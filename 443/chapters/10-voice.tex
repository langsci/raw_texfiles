\chapter{Voice}
\label{chap:voice}
\is{voice|(}
\section{Introduction}
\label{sec:introduction-10}
Kagayanen, like most Philippine languages, has an elaborate system for expressing the relationship between core \isi{grammatical relations} and \isi{semantic roles} in clauses. This system can be understood as a “voice” system, though it is more complex than the relatively simple “active/passive” systems found in many other languages of the world. Voice systems of all types provide speakers with ways of highlighting, or “perspectivizing” \citep{fillmore1977}\is{perspectivizing} different parts of their ideas. In this chapter, we give a brief overview of ongoing debates over voice in Philippine languages and describe the voice system of Kagayanen from a typological perspective.

The notions of \textit{semantic roles}\is{semantic roles} and \textit{grammatical relations} are crucial to understanding the Kagayanen voice system. This is because, as stated above, voice is a system for expressing and adjusting the relationship between semantic roles and grammatical relations in clauses. In \sectref{sec:semanticroles} and \sectref{sec:grammaticalrelations} we provide background to these notions and argumentation for why we consider \isi{ergative} and \isi{absolutive} to be the \textit{core grammatical relations}\is{core grammatical relations} in Kagayanen clauses. In \sectref{sec:voiceinphilippinelanguages} we provide some background as to how the notion of voice developed in linguistic studies of Philippine languages. In \sectref{sec:choir} we illustrate various constructions that may be considered the voices of Kagayanen. Finally, in \sectref{sec:othertransitivity} we describe certain other constructions that have recently been treated in the typological literature under the heading of ``voice", namely \isi{causative constructions}, \isi{reflexives} and \isi{reciprocals}. 

\section{Semantic roles}
\label{sec:semanticroles}
\is{semantic roles|(}
\textit{Semantic roles} are roles that participants play in the \textit{message world}. The \isi{message world} can be thought of as the shared imagined\footnote{By “imagined" we simply mean ideas and pictures formulated in the minds of people communicating with one another.} scenes being elaborated in any situation in which people are communicating. This world may correspond more or less closely to objective reality, but may be entirely fictitious, abstract or hypothetical. In any case, the message world is populated by participants and props whose properties, actions and relationships form the \textit{content} of linguistic communication. 

Though semantic roles influence morphosyntax profoundly, they are not primarily morphosyntactic (grammatical) categories. Technically, semantic roles exist quite apart from linguistic expression. For example, if in some imagined situation (which may or may not correspond to external reality), someone named Hiro purposely hits someone named Toshi, then Hiro is the agent and Toshi is the patient of the hitting event, regardless of whether anyone ever utters a clause like \textit{Hiro hit Toshi} to describe the event. If someone does want to describe this event, they will definitely need to communicate who is acting and who is undergoing the action. Therefore, every language provides grammatical tools for making that clear. Nevertheless, the roles themselves exist in the message world, while the means of representing them in language belong to grammar.

The term \textit{semantic macro-role} \citep{vanvalin1997} is a cover term for a range of semantic roles that tend to be treated similarly by individual languages.\footnote{The notion that semantic roles, as represented in the grammars of languages, are not discrete mind-external categories, but rather general “areas" of semantic space probably originated with \citet{dowty1991}. Dowty used the term “semantic proto-roles" to capture the fact that language structures reflect prototype notions of semantic roles. A similar idea is embodied in \citet{vanvalin1997} under the term “semantic macro-roles". All of this terminology has its roots in “case grammar" \citep{fillmore1977} who already was conceiving of “deep cases" as \textit{perceived} semantic roles, rather than objective, mind-external categories.} The macro-roles that have been most useful in this grammar are Actor and Undergoer. The Actor can be thought of as the participant that “does something". More specifically, the Actor controls, initiates, or is the starting point of a situation. Specific semantic roles that tend to be subsumed under the macro-role of Actor include agent (a conscious initiator), force (an unconscious/inanimate initiator), cognizer, experiencer and causer. The Undergoer can be thought of as the participant that is most “affected by" or is the end point of a situation. Specific semantic roles that tend to be subsumed under the macro-role of Undergoer include patient (something that undergoes a change of state), beneficiary, maleficiary, theme (an item transferred or described), location and instrument. The significance of these concepts will become clear in the following discussion.

It must be kept in mind that, as with all semantic notions, semantic roles are infinitely variable. There are no strict boundaries dividing agents, from patients, from experiencers, or from other semantic roles in the message world. There are very good examples of each of these roles, but there is an infinite range of possibilities between them. For example, the agent of an event described as “killing" is a very good (sometimes called \textit{prototypical}) agent, while the agent of “watching" is less prototypical. Someone who watches something does so on purpose, but the effect on the thing watched is less dire than it is on the thing killed. Because the human mind cannot deal adequately with infinite variability and in most cases there is little communicative value in assigning very precise “degrees of agentivity" to participants in a message world scene, there is a strong tendency for human beings to create bounded categories when they communicate with one another. In essence, that's what grammar is--the subconscious tendency to form categories out of infinitely variable semantic space. Therefore, the names that linguists give to particular semantic roles refer to “areas” in semantic space that grammars of individual languages categorize in different ways.

Furthermore, any given participant in an imagined scene may play more than one semantic role, to varying degrees. For example, the agent, Noel, in a situation described as \textit{Noel just ate three bananas} is presented as the controller of the situation, but he is also affected by the situation to a certain degree--not as much as the three bananas, but eating something certainly has some degree of effect on the eater.
\is{semantic roles|)}
\section{Grammatical relations}
\label{bkm:Ref482605816} \label{sec:grammaticalrelations}\is{grammatical relations|(}
The notion of \textit{grammatical relations} is crucial to understanding the Kagayanen voice system. In this section, we provide background to grammatical relations and argumentation for why we consider \textit{ergative}\is{ergative} and \textit{absolutive}\is{absolutive} to be the \textit{core grammatical relations}\is{core grammatical relations} in Kagayanen clauses.

Grammatical relations\is{grammatical relations} (GRs) are structurally-defined relations between elements in phrases and clauses. For example, ``subject of" is an important grammatical relation that holds between certain Referring Expressions and predicates in English clauses. The subject of a predicate is the Referring Expression (noun phrase or pronoun) that occurs in the nominative case (for pronouns) and normally occurs before the main verb or auxiliary in a predicate. Other referring expressions may have other grammatical relations, such as object of a verb or genitive of a noun phrase. These grammatical relations are tools that languages provide to help speakers express such important notions as who or what is affected in a situation being communicated, who or what is in control, and what is central versus peripheral \citep{comrie1989}.

Despite these clear and universal functions for grammatical relations, it is important to recognize that they are defined in terms of language-specific structural properties. One cannot simply “translate” the grammatical relations of one language into those of another language based on meaning alone. For example, the following two sentences are two expressions of a certain situation:

\ea \label{bkm:Ref482606160}
    \ea
    \label{ex:policecaughtthethief}
    \textit{The police caught the thieves}.
    \ex
    \label{ex:padakeptapulismgamatakaw}
    Padakep ta pulis mga matakaw. \\\smallskip
\gll Pa-dakep ta pulis mga matakaw. \\
    \textsc{t.r}-catch  \textsc{nabs}  police  \textsc{pl} thief   \\
    \z
\z

If we know something about traditional English grammar, we might say that \textit{the police} in \REF{ex:policecaughtthethief} has the grammatical relation of “subject of" the predicate in this sentence. If we assume that the subject relation translates directly from one language to the next, we might then identify \textit{pulis} ‘the police’ as the subject of \REF{ex:padakeptapulismgamatakaw}. Indeed, \textit{the police} in \REF{ex:policecaughtthethief} and \textit{pulis} in \REF{ex:padakeptapulismgamatakaw} both refer to the Actor (the controller, initiator or starting point) in the imagined scene described by the clause, and Actor is a semantic macro-role that is often expressed by the grammatical relation of subject in English. However, as we will see below, there are solid structural reasons for saying that \textit{mga matakaw} ‘the thieves’ is a better choice as the “subject of" \REF{ex:padakeptapulismgamatakaw}.

Grammatical relations must always be defined in terms of grammatical (structural) properties, rather than in terms of semantic roles like Actor. Even within one language, a grammatical relation like “subject” can represent message world participants with many different semantic roles. Consider the following English sentences:

\ea
\label{bkm:Ref482608282}
    \ea
    \label{ex:michellewasarrested}
    Michelle was arrested by the police. \\
    \ex
    \label{ex:michelleistall}
    Michelle is tall. \\
    \ex
    \label{ex:cobylooksstunning}
    Coby looks stunning tonight. \\
    \ex
    \label{ex:thesekeys}
    These keys belong to Igor. \\
    \ex
    \label{ex:inthishospital}
    In this hospital, sixteen orthopedic surgeries are performed every day.
    \z
\z

In \REF{ex:michellewasarrested} \textit{the police} is the Actor. In \REF{ex:michelleistall} through \REF{ex:inthishospital} there is no Referring Phrase that mentions an Actor at all. None of the participants mentioned in these last four examples are \textit{doing} anything. Yet, in these sentences “Michelle”, “Coby”, “these keys” and “sixteen orthopedic surgeries” can be identified as the subject on structural grounds: they each appear before the verb, each is the element the verb agrees with, and if pronominalized each would appear in the nominative (subject) case--\textit{she} in \REF{ex:michellewasarrested} and \REF{ex:michelleistall}, \textit{he} in \REF{ex:cobylooksstunning}, and \textit{they} in \REF{ex:thesekeys} and \REF{ex:inthishospital}. These structural facts are independent of the semantic function of Actor. Of course, it is very common for the subject to refer to an Actor, but this general tendency does not \textit{define} subjects in English or in any other language. Similarly, Referring Expressions with other grammatical relations may refer to the Actor in the right context. For example, in \REF{ex:michellewasarrested}, the Actor is expressed in an oblique phrase “by the police”. An analogy from everyday life may be helpful here. Hammers are tools that are often used for pounding nails. However, hammers may be used for other purposes, and other tools may be used to pound nails in the right context. In other words, there is not necessarily a one-to-one “mapping" between tools and their uses. Something similar is true for grammatical relations. Grammatical relations are structural tools that allow speakers to accomplish a range of communicative jobs.

Sometimes the term \textit{argument} is used to refer to any Referring Expression (RE) that has a grammatical relation to a verb or to some other syntactic element. This sense of “argument" is borrowed from mathematics, where an argument is an independent variable in a function; in other words, a thing that has a property or has a relation to some other thing. A nominal that doesn't have a grammatical relation to some other word is either called a “non-argument," or an \textit{oblique}. As with most terminology in linguistics, there is considerable literature and controversy surrounding the term “argument” (see, e.g., \citealt{grimshaw1990, malchukovcomrie2015, coon2017} and references cited therein). Nevertheless, for our purposes it is sufficient to consider REs that have grammatical relations within a clause to be “arguments” of that clause.

In order to discuss and compare systems of grammatical relations, it is convenient to identify three basic “semantico-syntactic roles” \is{semantico-syntactic roles!} termed S, A and O (\citealt{dixon1972, dixon1979, dixon1994}). Similar terms are used by \citet{silverstein1976}, and \citet{comrie1989}. These terms assume a basic and universal distinction between intransitive (single argument) and transitive (multi-argument) clause types:

\ea
\label{bkm:Ref482783415}
    \ea 
    \label{ex:gapanawpulisan}
    Intransitive (Single argument) clause: \\
    Gapanaw pulis an. \\
    \hspace{3cm}S \\
    \gll Ga-panaw [ pulis an ]. \\
    \textsc{i.r}-go/walk {} police \textsc{def.m} \\
    \glt ‘The police left.’
    \ex
    Transitive (multi-argument) clause: \\
    Padakep ta pulis mga matakaw. \\
    \hspace{3cm}A\hspace{3cm}O \\
    \gll Pa-dakep [ ta pulis ] [ mga matakaw ]. \\
    \textsc{t.r}-catch {} \textsc{nabs} police {} {} \textsc{pl} thief \\
    \glt ‘The police caught the thieves.’
    \z
\z
The S is defined as the only nominal argument of a single argument clause. While the term S may remind us of the grammatical relation “subject”, S as used in this framework refers only to the “Single core argument” of an intransitive clause.

The A is defined as the most “Actor-like argument” of a transitive clause. If there is no argument that is very Actor-like, the A is the argument that is treated grammatically in the same manner as prototypical Actors are treated. Usually there will be one argument in every verbal clause that exhibits this property, though there may not be. The O is the “Other argument” of a multi-argument clause.\footnote{Some clauses may have three obligatory “arguments”. English clauses containing verbs such as \textit{give} and \textit{put} are of this type. For the purposes of the current discussion, it will be sufficient to consider only one- and two-argument clauses.}  While the term O may remind us of Object, O refers only to the less Actor-like argument of a transitive (two-argument) clause.

Different languages may treat these three basic relations differently in terms of case marking, constituent order, pronominal form, and other grammatical features. If a language treats S and A as “the same” and O differently, we can say the language has a “Subject-Object”, or “Nominative-Accusative”, system. This is the system that is familiar from many Indo-European and other languages of the world. For example, English personal pronouns are organized on a Nominative-Accusative basis. The feminine singular pronoun \textit{she} functions in both A and S roles, while the distinct pronoun \textit{her} is required in the O role.

\ea 
        \ea She left.
	\ex She caught her.
	\z
\z
\begin{figure}
	\caption{A nominative-accusative system for organizing grammatical relations}
	\begin{forest} for tree = {no edge, s sep=0.9cm}
		[S,name=S [A,name=A] [O,name=O,draw,rounded corners]]
		\draw[rounded corners]  (S.south west) -- (S.north west) -- (S.north east) -- (S.south east) -- (A.north east) -- (A.south east) -- (A.south)  -- (A.south west) -- (A.north west) -- cycle;
		\node [below=5pt of A, align=center] {\textsc{nom}\\(subject)\\`she'};
		\node [below=5pt of O, align=center] {\textsc{acc}\\(object)\\`her'};
	\end{forest}
\end{figure}

In English, constituent order \is{constituent order} (sometimes referred to as “word order") also reveals that S and A together constitute a basic grammatical relation as opposed to O--in pragmatically neutral, declarative clauses, S and A occur before the verb or auxiliary, while O occurs after the verb. Other more subtle grammatical properties also confirm that S together with A constitute a consistent grammatical relation in English.

Such a nominative/accusative, or subject/object system seems very intuitive and “natural” to speakers of languages that employ such a system. However, there are several other possible ways a language might categorize S, A and O, some of which are attested in the world’s languages, while others are not (see e.g., \citealt[139--167]{payne1997}).

The most commonly attested alternative to the nominative/accusative system is called an ergative/absolutive system. In this way of organizing grammatical relations, the S and O are treated as “the same” grammatical relation, whereas the A is treated distinctly for purposes of case marking, pronominal form and other grammatical features. If any grammatical relation includes A alone it can be called the \textit{ergative} relation\is{case!ergative}\is{ergative case}, and any grammatical relation that encompasses S and O can be termed the \textit{absolutive} relation\is{case!absolutive}\is{absolutive case}. This is the system exhibited in basic clauses of many languages of the world, including Kagayanen. The examples in \REF{bkm:Ref482781620} illustrate an ergative/absolutive system in personal pronouns\is{ergative/absolutive system}:

\ea 
\label{bkm:Ref482781620}
	\ea 
        \label{ex:gapanawkanenan}
        \label{ex:sheleft}
	\hspace{2.6cm}  S \\
	\gll Ga-panaw \textbf{kanen} an. \\
        \textsc{i.r}-walk/leave 3\textsc{s.abs} \textsc{def.m} \\
	\glt ‘S/he left.’
	\ex 
        \label{ex:shecaughtthethief}
	\hspace{1.8cm}A \hspace{1cm} O \\\smallskip
\gll Pa-dakep \textbf{din} mga matakaw. \\
        \textsc{t.r}-catch 3\textsc{s.erg} \textsc{pl} thief \\
	\glt ‘S/he caught the thieves.’
	\ex 
        \label{ex:thepolicecaughther}
	\hspace{2cm}A \hspace{2.3cm}O \\
	\gll Pa-dakep ta pulis an \textbf{kanen} ya. \\
        \textsc{t.r}-catch \textsc{erg} police \textsc{def.m} 3\textsc{s.abs} \textsc{def.f} \\
	\glt ‘The police caught him/her.’
	\z
\z
\begin{figure}
	\caption{An ergative--absolutive system for organizing grammatical relations}
	\begin{forest} for tree = {no edge, s sep=0.75cm}
		[S,name=S [A,name=A,draw,rounded corners] [O,name=O]]
		\draw[rounded corners]  (S.south west) -- (S.north west) -- (S.north east) -- (S.south east) -- (O.north east) -- (O.south east) -- (O.south west) -- (O.north west) -- cycle;
		\node[right=1ex of O] {\strut{absolutive}};
		\node[left=1ex of A] {\strut{ergative}};
		\node[below=5pt of A] {\textit{din}};
		\node[below=5pt of O] {\textit{kanen}};
	\end{forest}
\end{figure}
In these very basic examples, we see that the pronoun referring to the S of \REF{ex:sheleft}, is \textit{kanen}. In \REF{ex:shecaughtthethief}, however, we see that the A argument is \textit{din}, even though both A and S are third person singular. The pronoun referring to the O argument in \REF{ex:thepolicecaughther} is \textit{kanen}, just like the S in \REF{ex:sheleft}. Since the A is the unique member of this system, being expressed by \textit{din}, we can say that \textit{din} expresses a third person singular referent in the \textit{ergative} relation. The S and the O, on the other hand, are both expressed by \textit{kanen}. Therefore, we can call this relation the \textit{absolutive}.

Unlike English, constituent order in Kagayanen, and in fact all dominantly verb-initial languages, does not directly illustrate the system for organizing grammatical relations. Although the dominant orders in intransitive and transitive clauses in Kagayanen are VS and VAO respectively, these orders do not lend themselves to grouping A, S and O in the same way as in a verb-medial language like English. One could say that S and A are treated alike because they both immediately follow the verb. But one could also say that S and O are treated alike because they both occur at the end of the basic clause. So constituent order is not a grammatical feature that can be used to identify the \textit{system} for organizing grammatical relations in Kagayanen.

However, there is another overt feature that does illustrate the ergative/absolutive character of basic grammatical relations in Kagayanen, namely the prenominal case marker \textit{ta}, as in the following examples:

\ea 
	\ea 
        \label{ex:gapanawpulis}
	\hspace{3cm}S \\
	\gll Ga-panaw [ pulis an ]. \\
        \textsc{i.r}-walk/leave {} police \textsc{def.m} \\
	\glt ‘The police left.’
	\ex 
        \label{ex:padakeptapulismatakawan}
	\hspace{3cm}A\hspace{2cm}O \\
	\gll Pa-dakep [ ta pulis an ] [ mga matakaw ]. \\
        \textsc{t.r}-catch {} \textsc{erg} police \textsc{def.m} {} {} \textsc{pl} thief \\
        \glt ‘The police caught the thieves.’
	\z
\z
\begin{figure}
	\caption{The ergative-absolutive system of Kagayanen: Case marking}
	\begin{forest} for tree = {no edge, s sep=0.75cm}
	[S,name=S [A,name=A,draw,rounded corners] [O,name=O]]
	\draw[rounded corners]  (S.south west) -- (S.north west) -- (S.north east) -- (S.south east) -- (O.north east) -- (O.south east) -- (O.south west) -- (O.north west) -- cycle;
	\node[right=1ex of O] {\strut{absolutive}};
	\node[left=1ex of A] {\strut{ergative}};
	\node[below=5pt of A] {\emph{ta}\,+\,N};
	\node[below=5pt of O] {0\,+\,N};
	\end{forest}
\end{figure}
In these examples, we see that the S of \REF{ex:gapanawpulis}, \textit{pulis an}, is not preceded by any overt prenominal case marker. In \REF{ex:padakeptapulismatakawan}, however, we see that the A argument, \textit{ta pulis an}, appears with the prenominal marker \textit{ta}. The O argument in \REF{ex:padakeptapulismatakawan} has no prenominal marker, just like the S in \REF{ex:gapanawpulis}. Since the A is the unique member of this system, being preceded by \textit{ta}, we can say that \textit{ta} marks the \textit{ergative} relation (or sometimes \textit{ergative case}). The S and the O, on the other hand, are both unmarked. Therefore, we can call this relation the \textit{absolutive} \citep{silverstein1976, comrie1989, dixon1994}.

In summary, any grammatical system that treats S and A alike as opposed to O is a nominative-accusative system for organizing grammatical relations. Any system that treats S and O alike as opposed to A is an ergative-absolutive system. In Kagayanen, personal pronouns and prenominal case markers clearly reveal an ergative-absolutive system in basic clauses.

One apparent complication to this story in Kagayanen is that, in addition to the transitive construction illustrated in \REF{ex:shecaughtthethief}, \REF{ex:thepolicecaughther} and \REF{ex:padakeptapulismatakawan}, there is another very common construction for expressing semantically transitive\is{semantic transitivity}\is{transitivity!semantic} situations (also known as “two-place predicates”, as in \citealt{polinsky2013b}, \textit{inter alia}). This is what we are calling the \textit{detransitive}\is{detransitive}, or \textit{actor voice}\is{actor voice} construction, as illustrated in \REF{ex:gadakeppulisantamatakaw} and \REF{ex:gadakeppulisankikanen}:

\ea
\label{bkm:Ref482789597} \label{ex:gadakeppulisantamatakaw}
    Gadakep pulis an ta matakaw. \\\smallskip
\gll Ga-dakep pulis an ta matakaw. \\
    \textsc{i.r}-catch police \textsc{def.m} \textsc{nabs} thief \\
    \glt ‘The police caught/are catching a/the thief/thieves.’
\z
In this construction, the grammatical relations seem to be reversed from what we have seen in transitive examples such as \REF{ex:padakeptapulismatakawan}. Now the most Actor-like argument occurs with no prenominal \textit{ta}, and instead the other argument is preceded by \textit{ta}. There are important reasons, however, for saying the grammatical relations are not simply reversed in these examples. To be sure, the Actors (\textit{pulis an} in both examples) are in the absolutive case--they have all the structural properties of the S and O arguments in examples \REF{ex:sheleft}, \REF{ex:gapanawpulis}
and \REF{ex:padakeptapulismatakawan}. However, the phrases preceded by \textit{ta} do not have the same structural properties of A arguments in examples \REF{ex:padakeptapulismatakawan} and \REF{ex:padakeptapulismatakawan}. The \textit{ta}-marked argument in \REF{bkm:Ref482789597} is in a different grammatical relation. We know this for several reasons. First, when non-agent arguments are pronominalized, they do not occur in the case we have identified as “ergative” above.

\ea
    \ea
    \label{ex:gadakeppulisankikanen}
    Gadakep pulis an ki kanen. \\\smallskip
\gll Ga-dakep pulis an ki kanen. \\
    \textsc{i.r}-catch police \textsc{def.m} \textsc{obl.p} 3s\\
    \glt ‘The police caught him/her.’
    \ex
    \label{ex:gadakeppulisan}
    \gll *Gadakep pulis an (ki) din. \\
         {} {} {} {} \textsc{3s.erg} \\
    \z
\z
In \REF{ex:gadakeppulisankikanen}, the Actor, \textit{pulis}, appears in the absolutive case, but the pronominalized human patient follows the locative preposition \textit{ki}. Again, the ergative pronoun \textit{din} is impossible in this context. So this is clear structural evidence that the RP preceded by \textit{ta} in a detransitive construction is in a different grammatical relation than the ergative RP in a transitive construction.

Furthermore, in example \REF{ex:gadakeppulisankikanen} the verb takes the prefix \textit{ga}-, whereas in the ergative constructions presented earlier, the verb takes the prefix \textit{pa}{}-. Recall that \textit{ga}{}- is the basic dynamic mode realis prefix that occurs on verbs that are unquestionably intransitive (e.g., one-place predicates), such as \textit{panaw} ‘to walk/leave’ in \REF{ex:gapanawpulisan} and \REF{ex:gapanawkanenan}. Whenever \textit{ga}- appears on the verb, the Actor is the only required argument and it always appears in the absolutive role, either as a Referring Phrase with no pre-nominal case marker or in the form of an absolutive case pronoun. For this reason, we have analyzed \textit{ga}{}- and its irrealis counterparts \textit{mag}- and \textit{m-} as markers of grammatical intransitivity. In other words, all the examples in \REF{ex:gadakeppulisantamatakaw} and \REF{ex:gadakeppulisankikanen} are \textit{grammatically intransitive}, even though they express ideas that have two major participants. Similar constructions have also been called “actor-voice” \citep {himmelmann1991, foley2008, kaufman2017} “actor-focus” \citep{schachter1972} or “antipassive” \citep{bell1979, aldridge2012} in studies of other Philippine languages.

Because these constructions describe situations that involve transfer of energy from an Actor to an Undergoer, they can be considered semantically transitive\is{semantic transitivity}\is{transitivity!semantic}. However, structurally they are intransitive. Therefore we refer to such structures as \textit{detransitive constructions}\is{detransitive constructions}. Many examples of detransitive constructions are found throughout this grammar. The following are some additional examples from the corpus:

\ea
\label{bkm:Ref125017724v}
absolutive = Actor: \\
Piro  tama  man  na  mga  inay  na  galuag  man  ta  kaanlao ... \\\smallskip
\gll Piro  tama  man  na  mga  inay  na  ga-luag  man  ta  kaanlao ... \\
but  many  also  \textsc{lk}  \textsc{pl}  mother  \textsc{lk}  \textsc{i.r}-watch  \textsc{emph}  \textsc{nabs}  lunar.eclipse \\
\glt ‘But there were many mothers who watched a lunar eclipse ...’ [JCOE-C-03 5.4]
\z
\ea
\label{bkm:Ref118364790v}
Pagtapos  nay  igma,  gatan-aw  kay  ta  sini. \\\smallskip
\gll Pag-tapos  nay  igma,  ga-tan-aw  kay  ta  sini. \\
\textsc{nr.act}-finish  1\textsc{p.excl.erg}  lunch  \textsc{i.r}-look  1\textsc{p.excl.abs}  \textsc{nabs}  movie \\
\glt ‘After we finished lunch, we were watching a movie.’ [AGWN-L-01 5.5]
\z
In example \REF{bkm:Ref118364790v} the root \textit{tan-aw} ‘to look’ describes an activity that necessarily involves an Actor (the looker) and an Undergoer (the thing looked at). However, in this example, the verb is marked as intransitive, and the Actor, “we”, is in the absolutive case. Like all intransitive clauses, this is a statement about the referent of the absolutive element; it is not about the particular movie. The Undergoer is mentioned, but in a non-absolutive role marked by the prenominal case marker \textit{ta}. The story goes on to talk about the Actors going home and then going other places.

Similarly, example \REF{bkm:Ref117000014} is about the unstated absolutive participant, “us”, and not about any particular trees. Again, as in all intransitive constructions, the absolutive is presented as the most salient or highly affected participant in the situation being described. The Undergoer is presented in a less “privileged” syntactic role. In fact, in such constructions, the Undergoer is often omitted altogether, as in example \REF{bkm:Ref117000205} further below.

\ea
\label{bkm:Ref117000014}
 ... basi  en  na  mananem  nang  ta  kaoy  en. \\\smallskip
\gll ... basi  en  na  ma-ng-tanem  nang  ta  kaoy  en. \\
 {} perhaps  \textsc{cm}  \textsc{lk}  \textsc{a.hap.ir-pl}-plant  only  \textsc{nabs}  tree  \textsc{cm} \\
\glt ‘... perhaps now (we) will be able to/might just plant trees.’ [ROOB-T-01 12.1]
\z
In \REF{bkm:Ref117000354} there is no particular orchid flower on stage in the situation being described. Rather the speaker is saying something about “you”. Seeing an orchid flower in the wild is more likely to affect the person than the flower.

\ea
\label{bkm:Ref117000354}
... makakita      ka      ta  buļak  na  dapo. \\\smallskip
\gll ... maka-kita      ka      ta  buļak  na  dapo. \\
{}  \textsc{i.hap.ir}-see  \textsc{2sabs}      \textsc{nabs}  flower  \textsc{lk}  orchid \\
\glt ‘... you may happen to see an orchid flower.’ [EFWE-T-05 3.4]
\z

Finally, in example \REF{bkm:Ref117000205}, there are two verbs that have the same absolutive argument, the enclitic \textit{ka} ‘you’. Both verbs \textit{butaļ} ‘fight’ and \textit{kaan} ‘eat’ describe semantically transitive\is{semantic transitivity}\is{transitivity!semantic} situations – fighting involves at least two people, and eating necessarily involves an eater and something that gets eaten. Yet each of these verbs is presented as grammatically intransitive with no mention of any Undergoer. This is because this is an admonition from a parent to a child about how the child should behave. It is all about the Actor, and not about any particular Undergoer.

\ea
\label{bkm:Ref117000205}
Dili    ka    mabutaļ  daw  naan  ta  lamisaan  na  gakaan. \\\smallskip
\gll Dili    ka    ma-butaļ  daw  naan  ta  lamisa-an  na  ga-kaan. \\
\textsc{neg.ir}  2\textsc{s.abs}    \textsc{a.hap.ir-}fight  if/when  \textsc{spat.def}  \textsc{nabs}  table-\textsc{nr}  \textsc{lk}  \textsc{i.r}-eat \\
\glt `You should not fight/argue when at the table when eating.' [ETOP-C-10 2.2]
\z

Detransitive constructions such as these serve a communicative function that is similar to what have been called “Oblique Object” or “Object Omission” constructions in English, illustrated in \REF{bkm:Ref117062761}:

\ea
\label{bkm:Ref117062761}
\begin{tabbing}
\hspace{5.2cm} \= \kill
Oblique Object:   \>   Object Omission: \\
Fido chewed on the bone. \>     Thompson already ate. \\
The hunter shot at the deer. \>     We just sat back and watched. \\
Lucretia ate of the poison apple. \>   Pele kicks, he scores!
\end{tabbing}
\z
All of these English sentences describe situations that can be considered semantically transitive\is{semantic transitivity}\is{transitivity!semantic}-they all involve action carrying over from an Actor to an Undergoer. However, they are presented as grammatically intransitive--they have a subject but no direct object. The Undergoers in Oblique Object constructions are expressed in an oblique phrase, and the Undergoers in Object Omission constructions are not present at all; they are understood because of the nature of the predicate, but their identity is either completely obvious or not relevant to the speaker’s message. All of these English constructions present a situation as something that primarily affects the Actor, while downplaying the Undergoer in some way, either by suggesting it is incompletely affected, obvious, or not relevant at all. Because these constructions present semantically transitive\is{semantic transitivity}\is{transitivity!semantic} situations in a grammatically intransitive way, we may think of them as “detransitive” constructions.

In a similar vein, situations described as \textit{Phillip eats pizza}, or \textit{We ate pizza for lunch} also present the activity of eating as the primary assertion, rather than the effect of the activity on a distinct patient. In fact, \citet[293]{halliday2014} describe such clauses in English as intransitive, with \textit{pizza} filling the participant function (similar to semantic role) of “Range” rather than patient. In other words, \textit{eating pizza} is a kind of eating, in which \textit{pizza} simply restricts the range of application of the verb, similar to the function of adverbial modifiers, as in \textit{eating in the kitchen}, or \textit{eating quickly}. 

Similar observations can be made regarding the functions of detransitive constructions in Kagayanen. As in all intransitive constructions, the primary assertion of a detransitive construction is the activity, and its effect on the Actor, rather than what happens to a distinct Undergoer.

While the functions of detransitive constructions in English and Kagayanen are similar, there are major structural differences that may obscure their common communicative functions. Unlike English, in which both transitive and intransitive expressions of the concepts of \textit{chewing, watching,} or \textit{eating} are expressed with the same set of verb forms, Philippine languages overtly inflect verbs as either grammatically transitive or intransitive.  The same verbs in a transitive frame take a different set of inflections (see \chapref{chap:verbstructure}, \tabref{tab:inflectionalaffixes}). This is consistent with the general morphosyntactic typology of most Philippine languages – much more communicative “work” is accomplished with verbal morphology than in English. Transitivity is not  a category that is very important to the verbal morphology in English, whereas it is in Philippine languages.

A second major difference between detransitive constructions in English and Kagayanen is that in English, the Actor is in the same form (called nominative case) in transitive or intransitive constructions. This is because in a nominative-accusative system, the S (single argument) of an intransitive clause and the A (most Actor-like argument) of a transitive clause are both nominative. However, since the system of grammatical relations in Kagayanen is basically ergative-absolutive, the Actor in a transitive construction is in the ergative case, while the Actor in all intransitive constructions, including detransitives, is in the absolutive case.

Another complication that makes the detransitive construction difficult to understand for speakers of non-Philippine type languages is the role of the prenominal marker \textit{ta}. The fact that \textit{ta} precedes Actors in basic transitive constructions and precedes Undergoers in detransitive constructions leads some linguists to conclude that these two different semantic roles are expressed in the same case. The name given to that case is sometimes “genitive” (see, e.g., \citealt{kroeger1993}, referring to \isi{Tagalog}). However, as we have argued above, \textit{ta} is multifunctional--it expresses ergative case, genitive (possessor) case, oblique case, and the case of the demoted Undergoer in detransitive constructions. These are different cases that represent different grammatical relations, as revealed by their usages (see \tabref{tab:cases}).

\begin{table}
\caption{The cases of Kagayanen}
\label{tab:cases}
\begin{tabularx}{\textwidth}{QQQl}
\lsptoprule
{\bfseries\itshape Prenominal case markers} & \bfseries\itshape Personal pronouns (3\textsc{sg} human)\footnote{For
        clarity, we are using the 3rd person singular human personal pronouns as representative of the entire set of pronouns that function in this column, as described in \chapref{chap:referringexpressions}, \sectref{sec:pronouns}.
        }
        & {\bfseries\itshape Usages} & {\bfseries\itshape Case} \\
\midrule
\emptyset{} & kanen & S and O arguments & ABS \\
\tablevspace
\emptyset{} & din & Personal name A arguments & ERG \\
\tablevspace
ta & din & Other A arguments & ERG \\
\tablevspace
ta & din & Adnominal possessors & GEN \\
\tablevspace
ki & ki kanen & Personal name demoted Undergoers & NABS \\
\tablevspace
ta & ki kanen & Other demoted Undergoers & NABS \\
\tablevspace
(LOC) ta / ki & ki kanen & Oblique elements & OBL \\
\lspbottomrule
\end{tabularx}
\end{table}

Although \textit{ta} has a role in four of the seven rows of \tabref{tab:cases}, the different pronoun sets and usages confirm that each row represents a different “case”. This kind of multifunctionality is not at all unusual in the world’s languages. There are situations even in English, a language with a very impoverished case marking system, in which identical forms are used to express different cases. For example, the pronoun \textit{her} is used for genitive (\textit{her house}), accusative (\textit{we saw her}), and oblique roles (\textit{they bought it from her}). Also, the pronouns \textit{you} and \textit{it} occur both in nominative (\textit{you rang}, \textit{it collapsed}), accusative (\textit{I saw you}, \textit{I saw it}), and oblique roles (\textit{I sent it to you}, \textit{I fell on it}). No one argues that these cases are the same just because some of the forms that express them happen to be the same. Multifunctionality is normal in language, and \textit{ta} is a very useful, multifunctional morpheme.

Finally, some linguists have argued that the Undergoer in examples such as \REF{bkm:Ref482789597} and \REF{ex:gadakeppulisankikanen} is not an oblique argument at all, but is a core argument, similar to a direct object (see, e.g., \citealt{kroeger1993, foley2008} for other Philippine languages). In Kagayanen, there is no doubt that the demoted Undergoer in a detransitive construction is structurally distinct from other oblique elements. For example, demoted Undergoers cannot naturally be expressed before the verb:

\ea
\label{bkm:Ref497135886}
*\textit{Ta bata gasagod bai an.} \\  
‘A child the woman cared for.’
\z
This example is highly awkward or ungrammatical because the demoted Undergoer, \textit{ta bata}, is presented before the verb. Some other oblique elements may easily occur in this position:

\ea
\label{bkm:Ref497135842}
Naan  ta  uma  nay  patanem  nay  kamuti  an. \\\smallskip
\gll Naan  ta  uma  nay  pa-tanem  nay  kamuti  an. \\
\textsc{spat.def}  \textsc{nabs}  field  1\textsc{p.excl.erg}  \textsc{t.r}-plant  1\textsc{p.excl.erg}  camote  \textsc{def.m} \\
\glt ‘In our field we planted camote.’
\z
\ea
\label{bkm:Ref497135845}
Naan  ki  kanen  paatag  ko  bayo  no  ya. \\\smallskip
\gll Naan  ki  kanen  pa-atag  ko  bayo  no  ya. \\
\textsc{spat.def}  \textsc{obl.p}  3s \textsc{t.r}-give  1\textsc{s.erg}  clothes  2\textsc{s.gen}  \textsc{def.f} \\
\glt ‘To him/her I gave your shirt/clothes.’
\z
However, in \REF{bkm:Ref497135842} and \REF{bkm:Ref497135845}, the spatial element \textit{naan} must occur before the fronted oblique phrase. These examples are as ungrammatical as \REF{bkm:Ref497135886} if \textit{naan} is omitted. In fact, it is very odd for any non-absolutive RP to be fronted without including \textit{naan}, though they may occur post-verbally without it:

\ea
Patanem  nay  kamuti  ta  uma  nay. \\\smallskip
\gll Pa-tanem  nay  kamuti  ta  uma  nay. \\
\textsc{t.r}-plant  1\textsc{p.excl.erg}  camote  \textsc{nabs}  field  1\textsc{p.excl.gen} \\
\glt ‘We planted camote in our field.’ \\\smallskip

\textit{*Ta uma nay patanem nay kamuti an}.
\z
\ea
Paatag  ko  bayo  no  ya  ki  kanen. \\\smallskip
\gll Pa-atag  ko  bayo  no  ya  ki  kanen. \\
\textsc{t.r}-give  1\textsc{s.erg}  clothes  2\textsc{s.gen}  \textsc{def.f}  \textsc{obl.p}  3\textsc{s} \\
\glt ‘I gave your shirt/clothes to him/her.’\\\smallskip
\textit{*Ki kanen paatag ko bayo no ya}.
\z

These Waray facts are not particularly surprising, given that several previous studies have argued that the core/oblique distinction is neither absolute nor universal. For example, \citep{arka2019} argues that in several languages of Indonesia clause constituents exhibit a range of syntactic properties such that “core" and “oblique", in Arka's view, are ends of a cline rather than absolute categories. Role and Reference Grammar \citep{vanvalin1997, vanvalin2000} proposes a distinction between “core obliques” versus “peripheral obliques.” Relational grammar \citep{perlmutter1986} famously introduced the concept of “chômeur” as a nominal relation that has syntactic properties of neither “terms” (core arguments) nor obliques. While we do not take a stand on any of these theoretical frameworks, we believe they do offer some insight into the clause structure of detransitive constructions in Waray. Genitive-marked Undergoers are in a pragmatically “deperspectivized” grammatical role, and as such lack some syntactic properties of core arguments (ergatives and absolutives) while simultaneously lacking some syntactic properties of prototypical obliques. 

In Kagayanen, and Philippine languages generally, there are relatively few pre-nominal case marking particles, therefore each one must serve several functions. This is another reason why \textit{ta} is used for genitive, ergative, oblique, and demoted Undergoer functions. Part of the genius of Philippine languages is that their rich systems of verbal morphology help to clarify the function of each Referring Phrase, making it unnecessary to mark the semantic role of each individual RP directly.
\is{grammatical relations|)}
\section{Voice in Philippine languages: An introduction}
\label{sec:voiceinphilippinelanguages}

We’ve seen that the system of organizing grammatical relations in Kagayanen is very different from what is found in European languages; the central grammatical relations in clauses are absolutive and ergative, rather than subject and object. This is a very common system in other Austronesian languages, and in languages of many other parts of the world, though it is often perplexing to speakers and linguists who are used to subject/object systems. In this section we will discuss the notion of “voice” as a grammatical system for adjusting the relationship between grammatical relations and semantic roles in clauses. We will begin with a brief outline of the history of scholarship surrounding voice in Philippine languages, most of which deals with \isi{Tagalog}.

\citet{blake1906} set the stage for the view of \isi{Tagalog} verbs as expressing four distinct “voices." Blake used the term “expression of case by the verb" to describe the ways in which verb morphology varies to express different semantic roles of the “subject" (our absolutive) argument. The \isi{Tagalog} system, according to Blake, represents a more fully developed version of the expression of case by the verb than the two-way active/passive voice systems common in Indo-European and Semitic languages. In other words, for Blake “voice" is a limited variety of case expression in the verb. More recent approaches (e.g., \citealt{wolff1967, payne1994, foley2008}, and \citealt{himmelmann2008}) have reversed Blake's characterization, instead extending the definition of voice to include Philippine-type systems, rather than extending the definition of case in the verb to include voice in the Indo-European and Semitic traditions. In this grammar, we will use the term voice in the more recent sense to refer to any system of verb (phrase) morphosyntax that specifies the alignment between grammatical relations and semantic roles in a clause.

Ever since \citet{blake1906}, linguists have used various terms to describe the basic clause structure of Philippine languages. \citet{bloomfield1917} treated \isi{Tagalog} as though it were an Indo-European language with an active/passive voice system. The “active voice” for Bloomfield was what we are calling the detransitive (see above), and the “passive voice” was what we are calling the basic transitive (ergative) construction. By the mid-1960s, however, linguists were beginning to realize that the active/passive perspective didn’t work very well for \isi{Tagalog}. For example, \citet{schachter1972} treated the basic clause structure of \isi{Tagalog} as a “predicate+topic” system, rather than a voice system. The various verbal categories relating to argument relations represented options for bringing different arguments into the topic position.\footnote{Of course, this is exactly what a voice system does, but because the \isi{Tagalog} system seemed so different from familiar active/passive systems, \citet{schachter1972} were reluctant to use the term “voice”.} \citet{schachter1972} were actually inconsistent in their terminology, in that they used the term “topic” when referring to nominal categories, signaled by the pre-nominal particles and certain pronouns, and “focus” when referring to verb forms. Thus, for example, the “Actor-focus” verb forms were ways of indicating that the Actor was the “topic” of the clause.\footnote{“\textsc{Focus} is the feature of a verbal predicate that determines the semantic relationship between a predicate verb and its topic.” (\citealt[69]{schachter1972}, emphasis in the original). This apparent treatment of “topic” and “focus” as the same caused a great deal of confusion, since most linguists consider these categories to be in opposition to each other (see, e.g., \citealt{chafe1976, lyons1968} \textit{inter alia}).} Some subsequent work changed the terminology to be more internally consistent, using “Actor-topic”, “Patient-topic” and so on for the verb forms, while others kept the focus terminology for both verbal and nominal categories. The result was a great deal of confusion and sense of mystery among linguists concerning the basic clause structure of \isi{Tagalog}, and Philippine languages in general.

In 1977 Resty Cena presented arguments for what he called “patient primacy” in \isi{Tagalog} \citep{cena1977}. Arguing from multiple perspectives, including frequency, syntactic simplicity, psycholinguistic salience \citep{tucker1971}, and age of acquisition \citep{segalowitz1978}, Cena showed that Bloomfield’s “passive” voice (Schachter \& Otanes’ “goal focus”) construction was the more “basic” or “natural” way of expressing a transitive situation. \citet{deguzman1992} followed up with morphological evidence for the same claim. Both Cena and De Guzman used the subject/object terminology similar to Bloomfield, but argued that the subject of a transitive clause in \isi{Tagalog} was a primary patient, rather than an Actor. In other words, the subject for Cena and De Guzman was what we are calling the absolutive in Kagayanen. They independently argued that universally there are two possible primary orientations to a transitive situation, one in which the role of Actor is primary, as in English and many other languages, and the other in which the role of patient is primary, as in \isi{Tagalog}. Though neither Cena nor De Guzman argued explicitly for an ergative/absolutive analysis of \isi{Tagalog}, they both recognized and acknowledged the conceptual similarity between patient primacy in \isi{Tagalog} and indisputably ergative systems \citep[87]{deguzman1992}.

Several linguists in recent years have followed the precedent set by Cena in referring to the absolutive relation in \isi{Tagalog} as the “subject”, acknowledging that this relation is defined very differently than the subject in a language like English (see, e.g., \citealt{himmelmann1991, kroeger1993, kaufman2009}). At some level, this may be considered mere terminological variation--whether one chooses to call the case that expresses intransitive S arguments and basic transitive O arguments “absolutive” or “subject” may not seem like a substantive analytic decision. However, in our opinion, using terminology that is common in analyses of nominative/accusative systems for languages like \isi{Tagalog} and Kagayanen, for which grammatical relations have very different properties, is confusing for speakers and linguists alike in that it may perpetuate the impression that these languages are “mysterious” and fundamentally dissimilar to all other languages of the world. In our opinion, viewing the system for organizing grammatical relations as basically ergative/absolutive, and the voice system as involving various ways of adjusting the relationships between semantic roles and grammatical relations in a clause provides a coherent way of understanding the basic clause structure of most Philippine languages that is consistent with what is known about language in general. The voice system is clearly more elaborate than in most languages commonly studied by linguists, but its fundamental character is well-understood from the extensive body of literature now available on voice and grammatical relations in general, and on ergative/absolutive systems in particular (see, e.g., \citealt{comrie1989, dixon1994, payne1997, dixon2000, bahrt2021}).

\section{“The Kagayanen choir”}
\label{sec:choir}

In order to understand the role of voice in any language, it may be helpful to start with the fairly simple voice system of a well-known language like English. Consider the sentences in \figref{fig:kagayanenchoir}. Examples (d) and (e) are marginal or ungrammatical to most English speakers, but they are, we believe, understandable enough to serve as a bridge to understanding voice in Kagayanen:

\begin{figure}
    \centering
    % \includegraphics{}
    \caption{The Kagayanen choir}
\label{fig:kagayanenchoir}
	\resizebox{\textwidth}{!}{\begin{tabular}[t]{@{}l@{~}l|l|l} \cline{3-3}
	a. & Actor:	      & I 	         & served fish (to the child, on a plate, with a spoon).\\
	b. & Patient:     & Fish 	     & was served (to the child, by me, on a plate, with a spoon).\\
	c. & Beneficiary: & The child	 & was served (fish by me, on a plate, with a spoon).\\
	d. & Location:	  & ??The plate	 & was served fish on (by me, with a spoon).\\
	e. & Instrument:  & ??The spoon	 & was served fish with (by me, on a plate).\\
	\cline{3-3}
	\end{tabular}} \\ 
\vspace{4pt}
\hspace{1.4cm}PSA (Privileged syntactic argument, \citealt{vanvalin2000})
\end{figure}

Using terminology from \citet{vanvalin1997}, we can understand these different syntactic constructions as ways of bringing various parts of a sentence into a “privileged” grammatical status (indicated in the box in \figref{fig:kagayanenchoir}, and called the “subject” in traditional English grammar). This privileged status is “special” in several ways. In terms of syntax, the \isi{Privileged Syntactic Argument} (PSA) is the argument that controls various syntactic patterns, like verb agreement, reflexivization, coreference in coordinate structures and others. In terms of pragmatics, the PSA tends to refer to the most “topical” participant. That is, the sentences in \figref{fig:kagayanenchoir} can be understood as making a comment “about” the Actor (\textit{I} in 10.4a), the patient (\textit{fish} in 10.4b), the beneficiary (\textit{child} in 10.4c), the location (\textit{plate} in  10.4d), and the instrument (\textit{spoon} in 10.4e) respectively. When these noun phrases are the PSA, they no longer carry any direct marking of their semantic role. In other words, they no longer follow any benefactive, locative or instrumental prepositions. The only overt clues as to the semantic role of the PSA are found in the verb phrase, namely the auxiliary (a form of \textit{be}), the past participle form of the main verb, and possible post-verbal particles, such as \textit{on}, \textit{in}, or \textit{with}.

The important feature to notice about the English system is that there are only two grammatical voices, which are traditionally referred to as “Active” and “Passive”\is{active voice}\is{passive voice}. In these examples, the verb form in the Active voice is \textit{served} and all the others are in the Passive voice, employing the auxiliary \textit{was} plus the past participle: \textit{was served}. However, one could say that the little words \textit{on} and \textit{with} following the passive verb forms in \figref{fig:kagayanenchoir}d and e indicate the semantic role of the PSA -- \textit{on} indicates that the PSA is a location, and \textit{with} indicates that the PSA is an instrument. In other words, the verb forms in \ref{fig:kagayanenchoir}d and e are really \textit{was served on} and \textit{was served with} respectively. These little words no longer function as prepositions preceding a noun phrase, but rather as particles in the verb phrase that indicate the semantic role of the PSA.

Constructions analogous to those in \figref{fig:kagayanenchoir} all occur in Kagayanen, with several important differences. First, the PSA in Kagayanen is the \textit{absolutive} argument, as described in \sectref{bkm:Ref482605816}. Second, Kagayanen speakers use all the corresponding constructions in Kagayanen fluently, with a wide range of verbs and arguments. While certain verb+argument combinations are limited in the number and type of voices that may be expressed (as discussed at length in \chapref{chap:verbclasses-1} and \chapref{chap:verbclasses-2}), in general the possibilities are much more productive and regular than analogous constructions in English. Third, elements that indicate the semantic role of the absolutive argument are verb forms, differentiated by verbal affixation, rather than by auxiliaries (\textit{was} in the English examples), or particles (\textit{on} and \textit{with}). While all these structural features make voice in English and Kagayanen very different, the voice systems of both languages share one major functional similarity: they are both ways of bringing particular clausal arguments into the Privileged Syntactic Argument position, while “demoting” or “deperspectivizing” other elements. Once an argument is in the PSA position, it is then available for many syntactic functions that are not possible for obliques, or other clausal arguments.  

In Kagayanen, verb semantics, grammatical transitivity, and the applicative constructions (see \chapref{chap:verbstructure} through \chapref{chap:verbclasses-2}) work together to express the various “voices”.  In the following discussion, we will describe the voices semantically according to the semantic role of the absolutive argument. Applicative constructions are used to express Benefactive, Locative, Instrumental, and Adversative voice, depending on the semantics of the predicate.

\largerpage
We hope to show that the voice system of Kagayanen is the result of the interaction between the dimension of \isi{transitivity} in the inflectional paradigm (\chapref{chap:verbstructure})\is{inflection} and applicative\is{applicative morphology} processes in the stem-forming\is{stem-forming morphology} group (\chapref{chap:stemformingprocesses}). These relationships are summarized in \REF{ex:voices}, and exemplified in the rest of this section:

\ea
\label{ex:voices}
\begin{tabbing}
\hspace{6.1cm} \= \kill    
Intransitive inflection: \>	“Actor voice" \\
Transitive inflection, no applicative: \>	“Patient/Undergoer voice" \\
Transitive inflection + applicative -\textit{an}:	\> “Benefactive/Locative voice" \\
Transitive inflection + applicative \textit{i}-: 	\> “Conveyance/Instrumental voice"
\end{tabbing}
\z

\subsection{Intransitive constructions (Actor voice)}
\largerpage
\label{sec:Actorvoice}\label{sec:actorvoice}\is{actor voice|(}\is{intransitive constructions|(}
The simplest clause type in Kagayanen is an \textit{intransitive}\is{intransitive clause} clause. It consists minimally of a predicating element and a Referring Expression (Referring Phrase or Pronoun) in the absolutive role. It may also contain optional oblique elements, including non-absolutive phrases, locative phrases, and adverbial elements. Examples of intransitive clauses are found throughout this grammar. Here are a few others from the corpus. In these examples we have bracketed and labeled the Predicate and the absolutive argument.

\ea
\label{bkm:Ref116980651}
Isya  adlaw tallo  i  na  mag-arey gapanaw ...\\\smallskip
\hspace{2cm}absolutive\hspace{2.8cm}          Predicate \\
\gll Isya  adlaw [ tallo  i  na  mag-arey ] [ ga-panaw ]... \\
one  day {} three  \textsc{def.n}  \textsc{lk}  \textsc{rel}-friend {} {} \textsc{i.r}-walk/leave {} \\
\glt ‘One day the three friends left . . .’ [CBWN-C-10 2.2]
\z
\ea
\label{bkm:Ref116980654}
Ganti taan gaistar ti  na  baļay ...\\\smallskip
absolutive \hspace{1cm}   Predicate \\
\gll [ Ganti ] taan [ ga-istar ] ti  na  baļay ... \\
{} giant {} perhaps {} \textsc{i.r}-live {} \textsc{d1nabs}  \textsc{lk}  house \\
\glt ‘Perhaps a giant lives in this house . . .’ [CBWN-C-10 6.2]’
\z
\ea
\label{bkm:Ref116980701}
... gailing kay ta  Iloilo.\\\smallskip
\hspace{.3cm} Predicate \hspace{.3cm}  absolutive \\
\gll ... [ ga-iling ] [ kay ] ta  Iloilo. \\
  {} {} \textsc{i.r}-go {} {} 1\textsc{p.excl.abs} {} \textsc{nabs}  Iloilo \\
\glt ‘...we went to Iloilo.’ [VAWN-T-18 2.3]
\z
\ea
\label{bkm:Ref116980703}
Miling ka tengnged  ya  patyo bag-o  mamangdan.\\\smallskip
Predicate\hspace{.3cm} absolutive \\
\gll [ M-iling ] [ ka ] tengnged  ya  patyo bag-o  ma-ng-pangdan. \\
{} \textsc{i.v.ir}-go {} {} 2\textsc{s.abs}  {} next  \textsc{def.f}  graveyard before  \textsc{a.hap.ir-pl}-pandan \\
\glt ‘Go to (the place) next to the graveyard then get pandan.’[NCOP-C-02 1.2]
\z
\ea
\label{bkm:Ref116980658}
Nya dayad  ya  en.\\\smallskip
absolutive  Predicate \\
\gll [ Nya ]  [ dayad  ya  ] en. \\
{} \textsc{d4adj} {} {} good \textsc{def.f}  {} \textsc{cm} \\
\glt `That is good now.’ [NCOP-C-02 3.5]
\z

In an intransitive clause, there is only one Referring Expression that has a grammatical relation to the Predicate. Any additional elements are optional. Referring Expressions that have a grammatical relation to the Predicate are called \textit{arguments}\is{arguments}, or sometimes \textit{core arguments}\is{core arguments}. In each of these examples, the absolutive is the only argument of the clause. It is marked as absolutive by either the lack of a prenominal case marker (\ref{bkm:Ref116980651} and \ref{bkm:Ref116980654}), or the absolutive form of a pronoun (\ref{bkm:Ref116980701}, \ref{bkm:Ref116980703} and \ref{bkm:Ref116980658}). In all the verbal predicates (\ref{bkm:Ref116980651}-\ref{bkm:Ref116980703}), the verb is marked as intransitive, either realis (\ref{bkm:Ref116980651}, \ref{bkm:Ref116980654} and \ref{bkm:Ref116980701}), or irrealis (twice in \ref{bkm:Ref116980703}). Example \ref{bkm:Ref116980658} is a non-verbal predicate. Non-verbal predicates\is{non-verbal predicates} are always intransitive, and are not marked for transitivity or modality (see \chapref{chap:non-verbalclauses}).

In all of these constructions (\ref{bkm:Ref116980651}-\ref{bkm:Ref116980658}), the speaker is making a statement about the absolutive case argument. The absolutive argument is presented as the person or thing that is most relevant or most highly affected by the situation described in the Predicate. It is very common for the absolutive argument of an intransitive clause to be a semantic Actor (\ref{bkm:Ref116980651}-\ref{bkm:Ref116980703}), but this is not always true. Example \REF{bkm:Ref116980658} illustrates a clause in which the absolutive does not do anything. It is not an Actor because it does not initiate or control the event. Intransitive verbal predicates in happenstantial modalities\is{happenstantial modality} also express situations in which the absolutive argument is not an Actor:

\ea
...   kami  naļabo  en. \\\smallskip
\gll ...   kami  na-ļabo  en. \\
{} 1\textsc{p.excl.abs}  \textsc{a.hap.r}-capsize  \textsc{cm} \\
\glt ‘... we had capsized.’[VAWN-T-18 6.9]
\z

The absolutive argument of this clause, \textit{kami}, did not capsize on purpose. The situation is one that happened to the speaker and associates. There is no controlling Actor expressed at all. For this reason, the realis happenstantial mood\is{happenstantial modality} prefix is called for (see \chapref{chap:verbstructure}, \sectref{sec:modality}).

There is a difference between \isi{semantic transitivity}\is{transitivity!semantic} and \isi{grammatical transitivity}\is{transitivity!grammatical}. As mentioned in \sectref{bkm:Ref482605816}, a situation that seems to require an Undergoer and a distinct Controller may be expressed in an intransitive construction, with the Undergoer simply omitted or expressed in a non-absolutive case, preceded by \textit{ta} or \textit{ki}. We describe such constructions as \textit{detransitive}\is{detransitive constructions}. They express semantically transitive\is{semantic transitivity}\is{transitivity!semantic} situations in grammatically intransitive frames. Such constructions are used when there is an Undergoer, but it is less salient in the communicative situation than the Actor in some way. The Actor is presented as the most affected participant in the situation, and the Undergoer, if present at all, is presented as less central in some way, either by being indefinite, non-specific, generic or incompletely affected. Examples of detransitive constructions in the corpus are presented in \ref{bkm:Ref125017724v}-\ref{bkm:Ref117000205} above.
\is{intransitive constructions|)}
\is{actor voice|)}
\subsection{Transitive constructions (Patient/Undergoer voice)}
\is{patient voice|(}\is{undergoer voice|(}\is{transitive constructions|(}
In English and other Indo-European languages, almost every complete clause has a subject. Some clauses also have a direct object. Clauses which include a subject (a nominative argument) and no object are considered intransitive, while those that include a subject and a direct object (an accusative argument) are considered transitive (see, e.g., Merriam Webster online 2022, definitions of “transitive” and “intransitive”). These definitions don’t fit Philippine languages very well, since grammatical notions of “subject” and “direct object” do not apply (see, e.g., \citealt{cena1977, nolasco2018}, and the discussion above in \sectref{sec:grammaticalrelations}). Rather, in Kagayanen and most other Philippine languages, it is more accurate to say that almost all clauses include an absolutive argument, either overtly present or strongly implied. Some also include an ergative argument. Intransitive clauses are those that include an absolutive and no ergative argument, as illustrated in the previous section. Transitive clauses are those that include an absolutive and an ergative argument. \citet{halliday2004}, elaborated in \citet[478--484]{lapolla2011}, describes the Indo-european definition of transitivity as the “\isi{transitive model of transitivity}”, and the Kagayanen definition as the “\isi{ergative model of transitivity}". Examples of basic transitive constructions occur throughout this grammar. Here are a few additional examples from the corpus. Notice that all of these examples contain an ergative pronoun that refers to the Actor--the person or persons who control or originate the situation--, in addition to an absolutive argument, either overtly expressed or strongly implied:

\ea
Patimod  eman  danen. \\\smallskip
\gll Pa-timod  eman  danen. \\
\textsc{t.r}-pick.up  again.as.before  3\textsc{p.erg} \\
\glt ‘They picked (it) up again as before.’ [CBWN-C-10 3.4]
\z



\ea
Tapos  likid  pangdan  ya  a.  Tapos  tapa. Diristso  no  banggod  daw  likiren  isab  daw  bļaren. \\\smallskip
\gll Tapos  likid-...  pangdan  ya  a.  Tapos  tapa-... Diristso  no  banggod-...\footnotemark  daw  likid-...  isab  daw  baļad-en. \\
then  roll.up-\textsc{t.ir}  pandan  \textsc{def.f}  \textsc{ctr}  then  dry.on.fire-\textsc{t.ir} straight   2\textsc{s.erg}  straighten.out-\textsc{t.ir}  and  roll.up-\textsc{t.ir}  again  and  dry.in.sun-\textsc{t.ir} \\
\footnotetext{Recall from \chapref{chap:verbstructure}, \sectref{sec:omissionofaffixes} that inflectional affixes are often omitted. One common context for such omission is sequences of closely linked events such as this.}
\glt `Then roll up the pandan. Then dry (it) on a fire. Right away straighten (it) out and roll (it) up again and dry (it) in the sun.’ [NCOP-C-02 2.2-4]
\z
\ea
... nakita  kon  danen sikad  tama  na  kwarta  daw  bļawan. \\\smallskip
\gll ... na-kita  kon  danen sikad  tama  na  kwarta  daw  bļawan. \\
{}  \textsc{a.hap.r}-see  \textsc{hsy}  3\textsc{p.erg} very  many/much  \textsc{lk}  money  and  gold \\
\glt ‘... they saw, it is said, much money and gold.’ [CBWN-C-10 6.3]
\z

Constructions such as these are sometimes called “Patient voice” or “Undergoer voice” constructions because they present the Undergoer in the absolutive case. That is, the Undergoer (often patient), is presented as the participant that is the most affected by the situation being expressed, and is the \isi{Privileged Syntactic Argument}. Like intransitives, these transitive constructions can be understood as statements “about” the absolutive. \citet{cena1977} clearly showed that in \isi{Tagalog}, transitive constructions such as these are the most common, simplest, and most psychologically salient construction type for expressing situations involving an Undergoer and a separate Actor. This is also the case in Kagayanen. In addition, children acquiring \isi{Tagalog} as their first language learn to use Patient voice transitive constructions fluently earlier than the corresponding Actor voice constructions (\citealt{tucker1971, segalowitz1978}). Semantically, the absolutive refers to the referent that is most affected by the event, usually a patient. Pragmatically, the absolutive patient tends to be highly \isi{referential}, \isi{individuated}, and \isi{specific}. 

The following examples illustrate transitive patient voice clauses similar to the English clauses illustrated in \figref{fig:kagayanenchoir}. In these examples, we indicate the absolutive argument by a zero (\emptyset{}) in the pre-nominal position, though we do not use this convention in other parts of the grammar:


\ea
Absolutive = Patient: \\
Paakid  ko  \emptyset{}  \textbf{sidda}  \textbf{an}  naan  ta  pinggan  no. \\\smallskip
\gll Pa-akid  ko  \emptyset{}  \textbf{sidda}  \textbf{an}  naan  ta  pinggan  no. \\
\textsc{t.r}-serve  1\textsc{s.erg}  \textsc{abs}  fish  \textsc{def.m}  \textsc{spat.def}  \textsc{nabs}  dish  2\textsc{s.gen} \\
\glt ‘I served \textbf{the fish} on your plate.’
\z

\newpage
\ea
Lutuon  ko  \emptyset{}  \textbf{sidda}  \textbf{an}  para  ta  mga  bisita. \\\smallskip
\gll Luto-en  ko  \emptyset{}  \textbf{sidda}  \textbf{an}  para  ta  mga  bisita. \\
cook-\textsc{t.ir}  1\textsc{s.erg}  \textsc{abs}  fish  \textsc{def.m}  for  \textsc{nabs}  \textsc{pl}  visitor \\
\glt ‘I will cook \textbf{the fish} for the visitors.’
\z
\ea
Iluto  no  \emptyset{}  \textbf{sidda}  \textbf{an}. \\\smallskip
\gll I-luto  no  \emptyset{}  \textbf{sidda}  \textbf{an}. \\
\textsc{t.deon}-cook  2\textsc{s.erg}  \textsc{abs}  fish  \textsc{def.m} \\
\glt ‘You must cook \textbf{the fish}.’
\is{transitive constructions|)}\is{undergoer voice|)}\is{patient voice|)}
\z
\subsection{Applicative constructions (Benefactive, Adversative, and Locative voice)}
\label{sec:applicativevoice}
\is{benefactive voice|(}\is{adversative voice|(}\is{locative voice|(}\is{applicative constructions|(}
Applicatives are often thought of as "transitivizing" constructions (see \citealt{polinsky2013, payne1997}  \textit{inter alia}). This is because applicative morphology on a verb expressing an intransitive situation creates a new verb stem that expresses a transitive situation in which activity is directed to an endpoint, usually a location or a beneficiary. The endpoint introduced by the applicative construction becomes a core argument of the verb--a direct object in a \isi{nominative/accusative system} or an absolutive in an \isi{ergative/absolutive system}. However, if the root that accepts applicative morphology already involves activity transferred to an endpoint (i.e., it is “semantically transitive\is{semantic transitivity}\is{transitivity!semantic}"), the applicative construction \textit{redirects} the activity to a different endpoint, again usually a location or beneficiary. In such situations, the applicative does not increase transitivity at all--it simply redirects the activity described by the verb to a different endpoint.

This characterization contrasts applicative constructions with causative constructions in that the applicative introduces an endpoint (an “object” in terms used by \citealt{polinsky2013}), whereas a causative adds a starting point--an initiator, or Actor to a scene (see \sectref{bkm:Ref500569782} on causatives). In Kagayanen, the verbal suffix -\textit{an} (and sometimes the prefix \textit{i}-) normally introduces an endpoint to a scene. That endpoint may be a semantic beneficiary/recipient, maleficiary\is{maleficiary}, locative or instrumental participant, depending on the semantics of the verb, its arguments, and the context. If the base clause is intransitive, the applicative is grammatically transitive. If the base clause is transitive, the clause remains transitive and the applicative in effect “replaces" the endpoint of the base clause.

\hspace*{-.2pt}Applicative morphology is central to the voice systems of Philippine languages. As mentioned in the introduction to \sectref{sec:choir}, transitivity marking in the inflectional paradigm, plus stem-forming applicative morphology is what gives rise to benefactive, adversative, locative, and instrumental voice in Kagayanen. In each of these construction types, the Actor, if present, appears in the ergative case, the Undergoer/patient, if present, appears in the non-absolutive (demoted Undergoer) case, and a “new"  beneficiary, maleficiary, location or instrument appears in the absolutive case.

Benefactive voice often functions to express situations in which something is transferred to a beneficiary, with the beneficiary presented in the absolutive case (examples \ref{bkm:Ref500489993} and \ref{bkm:Ref500489123}). However, transfer is not a necessary component of the meaning of this construction (examples \ref{ex:thecabinet} through \ref{ex:thetikling}).

\ea
\label{bkm:Ref500489993}
Absolutive = Beneficiary/recipient \\
Paakiran  ko  \emptyset{}   \textbf{kanen}  \textbf{an}  ta  sidda. \\\smallskip
\gll Pa-akid-an  ko  \emptyset{}   \textbf{kanen}  \textbf{an}  ta  sidda. \\
\textsc{t.r}-serve-\textsc{apl}  1\textsc{s.erg}  \textsc{abs}  3\textsc{s.abs}  \textsc{def.m}  \textsc{nabs}  fish \\
\glt ‘I served \textbf{her/him} some fish.’
\z

\ea
\label{bkm:Ref500489123}
Lutuan  ko  \emptyset{}  \textbf{kanen}  \textbf{an}  ta  sidda. \\\smallskip
\gll \emptyset{}-Luto-an  ko  \emptyset{}  \textbf{kanen}  \textbf{an}  ta  sidda. \\
\textsc{t.ir}-cook-\textsc{apl}  1\textsc{s.erg}  \textsc{abs}  3\textsc{s.abs}  \textsc{def.m}  \textsc{nabs}  fish \\
\glt ‘I will cook \textbf{him/her} some fish.’
\z

Without the applicative marker on the verb, example \REF{bkm:Ref500489123} could only be understood as patient voice, which would be anomalous with these arguments: “I will cook him/her (with/on/by) fish.”

The absolutive of a Benefactive applicative need not be the recipient of a transferred item. In example \REF{ex:thecabinet}, the absolutive is the pronominal enclitic \textit{a}, `1\textsc{s.abs}'. This is understood as the beneficiary of the cleaning event, but not necessarily the recipient of the cleaned item (the cabinet). Similar observations can be made for examples \REF{ex:nationalanthem} and \REF{ex:thetikling}:

\ea
\label{ex:thecabinet}
Palimpyuan a din ta aparador. \\\smallskip
\gll Pa-limpyo-an a din ta aparador. \\
\textsc{t.r}-clean-\textsc{apl} 1\textsc{s.abs} 3\textsc{s.erg} \textsc{nabs} cabinet \\
\glt `S/he cleaned the cabinet for me.' (Meaning I will not have to clean it.)
\z

\ea
\label{ex:nationalanthem}
Pakantaan kay din ta Bayang Magiliw. \\\smallskip
\gll Pa-kanta-an kay din ta Bayang Magiliw. \\
\textsc{t.r}-sing-\textsc{apl} 1\textsc{p.excl.abs} 3\textsc{s.erg} \textsc{nabs} country beloved \\
\glt `S/he sang for us Beloved Country (the National Anthem).'
\z

\ea
\label{ex:thetikling}
Pasayawan a danen ta tinikling. \\\smallskip
\gll Pa-sayaw-an a danen ta tiinkling. \\
\textsc{t.r}-dance-\textsc{apl} 1\textsc{s.abs} 3\textsc{p.erg} \textsc{nabs} \\
\glt `They danced for me the tinikling (dance).'
\z

As discussed in \chapref{chap:verbstructure}, \sectref{sec:modality} the Deontic prefix \textit{i-} may function as an applicative marker, as in \REF{bkm:Ref500489008}. As such, \textit{i}- may function as a deontic Benefactive marker:

\ea
  \label{bkm:Ref500489008}
Absolutive = Beneficiary: \\
Iluto \textbf{a}  no  ta  sidda. \\\smallskip
\gll I-luto \textbf{a}  no  ta  sidda. \\
\textsc{t.deon}-cook 1\textsc{s.abs}  2\textsc{s.erg}  \textsc{nabs}  fish \\
\glt ‘You must cook \textbf{me} some fish.’
\z

Technically, \ref{bkm:Ref500489008} is ambiguous between Benefactive and Patient voice. However, pragmatically it can only be understood as Benefactive voice, since people don’t cook other people. We are reminded of English sentences like “I will make you a sandwich” which are similarly “ambiguous”. The usage illustrated in \REF{bkm:Ref500489008} has also been described as “conveyance voice” in \isi{Cebuano} based on the fact that it presents the Undergoer (\textit{sidda} ‘fish’) as moving away from the Actor \citep[69]{wolff1973}.

\largerpage
Examples \REF{bkm:Ref500490084} through \REF{bkm:Ref500742432} illustrate Locative voice. Again, these are only understood as locative because of the context. In \REF{bkm:Ref500490084}, for example, the plate cannot logically be a beneficiary, but rather is a logical location for serving something:

\ea
\label{bkm:Ref500490084}
Absolutive = Location: \\
Paakiran  ko  \textbf{pinggan  no  an}  ta  sidda. \\\smallskip
\gll Pa-akid-an  ko \emptyset{} \textbf{pinggan}  \textbf{no}  \textbf{an}  ta  sidda. \\
\textsc{t.r}-serve-\textsc{apl}  1\textsc{s.erg}  \textsc{abs}  plate  2\textsc{s.gen}  \textsc{def.m}  \textsc{nabs}  fish \\
\glt ‘I served on \textbf{your plate} some fish.’
\z


\ea
Pakan-anan  din  \textbf{pinggan}  \textbf{ko}  \textbf{an} ta sinugba    ya  na  sidda. \\\smallskip
\gll Pa-kan-en-an\footnotemark{}  din  \emptyset{}  \textbf{pinggan}  \textbf{ko}  \textbf{an} ta s<in>ugba    ya  na  sidda. \\
\textsc{t.r}-cooked.rice-\textsc{apl}  3\textsc{s.erg}  \textsc{abs}  dish  1\textsc{s.gen}  \textsc{def.m}
\textsc{nabs} <\textsc{nr.res}>roast \textsc{def.f}  \textsc{lk}  fish \\
 \footnotetext{As mentioned earlier, the verb meaning `to eat' consists of suppletive stems; \textit{kaan} is the bare form, but when any suffixes are involved, the root changes to \textit{kan-en}, the nominal root meaning `cooked rice.' When the stem-forming suffix contains an \textit{a} vowel, such as the applicative -\textit{an}, vowel harmony applies, resulting in the stem form \textit{kan-anan}.}
\glt ‘S/he ate roasted fish in \textbf{my dish}.’
\z

We are calling example \REF{bkm:Ref500490264} Locative voice because the younger sibling is presented as the target of laughing. This sentence cannot be understood as Benefactive voice (“He laughed for my younger sibling”), even though the absolutive argument is animate and human.

\ea
\label{bkm:Ref500490264}
Patawaan  din    \textbf{mangngod}  \textbf{ko}  an. \\\smallskip
\gll Pa-tawa-an  din    \emptyset{}  \textbf{mangngod}  \textbf{ko}  an. \\
\textsc{t.r}-laugh-\textsc{apl}  3\textsc{s.erg}    \textsc{abs}  younger.sibling  1\textsc{s.gen}  \textsc{def.m} \\
\glt ‘S/he laughed at my younger sibling.’
\z

The following example illustrates a Locative voice construction in Irrealis modality:

\ea
\label{bkm:Ref500742432}
Lutuan  ko  \textbf{bakod}  \textbf{an}  \textbf{na}  \textbf{kandwagan}  ta  sidda   para  ta  bisita. \\\smallskip
\gll \emptyset{}-Luto-an  ko  \emptyset{}  \textbf{bakod}  \textbf{an}  \textbf{na}  \textbf{kandwagan}  ta  sidda   para  ta  bisita. \\
\textsc{t.ir}-cook-\textsc{apl}  1\textsc{s.erg}  \textsc{abs} big  \textsc{def.m}  \textsc{lk}  skillet  \textsc{nabs}  fish for  \textsc{nabs}  visitor \\
\glt ‘I will cook in \textbf{the big skillet} some fish for the visitor.’
\z

The following are some additional applicative (Locative or Benefactive/Recipient voice) constructions from the corpus:

\ea
Absolutive = Location: \\
\textbf{Papilitan}  din  kon  gantangan  ya  ta  saļeng… \\\smallskip
\gll \textbf{Pa-pilit-an}  din  kon  gantangan  ya  ta  saļeng… \\
\textsc{t.r}-stick.on-\textsc{apl}  3\textsc{s.erg}  \textsc{hsy}  3.quart.measurer  \textsc{def.f}  \textsc{nabs}   sap \\
\glt ‘He \textbf{stuck} the sap on the 3 quart measurer.’ [CBWN-C-22 6.5]
\z

\ea
Absolutive = Location: \\
yon  nang  \textbf{pakamangan}  nay  ta  ame  na pangabuian  pang-adlaw-adlaw \\\smallskip
\gll yon  nang  \textbf{pa-kamang-an}  nay  ta  ame  na pangabui-an  pang-adlaw-adlaw \\
\textsc{d}3\textsc{abs}  only/just  \textsc{t.r}-get-\textsc{apl}  1\textsc{p.excl.erg}  \textsc{nabs}  1\textsc{p.excl.gen}  \textsc{lk} living-\textsc{nr}  \textsc{inst}-\textsc{red}-sun/day \\
\glt `… that only is from where we get our livelihood for daily use.’ (This refers to the salary of the author of this text.) [VPWL-T-03 3.2]
\z

\ea
Absolutive = Location (Goal): \\
\textbf{Pabalikan}  din  pa  kwarta  ya  na  bilin  ta  baļay  ya danen. \\\smallskip
\gll … \textbf{pa-balik-an}  din  pa  kwarta  ya  na  bilin  ta  baļay  ya danen. \\
{} \textsc{t.r}-return-\textsc{apl}  3\textsc{s.erg}  \textsc{emph}  money  \textsc{def.f}  \textsc{lk}  leave.behind  \textsc{nabs}  house  \textsc{def.f} 3\textsc{p.gen} \\
\glt `…he even \textbf{returned} \textbf{for} the money that he left in their house.’ [BCWN-C-04 6.5]
\z

\ea
Absolutive = Beneficiary/recipient: \\
Manang  \textbf{padaļaan}  a  nyo  man  ta  nyo  na  kudak… \\\smallskip
\gll Manang  \emptyset{}-\textbf{pa-daļa-an}  a  nyo  man  ta  nyo  na  kudak… \\
Older.sister  \textsc{t.ir-}\textsc{caus}-carry/take-\textsc{apl}  1\textsc{s.abs}  2\textsc{p.erg}  also  \textsc{nabs}  2\textsc{p.gen}  \textsc{lk}  picture \\
\glt ‘Older sister, \textbf{send} me your picture…’ [PBWL-T-10 10.2]
\z

\ea
Absolutive = Beneficiary/recipient: \\
Taga-Mampyo  daw  Nusa  \textbf{pasuļatan}  na  mubos  uli  ta  Sintro tak  \textbf{indyiksyunan}. \\\smallskip
\gll Taga-Mampyo  daw  Nusa  \textbf{pa-suļat-an}  na  m-ubos  uli  ta Sintro tak  \emptyset{}-\textbf{indyiksyun-an}. \\
\textsc{orig}-Mampyo  and  Nusa  \textsc{t.r}-write-\textsc{apl}  \textsc{lk}  \textsc{i.v.ir}-all  go.home  \textsc{nabs}  Central because  \textsc{t.ir}-inject-\textsc{apl} \\
\glt `The ones from Mampyo and Nusa \textbf{were} \textbf{written} \textbf{to} all to come to Central because (they) \textbf{will} \textbf{be} \textbf{injected}.’ (This is about an outbreak of cholera.) [JCWN-T-21 10.1]
\z

Example \REF{bkm:Ref118204268} illustrates an applicative construction inside a relative clause. The head is \textit{basak} ‘land’, which is the location of planting.

\ea
\label{bkm:Ref118204268}
Absolutive = Location: \\
Kami  may  basak  \textbf{nataneman}  nay  ta  niog. \\\smallskip
\gll Kami  may  basak  \textbf{na-tanem-an}  nay  ta  niog. \\
1\textsc{p.excl.abs}  \textsc{ext.in}  land  \textsc{a.hap.r}-plant-\textsc{apl}  1\textsc{p.excl.erg}  \textsc{nabs}  coconut \\
\glt ‘As for us (we) have land on which we have planted coconut trees.’ [JCOE-T-06 5.5]
\z

In example \REF{bkm:Ref118204368} the composite pronoun \textit{ta kaw} expresses a first person singular ergative plus second person singular absolutive argument (see \chapref{chap:referringexpressions}, \sectref{sec:pronouns} on the composite pronouns):

\ea
\label{bkm:Ref118204368}
Erg.+Abs. = Agent+Addressee: \\
Nay,  kisyem   nang  en  \textbf{sugiran}  ta kaw. \\\smallskip
\gll Nay,  kisyem   nang  en  \emptyset{}-\textbf{sugid-an}  ta kaw.\footnotemark{} \\
mom  tomorrow  only/just  \textsc{cm}  \textsc{t.ir}-tell-\textsc{apl}  1\textsc{s.erg} 2\textsc{s.abs} \\
\footnotetext{The composite pronouns are usually spelled as two words in the Kagayanen orthography, even though they are pronounced as one word.}
\glt ‘Mom, just tomorrow I will tell you.’ (The content of what she will tell is implied.) [BMON-C-06 3.19]
\z

Example \REF{bkm:Ref118204569} illustrates three applicative constructions. The first applicative involves the root \textit{ibit} ‘hold’, which is a member of the class of roots that always appear in an applicative form when transitive (Class VIII). All such roots involve a patient that does not undergo a change in state, but is only slightly, superficially or invisibly affected by the action of the verb. In this case, the act of holding a child does not change or affect the child in any substantive way, and thus the child shares the semantic feature of “superficial affectedness” with locations. \chapref{chap:verbclasses-1}, \sectref{sec:volitionaltransitiveroots} contains a long list of such roots. In the second applicative in \REF{bkm:Ref118204569}, the location of the placing of water is the head of the child, which appears in the absolutive. Finally, in the third applicative, the absolutive argument of the stem \textit{garayan} ‘to chant for’ is ellipted because the child is the main topic in this stretch of text, and in the context it is clear that the midwife chants for the child.

\ea
\label{bkm:Ref118204569}
Nanligan  i  \textbf{pabitan}  din  bata  daw  \textbf{batangan} ta waig  a  ulo  ta  bata.  Tapos  \textbf{pagarayan}  din.\\\smallskip
\gll Nanligan  i  \textbf{pa-ibit-an}  din  bata  daw  ...-\textbf{batang-an} ta waig  a  ulo  ta  bata.  Tapos  \textbf{pa-garay-an}  din.\\
midwife  \textsc{def.n}  \textsc{t.r}-hold-\textsc{apl}  3\textsc{s.erg}  child  and  \textsc{t.r}-put-\textsc{apl} \textsc{nabs}  water  \textsc{inj}  head  \textsc{nabs}  child  then  \textsc{t.r}-chant-\textsc{apl}  3\textsc{s.erg} \\
\glt `As for the midwife, she holds the child and puts some water on the head of the child. Then she chants (for the child).’ (This is a naming ceremony for newborn babies similar to baptism.) [VAOE-J-06 1.4]
\z

absolutives in applicative constructions may be “Malefactees", as well as Benefactees. In happenstantial modality without a conscious controller, applicatives almost always imply a “malefactive” or “adversative” sense. Such constructions present an adversely affected participant as the absolutive. For example, \REF{ex:thepigdied} illustrates a basic intransitive clause, while \REF{ex:thepigdiedonus} adds an adversely affected participant to the scene:

\ea
   \ea 
   \label{ex:thepigdied}
   Absolutive = Patient: \\
    Napatay  \emptyset{}  baboy  an. \\\smallskip
\gll Na-patay  \emptyset{}  baboy  an. \\
    \textsc{a.hap.r}-die  \textsc{abs}  pig  \textsc{def.m} \\
    \glt ‘The pig died.’
    \ex \label{ex:thepigdiedonus}Absolutive = Maleficiary: \\
    Napatayan  kami  i  ta  baboy. \\\smallskip
\gll Na-patay-an  kami  i  ta  baboy. \\
    \textsc{a.hap.r}-die-\textsc{apl}  1\textsc{p.excl.abs}  \textsc{def.n}  \textsc{nabs}  pig \\
    \glt ‘The pig died on us (adversely affecting us).’
    \z
\z

The English free translation expresses the approximate sense of \REF{ex:thepigdiedonus}. Recall that \textit{na}{}- is an “ambitransitive” prefix, meaning that it may occur in a transitive or intransitive construction. Example \REF{ex:thepigdied} is obviously intransitive because there is only one participant, and that participant, the pig, is in the absolutive case. Example \REF{ex:thepigdiedonus}, on the other hand, is somewhat transitive in that \textit{kami} ‘us’ is presented as the most saliently affected participant, and the pig, though clearly not an Actor, is the source of the effect. A more literal but non-idiomatic English translation might be “The pig died us” (it died, adversely affecting us).

The following is an example of Adversative voice in irrealis, happenstantial modality:

\ea
Absolutive = Maleficiary: \\
Mabadbaran  kami  ta  igot  ta  katig. \\\smallskip
\gll Ma-badbad-an  kami  ta  igot  ta  katig. \\
\textsc{a.hap.ir}-untie-\textsc{apl}  1\textsc{p.excl.abs}  \textsc{nabs}  tie  \textsc{nabs}  outrigger \\
\glt ‘The tie of an outrigger will become untied on us (adversely affecting us).’
\z

Example \REF{bkm:Ref118208801} illustrates a happenstantial irrealis modality Adversative applicative construction from the corpus. We know the prefix on the bolded verb has been dropped because \textit{adlaw} `sun/day' used as a verb meaning `become daylight' must take happenstantial modality. The prefix \textit{ma}- may be added, with no change in meaning, but the dynamic intransitive irrealis prefixes \textit{mag}- or \textit{m}- are impossible:

\ea
\label{bkm:Ref118208801}
Maria,  pilak  no  duma  an  na  suwa  tak  kani \textbf{adlawan  ki  en}. \\\smallskip
\gll Maria,  pilak  no  duma  an  na  suwa  tak  kani ...\textbf{-adlaw-an}  \textbf{ki  en}. \\
Maria  throw.away  2\textsc{s.erg}  some  \textsc{def.m}  \textsc{lk}  citrus.fruit  because  later \textsc{a.hap.ir}-sun/day-\textsc{apl}  1\textsc{p.incl.abs}  \textsc{cm} \\
\glt `Maria, throw away some of the citrus fruit because later it will become daytime on us.’ (This is a story about a mother, who is a witch, and her daughter. They were invisible flying home carrying citrus fruit. So the mother tells the daughter to throw away some of the fruit so they will not be seen.) [MBON-T-06 5.8]
\z

In constructions with an Actor in dynamic modality the semantic role of the absolutive argument is sometimes technically ambiguous between beneficiary or maleficiary, though context usually disambiguates:

\ea
Pabadbaran  kay  din  ta  igot  ta  katig. \\\smallskip
\gll Pa-badbad-an  kay  din  ta  igot  ta  katig. \\
\textsc{i.r}-untie-\textsc{apl}  1\textsc{p.excl.abs}  3\textsc{s.erg}  \textsc{nabs}  tie  \textsc{nabs.gen}  outrigger \\
\glt ‘S/he untied the tie of an outrigger on us.’ Adversative \\
 Or ‘S/he untied the tie of an outrigger for us.’ Benefactive
\z

\ea
Pagubbaan  kay  din  ta  pungkuan. \\\smallskip
\gll Pa-gubba-an  kay  din  ta  pungko-an. \\
\textsc{t.r}-ruin-\textsc{apl}  1\textsc{p.excl.abs}  3\textsc{s.erg}  \textsc{nabs}  sit-\textsc{apl} \\
\glt ‘S/he ruined the chair on us.’ Adversative \\
Or: ‘S/he ruined the chair for us.’ (They wanted it ruined to use the wood for something else like firewood.) Benefactive
\z
\ea
Pataguan  kay  din  ta  pagkaan. \\\smallskip
\gll Pa-tago-an  kay  din  ta  pagkaan. \\
\textsc{t.r}-hide-\textsc{apl}  1\textsc{p.excl.abs}  3\textsc{s.erg}  \textsc{nabs}  food \\
\glt ‘S/he hid the food from us.’ (So we couldn’t eat it.) Adversative
Or: ‘S/he hid the food for us.’ (So we could eat it and nobody else could eat it first.) Benefactive
\z
\ea
Padalukan  kay  din  ta  pagkaan. \\\smallskip
\gll Pa-dalok-an  kay  din  ta  pagkaan. \\
\textsc{t.r}-greedy-\textsc{apl}  1\textsc{p.excl.abs}  3\textsc{s.erg}  \textsc{nabs} food\textsc{} \\
\glt ‘S/he was greedy towards us with food.’
\z

The following are some Benefactive and Adversative applicatives from the corpus:

\ea
… daw  \textbf{pamisaan}  ta  pari  patay  an  daw  \textbf{pabinditaan} din  man  patay  an. \\\smallskip
\gll … daw  \textbf{pa-misa-an}  ta  pari  patay  an  daw  \textbf{pa-bindita-an} din  man  patay  an. \\
{} and  \textsc{t.r}-mass-\textsc{apl}  \textsc{nabs}  priest  die  \textsc{def.m}  and \textsc{t.r}-bless-\textsc{apl} 3\textsc{s.erg}  also  die  \textsc{def.m} \\
\glt `…and the priest \textbf{conducts} \textbf{a} \textbf{mass} \textbf{for} the dead one and he also \textbf{blesses} the dead one.’ [CBWE-C-06 6.2]
\z

\ea
… daw  maimo  kon  palitan  nay  daen  iran  na mga  order. \\\smallskip
\gll … daw  ma-imo  kon  \emptyset{}-palit-an  nay  daen  iran  na mga  order. \\
{}   if/when  \textsc{a.hap.ir}-do/make  \textsc{hsy}  \textsc{t.ir}-buy-\textsc{apl}  1\textsc{p.excl.erg}  3\textsc{p.abs}  3\textsc{p.gen}  \textsc{lk} \textsc{pl}  order \\
\glt ‘... if it is possible they said we will \textbf{buy} \textbf{for} them what they ordered.’ [AFWL-L-01 8.3]
\z

\ea
Daw    gaduļot  kaw  ta  ittaw  na  dili   makita yaken  pa  na  dili  nyo  \textbf{duļutan}  na  bui  a  i? \\\smallskip
\gll Daw    ga-duļot  kaw  ta  ittaw  na  dili   ma-kita yaken  pa  na  dili  nyo  0-\textbf{duļot-an}  na  bui  a  i? \\
if/when  \textsc{i.r}-food.offering  2\textsc{p.abs}  \textsc{nabs}  person  \textsc{lk}  \textsc{neg.ir}  \textsc{a.hap.ir}-see
1\textsc{s.abs}  \textsc{emph}  \textsc{lk}  \textsc{neg.ir}  2\textsc{p.erg}  \textsc{t.ir}-food.offering-\textsc{apl}  \textsc{lk}  live  1\textsc{s.abs}  \textsc{att} \\
\glt `If you offer food for unseen people (spirits), even more for me \textbf{won’t} \textbf{you} \textbf{offer} \textbf{food} when I am living?’ [JCWN-T-20 10.5]
\z
\ea
 … kami  \textbf{napatayan}  ta  makina… \\\smallskip
\gll … kami  \textbf{na-patay-an}  ta  makina… \\
  {} 1\textsc{p.excl.abs}  \textsc{a.hap.r}-die-\textsc{apl}  \textsc{nabs}  engine \\
\glt ‘…the engine (of the boat we were riding) \textbf{died} on us….’ [MMWN-L-01 5.8]
\z

There is also a “partitive” usage of the applicative construction, which may be considered a variation on patient voice. Compare the following two examples:

\largerpage
\ea 
    \ea \label{bkm:Ref500573274}
    Absolutive = Patient: \\
    Pautod  din kaoy  an. \\\smallskip
\gll Pa-utod  din  \emptyset{}  kaoy  an. \\
    \textsc{t.r}-cut  3\textsc{s.erg}  \textsc{abs}  tree/wood  \textsc{def.m} \\
    \glt ‘S/he cut the wood.’
    \ex \label{bkm:Ref500573276}
    Absolutive = Partitive patient: \\
    \textbf{Pauturan}  din  \emptyset{}  kaoy  an. \\\smallskip
\gll \textbf{Pa-utod-an}  din  \emptyset{}  kaoy  an. \\
    \textsc{t.r}-cut-\textsc{apl}  3\textsc{s.erg}  \textsc{abs}  tree/wood  \textsc{def.m} \\
    \glt ‘S/he cut part of the wood.’
    \z
\z

Examples \REF{bkm:Ref500573274} and \REF{bkm:Ref500573276} have the same argument structure--an Actor in the ergative case and a patient in the absolutive, therefore they are both patient voice. The difference is that \REF{bkm:Ref500573274} is a simple transitive construction, while  \REF{bkm:Ref500573276} is an applicative. The meaning of \REF{bkm:Ref500573276} is that the Actor did not cut the wood completely in two parts, but only cut off a small portion. The following examples, repeated here from \chapref{chap:verbstructure}, also illustrate the partitive use of the applicative. Example \REF{bkm:Ref395171869} illustrates the verb \textit{panno} ‘fill’ in a non-applicative construction. In this example, the shirt is the absolutive, while the Actor, ‘he’ is ergative:

\ea
\label{bkm:Ref395171869}
\textbf{Papanno}  din  bayo  din  ya  ta  bļawan… \\\smallskip
\gll \textbf{Pa-panno}  din  bayo  din  ya  ta  bļawan… \\
\textsc{t.r}-fill  3\textsc{s.erg}  shirt  3\textsc{s.gen}  \textsc{def.f}  \textsc{nabs}  gold \\
\glt ‘He filled up his shirt with gold...’ [CBWN-C-22 4.23]
\z

Example \REF{bkm:Ref395171874v} illustrates the same verb in an applicative construction. Note, however, that the argument structure is the same. The shirt is still absolutive, and the Actor is still ergative:

\ea
\label{bkm:Ref395171874v}
\textbf{Papannuan}  din  bayo  din  ya  ta  bļawan. \\\smallskip
\gll \textbf{Pa-panno-an}  din  bayo  din  ya  ta  bļawan. \\
\textsc{t.r}-fill-\textsc{apl}  3\textsc{s.erg}  shirt  3\textsc{s.gen}  \textsc{def.f}  \textsc{nabs}  gold \\
\glt ‘He filled part way his shirt with gold.’
\z

The applicative has this effect for many Class VI verbs that are inherently transitive. Consider the following:

\ea
    \ea 
    Patient voice (absolutive = patient: roasted fish): \\
    \textbf{Pakaan}  din  sinugba  ya  na  sidda  naan  ta  pinggan  ko. \\\smallskip
\gll \textbf{Pa-kaan}  din  s<in>ugba  ya  na  sidda  naan  ta  pinggan  ko. \\
    \textsc{t.r}-eat  3\textsc{s.erg}  <\textsc{nr.res}>roast  \textsc{def.f}  \textsc{lk} fish  \textsc{spat.def}  \textsc{nabs} dish  1\textsc{s.gen} \\
    \glt ‘S/he ate the roasted fish in my dish.’
    \ex
    Partitive (absolutive = partitive patient: some of the roasted fish): \\
    \textbf{Pakan-anan}  din  sinugba  ya  na  sidda naan ta pinggan  ko.\\\smallskip
\gll \textbf{Pa-kan-en-an}  din  s<in>ugba  ya  na  sidda naan ta pinggan  ko.\\
    \textsc{t.r}-cooked.rice-\textsc{apl}  3\textsc{s.erg}  <\textsc{nr.res}>roast  \textsc{def.f}  \textsc{lk} fish
    \textsc{spat.def} \textsc{nabs}  dish  1\textsc{s.gen} \\
    \glt `S/he ate some of the roasted fish in my dish.’
    \z
\z

As mentioned in \chapref{chap:verbstructure}, one may understand this usage as indicating that the Undergoer is not a prototypical patient. In this case, it is understood that the patient is only partially affected. In other constructions, the applicative indicates that the absolutive is a locative, or beneficiary/maleficiary participant. These are also non-prototypical Patients, in that they are less than completely affected by the activity expressed by the verb.

This observation is corroborated by verbs that always take an applicative suffix in their basic transitive form (Class VIII). In each case, they describe situations in which the patient argument is not prototypical in that it does not undergo a change in state as a result of the event described by the verb. These verbs include \textit{amblig} ‘to care for’, \textit{arek} ‘to kiss’, \textit{tabang} ‘to help’, and many others listed in \chapref{chap:verbclasses-1}, \sectref{sec:volitionaltransitiveroots}.
\is{applicative constructions|)}\is{locative voice|)}\is{adversative voice|)}\is{benefactive voice|)}
\subsection{Instrumental voice}
\label{sec:instrumentalvoice}\is{instrumental voice|(}\is{conveyance voice|(}

Instrumental voice constructions are characterized by a grammatically transitive verb form, with the Actor in the ergative case, and an instrumental element in the absolutive. In realis modality, Instrumental and Patient voice constructions are structurally identical; the only way to distinguish the two is by semantics--if the absolutive argument is a semantic instrument, it is Instrumental voice, whereas if the absolutive argument is a patient, it is Patient voice:


\ea
\label{bkm:Ref499974867}
Absolutive = Instrument: \\
Paakid  ko  luag  i  na  bag-o  ta  sidda. \\\smallskip
\gll Pa-akid  ko  \emptyset{}  luag  i  na  bag-o  ta  sidda. \\
\textsc{t.r}-serve  1\textsc{s.erg}  \textsc{abs}  big.spoon  \textsc{def.n}  \textsc{lk}  new \textsc{nabs}  fish \\
\glt ‘I used the new serving spoon to serve some fish.’
\z

In irrealis modality, however, instrumental voice is distinguished by the presence of the transitive deontic prefix \textit{i}-. In such situations, the -\textit{i} prefix has an instrumental applicative function, as in \REF{ex:servesomefish} and \REF{bkm:Ref500142287}:

\ea
\label{ex:servesomefish}
Iakid	ko	luag	i	na	bag-o	ta	sidda. \\\smallskip
\gll I-akid	ko	\emptyset{}	luag	i	na	bag-o	ta	sidda. \\
\textsc{t.deon}-serve	1\textsc{s.erg}	\textsc{abs}	big.spoon	\textsc{def.n}	\textsc{lk}	new	\textsc{nabs}	fish \\
\glt ‘I must use the new serving spoon to serve some fish.’
\z

Without the \textit{i}- prefix in example \REF{ex:servesomefish}, the sentence becomes anomalous:

\ea
*Akiren	ko luag	i	na	bag-o	ta	sidda. \\\smallskip
\gll *Akid-en	ko	\emptyset{}	luag	i	na	bag-o	ta	sidda. \\
serve-\textsc{t.ir}	1\textsc{s.erg}	\textsc{abs}	big.spoon	\textsc{def.n}	\textsc{lk}	new	\textsc{nabs}	fish \\
\glt (‘I will serve the new serving spoon to/for fish.’)
\z

\ea
\label{bkm:Ref500142287}
Iprito  ko  bag-o  i  na  mantika  ta  sidda. \\\smallskip
\gll I-prito  ko  \emptyset{}  bag-o  i  na  mantika  ta  sidda. \\
\textsc{t.deon}-fry  1\textsc{s.erg}  \textsc{abs}  new  \textsc{def.n}  \textsc{lk}  oil \textsc{nabs}  fish \\
\glt ‘I must use the new oil to fry some fish.’
\z

Again, in transitive irrealis modality without the \textit{i-} prefix, example \REF{bkm:Ref500142287} becomes anomalous (??‘I will fry the new oil for/to/with fish.’). Because the absolutive nominal can be construed as an instrument in each instance, the only possibility is Instrumental voice. 

In several other Philippine languages, such as Waray \citep{payneoyzon2022}, reflexes of the same verbal prefix \textit{i}- are more strongly associated with Instrumental voice. However, in Kagayanen \textit{i-} is essentially a transitive deontic modality marker that occasionally has an irrealis instrumental applicative usage. 

The following are some additional examples of instrumental voice in realis and irrealis modality from the corpus:

\ea
\label{ex:inthehouse}
Daw   may kwarta, ipalit ta sabon, puspuro, agas, daw tanan na mga gamit ta baļay. \\\smallskip
\gll Daw   may kwarta, i-palit ta sabon, puspuro, agas, daw tanan na mga gamit ta baļay. \\
if/when	\textsc{ext.in}	money	\textsc{t.deon}-buy	\textsc{nabs}	soap	matches	kerosene and	all	\textsc{lk}	\textsc{pl}	use	\textsc{nabs}	house \\
\glt `If there is money, (it) should be used to buy soap, matches, kerosene, and all that is used in the house.’ [NWE-L-01 2.9]
\z

\ea
Gapasalamat	a	ta	ate	na	Dios	tak	uļa a	natabo	ta	laod	parti	ta	ake	na	pagpangita ta	isagod	ko	ta	ake	na	pamilya. \\\smallskip
\gll Ga-pa-salamat	a	ta	ate	na	Dios	tak	uļa a	na-tabo	ta	laod	parti	ta	ake	na	pag-pangita ta	i-sagod	ko	ta	ake	na	pamilya. \\
\textsc{i.r}-\textsc{caus}-thank	1\textsc{s.abs}	\textsc{nabs}	1\textsc{p.incl.gen}	\textsc{lk}	God	because	\textsc{neg.r} \textsc{inj}	\textsc{a.hap.r}-happen	\textsc{nabs}	deep.sea	concerning	\textsc{nabs}	1\textsc{s.gen}	\textsc{lk}	\textsc{nr.act}–search \textsc{nabs}	\textsc{t.deon}–take.care	1\textsc{s.erg}	\textsc{nabs}	1\textsc{s.gen}	\textsc{lk}	family \\
\glt `I give thanks to our God because nothing happened in the deep sea concerning my searching for something I have to use to take care of my family.’ (The speaker was fishing so that his family would have food.) [MCWN-L-01 2.21]
\z

\ea
Uwi	pabligya	danen	daw	ipalit	ta	kinangļanen	ta	pamaļay-baļay. \\\smallskip
\gll U-wi	pa-bligya	danen	daw	i-palit	ta	kinangļan-en	ta	pang-baļay-baļay. \\
\textsc{emph-d1abs}	\textsc{t.r}-sell	3\textsc{p.erg}	and	\textsc{t.deon}-buy	\textsc{nabs}	need-\textsc{nr}	\textsc{nabs} \textsc{inst-red}-house \\
\glt ‘These (mats, different kinds of baskets they have woven) they sell and (use the money) to buy what is needed for use in houses.’ [NBWE-T-01 3.2]
\z

\ea
Patikang	ko	waļa	ko	na	batiis	piro	daw	gakereg ake	na	mga	lawa. \\\smallskip
\gll Pa-tikang	ko	waļa	ko	na	batiis	piro	daw	ga-kereg ake	na	mga	lawa. \\
\textsc{t.r}-step	1\textsc{s.erg}	left	1\textsc{s.gen}	\textsc{lk}	foot	but	if/when	\textsc{i.r}-shake 1\textsc{s.gen}	\textsc{lk}	\textsc{pl}	body \\
\glt ‘I used my left foot to step but it was like the (parts of) my body were shaking.' [JCWN-L-34 2.9]
\z

\ea
Uļa	aren	iniler	na	pabumba	ta	ake	na	baba. \\\smallskip
\gll Uļa	aren	iniler	na	pa=bumba	ta	ake	na	baba. \\
\textsc{neg.r}	1\textsc{s.abs}	inhaler	\textsc{lk}	\textsc{t.r}-pump	\textsc{nabs}	1\textsc{s.gen}	\textsc{lk}	mouth \\
\glt ‘I had no inhaler to use to pump (medicine) into my mouth.’ [JCWN-T-22 2.7]
\z

\ea
\label{ex:herbalmedicine}
May	daon	ta	kaoy	na	patampeļ. \\\smallskip
\gll May	daon	ta	kaoy	na	pa-tampeļ. \\
\textsc{ext.in}	leaf	\textsc{nabs}	tree	\textsc{lk} \textsc{t.r}-put.on.herbal.medicine \\
\glt ‘There are some leaves of trees used to put on as herbal medicine (like a bandage).’ [CBWE-T-07 5.3]
\z

Because the absolutive nominal in each of the examples \REF{ex:inthehouse} through \REF{ex:herbalmedicine} can be construed as an instrument, the only possibility is Instrumental voice.
\is{conveyance voice|)}\is{instrumental voice|)}

\section{Other transitivity-related phenomena}
\label{sec:othertransitivity}
\largerpage
To this point, we have characterized voice in Kagayanen as a grammatical system for adjusting the relationship between semantic roles and grammatical relations. In particular, the various voices in the Kagayanen choir highlight different semantic roles by bringing them into the absolutive case. In this section we will describe three additional grammatical phenomena that linguists often view as transitivity increasing or decreasing constructions. These are \textit{morphological causatives}\is{morphological causatives}, \textit{reflexives}\is{reflexive constructions}. and \textit{reciprocals}\is{reciprocal constructions}.


\subsection{Causatives}
\label{bkm:Ref500569782}
\label{sec:causatives}\is{causative constructions|(}

A causative construction is one in which an external causal agent is added to the message-world situation described by a clause. Semantically, a causative situation consists of two main parts: the \textit{cause} and the \textit{effect}. The effect is the base situation, and the cause is what the external causal agent does in order to bring about the effect. We will call the added agent the causer, and we will call the central participant of the effect the causee. Consider the following English causative construction:

\ea
The devil made me do it.
\z

In this example, \textit{I did it} describes the effect, and \textit{the devil made (something happen)} describes the cause; \textit{the devil} is the causer (also known as the agent\textsubscript{cause}), and \textit{me} is the causee.  Typically, the notion of cause may involve permission (\textit{allow}), and/or facilitation (\textit{make possible}), as well as strict causation. For example, in the following English sentences, the semantic roles of the participants are given above the referring expressions:

\ea
\label{bkm:Ref501028100}
    \ea
    \label{ex:letthehorseseat}
    \hspace{.15cm} Causer \hspace{.95cm} Causee \\
    {}[ Corelli ]  fed [ the horses ]. ‘Corelli made/let the horses eat.’
    \ex
    \label{ex:youmademelaugh}
    Causer \hspace{.7cm} Causee \\
    {}[ You ] made [ me ] laugh. \\
\newpage
    \ex
    \hspace{.15cm} Causer \hspace{1.45cm} Causee \hspace{1.8cm} Patient \\
    {}[ Meribel ]  urged [  Calixto ]  to order [ Hawaiian Pizza ]. \\
    \ex
    \label{ex:simeonletmeribelsendmichaelflowers}
    \hspace{.15cm} Causer \hspace{.9cm} Causee  \hspace{1.2cm} Recipient \hspace{.2cm} Theme \\
    {} [ Simeon ]  let [ Meribel ] send   [ 
 Michael ] [ flowers ]. \\
    \z
\z

Grammatically, causation may be accomplished in the lexicon, the syntax, or the morphology of any language. A \textit{lexical causative}\is{lexical causatives} is a single verb that embodies both the notions of cause and effect. For example, the English verb \textit{die} evokes an inherently intransitive concept. The verb \textit{kill}, on the other hand, evokes the same intransitive concept as \textit{die} does, but with the addition of a causer--\textit{kill} means \textit{cause to die}. There are many such non-causative/causative pairs of verbs in English, including \textit{learn/teach}, \textit{see/show}, \textit{go/send}, \textit{eat/feed}, and so on.  Example \ref{ex:letthehorseseat} is a lexical causative construction (\textit{feed} means ‘cause/allow to eat’). Corelli is the participant who does something to cause, allow or make it possible for the horses to eat.

A \textit{syntactic causative construction}\is{syntactic causative construction}\is{causatives!syntactic} (also called an “analytic causative”\is{analytic causatives}\is{causatives!analytic} is one in which the notions of cause and effect are embodied in different verbs. For example, \textit{cause to die} is a syntactic causative because there are two verbs, one expressing the cause and the other expressing the effect (\textit{die}). Examples \REF{ex:youmademelaugh} through \REF{ex:simeonletmeribelsendmichaelflowers} are all syntactic causative constructions because they involve the addition of a main verb, \textit{make}, \textit{urge}, or \textit{let}, which expresses the cause, while the dependent verbs (\textit{laugh}, \textit{order}, and \textit{send}) express the effect. The subject of the causative verb is the causer, while the grammatical object of that verb is the causee.

\hspace*{-4.1pt}Finally, a \textit{morphological causative}\is{morphological causatives}\is{causatives!morphological} (also sometimes called a “synthetic causative”\is{synthetic causatives}\is{causatives!synthetic}) is one in which non-causative and causative forms are related morphologically, usually via some kind of affixation. English does not have very productive morphological causatives, though pairs such as \textit{dark/darken} (\textit{darken} ‘cause to be dark’), and \textit{mad/madden} (\textit{madden} ‘cause to be mad’) come close. In other languages, including Kagayanen, morphological causatives are a much more robust and regular part of the grammar.

Kagayanen exhibits many pairs of lexical roots that may be considered non-causative/causative pairs. A few we have encountered to date are given in \tabref{tab:non-causativeandlexicallycausativeverbs}.

\begin{table}
\caption{Non-causative and lexically causative verbs in Kagayanen}
\label{tab:non-causativeandlexicallycausativeverbs}
\begin{tabularx}{\textwidth}{XX}
\lsptoprule
Non-causative verbs &  Lexically causative counterparts \\
\midrule
kaan ‘eat’

bugtaw ‘to wake up’

apoy ‘fire to burn’

kuyog ‘to go with s.o.’

mara ‘to dry’

duļak ‘animals to fight’

alin ‘to go away from’

basa ‘to be wet’ & baog ‘feed an animal’

pukaw ‘to wake s.o. up’

maket ‘start a fire’

anggat ‘ask/have s.o. go with you’

baļad ‘to dry something in the sun’

sampok ‘to cause animals to fight’

tabog ‘shoo away an animal’

bunyag ‘to water plants’ \\
\lspbottomrule
\end{tabularx}
\end{table}
In addition to these lexical causative verbs, Kagayanen has several verbs that function as \isi{syntactic causatives}. These verbs evoke the notion of cause, while the effect is expressed in a separate verb in irrealis modality.\footnote{We term this use of the irrealis modality “subjunctive”\is{subjunctive} (see \chapref{chap:clausecombining}, \sectref{sec:subjunctiveclauses}).} A few of these verbs are listed in \REF{bkm:Ref501085954}, and some examples from the texts are given below, beginning with \REF{bkm:Ref501085964}:


 \ea
\label{bkm:Ref501085954}
\textit{pwirsa}     ‘to force someone to do something’ \\
\textit{pilit}     ‘to pressure/urge someone to do something’ \\
\textit{leges}     ‘to strongly force/coerce someone to do something’ \\
\textit{sugo}     ‘to order or command someone to do something’ \\
\textit{mandar/mando}  ‘to order subordinates to do something’ (more formal for military or government officials) \\
\textit{sigyet/segyet}  ‘to entice someone to do something’ \\
\textit{aļam-aļam/am-aļam} ‘to convince someone to do something’ \\
\textit{bawal}     ‘to forbid someone to do something’ \\
\textit{ambaļan}     ‘to warn, advise, speak directly to, order’\footnote{The verb \textit{ambaļ} ‘say’ only means ‘to warn/advise/order’ in the applicative form. In other forms it means simply ‘to say.’}
\z


\ea
\label{bkm:Ref501085964}
Kaysan  \textbf{pwirsa}  kay  na  mutang  ta  duma  bag-o pautang  en  man  ki  danen. \\\smallskip
\gll Kaysan  \textbf{pwirsa}  kay  na  m-utang  ta  duma  bag-o pa-utang\footnotemark{}  en  man  ki  danen. \\
sometimes  force  1\textsc{p.excl.abs}  \textsc{lk}  \textsc{i.v.ir}-borrow.money  \textsc{nabs}  other  before
\textsc{caus}-borrow.money  \textsc{cm}  too  \textsc{obl.p}  3p \\
\footnotetext{The \textit{pa-} in this verb is the morphological causative, described below. The idea of ‘to lend’ something is expressed literally as ‘cause to borrow’.}
\glt `Sometimes we were forced to borrow money from others before lending to them too.’ [TTOB-L-03 6.1]
\z
\ea
… \textbf{paleges}  din  gid  na  daw  maimo  batunon  gid Kristo  iya  na  Salvador… \\\smallskip
\gll … \textbf{pa-leges}  din  gid  na  daw  ma-imo  baton-en  gid Kristo  iya  na  Salvador … \\
 {} \textsc{t.r}-force  3\textsc{s.erg}  \textsc{int}  \textsc{lk}  if/when  \textsc{a.hap.ir}-do/make  receive-\textsc{t.ir}  \textsc{int} Christ  3\textsc{s.gen}  \textsc{lk}  Savior \\
\glt `… he really coerced (him) to receive Christ to  be his savior …’ [ETON-C-07 3.14]
\z

\ea
Isya  na  adlaw  kanen  \textbf{pasugo}  ta  iya  na  nanay  na mandok ta waig.\\\smallskip
\gll Isya  na  adlaw  kanen  \textbf{pa-sugo}  ta  iya  na  nanay  na m-sandok ta waig. \\
one  \textsc{lk}  day/sun  3\textsc{s.abs}  \textsc{t.r}-order  \textsc{nabs}  3\textsc{s.gen}  \textsc{lk}  mother  \textsc{lk}
\textsc{i.v.ir}-carry.water \textsc{nabs} water \\
\glt `One day, as for her, her mother ordered her to carry water.’ [VAWN-T-20 3.1]
\z

The most productive means of forming a causative construction in Kagayanen is morphologically,\is{morphological causatives} expressed with the verbal prefix \textit{pa}{}-. This is a very common stem-forming prefix, as discussed in \chapref{chap:stemformingprocesses}, \sectref{sec:morphologicalcausative}. A few common non-caus\-a\-tive{\slash}morphological causative verb pairs are given in \REF{bkm:Ref501438581}. However, the causative prefix may be added to practically any root to form a transitive, causative verb stem. Meanings expressed by the morphological causative include ‘to cause someone to do something’, ‘to ask someone to do something’, ‘to let someone or something do something’, ‘to make it possible for someone to do something’, or ‘just wait for someone to do something or something to happen.’

[2]
\ea
\label{bkm:Ref501438581}
\begin{tabbing}
\hspace{5cm} \= \kill
\textit{eļes} ‘to borrow’ \> \textit{paeļes/peļes} ‘to lend’ \\
\textit{utang} ‘to borrow money’ \> \textit{pautang} ‘to lend money’ \\
\textit{daļa} ‘to bring/take/carry X’   \> \textit{padaļa} ‘to send X’ \\
\textit{patay} ‘to die’ \> \textit{papatay} ‘to kill’
\end{tabbing}
\z


As discussed in \chapref{chap:verbstructure}, \sectref{sec:transitiverealis}, the causative prefix \textit{pa}{}- is homophonous with the transitive realis Inflectional prefix. Therefore, one transitive verb with \textit{pa}{}- can have two meanings: one simple transitive \REF{bkm:Ref501092121} and the other causative \REF{bkm:Ref329253035}.

\ea
\label{bkm:Ref501092121}
Pasuot  din  bayo  din  ya. \\\smallskip
\gll Pa-suot  din  bayo  din  ya. \\
\textsc{t.r}-wear  3\textsc{s.erg}  clothes/shirt  3\textsc{s.gen}  \textsc{def.f} \\
\glt ‘S/he wore his/her shirt.’
\z
\ea
\label{bkm:Ref329253035}
...-Pasuot  din  ki  yaken  bayo  din  ya. \\\smallskip
\gll pa-pa-suot  din  ki  yaken  bayo  din  ya. \\
\textsc{t.r-caus}-wear  3\textsc{s.erg}  \textsc{obl.p}  1s  clothes/shirt  3\textsc{s.gen}  \textsc{def.f} \\
\glt ‘S/he let me wear his/her shirt.’
\z

It is clear, however, that there are two distinct \textit{pa}{}- morphemes for at least two reasons. First, example \REF{bkm:Ref329253035} is ditransitive--it has three arguments: a causer \textit{din} '3serg', a causee \textit{ki yaken} `\textsc{obl.p 1s}', and a patient \textit{bayo din ya} `shirt 3\textsc{s.erg def.f}'. This can only be understood as causative. The only non-causative interpretation would be “s/he wore his/her shirt to/on me”, which makes no sense. Example \REF{bkm:Ref501092121}, on the other hand, has only two arguments, and may not be understood as causative (*‘S/he made the shirt wear something’ ...). Second, it is possible for both \textit{pa}{}- morphemes to occur on the same verb:

\ea
\label{ex:his/hershirt}
Papasuot  din  ki  yaken  bayo  din  ya. \\\smallskip
\gll Pa-pa-suot  din  ki  yaken  bayo  din  ya. \\
\textsc{t.r-caus}-wear  3\textsc{s.erg}  \textsc{obl.p}  1s  clothes/shirt  3\textsc{s.gen}  \textsc{def.f} \\
\glt ‘S/he made me wear his/her shirt.’
\z

As discussed in \chapref{chap:verbstructure} \sectref{sec:omissionofprefixes}, sequences of double \textit{pa}- often simplify. However, when a double \textit{pa}- does occur, it expresses a stronger causative, as roughly reflected in the English free translations of \REF{bkm:Ref329253035} and \REF{ex:his/hershirt}.

With ditransitive verb roots, when there is only one \textit{pa}{}- it is the transitive, realis prefix \REF{bkm:Ref236213080}. For the causative, the shortened form is not normally used, probably for clarity \REF{bkm:Ref329253571}, but it can be used in fast speech.

\ea
\label{bkm:Ref236213080}
Non-causative: \\
Pabligya  din  ki  yaken  baboy  din  an. \\\smallskip
\gll Pa-bligya  din  ki  yaken  baboy  din  an. \\
\textsc{t.r}-sell  3\textsc{s.erg}  \textsc{obl.p}  1s  pig  3\textsc{s.gen}  \textsc{def.m} \\
\glt ‘S/he sold his/her pig to me.’
\z
\ea
\label{bkm:Ref329253571}
Causative: \\
Papabligya  din  ki  yaken  baboy  din  an. \\\smallskip
\gll Pa-pa-bligya  din  ki  yaken  baboy  din  an. \\
\textsc{t.r}-\textsc{caus}-sell  3\textsc{s.erg}  \textsc{obl.p}  1s  pig  3\textsc{s.gen}  \textsc{def.m} \\
\glt ‘S/he made/let me sell his/her pig.’
\z

Some non-volitional intransitive verbs are understood as causative when used in a grammatically transitive frame, even without the causative prefix \textit{pa}{}-. These are described as \textit{patient-preserving labile}\is{patient-preserving labile verbs} verbs in \chapref{chap:verbclasses-1} (Classes I-II),  or “P-labile” verbs\is{P-labile verbs} \citep{haspelmath1993, kulikov2014}:

\ea
Non-volitional intransitive (Class II) root: \textit{pudpod} ‘to crumble to powder’: \\
\textbf{Gapudpod}  siminto  an  tak  lugay  en. \\\smallskip
\gll \textbf{Ga-pudpod}  siminto  an  tak  lugay  en. \\
\textsc{i.r}-crumble.into.powder  cement  \textsc{def.m}  because  long.time  \textsc{cm} \\
\glt ‘The cement is/was crumbling up into powder because it has been a long time.’
\z

\ea
\label{bkm:Ref501112294} \label{ex:onthewound}
Causative: \\
\textbf{Pudpuron} no buļong an aged batang din naan ta nina din. \\\smallskip
\gll \textbf{Pudpod-en}  no  buļong an aged \emptyset{}-batang din naan ta nina din. \\
\textbf{crumble}-\textsc{t.ir}  2\textsc{s.erg}  medicine \textsc{def.m} so.that \textsc{t.ir}-put  3\textsc{s.erg}   \textsc{spat.def} \textsc{nabs} wound  3\textsc{s.gen} \\
\glt `Crush the medicine into powder so that she can put (it) on her wound.' (lit. `Cause the medicine to crumble to powder ...’)
\z

% \ea
% Intransitive: \\
% {Nalunot}  mga  bunga  ta  bayabas  tak  sigi  nang  na  uran. \\\smallskip
% \gll {Na-lunot}  mga  bunga  ta  bayabas  tak  sigi  nang  na  uran. \\
% \textsc{a.hap.r}-decay  \textsc{pl}  fruit  \textsc{nabs}  guava  because  continually  just  \textsc{lk}   rain \\
% \glt ‘The fruit of the guava has decayed because it just keeps on raining.’
% \z

% \ea
% \label{bkm:Ref501112194} \label{ex:decayintheirfield}
% Causative: \\
% \textbf{Palunot}  din  gid  sagbet  naan  ta  iran  na  uma. \\\smallskip
% \gll \textbf{Pa-lunot}  din  gid  sagbet  naan  ta  iran  na  uma. \\
% \textsc{t.r}-decay  3\textsc{s.erg}  \textsc{int}  weeds/trash  \textsc{spat.def}  \textsc{nabs}  3\textsc{p.gen}  \textsc{lk}  field \\
% \glt ‘S/he let/waited for/caused the weeds to decay in their field.’
% \z


As with all semantically transitive\is{semantic transitivity}\is{transitivity!semantic} verbs, causative verb stems can appear in a Detransitive (Actor voice) construction. In this case, the \textit{pa}{}- causative does appear with some verbs (such as \textit{lunot} `decay' in example \ref{bkm:Ref501112712}) but not with others (such as \textit{pudpod}, ‘crumble into powder’ in example \ref{bkm:Ref329259219}):

\ea
\label{bkm:Ref501112712}
Detransitive of causative: \\
\textbf{Gapalunot}  kanen  ta  tampayas.  \\\smallskip
\gll \textbf{Ga-pa-lunot}  kanen  ta  tampayas.  \\
\textsc{i.r}-\textsc{caus}-decay  3\textsc{s.abs}  \textsc{nabs}  papaya \\
\glt ‘S/he let/caused the papaya to decay.’ \\
\z

\ea
Non-volitional intransitive verb, \textit{pudpod} ‘to crumble into powder’: \\
\textbf{Napudpod}  tanan  na  siminto  na  paubra  na  allublak. \\\smallskip
\gll \textbf{Na-pudpod}  tanan  na  siminto  na  pa-ubra  na  allublak. \\
\textsc{a.hap.r}-crumble.into.powder  all  \textsc{lk}  cement  \textsc{lk}  \textsc{t.r}-work  \textsc{lk}  hollow.block \\
\glt ‘All the cement made into hollow block has crumbled up into powder.’
\z

\ea
\label{ex:putitonwound}
Causative: \\
\textbf{Papudpod}  din  buļong  an  aged  batang din  naan  ta  nina  din. \\\smallskip
\gll \textbf{Pa-pudpod}  din  buļong  an  aged  \emptyset{}-batang din  naan  ta  nina  din. \\
\textsc{t.r}-crumble.into.powder  3\textsc{s.erg}  medicine  \textsc{def.m}  so.that  \textsc{t.ir}-put
3\textsc{s.erg}  \textsc{spat.def  nabs}  wound  3\textsc{s.gen} \\
\glt `S/he crumbled the medicine into powder so that s/he could put it on his/her wound.’
\z

We know that the \textit{pa}{}- in example \REF{ex:putitonwound} is the transitive realis prefix and not a causative because a) in irrealis modality, \textit{pa}{}- does not need to appear even though the frame is transitive, and the meaning is causative (\ref{ex:onthewound}), and b) a double \textit{pa}{}- is not possible for this subclass of verbs: *\textit{papapudpod}.

\ea
\label{bkm:Ref329259219}
Detransitive (Actor voice) of causative: \\
\textbf{Gapudpod}  kanen  an  ta  buļong. \\\smallskip
\gll \textbf{Ga-pudpod}  kanen  an  ta  buļong. \\
\textsc{i.r}-crumble.into.powder  3\textsc{s.abs}  \textsc{def.m}  \textsc{nabs}  medicine \\
\glt ‘S/he is/was crumbling medicine into powder.’
\z

We know that \REF{bkm:Ref329259219} is a detransitive of a causative construction and not a simple intransitive construction because a) the absolutive pronoun refers to a person, and persons don’t (normally) crumble into powder, and b) the understanding is that the medicine is what crumbles, with the absolutive argument referring to a causal agent.

The reason verbs like \textit{lunot} ‘to decay/disintegrate’ require an overt causative morpheme in the detransitive may be that they describe processes that are not accomplished by an external agent--there is nothing one can do to “decay something”. Rather, one can only wait or allow it to happen on its own. On the other hand, verbs like \textit{pudpod} ‘to crumble into powder’ can be construed as the result of someone’s act.

Causative constructions may also take the applicative suffix to express Locative, Benefactive, or Adversative voice. With P-labile roots, such causatives may be expressed without an overt causative prefix.

\ea
Causative and applicative (Locative voice). P-Labile verb: \\
\textbf{Padagsaan}  din  baybay  an  ta  buti. \\\smallskip
\gll \textbf{Pa-dagsa-an}  din  baybay  an  ta  buti. \\
\textsc{t.r}-wash.ashore-\textsc{apl}  3\textsc{s.erg}  beach  \textsc{def.m}  \textsc{nabs}  small.boat. \\
\glt ‘S/he caused/let the small boat wash ashore.’
\z

\ea
Causative and applicative (Benefactive voice). P-Labile verb: \\
\textbf{Pabuian}  a  din  ta  saging  na  tanem ko  naan  ta  silong  nay. \\\smallskip
\gll \textbf{Pa-bui-an}  a  din  ta  saging  na  tanem ko  naan  ta  silong  nay. \\
\textsc{t.r}-live-\textsc{apl}  1\textsc{s.abs}  3\textsc{s.erg}  \textsc{nabs}  banana  \textsc{lk}  plant
1\textsc{s.erg}  \textsc{spat.def}  \textsc{nabs}  yard  1\textsc{p.excl.gen} \\
\glt `S/he kept alive for me the banana that I planted in our yard.’
\z

Some non-volitional intransitive verbs are understood as “double causatives” when the causative prefix occurs:

\ea
Causee as absolutive: \\
\textbf{Papalunot}  ko  \textbf{kanen}  an  ta  tampayas  na  inog. \\\smallskip
\gll \textbf{Pa-pa-lunot}  ko  \textbf{kanen}  an  ta  tampayas  na  inog. \\
\textsc{t.r}-\textsc{caus}-decay  1\textsc{s.erg}  3\textsc{s.abs}  \textsc{def.m}  \textsc{nabs}  papaya  \textsc{lk}  ripe \\
\glt ‘I caused him/her to let the ripe papaya decay.’
\z

\ea
Patient as absolutive: \\
\textbf{Papalunot}  ko  ki  kanen  \textbf{tampayas}  an  na  inog. \\\smallskip
\gll \textbf{Pa-pa-lunot}  ko  ki  kanen  \textbf{tampayas}  an  na  inog. \\
\textsc{t.r}-\textsc{caus}-decay  1\textsc{s.erg}  \textsc{obl}  3s  papaya  \textsc{def.m}  \textsc{lk}  ripe \\
\glt ‘I caused him/her to let the ripe papaya decay.’
\z

\ea
Causee as absolutive (irrealis modality): \\
\textbf{Palunuton}  ko  \textbf{kanen}  an  ta  tampayas  na  inog. \\\smallskip
\gll \textbf{Pa-lunot-en}  ko  \textbf{kanen}  an  ta  tampayas  na  inog. \\
\textsc{caus}-decay-\textsc{t.ir}  1\textsc{s.erg}  3\textsc{s.abs}  \textsc{def.m}  \textsc{nabs}  papaya \textsc{lk} ripe\\
\glt ‘I will cause him/her to cause/let the ripe papaya decay.’
\z
\ea
Patient as absolutive (irrealis modality): \\
\textbf{Palunuton}  ko  ki  kanen  \textbf{tampayas}  an  na  inog \\\smallskip
\gll \textbf{Pa-lunot-en}  ko  ki  kanen  \textbf{tampayas}  an  na  inog \\
\textsc{caus}-decay-\textsc{t.ir}  1\textsc{s.erg}  \textsc{obl.p}  3s  papaya  \textsc{def.m}  \textsc{lk}  ripe \\
\glt ‘I will cause him/her to let the ripe papaya decay.’
\z

\ea
Detransitive with causer as absolutive: \\
\textbf{Gapalunot}  kanen  ki  yaken  ta  tampayas. \\\smallskip
\gll \textbf{Ga-pa-lunot}  kanen  ki  yaken  ta  tampayas. \\
\textsc{i.r}-\textsc{caus}-decay  3\textsc{s.abs}  \textsc{obl.p}  1s  \textsc{nabs}  papaya \\
\glt ‘S/he is/was causing me to let the papaya decay.’
\z

\ea
\textbf{Magpalunot}  ka  ta  tampayas  ki  kanen. \\\smallskip
\gll \textbf{Mag-pa-lunot}  ka  ta  tampayas  ki  kanen. \\
\textsc{i.ir}-\textsc{caus}-decay  2\textsc{s.abs}  \textsc{nabs}  papaya  \textsc{obl.p}  3s \\
\glt ‘You will cause him/her to let the papaya decay.’
\z

\ea
Double causative and applicative (Locative voice). P-Labile verb: \\
\textbf{Papadagsaan}  din  ki  yaken  \textbf{baybay}  \textbf{an}  ta  mga batang. \\\smallskip
\gll \textbf{Pa-pa-dagsa-an}  din  ki  yaken  \textbf{baybay}  \textbf{an}  ta  mga batang. \\
\textsc{t.r}-\textsc{caus}-wash.ashore-\textsc{apl}  3\textsc{s.erg}  \textsc{obl.p}  1s  beach  \textsc{def.m}  \textsc{nabs} \textsc{pl} driftwood \\
\glt `S/he made me let some driftwood wash ashore on the beach.’
\z

\ea
\label{bkm:Ref501176811}
\textbf{Padagsaan}  din  ki  yaken  \textbf{baybay}  \textbf{an}  ta  mga batang. \\\smallskip
\gll \emptyset{}-\textbf{Pa-dagsa-an}  din  ki  yaken  \textbf{baybay}  \textbf{an}  ta  mga batang. \\
\textsc{t.ir-caus}-wash.ashore-\textsc{apl}  3\textsc{s.erg}  \textsc{obl.p}  1s  beach  \textsc{def.m}  \textsc{nabs}  \textsc{pl} driftwood \\
\glt `S/he will make me let some driftwood wash ashore on the beach.’
\z

We know that the \textit{pa}{}- prefix is a causative, and the modality is irrealis in \REF{bkm:Ref501176811} because of the argument structure. The presence of an oblique argument, \textit{ki yaken}, makes it clear that this is a double causative. The absence of an overt transitivity marker on the verb indicates that this must be irrealis modality (recall that the presence of the applicative -\textit{an} precludes the use of the transitive irrealis suffix -\textit{en}, thus leaving “zero” as the only indicator of transitivity). If the \textit{pa}{}- in this example were the transitive realis \textit{pa}{}-, the only possible reading would be “You brought the driftwood to the beach at me,” and such a meaning is impossible.

The following are examples of causatives of semantically volitional intransitive verbs. In contrast to non-volitional P-labile verbs, these verbs must occur with the causative prefix in order to be understood as causative:

\ea
\textbf{Papalarga}  din  bata  din  ya  aged  mangita ta  ubra.\\\smallskip
\gll \textbf{Pa-pa-larga}  din  bata  din  ya  aged  ma-ng-ita ta  ubra.\\
\textsc{t.r}-\textsc{caus}-depart  3\textsc{s.erg}  child  3\textsc{s.gen}  \textsc{def.f}  so.that \textsc{a.hap.ir}-\textsc{pl}-search \textsc{nabs}  work \\
\glt ‘S/he caused his/her child to depart in order to search for work.’
\z

\ea
\textbf{Palargaen}  a  ta  mga  ginikanan  ko  tak  adlek  na mangasawa. \\\smallskip
\gll \textbf{Pa-larga-en}  a  ta  mga  ginikanan  ko  tak  adlek  na ma-ngasawa. \\
\textsc{caus}-depart-\textsc{t.ir}  1\textsc{s.abs}  \textsc{nabs}  \textsc{pl}  parent  1\textsc{s.gen}  because  afraid  \textsc{lk} \textsc{a.hap.r}-get.married \\
\glt `My parents will send me away (cause me to depart) because (they) are afraid I will get married soon.’
\z

\ea
Detransitive (Actor voice) of causative, absolutive = agent\textsubscript{cause}: \\
\textbf{Gapalarga}  kanen  an  ta  mga  lansa  tak  uļa  en  bagyo. \\\smallskip
\gll \textbf{Ga-pa-larga}  kanen  an  ta  mga  lansa  tak  uļa  en  bagyo. \\
\textsc{i.r}-\textsc{caus}-depart  3\textsc{s.abs}  \textsc{def.m}  \textsc{nabs}  \textsc{pl}  launch  because  \textsc{neg.r}  \textsc{cm}  typhoon \\
\glt ‘S/he is/was causing/letting the launches (to) depart because there was no more typhoon.’
\z

\ea
\textbf{Magpalarga}  kanen  an  ta  mga  lansa  daw  linaw  en. \\\smallskip
\gll \textbf{Mag-pa-larga}  kanen  an  ta  mga  lansa  daw  linaw  en. \\
\textsc{i.ir}-\textsc{caus}-depart  3\textsc{s.abs}  \textsc{def.m}  \textsc{nabs}  \textsc{pl}  launch  if/when  calm  \textsc{cm} \\
\glt ‘S/he will cause/let the launches (to) depart when it is calm.’
\z


\ea
Applicative (Locative voice) + Causative: \\
\textbf{Papailingan}  din  ki  yaken  baļay  din  an  na  pagpayag. \\\smallskip
\gll \textbf{Pa-pa-iling-an}  din  ki  yaken  baļay  din  an  na  pag-payag. \\
\textsc{t.r}-\textsc{caus}-go-\textsc{apl}  3\textsc{s.erg}  \textsc{obl.p}  1s  house  3\textsc{s.gen}  \textsc{def.m}  \textsc{lk}  \textsc{red}-nipa.house \\
\glt ‘S/he caused/let me go to his/her house that is a rest house.’
\z

The following examples illustrate causatives of inherently transitive and ditransitive verbs. Again, with these verbs the causative prefix is required in order to express the causative sense:

\ea
Patient = Absolutive: \\
\textbf{Papatangtang}  ko  ki  kanen  \textbf{igot}  \textbf{an}  ta  lunday. \\\smallskip
\gll \textbf{Pa-pa-tangtang}  ko  ki  kanen  \textbf{igot}  \textbf{an}  ta  lunday. \\
\textsc{t.r}-\textsc{caus}-remove  1\textsc{s.erg}  \textsc{obl.p}  3s  tie  \textsc{def.m}  \textsc{nabs}  outrigger.canoe \\
\glt ‘I had him/her remove the tie from the outrigger canoe.’
\z

\ea
Causee = Absolutive: \\
\textbf{Papatangtang}  ko  \textbf{kanen}  an  ta  igot  ta  lunday. \\\smallskip
\gll \textbf{Pa-pa-tangtang}  ko  \textbf{kanen}  an  ta  igot  ta  lunday. \\
\textsc{t.r}-\textsc{caus}-remove  1\textsc{s.erg}  3\textsc{s.abs}  \textsc{def.n}  \textsc{nabs}  tie  \textsc{nabs}  outrigger.canoe \\
\glt ‘I made him/her remove the tie from the outrigger canoe.’
\z

\ea
Causer = Absolutive (detransitive): \\
\textbf{Gapatangtang}  \textbf{a}  ki  kanen  ta  igot  ta  guso. \\\smallskip
\gll \textbf{Ga-pa-tangtang}  \textbf{a}  ki  kanen  ta  igot  ta  guso. \\
\textsc{i.r}-\textsc{caus}-remove  1\textsc{s.abs}  \textsc{obl.p}  3s  \textsc{nabs}  tie  \textsc{nabs}  seaweed \\
\glt `I am/was causing/allowing him/her to remove the tie of the agar-agar seaweed.’
\z

\ea
\textbf{Magpatangtang}  \textbf{ka}  nang  ki  kanen  ta  igot  ta  lunday. \\\smallskip
\gll \textbf{Mag-pa-tangtang}  \textbf{ka}  nang  ki  kanen  ta  igot  ta  lunday. \\
\textsc{i.ir}-\textsc{caus}-remove  2\textsc{s.abs}  just  \textsc{obl.p}  3s  \textsc{nabs}  tie  \textsc{nabs}  outrigger.canoe \\
\glt ‘You just cause/let him/her remove the tie of the outrigger canoe.’
\z



\ea
Instrument = Absolutive (Instrumental voice): \\
\textbf{Iparabas}  no  ki  kanen  \textbf{ļangkaw}  \textbf{an}  \textbf{na} \textbf{sundang}. \\\smallskip
\gll \textbf{I-pa-rabas}  no  ki  kanen  \textbf{ļangkaw}  \textbf{an}  \textbf{na} \textbf{sundang}. \\
\textsc{t.deon}-\textsc{caus}-clear.land  2\textsc{s.erg}  \textsc{obl.p}  3s  long  \textsc{def.m}  \textsc{lk}  machete \\
\glt ‘You must cause/let him/her clear the land with the long machete.’
\z

\ea
\textbf{Paparabas}  ko  ki  kanen  \textbf{ļangkaw}  \textbf{an}  \textbf{na}  \textbf{sundang}. \\\smallskip
\gll \textbf{Pa-pa-rabas}  ko  ki  kanen  \textbf{ļangkaw}  \textbf{an}  \textbf{na}  \textbf{sundang}. \\
\textsc{t.r}-\textsc{caus}-clear.land  1\textsc{s.erg}  \textsc{obl.p}  3s  long  \textsc{def.m}  \textsc{lk}  machete \\
\glt ‘I caused/let him/her clear the land with the long machete.’
\z

\ea
Theme = Absolutive (causative of ditransitive): \\
\textbf{Paatag}  no  kani  ki  kanen  kwarta  i  naan ta  tatay  din. \\\smallskip
\gll \emptyset{}-\textbf{pa-atag}  no  kani  ki  kanen  kwarta  i  naan ta  tatay  din. \\
\textsc{t.ir}-\textsc{caus}-give  2\textsc{s.erg}  later  \textsc{obl.p}  3s  money  \textsc{def.n}  \textsc{spat.def} \textsc{nabs}  father  3\textsc{s.gen} \\
\glt `Let him/her later give the money to his/her father.’
\z

\ea
\textbf{Ipaatag}  no  kani  ki  kanen  kwarta  i  naan ta  tatay  din. \\\smallskip
\gll \textbf{I-pa-atag}  no  kani  ki  kanen  kwarta  i  naan ta  tatay  din. \\
\textsc{t.deon}-\textsc{caus}-give  2\textsc{s.erg} later  \textsc{obl.p}  \textsc{3s} money \textsc{def.n}  \textsc{spat.def}
\textsc{nabs} father  3\textsc{s.gen} \\
\glt `You must later let him/her give the money to his/her father.’
\z

\ea
Patient as absolutive (Patient voice): \\
Papasabligan din ki yaken mga ittaw an ta waig. \\
\textbf{Pasablig}  no  ki  kanen  waig  an  naan  ta  mga  ittaw. \\\smallskip
\gll \emptyset{}-\textbf{Pa-sablig}  no  ki  kanen  waig  an  naan  ta  mga  ittaw. \\
\textsc{t.ir}-\textsc{caus}-splash  2\textsc{s.erg}  \textsc{obl.p}  3s  water  \textsc{def.m} \textsc{spat.def} \textsc{nabs}  \textsc{pl}  people \\
\glt ‘Cause/let him/her splash the water on people.’
\z



\ea
\textbf{Ipasablig}  no  ki  kanen  waig  an  naan  ta  mga  ittaw. \\\smallskip
\gll \textbf{I-pa-sablig}  no  ki  kanen  waig  an  naan  ta  mga  ittaw. \\
\textsc{t.deon}-\textsc{caus}-splash  2\textsc{s.erg}  \textsc{obl.p}  3s  water  \textsc{def.m} {spat.def}\textsc{} \textsc{nabs}  \textsc{pl}  people \\
\glt ‘You must cause/let him/her splash the water on people.’
\z

No example has been found of Benefactive or Adversative voice in a deontic causative construction (\textit{i-}+\textit{pa}{}-) in the corpus.

\ea
Applicative of a causative (Locative voice): \\\smallskip
\gll \textbf{Pa-pa-sablig-an}  din  ki  yaken  mga  ittaw  an  ta  waig. \\
\textsc{t.r}-\textsc{caus}-splash-\textsc{apl}  3\textsc{s.erg}  \textsc{obl.p}  1s  \textsc{pl}  people  \textsc{def.m}  \textsc{nabs}  water \\
\glt ‘S/he caused/let me splash at the people with water.’
\z

\ea
Applicative of a causative (Benefactive voice): \\
\textbf{Papabunakan}  ka  din  ki  yaken  ta  mga  bayo  no. \\\smallskip
\gll \textbf{Pa-pa-bunak-an}  ka  din  ki  yaken  ta  mga  bayo  no. \\
\textsc{t.r}-\textsc{caus}-laundry-\textsc{apl}  2\textsc{s.abs}  3\textsc{s.erg}  \textsc{obl.p}  1s  \textsc{nabs}  \textsc{pl}  clothes  2\textsc{s.gen} \\
\glt ‘S/he caused/let me wash your clothes for you.’
\z

\ea
Applicative of a Causative (Adversative voice): \\
\textbf{Papalagpukan}  din  ki  yaken  mga  ittaw  an  ta  baril. \\\smallskip
\gll \textbf{Pa-pa-lagpuk-an}  din  ki  yaken  mga  ittaw  an  ta  baril. \\
\textsc{t.r}-\textsc{caus}-explode-\textsc{apl}  3\textsc{s.erg}  \textsc{obl.p}  1s  \textsc{pl}  people  \textsc{def.m}  \textsc{nabs}  gun \\
\glt ‘S/he caused/let me fire off a gun on the people.’ (This does not mean the people were being shot at, but just being scared or intimidated by the firing of the gun.)
\z

The detransitive causative of some verbs has a reflexive sense, since the causer and the patient may be coreferential (\ref{bkm:Ref329267348}). The meaning is ‘to let oneself have something done to oneself.’ A distinct causee can occur optionally in an oblique phrase, as in \REF{bkm:Ref329267373} and \REF{bkm:Ref329267376}.

\ea
\label{bkm:Ref329267348}
\textbf{Magpasagod}  ka  nang  en  daw  manakem  ka. \\\smallskip
\gll \textbf{Mag-pa-sagod}  ka  nang  en  daw  manakem  ka. \\
\textsc{i.ir}-\textsc{caus}-take.care  2\textsc{s.abs}  just  \textsc{cm}  if/when  older  2\textsc{s.abs} \\
\glt ‘Just have yourself taken care of when you are elderly.’
\z

\ea
\label{bkm:Ref329267373}
\textbf{Gapabuļong}  a  naan  ta  duktor. \\\smallskip
\gll \textbf{Ga-pa-buļong}  a  naan  ta  duktor. \\
\textsc{i.r}-\textsc{caus}-medicine  1\textsc{s.abs}  \textsc{spat.def}  \textsc{nabs}  doctor \\
\glt ‘I am/was having myself medicated by the doctor.’
\z

\ea
\label{bkm:Ref329267376}
\textbf{Gapaluko}  ka  nang  imo  ta  yo  na  ittaw. \\\smallskip
\gll \textbf{Ga-pa-luko}  ka  nang  imo  ta  yo  na  ittaw. \\
\textsc{i.r}-\textsc{caus}-trick  2\textsc{s.abs}  just  \textsc{emph}  \textsc{nabs}  \textsc{d4adj}  \textsc{lk}  person \\
\glt ‘You just let yourself be tricked by that person.'
\z

Example \REF{bkm:Ref108689939} illustrates two instances of this “reflexive/causative” usage for two other verbs, \textit{kita} ‘to see’ and \textit{kasaļ} ‘to marry’:

\largerpage
\ea
\label{bkm:Ref108689939}
Sabat  ta  pangka  ya,  “\textbf{Magpakita}  a  ta  imo  na  mga ginikanan  daw  ambaļen  ko  danen  na  \textbf{magpakasaļ}  \textbf{ki}." \\\smallskip
\gll Sabat  ta  pangka  ya,  “\textbf{Mag-pa-kita}  a  ta  imo  na  mga ginikanan  daw  ambaļ-en  ko  danen  na  \textbf{mag-pa-kasaļ}  \textbf{ki}." \\
answer  \textsc{nasb}  frog  \textsc{def.f}  \textsc{i.ir}-\textsc{caus}-see  1\textsc{s.abs}  \textsc{nabs}  2\textsc{s.gen}  \textsc{lk}  \textsc{pl}
parent  and  say-\textsc{t.ir}  1\textsc{s.erg}  3\textsc{p.abs}  \textsc{lk}  \textsc{i.ir}-\textsc{caus}-wedding  1\textsc{p.incl.abs} \\
\glt `The frog answered, “I will \textbf{show} myself to your parents and I will tell them that we are \textbf{going} \textbf{to} \textbf{get} \textbf{(ourselves)} \textbf{wedded/married}.”' [CBWN-C-17 6.16]
\is{causative constructions|(}
\z
\subsection{Reflexives and reciprocals}
\label{sec:reflexivesandreciprocals}

Reflexive and Reciprocal constructions 
% are types of valence reducing constructions in that they 
express the idea that two main participants in a scene are “the same” in some way. A \textit{reflexive}\is{reflexives} construction is one in which the Actor (A) and an Undergoer (O) are the same entity, as in \textit{She saw herself}. All true reflexive constructions reduce the semantic transitivity of a transitive clause by specifying that there are not two separate entities involved. Rather, one entity fulfills two semantic roles and/or grammatical relations. A prototypical \textit{reciprocal}\is{reciprocal} clause is one in which two participants equally act upon each other, for example \textit{they saw each other}.

Certain verbs can be described as “lexical reflexives” in that, when they occur in an intransitive construction, they are automatically understood as reflexive. These are mostly ``grooming” verbs, such as \textit{langoy} ‘bathe’ and \textit{arbas} ‘shave’:

\ea
Galangoy  kanen  an. \\\smallskip
\gll Ga-langoy  kanen  an. \\
\textsc{i.ir}-bathe  3\textsc{s.abs} \textsc{def.m} \\
\glt ‘S/he bathed (him/herself).’
\z
\ea
Gabarbas  kanen  an. \\\smallskip
\gll Ga-barbas  kanen  an. \\
\textsc{i.ir}-shave  3\textsc{s.abs} \textsc{def.m} \\
\glt `He shaved (himself).’
\z

Except for these lexical reflexives, and the extended use of causative morphology to express a kind of reflexive described in the previous section, most reflexive constructions in Kagayanen are syntactic. They consist of a transitive or detransitive construction with a possessed Referring Phrase headed either by \textit{kaugalingen} ‘self’ or  \textit{lawa} ‘body’  occurring in the Undergoer role. These may either be transitive (\ref{bkm:Ref118617763}), or detransitive (\ref{bkm:Ref118617766}):

\ea
\label{bkm:Ref118617763}
Panina  din  \textbf{lawa}  \textbf{din}  \textbf{an}. \\\smallskip
\gll Pa-nina  din  \textbf{lawa}  \textbf{din}  \textbf{an}. \\
\textsc{t.r}-wound  3\textsc{s.erg}  body  3\textsc{s.gen}  \textsc{def.m} \\
\glt ‘S/he wounded him/herself.’ (lit. ’S/he wounded her/his own body.’)
\z

\ea
\label{bkm:Ref118617766}
Gaugas  kanen  ta  \textbf{kaugalingen}  \textbf{din}. \\\smallskip
\gll Ga-ugas  kanen  ta  \textbf{kaugalingen}  \textbf{din}. \\
\textsc{i.r}-wash  3\textsc{s.abs}  \textsc{nabs}  self  3\textsc{s.gen} \\
\glt ‘S/he washed her/himself.’
\z

These examples must be understood as reflexive; the possessor of the Undergoer may not refer to a distinct person. In other words, \REF{bkm:Ref118617763} cannot mean `S/he wounded someone else's body’. \textit{Kaugalingen} is the dedicated reflexive nominal. \textit{Lawa} ’physical body’ is sometimes used in place of \textit{kaugalingen}, possibly because it is shorter.

Examples \REF{bkm:Ref118617763} and \REF{bkm:Ref118618251}-\REF{bkm:Ref118618196} show that ergative case Actors may antecede absolutive case reflexive RPs. Examples \REF{bkm:Ref118617766} and \REF{bkm:Ref501865097} illustrate that an absolutive Actor (\textit{ka} ‘you’) may be the antecedent for a reflexive Referring Phrase in a detransitive construction:


\ea
\label{bkm:Ref501865097}
Dili  ka  gid  dyan  magpabaya  ta  \textbf{imo}  \textbf{na}  \textbf{kaugalingen}. \\\smallskip
\gll Dili  ka  gid  dyan  mag-pa-baya  ta  \textbf{imo}  \textbf{na}  \textbf{kaugalingen}. \\
\textsc{neg,ir}  2\textsc{s.abs}  \textsc{int}  \textsc{d}4\textsc{loc}  \textsc{i.ir}-\textsc{caus}-ignore  \textsc{nabs}  2\textsc{s.gen}  \textsc{lk}  self \\
\glt ‘Do not just ignore yourself there.’ [VBWL-T-06 2.4]
\z

The following are some additional examples of syntactic reflexives from the corpus:

\ea
\label{bkm:Ref118618251}
Reflexive RP = Absolutive (theme): \\
\textbf{Iran}  \textbf{nang}  \textbf{na}  \textbf{kaugalingen} isip  ya  danen. \\\smallskip
\gll \textbf{Iran}  \textbf{nang}  \textbf{na}  \textbf{kaugalingen} isip  ya  danen. \\
 3\textsc{p.gen}  only  \textsc{lk}  self  think  \textsc{def.f}  3\textsc{p.erg} \\
\glt ‘Themselves only is what they think about.’ [MOOE-C-01 189]
\z

\ea
Reflexive RP = Non-absolutive (addressee): \\
Yon  ambaļ  ko  ta  \textbf{ake}  \textbf{na}  \textbf{kaugalingen} ... \\\smallskip
\gll Yon  ambaļ  ko  ta  \textbf{ake}  \textbf{na}  \textbf{kaugalingen} ... \\
\textsc{d}3\textsc{abs}  say  1\textsc{s.erg}  \textsc{nabs}  1\textsc{s.gen} \textsc{lk}  self \\
\glt ‘That is what I said to myself ...’ [EMWN-T-05 7.4]
\z

\ea
\label{bkm:Ref118618196}
Reflexive RP = Absolutive (theme): \\
Piro  gusto  ko  gid  anay  ipakilala  ko  \textbf{ake}  \textbf{na}  \textbf{kaugalingen}. \\\smallskip
\gll Piro  gusto  ko  gid  anay  i-pa-kilala  ko  \textbf{ake}  \textbf{na}  \textbf{kaugalingen}. \\
but  want  1\textsc{s.erg}  \textsc{int}  first/for.awhile  \textsc{t.deon}-\textsc{caus}-know  1\textsc{s.erg}  1\textsc{s.gen}  \textsc{lk}  self \\
\glt ‘But I really want to first have to introduce myself.’ (\textit{kilala} is code switching from \isi{Tagalog}.) [EMWL-T-04 4.5]
\z

\ea
Reflexive RP = Absolutive (patient): \\
… pirmi  a  nang  en  na  pangamuyo  ta Dios,  "Daw anen ka  pa  man  gapamati,  tagan  no  ta dayad na disisyon para  malibri  ko  pambot,  malibri  ko  \textbf{lawa ko}."\\\smallskip
\gll … pirmi  a  nang  en  na  …-pangamuyo  ta Dios,  "Daw anen ka  pa  man  ga-pa-mati,  \emptyset{}-atag-an  no  ta dayad na disisyon para  ma-libri  ko  pambot,  ma-libri  ko  \textbf{lawa}  \textbf{ko}." \\
{} always  1\textsc{s.abs}  only/just  \textsc{cm}  \textsc{lk}  \textsc{i.r}-pray  \textsc{nabs}
God  if/when \textsc{ext.g}  2\textsc{s.abs}  \textsc{inc} also  \textsc{i.r}-\textsc{caus}-hear  \textsc{t.ir}-give-\textsc{apl}  2\textsc{s.erg}  \textsc{nabs} good \textsc{lk} decision for \textsc{a.hap.ir}-free  2\textsc{s.erg} motorboat \textsc{a.hap.ir}-free  2\textsc{s.erg} body 1\textsc{s.gen} \\
\glt `… I kept on praying to God, “If you are still here listening, give (me) a good decision that I may free the motorboat, I may free \textbf{myself}."’ (This is a story about a fisherman whose boat was turned upside down and he was riding on the bottom of the boat still holding on to his hook and line with a big fish on it.) [EFWN-T-10 4.7]
\z

\ea
Reflexive RP = Non-absolutive (patient) \\
Nang  Maria,  mag-amblig  ka  ta  imo  na  \textbf{lawa}  para dili  ka  magmasakit  daw  dili  man  malik  imo na  sakit. \\\smallskip
\gll Nang  Maria,  mag-amblig  ka  ta  imo  na  \textbf{lawa} para dili  ka  mag-ma-sakit  daw  dili  man  m-balik  imo na  sakit. \\
older.sister  Maria  \textsc{i.ir}-be.careful/take.care  2\textsc{s.abs}  \textsc{nabs}  2\textsc{s.gen}  \textsc{lk} body  for
\textsc{neg.ir} 2\textsc{s.abs}  \textsc{i.ir}-\textsc{adj}-sick  and  \textsc{neg.ir}  also  \textsc{i.v.ir}-return 2\textsc{s.gen}  \textsc{lk}  pain \\
\glt `Older sister Maria, take care of (your) \textbf{self} so that you will not become sick and your pain will not return.’ [BCWL-C-02 6.1]
\z

Though \textit{kaugalingen} is the dedicated reflexive nominal, there is evidence that the meaning of \textit{lawa} is also becoming a dedicated reflexive itself. Example \REF{bkm:Ref118699324} shows that \textit{lawa} does not always refer literally to someone’s physical body.

\ea
\label{bkm:Ref118699324}
Reflexive RP = Non-absolutive (figure of speech) \\
Yi ni  isya  na  liksyon  ta  isya  na  ittaw  na  sali  ta  bansag  ta iya  na  \textbf{lawa}  na  dili  man  kasarang. \\\smallskip
\gll Yi ni  isya  na  liksyon  ta  isya  na  ittaw  na  sali  ta  bansag  ta iya  na  \textbf{lawa}  na  dili  man  ka-sarang. \\
\textsc{d1abs} \textsc{d1pr}  one  \textsc{lk}  lesson  \textsc{nabs}  one  \textsc{lk}  person  \textsc{lk} always \textsc{nabs}  nickname  3\textsc{s.gen}
3\textsc{s.gen}  \textsc{lk}  body  \textsc{lk}  \textsc{neg.ir} also \textsc{i.exm}-able \\
\glt `This very thing is a lesson for a person who always is boasting of \textbf{him/herself} (lit. nicknaming him/herself) when (s/he) is not capable (of doing what s/he boasts about).’ [DBWN-T-26 12.1]
\z

Example \REF{bkm:Ref118618209} illustrates the \isi{Tagalog} borrowing \textit{sarili} ‘self’ used in place of \textit{kaugalingen} or \textit{lawa}:

\ea
\label{bkm:Ref118618209}
… dili  ki  magbugal  ta  \textbf{ate}  \textbf{na}  \textbf{sarili}  na  kiten mabiskeg… \\\smallskip
\gll … dili  ki  mag-bugal  ta  \textbf{ate}  \textbf{na}  \textbf{sarili}  na  kiten ma-biskeg… \\
{} \textsc{neg.ir}  1\textsc{p.incl.abs}  \textsc{i.ir}-proud  \textsc{nabs}  1\textsc{p.incl.gen}  \textsc{lk}  self  \textsc{lk} 1\textsc{p.incl.abs}
\textsc{ar}-strong \\
\glt `….let’s not be proud of ourselves that we are strong…’ [BCWN-T-05 12.19]
\z

Kagayanen employs morphological reciprocal constructions. The morphological reciprocal reduces semantic and grammatical transitivity by expressing a multi-argument situation in a grammatically intransitive clause. As described in \chapref{chap:stemformingprocesses}, \sectref{sec:reciprocals}, there are two verb suffixes that express reciprocal constructions, -\textit{anay} and -\textit{ay}.

Examples \REF{bkm:Ref414723974} through \REF{bkm:Ref414775243} illustrate several prototypical examples of the reciprocal suffixes from the text corpus. Following this, some extended usages are described. A prototypical reciprocal construction involves an inherently transitive verb in a grammatically intransitive construction, with a reciprocal suffix on the verb and an Actor understood or explicited expressed as plural. The plural Actor may appear as a plural pronoun (\ref{bkm:Ref414723974}) or full RP (\ref{bkm:Ref414775363}):

\ea
\label{bkm:Ref414723974}  
Dimalas  na  \textbf{gasugatay}  danen  i  naan  ta  Puerto. \\\smallskip
\gll Dimalas  na  \textbf{ga-sugat-ay}  danen  i  naan  ta  Puerto. \\
  bad.luck  \textsc{lk}  \textsc{i.r}-meet-\textsc{rec}  3\textsc{p.abs}  \textsc{def.n}  \textsc{spat.def}  \textsc{nabs}  Puerto \\
\glt  ‘(It was) bad luck that they met each other in Puerto.’ [BEWN-T-01 5.3]
\z


The plural Actor can also be expressed in coordinated noun phrases:

\newpage

\ea
\label{bkm:Ref414775363}
\textbf{Pagkitaay}  danen  na  tama  en  na  mga  umang,  pasugiran  en  na  kanen  an  daw  Pwikan  kisyem  sellem  \textbf{maglumbaay}  dļagan.\\\smallskip
\gll \textbf{Pag-kita-ay}  danen  na  tama  en  na  mga  umang,  pa-sugid-an  en  na  kanen  an  daw  Pwikan  kisyem  sellem  \textbf{mag-lumba-ay}  dļagan. \\
\textsc{nr.act}-see-\textsc{rec}  3\textsc{p.abs}  \textsc{lk}  many  \textsc{cm}  \textsc{lk}  \textsc{pl}  hermit.crab \textsc{t.r}-tell-\textsc{apl} \textsc{cm}  \textsc{lk} 3\textsc{s.abs}  \textsc{def.m}  and  sea.turtle  tomorrow  morning  \textsc{i.ir}-race-\textsc{rec}  run \\
\glt `When he and many hermit crabs met, (he) told (them) that he and Sea-turtle tomorrow morning would race each other running.’ [JCON-L-08 19.4]
\z

Notice the first word in example \REF{bkm:Ref414775363} carries the reciprocal suffix -\textit{ay} and also the action nominalization prefix \textit{pag}{}- (see \chapref{chap:referringexpressions}, \sectref{sec:pag}). The sense of this action nominalization is “seeing each other”, or “meeting together”. Examples \REF{bkm:Ref501776847} and \REF{bkm:Ref501776849} illustrate the same verb with the reciprocal suffix as the main predicator of the clause:

\ea
  \label{bkm:Ref501776847}
Uyi  \textbf{kitaay}  danen  ta  iya  na  sawa  paagi  man   ta  kaļat  nang. \\\smallskip
\gll U-yi  ...-\textbf{kita-ay}  danen  ta  iya  na  sawa  paagi  man   ta  kaļat  nang. \\
\textsc{emph-d}1\textsc{abs}  \textsc{i.r}-see-\textsc{rec}  3\textsc{p.abs}  \textsc{nabs}  3\textsc{s.gen}  \textsc{lk}  spouse  by.means  too \textsc{nabs}   rope  only \\
\glt `These ones, he and his spouse met/saw each other by means of just a rope…’ (His wife lowered a rope out her window and they eloped.)  [MBON-T-07a 13.1]
\z

\ea
\label{bkm:Ref501776849}
Mam,  taan  sadya  kaw  pirmi  tak  \textbf{gakitaay}  kaw  ta inyo  na  mga  utod  daw  arey. \\\smallskip
\gll Mam,  taan  sadya  kaw  pirmi  tak  \textbf{ga-kita-ay}  kaw  ta inyo  na  mga  utod  daw  arey. \\
Ma’am  maybe  happy  2\textsc{p.abs}  always  because  \textsc{i.r}-see-\textsc{rec}  2\textsc{p.abs} \textsc{nabs}
2\textsc{p.gen}  \textsc{lk}  \textsc{pl}  sibling  and  friend \\
\glt `Ma’am,  maybe you are always happy because you and your siblings and friends are seeing each other.’ [AFWL-L-014.2]
\z

Reciprocal constructions are very productive and common. The following are some additional examples from the corpus:

\ea
Dili  \textbf{mag-agaway}  ta  mga  pagkaan. \\\smallskip
\gll Dili  \textbf{mag-agaw-ay}  ta  mga  pagkaan. \\
\textsc{neg.ir}  \textsc{i.ir}-grab-\textsc{rec}  \textsc{nabs}  \textsc{pl}  food \\
\glt ‘Don’t grab food from each other.’ [ETOP-C-10 2.4]
\z

\ea
\label{bkm:Ref414775243}
Paambaļ  na  tangkeban,  sura  din,  may  darwa  na  bato na  \textbf{gatubangay}  na  may  kulor  na  gainog-inog. \\\smallskip
\gll Pa-ambaļ  na  tangkeb-an,  sura  din,  may  darwa  na  bato na  \textbf{ga-tubang-ay}  na  may  kulor  na  ga-inog-inog.\footnotemark{} \\
\textsc{t.r}-say  \textsc{lk} close-\textsc{nr}  looks  3\textsc{s.gen}  \textsc{ext.in}  two  \textsc{lk}  rock \textsc{lk}  \textsc{i.r}-facing-\textsc{rec}  \textsc{lk}  \textsc{ext.in} color  \textsc{lk}  \textsc{i.r}-\textsc{red}-red \\
\footnotetext{The usual root meaning ‘red’ is \textit{minog}. Sometimes the form is \textit{inog}, especially when reduplicated, as in this example, in which case it means ‘reddish’ or ‘getting red’ as a red fruit when ripening.} 
\glt `What is called \textit{tangkeban}, it looks like two rocks that have a reddish color \textbf{facing} \textbf{each} \textbf{other}.’ [EFWE-T-05 3.1]
\z

Reduplication may co-occur with a reciprocal suffix:
\ea
Daw  danen  gampang  dili  \textbf{magtangayaw-tangayaway} … Dili \textbf{makagat-kagatay}. \\\smallskip
\gll Daw  danen  ga-ampang  dili  \textbf{mag-tangayaw-tangayaw-ay} … Dili  \textbf{ma-kagat-kagat-ay}. \\
if/when  3\textsc{p.abs}  \textsc{i.r}-play  \textsc{neg.ir} \textsc{i.ir}-\textsc{red}-say.bad.things-\textsc{rec} {} \textsc{neg.ir} \textsc{a.hap.ir-red}-bite-\textsc{rec} \\
\glt `When they are playing together, (they) \textbf{will not keep saying bad things to each other}… (They) \textbf{will not keep biting each other}.’ [LBOP-C-04 1.2-3]
\z

\ea
Gaampang  danen  daw  \textbf{galagas-lagasay} danen  naan  ta  suba … \\\smallskip
\gll Ga-ampang  danen  daw  \textbf{ga-lagas-lagas-ay} danen  naan  ta  {suba …} \\
\textsc{i.r}-play  3\textsc{p.abs}  and  \textsc{i.r-red}-chase-\textsc{rec}  3\textsc{p.abs}  \textsc{spat.def}  \textsc{nabs}  river \\
\glt ‘They were  playing and they kept chasing each other in the river …’ [CBWN 6.2]
\z

The following are two non-prototypical usages of the reciprocal suffix. These uses do not occur in the corpus, but are often heard in conversation. First, the suffix -\textit{ay} can express “mutual action”, that is, a group of people doing an action together over a period of time, though not necessarily acting on each other:

\ea
\textbf{Gasayaway}  mga  bai  naan  ta  balkon. \\\smallskip
\gll \textbf{Ga-sayaw-ay}  mga  bai  naan  ta  balkon. \\
\textsc{i.r}-dance-\textsc{rec}  \textsc{pl} woman  \textsc{spat.def}  \textsc{nabs}  porch \\
\glt ‘The women danced together (as a group but not as dance partners) on the porch.’
\z
\ea
\textbf{Gainemay}  mga  mama  naan  ta  balkon. \\\smallskip
\gll \textbf{Ga-inem-ay}  mga  mama  naan  ta  balkon. \\
\textsc{  i.r}-drink-\textsc{rec}  \textsc{pl}  man  \textsc{spat.def}  \textsc{nabs}  porch \\
\glt ‘The men drank together for some time on the porch.’
\z

\ea
Ta  buļan  ta  Nobyembre  1988  ta  mga  bagyo  na  \textbf{gasunod-sunuray}  alos  galibeg  uļo  ta  mga  ittaw  di tak  naubusan  kay  ta  beggas.\\\smallskip
\gll Ta  buļan  ta  Nobyembre  1988  ta  mga  bagyo  na  \textbf{ga-sunod-sunod-ay}  alos  ga-libeg  uļo  ta  mga  ittaw  di tak  na-ubus-an  kay  ta  beggas.\\
\textsc{nabs}  month/moon  \textsc{nabs}  November  1988  \textsc{nabs}  \textsc{pl}  typhoon  \textsc{lk} \textsc{i.r}-\textsc{red}-follow-\textsc{rec}  almost  \textsc{i.r}-worry  head  \textsc{nabs}  \textsc{pl}  person \textsc{d}1\textsc{loc}  because  \textsc{a.hap.r}-use.up-\textsc{apl}  1\textsc{p.excl.abs}  \textsc{nabs}  milled.rice \\
\glt `In the month of November, 1988, because of typhoons that kept \textbf{following one after another}, mostly people here were worried because we ran out of rice.’ [EMWL-T-03 7.2]
\z
\ea
\textbf{Gaeļes-eļesay}  kami  ta  kanta. \\\smallskip
\gll \textbf{Ga-eļes-eļes-ay}  kami  ta  kanta. \\
\textsc{i.r}-\textsc{red}-take.turns-\textsc{rec}  1\textsc{p.excl.abs}  \textsc{nabs} sing \\
\glt ‘We keep taking turns singing.’
\z

\ea
\textbf{Gaeļes-eļesay}  kami  ta  bayo. \\\smallskip
\gll \textbf{Ga-eļes-eļes-ay}  kami  ta  bayo. \\
\textsc{i.r-red}-borrow-\textsc{rec}  1\textsc{p.excl.abs}  \textsc{nabs}  clothes \\
\glt ‘We keep borrowing each other’s clothes.’ (lit. ‘We keep on taking turns with clothes.’)
\z

Finally, the reciprocal may be used in a special sense when only one participant is acting (\ref{bkm:Ref402683617}-\ref{bkm:Ref118453570}). The free translations of these examples somewhat capture the sense of what this construction expresses, though they are not very literal. A literal translation of \REF{bkm:Ref402683617} would be something like “That is dancing of the woman with each other.” The idea is that the woman is dancing so well or vigorously that it seems as though there are several women dancing together. This is a very productive construction that can involve all kinds of actions and participants.

\ea
\label{bkm:Ref402683617}
Nyan  na  \textbf{sayaway}  ta  bai  an! \\\smallskip
\gll Nyan  na  \textbf{sayaw-ay}  ta  bai  an. \\
\textsc{d2abs}  \textsc{lk}  dance-\textsc{rec}  \textsc{nabs}  woman  \textsc{def.m} \\
\glt ‘Wow! Look at the woman dance!’
\z

\ea
Nyan  na  \textbf{ampangay}  ta  bata  an! \\\smallskip
\gll Nyan  na  \textbf{ampang-ay}  ta  bata  an. \\
\textsc{d2abs}  \textsc{lk}  play-\textsc{rec}  \textsc{nabs}  child  \textsc{def.m} \\
\glt ‘Wow! Look at the child play!’
\z

\ea
Nyan  na  \textbf{layugay}  ta  yupan  an! \\\smallskip
\gll Nyan  na  \textbf{layog-ay}  ta  yupan  an. \\
\textsc{d2abs}  \textsc{lk}  fly-\textsc{rec}  \textsc{nabs}  bird  \textsc{def.m} \\
\glt ‘Wow! Look at the bird fly!’
\z
\ea
Nyan  na  \textbf{kantaay}  ta  mama an na  ingaw! \\\smallskip
\gll Nyan  na  \textbf{kanta-ay}  ta  mama an na ingaw. \\
\textsc{d2abs}  \textsc{lk}  sing-\textsc{rec}  \textsc{nabs} man \textsc{def.m} \textsc{lk} drunk \\
\glt ‘Wow! Look at the drunk man sing!’
\z

\ea
Nyan  na  \textbf{kantaay}  ta  bata  an! \\\smallskip
\gll Nyan  na  \textbf{kanta-ay}  ta  bata  an. \\
\textsc{d2abs}  \textsc{lk}  sing-\textsc{rec}  \textsc{nabs}  chld  \textsc{def.m} \\
\glt ‘Wow! Look at the child sing!’
\z

\ea
\label{bkm:Ref118453570}
Nyan  na  \textbf{basaay}  ta  libro  bata  an! \\\smallskip
\gll Nyan  na  \textbf{basa-ay}  ta  libro  bata  an. \\
\textsc{d2abs}  \textsc{lk}  read-\textsc{rec}  \textsc{nabs}  book  chld  \textsc{def.m} \\
\glt ‘Wow! Look at the child reading books!’
\is{voice|)}
\z
% \begin{verbatim}%%move bib entries to  localbibliography.bib
% \end{verbatim}warning
    
