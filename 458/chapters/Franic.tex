\documentclass[output=paper,colorlinks,citecolor=brown,arabicfont,chinesefont,booklanguage=french]{langscibook}
\ChapterDOI{10.5281/zenodo.15394485}
\author{Ivana Franić \affiliation{University of Zagreb, Croatia }}
\title[Dictionnaires manuscrits dans l’histoire de la lexicographie croate]
      {Dictionnaires manuscrits dans l’histoire de la lexicographie croate~: Des recueils de mots aux trésors linguistiques et culturels}

\abstract{The paper discusses the status of manuscript dictionaries created in Dubrovnik from the end of the 16th to the beginning of the 19th century and their role in the history of the Croatian lexicography. Firstly, I provide a brief overview of these dictionaries, with a particular attention given to the \emph{Dictionarium Latino-Illiricum} (1715--1716), a bilingual dictionary, which is a combination of an explanatory and a translational general dictionary with encyclopaedic elements (\citealt{Franic2014}). 
With this analysis, we want to show that manuscript dictionaries, although they do not have the communication potential of published (or printed) dictionaries represent an important link in the unbroken chain of lexicographic influences. Only a thorough analysis of manuscript dictionaries can lead to a better evaluation of the overall lexicographic production of individual cultures \citep{Dubois1970}, both vertically within historical lexicographic movements in a particular linguistic and cultural community and horizontally at the level of mutual influences of European cultures and languages. Only if manuscript dictionaries stand alongside the published ones will it be possible to better highlight the mutual influences of models that exist in lexicography (\citealt{Landau2001, Rey2008, Samardzija2013}), so each dictionary can be considered as a sum of dictionaries that precede it and a model for dictionaries that follow.}

\IfFileExists{../localcommands.tex}{
  \addbibresource{../localbibliography.bib}
  % add all extra packages you need to load to this file

\usepackage{tabularx,multicol}
\usepackage{url}
\urlstyle{same}

\usepackage{listings}
\lstset{basicstyle=\ttfamily,tabsize=2,breaklines=true}

\usepackage{langsci-basic}
\usepackage{langsci-optional}
\usepackage{langsci-lgr}
\usepackage{langsci-osl}
% \usepackage{./langsci/styles/langsci-lgr}
% \usepackage{./langsci/styles/langsci-osl}
% \usepackage{langsci-gb4e}

\usepackage{tikz}
\usetikzlibrary{patterns,calc}
\pgfdeclarepatternformonly{south east lines}{\pgfqpoint{-0pt}{-0pt}}{\pgfqpoint{3pt}{3pt}}{\pgfqpoint{3pt}{3pt}}{
    \pgfsetlinewidth{0.6pt}
    \pgfpathmoveto{\pgfqpoint{0pt}{3pt}}
    \pgfpathlineto{\pgfqpoint{3pt}{0pt}}
    \pgfpathmoveto{\pgfqpoint{.2pt}{-.2pt}}
    \pgfpathlineto{\pgfqpoint{-.2pt}{.2pt}}
    \pgfpathmoveto{\pgfqpoint{3.2pt}{2.8pt}}
    \pgfpathlineto{\pgfqpoint{2.8pt}{3.2pt}}
    \pgfusepath{stroke}}
    
\usepackage{stmaryrd}
\usepackage{wasysym}
\usepackage{multirow}
\usepackage{caption}
\usepackage{subcaption}
\usepackage{mathrsfs}
\usepackage{qtree}

\usepackage{linguex}


  %pminos do not split footnotes
% \interfootnotelinepenalty=10000 %Footnote in Laporte chapters has to be split SN


%\DeclareIndexNameFormat{default}{%
%\nameparts{#1}%
%\usebibmacro{index:name}%
%{\index[names]}%
%{\namepartfamily}%
%{\namepartgiveni}%
% {}% L1
% {}% L2
%{\namepartprefix}% generates spurious space L3
%{\namepartsuffix}% generates spurious space L4
%}

%  {\DeclareIndexNameFormat{default}{%
%     \usebibmacro{index:name}{\index[names]}{#1}{#3}{#5}{#7}}}

%\DeclareIndexNameFormat{default}{%
%  \usebibmacro{index:name}{\sindex[nom]}{#1}{#3}{#5}{#7}}

%\DeclareIndexNameFormat{default}{%
%  \usebibmacro{index:name}{\sindex[person]}{#1}{#3}{#5}{#7}}
%\DeclareIndexNameFormat{default}{%
%\nameparts{#1} \usebibmacro{index:name}{\sindex[person]]}{\namepartfamily}{‌​\namepartgiven}{\nam‌​epartprefix}{\namepa‌​rtsuffix}}

%\newcommand{\smiley}{:)}

%\renewbibmacro*{index:name}[5]{%
%\usebibmacro{index:entry}{#1}%
%{\iffieldundef{usera}{}{\thefield{usera}\actualoperator}\mkbibindexname{#2}{#3}{#4}{#5}}}

% \newcommand{\noop}[1]{}

%remove for final
%\overfullrule=1mm

\newcommand{\tobi}[2]}}
\renewcommand{\S}[1]{\tobi{#1}{\textsc{*}}}

% this volume references
% puts: [this volume]
% already defined: \citetv
%\newcommand{\citepv}[1]{(\citeauthor{#1} \citeyear*{#1} [this volume])}
\newcommand{\citealtv}[1]{\citeauthor{#1} \citeyear*{#1} [this volume]}

%parentheses around example number
\newcommand{\pref}[1]{(\ref{#1})}

% in-text examples

\newcommand{\lnex}[1]{\textit{#1}} %target lang word
\newcommand{\lnlit}[1]{(lit.: `#1')} %literal reading
\newcommand{\lnlat}[1]{(#1)} % latinization
\newcommand{\lntrans}[1]{`#1'} %translation
\newcommand{\lnexl}[2]%
{\lnex{#1}{} \lnlat{#2}} % ex with latinization
\newcommand{\lnexlat}[3]{\lnex{#1}{} \lnlat{#2}{} \lntrans{#3}} % ex with latinization and tranl.

%ch01
\newcommand{\co}[1]{\mbox{\textbf{#1}}}

%ch09

\newcommand{\cyrbulg}[1]{\begin{otherlanguage*}{bulgarian}#1\end{otherlanguage*}}


%ch10
\newcommand{\nlp}{{\small NLP}}
\newcommand{\mwe}{{\small MWE}}
\newcommand{\rae}{{\small RAE}}
\newcommand{\lvc}{{\small LVC}}
\newcommand{\pos}{{\small P}o{\small S}}
%\newcommand{\todo}[1]{ \textcolor{red}{#1} }

%\renewcommand{\labelenumi}{\theenumi}
%\ainamefmt{{vv}{ll}{, ff}{, jj}} % fullname

\newcommand{\biberror}[1]{{\color{red}#1}}

\newcommand{\osenovaitem}{--~}
  %% hyphenation points for line breaks
%% Normally, automatic hyphenation in LaTeX is very good
%% If a word is mis-hyphenated, add it to this file
%%
%% add information to TeX file before \begin{document} with:
%% %% hyphenation points for line breaks
%% Normally, automatic hyphenation in LaTeX is very good
%% If a word is mis-hyphenated, add it to this file
%%
%% add information to TeX file before \begin{document} with:
%% %% hyphenation points for line breaks
%% Normally, automatic hyphenation in LaTeX is very good
%% If a word is mis-hyphenated, add it to this file
%%
%% add information to TeX file before \begin{document} with:
%% \include{localhyphenation}
\hyphenation{
    Beck-man
    Ngu-yen
    back-chan-nel
    back-chan-nels
    mo-not-o-nous
    ste-reo-typ-i-cal
}

\hyphenation{
    Beck-man
    Ngu-yen
    back-chan-nel
    back-chan-nels
    mo-not-o-nous
    ste-reo-typ-i-cal
}

\hyphenation{
    Beck-man
    Ngu-yen
    back-chan-nel
    back-chan-nels
    mo-not-o-nous
    ste-reo-typ-i-cal
}

  \togglepaper[8]%%chapternumber
}{}

\begin{document} 
\begin{otherlanguage}{french}
\maketitle

\section{Introduction}
Ouvrages fondamentaux créés par le travail lexicographique, les dictionnaires sont traditionnellement considérés comme des livres qui «~contiennent une liste de mots d’une langue avec des définitions de leur sens ou avec des équivalents dans une langue étrangère~» \citep[10]{Samardzija2019}. Grâce au développement du numérique et compte tenu des besoins de plus en plus croissants de la communication entre langues et cultures, différents enjeux méthodologiques se posent lors de la confection de dictionnaires. C’est pour cette raison que, de nos jours, les dictionnaires sont plutôt conçus comme 

\begin{quote}
    une collection d’éléments lexicaux présentés par certains médias, dans lesquels des informations sont données et qui doivent être organisées de manière à ce que l’utilisateur y ait un accès rapide et facile. (\citealt[11]{Samardzija2019}, notre traduction)
\end{quote}

Jean (\citealt[44]{Dubois1970}Jean Dubois parle du dictionnaire comme d’un «~trésor~» dans lequel se trouvent des mots inconnus et avec lequel chaque individu «~mesure~» son trésor linguistique et le compare au trésor de sa communauté linguistique~; néanmoins, bien que le dictionnaire ne soit pas un moyen d’expression, il représente la norme ou la mesure de la richesse ou de la pauvreté d’un énoncé. 

Il existe différentes typologies de dictionnaires, plusieurs auteurs abordent les critères de leur différenciation et leurs spécificités dont nous ne soulignons que \citet{Zgusta1971}, puis \citet{Landau2001}, \citet{Dubois1970} et surtout \citet{Rey2008} et \citet{Pruvost2021}. Plus récemment dans le contexte croate, il convient de mentionner l’ouvrage de \citet{Samardzija2019}.

Un dictionnaire bilingue représente un type particulier de dictionnaire dans lequel les unités lexicales de deux différents systèmes linguistiques sont juxtaposées :

\begin{quote}
    Les dictionnaires bilingues se contentent le plus souvent de donner le terme équivalent dans une seconde langue, ce qui implique que l’utilisateur ait une certaine connaissance du système lexical de celle-ci. (\citealt[32]{Collignon1978})
\end{quote} 

Selon Landau, le dictionnaire bilingue comprend deux fonctions~: 

\begin{quote}
    for comprehension, as in reading, of the source language, by a person who knows the source language~; or as an aid in expression, as in writing, of the target language, by a person who knows the target language. (\citealt[9]{Landau2001})
\end{quote} 

Le même auteur souligne le fait que parler de l’étendue d’un dictionnaire n’est pas chose facile, le lexique étant une catégorie ouverte. (\citealt{Landau2001} ~: 7)

Les dictionnaires servent d’habitude de «~pont interculturel~» (\citealt{Rey2007}), ils sont des moyens précieux de communication et de compréhension interculturelles (\citealt{Wierzbicka1988}:  selon \citealt{Tallarico2013, Murano2013b, Vaxelaire2005, Tallarico2013}). Les dictionnaires peuvent également contenir des données encyclopédiques hautement culturelles et rélèvent de l’histoire culturelle \citep{Mestrovic1994} ou de la \emph{culture partagée} \citep{Galisson1988}. Les dictionnaires fournissent des images linguistiques qui témoignent de l’essence linguistique de chacune des deux langues \citep{Mestrovic1994} et des représentations des deux langues et deux cultures. Le dictionnaire a alors aussi une fonction symbolique car il devient :

\begin{quote}
    instrument de reconnaissance identitaire et égalitaire qui témoigne de la culture de l’Autre et amène à le connaître et à le reconnaître. (\citealt{Frey2007} ~: 209)
\end{quote}

Les dictionnaires bilingues ont aussi la particularité d’aborder inévitablement le champ des spécificités culturelles~: ils regroupent bien souvent des notions propres à certaines cultures et qui n’ont pas d’équivalents univoques dans la langue cible (coutumes, croyances, création artistique, artisanat, alimentation, habitat, etc.). Cela inclut une compréhension très variable et culturellement coloriée de certaines parties de la réalité extralinguistique (faune, flore) \citep{Rey1991}.
C’est la finalité du dictionnaire qui détermine aussi bien son contenu que sa structure, comme le prétendent \citep[12]{Collignon1978}. Gabrić-Bagarić soutient que dans le cas du dictionnaire conçu par un auteur, qui sélectionne les entrées indépendamment du corpus, la portée du dictionnaire est conditionnée par le but et le mode de création \citep[19]{Gabric_bagaric2010}.

\section{Débuts de l’activité lexicographique}

Les débuts de l’activité lexicographique sont généralement liés à l’histoire ancienne, d’où proviennent des listes plus ou moins longues de mots moins connus (mots obsolètes, dialectaux ou étrangers), ou glossaires. \citep[9]{Samardzija2019}. Le regain d’intérêt tant pour les langues classiques que pour les langues vivantes grandit à la Renaissance, ce qui donne un nouvel élan à la création d’œuvres similaires. L’invention de l’imprimerie marque sans doute un tournant important quant à l’élaboration de ces «~listes de mots~», si bien que les dictionnaires – d’abord mis en annexe puis volumes indépendants ou autonomes – acquièrent peu à peu l’attribut de livres prestigieux de l’époque. Ainsi, depuis le 15ème siècle, de nombreux dictionnaires de langues classiques et modernes ont été publiés sous des titres différents.

A la différence du dictionnaire élaboré avant la période de l’imprimerie, comprenant pour la plupart des explications de mots inconnus ou peu connus, la tâche particulière du dictionnaire à l’époque de la Renaissance, est de relier les contextes culturels, de permettre la communication entre les individus et surtout de rapprocher les oeuvres littéraires aux lecteurs \citep{Samardzija2019}. Un siècle plus tard, la Réforme catholique va radicalement changer la pratique lexicographique ainsi que la finalité des dictionnaires, qui ont depuis lors une part active en tant que supports au service pastoral. \citep[42]{Katicic1981} C’est ainsi que les premiers manuels, dictionnaires et grammaires sont confectionnés justement pendant cette période. En ce temps-là, c’est par rapport au latin que les langues «~se mesurent~», l’objectif étant de répondre, en premier lieu, aux besoins éducatifs des élèves et des missionnaires ainsi que de tous ceux qui ont besoin d'une bonne maîtrise du latin. De fait, le latin représentait à l’époque la «~langue littéraire de l’érudition baroque~» \citep[13]{Katicic2013}, toute cette époque étant, selon Katičić, la période de l’ «~érudition vivante~» où naissent l’historiographie critique, la lexicographie et la grammaire, l’époque de l’émergence de l’ensemble des sciences humaines \citep{Katicic2013}.\footnote{Les activités de la Compagnie de Jésus (S. I.) portaient à l’époque, entre autres, sur la rédaction et l’élaboration de grammaires et manuels d’apprentissage du latin en combinaison avec les langues de différentes populations (vernaculaires), entre autres le croate, afin de répandre son enseignement parmi le peuple.}  Le latin était à la fois la langue de la liturgie, mais aussi des documents officiels, de la science et de l’alphabétisation, tandis que le droit canonique, partout reconnu, influençait fortement les codes civils \citep[19]{Krasic2009}. Les grammaires et les dictionnaires représentaient donc, à cette époque-là, des ouvrages indispensables, une sorte d’outil linguistique qui aide à la maîtrise du latin.

En ce sens, un bon dictionnaire bilingue, comportant de bons équivalents est particulièrement important~: on y répertorie et explique les mots latins. Ce répertoire de mots latins doit aider les utilisateurs à comprendre les mots et constructions latins. Si l’on ajoute à tout cela l’activité très fructueuse de nombreuses académies qui recensent les trésors lexicaux de nombreuses langues depuis la seconde moitié du 17ème siècle, on peut dire que la lexicographie acquiert alors le statut d’une véritable discipline. 


A cette époque, une attention particulière est portée aux grands dictionnaires nationaux qui revêtent une grande importance pour l'histoire linguistique et culturelle en général. Comme le souligne Matasović, les résultats les plus significatifs des efforts de normalisation des langues en Europe au 17ème et au 18ème siècle sont les dictionnaires des académies nationales~: Academia della Crusca (1611), Dictionnaire de l’Académie française (1694), de l’Académie espagnole (1726--1739) et de l’Académie russe (1789--1794). \citep[514]{Matasovic2013}

En ce qui concerne le rapport entre deux langues en combinaison lexicographique, très souvent dans les dictionnaires ayant la composante latine, on observe un certain déséquilibre. Plus précisément, les équivalents proposés dans d’autres langues ne sont pas toujours en mesure d’expliquer de manière simple ou en un seul mot (un synonyme ou une suite de synonymes) la signification du mot latin figurant dans l’entrée. 

Par conséquent, dans ce type de dictionnaire (bilingues, mais souvent trilingues), la ou les langues cibles sont vues à travers le miroir du latin \citep[34]{Furno2003}.

C’est ainsi que les lexicographes de l’époque, guidés par le latin comme langue de départ, mais aussi la langue prestigieuse autour de laquelle tout était mesuré, et en même temps désireux de préserver l’esprit et l’expression de la langue vernaculaire, ajustent soigneusement la langue cible au latin afin d’expliquer les mots latins de la manière la plus fiable possible, en fonction des besoins de la description lexicographique et de la finalité du dictionnaire. \citep[28--29]{Furno2003}.

Outre les grands dictionnaires imprimés bilingues et trilingues, les ouvrages lexicographiques qui sont alors créés sont très souvent manuscrits. 

Dans ce qui suit, nous allons d’abord identifier quelques particularités des œuvres lexicographiques manuscrites et examiner de plus près quelques dictionnaires manuscrits créés par des auteurs qui appartiennent à ce que nous désignerons de «~cercle lexicographique de Dubrovnik~» (cf. § 4).

\section{Dictionnaires manuscrits au sein de la lexicographie~: Leurs propriétés et leur(s) destin(s) }

Les dictionnaires manuscrits constituent des maillons importants dans l’histoire de la lexicographie. Bien que de par leur structure, leur forme, leur méthodologie et leur objectif ils diffèrent largement des dictionnaires imprimés, leur étude est indispensable pour une présentation complète des réalisations lexicographiques dans un contexte linguistique et historico-culturel.

Le destin d’un dictionnaire fut le plus souvent étroitement lié, au cours de l’histoire, au sort de son auteur, ou comme le soutient \citet[118]{Rey2008}, à des modèles analogues des dictionnaires antérieurs et notamment à l’idiolecte du lexicographe. En particulier, dans les dictionnaires manuscrits plus petits, les lexicographes s’appuient souvent sur leur intuition linguistique et répertorient des éléments dialectaux ainsi que des éléments de la langue parlée. Bien entendu, de telles œuvres refléteront également la créativité de l’auteur puisque l’auteur, en l’absence de bons équivalents en un mot, essaie souvent de trouver la meilleure expression (tournure ou explication) dans la langue cible.

On peut donc dire que le rôle du lexicographe dans la création d’un dictionnaire est très souvent crucial. En fait, l’auteur du dictionnaire «~tend à l’anonymat~» et s’identifie en même temps avec l’objet de sa création, à son dictionnaire (ainsi les dictionnaires portent souvent le nom de leurs auteurs), mais aussi à la langue décrite dans le dictionnaire (\citet[41]{Dubois1970}.

Les lexicographes ne sont pas seulement des locuteurs de cette langue et donc «~possesseurs~» de ce «~trésor commun~», ils sont aussi des médiateurs de ce trésor en direction des lecteurs (utilisateurs). Et comme les lexicographes sont guidés presque régulièrement par les modèles de dictionnaires de leurs précurseurs, on peut constater que leur tâche consiste non seulement à transmettre ce trésor à leurs lecteurs, mais également de le laisser comme un héritage et une marque authentique dans le temps. C’est ainsi que le dictionnaire devient un véritable pont qui relie des générations de locuteurs, séparées par des années ou des décennies, parfois par des siècles entiers, mais aussi réunies par l’identité linguistique et par l’utilisation du même idiome.  

Reprenons ici le constat de Collignon et Glatigny sur le rôle décisif de la destination du dictionnaire qui détermine son contenu ainsi que son organisation (cf. Introduction). C’est précisément une partie de la réponse à la question de savoir pourquoi, d’une part, certains dictionnaires ont vu le jour sous forme imprimée et pourquoi, d’autre part, certains dictionnaires sont restés en manuscrit et tombés dans l’oubli. Nous essaierons de fournir au moins une partie de réponses à partir de données dont nous disposons sur les dictionnaires manuscrits que nous avons étudiés.

A l’instar de tous les manuscrits, les dictionnaires manuscrits sont des œuvres écrits par la main de leurs auteurs, lexicographes. Tout dictionnaire manuscrit est donc unique~: il représente un texte authentique (ou le matériel lexical), ce texte n’est pas multiplié par des procédés de reproduction mécanique. En outre, le manuscrit, y compris le dictionnaire, a son propre format, nombre de lignes par page (souvent variable), style d’écriture, qui peut être rempli de caractères spéciaux d’auteur, d’abréviations, de corrections, d’interpolations, de notes en marge, etc. Il n’y a donc pas deux dictionnaires manuscrits identiques, tout dictionnaire manuscrit représente un recueil original~: de par la structure du répertoire, des déterminants, des types de définition, des manières de proposer l’équivalent. Cependant, cet «~autographe authentique~» témoigne des efforts des individus pour composer eux-mêmes de telles œuvres, en s’appuyant sur les modèles existants, souhaitant laisser leur propre trace ou interprétation des mots d’une autre langue.

Un dictionnaire peut rester sous forme de manuscrit pour différentes raisons : la finalité du dictionnaire, le caractère inachevé du dictionnaire, son caractère pratique ou tout simplement le fait que le manuscrit d’un dictionnaire n’est souvent que la transcription d’un autre dictionnaire (que l’auteur a consulté). Par ailleurs, le contexte social et économique, les circonstances culturelles et historiques ont dans la plupart des cas déterminé le sort du dictionnaire. Comme le soutient Rey~:

\begin{quote}
    la décision de publier un dictionnaire relève plus de facteurs économiques ou politiques que d’un noble souci didactique ou scientifique \citet[14]{Rey2008}.
\end{quote}

Il convient de mentionner ici le fait que de nombreux dictionnaires manuscrits et leurs différentes rédactions ont été perdus, puisqu’ils ont été écrits en un seul exemplaire. En tant que manuscrits, ils n’ont été utilisés que par un nombre bien restreint de lecteurs et leur utilisation était donc limitée.

Malgré leur faible potentiel de communication et leur réception assez limitée, nous pouvons supposer que ces répertoires manuscrits ont servi d’appui aux autres lexicographes, dans leurs recherches de modèle convenable, comme le souligne \citet[118]{Rey2008}~: 

\begin{quote}
    La lexicographie décrit habituellement un système lexical qu’elle définit a priori et qui correspond à un modèle élaboré selon des règles non formulées, subjectives et souvent variables~: intuition de l’unité lexicale, de son importance fonctionnelle, de sa valeur culturelle (dans le cas du purisme normatif). Ce modèle est défini en partie par référence à l’idiolecte du lexicographe, en partie par référence aux modèles analogues (les dictionnaires précédents) et en partie par référence à une sélection du corpus indéfini des phrases, c’est à dire un ensemble de faits de discours.
\end{quote}

En fait, aucun dictionnaire n’aurait pu être créé \emph{ex nihilo}, sans modèles et précurseurs, comme le prétend Landau~: 

\begin{quote}
The earliest English lexicographers by and large copied the definitions of their predecessors. \citep[190]{Landau2001}    
\end{quote}

C’est là que réside la plus grande valeur des dictionnaires manuscrits plus petits ou plus grands, plus pauvres ou plus riches. Sans doute est-il parfois difficile de prouver la chaîne des influences lexicographiques, notamment celles qui lient un dictionnaire plus ancien et un autre plus récent. Cependant, il existe de nombreux exemples qui nous permettent de déterminer assez facilement ces influences (dans \citet{Franic2014} nous avons expliqué plus en détail les rapports qui existent à l’intérieur d’un corpus qui a précédé la confection de \emph{Dictionarium Latino-Illiricum}, cf. § 4.2., ainsi que dans les dictionnaires qui ont été rédigés postérieurement à celui-ci).

En conséquence, si les œuvres manuscrites n’avaient pas un potentiel de communication égal à celui des œuvres imprimées – n’étant pas largement répandues ni connues – nous trouvons que, dans les circonstances de l’époque, elles étaient suffisamment disponibles aux lexicographes qui auraient pu les atteindre au moment où ils en avaient besoin. juste titre que les grands dictionnaires (imprimés) doivent leur naissance, au moins partiellement, entre autres, à un nombre d’œuvres manuscrites.

C’est ainsi que dans l’histoire de la lexicographie croate, de nombreux dictionnaires manuscrits, de portée plus ou moins grande, pour la plupart bilingues, mais aussi trilingues, ont marqué certaines périodes, comme l’estime Samardžija en ajoutant que ces dictionnaires représentent une source précieuse pour l’histoire de la langue littéraire croate~: 

\begin{quote}
En plus d’être un riche trésor du lexique croate, les dictionnaires croates de l’époque, tant ceux publiés que ceux qui nous sont parvenus en manuscrits à cause des circonstances (non) favorables, sont des sources précieuses pour l’histoire de la langue littéraire croate […]. (\citet[453]{Samardzija2013}, notre traduction) 
\end{quote}
 
Au cours des décennies passées, de nombreux auteurs ont réalisés des recherches dans ce domaine (\citealt{Gabric_bagaric1996,Gabric_bagaric1998,Gabric_bagaric2000,Gabric_bagaric2002, Vajs2003, Voncina1992,Voncina2003, Sironic_bonefacic1992, Putanec1991, Putanec2000, Mestrovic1995, Horvat1990}). Il convient de souligner tout particulièrement le \emph{Lexicon Latino-Illyricum} (1700--1709) de Pavao Ritter Vitezović, trois dictionnaires de Ivan Tanzlingher Zanotti~: \emph{Talijansko-ilirsko-latinski rječnik} (1699), \emph{Vocabolario di tre nobilissimi linguaggi, italiano, illirico e latino}\footnote{Le dictionnaire a été numérisé en 2008 (avec une autre édition en 2020) et peut être consulté sur le site \url{https://tanzlingher.disll.unipd.it/xml/index.php?\_lang=it}.} (1699) et \emph{Vocabolario italiano ed illirico} (1699) ainsi que \emph{Dictionarium latino-illyricum et germanicum} (1772--1779) de Adam Patačić. Etant donné que Vitezović et Tanzlingher Zanotti avaient une conception bien claire de la langue littéraire croate fondée sur trois dialectes~: shtokavien, kaïkavien et ikavien \citep[47--48]{Samardzija2019}, leurs dictionnaires s’avèrent être incontournables pour donner un aperçu tout d’abord d’un corpus intégral lexicographique et lexical de la langue croate et puis pour faire un bilan des contacts linguistiques du croate, à l’époque, avec les langues de communication, aussi bien générale que scientifique, les langues de l’art, avant tout le latin et l’italien.

Nous nous proposons d’examiner, dans la section suivante, un corpus lexicographique manuscrit qui d’ailleurs rejoint les grands dictionnaires imprimés de l’époque ayant pour base l’idiome de Dubrovnik – \emph{Blago jezika slovinskoga} (1649--1651) de Jakov Mikalja et \emph{Dizionario italiano-latino-illirico} (1728) de Ardelio Della Bella. Ce corpus manuscrit, par sa portée et son étendue s’inscrit par-là dans le patrimoine linguistique et culturel croate et européen. 

En conséquence, nous ne décrirons pas ici les grands dictionnaires imprimés du cercle lexicographique de Dubrovnik car ils sont déjà bien décrits dans de nombreuses études menées dans les décennies passées. En revanche, nous nous pencherons sur les dictionnaires manuscrits moins connus ou presque inconnus qui ont sans doute leur part dans la production lexicographique intégrale de Dubrovnik. 

\section{Corpus lexicographique manuscrit au sein du cercle lexicographique de Dubrovnik}

Le statut de Dubrovnik comme point de rencontre d’événements culturels et littéraires au cours des siècles, le prestige de l’idiome de Dubrovnik, mais aussi le besoin constant d’enseignement, notamment l’enseignement du latin et l’interprétation de ses spécificités ont favorablement influencé l’intérêt pour la réalisation de divers dictionnaires manuscrits. Ces œuvres provenaient principalement des auteurs qui maitrisaient bien le latin, des prêtres et des religieux de l’époque. Dans les couvents de Dubrovnik, cette activité était souvent accessoire, parallèle à la transcription et à l’interprétation d’œuvres religieuses, à la prédication, etc. De nombreux dictionnaires manuscrits, soit autonomes soit figurant dans l’annexe d’un livre, furent rédigés dans les bibliothèques de couvents de Dubrovnik, celui des Frères mineurs, des Dominicains et de Collegium Ragusinum Jésuite.

Ce corpus manuscrit témoigne du fait que les sept décennies qui divisent le dictionnaire \emph{Blago jezika slovinskoga} (1649--1651) de Mikalja et le \emph{Dizionario italiano-latino-illirico} (1728) de Ardelio Della Bella abritent une activité lexicographique très vive et surtout originale. Nous considérons que tous les dictionnaires, imprimés ou manuscrits, qu’ils furent rédigés à Dubrovnik ou que l’une de ses composantes fut l’idiome de Dubrovnik, forment le cercle lexicographique (bicentenaire) de Dubrovnik. 

Les auteurs d’œuvres lexicographiques plus importantes (publiés et imprimés) de l’époque ont sans doute puisé dans des recueils manuscrits, plus ou moins grands, dans différentes paires de langues ou combinaisons linguistiques, créés au sein du cercle lexicographique de Dubrovnik. Dès lors, si les dictionnaires manuscrits n’avaient pas le potentiel de communication des ouvrages imprimés, on ne peut pas leur nier une part quasi certaine dans l’émergence de prestigieux grands dictionnaires imprimés. 

Outre l’idiome de Dubrovnik en tant que langue au patrimoine linguistique et culturel très riche que Muljačić désigne de langue de Dubrovnik «~dubrovački jezik~»\footnote{Ensemble de réalisations écrites de la langue littéraire de Dubrovnik \citep{Muljacic2001}.},  nous nous intéressons également à la langue vivante (vernaculaire), ce parler que les lexicographes ont enregistré dans leurs manuscrits et qui est utilisé dans sa continuité historique à Dubrovnik. Dans ses études récentes sur l’idiome de Dubrovnik, \citet{Lovric_jovic2014} prétend que cet idiome comprend la «~langue littéraire de Dubrovnik~» et le «~parler de Dubrovnik~». Tout cela pris en compte, nous mettrons en avant, dans notre analyse des dictionnaires manuscrits, un corpus représentatif d’unités lexicales qui relèvent de ces deux groupes. Ces unités sont encore appelées les «~mots ragusains~» ou «~raguséismes~» \citep{Gabric_bagaric2002}.

\subsection{Dictionnaire croate-italien (1597) de Bartol Kašić}

C’est le Dictionnaire croate-italien de Kašić, rédigé entre 1597 et 1599, qui témoigne d’une activité lexicographique bien vive à Dubrovnik. Le manuscrit, conservé à la Bibliothèque des Frères mineurs à Dubrovnik, fut étudié et publié en 1990 par père Vladimir Horvat, SI.\footnote{L’ouvrage s’intitule \emph{Hrvatsko-talijanski rječnik} ‘Dictionnaire croate-italien’, son \emph{editio princeps} fut préparée par père Vladimir Horvat, SI, publié dans les éditions de Kršćanska sadašnjost et de l’Institut de la langue croate \citep{Horvat1990}.}  \emph{Dizionario illirico-dalmatino italiano – Slovoslovje dalmatinsko italijansko}, ou le dictionnaire chakavien, comme le désigne  \citet{Mitrovic1909} est souvent associé, dans un nombre d’études, au rapport du Ragusain Marin Temperica, fait à l’intention de général Acquaviva, qui ordonna à Kašić de rédiger une grammaire.\footnote{Comme le prétend \citet{Putanec2000}, Temperica propose, en 1582, de confectionner un dictionnaire «~croate~».}  Comme le prétend Horvat, Kašić avait déjà achevé son dictionnaire au moment du rapport de Temperica. \citep[XV]{Horvat1990}. De toute évidence, vers les années 1595, Kašić a rencontré Faust Vrančić à Rome qui, à ce moment-là, avait son \emph{Dictionarium quinque nobilissimarum Europae linguarum} (1595) avec lui, imprimé à Venise~; Kašić l’a probablement utilisé lors de la création de son dictionnaire au moyen du procédé dit de «~renversement~» \citep[97]{Horvat1999}.

Par ailleurs, le nombre de lexèmes empruntés à Vrančić, environ 3 000 mots selon \citet{Gabric_bagaric1990} et \citet[98]{Horvat1999}, parle en faveur de la base chakavienne-ikavienne du dictionnaire, avec un nombre significatif de raguséismes. Selon \citet{Gabric_bagaric1990}, les lexèmes les plus nombreux relèvent d’agronomie et d’élevage du bétail (y compris les noms d’animaux), ainsi que du quotidien. On y trouve également le vocabulaire en lien avec l’église et la liturgie, ainsi qu’un nombre significatif du vocabulaire marin et gastronomique.

Le \emph{Dictionnaire croate-italien} était la source pour les dictionnaires qui ont suivi, surtout pour le \emph{Dizionario} (1728) de Della Bella et \emph{Rječosložje} (1806) de J. Stulli.  

Le \emph{Dictionnaire} de Kašić témoigne du fait que l’auteur lexicographe, dans sa tentative de rédiger le dictionnaire ayant le croate dans la colonne de gauche, s’appuie sur le Dictionnaire de Vrančić, le prend pour son point de départ et y ajoute une partie importante de son fonds lexical. Une orientation claire sur la base du dialecte shtokavien rend ce dictionnaire encore plus intéressant car il témoigne en outre du fait que dans ce corpus, les éléments chakavien et shtokavien se trouvent côte à côte, ce qui fait de ce dictionnaire un lien entre les régions croates du nord et du sud.

\subsection{Dictionarium Latino-Illiricum (1715--1716) de Đuro Matijašević  }

Ce dictionnaire constitue un des maillons les plus importants de la chaîne lexicographique de Dubrovnik. C’est la première fois qu’un locuteur natif fait un recueil lexicographique de l’idiome de Dubrovnik. En plus, c’est la seule trace du projet lexicographique inscrit au programme de l’Académie des Otieux (\emph{Academia degli Oziosi Eruditi}) fondée à Dubrovnik vers 1690 avec une attention particulière axée sur les questions de langue littéraire et d’orthographe, ainsi que sur le développement culturel général (les travaux des membres de l’Académie portaient également sur les thèmes d’histoire, en particulier la tradition populaire). Le début du 18ème siècle à Dubrovnik étant marqué par un retard général dans la création littéraire (et linguistique), on peut constater l’impossibilité de dresser un véritable portrait linguistique de la ville-république. Ainsi le \emph{Dictionarium} vient-il combler le vide créé dans le travail lexicographique de Dubrovnik en tant que centre culturel de l’époque. 

Ce dictionnaire fait partie du Codex numéro 194 intitulé \emph{Razlika skladanja slovinska} (abritant également le Dictionnaire croate-italien de Kašić), qui est conservé à la Bibliothèque du Couvent des Frères mineurs à Dubrovnik.\footnote{Ce Codex manuscrit fut décrit par P. Mitrović en 1909, dans son étude \emph{Četiri nepoznata dubrovačka rječnika} ‘Quatre dictionnaires ragusains inconnus’. Outre le \emph{Dictionarium} de Matijašević, le codex abrite le \emph{Vocabolario italiano-illirico} de L. Cekinić, (cf. § 4.3.), le \emph{Dictionnaire croate-italien} de B. Kašić (cf. § 4.1.), et un quatrième dictionnaire italien-slovène dont l’auteur est inconnu.}  Les Figures 1 et 2 montrent la page de titre ainsi qu’une page choisie du \emph{Dictionarium}.

\begin{figure}
\includegraphics[width=.75\textwidth]{images/Franic_figure1.jpg}
\caption{\emph{Dictionarium Latino-Illiricum} (1715--1716) de Đuro Matijašević, page de titre, manuscrit 194 conservé à la Bibliothèque du Couvent des Frères mineurs à Dubrovnik}
\label{Figure 1}
\end{figure}

\begin{figure}
\includegraphics[width=.75\textwidth]{images/Franic_figure2.jpg}
\caption{\emph{Dictionarium Latino-Illiricum} (1715--1716) de Đuro Matijašević, page 18, manuscrit 194 conservé à la Bibliothèque du Couvent des Frères mineurs à Dubrovnik}
\label{Figure 2}
\end{figure}

Le latin figure dans la colonne de gauche et l’idiome croate ragusain dans la colonne de droite. Le principe lexicographique est explicatif, il y a beaucoup de synonymes de contact~; la colonne de droite comporte bien souvent des définitions ou formules explicatives et la composante syntaxique est assez riche. Pour le lexique, on peut noter les influences significatives de langues romanes (italien) ainsi qu’un nombre de lexèmes d’origine dalmate. 

Le \emph{Dictionarium} représente un recueil d’environ 13 000 entrées, il s’agit de la combinaison d’un dictionnaire général explicatif et d’un dictionnaire général à forte démarche traductive avec des éléments encyclopédiques et lexicaux. Dans l’orthographe, Matijašević s’appuie largement sur les solutions de ses précurseurs (R. Džamanjić et B. Kašić), mais introduit également quelques améliorations intéressantes~: le phonème /ӡ/ sous forme du digramme zſ, et le phonème syllabique /r/ sous forme du digramme ær.

Dans sa méthode lexicographique, Matijašević s’inspire considérablement de Mikalja. Il convient de mentionner le caractère descriptif de la colonne de droite \citep{BilicFranic2007}, qu’elle soit constituée d’un seul ou plusieurs équivalents (synonymes), ou bien des éléments encyclopédiques~:
\begin{itemize}
    \item équivalent en un mot, ex. \emph{primam} ‘je reçois’ s. v. \emph{accipio}~;
    \item bloc syntagmatique, ex. \emph{cinim gorjet} ‘je fais brûler’ s. v. \emph{igneo} ou locution \emph{npr}. \emph{mala snaga} ‘manque de force’ s. v. anxietas~;
    \item série de deux ou trois synonymes, ex. \emph{kantun, nugao} ‘coin’ s. v. \emph{angulus}, \emph{gredem, idem, hodim} ‘je marche’ s. v. eo~;
    \item équivalent d’un mot + définition, ex. \emph{mencetta, sgradda tvarda} ‘bâtiment fort’ s. v. \emph{aro~}; \emph{ʃcilla ʃcivotna, duscnik} ‘larynx’ s. v. \emph{arteria~};
    \item définition à deux ou plusieurs éléments~: ex. \emph{cistim klacinatu, i sovarnu od ʃida staroga} ‘j’enlève les décombres’ s. v. \emph{erudero}.
\end{itemize} 

Il convient de souligner ici une autre structure, notamment la structure «~A de B ou de + génitif~» (ou expression de la possession par le génitif, selon \citet{Gabric_bagaric2002}, ayant différentes fonctions~: attribut, origine, qualité, apposition, CC de but. En voici quelques exemples~:
\begin{itemize}
    \item \emph{Ambrices}. Kupe od kucchje ‘tuiles de la maison’ (attribut)
    \item \emph{Laurinum}. Ulje od lovorikke ‘huile de laurier’ (origine)
    \item \emph{Litus}. Kraj od morra ‘bord de la mer’ (qualité)
    \item\emph{Boreas}. Vjettar od burré ‘vent de bura (nord)’
    \item\emph{Epidaurus}. Stari Grad od Zaptata ‘vieille ville de Cavtat’ (apposition)
    \item\emph{Cretera}. Romjenca od vode ‘pot à l’eau’ (CC de but)
\end{itemize}

Citons également la présence de nombreux éléments encyclopédiques, ce qui ne surprend pas, vu l’objectif de l’auteur d’expliquer un certain nombre d’éléments encyclopédiques~:

\begin{itemize}
    \item \emph{\textbf{Scabellum.} Podnoʃcje, illi ono na cem Biskup klecí, alli Kneʃ} ‘plancher ou pied où l’évêque s’agenouille ou le prince’
    \item \emph{\textbf{Septentrio.} Strana od svjeta kudse obratisc put istoka kojati na ljevu ruku ostanne} ‘le côté quand tu tournes vers l’Est puis le côté qui reste sur ta main gauche’
    \item \emph{\textbf{Styx.} Jedan kladenaƷ vode u Arkadii od tolike studeni svaki koijese napie od tolike studeni svake umre illiti ʃcivina, illi covjek, i svakí sud puƷza od gne} ‘une source en Arcadie tellement froide que chacun qui en boit cette eau froide meurt, soit animal, soit humain, et tout récipient en est cassé’
    \item \emph{\textbf{Tigris.} ʃcivina bærʃa kako strjela i rjeka koja istjece iʃ Raja ʃemaglskoga} ‘animal rapide comme une flèche et la rivière qui prend sa source dans le Paradis terrestre’
\end{itemize}

Pour ce qui est de la structure du lexique, \emph{Dictionarium} comprend de nombreux éléments alloglottes (italiens et dalmates) ainsi qu’un nombre impressionnant de mots nouveaux, fruit de la créativité de l’auteur.

Le vocabulaire de spécialité semble assez riche, si l’on considère les phytonymes par exemple. De cette catégorie nous avons choisi une dizaine d’éléments qui d’ailleurs font partie du lexique habituel de nos jours à Dubrovnik, et dont l’origine est clairement romane (soit les emprunts à l’italien, soit au dalmate).

\begin{itemize}
    \item \emph{bosilak} s. v. acapnos ‘basilic’
    \item \emph{cempres} s. v. cuprenus ‘cyprès’
    \item \emph{cesvinna} s. v. ilex ‘houx’
    \item \emph{dub cedar} s. v.  citrus ‘cèdre’
    \item \emph{fjerla} s. v. ferula ‘férule’
    \item \emph{garofao} s. v. gariophilum ‘girofle’
    \item \emph{kukumar} s. v. cucumer ‘concombre’
    \item \emph{lovorika} s. v. laurus ‘laurier’
    \item \emph{mjendeo} s. v. amigdala ‘amande’
    \item \emph{morac} s. v. fęniculum ‘fenouil’
    \item \emph{mærkatugna} s. v. cotoneum ‘coing’
    \item \emph{smokva} s. v. ficus ‘figue’
    \item \emph{sparoga} s. v. asparagus ‘asperge’
    \item \emph{smokva suha} s. v. carica ‘figue sèche’
    \item \emph{ʃcukka} s. v. juncus ‘jonc’
    \item \emph{Ʒærnika} s. v. ilex ‘houx’
\end{itemize}

Dans ce qui suit, nous présentons un répertoire choisi de lexèmes qui tirent leur origine du dalmate, groupe de parlers romans autochtones répandus le long de la côte adriatique, dont les influences persistent de nos jours en croate, surtout au niveau lexical. En voici des attestations recueillies par Matijašević dans le \emph{Dictionarium} (pour l’explication des origines de mots, cf. \citealt[146]{Franic2014})~:

\begin{itemize}
    \item \emph{arkuo} s. v. lecythus ‘flacon’
    \item \emph{bumbak} s. v. bambacium ‘coton’
    \item \emph{fjerla} s. v. ferula ‘férule’
    \item \emph{kacuo} s. v. cacabus ‘pot, chaudron’
    \item \emph{kelomna} s. v. basis ‘bas de pilier’
    \item \emph{kimak} s. v. cimex ‘punaise’
    \item \emph{klacina} s. v. calcaria ‘four à chaux’
    \item \emph{kupjerta} s. v. doma ‘toit’
    \item \emph{lukjerna} s. v. lucerna ‘lucerne, lanterne’
    \item \emph{mjendeo} s. v. amigdala ‘amande’
    \item \emph{pikat} s. v. ficatum ‘figue’ 
    \item \emph{plakjeri} [cas nominatif, pluriel] s. v. apkrodisia ‘plaisir’
    \item \emph{prigam} s. v. frigo ‘frire’
    \item \emph{rusa} s. v. rosa ‘rose’ 
\end{itemize}

L’auteur emploie fréquemment le verbe \emph{činiti} ‘faire’ qui peut servir de semi-auxiliaire comme dans l’exemple \emph{cinim rat} s. v. belligero ‘je fais la guerre’ ou \emph{cinim gosbu} s. v. convivor ‘je fais la fête’. Ce même verbe peut introduire la diathèse causative comme dans l’exemple \emph{ranu činju} s. v. exulcero ‘je provoque une blessure’ ou \emph{cinim gorjet} s. v. igneo ‘je fais brûler’.

L’analyse du lexique du \emph{Dictionarium}, faite dans \citet{Franic2014} montre qu’il existe un nombre relativement significatif de lexèmes que Matijašević emprunte au \emph{Blago} de Mikalja, cependant, on observe dans le \emph{Dictionarium} un nombre important de lexèmes qui ne figurent pas dans le \emph{Blago} de Mikalja. Par ailleurs, Della Bella a surement dû prendre en compte le lexique de Matijašević lors de la confection de son \emph{Dizionario}. On trouve également un nombre de lexèmes qui ne sont pas répertoriés chez Della Bella. Néanmoins, il existe un répertoire général significatif qui est commun aux \emph{Dictionarium} (Matijašević), \emph{Blago} (Mikalja) et \emph{Dizionario} (Della Bella).   

D’une part, le \emph{Dictionarium} regroupe de nombreuses attestations lexicographiques d’un nombre de lexèmes ragusains qui ne figurent pas dans le \emph{Blago} et le \emph{Dizionario}~: \emph{art} ‘art’, \emph{bankjer} ‘banquier’, \emph{barun} ‘baron’, \emph{batessa}  'abbesse', \emph{duplavam} ‘je double’, \emph{febra} ‘fièvre’, \emph{kalamar} ‘calamar’, \emph{kantun} ‘coin’, \emph{klacinata} ‘chaux’, \emph{konao} ‘canal’, \emph{kortigl} ‘cour’, \emph{kukumar} ‘concombre’, \emph{kupjerta} ‘toit’, \emph{libro} ‘livre’, \emph{mencetta} ‘bâtiment fort’, \emph{merit} ‘mérite’, \emph{modeo} ‘modèle’, \emph{segnavam} ‘je fais signe’, \emph{tenero} ‘tendre’, \emph{lukjerna} ‘lucerne, lanterne’.

D’autre part, le \emph{Dictionarium} comprend de nombreux éléments alloglottes (italiens et dalmates) ainsi que des premières attestations des lexèmes ragusains typiques~: \emph{bankjer} ‘banquier’, \emph{ciret} ‘onguent obtenue en mélangeant l’huile et la cire’, \emph{kamelja} ‘chameau’, \emph{klacinata} ‘chaux’, \emph{romjenciza} ‘petit pot’, \emph{segnavam} ‘je fais signe’, \emph{moskar} ‘éventail’. Si on le compare avec le dictionnaire \emph{Rječosložje} (1806) de Stulli,\footnote{\emph{Rječosložje iliričko-italijansko-latinsko} (1806).}  publié presqu’un siècle après le \emph{Dictionarium}, on peut supposer que Stulli aurait pu emprunter au \emph{Dictionarium} les raguséismes \emph{kacuo} ‘pot, chaudron’, \emph{duplavati} ‘doubler’ ou certains autres, ayant des formes intéressantes~: a) substantifs \emph{okuscaj} ‘goût’, \emph{samos} ‘solitude’, \emph{skaʃan} ‘exemple’, \emph{skupgljevina} ‘pile, amas’, \emph{vacellanje} ‘hallucination’, \emph{vosctanizza} ‘chandelle’, et b) verbes \emph{bludnujem} ‘je suis surabondant’, \emph{kakochjem} ‘je caquette’, \emph{vacellam} ‘j’hallucine’, \emph{spjegnam} ‘j’écume’ (cf. \citealt{Franic2014}).

Désireux de proposer des équivalents en un mot pour les termes latins, l’auteur créé des formes «~pures~» en s’appuyant sur la langue littéraire de Dubrovnik, mais aussi sur sa propre intuition~: ainsi propose-t-il plusieurs néologismes, dont nous citerons des substantifs \emph{nomina agentis}~: \emph{bdilaƷ} ‘celui qui veille’, \emph{ctioz} ‘lecteur’, \emph{daronosca} ‘celui qui apporte des cadeaux’, \emph{duggoʃcivaƷ} ‘celui qui vit longtemps’, \emph{iʃjelaƷ} ‘celui qui mange beaucoup’, \emph{iʃostalaƷ} ‘celui qui est absent’, \emph{iʃgovaralaƷ} ‘celui qui dit quelque chose’, \emph{kuhaciƷza} ‘cuisinière’, \emph{kupilaƷ} ‘acheteur’, \emph{velegovorilaƷ} ‘qui parle beaucoup’, \emph{tesckogovorilaƷ} ‘celui qui parle avec difficulté’, \emph{vinonos} ‘qui porte du vin’. 

\largerpage
Sur l’exemple des vocabulaires de spécialité en rhétorique, philosophie et philologie on observe les efforts de l’auteur de proposer des explications originales de plusieurs notions abstraites~: 

\begin{itemize}
    \item \emph{spodobnos, kad jedno drugomu odgovara} ‘ressemblance, quand une chose correspond à une autre’ s. v. analogia; 
    \item \emph{reccenja u kratko nu spametno} ‘mots courts mais intelligents’ s. v. dicteria; 
    \item \emph{govorilaƷ koga se govor na dva nacina raʃumje} ‘locuteur dont les paroles sont compris de deux manières’ s. v. flexiloquus;
    \item \emph{ʃlamenje u pismu na nacin male ʃvjeʃdizze} ‘signe écrit à la façon d’une petite étoile’ s. v. asteriscus; 
    \item \emph{pismo od svoje ruke} ‘écriture de sa propre main’ s. v. autographum; 
    \item \emph{tko ispiscuje, alli iʃ jednoga pisma na druggo stavglja} ‘qui écrit ou transmet d’une écriture à une autre’ s. v. ascriptor;
    \item \emph{art ʃa dobro govorit i pisat} ‘art de bien parler et écrire’ s. v. grammatica. 
\end{itemize}

Les résultats de l’analyse du lexique montrent que le \emph{Dictionarium} de Matijašević, bien qu’il soit resté manuscrit, a servi de modèle ou de source pour d’autres manuels lexicographiques, en particulier les dictionnaires imprimés, surtout ceux de Della Bella et de Stulli. En outre, ce dictionnaire est unique en ce qui concerne les efforts de l’auteur en matière de pureté linguistique, en d’autres termes, de purification du lexique figurant dans la colonne de droite des influences venant d’autres langues, surtout de l’italien.

\subsection{Vocabolario italiano-illirico (\~{}1745) de frère Lovro Cekinić}

Frère Lovro Cekinić (1692--1752), franciscain de l’Ordre des Frères mineurs, installé à Dubrovnik, collectionneur assidu d’œuvres littéraires et historiques, excellent prédicateur, est bien fréquemment mentionné dans les chroniques de l’époque. Les biographes en parlent souvent, pour la plupart ils font mention de ses activités dans la rédaction des chroniques, des transcriptions et de la prédication.\footnote{Cf. \citealt{Franic2007} et \citealt{Franic2008}.}  Bien que son activité lexicographique soit rarement mentionnée, c’est en 1909 que P. Mitrović décrit ses efforts en lexicographie.\footnote{Cf. \citealt[528--533]{Mitrovic1909}.  }

Le \emph{Vocabolario} figure parmi quatre dictionnaires manuscrits reliés dans le Codex \emph{Razlika skladanja slovinska}, conservé à la Bibliothèque des Frères mineurs. De par son étendue, il est assez petit~: il regroupe environ 5\,080 entrées, dont un nombre significatif de lacunes (absence d’équivalent), presque 600.\footnote{Il est fort probable que le travail sur la traduction de l’œuvre \emph{Duhovne zabave duše bogoljubne} fut la principale motivation pour la création du dictionnaire (vers 1745), ce dont nous fait part D. \citet[269]{Berić1956}.} .  Les Figures 3 et 4 montrent la page de titre du \emph{Vocabolario} ainsi qu’une page de dictionnaire~:

\begin{figure}
\includegraphics[height=.75\textheight]{images/Franic_figure3.jpg}
\caption{Vocabolario italiano-illirico (\~{}1745) de Lovro Cekinić, page de titre, manuscrit 194 conservé à la Bibliothèque des Frères mineurs de Dubrovnik}
\label{Figure 3}
\end{figure}

\begin{figure}
\includegraphics[height=.75\textheight]{images/Franic_figure4.jpg}
\caption{Vocabolario italiano-illirico (\~{}1745) de Lovro Cekinić, page 203, manuscrit 194 conservé à la Bibliothèque des Frères mineurs de Dubrovnik}
\label{Figure 4}
\end{figure}

Le \emph{Vocabolario} fut rédigé presqu’un siècle après le \emph{Blago} de Mikalja et quelques années après la parution de \emph{Dizionario} de Della Bella. En conséquence, on aurait pu s’attendre à une influence significative des deux grands dictionnaires sur le corpus lexical de Cekinić. Malgré l'existence de ces deux grands dictionnaires et leur riche fonds lexical, on peut observer un haut degré de créativité lexicale de Cekinić. En effet, un nombre relativement important de mots ont leur première attestation lexicographique dans le \emph{Vocabolario} (ces mots trouveront plus tard leur place dans le \emph{Rječosložje} de Stulli sans aucune mention de source), nous n’en citerons que quelques-uns~: \emph{boghoneuiernik} ‘infidèle’, \emph{duorkigniza} ‘demoiselle’, \emph{neinako} ‘non autrement’, \emph{neufati} ‘désespérer’ \emph{nochiascni} ‘nocturne’, \emph{odredba} ‘disposition’, \emph{okorusciati} ‘écorcer’, \emph{posliednizi} ‘descendants’, \emph{posnasgiti} ‘renforcer, encourager’, \emph{prasnoruk} ‘aux mains vides’, \emph{rasiediniti} ‘séparer’, \emph{raslauiti} ‘\emph{diluer}’, \emph{rasrok} ‘louche’, \emph{suietostuornik} ‘qui crée le monde’ (\citealt{Franic2007} et \citealt{Franic2008}).

Par ailleurs, on observe rarement, parmi les solutions lexicographiques de Cekinić, des lexèmes d’origine étrangère. Les emprunts à l’italien sont les plus nombreux, par exemple~:  \emph{aier} ‘air’ s. v. aere~; \emph{casctio} ‘forteresse’ s. v. fortezza, castello~; \emph{kollaina} ‘collier’ s. v. collana~; \emph{legatti} ‘lire’ s. v. leggere~; \emph{presciti} ‘se dépêcher’, être pressé’ s. v. affrettare, puis des descendants lexicaux (ou reliques) du dalmate~:  \emph{cersa} ‘blanchisseur’ s. v. biaxa~; \emph{kaipa} ‘cage’ s. v. gabbia~; \emph{pedepsa} ‘punition’ s. v. flagello~; \emph{podumienta} ‘fondement’ s. v. fondamento~; \emph{prigati} ‘frire’ s. v. lessare.

Malgré sa finalité pratique et une ampleur moins importante, le \emph{Vocabolario} reflète des tentatives de Cekinić de chercher de bons équivalents dans le vernaculaire. En effet, bien qu’il propose un emprunt, Cekinić ajoute à titre égal, son équivalent local, bien souvent un raguséisme~:  \emph{pouietarze} à côté de \emph{aier} ‘air’ s. v. aere~; \emph{bielilos} à côté de \emph{cersa} ‘blanchisseur’ s. v. biaxa~; \emph{smai} à côté de \emph{drokun} ‘dragon’ s. v. dragone~; \emph{pustignak} à côté de \emph{remeta} ‘ermite’ s. v. eremita~; \emph{temegl} à côté de \emph{podumienta} ‘fondement’ s. v. fondamento. 

Sa recherche des équivalents issus de l’idiome de Dubrovnik est particulièrement évidente en ce qui concerne les notions de divers domaines scientifiques, de sorte que le \emph{Vocabolario} répertorie pour la première fois certains termes de philosophie, rhétorique et théologie~: \emph{boghoneuiernik} ‘infidèle’, \emph{daunoroditegl} ‘qui est né avant’, \emph{boghociastiti} ‘honorer Dieu’, \emph{istomacnik} ‘interprète’, \emph{kgnigostuorni} ‘académique’, \emph{mudrogouornos} ‘éloquence’, \emph{saglauglie} ‘conclusion’, \emph{snanosakonik} ‘législateur’. Ainsi, les recherches lexicales de Cekinić suivent-elles les traces de tentatives entamées par Matijašević et ses contemporains au sein de l’Académie des Otieux, vers la fin du 17ème siècle. Un taux peu significatif d’éléments alloglottes parle en faveur du purisme de Cekinić dans la recherche d’équivalents. Ceci fait du \emph{Vocabolario} un maillon important dans l’histoire de la construction de la langue littéraire croate et marque un pas en avant dans le développement de la lexicographie bilingue.

\subsection{Dizionarietto italiano-slavo-moscovitico (1751) de Ivan Marija Matijašević}


Ce fut le chercheur croate M. Deanović qui a publié, dans les années 1950, le dictionnaire manuscrit trilingue italien-croate-russe Dizionarietto italiano-slavo- moscovitico\footnote{Ce dictionnaire fait partie du recueil R 4091 qui est conservé à la Bibliothèque nationale et universitaire de Zagreb} \citep{Deanovic1951, Deanovic1952} du jésuite et Ragusain, père Ivan Marija Matijašević\footnote{Il obtient son doctorat en 1733 à Rome, où il passe une période en compagnie de Ruđer Bošković, chercheur, philosophe, mathématicien, astronome et diplomate croate (1711--1787). Bošković, Ruđer Josip. Hrvatska enciklopedija, mrežno izdanje. Leksikografski zavod Miroslav Krleža, 2021. Accédé 27. 4. 2023. http://www.enciklopedija.hr/Natuknica.aspx?ID=8948} (1714--1791) (Johannes Maria Mattei, Gian Maria Mattei, Giammaria Mattei), auteur d’œuvres religieuses et prédicateur, gardien du patrimoine jésuite à Dubrovnik et neveu du père Đuro Matijašević \parencites[34]{Dolci1767}{Brlek1952}{Lucic1997}.

Ce dictionnaire trilingue est bien spécifique~: il fut créé au bord d'un navire, en trois semaines, entre le 6 et le 28 octobre 1751, lors du trajet de l’auteur de Ancona à Dubrovnik.  Ce voyage fut décrit dans un journal, publié aussi par Deanović en 1952.  

Le dictionnaire regroupe 1087 entrées italiennes, avec 35 entrées en annexe. Compte tenu des circonstances de l’'élaboration du dictionnaire (le texte fut écrit lors d'un trajet), les entrées sont dispersées sans aucun critère, la plupart des entrées étant des substantifs et des adjectifs, avec quelques expressions. (\citealt{Deanovic1951}~: 578) Avec une tendance à la pureté de la langue, l’auteur ne note pas d’éléments alloglottes, c’est pour cela que l’on ne trouve pas d’équivalents pour les mots italiens \emph{arte, artegiano, bagatella, bandiera, bestimiare, brocca, camera, canale, carico, carro, candeliere, cassa, cipresso, consolazione, collana, colonna, cresima, coniglio, danari, disperato, disgrazia, differenza, dispiacere, doppio, eccellente, elemosina, erede, esame, esperienza, evangelio, favore}. Néanmoins, Deanović observe que tous ces mots italiens ont leurs équivalents dans le parler ragusain et qu’il est peu probable que l’auteur ait oublié ces mots de son enfance (nous ne citons que \emph{bagatela, kamara, kono, kandilijer, kasa, čempres, konsolacion, kolana, kolona ou kelomna, dizgracija, diferenca, lemozina, ered, esperjenca, vanđelje, favor}, \citealt[581]{Deanovic1951}) . Dès son retour à Dubrovnik, Matijašević continue à travailler sur le dictionnaire en ajoutant certains mots.  

L’importance de ce petit recueil manuscrit repose surtout sur le lien familial entre le neveu Ivan Marija et l’oncle Đuro Matijašević. De fait, Ivan Marija a dû disposer, à un certain moment, après la mort de son oncle Đuro, du dictionnaire \emph{Dictionarium latino-illiricum}, car son oncle l’a laissé, ainsi que tout son héritage, aux jésuites ragusains. Par ailleurs, la remarque de Deanović disant que 

\begin{quote}
    l’idée même de cette improvisation pourrait être plus importante du point de vue historico-culturel, que son exécution peu réussie, qui d’ailleurs a peu de signification linguistique  (\citealt[581]{Deanovic1951}, notre traduction). 
\end{quote}

fournit un bon résumé sur ce manuscrit. Il convient de constater, malgré tout, que ce dictionnaire s’inscrit dans la continuité des activités lexicographiques à Dubrovnik et que Matijašević nous laisse un contact bien intéressant de trois langues.

\subsection{Vocabula italico-illyrica (1760) de Mato Klašić}

Ce dictionnaire manuscrit n’a pas encore attiré une attention particulière. C’est M. Brlek qui le décrit dans son étude de 1952, intitulée \emph{Rukopisi knjižnice Male braće u Dubrovniku} ‘Les manuscrits de la Bibliothèque des Frères mineurs’. Plus précisément, il cite le dictionnaire «~italien-illyrique~» de Mato Klašić \emph{Vocabula italico-illyrica}.\footnote{Il s’agit du manuscrit no 237, intitulé Cliasci Matthaeus, \emph{Vocabula italico-illyrica}. D’après la datation sur la page de titre, le dictionnaire fut créé en 1760. \citet{Brlek1952} (236--237) fournit plus de détails sur ce recueil (1952~: 236--237). Une courte description est donnée dans \citet{Franic2014}.}  Par endroits, l'auteur n'a pas inscrit de mots, de sorte que certains passages dans le texte sont restés vides. Après avoir terminé la dernière lettre Z (feuille 115), l’auteur recommence sur la feuille 116 et continue jusqu’à la feuille 162. 

Pour les entrées de l’italien (colonne de gauche) on observe parfois plusieurs équivalents~:

\begin{itemize}
    \item \emph{Filosofo, mudrosnanaz, mudaraz, rasumnik} ‘philosophe’, ‘sage’~; 
    \item \emph{Asino, touar, osso, magarre, poslenik} ‘âne’~; 
    \item \emph{Parlare, Gouoriti, Bessiediti, Isriʒati, Rieti, Sborrit, Slouit} ‘parler’.
\end{itemize}

Klašić emprunte une partie de lexèmes aux dictionnaires déjà publiés, tout en formant certains mots lui-même~:

\begin{itemize}
    \item \emph{Canale, legno cauato, Vodotocie, Vodeni prouod} ‘canal’~;
    \item \emph{Pittore, Slikouaç, Pengatur} ‘peintre’~; 
    \item \emph{Prossimo, Iskargni, Sussied, Blixnik} ‘proche’, ‘voisin’. 
\end{itemize}

Parfois, l’auteur apporte des locutions intéressantes et typiques pour la langue parlée~: nous citons les entrées \emph{lingua} ‘langue’ et \emph{ora} ‘heure’~: 

\emph{Lingua, Jesik~; Lingua cauata fuori, Isplaʃen iesik; Auer sula punta della lingua, Imat scto na iesiku; Lingua /fauella/, Jesik; Lingua paesana, Materin Jesik; Lingua di cane /erba/ Tarput; La punta della lingua, Varh od iesika.} 

Voici quelques traits spécifiques morphologiques et lexicaux de l’idiome de Dubrovnik~:

\begin{itemize}
    \item \emph{Cardinale, Kardeno} [langue standard croate~: kardinal]~; 
    \item \emph{Cantilena, Popieuka, Colenda}~; 
    \item (sortes de figue)~: \emph{Fico albero, Smokva, Braxetta, Petrouaccia, GlusciƷa, Prizrella, Mletkigna, NeƷrella, OƷimiƷa, Suha smokva}; 
    \item \emph{Sapone, Midlo, Saplun} [langue standard croate~: sapun]. 

\end{itemize}

Ces exemples ne sont qu’un petit extrait du trésor de l’idiome de Dubrovnik en contact avec l’italien, tel qu’il fut enregistré par Klašić. Bien que ce dictionnaire soit créé après le grand \emph{Dizionario} de Della Bella, ce petit recueil mérite une analyse plus détaillée qui pourrait lui attribuer sa véritable place dans la chaîne ininterrompue de la lexicographie ragusaine et partant la lexicographie croate. 

\subsection{Dictionnaire anonyme, manuscrit n° 278}

Dans son étude sur les manuscrits ragusains, \citet{Brlek1952} aborde, entre autres, le dictionnaire anonyme italien-croate, conservé à la Bibliothèque des Frères mineurs à Dubrovnik, numéro 278. Le texte du dictionnaire commence par la lettre D et le mot «~dissimile, nesličan~». En effet, Brlek constate que le manuscrit contient plusieurs écritures, faites par différentes mains. Les lettres A, B et C viennent à la fin du dictionnaire, ce qui fait penser soit au fait que ces pages ont été déchirées, soit que quelqu’un a ajouté ces entrées plus tard. \citep[283]{Brlek1952}.

Parfois, l’auteur ajoute plusieurs équivalents correspondant aux entrées italiennes, p. ex.~: 

\begin{itemize}
    \item Barcha ‘navire’~: lagia, brod, plov~; 
    \item Legare ‘lire’~: sctiti, legatti~; 
    \item Malizioso ‘malicieux’~: himben, priuarni, sciarovit~; 
    \item Monte ‘montagne’~: bardo, planina, gora, glauiza.
\end{itemize}

On y trouve des explications et définitions, p. ex.~: 

\begin{itemize}
    \item logica ‘logique’~: nauk za razložiti dobro~; 
    \item cerbero ‘cerbère’~: pas is pakla od tri glaue~; 
    \item idolatrare ‘idolâtrer’~: klagnatse Bogovom laʃciviem~; 
    \item lido ‘sable’~: igalo, kraj od morra~; 
    \item metafora ‘métaphore’~: prinescenie od rieci~; 
    \item metropolli ‘métropole’~: paruostolni grad. 
\end{itemize}

A la page 192 du manuscrit se trouve un court inventaire de toponymes ragusains, ce qui fait \citet{Brlek1952} constater que le dictionnaire fut conçu à Dubrovnik, par un locuteur natif, probablement dans le Couvent des franciscains. Etant donné que ce recueil abrite un bon nombre d’éléments lexicaux d’origine ragusaine et vu la liste de toponymes ragusains, ce dictionnaire constitue un apport significatif aux recherches des contacts du croate et de l’idiome ragusain avec la langue italienne.

\section{Conclusion}

Les dictionnaires manuscrits constituent un maillon très important dans l’histoire de la lexicographie en général, ainsi que dans l’histoire de la lexicographie croate. Sur l’exemple de six dictionnaires manuscrits du cercle lexicographique de Dubrovnik et du corpus représentatif d’éléments de l’idiome de Dubrovnik, en particulier les raguséismes lexicaux qui y sont répertoriés, nous constatons que ces dictionnaires ont joué un rôle fort important dans la création de grands dictionnaires imprimés, en particulier aux 17ème et 18ème siècle. 

Seule une étude approfondie des dictionnaires manuscrits peut permettre une meilleure évaluation de la production lexicographique dans sa continuité historique au sein d’une communauté linguistique. En outre, une telle évaluation donnera un aperçu au niveau horizontal et témoignera des contacts linguistiques et culturels entre les langues et cultures européennes de l’époque \citep{Dubois1970}. Ce n’est que si les dictionnaires manuscrits côtoient les dictionnaires imprimés qu’il sera possible de mieux mettre en évidence la continuité des modèles et influences \citep{Samardzija2013}, de sorte que chaque dictionnaire représente une somme de dictionnaires qui le précèdent ainsi qu’un modèle pour les dictionnaires qui suivent.

{\sloppy\printbibliography[heading=subbibliography,notkeyword=this]}
\end{otherlanguage}
\end{document}
