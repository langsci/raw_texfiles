\documentclass[output=paper,colorlinks,citecolor=brown,arabicfont,chinesefont]{langscibook}
\ChapterDOI{10.5281/zenodo.15394493}
\author{Agata Pawlina\affiliation{Jagiellonian University, Poland}}
\title[Musical terms of Greek and Italian origin]
      {Musical terms of Greek and Italian origin in Ottoman Turkish lexicography}

\abstract{Turkish musical terminology is a highly specialized field of inquiry, dealing not only with the linguistic aspects of the contemporary Language for Specific Purposes (LSP) used by Turkish musicians and musicologists, but also with complex socio-cultural issues originating from the history of the reforms of both Turkish music and Turkish language. In this chapter, the author focuses solely on the Ottoman Turkish era. She presents 64 musical terms of Greek and Italian origin collected from 10 Ottoman Turkish dictionaries (17th-19th centuries). She puts the linguistic data into a wider context focusing on two specific diachronic aspects of her broader research on Turkish terminology of art music: 1) how the Greek and Italian languages used by the inhabitants of the Ottoman Empire influenced terminology of the makam music, and 2) to what extent the westernizing reforms of the Ottoman musical culture conducted in the first half of the 19th century are reflected in the lexicography of the period. To answer both questions, she employs a sociocultural understanding of lexicography by seeing the lexicographer as an active observer of not only the language which he describes in a dictionary, but also of the culture of the people who use that language.}

\usepackage{fontspec}
\usepackage{parskip}
\IfFileExists{../localcommands.tex}{
  \addbibresource{../localbibliography.bib}
  % add all extra packages you need to load to this file

\usepackage{tabularx,multicol}
\usepackage{url}
\urlstyle{same}

\usepackage{listings}
\lstset{basicstyle=\ttfamily,tabsize=2,breaklines=true}

\usepackage{langsci-basic}
\usepackage{langsci-optional}
\usepackage{langsci-lgr}
\usepackage{langsci-osl}
% \usepackage{./langsci/styles/langsci-lgr}
% \usepackage{./langsci/styles/langsci-osl}
% \usepackage{langsci-gb4e}

\usepackage{tikz}
\usetikzlibrary{patterns,calc}
\pgfdeclarepatternformonly{south east lines}{\pgfqpoint{-0pt}{-0pt}}{\pgfqpoint{3pt}{3pt}}{\pgfqpoint{3pt}{3pt}}{
    \pgfsetlinewidth{0.6pt}
    \pgfpathmoveto{\pgfqpoint{0pt}{3pt}}
    \pgfpathlineto{\pgfqpoint{3pt}{0pt}}
    \pgfpathmoveto{\pgfqpoint{.2pt}{-.2pt}}
    \pgfpathlineto{\pgfqpoint{-.2pt}{.2pt}}
    \pgfpathmoveto{\pgfqpoint{3.2pt}{2.8pt}}
    \pgfpathlineto{\pgfqpoint{2.8pt}{3.2pt}}
    \pgfusepath{stroke}}
    
\usepackage{stmaryrd}
\usepackage{wasysym}
\usepackage{multirow}
\usepackage{caption}
\usepackage{subcaption}
\usepackage{mathrsfs}
\usepackage{qtree}

\usepackage{linguex}


  %pminos do not split footnotes
% \interfootnotelinepenalty=10000 %Footnote in Laporte chapters has to be split SN


%\DeclareIndexNameFormat{default}{%
%\nameparts{#1}%
%\usebibmacro{index:name}%
%{\index[names]}%
%{\namepartfamily}%
%{\namepartgiveni}%
% {}% L1
% {}% L2
%{\namepartprefix}% generates spurious space L3
%{\namepartsuffix}% generates spurious space L4
%}

%  {\DeclareIndexNameFormat{default}{%
%     \usebibmacro{index:name}{\index[names]}{#1}{#3}{#5}{#7}}}

%\DeclareIndexNameFormat{default}{%
%  \usebibmacro{index:name}{\sindex[nom]}{#1}{#3}{#5}{#7}}

%\DeclareIndexNameFormat{default}{%
%  \usebibmacro{index:name}{\sindex[person]}{#1}{#3}{#5}{#7}}
%\DeclareIndexNameFormat{default}{%
%\nameparts{#1} \usebibmacro{index:name}{\sindex[person]]}{\namepartfamily}{‌​\namepartgiven}{\nam‌​epartprefix}{\namepa‌​rtsuffix}}

%\newcommand{\smiley}{:)}

%\renewbibmacro*{index:name}[5]{%
%\usebibmacro{index:entry}{#1}%
%{\iffieldundef{usera}{}{\thefield{usera}\actualoperator}\mkbibindexname{#2}{#3}{#4}{#5}}}

% \newcommand{\noop}[1]{}

%remove for final
%\overfullrule=1mm

\newcommand{\tobi}[2]}}
\renewcommand{\S}[1]{\tobi{#1}{\textsc{*}}}

% this volume references
% puts: [this volume]
% already defined: \citetv
%\newcommand{\citepv}[1]{(\citeauthor{#1} \citeyear*{#1} [this volume])}
\newcommand{\citealtv}[1]{\citeauthor{#1} \citeyear*{#1} [this volume]}

%parentheses around example number
\newcommand{\pref}[1]{(\ref{#1})}

% in-text examples

\newcommand{\lnex}[1]{\textit{#1}} %target lang word
\newcommand{\lnlit}[1]{(lit.: `#1')} %literal reading
\newcommand{\lnlat}[1]{(#1)} % latinization
\newcommand{\lntrans}[1]{`#1'} %translation
\newcommand{\lnexl}[2]%
{\lnex{#1}{} \lnlat{#2}} % ex with latinization
\newcommand{\lnexlat}[3]{\lnex{#1}{} \lnlat{#2}{} \lntrans{#3}} % ex with latinization and tranl.

%ch01
\newcommand{\co}[1]{\mbox{\textbf{#1}}}

%ch09

\newcommand{\cyrbulg}[1]{\begin{otherlanguage*}{bulgarian}#1\end{otherlanguage*}}


%ch10
\newcommand{\nlp}{{\small NLP}}
\newcommand{\mwe}{{\small MWE}}
\newcommand{\rae}{{\small RAE}}
\newcommand{\lvc}{{\small LVC}}
\newcommand{\pos}{{\small P}o{\small S}}
%\newcommand{\todo}[1]{ \textcolor{red}{#1} }

%\renewcommand{\labelenumi}{\theenumi}
%\ainamefmt{{vv}{ll}{, ff}{, jj}} % fullname

\newcommand{\biberror}[1]{{\color{red}#1}}

\newcommand{\osenovaitem}{--~}
  %% hyphenation points for line breaks
%% Normally, automatic hyphenation in LaTeX is very good
%% If a word is mis-hyphenated, add it to this file
%%
%% add information to TeX file before \begin{document} with:
%% %% hyphenation points for line breaks
%% Normally, automatic hyphenation in LaTeX is very good
%% If a word is mis-hyphenated, add it to this file
%%
%% add information to TeX file before \begin{document} with:
%% %% hyphenation points for line breaks
%% Normally, automatic hyphenation in LaTeX is very good
%% If a word is mis-hyphenated, add it to this file
%%
%% add information to TeX file before \begin{document} with:
%% \include{localhyphenation}
\hyphenation{
    Beck-man
    Ngu-yen
    back-chan-nel
    back-chan-nels
    mo-not-o-nous
    ste-reo-typ-i-cal
}

\hyphenation{
    Beck-man
    Ngu-yen
    back-chan-nel
    back-chan-nels
    mo-not-o-nous
    ste-reo-typ-i-cal
}

\hyphenation{
    Beck-man
    Ngu-yen
    back-chan-nel
    back-chan-nels
    mo-not-o-nous
    ste-reo-typ-i-cal
}

  \togglepaper[13]%%chapternumber
}{}

\begin{document} 
\maketitle

\section{Introduction}\label{sec:pawlina:1}

Turkish musical terminology is a highly specialized field of the Oriental Studies, which until now has not been researched by linguists. It has gained some interest amongst musicologists (\citealt{Farmer1936, Kalender1981, Bulut2009, Tohumcu2009, Yalcin2013}) and experts specialized in the Ottoman literature (\citealt{Cetin, Kadioglu2012}), however not as a leading research field of their scholarly work. After spending the last few years examining and analysing the Ottoman Turkish (\emph{Osmanlıca}) and Turkish (\emph{Türkçe}) musical terminology, it has become apparent to me, that to fully understand this interdisciplinary subject one must be well versed in both, the history of Turkish music and the – not less turbulent – history of Turkish language. The socio-cultural changes and the westernizing reforms that took place in Turkey in the 19th and 20th centuries affected both of those disciplines of human creativity. My main goal was to establish whether one can trace a change in Turkish musical culture happening over time by investigating pages of the lexicographical sources from several historical periods. In this chapter, I will solely focus on the Ottoman era, i.e., before 1923.

The turn of the 16th and 17th centuries witnessed the emergence of the \emph{Ottoman idiom} within Middle Eastern art/urban music, distinct from the earlier Arabo-Persian classical canon (\citealt{Behar2006, Feldman1996, Greve2015, ReinhardReinhard1984, ReinhardReinhardStokes2001, Wright1992}). Later, the modernizing and westernizing reforms of the 19th century led to the introduction of the European (especially Italian) art music in Istanbul (\citealt{Araci2002,Araci2006, KomsuogluTuran2007, Ozalp2000, Turan2004,Turan2011a,Turan2011b}, \citealt{Yore2011}). Since then, both musical traditions have professionally developed concurrently, but independently from each other in Turkey. To describe this phenomenon I use the term \emph{dualism of Turkish art music}. It means that the genre of \emph{art music} in Turkey can be further divided into two sub-genres, which (in oversimplified terms useful for my terminological, not an in-depth musicological research) I titled the \emph{Eastern-style art music} and the \emph{Western-style art music}.

The LSP (Language for Specific Purposes) used by Turkish musicians and musicologists is also \emph{dualistic}. It signifies that a practitioner of the Eastern-style Turkish art music uses a different vocabulary while describing musical pieces, than a musician educated in the Western-style music and who works in an opera theatre or a philharmonic orchestra. However, there is a part of specialized vocabulary that is mutually used by practitioners and theoreticians of both musical genres (e.g., verbs, some adjectives and denominations of such \emph{objective} acoustic phenomena like ‘sound’, ‘pitch’, ‘timbre’, etc.). My recent research \citep{Pawlina2021} proved that the dualism of Turkish musical terminology can be traced back to the mid-19th century – the same period when European-style music was officially incorporated into the musical culture of Constantinople. This multi-layered complexity makes the lexicon of the LSP used by the Turkish musicians and musicologists, a very interesting, however currently under-researched, field of Turkish linguistics.

In this chapter, I would like to focus on a specific diachronic, etymological aspect of the subject matter. My investigations showed that over the centuries, the Turkish musical terminology expanded by borrowings mainly, while word-formation and terminologization played a minor role in the process.\footnote{Consequently, native Turkish words account for only 25\% of terms used to describe the dualistic art music. Such etymological structure – with the minority of vocabulary being native words – is not surprising. Similar structure can be observed in musical terminology of the most of the European languages \citep{Dabkowski1997}.} Thus, it should be regarded as a “lexicon built on the conceptual foundation of a foreign language” \citep[59]{LukszynZmarzer2006}. To be precise, the two groups of foreign languages: Persian and Arabic ‒ in the case of terminology of the Eastern-style art music; and European languages (Italian in the 19th century, currently mainly French) ‒ in describing theory and practice of the Western-style art music.

It is popularly believed that musical terminology consists mainly of the Italian and Greek (or Latin < Greek) borrowings, hence, the idea for the subject of this paper. It has already been established that this belief is not entirely correct within the European context \citep{Dabkowski1997}. Here, I would like to ascertain the origin and the role that the Greek and Italian loan-words played in the Ottoman Turkish music vocabulary that was generally dominated by the Arabic and Persian borrowings. To do so, I will present all terminological units of this type, which I have collected from 10 mono-, bi- and multilingual, general and specialized Ottoman Turkish dictionaries. I will provide their lexicological analysis and describe historical and socio-cultural background that led to the introduction of this vocabulary into Turkish language.

\section{The Ottoman Turkish lexicography: Methodology \& sources}\label{sec:pawlina:2}

The history of the Ottoman Turkish lexicography in the broadest methodological approach encompasses all dictionaries written in the Ottoman lands since the emergence of the dynasty in the late 13th century until the proclamation of the Republic of Turkey in 1923 (\citealt{Aksan1998, Olmez1994, Prokosch2011}). For centuries, lexicographical works authored by Ottomans focused on Arabic and Persian vocabulary rather than on Turkish lexicon. This situation changed in the 19th century, when the Ottoman intellectuals deemed the reforms necessary and they simplified their written language. At the same time the first monolingual Ottoman Turkish dictionaries and grammars were published in Istanbul (\citealt{Lewis1999}:  5–32, \citealt{Prokosch2013, Siemieniec-golas2015}, \citealt[335]{Stein1990}). Before that, those who believed essential to compile Ottoman Turkish glossaries for practical and, later on, also scientific purposes, were European merchants, missionaries, diplomats and scholars-orientalists (\citealt{Rocchi2011, Schweickard2014, Stachowski1995}).

Dictionaries have proven to be a reliable source of terminological data (\citealt{GalinskiBudin1999, Harras1989, Miodunka1989, Sager1989}), therefore, for the purposes of this current research, I have selected 10 Ottoman Turkish lexicographical works as a source of material. I have narrowed the vast period of the history of Turkish lexicography down to the 17th-19th centuries according to the state-of-the-art on the periodization of the Ottoman Turkish music summarized in the previous section. Of the selected dictionaries, six represent the general bilingual type of dictionaries: Italian-Turkish \citep{Siemieniec-golas2005}, French-Turkish (\citealt{KiefferBianchi1835, Sami1885, Stachowski2002}) and English-Turkish (\citealt{Redhouse1861, RedhouseSozlugu2017}); two are general multilingual dictionaries (\citealt{Korabinsky1788, Meninski1680}); one is a French-Turkish terminological dictionary of arts, science and medicine (\citealt{TinghirSinapian1891}), and one is a monolingual specialized dictionary of the Ottoman Turkish art music \citep{Uz1892}.

They represent several varieties of sources that are typical for the research on Turkish historical lexicography: 

\begin{enumerate}
    \item dictionaries written in both the Arabic\footnote{I use the term “Arabic script” as a simplification for Ottoman Turkish alphabet based on Arabic and Persian letters, which was used in the Ottoman Empire until its replacement by the Latin-based alphabet in 1928 \citep{Lewis1999}.} and Latin scripts, for the Ottoman Turkish lexical items, and their definitions in the European languages, respectively (\citealt{KiefferBianchi1835, Meninski1680, Sami1885, Redhouse1861, RedhouseSozlugu2017, TinghirSinapian1891});
    \item dictionaries written in the Latin script for both the Turkish lexical items and their definitions (\citealt{Korabinsky1788, Siemieniec-golas2005, Stachowski2002});\footnote{These dictionaries can be considered as a part of the vast group of the so-called Turkish \emph{Transkriptionstexte} \citep[185--86]{Stachowski1995} in which Turkish words are transcribed into Latin script influenced by the orthography of the language native to the author of a given text (in the case of my sources – German, Italian and French).}
    \item dictionaries written in the Ottoman Turkish in the Arabic script \citep{Uz1892}.
\end{enumerate}

The latter deserves a special attention. \emph{Tʿalīm-i mūsīḳi, yaḫud, mūsīḳi ıṣṭılāḥātı} written by İsmail Kâzım Uz, printed in Constantinople in 1892--93 \emph{(AH1310)}, is the first and (to the best of my knowledge) the only specialized dictionary of the Ottoman Turkish music written by an Ottoman. It is very modern in its macro- and microstructure. It describes ca. 500 musical terms presented in an alphabetical order. Given the turbulent history of the Ottoman Turkish art music after the proclamation of the Republic of Turkey, the value of the musicological and linguistic content of Uz’s definitions is invaluable. These contain as well as the semantic data that is important for the subject matter of this paper, which I will describe below.

From those selected sources, in total I have collected ca. 1300 musical terms. I analysed them in terms of semantics, morpho-syntactic structure and etymology. Below, in Sections~\ref{sec:pawlina:3} and~\ref{sec:pawlina:4}. I present lexical items of the Greek and Italian origin. Etymological information provided in these sections, has been derived from the historical dictionaries of Turkish and the state-of-the-art dictionaries of the Ottoman Turkish (\citealt{AksoyDilsin2009, Kubbealti-Lugati, Parlatir2017, RedhouseSozlugu2017}, M. \citealt{Stachowski2019, Tietze2016}) and it has been compared with the information on the European terminology and its origins (\citealt{Dabkowski1991,Dabkowski1997,Dabkowski2004}).

In \sectref{sec:pawlina:5}, I examine the data in a wider context in order to answer the following questions: 1) how Greek and Italian languages used by the inhabitants of the Ottoman Empire influenced terminology of the Ottoman Turkish art music, and 2) to what extent the westernizing reforms of the Turkish musical culture conducted in the first half of the 19th century are reflected in Turkish lexicography of the period. To answer both of these questions, I employ a sociocultural understanding of lexicography \parencites[37]{Doroszewski1970}{Hausmann1989}. I regard the lexicographer as an active observer of not only the language that he describes in his dictionary, but also of the culture of people who use that language \citep{Ligara2017}.
\section{Transcription and abbreviations}
For convenience, in the following lists of the Ottoman Turkish vocabulary (Sections~\ref{sec:pawlina:3} and \ref{sec:pawlina:4}) terms are given in the alphabetical order of the Latin script. I employ transcription system from Arabic into the Latin characters developed by the scholars of the Project \emph{Corpus Musicae Ottomnicae} based in Münster, published on the CMO website in October 2022.\footnote{See: \emph{Guidelines for the Transcription of the Ottoman Lyric from Arabic into Latin Characters – Revised edition}: https://www.uni-muenster.de/CMO-Edition/en/publikationen/publikationen.html (15-03-2023-03-15).}

Each of the Ottoman Turkish terms is numbered and given in the Latin script followed by all spelling variations in the Arabic script found in the sources. Below, I provide the English meaning of the term or – when there is no English equivalent − its short definition. In the next line, I present etymology and (if applicable) the morphological and/or syntactic structure of the term. The last part of the description of each numbered lexical item is a list of abbreviated primary sources in which a term has been found. The sources are provided in chronological order, from the oldest, with the volume (if applicable), and page number in brackets (  ), along with the original Latin transcription given by the author of the dictionary (if applicable). For example: Men(144: \emph{arghanūn}) means that Meniński in his \emph{Thesaurus Linguarum Orientalium} transcribed Ottoman Turkish {\textarab{ارغانون}} [erġanun] as \emph{arghanūn}.

The following is the list of the abbreviated sources:

\begin{multicols}{2}
\begin{tabbing}
MMM \= Source\kill
K-B \> \citealt{KiefferBianchi1835}\\
Kor \> \citealt{Korabinsky1788}\\
Men \> \citealt{Meninski1680}\\
Mol \> \citealt{Siemieniec-golas2005}\\
NR  \> \citetitle{RedhouseSozlugu2017} (\citeyear{RedhouseSozlugu2017})\\
R   \> \citealt{Redhouse1861}\\
ŞS  \> \citealt{Sami1885}\\
T-S \> \citealt{TinghirSinapian1891}\\
Uz  \> \citealt{Uz1892}\\
Vig \> \citealt{Stachowski2002}
\end{tabbing}
\end{multicols}

When the definition of the given term in the source is particularly interesting, it is quoted in English translation. If the exact etymological root is not confirmed in the sources, the indication [?] is stated next to the origin proposed by me.

\section{Musical terms of Greek origin}\label{sec:pawlina:3}

Among the Ottoman Turkish musical terms collected from the sources described earlier, I have found only 10 Greek loan words and 3 derivatives consisting of a Greek stem with a Turkish suffix creating \emph{nomina agentis}: {\textarab{جى}} [read as: -ci, -cı, -cu or -cü]. These 13 terms belong to the following semantic fields: a) the general musical terms (i.e., those that can be used in the context of both, the Eastern- and the Western-style music), b) the theory of the Ottoman Turkish art music, c) the practice of the Ottoman Turkish art music, d) the organology (i.e., the names of instruments and their parts):

\begin{itemize}
    \item[(1)] erġanun {\textarab{ارغانون , ارغنون}}
    
    ‘organ’\footnote{Even though my sources translate this word as a name of the specific instrument known as an ‘organ’, other Ottoman sources indicate that it was a rather ambiguous term designating “any wind instrument” (see: \citealt[3]{Farmer1931}).}
    
    < Greek ὄργᾰνον [órganon] ‘instrument, tool; musical instrument’

    Mol(96: \emph{arghanan}); Men(144: \emph{arghanūn}); Kor(4: \emph{arganon}); Vig(95: \emph{èrghanoun, èrghoun}); K-B(I,24: \emph{arghanoun}); R(549); T-S(II,202); ŞS(69: \emph{erghanoûn}); NR(345)\footnote{\emph{Kubbealtı Lugati} informs that the name of a zither-like \emph{kanun} in its original Arabic form {\textarab{قانون}} [qānūn] was derived from Greek ὄργᾰνον, see: http://lugatim.com/s/kanun.}

    \item[(2)] erġanuncu {\textarab{ارغانونجى}}

    ‘organist’

    < {\textarab{ارغانون}} [erġanun] + {\textarab{جى}} [-cu]

    T-S(II,201)
    
    \item[(3)] fiske {\textarab{فسكه}}

    a name of a playing technique on the plucked string instruments

    < Greek φούσκα [foúskē] ‘to flip, to flick’

    NR(376)
    
    \item[(4)] ḳataḳofti {\textarab{قاتاقوفتى}}

    a name of an \emph{usul} (i.e., a rhythmic pattern in the Ottoman Turkish art music)

    < Greek κατακόπτω [katakoptō] ‘to cut’ [?]

    NR(617) “G[ree]k \emph{slang} swagger; \emph{Or[iental] mus[ic]} same as \emph{müsemmen}”
    
    \item[(5)] ḳatiḳofti {\textarab{قاتيقوفتى}}

    see above: \emph{ḳataḳofti}

    NR(619) “\emph{Or[iental] mus[ic]} a fast rhythmic pattern of eight beats”
    
    \item[(6)] ḳomma {\textarab{قومّا}}

    ‘comma’, a name of the smallest interval

    < Greek κόμμα [kómma] ‘piece’

    Uz(43); NR(672)
    
    \item[(7)] laġuta {\textarab{لاغوته}}
    
    see below: \emph{lauta}

    Uz(46)\footnote{This spelling is also confirmed in the, so-called, \emph{Tarama Sözlüğü} – the biggest historical dictionary of Turkish language − based on the Ottoman Turkish sources from the 16th, 18th and 19th centuries (Aksoy and Dilc̜in 2009, IV,2661; V,3395).}

    \item[(8)] lauta {\textarab{لاوته}}, lauṭa {\textarab{لاوطه}}

    ‘lute’ (of the European lutes family)

    < Greek λαούτο [laouto] ‘lute’

    Mol(96: \emph{lauta}) “liuto, stromento di sonare”; R(492); ŞS(936: \emph{laouta}); NR(705: \emph{lavta}) “an instrument like \emph{ud}, now out of use”\footnote{The, so-called, \emph{Kieffer-Bianchi Dictionary} is the only source to list another form of this term: {\textarab{لاوت}} [laut] transcribed into Latin characters as \emph{laut} and \emph{lāvout} − K-B(II,693).}
    
    \item[(9)] lauṭacı {\textarab{لاوطه جى}}

    1. ‘lute player’, 2. ‘maker of seller of lutes’

    < {\textarab{لاوطه}} [lauṭa] + {\textarab{جى}} [-cı]

    NR(705)

    \item[(10)] mandal {\textarab{مندال ,ماندال}}

    a name of a metal peg of a \emph{kanun} (a zither-like instrument used in the Ottoman Turkish art music)

    < Greek μανταλι [mantáli] ‘bolt, lock’

    NR(729)
    
    \item[(11)] mūsīḳi {\textarab{موسيقى}}, mūsiḳi {\textarab{موسقى}}

    ‘music’

    < Arabic {\textarab{موسيقى}}
    [mūsīqi] ‘music’ < Greek μουσική [mousiké] ‘the art of muses, music’

    Mol(150: \emph{musiki}); Men(5024: \emph{mūsykȳ}); Men(5025: \emph{mūsȳky}); Vig(204: \emph{mousiqi}); K-B(II,1046: \emph{mouciqy, moucyqy}); R(526); ŞS(1098); T-S(II,153); Uz(49); NR(801)

    \item[(12)] mūsīḳici {\textarab{موسيقجى}}

    ‘musician’

    < {\textarab{موسيقى}} [mūsīḳi] + {\textarab{جى}} [-ci]

    Mol(41: \emph{miskgi}); Vig(204: \emph{mousiqidji})

    \item[(13)] sirto {\textarab{سيرتو}}

    a name of an instrumental genre in the Ottoman Turkish art music; since the 19th century it was a part of the \emph{fasıl} suite

    < Greek συρτός [sirtós], a name of a folk dance

    Uz(30-31); NR(1022)
\end{itemize}

\section{Musical terms of Italian origin}\label{sec:pawlina:4}

In the collected material, there are more Ottoman Turkish musical terms borrowed from Italian. In total, I have found 50 lexical items comprizing: 41 single-word loan-words, 3 derivatives with an Italian stem and a Turkish suffix creating \emph{nomina agentis}: {\textarab{جى}} [read as: -ci, -cı, -cu or -cü] and 1 substantivized derivative with an Italian stem and a Turkish suffix creating adjectives {\textarab{لى}} [read as: -li, -lı, -lu or -lü]. There are also 6 compounds in the form of the \emph{indefinite izafet} (\citealt{Lewis1991}:  41), i.e., a group of two nouns, where the first noun remains in the absolute form and qualifies the second noun, which has the possesive suffix of the third person: {\textarab{سى}} [read as: -si, -sı, -su or -sü] or {\textarab{ى}} [read as: -i, -ı, -u or -ü].

The collected terms belong to the following semantic fields: a) the general musical terms, b) the theory of the Western-style art music, c) the names of the symbols in the musical notation, d) the musical practice of the Western-style art music, e) the musical practice of the Eastern-style art music, f) organology.

\begin{itemize}
    \item[(14)] alafranġa  {\textarab{آلا فرانغه}}

    a general term for the ‘European music’ (i.e., the Western-style music)

    < Italian \emph{alla Franca} ‘in the French style’

    NR(42)
    \item[(15)] alaturḳa {\textarab{آلا تورقه}}

    a general term for the ‘Turkish music’ (i.e., the Eastern-style music)\footnote{Both terms, \emph{alafranga} and \emph{alaturka}, emerged in the 19th century Ottoman Empire and were used to describe ‘Occidental-’ and ‘Oriental way’ in many other elements of culture, e.g., calendar, hour, clothing, etc. However, in my sources and in literature on music from the period they are often used not only as adjectives but also as nouns designating ‘Turkish music’ and ‘European music’ (see: \citealt[10]{Ozalp2000}, \citealt[2--27]{Oztuna1969}).}

    < Italian \emph{alla Turca} ‘in the Turkish style’

    NR(44)
    \item[(16)] alaturḳacı {\textarab{آلاتورقه جى}}

    1. a practicioner of Turkish music (i.e. the Eastern-style music), 2. a lover of Turkish music

    < {\textarab{آلا تورقه}} [alaturḳa] + {\textarab{جى}} [-cı]

    NR(44)
    \item[(17)] alto {\textarab{آلتو}}

    ‘alto, contralto’

    < Italian \emph{alto} ‘alto, contralto’

    NR(54)
    \item[(18)] armonyum {\textarab{آرمونيوم}}

    ‘harmonium, pump organ’

    < Italian \emph{armonium} (< French) ‘harmonium, pump organ’

    T-S(I,394); NR(74) 
    \item[(19)] arya {\textarab{آريا}}

    ‘air, song’

    < Italian \emph{aria} ‘aria, air, song’

    NR(72,76)
    \item[(20)] banda {\textarab{بانده}}

    see below: \emph{bando}

    NR(129)
    \item[(21)] bando {\textarab{باندو}}

    1. ‘military band’, 2. a name of the 19th-century official Ottoman military band, which replaced the former \emph{mehterhane} after the abolishment of the Janissary corps in 1826; its first conductor was Giuseppe Donizetti, who had been invited by sultan Mahmud II to perform the westernizing reforms in the musical culture of the Ottoman court (\citealt{Araci2002,Araci2006})

    < Italian \emph{banda (musicale)} ‘(musical) band’ 

    NR(130)
    \item[(22)] baṣo {\textarab{باصو}}
    
    ‘bass’
    
    < Italian basso ‘bass’

    T-S(I,53); NR(134)\footnote{Names of voices are also listed in the first, English-Turkish edition of the so-called \emph{Redhouse Dictionary} \citep{Redhouse1861} but there are no Ottoman Turkish equivalents yet, only a short description of each voice type.}
    \item[(23)] baṣo kemānı {\textarab{باصو كمانى}}

    ‘double bass’

    < Italian \emph{basso} ‘bass’ + Ottoman Turkish (<Persian) {\textarab{كمان}} [kemān] ‘violin’ + 3SG.POSS

    T-S(I,148) 
    \item[(24)] batuṭa {\textarab{باتوطه}}

    ‘bar, measure’

    < Italian \emph{battuta} ‘bar, measure’

    Uz(11)
    \item[(25)] çimbalo {\textarab{چمبالو , چيمبالو}}

    ‘harpsichord, cembalo’ or/and ‘piano’

    < Italian \emph{cembalo} ‘harpsichord, cembalo’	

    R(461) “keyboard”; R(576) “piano”, “pianoforte”; 

    T-S(I,120) “clavicorde”
    \item[(26)] çinbalo {\textarab{چنبالو}}
    
    see above: \emph{çimbalo}
    \item[(27)] do {\textarab{دو}}
    
    ‘do (in solmisation)’

    < Italian \emph{do}

    R(256); T-S(I,225) “a name of the first note in singing [sic!]; [same as] \emph{çargâh} [in the scale used in the Ottoman Turkish music]”; Uz(24) “this is a name of the [pitch] \emph{çargâh} in the \emph{alafranga} [music]; its other name is \emph{ut}; it is the first pitch in the musical scale of the \emph{alafranga} music”; NR(306)
    \item[(28)] fa {\textarab{فا}}

    ‘fa (in solmisation)’ 

    < Italian \emph{fa}

    R(314); T-S(I,314); Uz(41); NR(357)
    \item[(29)] falṣo {\textarab{فالصو}}

    ‘false note’

    < Italian \emph{falso} ‘false [note]’

    Uz(41--42) “with this word one can describe a mistake in singing”; NR(359)
    \item[(30)] fanfar {\textarab{فانفار}}

    ‘fanfare, trumpet call’

    < Italian \emph{fanfara} ‘fanfare, trumpet call’

    T-S(I,317)
    \item[(31)] fantaziya {\textarab{فانتازيا}} , fanṭaziya {\textarab{فانطازيه}}

    ‘fantasy, fantasia’

    	< Italian \emph{fantasia} ‘fantasy, fantasia’

     T-S(I,314); NR(360)
    \item[(32)] flauta {\textarab{فلاوته}}

    ‘flute’

    < Italian \emph{flauto} ‘flute’

    R(338); T-S(I,337); Uz(42); NR(377: \emph{flâvta})
    \item[(33)] flautacı {\textarab{فلاوته جى}}

    ‘flutist’

    < {\textarab{فلاوته}} [flauta] + {\textarab{جى}} [-cı]

    R(338); NR(377: \emph{flâvtacı})
    \item[(34)] ısḳala {\textarab{اسقله , اسقاله}}

    ‘musical scale’

    < Italian \emph{scala} ‘scale’

\largerpage
    T-S(I,241) “echelle diatonique”; Uz(6--7); NR(503)
    \item[(35)] ḳıranta {\textarab{قرانته}}  [sic!]

    ‘clarinet’

    < Italian \emph{clarinetto} [?]

    Uz(43)\footnote{It is possible that K. Uz made a spelling mistake here. According to H.G. Farmer a different spelling of this term: {\textarab{قرناطه}} [ḳırnaṭa] was used in the Ottoman Turkish sources at least since the 17th century \citet[24--25]{Farmer1936}. However, I did not find it in the dictionaries.}
    \item[(36)] kitara {\textarab{كتاره}}

    ‘guitar’

    < Italian \emph{chitarra} ‘guitar’

    Uz(44)
    \item[(37)] konçerto {\textarab{كونچرتو}} 

    ‘concert’ (an event, not a genre)

    < Italiano \emph{concerto} ‘concert’

    T-S(II,153) “musique de concert”
    \item[(38)] ḳontrabaṣo {\textarab{قونترا باصو}} 

    ‘double bass’

    < Italian \emph{contrabasso} ‘double bass’

    T-S(I,53)
    \item[(39)] ḳoro {\textarab{قورو}}

    ‘choir’

    < Italian \emph{coro} ‘choir’

    NR(675)
    \item[(40)] la {\textarab{لا}}

    ‘la (in solmisation)’ 

    < Italian \emph{la}

    R(464); T-S(II,45); Uz(45); NR(700)
    \item[(41)] lonġa {\textarab{لونغه}}

    an instrumental genre in the Ottoman Turkish art music; emerged in the late 19th century from the Rumelian Gypsy music; sometimes used as the last part of the \emph{fasıl} suite

    < Italian \emph{lungo} ‘long’

    R(713); NR(713)
    \item[(42)] mayestro {\textarab{مايسترو}} 

    ‘conductor of the orchestra’

    < Italian \emph{maestro} ‘master,  teacher’

    NR(739)		
    \item[(43)] mezosoprano {\textarab{مه زوسوپرانو}}

    ‘mezzo-soprano’

    < Italian \emph{mezzo soprano}

    NR(772)
    \item[(44)] mızıḳa {\textarab{مزيقه}} 

    see below: \emph{muzıḳa}

    NR(774)
    \item[(45)] mızıḳalı {\textarab{مزيقه لى}}

    a member of the Sultan’s military orchestra

    < {\textarab{مزيقه}}  [mızıḳa] + {\textarab{لى}} [-lı]

    NR(774) 
    \item[(46)] mi {\textarab{مى}}

    ‘mi (in solmisation)’ 

    < Italian \emph{mi}

    R(511); T-S(I,237); T-S(II,124); Uz(49) 
    \item[(47)] a. muzıḳa {\textarab{موزيقا}} 

    1. ‘military band’, 2. ‘military music’

    < Italian \emph{musica < banda musicale}

    		ŞS(1097: \emph{mouzyqa}); NR(806)\footnote{The official name of the \emph{Bando} (see above: 2. meaning of the term no. 21) was {\textarab{موزيقهٔ همايون}} [muzıḳa-yı hümāyūn], literally ‘the military band of the sultan’. It is an interesting compound term in the form of the \emph{Persian izafet} combining Italian and Persian loan-words together. It was also a name of the palace music school founded by Giuseppe Donizetti to train musicians in the Western-style art music \citep{Ozcan2020}.}


    b. ʿasker muzıḳası {\textarab{عسكر موزيقه سى}}

    1. ‘military band’ 2. ‘military music’

    < Ottoman Turkish (<Arabic) {\textarab{عسكر}} [ʿasker] ‘soldier’ + {\textarab{موزيقه}} [muzıḳa] + 3SG.POSS

    R(526)
    \item[(48)] muzıḳacı {\textarab{موزيقاجى}}

    a member of a military band

    < {\textarab{موزيقه}} [muzıḳa] + {\textarab{جى}} [-cı]

    T-S(II,152) “musicien.enne”
    \item[(49)] a. noṭa {\textarab{نوطه}} , nota {\textarab{نوته}} 

    1. ‘musical note’ 2. ‘written music’

    < Italian \emph{nota} ‘musical note’

    		ŞS(1138: \emph{nota}); T-S(I,56); Uz(51); NR(891)

      	b. musīḳi noṭası {\textarab{موسيقى نوطه سى}} 

       1. ‘musical note’ 2. ‘musical notation’

       < {\textarab{موسيقى}} [musīḳi] + {\textarab{نوطه}} [noṭa] + 3SG.POSS

       T-S(II,176)
    \item[(50)] nota kitābı {\textarab{نوته كيتابى}} 

    ‘musical score’

    < {\textarab{نوته}} [nota] + Ottoman Turkish (<Arabic) {\textarab{كتاب}} [kitāb] ‘book’ + 3SG.POSS

    R(526)
    \item[(51)] noṭa miḳyāsı {\textarab{نوطه مقياسى}}
    
    ‘diatonique scale’

    < {\textarab{نوطه}} [noṭa] + Ottoman Turkish (<Arabic) {\textarab{مقياس}} [miḳyās] ‘scale’ + 3SG.POSS

    T-S(I,241) 	
    \item[(52)] opera {\textarab{اوپرا}} 

    ‘opera’

    < Italian \emph{opera}

    R(547); ŞS(162); NR(901)
    \item[(53)] opera tiyatrosu {\textarab{اوپرا تياتروسى}}

    ‘opera theater’

    < {\textarab{اوپرا}} [opera] + {\textarab{تياترو}} [tiyatro] < Italian \emph{teatro} ‘theater’ + 3SG.POSS

    R(547)
    \item[(54)] orkestra {\textarab{اوركستره}} 

    ‘orchestra’

    < Italian \emph{orchestra}

    T-S(II,199); NR(902)
    \item[(55)] primo {\textarab{پريمو}}   

    a term describing the main melody in a musical piece

    < Italian \emph{primo} ‘the first’

    Uz(14) “we call that a melody which is played by such instruments as [a lute] \emph{ut}, \emph{keman} [‘violin’], [zither-like] \emph{kanun}, [a lute] \emph{tanbur}, etc.”
    \item[(56)] piyano {\textarab{پيانو}} 

    ‘piano’

    < Italian \emph{piano}, \emph{pianoforte}

    R(596); ŞS(297: \emph{pïano}); T-S(II,267); Uz(15); NR(937)
    \item[(57)] re {\textarab{ره}}

    ‘re (in solmisation)’ 

    < Italian \emph{re}

    R(623); T-S(I,180); T-S(II,353); Uz(28); NR(950)
    \item[(58)] si {\textarab{سى}}

    ‘si (in solmisation)’ 

    < Italian \emph{si}

    R(685); T-S(I,390); T-S(II,421); NR(1015)
    \item[(59)] ṣol {\textarab{صول}}
    
    ‘sol (in solmisation)’ 
    
    < Italian \emph{sol}

    R(696); T-S(II,428); Uz(37); NR(1026) 
    \item[(60)] ṣoprano {\textarab{صوپرانو}}

    ‘soprano’

    < Italian \emph{soprano}

    T-S(II,431)
    \item[(61)] ṭono {\textarab{طونو}} 

    ‘tone, pitch’

    < Italian \emph{tono} ‘tone, pitch’

    T-S(II,496)
    \item[(62)]	viyola {\textarab{ويوله}}

    ‘viola’

    < Italian \emph{viola} 

    NR(1230)
\end{itemize}

\section{Conclusions}\label{sec:pawlina:5}

The Ottoman Turkish musical terms of  Italian origin belong to the newest stratum of that language, the so-called \emph{New Ottoman Turkish} period of its history. The oldest dictionary that recorded them in its definitions is \emph{A Lexicon, English and Turkish Shewing in Turkish, the Literal, Incidental, Figurative, Colloquial, and Technical Significations of the English Terms} written by Sir James W. Redhouse, published in London in 1861. They were absorbed into the Ottoman Turkish during the initial period of the westernization of the Turkish musical culture. Its forerunner was Giuseppe Donizetti (the elder brother of the famous opera composer ‒ Gaetano), who, at the request of Sultan Mahmud II, undertook a complete reorganization of the Ottoman military band in accordance with the European fashion and musical style. His work had begun in 1827, and within a few years, the former “Eastern-style” official military band of the Sultan called \emph{Mehterḫāne}, was replaced by the \emph{Muzıḳa-yı Hümāyūn} (colloquially called \emph{Bando}) operating within the framework of the new palace music school of the same name (\citealt{Araci2002,Araci2006, KomsuogluTuran2007, Pawlina2014}, \citealt[278--95]{Sanal1964, Yore2011}). It seems that, along with the Western instruments and the Italian musicians, that Donizetti had also brought to Constantinople was the basic terms used in the teaching of the European musical theory and practice.

The most prevalent terms incorporated into Turkish during that period are the pitch names (lexical items no. 27, 28, 40, 46, 57, 58, 59). Until today, the solmisation is the main system used by the Turkish musicians to describe the pitches and keys in the Western-style art music (e.g., Tur. \emph{La-majör} ‘A major’, Tur. \emph{la-minör} ‘A minor’).\footnote{This is the system used in the musical terminology of Romance languages and Russian, while in LSP of English, German, Czech and Polish musicians the basic system of pitch/key naming employs first seven letters of the alphabet \citep[82]{Dabkowski1997}.} The continuous use of solmisation in the LSP of Turkish musicians and musicologists can be traced back to the reforming and pedagogical efforts of Donizetti and his colleagues. The same can be said about incorporating the names of the voice types (lexical items no. 17, 22, 43, 60) and their use within the names of instruments (no. 23). The derivatives and the compound terms of mixed origin such as \emph{flautacı, muzıḳacı, baṣo kemānı, ʿasker muzıḳası} and \emph{noṭa miḳyāsı} prove that in the second half of the 19th century new Italian loan-words were fully assimilated.\footnote{It is worth noting that in the 20th century most of them have been replaced by French borrowings and now, in modern Turkish (\emph{Türkçe}) are out of use, for example: Ott.Tur. \emph{flauta} had been replaced by Tur. \emph{flüt} < Fr. \emph{flûte}; Ott.Tur. \emph{konçerto} − Tur. \emph{konser} < Fr. \emph{concert}, Ott.Tur. \emph{çimbalo} – Tur. \emph{klavsen} < Fr. \emph{clavecin}. An investigation of change in Turkish musical terminology after the 20th-century language reform can be found in my unpublished PhD dissertation \citep{Pawlina2021}.}

Most importantly, the analysis of the Italian vocabulary presented above allows us to conclude that the 19th-century westernizing reforms of the Turkish musical culture are indeed perfectly reflected in the Ottoman Turkish lexicography of the period. The names of the European instruments, instrument players, musical bands and genres mirror the new elements of the cultural life of the Ottoman elite in Constantinople: the state military band (lexical items no. 20, 21, 30, 32, 33, 35, 42, 44, 45, 47, 48), opera (no. 17, 19, 22, 29, 39, 43, 52, 53, 60) and its orchestra (no. 23, 31, 37, 38, 54, 62), and the chamber ensemble instruments that would be played during private gatherings at the palaces (no. 18, 25, 26, 36, 56). Additionally, the theoretical terms recorded in specialized dictionaries of the period reflect the new pedagogical methods (no. 24, 34, 49, 50, 51, 55, 61), especially the five-stave musical notation, which until the 19th century had not been used in Turkey as a method of instruction of music. The new approach to the musical practice and theory can also be seen in the musicological content of definitions published in the specialized French-Turkish dictionary prepared by \citet{TinghirSinapian1891}. The Ottoman scholars described the solmisation names of pitches by comparing them to their “equivalents” on the Ottoman Turkish musical scale (see: lexical item no. 27). The same practice had been employed by the musicologist and pedagogue İsmail Kâzım \citet{Uz1892}.

In \sectref{sec:pawlina:4} there is only one Italian borrowing with a different historical background. The name of an Ottoman Turkish Eastern-style genre: \emph{lonġa} is enlisted as an Italian loan-word only in the \emph{Kubbealtı Lugati}\footnote{See: \url{http://lugatim.com/s/longa} (Accessed: 17.03.2023).}. Other historical and etymological dictionaries do not confirm this information. This fact, along with the musicological data on the genre, questions the presented etymology. However, if it is correct, it would be an important clue motivating further investigations regarding the impact of the language used by Italian Levantines on the musical LSP of the period.

Musicological research has already proved that the Greek (\emph{Rūm}) inhabitants of the Ottoman Empire played an important role in the history of the Ottoman art music since the late 18th century. The peak of this trend in musical practice happened in the 19th century (\citealt{Behar2006, Feldman1996, Greve2015}). In this context, it does not come as a surprise that the several technical terms denoting the phenomena of musical practice (lexical items no. 3, 4, 5, 10, 13) and the theory of music (no. 6) had only been recorded during the last period of the history of the Ottoman Turkish language (i.e., the \emph{New Ottoman Turkish}). These terms confirm that the Ottoman Turkish urban music of the 19th century was multicultural not only in the sphere of the sound, but also in its terminology, combining elements of traditions of various ethnic groups that lived in the capital of the Ottoman Empire.

On the other hand, among the Ottoman Turkish musical terms of Greek origin there are also borrowings belonging to the oldest stratum of the language (lexical items no. 1, 2, 8, 9, 11, 12). A denomination of ‘music’ − \emph{mūsiḳi} − has the most interesting history. Unsurprisingly, it is the only term that has been found in all 10 dictionaries selected for the current research. It was in use in the Ottoman Turkish since the 14/15th century, when the first Turkish translations of the Arabic and Persian treatises on the theory of music began to emerge \citep{Kalender1981}. The Greek term μουσική [mousiké] had been included into Arabic lexicon ({\textarab{موسيقى}} [mūsīqi]) much earlier, when in the 9th-10th centuries Arabic scholars adapted the theory of music of ancient Greeks and transformed it into the roots of the theory of Middle-Eastern music (\citealt{Danielson2002}, \citealt[215,290,341]{Oztuna1969}, \citealt[52--53]{ReinhardReinhard1984}). In the 20th century, the term \emph{mūsiḳi} was removed from the modern Turkish (\emph{Türkçe}) to be – eventually~− replaced by a French loan\hyp word: \emph{müzik} (< Fr. \emph{musique}) used today. Ottoman Turkish term in its contemporary spelling, \emph{musiki}, is described as an “obsolete” or omitted in most of the dictionaries written in the 20th century. Interestingly, in the newest dictionaries, it made a come back: it is sometimes enlisted as a name for the ‘Eastern-style music’ only (not ‘music’ in its broadest sense). The investigation of the written and spoken LSP of the contemporary Turkish musicians and musicologists in terms of the use of both terms, \emph{musiki} and \emph{müzik}, is an open research field.

Both groups of loan-words analysed in this paper – Greek and Italian – should be seen as \emph{cultural borrowings} \citep{Bloomfield1984}. In historical periods indicated in this paper, the Ottomans borrowed both the ideas (theory) and the physical objects (instruments) related to music, and along with them, they incorporated their names with a little or no change in their linguistic form. Based solely on the lexicographical data presented here, it is difficult to ascertain to what extent Greek and Italian languages used by the non-Turkish inhabitants of the Ottoman lands influenced the LSP of Ottoman musicians practicing the art/urban music. However, it can be said that such a linguistic influence existed (especially from Greek) and it requires further investigation. The answer to the second question raised above – how the westernization of the 19th-century Turkish music is reflected in the Ottoman Turkish lexicography − is straightforward: the ongoing change in the musical culture and even the pace of the assimilation of the ideas and the changing popularity of newly brought instruments can easily be traced by investigating the dictionaries.

It is worth emphasizing that my research on musical terms of the Greek and Italian origin found in the Ottoman Turkish lexicography proves that dictionary makers can be interpreted as the attentive observers of the socio-cultural processes surrounding any linguistic \citep{Siemieniec-golas2015} and extralinguistic (\citealt{Doroszewski1970, Ligara2017}) change. Because of that, a diligent reader can trace important extra-linguistic historical data on the pages of dictionaries. In the Ottoman Turkish context, it is particularly important to appreciate the value of information given in definitions written by Ottoman lexicographers (in my case: Şemseddin Sami, Anton B. Tinghir, Krikor Sinapian, İsmail Kâzım Uz), who were not only observers (like their contemporaries of European origin, such as James W. Redhouse, Jean D. Kieffer or Julius T. Zenker) but also, more importantly, the active participants of the Ottoman Turkish culture undergoing a significant change in the 19th century. Current research proves that the analysis of the extra-linguistic content of dictionary articles prepared by such authors may be useful information for research in other scientific disciplines within the framework of the Ottoman Studies – in my case, musicology. Lexicographical data perfectly reflects the socio-cultural change in this very specific element of the Turkish culture, and it can be used as a new source of information for future research on the history of Turkish music.

{\sloppy\printbibliography[heading=subbibliography,notkeyword=this]}
\end{document}

