\documentclass[output=paper,colorlinks,citecolor=brown,arabicfont,chinesefont]{langscibook}
\ChapterDOI{10.5281/zenodo.15394501}
\author{Ivett Kelemen\affiliation{University of Debrecen}}
\title[Hungarian aspects of North Saami dictionary writing]
      {Lexicon Lapponicum Bipartitum.....ungarice scriptum: Hungarian aspects of North Saami dictionary writing}

\abstract{The birth of North Saami lexicography in 18th-century Norway is closely related to the work of the Norwegian priest, Knud Leem (1697--1774). He was the first to publish a Danish–North Saami dictionary (En lappisk Nomenclator, 1756) to help the language learning of missionaries setting off for Saami territories. Later on, the “father” of the North Saami language continued his lexicographical work with a larger endeavour: he started to write a trilingual (North Saami–Danish–Latin) dictionary. The printing of the first volume of the 1600-page-long Lexicon Lapponicum started in 1768, but a year later, the work was suspended for a long period of time, because of unexpected “Hungarian happenings”. It is a well-known fact among Finno-Ugric scholars that during their 1768--1769 astronomical expedition to the Norwegian Island of Vardø, Maximilian Hell (1720--1792) and János Sajnovics (1733--1778) also studied the Saami language that was already considered by contemporary international linguists as potentially related to Hungarian. Sajnovics thought that the Hungarian way of spelling would have been more suitable for the recording of the Saami language and for the demonstration of linguistic affinity, as “the Hungarian and the Lapp languages must be identical just on the basis of pronunciation alone”. He put his theory into practice and with the help of the Saami-origin Anders Porsanger (1735--1780), he revised Leem’s lexicon. During their work, in addition to applying Hungarian spelling, they also complemented the entries with Hungarian data. The current paper presents the history of the Saami dictionary written in “Hungarian”.}

\IfFileExists{../localcommands.tex}{
  \addbibresource{../localbibliography.bib}
  % add all extra packages you need to load to this file

\usepackage{tabularx,multicol}
\usepackage{url}
\urlstyle{same}

\usepackage{listings}
\lstset{basicstyle=\ttfamily,tabsize=2,breaklines=true}

\usepackage{langsci-basic}
\usepackage{langsci-optional}
\usepackage{langsci-lgr}
\usepackage{langsci-osl}
% \usepackage{./langsci/styles/langsci-lgr}
% \usepackage{./langsci/styles/langsci-osl}
% \usepackage{langsci-gb4e}

\usepackage{tikz}
\usetikzlibrary{patterns,calc}
\pgfdeclarepatternformonly{south east lines}{\pgfqpoint{-0pt}{-0pt}}{\pgfqpoint{3pt}{3pt}}{\pgfqpoint{3pt}{3pt}}{
    \pgfsetlinewidth{0.6pt}
    \pgfpathmoveto{\pgfqpoint{0pt}{3pt}}
    \pgfpathlineto{\pgfqpoint{3pt}{0pt}}
    \pgfpathmoveto{\pgfqpoint{.2pt}{-.2pt}}
    \pgfpathlineto{\pgfqpoint{-.2pt}{.2pt}}
    \pgfpathmoveto{\pgfqpoint{3.2pt}{2.8pt}}
    \pgfpathlineto{\pgfqpoint{2.8pt}{3.2pt}}
    \pgfusepath{stroke}}
    
\usepackage{stmaryrd}
\usepackage{wasysym}
\usepackage{multirow}
\usepackage{caption}
\usepackage{subcaption}
\usepackage{mathrsfs}
\usepackage{qtree}

\usepackage{linguex}


  %pminos do not split footnotes
% \interfootnotelinepenalty=10000 %Footnote in Laporte chapters has to be split SN


%\DeclareIndexNameFormat{default}{%
%\nameparts{#1}%
%\usebibmacro{index:name}%
%{\index[names]}%
%{\namepartfamily}%
%{\namepartgiveni}%
% {}% L1
% {}% L2
%{\namepartprefix}% generates spurious space L3
%{\namepartsuffix}% generates spurious space L4
%}

%  {\DeclareIndexNameFormat{default}{%
%     \usebibmacro{index:name}{\index[names]}{#1}{#3}{#5}{#7}}}

%\DeclareIndexNameFormat{default}{%
%  \usebibmacro{index:name}{\sindex[nom]}{#1}{#3}{#5}{#7}}

%\DeclareIndexNameFormat{default}{%
%  \usebibmacro{index:name}{\sindex[person]}{#1}{#3}{#5}{#7}}
%\DeclareIndexNameFormat{default}{%
%\nameparts{#1} \usebibmacro{index:name}{\sindex[person]]}{\namepartfamily}{‌​\namepartgiven}{\nam‌​epartprefix}{\namepa‌​rtsuffix}}

%\newcommand{\smiley}{:)}

%\renewbibmacro*{index:name}[5]{%
%\usebibmacro{index:entry}{#1}%
%{\iffieldundef{usera}{}{\thefield{usera}\actualoperator}\mkbibindexname{#2}{#3}{#4}{#5}}}

% \newcommand{\noop}[1]{}

%remove for final
%\overfullrule=1mm

\newcommand{\tobi}[2]}}
\renewcommand{\S}[1]{\tobi{#1}{\textsc{*}}}

% this volume references
% puts: [this volume]
% already defined: \citetv
%\newcommand{\citepv}[1]{(\citeauthor{#1} \citeyear*{#1} [this volume])}
\newcommand{\citealtv}[1]{\citeauthor{#1} \citeyear*{#1} [this volume]}

%parentheses around example number
\newcommand{\pref}[1]{(\ref{#1})}

% in-text examples

\newcommand{\lnex}[1]{\textit{#1}} %target lang word
\newcommand{\lnlit}[1]{(lit.: `#1')} %literal reading
\newcommand{\lnlat}[1]{(#1)} % latinization
\newcommand{\lntrans}[1]{`#1'} %translation
\newcommand{\lnexl}[2]%
{\lnex{#1}{} \lnlat{#2}} % ex with latinization
\newcommand{\lnexlat}[3]{\lnex{#1}{} \lnlat{#2}{} \lntrans{#3}} % ex with latinization and tranl.

%ch01
\newcommand{\co}[1]{\mbox{\textbf{#1}}}

%ch09

\newcommand{\cyrbulg}[1]{\begin{otherlanguage*}{bulgarian}#1\end{otherlanguage*}}


%ch10
\newcommand{\nlp}{{\small NLP}}
\newcommand{\mwe}{{\small MWE}}
\newcommand{\rae}{{\small RAE}}
\newcommand{\lvc}{{\small LVC}}
\newcommand{\pos}{{\small P}o{\small S}}
%\newcommand{\todo}[1]{ \textcolor{red}{#1} }

%\renewcommand{\labelenumi}{\theenumi}
%\ainamefmt{{vv}{ll}{, ff}{, jj}} % fullname

\newcommand{\biberror}[1]{{\color{red}#1}}

\newcommand{\osenovaitem}{--~}
  %% hyphenation points for line breaks
%% Normally, automatic hyphenation in LaTeX is very good
%% If a word is mis-hyphenated, add it to this file
%%
%% add information to TeX file before \begin{document} with:
%% %% hyphenation points for line breaks
%% Normally, automatic hyphenation in LaTeX is very good
%% If a word is mis-hyphenated, add it to this file
%%
%% add information to TeX file before \begin{document} with:
%% %% hyphenation points for line breaks
%% Normally, automatic hyphenation in LaTeX is very good
%% If a word is mis-hyphenated, add it to this file
%%
%% add information to TeX file before \begin{document} with:
%% \include{localhyphenation}
\hyphenation{
    Beck-man
    Ngu-yen
    back-chan-nel
    back-chan-nels
    mo-not-o-nous
    ste-reo-typ-i-cal
}

\hyphenation{
    Beck-man
    Ngu-yen
    back-chan-nel
    back-chan-nels
    mo-not-o-nous
    ste-reo-typ-i-cal
}

\hyphenation{
    Beck-man
    Ngu-yen
    back-chan-nel
    back-chan-nels
    mo-not-o-nous
    ste-reo-typ-i-cal
}

  \togglepaper[17]%%chapternumber
}{}

\begin{document}
\maketitle

\section{Introduction}

1769 was a very promising year for the international scientific and academic society, with opportunities and prospects of historical scientific significance. On 3 June of this year, Venus passed between the Sun and the Earth. Through the observation of this astronomical event, astronomers had the means to calculate the distance between the Sun and the Earth. This explains why the transit of Venus triggered the most prominent scientific collaboration in the history of the world until that time, which made it possible for 151 scientists to travel to 77 locations across the Earth to achieve international recognition and glory for the most exact measurement of the parallax\footnote{The angle at which an observer would see the radius of the Earth from the Sun, and which can be used to calculate the distance between the Sun and the Earth.} of the Sun. Furthermore, these expeditions were of paramount importance not only from the aspect of astronomy but also contributed to several other areas of humanities and science (e.g. linguistics, ethnography, botany, zoology, etc.)

This is how Maximilian Hell (1720--1792) and János Sajnovics (1733--1785), Hungarian Jesuit astronomers, also made close observations on the transit of Venus on the island of Vardø, and at the same time discovered the linguistic affinity between the Hungarian and Saami languages through the examination of the native local people’s language.\footnote{Saami people (formerly known as Lapps or Laplanders) had never formed a separate, individual state. They have always lived in Sápmi (or Saamiland, formerly known as Lapland), a cultural region of 400,000 square kilometres, stretching across four countries (Norway, Sweden, Finland, and Russia). Saami belongs to the Uralic language family and it is not a unified language, but the sum of nine, separate variants: Southern Saami, Ume Saami, Pite Saami, Lule Saami, North Saami, Skolt Saami, Inari Saami, Kildin Saami, and Ter Saami. During their expedition, Sajnovics and his colleagues researched the North Saami dialect, which is the central Saami language variant, with the most speakers today.} In his work titled \emph{Demonstratio. Idioma ungarorum et lapponum idem esse} (1770), János Sajnovics provided a scientific demonstration of the relationship between the two languages by comparing the respective linguistic levels. This book is still considered one of the fundamental works of Finno-Ugric studies and comparative linguistics. Sajnovics also emerged as a prominent figure of the history of North Saami\footnote{The North Saami language is the largest, central Saami language, spoken by ca. 20,000 people. As the borders of the neighbouring countries never coincided with the borders of the Saami language, North Saami is spoken in three countries (Northern Norway, Northern Sweden, and Northern Finland). From the linguistic point of view, this dialect is the most researched one, and the majority of Saami dictionaries and grammars were written in this dialect.} orthography and lexicography by making observations and suggestions on the Saami language.

\section{The beginnings of Saami lexicography (Sweden)}

The beginning of the Christian missionaries’ activity was an important milestone in the evolution of the Saami languages – similarly to a number of other languages. Only Protestant missions were able to show significant results in this respect (unlike Catholic ones). The underlying reason was Reformation’s strategic emphasis on peoples’ native languages, on the importance of making the word of God available for everybody in their mother tongues. This is why the very first written works of the Saami culture were created by translating ecclesiastical texts into Saami. The literacy of Saami languages is very closely linked to missionary activities, as missionaries working among the Saami were the first to write and publish texts in the Saami language. In the 17th and 18th centuries, the Saami texts published in Sweden (Finland included) and (Denmark–)Norway\footnote{In 1536,  the Kingdom of Denmark and the Kingdom of Norway entered into a personal union, which lasted until 1814. During this time, political and economic control was exercised in Copenhagen, Denmark.} were of three kinds. In addition to descriptions of the Saami people and their traditions (especially regarding their pagan beliefs), translations of religious texts (catechisms and Bible excerpts) constituted the second type of text, while the third type consisted of dictionaries, grammars, and alphabet books for educational purposes to assist missionaries in their linguistic training \citep[8]{Aspaas2021}.

Christian missionary work was the most intense in Sweden, beginning to produce remarkable results in the 17th century. However, Saami lexicography only started to thrive in the 18th century. The very first instance of educational texts was the first Saami dictionary, published in Stockholm in 1738, by a Swedish school teacher and dean, Per Fjellström (1697--1764) from Lycksele.\footnote{The oldest settlement in the Swedish Lapland.} The book was a Swedish-Saami dictionary titled \emph{Dictionarium sueco-lapponicum}. It contained mainly Ume material, but also some Pite and Lule words,\footnote{Although Fjellström travelled to Pite and Lule areas in 1737, his work was based mainly on the Ume dialect, which he presumably acquired as a child \citep[154]{Toth2015}.} not being identified as such in the entries. This dictionary marked the first milestone of Southern Saami lexicography, and was the sole help of Saami translators for over 40 years until the Norwegian Knud Leem (1697–1774) started his activity as a linguist.

\section{The beginnings of Saami lexicology in (Denmark–)Norway: Knud Leem}

In Norway, missionary activity intensified in the 1700s, partly as a result of the devastating reports of the church and closely related school conditions in Northern Norway \citep[26--27]{Kjolaas1995}. In 1714,  Frederick IV established the College of Missions (Missionskollegiet) in Copenhagen to finance and manage Protestant missions under royal patronage. The principal goal of the institution was to organize a mission to India, but from 1716, it also managed the Saami missions.

The first prominent representative of Saami lexicography in (Denmark–)Nor\-way was the Norwegian scholar Knud Leem.

\subsection{The life and work of Knud Leem}

Knud Leem was born in 1715. At the age of 18, he received a degree in theology from the University of Copenhagen. As a fresh graduate, he was interested in Finnmark\footnote{It is the north-easternmost and largest (74 293 km2) county of Norway. It constitutes a part of Saamiland and a quarter of its population is made up of the Saami people.} and the Saami people living there. In 1723,  he requested the College of Missions to be appointed as a missionary among the Saami people of Finnmark. As it was important to him to be able to speak to the Saami in their own language, he started to learn it. Between 1723 and 1725, he studied North Saami at the Seminarium\footnote{Between 1717 and 1727, the institution provided Saami linguistic training for future priests and missionaries, preceding their theological studies in Copenhagen.} in Trondheim. In 1725,  he was granted permission to work as a missionary in Porsanger and Laksefjorden in Finnmark and thus had the opportunity to study the Saami language and culture more closely. Twenty-five years after its closure, the Seminarium Scholasticum was reopened in 1752, now called the Seminarium Lapponicum. Leem was appointed as head of the institution. He taught the Saami language at the Seminarium Lapponicum from 1752 to his death in 1774. In the meantime, he published a number of works. After his 400 page long grammar book,\footnote{\emph{En lappisk Grammatica efter den Dialect, som bruges af Field-Lapperne udi Porsanger-Fiorden: samt et Register over de udi samme Grammatica anførte Observationers Indhold, hvorhos er føyet et Blad af den berømmelige Historie-Skriveres Hr. Baron Ludvig Holbergs Kirke-Historie oversat i det Lappiske Tungemaal med en Analyse over hvert Ord.}} his first paper as a professor\footnote{In 1752, Leem was awarded the title „professor lingvæ lapponiæ” (for more details, c.f. \citealt{Magga2003}).} was a Danish–Saami glossary titled \textit{En Lappisk Nomenclator efter den Dialect, som bruges af Fjeld-Lapperne i Porsanger-Fjorden} [Saami glossary on the basis of the Mountain Saami dialect spoken around the Porsanger Fjord] published in 1756 in Trondheim.

\section{About work on the \textit{Lexicon Lapponicum}}

From 1753, \citeauthor{Leem1768} was working on his great work, the trilingual (North Saami–Danish–Latin) dictionary titled \textit{Lexicon Lapponicum}. The printing started in Trondheim\footnote{The first volume (ca. 42 arcs) was printed before the Sajnovics “incident”.} in  June of \citeyear{Leem1768}. The elderly Leem needed a permanent deputy due to his deteriorating health, which position was filled by Anders Porsanger (1735--1780).

\subsection{The conflict between Anders Porsanger and Knud Leem}

Anders Porsanger\footnote{His original name was Anders (Biret\hyp Ánde) Andersen, and he took on the Porsanger name as a tribute to his homeland \citep[9]{Solbakk1998}.} was also a pioneer in a way, as he was the first Saami person to hold a degree in theology. He was 15 years old when he was invited by Johan Ernst Gunnerus (1718–1773), Bishop of Trondheim\footnote{Johan Ernst Gunnerus was a patron of sciences, a botanist himself, and professor of theology at the University of Copenhagen. He was the Bishop of Trondheim from 1758 until his death. He was also among the founding members of the Trondheim Society of Sciences.} to help Leem. He was studying at the local cathedral school at that time, completed his theological studies in Copenhagen in 1716, and became a missionary in Varanger\footnote{The easternmost administrative unit of Finnmark.} in the following year. In 1764,  he returned to Trondheim and started to work as a senior lecturer at the Seminarium Lapponicum, and as a hospital priest at the local hospital. Porsanger had numerous duties in addition to dictionary writing since he was obliged to undertake other jobs to earn a living. This hindered work on the dictionary, which made Leem very dissatisfied. Although the manuscript of the second volume was already in progress during the publishing work on the first volume, the tension that was present between Leem and Porsanger from the beginning continuously heightened. The fact that Porsanger was expected to undertake more and more responsibilities added to the issue. Leem only taught a few classes in the Seminary by this time, and he also did little work on the dictionary. Furthermore, Porsanger was asked to write a report\footnote{The report is available in \citet{Martinussen1992}.} for the College on Saami orthography in September of 1767, in which he criticised Leem’s difficult spelling system.\footnote{For more details, see \citet{Kelemen2019}.} The conflict between them reached its peak in 1769 when the \emph{Lexicon} suffered another “attack”.

\section{Hungarians in the North}

The 1769 transit of Venus attracted the attention of Christian VII (1749–1808), the Danish-Norwegian king as well, who was interested in astronomy. He had an observatory built on the Norwegian Island of Vardø and organised an expedition, providing considerable financial support. Christian wrote a letter to Maria Theresa and asked her to nominate an expert from the Vienna Observatory who had the required skills. The empress was well-known as a patron of the sciences, so she supported the planned expedition and invited the director of the Vienna observatory, Maximilian Hell\footnote{Maximilian Hell was born in 1720, in Banská Štiavnica (hun. Selmecbánya), in today’s Slovakia. He joined the Jesuit Order in 1738 and continued his studies in Vienna from 1740. He got acquainted with the science of astronomy in the Jesuit Observatory of Vienna. In the September of 1755, he was appointed as director of the newly founded Observatory and as professor of mechanics at the University of Vienna. After the dissolution of the Jesuit order in 1773, Hell joined the secular clergy and continued his versatile academic work. He died in Vienna, in 1792, at the age of 72.} for the mission. He gladly accepted the invitation, and as he needed an assistant, he chose a former colleague of his, who was also a Jesuit: János Sajnovics (for further details, see \citealt[403--404]{MaticsakKeleman2021}).\footnote{It is not known why the king needed an expert from Vienna. The assignment of Hell (and Sajnovics) was especially interesting, because the presence of Jesuits was forbidden at the time in (Denmark-)Norway. It seems likely that the prestige of this scientific exploration was more significant than the religious ban (c.f. \citealt[186]{AspaasKontler2019}).}

\subsection{About János Sajnovics}

János Sajnovics was born in Tordas,\footnote{A village in Central Transdanubia, Hungary.} Hungary, in 1733. He studied at the prestigious Jesuit Grammar School of Győr\footnote{Győr is the most important city of northwest Hungary, the capital of Győr-Moson-Sopron County.} and then in Buda\footnote{It was one of the historical capital cities of the Kingdom of Hungary. In 1873, Buda, Óbuda, and Pest were officially unified and given the new name Budapest.} from 1747. He showed an early interest in science, especially mathematics and astronomy. In 1748, he joined the Jesuit Order and in 1751 he began his philosophical studies in Trnava.\footnote{Nagyszombat in Hungarian.} After completing his studies, he started working as a teacher in Pozsony (modern-day Bratislava). A few years later, he started to study theology at the College of Vienna, also supervised by the Order, where he became acquainted with astronomy at a professional level.  His superiors soon recognized his talent and moved him to the observatory of Vienna to work with Maximilian Hell (who was already well-known internationally), and later, in 1766, to the Trnava observatory to work with Ferenc Weiss (1717–1785), prominent mathematician, astronomer, and Jesuit priest. The 1773 dissolution of the Jesuit Order caused a break in Sajnovics’s career as well: he became a deputy astronomer in the Buda observatory. Toward the end of his life, he became more and more withdrawn and died at the young age of 52, in 1785.

\subsection{Hungarian and Saami are “identical”}

After Hell’s invitation, Sajnovics arrived in Vienna on 12 April 1768,  and at the end of the month, they set off on the long journey of nearly six months. They traveled through Prague, Dresden, Leipzig, Hamburg, and Lübeck to reach their destination: Copenhagen, then called Hafnia. The rough sea voyage finally came to a happy end on 11 October: the two scientists arrived at the Island of Vardø. During the months spent in Vardø, they made preparations to observe Venus and did other types of scientific research as well (e.g., on the sparkling of the Arctic Ocean, auroras, and tidal patterns). In addition to scientific observations, they also started to study the Saami language.

Hell already had some preliminary information on the presupposed linguistic affinity between the Hungarian and the Saami languages. This is why it was thought for a long time that linguistic research was also planned even before the journey and, as Hell did not speak good Hungarian, he chose Sajnovics knowingly, because of his linguistic competence. However, now it has become evident that the two scholars started their linguistic research so “unprepared”, that it could not have been deliberately planned.\footnote{They did not take any kind of literature along. They received all material – even the Hungarian ones – during their stay abroad.}

\subsection{“Decoding” Leem’s spelling system}

For his linguistic research, Sajnovics used Knud Leem’s above-mentioned grammar book and Danish–Saami glossary; however, he had considerable difficulty with the interpretation of these works:

\begin{quote}
    When I opened up the grammar book, I saw a long list of rules, however, they were in Danish, a language I was completely unfamiliar with ... When I took a look at the Nomenclator, I saw Lapp words, consisting almost exclusively of vowels and some curiously selected consonants \citep[36]{Sajnovics1994}.
\end{quote}

He was very discouraged and sadly reported to Hell that he saw no connection between the two languages. However, they did not give up, and took the following steps to find useful language material. First, they collected words from the \textit{Nomenclator} that were similar to Hungarian words in form. Then they looked for the Latin equivalents of their Danish meaning. For this step, they used historian Hans Gram’s (1685--1748) Latin–Danish dictionary titled \emph{Nucleus Latinitatis} [The nucleus of Latin] (published in Copenhagen in 1722). The extremely tedious work often ended in failure, either because the word chosen was not in the \textit{Nucleus}, or because the meaning of the Saami word and the Hungarian word it was intended to be related to were completely different after all \citep[36]{Sajnovics1994}. They were already experimenting with novel phonetic notations at this time, but their work only yielded reliable results when native Saami speakers arrived and, with the help of an interpreter, they started asking them about various words. In these words, they saw a such similarity – both in their pronunciation and their meaning – that Sajnovics went as far as to say that these were in fact Hungarian words \citep[37]{Sajnovics1994}. During these personal encounters, Sajnovics put down the words heard in Hungarian spelling, and he managed to compile a very rich corpus. From this point on, he worked with much more enthusiasm, and the work progressed faster. As a next step, within a few weeks’ time, he detected the Latin equivalents of all the headings of the \textit{Nomenclator}, making it easier to observe the spelling system and provide the rules for converting it to Hungarian spelling.

The painstaking work was finally completed in 1770: as an argument for the linguistic affinity between the Hungarian and Saami languages, Sajnovics published his work titled \emph{Demonstratio. Idioma Ungarorum et Lapponum idem esse} in Copenhagen. This work is considered one of the fundamental works of Finno-Ugric studies and comparative linguistics today.\footnote{In recognition of his work, he was admitted as a member of the Norwegian Academy of Sciences on 5 February 1770 and of the Danish Academy of Sciences on 9 February 1770 – as the second foreigner after Hell \citep[135, 137]{Sajnovics1990}.}

Sajnovics’s work also concerned the field of North Saami orthography and lexicography: he confidently claimed that Hungarian spelling would facilitate the description of the North Saami language. This was not a mere suggestion on his part, as he also intended to realize his idea: he wanted to publish the \textit{Lexicon Lapponicum}\footnote{For a history of the lexicon see \citet{Kelemen2018}:  and for a detailed description see \citet{Djarv2003}} with Hungarian spelling, even though it was already in press.

\section{The dictionary of the Hungarian Jesuit and the Saami theologian}

In fact, the idea of introducing Hungarian orthography had already come up on the Island of Vardø, during the study of the Saami language and the description of the linguistic affinity. Hell (as the idea was his) decided to take concrete steps to reform Saami spelling:

\begin{quote}
    Reverend Father HELL’s main wish was to introduce the solid Hungarian spelling practice into the Lapp books and texts ... Reverend Father HELL considered two more things necessary. \uline{Firstly}, LEEM’s dictionary, which is about to be republished, and the Grammatica should be given a Hungarian appearance. \uline{Secondly}, there should be a Laplander who, having learned the Hungarian spelling from me, would supervise the printing of the dictionary and the Lapp training of the missionaries and teachers to be sent to Laponia. In addition, it is necessary to invite to Copenhagen ... a man who is well versed in the Lapp language, whom I could personally teach to the Hungarian spelling, and with whom I could adapt the LEEM dictionary to the Hungarian spelling \citep[42--43]{Sajnovics1994}.
\end{quote}

With this agenda, they approached Otto Thott (1703--1785),\footnote{Danish count and statesman. He held several positions at the time, e.g. he was a member of the Privy Council, Highest Chancellor of the Danish Chancery, etc.} then-president of The Royal Danish Academy of Sciences on 13 February 1770: “We handed him in writing the arguments that would convince him that the great Lapp lexicon ought to be printed with Hungarian orthography” \citep[135]{Sajnovics1990}. Thott was very enthusiastic about the idea. For his request, Bishop Gunnerus appointed Leem’s assistant, Anders Porsanger, to help Sajnovics re-edit the dictionary. The choice could not have been more perfect, not only because Porsanger was a native speaker of Saami, but also because he had worked with the professor since he was 15 years old, and had helped him edit the dictionary he was transcribing, so he knew the material well. Sajnovics was aware of how unique a Saami man his “colleague” was, having been raised “in a way very different from the ways of his homeland” and having chosen to become a scholar in a period when the majority of Saami people were undereducated or outright illiterate.\footnote{“This man came from Porsanger and studied in Nidrosia (i.e. Trondheim) as a young man. He received an excellent education in the field of humanities there. At the University of Copenhagen, he studied philosophy and theology. He spent three years as a missionary among his fellow Saami people, and then was appointed as a priest at the church of the Trondheim hospital. […] This excellent scholar spoke not only his Saami mother tongue but also Danish, German, Latin, Greek and Hebrew, to the glory of the talented Saami people” \citep[43]{Sajnovics1994}.}

Sajnovics provided a very detailed description of the nature of changes they made in Leem’s material in Chapter IV of the \textit{Demonstratio}’s second, Trnava edition.\footnote{The first edition, published in Copenhagen in 1770, did not contain the chapter describing the circumstances of the dictionary’s transcription.}

Porsanger himself also read the \textit{Demonstratio} and thus regarded the linguistic affinity of the two languages as established. As Sajnovics reported, Porsanger was quick to get acquainted with the Hungarian way of spelling. They spent entire days working together, to have the dictionary ready for printing. Porsanger was reading the Saami words aloud while Sajnovics was recording in writing. Afterwards, the glossary was organized in alphabetical order. In Sajnovics’s version of the dictionary, the headwords were spelt according to Hungarian orthography, each followed by a Hungarian synonym (if available), then the Danish and Latin equivalents. At the end of each entry such Hungarian words were included in italics (together with their Latin meanings) as were similar to the headword but had different meanings (see Table 1).\footnote{Sajnovics noted these for future reference, in case “another meaning of the Lapp word should be found” \citep[44]{Sajnovics1994}.}

\begin{table}
\caption {A comparison of dictionary entries by \textcite[44]{Sajnovics1994} and by \textcite[7]{Leem1768}}
\label{tab:kelemen:frequencies}
\begin{tabularx}{\linewidth}{*{2}{lQ}}
\lsptoprule
Leem’s entry: & ABOM, syder over, efferveo. 3 sing. præs. ind. Abbo. 2 sing. præs. imp. Abo. \\
Sajnovics’s entry: & ABOM forrok, Syder Ower, efferveo. \emph{hab} (spuma, unda), \emph{habzom} (spumo, undo). \\
\lspbottomrule
\end{tabularx}
\end{table}

Finally, Sajnovics wrote the preface of the book, providing extensive information on a) the grounds for changing the original spelling, b) the changes themselves, c) the status and rules of the contemporary Hungarian orthography, d) the pronunciation of the vowels and the consonants, and e) the benefit of the endeavor in general \citet[44]{Sajnovics1994}.

They completed their work in May 1770  and handed over the manuscript to Otto Thott, who promised to send some copies to Hungary when the Saami dictionary with the “Hungarian spelling” was printed. Sajnovics and Hell set off from Copenhagen to Hungary on 22 May 1770,  leaving behind the Danish-Norwegian academic community in turmoil because of the rewritten dictionary.

\section{The fate of the manuscript}

The fate of the dictionary manuscript was documented in Bishop Gunnerus’s correspondence: in his letter written on 13 December 1770,  the bishop asked for Leem’s opinion on Sajnovics’s spelling proposal. Two days later, he also notified the Missionary College about handing the “material” to Leem, who was still reviewing the work \citep[49--50]{Dahl1906}.

On 22 April 1771, Gunnerus sent two copies of the dictionary, one of Leem’s and one of Sajnovics’s version, to Christian Weldingh (1728–1801), a Danish-born missionary. His task was, similarly to that of another missionary, Gerhard Sandberg (1741--1805) (who published the book at the end), to evaluate which way of spelling would be ideal in the dictionary independently of each other.

The issue was left unresolved. Gunnerus made no more mention of Saami orthography or the issue of the dictionary in his surviving letters. The next letter touching upon this subject informed the readers that the publication of the dictionary was moving forward with the letter M, in Leem’s original version \citet[54]{Dahl1906}. The full dictionary was published as late as 1781, together with its reverse version (Danish–Latin–Saami).

The manuscript of Sajnovics’s dictionary disappeared without a trace. Neither the (Danish–)Norwegian nor the Hungarian side had any more information about it. However, it is obvious that everybody who was involved abandoned the subject. Sajnovics and Hell returned home, and Porsanger’s career took an unfortunate turn after their joint project (as he was sent on a mission again and his academic work ended for good). In December 1770,  Christian VII’s personal physician, Johann Friedrich Struensee (1737–1772) was appointed, and Otto Thott was removed from office. Thus, the idea of rewriting the dictionary remained without any local supporters. In addition, the Missionary College of Copenhagen intended to publish Leem’s dictionary very quickly and in a cost-effective manner, and as the Hungarian diacritical marks were new to printing technology, the Hungarian spelling would have made the printing more expensive. Not to mention the fact that these signs would have seemed a bit out-of-place in the Scandinavian linguistic context, not accommodating well to the then popular Gothic script either.

\section{Summary}

Although research on Sajnovics’s work is extensive both within Hungary and abroad, discussing almost all aspects of his work, the history of the Saami dictionary’s manuscript, and Sajnovics’s efforts to apply Hungarian spelling in it have received very little attention. I was the first to present this topic to the Hungarian research community in 2020 and 2021, within the scope of the 250-year centenary lecture series on the \emph{Demonstratio}. The present paper is intended to reflect on the 1770 Copenhagen events, the trilingual Saami dictionary, and János Sajnovics’s dictionary writing activity in the international context as well.

This research opens up new perspectives on the Saami history of science: besides the presentation of the manuscript’s history (which already addresses a research niche), my ultimate goal is to retrieve the manuscript through international, in particular Norwegian/Danish, scientific cooperation, thus adding to Sajnovics’s legacy. If the manuscript exists, after a detailed linguistic (lexical, lexicological, orthographic, etc.) analysis, I intend to present it in a monograph, first to the Hungarian and then to the international researchers interested in this topic. 

{\sloppy\printbibliography[heading=subbibliography,notkeyword=this]}
\end{document}
