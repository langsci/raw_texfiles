\documentclass[output=paper]{langscibook}
\ChapterDOI{10.5281/zenodo.15394473}
\author{Daphne Preston-Kendal\affiliation{Humboldt-Universität zu Berlin}}
\title[On the dangerous fragility of digital publication]
      {Dictionaries in the web of Alexandria: On the dangerous fragility of digital publication}

\abstract{Digital publication is an attractive means of making dictionaries available to the general public, as has been widely documented by lexicographers and dictionary readers alike since the beginning of the “digital revolution” in dictionary publication over thirty years ago. However, commonly adopted means of publishing dictionaries pose dangers to scholarly integrity and to the long-term survival of lexicographical works. Adoption of practices used for the publication of some other online reference works and academic papers, indeed mirroring the survival characteristics of paper dictionaries, can help to mitigate these issues.}

\IfFileExists{../localcommands.tex}{
  \addbibresource{../localbibliography.bib}
  % add all extra packages you need to load to this file

\usepackage{tabularx,multicol}
\usepackage{url}
\urlstyle{same}

\usepackage{listings}
\lstset{basicstyle=\ttfamily,tabsize=2,breaklines=true}

\usepackage{langsci-basic}
\usepackage{langsci-optional}
\usepackage{langsci-lgr}
\usepackage{langsci-osl}
% \usepackage{./langsci/styles/langsci-lgr}
% \usepackage{./langsci/styles/langsci-osl}
% \usepackage{langsci-gb4e}

\usepackage{tikz}
\usetikzlibrary{patterns,calc}
\pgfdeclarepatternformonly{south east lines}{\pgfqpoint{-0pt}{-0pt}}{\pgfqpoint{3pt}{3pt}}{\pgfqpoint{3pt}{3pt}}{
    \pgfsetlinewidth{0.6pt}
    \pgfpathmoveto{\pgfqpoint{0pt}{3pt}}
    \pgfpathlineto{\pgfqpoint{3pt}{0pt}}
    \pgfpathmoveto{\pgfqpoint{.2pt}{-.2pt}}
    \pgfpathlineto{\pgfqpoint{-.2pt}{.2pt}}
    \pgfpathmoveto{\pgfqpoint{3.2pt}{2.8pt}}
    \pgfpathlineto{\pgfqpoint{2.8pt}{3.2pt}}
    \pgfusepath{stroke}}
    
\usepackage{stmaryrd}
\usepackage{wasysym}
\usepackage{multirow}
\usepackage{caption}
\usepackage{subcaption}
\usepackage{mathrsfs}
\usepackage{qtree}

\usepackage{linguex}


  %pminos do not split footnotes
% \interfootnotelinepenalty=10000 %Footnote in Laporte chapters has to be split SN


%\DeclareIndexNameFormat{default}{%
%\nameparts{#1}%
%\usebibmacro{index:name}%
%{\index[names]}%
%{\namepartfamily}%
%{\namepartgiveni}%
% {}% L1
% {}% L2
%{\namepartprefix}% generates spurious space L3
%{\namepartsuffix}% generates spurious space L4
%}

%  {\DeclareIndexNameFormat{default}{%
%     \usebibmacro{index:name}{\index[names]}{#1}{#3}{#5}{#7}}}

%\DeclareIndexNameFormat{default}{%
%  \usebibmacro{index:name}{\sindex[nom]}{#1}{#3}{#5}{#7}}

%\DeclareIndexNameFormat{default}{%
%  \usebibmacro{index:name}{\sindex[person]}{#1}{#3}{#5}{#7}}
%\DeclareIndexNameFormat{default}{%
%\nameparts{#1} \usebibmacro{index:name}{\sindex[person]]}{\namepartfamily}{‌​\namepartgiven}{\nam‌​epartprefix}{\namepa‌​rtsuffix}}

%\newcommand{\smiley}{:)}

%\renewbibmacro*{index:name}[5]{%
%\usebibmacro{index:entry}{#1}%
%{\iffieldundef{usera}{}{\thefield{usera}\actualoperator}\mkbibindexname{#2}{#3}{#4}{#5}}}

% \newcommand{\noop}[1]{}

%remove for final
%\overfullrule=1mm

\newcommand{\tobi}[2]}}
\renewcommand{\S}[1]{\tobi{#1}{\textsc{*}}}

% this volume references
% puts: [this volume]
% already defined: \citetv
%\newcommand{\citepv}[1]{(\citeauthor{#1} \citeyear*{#1} [this volume])}
\newcommand{\citealtv}[1]{\citeauthor{#1} \citeyear*{#1} [this volume]}

%parentheses around example number
\newcommand{\pref}[1]{(\ref{#1})}

% in-text examples

\newcommand{\lnex}[1]{\textit{#1}} %target lang word
\newcommand{\lnlit}[1]{(lit.: `#1')} %literal reading
\newcommand{\lnlat}[1]{(#1)} % latinization
\newcommand{\lntrans}[1]{`#1'} %translation
\newcommand{\lnexl}[2]%
{\lnex{#1}{} \lnlat{#2}} % ex with latinization
\newcommand{\lnexlat}[3]{\lnex{#1}{} \lnlat{#2}{} \lntrans{#3}} % ex with latinization and tranl.

%ch01
\newcommand{\co}[1]{\mbox{\textbf{#1}}}

%ch09

\newcommand{\cyrbulg}[1]{\begin{otherlanguage*}{bulgarian}#1\end{otherlanguage*}}


%ch10
\newcommand{\nlp}{{\small NLP}}
\newcommand{\mwe}{{\small MWE}}
\newcommand{\rae}{{\small RAE}}
\newcommand{\lvc}{{\small LVC}}
\newcommand{\pos}{{\small P}o{\small S}}
%\newcommand{\todo}[1]{ \textcolor{red}{#1} }

%\renewcommand{\labelenumi}{\theenumi}
%\ainamefmt{{vv}{ll}{, ff}{, jj}} % fullname

\newcommand{\biberror}[1]{{\color{red}#1}}

\newcommand{\osenovaitem}{--~}
  %% hyphenation points for line breaks
%% Normally, automatic hyphenation in LaTeX is very good
%% If a word is mis-hyphenated, add it to this file
%%
%% add information to TeX file before \begin{document} with:
%% %% hyphenation points for line breaks
%% Normally, automatic hyphenation in LaTeX is very good
%% If a word is mis-hyphenated, add it to this file
%%
%% add information to TeX file before \begin{document} with:
%% %% hyphenation points for line breaks
%% Normally, automatic hyphenation in LaTeX is very good
%% If a word is mis-hyphenated, add it to this file
%%
%% add information to TeX file before \begin{document} with:
%% \include{localhyphenation}
\hyphenation{
    Beck-man
    Ngu-yen
    back-chan-nel
    back-chan-nels
    mo-not-o-nous
    ste-reo-typ-i-cal
}

\hyphenation{
    Beck-man
    Ngu-yen
    back-chan-nel
    back-chan-nels
    mo-not-o-nous
    ste-reo-typ-i-cal
}

\hyphenation{
    Beck-man
    Ngu-yen
    back-chan-nel
    back-chan-nels
    mo-not-o-nous
    ste-reo-typ-i-cal
}

  \togglepaper[3]%%chapternumber
}{}

\begin{document} 
\maketitle

\section{Models of digital reference publishing}\label{models-of-digital-reference-publishing}

The most common model adopted when publishing dictionaries online today
is one which I have called the Oxford Model, after the \emph{Oxford
English Dictionary} Online which was probably the first to adopt it in
full at its launch in 2000.\footnote{To be clear, I do not blame the
OED editorial team nor Oxford University Press for adopting this model, though I am about to criticize it heavily. The digitization of the OED in the 1980s and development through,the 1990s into the online dictionary which launched in 2000 was, in every way, a pioneering process. The OED website first went online during the comparatively early days of the World Wide Web, when it was not yet clear what the implications and dangers of this model would be.

This is also not to say I consider the Oxford University Press entirely responsible stewards of the OED in this regard. However, 23 years later, a shift to a more sustainable model should certainly have been made long ago. The Oxford University Press has declined requests from scholars for access to digital copies of older OED editions. Decision-making with regards to the OED in recent years has appeared to be increasingly profit-motivated, to the detriment of the long-term scholarly value of the work.}

In the Oxford Model, nobody can purchase their own complete copy of the
dictionary to keep on their own computer. Instead, a “subscription” is
offered for sale, allowing the subscriber to view individual dictionary
entries over the Internet. These entries are loaded one at at time from
the one and only complete copy of the dictionary, which exists only on
the servers of the publisher.

Because there is only one complete copy of the dictionary held on the
publisher's servers, and because readers as subscribers can only load
entries as they currently appear in that copy, readers have no guarantee
that the entry they load today will have the same content as when they
viewed it previously, or whether it might change (again) before they
view it again in the future. The ability to publish the newest research
immediately and perform ongoing revision of a dictionary is of course a
great benefit of digital publication, but under the Oxford Model,
readers have no way to go back and see exactly how an entry has been
revised and when.

As subscribers to the dictionary website, rather than owners of a
complete copy, readers are also beholden to the dictionary publisher on
an ongoing basis. If they stop paying their subscription fee, they lose
the ability to read the dictionary and cannot even see entries they have
previously looked at, unless they had the foresight to save them to
their local disk. On the other hand, if the publisher were to go out of
business, decide that hosting the website was too unprofitable, or
otherwise lose interest, they may simply take the website offline, at
which point nobody can read the dictionary any more.

The Oxford Model in its complete form is the most irresponsible way to
publish dictionaries online from the point of view of ensuring
long-term, sustainable accessibility. The reasons for this, if they were
not immediately clear from this description, will be explored in more
detail below. Despite this, the Oxford Model is also very widespread: in
the German national database of electronic academic resources, the
\emph{Datenbank\hyp Infosystem},\footnote{\url{https://dbis.ur.de/}} 1,373 resources described by “dictionary,
encyclopedia, reference work” (\emph{Wörterbuch, Enzyklopädie,
Nachschlagewerk}) are accessed over the web in exchange for a licence
fee as of writing.

There are a few variants of the Oxford Model in use by some
dictionaries, each of which tempers the problems with it. Some online
dictionaries are published under an Open Access Oxford Model, in which
there is no subscription fee, but the publication in other respects
follows the system outlined above. Others have adopted an Oxford Model
Without Revisions, where there is at least an informal policy that
entries are not revised after they have been published (except perhaps
to fix typographical errors and other, very minor details, of the same
variety of change which may also be made between printings of the same
edition of a paper dictionary). While both of these variants maintain
some problematic aspects of the Oxford Model, they do at least mitigate
some of the others. How and why will also be explored below.

\section{The dangers of centralization}\label{the-dangers-of-centralization}

A key aspect of the paper model of publication is the distribution of
complete copies for a one-time cost borne by the purchaser. While this
may seem obvious, it is also notably different from the Oxford Model
described above, to the extent that proposals to reinstate this crucial
aspect within the context of digital publication can seem radical by
comparison.

Why, then, is the distribution of dictionaries from a centralized
authority only so problematic? Some problems of models in which the only
complete copy of a text is held by the publisher are outlined above. The
technological structure of the World Wide Web essentially encourages
this model for the publication of all kinds of media online, but this
situation is bad for the long-term accessibility of that media. New
media researcher Bret Victor has informally called this situation the
“Web of Alexandria”, in reference to the burning of the library of
Alexandria, a disastrous loss of knowledge caused by the failure to make
and distribute duplicate copies of works which were instead held in one
central location \citep{Victor2015}. While physical natural disasters or
human accidents such as file deletion remain a danger to digital texts
stored on the servers of a single institution, this model is insidious
in other ways.

One has to do with how one of the advantages of digital publication~--
the ability for scholarly dictionaries to be continually revised to
reflect the latest research~-- is applied in a centralized system. It
has become quite normal for entries to be revised in their online
editions. A description of how this occurs in practice under the Oxford
Model is described in \citet{Simpson2012}: the OED1 and OED2
sub-entry for \emph{market garden} was upgraded to a full entry in 2000,
the first year of OED3 entry publication. At the time, the oldest
quotation was from 1811, an improvement on the 1840 date given in the
older edition. Two subsequent revisions pushed this back first to 1793,
then to 1727. However, these revisions are “invisible” to the reader,
and there is no way to trace when these antedatings were added. Indeed,
there is no way to trace the specific dates of any changes to
OED3 articles after their initial publication, nor what they
looked like at a particular previous point in time.\footnote{Recently
  the OED3 website has added an indication of the month and year
  in which each entry was last revised, but there remains no indication
  of what was actually changed at each revision, nor any way to actually
  go back to an older version. This indication also often appears to be
  inaccurately forward-dated: as of writing, the entry for \emph{market
  garden} claims the entry was last revised in March 2022:  but there are
  no textual differences compared to the version of 2012 which appears
  in Simpson's paper.} It is entirely conceivable that a study of, for
example, word coinages of the early 19th century might have cited this
OED3 entry at the point in time when it claimed the word was
coined in 1811. Future readers of that study following this claim back
to the source are left to wonder why its author missed or disregarded
the earlier uses since added. This alone undermines basic scholarly
practice.

Market forces and changing institutional interests can also pose a
danger to the long-term accessibility of dictionaries under the
centralized digital publication model. If a publisher decides a
dictionary project is too unprofitable or loses interest, it may simply
decide to remove the dictionary from the web entirely, so it becomes
effectively entirely lost to history. We have already witnessed cases
where publishers have made decisions like these: the tragic case of the
\emph{Historical Dictionary of American Slang} (henceforth HDAL)(\citealt{Lighter1994, Lighter1997}),
cancelled by its publisher after only two of its planned four volumes,
is well known. In a paper publication model, the completed dictionary
parts are at least accessible by those who have already purchased
copies; in an Oxford Model online environment, this kind of
commercially-motivated decision would simply result in the total
inaccessibility of the work already done. In fact, we have even seen a
very similar dictionary become a victim of exactly this failure mode of
the Oxford Model: \emph{Partridge Slang Online}, an online edition of
the \emph{New Partridge Dictionary of Slang and Unconventional English}
(\citealt{Dalzell2005}), disappeared without trace from the web after
about three years without any apparent explanation.\footnote{\emph{Partridge
  Slang Online} was online at
  \url{http://www.partridgeslangonline.com/} from approximately January
  2013 to September 2016  according to the Internet Archive WayBack
  Machine, which has archived the home page, front matter, and a small
  number of sample entries, but not the dictionary as a whole. After
  September 2016:  only error pages are archived.} Fortunately in this
case, the dictionary had also previously been published on paper, so the
dictionary text remains accessible in libraries and on private
bookshelves around the world, albeit without updates, advanced search
tools beyond alphabetical headword lookup, and other benefits of digital
publication.

Finally, while the ability to update dictionary entries on an ongoing
basis is certainly a benefit of digital publication, the practice of
doing this in a centralized publication model makes the dictionary
vulnerable to political attacks. Numerous dictionary projects, even
those of a scholarly nature, have been the victims of political
interference typically by totalitarian regimes,\footnote{As examples,
  see for descriptions of the ideological pressures exercised on
  lexicographers under the two dictatorial regimes in Germany in the
  20\textsuperscript{th} century, \citep{Lea2009} and \citep{Zielinski2010}.} but those regimes could
not retroactively alter the contents of volumes of dictionaries already
published (except by official edicts to libraries to censor their
copies, which is reported to have happened in some cases). With
centralized digital publication, this kind of change to entries for
political reasons can be carried out immediately, affecting all users,
and potentially without any trace of the older versions remaining.

None of this, however, need dissuade us from digital publication and its
benefits entirely. Instead, an alternative model is required which
retains these advantages over print while providing the long-term
stability which was intrinsic to printed copies. As we will see, the
low cost of digital publication can actually even lead to better and
more widespread long-term accessibility than is possible with paper,
provided the right technical and administrative choices are made about
how the dictionary is actually published.

\section{An alternative model}\label{an-alternative-model}

While digital dictionaries, especially scholarly dictionaries, are
unique among digital texts in the extent and importance of their
microstructure, lexicographers can still hope to learn from the
experiences of other kinds of scholarly resources in exploring the
possible models of digital publication.

A particularly instructive model, one which I hold to be the gold
standard for digital publication of scholarly works, is that adopted by
the \emph{Stanford Encyclopedia of Philosophy} (SEP, \citealt{Zalta2006}; see also \citealt{HammerZalta1997, AllenNodelmanZalta2002}). In the Stanford
Model, an endowment has been established under a membership model which
universities, libraries, and other institutions may join for a
(significant, but one-time) fee. In exchange for this fee, the
institutions obtain the right (and implicitly the responsiblity) to
archive the SEP on their own computers in perpetuity; if the
SEP project ever closes down or becomes otherwise unavailable,
they then gain the further right to re-host the entire encyclopedia on a
public website to ensure general access to the text remains available.
The encyclopedia is revised in versions which carry the dates of their
publication explicitly, and all older versions of the encyclopedia and
of the individual articles remain available in archived editions, both
on the official website and in the archived versions held by member
institutions. Finally, because the membership scheme supports an
endowment and only the returns on the investment are used to actually
support the project, Stanford promises to return the membership fee with
interest to members should the SEP project ever shut down. As of
writing, over 500 institutions worldwide have become members of this
scheme.\footnote{\url{https://plato.stanford.edu/support/commitments.html}}

This model in its entirety is incredibly ambitious. However, I will
outline the most important aspects which can be applied to digital
lexicography, dealing first with what might be termed the technical
aspects, then an overview of the financial aspects.

The first technical aspect, as implied by the previous section, is that
each member institution has under its control complete copies of the
encyclopedia. They thus retain access to it, no matter what should
happen to the producers and publishers. As applied to digital scholarly
dictionaries, however, it is important not only that the text itself be
made available, but that it be in machine-readable form; that is, with
the microstructure encoded in a database file usually using descriptive
XML or SGML markup. It is this machine-readable database structure which
enables the advanced search capabilities which users of digital
scholarly dictionaries have come to expect and rely upon. With
machine-readable database files, this functionality can be made to work
again long after the original software has become technologically
obsolete.

The second is that this archival capability is, de facto, not strictly
limited to member institutions; rather, because the SEP is
available as open access in web page form without any authentication of
users, the web pages themselves are picked up and archived by web
crawlers such as the Internet Archive WayBack Machine and potentially by
other private “webpage capture” services such as archive.is. These
static web page archives can only be considered machine readable in a
limited sense, and are not ideal for the use cases for machine-readable
text previously outlined~-- but simply making an existing dictionary
website open access represents an important and easy first step to allow
archiving by third parties.

The third technical aspect is that not only do older versions happen to
be accrued by member institutions over the period of their membership,
continued distribution of older versions of the text is actually
explicitly foreseen by the model. The ability to continue to distribute
older editions and thus maintain the value of scholarly citations (as
mentioned above) is an advantage of digital publication which has, in
lexicography, remained thus far comparatively unexplored. It is another
consequence of the fact that digital reproduction is close to free,
compared to the unattractive paper and ink costs associated with
continually making older editions of a printed dictionary available.

These first two of these three technical aspects are not unique to the
SEP but are in widespread use for the distribution of academic
papers under the open access model.

I will now briefly discuss the financial aspects of the Stanford Model,
though with an acknowledgement that the specific funding situations of
individual lexicography projects differ significantly from one another.

Notably, the Stanford Model as applied for the SEP is concerned
only with covering the costs of publication of the encyclopedia, not of
paying the authors of its articles, which they write on a pro bono
basis. Most lexicography projects involve a small integrated team,
rather than a large distributed worldwide team of experts each writing
articles in their own field of specialization~-- this difference
necessarily implies that the SEP model, whereby authors consider
their work on the dictionary to be part of their output of academic
writing in their careers, is not workable for lexicography. While most
European historical dictionary projects are funded as research projects
under the auspices of a university or academy of sciences, and the costs
of producing the text thus subsidized away for the eventual publisher
who must only cover typesetting and printing costs, the OED and
some other dictionaries of potential scholarly relevance (such as the
“Unabridged” dictionary published by the Merriam-Webster company in the
US) are commercial undertakings. This, indeed, was the case with
HDAS, where it was likely not the printing but the editorial
costs which motivated the early cancellation of the project.

It is difficult to know what to suggest in such cases, other than that
the preparation of scholarly dictionaries in the Anglosphere, where
currently carried out for commercial or quasi-commercial publishers,
should move towards the European model. The association between the OED and the Oxford University Press goes back to the first
edition, but perhaps it is time for this arrangement to be reconsidered,
and for the editorial work of the OED to come under the aegis of
and be funded by, for example, the University of Oxford's Faculty of
English, rather than its University Press. This is merely an idle
suggestion, however, and made in the acknowledgement that European
scholarly dictionary projects have historically moved much slower even
than English ones, probably largely because of this model in which
lexicographers are also employed as academics with their own teaching
and research schedules, rather than as full-time dictionary writers.

\section{Further work}\label{further-work}

It was suggested at the conference that some clear guidelines should be
prepared for the sustainable publication of digital dictionaries. If
lexicographers, lexicographical historians, and dictionary publishers
feel such guidelines would be useful, an expert group could be assembled
to use the experiences of lexicographers working on projects of various
kinds, as well as those of librarians and archivists, to produce a
thorough but realistic set of recommendations. Since scholarly
dictionaries in particular are, in the digital age, increasingly being
published “in house” by the department of the university or other
institution which produces them, leaving decisions about publication
mostly in the hands of lexicographers, such guidelines may be valuable
especially for smaller teams which may lack specific expertise in this
area. Several such smaller projects were presented at ICHLL12.

Nonetheless, I hope this paper's recommendations~-- that wide
distribution of machine-readable dictionary texts in their entirety and
with their complete revision history should be made an integral part of
the publication of lexicographical projects~-- offer a useful starting
point until such guidelines are published.

The existing infrastructure for open access journal and data-set
distribution provides a useful starting point for the technical means of
achieving this. Another project of Stanford University, LOCKSS (\citealt{Reich2000}),\footnote{\url{https://lockss.org/}} provides open-source software
used by a number of “preservation networks” which automatically create
copies of digital resources in linked member institutions around the
globe. One network established by Stanford currently archives 52 million
journal articles and nearly half a million ebooks; in addition, archives
established by the governments of the United States and of Canada
distribute and save those countries' digital national archives, among
other networks which preserve data for the future which cannot yet be
made available to the public. LOCKSS is clearly a proven technology for
this purpose, but given the unique requirements of digital dictionaries
as structured texts, it would perhaps make most sense to establish a new
network using this software which all historical lexicography projects
could use to share their published work.

Until sustainable electronic dictionary publication as the norm becomes
a reality, stopgap solutions must be found. One simple such stopgap is
to continue to publish and buy dictionaries in paper form. This can be
achieved through pressure on publishers and libraries to keep print
alive even if digital publication is still perceived as the superior
option. The paper publication model, as outlined above, has
intrinsically good survival characteristics: dictionaries such as
\emph{Partridge Slang Online} which have already died in digital form
survive in paper form. “Retro\hyp redigitalization” from a paper source of a
dictionary which was originally made digitally, but whose digital form
has been lost, may be a realistic last-resort possibility to recover a
digital text.

In the era of online journal publication, it has become the norm for
(sometimes extensive) supplementary material to be made available with
the electronic version of an article. As a further stopgap measure, as
long as Oxford Model reference resources exist, I propose that it should
be made the norm that these supplementary materials should include PDF
archival copies of the complete versions of any dictionary entries
cited, when those entries were accessed from an Oxford Model
publication. Given the low cost of electronic storage and distribution,
this is technically plausible even for articles which cite very many
dictionary entries, though in this case legal difficulties may arise due
to copyright law. However, this practice alone should be sufficient to
guard against the problem of being unable to trace back citations after
an Oxford Model dictionary has been revised to obscure an original
reference.

A further consideration related to the long-term retraceability of
dictionary citations is the long-term reproduceability of research
conducted using text search tools on online dictionaries. This implies that the exact search algorithms used need to somehow be reproducible. Even apparently simple algorithms
can have very complex requirements in reality: a case-insensitive search
is actually intrinsically language sensitive (as one example, ⟨i⟩ is not
the lower case form of ⟨I⟩ in Turkish, unlike in most languages). This
alone makes the task of searching an etymological dictionary, which
typically mentions words from dozens of different languages in its text,
much more complex. This complexity affects the reproducibility of
research based on text searches, because the approaches used by a
particular search software must be fairly exactly replicated in order to
obtain the same results. The simplest approach would be to make all
dictionary search software open source so that the same software can be
used by everyone on their own computers. Code, though, intrinsically has
worse survival characteristics than data: software platforms are
constantly evolving, but data storage standards such as XML are
essentially permanent. Making software which can still be used many
years in the future is a broader problem in computing research, but the
number of complicating factors in the design of linguistic applications
makes the issue even trickier for our use cases.

\section{Summary}\label{summary}

My purpose in this article is certainly not to dampen the great enthusiasm for digital publication which lexicography has shown over the last few decades. On the contrary, I welcome digital publication and
recognize the great power it offers~-- much of it as-yet still totally
unexplored by lexicographers. I instead wish to heed caution and
encourage more careful thought about the precise means by which digital
dictionaries as texts are made available to the public.

Fixing this problem will require changes in the technical means of
dictionary publication, as well as potentially in how scholarly
dictionary writing is financed. Some of the technical problems involved
have already been solved by other digital publications facing similar
issues; others will require active research to resolve. The financial
problems, where they exist, will likely have to be resolved by
individual dictionary projects, since sources of funding already differ
greatly between projects. Above all, solving this problem requires a
recognition by dictionary sponsors that a scholarly dictionary is not a
source of profit, but a contribution to human knowledge, and must be
treated as such.

Indeed, solving this problem is a necessity if dictionary making as a
scholarly pursuit is to continue, because matters of basic scholarly
integrity are at stake. We must save our dictionaries from the Web of
Alexandria.

{\sloppy\printbibliography[heading=subbibliography,notkeyword=this]}
\end{document}
