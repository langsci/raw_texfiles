\documentclass[output=paper,colorlinks,citecolor=brown,arabicfont,chinesefont]{langscibook}
\ChapterDOI{10.5281/zenodo.15394471}
\author{Lynda Mugglestone\affiliation{Pembroke College, University of Oxford}}
\title[Examining the editors’ proofs of OED1]{On closure and its challenges: Examining the editors’ proofs of OED1}

\abstract{This chapter examines the newly discovered Editors’ Proofs of the first edition of the \textit{Oxford English Dictionary}. Long considered lost (with the exception of a single volume in the OED archives at Oxford University Press), these were a serendipitous find in the Bodleian Library in 2021 where they were filed under the heading “Amusements”. Signed and dated, as well as richly annotated, they offer a complex narrative of the final stages of the making of the OED, documenting the working practices of the different editors, and presenting a heteroglossic space where the interactions of editors, assistants, and press workers reveal a unique community of practice, united – and at times divided – by the different agendas at stake. On one hand, these offer compelling insights into the material historicity of the original text, where pen, ink, and the demands of hot metal can reveal an uneasy symbiosis. On the other, the proofs emerge as  a site of new historical inquiry, marked by the visible shaping of the final stage of the text in which e.g. the demands of in-text justification can drive a range of conflicted decisions while the salience of substitutive revision – by which any addition must be compensated by an equivalent process of loss –  impacts in critical ways on the presentation of evidence, labelling, semiotic analysis, as well as entries as a whole.}

\IfFileExists{../localcommands.tex}{
  \addbibresource{../localbibliography.bib}
  % add all extra packages you need to load to this file

\usepackage{tabularx,multicol}
\usepackage{url}
\urlstyle{same}

\usepackage{listings}
\lstset{basicstyle=\ttfamily,tabsize=2,breaklines=true}

\usepackage{langsci-basic}
\usepackage{langsci-optional}
\usepackage{langsci-lgr}
\usepackage{langsci-osl}
% \usepackage{./langsci/styles/langsci-lgr}
% \usepackage{./langsci/styles/langsci-osl}
% \usepackage{langsci-gb4e}

\usepackage{tikz}
\usetikzlibrary{patterns,calc}
\pgfdeclarepatternformonly{south east lines}{\pgfqpoint{-0pt}{-0pt}}{\pgfqpoint{3pt}{3pt}}{\pgfqpoint{3pt}{3pt}}{
    \pgfsetlinewidth{0.6pt}
    \pgfpathmoveto{\pgfqpoint{0pt}{3pt}}
    \pgfpathlineto{\pgfqpoint{3pt}{0pt}}
    \pgfpathmoveto{\pgfqpoint{.2pt}{-.2pt}}
    \pgfpathlineto{\pgfqpoint{-.2pt}{.2pt}}
    \pgfpathmoveto{\pgfqpoint{3.2pt}{2.8pt}}
    \pgfpathlineto{\pgfqpoint{2.8pt}{3.2pt}}
    \pgfusepath{stroke}}
    
\usepackage{stmaryrd}
\usepackage{wasysym}
\usepackage{multirow}
\usepackage{caption}
\usepackage{subcaption}
\usepackage{mathrsfs}
\usepackage{qtree}

\usepackage{linguex}


  %pminos do not split footnotes
% \interfootnotelinepenalty=10000 %Footnote in Laporte chapters has to be split SN


%\DeclareIndexNameFormat{default}{%
%\nameparts{#1}%
%\usebibmacro{index:name}%
%{\index[names]}%
%{\namepartfamily}%
%{\namepartgiveni}%
% {}% L1
% {}% L2
%{\namepartprefix}% generates spurious space L3
%{\namepartsuffix}% generates spurious space L4
%}

%  {\DeclareIndexNameFormat{default}{%
%     \usebibmacro{index:name}{\index[names]}{#1}{#3}{#5}{#7}}}

%\DeclareIndexNameFormat{default}{%
%  \usebibmacro{index:name}{\sindex[nom]}{#1}{#3}{#5}{#7}}

%\DeclareIndexNameFormat{default}{%
%  \usebibmacro{index:name}{\sindex[person]}{#1}{#3}{#5}{#7}}
%\DeclareIndexNameFormat{default}{%
%\nameparts{#1} \usebibmacro{index:name}{\sindex[person]]}{\namepartfamily}{‌​\namepartgiven}{\nam‌​epartprefix}{\namepa‌​rtsuffix}}

%\newcommand{\smiley}{:)}

%\renewbibmacro*{index:name}[5]{%
%\usebibmacro{index:entry}{#1}%
%{\iffieldundef{usera}{}{\thefield{usera}\actualoperator}\mkbibindexname{#2}{#3}{#4}{#5}}}

% \newcommand{\noop}[1]{}

%remove for final
%\overfullrule=1mm

\newcommand{\tobi}[2]}}
\renewcommand{\S}[1]{\tobi{#1}{\textsc{*}}}

% this volume references
% puts: [this volume]
% already defined: \citetv
%\newcommand{\citepv}[1]{(\citeauthor{#1} \citeyear*{#1} [this volume])}
\newcommand{\citealtv}[1]{\citeauthor{#1} \citeyear*{#1} [this volume]}

%parentheses around example number
\newcommand{\pref}[1]{(\ref{#1})}

% in-text examples

\newcommand{\lnex}[1]{\textit{#1}} %target lang word
\newcommand{\lnlit}[1]{(lit.: `#1')} %literal reading
\newcommand{\lnlat}[1]{(#1)} % latinization
\newcommand{\lntrans}[1]{`#1'} %translation
\newcommand{\lnexl}[2]%
{\lnex{#1}{} \lnlat{#2}} % ex with latinization
\newcommand{\lnexlat}[3]{\lnex{#1}{} \lnlat{#2}{} \lntrans{#3}} % ex with latinization and tranl.

%ch01
\newcommand{\co}[1]{\mbox{\textbf{#1}}}

%ch09

\newcommand{\cyrbulg}[1]{\begin{otherlanguage*}{bulgarian}#1\end{otherlanguage*}}


%ch10
\newcommand{\nlp}{{\small NLP}}
\newcommand{\mwe}{{\small MWE}}
\newcommand{\rae}{{\small RAE}}
\newcommand{\lvc}{{\small LVC}}
\newcommand{\pos}{{\small P}o{\small S}}
%\newcommand{\todo}[1]{ \textcolor{red}{#1} }

%\renewcommand{\labelenumi}{\theenumi}
%\ainamefmt{{vv}{ll}{, ff}{, jj}} % fullname

\newcommand{\biberror}[1]{{\color{red}#1}}

\newcommand{\osenovaitem}{--~}
  %% hyphenation points for line breaks
%% Normally, automatic hyphenation in LaTeX is very good
%% If a word is mis-hyphenated, add it to this file
%%
%% add information to TeX file before \begin{document} with:
%% %% hyphenation points for line breaks
%% Normally, automatic hyphenation in LaTeX is very good
%% If a word is mis-hyphenated, add it to this file
%%
%% add information to TeX file before \begin{document} with:
%% %% hyphenation points for line breaks
%% Normally, automatic hyphenation in LaTeX is very good
%% If a word is mis-hyphenated, add it to this file
%%
%% add information to TeX file before \begin{document} with:
%% \include{localhyphenation}
\hyphenation{
    Beck-man
    Ngu-yen
    back-chan-nel
    back-chan-nels
    mo-not-o-nous
    ste-reo-typ-i-cal
}

\hyphenation{
    Beck-man
    Ngu-yen
    back-chan-nel
    back-chan-nels
    mo-not-o-nous
    ste-reo-typ-i-cal
}

\hyphenation{
    Beck-man
    Ngu-yen
    back-chan-nel
    back-chan-nels
    mo-not-o-nous
    ste-reo-typ-i-cal
}

  \togglepaper[2]%%chapternumber
}{}

\begin{document} 
\maketitle

\section{Introduction}

\emph{Closure} (‘a bringing to a conclusion; end, close’), edited by James Murray, appeared in the fascicle \emph{Clo--Consigner} in 1891 as part of the serial publication of the \emph{New English Dictionary on Historical Principles} (henceforth OED1) \citep{Murrayetal1884}. The opening fascicles \emph{A--Ant}, and \emph{Ant--Batten} had been published, respectively, in 1884 and 1885. If Murray had been appointed as editor on the assumption that the dictionary would be completed in ten years, closure remained elusive. Henry Bradley (appointed as a second editor in 1886), produced his second independent fascicle, \emph{Everybody--Ezod}, in 1894 (three years after \emph{E--Every}). Ten years later, Murray had advanced to \emph{O}, Bradley was at work on \emph{M}, and another editor, William Craigie, promoted in 1901 from his earlier role as dictionary assistant, had completed work on \emph{Q--Ree}. Only after another decade did closure start to seem a realistic proposition. “We may… reckon that the end of 1916 will see the \emph{Dictionary} finished”, Murray wrote to Walter Skeat in 1912: “If I live to then, I shall be 80, and it will also be my Golden Wedding: Let us hope that the Grand Conjunction of all these cycles will really take place”\footnote{MP/22/7/1912. James A. H. Murray to Walter Skeat. Murray Papers, Bodleian Library, Oxford.}. A flurry of activity ensued. \emph{Su--Subterranously}, the first independent fascicle by a fourth editor, Charles Onions, appeared on New Year’s Day 1915. This was followed, if discontinuously, by William Craigie’s \emph{Spring--Squoyle} in April, and Bradley’s \emph{Standard--Stead} in October. Murray’s \emph{Trink--Turn--down}, a fascicle published in July 1915,  was, however, to be his last. He died on July 26\textsuperscript{th}. \textit{T} was completed by Craigie, with the assistance of Murray’s Scriptorium staff, as well as Murray’s extant drafts. Closure, for the \emph{Dictionary}, eventually took place in 1928. Craigie and Onions were, by that point, the only surviving editors.

A richly documented history of OED1 (see e.g. \citealt{Brewer2007, Gilliver2016, Mugglestone2005, Murray1977}) traces this lengthy gestation, aided by substantial archival collections in both Oxford University Press and in the Bodleian Library, Oxford. This article, however, will focus on the contributions of a newly-discovered resource, the final “Editors’ Proofs” of OED1, found by chance in July 2021  in the Bodleian Library where it had been tagged under the heading “OED. Amusements”. “Every so often we get lucky, and a long-lost work turns up” writes \citet[158]{Lynch2016} in a chapter on lost linguistic projects and the fragility of the historical records we possess. Lynch’s words were prompted by the discovery, also in the Bodleian Library, of James Boswell’s manuscript dictionary of Scots. The Editors’ Proofs reveal similar fortuity, as well as undeniable salience in bibliographic as well as linguistic terms. Long considered lost (apart from a single volume preserved at Oxford University Press), they present a vast and hitherto unexamined resource, spanning twenty-three volumes from the mid-1880s to 1928. Documenting the penultimate versions before each fascicle went to press, they shed new light on a text on the very point of closure, alongside the sustained challenge of shaping definitive histories on the page. Pasted “slips” containing new material as well as in-text epistolarities that, in various ways, negotiate matters of both form and inclusion are common. Marginal asides, in a variety of hands, create an intriguingly heteroglossic text. The Proofs are replete with annotations, deletions, and revisions at a range of levels.

As this suggests, the Editors’ Proofs exist as a text of complex materiality, making plain not only the practical difficulties of fixing the text but the role (and significance) of different communities of practice as part of this process. As such, they present a useful corrective to the familiar historical alignment of OED1 with the history of Murray himself – a consequence both of archival materials (see e.g. the \textit{Murray Papers} in the Bodleian Library) as well as popular print history in which \textit{Murray’s Dictionary} was early established as an alternative title. “The college to subscribe to James Augustus Henry Murray’s \emph{Dictionary of the English Language}”, as the Chapter Book of Corpus Christi College (CBCCC)\footnote{\url{https://www.corpus.cam.ac.uk/sites/default/files/downloads/chapter_book_8.pdf}} recorded in 1890 (“Chapter Book”, 22). That Walter Worrall, employed as an assistant on the \emph{Dictionary} since 1885, was the \emph{de facto} editor of \emph{W} after Bradley’s death in 1923 is, for example, unambiguous in the Proofs. So, too, are the decisive contributions of, say, Charles Balk who worked on the \emph{Dictionary} for some forty years, initially alongside Murray (and later with Onions). Across the Proofs, the surviving records of in-text correspondence between different editors and dictionary assistants, as well as the contributions of press-room and compositors, vividly document a range of alternative histories behind the finished text. 

\section{Examining the text in time}

“The great dictionaries” of the past “were created using basic technologies; pen, paper, and index cards for the lexicography, hot metal for the typesetting and printing” write Michael Rundell and Adam Kilgariff (\citealt[258]{Rundell2011}). This sense of print history is palpable in the Editors’ Proofs. References to the “foundry” (“To be sent to the foundry at once”, as the corrected sheet of \emph{contemplation} declares), or to the need to “impose for Electro” (`used \emph{colloq.} as an abbreviation for […] ELECTRO-PLATE’, as Bradley explained in the relevant fascicle) are common. Press-room warnings that “type wanted” (such that the text must be finalized, printed off, and type released for further sections) reveal other imperatives of historical production. Apparent, too, is the occasional striking instance of “printers pie”. “Please have this page reset, they have had an accident and dropped the matter out before moulding”, as an in-text missive written alongside \emph{devil} records in March 1895. The final column was immaculate. The “matter” of the previous two was missing, presumably scattered on the composing-room floor.

\citet{Rundell2011} nevertheless describe a far more binary process of historical composition than is presented by the Editors’ Proofs. On one hand, they suggest, is the “lexicography”, vested in pen and ink in the primary collection of data. On the other, as they indicate, was the press, and the technology of hot metal on which it relied. The Editors’ Proofs, however, routinely place lexicography in pen and ink as an ongoing process – as revision in proof, and in symbiosis with “hot metal” to the last possible moment. As in-text annotation confirms, for example, the extent to which lexicographical desiderata could be accommodated before closure of this kind was a recurrent concern. Hand-written injunctions such as Bradley’s “If it is not too late, I should like the enclosed correction to be made” (here with reference to a last-minute revision of the etymology of \emph{lanchara}) or “Please make this correction and addition if in time” (a comment written by Murray above \emph{to-queme} in a change that affected both derivation and cognate forms) routinely reveal the provisionalities (and hazards) of late-stage change. 

The iterated use of “if” claims particular significance in this respect. Murray’s “Please alter this if it can be done” (under \emph{dragonate}) or, say, “I shall be glad if the enclosed alteration can be made on the latest final”, written by Craigie alongside the entry for \emph{quandary} in January 1902  (by which an earlier attributive sense was to be removed) stand as other marginal annotations, directed to the press-room, and in which the pragmatic contingency of emendation is overt. Balk’s “If it is not too late (i.e. the sheet printed off) we should like to have this correction made, as the original statement is incorrect”, inscribed alongside \emph{discipline} in the Editors’ Proofs (and counter-signed by Murray), is, even so, particularly illuminating in the detail it provides. On one hand, the proposed emendations were self-evidently important for the authority (and authoritativeness) at which OED aimed. On the other, the information provided might, as Balk recognised, remain a casualty of production, and the processes it required. “If not too late, please make corrections marked”, as another annotation by Balk, here against \emph{obmit}, records: “if too late i.e. the pages already electro’d, please return”.

The text was, in this light, routinely shaped by its own unfolding history even as it endeavoured to tell the full history of words. In-text annotations such as “needed for the date” (placed alongside \emph{disc} (n.) where Murray wanted to include lengthy if last-minute citational evidence from the \emph{Engineer} of 1856) can, for instance, set out an intentionally persuasive rationale for the changes to be made~-- even if, of course, the individual compositor might think otherwise given the resetting and laborious readjustment that was, in turn, required. Bradley’s use of “small”, in an in-text missive pinned to \emph{fight} in 1895 (“Will you kindly make the small corrections indicated on the enclosed cuttings from the pages”) was equally tactical. His further comment (“I fear some of the pages in question will have been printed off, but the alteration might be made in the plate so that subsequent copies printed may be correct”) arrestingly suggests the process of in-text change even as the text was printing, and by which variant versions might be incorporated in a single print run. Seen in hindsight, the decision, made in Proof, to delete the evidence and entry for \emph{final proof} as a further aspect of late-stage change acquires a sharply ironic charge, “\sout{\emph{Printing}. The last proof sheet before sending to press}”, the relevant sheet still records. “Final”, as other comments confirm, could be a highly relative term.  

Visibilities of this kind vividly inscribe the complex realities of collaborative and multi-author composition while reminding us of the gaps that, as part of late-stage emendation, could intervene between lexicographical intent and the text as set. A surprising number of revisions remain, for example, within the confines of the Editors’ Proofs where editorial “last words” have, for whatever reason, failed to be reproduced in print. Craigie decided, for instance, to undertake a stringent process of revision in the fascicle \emph{Unright-Uzzle}, deleting the entry (and evidence) he had previously included for \emph{unsonned} and adding new evidence (and a new reference) to \emph{unstoicized} instead. A new quotation from Robert Wallace’s biography of the Scottish historian George Buchanan was carefully written in the bottom margin: “Had he been all his detractors call him, that would not have *unstoicized him”. At the same time, the previous evidence, from George MacDonald’s \emph{Weighed and Wanting} of 1882 was crossed through (“\sout{Determined that his son, having *unsonned himself, should no longer be treated as a son}”). Yet neither change made its way into print or, indeed, into later iterations of the OED. Even in OED Online, \emph{unsonned} remains in place while \emph{unstoicized} remains unrecorded and unknown.

The final revisions of \emph{clitoris} (n.), retained in full on the Editors’ Proofs (see \figref{fig:muggle:1}), provide an even more striking example, problematizing the standard narrative on this contested word in OED history. Edited by Murray in a fascicle published in 1889, its definition as `a homologue of the male penis, present, as a rudimentary organ, in the females of many of the higher vertebrata' is, as by \citet{Rees2019}, prototypically seen as a product of Victorian reticence as well as androcentrism. “The iconic English dictionary’s misogynist descriptions were ripe for revision”, they assert. In the modern OED, a new definition (`The female genital organ located in the anterior part of the vulva, which contains numerous nerve endings and plays a major role in sexual arousal and pleasure') articulates meanings which were hitherto obscured or denied, and in which new interpretative trajectories in relation to female sexuality displace the hegemonies of the past.

The Editors’ Proofs, however, provide their own contested readings of this entry in ways which affect both evidence and definition. In one version of the surviving text, dated October 1889,  Murray’s definition had already been set. It drew closely on Thomas Huxley’s \emph{Manual of the Anatomy of Vertebrated Animals} of 1871 (from which Murray’s final illustrative citation also derived). “In the female sex, the homologue of a penis frequently makes its appearance as a \emph{clitoris}”, as \citet[111]{Huxley1871} had explained. Yet, as \figref{fig:muggle:1} illustrates, pasted palimpsestically to the same page in the Proofs is an alternative entry in which evidence and definition assume a different shape. Murray’s final citation (“In some few mammals (e.g. the \emph{Lemuridæ}) the clitoris is traversed by a urethral canal”), included from Huxley in the original version, was now displaced.  Evidence from Todd’s \emph{Cyclopaedia} (“The clitoris lies in the upper part of the vulvar fissure”) takes its place. Redefinition meanwhile removed the male frame of reference, and the further link to Huxley, even if a certain euphemistic coding remains in place: “Part of the female generative organs in the mammalia”, the revised entry explained. “Please make correction to final”, states a neat annotation by Balk above these revisions. In effect, all subsequent editions, from the fascicle of 1899 to the reprinted text of 1933, as well as OED2 \citep{Simpson1989}, have been based on a version that had already been rejected.

\begin{figure}
\includegraphics[width=\textwidth]{images/Mugglestone_Figure1.png}
\captionof{figure}{Editors’ Proofs: revised entry for \textbf{clitoris}, November 1889. Reproduced with permission of the Bodleian Library Oxford from \emph{A New English Dictionary on Historical Principles} (annotated proof text) 30255 c. 17 (2), p. 508.}
\label{fig:muggle:1}
\end{figure}

Precisely why the published text assumed the form it did nevertheless remains unknown. Balk was, by 1889, an experienced co-worker whose notes on the Proofs routinely reflect the consensus of Scriptorium (and Murray) on the changes to be implemented.  Evidence from other sources, as in the compositorial walk-out that attended the printing of Farmer and Henley’s \emph{Dictionary of Slang}, or indeed in contemporary Philological Society discussions (whereby it was decided that entries for \emph{cunt} and \emph{condom} should be suppressed in OED1 (see e.g. \citealt[1--22]{Mugglestone2007}), clearly suggest, however, the presence of extra-textual pressures by which the immediacies of print history could be shaped. Other evidence in the Editors’ Proofs readily confirms the effect of related sensibilities on the matter of cultural propriety and linguistic representation, evident in e.g. the tactical elision within \emph{squitter} (`To \sout{have or suffer from diarrhea}; void thin excrement’) just before it went to press or the deletion from \emph{lap} of “Lap it hot about the privy parts” (a citation from Gervase Markham’s \emph{The English Housewife} (1615)) while equivalent if less indelicate material from Moxon, Goldsmith, and Swift in the same entry remained intact.

Murray’s careful rewriting (and decorous code-switching) in sense 20 of \emph{cock} (= ‘penis’) some fifty pages after the disputed entry for \emph{clitoris} presents a further instance where contemporary socio-cultural taboo arguably reshaped the form it later assumed in print. Murray’s earlier version~-- “Only in low vulgar use, and considered indecent”, with its embedded acknowledgement of contemporary language attitudes~-- was edited out. “\emph{Pudoris causa}, inadmissible in polite speech or literature” (complete with Latinate indirection) took its place, courtesy of a series of in-text annotations in Murray’s hand. This, too, documents a history of change and revision that has hitherto remained hidden in the ancillary histories that the Proofs provide. We don’t, of course, have a variorum edition of OED1. At a range of points, however, the Editors’ Proofs suggest that, in whole or in part, this might constitute an interesting and, indeed, productive resource.

\section{Rereading the text in Proof}

How to use data of this kind nevertheless brings its own challenges. Auden’s skepticism about the value (and insight) of academic scrutiny with reference to late revision was palpable.  Revisions he made while preparing his \emph{Collected Shorter Poems} for the press in the early 1960s were, as he wrote to Neville Coghill, “a gift to any anal-minded American Ph.D. student” \citep{Crook1990}. Nevertheless, as with pre-print versions of Sally Rooney’s \emph{Beautiful World, Where Are You?} (published in 2021), such anterior texts have, in recent years, often claimed considerable attention in relation both to their monetary cost \citep{Strzyzynska2021} as well as the critical exegesis they might afford. Pre-print text with annotations, serving to highlight e.g. revisions between different versions or which are seen as indicative of a writer’s later thoughts, are typically regarded as even more collectible. Proof revision or the existence of an emended text in galley proofs can, in similar ways, claim a valued materiality, especially in an era of born-digital texts where the processes of writerly revision and rethinking are all too easily elided by the affordances of new technology. The unexpected recovery of the Editors’ Proofs of \citet{Murrayetal1884}, with all their complex materiality, will not in future years be paralleled by those of OED Online. 

Revision in proof, and the survival of texts of this kind, can, for example, provide us with useful testimony of what \citet[103]{Lillis2013} describes as the text trajectory of a particular work~-- in other words, with historical evidence of its contextualization, its coming into being as text~-- and, by extension, of what is, or not, suddenly deemed expendable. Proofs and page proofs, as Hannah Sullivan argues in her book \emph{The Work of Revision} \citep{Sullivan2013}, can often be seen as constituting a site of continued writerly experimentation, complete with visible evidence of re-composition, whether this involves the rewriting of a particular section or phrase or wider patterns of restructuring. On one level, as Sullivan comments \citep[4]{Sullivan2013}, the process of late-stage revision clearly signals “the possibility of the text being fixed in some material form”. On another, as she adds, evidence of this kind also remains profoundly evocative of the dialectic between “what changes and what stays the same” – and, crucially, why this might be so.

Sullivan’s critical focus is, of course, directed to literary rather than linguistic exegesis in which surviving proofs and their embedded revision by writers such as Henry James and Virginia Woolf are explored as “classic documents” of high modernism. If the text in proof occupies, as she argues \citep[6]{Sullivan2013}, a “materially intermediate” status (“neither manuscript nor print”), it is also made reflective of the on-going “friction between possibility and constraint”, and a locus of new insights that encompass both the status of the published text, and the interpretative trajectories that anterior in-text annotation reveals. “High modernism”, not least given the OED’s beginnings some forty years before Victoria’s death, might seem remote from the critical debates that inform historical lexicography. There are nevertheless some illuminating correspondences in the visible tensions of fixity and flux, and “friction” and “constraint”, that the Editors’ Proofs variously attest~-- alongside the decisions that the final and published text can, in turn, reveal.

The status of Editors’ Proofs per se has, in this respect, particular salience. They exist, by definition, at a point of “almost closure” or “almost finality”~-- a liminal state where, as contemporary print manuals affirmed (see e.g. \citealt{Ford1854, Graham1848}), only minor emendations were expected. Any whole-scale revision was disfavoured, especially if it threatened to disturb the pagination that had been carefully set up (and, importantly, in terms of the OED1, calibrated across different fascicles for which printing was, as we have seen, often alphabetically discontinuous). The fact that the Editors’ Proofs drew on a pre-existing sequence of first and second proofs, third proofs when necessary, as well as revises and second revises, likewise meant any major processes of revision should already have taken place. Some remaining changes were, even so, obligatory. Each page was attentively read by the press reader whose oversight ensured the systematic correction of typography and font in a text where being “eloquent to the eye” \citep[vi]{Murray1888} was not only aesthetic but functional, enabling the ready discrimination of different kinds of information. The OED, with its twenty-one contrasts of form and font, was, as \citet{Luna2004} notes, particularly “structure-rich” in this respect.

Other forms of obligatory change are, however, also apparent. Transposition (often indicated by a marginal “tr.”) was a surprisingly common injunction, triggered by the need to maintain dictionary order. The entries for \emph{distrainment} and \emph{distrainor} were, for example, awry until just before publication, as were those for \emph{cradler}, \emph{cradle-hood}, and \emph{cradling}. Historically\hyp motivated transposition (and the diachronic sequencing of evidence within a given sense or entry) claimed similar importance, especially given the historical principles on which the OED was based. “Should have been put in according to date”, Murray firmly instructs alongside the entry for \emph{beholding} where, in the Proofs, evidence from Shakespeare’s \emph{Merry Wives of Windsor}, dated 1598, unaccountably came after that from Henry More’s \emph{An Antidote Against Atheisme} of 1662. “[I] shall be glad if the reader will point out accidental derangements of date, such as this, when he sees them”, another marginal note from Murray confirms. Even in emendations of this kind, the “friction between possibility and constraint” can be perceptible. Reading the Proofs of \textit{D}, Murray had urgently written to alert the press-room that the entry for \emph{denourishment} had been misplaced such that it followed \emph{denunciatory} rather than appearing four pages earlier (and following \emph{denouncing}). Yet, as an in-text and hand-written response from the press-room confirms, change was no longer possible. The relevant pages had already been cast. Even in the reprinted text of 1933, the disrupted order remains in place. 

In general, however, emendations that aimed to secure an authoritative text had self-evident priority. Craigie, for example, rewrote \emph{quick time} on a new slip pinned to the side of the relevant page on realizing that his earlier facts had been awry (it was not, he confirmed a “rate of marching [that] consists of 120 steps of 30 inches each (= 300 feet)” but “128 paces of 33 inches (=118 yards) or four miles an hour”). Bradley eliminated other late-stage errors under \emph{lychnic}. As his deletions confirm, this was not “an office which accompanies the lighting of lamps, forming part of the vespers on certain occasions” but referred, with marked specificity, to “the introductory part of the vespers” alone. The text was, in both cases, reset accordingly. Early evidence for \emph{flump} (“To fall or move heavily with a dull noise”), which had previously dated the first English usage to 1677, was likewise removed. “Found in the Bodleian today”, an annotation in the adjacent margin by George Sykes (one of the editorial staff) announced. The relevant word was, in fact, \emph{slump} while the submitted evidence (and the entry up to this point) had been based on a misreading of long tailed <s> as <f>. \emph{Flump} in English was suddenly postdated to 1816. “The perfection of the dictionary in its \emph{data}” was vital, \citet[129]{Murray1880} had early stressed. Here and elsewhere, rereading continued until the text went to press.

\section{Willed and willing change}

Changes of this kind constitute what we might see as willing editorial revisions, impelled, if in different ways, by a sense of the language, and the intent to secure the linguistic (and historical) rigour on which the OED project was founded. Etymology, semantics, definition, and sense-history (as well as bibliographical information) can all be reworked to this end. Murray’s anxious revision (in Proof) of \emph{toast-stand} shortly before he died is part of this. Was ‘a \sout{stand} utensil for standing toast, etc. by the fire’ best, as a hand-written emendation in the margin records? Or ‘a stand for toast, etc. by the fire’? The former was crossed though. Bradley’s in-text revision of \emph{glaciarium} (not a ‘skating rink with artificial ice’ but ‘a skating rink with ice artificially produced’, he decided) was similar; artifice was, rightly, relocated to the means of production rather than the ice itself. Across the Editors’ Proofs, Craigie proved a particularly inveterate rephraser changing e.g. ‘noodle, simpleton’ to ‘thickhead, dolt’ in revising \emph{numskull}, and amplifying the already expansive definition of \emph{narrow-minded} (‘lacking in breadth of mind; illiberal, bigoted, prejudiced’) by adding ‘incapable of broad views’. If the tensions of possibility and constraint thereby remain in view, so do related questions of where revision in proof could or should legitimately stop.

Nevertheless, as the annotated text confirms, each dictionary-maker and assistant also worked with the conscious knowledge that any new addition had a cost~-- not only monetary (though this had its own significance) but, more critically, a spatial cost, enacted within a textual economy that had to be rigidly maintained. A new page was not forthcoming. Space itself was a limited commodity in ways that yield not only other manifestations of constraint but processes of textual reshaping by which evidence and exposition can be forcibly recalibrated on the page. Antedatings and the emergence, in Proof, of new evidence on the first use of a given word or sense provide a useful text case in this respect. Linguistically, these had obvious salience; “quotations illustrating the first appearance” of a word or sense, and data on “a word’s first appearance in the language” had been identified by both Murray (\citeyear[46]{Murray1900}) and Richard Chenevix Trench (\citet[23]{Trench1857} as primary requisites for a robust historical lexicography. Late-stage additions of this kind were pinned or pasted to a wide range of entries in the extant Proofs. A new citation from Mulcaster’s \emph{Positions Concerning the Training Up of Children}, published in 1581, appeared, for instance, in the margin alongside \emph{gay} ((vb.) ‘to make gay…to give a bright and pleasant look to’), here in ways that served to revise its early use by some sixty years. \emph{Gather} and \emph{coaxingly} provide similar examples. An antedating from George Gascoigne’s \emph{The Steel Glass} (1576) secured a new date of birth for the former (versus a previous first usage in 1704). Likewise, under \emph{coaxingly}, an antedating from Susannah Centlivre documented a life-history that now began over a century earlier than had previously been supposed. “Prithee, my dear, moderate the passion (\emph{coaxingly})”, as she had written in 1714 in \emph{The Wonder: A Woman Keeps a Secret}.

Information of this kind served, at least in principle, to supplement text and historical record alike. Nevertheless, as the Proofs attest, the wider textual consequences can be arresting. Equally visible, for example, is not only the presence of newly-inserted citational evidence but a corresponding process of elision by which any material added must, given the immediacies of historical production, be accommodated by excising the equivalent space from elsewhere. As under \emph{coaxingly} therefore, if Centlivre was added, a citation from Charles Lamb’s \emph{Letters} in 1835 (“There was a rough earnest to the request though it was put \emph{coaxingly}”) was removed while, in parallel ways, the decision to add Gascoigne under \emph{gather} necessitated the loss of Colley Cibber’s \emph{The Careless Husband} in a citation recorded from 1704. Mulcaster’s new-found evidence of \emph{gay} yields a corresponding narrative of cause and effect. The relevant life-history of \emph{gay}, sense 1, was now beneficially extended to the late sixteenth century. Nevertheless, a minor entry (\emph{gay}, defined as an obsolete form of \emph{go}) as well as earlier embedded references to \emph{gayed} and \emph{gaying} as subordinate forms were all deleted in response. As the Proofs confirm, the process known as substitutive revision was widely adopted as a means of crafting late-stage lexicography on the page. As the antedatings discussed confirm, the text might therefore be supplemented, and the historical record improved. But gain could not take place without loss, nor insertion without excision.

Surviving in-text annotation can make this process of exchange strikingly overt. As under \emph{convent} (n.), sense 6, a new slip was pinned to the left-hand margin, providing an antedating from Thomas Trusler’s \emph{The Distinction between Words Esteemed Synonymous}, dated 1795, and revising evidence of its earliest use (hitherto recorded from Stanley Lane-Poole’s biography of Lord Stratford de Redcliffe in 1814). An explanatory note in Balk’s hand, however, sets out the cuts and consequences which must, in turn, take place. As the press-room was instructed, if Trusler is added, a later citation from Guilt’s \emph{Architectural Glossary} (1842) was to be deleted to secure the requisite space.  As this suggests, excisions of this kind were inherently opportunistic, driven by patterns of revision that play out at the local level of column or page. “If necessary to save line", Murray likewise wrote alongside \emph{legitimacy}, for example, excising a quotation from the \emph{Contemporary Review} in 1885 in order to include new evidence under \emph{legister} instead. Similar was e.g. a new antedating from Nathan Bailey introduced under \emph{demongrapher} (which took usage back to 1736) while previously inserted evidence from the British traveller Rachel Harriette Busk was deftly truncated instead in order to accommodate the desired change. 

An identical trajectory of gain and loss played out between Thomas Erskine and Jane Austen under \emph{blowzy} (adj.), sense 2, defined as ‘Of hair, dress: Dishevelled, frowzy, slatternly’. In the Editors’ Proofs as originally printed, Austen’s \emph{Pride and Prejudice} (courtesy of Miss Bingley’s description of Elizabeth Bennett) had supplied a prized first citation: “her hair was so untidy, so blowzy”. Newly-arrived data, in an antedating from Erskine’s \emph{The Barber} of 1778, meant, however, that substitutive revision again took its toll, here in ways that displaced Austen and her evidence from the published text. Austen’s \emph{The Watsons} (1804) revealed a similar vulnerability under \emph{full} (sense 1a). Bradley deleted  Austen’s “quite as full as ever…there seemed no vacancy whatever” under the pressures of space that were imposed by an antedating he introduced under \emph{full}, sense 1e (referring to ‘Of an animal: Pregnant. Of a fish: Charged with roe’). Changes of this kind might, as we have seen, be opportunistic but Bradley’s choice can, even so, seem surprising.

The extent to which certain types of text seem more vulnerable to the forms of attrition that occur is, however, clearly of interest. Shakespeare in a new citation from \emph{Timon of Athens} could, as under \emph{tithed} (‘Subject to, charged with, or liable for the payment of tithes’) extend his numerical dominance in the OED, courtesy of a citation written by hand in the upper margin of the relevant page. Yet while the entry for \emph{tithed} was adjusted accordingly (its “birth” moved backwards by almost 250 years), at the other end of the entry a different process was taking place such that evidence from the \emph{Manchester Evening News} in 1884, referring to ‘The heavily rented and tithed British farmer’, was crossed though, having been judged expendable against Shakespeare’s greater claim. Craigie’s insertion of a new citation from Wordsworth’s \textit{Prelude} in 1805 mid-way through \emph{quartered} (‘Through three divisions of the quartered year’) can likewise be set against the corresponding loss of the \emph{Daily News}, in final position, in what had originally provided a far more modern citation from 1900. In a related process of exchange, a decision to include a new citation from Blackstone mid-way though \emph{tourn} (referring to the tour or circuit made by the sheriff of a county) was clearly facilitated by the selective loss of the \emph{Times}, in a 1904 citation that had previously concluded the previous entry for \emph{tourmaline}. Likewise, evidence from the \emph{Leeds Mercury} in 1887 disappears under \emph{fleeting} (here, as prompted by the decision to include an antedating from Sir Thomas Smith), a citation from \emph{Harper’s Magazine} in 1887 disappears under \emph{burgraviate} (sb.), displaced by one from 1762, taken from Patrick Murdoch’s translation of \emph{A New System of Geography}, and a citation from the \emph{Nation} in 1892 disappears under \emph{give} while adjacent evidence from Scott, Thackeray, and Wilkie Collins remains intact. “Please make this correction if not too late”, an accompanying note from Bradley confirmed. While the canonical and gender prosodies of OED1 have long attracted comment (see e.g. \citealt{Mugglestone2005}, \citealt{Brewer2012}), the process of late-stage revision in the Proofs suggest other productive avenues for investigation in this respect, not least in the cuts and consequences that are revealed.\footnote{The transfer of the data from the Editors’ Proofs to a database (funded by the John Fell Fund at Oxford University) will, as part of the next stage of the project on which this paper is based, enable far closer and systematic scrutiny of patterns of this kind.}

In principle, of course, evidence across OED1 was treated with impartiality. “The literary merit or demerit of any particular writer, like the comparative elegance or inelegance of any given word, is a subject upon which the Lexicographer is bound to be almost indifferent”, as Herbert Coleridge, the \emph{Dictionary}’s first editor, had affirmed \parencite[2]{Coleridge1858}. Canonical and non-canonical writers were targets of substitutive revision in the Proofs; if a new quotation from Longfellow was inserted under \emph{birdlet}, it was, for example, Keats who was deleted under \emph{bird-lime} in order to make room. Even so, the relative patterns of loss and gain in relation to e.g. inserted evidence from Scott are striking. As under \emph{name}, Scott is added, both in a personal letter and in a citation from Lockhart’s \emph{Life}, while evidence from the minor Anglo-Caribbean writer Annie Holdsworth (in a citation from 1890) is taken away. In related ways, in a textual battle for space between Surtees and Scott inscribed in the entry for \emph{bartizan}, Scott (and a citation from \emph{The Eve of St. John}) clearly wins. So does, under \emph{neighbour}, a new mid-entry citation inserted from Scott’s \emph{The Fortunes of Nigel} (1822) which preferentially displaced the pithy “Neighbours especially are often far from neighbourly”, as previously recorded from Charles Spurgeon in 1874. Under \emph{qualifying}, too, a similar process can be seen at work such that a new citation from \emph{Rob Roy} in 1817 drove the loss of a mid-entry citation from Worcester under the adjacent \emph{qualify}; under \emph{dine}, a new citation from \emph{Guy Mannering} displaced one from the \emph{American} in 1884. Under \emph{spinster}, new evidence from Scott’s \emph{Rob Roy}, inserted in final position, likewise triggered the compensatory clipping of evidence that, under the adjacent \emph{spinous}, had hitherto been recorded from the \emph{Proceedings of the Berwickshire Naturalists’ Club} in 1854. After Shakespeare, Scott was, \citet{Brewernodate} records, the most cited individual writer in OED1. This was, self-evidently, reinforced rather than reduced in late-stage editing of this kind. On one level therefore, substitutive revision provides valuable evidence of the recomposition of the text under constraints of space. But, on another, it also makes particular aspects of editorial choice open to our view. Each editor (and editorial assistant) had, in reality, some latitude as to where the revising hand might fall. 

The iterated vulnerability to attrition of e.g. citations from periodical literature, from non-UK writers, and minor writers more generally when aggregated across the Proofs can thereby both document the fact of wider patterns of representation than later appeared in the published text, as well as the praxis by which such representativeness was reduced. As in Craigie’s re-editing of \emph{nameless}, for example, evidence from Pope, Cowper, Tennyson, and Ruskin was, expansively, retained in full while, in order to satisfy the need for space, a citation from Adelaide Procter’s \emph{Legends and Lyrics} (1859) was instead reduced to the sharply truncated “Over […] a nameless grave”. In parallel ways, Edward Bulwer Lytton was clearly deemed of greater salience under \emph{bulky} than the evidence (from the \emph{Scotsman}) that Murray now deleted under the adjacent \emph{bulking} while, under \emph{object}, it was Frances Anne Kemble’s \emph{Journal of a Residence on a Georgia Plantation}~-- now recognized as a key text within the literature of American slavery~-- that was deleted (also by Murray) while evidence from J. H. Newman and Mark Pattison remained intact. Such instances can be multiplied across the extant text. Editing was, as the Proofs widely confirm, a process in which push routinely came to shove under the conditions of substitutive revision. Whether, given adequate space, such cuts would have been made is doubtful. 

\section{On being too long}

No single change had, in reality, a monopoly over substitutive revision and the impacts it had. Etymology, sense-history, and definition (as well as citational range, as we have seen) can all be affected. As in the reshaping of \emph{dweomercraeft} (an obsolete word signifying ‘Jugglery, magic art’) and \emph{dwelth} (an obsolete word for ‘Error, delusion’), both evidence and entries are visibly renegotiated on the page. “1 line more”, as a comment in the margin attests, was necessitated by the addition of new etymological detail for \emph{dweomercraeft}. “1 line less” was, however, secured by cutting equivalent detail under the adjacent \emph{dwelth}~-- even if the information so cut had, in earlier iterations of the text, been deemed essential for the history and derivation that the OED should provide. Given the textual limits on space, the editor (Murray in this instance) had to choose and cut accordingly. Lexicographical fluency in the praxis and metalanguage of omission was conspicuous.  

Particularly interesting, however, is the related evidence that the Proofs provide of a set of changes that were driven, in entirety, by the processes of the press and the calibration of allowable space that language history might occupy. What we might regard as the trivial matter of textual justification~-- and the desire to secure the requisite neatness of the footer or bottom margin – can, in this light, emerge as a further directive for obligatory change. Signalled by the press-room annotation “long” (or, conversely, “short”) inscribed at the bottom of a printed column of text, press-room desiderata overtly took precedence while the dictionary-makers were compelled to adjust evidence and entries accordingly. 

“Short”, for example, routinely appeared in marking a column that, under the conditions of printing, fell short of its companions on the same page. It was, inevitably, far easier to remedy, given that it provided a rare opportunity for expansion. As on the Proof sheet of \emph{narratively}, for example, Craigie’s handwritten annotations (“In a narrative manner; \sout{after the style of} also considered as a narrative”) still illustrate his careful crafting of a definition that, under earlier pressures of space, had previously been omitted. Other examples can be located in e.g. \emph{cockney} where Murray likewise seized the opportunity for amplification “to make up the line or 2 lines short”, here by means of a new cross-reference as well as new embedded supporting evidence for \emph{cock’s egg}. He detailed a range of variables that the compositor might make use of, depending on the precise space required. Similar forms of delegated decision-making in relation to the published text are recorded elsewhere. “These are merely expansions \& fill up the space” and so “may be dealt with as is most convenient to you”, Murray likewise instructed under \emph{commission}. Not all, as he realized, could be included. But the act of selection was formally transferred from dictionary\hyp maker to press-room in the  final setting of the text. 

Compensatory expansion of this kind was, however, the exception rather than the rule. “Long” was far more prevalent as a press-room annotation while definitions, evidence, etymology, usage notes and labelling could all be affected by the excisions that result. As under the Proof text of \emph{oliva}, a word can be made to lose the senses it was previously thought to require or, as under \emph{qualm} (v.) or \emph{shiva}, to shed the chains of derivation they had previously possessed. In relation to definitions, too, a range of words were suddenly identified as targets for compression, gaining, by virtue of the injunction “long”, interpretative nuances that the relevant editor had not originally intended. Conditions of this kind attend, for example, the redefinition of \emph{outwittal} as ‘the point of outwitting’ (from the earlier ‘outdoing or frustrating by trickery’) or dictated the appearance in print of ‘to preach above and beyond’ (under \emph{overpreach} (v.) sense 1) instead of the earlier and more specific ‘To preach above or beyond (a person’s capacity)’. Craigie’s decision to compress ‘The study of the clouds as a branch of meteorology’ to ‘The study of the clouds’ (under \emph{nephology}) was part of the same process, as was the truncated definition of \emph{quietistic} so that if it applied to “quietists”, the earlier elaboration in terms of “quietism” was now removed. Other repercussions can be seen in the changed representation of supporting evidence, as dictated by an inscribed “long”, such that e.g. a vivid citation from Hawthorne’s \emph{Scarlet Letter} (used in illustration of \emph{frizzle}) was now reduced to its final five words: “\sout{Nothing was left of my respected predecessor, save an imperfect skeleton …and} a wig of majestic frizzle”. Related casualties included e.g. “\sout{His spontaneous emotion is} the result of two over-night glasses of brandy” (under \emph{overnight}); ‘\sout{A somewhat uncommon variety of the verruca simplex is} the subungual wart’ (under \emph{subungual}), ‘\sout{Their heads were full of} the languid dreams of commentators’ (under \emph{languid}), or ‘\sout{When He healed the sick,} that was an earnest...  of a sickless state’ (under \emph{sickless}). The relevant headword was, in each case, still illustrated. But, at a number of points,  loss arguably extends beyond the process of textual remediation that was pragmatically required. 

Labelling, and evidence of in-text decision-making on whether field or status labels could be retained, reveal related forms of textual adaptation. Prefatory matter for the individual fascicles of OED1 often provided a neat tabulation and quasi-quantification of the labels that were deployed. Nevertheless, scrutiny of the Proofs confirms that deleting a label was early recognized as a relatively easy way to secure an additional line in order to meet compositorial requirements of this kind – even if this also problematizes our understanding of whether a word was (or was not) considered as obsolete or nonce, or attributive, or regional in OED1. The entry for \emph{siest} (‘to take a siesta’) was, for example, “nonce” until the injunction “long” appeared on the relevant page proof; Murray’s accompanying marginal confirms that the resulting excision was “to enable the line to be saved” and makes the direction of willed change overt, alongside the priorities that textual revision had enforced. Similar is e.g. the annotation “omit if needed” which appears alongside a range of labels in the Proofs (and where “need” was delegated to the press-room, and decisions on what the final print text might require). Rare words can, by the same token, lose their rarity, obsolete words their obsolescence, and, as under \emph{squatmore}, dialectal words their regional status. 

Examples of this kind can be multiplied  over the course of the Editors’ Proofs.  The textual imperatives of closure, on various levels, drove a range of changes as part of late-stage editing, and in the intent to secure the neatness of the individual print page. Nevertheless, in examining the intricate play of form and revision across the annotated text, it is interesting to contemplate whether a better historical record, or more representative data, can in fact be found in these antecedent Proofs and in, say, the definitions that existed before enforced compression took place, or in the citational range which can likewise suggest other more inclusive stories before space and time took their toll. In the published fascicles of OED1, the print text might thereby be finalized and closure achieved. But across the Proofs, the underlying dialectics of choice and change are made visible, presenting the possibility of new forms of exegesis as well as alternative histories that undercut the apparent certainties of the published text, and the depersonalized “the dictionary” that results.

\addsec{Archival material}


\begin{description}[font=\normalfont]
\item Bodleian Library, Oxford. 30255
c. 17. Murray, James A. H., Henry Bradley, William Craigie \& Charles Onions, A New
English Dictionary on Historical Principles (annotated Editors’ Proofs). 1882--1928.
\item[OED/B/1/1/2.] Herbert Coleridge. [c. 1858]. Draft report on the New English Dictionary. OED Archives at Oxford University Press. 
\item[MP/22/7/1912.] James A. H. Murray to Walter Skeat. Murray Papers, Bodleian Library, Oxford.

\end{description}

\printbibliography[heading=subbibliography,notkeyword=this]
\end{document}
