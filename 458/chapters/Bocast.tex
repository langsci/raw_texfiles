\documentclass[output=paper]{langscibook}
\ChapterDOI{10.5281/zenodo.15394479}
\author{Alexander K. Bocast\affiliation{Berkeley Bridge Press, Anacortes, Washington, USA}}
\title[Disattributing Formey]
      {Disattributing the \textit{Encyclopédie} article on \textit{définition en logique} from Jean-Henri-Samuel Formey}

\abstract{I argue that Denis Diderot’s attribution of the article \emph{définition en logique} to Jean-Henri-Samuel Formey in the \emph{Encyclopédie} entry for \textsc{DÉFINITION} cannot be justified. First, I lay the groundwork for the argument by analyzing the sources, roles, and attributions for each paragraph in the whole entry on \textsc{DÉFINITION}. The origins of 40 of the 42 paragraphs in \textsc{DÉFINITION} are unambiguously identified; these 40 paragraphs are either original contributions or traceable to specific sources. These data demonstrate that the \emph{Encyclopédie} attributions for \emph{définition en mathématiques} to Jean le Rond d’Alembert and \emph{définition en rhétorique} to Edme-François Mallet and Jean François Marmontel are warranted. However, no paragraphs or passages in \emph{définition en logique} can be traced to Formey as \emph{author}. Second, having found no reason to consider Formey the author of \emph{définition en logique}, I ask whether Formey could have translated, compiled, or edited any part of this article. Regarding Formey as translator, having analyzed Formey’s bilingual edition of Savile’s \emph{Advice to a Daughter}, I find no evidence that Formey could have translated the English sources found in the article. Regarding Formey as compiler, I show that the treatment of definition in \emph{définition en logique} is not consistent with Formey’s writings on \emph{definition}. When Formey wrote about \emph{definition}, he consistently used Christian Wolff’s philosophical terms. However, three paragraphs in \emph{définition en logique} ridicule precisely those terms, which are not found elsewhere in the article. There is no reason to believe that Formey would have compiled an article from sources hostile to his own views. Regarding Formey as editor, I find no evidence that he had any editorial role in the preparation of \emph{définition en logique} or in the larger production of the \emph{Encyclopédie} itself. Instead, Étienne Bonnot de Condillac and d’Alembert are the editors most likely to have touched \emph{définition en logique}. Thus, my argument stands that Formey should be \emph{disattributed} from the article \emph{définition en logique}.}

\IfFileExists{../localcommands.tex}{
  \addbibresource{../localbibliography.bib}
  % add all extra packages you need to load to this file

\usepackage{tabularx,multicol}
\usepackage{url}
\urlstyle{same}

\usepackage{listings}
\lstset{basicstyle=\ttfamily,tabsize=2,breaklines=true}

\usepackage{langsci-basic}
\usepackage{langsci-optional}
\usepackage{langsci-lgr}
\usepackage{langsci-osl}
% \usepackage{./langsci/styles/langsci-lgr}
% \usepackage{./langsci/styles/langsci-osl}
% \usepackage{langsci-gb4e}

\usepackage{tikz}
\usetikzlibrary{patterns,calc}
\pgfdeclarepatternformonly{south east lines}{\pgfqpoint{-0pt}{-0pt}}{\pgfqpoint{3pt}{3pt}}{\pgfqpoint{3pt}{3pt}}{
    \pgfsetlinewidth{0.6pt}
    \pgfpathmoveto{\pgfqpoint{0pt}{3pt}}
    \pgfpathlineto{\pgfqpoint{3pt}{0pt}}
    \pgfpathmoveto{\pgfqpoint{.2pt}{-.2pt}}
    \pgfpathlineto{\pgfqpoint{-.2pt}{.2pt}}
    \pgfpathmoveto{\pgfqpoint{3.2pt}{2.8pt}}
    \pgfpathlineto{\pgfqpoint{2.8pt}{3.2pt}}
    \pgfusepath{stroke}}
    
\usepackage{stmaryrd}
\usepackage{wasysym}
\usepackage{multirow}
\usepackage{caption}
\usepackage{subcaption}
\usepackage{mathrsfs}
\usepackage{qtree}

\usepackage{linguex}


  %pminos do not split footnotes
% \interfootnotelinepenalty=10000 %Footnote in Laporte chapters has to be split SN


%\DeclareIndexNameFormat{default}{%
%\nameparts{#1}%
%\usebibmacro{index:name}%
%{\index[names]}%
%{\namepartfamily}%
%{\namepartgiveni}%
% {}% L1
% {}% L2
%{\namepartprefix}% generates spurious space L3
%{\namepartsuffix}% generates spurious space L4
%}

%  {\DeclareIndexNameFormat{default}{%
%     \usebibmacro{index:name}{\index[names]}{#1}{#3}{#5}{#7}}}

%\DeclareIndexNameFormat{default}{%
%  \usebibmacro{index:name}{\sindex[nom]}{#1}{#3}{#5}{#7}}

%\DeclareIndexNameFormat{default}{%
%  \usebibmacro{index:name}{\sindex[person]}{#1}{#3}{#5}{#7}}
%\DeclareIndexNameFormat{default}{%
%\nameparts{#1} \usebibmacro{index:name}{\sindex[person]]}{\namepartfamily}{‌​\namepartgiven}{\nam‌​epartprefix}{\namepa‌​rtsuffix}}

%\newcommand{\smiley}{:)}

%\renewbibmacro*{index:name}[5]{%
%\usebibmacro{index:entry}{#1}%
%{\iffieldundef{usera}{}{\thefield{usera}\actualoperator}\mkbibindexname{#2}{#3}{#4}{#5}}}

% \newcommand{\noop}[1]{}

%remove for final
%\overfullrule=1mm

\newcommand{\tobi}[2]}}
\renewcommand{\S}[1]{\tobi{#1}{\textsc{*}}}

% this volume references
% puts: [this volume]
% already defined: \citetv
%\newcommand{\citepv}[1]{(\citeauthor{#1} \citeyear*{#1} [this volume])}
\newcommand{\citealtv}[1]{\citeauthor{#1} \citeyear*{#1} [this volume]}

%parentheses around example number
\newcommand{\pref}[1]{(\ref{#1})}

% in-text examples

\newcommand{\lnex}[1]{\textit{#1}} %target lang word
\newcommand{\lnlit}[1]{(lit.: `#1')} %literal reading
\newcommand{\lnlat}[1]{(#1)} % latinization
\newcommand{\lntrans}[1]{`#1'} %translation
\newcommand{\lnexl}[2]%
{\lnex{#1}{} \lnlat{#2}} % ex with latinization
\newcommand{\lnexlat}[3]{\lnex{#1}{} \lnlat{#2}{} \lntrans{#3}} % ex with latinization and tranl.

%ch01
\newcommand{\co}[1]{\mbox{\textbf{#1}}}

%ch09

\newcommand{\cyrbulg}[1]{\begin{otherlanguage*}{bulgarian}#1\end{otherlanguage*}}


%ch10
\newcommand{\nlp}{{\small NLP}}
\newcommand{\mwe}{{\small MWE}}
\newcommand{\rae}{{\small RAE}}
\newcommand{\lvc}{{\small LVC}}
\newcommand{\pos}{{\small P}o{\small S}}
%\newcommand{\todo}[1]{ \textcolor{red}{#1} }

%\renewcommand{\labelenumi}{\theenumi}
%\ainamefmt{{vv}{ll}{, ff}{, jj}} % fullname

\newcommand{\biberror}[1]{{\color{red}#1}}

\newcommand{\osenovaitem}{--~}
  %% hyphenation points for line breaks
%% Normally, automatic hyphenation in LaTeX is very good
%% If a word is mis-hyphenated, add it to this file
%%
%% add information to TeX file before \begin{document} with:
%% %% hyphenation points for line breaks
%% Normally, automatic hyphenation in LaTeX is very good
%% If a word is mis-hyphenated, add it to this file
%%
%% add information to TeX file before \begin{document} with:
%% %% hyphenation points for line breaks
%% Normally, automatic hyphenation in LaTeX is very good
%% If a word is mis-hyphenated, add it to this file
%%
%% add information to TeX file before \begin{document} with:
%% \include{localhyphenation}
\hyphenation{
    Beck-man
    Ngu-yen
    back-chan-nel
    back-chan-nels
    mo-not-o-nous
    ste-reo-typ-i-cal
}

\hyphenation{
    Beck-man
    Ngu-yen
    back-chan-nel
    back-chan-nels
    mo-not-o-nous
    ste-reo-typ-i-cal
}

\hyphenation{
    Beck-man
    Ngu-yen
    back-chan-nel
    back-chan-nels
    mo-not-o-nous
    ste-reo-typ-i-cal
}

  \togglepaper[6]%%chapternumber
}{}

\begin{document} 
\maketitle

\section{Introduction and plan of analysis}
While translating the \emph{Encyclopédie}’s \textsc{définition} from French into English for a project on Gergonne’s theory of \emph{definition},\footnote{See Diderot and d’Alembert (1751‒1772: s.v. définition). This entry appears in the 4th volume, published in 1754. For my translation of this entry, see its three parts (\citealt{Formey2016, Dalembert2016, Mallet2016}) at \emph{The Encyclopedia of Diderot \& d’Alembert Collaborative Translation Project} \url{https://quod.lib.umich.edu/d/did/}. For Gergonne’s thoughts on definition, see \citet{Gergonne1818}.}  I found this anomaly: although the article \emph{définition en logique} is attributed to Jean-Henri-Samuel Formey, I could find no trace of Formey in the article itself. To understand this anomaly, I ask what relation, if any, did Formey have to this article and, if so, would this relation merit this attribution? Few detailed analyses of individual \emph{Encyclopédie} articles have been published,\footnote{One such study is \citet{Coste2009}.}  so this information may also contribute to other studies of the \emph{Encyclopédie}.

First, I look at the textual sources for the \emph{Encyclopédie}’s entry for \textsc{définition} and the articles it contains. With this detailed information in hand, gathered at the level of individual paragraphs, I look at relationships, if any, between attributions made within \textsc{définition} by the editors of the \emph{Encyclopédie} and the articles so attributed.

Second, I ask whether the attribution of the article on \emph{définition en logique} to Formey is warranted. Attributions in the \emph{Encyclopédie} tend to be ambiguous; they lump authors, compilers, translators, and editors together. They generally fail to distinguish whether an attributed person provided an original composition, assembled material from existing texts, translated an existing text, or edited an article supplied by other contributors. Would it be reasonable to assert that Formey is associated with this article in any of those roles?

\section{Analysis of the whole entry \textsc{définition}}
\subsection{Overview of \textsc{définition}}
The \emph{Encyclopédie} entry \textsc{définition} is substantial; it weighs in at seven and a half columns. It contains three named articles: the first treats \emph{définition en logique}; the second discusses \emph{définition en mathématiques}; and the third is focused on \emph{définition en rhétorique}. The longest of these articles, by far, is the article on \emph{définition en logique}, at almost six columns.

The article on \emph{définition en logique} is attributed as an ‘article of monsieur Formey’,\footnote{“Article de M. Formey.”}  implying that Formey wrote that article. The article on \emph{définition en mathématiques} is attributed to Jean le Rond d’Alembert, and I suppose that, as an editor of the \emph{Encyclopédie}, d’Alembert would know what d’Alembert wrote. The third article, on \emph{définition en rhétorique}, is attributed in part to Edme-François Mallet and in part to Jean François Marmontel.

Now to the sources of these texts. The first ten paragraphs of \emph{définition en logique} and the first four paragraphs of \emph{définition en rhétorique} are taken directly from the entry  \textsc{definition} in Ephraim Chambers’s \emph{Cyclopædia}, in the order that those paragraphs are found in Chambers’s entry. These two articles are also presented in the same order, that is, \emph{definition in logic} precedes \emph{definition in rhetoric} as \emph{définition en logique} comes before \emph{définition en rhétorique}. However, Chambers’s account of \textsc{definition} does not contain an article that corresponds to the \emph{Encyclopédie} article on \emph{définition en mathématiques}.\footnote{For a detailed analysis of Chambers’s entry on \emph{definition} and his sources, see \citet{Bocast2016}.} 

\subsection{Définition en logique}
The article \emph{définition en logique} comprises 32 paragraphs. The origin of 30 of these paragraphs is definitely known. The source of one paragraph can be confidently deduced. One paragraph is merely a one-sentence editorial segue, without substance. \tabref{tab:bocast:frequencies1} summarizes the sources, roles, and attributions for each paragraph in \emph{définition en logique}. It shows that Diderot’s attribution of this article to Formey has no grounds in any role related to the content of this article.


\begin{table}
\caption {Sources and roles for the \emph{Encyclopédie} article “définition en logique”.}
\label{tab:bocast:frequencies1}
\small
\begin{tabularx}{\linewidth}{p{15mm}Q@{~}cc@{~}Q@{}Q@{}}
\lsptoprule
\multicolumn{2}{l}{définition en logique} & \multicolumn{2}{c}{editors} & paragraphs &\multirow{2}{2.2cm}{\emph{Encyclopédie}\newline source}\\
\cmidrule(lr){1-2}\cmidrule(lr){3-4}
Author & Compiler & first & final & count=32 &  \\
\midrule
 & Chambers ; translated by  Mills \& Sellius &  &  & {\small ¶¶ 1‒10
 
 ¶¶ 22} & { Cyclopædia \citep[s.v. definition]{Chambers1741} } \\\addlinespace
Buffier & Condillac &  &  & {\small ¶¶ 11‒13 

¶¶ 20‒21} & { Traité des premiéres veritez (\citealt{Buffier1724}:  pt. 2, ch. 7)} \\\addlinespace
Condillac & Condillac &  &  & {\small ¶¶ 14‒19 

¶¶ 24 

¶¶ 26‒27} & { Essai sur l’origine des connoissances humaines (\citealt{Condillac1746}:  pt. 1, §§ 3 \& 5; pt. 2, § 1)} \\
 &  &  &  & {\small ¶¶ 23} & { one sentence segue; editorial glue} \\\addlinespace
Locke;\newline translated\newline by Coste & Condillac &  {Condillac} &  {d'Alembert} & {\small ¶¶ 25} & { Essai philosophique concernant l’entendement humain (\citealt{Locke1700b}:  bk. 3, ch. 4)} \\\addlinespace
Condillac &  &  &  & {\small ¶¶ 28} & { presumably Condillac, but unresolved}  \\\addlinespace
Port-Royal & Condillac &  &  & {\small ¶¶ 29‒32} & { La Logique, 6th ed. (\citealt[pt. 1, ch. 12 \& 13]{ArnauldNicole1712}}  \\\addlinespace
\lspbottomrule
\end{tabularx}
\end{table}


In all, eleven of the 32 paragraphs of \emph{définition en logique} are taken from Chambers; this is more than one-third of the paragraphs in the \emph{Encyclopédie} article. As well, four of the five paragraphs in \emph{définition en rhétorique} were taken from Chambers. This preponderance of Chambers in their corresponding \emph{Encyclopédie} articles strongly suggests that those paragraphs are based upon the translation delivered by John Mills and Gottfried Sellius to the unscrupulous publisher André le Breton in 1745. I will return to this when I ask whether Formey could himself have translated Chambers’s articles from English into French.

The next major source for \emph{définition en logique} is the writings of Étienne Bonnot de Condillac, specifically his \emph{Essai sur l’origine des connoissances humaines} \citep{Condillac1746}.\footnote{A few years later, Thomas Nugent translated Condillac’s \emph{Essai sur l'origine des connoissances humaines} into English. Nugent emphasized Condillac’s relation to John Locke when he titled his translation \emph{An Essay on the Origin of Human Knowledge: Being a Supplement to Mr. Locke’s Essay on the Human Understanding} \citep{Condillac1756}.}  The 1746 date of publication of Condillac’s \emph{Essai} is, I think, significant. Condillac was at the time in Paris, dining weekly with Denis Diderot and Jean Jacques Rousseau, and contributed somewhat to the early editing of the \emph{Encyclopédie} (\citealt{Dalembert1995}:  xiii‒xiv; noted in Schwab’s introduction). I believe that both the Mills/Sellius translation \emph{and} a treatment of \emph{definition} from Formey fell into Condillac’s hands for editing.

At that time, two intellectual movements dominated the Enlightenment of western Europe. One centered on the Englishmen Isaac Newton and John Locke; and the other followed the Germans Gottfried Wilhelm Leibniz and Christian Wolff. (Remember that Newton and Leibniz both claimed to have invented the calculus.) Condillac favored Newton and Locke, while Formey was a quite orthodox proponent of Wolff’s philosophy. Again, I will come back to Condillac’s contribution when I ask whether the \emph{Encyclopédie} article on \emph{définition en logique} is consistent with Formey’s commitment to Wolff.

Next comes Claude Buffier, with extracts taken from his \emph{Traité des premiéres veritez, et de la source de nos jugemens} \citep{Buffier1724}. His five paragraphs bracket the longest run of paragraphs taken from Condillac’s \emph{Essai}. Buffier was a Jesuit, anathema to the Huguenot Formey, who took a psychological approach to epistemology that was not inconsistent with Condillac’s primacy of sensation as the origin of knowledge.

Buffered between paragraphs traced to Condillac, there is a paragraph taken from John Locke’s \emph{Essay Concerning Human Understanding} \citep{Locke1700a}. In consultation with Locke, Pierre Coste translated Locke’s \emph{Essay} into French as the \emph{Essai philosophique concernant l’entendement humain} \citep{Locke1700b}. The long paragraph found in the \emph{Encyclopédie} article is taken directly from Coste’s translation.

Of course, if a Francophone intellectual discussed \emph{definition} in the 18th century, there was sure to be some recourse to the \emph{Port Royal Logic}, that is, to the \emph{La Logique ou l’Art de penser} of Antoine Arnauld et Pierre Nicole \citep{ArnauldNicole1712}. The final four paragraphs of the article on \emph{définition en logique} are taken from part 1, chapters 12 and 13, it seems of the 6th edition of 1712.

There are only two other paragraphs in the article \emph{définition en logique} to mention. One consists of a single short sentence: 'In general, we can make the following \emph{observations} about definition".\footnote{On peut faire sur la définition en général les réflexions suivantes.'} It is followed by six ordinally-identified observations. The first two were taken from Condillac, the third is of unknown provenance, and those remaining are found in the \emph{Port-Royal Logic}. That third observation is given by the 28th paragraph; it consists of a single sentence, but  is of respectable length: 'Third, definitions never help us know the nature of substances, but only essences that we conflate with notions that we form of things, notions based on \emph{archetypal} ideas and not on actually existing real things such as substances.'\footnote{“3º. Les définitions ne nous aident jamais à connoître la nature des substances, mais seulement les essences qui se confondent avec les notions que nous nous faisons des choses ; notions fondées sur des idées archétypes, \& non pas d’après des modeles réellement existans, ainsi que sont les substances.”}  The word \emph{archetypal} certainly suggests Condillac as \emph{author}, just as the ordinal ordering of these observations suggests Condillac as \emph{editor}. However, I have not traced this fragment to some specific passage in Condillac; it may have been constructed from Condillac’s words by another \emph{Encyclopédie} editor.

This accounts for the whole of the text of the article on \emph{définition en logique}, excepting of course a number of details and transitions within paragraphs. Of the 32 paragraphs of the article on \emph{définition en logique}, 30 paragraphs can be definitively traced to specific passages in the works of Chambers, Buffier, Condillac, Coste’s Locke, and the \emph{Port-Royal Logic}. No passages can be traced to the pen of Formey.

% \subsection{Définition en mathématiques}

The article on \emph{définition en mathématiques} is not related to Chambers’s \textsc{définition}, which does not contain an article on \emph{definition in mathematics}. It is attributed to d’Alembert, perhaps by d’Alembert, and I have found no reasons to doubt that assertion. This attribution assures a reader that d’Alembert wrote \emph{définition en mathématiques}. Further, \emph{définition en mathématiques} contains references to the physically-preceding \emph{définition en logique}; to me, these references argue that \emph{définition en mathématiques} was prepared after or alongside \emph{définition en logique}. Throughout the \emph{Encyclopédie}, signatures of d’Alembert are frequently associated with attributions to Formey;\footnote{For an example, see \citet{Coste2009}.}  it is reasonable to suppose that d’Alembert also played some editorial role in the production of \emph{définition en logique}. The attribution to d’Alembert recognizes his role as \emph{author} of the article.

\subsection{Définition en rhétorique}

Now I can briefly consider the article on \emph{définition en rhétorique}. The first four paragraphs account for the whole of Chambers’s corresponding \emph{definition in rhetoric} as translated by Mills \& Sellius;\footnote{Chambers’s sources for these paragraphs are examined by \citet[203-225]{Bocast2016}.}  I assume their translation was fairly literal. Diderot presumably tasked Edme-François Mallet ‘to review, correct, and augment’ the article seeded by that translation.\footnote{“Mais en présentant à chacun un rouleau de papiers, qu’il ne s’agissoit que de revoir, corriger, augmenter”. (\citealt{Diderot1751}: s.v. encyclopédie)}  I deduce that Mallet completed this task because Diderot attributed these paragraphs to him.\footnote{The attribution symbol for Mallet is (G). It is placed at the end of the fourth paragraph in the article.}  Subsequently, Diderot asked Marmontel to review and improve articles prepared by Mallet, who was prolific but may have been sometimes overly hasty.\footnote{“Beginning in volume iii, Marmontel was assigned . . . to revise before publication other articles that Mallet had drafted.” \citep[240]{KafkerKafker1988}.}  Diderot’s attribution “Article de M. Marmontel” at the end of the article appears to confirm Marmontel’s editorial review of Mallet’s editorial review.

As printed, the text given to Mallet has been carefully improved and expanded, but the article does not venture further than Chambers’s original paragraphs. Cumulative changes to Chambers’s article by Mills and Sellius, Mallet, and Marmontel cannot be sorted out, but whatever changes Marmontel might have made did not displace the attribution to Mallet at the end of the fourth paragraph. The attribution to Marmontel is found later in the text, at the end of the fifth and final paragraph. 

However, beyond physical proximity, the fifth paragraph is wholly unrelated to the preceding article \emph{définition en rhétorique}. This divide and this paragraph are seen again in Panckoucke’s \emph{Encyclopédie méthodique} \citep{Marmontel1782}, and Marmontel definitely claimed this text as his own in his \emph{Oeuvres complettes}\citep[7, 368-369]{Marmontel1787}. It seems that Marmontel appended this paragraph to Mallet’s manuscript before returning it to Diderot; this new material addresses an aspect of ‘philosophical definitions’, specifically a metaphysical role in unifying intersubjective understandings of things held to be ontologically real. Mallet’s \emph{définition en rhétorique} and Marmontel’s \emph{définitions philosophiques} are distinguished by their respective use of internal references. Following Chambers’s \emph{definition in rhetoric}, Mallet’s \emph{définition en rhétoriqu}e contains no pointers to other entries. In contrast, Marmontel supplied explicit references to entries for \emph{idea}, \emph{truth}, \emph{evidence}, and \emph{error}, entries primarily tied to philosophy rather than rhetoric.\footnote{That is, references to the entries \textsc{IDÉE}, \textsc{VÉRITE}, \textsc{ÉVIDENCE}, and \textsc{ERREUR}.}  Table 3 summarizes the sources, roles, and attributions for each paragraph in \emph{définition en rhétorique}. It shows an attribution to Mallet that recognizes his role as primary editor of the paragraphs constituting the article \emph{définition en rhétorique} and an attribution to Marmontel that recognizes his role as author of the final paragraph of the entry on \emph{définition}.



\begin{table}
\caption{Sources, roles, and attributions for \emph{définition en rhétorique}}
\label{tab:bocast:frequencies2}
\footnotesize
\begin{tabularx}{\linewidth}{ Qp{14mm}@{} cccc Q}
\lsptoprule
\multicolumn{2}{l}{\emph{définition en rhétorique}} & \multicolumn{2}{c}{editors} & \multicolumn{2}{c}{paragraphs count=5} & \multirow  {2}{1.7cm}{\emph{Encyclopédie} source}\\
{Author} & {Compiler} & {first} & {final} & {attribution} &  &  \\
\midrule
{de Colonia; translated by Chambers} & {Chambers; translated by Mills \& Sellius} & {Mallet} & {Marmontel} & {Mallet} & {¶¶ 1‒4} & {\citet{Chambers1741}: s.v. definition} \\
{Marmontel} &  & {Marmontel} & {Marmontel} & {Marmontel} & {¶¶ 5} & {Marmontel} \\
\lspbottomrule
\end{tabularx}
\end{table}

Thus, Diderot’s attribution of \emph{définition en rhétorique}, especially given internal evidence and retrospective testimony from the \emph{Encyclopédie méthodique} and Marmontel’s \emph{Œuvres complettes}, should be considered in light of these interwoven roles: \emph{author}, Dominique de Colonia;\footnote{Chambers’s \emph{definition in rhetoric} is taken almost entirely from Dominique \citet{De_colonia1710}. For a detailed look at de Colonia as a source for Chambers, see \citet[206-228]{Bocast2016}.}  \emph{translator \& compiler}, Ephraim Chambers; \emph{translator}, John Mills \& Gottfried Sellius; \emph{primary Encyclopédie editor}, Edme-François Mallet; and \emph{secondary Encyclopédie editor}, Jean François Marmontel. Diderot’s attribution of \emph{définition en rhétorique} should be read as affirming Mallet’s role as its \emph{primary editor}.

Marmontel’s lone paragraph is wholly inadequate to encompass \emph{definition} in philosophy.\footnote{Indeed, Marmontel’s expansion of this paragraph in the \emph{Encyclopédie méthodique} is more than five columns long \citep{Marmontel1782}. There Marmontel treats \emph{definition} in legal argumentation, in laudatory speech and elegies, in poetry, and, finally, in philosophy.}  I suspect that Diderot simply did not notice that Marmontel had gone beyond his remit to review and revise Mallet’s article. Thus, Diderot did not insert the typographic and syntactic cues that typically introduce an article and a new attributee within an \emph{Encyclopédie} entry.\footnote{As printed, the article structure of the entry \emph{définition} is revealed by typographic and syntactic cues, except for the final paragraph:

\textbf{DÉFINITION}, \emph{en logique}, est . . . \emph{Article de M.} \textbf{FORMEY}.

\textbf{DÉFINITION}, \emph{en mathématiques}, c’est . . . (O)

\textbf{DÉFINITION}, \emph{en rhétorique}, c’est . . . (G)

A l’égard des \emph{définitions} philosophiques, elles sont . . . \emph{Article de M.} \textbf{MARMONTEL}.

 
A sequence of consistent cues might have been:


\textbf{DEFINITION}, \emph{en logique}, est . . . \emph{Article de M.} \textbf{FORMEY}.

\textbf{DÉFINITION}, \emph{en mathématiques}, c’est . . . (O)

\textbf{DÉFINITION}, \emph{en rhétorique}, c’est . . . (G)

\textbf{DÉFINITION}, \emph{en philosophie}, c’est . . . \emph{Article de M.} \textbf{MARMONTEL}.


These cues mark four articles in this entry rather than the customary three.
}  I think he simply added the attribution \emph{“Article de M. Marmontel”} to the end of the manuscript before sending it off to the printers. Whether I have correctly intuited this story, sufficient evidence indicates that the intent of Diderot’s attribution to Marmontel was to recognize his contribution as the secondary \emph{Encyclopédie} editor of \emph{définition en rhétorique}; instead, I believe this attribution should be read to acknowledge Marmontel as author of the last paragraph rather than as editor of the preceding four paragraphs.

\section{Analysis of définition en logique}
\subsection{Formey considered as author}

Given this analysis of textual sources, the notion that Formey might have authored \emph{définition en logique} is quickly disposed of. I have identified specific authors and sources for 30 of the 32 paragraphs in the article, and putative editors for the remaining two. None of these 32 paragraphs can be traced to Formey’s writings. Thus, Formey is clearly \emph{not} the \emph{author} of this article.

\subsection{Formey considered as translator}
\subsubsection{… regarding Chambers’s \textsc{definition}}

Because a third of the article on \emph{définition en logique} is taken from English sources, I ask whether Formey was himself capable of translating such material from English to French. Formey was of course intimate with French, and as Secretary of the Berlin Academy he translated works into French from German and from Latin.\footnote{“Er übernahm die Verpflichtung, deutsch oder lateinisch abgefaßte wissenschaftliche Referate für die Sammelbände der Akademie zu übersetzen”. \citep[55]{Krauss1963Akademiesekretär}.

\citet{Thomson1980} related that “De 1745 à 1747, il servit comme traducteur du roi dans le Département des Affaires étrangères.” This was echoed by \citet[140]{KafkerKafker1988}: “from 1745 to 1747, [Formey] served as the royal translator for the Prussian Department of Foreign Affairs.”

Grosse notes: “Formey was engaged in the translation and publication of other authors' works: beginning with sermons and the writings of the German philosopher Christian Wolff, he later also translated the works of ancient philosophers and the scientific contributions of his German colleagues at the Academy.” \citep[10]{Grosse2018}

None of these suggest any knowledge of English.
}  However, I have no evidence that Formey conversed or wrote in English. He may have had some ability to read English in correspondence, but few letters directed to him as Secretary of the Academy were written in English.\footnote{“Les éditeurs du présent livre ont choisi de publier 116 lettres écrites en français (trois seulement sont en anglais) de la correspondance passive de Formey. Leur unité affichée est dans leur commune provenance géographique : elles ont toutes été ecrites, entre 1737 et 1788, d'Angleterre à Formey résidant à Berlin.” \citep[137]{Hubac2007}}

Formey’s \emph{Conseils pour former une bibliotheque} is indicative. In the first edition of 1746, he wrote ‘but of more value than these [other dictionaries] for those who understand English is Chambers’s \emph{Encyclopaedia}; we continue to look forward to a French edition’,\footnote{“Enfin les Dictionaires Philosophiques sont des repertoires utiles. Il y a en Latin celui de Chauvin, in folio, \& en Allemand celui de Walch, en un gros in 8vo, mais ce qui vaut mieux que tout cela pour ceux qui entendent l’Anglois, c’est l’\emph{Encyclopedie de Chambers}, dont on avoit fait esperer une Edition Françoise, que je ne sai quels contretems ont derangé.” \citep[13]{Formey1746}.

On p. 56, Formey recommends the \emph{“Encyclopédie de Chambers”} as a work of philosophy.}  where the words “pour ceux” imply that Formey himself is \emph{not} among those who understand English. Here and consistently thereafter in those Conseils he mispresented the \emph{Cyclopædia} as the \emph{“Encyclopaedia”}, which indicates that he did not possess a copy of Chambers’s work but knew it only by reputation. In the 2nd edition of 1750~-- \emph{after} he had delivered his manuscripts to the \emph{Encyclopédie}~-- Formey added that a translation of the \emph{Encyclopaedia} of Chambers was being printed in Paris under the direction of d’Alembert and Diderot.\footnote{“Enfin les Dictionaires Philosophiques sont des répertoires utiles. Il y a en Latin celui de Chauvin, in folio, \& en Allemand celui de Walch, en un gros in 8vo, mais ce qui vaut mieux que tout cela pour ceux qui entendent l’Anglois, c’est l’\emph{Encyclopedie de Chambers}, dont la Traduction s’imprime actuellement à Paris, sous la direction de Mrs. d’Alembert \& Diderot, avec des additions \& des ameliorations, qui feront de cet Ouvrage un vrai Trésor.” \citep[18]{Formey1750}.

On p. 80, Formey again recommends the \emph{Encyclopédie de Chambers} as a work of philosophy.} He made no claim to have a stake or a role in this translation.

\subsubsection{… regarding Savile’s Ladies-New-Years Gift}

Chambers disappeared from the 1755 edition of Formey’s \emph{Conseils}; the \emph{Encyclopédie} replaced the \emph{Cyclopaedia}. However, this edition intimates that perhaps Formey could translate from English to French. Formey reported that the Berlin firm of \emph{Haude \& Spener} had asked him to prepare a new edition of George Savile’s \emph{The Ladies New-Years Gift: or, Advice to a Daughter}.\footnote{See \citep{Savile1688}. Three editions were published that year in London and Edinburgh. The 4th edition is the last before Savile’s death; it is the edition that was translated into French. Subsequent editions are merely reprints from different publishers; fifteen of these were published in London from 1688 through 1765. This work was also published in the three editions of \emph{Miscellanies}, a collection of Savile’s writings; see \citet{Savile1700, Savile1704, Savile1717}).}  This he did ‘by putting to the side of the English text a French translation,\footnote{For his English text, Formey drew from the 13\textsuperscript{th} (1748) London edition; see \citet["Avertissement"]{Savile1752, }. I have been unable to consult that edition. However, the 12th edition and the 14th edition seem to be identical (\citealt{Savile1741,Savile1756}), and I have found no discrepancies between the English text in Formey’s edition and either of those editions.

For his French text, Formey had in hand the 1698 Den Haag translation; see  \citet[“Avertissement”]{Savile1698,Savile1752}. The 1698 edition is practically identical to the first published French translation \citep{Savile1692}. Indeed, subsequent editions are merely reprints of the 1692 edition; thus, Formey \emph{could} have used any of those four printings; see \citet{Savile1692,Savile1699,Savile1748}.
} which already existed, but which I retouched in several places.’\footnote{“Le Libraire \emph{Spener} m’ayant prié de procurer une nouvelle Edition d’un Ouvrage Anglois du Marquis d’\emph{Halifax}, \emph{Conseils d’un Père à sa Fille}, je le fis en mettant à côté du Texte une Traduction Françoise, qui existoit déjà, mais que je retouchai en plusieurs endroits. Cela fit un Volume \emph{in octavo}, qui parut au commencement de 1753.” (\citealt{Formey1755}:  120)

The resulting book had a different title: \emph{Conseils d’un homme de qualité à sa fille}; see \citet{Savile1752}. 
}  This might be read to suggest that Formey drew his changes directly from the English work, but an announcement in the \emph{Bibliothèque impartiale} in 1752 should discourage such a reading: ‘The \emph{Conseils d’un Père à sa fille} in French and English has just been reprinted by \emph{Haude \& Spener}. M. Formey, the editor, has recast the French version which was already out of date.’\footnote{“Les \emph{Conseils d’un Père à sa fille} en Anglois \& en François, ouvrage du célèbre Lord \emph{Halifax}, viennent d’être réimprimés chez \emph{Haude \& Spencer}. Mr \emph{Formey}, qui en est l’Editeur \& qui a refondu la Version Françoise \& qui étoit déjà surannée”. \citep[1, 158]{Luzac1752}.

The name of the bookseller \emph{Johann Carl Spener}, of \emph{Haude \& Spener}, was misprinted in this passage. 
} 

Formey went on to say that the 1698 “translation has aged and, moreover, it is not free from mistakes. I have carefully retouched it; and my goal, by setting the English and French texts side by side, is at the same time to provide a small workbook for people who are applying themselves to the study of the English language.”\footnote{“Cette Traduction avoit vieilli; \& n’etoit pas d’ailleurs exemte de défauts. Je l’ai soigneusement retouchée ; \& mon but, en mettant le Texte à côté, est en même tems de fournir un petit Manuel aux personnes qui s’appliquent à l’etude de la Langue Anglois.” (\citealt{Savile1752}:  “Avertissement”)}  Thus, the Haude \& Spener edition is bilingual, setting French and English in parallel columns. The French text, before and after Formey, is more a paraphrastic approximation of the English text than an actual translation; it abounds with departures from Savile’s stated intent.

\largerpage
Well, I looked in detail at Formey’s \emph{Conseils}, both the English and the French, and I compared them with each other, with Savile’s original English text, and with the previous printings of the \emph{Conseils}.\footnote{I examined three chapters in Formey’s edition: the introduction, the 3rd chapter, and the 9th and final chapter. These sections contain 40 pages; the whole book has just 152 pages. In addition, I looked closely at the first five lines of every fifth page, starting with page 6. Altogether, I have examined about 30\% of Formey’s French text and the corresponding texts in Savile’s English and in the previous \emph{Conseils}.}  I studied punctuation, orthography, verb tense, and end-of-sentence hyphenation;\footnote{French hyphenation was used in both columns. This leads to such oddities in the English column as \emph{enti-rely}, \emph{li-ne}, \emph{ru-le}, \emph{mista-kes}, \emph{star-ve}, and \emph{ushe-red}.}  relative density of typographical errors in both English and French texts; alignment of English to French text across page boundaries; widows, orphans, and runts, and the allotment and distribution of whitespace; alignment of lexical cognate terms and terminal punctuation in English and French; textual differences between previous editions of the \emph{Conseils} and Formey’s edition, including chapter titling;\footnote{Formey gives the mysterious title “Discours sur le Prochain” to a chapter titled “Censure” in Savile’s English and “Des circonspections que doit aporter une Dame en parlant des gens” in previous \emph{Conseils}.}  Formey’s substitution of synonyms for their equivalents; the fidelity of the previous French translation to Savile’s original text; the changes made in the French text by Formey and how these changes relate to both Savile’s text and the previous French translation; whether Formey’s changes did fix mistakes or improve the translation;\footnote{In at least one case, Formey found a grammatical mistake but his fix degraded the translation. On the first page of Formey’s edition \citep[1]{Savile1752} is the passage “il est occupé d’une Passion \emph{qui lui plait}.” The expression found in earlier editions (e.g., \citealt[2]{Savile1748}) is “\emph{qui le flate.}” This difference may be read as ‘that he likes’ in contrast to ‘that flatters him’. Both misrepresent Savile’s intent when he wrote “it is possess’d by a darling Passion.” Here, Formey did correctly identify a flaw in the original translation, which should have been \emph{qu’il flatte}. Unfortunately, instead of just correcting the grammatical error, which put the cart before the horse, Formey changed the sense of \emph{darling passion}, which is reflected by \emph{qu’il flatte}, to the weak notion ‘a passion that he likes’. This suggests that Formey did not consult the English text.}  and errors that Formey did \emph{not} correct, among other considerations.\footnote{At one point, Formey fell into an interesting error. An English passage reads “and the old Housekeeper shall make a better figure in the Family”; the parallel French passage is “Et une vieille \emph{Ausquiper} sera meilleure figure dans la famille”, with a footnote attached to the word \emph{ausquiper} (\citealt{Savile1752}:  64). Now, \emph{ausquiper} is a nonce word; it is simply the English word \emph{housekeeper} phonetically spelled for French ears. Nonetheless, Formey mistook \emph{ausquiper} as an ordinary word, although it is not clear whether he took it for an English word or a French word. To the existing footnote he appended a claim that the word \emph{ausquiper} is an equivalent of the German word \emph{Hausgeber}. This again suggests that Formey did not read the English text.}

\largerpage[2]
With one possible exception,\footnote{Savile’s salutation “Dear Daughter” appears like a title on the introductory page \citep[3]{Savile1741}. Savile’s \emph{Conseils} take “Dear Daughter” as a salutation rather than a section title and remove it to the introduction’s first sentence: “J’éprouve tous les jours, \emph{Ma chére Fille}, que nos pensées les plus agréables sont, malgré nous, mêlées d'inquiétude”, leaving the introduction without a title \citep[1]{Savile1748}. The titles of Savile’s (1741) chapters are all formed by one to three nouns assisted by the conjunction \emph{and}. The titles of the \emph{Conseils} chapters are longer phrases, such as “Comment il faut qu’une Dame se conduise sur tout en Conversation” (\citealt{Savile1748}:  table des matieres). Set side by side with Savile’s English titles, Formey’s French titles are modelled on the \emph{form} of their English companions, but take their \emph{nouns and connective tissue} from the previous French titles (\citealt{Savile1752}:  Table). However, the introductory section did not have a French title \citep[1]{Savile1748}. As the compositor at Haude \& Spener set up this page (\citealt{Savile1752}:  1), an unwanted gap would then appear in the French column opposite the salutation “Dear Daughter” in the English column. To fill this gap, “Ma chere Fille” was promoted from the first sentence in the French column to the title position. \emph{Someone} recognized that it would be appropriate to set “Ma chere Fille” in the French column opposite “Dear Daughter” in the English column. \emph{Someone} recognized that “Ma chere Fille” and “Dear Daughter” are functionally equivalent as salutations, even though they are not quite identical in meaning. This someone may have been Formey. However, this is the \emph{only} change in Formey’s edition that seems related to Savile’s English original, so my money is on a balance-seeking compositor.}  I found no evidence that Formey consulted Savile’s English text. I found no passage in Formey’s edition that corrected any of the many infelicities in the original translation of 1698.
\clearpage

Instead, I found many suggestions that Formey ignored or did not understand the English text, to which he made no changes. The few changes that he made to the French text were editorial: orthography, punctuation, choice of word, choice of tense. These changes were made with respect to the previous French text; they are unrelated to the English text. This is to say, these changes could have been made without having in hand Savile’s English text. Thus, Formey’s \emph{Conseils} does not support the supposition that Formey could translate from English to French.

\subsubsection{… regarding Locke’s \textit{Essay}}

The one paragraph in \emph{définition en logique} that was taken from Locke’s \emph{Essay} was lifted, with some editorial compression, from Coste’s translation.\footnote{See \citet[336]{Locke1729}.}  

Formey noted three books written by Locke in his \emph{Conseils pour former une bibliotheque}. Coincidentally, but perhaps not, these three books had been translated into French \emph{before} Formey made his recommendations. These books are Mazel’s \emph{Du gouvernement civil} \citep{Locke1691}; Coste’s \emph{Education des enfans} \citep{Locke1695}; and Coste’s \emph{Essai sur l’entendement} \citep{Locke1700b}.\footnote{It would have been quite impertinent of Formey to offer his own translation of Locke’s \emph{Essay}. Coste had worked closely with Locke, who was himself competent in French, to ensure that his translation (and Locke’s text itself) suitably and adequately conveyed Locke’s ideas \citep{Soulard2011}.}  It might be thought that, addressed to a French audience, only works in French would be considered by Formey’s \emph{Conseils pour former une bibliotheque}; but recall that Formey had also recommended Chambers’s \emph{Cyclopædia}, then still an English work. One explanation of Formey’s selection of these three books may be that he had read or knew their French incarnations but he did not recommend other works by Locke because Formey could not read English.\footnote{For another example of Formey copying a French translation of an English original rather than translating the English himself, see \citet[10]{Coste2009}.}

It still might be so that Formey copied Coste’s translation and included it among the pages received from Formey by the editors of the \emph{Encyclopédie}. However, the passage from Locke is bracketed, fore and aft, by paragraphs from Condillac’s \emph{Essai}. This context suggests that Condillac, not Formey, supplied this passage.

Nonetheless, it does seem clear that Formey did not possess or at least did not deploy a competence to translate Chambers for his own article on \emph{définition en logique}. Thus, translation by Formey of Chambers’s article on \emph{definition in logic} seems far-fetched.

\subsection{Formey considered as compiler}
\subsubsection{Formey’s writings on definition}

If Formey neither wrote nor translated paragraphs in \emph{définition en logique}, could he have assembled that text? For this, I examine Formey’s known writings on \emph{definition}, compare them with the substance of this article, and assess whether Condillac’s critique of Wolff is consistent with Formey’s known views.

The manuscripts that Formey sold to the \emph{Encyclopédie} editors were eventually returned to him at the Prussian Academy, but they disappeared, were looted, during the Russian occupation of Berlin in the aftermath of the Second World War; they have not been seen again \citep[299]{Marcu1953}. Thus, I have been unable to consult Formey’s papers to compare them with the printed \emph{Encyclopédie} article. However, Formey did write on \emph{definition} in other places; these include his \emph{La belle Wolfienne} in 1741;\footnote{“On ne doit faire entrer dans la Définition, que les Qualitez intrinseques du Sujet, c’est-à-dire, ses Qualitez essentielles \& ses Attributs, qui fournissent les \textbf{marques}, auxquelles il est reconnoissable. Les \textbf{Possibilitez} intrinseques peuvent néanmoins y entrer. . . . Cette propriété donc de se fondre, étant un résultat des Qualitez essentielles du Plomb, est une \textbf{Possibilité} intrinseque, qui convient à toute masse de ce Métal, \& qui peut, par conséquent, entrer dans sa Définition. On conçoit alors la Liquebilité, comme une Qualité intrinseque du Plomb, \& on la range parmi ses \textbf{marques}, comme servant à la distinguer des autres Corps. La \textbf{Possibilité} même des Modes \& des Relations, étant une suite des Qualitez essentielles, en vertu desquelles un Sujet est susceptible de telles ou telles Déterminations extérieures ; cette \textbf{Possibilité}, dis-je peut encore être considérée comme intrinseque au Sujet, \& admissible dans sa Définition. . . . Tout ce qui est, étant \textbf{possible}, puis qu’il est ; \& cette \textbf{Possibilité} supposant une raison suffisante pourquoi il est ainsi, \& non autrement ; la Définition, qui ne rend pas cette raison, n’est que nominale : celle, qui la rend, est réelle. . . . En effet, elles ne font qu’indiquer l’assemblage d’un certain nombre de Qualitez, dont l’une ne détermine pas nécessairement l’autre, \& d’où il ne résulte rien, ni pour, ni contre, la \textbf{Possibilité} du Défini.” \citep[1-120 et seqq]{Formey1741}.} his \emph{Abrégé de toutes les sciences, à l’usage des enfans} (1760);\footnote{“D. Qu’entendez-vous par différence de la définition? R. J’entends par différence de la définition, ce qui fait différer essentiallement la chose définie de toute autre chose \textbf{existante}. D. Rendez sensible par un exemple tout ce que vous venez de dire. R. Le voici par la définition de l’homme ; quand je dis, l’homme est un Etre raisonnable, je donne sa définition par son genre \& sa différence. Le mot Etre est le genre, parce qu’il confond l’existence de l’homme avec l’\textbf{existence} de tout autre être sans en spécifier aucun.” \citep[3]{Formey1760}}  his \emph{Abrégé de toutes les sciences, à l’usage des adolescens} (1764);\footnote{“Cependant, dès qu’une idée est négative, ou destructive de l’autre, le terme est nul, il porte le caractère évident de sa réprobation dans la \textbf{contradiction} des deux idées simples auxquelles il répond. On peut inventer des exemples dans ce genre où la chose faute aux yeux. Une figure rectiligne formée par deux lignes, \emph{bilineum rectilineum}, est \textbf{impossible}, \& le terme qui la désigne ne se rapporte à rien.” \citep[295-296]{Formey1764}}  and his \emph{Dictionnaire instructif} (1767).\footnote{“Énumération des principales marques, qui caractérisent un objet, et en donnent une idée distincte. C’est en cela proprement que consiste la définition nominale. Mais la définition réelle comprend l’idée de la \textbf{possibilité} de ce qu’on définit et l’explication intelligible de ce qui constitue cette \textbf{possibilité}.” (\citealt{Formey1767}:  s.v. définition)}.  These examples of Formey’s definition of \emph{definition} faithfully reflect Wolff’s own definitions of \emph{definition}. Wolff presented his definitions most concisely in the many editions of his \emph{Mathematische Lexicon} and of his \emph{Philosophisches Lexicon};\footnote{See, for example, his \emph{Mathematisches Lexicon} \citep{Wolff1716}; his \emph{Philosophisches Lexicon}, \citep{Wolff1737}; and his \emph{Vollständiges Mathematisches Lexicon} \citep{Wolff1747}. In his \emph{Vollständiges Mathematisches Lexicon}, Wolff defines \emph{definition} so: “heisset ein deutlicher Begriff dessen, wodurch eine Sache van allen andern unterschieden wird. Man hat aber einen deutlichen Begriff, wenn die Vorstellung der Sache so beschaffen, dass man sagen kan, aus was vor \textbf{Merckmahlen} man sie erkennet, oder wodurch man sie von andern unterscheidet. Wird nun die Sache von andern ihres gleichens durch ihre besondere Eigenschafften unterschieden, so heisset es eine Wort-Erklärung, \emph{Definitio Nominalis}, dergleichen ist, wenn ich sage: Ein Uhrmerckt sey eine Maschine, welche die Stunden anzeiget, so erkläre ich das Wort Uhrmerck. Wird aber die Art and Weise angegeben, wie eine Sache \textbf{entstehen} kan, und ich zeige z. E. aus was vor Rädern und anderm nöthigen Zugehör in einer gewissen Absicht das Uhrmerck zusammen gesetzet, so ist dieses eine Sach-Erklärung, \emph{Definitio Realis}. Diese letzte hat vor der erstern diesen Vorzug, dass man gleich daraus erkennen kan, ob die Sachen \textbf{möglich} sind oder nicht, daher sie ohne fernern Beweis zum Grund einer richtigen Demonstration gebrauchet werden kan; da hingegen bey der erstern zu erweisen nöthig ist, ob sie \textbf{möglich} sey oder nicht. Uiber dieses lassen sich gar öffters auch aus den Sach-Erklärungen viele Dinge ohne all Umwege demonstriren, die man sonst mit Mühe erweisen muss. . . . Im übrigen ist sehr wohl zu behalten, dass zu einer Sach-Erklärung zweyerley erfodert werde, einmal, dass man wisse, was vor Dinge darzu gehören, wenn sie entstehen soll, und sodann auch, wie viel ein jedes von ihnen zur Sache beyträget.” (\citealt{Wolff1747}:  s.v. Erklärung)}  with admirable consistency, Wolff’s definitions of \emph{definition} in these \emph{Lexicons} bind Wolff’s notions of real or genitive definition to his essential ontological ideas of \emph{contradiction}, \emph{impossibility}, \emph{possibility}, and \emph{existence}.\footnote{In turn, these concepts borrow heavily from Gottfried Wilhelm Leibniz, but that is well outside the scope of this chapter.}

\largerpage[-3]
Without dissecting Formey’s passages on \emph{definition}, it is obvious that he deployed terms closely associated with Wolffian philosophy: \emph{possibility}, \emph{impossibility}, \emph{contradiction}, \emph{existence}, and, particularly with respect to \emph{definition}, the noun \emph{mark}.\footnote{The English “mark” or “characteristic” corresponds to Wolff’s use of “Merckmahl” and “Kennmerck” when he wrote in German and “nota” when he wrote in Latin.}  The presence or absence of these terms in the \emph{Encyclopédie} article on \emph{définition en logique} is therefore significant, particularly given that the first paragraphs of the article treat Wolff’s notions of definition favorably.\footnote{“La \emph{définition} de nom est celle qui explique le sens ou la signification propre d’un mot ; ou, comme le dit plus exactement M. \emph{Wolf}, c’est l’énumération qu’on fait d’un certain nombre de marques ou de caracteres suffisans pour faire distinguer la chose qu’on définit, d’avec toute autre ; de sorte qu’il ne reste point de doute sur ce que c’est que la chose qu’on a voulu faire entendre \& désigner par le nom. . . . M. \emph{Wolf} dit que la \emph{définition} de chose est une notion distincte qui explique la génération de cette chose, c’est-à-dire la maniere dont elle est faite ou dont elle se fait. . . . La notion que nous avons donnée de la \emph{définition} de chose, d’après plusieurs philosophes, suffit pour faire connoître en quoi elle differe de la \emph{définition} de nom. Mais quoique cette notion ait de son côté l’avantage de l’analogie, de la clarté \& de la convenance, cependant comme elle n’est elle-même qu’une \emph{définition} de nom, c’est-à-dire une \emph{définition} du mot, c’est sous ce point de vûe principalement que nous devons la considérer, en la regardant comme une idée attachée arbitrairement à ce mot, \& que l’auteur doit toûjours y conserver attachée dans toute la suite de son ouvrage. Mais cette notion ne renferme point en effet le sens ou la signification ordinaire qu’on a coûtume de donner à ce mot, \& qui est beaucoup moins juste \& moins distincte ; \& c’est à cette signification ordinaire que nous devons principalement avoir égard.” (\citealt{Diderot1751}: s.v. définition en logique)}  However, these paragraphs were taken in translation from Chambers’s article on \emph{definition}, who was impressed by Wolff’s notions of \emph{definition} as Wolff expressed them in his \emph{mathematical} texts rather than his \emph{philosophical} works.\footnote{“\emph{Definition of the Name}, or \emph{Nominal Definition}, is that which explains the sense, or signification appropriated to a word: or, as Wolfius more accurately considers it, an enumeration of certain \textbf{marks}, or characters, sufficient to distinguish the thing \emph{defined} from any other thing; so to leave it out of doubt, what the subject is that is intended or denoted by the name. . . . Wolfius \emph{defines} a \emph{real Definition} to be a distinct notion explaining the genesis of a thing, that is, the manner wherein the thing is made, or done: . . . This notion of a \emph{Real Definition} is very strict and just; and affords a sufficient distinction between a \emph{real} and a \emph{nominal} one. But, though it has the Advantages of analogy, distinctness, and conveniency on its side; yet being only itself a \emph{Nominal Definition}, i.e., a \emph{Definition} of the term \emph{Real Definition}, we must consider it in that light, that is, as an idea fixed arbitrarily to that word, and which the author always denotes by that word in the course of his book. But, in effect, it is not the usual sense, or acceptation of the term; which is much less convenient and distinct. And it is to that usual acceptation we are here chiefly to have regard.” (\citealt{Chambers1741}:  s.v. definition)}  In particular, Chambers applauded Wolff’s notion of genitive definition, which Wolff terms \emph{real definition}. Chambers’s applause for Wolff’s notion of genitive definition is echoed in the \emph{Encyclopédie} article on \emph{définition en logique}. There the Wolffian term \emph{mark} was taken from Chambers. However, non-mathematical Wolffian terms~-- \emph{possibility}, \emph{impossibility}, \emph{contradiction}, \emph{existence}~-- do not appear in Chambers’s article and they appear in the corresponding \emph{Encyclopédie} article only in Condillac’s criticism of Wolff’s philosophy.\footnote{For Chambers’s use of Wolff’s work in his \emph{Cyclopædia} article on \emph{definition}, see \citet{Bocast2016}.}

Thus, when Formey \emph{did} write on \emph{definition} in other places, he consistently used Wolffian terms to invoke certain Wolffian concepts. These terms are absent from the \emph{Encyclopédie} article on \emph{définition en logique}.

\subsubsection{Condillac’s critique of Wolff’s philosophy}

And now I can lay out a central argument for \emph{disattribution} in this case. Recall those Wolffian terms, \emph{possibility}, \emph{impossibility}, \emph{contradiction}, and \emph{existence}. Recall as well Condillac’s words in paragraphs 16 through 18, which quite plainly \emph{attack} the Wolffian notions signified by those terms. Suddenly, in the flow of the article, seemingly without provocation,\footnote{And (1) that Formey’s article contained those Wolffian terms; (2) that these Wolffian terms offended Condillac’s sensibilities, prompting him to include just these three paragraphs from his \emph{Essai}; and (3) that Condillac simply tossed Formey’s manuscript into his fireplace, for winters are cold in Paris.}  Condillac lashes out: 

\begin{quote}
After having defined \emph{impossible} as that which implies contradiction, \emph{possible} as that which does not imply contradiction, and \emph{being} as that which can exist, these philosophers have not known how to give any other definition of \emph{existence} than the complement of possibility. But I ask whether this definition presents any actual idea, and whether we should not show it the ridicule that we have given to some of those of Aristotle. 

If the \emph{possible} is that which does not imply contradiction, \emph{possibility} is the non-implication of contradiction. \emph{Existence} is then the complement of the non-implication of contradiction. What a way to talk!\footnote{“§. 14. Il est donc bien important de ne pas réaliser nos abstractions. Pour éviter cet inconvénient, je ne connois qu’un moyen, c’est de savoir développer l’origine \& la génération de toutes nos notions abstraites. Mais ce moyen a été inconnu aux Philosophes, \& c’est envain qu’ils ont tâché d’y suppléer par des définitions. La cause de leur ignorance à cet égard, c’est le préjugé où ils ont toûjours été qu’il falloit commencer par les idées générales ; car lorsqu’on s’est défendu de commencer par les particuliéres, il n’est pas possible d’expliquer les plus abstraites qui en tirent leur origine. En voici un exemple.

Après avoir défini l’impossible par \emph{ce qui implique contradiction} ; le possible, par \emph{ce qui ne l’implique pas} ; \& l’être, par ce qui peut exister : on n’a pas sû donner d’autre définition de l’existence, sinon, qu’elle est \emph{le complément de la possibilité}.\footnotemark Mais je demande si cette définition présente quelque idée, \& si l’on ne seroit pas en droit de jetter sur elle le ridicule qu’on a donné à quelques-unes de celles d’Aristote. 

Si le possible est \emph{ce qui n’implique pas contradiction}, la possibilité est la \emph{non-implication de contradiction}. L’existence est donc \emph{le complément de la non-implication de contradiction}. Quel langage ! En observant mieux l’ordre naturel des idées, on auroit vû que la notion de la possibilité ne se forme que d’après celle de l’existence. 

Je pense qu’on n’adopte ces sortes de définitions, que parce que connoissant d’ailleurs la chose définie, on n’y regarde pas de si près. L’esprit qui est frappé de quelque clarté, la leur attribue, \& ne s’apperçoit point qu’elles sont inintelligibles.” \citep[234-236]{Condillac1746}.
} \footnotetext {Wolff gave his \emph{definition} of existence in German in his \emph{Deutsche Metaphysik} as “der Erfüllung der Möglichkeit” and in Latin in his \emph{Philosophia Prima, sive Ontologia} as “per complimentum possibilitatis”. See \textcite[8]{Wolff1722} and \citet[143]{Wolff1736}.

Condillac translated Wolff’s Latin into French as “le complément de la possibilité”. However, other possible translations of Wolff’s definition of \emph{existence} include “the fulfillment of possibility” and “the completion of possibility”; these alternate readings might be sensible.
} 
\end{quote}

Can it be imagined that Formey~-- known across Europe as an ardent promulgator of Wolff’s philosophy~-- would attach his own name~-- as author, as compiler, as translator, as editor, or in any other way~-- to an article that, at its core, rudely dismisses ideas fundamental to Wolffian philosophy and to the six volumes of his own \emph{La belle Wolffiene}? I cannot believe that he would do so. 

\subsection{Formey considered as editor}

Considering Formey as editor is trivial: Formey in Berlin had no more to do with the manuscripts of his erstwhile “Philosophical Dictionary” once they had been put into the hands of the editors of the \emph{Encyclopédie} in Paris \citep[145]{Moureau1987}. I suggest that Diderot handed Formey’s notes on \emph{definition} to Condillac soon after they arrived in Paris, perhaps in 1747 \citep[128-9]{Moureau1987}. Later, d’Alembert took up Condillac’s draft of \emph{définition en logique} when he wrote his own article on \emph{définition en mathématiques}. Thus, given this spatiotemporal scenario, Formey could \emph{not} have been the editor of the printed article, even could he have countenanced Condillac’s attack on Wolffian philosophy.

\subsection{Conclusions}

On these grounds, the attribution of the article on \emph{définition en logique} to Jean-Henri-Samuel Formey appears to have no merit. The evidence shows that he did not \emph{author} the article nor did he \emph{translate} it from Chambers’s English. The evidence gives no grounds to believe that Formey \emph{compiled} the contents of the article, and he \emph{certainly} did not \emph{edit} the article that was printed. 

I suspect that Formey \emph{did} provide an article on \emph{definition} to the editors of the \emph{Encyclopédie } as, without it, what could explain Condillac’s eruption but that this article disappeared in Paris long before it was lost in Berlin. Nonetheless, the article on \emph{définition en logique} as \emph{actually printed} can no longer sustain an attribution to Formey. 

Oddly, the editors of the \emph{Encyclopédie did} nevertheless place that attribution to Formey at the end of the printed article. This may have been part of a \emph{quid-pro-quo} leading to Diderot’s invitation to the Berlin Academy in 1751 \citep[136]{Moureau1987}. But perhaps there is another rationale for this attribution. Recall that neither Diderot nor d’Alembert had a high regard for Formey or for the Wolffian school of thought he professed \citep[141]{Kafker1989}; perhaps together they found this attribution a great joke~-- a great \emph{contradiction}~-- \emph{precisely} because the printed article so bluntly mocks Formey’s intellectual hero.

In the end, however, those who study the \emph{Encyclopédie} should \emph{disattribute} this article from Formey.

{\sloppy\printbibliography[heading=subbibliography,notkeyword=this]}

\end{document}
