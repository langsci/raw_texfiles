\documentclass[output=paper,colorlinks,citecolor=brown,arabicfont,chinesefont,booklanguage=french]{langscibook}
\ChapterDOI{10.5281/zenodo.15394487}
\author{Clarissa Stincone\affiliation{Université Sorbonne Nouvelle – Paris 3}}
\title[Richelet dans le Dictionnaire universel de Basnage 1701]
      {Évaluer la dette: L’étendue de la présence de Richelet dans le \textit{Dictionnaire universel} de Basnage (1701)}

\abstract{Although he does not quote Richelet in the preface to his edition of the \textit{Dictionnaire universel} and attributes exclusively to the Académie française the role of guide and guarantee of his lexicographical choices, Basnage refers to Richelet even more than to the Académie française. Basnage not only transcribes entries or senses, syntactic constructions and authorial quotations from the \emph{Dictionnaire françois} (1693), but also takes up its didactic and descriptive intentions. As both lexicographers are familiar, for different reasons, with the doubts of those learning a foreign language, they make similar lexicographical choices. 
The didactic intentions can be seen in the spelling chosen as well as in the author citations and grammatical elements provided in their dictionaries. Descriptive intentions are evident in the fact that Richelet and Basnage record all the linguistic usages of which they are aware and at most suggest their preference by relying on the authority of writers and linguists who have used or spoken in favour of a specific usage. 
The aim of this article is to assess the extent to which Basnage drew inspiration from Richelet by analysing the common features of the two lexicographers' dictionaries, while seeking to understand the reasons why Basnage neglected to refer to Richelet.}

\IfFileExists{../localcommands.tex}{
  \addbibresource{../localbibliography.bib}
  \usepackage{langsci-optional}
\usepackage{langsci-gb4e}
\usepackage{langsci-lgr}

\usepackage{listings}
\lstset{basicstyle=\ttfamily,tabsize=2,breaklines=true}

%added by author
% \usepackage{tipa}
\usepackage{multirow}
\graphicspath{{figures/}}
\usepackage{langsci-branding}

  
\newcommand{\sent}{\enumsentence}
\newcommand{\sents}{\eenumsentence}
\let\citeasnoun\citet

\renewcommand{\lsCoverTitleFont}[1]{\sffamily\addfontfeatures{Scale=MatchUppercase}\fontsize{44pt}{16mm}\selectfont #1}
  
  %% hyphenation points for line breaks
%% Normally, automatic hyphenation in LaTeX is very good
%% If a word is mis-hyphenated, add it to this file
%%
%% add information to TeX file before \begin{document} with:
%% %% hyphenation points for line breaks
%% Normally, automatic hyphenation in LaTeX is very good
%% If a word is mis-hyphenated, add it to this file
%%
%% add information to TeX file before \begin{document} with:
%% %% hyphenation points for line breaks
%% Normally, automatic hyphenation in LaTeX is very good
%% If a word is mis-hyphenated, add it to this file
%%
%% add information to TeX file before \begin{document} with:
%% \include{localhyphenation}
\hyphenation{
affri-ca-te
affri-ca-tes
an-no-tated
com-ple-ments
com-po-si-tio-na-li-ty
non-com-po-si-tio-na-li-ty
Gon-zá-lez
out-side
Ri-chárd
se-man-tics
STREU-SLE
Tie-de-mann
}
\hyphenation{
affri-ca-te
affri-ca-tes
an-no-tated
com-ple-ments
com-po-si-tio-na-li-ty
non-com-po-si-tio-na-li-ty
Gon-zá-lez
out-side
Ri-chárd
se-man-tics
STREU-SLE
Tie-de-mann
}
\hyphenation{
affri-ca-te
affri-ca-tes
an-no-tated
com-ple-ments
com-po-si-tio-na-li-ty
non-com-po-si-tio-na-li-ty
Gon-zá-lez
out-side
Ri-chárd
se-man-tics
STREU-SLE
Tie-de-mann
}
  \togglepaper[10]%%chapternumber
}{}

\begin{document} 
\begin{otherlanguage}{french}
\maketitle

\section{Introduction}

Henri Basnage de Beauval révise en 1701 le Dictionnaire universel d’Antoine \citet{Furetière1690}, dorénavant [DU1690]. Le travail lexicographique de Basnage évolue par rapport à celui de Furetière sous plusieurs égards. En plus d’augmenter le nombre d’entrées et d’enrichir les entrées existantes de précisions au niveau du contenu, Basnage introduit dans son édition nombre d’informations lexicographiques que Furetière n’avait pas prises en compte. Il s’agit de citations d’écrivains, de marques d’usage contenant les jugements du lexicographe sur la fréquence d’utilisation des mots et les axes de la variation linguistique, d’éléments grammaticaux comme les flexions verbales et les constructions syntaxiques, d’indications concernant l’orthographe et la prononciation, de discussions des experts en matière de langage, de collocations. L’analyse de ces éléments permet de découvrir les sources de Basnage, parmi lesquelles le Dictionnaire de l’Académie française (\citealt{AcadFr1694}) et le \emph{Dictionnaire françois} \citep{Richelet1693}, (première édition : \citealt{Richelet1680}) de César-Pierre Richelet occupent une place charnière. 

Bien que Basnage s'inspire fortement des deux dictionnaires, il leur réserve un traitement bien différent. Le rôle de l'Académie française est reconnu dès la préface où Basnage affirme que « dans le partage des sentimens, l’Académie Françoise est l’autorité suprême à laquelle j’ai toûjours deferé » (Préface, \citealt{Furetiere1701}) dorénavant [DU1701]. Il est confirmé dans la suite de l'ouvrage où les citations de l'Académie sous les formes « Ac. », « L'Acad. » et « Mrs. de l'Acad. » sont au nombre de 630 environ. En revanche, le nom de Richelet n'apparaît pas dans la préface tout comme il n'apparaît pas dans la liste de 130 abréviations d'auteurs et d'œuvres au début du dictionnaire. Dans le corps de l'ouvrage, Richelet n'est mentionné que 66 fois, dont deux sous la forme abrégée « Rich. » et souvent de manière critique. Il n'est pas rare que Basnage associe au nom de Richelet le verbe \textit{pretendre} en invitant à préférer la variante orthographique proposée par l’Académie française~: 

\begin{quote}
    «~Richelet pretend qu’il faut prononcer, \& même écrire \emph{extrordinaire}~; mais l’Academie dit positivement qu’il faut prononcer \emph{extraordinaire}~; \& par con\-se\-quent, c’est ainsi que les honnêtes gens prononceront toûjours, au moins dans les discours graves~» (EXTRAORDINAIRE, DU1701).\footnote{Les mots en majuscules entre parenthèses indiquent l’entrée où se trouve la partie de texte citée. S'ils sont suivis de «~DU1690~», cela signifie que la portion de texte ne se trouve que dans le \emph{Dictionnaire universel} de Furetière et non dans celui de Basnage. S'ils sont suivis de «~DU1701~», cela signifie, au contraire, que la portion de texte ne se trouve que dans le \emph{Dictionnaire universel} de Basnage et non dans celui de Furetière. S'ils ne sont suivis de rien, cela signifie que le texte se trouve dans les deux éditions. Dans ce cas, l'orthographe adoptée est celle de Basnage.} 
\end{quote}

Même s’il ne le reconnaît pas, Basnage emprunte à Richelet plusieurs éléments lexicographiques, voire des acceptions et des entrées entières, ainsi que deux grandes orientations de son travail, à savoir l’orientation didactique et l’orientation sociolinguistique\footnote{Pour plus de détails sur les orientations didactique et sociolinguistique du \emph{Dictionnaire françois} (1680) de Richelet cf. \citealt[7]{Quemada2004}.} . Comme Richelet, Basnage s’efforce de répondre aux besoins d'apprentissage de son public. Richelet est un professeur de français\footnote{Richelet « a donné des cours particuliers de français à des étrangers distingués, principalement allemands, anglais ou scandinaves, qui séjournaient à Paris et avec lesquels il s’est lié d’amitié~» \citep[2]{Quemada2004}.}  et s'adresse, entre autres, aux étrangers désireux d'apprendre sa langue, alors que Basnage est un émigré aux Pays-Bas, et s'adresse, parmi d’autres, à la communauté française exilée qui ne peut pas parler le français dans la vie quotidienne. Pour des raisons différentes, les deux lexicographes sont familiers avec les doutes de ceux qui (ré)apprennent une langue. Il n'est donc pas surprenant qu'ils fassent des choix lexicographiques similaires. Les intentions pédagogiques ressortent à partir de l’orthographe. Tous deux suppriment les lettres qui ne se prononcent pas, en veillant à ne pas \emph{défigurer} les mots, c'est-à-dire à en préserver l'étymologie. En ce sens, la position de Basnage est très forte car elle implique la réécriture de l'ensemble du dictionnaire de Furetière, dont l'orthographe est, par ailleurs, quasiment la même que celle des académiciens. Quant aux citations d'auteurs, seuls Richelet et Basnage les utilisent, parmi les grands lexicographes de l'époque classique ; si d'une part elles garantissent l’exhaustivité du travail lexicographique, d'autre part elles constituent un support valable pour les apprenants car elles offrent des exemples d'utilisation ainsi que des modèles linguistiques à imiter. Toujours dans le but d'aider les apprenants, les deux lexicographes enregistrent un grand nombre de constructions syntaxiques, de flexions verbales, des collocations et de règles grammaticales. L'approche descriptive est un autre trait commun aux deux lexicographes : ni Richelet ni Basnage ne s'érigent en prescripteurs de certaines formes ou constructions. Plutôt que de proposer uniquement leur forme préférée, ils enregistrent tous les usages linguistiques dont ils ont connaissance et suggèrent tout au plus leur préférence en s'appuyant sur l'autorité d'écrivains et des experts en langues qui ont utilisé ou se sont prononcés en faveur d'un usage spécifique. Tout en prêtant attention aux éléments qui facilitent l'apprentissage de la langue, les deux lexicographes cherchent donc à répondre à leur besoin d'exhaustivité dans la description linguistique.

\largerpage
Dans ce qui suit, après avoir montré les analogies entre le \emph{Dictionnaire françois} de \citealt{Richelet1693} et le \emph{Dictionnaire universel} de Basnage \cite{[DU1701]}, notamment en ce qui concerne l’orthographe, les renseignements sur la prononciation, les citations, les marques de domaine et d’usage, le traitement des mots grammaticaux, les flexions verbales, les constructions syntaxiques et les collocations, nous tenterons de trouver les raisons de la dissimulation de la source.

\section{Orthographe}

En feuilletant le \emph{Dictionnaire universel} de Furetière et celui de Basnage, la première chose qui saute aux yeux est que l'orthographe adoptée par les deux lexicographes est différente. Furetière était un académicien, dont l'avis a certainement influé sur les choix orthographiques du \emph{Dictionnaire de l'Académie française} (1694)\footnote{Les choix orthographiques du Dictionnaire de l'Académie sont décrits dans les Cahiers de Mézeray. Bien que l'Académie déclare dans la préface du dictionnaire qu'elle souhaite maintenir l'ancienne orthographe « par tout », les Cahiers contiennent « en partie, une condamnation de l’orthographe traditionnelle » \citep[186]{Catach2001}.} . Par conséquent, il n'est pas étonnant que l’orthographe adoptée par Furetière est \emph{grosso modo} la même que celle des académiciens. Basnage est le rédacteur en chef d'un journal ouvert sur le monde et l'actualité, qui donne la parole aux contemporains et qui défend souvent des positions novatrices, y compris en matière d’orthographe. Basnage affirme s’être tenu à l’orthographe « de Furetière, qui est aussi celle de Mrs. de l’Academie françoise » (Préface, DU1701). Il fait ainsi référence à la macrostructure du dictionnaire qui présente « les mots dans le même ordre, \& la même disposition » (ibid.) que le \emph{Dictionnaire universel} de Furetière\footnote{Il s'agit d'une décision de convenance : changer l'orthographe des mots vedettes aurait entraîné une perturbation de leur agencement macrostructurel.}. Néanmoins, il reconnaît avoir supprimé les\textit{s} « en beaucoup de mots où elle ne se prononce point » (ibid.). Il fait ainsi référence à la microstructure qui présente une orthographe assez différente par rapport à celle de son prédécesseur. Bien que le seul changement envisagé dans la préface soit la suppression de \textit{s}, Basnage applique trois types de modifications à l'orthographe de la première édition du dictionnaire : la suppression, le remplacement et, plus rarement, l'ajout de caractères. Les changements concernent non seulement les lettres, mais aussi les signes diacritiques. Lorsque Basnage commence la révision du dictionnaire, la discussion sur l'orthographe en France est ouverte. 

Conscient des débats en cours, Basnage se déclare attentif à préserver l'étymologie des mots qu'il craint de défigurer ou déguiser. Furetière, après la définition du lemme ORTHOGRAPHE, précise que ceux qui ont essayé, à la suite de « Jacques Pelletier du Mans, qui soutint qu’il falloit écrire comme on parle » (ORTHOGRAPHE, DU1690), de modifier l'orthographe du français ont été mal reçus. Il s’agit de « Louïs Maigret, Pierre la Ramée dit Ramus, Jean Antoine de Baïf, et de notre temps l'Esclache » (ibid.) . Basnage efface la référence à la réception des avis en matière de changement d’orthographe ainsi que celle au contemporain l’Esclache\footnote{Dans « les systèmes extrêmes de phonéticiens » \citep[120]{Catach2001}, chaque son correspond à un et un seul signe.}. Louis de L’Esclache est un professeur de philosophie qui dénonce l'exclusion des femmes de la société et demande que la langue française soit émancipée du joug de l'étymologie . Le titre de son traité, \textit{Les Véritables régles de l’ortographe Franceze, ou l’art d’aprandre an peu de tams à écrire côrectemant}, donne déjà une idée de certaines des positions qu'il défend comme le remplacement des \textit{en} et \textit{em} par \textit{an} et \textit{am}. Basnage, se prononce en faveur d'une orthographe qui, bien que novatrice, respecte l'étymologie des mots. C'est précisément au titre du traité de L’Esclache qu’il semble se référer lorsqu'il affirme que les auteurs qui écrivent « tans » ou « ortographe » semblent défigurer les mots « temps » et « orthographe » en empêchant de reconnaître leurs origines latine et grecque respectivement. Il n'est donc pas étonnant que Basnage ait supprimé la mention de cet auteur. Basnage écrit que « la chose est encore en contestation, \& on n'a pu convenir d'une manière uniforme d'orthographe » (ORTHOGRAPHE, DU1701). Il est conscient de l'existence d'au moins trois positions : celle, dépassée, de ceux qui entendent écrire comme on parle ; celle qui reste ancrée dans la tradition, comme celle des académiciens, qui « ont retenu l'ancienne orthographe, \& ont gardé les lettres mêmes qui ne se prononcent point » (ibid.), et enfin celle qui adopte « un tempérament entre l'une, \& l'autre orthographe, en retranchant les lettres inutiles, comme le s, dans une infinité de mots, \& en conservant pourtant dans les mots l'orthographe qui marque leur origine » (ibid.). De caractère conciliant, après avoir affirmé qu' « on peut donc choisir entre les manières d'orthographe » (ibid.), Basnage déclare préférer la dernière position puisque « le plus sûr est de ne point trop s’éloigner de l’ancien usage pour ne pas trop changer les mots aux yeux, en les voulant accomoder à l’oreille » (ibid.). Parmi ceux qui ont adopté une position intermédiaire entre l’ancienne et la nouvelle orthographe, il y a Richelet, que Basnage ne mentionne pas dans l’entrée ORTHOGRAPHE. En décrivant l’orthographe adoptée dans son dictionnaire, Richelet explique sa position dès la première édition de son \emph{Dictionnaire françois} (\citealt{Richelet1680}) :

\begin{quote}
    Touchant l'Ortographe, on a gardé un milieu entre l'ancienne, \& celle qui est tout à fait moderne, \& qui défigure la Langue. On a seulement retranché de plusieurs mots les lettres qui ne rendent pas les mots méconnoissables quand elles en sont otées, \& qui ne se prononçant point, embarassent les Etrangers, \& la plu-part des Provinciaux.~» (\citealt{Richelet1680}:  Avertissement, \emph{Dictionnaire François})
\end{quote}

La position de Basnage est près de celle de Richelet qui propose d'éliminer les lettres qui ne se prononcent pas, tout en faisant attention à ne pas \emph{défigurer} la langue. Le souci de Richelet pour son public est explicite puisqu'il affirme que les lettres qui ne sont pas prononcées «~embarassent les Etrangers~» (Avertissement, \emph{Dictionnaire françois}, 1680). Bien que Basnage ne mentionne pas explicitement ses lecteurs, il est possible de discerner dans l'adoption d'une orthographe simplifiée la volonté de leur rendre service. A différence de Richelet, Basnage accorde plus d'attention à la préservation des étymologies. Par exemple, là où Richelet écrit \emph{ortographe}, Basnage invite à conserver le \emph{h} car cette lettre est la marque de la dérivation grecque du mot. 

Au-delà des affirmations des préfaces, outre la suppression des \emph{s} pré-consonnes Richelet et Basnage partagent au moins deux autres positions orthographiques dans les corps de leurs ouvrages. La première est la syncope de t des pluriels des adjectifs, voire substantifs, en \emph{-ant} et \emph{-ent} qui dérivent du participe présent des verbes latins, ainsi que des substantifs composés du suffixe \textit{-ment}. Alors que Furetière garde la dentale entre nasale et sibilante (ex. : \emph{absents, compliments, correspondants})\footnote{Furetière n'est pas toujours irréprochable dans cette position. Par exemple, le DU1690 enregistre 88 occurrences d' « amants » et 37 d' « amans ».} , Richelet et Basnage l'effacent systématiquement (ex. : \emph{absens, complimens, correspondans}). La deuxième position que Richelet et Basnage partagent concerne la préférence d’\emph{i}- à \emph{y} à la fin des mots que Furetière orthographie avec un \emph{y}, comme les pronoms personnels complément et leurs dérivés (ex. : \emph{moi, toi, celui} au lieu de \emph{moy, toy, celuy}). Basnage déclare qu’« il n’y a point de raison de preferer l'\emph{y} Grec à un \emph{i} simple » (Y, DU1701) au moins à la fin des mots où « ce n’est que pour observer la vieille orthographe qu’on écrit, \emph{Roy, Loy, \& }[...] \emph{j’ay, j’allay} » (ibid.). Cette position est essentiellement celle de Richelet qu’« à l’imitation de l’illustre Monsieur d’Ablancourt, \emph{Preface de Thucidide, Apophtegmes des Anciens, Marmol,} etc. et de quelques Auteurs celébres, [...] change presque toujours l’y grec en i simple » (\citealt{Richelet1680}:  \emph{Avertissement}).

\section{Prononciation}

L'indication de la prononciation des mots augmente fortement dans le \textit{Dictionnaire universel} de Basnage 1701 par rapport à l’édition précédente. Furetière enregistre la prononciation d’environ 50 mots (cf. \citealt{Rey2020})  tandis que Basnage d'environ 800, tout comme Richelet dont il s’inspire souvent.

La plupart des prononciations enregistrées par Furetière concernent des mots contenant un groupe de consonnes commençant par la lettre \emph{s}. Comme il adopte l'ancienne orthographe qui enregistre les lettres d'origine latine, et notamment \emph{s}, même lorsqu'elles ne sont pas prononcées, Furetière sent parfois le besoin de fournir des indications sur les endroits où cette lettre est ou n'est pas prononcée. Cependant, le lexicographe enregistre également des informations sur d'autres lettres, notamment \emph{c} et \emph{d}. En général, Basnage maintient les informations fournies par Furetière en en introduisant de nombreuses d’autres. Les formulations de Furetière sont impersonnelles ou ont pour sujet la lettre en question tandis que celles de Basnage s'adressent souvent au lecteur par le biais de la deuxième personne du pluriel de l'impératif du verbe \emph{prononcer}. Basnage indique la prononciation de différentes manières, parmi lesquelles on peut reconnaître trois schémas qui se combinent parfois entre eux~:

\begin{itemize}
    \item L'invitation à lire ou à ne pas lire une des lettres contenues dans le mot-vedette, par exemple~: \emph{CHRIST [...] Prononcez I's}. 
    \item L'indication de la prononciation du mot entier, par exemple~: \emph{AIGUADE [...] Prononcez EGADE}. 
    \item L'indication de la quantité syllabique d’une syllabe du mot, par exemple~: \emph{PATE [...] La premiere syllabe de ce mot est breve, \& c'est pour cela qu'il est mieux d'écrire avec l'Academie, Patte}.
\end{itemize}

Les trois exemples donnés ci-dessus sont tirés de Richelet, y compris le dernier dans lequel l'Académie française est mentionnée pour confirmer ce qui vient de Richelet qui est occulté par Basnage. Bien qu'il s'inspire souvent de Richelet, Basnage modifie fréquemment les formulations introduisant les renseignements sur la prononciation. Alors que les formulations de Richelet sont presque toujours introduites par «~Prononcez~»\footnote{Richelet utilise «~Prononcez~» dans 636 entrées, «~se prononce~» dans 82 entrées, «~on prononce~» dans 93 entrées.},  l’alternance de formulations personnelles et impersonnelles montre chez Basnage\footnote{Basnage utilise «~Prononcez~» dans 333 entrées, «~se prononce~» dans 258 entrées, «~on prononce~» dans 108 entrées.} une plus forte tension entre la volonté d'offrir un outil qui guide le lecteur au moyen des prescriptions et une autre volonté, tout aussi forte, d'offrir une image objective de la langue telle qu'elle émerge des pratiques des locuteurs. 

\section{Mots grammaticaux}

Bien que Furetière prenne en compte le domaine de la grammaire\footnote{Furetière utilise 84 fois la marque de domaine «~Terme de Grammaire~». Il décrit également 86 mots plurisémiques qui trouvent une application «~en termes de Grammaire~» (37) ou «~en Grammaire~» (49).}, ses définitions des mots grammaticaux sont concises et fournissent peu d'exemples illustratifs. Basnage, pour sa part, au-delà de l'augmentation quantitative des articles consacrés à la grammaire - 206 au total\footnote{Basnage utilise 114 fois la marque de domaine «~Terme de Grammaire~». Il décrit également 92 mots plurisémiques qui s’appliquent «~en termes de Grammaire~» (41) et «~en Grammaire~» (51).} contre les 170 de Furetière -, accorde une large place aux règles grammaticales, notamment dans les entrées des mots grammaticaux. Basnage est conscient, comme il le dit dans la préface de son édition, du fait que «~les minuties de Grammaire dont ce Dictionnaire est rempli, ne sont pas fort essentielles pour parler, quand on ne parle que pour se faire entendre~» (Préface, \emph{DU1701}). Par contre, elles sont précieuses pour ceux qui cherchent à parler «~exactement, poliment, \& noblement~» (ibid.). Les règles grammaticales introduites par Basnage dans le \emph{Dictionnaire universel} de Furetière proviennent souvent du \emph{Dictionnaire françois} de Richelet ainsi que de grammaires et recueils de remarques sur la langue française. Par exemple, dans l’entrée JE, fortement augmentée, Basnage explicite par le moyen de règles et d’exemples les informations que Furetière avait données sous forme implicite, et les complète par d'autres informations de caractère phonétique et morphologique. Il introduit des remarques de \citet[115]{Menage1672} et de \citet{Richelet1693} (JE), reprises à leur tour da la \emph{Grammaire générale et raisonnée} de \citet[35]{ArnaudLancelot1660}. Basnage insère des renvois extratextuels à «~Ménage, \& la Grammaire raisonnée~» (JE, \emph{DU1701}), qu'il invite à consulter mais pas à Richelet, dont il occulte encore une fois la consultation.

\section{Citations d’écrivains}

Les citations d’écrivains sont marginales dans le dictionnaire de Furetière et «~y répondent avant tout à des intentions littéraires~» \citep[49]{Lehmann1995}. Au contraire, dans le dictionnaire de Basnage, elles jouent un rôle de premier plan, puisqu'elles figurent dans toutes les entrées pour lesquelles le lexicographe a pu en trouver.\footnote{La plupart des entrées terminologiques manquent de citations car les termes sont rarement utilisés par les écrivains.}

Les citations de Basnage ne sont pas toutes de première main. Le lexicographe transcrit parfois les citations que d’autres lexicographes avaient incluses dans leurs ditionnaires. Richelet est la source majeure des citations de Basnage. A l'exception de quelques-uns, tous les auteurs cités dans la «~table alphabetique de la plupart des Auteurs et des Livres citez dans ce dictionnaire~» au début du \emph{Dictionnaire françois} de \citet{Richelet1693} sont présents dans le \emph{Dictionnaire universel} de Basnage. Toutefois, seuls certains d'entre eux – cinquante-un plus précisément - figurent dans la liste des abréviations de Basnage. Cela ne s'explique que partiellement par le fait que Basnage inclut dans la liste les seuls auteurs qu'il cite au moyen d'une abréviation tandis qu’il ne mentionne pas les auteurs dont il donne le nom complet. En fait, bien que beaucoup d’auteurs de la table alphabétique de Richelet soient cités par Basnage au moyen d'une ou plusieurs abréviations, ils ne figurent pas dans sa liste des abréviations. Probablement, il s'agit d'un oubli. Au moment de la compilation de la table des abréviations, le manque de recherches personnelles empêche Basnage de se souvenir de tous les auteurs et de toutes les œuvres pour lesquels il s'est appuyé sur les recherches d’autres. Alors que Richelet indique souvent ses sources de manière très précise, fournissant en plus du nom de l'auteur également le titre de l'œuvre et la page où la citation se trouve, Basnage est beaucoup plus hâtif et, en même temps, systématique, en ne fournissant que l'abréviation du nom de l'auteur. Néanmoins, bien que «~la dette de Basnage envers Richelet [soit] grande, [...] il a lu tous ces auteurs ou presque (une exception semble Voiture)~» \citep[86]{Fennis1988}. Le fait que les citations identifiées par Basnage et ses collaborateurs s'additionnent à celles de Richelet et de ses collaborateurs explique le nombre très élevé de citations de d'Ablancourt et de Boileau\footnote{Richelet le cite comme «~Dépreaux~».} ainsi que de Saint-Évremond\footnote{Charles de Marguetel de Saint-Denis, seigneur de Saint Evremond (1616--1703), était un correspondant de Basnage et deux de ses lettres de Londres ont été même publiées dans le HOS en décembre 1698 et mars 1699 (cf. \citealt[743]{Graveleau2018}).} et de ses \emph{Œuvres mêlées}. Il est très probable que, s'inspirant de la méthode de travail de Richelet qui se faisait aider par des collaborateurs dans la recherche de citations, \footnote{Cf. \citealt[543]{Quemada1967}.} Basnage se soit fait envoyer les citations par les mêmes écrivains qu’il cite.

L’entrée consacrée au pronom relatif \emph{que}, récrite à partir du \emph{Dictionnaire françois} (1693) de Richelet, montre que Basnage fait aveuglément confiance aux informations de sa source qu’il ne vérifie pas toujours~:

\begin{quote}
    QUE. Pronom relatif \& indeclinable, qui perd son e devant une voyelle, \& qui s’employe diversement. Par exemple, c’est un mechant que le Ciel châtie. Arn. \emph{Que} est mis là pour \emph{lequel}. Ce n’est pas à toi que je voudrois deguiser la verité. Ablan. \emph{Que} est mis là pour \emph{auquel}, ou à qui. Les mechans sont semblables à ces petites pailles que le vent emporte. Port-R. \emph{Que} est mis là pour \emph{lesquelles}. 
\end{quote}

Après avoir remplacé la définition de Furetière\footnote{«~Pronom relatif \& interrogatif~» (QUE, DU1690).} avec celle de Richelet qu’il ne cite pas, Basnage transcrit même les citations du \emph{Dictionnaire françois} (1693). En général, Basnage utilise l’abréviation «~Port. R.~» pour se référer au traité de Claude Lancelot, \emph{Nouvelle Méthode pour apprendre facilement la langue latine} \citep{Lancelot1662}, mais dans le cas en espèce l’abréviation désigne ladite «~Bible de Port Royal~» ou «~Bible de Sacy~»\footnote{Louis-Isaac Lemaistre De Sacy est le principal auteur de l’ensemble des traductions de la Bible en français, publiées entre 1665 et 1696 par les imprimeurs de Port-Royal.}, d’où proviennent les exemples du \emph{Dictionnaire françois} de \citet{Richelet1693}. En somme, Basnage reprend les citations de Richelet sans se demander si le collègue attribue la même valeur à l'abréviation que lui-même.

\section{Marques de domaines et marques d’usage}

Les marques de domaine et d'usage\footnote{Par ‘marque d'usage’, nous entendons exclusivement les indications de nature rhétorique et sociolinguistique placées à la fin de l'entrée, et non les déterminations de domaine qui se trouvent immédiatement après la catégorie grammaticale que nous appelons ‘marques de domaine’.} sont utilisées par Furetière plus largement que les citations\footnote{Cf. \citet{Rey1990}.}. Cependant, elles semblent être extrêmement peu nombreuses par rapport à la quantité de marques de domaine et d'usage enregistrées par Basnage. Il déclare que dans son ouvrage la distinction entre \emph{mots} et \emph{choses} «~ne subsiste plus~» (Préface, \textit{DU1701}). Si Furetière entendait décrire essentiellement les sciences et les arts et «~le reste n’étoit qu’accessoire~» (ibid.), dans son \emph{Dictionnaire universel} Basnage vise à «~apprendre à parler poliment, aussi bien qu’à parler juste, \& dans les termes propres à chaque Art~» (ibid.). Richelet est le premier lexicographe à avoir appelé le sien un «~dictionnaire des mots et des choses~» (\citealt{Quemada1967}:  85).

Les marques de domaine et les marques d'usage permettent aux lexicographes de signaler les mots qui s’écartent du \emph{bel usage} de la langue française. Elles sont appelées \emph{nottes} par les lexicographes de l'âge classique. Comme on lit dans les deux éditions du \emph{Dictionnaire universel}, «~dans un Dictionnaire on doit mettre une \emph{notte} à un mot, quand il est vieux, ou particulier à quelque art ou science. Quand il est dans l'usage commun, il n'y faut point de \emph{notte}~» (NOTTE). Cet exemple décrit la fonction des marques de domaine et d’usage~: les premières, placées après les mots-vedettes, détectent les langues spécialisées en indiquant le domaine auquel appartient l'entrée, et les deuxièmes, placées en fin d'entrée, renvoient aux dimensions de la variation linguistique et à la fréquence d’usage des mots. Basnage augmente considérablement les unes et les autres. Le domaine est introduit par diverses formulations, dont les plus courantes sont «~(en) terme(s) de [nom du domaine]~». Quant aux marques d’usage, bien que l'entrée NOTTE ne mentionne que des archaïsmes, le \emph{Dictionnaire universel} de Basnage décrit un spectre assez large de la variation linguistique. Le lexicographe ne néglige la description d'aucune dimension du langage, à savoir diachronique, diatopique, diastratique, diaphasique et diamésique (cf. \citet{Coseriu1977}. La description de ces différentes dimensions n'est pas systématique et uniforme. Les marques d'usage sont introduites par «~ce mot~» dans un certain nombre d'entrées, par le pronom personnel «~il~» dans d'autres, par des périphrases impersonnelles telles que «~on dit~» ou «~on s'en sert~» dans d'autres encore. Parfois, elles se réfèrent à tous les sens du mot, parfois au dernier sens seulement. Dans certains cas, Basnage le précise au moyen du complément de limitation «~en ce sens~», dans d'autres il ne le fait pas.

Furetière tente de regrouper les entrées sous un nombre plus restreint de domaines par rapport à Basnage, en offrant un texte plutôt systématique. De son côté, Basnage ne se soucie pas de préserver la cohérence du texte de Furetière et multiplie les marques de domaine en introduisant de nombreuses désignations de catégories de personnes auxquelles il n’attribue souvent qu’un seul terme. Elles sont généralement reprises du \emph{Dictionnaire françois} de \citep{Richelet1693}. Si chez Furetière, qui toutefois enregistre plus de 250 métiers et activités, il y a un «~début de systématisation~» \citet[306]{Quemada1967}, Basnage le brise car son \emph{Dictionnaire universel} contient une quantité bien plus imposante de marques de domaine provenant tout droit du \emph{Dictionnaire françois} \citet{Richelet1693}. Cela implique que, bien que moins systématiques, Richelet et Basnage sont plus précis dans la représentation des catégories de travailleurs et de travailleuses. Par exemple, lorsque Furetière ignore les désignations féminines, Richelet et Basnage les introduisent souvent. Dans la société décrite par les deux lexicographes les femmes travaillent, produisent et vendent, comme en témoignent les termes de \emph{blanchisseuses, bouquetiere, coiffeuse, couturiere, couturiere en linge, faiseuse de point, faiseuse de point \& de dentelle, fileuse, fileuse au roüet, vendeuse de sel}. En outre, les domaines de Richelet et Basnage contiennent souvent une référence à Paris à laquelle sont associés de termes liés aux métiers ou aux bâtiments de Paris tels que \emph{batelier de Paris, chaudronnier de Paris, college de Paris, executeur de Paris, hôpital general de Paris, libraires de Paris, marchand de bois de Paris, Nôtre Dame de Paris, oiselier de Paris, rotisseur de Paris, seminariste de Paris, theologien de Paris, Université de Paris}. Basnage, d'origine normande, exilé aux Provinces-Unies, a peu de liens avec Paris, et ne peut pas connaître le vocabulaire utilisé par les différentes catégories de travailleurs de la ville. Toutes les désignations renvoyant à Paris proviennent de Richelet qui, au contraire habite à Paris et «~pour préciser ses définitions, va descendre dans la rue, il va consulter les gens de métier~» (\citealt{Bray1990}:  49). L'absence de réflexion personnelle dans la sélection des domaines est confirmée par le fait que pour certaines de ces désignations, il n'existe aucune entrée dans le \emph{Dictionnaire universel} de Basnage.

En ce qui concerne les marques d’usage, Basnage puise le plus souvent dans Richelet, puis dans \citet{Pomey1664}, \citet{Menage1672}, \citet{La_touche1696} et \citet{Corneille1694}. Parfois il copie le texte de la source presque mot à mot, d'autres fois il le reformule. Dans quelques cas, Basnage aborde ses sources de manière critique. Par exemple, alors que Richelet affirme que «~Ce mot ne se dit point~» (INTOLERANCE, \citet{Richelet1693}, Basnage soutient qu'il est utilisé depuis quelques années et par une catégorie spécifique de personnes, les théologiens\footnote{Cf. INTOLERANCE, \emph{DU1701}.}. Parfois Basnage ramène dans la même entrée des informations rencontrées dans des sources différentes. Dans un cas, après avoir affirmé qu’aucun bon auteur ne s’est servi du mot, ce qui reprend les mots de \citet{La_touche1696} qui à son tour cite \citet{Boisregard1689}, Basnage écrit qu’«~à la verité Mazeray l’a employé~» (AFFLUER, DU1701) comme l’indique Richelet. De même, pour presque toutes les marques renvoyant aux styles burlesques, comiques et satyriques, Basnage s’appuie sur les mots de Richelet. Voici quelques exemples~:

\begin{quote}
    BARILLET. s. m. [...] Il est hors d'usage en ce sens. 
    
    LANTIPONAGE. s. m. Mot bas \& burlesque, [...]

    PARFAUTE [...] Ce mot est un peu vieux.

    RENGAINE [...] Mot tout-à-fait bas \& du petit peuple de Paris. 

\end{quote}

Toutes ces marques de fréquence et d’usage sont tirées de Richelet. Néanmoins, Richelet n’est pas la seule source des marques d’usage de Basnage qui ressent le besoin de fournir également les prescriptions de l'Académie française ainsi que d’autres lexicographes, grammairiens et experts en langue dans le but d’offrir un outil qui décrit la langue telle qu'elle est, dans toutes les situations et contextes de communication possibles. Parfois Basnage fait aussi confiance à sa propre conscience linguistique pour créer des marques d'usage. 

\section{Flexions verbales}

Alors que Furetière ne donne des conjugaisons verbales qu’implicitement à travers des exemples illustratifs, Basnage donne des conjugaisons verbales de manière explicite pour environ 240 verbes. Elles sont le plus souvent tirées du \emph{Dictionnaire françois} de \citet{Richelet1693}. Si l’on considère le coût du papier, il est très probable que les lexicographes aient sélectionné les seules formes irrégulières de verbes qui, à leurs avis, ne sont pas déductibles de l’infinitif lemmatisé. Même si fonctionnel, ce choix se traduit par une présentation chaotique des personnes, des temps et des modes verbaux. Alors que Richelet n’introduit pas les flexions verbales au moyen de formules, chez Basnage la conjugaison est précédée quelquefois par «~Conjuguez~» qui peut être abrégé en «~Conj.~», quelquefois par «~On conjugue~» qui peut être abrégé en «~On conj.~», quelquefois par «~Ce verbe se conjugue ainsi~». 

En général, Basnage tend à réduire le nombre de formes verbales qu'il trouve chez Richelet. Voir l'exemple où les formes du \emph{Dictionnaire françois} (1693) de Richelet qui ne se trouvent pas dans le \emph{Dictionnaire universel} de Basnage \citep{Furetiere1701} sont indiquées entre crochets.~:

\begin{quote}
    ABSTENIR v. n. […] Conjuguez Je m'\emph{abstiens}~; je m'\emph{abstenois}~; je m'\emph{abstins}~; [je me suis abstenu]~; je m'\emph{abstiendrai}~; [abstiens toi~; qu’il s’abstienne]~; je m’\emph{abstiendrois} [Je m’abstinsse, que je me suis abstenu] \&c.
\end{quote}

\section{Constructions syntaxiques}

Contrairement à Furetière, qui ignore quasiment les constructions syntaxiques, Basnage leur donne une large place. Elles sont quelquefois introduites par le verbe \emph{joindre}\footnote{Ex. «~Il se joint souvent avec le pronom relatif~» (ACCREDITER, \emph{DU1701})}, dont s’était sporadiquement déjà servi Furetière. Cependant, Basnage utilise plus fréquemment la forme «~se construit~» pour indiquer principalement des constructions verbales. Un autre introducteur est la forme «~regit~», utilisée dans la plupart de cas pour décrire le régime des conjonctions et prépositions qui sont lemmatisées par Basnage de manière beaucoup plus systématique que par Furetière. Exceptionnellement, le régime des autres parties du discours comme les adjectifs, les adverbes et les substantifs est donnée par Basnage. En général, en ce qui concerne les verbes, Basnage propose parfois des verbes auxiliaires, parfois la construction pronominale, parfois la préposition qui suit le verbe, parfois la diathèse. En ce qui concerne les prépositions, il indique le cas modulé sur la déclinaison latine (génitif, datif, accusatif, ablatif). Concernant les conjonctions, Basnage précise généralement le mode verbal qui les accompagne (indicatif, infinitif, subjonctif). Bien qu’il adopte une approche plus descriptive que son prédécesseur, Basnage indique aussi des constructions à éviter. Le régime des parties du discours est repris presque toujours du \emph{Dictionnaire françois} de \citet{Richelet1693} et, plus rarement, du \emph{Dictionnaire de l’Académie française} \citep{AcadFr1694}. Dans les entrées suivantes nous avons souligné les mots qui proviennent du \emph{Dictionnaire françois} de Richelet~:

\begin{quote}
    AFIN. Conjonction qui denote l’intention, \& signifie, Pour, à dessein. Elle \uline{est toujours suivie d’un de ou d’un que}. Quand elle \uline{est suivie d’un de, elle regit l’infinitif}

    AUPRÉS. Preposition qui marque un lieu proche, \uline{\& qui regit le genitif}.

    PASSIONNER [...] \uline{Le mot de \emph{passionner}} […] dans le regime actif \uline{est assez nouveau}.
\end{quote}

\section{Collocations}

Les lexicographes de l’âge classique sont très conscients du fait que l’usage de certains mots dépend de la présence d’autres et que le sens d’un mot peut varier en fonction des mots qui l’entourent. Furetière indique quelques collocations que Basnage garde en en précisant d’autres. En général, il informe le lecteur que l’utilisation d’un mot tout seul est sporadique, tout en assignant à des exemples et des citations la tâche de montrer ses collocations. Basnage reprend souvent les collocations de Richelet. Voir quelques exemples de collocations transcrites du \emph{Dictionnaire françois} de Richelet~: 

\begin{quote}
    BATTEUR. s. m. Ce mot ne se dit presque jamais seul. [...] Les batteurs de gens trouvent souvent de plus mechans qu'eux qui les battent. 

    EMBLÉE. s. f. [...] Ce mot ne se dit pas seul. [...] Ce Capitaine étoit heureux à prendre les villes d'emblée. La ville étoit trop bien munie pour l'emporter d'emblée. Vaug. 

    GAGNEUR. s. m. Ce mot [...] ne se dit pas seul, \& même il ne se dit guere. Ce gagneur de tant de batailles. Voit.
\end{quote}

\section{Conclusions}

Basnage puise dans le \emph{Dictionnaire françois} de \citet{Richelet1693} de nombreuses informations afin d'enrichir son \emph{Dictionnaire universel} (1701). En plus de transcrire des sens et des entrées entières que Furetière avait négligés, Basnage reprend des informations concernant la prononciation, la grammaire, les citations d’écrivains, les marques d'usage, les flexions verbales, les constructions syntaxiques, les collocations. Même l’orthographe simplifiée de Basnage ressemble celle de Richelet. Tous ces éléments sont inhérents à un outil conçu pour soutenir l'apprentissage du français. En même temps, ils permettent aux deux lexicographes d'offrir une description large et objective de la langue. Néanmoins, Basnage ne reconnaît pas sa dette envers le collègue. Il est plausible qu’afin de plaire aux membres de l'Académie française qui, comme le dit Furetière dans le premier \emph{Factum}, «~traitent de misérable~» \citep[9]{Asselineau1859}, Richelet, Basnage ne reconnaît pas le rôle de Richelet en évitant de le citer et en le dépréciant lorsqu'il le cite. Par ailleurs, le choix de l'édition de 1693 n'est pas fortuit. D'abord, elle n'avait pas été utilisée par Furetière et permettait donc à Basnage de mettre à jour le \emph{Dictionnaire universel} avec les informations d’une source plus récente que celui-là. Par ailleurs, le choix de l'édition de 1693 n'est pas fortuit. En n’étant pas utilisée par Furetière elle permet à Basnage de mettre à jour le Dictionnaire universel en peu de temps avec une grande quantité d'informations récentes, sélectionnées et ordonnées.

\section*{Acknowledgement}
Ce travail a été réalisé dans le cadre du projet ANR-18-CE38-0003 \emph{Numérisation et analyse du Dictionnaire universel de Basnage de Beauval: lexicographie et réseaux scientifiques – BASNUM}

{\sloppy\printbibliography[heading=subbibliography,notkeyword=this]}
\end{otherlanguage}
\end{document}
