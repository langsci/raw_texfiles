\documentclass[output=paper,colorlinks,citecolor=brown,arabicfont,chinesefont]{langscibook}
\ChapterDOI{10.5281/zenodo.15394507}
\author{Rafał Rosół\affiliation{Adam Mickiewicz University in Poznań}}
\title{The \textit{Lexicon of Oriental Words in Ancient Greek}}

\abstract{The aim of this chapter is to present the \emph{Lexicon of Oriental words in Ancient Greek}, which currently being prepared in both electronic and book form. The lexicon will contain common words (loanwords and glosses) that derive from Anatolian, Armenian, Austroasiatic, Cushitic, Dravidian, Egyptian, Hurro-Urartian, Indo-Aryan, Iranian, Kartvelian, Phrygian, Semitic, Sino-Tibetan and other Eastern languages. It will also include vocabulary of uncertain provenance (especially \emph{Wanderwörter}). The data will cover two millennia of the history of the Greek language, i.e. from the Mycenaean period to the 6th cent. CE.} 

\IfFileExists{../localcommands.tex}{
  \addbibresource{../localbibliography.bib}
  \usepackage{langsci-optional}
\usepackage{langsci-gb4e}
\usepackage{langsci-lgr}

\usepackage{listings}
\lstset{basicstyle=\ttfamily,tabsize=2,breaklines=true}

%added by author
% \usepackage{tipa}
\usepackage{multirow}
\graphicspath{{figures/}}
\usepackage{langsci-branding}

  
\newcommand{\sent}{\enumsentence}
\newcommand{\sents}{\eenumsentence}
\let\citeasnoun\citet

\renewcommand{\lsCoverTitleFont}[1]{\sffamily\addfontfeatures{Scale=MatchUppercase}\fontsize{44pt}{16mm}\selectfont #1}
  
  %% hyphenation points for line breaks
%% Normally, automatic hyphenation in LaTeX is very good
%% If a word is mis-hyphenated, add it to this file
%%
%% add information to TeX file before \begin{document} with:
%% %% hyphenation points for line breaks
%% Normally, automatic hyphenation in LaTeX is very good
%% If a word is mis-hyphenated, add it to this file
%%
%% add information to TeX file before \begin{document} with:
%% %% hyphenation points for line breaks
%% Normally, automatic hyphenation in LaTeX is very good
%% If a word is mis-hyphenated, add it to this file
%%
%% add information to TeX file before \begin{document} with:
%% \include{localhyphenation}
\hyphenation{
affri-ca-te
affri-ca-tes
an-no-tated
com-ple-ments
com-po-si-tio-na-li-ty
non-com-po-si-tio-na-li-ty
Gon-zá-lez
out-side
Ri-chárd
se-man-tics
STREU-SLE
Tie-de-mann
}
\hyphenation{
affri-ca-te
affri-ca-tes
an-no-tated
com-ple-ments
com-po-si-tio-na-li-ty
non-com-po-si-tio-na-li-ty
Gon-zá-lez
out-side
Ri-chárd
se-man-tics
STREU-SLE
Tie-de-mann
}
\hyphenation{
affri-ca-te
affri-ca-tes
an-no-tated
com-ple-ments
com-po-si-tio-na-li-ty
non-com-po-si-tio-na-li-ty
Gon-zá-lez
out-side
Ri-chárd
se-man-tics
STREU-SLE
Tie-de-mann
}
  \togglepaper[20]%%chapternumber
}{}

\begin{document} 
\maketitle

The \emph{Lexicon of Oriental Words in Ancient Greek}, which is currently in preparation, will have an electronic and a printed version. The entries will look the same in both versions, but there will be some differences largely due to their specific nature. The former will provide tools for searching and grouping words, while the latter will contain a much expanded introduction and indexes of linguistic data. The online lexicon will be fully completed in 2025, but is already partially available (\url{https://orientalwordsingreek.org/}). Shortly afterwards, work will begin on preparing the book for publication.

The goal of the project is to study words borrowed by the Greeks from Oriental languages, i.e. languages used in Asia (along with the European part of Scythia and the Caucasus region), Egypt and, because of the ancient maritime trade route from India to Europe, on both sides of the Gulf of Aden and the Red Sea. Accordingly, the vocabulary under investigation will originate in Anatolian, Indo-Aryan, Iranian, Phrygian, Armenian, Semitic, Egyptian, Cushitic, Hurro-Urartian, Kartvelian, Dravidian, Austroasiatic, Sino-Tibetan, etc. It will also encompass vocabulary without a determined source (especially \emph{Wanderwörter}). The data will cover two millennia of the history of the Greek language, i.e. from the Mycenaean period (the Linear B tablets) to the 6th cent. CE.

The lexicon will generally contain only common nouns. However, it will also include words that are not proper nouns in principle, but were treated as such by Greek authors and therefore usually appear in capital letters in modern editions, e.g. Βραχμᾶνες m. pl. ‘Indian sages, philosophers (often treated as a tribe)’, which goes back to Sanskrit \emph{brāhmaṇa}- ‘one who has divine knowledge; a Brahman’ (cf. \emph{brahman}- ‘the class of men who are the repositories and communicators of sacred knowledge, the Brāhmanical caste as a body’). Moreover, words for which it is difficult to determine whether they are proper nouns or common nouns will also be taken into account.

It should be noted that common nouns derived from proper nouns, if they are formed according to the rules of Greek morphology, will be excluded, e.g. Δαρεικός m. ‘a Persian gold coin’ (from Δαρεῖος). On the other hand, words like Μόσυλον (also μόσυλλον) n. ‘a kind of cinnamon of fine quality’ (cf. Μοσυλῖτις f. ‘a kind of cinnamon’) will be put into the lexicon; this term derives from the name of the city (and the cape) Mosyl(l)on located on the northern coast of Somalia and attested in Greek sources as Μόσυλον or Μόσυλλον (cf. Μόσυλοι, the name of a people living nearby),\footnote{That cinnamon was imported from Mosyl(l)on is attested to by a reference in the \emph{Periplus of the Erythraean Sea} (10): Ἐξάγεται δὲ ἀπὸ τῶν τόπων [scil. ἀπὸ τοῦ Μοσύλλου] κασίας χύμα πλεῖστον – (<δι>ὸ καὶ μειζόνων πλοίων χρῄζει τὸ ἐμπόριον – καὶ ἄλλη εὐωδία καὶ ἀρώματα καὶ χελωνάρια ὀλίγα καὶ θυμίαμα μοκρότου [...] καὶ λίβανος ὁ περατικὸς, ἐλέφας δὲ καὶ σμύρνα σπανίως. – “Exports from this area [scil. from Mosyllon] are: a great quantity of cassia (for this reason the port requires bigger ships); other spices and aromatics; a little low-quality tortoise shell; \emph{mokrotu} incense [...]; “far-side” frankincense; ivory and myrrh but only on rare occasions.” (trans. L.  Casson).}  but morphologically it does not show a Greek derivation with any of the possible suffixes.

An important part of the project are Oriental glosses that appear in Greek sources. We are referring to words that ancient and early Byzantine authors explicitly identify as Egyptian, Phoenician, Carian, etc., e.g. the plant name μίθ ‘celery (\emph{Apium graveolens} L.)’ that is quoted in the appendix (the so-called \emph{Recensio Vindobonensis}) to \emph{De materia medica} by Pedanius Dioscorides (3.64 RV):

\begin{quote}
    σέλινον κηπαῖον· [...] Αἰγύπτιοι μίθ [...].
    \newline
    \emph{Selinon kēpaion} (celery): [...] the Egyptians [call it] \emph{mith} [..,]
\end{quote}

This gloss comes from Egyptian \emph{m3t.t} ‘id.’ (cf. Coptic \emph{mit} (Sahidic and Bohairic), \emph{emit} (Bohairic) ‘parsley, celery’). Another example is the gloss ἄγλυ ‘swan’ transmitted by Hesychius of Alexandria (α 621): 

\begin{quote}
    ἄγλυ· ὁ κύκνος, ὑπὸ Σκυθῶν.
    \newline
    \emph{agly}: swan; [it is called so] by Scythians. 
\end{quote}

The word still has no convincing etymology. A large group of glosses are transliterations of Hebrew or Aramaic words found in the Septuagint (and in other Greek translations of the Old Testament), the New Testament, in the writings of Joseph Flavius and in other texts related to Judaism and Christianity. The following words may be cited as examples: γαι ‘valley’ (LXX, Ez. 39.11, 39.15) from Biblical Hebrew \emph{gayʾ} ‘id.’ and ταλιθα κουμ ‘little girl, get up!’ (NT, Mar. 5.41) from Aramaic *\emph{ṭalîṯā qûm} ‘id.’.

To date, no attempt has been made to create a lexicon containing all the possible Oriental loanwords and glosses found in Greek sources. Although there are many important books and meticulously researched papers devoted to the subject, there are still significant gaps. Of course, we can rely on some works concerning loanwords and glosses from Semitic,\footnote{E.g. \citealt{Masson1967},  \citealt{Bai2009},  \citealt{Rosol2013} and \citealt{Rosol2018}, as well as an important publication on Hebrew and Aramaic words in the Septuagint by \citealt{Simotas1968} .}  Indo-Iranian,\footnote{The most important publication on the subject is the comprehensive book by \citealt{Brust2008} ; moreover, see e.g. \citealt{Schmitt1967}:  \citealt{Hinz1975}:  \citealt{Huyse1990} .}  Egyptian,\footnote{First of all, the paper by \citealt{Fournet1989} ; furthermore, some overviews: \citealt{Hemmerdinger1968},  \citealt{Mcgready1968},  Torallas \citealt{Tovar2004} ; cf. also \citealt{Pierce1971} .}  Anatolian\footnote{E.g. \citealt{Neumann1961}  and an excellent overview by \citealt{Simon2018} ; the glosses from Anatolian languages, i.e. from Carian, Lydian, etc., can be found in the digital philological-etymological dictionary \emph{eDiAna}: https://www.ediana.gwi.uni-muenchen.de/.}  or Phrygian.\footnote{\citealt{Haas1960}, \citealt[39--68]{Sowa2008}, Obrador \citealt[412--424]{Obrador-cursach2020}.}  However, many aspects of Oriental borrowings and glosses in ancient Greek have not been sufficiently studied and described. There is still no up-to-date monograph on Semitic words attested in the Hellenistic and Roman periods.\footnote{Cf. old works by \citealt{Lewy1895}  and \citealt{Muss-arnolt1892} .}  The lack of a monograph concerning Egyptian words in Greek must also be emphasised, since the papers mentioned above comprise only a partial material. There is also no comprehensive work on “wandering” words in antiquity or on Oriental words with undefined source language. Moreover, it is worth noting that the existing etymological dictionaries of Greek\footnote{\citealt[72]{Frisk1960}, \citealt{Chantraine1999},  \citealt{Beekes2010} .}  are often inadequate in their treatment of the problem of Oriental borrowings and very often do not represent the current state of research, as well as they do not include many foreign words and the vast majority of glosses. All of this points to the need for a comprehensive lexicon of the Oriental vocabulary that appears in ancient and early Byzantine Greek sources.

Regarding the structure and composition of entries in the lexicon, they will consist of 4 or, if sufficient data is available, 5 sections. The basic section will be a headword with three types of tags. The first indicates the general provenance of a given word: Anatolian, Armenian, Austroasiatic, Cushitic, Dravidian, Egyptian, Hurro-Urartian, Indo-Aryan, Iranian, Kartvelian, Phrygian, Semitic, Sino-Tibetan or, simply, Unspecified origin. The second tag refers to the earliest attestation of a word in Greek texts. Accordingly, the words will be classified into six historical periods: Mycenaean (15th–12th cent. BCE), Archaic (8th–6th cent. BCE), Classical (5th–4th cent. BCE), Hellenistic (3rd–1st cent. BCE), Roman (1st–4th cent. CE), and Early Byzantine (5th–6th cent. CE). The third tag allows a general differentiation of the word type: loanword or gloss. In doubtful cases, more than one tag may appear in a given category, e.g. Iranian?/Semitic?, Classical/Hellenistic period, Loanword/Gloss. The information in this section will be particularly useful when searching the electronic version of the dictionary, as the browsing tools will enable to search for words in two different ways: a) by typing a headword in the search box (in the Greek alphabet); b) by selecting tags. More tags and search tools will be introduced in the future.

The second section (marked \RosolSymbol{1F449}) will present word forms (including grammatical features and, where possible, vowel length, together with definitions and sources), as well as variants and \emph{variae lectiones} in manuscripts. An entry will also contain the most important derivatives and compounds. In addition, this section will draw attention to possible difficulties in determining the exact meaning of a word, such as plant identification of phytonyms or measurements for unit names. In the case of rare words (especially, \emph{hapax legomena}), suggestions for conjunctures will also be cited. Furthermore, possible Latin counterparts will be given, if a word (usually being a borrowing from Greek) appears in Latin sources.

The third section (marker: \RosolSymbol{26A0}) is something new in lexicographic practice. It will contain quotations and pieces of information that may directly or indirectly point to possible Oriental roots of a word. Each Greek (or Latin) quotation will be provided with an English translation. For instance, the gloss διβιρουπτ m. ‘chief scribe (an Iranian title)’ (attested in the inscription abbreviated as \emph{ŠKZ}, v. 57) will be accompanied only by the following comment: 

\begin{quote}
    \RosolSymbol{26A0} \emph{ŠKZ} is a trilingual, i.e. Middle Persian, Parthian and Greek inscription of Shapur I written on the walls of the Ka’ba-ye Zartosht (Cube of Zoroaster) in Naqsh-e Rustam (ca. 12 km from Persepolis).
\end{quote}

In the case of βωσαρή f. ‘a kind of elephant (in India)’, a relevant passage from the \emph{Periplus of the Erythraean Sea} will be quoted in Greek and English:

\begin{quote}
    \sloppy
    \RosolSymbol{26A0} \emph{PME} 62: Ἀπὸ δὲ ταύτης [scil. Μασαλίας] εἰς αὐτὴν τὴν ἀνατολὴν διαπεράσαντι τὸν παρακείμενον κόλπον ἡ Δησαρηνὴ χώρα, φέρουσα ἐλέφαντα τὸν λεγόμενον βωσαρὴ [...].
    
    If you go due east from it [scil. from the region of Masalia, i.e. Machilipatnam] across the bay that lies alongside, you come to the Desarene region [i.e. probably Odisha], the habitat of a kind of elephant called \emph{bōsarē}. (trans. L. Casson).
\end{quote}

Of course, not every entry will have this section, but in some cases it can be quite extensive (even two or three pages long), depending on the material available.

The fourth section (marker: \RosolSymbol{1F154}) will provide results of etymological investigations along with extensive linguistic data from Oriental languages, e.g. ἀγάλοχον (also ἀγάλλοχον) n. ‘agarwood (a fragrant wood obtained from some tree species of the genus \emph{Aquilaria}, especially \emph{Aquilaria malaccensis} Lam.)’ is explained in this way: 

\begin{quote}
    \RosolSymbol{1F154} An Oriental word, originally Indo-Aryan or Dravidian – Indo-Aryan: Sanskrit \emph{agaru-} and \emph{aguru-} ‘agarwood tree (\emph{Aquilaria malaccensis} Lam.)’, Pali \emph{agaru-}, \emph{agalu-}, \emph{agaḷu-}, \emph{akalu-} ‘the fragrant resin or wood of agarwood tree (\emph{Aquilaria malaccensis} Lam.)’ etc.; Dravidian: Tamil \emph{akil} ‘agarwood tree (\emph{Aquilaria malaccensis} Lam.); the drug agar obtained from the tree’, Malayalam \emph{akil} ‘agarwood tree (\emph{Aquilaria malaccensis} Lam.)’ etc.; cf. Biblical Hebrew ʾ\emph{ăhālôṯ} (formally pl.) and ʾ\emph{ăhālîm} (formally pl.) ‘agarwood (of \emph{Aquilaria malaccensis} Lam.)’ along with Greek ἀλωθ which is a transliteration of ʾ\emph{ăhālôṯ} in the Septuagint, Middle Persian \emph{awalūg} ‘agarwood’ (probably from earlier *\emph{aγalūk}; cf. \emph{w} in Middle Persian \emph{murw} ‘bird’ vs. Young Avestan \emph{mərəγa}- ‘id.’), Modern Persian \emph{āluwwa} ‘id.’, Syriac ʾ\emph{wlwg}, ʾ\emph{blwg} ‘id.’ (from Iranian). The closest to Greek ἀγάλοχον is the early Middle Persian form *\emph{aγalūk}, but, probably, the Greeks did not adopt this word from Persian (we would expect κ instead of χ). Cf. an unlikely suggestion that ἀγάλοχον comes from Indo-Aryan *\emph{agalukha-} with the suffix -\emph{kha}- (this view has no confirmation in the available data).
\end{quote}

If there are different hypotheses about the etymology of a word, these are presented in such a way that the dictionary user can easily find them out without having to consult the relevant publications. As a rule, the etymologies proposed in the last 150 years are cited, and only in special situations are older theories given. For example, in the case of the Carian gloss γίσσα ‘stone’ six proposals are listed (the hypotheses are so unlikely as to require no further discussion):

\begin{quote}
    \RosolSymbol{1F154} A Carian word without etymology. The gloss was juxtaposed with: 1. γεῖσον n. ‘projecting part of the roof, cornice’; 2. κίσηρις, -εως and -ιδος f. ‘pumice’; 3. Georgian \emph{kviša} ‘sand’, etc.; 4. Basque \emph{gisu}, \emph{khisu} ‘lime’ (note that this word comes from Latin \emph{gypsum} ‘lime’); 5. Akkadian \emph{ḫiṣṣu(m)} ‘gravel’, Biblical Hebrew \emph{ḥāṣāṣ} ‘id.’, etc.; 6. Old High German \emph{kisil} ‘pebble’, Old English \emph{ceosel} ‘gravel’, etc.
\end{quote}

As regards etymological investigations, methodological approaches developed in the field of comparative-historical linguistics are largely applied. This approach will allow for phonetic and morphological analyses. Of course, instead of sound laws (known e.g. from Indo-European studies), phonetic counterparts between the source and target languages are crucial here.\footnote{Cf. considerations made by \citealt{Brust2008}:  I–LXI. in the introduction to his book on Indo-Iranian vocabulary in Greek.}  It would be a methodological error to suppose, for example, some irregularities without a determined cause or to postulate a semantic change without sufficient evidence for it.\footnote{See \citealt{JasanoffNussbaum1996}  who levelled criticism at an overly loose approach to phonetic and semantic change of \citealt{Bernal1987} –2006; cf. \citealt{Nardelli2013} . For the methodology, see also \citealt{Hoch1994}:  6–14, \citealt{Griffith1997},   \citealt[1--91]{Ciancaglini2008}, as well as, more general, \citealt{ThomasonKaufmann1988}.}

The last section of an entry (marker: \RosolSymbol{1F56E}) will deliver bibliographical notes divided into two groups: a) Data – publications (especially dictionaries) that provide linguistic data from Oriental languages; b) References – the most important publications concerning the etymology of a given word. The bibliography (located under a separate tab in the electronic version or at the end of the publication) will be divided into three groups: a) Editions of sources (ancient literary texts, inscriptions, papyri) with  abbreviations used throughout the lexicon; b) Book abbreviations (especially dictionaries and reference books); c) Books and articles.

In summary, the \emph{Lexicon of Oriental words in Ancient Greek}, which will contain approximately 1200 entries, aims to be a useful reference work in the fields of etymological, philological, historical and cultural studies. First of all, it seeks to constitute an important addition to the knowledge of Greek etymology in general, as well as the presence of Oriental words in Greek sources. The data collected will also be a significant voice in the discussion on intercultural contacts in antiquity, namely between the West and the East over the course of two thousand years of history. The investigations will provide linguistic evidence for studies concerning the mutual relations and Eastern impact on Greek culture. The lexicon may also be useful for the study of other languages, which have adopted words of Oriental origin, either directly from Greek or via Latin.

\section*{Acknowledgments}

This research was funded by National Science Centre, Poland (research project no. UMO-2020/39/B/HS2/00934).

{\sloppy\printbibliography[heading=subbibliography,notkeyword=this]}
\end{document}
