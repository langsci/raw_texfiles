\documentclass[output=paper,colorlinks,citecolor=brown,arabicfont,chinesefont]{langscibook}
\ChapterDOI{10.5281/zenodo.15394483}
\author{Alenka Jelovšek\affiliation{Research Centre of the Slovenian Academy of Sciences and Arts, Ljubljana, Slovenia}}
\title[Nouns of non-Slavic origin in 16th c. Slovenian literary language]
      {The incorporation of proper nouns of non-Slavic origin into the 16th-century Slovenian literary language}

\abstract{After the establishment of the Slovenian literary language as the result of the Reformation movement in the 16th century, a significant number of proper nouns of Semitic, Romanesque and Germanic origin were introduced into it, which posed a great challenge in terms of their orthography, phonology and inflection. As the Protestant translated mainly from Luther’s works and some of them also used Latin translations of the Bible, the use of proper names was highly influenced by those languages. The analysis of proper names used in 16th-century Slovenian literary language, based on the complete excerpt of the Slovenian printed works published between 1550 and 1595, shows that the authors various strategies for transferring proper names of different origins: in some cases they were transferred in their original form, but generally they were at least morphologically adapted, as Slovenian is an inflection-rich language; they were also inconsistently adapted orthographically and phonologically, creating various exonyms, which resulted in high degree of variation in the literary language of the period.}

\IfFileExists{../localcommands.tex}{
  \addbibresource{../localbibliography.bib}
  % add all extra packages you need to load to this file

\usepackage{tabularx,multicol}
\usepackage{url}
\urlstyle{same}

\usepackage{listings}
\lstset{basicstyle=\ttfamily,tabsize=2,breaklines=true}

\usepackage{langsci-basic}
\usepackage{langsci-optional}
\usepackage{langsci-lgr}
\usepackage{langsci-osl}
% \usepackage{./langsci/styles/langsci-lgr}
% \usepackage{./langsci/styles/langsci-osl}
% \usepackage{langsci-gb4e}

\usepackage{tikz}
\usetikzlibrary{patterns,calc}
\pgfdeclarepatternformonly{south east lines}{\pgfqpoint{-0pt}{-0pt}}{\pgfqpoint{3pt}{3pt}}{\pgfqpoint{3pt}{3pt}}{
    \pgfsetlinewidth{0.6pt}
    \pgfpathmoveto{\pgfqpoint{0pt}{3pt}}
    \pgfpathlineto{\pgfqpoint{3pt}{0pt}}
    \pgfpathmoveto{\pgfqpoint{.2pt}{-.2pt}}
    \pgfpathlineto{\pgfqpoint{-.2pt}{.2pt}}
    \pgfpathmoveto{\pgfqpoint{3.2pt}{2.8pt}}
    \pgfpathlineto{\pgfqpoint{2.8pt}{3.2pt}}
    \pgfusepath{stroke}}
    
\usepackage{stmaryrd}
\usepackage{wasysym}
\usepackage{multirow}
\usepackage{caption}
\usepackage{subcaption}
\usepackage{mathrsfs}
\usepackage{qtree}

\usepackage{linguex}


  %pminos do not split footnotes
% \interfootnotelinepenalty=10000 %Footnote in Laporte chapters has to be split SN


%\DeclareIndexNameFormat{default}{%
%\nameparts{#1}%
%\usebibmacro{index:name}%
%{\index[names]}%
%{\namepartfamily}%
%{\namepartgiveni}%
% {}% L1
% {}% L2
%{\namepartprefix}% generates spurious space L3
%{\namepartsuffix}% generates spurious space L4
%}

%  {\DeclareIndexNameFormat{default}{%
%     \usebibmacro{index:name}{\index[names]}{#1}{#3}{#5}{#7}}}

%\DeclareIndexNameFormat{default}{%
%  \usebibmacro{index:name}{\sindex[nom]}{#1}{#3}{#5}{#7}}

%\DeclareIndexNameFormat{default}{%
%  \usebibmacro{index:name}{\sindex[person]}{#1}{#3}{#5}{#7}}
%\DeclareIndexNameFormat{default}{%
%\nameparts{#1} \usebibmacro{index:name}{\sindex[person]]}{\namepartfamily}{‌​\namepartgiven}{\nam‌​epartprefix}{\namepa‌​rtsuffix}}

%\newcommand{\smiley}{:)}

%\renewbibmacro*{index:name}[5]{%
%\usebibmacro{index:entry}{#1}%
%{\iffieldundef{usera}{}{\thefield{usera}\actualoperator}\mkbibindexname{#2}{#3}{#4}{#5}}}

% \newcommand{\noop}[1]{}

%remove for final
%\overfullrule=1mm

\newcommand{\tobi}[2]}}
\renewcommand{\S}[1]{\tobi{#1}{\textsc{*}}}

% this volume references
% puts: [this volume]
% already defined: \citetv
%\newcommand{\citepv}[1]{(\citeauthor{#1} \citeyear*{#1} [this volume])}
\newcommand{\citealtv}[1]{\citeauthor{#1} \citeyear*{#1} [this volume]}

%parentheses around example number
\newcommand{\pref}[1]{(\ref{#1})}

% in-text examples

\newcommand{\lnex}[1]{\textit{#1}} %target lang word
\newcommand{\lnlit}[1]{(lit.: `#1')} %literal reading
\newcommand{\lnlat}[1]{(#1)} % latinization
\newcommand{\lntrans}[1]{`#1'} %translation
\newcommand{\lnexl}[2]%
{\lnex{#1}{} \lnlat{#2}} % ex with latinization
\newcommand{\lnexlat}[3]{\lnex{#1}{} \lnlat{#2}{} \lntrans{#3}} % ex with latinization and tranl.

%ch01
\newcommand{\co}[1]{\mbox{\textbf{#1}}}

%ch09

\newcommand{\cyrbulg}[1]{\begin{otherlanguage*}{bulgarian}#1\end{otherlanguage*}}


%ch10
\newcommand{\nlp}{{\small NLP}}
\newcommand{\mwe}{{\small MWE}}
\newcommand{\rae}{{\small RAE}}
\newcommand{\lvc}{{\small LVC}}
\newcommand{\pos}{{\small P}o{\small S}}
%\newcommand{\todo}[1]{ \textcolor{red}{#1} }

%\renewcommand{\labelenumi}{\theenumi}
%\ainamefmt{{vv}{ll}{, ff}{, jj}} % fullname

\newcommand{\biberror}[1]{{\color{red}#1}}

\newcommand{\osenovaitem}{--~}
  %% hyphenation points for line breaks
%% Normally, automatic hyphenation in LaTeX is very good
%% If a word is mis-hyphenated, add it to this file
%%
%% add information to TeX file before \begin{document} with:
%% %% hyphenation points for line breaks
%% Normally, automatic hyphenation in LaTeX is very good
%% If a word is mis-hyphenated, add it to this file
%%
%% add information to TeX file before \begin{document} with:
%% %% hyphenation points for line breaks
%% Normally, automatic hyphenation in LaTeX is very good
%% If a word is mis-hyphenated, add it to this file
%%
%% add information to TeX file before \begin{document} with:
%% \include{localhyphenation}
\hyphenation{
    Beck-man
    Ngu-yen
    back-chan-nel
    back-chan-nels
    mo-not-o-nous
    ste-reo-typ-i-cal
}

\hyphenation{
    Beck-man
    Ngu-yen
    back-chan-nel
    back-chan-nels
    mo-not-o-nous
    ste-reo-typ-i-cal
}

\hyphenation{
    Beck-man
    Ngu-yen
    back-chan-nel
    back-chan-nels
    mo-not-o-nous
    ste-reo-typ-i-cal
}

  \togglepaper[9]%%chapternumber
}{}

\begin{document}
\maketitle

\section{Introduction}

Proper names have received much scholarly attention in various disciplines of linguistics in recent decades, including contact linguistics and translation studies, but not many studies deal with the question of integration of foreign proper names into a (literary) language in its older periods, especially those originating in languages which are not genetically closely related or are even unrelated.\footnote{For historical integration of names from closely related languages see e.g. \citealt{Casanova2012}; for integration of names from classical languages, see e.g. \citealt{Gordon2012}; for adaptation of Hebrew names in various translations of the Bible, including Slovenian, see \citealt{Krasovec2006} and \citealt{Krasovec2010}.} This study aims to present the methods of incorporation of proper names of non-Slavic origin into the (Slavic) Slovenian literary language at the time of its establishment in the 16th century, taking into the account the specific linguistic situation in Slovenian lands in the analysed period and the distinctive features of proper names in a language system.

Proper names, in contrast to common nouns/semantic words, do not have lexical meaning, i.e. descriptive content with which they would denote concepts. Their function is to refer and identify; accordingly, their relationship to their referent is direct and not mediated via the lexical content \citep[312]{SchluckerAckermann2017}. \citet[549]{Nicolaisen1996} describes them as semantically opaque: “Proper names have content, not meaning, in order to function as identifying and individuating devices".

\citet{Hermans2015} points out an additional formal property of proper names: their minimal integration into the language system. “Phonologically as well as morphologically they exploit the possibilities available to a given language to a larger extent than is the case in the rest of the lexicon" \citep[12]{Hermans2015}. Similarly, \citet{SchluckerAckermann2017} argue that proper names differ from common nouns morphosyntactically.

Because of their specific features, the transfer of proper nouns from one language to another in language contact situation is easier in comparison with semantic words. As \citet[94]{Raukko2017} states, “there is no requirement for target language users to understand the semantics of the foreign items, as long as the identifying function is evident”. Similarly, \citet[549]{Nicolaisen1996} claims that “[a]s long as at least minimal content is perceived in their usage it can be transferred in a linguistic shell and thus keep the name alive”. 

The aim of this chapter is to illustrate that such proper names transfer between genetically unrelated or distantly related languages based on the historical material from the period of the establishment of the Slovenian literary language in the 16th century. As a result of numerous translations created in a short period of time, there was a large influx of proper names of non-Slavic origin into Slovenian, which posed a great challenge in terms of their orthography, phonology and inflection. Lacking an established tradition, the translators applied various strategies of transfer which resulted in a high level of variation. 

The structure of the paper is as follows: Section 2 brings a general discussion on the strategies of transferring proper names between languages  and outlines the general terminology used in the remainder of the chapter. Section 3 gives a brief historical overview of the Slovenian literary language in the analysed period, and Section 4 outlines the general properties of proper names of foreign origin in Slovenian before and at the time of the establishment of the literary languages, as well as giving the description of the material for the analysis.  Following this general discussion, Section 5 presents and discusses the findings from the analysis, divided into three subsections, described in more detail at the end of the Section 4. Finally, the concluding remarks section considers the main findings of the analysis and outlines their meaning for Slovenian linguistics.

\section{The strategies of inter-lingual proper name transfer}

Linguists have enumerated various strategies of transferring proper names from one language into another. \citet{Hermans2015} speaks of \emph{copying}, i.e reproducing proper names in the target text exactly as they were in the source text; \emph{transcription}, under which he subsumes transliteration and adaptation on the level of spelling, phonology, etc.; \emph{substitution}; and \emph{translation} in cases when a proper name acquires "meaning" (quotation marks in original text) in the source language. Somewhat differently, \citet{Vermes2003} speaks of \emph{transference} (= Hermans's copying); \emph{substitution}, which for Vermes encompasses replacing a proper name from the source language with a conventional correspondent in the target language (= Hermans's substitution), as well as traditional transliteration (defined as the replacement of graphological units of the source language by graphological units of the target language, based on conventionally established correspondences, where the target language form makes explicit the phonological value of the original expression \citep[93]{Vermes2003}); \emph{translation} (rendering the meaning into another language); and \emph{modification} (replacing the original name with one which gives rise to similar contextual implications). \citet{Raukko2017} assumes Vermes's division but considers substitution more closely. He considers any adaptated form of a proper name in the target language that differs from the form in the source language an exonym, extending the term from geographical names to any type of proper names \citep[97]{Raukko2017}. According to the types and extent of adaptation of proper names, he distinguished between a) exonyms formed as a result of historical and cultural importance of its referent; b) exonyms arising from shared onomastic heritage (such as names of monarchs); c) transliterated forms, in which graphemes from the source language alphabet are replaced with the graphemes of the target language alphabet; d) exographs, which change orthography but, ideally, keep the original pronunciation (or change the orthography in order to bring the pronunciation closer to the original; e) grammatically adapted forms that are a result of a grammatical necessity, such as adding a required suffix to express gender, f) exophones, whose written form is not changed, but the phonetic or phonological form is: ``In the target language form the individual sounds and the phonotactic form are adjusted to the target language phonological system" \citep[99]{Raukko2017}. “Phonological translation” is typical not only for exophones but generally also for full exonyms. The author also mentions “semantic translation” which occurs only in cases where the original name can be seen not only as a name but as a carrier of semantic content \citep[108]{Raukko2017}. He also draws attention to the influence of “transmitting” or “intermitting” languages that spread their own exonyms to other languages \citep[106]{Raukko2017}.

In the context of 16th-century literary Slovenian, Raukko’s detailed division of forms of adaptation is the most appropriate as a basis for the study of the incorporation of proper names into Slovenian, and will serve as a framework for arranging the analysed data. His use of the term \emph{exonym} for adapted forms of various classes of proper names will also be adopted in the analysis, and the verb \emph{transfer} will be from now on used for describing cases which would fall into the Hermans’s category of \emph{copying} and Vermes’s \emph{transference}.

\section{The establishment and development of the Slovenian literary language in the 16th century}

Literary Slovenian was established in the 16th century as a direct result of a Reformation movement in so-called Inner Austrian lands that were ruled by the Hapsburgs and inhabited by mainly speakers of Slovenian dialects (duchies Carniola with a Slovenian-speaking majority and Carinthia and Styria with significant Slovenian-speaking minorities alongside German-speaking majority \emph {Historical atlas of East Central Europe} \citep[99]{MagocsiMatthews1993}). The first two printed Slovenian books, a catechism and a primer, were published in 1550 by an exiled Catholic priest turned Protestant, Primož Trubar, who thus established the first model for literary Slovenian, which he upgraded and partially modified in his subsequent translations of the New Testament (cf. \citealt{Cooper1985}). In 45 years, 49 Slovenian books, mainly with religious content, were printed by the Slovenian Protestant authors, with the financial help of the Inner Austrian estates and some German Protestant princes (cf. \citealt[64--73]{Ahacic2014}). In addition, Slovenian was also included in multi-language dictionaries compiled by the German scholar Hieronymus Megiser (1592, 1603). The main achievement of the period was the translation of the entire Bible by Jurij Dalmatin in 1584. Slovenian was thus the 14th world language with a complete translation of the Bible. In the same year, Adam Bohorič published his Slovenian grammar \emph{Arcticae horulae} (written in Latin). With the publication of the Bible and the grammar, Slovenian literary language achieved the level of standardization and the Bible became the basis of the Slovenian literary standard for the following two centuries \citep[35]{Pogorelec2011}.

\section{Proper names in the 16th-century literary Slovenian}

As the majority of the published texts were translations of either the Bible texts or German works with religious content, such as postils, hymnals or prayer books, a significant number of proper nouns related to the Christian faith and the Church (both Catholic or Protestant) of Semitic, Romanesque and Germanic origin were introduced into Slovenian, which posed a great challenge in terms of their orthography, phonology and inflection, especially because, before the establishment of the literary language, Slovenian was a seldom written vernacular with no existing orthographic tradition (although some oral supradialectal tradition is presumed to have existed in clerical circles; cf. \citet{Pogorelec2011}. Although some records of proper names exist, e. g. in medieval land registers \citep[73--74]{Jakopin1986} and in the Černjeja manuscript (cf. \citealt[109--133]{Mikhailov1998}), these were ad-hoc recordings and show high degree of variation.

Not all names of non-Slavic origin appearing in the works of the Slovenian Protestants were equally foreign to speakers of Slovenian in the analysed period. Some of them were familiar to them from church services and sermons, others from their contacts with non-Slavic surroundings, especially in the context of the Holy Roman Empire as the wider political frame into which Slovenian lands were incorporated and to which they were culturally connected; some personal names were also already incorporated into the Slovenian language as Christian names which gradually replaced original Slavic first names by the 13th century \citep[70]{Jakopin1986}. But many names, especially less frequent ones from the Old Testament, were introduced into Slovenian for the first time in the translations of the Bible. As the Protestants translated mainly from Luther’s works and some of them also used Latin translations of the Bible, the use of proper names was highly influenced by the two source languages. Their influence extended not only to newly introduced proper nouns but also to the already established Christian names, most often at the orthographical level. Similarly, many Aramaic and Hebrew biblical names were adopted in their Latinized form and therefore exhibit similar adaptations as originally Latin names.

The following Section 5 describes the strategies of incorporating non-Slavic proper names into the 16th century Slovenian. The analysis is limited to proper nouns as the prototypical proper names \citep[92--93]{Vermes2003}, more precisely, to personal names (anthroponyms) and place names (toponyms).  As their number in the analysed material is still high, as is the degree of variation on the levels of orthography, phonology and morphology, an exhaustive description of the topic is not possible in the limits of this chapter. Therefore, the analysis will focus on selected proper nouns with higher numbers of occurrences in the 16th-century Slovenian texts.

The first subsection discusses the incorporation of names of Latin origin and those adopted in their Latinized form; the second subsection deals with the names of Germanic origin (mostly from the southern German dialects and New High German; and the third subsection presents names of Semitic (mostly Aramaic and Hebrew) origin, with a high emphasis on the “intermitting” languages – Latin and German. 

The material for the analysis of proper names used in 16th-century Slovenian literary language is the complete excerpt of the Slovenian printed works published between 1550 and 1595. The excerpt is in the form of a paper concordance which was compiled as part of the collection of material for the \emph{Dictionary of the 16th-Century Slovenian Literary Language}.\footnote{The first volume of the \emph{Dictionary of the 16th-Century Slovenian Literary Language}, containing entries from A to D, was published in 2021 and is available online at the Slovenian dictionary portal \emph{Fran.si} since 2022. The dictionary does not contain proprial lemmas, i.e. proper names as dictionary entries (cf. \citealt{Langendonck2007}). } The paper concordance is kept at the Section for the History of the Slovenian Language, Fran Ramovš Institute of the Slovenian Language, Research Centre of the Slovenian Academy of Sciences and Arts.

\section{The incorporation of non-Slavic proper names into 16th-century Slovenian}

\subsection{Latin or Latinized names}

In the case of names taken from or via Latin, we must distinguish between names from the Christian tradition which were, in the time of the establishment of the literary language, already relatively well incorporated into Slovenian, and other names which only entered the language via Bible translation. 

The names from the first group were already phonologically and morphologically adapted to Slovenian: the Latin nominative endings \emph{-us} and \emph{-as} were commonly replaced with \emph{-už/-ož/-o} and \emph{-ež}, respectively (Lat. \emph{Primus} > Sln. \emph{Primush/Primosh} [Primož], Lat. \emph{Lucas} > Sln. \emph{Lukesh}, Lat. \emph{Marcus} > Sln. \emph{Marko}), or even omitted (Lat. \emph{Augustinus} > Sln. \emph{Augustin}, Lat. \emph{Paulus} > Sln. \emph{Paul/Paval}, Lat. \emph{Petrus} > Sln. \emph{Peter} with an added \emph{schwa} between the final two consonants); the palatalization of sibilants was common not only in the endings but also in the middle of a word (Lat. \emph{Augustinus} > Sln. \emph{Augu\textbf{sh}tin}); some of the names underwent further phonological adaptations, such as vowel and consonant reductions etc. (eg. Lat. \emph{Iohannes} > Sln. \emph{Jansh} [Janž] > \emph{Ansh}, Lat. \emph{Georgius} > Sln. \emph{Iuri}). Such names can be considered full exonyms, according to Raukko (see Section 2), differing from the original names orthographically as well as phonologically and morphologically. Some of them are attested as Christian names of the Protestant writers: \emph{Primosh (Truber), Iuri (Dalmatin), Iansh (Shvvager)}, but the authors also sometimes signed their names in the Latinized forms: \emph{Primus Truber, Georgius Jurishitz}. The Latin (or German) influence can also be seen in the orthography of the first names: the author of Slovenian Protestant hymns, Marko Kumpreht, spelled his name as \emph{Marco} (< Lat. \emph{Marcus}) with the <c> and not with the <k> that was occasionally used in Slovenian works when referring to Mark the Evangelist (\emph{Marko}, \emph{Markus}).\footnote{Such variants of personal names also reflect the multilingualism of the authors: as no Slovenian schools existed in the 16th century, intellectuals were by necessity multilingual \citep[23, 43--44]{Ahacic2014} and, as a result of their education in German and Latin schools, often more fluent in those languages than in their native tongue, especially in written communication. As a result, German and Latin orthographical norms greatly influenced their writing in Slovenian.}  

\largerpage
Although such names had been well established as exonyms in Slovenian, under the influence of the German and Latin translations of the Bible as their source texts, Slovenian translators often transmitted source forms of proper names into their translations. Therefore, in the Bible translations the Latin(ized)\footnote{See further discussion on orthography in possible phonological variation of the original forms.}    forms of names were predominantly used, while in other texts exonyms were more frequent, although sporadically they can also be found in the Bible translations; e.g. in Dalmatin’s Bible there are 8 occurrences of the exonym \emph{Peter} and 104 occurrences of the Latin form \emph{Petrus} in the Bible verses, while the exonym \emph{Iansh} does not appear in the Bible verses; the only name where the exonym is almost as frequent as the Latin form is Paul (34 occurrences of \emph{Paul} vs. 49 occurrences of \emph{Paulus}). In comparison, Dalmatin’s predecessor Trubar used \emph{Paulus} much more frequently; in the same Bible verses, he used \emph{Paulus} only in four instances.  But exonyms are often used in other Slovenian texts, as well as in titles, marginal notes and other additions to the Bible verses in the Bible translations. 

\begin{itemize}
    \item[(1)] INu \textbf{Ioanneſu} ſo njegovi Iogri letu vſe osnanili <S. \textbf{Iansh} ſvoja dva Iogra h'Chriſtuſu poſhle> (Dalmatin, \emph{Biblia}, 1584)
    
    ‘And the disciples of John shewed him of all these things <Saint John sends two of his disciples to Christ>’.\footnote{The sources for the Slovenian quotations in this chapter were either the \emph{Corpus of texts by the 16th-Century Slovenian Protestant writers} (Sln. \emph{Korpus besedil slovenskih protestantskih piscev 16. stoletja}), available at \url{https://fran.si/korpus16} or the \emph{Slovenian Bible Portal} \citep{[Biblija]}, which includes several Slovenian historical and modern translations, as well as some foreign language translations of the Bible, including Luther’s \emph{Bible} (1545) and the Vulgate, from which German and Latin equivalents of Slovenian quotations are copied. English translations of the Bible texts are copied from the King James Version, also available at the \emph{Slovenian Bible Portal}; marginal notes (marked by <>) and other non-Biblical texts were translated by the author of the chapter.}
\end{itemize}

Also, some of the authors, such as Sebastijan Krelj and Jurij Juričič, showed preference to source language forms of proper names over exonyms also outside Bible translation \citep[71]{Jakopin1986}.

The second group consists of names of Latin origin that had not already been incorporated into the Slovenian at the time of the establishment of the literary language. Their nominal forms were typically transferred into Slovenian texts, although for Latin names ending in \emph{-ius} and \emph{-as}, apparent exonyms with nominative ending \emph{-a} or \emph{-i} can be found in the Bible translations. But the comparison with the Latin and German translations shows that such forms were usually not true exonyms but transfers of Latin or Luther’s German vocative forms into Slovenian, which at the analysed period no longer had distinctive case endings for vocative, which had been replaced with nominative endings. 

\begin{itemize}
    \item[(2)] \textbf{Trubar 1557:} Ta iſti ie uidil utim uidenu ozhitu okuli te deuete ure tiga dne, eniga Boshyga Angela noter kſebi gredozh, inu ie dial knemu, \textbf{Corneli}.
    
    \textbf{Dalmatin 1584:} Ta je vidil v'eni prikasni ozhitu, ob deveti uri, po dnevi, eniga Boshjiga Angela k'ſebi notèr gredozh, ta je djal k'njemu: \textbf{Corneli}.
    
    \textbf{Vulgate:} vidit in visu manifeste quasi hora nona diei angelum Dei introeuntem ad se et dicentem sibi \textbf{Corneli}
    
    \textbf{Luther:} Der sahe in einem Gesichte offenbarlich / vmb die neunde stunde am tage / einen Engel Gottes zu jm eingehen / der sprach zu jm / \textbf{Corneli}.
    
    ‘He saw in a vision evidently about the ninth hour of the day an angel of God coming in to him, and saying unto him, Cornelius.’

    \item[(3)] \textbf{Trubar 1577:} INu \textbf{Agrippas}\textsubscript{NOM} rezhe Hpauluſu. Tebi ie perpuſzhenu ſamimu ſa ſe gouoriti. Tedai Paulus iſtegne to roko, inu ie ſa ſe uti maſſi gouuril. Lubi Kral \textbf{Agrippa}\textsubscript{VOC}, Ieſt ſe shtimam ſa ſrezhniga, de ſe ieſt bõ danas pred tebo odgouariel, od uſeh tih rizhi, ſa katerih uolo ſem od tih Iudou ſatoshen
    
    \textbf{Dalmatin 1584:} \textbf{AGrippas} pak je rekàl h'Paulu: Tebi je pèrpuſzhenu, de ſam sa ſe govoriſh. Natu ſe je Paul sagovarjal, inu je roko istegnil: Meni je ſilnu lubu, lubi Krajl \textbf{Agrippa}, de ſe jeſt imam danas pred tabo odgovoriti, vſiga tiga, kar mene Iudje dolshé:
    
    \textbf{Vulgate:} \uline{Agrippa} vero ad Paulum ait permittitur tibi loqui pro temet ipso tunc Paulus extenta manu coepit rationem reddere de omnibus quibus accusor a Iudaeis rex \textbf{Agrippa} aestimo me beatum apud te cum sim defensurus me hodie

    \textbf{Luther:} \textbf{AGrippas} aber sprach zu Paulo / Es ist dir erleubet fur dich zu reden. Da verantwortet sich Paulus / vnd recket die hand aus / Es ist mir seer lieb / lieber könig \textbf{Agrippa} / das ich mich heute fur dir verantworten sol / alles des ich von den Jüden beschüldiget werde

    ‘Then Agrippa said unto Paul, Thou art permitted to speak for thyself. Then Paul stretched forth the hand, and answered for himself: I think myself happy, king Agrippa, because I shall answer for myself this day before thee touching all the things whereof I am accused of the Jews.’
\end{itemize}

Besides transferred nominal forms and full exonyms, names from both groups forms were frequently adapted orthographically, forming \textbf{exographs}: the most common changes were replacing Latin dipthongs and digraphs representing aspirated plosives, which are not a part of Slovenian phonological system, as well as omitting Latin (post-Classical) silent <h> and replacing Latin <c> with Slovenian <k>: e.g.,; for Lat. \emph{Iohannes}, besides the transferred form exographs \emph{Ioannes} and \emph{Ioanes} are attested; similarly, for Lat. \emph{Philippus}, we find exographs \emph{Philipus}, \emph{Phillipus}, \emph{Filippus} and \emph{Filipus} (besides equally orthographically variant exonyms \emph{Philip, Fillip, Filip}); for the Latinized name \emph{Caiphas} exographs \emph{Kaifas}, \emph{Kajfas}, \emph{Kajphas}, \emph{Kaiffas} are attested as well as exonyms \emph{Kajfash}, \emph{Kaifeſ}, \emph{Kaifesh}, \emph{Caifesh} and \emph{Kajphesh}; for Lat. \emph{Mathaeus} we find only exographs \emph{Matheus}, \emph{Mateus}, \emph{Matteus} (besides exonyms \emph{Matheuſh}, \emph{Mateush}, \emph{Matthevs}, \emph{Matteush}), original Latin spelling is not attested. The exonyms, as well as exographs, show high variation in spelling and demonstrate the high influence of Latin on the orthography of proper names in the 16th-century Slovenian literary language. It should be noted that in the entire Bible translation (Dalmatin, 1584), some of the earlier substitutions (such as \emph{ph} > \emph{f}) were omitted in favour of spelling closer to that in Luther’s Bible, while the others, such as the omission of the silent \emph{h}, were retained.

\begin{itemize}
    \item[(4)]\textbf{Trubar 1557:} Sakai Paulus ie bil naprei uſel, mumu \textbf{Efeſa} ſe pelati
    
    \textbf{Dalmatin 1584:} Sakaj Paulus ſi je bil naprej vsel mimu \textbf{Ephesa} ſe zhes pelati

    \textbf{Luther:} Denn Paulus hatte beschlossen / fur \textbf{Epheso} vber zu schiffen 
    
    ‘For Paul had determined to sail by Ephesus.’
    \item[(5)]\textbf{Dalmatin 1578:}\footnote{While other Bible quotations were copied from www.biblija.net, the quotation from Dalmatin's 1578 translation of the Pentateuch was taken from the \emph{Corpus of texts by the 16th-Century Slovenian Protestant writers}.}  Od Keelate ſo shli inu ſo legli vtih Gorrah \textbf{Saffer}. Od Gur \textbf{Saffer} ſo shli inu ſo legli \uline{vAradi}
    
    \textbf{Dalmatin 1584:} Od Keelate ſo ſhli, inu ſo legli v'téh gorrah \textbf{Sapher}. Od gur \textbf{Sapher} ſo ſhli inu ſo legli \uline{v'Aaradi} (Dalmatin, \emph{Biblia}, 1584)
    
    \textbf{Luther:} Von Kehelatha zogen sie aus / vnd lagerten sich im gebirge \textbf{Sapher}. Vom gebirge \textbf{Sapher} zogen sie aus / vnd lagerten sich in \uline{Harada}.
    
    ‘And they went from Kehelathah, and pitched in mount Shapher.  And they removed from mount Shapher, and encamped in Haradah.’
\end{itemize}

While the written source material does not enable us to definitively prove the existence of \textbf{exophones} in the 16th-century, it is probable that all forms of proper names that correspond to their Latin equivalents were not pronounced as in Latin. \citet[72--73]{Jakopin1986} believes that masculine names orthographically ending in \emph{-us} (\emph{-os}), \emph{-as}, \emph{-es} were pronounced with a final post-alveolar fricative [ʃ], same as their exonyms where such pronunciation is orthographically confirmed, or that variant high/low pronunciation existed. Some variants of inflectional forms and word-forming morphemes seem to corroborate his thesis. For example, possessive adjectives derived from the names \emph{Jesus} and \emph{Kristus} ‘Christ’, which are predominantly spelled according to Latin and German traditions (\emph{Christus}), are often variantly formed with the suffix \emph{-ev}, which is usually used after palatal consonants, instead of the expected \emph{-ov} (\emph{Iesusev} : \emph{Iesusov}, \emph{Christusev} : \emph{Christusov}), indicating the pronunciation [Jezuʃ] (or after the trans-syllabic consonant assimilation [Jeʒuʃ] and [Kristuʃ/Kriʃtuʃ]). Such pronunciation seems to be confirmed by rare variants of the name \emph{Iesu\textbf{sh}}, \emph{Je\textbf{sh}u\textbf{sh}} and the possessive adjective \emph{Kristu\textbf{sh}ev}, as well as the forms \emph{Kri\textbf{sh}tus}, \emph{Chri\textbf{sh}tusev} which retain the prevalent final <s> but the change \emph{s} > ʃ in the middle of the word indicates trans-syllabic consonant assimilation and therefore post-alveolar pronunciation of the final consonant. However, rare examples of possessive adjectives with the suffix \emph{-ev} used after non-palatal consonants (e.g., \emph{Aristobulev} from \emph{Aristobulus}) make it impossible to confirm Jakopin’s thesis. 

So far, the discussion focused on nominative forms which could be transferred from Latin to Slovenian without the change of the ending. But because Slovenian is an inflection-rich language, it requires distinctive endings in the oblique cases and therefore morphological adaptation of proper names is almost obligatory to enable them to fully function in texts.  Some transfers of Latin names in oblique cases are attested in the analysed material, but they are extremely limited: in some texts, Latin oblique forms were used when referring to various Bible books (6) or, in Dalmatin’s Bible, oblique forms of toponyms were transferred from Latin (7, 8); in earlier Trubar’s translation such toponyms were morphologically adapted.

\begin{itemize}
    \item[(6)] Sato tudi Chriſtus gouori, \textbf{Marci} na 9. cap. Vſe rizhi ſo mogozhe, temu ker Veruie. (Juričič, \emph{POSTIILLA},  1578)
    
    ‘So says Christ in Mark’s Gospel, chapter 9,  all things \emph{are} possible to him that believeth’
    \item[(7)] \textbf{Trubar 1557:} Inu od Mileta poshle on uto \textbf{Efeſo}

    \textbf{Dalmatin 1584:} OD Mileta pak je on poſlal v'\textbf{Epheſum}

    \textbf{Luther:} ABer von Mileto sandte er gen \textbf{Ephesum}

    \textbf{Vulgate:} a Mileto autem mittens \textbf{Ephesum}
    
    ‘And from Miletus he sent to Ephesus.’

    \item[(8)] \textbf{Trubar 1557:} Ty Sholnery pag, koker ie nim bilu naloshenu, uſamo Pauluſa, inu ga pelaio po nozhi uto \textbf{Antipatrido}
    
    \textbf{Dalmatin 1584:} Ty Sholnerji ſo vseli Paula, kakòr je nym bilu porozhenu, inu ſo ga po nozhi pelali v'\textbf{Antipatriden}.
    
    \textbf{Luther:} DJe Kriegsknechte / wie jnen befolhen war / namen Paulum / vnd füreten jn hin bey der nacht gen \textbf{Antipatriden}
    
    \textbf{Vulgate:} milites ergo secundum praeceptum sibi adsumentes Paulum duxerunt per noctem in \textbf{Antipatridem}
\end{itemize}

There are also some cases of the transfer of oblique forms of rarely used names of Latin origin in non-Biblical texts, e.g. for names of pagan deities. 

\begin{itemize}
    \item[(9)] TY Babilonery ſo imeili nih Boga \textbf{Belum}. Egiptary \textbf{Iſidem} inu \textbf{Oſyrin}. Affri \textbf{Neptunum}, Rodiſary Sonce. Ta Meiſta Samos \textbf{Iunonem}, Pafos \textbf{Venerem} Delfos \textbf{Apollinem}, Rym \textbf{Quirinum}, Atene \textbf{Mineruam}, Epheſus \textbf{Dianam} (Trubar, \emph{CATEHISMVS SDVEIMA ISLAGAMA}, 1575)
    
    ‘The Babilonians had their God Baal. The Egyptians had Isis and Osiris. The Africans(?) had Neptun, the Rhodians had the Sun. The city of Samos had Iuno, Paphos had Venus, Delphi had Appolo, Rome had Quirinus, the Athens had Minerva, the Ephesus had Diana.’

\end{itemize}

In rare cases, the degree of morphological adaptation of a name depended on its referent: thus for Thomas the Apostle an exonym \emph{Thomash} (and rarely an exograph \emph{Tomas}) was used and inflected according to the rules of Slovenian morphology, while for (rarely mentioned) Saint Thomas of Aquinas, the Latin vocative form \emph{Thoma} was used in nominative and locative cases.

In general, masculine proper names of Latin origin formed oblique forms from the same base as in Latin but some retained Latin nominative suffixes as a part of a Slovenian base word.  For some names, such forms were more common than for others: e.g., \emph{Ioannes} (< Lat. \emph{Iohannes}) always retains its final -es in oblique cases (Gen \emph{Ioannesa}, Dat \emph{Ioannesu}); \emph{Paulus} forms the genitive form with Latin nominative suffix much more frequently than \emph{Petrus}: while the genitive form \emph{Petrusa} occurs only in two works, genitive form \emph{Paulusa} as a variant to the form \emph{Paula} can be found in various texts, including (Bible) translations, sermons, catechisms, hymns, etc. Also, some authors used the longer forms more frequently than others. 

\begin{itemize}
    \item[(10)] \textbf{Trubar 1557:} Ampag ty Iudi ſo gori ſdrashili ene andohtiue inu poshtene shene, inu te Viſshe tiga Meiſta, inu ſo obudili enu preganene zhes \textbf{Pauluſa}\textsubscript{ACC} inu \textbf{Barnabaſa}\textsubscript{ACC}, inu ſo nyu uunkai is shnih krayeu iſpahnili
    
    \textbf{Dalmatin 1584:} Iudje pak ſo nadraſtili te andohtlive inu poſhtene Shene, inu Viſhe tiga Méſta, inu ſo enu preganenje obudili zhes \textbf{Paula} inu \textbf{Barnaba}, inu ſo nyu vun is ſvoih pokrain pahnili.
    
    \textbf{Luther:} Aber die Jüden bewegeten die andechtigen vnd erbarn Weiber / vnd der stad Obersten / vnd erweckten eine verfolgunge vber \textbf{Paulum} vnd \textbf{Barnaban} / vnd stiessen sie zu jren Grentzen hinaus.
    
    \textbf{Vulgate:} Iudaei autem concitaverunt religiosas mulieres et honestas et primos civitatis et excitaverunt persecutionem in \textbf{Paulum} et \textbf{Barnaban} et eiecerunt eos de finibus suis

\end{itemize}

In some cases, both types of declension can be found in the same passage:

\begin{itemize}
    \item[(11)] \textbf{Petruſ}\textsubscript{NOM} pak ta zhaſ ſtoij sûnai pred vratmi, Nato gre ta drugi Ioger, kijr ie Velikimufariu snan bil, inu govori s'vratarico, inu pelia \textbf{Petruſa}\textsubscript{ACC} notàr. [Vratarica] Pravi k'\textbf{Petru}\textsubscript{DAT}: Néſili tudi tiga Zhloveka mlaiſhih edan? (Krelj, \emph{Postilla Slovenska}, 1567)
    
    ‘But Peter stood at the door without. Then went out that other disciple, which was known unto the high priest, and spake unto her that kept the door, and brought in Peter. Then saith the damsel that kept the door unto Peter, Art not thou also \emph{one} of this man's disciples?’
\end{itemize}

On the other hand, genitive \emph{Marcusa/Markusa} < \emph{Marcus/Markus} is not attested, oblique forms of the name are only formed from Latin base form (Gen \emph{Marca/Marka}, Dat \emph{Marcu/Marku}). 

As the base form for transferred Latin nominative forms and their Slovenian exonyms are often the same (\emph{Petrus} > Gen \emph{Petra}; \emph{Peter} > Gen \emph{Petra}), it is often impossible to determine  to which nominative form oblique cases should be attributed if more than one is attested. An example of this is the name \emph{Andreas} or \emph{Andrej} ‘Andrew’. In nominative the Latin nominal form Andreas is the most common (see Table 1; numbers in brackets indicate number of occurrences in the 16th-century Slovenian). The adapted nominal form is \emph{Andrei}, which also occurs in phonologically reduced form \emph{Andri}. In one case, a contaminated form \emph{Andreias} appears. In oblique cases the root without suffix -as, typical for Latin and for Slovenian, is the norm, with only one occurrence of the extended base \emph{Andreasa} in the accusative. Therefore, it is difficult to determine whether those oblique cases should be counted as occurrences of Andreas or Andrei.

\begin{table}
\caption {The declension of `Andrew’ in Latin and Slovenian.}
\label{tab:jelovsek:frequencies1}
\small
\begin{tabularx}{\linewidth}{QllQQ}
\lsptoprule
\multicolumn{2}{c}{\textbf{Latin declension}}  & \multicolumn{2}{c}{\textbf{16th-century Slovenian forms}} & \textbf{Modern Slovenian declension} \\
\cmidrule(lr){1-2}\cmidrule(lr){3-4}\cmidrule(lr){5-5}
\textbf{Case} & Singular	& Case & Singular &  \\
\midrule
{Nominative} & Andrē̆ās	& Nominative & Andreas (33) / Andrei (10)/ Andri (3) / Andreias (1)  & Andrej \\
{Genitive} & Andrē̆ae	& Genitive & Andrea (12) / Andreja (1) & Andreja\\
{Dative} & Andrē̆ae & Dative & Andreiu (3) / Andreju (2) & Andreju\\
{Accusative} & Andrē̆am Andrē̆ān  & Accusative & Andrea (19) & Andreja\\
{Ablative} & Andrē̆ā & Locative & Andreiu & Andreju\\
{Vocative} & Andrē̆ā & Instrumental & Andreom (2) / Andrejom (1) & Andrejem\\
\lspbottomrule
\end{tabularx}
\end{table}

One criterion for making the distinction could be the orthographic closing of the hiatus with <j> or its spelling variant <i>, but such closing of hiatus is a matter of orthographical convention and not phonological variation (it is always orthographically closed for \emph{eu}, as can be seen in the dative and the locative, but not necessarily for \emph{ea} and \emph{eo}, while the pronunciation of all vocal clusters is identical – the hiatus is closed). 

\largerpage
Also, same oblique forms appear in the same context as different nominative forms.

\begin{itemize}
    \item[(12)] Filipus pride inu pouei \textbf{Andreiu}\textsubscript{DAT}, \textbf{Andreas}\textsubscript{NOM} ſpet inu Filipus poueiſta Ieſuſu (Trubar, \emph{TA PERVI DEIL TIGA NOVIGA TESTAMENTA}, 1557)
    
    Filipus pride inu pouei \textbf{Andreiu}\textsubscript{DAT}, \textbf{Andreias}\textsubscript{NOM} ſpet inu Filipus poueiſta Ieſuſu (Trubar, \emph{TA CELI NOVI TESTAMENT}, 1581--82)

    ‘Philip cometh and telleth Andrew: and again Andrew and Philip tell Jesus.’

    \item[(13)] Kadar je pak Iesus raven Galilejſkiga Morja hodil, je sagledal dva Brata, Simona, kateri je imenovan Petrus, inu \textbf{Andrea}\textsubscript{ACC}, njegoviga Brata, ta ſta ſvoje Mréshe v'Murje metala: Sakaj ona ſta bila Ribizha <Peter inu S. \textbf{Andri}\textsubscript{NOM} k'Apoſtolom poklizana> (Dalmatin, \emph{BIBLIA}, 1584)
    
    ‘And Jesus, walking by the sea of Galilee, saw two brethren, Simon called Peter, and Andrew his brother, casting a net into the sea: for they were fishers <Peter and St. Andrew called to be the Apostles>.

    \item[(14)] Kerſtnik perpravi \textbf{Andrea}\textsubscript{ACC} hChriſtuſu, \textbf{Andreas}\textsubscript{NOM} Petra (Trubar, \emph{HISHNA POSTILLA}, 1595)
    
    ‘The Baptist sends Andrew to Christ, Andrew brings Peter.’

\end{itemize}

Another morphological variation occurs in toponyms. As, unlike the majority of anthroponyms, they do not have a natural gender, a grammatical gender had to be assigned to them in Slovenian for them to be included in the appropriate declension and for the purposes of syntactic agreement. For several toponyms, variant gender assignment is attested, either according to the general tendencies of gender assignment for Latin words in Slovenian based on their form \citep{Ahacic2011} or  according to their Latin gender. In nominative, they could either retain the Latin ending or an exonym without the Latin ending or – for neuter and feminine gender  – an exonym with the appropriate Slovenian ending was used: e.g., for Lat. \emph{Corinthus}\textsubscript{f} exograph \emph{Corintus}\textsubscript{m/n?} as well as exonyms \emph{Corint}/\emph{Korint}/\emph{Korinth}\textsubscript{m} were used, as the Latin words ending in \emph{-us} were generally assigned masculine gender when borrowed into Slovenian \citep[116]{Ahacic2011}, but the feminine accusative form \emph{Corinto}\textsubscript{f} is also attested, as well as locative form \emph{Corintij} which could be interpreted as a feminine form with an added suffix \emph{–ija} (\emph{Corintija}\textsubscript{NOM}, \emph{Corintiji}\textsubscript{LOC}), but as the spelling for /i/ and /j/ and their combinations was highly irregular in the 16th-century Slovenian literary language (the combination <ij> could stand for /i/, /j/, /ij/, /ji/ and /iji/ (\citealt{Jelovsek2020}: 293), it is impossible to reliably determine the base word. In Trubar's works, we also find sporadic cases in the neuter gender, probably under the influence of the gender of the noun \emph{mesto} ‘town, city’: \emph{Corintu\textsubscript{n}}.  

Similarly, \emph{Jerusalem} (which was adopted from its German and not Latin form) could function as a masculine noun, as it ends in a consonant (\citealt{Ahacic2011}: 116), but also as a feminine or neuter noun (both genders were attested in Latin); unlike \emph{Korinth}, when feminine or neuter, \emph{Jerusalem} did not acquire an appropriate case ending in nominative; the gender is determined by the verb form or the definite article used.

\begin{itemize}
    \item[(15)] v'ta nebeſki Ierusalem (Krelj, \emph{POSTILLA SLOVENSKA}, 1567) 
    
    to the A\textsubscript{CC.SG.MASC}. heavenly A\textsubscript{CC.SG.MASC}. Ierusalem\textsubscript{ACC.SG}.
    
    „to the heavenly Ierusalem”

    \item[(16)] \emph{Ierusalem puſta leshy} (Dalmatin, \emph{BIBLIA}, 1584) 
    
    Ierusalem\textsubscript{NOM.SG}. bare\textsubscript{NOMSG.MASC}. lies

    „Ierusalem lies bare”

    \item[(17)] \emph{terpi tu Ieruſalem inu tui Folk leto ſramoto} (Trubar, ENA DVHOVSKA PEISSEN SVBPER TVRKE, 1567)
    
    suffers\textsubscript{3RD.SG}. the\textsubscript{NOM.SG.NEUT}. Ierusalem\textsubscript{NOM.SG}. and your people this shame
    
    „Ierusalem and your people suffers this shame”

\end{itemize}

\subsection{Names of Germanic origin}

For originally Germanic names, determining the exact type of transfer strategy is often much more difficult, as the majority of them were adopted into Slovenian from various dialects and chronolects of German, in which case the source and the target language were not only in direct geographical contact but German played the role of the prestige language in the Slovenian lands for centuries \parencites[17--20]{Ahacic2014}[590--591]{Javor2012}; furthermore, it was itself undergoing intense dialectal development in the centuries before the establishment of the Slovenian literary language (cf. \citealt{Sekli2020}). In the 16th century, Slovenian authors were not only acquainted with the newly established Luther’s literary norm but they were also in direct contact with a variety of German dialects in the southern parts of the Holy Roman Empire. Therefore, it is difficult to determine whether a nominative form of a proper name is a true exonym, phonologically as well as orthographically distinct from the source form, or merely an exograph, orthographically modified to better reflect original  (regional) German pronunciation in the period when the form entered the Slovenian language. For example, the locative form \emph{Vshalspurgi} ‘in Salzburg’ could, at least theoretically, reflect Bavarian post-alveolar pronunciation of the Middle High German \emph{s as} [ṡ/ż], typical for the period 750–1500, and the development of Bavarian Old High German \emph{b}- > \emph{p}- (appearing around 770) \citep[10]{Sekli2020}, while the form \emph{Salzburg} could be borrowed from the later literary form of the name; as we see from the first variant, which appears in the oblique case, a morphological adaptation (addition of Slovenian case endings) was also required in majority of cases. 

On the other hand, rare adaptations from non-German Germanic names, such as \emph{Viklef} for English Wycliffe, could be confidently labelled as exographs, replacing the foreign orthography in a way to reflect original pronunciation. 

The orthographical adaptations were the dropping of the original silent <h> and the replacement of the <ch> with <h>; in cases where it was not phonologically adapted to /b/, German <w> could be either retained or orthographically replaced by <vv>, <vu> or <v>:  \emph{Vuilelm} (< \emph{Wilhelm}), \emph{Vvolfgang}/\emph{Vuolfgang} (< \emph{Wolfgang}) alongside \emph{Bolfgang}

The most common phonological adaptations attested in names adapted from German were the palatalization of alveolar sibilants (similar as in names of Latin origin), the de-voicing of voiced stops (or, less frequently, voicing of the aspirated voiceless stops) and the sporadic replacement of German /w/ with /b/, as well as the simplification of consonant clusters. In some cases, vowel rounding also occurred, while German rounded vowels /œ/ and /ʏ/ were commonly replaced with /e/ and /i/. For illustration, the exonyms and exographs of German toponyms \emph{Wittemberg} and \emph{Augspurg} are \emph{Wittemberg}, \emph{Witeberg}, \emph{Biteberg}, \emph{Bitemberg}; \emph{Augspurg}, \emph{Augshpurg}, \emph{Auspurg}, \emph{Aushpurg}; some of less frequent, but interesting examples are \emph{Ludvih}/\emph{Ludbih} (< \emph{Ludwig}), \emph{Hiltepront} (< \emph{Hildebrand}) and \emph{Tonaberd} (< \emph{Donauwörth}). 

For some anthroponyms, forms influenced by the Latinized forms of the name can also be found in the analysed material: \emph{Ludovik/Ludovic} (< Lat. \emph{Ludovicus}, Ger. \emph{Ludwig}) ‘Louis’, \emph{Bernardus} (< Lat. \emph{Bernardus}, Ger. \emph{Bernhard}) alongside \emph{Bernhard} and a contaminated form \emph{Bernhardus}. For German \emph{Karl} ‘Charles’ only the form \emph{Karol} from Latinized \emph{Carolus} is attested.
 
In cases where German and Slovenian exonyms existed for an anthroponym from the shared onomastic heritage (cf. \citealt[98]{Raukko2017}), Slovenian exonym was predominantly used:  \emph{Vyd Djetrih} < Ger. \emph{Veit Dietrich}, \emph{Iansh Brencius} < Ger. \emph{Johannes Brentz}, Latinized \emph{Brentius}.

In rare cases, a partial or a full translation of German proper names is also attested. The name \emph{Felix od Barteperg} < Ger. Felix von Wartenberg ‘lit. Felix from Wartenberg’ could be understood as an example of a partial translation, in which only the nobility particle \emph{von} was translated with the Slovenian preposition \emph{od}, while the family name was adapted phonologically, but not morphologically (the full translation would require the family name in the genitive case after the preposition \emph{od}). 

For rare frequently used and historically important toponyms, unrelated Slovenian exonyms existed (and still exist), such as \emph{Dunaj} for \emph{Wien} ‘Vienna’, but an exonym \emph{Kostnica} is also attested for the German town of Konstanz which has no direct historical connection to Slovenian lands and was probably a result of morphological re-analysis in Slovenian (cf. \citealt[553]{Nicolaisen1996}), in which German name was reinterpreted as a similarly sounding  but semantically unrelated meaningful Slovenian word \emph{kostnica} ‘ossuary’.

\subsection{Names of Semitic origin}

Names of Semitic origin, especially from the Old Testament, which did not become a part of the Christian names inventory and also did not acquire typical Latin endings in the Latin or German translations of the Bible, nevertheless show high influence of the intermediary languages – Latin and German – in 16th-century Slovenian texts, as direct transcriptions from Luther’s Bible or the Vulgate were even more frequent than in the other two groups of proper names. Therefore, the attested variation is often the result of the similar variation in the source (inconsistent adaptation of Semitic names is typical for both Luther’s Bible and the Vulgate; cf. \citet[352ff]{Krasovec2006}. 

\begin{itemize}
    \item[] \textbf{Dalmatin 1584}: Osnanite v'Egypti, povéjte tu v'\textbf{Migdali}
    
    \textbf{Luther:} Verkündigt in Egypten / vnd sagts an zu \textbf{Migdal}
    
    \textbf{Vulgate:} adnuntiate Aegypto et auditum facite \textbf{Magdolo}
    
    ‘Declare ye in Egypt, and publish in Migdol.’

    \item[] \textbf{Dalmatin 1584:} Od Etame ſo ſhli, inu ſo oſtali v'ti dolini Ahirot, katera leshy pruti BaalZephonu, inu ſo legli pruti \textbf{Migdoli}.
    
    \textbf{Luther:} Von Etham zogen sie aus / vnd blieben im grund Hahiroth / welchs ligt gegen Baal Zephon / vnd lagerten sich gegen \textbf{Migdol}

    \textbf{Vulgate:} inde egressi venerunt contra Phiahiroth quae respicit Beelsephon et castrametati sunt ante \textbf{Magdolum}

    ‘And they removed from Etham, and turned again unto Pihahiroth, which \emph{is} before Baalzephon: and they pitched before Migdol.’

    \item[] \textbf{Dalmatin 1584:} Taku pravi \textbf{Sanherib}, Aſſyrerſki Krajl
    
    \textbf{Luther:} so spricht \textbf{Sanherib} der könig zu Assur 
    
    \textbf{Vulgate:} haec dicit \textbf{Sennacherib} rex Assyriorum
    
    ‘Thus saith Sennacherib king of Assyria’

    \item[] \textbf{Dalmatin 1584:} Koku je \textbf{Senaherib}, inu ſtu inu pet inu oſſemdeſſet taushent Mosh shnym red, v'eni nozhi bilu konzhanih.
    
    \textbf{Luther:} Da dich die Boten des königes \textbf{Sennaherib} lesterten / schicktest du einen Engel / der schlug tod / hundert vnd fünff vnd achzig tausent man.
    
    \textbf{Vulgate:} et sub \textbf{Sennacherim} centum octoginta quinque milia ut perierunt
    
    ‘how they were delivered, when under Sennacherib an hundred fourscore and five thousand perished’

\end{itemize}

As can be seen from the examples, Dalmatin made similar orthographical adaptations for Semitic names as for Latin(ized) names (ommiting the silent \emph{h}, replacing the aspirated stops etc.). 

Besides frequent vowel variation, we can also find some typical variation in consonants, such as final -\emph{n} /-\emph{m}, which was not necessarily a result of a variation in source texts: e.g. \emph{Absalom} – \emph{Absalon} – \emph{Absolom} – \emph{Absolon}. 

Some variation in Semitic proper names in the 16th-century Slovenian arose because the authors used different source translations of the Bible. While Dalmatin in his translation of the Old Testament closely followed Luther \citep[3]{Krasovec2010}, other authors, when writing of persons from the Old Testament in their non-Biblical texts, often used the name forms from the Vulgate. The best example for this is the name of the king Nebuchadnezzar with thirteen variants (see Table 2). This name also illustrates other typical variations in Semitic proper names, such as the trans-syllabic vowel and consonant assimilation, as well as the variation between /z/ (spelled with <s>), when based on the Latin form, and /c/ (spelled with <z>), when based on the German form, and the variation between the final \emph{-er} and \emph{-ar} which is not limited to proper names but is also typical for common nouns borrowed from German into Slovenian at the period \citep[141]{Legan_ravnikar2012}. On the other hand, two occurrences of \emph{NebudNezar} with a missing syllable are probably a mistake.

\begin{table}
\caption {Forms of the name \textit{Nebuchadnezzar} in the 16th-century Slovenian.}
\label{tab:jelovsek:frequencies2}
\begin{tabular}{ll}
\lsptoprule
{Via German: \emph{NebucadNezar}} & {Via Latin: \emph{Nabuchodonosor}} \\
\midrule
\emph{\textbf{N}a\textbf{bu}k\textbf{ad}n\textbf{ec}ar} & \emph{Nabugdnosor} \\
\emph{\textbf{N}a\textbf{bu}k\textbf{ad}n\textbf{ec}er} & \emph{Nabuhodonosar} \\
\emph{\textbf{N}a\textbf{bu}k\textbf{ad}r\textbf{ec}ar} & \emph{Nabuhodonosor} \\
\emph{\textbf{N}e\textbf{bud}n\textbf{e}car}            & \emph{Nebuhodnosor} \\
\emph{\textbf{N}e\textbf{bu}k\textbf{ad}n\textbf{ec}ar} & Nubugdonosor \\
\emph{\textbf{N}e\textbf{bu}k\textbf{ad}n\textbf{ec}er} &  \\
\emph{\textbf{N}e\textbf{bu}k\textbf{ad}r\textbf{ec}ar} &  \\
\emph{\textbf{N}e\textbf{bu}k\textbf{ad}r\textbf{ec}er} &  \\
\lspbottomrule
\end{tabular}
\end{table}

\section{Conclusion}

The analysis of the methods of incorporation of proper names into the 16th-century Slovenian literary language has shown some differences based on the origin of the name: Latin and Latinized names were more often transferred than Germanic names, who underwent more extensive phonological adaptation that was reflected also in their orthography, or their orthography was changes to more accurately reflect their original phonological form in Slovenian; as only written sources can be studied, it is hard to determine whether Latin names also underwent phonological adaptation which was not reflected in their orthographical form, due to the prevailing influence of Latin orthography, as Latin was prized above all other languages in intellectual circles \citep[23]{Ahacic2014}. For Semitic names, high influence of source translations is attested.

The analysis has also shown the high level of prestige the Bible text had among the translators, who predominantly kept the original forms of proper names in their Bible translations despite using existing exonyms in their other texts. The changes made in the entire translation of the Bible in 1584, which decreased the level of (orthographical and morphological) adaptation of some proper names compared to previous translations, and brought them again closer to German and Latin forms, further corroborate the prestigious status of the source translations, especially Luther’s Bible 1545.

In some aspects, the findings of the analysis agree with the findings of the previous research on linguistic borrowing of semantic words (e.g. \citealt{Striedter-temps1963}  and \citealt{Legan_ravnikar2012}  for German; \citealt{Ahacic2011} for Latin); proper names underwent some similar orthographical and phonological adaptations as semantic words, but due to the their specific features, they were much more frequently transferred (even if in a form of an exograph) from the source language  than semantic words; they also exhibit some special adaptations in terms of morphology and a higher degree of variation. The findings of the study therefore provide further insight into issues of language borrowing and adaptation of loanwords, enriching the findings of Slovenian contact linguistics by encompassing heretofore rarely considered material.

\section*{Acknowledgements}
The article was written in the context of the research programme IMAGE – WORD – KNOWLEDGE. The Transmission and Transformation of Ideas on the Territory Between the Eastern Alps and the North Adriatic 1400--1800 (P6-0437). The author acknowledges the project was financially supported by the Slovenian Research Agency from the state budget. 


{\sloppy\printbibliography[heading=subbibliography,notkeyword=this]}
\end{document}
