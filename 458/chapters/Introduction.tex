\documentclass[output=paper]{langscibook}
\ChapterDOI{10.5281/zenodo.15394469}
\author{Geoffrey Williams\affiliation{Université de Bretagne Sud; Université Grenoble Alpes}}
\title{Introduction}

\abstract{This work consists of a series of papers written following the International Conference for Historical Lexicography and Lexicology (ISHLL) held in Lorient, France in May 2022. The selection is in two languages -- English and French, the language of the host country -- but is organised thematically so as to underline content. The papers illustrate the vitality of this field by analysing historical dictionaries that reflect languages from across the world.}

\IfFileExists{../localcommands.tex}{
  \addbibresource{../localbibliography.bib}
  % add all extra packages you need to load to this file

\usepackage{tabularx,multicol}
\usepackage{url}
\urlstyle{same}

\usepackage{listings}
\lstset{basicstyle=\ttfamily,tabsize=2,breaklines=true}

\usepackage{langsci-basic}
\usepackage{langsci-optional}
\usepackage{langsci-lgr}
\usepackage{langsci-osl}
% \usepackage{./langsci/styles/langsci-lgr}
% \usepackage{./langsci/styles/langsci-osl}
% \usepackage{langsci-gb4e}

\usepackage{tikz}
\usetikzlibrary{patterns,calc}
\pgfdeclarepatternformonly{south east lines}{\pgfqpoint{-0pt}{-0pt}}{\pgfqpoint{3pt}{3pt}}{\pgfqpoint{3pt}{3pt}}{
    \pgfsetlinewidth{0.6pt}
    \pgfpathmoveto{\pgfqpoint{0pt}{3pt}}
    \pgfpathlineto{\pgfqpoint{3pt}{0pt}}
    \pgfpathmoveto{\pgfqpoint{.2pt}{-.2pt}}
    \pgfpathlineto{\pgfqpoint{-.2pt}{.2pt}}
    \pgfpathmoveto{\pgfqpoint{3.2pt}{2.8pt}}
    \pgfpathlineto{\pgfqpoint{2.8pt}{3.2pt}}
    \pgfusepath{stroke}}
    
\usepackage{stmaryrd}
\usepackage{wasysym}
\usepackage{multirow}
\usepackage{caption}
\usepackage{subcaption}
\usepackage{mathrsfs}
\usepackage{qtree}

\usepackage{linguex}


  %pminos do not split footnotes
% \interfootnotelinepenalty=10000 %Footnote in Laporte chapters has to be split SN


%\DeclareIndexNameFormat{default}{%
%\nameparts{#1}%
%\usebibmacro{index:name}%
%{\index[names]}%
%{\namepartfamily}%
%{\namepartgiveni}%
% {}% L1
% {}% L2
%{\namepartprefix}% generates spurious space L3
%{\namepartsuffix}% generates spurious space L4
%}

%  {\DeclareIndexNameFormat{default}{%
%     \usebibmacro{index:name}{\index[names]}{#1}{#3}{#5}{#7}}}

%\DeclareIndexNameFormat{default}{%
%  \usebibmacro{index:name}{\sindex[nom]}{#1}{#3}{#5}{#7}}

%\DeclareIndexNameFormat{default}{%
%  \usebibmacro{index:name}{\sindex[person]}{#1}{#3}{#5}{#7}}
%\DeclareIndexNameFormat{default}{%
%\nameparts{#1} \usebibmacro{index:name}{\sindex[person]]}{\namepartfamily}{‌​\namepartgiven}{\nam‌​epartprefix}{\namepa‌​rtsuffix}}

%\newcommand{\smiley}{:)}

%\renewbibmacro*{index:name}[5]{%
%\usebibmacro{index:entry}{#1}%
%{\iffieldundef{usera}{}{\thefield{usera}\actualoperator}\mkbibindexname{#2}{#3}{#4}{#5}}}

% \newcommand{\noop}[1]{}

%remove for final
%\overfullrule=1mm

\newcommand{\tobi}[2]}}
\renewcommand{\S}[1]{\tobi{#1}{\textsc{*}}}

% this volume references
% puts: [this volume]
% already defined: \citetv
%\newcommand{\citepv}[1]{(\citeauthor{#1} \citeyear*{#1} [this volume])}
\newcommand{\citealtv}[1]{\citeauthor{#1} \citeyear*{#1} [this volume]}

%parentheses around example number
\newcommand{\pref}[1]{(\ref{#1})}

% in-text examples

\newcommand{\lnex}[1]{\textit{#1}} %target lang word
\newcommand{\lnlit}[1]{(lit.: `#1')} %literal reading
\newcommand{\lnlat}[1]{(#1)} % latinization
\newcommand{\lntrans}[1]{`#1'} %translation
\newcommand{\lnexl}[2]%
{\lnex{#1}{} \lnlat{#2}} % ex with latinization
\newcommand{\lnexlat}[3]{\lnex{#1}{} \lnlat{#2}{} \lntrans{#3}} % ex with latinization and tranl.

%ch01
\newcommand{\co}[1]{\mbox{\textbf{#1}}}

%ch09

\newcommand{\cyrbulg}[1]{\begin{otherlanguage*}{bulgarian}#1\end{otherlanguage*}}


%ch10
\newcommand{\nlp}{{\small NLP}}
\newcommand{\mwe}{{\small MWE}}
\newcommand{\rae}{{\small RAE}}
\newcommand{\lvc}{{\small LVC}}
\newcommand{\pos}{{\small P}o{\small S}}
%\newcommand{\todo}[1]{ \textcolor{red}{#1} }

%\renewcommand{\labelenumi}{\theenumi}
%\ainamefmt{{vv}{ll}{, ff}{, jj}} % fullname

\newcommand{\biberror}[1]{{\color{red}#1}}

\newcommand{\osenovaitem}{--~}
  %% hyphenation points for line breaks
%% Normally, automatic hyphenation in LaTeX is very good
%% If a word is mis-hyphenated, add it to this file
%%
%% add information to TeX file before \begin{document} with:
%% %% hyphenation points for line breaks
%% Normally, automatic hyphenation in LaTeX is very good
%% If a word is mis-hyphenated, add it to this file
%%
%% add information to TeX file before \begin{document} with:
%% %% hyphenation points for line breaks
%% Normally, automatic hyphenation in LaTeX is very good
%% If a word is mis-hyphenated, add it to this file
%%
%% add information to TeX file before \begin{document} with:
%% \include{localhyphenation}
\hyphenation{
    Beck-man
    Ngu-yen
    back-chan-nel
    back-chan-nels
    mo-not-o-nous
    ste-reo-typ-i-cal
}

\hyphenation{
    Beck-man
    Ngu-yen
    back-chan-nel
    back-chan-nels
    mo-not-o-nous
    ste-reo-typ-i-cal
}

\hyphenation{
    Beck-man
    Ngu-yen
    back-chan-nel
    back-chan-nels
    mo-not-o-nous
    ste-reo-typ-i-cal
}

  \togglepaper[1]%%chapternumber
}{}

\begin{document}
\maketitle
\noindent
As we all know, lexicography in its many forms is a very old practical discipline solving practical problems concerning word usage. “Word" seems more appropriate than “language" as lexicography addresses more questions relating to what we now call lexicology than other issues. As with all areas of human endeavour, what grew gradually through trial and lots of error is eventually subjected to a theoretical framework. The role of historical lexicography is to look back on the development of these highly varied word lists to see how we reached the tremendous variety that characterises practice throughout the world. One issue is the word “dictionary” as this in fact severely reduces our scope of study, and this is why we insist on the sister discipline of historical lexicography as the analysis of word\hyp usage and the description of words in formalised reference works are totally intertwined.

This volume is both a selection of papers from one conference on historical lexicography and lexicology held under the aegis of the International Society for Historical Lexicography and Lexicology (ISHLL) in Lorient, France in May 2022  and also the first in a new  series dedicated to the area and which represents a collaboration between two sister associations ISHLL and HEL-Lex.

Our current vision of what we now term “dictionaries" is a relatively modern phenomenon developing largely from the late 16th century with major works such as Robert \citegen{Estienne1539} \textit{Dictionnaire françois-latin} being called a “dictionary”, whereas he had named his previous work a “thesaurus”. As late as 1604, Robert Cawdrey called his work \citetitle{Cawdrey1604} (\citeyear{Cawdrey1604}), emphasizing simply the order in which the words were given, whilst at the same period the Accademia della Crusca were calling their far more consequential work a \emph{Vocabolario} (\citealt{Accademia_della_crusca1612}). These are the origins of the European tradition of lexicography with the different terms gradually becoming stabilized as pointing to different objects. However, the use of such essential reference works is far older and can be traced through all written civilisations, notably China and India (\citealt{YongPeng2008}). Whilst the chapters in this volume essentially deal with European style dictionaries, both monolingual and bi- or multilingual, this international perspective must be borne in mind.

The papers published here seek to demonstrate a world perspective in the array of languages covered, moving quite literally from west to east. This is a selection of papers presented at the \textit{12th International Conference of Historical Lexicography and Lexicology} held in Lorient, France, in 2022 under the patronage of the \textit{Académie française} and with the assistance of the French Ministry of Culture. This is a collective work with the chapters being rewritten and expanded, reflecting the interaction and dynamism of the event. Being held in France, the languages of the conference were English and French, two internationally used \textit{linguae francae}, with French being very much the dominating language in the 18th century, a fact reflected in the fact that three of the chapters concern 18th century French lexicographical works. The event was also sponsored by the BasNum project\footnote{\url{https://anr.fr/Projet-ANR-18-CE38-0003}} based at the \textit{Université Grenoble Alpes} that is digitising the major encyclopaedic work of Antoine Furetière, the \textit{Dictionnaire universel} in its 1701 edition. 


The theme of this event was “West meets East” with the aim of looking at historical dictionaries right across Europe from Portugal and Great Britain in the West to Georgia in the East and in global terms from across the world. 

The theme is also a play on place names as the named city in which it was held is literally “The Orient”, Lorient. Lorient was founded \textit{ex nihilo} in 1666 as a port to build ships for the French East Indies Company, la \textit{Compagnie des Indes}. At this time, the bay had no coastal town, just dispersed fishing villages. The first ship built was the Soleil de l'Orient and the workers gradually became referred to as those from the l’Orient with the name sticking so that the future parish and burgeoning town adopted this name. Lorient is thus a city open to the world with strong international connection making it an ideal location to celebrate the East.

The chapters represent a linguistic diversity through a variety of language combinations. Ordering them thematically is clearly a necessary impossibility in that the parameters are numerous. We have decided to order first by monolingual and bilingual and then to regroup by relatedness in the themes. However, it must be noted that whilst a thematic grouping has been attempted, we have deliberately decided to mix papers published in the two languages of the event: French and English. 

If dictionaries are tools for the understanding and diffusion of language, then multilingualism as expressed through the Helsinki Initiative\footnote{\url{https://www.helsinki-initiative.org}} is an important issue. The language of lexicography is not English, but English is just one of many vehicles for discussing research related to dictionaries. This is why this work has been compiled according to themes rather than language and why we have not created artificial sections by language. As in any international event, the majority of papers tend to be presented in English, whatever the native language of the writer. Thus, we consider that, as English and French are both international languages, it is not necessary to group productions. It is also important to stress that texts in French are not simply those of French native speakers, but scholars who work in this language in the same way as other non-native speakers adopt English. True multilingualism is not about privileging one language above another, but promoting variety and choice. However, the abstracts are systematically in English so that a wider audience can access introductory information.

The first paper is that of Lynda Mugglestone \textit{On Closure and its Challenges: Examining the Editors’ Proofs of OED1}. This chapter represents an entirely new look at the genesis of the \textit{Oxford English Dictionary} using recently discovered the final Editors' Proofs that show the process of closure in detail. Not only do these show often amusing details as to fixing the text, they shed a great deal of light on the finalising of this important text in the pre-digital hot-press era. This clearly shows the trials and tribulations of putting a printed work to press with the question of access to historical data in the digital age. 

In \textit{Dictionaries in the Web of Alexandria: On the Dangerous Fragility of Digital Publication}, Daphne Preston-Kendal discusses long-term access to data in a situation where a work may be constantly updated, but where no paper reference edition exists. This “web of Alexandria” highlights the complexities of digital management and citing entries which are not necessarily as stable over time as a print edition. For example, if we look up any keyword like, say, 'dictionary', in any previous edition of OED, we can give an exact reference, but a dynamic evolving work can see its entries change from one day to the next. It is also a question of long term access as the number of copies of a file is limited to a small number of servers. This may seem a technical issue, and it is, but in a digital world such issues have very major consequences.

Two papers address issues relating to early English dictionaries: \textit{A dictionary of the languages of medieval England: issues and implications} (Gloria Mambelli) and \textit{The Treatment of English High-frequency Verbs in the Promptorium Parvulorum 1440} (Kusujiro Miyoshi). Mambelli tackles the complex issue of triglossia through the analysis of three historical dictionaries of English: the \textit{Middle English Dictionary (MED)}, the \textit{Anglo-Norman Dictionary (AND)}, and the \textit{Dictionary of Medieval Latin from British Sources (DMLBS)}. The discussion looks at how they handle a situation where a literate medieval Englishman had to master his variety of English, Anglo-French, which was the language of the court, and Latin, the language of the church, law and administration. The outcome is that existing historical dictionaries have a highly inconsistent approach to the three language groups and that a more recent work is much needed. Miyoshi tackles a different controversy, the treatment of high-frequency verbs in English. The preeminent place is traditionally awarded to Samuel Johnson, but here it is posited that a wider definition of dictionaries than simply monolingual works should be adopted and that in this case, the first lexicographer to tackle such an issue would be the medieval monk and grammarian Galfredus Grammaticus, the probable compiler of the otherwise anonymous \textit{Promptorium Parvulorium}. Whatever position is adopted, this paper demonstrates the debt we have to early lexicographical endeavours.

Alexander Bocast's chapter \textit{Disattributing the Encyclopédie article on  définition en logique from Jean-Henri Samuel Formey} takes us across to 18th century France and another controversy. In this case, the issue is  whether the attribution of the \textit{définition en loqique}  (definition in logic) in the great Encyclopaedia of d'Alembert and Diderot to Jean-Henri Samuel Formey, a German pastor of French Huguenot origin can be justified. After looking at whether Formey could have written, translated or edited the article, the author concludes that even if Formey did indeed provide material to Diderot it was not his text that was finally used. 

This initial set of chapters discusses works aimed essentially at scholarly audiences, the contribution entitled \textit{Project Cleveland} by Alenka Vrbinc, Donna Farina \& Marjeta Vrbinc looks at the lexicographical production of Slovenian immigrants in the USA in the 20th century. The paper looks at several bilingual Slovenian-English and English-Slovenian dictionaries in order to ascertain their pedagogical role in the rapid language assimilation of this immigrant community. The paper provides a valuable insight into dictionary use, and also into the arrival and integration of Slovenians into the New World. It provides important input as to how dictionaries can be developed to handle the language needs of very precise communities in particular historical contexts. Whilst it is fascinating from a historical perspective, there are also important lessons to be learnt about how immigrant communities achieve integration through language whilst maintaining their own cultural specificity.

Remaining with the Slovenian language, Alenka Jelovšek takes us far further back in time to 16th century Slovenian through \textit{The Incorporation of Proper Nouns of Non-Slavic Origin into the 16th-Century Slovenian Literary Language}. The previous chapter sees Slovenians adapting to a new and emerging cultural environment, but here we see how dictionaries adapt proper nouns from other cultural environments, including different scripts, into the Slovenian language. Nowadays ISNI\footnote{\url{https://isni.org}} assists modern scholars in handling variations across languages, but begs the question as to how they enter a language, which is what is handled in this text through the theory and practice of proper noun integration through the development of a Slovenian literary language. The texts, often religious in character, allow an analysis as to how Semitic, Romance language and German were brought into Slovenian allowing the move from a largely vernacular language to a written literary one.

The chapter entitled \textit{Dictionnaires manuscrits dans l’histoire de la lexicographie croate : Des recueils de mots aux trésors linguistiques et culturels} by Ivana Franić deals with early Croatian manuscript dictionaries. As with the previous chapter, we see the essential role of dictionaries in the evolution of a language through the work of the Dubrovnik circle. The dictionaries discussed are a series of bilingual works which are in a manuscript format which the authors now bring to the attention of a wider public.

By the end of the 17th century, France had already a well established tradition of lexicography, but one hampered throughout the century by the monopoly of the \textit{Académie française}, which forbade the publication of any monolingual dictionary of the French language until, and for twenty years after, the publication of their own dictionary, which took over sixty years to see the light of day. There were two serious breaches of this monopoly, that of Pierre Richelet published in Switzerland in 1680 and that of Antoine Furetière's \textit{Dictionnaire universel} which, after a notorious scandal in which the Academy endeavoured to prevent its publication, was published posthumously in Rotterdam in 1690. The paper \textit{Évaluer la dette : l’étendue de la présence de Richelet dans le Dictionnaire universel de Basnage 1701} by Clarissa Stincone, working within the context of the BasNum project, deals with the second edition of the \textit{Dictionnaire universel} which had been totally revised and updated by Henri Basnage de Beauval in 1701. Working within very tight deadlines, Basnage was a brilliant compiler making use of many specialised works. Stincone posits that Basnage owed a great debt to the second edition of the Richelet for language and grammar related entries to which he failed to give full justice.

If Basnage had recourse to many specialised sources, one of these was Félibien, whose role in the development of terminology relating to art is discussed by Rosa Cetro in her paper \textit{De Félibien à Boutard : l’évolution du dictionnaire artistique entre le XVIIème et le début du XIXème siècle}. This paper charts the development of artistic terminology from the \textit{Des Principes de l’Architecture, de la Sculpture et de la Peinture et des autres Arts qui en dépendent } of Félibien to that of Boutard, the \textit{Dictionnaire des arts du dessin, la peinture, la sculpture, la gravure et l’architecture}. As the author had already digitised the work of Félibien \citep{Cetro2022}, she was able to trace words in the non-digital works based on a sample drawn from the letter C. The evolution of this type of dictionary also reflects changing audiences from a restricted élite to a wider public for whom the history of art is a factor. The challenge facing the author is also the width of the fields covered from architecture through painting and sculpture, which requires looking at the terms from a terminological as well as a lexicographical point of view.

A link to Henri Basnage also appears in the text of Georgios Kassiteridis who looks at \textit{La valeur pragmatique des langues dites « orientales » dans le Dictionnaire universel de Trévoux}. The so-called \textit{Dictionnaire de Trévoux}, a title the Jesuits only accepted much later, was a plagiarised version of Basnage's 1701 \textit{Dictionnaire universel}, which first came out in 1704. Published in the city of Trévoux by an anonymous group of Jesuits, the work only differed from the original by the addition of Latin equivalents and the rewriting or removal of usage that intolerant Catholics might see as heretical because linked to Protestantism. Their initial aim was simply to undermine the market of the Protestant publishers based in the Netherlands. The edition under discussion here is the second one, which was already moving away from being a pale copy to an encyclopaedic work which would later try to attack the great \textit{Encyclopédie} of Didérot and d'Alembert, which the Jesuits also saw as heretical. The work of George Kissiteridis looks particularly at the introduction of Hebrew, Arabic, and Syriac, which at the time were termed “oriental languages”. As the author points out, this 1721 edition demonstrates an increase in religious terminology, but also reflects input from other fields such as geography and botany. However, what is studied here is the large increase in the use of non-European languages. The notable inclusion of Hebrew demonstrates the clear influence of the theologian and bible critic, Richard Simon, who had always tried to hide his involvement in this work. Simon's knowledge of the bible and his mastery of Hebrew and other “oriental” languages, explains an improved etymology of bible-related language and the increased presence of Hebrew.

Agata Pawlina's chapter brings an interesting change of perspective as we move from “oriental” language in a French dictionary to the use of Italian and Greek in a Turkish lexicography, in this case Ottoman Turkish. Turkey has a unique place in Europe in being an interface between cultures of East and West, but also importantly in its covering much of the Roman Byzantine provinces and hence with a strong cultural contact with Italian and Greek. This work is based on a corpus of ten Ottoman Turk dictionaries. Her work is an exciting multidisciplinary study based on a strong knowledge of both music and its terminology and the history of the Turkish language, which allows a full analysis of the socio-cultural aspects. The result is a fascinating insight into Turkish lexicography.

The following chapter, \textit{Exploring the unique method for encoding sinograms in the first known Chinese-Polish dictionary} from Andrzej Swoboda is again an interface between very different languages and cultures, but in this case through a single bilingual dictionary of Chinese-Polish. The work under consideration is from 1936, \textit{Słownik chińsko-polski do czytania tekstów chińskich bez przygotowania}, and represents the first attempt to create an interface between these two languages. With reference to earlier studies, it analyses the challenges faced by the author of the dictionary, Doman Wieluch, and the means he employed to overcome these difficulties and structure his dictionary.

The chapter on \textit{Les travaux lexicographiques de Carlo da Castorano et ses tentatives pour faire imprimer un dictionnaire européen de chinois} by Mariarosaria Gianninotoa \& Michela Bussotti continues with the European language-Chinese interface, but takes us back to the 17th--18th century with work by an Italian Franciscan, Carlo Orazi da Castorano, and his difficulties in publishing a trilingual Latin-Italian-Chinese dictionary. The work was not published, as European printers were not yet handling  Chinese characters well, but the attempts of da Castorano mark a change in the handling of non-European languages. The chapter traces the origin of dictionaries of Chinese by missionaries from the 16th century up to the work of da Castorana. Choices inevitably meant which Chinese, notably Mandarin, and the difficulty in finding suitable equivalents. It then goes on to the trials and tribulations of getting the work published in print, a task in which da Castorano ultimately failed.

The chapter by Mirosława Podhajecka on \textit{The bilingual dictionary as a mediator between West and East: The beginnings of English-Polish lexicography},  yet again a manuscript dictionary, this time from the 18th century. Amongst the interesting factors is the date, which underlines the influence of French in Eastern Europe at a time when English was already becoming the dominant language of trade. A further twist in the tail is that analysis shows that the author based much of his work on a series of English-French dictionaries, including Boyer's dictionary of 1699. The article begins with the historical context of a large central European country that finds itself divided up amongst other powers, leaving Polish as a national language without a country. It also explains the dominance of French and German in the educational system, leading to a lack of interest in the English language. It then describes the manuscript of the anonymous \textit{Do Słownika Angielsko-Polskiego Dodatek ktorego układanie przedsięwzięte zostało Roku 1799  Dnia 17 czerwca [An Addendum to an English-Polish Dictionary Whose Compilation was Undertaken on 17 June 1799}] before considering issues relating to the compilation of this work.

The chapter \textit{Lexicon Lapponicum Bipartitum.....ungarice scriptum: Hungarian aspects of North Saami dictionary writing} by Ivette Keleman deals with the very different language combination of Northern Saami, spoken in Norway, and Hungarian, both of the Finno-Ugric group. The historical context is of great importance as missionaries sought to convert the Lapps and as scientific exploration also took place. This paper thus describes the work of two Jesuit astronomers, Maximilian Hell and János Sajnovics, their discovery of the affinities between the two languages and their creation of a dictionary using Hungarian spelling conventions. It provides a history of Saami lexicography and the creation of a new writing system.

Remaining with Hungarian, Gábor Tillinger looks at French-Hungarian dictionaries in a paper entitled \textit{Les exemples dans les dictionnaires français–hongrois à travers les siècles}. Starting from 1844, Tillinger takes us through the microstructure of dictionaries with an emphasis on examples. in a series of highly used verbs. The collection under study represents a series of large general language bilingual dictionaries to the exclusion of smaller specialised works. It then traces the choice of examples in these works and points to the particularities engendered by comparing two very different language structures. Covering a period from 1844 to 2007, examples show clearly the socio-cultural contexts.

Michela Murano takes us to the comparison of two French-Italian dictionaries of the 19th century in her chapter entitled \textit{“Sul finir d’imparare la Grammatica Francese, fa d’uopo studiar il Dizionario delle Frasi” : deux recueils phraséologiques bilingues franco-italiens de la première moitié du 19e siècle.} The chapter, in French, looks at two phraseological works, the \textit{Fraseologia italiana-francese} \citep{Trucchi1825} and the \textit{Dictionnaire des phrases françaises-italiennes} \citep{Lanza1837}. In 19th century Italy, the teaching of French affected all levels of society and works like those under study sought to approach grammar through phraseology. This is very much French as a Foreign Language, currently an area that attracts many students in France, but more than a century before the current perspectives. This chapter takes us through the macro and micro structures of dictionaries that were revolutionary in their time, and which contain lessons for current practice.

As can be seen in this collection of papers, historical lexicography and lexicology is about legacy dictionaries and also about works that describe past usage of language. Thus Rafał Rosół takes us to the foundations of Western European culture with a dictionary concerning the world of Ancient Greek, the \textit{Lexicon of Oriental Words in Ancient Greek}. The lexis discussed here covers a period of some two thousand years from the Mycenaean period to the 6th century AD, also termed Common Era or CE. The language covered concerns the incredible melting pot of civilisations in the Middle East up to the end of the Roman Empire. In many ways, this chapter shows the total relativity East meets West, because the question is as to whose East and whose West. The Greeks placed themselves in the centre and regarded the remainder of the world as barbarians, that is, non-Greek speakers. We are in a constant continuum of centre as we cross the world, which is why a historical approach to language is so important.

The final chapter, \textit{The discovery of a Russian-Tajik dictionary} by Abdusalom Mamadnazarova \& Bahriddin Navruzshoev, raises several issues which include the East to West continuum where Tajikstan is at the cross-roads between what we term Europe and what we term Asia, and the key issue of forgotten lexicographers that has pervaded this entire work. In this case, the author, Sh. Khatymtayev, is known, as is the date of publication; it is simply that the dictionary remains only in a fragmentary form. After a brief discussion of the history of Tajik dictionaries, the paper looks into the structure of this forgotten work and its equally forgotten compiler.

Dictionaries are tools of mediation across time and cultures. They are windows on the past. Historical lexicography and lexicology are vibrant, dynamic disciplines that open these windows to allow us to better understand the development of lexicography and the world in which we live. This collection of papers gives insights into a variety of worlds with shifting centres as languages evolve over time. East to West across Europe, East to West across the world.

{\sloppy\printbibliography[heading=subbibliography,notkeyword=this]}
\end{document}
