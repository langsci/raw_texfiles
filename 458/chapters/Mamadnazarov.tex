\documentclass[output=paper,colorlinks,citecolor=brown,arabicfont,chinesefont]{langscibook}
\ChapterDOI{10.5281/zenodo.15394509}
\author{Abdusalom Mamadnazarov\affiliation{Tajik National University} and
        Bahriddin Navruzshoev\affiliation{Tajik National University}}
\title{The discovery of a Russian-Tajik dictionary}

\abstract{The purpose of this article is to analyze the lexicographic and lexical peculiarities of the Russian-Tajik Dictionary (RTD) by Sh. Khatymtayev, published in 1899, and recently discovered in the library of the Tajik National Academy of Sciences. The lexicographic and lexical peculiarities of this publication remain almost unknown and unexplored up to now. The present analysis of this dictionary is being carried out based on the single copy, stored in the Academy library and numbered 337973A. Largely unknown by linguists, this Russian-Tajik Dictionary can be considered as the first thematic Russian-Tajik dictionary ever published. 

The dictionary consists of 72 pages, has no original cover, the binding in which it is currently discovered appeared much later. The absence of the flyleaf, as well as the last pages, seriously complicates the work of identifying additional data about this publication. The RTD is a thematic dictionary, contains 1370 entries divided into 49 topics. 

The dictionary suffers from a serious deficiency: it fails to arrange either the topics of the dictionary, or the words – entries inside the topics in alphabetical order.}

\IfFileExists{../localcommands.tex}{
  \addbibresource{../localbibliography.bib}
  \usepackage{langsci-optional}
\usepackage{langsci-gb4e}
\usepackage{langsci-lgr}

\usepackage{listings}
\lstset{basicstyle=\ttfamily,tabsize=2,breaklines=true}

%added by author
% \usepackage{tipa}
\usepackage{multirow}
\graphicspath{{figures/}}
\usepackage{langsci-branding}

  
\newcommand{\sent}{\enumsentence}
\newcommand{\sents}{\eenumsentence}
\let\citeasnoun\citet

\renewcommand{\lsCoverTitleFont}[1]{\sffamily\addfontfeatures{Scale=MatchUppercase}\fontsize{44pt}{16mm}\selectfont #1}
  
  %% hyphenation points for line breaks
%% Normally, automatic hyphenation in LaTeX is very good
%% If a word is mis-hyphenated, add it to this file
%%
%% add information to TeX file before \begin{document} with:
%% %% hyphenation points for line breaks
%% Normally, automatic hyphenation in LaTeX is very good
%% If a word is mis-hyphenated, add it to this file
%%
%% add information to TeX file before \begin{document} with:
%% %% hyphenation points for line breaks
%% Normally, automatic hyphenation in LaTeX is very good
%% If a word is mis-hyphenated, add it to this file
%%
%% add information to TeX file before \begin{document} with:
%% \include{localhyphenation}
\hyphenation{
affri-ca-te
affri-ca-tes
an-no-tated
com-ple-ments
com-po-si-tio-na-li-ty
non-com-po-si-tio-na-li-ty
Gon-zá-lez
out-side
Ri-chárd
se-man-tics
STREU-SLE
Tie-de-mann
}
\hyphenation{
affri-ca-te
affri-ca-tes
an-no-tated
com-ple-ments
com-po-si-tio-na-li-ty
non-com-po-si-tio-na-li-ty
Gon-zá-lez
out-side
Ri-chárd
se-man-tics
STREU-SLE
Tie-de-mann
}
\hyphenation{
affri-ca-te
affri-ca-tes
an-no-tated
com-ple-ments
com-po-si-tio-na-li-ty
non-com-po-si-tio-na-li-ty
Gon-zá-lez
out-side
Ri-chárd
se-man-tics
STREU-SLE
Tie-de-mann
}
  \togglepaper[21]%%chapternumber
}{}

\begin{document}
\maketitle

\section{Introduction}

The purpose of this article is to analyze the lexicographic and lexical peculiarities of the recently discovered \emph{Russian-Tajik Dictionary} (\citealt{RTD}) by Sh. Khatymtayev, published in 1899,  which builds on previous work by \citep{Mamadnazarov2013,Mamadnazarov2016,Mamadnazarov2021}. The lexicographic and lexical peculiarities of this publication have remained largely unknown and unexplored until now. The present analysis of this dictionary is based on the single copy stored in the library of the National Academy of Sciences of Tajikistan, numbered 337973A. The author, Khatymtayev, is to date poorly documented, and work is underway to compile a biography and locate copies of the text extant in Russia.

Largely unknown by linguists, this \emph{Russian-Tajik Dictionary} can be considered as the second Russian-Tajik dictionary ever published after V.F. Nalivkin's \emph{Russian-Persian Dictionary of the Dialects of the Turkestan Region}, published some ten years earlier. It is, however, the first \emph{thematic} Russian-Tajik dictionary ever published. It was compiled by Shohhaidar Khatymtayev and published at the Y. Baranovskiy Printing House in New Bukhara (Central Asia) and approved by the Russian Imperial Political Agency in Bukhara. The \emph{Russian-Tajik Dictionary} (henceforth RTD) was first published in 1899 in the city of New Bukhara (Kagan), and then in 1913 in Tashkent under the title \emph{Brief Russian-Tajik Dictionary}.

The analysis of this dictionary is being carried out using the only surviving copy, a valuable lexicographic publication of its time, that was preserved in the personal book collection of Professor M. Andreev, acquired by the Library of the Academy of Sciences of the Republic of Tajikistan in 1967. M. Andreev is widely known for his scientific publications on the history, culture and ethnography of Tajiks. Throughout his creative life, M. Andreev collected books and historical sources,  forming a unique book collection dedicated to the history and culture of the Central Asian region. The literature review indicates that this \emph{Russian-Tajik Dictionary} has not been described or studied by linguists and lexicographers in the Republic of Tajikistan or beyond.

\section{Literature review}

The introduction to the first \emph{Russian-Tajik Dictionary}, published in two volumes in the Soviet period in 1933--1934 in Leningrad, and compiled by S. Alizade, A. Ismailzade, R. Hashim and M. Yusupov, gives no hint about even the existence of Khatymtayev's dictionary  \citep[5--6]{Alizade1933}.

In the introduction to the second \emph{Russian-Tajik Dictionary}, edited by A.P. Dehoti and N.N. Yershov, we again do not observe any information about this work \citep[5--6]{DehotiYershov1949}.

The \citeyear{Asimov1985} \emph{Russian-Tajik Dictionary}, edited by M. Asimov, contains only one sentence in the introduction: “In 1899  and in 1913, two short \emph{Russian-Tajik} dictionaries were published in Tashkent (compiled by Shohhaidar Hotamtoev)" \citep[7]{Asimov1985}. As we see, the place of publication of the RTD is incorrect in this dictionary.

A small article consisting of only eight sentences devoted to this dictionary was published in the \emph{Tajik Soviet Encyclopedia} by A. Sayfullaev in Tajik, which provides general information about the dictionary by Hotamtoev. In particular, it is noted that this dictionary contains 1370 words and is of great importance as a lexicographic, dialectological, historical and language publication. The words are arranged by topics, such as Pronouns, Verbs, Creation in General, Airy Phenomena, etc. \citep[72]{Sayfullaev1983}.

To date, that is all the information about this dictionary in linguistic literature in Tajik and Russian. As this survey of the linguistic literature shows, this \emph{Russian-Tajik Dictionary} has not been described and studied by linguists and lexicographers in the Republic of Tajikistan or beyond.

It is impossible to determine when and how this dictionary entered Andreev’s collection, but its presence in the library is of great value for the National Academy of Sciences library, as the dictionary is one of the first Russian-Tajik dictionaries in the history of Russian-Tajik bilingual lexicography.

The dictionary consists of 72 pages, has no original cover, the binding it currently has having been added much later. The absence of the flyleaf, as well as the final pages, seriously complicates the work of identifying additional data about this publication; for example, we cannot estimate the number of published copies of the book. According to the author, there is a word list at the end of the dictionary, but this copy lacks it, and we cannot say in what order this list was compiled.

The Tajik scholars A. Mukhtarov and A. Sayfullaev in their article published in a newspaper entitled – \emph{This was the first dictionary}, dated July 9, 1981 noted “the uniqueness and significance” \citep[4]{MukhtorovSayfullaev1981} of this dictionary, unknown not only to the public, but also to lexicographers. At the same time, it is interesting that they wrote about a copy of the book found in the Russian State Library in Moscow. These scholars clearly did not know that a copy of this publication existed in the Library of the Academy of Sciences of Tajikistan. 

The RTD is a thematic dictionary containing 1370 entries divided into 49 topics. From the point of view of macrostructure [Hartman, 2002, 91], the dictionary consists of a \emph{Preface} (in Russian and Tajik), the thematic dictionary itself and the table of contents at the end of the dictionary. The sub-title reads: “A guide to how Russians can learn to speak Tajik and Tajiks can learn to speak Russian without help” \citep[1]{RTD}. 

It is interesting to note that the same page in Tajik, in Arabic script, calls this the \emph{Russian-Persian Dictionary} and then states that: “According to the content of this book, a Russian-speaking teacher can speak \emph{Persian}, and \emph{Muslims} can speak Russian” \citep[1]{RTD}. Further, we  find below in Uzbek: “Basmakhona Baranovskiy Yangi Bukhoroda”\footnote{trans. Printed in Baranovskiy’s Printing House}. We have no idea why this sentence is in Uzbek.

The introduction in Russian, which consists of only five sentences, notes that, “For the convenience of finding the right word, the words are divided into groups according to categories of appropriate meaning. The category titles are placed at the end of the dictionary” \citep[3]{RTD}. In addition, on the same page, a small text appears in Tajik under the headline: “From the author”, the content of which differs from the Russian introduction. In particular, it is claimed that: “With this dictionary \emph{Persian} speaking people can quickly and easily learn the Russian language, because Russian and \emph{Muslim} words and expressions are written in Russian and \emph{Muslim} script and anyone who has set a goal can find topics in the table of contents and find them in the dictionary" \citep[3]{RTD}.

Comparing the Russian original of the following: “With this dictionary, Russians can easily get acquainted with the \emph{Tajik} language, and \emph{Tajiks} with Russian, not knowing the script of these languages, because Russian words are written in Russian and \emph{Tajik} letters and vice versa” \citep[3]{RTD} with the text in Tajik (above), it can be seen that the word \textit{Tajiks} is replaced by \emph{Forisyon} (`Persians'), and \emph{Tajik language} is translated as `Muslim language'. We may thus conclude that, on the title page and the original Tajik text in the “Introduction”, the author avoided  using the words \emph{“Tajiks”} and \emph{“Tajik language”} several times and replaced them with \emph{“Persians”} and \emph{“Muslim Language”}. For what reason we do not know. We can only presume that at that period the authors did not distinguish between the Persian and Tajik languages. Tajik and Persian were used interchangeably. 

This work can be considered a bilingual thematic translation dictionary, since it is divided into 49 topics. Some topics are divided into several sub-topics, for example: “Man and parts of the human body" [p. 32], “Gender, Age and Family” [p. 41], “Tools, Weapons and Guns” [p. 57]. Taking this division into account, the total number of topics in the RTD increases to 70.

This figure is not accurate, however, since this copy consists of only 72 pages, with some pages missing. Thus, for example, whilst A. Sayfullaev talks about the topic “Phrases for exercise” saying that: “This section consists of 29 sentences with translation into Tajik” \citep[4]{Sayfullaev1983}, this topic is missing from our copy. It follows that in all probability the number of topics and pages of the dictionary were greater than we are aware of.

The entries are arranged in two columns on various topics including: Pronouns, Verbs, Conjunctions, Adverbs and Prepositions, Ordinal Numbers, Spiritual Words, Creation In General, Airy Phenomena, Numbers, etc.

The dictionary suffers from serious deficiencies in that it fails to arrange either the topics of the dictionary or the words within the topics in alphabetical order or in any other order or system of the Russian language. This makes it difficult to use when searching for a particular word. In this regard, is it not possible to call this publication a dictionary at all as, like all linguists, by a dictionary we understand “A book containing a systematic description of the vocabulary of the language” Akhmanova 2009, 421. When 10, 20 or even 30 words are given on a particular topic in a book, finding the right word does not seem to be a problem, but when 321 words are randomly given on a topic (for example, “Verbs” in this dictionary), finding the right word becomes highly problematic. Thus, this publication should more accurately be called a thematic \emph{‘word list’}, and not \emph{‘a dictionary’}. 

The description of Russian words and their translations in the dictionary begins with a capital letter without a paragraph in two ways: in both Cyrillic and in the Arabic script of the Tajik language. For the convenience of readers of this article, instead of the Arabic script, we have used a transliteration in Roman script. For example:

\ea (RTD, 1899, 6)
    \ea \gll Быть.		Шуданъ\\
             Bitь.		Shudanъ\\
        \glt `to be'
    \ex \gll    Родиться.      Таваллудъ-ёфтанъ.\\
                Roditsya.              {Tavallud yoftan.}\\
        \glt `to be born'
    \ex \gll    Шевелить.            Джумониданъ.\\
                Shеvelit            	 Jumonidan\\
        \glt `to shake' 		
    \ex \gll  Дышать.               Дамъ-гирифтанъ.\\
              Dishat.                   {Dam giriftan.}\\
        \glt `to breathe'
    \ex \gll Рассматривать.   Дида-гаштанъ.\\
             Rassmatrivat, 	{dida gashtаn.} \\
         \glt `to consider'
     \z
\z

As these examples show, Russian words are not arranged in alphabetical order; they begin with a capital letter and end with a full stop. Transliteration into Russian and translation into Tajik was carried out in Cyrillic and the Arabic script, which we replaced with Roman script.

A common RTD entry consists of a Russian entry with a capital letter without a paragraph and a Tajik equivalent or equivalents arranged in two columns, without any grammatical and stylistic content, special notes or illustrative examples.

The entire dictionary is divided into forty-six topics. The translation of Russian words into Tajik is given in Cyrillic and the Arabic script, so that those who do not know the Tajik script can use the dictionary. It consists of a variety of topics, but the size and content of the topics differ significantly from each other. The largest topic, for example, “Verbs”, is represented by 321 entries,  “Man and parts of the human body” is represented by 90, and “Conjunctions, adverbs and prepositions” by 74 words and expressions. The remaining topics have from 4 to 12 entries and from 30 to 45 words and expressions.

The first topic, “Pronouns” (Zamirho), consists of 32 entries (p. 5--6). This includes personal pronouns, as well as interrogative, negative and other types of pronoun and their translation into Tajik, but the pronouns themselves are not divided into types. For example: 

\ea
    \ea Kto.		Ki.
        \glt `who?'

    \ex Chto.		Chi.
        \glt `what?' 		

    \ex Kotoriy. 	Kadom.
        \glt `which?'  	 

    \ex Chey etot.   In kist.
        \glt `Whose is this?'  	

    \ex Nekto. 	Shakhse.
        \glt `somebody'  
    \z
\z

Lexical analysis of the examples shows that there are also some incorrect  Russian  Tajik translations. For example, the Russian phrase \textit{Chey etot} `whose is this' is translated as \textit{in kist} (`who is it'), and the correct translation in Tajik is \textit{In az kist}. Some words are translated into colloquial Tajik variants, for example: 

\ea
    \ea Moy etotъ. 		 Ini-manъ.
    \glt `this is mine' 		
    \ex Tvoy etotъ.		 Ini-ty.
    \glt `this is yours' 		
    \ex Ego etotъ.		 Ini-vay.
    \glt `this is his' 		
    \ex Nash etotъ.  	 Ini-mo.
    \glt `this is ours'  		
    \ex Vash etotъ. 	Ini-shumo.
    \glt `This is yours' 
    \z
\z

The fact that the dictionary also contains colloquial words suggests that the dictionary was intended to teach the spoken language as well, as indicated in the preface.

The lexical units of the largest topic in this dictionary, namely “Verbs”, as well as words and expressions in other topics, are not arranged in the Russian alphabetical order, which makes it very difficult to find the right word. The peculiarity of this section is that the verbs that have different forms in the imperative mood are also listed in the dictionary as independent lexicographic entries or lexicographic articles. For example:

\ea (RTD, p. 7)
    \ea Begatъ.		          Davidan.
        \glt `to run'  		
    \ex Bdzhite.       		Bezhite.
        \glt `run' 		
    \ex Vstatъ.        		Barkhestanъ.
        \glt `to stand up'  	
    \ex Vstavayte 			Khizidъ.
        \glt `stand up'  		
    \ex Slishat. 			Gush kardanъ.
        \glt `to listen'  	
    \ex Slushayte.    		Gush Kunidъ.
        \glt `listen'
    \z
\z

The dictionary also contains phrases, expressions and sentences in the imperative mood:

\ea 
    \ea Lozhites spatъ. 		Khobъ ravidъ.
        \glt `Go to bed'  	 [p. 9].
    \ex Imya dayte. 		Nomъ monidъ.
        \glt `Name it'  	 [p. 13].
    \ex Zaprygayte loshadey.    Asbhoro bandidъ.
        \glt `Harness the horses.'  [p. 14].
    \ex Vipraygayte loshadey.  Asbhoro yala kunidъ.
        \glt `Unharness the horses.'  [p. 14].
    \z
\z

There are also examples in which Russian words are translated into Uzbek with the Tajik auxiliary verb \textit{mondan}.

\ea
    \ea Plavat. 	Suzmishъ-kardanъ. 	\emph{Taj}. shino kardan
    \glt `to swim'  [p. 15].
    \ex Prosverlit.  Tishikъ-kardanъ. 		\emph{Taj}. surokh kardan
    \glt `to drill'  [p. 16].
    \ex Kleit. 	Elimъ-kardanъ. 		\emph{Taj}. shiresh kardan
    \glt `to glue'  [p. 16].
    \z
\z

We also find that Russian verbs  are translated into Tajik incorrectly (we have given the correct English translation in parentheses). For example, the phrase in the imperative mood \emph{Gum shaved!} has a rough sense corresponding to the English \emph{Get out!}

\ea
   \ea Bit. 		   Shudan.
       \glt `to become (to be)'  	[p. 16].
   \ex Meritь. 	   Paymudanъ.
       \glt `to understand 	(to measure)' 	[p. 16].
   \ex Meryayte. 	   Barkashidъ.
       \glt `measure (weigh)'   	[p. 16].
   \ex Spryachtes.   Gum shavidъ.
       \glt `get out (hide)'  [p. 16].
   \z
\z

There are also verbs that are translated correctly into  Tajik (with the Arabic script) (the third column), but words in Cyrillic are not accurately translated into  Tajik, with errors and with a hard sign at the end of each word (the second column). For example: 

\ea
    \ea Bolit golova.     Saramъ-dardъ-mikinatъ.
        \glt `My head aches.'   [c. 17].
    \ex Boznositshe.      Migrozh raftanъ.
        \glt `to ascend'                [c. 18].
    \ex Vospreshat.       Mangъ kardanъ.
        \glt `to forbid'                 [c. 18].
    \ex Naznachat.       Taganeъ kardanъ.
        \glt `to appoint'                 [c. 19].
    \ex Khudet.             Loghar-Gardidanъ.
        \glt `to lose weight'              [c. 21].
    \z
\z

In these  words, the translation into  Tajik with the Arabic script is mostly accurate without any problems, but the translation into  Tajik in Cyrillic has errors and inaccuracies, and each word ends with a solid sign (ъ). It seems to us that it was problematic for Russians to understand and pronounce these words correctly.

The second largest topic in this dictionary is \emph{Man and parts of the human body}, represented by 90 words and expressions. Some examples from p. 33 with our English translations are:

\ea
    \ea Lob\textbf{ъ}.	Pishona.
    \glt `forehead'
    \ex Glaz\textbf{ъ}.	Chashm\textbf{ъ}.
    \glt `eye'	
    \ex Brov\textbf{ъ}.	Abru\textbf{ъ}.
    \glt `eyebrow'
    \ex Zrachek\textbf{ъ}.	Siyohii Chashm.
    \glt `eyeball'
     \z
\z

From the examples given, we see that the words of the Russian language were mostly written with a final hard sign ending (ъ), which was preserved in translation and spelling in the Tajik language in Cyrillic. Spelling with a final hard sign (ъ) was used in Russian until the beginning of the 20th century. For example, a famous Russian lexicographer V.I. Dal used the same spelling when compiling his \emph{Explanatory Dictionary of the Live Great Russian Language} in the 19th century in four volumes \citep{Dal1966}. The reason why Sh. Khatymtayev used the hard sign in the RTD is, in our opinion, that the translation into Tajik with Cyrillic spelling was intended for Russians who could use Tajik without knowing the Arabic script. Apparently, the compiler of the dictionary did not know that in Tajik the sounds do not have a category of hardness and softness. However, in translations into Tajik in the same words with the Arabic script, such a phenomenon is not observed, that is, the firm sign (ъ) was not written.

The analysis of the dictionary entries and their translations into Tajik shows that even in Russian words ending in a soft sign (b) for some reason a hard sign (ъ) was used instead in the Tajik translations. For example:

\ea
    \ea Ochen\textbf{ъ}.	Bisyor\textbf{ъ}.
        \glt `very' 		[p. 29].
    \ex Zhizn\textbf{ъ}. 	Umur\textbf{ъ}.
        \glt `life'  		[p. 29].
    \ex Smert\textbf{ъ}. 	Marg\textbf{ъ}.
        \glt `death' 		 [p. 29].
    \ex Ogon\textbf{ъ}. 	Otash\textbf{ъ}.
        \glt `fire'  		[p. 30].
    \z
\z

There are also several words in the dictionary where the soft sign (ь) was preserved in translations into Tajik in Cyrillic. For example:

\ea
    \ea Step\textbf{ъ}. 	Chul\textbf{ъ}.
        \glt `desert'  [p. 30].
    \ex Bolezn\textbf{ъ}.      Kasal\textbf{ъ}.
        \glt `illness'   	 	[p. 39].
    \ex Inbir\textbf{ъ}.	Zandzhabil\textbf{ъ}.
        \glt `ginger'  	[p. 54].
    \z
\z

The dictionary also contains the words where there is no hard sign in Russian, but  the hard sign has been used in translations into Tajik in Cyrillic. For example:

\ea
    \ea Boroda. 	Rish\textbf{ъ}.
        \glt `beard' 	[p. 33].
    \ex Sheya.  	Gardan\textbf{ъ}.
        \glt `neck'  [p. 33].
    \ex Gorlo. 	Halq\textbf{ъ}.
        \glt `throat' 	[p. 33]	.
    \ex Plecho. 	Dush\textbf{ъ}.
        \glt `shoulder' 	[p. 33].
    \z
\z

In the section entitled “Flowers” (\textit{gulho}), on page 51, there are only six words, of which two  could hardly be related to the topic of flowers: For example:

\ea
    \ea Podsolnechnikъ. Oftobъ-farastъ.
    \glt `sunflower'
    \ex Opiumъ. 		Afyunъ.
    \glt `opium'
    \z
\z

Seventy-five lexical units represent the third biggest topic described in the RTD – \emph{Conjunctions, Adverbs and Prepositions}. As in the previous sections, lexical units are arranged in an irregular manner, not observing the alphabetical order of the Russian language. The section \emph{Conjunctions, Adverbs and Prepositions} is not divided into subgroups. For example:

\ea
   \ea Esli. 		Agarъ.
       \glt `if' 	 [p. 21].
   \ex Khotya. 	Agerchi.
       \glt `although'  	[p. 21].
   \ex Dazhe. 	Hatto.
       \glt `even' 	[p. 21].
   \ex Okolo. 	Nazdi.
       \glt `by'  	 [p. 22].
   \ex Vnezapno.   Nogohъ.
       \glt `suddenly'  	[p. 22].
   \z
\z

From the analysis of these  words and others in the dictionary, it is clear that \emph{Conjunctions, Adverbs and Prepositions} are given in the dictionary without any system or order, without indicating that they belong to one or another part of speech. Here, among other words, we can find \emph{Interrogative Pronouns} that are not related to this topic. These and some other Interrogative Pronouns should have been described in the first topic – Pronouns. For example:

\ea
   \ea Kak. 		Ch-tariqa.
       \glt `how'    [p.22]
   \ex Skolko. 	Chandъ.
       \glt `how much' 	 [p. 21].
   \ex Gde. 		Gdzho.
       \glt `where'  	[p. 23].
   \ex Otkuda. 	Azgdzho.
       \glt `where from'  	 [p. 23].
   \ex Kuda. 	Bakujo.
       \glt `where to'  	 [p. 23].
   \z
\z

Native Tajik letters and sounds, such as: [q], [ʁ], [h], [ə:], [i:], [dʒ] in the translations of Russian words into Tajik, in this dictionary in Cyrillic, are not represented in any way, although in words with the Arabic script of the Tajik language they are mostly correctly transliterated. This indicates that Russians using this dictionary could not accurately and correctly read or use words and expressions from the RTD in their speech. Since it is not known what the circulation of the book was or how many copies had been printed, we cannot say how often the dictionary was used, or whether it was used at all.

\ea
    \ea Vmeste. 	Khamro.
        \glt `together' 		 [p. 22].
    \ex Otkuda. 	Azgadzho.
        \glt `where from' 	 [p. 23].
    \ex Pozadi. 	Darъ-Aqibъ.
        \glt `in the back'  	 [p. 23].
    \ex Togda. 	Hamon-baqtъ.
        \glt `at that time'  	 [p.24].
    \z
\z

We also observed that in the dictionary the translations of some compound words, phrases and expressions into Tajik in Cyrillic were written with a hyphen and a capital letter, which is not observed in translations into Tajik with Arabic script. For example:

\ea
    \ea Potomu. 	Azъ-onъ-sababъ.
        \glt `For that reason' 	[p. 22].
    \ex Vnizu. 	Darъ taki-poinъ.
        \glt `downstairs'  	[p. 23].
    \ex Nigde. 	Darъ-khichъ-dzhoъ.
        \glt `nowhere'  	[p. 23].
    \ex Togda.  	Hamonъ-vaqtь.
        \glt `at that time'  	[p. 24].
    \ex Dnyomъ. 	Vaqtiъ-ruzъ.
        \glt `time of the day' 		[p. 24].
    \ex Oreshnikъ.  Chahor-mazhzъ.
        \glt `nut'  	[p. 52].
    \z
\z

The fourth largest topic in Khatymtayev's idictionary is called \emph{Arts and Crafts}, which contains 46 Russian language words and expressions with translation into Tajik in two versions – in Cyrillic and the Arabic script (pp. 59--60). As in other sections, lexical analysis of translations shows that the main problem is translation into Tajik with Cyrillic letters. For example:

\ea
   \ea Kupetsъ.		          Tuzharъ.  Tojir.
       \glt `trader'  		
   \ex Torgoviy-domъ. 		Tizharъ-khane.	tojirkhona
       \glt `trade house'  
   \ex Torgovlya bratevъ.   	Savdo-barodaronъ. 	savdoi barodaron
       \glt `brothers trade'  
   \ex Torgovlya. 			Migamila. 			muomila
       \glt `– trade'  
   \ex Kamenotyosъ. 		Sanъ-tarashъ. 		sangtarosh
       \glt `mason'  
   \ex Shveya. 			Zani-darizduzъ. 	 	zani darzduz
       \glt `tailor'  
   \z
\z

As was noted previously, the translation into Tajik written in Cyrillic in this section does not correspond in most cases to the translation represented in the Arabic script. In the words above, for example, \textit{mgamilya} is very far from the word \textit{muomila} in Tajik, in both writing and sound. These considerations explain why we doubt that Russians could benefit from this dictionary effectively. There are also mistakes in the words \textit{tizharkhane} and \textit{savdo-barodaron} and other words.

However, the originality of the translation of some words is of interest for modern Tajik, such as: \emph{bankrotstovo – bankruptcy – tazhar-shikastagi,  skvorknyak – furrier – pustin-soz, drevosek – woodcutter – ezumbur, tokar – turner – dukchi, raznoschik – peddler – charchinfurysh, cyryulnik – sartarosh} – barber and other words. 

One of the sections located at the end of the dictionary is called \emph{Time in General}, where 20 Russian words, expressions and 4 sentences are translated into Tajik on page 72. The translations of most Russian words and phrases are conveyed adequately and correctly. For example:

\ea
    \ea Denь. 		Ruzъ.
        \glt `day' 		
    \ex Nochь. 		Shabъ.
        \glt `night'  		 
    \ex Minuta. 		Daqia.
        \glt `Minute' 		
    \ex Zima. 		Zimiston.
        \glt `Winter'  		  
    \ex Sneg idyotъ. 	Barf-borida-istoddst.
        \glt `It is raining.'  
    \ex Segodnya ochenь zharko.  Imruzъ-bisyorъ-garmъ.
        \glt `It is very hot today.' 
    \z
\z

However, there are also inaccurate and inadequate translations or errors. For example, the word \emph{Vedro} `bucket', which apparently got into this topic by mistake, is translated as \textit{ruzi khubъ} `good day'.

\ea
    \ea Vedro. 	Rusi khubъ.
        \glt `Good day.'
    \ex Nepogoda. 	Ruzi-badъ.
        \glt `bad day' (should be: `bad weather')
    \ex Dozhdь. 	Buranъ-borida-istodaastъ.
        \glt `It is raining' (should be: `rain')
    \z
\z

The dictionary suffers from another serious deficiency, failing to arrange either the topics of the dictionary or the words – entries within the topics in alphabetical order or in any recognisable Russian language system. This makes it difficult to use and to find the right word. 

In Khatymtayev’s dictionary, the scope and content of the topics, in terms of spelling and semantics, differ significantly from each other. The biggest topic, for example, \emph{Verbs}, is represented by 321 words and collocations. The topic \emph{Conjunctions, adverbs and prepositions} is represented by only 74 words and expressions. The remaining topics have fewer, from six to 12 and from 30 to 35 words and expressions.

An analysis of the translation into equivalent Russian words shows that the translations into Tajik in the Arabic alphabet is carried out fairly accurately without much difficulty, but the translations into Tajik in Cyrillic suffer from errors. Russians faced great difficulty understanding and pronouncing the Tajik words and collocations correctly with the spelling provided in Cyrillic.

Analysis of the examples reveals that Russian language words were mostly written with a final hard sign (ъ). Russian spelling used this sign (ъ) at the end of the words until the beginning of the 20th century, something which this dictionary preserved in translations and spellings in Tajik in Cyrillic.  Khatymtayev probably used the hard sign (ъ) in the Tajik translations because the translation into  Tajik with Cyrillic script was intended for Russians who could use Tajik without knowing the Arabic script. Apparently, Khatymtayev did not know that the Tajik language lexical units do not have the category of \emph{hardness and softness}. Analysis of translations into Tajik also revealed that the author translates most of the Russian language words and expressions correctly. However, he also makes many mistakes and, in some cases, he uses translations into Uzbek. 

The analysis of the spelling of the RTD shows that it fails to represent or convey important native Tajik sounds, such as: [q], [ʁ], [h], [ə:], [i:], [dʒ] in the translations into Tajik in this dictionary in Cyrillic. However, it renders them mostly correctly in words with the Tajik/Persian script. This fact indicates that Russians using this dictionary could not accurately and correctly read or use words from this dictionary in their speech.

\section{Conclusion}

Lexicographic and lexical analysis of the Khatymtayev's RTD identified the features and peculiarities of this dictionary both from the point of view of translation and transliteration. As a result, it can be stated that this work, despite its lexicographic shortcomings from the point of view of modern principles of lexicography, is an important late 19th century lexicographic and lexical source of these languages. It reflects the vocabulary and spelling of the Russian and Tajik languages of its time and is of interest as a historical, lexical and dialectological monument of these languages in general and Bukhara region colloquial speech in particular. It seems to us that the compiler of the dictionary was neither a professional lexicographer nor a linguist. We cannot judge how widely the dictionary could have been used, since its circulation is not known. In conclusion, it should be emphasized that a thorough and detailed study of this source is necessary, since this cannot be done within the framework of an article.

{\sloppy\printbibliography[heading=subbibliography,notkeyword=this]}
\end{document}
