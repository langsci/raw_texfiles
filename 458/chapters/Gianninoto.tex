\documentclass[output=paper,colorlinks,citecolor=brown,arabicfont,chinesefont,booklanguage=french]{langscibook}
\ChapterDOI{10.5281/zenodo.15394497}
\author{Mariarosaria Gianninoto\affiliation{ReSO, Université Paul Valéry-Montpellier 3, France} and
        Michela Bussotti\affiliation{Ecole française d’Extrême-Orient, UMR 8173 (CNRS, EHESS, Université de Paris Cité, France)}}
\title[Les travaux lexicographiques de Carlo da Castorano]
      {Les travaux lexicographiques de Carlo da Castorano et ses tentatives pour faire imprimer un dictionnaire européen de chinois}

\abstract{The Franciscan missionary Carlo Orazi da Castorano (1673--1755) compiled a monumental trilingual Latin-Italian-Chinese dictionary (completed in 1732). Castorano tried several times to have his dictionary published, but only a few pages were finally printed. This article describes the elaboration of this dictionary by situating it in the context of the development of Sino-European lexicography, at the crossroads of various linguistic traditions, and then focuses on the attempts made by its author to have a European Chinese dictionary printed, several decades before the first Chinese-Latin-French dictionary printed in France in 1813 and the Chinese-English dictionary published in Macao by Robert Morrison (1782--1834) from 1815. These attempts came at a time when Etienne Fourmont (1683--1745) was engaged in engraving Chinese characters in wood in Paris. Despite his failure, Carlo Orazi da Castorano was one of the first missionaries who understood the importance of print for the dissemination and standardization of knowledge about the Chinese language and script in a foreign context.}

\IfFileExists{../localcommands.tex}{
  \addbibresource{../localbibliography.bib}
  \usepackage{langsci-optional}
\usepackage{langsci-gb4e}
\usepackage{langsci-lgr}

\usepackage{listings}
\lstset{basicstyle=\ttfamily,tabsize=2,breaklines=true}

%added by author
% \usepackage{tipa}
\usepackage{multirow}
\graphicspath{{figures/}}
\usepackage{langsci-branding}

  
\newcommand{\sent}{\enumsentence}
\newcommand{\sents}{\eenumsentence}
\let\citeasnoun\citet

\renewcommand{\lsCoverTitleFont}[1]{\sffamily\addfontfeatures{Scale=MatchUppercase}\fontsize{44pt}{16mm}\selectfont #1}
  
  %% hyphenation points for line breaks
%% Normally, automatic hyphenation in LaTeX is very good
%% If a word is mis-hyphenated, add it to this file
%%
%% add information to TeX file before \begin{document} with:
%% %% hyphenation points for line breaks
%% Normally, automatic hyphenation in LaTeX is very good
%% If a word is mis-hyphenated, add it to this file
%%
%% add information to TeX file before \begin{document} with:
%% %% hyphenation points for line breaks
%% Normally, automatic hyphenation in LaTeX is very good
%% If a word is mis-hyphenated, add it to this file
%%
%% add information to TeX file before \begin{document} with:
%% \include{localhyphenation}
\hyphenation{
affri-ca-te
affri-ca-tes
an-no-tated
com-ple-ments
com-po-si-tio-na-li-ty
non-com-po-si-tio-na-li-ty
Gon-zá-lez
out-side
Ri-chárd
se-man-tics
STREU-SLE
Tie-de-mann
}
\hyphenation{
affri-ca-te
affri-ca-tes
an-no-tated
com-ple-ments
com-po-si-tio-na-li-ty
non-com-po-si-tio-na-li-ty
Gon-zá-lez
out-side
Ri-chárd
se-man-tics
STREU-SLE
Tie-de-mann
}
\hyphenation{
affri-ca-te
affri-ca-tes
an-no-tated
com-ple-ments
com-po-si-tio-na-li-ty
non-com-po-si-tio-na-li-ty
Gon-zá-lez
out-side
Ri-chárd
se-man-tics
STREU-SLE
Tie-de-mann
}
  \togglepaper[16]%%chapternumber 
}{}

\begin{document} 

\renewcommand{\tabref}[1]{Tableau~\ref{#1}}
\begin{otherlanguage}{french}
\maketitle

\section{Introduction}

Les premiers dictionnaires européens des langues chinoises ont été conçus à la fin du XVIe siècle. Cette première étape de la lexicographie sino-européenne est associée à l’établissement de missions occidentales en Chine, les missionnaires ayant été les premiers Européens à avoir eu besoin de maitriser le chinois pour leur vie quotidienne dans le pays, ainsi que pour leurs activités de prosélytisme \citep{Raini2010, Paternico2013}. Ces dictionnaires bilingues ou multilingues, tout comme les grammaires européennes des variétés de chinois,\footnote{Le terme «~chinois~» désigne en français la langue standard contemporaine, mais il est aussi utilisé pour indiquer une macro-langue, caractérisée par une forte variation diatopique et diachronique. Sur les langues chinoises, cf. \citet{Peyraube2011} et \citet[6--61]{ArcodiaBascian2021}.} étaient ainsi des outils pédagogiques indispensables pour apprendre les langues locales. Un nombre considérable de dictionnaires bilingues ou trilingues apparaissent dans les siècles suivants, essentiellement dans des éditions manuscrites, avant l’essor impressionnant de la lexicographie bilingue au XIXe siècle (\citealt{Yang2014, Kloter2019}).

Parmi les travaux réalisés au cours du XVIIIe siècle, ceux du franciscain italien Carlo Orazi da Castorano (1673--1755) méritent notre attention. Auteur prolifique, Castorano rédige aussi des ouvrages de nature linguistique. Il convient de mentionner notamment ses grammaires du chinois en latin~: une version composée de 38 folios datée de 1706 et une version étendue composée de 88 folios et achevée en 1732. Castorano a également élaboré un monumental dictionnaire trilingue, le \emph{Dictionarium Latino-Italico-Sinicum} [Dictionnaire latin-italien-chinois], achevé en 1732, qu’il a cherché à plusieurs reprises à faire publier, mais dont seules quelques pages ont été finalement imprimées.

Dans cet article, après avoir brièvement présenté les dictionnaires des langues chinoises conçus par les missionnaires européens, nous décrivons la contribution de Carlo Orazi da Castorano à la lexicographie sino-européenne, en analysant en particulier l’un des exemplaires de son dictionnaire trilingue. Nous nous penchons ensuite sur ses tentatives pour imprimer le premier dictionnaire européen de chinois.

\section{La lexicographie missionnaire sino-européenne}
\largerpage
La Chine possède l’une des traditions lexicographiques les plus anciennes et les plus riches au monde. Les plus anciens ouvrages lexicographiques sont rédigés au cours du premier siècle avant J.-C. \citep{Bottero2019} et la lexicographie chinoise se développera tout d’abord sous la forme de dictionnaires de caractères et de recueils de gloses~; les dictionnaires organisés par rimes figurent aussi plus tard parmi ses genres principaux \citep{Vedal2019}. Cette tradition lexicographique est essentiellement représentée par des dictionnaires monolingues.\footnote{La lexicographie bilingue et multilingue était beaucoup moins présente dans la tradition linguistique chinoise. Ceci est probablement lié au statut des langues étrangères en Chine, qui n’étaient pas valorisées dans le système éducatif traditionnel (\cites[28]{Peverelli1986}[40]{Pellin2009}) et dans les concours mandarinaux (avec quelques exceptions, \citealt[166--167, 223]{Elman2000}), ne faisant pas partie de la formation des élites \citep[358]{Lackner2001}.}

Les premiers dictionnaires des variétés de chinois en langues occidentales sont écrits par les missionnaires établis en Chine et en Asie sud-orientale. Pour répondre à leurs besoins d’apprentissage des langues locales, les premiers missionnaires adaptent tout d’abord des ouvrages chinois d’apprentissage des caractères \citep[247--249]{Brockey2007}, puis rédigent également des grammaires, des méthodes et des dictionnaires des variétés de chinois en langues européennes. Les premiers dictionnaires missionnaires d’une variété de chinois ont été très probablement compilés à Manille à la fin du XVIe siècle, où résidait une communauté d’origine chinoise (\cites[615]{Zwartjes2019}{Kloter2011}). Le premier dictionnaire européen du chinois mandarin\footnote{Pour une analyse détaillée de la langue décrite dans ce dictionnaire, cf. \citet{Yang2001}.} est un dictionnaire manuscrit trilingue ayant le portugais comme langue source, des entrées traduites en italien, et le chinois comme langue cible. Compilé probablement dans la dernière décennie du XVIe siècle \citep[51]{Raini2010}, ce dictionnaire est attribué aux missionnaires jésuites italiens Michele Ruggieri (1543--1607) et Matteo Ricci (1552--1610).\footnote{Ce manuscrit a été découvert par le jésuite Pasquale D’Elia dans les archives romaines de la Compagnie de Jésus en 1934. D’Elia l’a appelé \emph{Dizionario Portoghese-Cinese} [Dictionnaire portugais-chinois] et en a attribué la paternité aux missionnaires jésuites Michele Ruggieri et Matteo Ricci, même si Ruggieri devrait probablement être considéré comme l'auteur principal. Sur ce dictionnaire cf. \citet{Yang2001}.} L’élaboration de dictionnaires de chinois en langues européennes doit répondre au défi d’inclure un système d’écriture non alphabétique, les caractères chinois ou sinogrammes, dans des ouvrages en langues alphabétiques. Des systèmes de romanisation sont ainsi élaborés pour noter la prononciation et faciliter la mémorisation des mots. Nous trouvons des dictionnaires entièrement rédigés en lettres latines et n’indiquant donc les mots chinois qu’en romanisation. Le dictionnaire espagnol-chinois \emph{Vocabulario de la lengua Mandarina} (1687) du missionnaire dominicain Francisco Varo (1627--1687) ne comporte par exemple aucun caractère chinois.\footnote{Selon \citet[12]{Coblin2006}, ce dictionnaire a été élaboré par Varo «~\emph{by the final decade of his life}~». Pour ce dictionnaire, nous avons consulté la version numérisée du manuscrit «~Libr. Sin. 29~» de la Bibliothèque nationale d'Allemagne (accessible à l’adresse suivante~: \url{https://digital.staatsbibliothek-berlin.de/werkansicht?PPN=PPN3308102102\&PHYSID=PHYS\_0005\&DMDID=DMDLOG\_0001}) ainsi que l’édition critique de \citet{Coblin2006}.} Toutefois, dans la plupart des dictionnaires, caractères chinois et romanisation coexistent. Ces dictionnaires doivent ainsi prévoir «~\emph{a digraphic arrangement of entries involving Chinese characters and transcriptions using the Roman alphabet}~» \citep[325]{Kloter2019}. C’est le cas par exemple du \emph{Diccionario de la lengua Mandarina}, écrit par le missionnaire dominicain espagnol Francisco Díaz (1606--1646), dont les entrées sont représentées par les caractères, dans le cadre de la tradition lexicographique chinoise. Le dictionnaire de Díaz marque un développement important de la lexicographie bilingue, les entrées en chinois y étant disposées selon l’ordre alphabétique de prononciation de l’initiale, suivant la tradition lexicographique occidentale \citep[236]{Masini2006}. Cette innovation dans l’organisation des entrées avait été rendue possible par l’introduction des systèmes de romanisation\footnote{Les dictionnaires monolingues chinois étaient traditionnellement classés par catégories sémantiques, par radicaux et traits, ou encore par tons et rimes.} \citep{Luo1930}. Ces ouvrages pionniers dus aux missionnaires jésuites et dominicains ouvrent la voie à une riche production de dictionnaires bilingues ou plurilingues, souvent rédigés par des missionnaires franciscains, et qui proposent le chinois comme langue source, plus rarement comme langue cible. Ces ouvrages nous sont parvenus dans des éditions manuscrites, datant du XVIIe et du XVIIIe siècles, tandis qu’il faut attendre le début du XIXe siècle pour les premiers dictionnaires imprimés. C’est dans la tradition de cette production que s’inscrit l’ouvrage trilingue de Carlo Orazi da Castorano et dans ce contexte historique que se situent ses tentatives de faire imprimer un dictionnaire européen du chinois.

\section{Carlo Orazi da Castorano et son dictionnaire latin-italien-chinois}

\subsection{Carlo Orazi da Castorano}

Carlo Orazi (ou Carlo Horatii da Castorano, Carlo di Orazio da Castorano~; en chinois Kāng Hézi\footnote{Dans le présent article, le système de transcription \emph{pīnyīn} (système de romanisation officiel de la République Populaire de Chine), est utilisé. Les romanisations originales sont maintenues dans les citations. Les caractères traditionnels (non simplifiés) sont utilisés pour toutes les sources de l’époque impériale. Pour les sources contemporaines, le choix entre caractères simplifiés ou traditionnels dépend de la source elle-même.} {\cn{康和子}}) est un missionnaire franciscain, né en 1673 à Castorano (près d’Ascoli Piceno, dans la région italienne des Marches). Il entre dans les ordres mineurs en 1690. Arrivé en Chine vers 1700, il y vivra pendant plus de trois décennies. Carlo Orazi travaillera dans les deux provinces du diocèse de Pékin (Shandong et Zhili), cf. \citealt{Catto2013}). En 1724,  il se retire dans la résidence des missionnaires de Propaganda Fide près de Pékin. Il est très impliqué dans la controverse sur les rites chinois (\citealt{Von_collani2016}).\footnote{Comme le precise \citet[479]{Von_collani2016}, «~\emph{The so-called Chinese Rites Controversy is the terminus technicus for the quite heated debates and arguments about the meaning of Chinese terms and Rites. It is a story which lasted more than 100 years and absorbed much energy in Europe and in China}~». Sur la question des rites cf. aussi \citet{Mungello1994}.} En 1733, il quitte la Chine pour se rendre à Rome, puis retourne à Castorano en 1741, où il meurt en 1755 \citep{Catto2013}.

Ce missionnaire était un auteur prolifique, ayant écrit de nombreux textes sur la controverse des rites chinois, sur la mission en Chine, sur les activités de prosélytisme \citep{Yu2017}. Parmi ses écrits, nous pouvons citer la \emph{Parva elucubratio super quosdam libros sinenses} [Brève explication sur quelques livres chinois], catalogue manuscrit décrivant les livres chinois ayant appartenu au missionnaire franciscain Giovanni Francesco Nicolai da Leonessa et le \emph{Brevis narratio itineris ex Italia usque ad Chinam} (1739) [Bref récit du voyage de l’Italie en Chine], qui, comme le précise Stafutti, est un «~récit des activités des missionnaires catholiques en Chine, [décrivant] les critères suivis par les différents ordres religieux pour mener à bien leur travail d’évangélisation, et les controverses qui ont surgi entre les ordres et le Saint-Siège en raison de ces critères\footnote{Texte original en italien~: «~\emph{testimonianza sull’attività dei missionari cattolici in Cina, sui criteri seguiti dai vari ordini religiosi nel condurre la propria opera di evangelizzazione e sulle controversie che proprio a causa di questi criteri sorsero tra gli ordini stessi e la S. Sede}~».}~» (\citet[35]{Stafutti1981}, notre traduction).

Parmi ses ouvrages, figurent aussi deux grammaires. Tout d’abord, la \emph{Manuductio ad linguam sinicam}, une «~Introduction à la langue chinoise~», rédigée pour la première fois en 1706, dans une version composée de 38 folios. La \emph{Grammatica seu Manuductio ad linguam sinicam} [Grammaire ou Introduction à la langue chinoise, 1732] est une version étendue de la première grammaire.\footnote{Sur la \emph{Grammatica seu Manuductio ad linguam sinicam}, cf. \citet{Li2017}.} La \emph{Grammatica} comprend 88 folios et inclut les caractères chinois, alors que seules les transcriptions étaient utilisées pour écrire les mots chinois dans la grammaire précédente.

Enfin, Castorano a également élaboré un dictionnaire trilingue et a participé à d’autres projets lexicographiques, notamment en copiant des dictionnaires manuscrits et en cherchant à faire imprimer son propre dictionnaire. 

\subsection{Le \emph{Dictionarium Latino-Italico-Sinicum}}

L’exemplaire consulté du \emph{Dictionarium Latino-Italico-Sinicum} [Dictionnaire latin-italien-chinois], achevé en 1732 à Pékin, comprend 1 008 folios.\footnote{L’édition consultée est la version numérisée du manuscrit «~Vat.estr.or.4~» de la Bibliothèque Apostolique Vaticane, qui correspond à l’une des versions manuscrites autographes (écrite par Carlo da Castorano, à l’exception des caractères chinois du dictionnaire, dus à des lettrés chinois, cf. \citealt[180]{Li2017}). Manuscrit accessible à l’adresse suivante~: \url{https://digi.vatlib.it/view/MSS\_Vat.estr.or.4}.} Le titre complet de cet ouvrage en explicite la nature de dictionnaire trilingue et donne des indications sur son lectorat cible~: \emph{Dictionarium Latino-Italico-Sinicum Tàm vocum, quàm Litterarum seu Characterum usualium Sinensium ad usum et commoditatem PP. Missionariorum in hanc Sinicam Missionem noviter adventantium} [Dictionnaire latin-italien-chinois des mots, des lettres ou caractères du chinois courant pour l’usage et le profit des missionnaires arrivant dans cette mission chinoise].

Ce manuscrit s’ouvre par des «~Avertissements aux lecteurs~» (\emph{Ad Lectores Monita}\footnote{Pages 3 à 15 du manuscrit consulté.}), où l’auteur décrit le contexte dans lequel son ouvrage a vu le jour et donne des précisions sur son processus de rédaction. Cette préface est suivie par la \emph{Grammatica seu Manuductio ad linguam sinicam} [Grammaire ou introduction à la langue chinoise\footnote{Pages 17 à 103.}], la deuxième version, étendue, de la grammaire chinoise de Carlo da Castorano. Puis, nous avons le dictionnaire, dont la langue cible est le chinois\footnote{Mandarin septentrional, enrichi de mots et d’expressions venant de la région du Shandong, en raison du long séjour de Carlo da Castorano à Línqīng {\cn{臨清}} \citet[177]{Li2017}.} et la langue source est le latin, bien que les entrées soient aussi traduites en italien. Ce dictionnaire devait en effet combler l’absence d’outils adaptés pour les missionnaires italophones, lacune dont l’auteur se rendra compte dans son processus d’apprentissage de la langue. Carlo da Castorano commence ainsi la rédaction de son ouvrage sur incitation du père Giovanni Battista da Illiceto, missionnaire de l’Ordre des Frères Mineurs \citep[178]{Li2017}. Toutefois, Carlo Orazi opte pour un dictionnaire latin et dans «~Les avertissements aux lecteurs~» explique ce choix par sa volonté d’élaborer un ouvrage accessible et utile à un public plus large~:

\begin{quote}
    Il vaut mieux que je fasse un dictionnaire latin-chinois, car il sera plus utile, y compris pour les missionnaires des autres pays, et pour beaucoup d’autres personnes.\footnote{Texte original~: «~\emph{Verum melius est si faciam dictionarium Latino-Sinicum, sic enim erit magis comunicabile bonum, etiam pro aliis Ppbus Misionariis aliarum Nationum, et pro multis Tertibus}~» \citet[f. 4]{Orazi1732}.}
\end{quote}

Dans la préface, Castorano décrit le processus de rédaction de son dictionnaire, en commençant par reprendre les entrées du dictionnaire de Cesare Calderino Mariani,\footnote{Il s’agit du \emph{Dictionarium tum latini, tum italici sermonis studiosis, apprime congruens}, dont la première édition date de 1587 et qui a fait l’objet de nombreuses éditions aux XVIIe et XVIIIe siècles \citet[178]{Li2017}.} ne pouvant pas disposer du dictionnaire d’Ambrogio Calepino\footnote{La première édition du dictionnaire latin de l’augustinien Ambrogio Calepino (1435–1511), intitulée \emph{Ambrosii calepini bergomatis eremitani dictionarium}, a paru en 1502. Ensuite «~\emph{at least two hundred editions were published before 1779, starting with three languages (Latin, Greek, and Italian) before being expanded to a total of eleven languages}~» \citep{Zwartjes2021}.} («~\emph{statim misi manum ad opus, deferibendo scilicet Dictionarium Calderinum [carebam enim Calepino]}.\footnote{\citet[f.4]{Orazi1732}} Partant de cette base, l’auteur ajoute, lorsque cela était possible, les traductions chinoises des entrées, aussi bien en transcription qu’en caractères chinois («~\emph{ut dum liceret verba sinica pomerem, et tandem Litteras seu characteres sinicos opponi facerem}\footnote{\emph{Ibidem.}}~»). Il abandonna ensuite ce travail jusqu’en 1728, année où il en reprit la rédaction. Il disposera alors du dictionnaire de Calepino, qui sera une autre source importante pour les entrées en latin (cf. \citealt{Li2017}  et \citealt{Castorina2019}).

Les sources des équivalents chinois sont variées. Nous pouvons tout d’abord mentionner les dictionnaires chinois-latin de son confrère Basilio Brollo (1684-1704), nom chinois Yè Zūnxiào {\cn{葉尊孝}}.\footnote{Mattia Andrea Brollo (1648-1704) est également un missionnaire franciscain. Né à Gemona près de Udine (d’où son nom Père Basilio «~da Gemona~», parfois transformé en «~da Glemona~»), il est ordonné en 1674 et arrive en Chine en 1684. Il est nommé vicaire apostolique du Shaanxi en 1696 où il se rendra quelques années plus tard et où il restera jusqu’à sa mort en 1704. À son propos, voir \citet{Polmonari2008}. } Brollo est l’auteur de deux dictionnaires. Le premier dictionnaire, complété en 1694, était organisé selon le système chinois des radicaux et des traits et contenait plus de 7 000 entrées~; le second dictionnaire, achevé en 1699, était classé par ordre alphabétique, en fonction de l’initiale du mot en romanisation et contenait plus de 9 000 entrées. Il est important de mentionner que l’un des exemplaires du dictionnaire alphabétique de Brollo a été recopié par Castorano,\footnote{Il s’agit du manuscrit «~Vat. estr.-or. 8~» de la Bibliothèque apostolique vaticane, intitulé \emph{Dictionarium Sinico-Latinum Reverendissimi Patris Basilij à Glemona} [Dictionnaire chinois-latin du très révérend père Basile da Gemona].} qui a inclus dans ce même manuscrit la première version de sa grammaire (\emph{Manuductio ad linguam sinicam}, 1706). Raini souligne également les similitudes au niveau des systèmes de transcription, affirmant que «~la romanisation utilisée dans le dictionnaire et la grammaire de Castorano, coïncide parfaitement avec la romanisation principale utilisée dans la copie du dictionnaire de Brollo datée de 1699~» et qu’il s’agit «~d’une transcription fortement marquée par la prononciation italienne~» (\citealt[236]{Raini2010}; notre traduction).\footnote{Texte original italien~: «~\emph{la romanizzazione usata nel vocabolario e nella grammatica di Castorano, coincide perfettamente con la romanizzazione principale utilizzata nella copia del dizionario di Brollo datata 1699}~»; «~\emph{una trascrizione a forte base italofona}~».}

Dans la préface, Castorano explique combien il est difficile de trouver les équivalents de nombreuses entrées latines. Plusieurs équivalents ou explications sont fournis par des informateurs chinois. Ces informateurs étaient des domestiques et paysans qu’il avait pu rencontrer dans son entourage, puis, dans un deuxième temps, un enseignant (cf. \citealt[179]{Li2017}). Les caractères chinois manquaient pour de nombreux mots de la langue parlée («~\emph{multa verba esse in locutione vulgari sinico, quibus Litterae seu characteres Sinici vulgares non occurant}~»,  \citealt[6]{Orazi1732}). Des équivalents écrits sont repris de la lexicographie sino-mandchoue.\footnote{\citet[6]{Orazi1732} parle d’un «~dictionnaire mandchou-chinois~» («~\emph{Tartaro=Sinico Dictionario}~»). La Chine de l’époque Qing (1644-1911) est un empire multilingue où le mandchou (langue de la dynastie régnante) est une langue officielle. Une riche production de manuels et de dictionnaires bilingues chinois-mandchou ou trilingues chinois-mandchou-mongol voit le jour (cf. \citealt{Soderblom_saarela2019}). Ces manuels et dictionnaires constituent une source importante pour l’étude du mandarin de l’époque.} Le dictionnaire de Castorano, enrichi de tous ces apports, constitue une source importante pour la langue de l’époque. L’intérêt pour la langue parlée, le registre familier, les mots de la vie quotidienne, tout ce qui était motivé par les besoins de la vie de tous les jours mais aussi pour les activités d’évangélisation, mérite d’être souligné.\footnote{\citet[183]{Li2017} écrit à cet égard que «~l’auteur a transcrit pratiquement toutes les entrées qui concernaient chaque aspect de la vie quotidienne~» («~\emph{l’autore ha trascritto quasi tutti i lemmi che hanno a che fare con qualsiasi ambito della vita quotidiana}~»).} Il s’agit d’une caractéristique importante de la lexicographie missionnaire, par rapport à la tradition lexicographique autochtone chinoise, centrée sur la norme écrite.

En ce qui concerne la macrostructure, les entrées sont classées en fonction de la première lettre du mot latin~; la première entrée est constituée par les prépositions latines \emph{A, ab, ex.} «~de, depuis~»~; la dernière entrée est \emph{zhytyum} [en latin] \emph{/acqua d’orzo tostato} [en italien], «~zythum~» ou «~zython~». Cette organisation par ordre alphabétique selon l’initiale s’inscrit dans le cadre de la tradition lexicographique occidentale et elle est liée au choix de la langue source, mais aussi aux modèles suivis et aux sources adoptées par Castorano. En effet, la première entrée correspond, y compris en ce qui concerne la définition, à une entrée du dictionnaire de \citet[12]{Calderino1586}; la dernière entrée, par contre, correspond à la dernière entrée du dictionnaire de Calepino.\footnote{Dans l’édition que nous avons pu consulter, datée de 1513.} 

 Chaque page compte dix-huit entrées. Les entrées se situent à gauche et les équivalents chinois à droite de la page. Comme c’est le cas dans la plupart de ces dictionnaires, bilingues ou multilingues, nous retrouvons une mise en page à l’occidentale~: la page est organisée par lignes et non par colonnes (ce qui est habituellement le cas dans la lexicographie traditionnelle chinoise), et écrite de gauche à droite (tandis que les colonnes sont écrites de droite à gauche dans les ouvrages chinois de l’époque).
 
En ce qui concerne la microstructure, les entrées se composent de mots ou de locutions d’abord en latin (éventuellement accompagnés de synonymes), puis en italien (les traductions en italien sont placées en dessous de l’entrée en latin)~; les équivalents chinois sont indiqués en caractères chinois mais aussi en romanisation (qui apparait sous les caractères correspondants)~; certaines entrées incluent également des exemples illustratifs en caractères chinois et en romanisation. Cela correspond à un objectif pédagogique~: la romanisation était utile pour les missionnaires et les apprenants visant à étudier uniquement la langue parlée, les caractères et la romanisation pour ceux souhaitant étudier à la fois la langue parlée et la langue écrite, comme l’explique Castorano dans sa préface~: 

\begin{quote}
    Ainsi, les missionnaires [ou bien les autres (lecteurs)] qui souhaitent uniquement apprendre la langue chinoise (parlée), ne doivent lire que les mots placés en bas à droite~; mais ceux qui souhaitent apprendre aussi les caractères chinois, doivent étudier les caractères disposés sur la même ligne ou au-dessus.\footnote{Texte original en latin~: «~\emph{Ergo PP Missionarij [aut etiam Alii] qui voulerint addiscere solumodò Linguam sinicam, legant solum nomina seu verba in linea recta inferiorius posita~; Qui autem addiscere voluerint etiam characteres Sinicos, inspiaciant eos intra eadem lineam seu areolam superius dispositos}~» \citet[6]{Orazi1732}}
\end{quote}

Nous citons des exemples constitués par des entrées successives, pour donner une idée de l’organisation des entrées dans ce dictionnaire (notre traduction française est ajoutée entre parenthèse) (cf. \tabref{tab:gianninoto:1}).

\begin{table}
\caption{L'organisation des entrées}
\label{tab:gianninoto:1}
\label{tab:gianninoto:frequencies1}
\begin{tabularx}{\linewidth}{QQ}
\lsptoprule
 Abloco, cas
 
\footnotesize (Je loue, tu loues)

\normalsize Vide Loco 

\footnotesize (cf. [entrée] «~Loco~»)

\normalsize Locare, affittare 

\footnotesize (Louer) & \normalsize {\cn{賃。   租。  賃房子 。   租房屋。}}

jín lín  zū    lín fângzù   zū fângwŭ

\footnotesize (Donner ou prendre à loyer. Louer. Donner à loyer une maison. Louer une maison) \\
\hline
 Abluo, is 
 
\footnotesize (Je lave, tu laves)

\normalsize Lavare

\footnotesize (laver) & {\cn{洗。洗一洗。│洗手。 洗臉。 │ 洗滌。盥。澣 。盥手}}

\normalsize sì     sì ĭ  sì     v.g. sì scèu  sì lièn    sì tĭ     kuòn   han    kuòn scèu

\footnotesize (Laver. Laver un peu.  p. ex. Se laver les mains. Se laver le visage. Nettoyer. Se débarbouiller.  Se laver. Se laver les mains) \\
\hline
 Abluere seu lavare corpus
 
\footnotesize (laver ou se laver le corps) & {\cn{洗澡。 洗身子。  沐浴}}

\normalsize sì zào    sì scīnzù     mŭiû mŭiŭ

\footnotesize (Prendre un bain. Laver son corps. Se baigner) \\
\lspbottomrule
\end{tabularx}
\end{table}

Dans cette séquence d’entrées,\footnote{F. 2v du dictionnaire. Il est important de préciser que la première page du dictionnaire se trouve à la page 109 du manuscrit (les pages précédentes étant numérotées jusqu’à 108, en haut de la page).} la première entrée est constituée par le lemme latin «~\emph{Abloco, as}~», correspondant à la première et à la deuxième personne du présent du verbe \emph{ablocare} «~louer~». À côté de ce lemme, nous trouvons un renvoi au synonyme «~\emph{Loco}~» (première personne du verbe \emph{locare} «~louer~»). Ce renvoi est présent dans le dictionnaire de Calepino, où à l’entrée \emph{Abloco, as}, le lecteur est invité à «~se rapporter au mot ‘Loco’~» («~\emph{vide in dictione loco}~», \citealt[3]{Calepino1513}). Dans son dictionnaire, Carlo da Castorano fournit également l’entrée en italien, sous forme des verbes à l’infinitif \emph{locare} et \emph{affitare} «~louer~» (placé en dessous de l’entrée en latin). Comme le montre cet exemple, c’est le latin qui est pris en compte pour la disposition alphabétique. Deux équivalents chinois figurent pour cette entrée. Pour le premier équivalent, deux prononciations sont indiquées. Ces équivalents sont suivis par les exemples de collocations (ayant le sens de «~louer une maison~»).

Dans le dictionnaire de Carlo da Castorano, l’entrée «~\emph{Abloco, as}~» est suivie par l’entrée «~\emph{Abluo, is}~». L’ordre ne correspond pas à celui du dictionnaire de Calepino (qui place \emph{Abluo} devant \emph{Abloco}~; cf. \citealt{Calepino1513} ~: 3), mais correspond à l’ordre du dictionnaire de Calderino Mariani, lequel avait constitué sa principale source pour établir la structure des entrées latines de son dictionnaire (\citealt{Calderino1586} ~: 13).\footnote{Toutefois, ce ne sont pas toutes les entrées latines qui sont reprises, du moins dans l’édition que nous avons pu consulter, le \emph{Dictionarium D. Caesaris Calderini Mirani Veronensis, tum latini, tum italici sermonis studiosis, apprimè congruens}, 1586, accessible à l’adresse suivante~:  http://bdh.bne.es/bnesearch/detalle/bdh0000209967.} L’entrée «~\emph{Abluo, is}~», est accompagnée de sa traduction italienne «~\emph{Lavare}~» («~laver~»). L’entrée suivante est «~\emph{Abluere seu lavare corpus}~» («~laver ou se laver le corps~»), qui ne figure pas à la suite de l’entrée «~\emph{Abluo}~» dans le Calderino et qui est ajoutée par Castorano. Dans ces deux entrées, nous avons des synonymes ou quasi-synonymes, appartenant souvent à des registres différents~: par exemple des mots de registre soutenu, comme \emph{guàn} {\cn{盥}} et \emph{huàn} {\cn{澣}}, figurent à côté d’expressions et tournures relevant davantage de la langue parlée, comme \emph{xǐ yī xǐ} {\cn{洗一洗}} «~se laver (un peu)~». Nous trouvons aussi des collocations et des exemples illustratifs, comme \emph{xǐ shǒu} {\cn{洗手}} «~se laver les mains~» et \emph{xǐ liǎn} {\cn{洗臉}} «~se laver le visage~». Les divers exemples sont introduits par l’abréviation latine «~\emph{v.g.}~» («~p. ex.~») mais uniquement dans la ligne écrite en romanisation, tandis qu’ils sont séparés par le symbole | dans la ligne écrite en caractères.

Un autre exemple d’entrée\footnote{F. 61r du dictionnaire.}~: 

\begin{table}
\caption{L'entrée de 'Canus, Canutus'}
\label{tab:gianninoto:2}
\label{tab:gianninoto:frequencies2}
\begin{tabularx}{\linewidth}{ QQ }
\lsptoprule
Canus, ni et Canutus, tis 

Cano Canuto 

\footnotesize (chenu vieux) & {\cn{頭髮白了。白頭髮的人。年老了。年老}}

{\cn{的人}}
 
têùfă pĕ lèao  pĕ têùfă tĭ jìn  niên lào leào niên lào tĭè jìn

\^ id est Senex

\footnotesize (Les cheveux sont devenus blancs. Une personne qui a les cheveux blancs. Il a vieilli. Une personne âgée.

C’est un vieil homme). \\
\lspbottomrule
\end{tabularx}
\end{table}

Dans cette entrée, Castorano propose des périphrases pour rendre les mots latins (entrées au nominatif et génitif). La traduction italienne est fournie (\emph{cano}, \emph{canuto}, «~aux cheveux blancs~»). Pour les traductions en chinois, nous avons d’abord une phrase (p. ex. «~Les cheveux sont devenus blancs~»), qui est reprise pour former un syntagme («~Une personne qui a les cheveux blancs~» ou «~Une personne aux cheveux blancs~»). Le même procédé est utilisé pour le deuxième équivalent («~Il a vieilli~»~; «~Une personne qui a vieilli~» et donc «~Une personne âgée~») afin de rendre avec une périphrase le sens des mots latins. Il convient de préciser que les deux équivalents sont séparés graphiquement par un symbole vertical à l’encre rouge. 

\largerpage
L’entrée suivante, concernant la conjonction \emph{tum},\footnote{F. 477v.} est intéressante du fait des exemples illustratifs en langue parlée qu’on y retrouve (cf. \tabref{tab:gianninoto:3}).

\begin{table}
\caption{l'entrée 'Tum'}
\label{tab:gianninoto:3}
\label{tab:gianninoto:frequencies3}
\begin{tabularx}{\linewidth}{ QQ }
\lsptoprule
Tum copula

\footnotesize (alors, copule)

\normalsize Si. cosi. ancora.

\footnotesize (Oui, ainsi, encore) & {\cn{也。 亦。又。 這個也用著那個亦要。又要這個又要那個}}
 
iè   ĭ  ieú  v.g. cé kó iè iúng ciŏ, nákó ĭ iáo ieú iáo cé kó ieú iáo ná kó

\footnotesize (Aussi. Ainsi que. Encore. p. ex. J’utilise celui-ci, je veux aussi celui-là. Je veux celui-ci et aussi celui-là.) \\
\lspbottomrule
\end{tabularx}
\end{table}

Dans cette entrée, une série de synonymes sont présentés comme équivalents du mot latin, accompagnés de phrases de la vie quotidienne. Aucune définition ou explication de nature grammaticale n’est fournie. 

La première entrée du dictionnaire constitue un autre exemple d’entrée de nature grammaticale, où une série de mots fonctionnels (des prépositions) sont définis puis traduits par des équivalents (cf. \tabref{tab:gianninoto:4}).\footnote{F. 1r du dictionnaire.}

\begin{table}
\caption{L'entrée 'A, ab, abs, ex'}
\label{tab:gianninoto:4}
\label{tab:gianninoto:frequencies4}
\begin{tabularx}{\linewidth} { QQ }
\lsptoprule
A, ab, abs, ex

Praeposit.\textsuperscript{nes} servie\textsuperscript{tes} ablat.o casui

\footnotesize (de, dès, depuis, par

 Prépositions utilisées avec les cas ablatif.) & {\cn{從.  自.  由.  從小慣的}}
 
zûng. zú. iêu   v.g. zûng siào kuón dĭ

a puero usus \&c.

\footnotesize{(De, Depuis. Par.  P. ex. Utilisé par les enfants.)}\\
\lspbottomrule
\end{tabularx}
\end{table}

Il convient de souligner que cette première entrée, ainsi que les deux suivantes, sont écrites à l’encre rouge (à la différence des autres entrées, rédigées à l’encre noire). Par ailleurs, un caractère est raturé dans l’entrée et corrigé ({\cn{慣}}) sur le côté gauche de la page. Cette première entrée introduit une série de prépositions latines, dont les premières commencent par la lettre «~a~», et qui sont employées avec le cas ablatif pour exprimer l’origine, le point de départ spatial et temporel, la cause et l’agent. Trois prépositions chinoises sont proposées comme équivalents~: \emph{cóng} {\cn{從}} et \emph{zì} {\cn{自}} «~de, depuis~» et \emph{yóu} {\cn{由}} «~de, par~», dont la deuxième relève d’un registre plus soutenu et la troisième peut introduire l’agent. Dans cette entrée, nous trouvons par ailleurs une définition de ces mots qui en explique l’emploi~: «~prépositions utilisées avec les cas ablatif~», définition qui est reprise du dictionnaire de Calepino,\footnote{Où nous lisons «~\emph{A \& Ab praepositiones ablativo casui servientes significant}~» \citet[12]{Calderino1585}.} et abrégée dans l’entrée de Carlo da Castorano.\footnote{Dans l’entrée~: «~\emph{praeposit.\textsuperscript{nes} servie\textsuperscript{tes} ablat.\textsuperscript{o} casui}~», c’est-à-dire \emph{praepositiones servientes ablativo casui}.}

Cette entrée témoigne ainsi de l’influence des modèles occidentaux sur ce dictionnaire. L’auteur y introduit une définition de nature grammaticale reprise des sources latines. Il convient toutefois de souligner que ce dictionnaire ne propose généralement que les équivalents en chinois et des exemples, sans indications concernant leur emploi \citet[185]{Li2017}. Cet ouvrage était probablement conçu comme un outil pédagogique à utiliser en complément de la grammaire de l’auteur, contenue dans le même manuscrit.

\section{Les tentatives avortées de publier les dictionnaires}

Le trait le plus marquant que l’on retient toujours à propos du missionnaire franciscain est son opposition aux rites chinois, avec une telle véhémence que, une fois à Rome, il fut invité à limiter ses actions à ce propos.\footnote{Auteur de 37 mémoriaux sur la question, même après la publication de la bulle papale de condamnation définitive des rites chinois (1742), Castorano continua à solliciter les autorités pontificales sur ces questions, jusqu’en 1750 quand on lui ordonna de s’abstenir de ces démarches. Cf. \citet{Catto2013}, voir la version en ligne https://www.treccani.it/enciclopedia/antonio-orazi\_\%28Dizionario-Biografico\%29/ (dernier accès, le 09/04/2023).} Mais dans le cadre de notre présentation, le point marquant est que, contrairement à ses contemporains et ses confrères qui semblent se satisfaire et s’accommoder de l’usage des manuscrits, Castorano avait compris le potentiel de l’imprimerie pour diffuser ses travaux et standardiser l’information linguistique. Il chercha donc à plusieurs reprises à mettre en œuvre ce projet, car – comme c’est expliqué dans certains de ses écrits – il aspirait à un cercle de lecteurs plus large que les seuls religieux. Cependant, il semble avoir manqué une occasion unique de publier ses dictionnaires en passant par Paris, lors de son voyage de retour de la Chine en Europe, justement parce qu’il était trop préoccupé par les questions religieuses et donc pressé de se rendre à Rome.

\subsection{Le passage à Paris}

Castorano arrive en France le 23 août 1734~; pendant le voyage, il s’est occupé de la copie de l’un de ses dictionnaires (\citealt{Li2017} ~: 182). Selon un petit livre à son nom publié de manière posthume à Livourne en 1759 (\emph{Brevissima notizia, o relazione di varj viaggi, fatiche, patimenti, opere ec. nell'Imperio della Cina, ec . Del Reverendo padre F. Carlo Horatii da Castorano}), son navire arrive en Bretagne, dans le port Luigi o Ludovico (la citadelle de Port-Louis à Lorient~?). De ce lieu, Castorano se rend à Rehen (Rouen~?), puis à Versailles, où il aurait été reçu par le roi, la reine et le cardinal de Fleury, précepteur de Louis XV. Il va s’arrêter ensuite à Paris, avant de repartir pour le Sud, en s’embarquant à Marseille pour Livourne, puis en traversant la Toscane~: il arrive à Rome en novembre 1934 \citep[60--62]{Orazi1759}.

Pendant son séjourner parisien, il loge chez Monseigneur Raniero Delci (1670-1761), nonce apostolique nommé auprès du roi de France entre 1730 et 1738. À cette occasion Castorano rend visite au Cardinal de Polignac (Melchior de Polignac, 1661-1741) qui lui propose de rester à Paris pour imprimer aux frais du roi le dictionnaire latin-italien-chinois. C’est probablement à ce moment que le missionnaire italien a été introduit auprès d’Étienne Fourmont (1683-1745). Nous savons que Polignac communique avec Fourmont à propos des «~choses chinoises~»~: par exemple, lors d’une rencontre à l'Académie Royale des Inscriptions et Médailles portant sur la «~Bussole chinoise~» en date du 26 janvier 1734, Polignac, Fourmont et Nicolas Fréret (1688-1749) échangent sur le figurisme \citep[56--58, 275--276]{Leung2002}. Polignac avait donc la position pour mettre Castorano en contact avec Fourmont qui, à ce moment-là, était déjà engagé depuis une vingtaine d’années dans le projet de gravure de caractères mobiles en bois, afin de réaliser une «~fonte~» la plus complète possible de types chinois, et ainsi imprimer à Paris cette langue d’une façon adaptée aux pratiques typographiques d’impression européennes. Le projet de Fourmont est mieux connu par le nom des caractères chinois, appelés les «~buis du régent~»\footnote{Ces caractères mobiles en bois sont ainsi désignés parce que leur gravure a commencé sous la régence (1715--1723) de Philippe d’Orléans (1674--1723). Ils existent encore et sont conservés à l’Imprimerie nationale.}. Il avait débuté en 1615, lorsque l’abbé Jean-Paul Bignon (1662--1743), directeur des Académies royales et de la Bibliothèque du Roi, en confia la direction à Étienne Fourmont, assisté de son frère cadet, l’abbé Michel (1690-1746) et d’un chrétien chinois présent à Paris, Arcade Huang (Huáng Jiālüè {\cn{黃嘉略}}).\footnote{À propos d’Arcade Huang et sa collaboration avec Fourmont, voir \citet{Xu_minglong_2004, Sieber2013}.} Mais Huang décède en 1716. On comprend l’apport potentiel pour le projet français d’une collaboration avec Castorano, vu les connaissances linguistiques acquises sur le terrain par le missionnaire italien, à un moment où les personnes pouvant prétendre connaître la langue sont extrêmement rares.

\largerpage
Fourmont écrit d’ailleurs dans ses \emph{Meditationes sinicae} (1737) au sujet d’une rencontre avec Castorano – sous le nom de «~Casaranus~»\footnote{Il donne comme nom complet Carulus Oratius de Casarano, et il se trompe d’année, car il parle de 1735, donc un an plus tard que le passage de l’Italien à Paris, «~Praefatio~»: XXIV.} –, il cite sa grammaire et son \emph{Dictionarium italico-sinicum}, et il fait aussi mention d’une admiration réciproque. Plus tard, cette information est reprise par Jean-Pierre Abel-Rémusat (1788--1832) dans son \emph{Plan d’un dictionnaire chinois}. Le premier sinologue du Collège de France, non moins attaqué par ses détracteurs à propos de ses limites linguistiques que le fut Fourmont (les deux hommes ayant manqué d’un apprentissage direct de la langue chinoise, notamment de la langue orale), nomme élogieusement Castorano, tout en se référant aux manuscrits de Brollo présents à Paris (\citealt[9--12]{Abel-remusat1814}). Nous ne pouvons pas établir avec certitude si, dans ce passage, Abel-Rémusat commente le manuscrit «~Chinois 9278~» – un dictionnaire chinois-latin qui se trouve aujourd’hui à la Bibliothèque nationale de France et qui inclut une version de la grammaire de Castorano –, ou d’autres dictionnaires liés au nom de Brollo, par exemple le manuscrit «~Chinois 9279~», conservé dans la même bibliothèque.

\largerpage
\subsection{Les difficultés romaines}

Dès son retour en Italie, Castorano souhaite imprimer un volume de son dictionnaire trilingue. Deux versions de ce dictionnaire sont aujourd’hui conservées à la bibliothèque Vaticane, ainsi que quelques pages et des fragments d’un volume inabouti, témoignant de malheureuses tentatives de publier l’ouvrage.\footnote{Un volume est «~Vat. Est Or 4~», décrit plus haut. Mais l’exemplaire que Castorano souhaitait imprimer est un autre~: «~Est Or 5 A~»~; ce volume, avant d’entrer à la Bibliothèque Vaticane, aurait appartenu à la Propaganda. C’est probablement la même chose pour les quelques pages imprimées d’un volume incomplet, répertorié sous la côte Est Or 5 B.  Castorano aurait réalisé quatre copies manuscrites du dictionnaire, et une autre copie aurait été faite par un tiers (\citealt[95]{Civezza1879}).} Dans ce volume, le titre en haut de la première page est \emph{Dictionarium // Latino-Italico-Sinicum// Tàm vocum, quàm Charàcterum//usualium Sinensium} [Dictionnaire Latin, Italien, Chinois, avec les caractères usuels et leur prononciation] et le contenu est organisé dans la page imprimée d’une façon assez proche de celle utilisée dans les manuscrits décrits plus haut. Le latin, avec l’italien, est placé dans une colonne à gauche de la page, occupant moins d’un quart de la feuille~; les entrées et les équivalents sont écrits sur une, deux ou trois lignes superposées~; on y fait recours à des abréviations si le texte est trop long. Le chinois à droite occupe le reste de la page, avec la transcription phonétique et tonale au-dessous de chaque caractère~; points et marqueurs de division interviennent entre les exemples. Un examen ultérieur de l’original serait nécessaire pour le confirmer, mais ce qu’on observe semble être une association de types occidentaux et de matrices de bois, correspondant à la ligne ou à un segment de la ligne, sur lesquelles ont été gravés les caractères chinois. Le tout a été composé dans une grille qui divise chaque entrée et reconstitue la disposition des contenus d’une façon proche à celle du manuscrit. Au verso de la deuxième page est écrit «~Premières épreuves du dictionnaire malchanceux Latin, Italien, Chinois, à cause de l’iniquité et de la trahison du graveur Filippo Zenoby~» (\emph{Prime prove dell'//infausto dizionario // Latino=Italiano+Cinese // Per l’iniquità e tradimento // di Filippo Zenoby incisore//}).

Filippo Zenobi est le descendant d’une famille d’imprimeurs et de graveurs, originaire de Macerata (dans les Marches)~: il est légitime de se demander si la région d'origine, la même que Castorano, peut avoir influencé ce dernier dans le choix d’un collaborateur de cette famille spécialisée dans la profession depuis plusieurs générations. Filippo est le fils d’Antonio (graveur du bois, mort avant 1729), qui était un de trois fils de Gaetano Zenobi (1663?--1726). Gaetano, le grand-père de Filippo, est le petit-fils de Carlo, déjà dans la profession à Macerata~; il a commencé son activité en tant que \emph{stampatore pubblico}, dans la ville de Foligono, près de Pérouse (Ombrie) et il s’est ensuite déplacé à Rome, où il semble être présent au moins dès 1691 ou 1693 \citep[3]{Marinellinodate}. Gaetano travaille pour le pape Clément X (1700-1721) comme imprimeur et graveur, et il sera aussi au service de deux papes successifs, Innocent XIII (1721-1724) et Benoît XIII (1724-1730). Les générations suivantes de la famille Zenobi restent dans la profession, mais les descendants semblent avoir moins de succès. En plus des publications pour l'Église, ils travaillent pour des éditions théâtrales romaines~:  Filippo Zenobi apparaît comme éditeur installé à la place Capranica, près du théâtre Capranica \citep[139--140]{FranchiSartori2002}, dans les mêmes années où Castorano lui aurait consigné son ouvrage plurilingue. Il est proche de Girolamo Mainardi (1679?-1763) qui, depuis 1726, a obtenu la gestion de la «~Stamperia Camerale~» à côté du théâtre.\footnote{À propos de Mainardi, éditeur bien connu qui aurait même relevé la gestion du théâtre pendant quelques mois entre 1728 et 1729, voir la notice de \citet{Franchi2006} en ligne dans le \emph{Dizionario biografico italiano} (dernier accès le 2023-04-08) \url{https://www.treccani.it/enciclopedia/girolamo-mainardi_\%28Dizionario-Biografico\%29}.} Filippo Zenobi travaille en tant que graveur sur bois pour certaines publications de Mainardi, et il est encore en activité dans les années 1750.\footnote{Dans le catalogue OPAC SBN (‘Catalogue collectif des bibliothèques du Service national des bibliothèques’ d’Italie), on compte une dizaine de volumes au nom de Filippo Zenobi, publiés pour la plupart vers 1734, puis dans les années 1750; signalons aussi un \emph{Processionario} de 1754, publié par Mainardi, présentant un frontispice illustré avec les noms de F. Zenobi et C. Grandi (en ligne: “Online ecclesiastical heritage”, \url{https://www.beweb.chiesacattolica.it/subeweb/}, dernier accès le 08/04/2023).}  Il est tout à fait vraisemblable que Castorano l’ait contacté pour graver les caractères chinois sur bois. Un autre franciscain, Joseph Maria da Evora (1690-1752), aurait initialement aidé pour réunir les fonds nécessaires à cette entreprise (\cites[9]{Franceschi2017}[96]{Civezza1879}), mais sa nomination à Porto l’éloigna finalement de son confrère et de ces projets éditoriaux \citep[179]{Li2017}.
 Dans les Archives de la Propagande se trouve un mémorial adressé au Pape par Castorano, en date du 10 janvier 1735, juste après son retour à Rome.\footnote{Congregazione “de Propagande Fide”, Archivio Storico, Sc. Stamperia, vol. 2 (1721--1770), ff. 109--110.} Le missionnaire explique que, pendant son séjour en Chine, il s’est donné beaucoup de mal afin d’apprendre les caractères chinois qui sont extrêmement difficiles (\emph{...affatigato ad imparare li caratteri di quella Nazione, quali sono molti, e difficili}). Il présente ensuite son dictionnaire trilingue~: il ne contient pas seulement les mots chinois en romanisation, mais aussi des caractères chinois, et ce qui est utile dans la langue courante~; il recense toutes les locutions et les phrases nécessaires pour rendre la langue plus accessible~; il est accompagné d’une grammaire. Castorano explique avoir compilé ce dictionnaire pour les missionnaires destinés à se rendre en Chine, mais aussi afin qu’il soit mis au service des marchands actifs à l’international et des personnes qui étudient les langues étrangères et les ‘choses curieuses’ (\emph{...per comodità e utilità non solo dei nuovi Padri Missionari che andaranno a  imphiegarsi nelle Sacre Missioni della Cina, ma anche dei Mercanti Europei che trafficano in quel Regno, e di chiunque si diletta di lingue e studii forastieri e curiosi}). Cette compilation lui a pris beaucoup de temps et de travail, et Castorano supplie Sa Sainteté de donner l’ordre d’imprimer immédiatement son dictionnaire, car il veut s’occuper de ce travail pendant qu’il se trouve encore à Rome (\emph{...per il tempo che restarà in Roma, potria dirigere la stampa al modo cinese...}). Il explique qu’il sera trop difficile de publier le dictionnaire pour les personnes qui ne connaissent pas la langue chinoise (\emph{...per essere molto difficile tale dittionario à caratteri cinesi stampati da chi ne sa poco, e molto più da chi non ne sa niente...}), point sur lequel nous ne pouvons que lui donner raison~: Castorano connait le chinois et il est conscient de la difficulté de cette entreprise éditoriale. Effectivement, vu le nombre extrêmement élevé des caractères chinois, ce n’est pas seulement l’impression qui pose des difficultés à des non-sinophones, mais la gestion et le rangement de l’ensemble des pièces, comme on le sait pour le cas français.\footnote{Avec des milliers de caractères, chaque pièce mal rangée correspond à un caractère perdu définitivement~; c’est pour cette raison qu’en France, pour les «~buis de régent~», est théorisée au XVIIIe siècle une gestion rigoureuse. Il faut éviter que les caractères soient divisés ou déplacés inutilement par des personnes qui, ne connaissant pas le chinois, ne pourraient pas les remettre à leur place \citep[77--78]{De_guignes1790}} Dans le même texte il est encore question des mérites que le représentant de l’église tirera d’une telle entreprise nécessaire aux activités de prosélytisme. Enfin, la conclusion nous apprend que l’impression du dictionnaire original de Brollo a été interrompue parce que ce dictionnaire contenait trop d’erreurs~; Castorano suggère donc d’utiliser sa propre copie (corrigée) du volume de Brollo à cette fin.\footnote{Il s’agit probablement, pour l’original, du manuscrit «~BAV EO 2~» du dictionnaire de Brollo~; quant à la copie du dictionnaire de Brollo par Castorano, il devrait s’agir de «~BAV EO 8~». À propos des tentatives inachevés d’imprimer les dictionnaires de Brollo, pour lesquelles fut vraisemblablement sollicité Matteo Ripa (1682-1746) dans les années immédiatement précédentes au retour de Castorano en Italie et sans que cela aboutisse, voir \citet[376--378]{Bussotti2015}.}

Un autre texte conservé à la Propaganda est un mémorial de l’année suivante (1736) au pape Clément XII.\footnote{Cf. Congregazione de Propagande Fide, Archivio Storico, Sc. Stamperia, vol. 2 (1721--1770), ff. 111--113.}  On comprend qu’aucune version de dictionnaire chinois n’a été finalement retenue pour l’impression.\footnote{En plus du dictionnaire trilingue (BAV EO 5) et de la version de Castorano du dictionnaire Chinois-Latin de Brollo (BAV EO 8), dans la lettre est mentionné le dictionnaire de Brollo transcrit en Chine par Telli (probablement BAV EO 2) auquel serait allée la préférence de Mainardi «~imprimeur caméral~», c’est-à-dire gérant la Stamperia Camerale pour le Pape: «~\emph{il Signor Mainardi stampatore camerale diede memoriale a vostra Santità di voler stampare non essi,  ma quello addotto in mezzo dal signor Filippo Telli.}~»} Castorano présente des excuses pour les modifications qu’il a apportées au dictionnaire de Brollo et qui ont été mal accueillies. Mais il se justifie par le fait qu’il n’existait pas de copie originale en Europe et que toutes les versions du dictionnaire étaient discordantes. Pour ses corrections, il s’est basé donc sur sa propre copie du dictionnaire de Brollo en faisant appel à ses connaissances,\footnote{Il écrit~: «~\emph{onde non non essendovi in Roma, né in Europa, l'originale del dizionario del R. P. Gemona e gli altri esemplari trascritti da mano in mano, essendo con molti errori, nè concordando l’uno con l’altro, l’oratore stimo bene dovere correggerlo e secondo quello trascritto da sé in Cina (nel 1706) e secondo la scienza è notizia che esso ha della lingua e caratteri cinesi}~».} tout en signalant ses éventuels ajouts~; ces amendements lui semblaient d’autant plus nécessaires que l’impression serait couteuse, et il était donc inopportun de proposer un texte erroné. Dans cet écrit, le franciscain demande finalement à être libéré de cette tâche ingrate et il mentionne Mainardi comme l’imprimeur chargé par le pontife de publier le dictionnaire.\footnote{Voir les notes n. 39 et 45.} On peut imaginer que, dans ce contexte, Castorano soit entré facilement en contact avec les Zenobi~: non seulement Filippo est voisin de Mainardi vers 1730, mais il travaille parfois pour lui dans les années qui suivent.

Enfin, un fragment du journal de Castorano nous apprend qu’en 1742, il présenta au Pape deux livres «~à imprimer pour le bien public~», à savoir son dictionnaire trilingue et la \emph{Parva elocubratio} comprenant le catalogue de la bibliothèque de Giovanni Francesco Nicolai, décrite plus haut (§ 3.1). Castorano explique qu’il était déjà en contact avec la \emph{Propaganda fide} pour ces impressions, mais qu’il aurait aimé recevoir des fonds du pontife pour ce projet. Mais ce dernier ne donna pas une réponse positive, en ordonnant même à Castorano de reprendre ces livres que, par ailleurs, il connaissait déjà.\footnote{\citet[450]{DiFiori1989}, qui cite le «~Document 26~», Fragment du journal de Carlo Orazi da Castorano 1735 -1742, Bibliothèque nationale de Naples, \emph{Manoscritti Castorano seng}. XI B 73.}

\section{Conclusion}

Le \emph{Dictionarium Latino Italico Sinicum} est un dictionnaire monumental, d’un grand intérêt non seulement en raison du nombre de ses entrées, mais aussi des exemples illustratifs proposés, ainsi que de la présence de mots, expressions et collocations relevant de divers registres. Dans le processus de rédaction, l’auteur s’est appuyé sur deux des dictionnaires européens les plus répandus (les ouvrages de Calderino et Calepino), mais aussi sur les dictionnaires de chinois écrits en langues européennes par ses confères (notamment par Basilio Brollo), ainsi que sur la lexicographie sino-mandchoue. Le dictionnaire trilingue de Castorano se situe au carrefour des traditions lexicographiques occidentales et chinoises, tout en ayant également recours à des informateurs. 

Comme le souligne \citet[176]{Li2017}, la richesse des entrées, les nombreux exemples et la présence de mots de la langue parlée contribuent à la valeur de ce dictionnaire pour la reconstitution du chinois mandarin de l’époque, ainsi que dans l’histoire de l’apprentissage du chinois. 

L’ouvrage de Carlo da Castorano est représentatif de l’apogée de la lexicographie missionnaire italophone en Chine, mais aussi de sa dernière étape. Comme le rappelle Castorina~: «~avec Brollo et Castorano, la période dorée des études italiennes sur la Chine se clôt\footnote{Texte original italien~: «~\emph{Con Brollo e Castorano si chiude così il periodo d’oro degli studi italiani sulla Cina}~».}~» (\citealt[79]{Castorina2019}; notre traduction). Mais les tentatives infructueuses de Castorano pour faire financer l’impression témoignent aussi des difficultés liées au contextes social, économique et politique (\citealt{Raini2017, Castorina2019}), en particulier de la difficulté à imprimer des ouvrages en chinois en Europe, sans un soutien extérieur important, comme ce fut le cas à Paris pour Fourmont et ses «~buis du régent~». Utilisés en petite partie pour les œuvres de Fourmont, ces types de bois devront attendre Napoléon et l’Empire pour servir encore une fois à l’édition du dictionnaire Chinois-Latin-Français de 1813, un projet politique et de prestige (cf. \citealt{BussottiLandry-deron2020}). La papauté, soucieuse des questions doctrinales et rituelles, tournée vers d’autres terres de mission que la Chine, ne développe pas un vrai programme éditorial concernant le chinois et ceci malgré le fait que les missionnaires, hommes de terrain, auraient été les premiers occidentaux à détenir toutes les compétences pour envisager en Europe un projet d’édition et d’impression du chinois viable dans le temps.

{\sloppy\printbibliography[heading=subbibliography,notkeyword=this]}
\end{otherlanguage}

\renewcommand{\tabref}[1]{Table~\ref{#1}}
\end{document}
