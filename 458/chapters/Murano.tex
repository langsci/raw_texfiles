\documentclass[output=paper,booklanguage=french]{langscibook}
\ChapterDOI{10.5281/zenodo.15394505}
\author{Michela Murano\affiliation{Università Cattolica del Sacro Cuore, Milan, Italy}}
\title[Deux recueils phraséologiques bilingues franco-italiens du 19e siècle]
      {« Sul finir d’imparare la Grammatica Francese, fa d’uopo studiar il Dizionario delle Frasi » : Deux recueils phraséologiques bilingues franco-italiens de la première moitié du 19e siècle}

\abstract{In this paper we provide a metalexicographical analysis of two French-Italian phraseological dictionaries published in Naples, in Italy, in the first half of the 19th century~: the \emph{Fraseologia italiana-francese} by Giuseppe \citet{Trucchi1825} and the \emph{Dictionnaire des phrases françaises-italiennes} by Nicolas \citet{Lanza1837}. These two works are composite but appear to be devoted mainly to the description of phraseological units of various types~: in fact, both authors are convinced that knowing phraseology is complementary to the study of grammar and is indispensable for perfecting the knowledge of the French language. Our aim is to detect the specificities of these works at the level of the target audience, the chosen nomenclature and the lexicographic techniques implemented, through the exploration of the paratext, the macrostructure and the microstructure.}

\IfFileExists{../localcommands.tex}{
  \addbibresource{../localbibliography.bib}
  \usepackage{langsci-optional}
\usepackage{langsci-gb4e}
\usepackage{langsci-lgr}

\usepackage{listings}
\lstset{basicstyle=\ttfamily,tabsize=2,breaklines=true}

%added by author
% \usepackage{tipa}
\usepackage{multirow}
\graphicspath{{figures/}}
\usepackage{langsci-branding}

  
\newcommand{\sent}{\enumsentence}
\newcommand{\sents}{\eenumsentence}
\let\citeasnoun\citet

\renewcommand{\lsCoverTitleFont}[1]{\sffamily\addfontfeatures{Scale=MatchUppercase}\fontsize{44pt}{16mm}\selectfont #1}
  
  %% hyphenation points for line breaks
%% Normally, automatic hyphenation in LaTeX is very good
%% If a word is mis-hyphenated, add it to this file
%%
%% add information to TeX file before \begin{document} with:
%% %% hyphenation points for line breaks
%% Normally, automatic hyphenation in LaTeX is very good
%% If a word is mis-hyphenated, add it to this file
%%
%% add information to TeX file before \begin{document} with:
%% %% hyphenation points for line breaks
%% Normally, automatic hyphenation in LaTeX is very good
%% If a word is mis-hyphenated, add it to this file
%%
%% add information to TeX file before \begin{document} with:
%% \include{localhyphenation}
\hyphenation{
affri-ca-te
affri-ca-tes
an-no-tated
com-ple-ments
com-po-si-tio-na-li-ty
non-com-po-si-tio-na-li-ty
Gon-zá-lez
out-side
Ri-chárd
se-man-tics
STREU-SLE
Tie-de-mann
}
\hyphenation{
affri-ca-te
affri-ca-tes
an-no-tated
com-ple-ments
com-po-si-tio-na-li-ty
non-com-po-si-tio-na-li-ty
Gon-zá-lez
out-side
Ri-chárd
se-man-tics
STREU-SLE
Tie-de-mann
}
\hyphenation{
affri-ca-te
affri-ca-tes
an-no-tated
com-ple-ments
com-po-si-tio-na-li-ty
non-com-po-si-tio-na-li-ty
Gon-zá-lez
out-side
Ri-chárd
se-man-tics
STREU-SLE
Tie-de-mann
}
  \togglepaper[18]%%chapternumber
}{}

\begin{document} 
\begin{otherlanguage}{french}
\maketitle

\section{Introduction}

Au 19e siècle, la présence d’un nombre considérable de recueils qui mettent en regard les phraséologies française et italienne témoigne de l’intérêt que les didacticiens et les lexicographes portent à la partie figée du lexique. On repère aussi bien des parties d’ouvrages de grammaire ou de conversation consacrées à la phraséologie que des recueils indépendants, dont certains relèvent de véritables projets lexicographiques.

En Italie, l’enseignement de la langue française devient à cette époque une discipline scolaire qui touche un vaste public, comprenant des élèves issus de couches sociales même modestes (\citealt{Pellandra1997, Minerva1998}). Les recueils phraséologiques, qui font partie du matériel didactique complémentaire élaboré par des professeurs, tout comme les manuels de lecture ou les recueils de dialogues, n’ont pas encore fait l’objet d’études approfondies pour le couple de langues français-italien\footnote{Un ouvrage magistral a par contre vu le jour pour le couple de langues français-hongrois \citep{Bardosi2017}.}~: nous avons récemment constitué un premier corpus de recueils \citep{Murano2017} et entamé l’analyse par deux des rares ouvrages publiés en France \citep{Murano2020}~; des études pointues sur un manuscrit de Goudar \citep{Lillo2014} et sur les dictionnaires phraséologiques commerciaux \citep{Sclafani2019} ont également vu le jour~; la corrélation entre la terminologie phraséologique adoptée et la typologie des phrasèmes recensés a été étudiée par \citet{Autelli2022} dans les \emph{phraséologies} de la fin du 19e siècle.

Dans cet article, nous nous proposons de présenter deux recueils phraséologiques parus en Italie dans la première moitié du siècle, dans les décennies 1820 et 1830 et publiés chez un même éditeur à Naples. Tout en étant des textes composites, ils paraissent consacrés en priorité à la phraséologie, si l’on en juge les titres~: Giuseppe Trucchi a publié une \emph{Fraseologia italiana-francese} en 1825, Lanza un \emph{Dictionnaire des phrases françaises-italiennes} en 1837. 

Après quelques considérations sur la phraséodidactique du Français Langue Étrangère en Italie au 19e siècle par le biais des études de manuels et un bref aperçu sur les recueils phraséologiques bilingues publiés à la même époque, les recueils de Trucchi e Lanza seront situés dans la production didactique et lexicographique de chaque auteur~; ensuite, nous analyserons la composition des ouvrages et le paratexte (titres, frontispices, préfaces), afin d’établir les motivations qui ont poussé les auteurs à les écrire, les objectifs pédagogiques, le public visé et les sources~; puis nous estimerons la nomenclature et nous essaierons de déterminer les critères de rangement des entrées. Enfin, l’analyse de la microstructure permettra de vérifier la présence éventuelle de rubriques lexicographiques.

\section{La phraséologie dans le matériel pédagogique (1820--1830)}

Dans les années 1990, les historiens de l’enseignement du Français Langue Étrangère (FLE) membres du CIRSIL (\emph{Centro Interuniversitario per la Ricerca sugli Insegnamenti Linguistici}\footnote{\url{https://cirsil.it}}) se sont intéressés à la phraséologie dans le cadre d’études sur les manuels de FLE en Italie. L’étude la plus significative est celle de Lillo en 1994, dans laquelle, en prenant en compte la structure générale des manuels de l’époque, la phraséologie est définie d’un point de vue pragmatique~: «~ce qui n’est ni phonétique, ni morphologie, ni syntaxe, ni lexique, ni dialogues, ni correspondances ou morceaux choisis, ni exercice, ni versification~» \citep[70--71]{Lillo1994}. Sur la base de cette définition et se référant aux manuels publiés entre 1625 et 1860, Lillo repère trois façons différentes de présentation de la phraséologie~: les deux premières se réfèrent à la présentation à l’intérieur des manuels, dans une section à part bien distincte ou disséminée dans les autres sections («~phraséologie périphérique~», \citep[71]{Lillo1994}; la troisième correspond à l’organisation en recueils ou en dictionnaires.

Concernant ce troisième mode de présentation du matériel phraséologique, dans une étude précédente \citep{Murano2017} nous avons essayé d’évaluer le nombre de recueils phraséologiques bilingues franco-italiens en croisant les données des répertoires existants pour les dictionnaires bilingues \citep{Lillo2019} et les manuels de FLE (\citealt{Pellandra1997, Minerva1996}), auxquelles nous avons ajouté les résultats de recherches dans les catalogues nationaux \emph{Opac Servizio Bibliotecario Nazionale} et \citet{[CCF]}. Une augmentation exponentielle du nombre de recueils peut être observée au 19e siècle, au cours duquel 66 recueils\footnote{Nous adoptons ici la dénomination \emph{recueil} en tant qu’hyperonyme, car à l’étape actuelle de nos recherches, n’ayant pu consulter qu’une partie des ouvrages du corpus, il ne nous est pas possible de savoir dans quels ouvrages les entrées sont rangées selon l’ordre alphabétique et combien d’ouvrages relèvent d’un projet lexicographique.} sont publiés, contre 8 au total dans les deux siècles précédents. La publication de la plupart des recueils se situe dans la deuxième moitié du siècle, mais dans la période 1820--1839 on enregistre déjà la parution d’un nombre appréciable de recueils (15).

Si l’on prend en compte la périodisation proposée par Lillo pour les manuels de FLE, les ouvrages que nous allons examiner dans cette étude, publiés en 1825 et 1837, se situent au tournant entre une période (1780--1830) dans laquelle les manuels privilégient la partie syntaxique et accordent peu de pages à la phraséologie, située surtout en position périphérique, et une période (1830--1860) dans laquelle l’intérêt pour la phraséologie augmente et plus de pages y sont consacrées. L’apparition d’un grand nombre de recueils phraséologiques bilingues s’expliquerait donc jusqu’à 1830 par la nécessité de compenser «~la pénurie de pages~» consacrées à la phraséologie dans les manuels, et après 1830 par un regain d’intérêt pour ce sujet.

Dans une étude récente \citep{Murano2020}, nous nous sommes penchée sur deux des rares recueils phraséologiques bilingues publiés dans les années 1820 en France, le \emph{Petit trésor de la langue française et de la langue italienne} de Giuseppe Filippo \citet{Barberi1821} et le \emph{Dictionnaire des idiotismes Italiens-Français} de Giacomo \citet{Polesi1829}, qui, tout en poursuivant un même objectif, celui de faciliter la compréhension des textes littéraires italiens, correspondent à deux typologies de recueils très différentes. Le \emph{Petit trésor} de Barberi se présente comme un \emph{traité de tropes} composé de plusieurs parties, parmi lesquelles les «~Figures dans l’association de plusieurs mots~» et une «~Phraséologie familière~», qui ne suivent aucun critère de présentation des entrées. Les «~Figures dans l’association de plusieurs mots~» présentent cependant l’intérêt d’être structurées en blocs de texte qui suivent un ordre de présentation régulier des informations (citation tirée d’un texte littéraire, forme lemmatisée, explication du sens). Le \emph{Dictionnaire des idiotismes de Polesi}, comme son titre l’indique, est un dictionnaire phraséologique au sens moderne du terme, qui suit un ordre alphabétique strict de présentation des entrées et différencie les informations fournies (traduction littérale, équivalent, explication, renvois synonymiques) par l’emploi de différents caractères typographiques.

Dans les paragraphes suivants, nous présenterons deux recueils publié, comme ceux de \citet{Barberi1821} et \citet{Polesi1829}, après 1820 et avant 1840, au moment de la première hausse considérable dans le nombre de publications de recueils phraséologiques franco-italiens. Ce premier critère de sélection chronologique se double d’un critère d’unité spatiale, car il s’agit de deux ouvrages publiés dans une ville de l’Italie du Sud, et donc vraisemblablement destinés à un public italophone. Les objectifs didactiques seront donc foncièrement différents par rapport à ceux des recueils de Barberi et Polesi, destinés à un public francophone apprenant l’italien «~sans maître~» \citep[16]{Dubois2015}.

\section{La \emph{Fraseologia italiana-francese} (1825) et le \emph{Dictionnaire des phrases françaises-italiennes} (1837)}

\subsection{Les auteurs et leur production}

Le recueil phraséologique de Giuseppe Trucchi, qui se définit \emph{maître de la langue française} («~maestro di lingua francese~») a eu plus de deux éditions, ce qui est assez rare pour ce type d’ouvrages \citep[171]{Murano2017}~: la \emph{Fraseologia italiana-francese} publiée en 1825 chez Seguin, que nous analysons dans cet article, est en effet la deuxième édition augmentée (on lit dans le frontispice «~Corredata di utili giunte~» ) d’un premier ouvrage publié en 1812 avec le même titre~; une troisième édition voit le jour en 1854. La \emph{Prefazione} de 1825 fournit des renseignements précieux sur le succès de la première édition de 1812, imprimée et vendue en 1500 exemplaires, qui a poussé l’auteur à travailler à la deuxième édition plutôt qu’à réimprimer la première~:

\begin{quote}
    «~La pubblica approvazione ch’ebbe allora la mia \emph{Fraseologia}, e le continue istanze che n‘ebbi, dopo averne esaurite più di mille cinquecento copie, per farla ristampare, mi hanno fatto decidere di procedere alla seconda edizione […]»
\end{quote}

Trucchi est également l’auteur d’autres ouvrages didactiques publiés dans les décennies 1820 et 1830 chez différents éditeurs à Naples~: il s’agit d’un recueil de dialogues \citep{Trucchi1834} et deux suppléments aux grammaires (\citep{Trucchi1831}  et \citep{Trucchi1837}, qui annoncent dans le frontispiece la présence d’une partie consacrée à la phraséologie (un «~elenco di frasi particolari, proverbj e gallicismi~», «~una lunga lista di gallicismi~»)\footnote{Une «~liste de phrases particulières, proverbes et gallicismes~» \citep{Trucchi1834}, «~une longue liste de gallicismes~» \citep{Trucchi1825}. Notre traduction.}.

De son côté, Nicolas Lanza paraît n’avoir publié que le \emph{Dictionnaire des phrases françaises-italiennes} en 1837. 

Les deux ouvrages sont publiés chez Giovanni Battista (Jean Baptiste) Seguin à Naples\footnote{En croisant nos données avec celles du répertoire des dictionnaires bilingues \citep{Lillo2019}, nous relevons que Seguin a édité en 1846--1847 un seul dictionnaire bilingue, le \emph{Vocabolario militare di marineria francese-italiano} de Giuseppe Parrilli. Le catalogue SBN révèle en outre qu’au moins deux grammaires du français ont paru chez Seguin dans les mêmes années, celle de Merle en 1826 et celle de De Willot en 1829.}, une ville qui avait été, avec Venise et Turin, l’un des centres d’édition les plus actifs pour les manuels de français au 18e siècle \citep[102]{Minerva1996}~: dans la première moitié du 19e siècle, Naples est encore un centre d’édition bien représenté en ce qui concerne les recueils phraséologiques, mais sera devancé par Milan dans la deuxième moitié du siècle.

Il est difficile d’estimer la diffusion de ces recueils sur le territoire, mais il nous semble intéressant de prendre en compte les exemplaires recensés dans les catalogues nationaux des bibliothèques italiennes et françaises (\citep{[OPAC-SBN]} et \citep{[CCF]})~: aucun exemplaire de ces recueils n’est repérable dans les catalogues français, ce qui laisse supposer qu’ils n’ont pas circulé en France~; au contraire, dans les bibliothèques italiennes on trouve 5 copies du \emph{Dictionnaire} de Lanza (3 dans des villes du Sud de l’Italie, 2 dans le Centre) et 8 exemplaires de la \emph{Fraseologia} de Trucchi de 1825\footnote{Dans \citep{[OPAC-SBN]} on repère également 4 exemplaires de l’édition de 1812 et 7 exemplaires de l’édition de 1854. } (à Naples, Palerme, Bari, Potenza). Ces chiffres sont comparables avec ceux d’un dictionnaire portatif publié dans les mêmes années, comme la quatrième édition du dictionnaire de Cormon et Manni publiée à Paris en 1823, dont on dénombre 6 exemplaires dans le catalogue OPAC SBN.

\subsection{Le titre et la composition des ouvrages}

Les titres des recueils pris en compte dans ce travail reflètent les deux tendances majeures repérées dans les recueils phraséologiques franco-italiens \citep[174]{Murano2017}, à savoir l’emploi comme premier mot d’un hyperonyme ou d’un terme collectif~: Lanza choisit, comme l’avait fait pour la première fois Giacomo Polesi en 1829, la dénomination \emph{Dictionnaire} «~qui ferait supposer l’existence d’un ordre (alphabétique ou autre) et d’un programme d’information régulier dans [la] microstructure~» \citep[174]{Murano2017}~; quant à Trucchi, il utilise à partir de 1812, premier parmi les auteurs de recueils phraséologiques, le nom collectif \emph{Fraseologia}, qui reflète le contenu de l’ouvrage. Le mot \emph{fraseologia}, absent des dictionnaires de la Crusca jusqu’à la quatrième édition de 1729--1738\footnote{Comme le signale le \emph{Dizionario Etimologico della Lingua Italiana} \citep{CortelazzoZolli1980}), le mot se trouve dans un ouvrage \citep{Bergantini1745} qui recense les mots utilisés par des auteurs approuvés par la Crusca, mais non enregistrés dans le \emph{Vocabolario}~: sous la forme \emph{frasilogía}, il avait était employé par G. Caracciolo au 18e siècle avec le sens de «~Arte, che infegna la fcelta delle buone frafi~» («~Art qui apprend le choix des bonnes phrases~», notre traduction).}, avait désormais acquis au début du 19e siècle le sens de ‘recueil de phrases’ («~raccolta di frasi, o di espressioni eleganti in qualche lingua~»\footnote{«~Recueil de phrases, ou d’expressions élégantes dans une langue~» (notre traduction).}, \citep{Bonavilla1819}).

Loin du modèle contemporain de dictionnaire bilingue, qui comprend deux parties principales de même ampleur, chacune ayant une nomenclature dans une langue et fournissant des équivalents dans une autre langue, ces ouvrages ne présentent pas une organisation spéculaire et sont formés de plusieurs parties, qui peuvent ne pas s’apparenter au dictionnaire. La \emph{Fraseologia} de Trucchi contient d’abord trois parties qui recueillent des \emph{phrases}~: \emph{Frasi / Phrases}\footnote{Nous rapportons ici les titres des trois premières parties tels qu’ils sont imprimés au début de chaque partie. Dans le texte de présentation de la \emph{Fraseologia}, à la p.4 Trucchi emploie d’autres dénominations, toutes en italien, comme \emph{Frasi per ordine alfabetico} qui correspond à \emph{Frasi / Phrases}.} (pp.5--49)~; \emph{Frasi particolari / Phrases particulières}, (pp.50--64)~; \emph{Proverbi / Proverbes} (pp. 54 --69). Les autres parties constituent des compléments variés à la grammaire, dont le poids en nombre de pages (environ la moitié du total) ne permet pas de les considérer comme du simple paratexte. Il s’agit d’un traité d’orthographe française (\emph{Trattato di ortografia}, pp.70--82), suivi de plusieurs listes de mots~: des homonymes et paronymes français avec leurs équivalents (\emph{Catalogo di voci di quasi simile pronuncia, ma di significato o d’ortografia dissomiglianti}, pp.83--94)~; des mots qui ont un genre différent en italien et en français (\emph{Catalogo di parecchi nomi sostantivi che in italiano hanno un genere, ed un altro in francese} […], pp.95--101)~; des mots français qui ont deux genres (\emph{Sostantivi di doppio genere del medesimo significato}, pp.102--103)~; une liste thématique de verbes français (\emph{Raccolta dei verbi i più usati nel discorso}, pp.104--112)~; des mots italiens polysémiques ou homonymiques avec leurs équivalents (\emph{Espressioni teoriche particolari della lingua francese}, pp.113--125).

Le \emph{Dictionnaire} de Lanza est également un ouvrage composite, dont voici la liste des sept parties~: \emph{Phrases précédés des Verbes / Frasi precedute dai verbi} (pp.7--136)~; \emph{Façons de parler / Modi di spiegarsi} (pp.137--273)~; \emph{Expressions adverbiales / Espressioni avverbiali} (pp.274--300)~; \emph{Termes élégants / Termini eleganti} (pp.301--352)~; \emph{Phrases proverbiales / Frasi proverbiali} (pp.353--378)~; \emph{Proverbes / Proverbj} (pp.379--386)~; \emph{Mots d’usage / Voci di usanza} (pp.387--391). Excepté les \emph{Termes élégants} et les \emph{Mots d’usage}, les titres des autres parties renvoient à la présentation d’items polylexicaux relevant de catégories grammaticales spécifiques (\emph{Phrases précédés des Verbes~; Expressions adverbiales}) ou non (\emph{Façons de parler}), ou bien d’unités phraséologiques atteignant la dimension de la phrase (\emph{Proverbes~; Phrases proverbiales}).

\subsection{Destinataires, objectifs et sources des ouvrages}

L’analyse du paratexte révèle les intentions pédagogiques des auteurs et les raisons qui les ont poussés à écrire les recueils.

La \emph{Fraseologia} de Trucchi est destinée aux jeunes amateurs des lettres napolitains («~[…] la studiosa gioventù amante delle lettere~» p.3, «~la virtuosa gioventù Napoletana~» p.4). Dans sa \emph{Prefazione} \citep[3--4]{Trucchi1825}, l’auteur affirme qu’en dépit du large nombre d’étudiants qui entreprennent l’étude de la langue française, rares sont ceux qui la maîtrisent parfaitement, car la plupart pensent connaître la langue après avoir étudié la grammaire et appris par cœur quelques mots. La publication d’un recueil des \emph{phrases} les plus utilisées («~più usitate~») en français et en italien, entreprise à laquelle il se consacre à partir de 1810, s’avère d’autant plus nécessaire que ce \emph{maître de la langue française}\footnote{«~Maestro della lingua francese~», dans le frontispiece.} constate que peu de précepteurs se donnent la peine d’enseigner les \emph{phrases} du français («~rari sono i Precettori che si danno l’incommodo d’insegnarne le frasi~», \citep[4]{Trucchi1825}.

Dans son texte de présentation (\emph{L’Autore}, pp. 3--4), Lanza dénonce les risques de la traduction mot à mot («~alla lettera~», «~con semplici vocaboli~», «~parola per parola~», p.4) comme technique d’apprentissage, car elle ne peut pas mener à une connaissance parfaite de la langue française pour des apprenants italiens et au contraire est source de malentendus en compréhension et en production~: 

\begin{quote}
    «~coloro che così imparato l’avessero, se trovansi poi a parlare co’ Francesi istessi, difficile li si rende d’intendere perfettamente il di loro linguaggio, e maggior difficoltà li si appresenta se esprimer si volessero co’ medesimi traducendo parola per parola il proprio discorso~» (\emph{ibidem}) 
\end{quote}

La maîtrise de la phraséologie parfait la connaissance du français et son apprentissage par cœur («~se n’impara il difficile a memoria~», \emph{ibidem}) doit survenir vers la fin de l’apprentissage de la grammaire~: 

\begin{quote}
    «~È perciò che sul finir d’imparare la Grammatica francese, fa d’uopo studiar il Dizionario delle Frasi~»
\end{quote}

L’objectif de son Dictionnaire est de faciliter et de rendre plus rapide l’étude de la langue française («~un largo capo ho aperto nello studio di questa bella lingua, in breve tempo e con pochissima fatica~», \emph{ibidem}), car dans les textes français et dans les autres ouvrages à visée didactique, comme les dictionnaires et les dialogues, l’apprenant est obligé de «~mendier~» les expressions, qui s’y trouvent éparpillées sans aucun ordre.

Concernant les sources des recueils, Trucchi affirme avoir puisé dans les meilleurs auteurs classiques, alors que le frontispiece de Lanza cite «~les dialogues les plus accrédités, et […] les plus beaux traits d’éloquence de la rhétorique française~». On ne repère cependant aucune citation datée ou signée, qui nous permette de remonter à des auteurs ou ouvrages précis. 

Les deux auteurs affirment en outre avoir puisé dans le \emph{Dictionnaire de l’Académie française}~: vu les dates de publication, il s’agit probablement de la cinquième édition de 1798 pour Trucchi, qui publie la \emph{Fraseologia} en 1812 et en 1825, et de la sixième édition de 1835 pour Lanza, qui publie le \emph{Dictionnaire des phrases} en 1837.

\subsection{La macrostructure~: taille de la nomenclature, critères de rangement des entrées}

En prenant en compte uniquement les parties que ces ouvrages consacrent à la phraséologie (les trois premières dans Trucchi, toutes sauf les \emph{Termes élégants} et les \emph{Mots d’usage} dans Lanza), on peut estimer la nomenclature de Trucchi autour de 1800 \emph{phrases} en 69 pages~; celle de Lanza s’élève à presque 3700 \emph{phrases}\footnote{Ce calcul approximatif a été fait en considérant le nombre moyen des phrases présentes dans les dix premières pages, 28 environ dans la \emph{Fraseologia} de Trucchi et 11 environ dans le \emph{Dictionnaire} de Lanza, dans lequel les interlignes sont plus grands.} en 334 pages.

Du point de vue de l’organisation typographique de la page, dans les deux recueils le texte est imprimé sur deux colonnes, dont celle de gauche est en italiques et celle de droite en caractères droits. Une lettrine est présente au-dessus des colonnes de texte dans la \emph{Fraseologia} de Trucchi, dont la densité du texte dans la colonne explique le nombre si élevé de phrases dans un nombre de pages limité.

En ce qui concerne les critères de rangement des entrées dans les recueils phraséologiques, Lillo a pu affirmer que «~ces recueils sont le plus souvent agencés en fonction de regroupements plus ou moins arbitraires de phrases et constituent des amas d’éléments plutôt disparates~» \citep[76]{Lillo1994}. En effet, comme nous avons pu le constater lors de notre étude sur les recueils de Polesi et Barberi \citep{Murano2020}, seule une analyse fine de chaque ouvrage et de chaque partie d’ouvrage permet de déceler quelques critères, qui peuvent être appliqués de manière plus ou moins systématique.

Dans la \emph{Fraseologia} de Trucchi, la colonne de gauche présente des items en langue italienne, la colonne de droite leurs équivalents en français.

La section \emph{Phrases} contient des séquences à base verbale, plus ou moins figées et compositionnelles, comprenant des locutions («~\emph{Far castelli in aria} – Faire des chateaux en Espagne~»), des collocations («~\emph{Indur sonno} – Faire venir le sommeil~»~; «~\emph{Innalzar una Statua} – Dresser une Statue~»), des séquences libres («~\emph{Giuocar alla palla} – Jouer à la paume~»). Les entrées sont rangées en suivant un ordre alphabétique strict selon le premier mot, puis selon le deuxième mot\footnote{Quelques hésitations sont présentes concernant l’ordre alphabétique du deuxième mot~: par exemple, \emph{Giuocar alle pugna} précède \emph{Giuocar a chi deve pagar tutto} (Voir \figref{fig:murano:Figure 1}).}~: par exemple à la page 30 (voir \figref{fig:murano:Figure 1}) \emph{Girar gli occhi di quà e di là} précède \emph{Giugner in porto}, qui précède à son tour \emph{Giugner nuova alcuna cosa}.

\begin{figure}
\includegraphics[scale=1]{images/Murano_Figure1.png}
\caption{Fraseologia italiana-francese, p.30}
\label{fig:murano:Figure 1}
\end{figure}

La section dénommée \emph{Proverbes} contient en réalité des séquences figées aussi bien phrastiques et proverbiales («~\emph{E’ meglio un ovo oggi, che domani una gallina} – Il vaut mieux un tiens, que deux tu l’auras~») que verbales («~\emph{Far vedere la luna nel pozzo} – En donner à garder~»), ces dernières étant bien moins nombreuses. Par rapport à la section précédente, on ne repère ici que des séquences verbales dont le sens est figuré et non compositionnel. L’ordre alphabétique y est également adopté, mais de manière bien moins rigoureuse concernant le deuxième mot~: \emph{Cane che abbaja, non morde mai} précède tous les proverbes en \emph{Chi…}, mais parmi ceux-ci, «\emph{Chi compra terra, compra guerra} précède \emph{Chi cerca d’ingannare resta ingannato}.

Aucun ordre, ni alphabétique, ni thématique, n’est repérable par contre dans la partie \emph{Phrases particulières}, que l’on prenne en compte les entrées ou les équivalents. Cette section, appelée \emph{Frasi particolari, e gallicismi} dans la \emph{Prefazione}, présente un contenu plutôt hétéroclite, allant des proverbes («~\emph{Chi vuole acquistare bisogna che affatichi} – Nul bien sans peine~») aux séquences adverbiales («~\emph{Nel maggior impeto della collera} – Dans le fort de la colère~») aux noms composés («~\emph{Carta suga} – Papier qui boit~»). Les séquences phrastiques sont toutefois les plus nombreuses~: la plupart sont libres («~\emph{Egli è uomo aspro} – C’est un homme brusque~»), mais on repère quelques séquences figées verbales présentées en discours («~\emph{Ha il cervello sopra la berretta} – Il a la cervelle au bout du pied~»). Les items phrastiques libres et compositionnels en italien présentent souvent un équivalent français figé et non compositionnel, ce qui est probablement à l’origine de leur enregistrement dans le recueil~: 

\begin{quote}
    \emph{Sono tutto sudato, tutto bagnato di sudore} – Je suis tout en nage.
    
    \emph{E' molto vecchio, è cadente} – Il est sur le bord de la fosse.

    \emph{E’ ridotto alla mendicità} – Il est réduit au bâton blanc à la main.
\end{quote}


Dans le \emph{Dictionnaire} de Lanza, les \emph{phrases} françaises se trouvent dans la colonne de gauche et les équivalents italiens dans la colonne de droite.

Les \emph{Phrases précédées des Verbes} sont des séquences verbales, rangées selon l’ordre alphabétique du premier mot, qui est toujours un verbe à l’infinitif, puis, avec quelques hésitations, selon l’ordre alphabétique du deuxième mot~: dans la \figref{fig:murano:Figure 2} on peut voir que les séquences qui commencent par \emph{Couper} précèdent celles qui commencent avec \emph{Courir}, mais \emph{Couper jeu} suit, au lieu de précéder, \emph{Couper par le plus court}. Cette section contient des locutions verbales non compositionnelles («~\emph{Couver des yeux} – Mirar con occhio affettuoso e contento~»), des verbes simples qui ont un équivalent polylexical en italien («~\emph{se Dégarnir} – Alleggerirsi di panni~»), ainsi que des séquences dont le sens est compositionnel («~\emph{Avoir le visage effilé} - Aver il volto sottile~; sfilato~»).

\begin{figure}
\includegraphics[scale=1]{images/Murano_figure2.jpg}
\caption{Dictionnaire des phrases françaises-italiennes, p.22.}
\label{fig:murano:Figure 2}
\end{figure}

L’ordre alphabétique est suivi également dans toutes les autres parties, à commencer par les \emph{Façons de parler}, qui regroupent des séquences phrastiques avec un verbe conjugué, aussi bien figées («~\emph{Il l’a manqué belle} – Egli ha schivato un gran pericolo~») que libres («~\emph{As-tu songé à cela~?} – Hai tu pensato a ciò~?~»). Les \emph{Phrases adverbiales} comprennent des items relevant de plusieurs catégories grammaticales~: locutions adverbiales (\emph{D’un seul coup, En abrégé}), prépositionnelles (\emph{Depuis que}), pronoms indéfinis (\emph{Qui que ce soit}), phrases impersonnelles (\emph{Il s’en est peu fallu}).

Les \emph{Termes élégants} sont pour la plupart des noms composés, comme \emph{Heures perdues} (it. «~Ore libere, d’ozio~»), \emph{Homme à projet} (it. «~Progettista~»), \emph{Jours de Palais} (it. «~Giorni di Tribunali, giorni non feriati~»).

La section \emph{Phrases proverbiales} contient des séquences figées verbales présentées avec le verbe à l’infinitif («~\emph{Marcher à pas de loup} - Caminar quatto quatto~»; «~\emph{Mettre le doigt entre le bois et l’écorce} – Trovarsi tra l’incudine e ‘l martello~»), alors que les \emph{Façons de parler proverbialement} sont des séquences figées verbales présentées avec le verbe conjugué («~\emph{C’est un roseau qui plie à tout vent} – Egli è un uomo volubile~») ou des séquences figées phrastiques («~\emph{Votre affaire est faite} – Ella è finita per voi~; non bisogna darvi più speranza~»). 

La section \emph{Proverbes} contient de véritables proverbes comme «~\emph{Le loup mourra dans sa peau} – Il lupo cangia il pelo, ma non il vizio~» ou «~\emph{À brebis tondue, Dieu mesure le vent} – Dio manda il freddo, secondo i panni~». 

Les pages finales de la section \emph{Mots d’usage} contiennent des formules utilisées dans différentes situations de communication, comme la chasse ou le jeu~:

\begin{quote}
    \emph{Chou chou là} – Badalo, badalo, badalo (\emph{termine, che si usa da’cacciatori, per eccitar il cane alla ricerca, che significa} Piglia, piglia (detto popolarmente)

    \emph{Coup de malheur} – Colpo disgraziato (\emph{espressione usata da’ giuocatori allorché perdono})

    \emph{Je m’y tiens} – Non ne voglio più (\emph{risposta ad alcuni giochi di carte che si danno una per volta})
\end{quote}

La présentation des sections qui composent la \emph{Fraseologia italiana-francese} et le \emph{Dictionnaire des phrases françaises-italiennes} met en évidence l’hétérogénéité des items qu’elles recensent et la volonté de rendre disponible aux lecteurs l’ensemble des unités phraséologiques du français, quel que soit leur degré de figement formel, sémantique ou pragmatique.

Les dénominations des différentes sections nous paraissent délimiter de manière imparfaite leurs contenus. Toutefois, il se peut qu’elles reflètent les emplois de l’époque 1820--1840 et s’avèrent pour cela précieuses~: sous l’étiquette \emph{Façons de parler}, Lanza réunit des séquences avec un verbe conjugué, ce qui les oppose aux \emph{Phrases}, qui présentent le verbe à l’infinitif aussi bien dans le \emph{Dictionnaires des phrases} de Lanza que dans la \emph{Fraseologia} de Trucchi\footnote{Sauf dans la section \emph{Phrases particulières}, comme nous l’avons vu plus haut.}. 

Tout comme Lillo l’avait repéré dans les manuels, dans ces recueils les proverbes «~sont souvent confondus avec les expressions imagées~» \citet[71]{Lillo1994}. La dénomination \emph{Proverbe} avait probablement un sens plus large que celui que nous lui attribuons aujourd’hui, car elle désigne chez Trucchi des séquences aussi bien phrastiques que verbales~: la preuve en est que chez Lanza, l’adverbe \emph{proverbialement} et l’adjectif \emph{proverbial} ont le sens de ‘figé et non compositionnel’ lorsqu’ils sont ajoutés aux dénominations \emph{Phrases et Façons de parler}.

\subsection{La microstructure}

Les recueils phraséologiques de Trucchi et Lanza ne sont pas composés d’articles de dictionnaire au sens moderne du terme~: ils se limitent à mettre en regard les séquences plus ou moins figées italiennes et françaises, sans qu’un véritable bloc de texte ayant une structure hiérarchique et typographique interne soit consacré à chaque item. Cette composition typographique fondée sur le dénombrement des adresses était typique de la lexicographie ancienne, ainsi que l’a montré \citet[467]{Quemada1967}, et avait été employée dans la lexicographie bilingue français-italien jusqu’au 18e siècle \citep[85--6]{Murano2013a}.

Dans la \emph{Fraseologia} de Trucchi, on repère dans de rares cas quelques éléments typiques de la microstructure lexicographique, aussi bien dans la colonne de gauche qui contient les entrées que dans celle de droite qui concerne les équivalents~: 
\begin{itemize}
    \item [-] Des variantes formelles introduites par la conjonction o, comme dans «~\emph{Darsi intieramente in preda, o in balia alle passioni} – Lâcher la bride à ses passions~» ou dans le proverbe «~\emph{Tanto va la gatta al lardo, che vi lascia lo zampino} […] – Tant va la cruche à l’eau qu’à la fin elle se casse ou qu’enfin elle se brise~».
    \item [-] Des gloses qui expliquent le sens des entrées, éventuellement introduites par la formule métalinguistique d’explication \emph{vale}\footnote{Dans un cas, on repère la formule métalinguistique \emph{Si dice anche}, qui paraît suivre l’explication et précéder la locution verbale : «~\emph{Accertarsi di checchessia con dimostrazione certissima ; si dice anche toccar con mano} – Toucher au doigt et à l’œil~».}~: par exemple «~\emph{Far la nanna, vale dormire} – Faire dodo~» et «~\emph{S’ella è Rosa fiorirà, s’ella è spina pungerà, vale il tempo scuopre la verità} – La fin découvrira le vrai~».
    \item [-] Des notes en bas de page, aux contenus divers~: la note reliée à \emph{Fare la nanna} contient la glose «~Frase usata dalle balie per addormentar i bambini, dicendo \emph{ninna nanna}.~»~; pour \emph{Essere un ciarlone}, la note reliée à l’équivalent «~être bavard~» indique la connotation négative de l’expression~: «~Prendesi sempre in cattiva parte~».
    \item [-] Des locutions ou des équivalents synonymiques de forme proche ou éloignée, introduits par la conjonction \emph{o} ou tout simplement séparés par le point virgule, par exemple «~\emph{Aver poco sale in zucca, o aver la zucca vuota} – N’avoir pas un grain de sel dans sa caboche~» ou «~\emph{Esser fuori di sé~; essere trasportato dalla collera} – Ne se connaître point~; n’être pas à soi~».
\end{itemize}

De même, des éléments de microstructure sont présents dans le \emph{Dictionnaire des phrases} de Lanza. Dans la colonne de gauche qui contient les \emph{phrases} - entrées en français, on repère essentiellement des variantes, comme dans «~\emph{Conter des sornettes~; conter des fagots~;} ou \emph{conter de belles~; en conter bien} – Contar baje […]~». 

Dans la colonne de droite, qui contient les équivalent(s) en italien, les éléments de microstructure sont plus nombreux et variés~:

\begin{itemize}
    \item [-] des gloses explicatives, par ex. «~\emph{se Marier par amourette} – prender moglie per innamoramento (\emph{matrimonio ineguale})~»~,
    \item [-] des variantes, par ex. «~\emph{Mettre quelqu’un en cervelle~; le tenir en cervelle} – Metter altrui il cervello a partito~; recarlo in dubbio~; farlo star sospeso~»; «~Nager entre deux eaux – Star sospeso~; in dubbio~»~,
    \item [-] des restrictions de sujet, par ex. «~\emph{Avoir un beau toucher, un toucher délicat, brillant} – Suonar con eccellenza (\emph{si dice de’ sonatori d’istrumenti di tasti})~»~; «~\emph{Mettre bas} – partorire (\emph{parlando di animali})~». Ces indications s’avèrent particulièrement utiles dans le cas de \emph{phrases} polysémiques, comme \emph{Flotter au gré~; à la merci du vent}, enregistrée deux fois, une première fois avec l’ équivalent «~Essere portato in balia del vento (\emph{per un vascello mal governato})~», puis avec l’équivalent «~Ondeggiar sulle spalle agitati dal vento (\emph{parlando de’ capelli sciolti})~»,
    \item [-] des marques diatechniques (caccia, giuoco di carte / di bigliardo, guerra, marina, maneggio, scherma etc.), par ex. «~Ménager bien sa voix – Regolar bene la sua voce (\emph{musica})~»~; «~\emph{Méler un cheval} – imbrogliare un cavallo (\emph{maneggio})~».
\end{itemize}

\section{Conclusion}

L’analyse de deux recueils publiés à Naples entre 1825 et 1837 nous a permis d’approfondir notre réflexion récente (\citealt{Murano2017}  et 2020) sur les recueils phraséologiques franco-italiens. Si le critère de choix des recueils a été éminemment chronologique au départ, car nous voulions étudier la production phraséographique dans les décennies 1820 et 1830, il s’est avéré intéressant d’examiner deux ouvrages publiés chez le même éditeur qui ont vraisemblablement circulé simultanément dans les mêmes régions d’Italie.

Nous avons d’abord présenté la production de chaque auteur, qui dans le cas de Lanza paraît limitée au seul \emph{Dictionnaire des phrases françaises-italiennes}, comme c’était le cas pour Polesi (cf. \citealt{Murano2020}). Des informations précieuses dans le paratexte et une recherche dans les catalogues des bibliothèques nationales et universitaires laissent supposer une bonne diffusion des recueils phraséologiques, ou tout du moins comparable à celle de quelques dictionnaires portatifs, même si nous ne possédons aucune indication précise sur le public visé (écoles ou étude en autonomie).

La composition de ces ouvrages nous confirme dans le choix du mot \emph{recueil} pour les désigner, car l’absence d’unité textuelle et la fragmentation en sections hétérogènes éloignent ces ouvrages de la conception traditionnelle de dictionnaire~: la dénomination \emph{recueil} nous sert alors d’hyperonyme et embrasse la variété des ouvrages qui ont pour objectif de mettre en regard les phraséologies française et italienne. Les deux ouvrages analysés dans cette recherche montrent bien ce statut hybride et fluctuant entre texte de lecture et texte de consultation~: alors que la structuration typographique de la page en séries d’items successifs suggèrerait une lecture suivie, l’ordre alphabétique est adopté dans presque toutes les sections, ce qui permet une recherche ponctuelle des informations. 

Concernant les raisons qui ont motivé l’écriture des recueils, les auteurs constatent l’absence d’enseignement de la phraséologie par les maîtres et la nécessité de l’apprentissage de la phraséologie pour la traduction et la conversation. Complément nécessaire de l’étude de la grammaire au niveau avancé de l’apprentissage, le recueil phraséologique est l’instrument qui permet d’apprendre par cœur ce qui est \emph{difficile} dans la langue, pour que l’apprenant ne soit pas tenté de faire une traduction mot à mot. 

Du point de vue des items linguistiques présentés dans la \emph{Fraseologia} et dans le \emph{Dictionnaire des phrases}, on remarque que le contenu des recueils n’est pas entièrement phraséologique au sens moderne du terme, ce qui impliquerait la dimension du figement~: les \emph{phrases} peuvent être figées ou libres, leur sens peut être compositionnel ou non compositionnel. Par contre, l’emploi du nom \emph{proverbe} (et de ses dérivés \emph{proverbial} et \emph{proverbialement}), qui s’étend au-delà de la désignation de séquences phrastiques, indique sans hésitation des séquences caractérisés par le figement, syntaxique et/ou sémantique. 

Dans ces deux recueils, on repère des éléments de microstructure ponctuant les textes de manière irrégulière~: les variantes des \emph{phrases} et de leurs équivalents sont nombreuses dans les deux ouvrages~; au contraire, les paraphrases en langue source sont rares car l’objectif premier est de poser la relation d’équivalence entre les \emph{phrases} des deux langues~; dans le \emph{Dictionnaire des phrases} de Lanza on peut repérer également des restrictions de sujet et de nombreuses marques diatechniques qui témoignent de l’ouverture vers la phraséologie des langues de spécialité. Ce dernier trait avait déjà repéré par Lillo, qui affirmait que les recueils phraséologiques ne dédaignaient plus «~les expressions burlesques, familières, ni les langues de spécialité~» \citep[76]{Lillo1994}. Cette citation nous amène à constater l’absence quasi-totale\footnote{Voir plus haut dans le \emph{Dictionnaire des phrases}, «~detto popolarmente~» qui s’applique à l’équivalent s.v. \emph{Chou chou là}.} de marques diastratiques et diaphasiques ou d’une section consacrée à la \emph{phraséologie familière}.

Les recherches ultérieures que nous comptons conduire sur d’autres recueils de la même période et postérieurs nous permettront de dresser un état de lieu plus complet sur la phraséographie bilingue franco-italienne au 19e siècle et sur le rôle que les recueils phraséologiques pouvaient jouer dans l’enseignement de la phraséologie (phraséodidactique) et en général dans l’enseignement du français à un public pour lequel la maîtrise de cette langue était essentielle, comme le rappelle Nicolas Lanza~: 

\begin{quote}
    «~lo studio di essa potrebbe con ragion dirsi esser una essenziale parte della buona educazione, facendo imparare a’ Giovani una lingua che potrebbe loro ben servire nelle diverse vicende della vita in estero paese, e nella propria patria esser li può di gran risorsa perché li facilita il mezzo di trattar con esteri, ed apprendere le di loro cognizioni, come pure far loro intendere le proprie idee parlando e scrivendo in tal idioma~»\footnote{«~Son étude [de la langue française] pourrait à juste titre être considérée comme une composante essentielle de la bonne éducation, car on ferait apprendre aux jeunes une langue qui pourrait leur servir dans les situations de la vie dans un pays étranger, et qui, dans leur pays, peut s’avérer très utile parce qu’elle facilite les relations avec les étrangers et la compréhension des connaissances de ces derniers, et leur [= aux jeunes] permet également de faire comprendre leurs idées en parlant et en écrivant dans cette langue~» (notre traduction).} \citep[3]{Lanza1837}.
\end{quote}

{\sloppy\printbibliography[heading=subbibliography,notkeyword=this]}
\end{otherlanguage}
\end{document}
