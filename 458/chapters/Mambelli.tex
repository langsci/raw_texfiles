\documentclass[output=paper,colorlinks,citecolor=brown,arabicfont,chinesefont]{langscibook}
\ChapterDOI{10.5281/zenodo.15394475}
\author{Gloria Mambelli\affiliation{University of Verona, Italy }}
\title[A dictionary of the languages of medieval England]
      {A dictionary of the languages of medieval England: Issues and implications}

\abstract{The coexistence of three languages in England during the post-Conquest period led to a complex multilingual situation. The fact that Middle English, Anglo-French, and Latin showed mutual borrowing and shared lexis as a result of multilingual practices makes them difficult to handle from a lexicographic perspective. Starting from the assumption that “monolingual lexicography in a multilingual environment does not work” \citep[4]{Trotter2000a}, this paper illustrates the issues related to the treatment of these languages in the three major historical dictionaries and advocates the compilation of a multilingual dictionary, which would be able to provide a more accurate perspective of the linguistic situation of the time. The problems and implications connected to the designing of such a long-needed resource are tackled and some methodological proposals are presented by taking into consideration two existing databases which treat specific lexical domains related to everyday life in medieval England from a multilingual perspective. The aim of this contribution is to urge further research on the possible ways to reproduce the fuzziness of language boundaries in the post-Conquest period in order to make the compilation of a multilingual dictionary of the languages of medieval England an attainable goal.}

\IfFileExists{../localcommands.tex}{
  \addbibresource{../localbibliography.bib}
  % add all extra packages you need to load to this file

\usepackage{tabularx,multicol}
\usepackage{url}
\urlstyle{same}

\usepackage{listings}
\lstset{basicstyle=\ttfamily,tabsize=2,breaklines=true}

\usepackage{langsci-basic}
\usepackage{langsci-optional}
\usepackage{langsci-lgr}
\usepackage{langsci-osl}
% \usepackage{./langsci/styles/langsci-lgr}
% \usepackage{./langsci/styles/langsci-osl}
% \usepackage{langsci-gb4e}

\usepackage{tikz}
\usetikzlibrary{patterns,calc}
\pgfdeclarepatternformonly{south east lines}{\pgfqpoint{-0pt}{-0pt}}{\pgfqpoint{3pt}{3pt}}{\pgfqpoint{3pt}{3pt}}{
    \pgfsetlinewidth{0.6pt}
    \pgfpathmoveto{\pgfqpoint{0pt}{3pt}}
    \pgfpathlineto{\pgfqpoint{3pt}{0pt}}
    \pgfpathmoveto{\pgfqpoint{.2pt}{-.2pt}}
    \pgfpathlineto{\pgfqpoint{-.2pt}{.2pt}}
    \pgfpathmoveto{\pgfqpoint{3.2pt}{2.8pt}}
    \pgfpathlineto{\pgfqpoint{2.8pt}{3.2pt}}
    \pgfusepath{stroke}}
    
\usepackage{stmaryrd}
\usepackage{wasysym}
\usepackage{multirow}
\usepackage{caption}
\usepackage{subcaption}
\usepackage{mathrsfs}
\usepackage{qtree}

\usepackage{linguex}


  %pminos do not split footnotes
% \interfootnotelinepenalty=10000 %Footnote in Laporte chapters has to be split SN


%\DeclareIndexNameFormat{default}{%
%\nameparts{#1}%
%\usebibmacro{index:name}%
%{\index[names]}%
%{\namepartfamily}%
%{\namepartgiveni}%
% {}% L1
% {}% L2
%{\namepartprefix}% generates spurious space L3
%{\namepartsuffix}% generates spurious space L4
%}

%  {\DeclareIndexNameFormat{default}{%
%     \usebibmacro{index:name}{\index[names]}{#1}{#3}{#5}{#7}}}

%\DeclareIndexNameFormat{default}{%
%  \usebibmacro{index:name}{\sindex[nom]}{#1}{#3}{#5}{#7}}

%\DeclareIndexNameFormat{default}{%
%  \usebibmacro{index:name}{\sindex[person]}{#1}{#3}{#5}{#7}}
%\DeclareIndexNameFormat{default}{%
%\nameparts{#1} \usebibmacro{index:name}{\sindex[person]]}{\namepartfamily}{‌​\namepartgiven}{\nam‌​epartprefix}{\namepa‌​rtsuffix}}

%\newcommand{\smiley}{:)}

%\renewbibmacro*{index:name}[5]{%
%\usebibmacro{index:entry}{#1}%
%{\iffieldundef{usera}{}{\thefield{usera}\actualoperator}\mkbibindexname{#2}{#3}{#4}{#5}}}

% \newcommand{\noop}[1]{}

%remove for final
%\overfullrule=1mm

\newcommand{\tobi}[2]}}
\renewcommand{\S}[1]{\tobi{#1}{\textsc{*}}}

% this volume references
% puts: [this volume]
% already defined: \citetv
%\newcommand{\citepv}[1]{(\citeauthor{#1} \citeyear*{#1} [this volume])}
\newcommand{\citealtv}[1]{\citeauthor{#1} \citeyear*{#1} [this volume]}

%parentheses around example number
\newcommand{\pref}[1]{(\ref{#1})}

% in-text examples

\newcommand{\lnex}[1]{\textit{#1}} %target lang word
\newcommand{\lnlit}[1]{(lit.: `#1')} %literal reading
\newcommand{\lnlat}[1]{(#1)} % latinization
\newcommand{\lntrans}[1]{`#1'} %translation
\newcommand{\lnexl}[2]%
{\lnex{#1}{} \lnlat{#2}} % ex with latinization
\newcommand{\lnexlat}[3]{\lnex{#1}{} \lnlat{#2}{} \lntrans{#3}} % ex with latinization and tranl.

%ch01
\newcommand{\co}[1]{\mbox{\textbf{#1}}}

%ch09

\newcommand{\cyrbulg}[1]{\begin{otherlanguage*}{bulgarian}#1\end{otherlanguage*}}


%ch10
\newcommand{\nlp}{{\small NLP}}
\newcommand{\mwe}{{\small MWE}}
\newcommand{\rae}{{\small RAE}}
\newcommand{\lvc}{{\small LVC}}
\newcommand{\pos}{{\small P}o{\small S}}
%\newcommand{\todo}[1]{ \textcolor{red}{#1} }

%\renewcommand{\labelenumi}{\theenumi}
%\ainamefmt{{vv}{ll}{, ff}{, jj}} % fullname

\newcommand{\biberror}[1]{{\color{red}#1}}

\newcommand{\osenovaitem}{--~}
  %% hyphenation points for line breaks
%% Normally, automatic hyphenation in LaTeX is very good
%% If a word is mis-hyphenated, add it to this file
%%
%% add information to TeX file before \begin{document} with:
%% %% hyphenation points for line breaks
%% Normally, automatic hyphenation in LaTeX is very good
%% If a word is mis-hyphenated, add it to this file
%%
%% add information to TeX file before \begin{document} with:
%% %% hyphenation points for line breaks
%% Normally, automatic hyphenation in LaTeX is very good
%% If a word is mis-hyphenated, add it to this file
%%
%% add information to TeX file before \begin{document} with:
%% \include{localhyphenation}
\hyphenation{
    Beck-man
    Ngu-yen
    back-chan-nel
    back-chan-nels
    mo-not-o-nous
    ste-reo-typ-i-cal
}

\hyphenation{
    Beck-man
    Ngu-yen
    back-chan-nel
    back-chan-nels
    mo-not-o-nous
    ste-reo-typ-i-cal
}

\hyphenation{
    Beck-man
    Ngu-yen
    back-chan-nel
    back-chan-nels
    mo-not-o-nous
    ste-reo-typ-i-cal
}

  \togglepaper[4]%%chapternumber
}{}

\begin{document} 
\maketitle

\section{Introduction}

The situation of linguistic contact which gave rise to “the trilingual civilization of medieval England” \citep[164]{Rothwell1998} has attracted considerable attention in recent scholarship \citep{Trotter2000b, Wogan-browne2009, Ingham2010, JeffersonPutter2013, Wright2020b}. Textual evidence shows that, during the centuries following the Norman Conquest, any form of literacy required “considerable, probably native-like, competence in Anglo-French and Latin, as well as in English” \citep[229]{Durkin2014}. The coexistence of these three major languages\footnote{These were not the only languages in use in post-Conquest England: the Celtic languages survived in peripheral areas (Cornwall, Wales, Scotland, etc.), and “Norse continued to be spoken […] quite possibly into the twelfth [century] in some places” \citep[66]{Townend2006}.}  is generally referred to as triglossia in view of the fact that language choice usually depended on domains and text types; however, increasing interest in “the non-literary material dealing (broadly speaking) with everyday life” (\citealt{Trotter2006}: 73) has revealed a more complex situation in which the use of these languages overlapped. Specifically, the multilingual condition of medieval England is not only reflected in monolingual texts drawn up in each of the three languages, often found in multilingual collections intended for a multilingual readership, but also in multilingual texts in which the languages are mixed by means of code-switching practices (intended as the mixing of different codes in the same text).

Mixed-language writing has been extensively discussed by Wright in her studies of business documents (\citealt{Wright2000,Wright2002, Wright2011,Wright2013a, Wright2013b}, etc.), a “text type where a mixing of two or more languages is the norm” \citep[149]{Wright2000}. Accounts and inventories regularly include instances of code-switching and are characterised by considerable language overlap, mainly by means of the abbreviation and suspension system, which, besides allowing scribes to save parchment and ink, allowed readers to interpret words according to their linguistic competence (see \citealt{Wright2000}). These findings show that, contrary to the traditional views of early language mixing as a “broken” or “macaronic” language,

\begin{quote}
    mixed-language texts are typically not the product of incompetent or non-native authors or scribes, but rather reflect the – often very high – multilingual competence of the authors and scribes who wrote or copied them. (\citealt{SchendlWright2011}:  20)
\end{quote}

Mixed-language practices inevitably add to the complexities underlying the linguistic condition of medieval England, which was typified by mutual borrowing and shared lexis. Investigations of medieval multilingual texts have highlighted the difficulty of assigning word-forms to a specific language: according to Hunt, “from a lexical point of view attempts at a clear distinction of language identity is futile, for such distinctions were certainly blurred in medieval England” (\citealt{Hunt2011}, 64). Even though the study of code-switching and borrowing relies on the separation of codes, according to Trotter, we cannot “assume […] that what is perceived as separate now (after centuries of standardization and monolingualism) was regarded as separate or indeed separable in the Middle Ages” \citep[157]{Trotter2011}. If, as claimed by Rothwell, “a cut-and-dried divide between languages did not exist” \citep[545]{Rothwell1998}, then medieval speakers might not have been aware of language boundaries. 

From a lexicographic point of view, the fuzziness of language boundaries poses problems in terms of language labels and leads to the following question: how can the medieval languages be effectively represented in lexicographic terms? 

The following section addresses the issues and implications of a lexicographic treatment of the languages in use in post-Conquest England by presenting an analysis of existing historical dictionaries.

\section{Medieval multilingualism and lexicography}

At present, the lexicon of the languages of medieval England is recorded in three historical dictionaries: the \emph{Middle English Dictionary} (MED, \citealt{MED}), the \emph{Anglo-Norman Dictionary} (AND), and the \emph{Dictionary of Medieval Latin from British Sources} (DMLBS, \citealt{DMLBS}), all of which have been transferred into their online versions. The work of linguists heavily relies on these lexicographic resources, which, in the absence of actual speakers of the languages in question, represent essential tools for historical linguistic studies and investigations of medieval texts.

Unfortunately, historical dictionaries do not comprehensively record the extant attestations of the medieval languages, most of which are scattered in archives and not easily accessible. A considerable number of historical documents still remains unpublished and therefore its content has yet to find a place among dictionary citations. As a result, what we are offered by the aforementioned dictionaries is skewed towards attestations found in published texts.

The entries provided by the MED, AND, and DMLBS include Present-Day English definitions and related quotations. In addition, some of them provide information concerning spelling variants (MED, AND), etymology (MED, DMLBS), and links to other dictionaries (MED, AND). Users interested in one of the three languages are likely to be used to consulting the relevant dictionary and they certainly benefit from its content; however, it is only by consulting these three dictionaries concurrently that we are offered a glimpse of the overlapping use of the medieval languages and the complexities arising from it. For the reasons illustrated below, these dictionaries should necessarily be considered complementary resources. 

As a result of the multilingual environment of post-Conquest England, each of these dictionaries inevitably includes words derived from the languages treated in the other two. The DMLBS, for instance, contains a considerable number of vernacular words (i.e. of French or English etymology) with Latin inflections, and, similarly, it is not unusual to find terms that are derived from English in the AND and French- or Latin-origin ones in the MED. This is due to the situation of language contact which characterised the Middle English period and, more specifically, to the existence of mixed-language attestations. As mentioned before, administrative documents include instances of code-switching, i.e. vocabulary from a language that is integrated into a different matrix language (generally vernacular words embedded in a Latin text). As a result, Middle English and Anglo-French words found in Latin texts are included in the DMLBS, and Latin and English words found in Anglo-French texts are included in the AND. 

Table 1 illustrates a sample of lemmata of English native origin drawn from the \emph{Oxford English Dictionary (OED)} which are found in two of or all the three dictionaries. These terms were selected as examples of vocabulary relating to concepts and activities connected to everyday life in medieval rural areas. They are all of native origin, either totally (e.g. \emph{frith}, \emph{hide}) or partially (e.g. \emph{gavel}, from Old English and Medieval Latin, and \emph{selfode}, probably combining an Old English element with an Old Norse or Middle English one), and are recorded in attestations displaying different matrix languages. It is intended as a sample of shared lexis, since the same word is recorded in different historical dictionaries, although in different forms. With respect to form, it is necessary to bear in mind that Middle English and Anglo-French spelling was not standardised, hence spelling variation is a prominent feature in surviving texts; the word-forms reported in Table 1 are the variants found as headwords.

\begin{table}
\caption {Sample of English-origin words recorded in more than one historical dictionary}
\label{tab:myname:frequencies}
\begin{tabular}{llll}
\lsptoprule
OED & MED & AND & DMLBS\\
\midrule
acreman & aker-man & - & acremannus \\
craft & craft & craft & - \\
croft & croft & croft & crofta \\
frith & frith & frith & frithum \\
gavel & gavel & gavel & gabulum \\
hayward & hei-ward & hayward & haiwardus \\
hide & hide & hide & hida \\
husbandry & hus-bondri(e) & husbandrie & husbonderia \\
mill & milne & mille & - \\
ridding & ridding(e)  & - & riddinga  \\
selfode & self-ode & - & selfodis  \\
steward & steward & estiward & steuardus  \\
\lspbottomrule
\end{tabular}
\end{table}

The word \emph{hayward}, an English term denoting a manorial or town officer in charge of keeping fences and enclosures under surveillance, is found in the three dictionaries. The AND entry is linked to the OED, MED, and DMLBS: this indicates that we are in the presence of a word of non-French etymology, which does not appear in any dictionary of Old or Middle French. The following citations are recorded in the AND:

\begin{itemize}
    \item [(1)] Le \textbf{hayward} [...] deit tard e tempre espier, e environer, e garder lez boys, e lez blez, e lez prez (c.1270) 
    \item [(2)] Un malvois ribaut baillif ou \textbf{hayward} par cohercion fra un povre om fere une suite (1310)
    \item [(3)] messatorc: \textbf{haiuard} (MS: s.xii/xiii)
\end{itemize}

The DMLBS records the Latinised form \emph{haiwardus} and provides the Middle English form \emph{heiward} at the beginning of the entry. The word is attested with different spelling forms, a typical feature of Middle English, in Latin documents dating between 1209 and 1410. Two examples are given below.

\begin{itemize}
    \item [(4)] \textbf{haywardus} de B. adquietat annuatim de gabulo per servicium suum xx d., et debet habere in defenso domini, quamdiu animalia fuerint ad herbam, j bovem vel j vaccam (1234)
    \item [(5)] \textbf{heiwardus} habet penes se particulas et nomina plegiorum (1247)
\end{itemize}

The Middle English term \emph{steuard} (Present-Day English \emph{steward}), denoting an official in charge of the administration of a manorial estate, is recorded as \emph{estiward} in Anglo-French and \emph{steuardus} in British Medieval Latin. Both the AND and DMLBS signal its etymological origin, the former by referencing the corresponding entries in the OED and MED and the latter by reporting the Old and Middle English corresponding forms within square brackets before providing the definition. A different picture emerges when we look up the term \emph{husbandry}, used to refer to the management of a household or estate and agricultural operations, in the three dictionaries. The AND explicitly invites the user to consult the MED entry, whilst the DMLBS, in turn, references the Anglo-French term, suggesting that the word was borrowed into Latin through Anglo-French rather than directly from Middle English.

As shown in the table above, not all words are shared among the three languages. For instance, the English term \emph{acreman}, denoting a class of tenants in medieval England, is recorded in the MED and DMLBS but not in the AND. This is because the word is attested in English- and Latin-matrix texts but not in documents with an Anglo-French matrix. The headword recorded in the DMLBS is the Latinised form \emph{acremannus}, whose English origin is indicated by the presence of the Old English form \emph{æcerman} at the beginning of the entry.

Mixed-language quotations are occasionally signalled in the MED by means of square brackets and presented as additional illustrative material, as explained by the editor:

\begin{quote}
    Occasionally non-Middle English quotations (Latin, Old French, Old English, Middle Scots, etc.) are used for various illustrative purposes, e.g., for Latin or Old French, to exhibit the meaning or construction of the foreign word taken over into Middle English. \hfill\citep[19]{Lewis2007}\hbox{}
\end{quote}

For instance, the MED entry for \emph{croft} includes five Latin quotations below the sense ‘small piece of ground used for agricultural purposes (usually enclosed, often adjoining a house)’ with dates ranging from 1160 to 1400. All these are provided within brackets at the beginning of the block of quotations, separated from the English-matrix ones. However, this is not always the case.

If we consult the MED entry for \emph{ridding(e)}, we find that square brackets are not used consistently and that different dictionaries show a different picture. \emph{Ridding(e)} is a Middle English gerund form used to refer firstly to actions of clearing or cleaning of a site and, secondly, to an assart, that is an area of ‘forest or waste land converted to arable use by clearing trees, bushes, etc.’ OED. It is recorded as \emph{riddinga} in the DMLBS, but only in the sense referring to an area of cleared land, which seems to suggest that the senses referring to clearing and cleaning are not attested in Medieval Latin. Nevertheless, the MED records several Latin attestations of the word, i.e. quotations in which the word is embedded in a Latin matrix, in the following senses: ‘clearing away of debris, scaffolding, etc.’, ‘clearing of a house, garden, pool, stables, etc.’, and ‘clearing of a site, esp. in preparation for construction’. None of these is marked by square brackets.

Interestingly enough, among these Latin quotations, the ones drawn from the \emph{Account Rolls of the Abbey of Durham} are distinguished by the presence of a French grammatical function word preceding the English word, as illustrated by the example below. This is a phenomenon which, according to \citet{Ingham2009}, represents evidence of the use of French as a medium of communication among estate professionals.

\begin{itemize}
    \item [(6)] Joh'i Thomson pro \emph{le} \textbf{riddynge} et deposicione finali omnium murorum, lapidum, et lez butteressez veteris ecclesie Beate Marie Magd (1449)
\end{itemize}

The problems related to the inconsistency through which mixed-language attestations are signalled in historical dictionaries is discussed by \citet{Wright2011,Wright2013a}. In particular, she highlights errors and inconsistencies concerning first attestation dates, demonstrating how the DMLBS often provides antedatings of usage of English words. 

If we look up the word \emph{selfode}, an English term which denoted a specific kind of land holding and the class of tenants occupying it, we find that it is recorded in both the MED and DMLBS by means of Latin-matrix quotations only. In addition, the two earliest attestations provided by the OED are Latin quotations. Nevertheless, if we take only the MED and OED into account, we are led to believe that the word was not attested before the end of the thirteenth century in the sense relating to the type of tenant, since the earliest attestations appear to be the 1271 quotation (\emph{Inquisition Post Mortem}) recorded in the OED and the 1281--1282 one (\emph{Calendar of Documents Relating to Scotland Preserved in Her Majesty's Public Record Office}) in the MED. In fact, the DMLBS provides an earlier attestation, from 1209 (\emph{Cartulary of Cirencester Abbey}).

Due to the status of Latin, the main language of record at the time, it is not unusual to find that certain English words were attested in Latin documents before their first appearance in English-matrix texts, hence consultation of the MED alone might lead to incorrect interpretation of attestations of usage. As illustrated in the example above, the first attestation of the English term \emph{selfode} (1209) is found in the DMLBS but not in the MED or OED, which record later attestations.

The commonalities, differences, and inconsistencies discussed above show that taking into consideration only one of the languages of post-Conquest England is likely to be misleading since “monolingual lexicography in a multilingual environment does not work” \citep[4]{Trotter2000b}. In addition, the presence of the same term, albeit in different forms, in the three historical dictionaries raises questions about the decisions made by lexicographers in terms of assigning words to languages.

Is \emph{acremannus} a Latin word simply because it occurs in a Latin text with a Latin inflection notwithstanding its non-Latin etymology? If this is the case, how should we treat nonce words or hapaxes such as \emph{mille}, an English word attested only once in an Anglo-French document? 

In general, “it is very difficult to assess the degree of assimilation of a borrowed word to a given language at a given point in time” \citep[155]{Wright2000}, even with regard to contemporary phenomena. When it comes to lexical borrowing in the medieval period, the scarcity of surviving evidence and the presence of the same term in two or all the three historical dictionaries makes such assessment even more difficult to undertake. The inclusion of a foreign word in a dictionary would normally suggest its lexicalisation in the language in question; however, in the case of historical dictionaries, editorial policies differ from dictionary to dictionary, as well as from editor to editor, and are not always consistently applied, especially when it comes to mixed-language attestations, as previously illustrated. Evidently, “[t]he mixed-language text-type is difficult for monolingual dictionaries to handle” \citep[131]{Wright2013b}. After all, lexicographers and scholars interested in the history of English have traditionally focused on monolingual literary texts, disregarding the fact that documents drawn up in everyday situations in medieval England display “varying proportions of Medieval Latin, Anglo-Norman and Middle English” \citep[5]{Wright2020a}.

Since it is not an easy task to decide whether a certain word should be labelled as English, Latin, or Anglo-French when it comes to words attested in mixed-language texts, the overlapping use of the medieval languages represents a major issue from the point of view of lexicography.

\section{A dictionary of the languages of medieval England}

Even though we cannot confidently assess the medieval perception of language boundaries, we can certainly observe that the modern conceptions which take the “separatedness of languages” for granted \citep[155]{Trotter2009} cannot be applied to the languages of medieval England, which manifested themselves as a continuum. The inconsistencies observed in historical dictionaries call into question the reliability of monolingual investigations in the light of mixed-language evidence, suggesting that “the different medieval languages cannot be studied in isolation” \citep[15]{Schendl2015}.

Since we are necessarily dependent on historical dictionaries, research in the field of English historical linguistics, lexicology, and semantics would certainly benefit from a consistently designed resource assembling the lexicon(s) of the three languages of medieval England. A dictionary of the medieval languages would enhance users’ awareness of the multilingual environment of the period and the “symbiotic relationship” by means of which English, Anglo-French, and Latin were inextricably interconnected \citep[181]{Turville-petre1996}.

Notwithstanding the validity of these assumptions, such work represents a major lexicographic challenge. From a methodological and practical perspective, the following questions arise:

\begin{enumerate}
    \item Which lexicographic approach would be the most appropriate to tackle the multilingual environment of post-Conquest England?
    \item Would it be possible for lexicographers to reproduce the medieval fuzziness of language boundaries?
\end{enumerate}

So far, only two resources treating the lexis of medieval England from a multilingual perspective have been compiled and made available. These are the \emph{Lexis of Cloth and Clothing Project \citealt{[LexP]}} and the \emph{Bilingual Thesaurus of Everyday Life in Medieval England (\citealt{[BTh]})}, two databases focusing on specific lexical domains.

The LexP assembles the vocabulary in use in Britain between 700 and 1450 to refer to dress and textiles. In addition, it gathers non-textual evidence provided by surviving textiles, accessories, and art from the medieval period. The lexical data are drawn from historical dictionaries and displayed in alphabetical order, implying a semasiological approach aimed at investigating meaning. At the same time, however, the presence of images prompts an onomasiological approach in that users are urged to investigate the ways in which the visual item was described in language. Indeed, the compilers of the LexP aimed at creating “a tool which weaves together dictionary, encyclopedia and bibliography” \citep[772]{Rutten2008}. Moreover, in addition to definitions, citations, and etymological information, each entry offers the different forms in which the word is attested in the medieval languages. In fact, the project does not focus solely on the three languages of literacy of post-Conquest England but it records vocabulary in Old and Middle English, Anglo-French, Medieval Latin, Anglo-Norse, and Celtic languages such as Welsh, Old Irish, etc. As a result, we can indeed refer to it as a multilingual resource. For instance, the terms \emph{baldekin} (Anglo-French), \emph{baldekinus} (Latin), \emph{baudekin} (Middle English), \emph{boytkin} (Cornish), \emph{baudkin} (Older Scots), denoting a specific kind of textile used for drapes and canopies, are all recorded in the entry below the Present-Day English headword \emph{baldachin}. Users can either browse the list of headwords or search for a specific form (e.g. \emph{baldekin}) in the database.

The \citet{[BTh]} organises the Middle English and Anglo-French lexis relating to medieval professions and everyday life activities in conceptual groupings, i.e. by means of an onomasiological approach. More specifically, it focuses on the vocabulary of seven occupational domains of late medieval England (Building, Domestic Activities, Farming, Food Preparation, Manufacture, Trade, Travel by Water) in use between 1150 and 1450. The lexical data come from the MED and AND and the conceptual classification devised for the project is based on the category structures of the \emph{Historical Thesaurus of the Oxford English Dictionary (\citealt{[HTOED]})}, occasionally modified because the editors aimed to assemble associative fields rather than build a typically semantic hierarchy.

This resource represents one of the first attempts “to capture the characteristics of two different languages in overlapping use during the same period” \citep[318]{Sylvester2017}. Firstly, the macrostructure of the \citet{[BTh]} allows us to gain an overview of the terms that were used as (near) synonyms\footnote{When the term \emph{synonym} is used, a caveat is in order. According to \citet{Samuels1972}, “language possesses no pure synonyms”. Synonymy produced by the introduction of borrowings into the language is expected to exist only temporarily: after a period of time in which variants coexist, “either one of them will become less and less selected and eventually discarded, or a difference of meaning, connotation, nuance or register will arise to distinguish them” (1972, 65).}  or co-hyponyms in post-Conquest England: Middle English and Anglo-French words are grouped in alphabeti     cal order below a Present-Day English heading expressing the concept they refer to. In this way, referents become the major focus of investigation, making language boundaries less prominent. However, the two languages are treated as distinct entities in that lexical items are presented in two separate sections, below the headings “Anglo Norman”\footnote{The language is referred to as “Anglo Norman” in the \citet{[BTh]}, although some scholars favour the use of the more generic term “Anglo-French”, which better reflects the features of the language (see \citealt[230]{Durkin2014}).} and “Middle English”. This inevitably conveys a modern-day perspective which might result in users not being aware of the overlapping condition of the medieval languages.

Secondly, attention to the issues related to mixed-language attestations and shared lexis clearly emerges in the dictionary microstructure. The matrix language of the documents in which the word in question is recorded is signalled by means of labels (“French”, “Latin”, etc.) and referred to as “language of citation”. This appears to be an effective solution to make users aware of medieval language-mixing practices. Moreover, by means of labels (“Old English”, “Old French”, “Old Norse”, etc.) derived from the etymological information found in the MED, the Middle English words are provided with information about their language of origin. Unfortunately, the same does not apply to the Anglo-French lexical items, for which only the language of citation is signalled. For instance, the term \emph{croft} appears in both sections of the category “small holding or croft” because it is attested in both Middle English and Anglo-French citations (and is thus recorded in both the MED and AND), as shown in the example below. Even though the link can easily be inferred because of the presence of the term in both sections, its Old English origin is indicated in the Middle English section only.

\begin{quote}
SMALL HOLDING OR CROFT (N.)

ANGLO NORMAN:

\textbf{Croft} 1200--1421

Languages of citation: French, ?Latin

\textbf{Fin} 1200

Languages of citation: French, ?Latin

MIDDLE ENGLISH:

\textbf{Croft} 1160--1450

Language of origin: Old English

Languages of citation: Latin, English

\textbf{Curtilage} 1439--1450

Language of origin: Old French

Language of citation: English

\textbf{Graunge} 1440--1447

Language of origin: Old French

Language of citation: English
\end{quote}

Nevertheless, the onomasiological arrangement of the medieval lexis devised by the \citet{[BTh]} editors is effectively able to shift the focus away from language boundaries, since referents rather than word-forms are used as the starting point of investigation. This helps us to address the question relating to the most appropriate lexicographical approach to be adopted in a dictionary of the languages of medieval England: in a note to an essay discussing language contact in the medieval period, Trotter observes that “[t]here is a strong case for an onomasiological treatment of later medieval England, which would alone be able really to show how the three languages (Anglo-French, Middle English and Latin) functioned” \citep[35]{Trotter1996}.

Moreover, by means of a new conceptual classification which takes into account the cognitive experience of the Middle Ages, the \citet{[BTh]} brings users closer to the environment of medieval England from both a linguistic and cognitive perspective, without imposing “a modern day world view onto the medieval data” \citep[310]{Sylvester2017}. Consequently, it could be considered a model for the creation of a dictionary, or rather, a thesaurus of the languages of medieval England, which would offer users the possibility to browse medieval concepts to find out how they were referred to in the three languages during the late medieval period.

To this end, it would be advisable to avoid internal subdivisions which might imply a separation of the medieval languages, in order to obtain a multilingual lexicographic resource challenging the monolingual lexicographic tradition from which the historical dictionaries now available originated. In addition, on the model of the \cite{[HTOED]}, (near) synonyms and/or co-hyponyms in the three languages should be arranged chronologically: in this way, users would be able to trace the diachronic development of the semantic space and see when lexical items were adopted and when they fell out of use. 

Nevertheless, the question of language labelling still remains unsolved. Nowadays, we are used to a concept of language boundaries that differs from the medieval one and to a lexicographic tradition that handles languages as separate entities. As a result, in order to meet modern users’ needs, each term in a potential thesaurus of medieval vocabulary would still need to be assigned to one of the three languages. As previously explained, data from mixed-language texts inevitably pose problems in terms of language labels, making the designing of the tool proposed here a challenging, controversial task. So far, the solution adopted by the \citet{[BTh]} editors, who opted for signalling the matrix language(s) of the citations in which every lexical item appears, seems the most appropriate and effective for addressing the problem of language mixing in medieval England.

Even though the creation of such a long-needed work seems to still lie in the future, existing resources such as the LexP and \citet{[BTh]} demonstrate that, by abandoning the monolingual approach of traditional historical lexicography, it is possible to convey a perspective that is closer to the one of medieval speakers, getting away from the modern one which stems from “the ideology of modern nationalism and the comforting doctrines of monolingual simplicity” \citep[3]{Trotter2000a}. Obviously, were such an ambitious project to be initiated, issues of a different nature would arise. In particular, the problem of the long-term survival of online lexicographic works described by Preston-Kendal in Chapter 3 will need to be addressed in order to ensure the preservation and accessibility of the achieved results. However, this is beyond the scope of this paper.

\section{Conclusions}

The inconsistencies observed in the treatment of shared lexis and mixed-language attestations in the existing historical dictionaries, i.e. the MED, AND, and DMLBS, have shown that “a perspective drawn from one language alone […] is manifestly inadequate” (\citealt{Trotter1996}:  31). Consequently, this paper advocates the compilation of a multilingual dictionary of the languages of medieval England, which would bridge the gap between existing monolingual dictionaries by providing a more accurate perspective of the linguistic situation of the time. However, the fuzziness of boundaries typifying the medieval languages and their being “in continuous, concurrent use throughout the Middle Ages and in constant contact” \citep[68]{Hunt2011} makes them extremely difficult to reproduce in lexicographic terms. Further research is needed to answer the question of how to handle shared lexis in a multilingual context and reconcile the needs of modern dictionary users with the actual features of the medieval languages.

Even though these problems are still unresolved and represent a major obstacle to the realisation of a multilingual lexicographic resource, the advantages offered by an onomasiological approach in the handling of multilingual data should be taken into account. The fact that language boundaries are perceived as less prominent when lexical investigation starts from the content side rather than from the formal one will possibly prove beneficial in the making of such a long-needed work, at least in terms of overcoming the modern urge to treat languages as distinct entities in favour of conceptions of language boundaries in line with the medieval ones. As explained in this paper, in order to achieve better results in this sense, the multilingual lexical material should be presented in chronological order in each conceptual group without being separated according to the language.

In sum, the existing historical dictionaries treating the languages of medieval England, on which historical linguistic studies necessarily rely, present issues and inconsistencies that result in an imprecise overview of the linguistic environment of medieval England. Hence, a new, multilingual lexicographic resource is an important desideratum for historical linguists, notwithstanding the issues and implications involved in the initiation of such an ambitious project. In light of the recent findings concerning the status of the languages of medieval England, we necessarily need to abandon the traditional monolingual approach, which “is neither appropriate nor adequate for the investigation of language use in a society where multilingualism was endemic and where, for the educated at least, monolingualism was the exception and not the norm” \citep[3]{Trotter2000a}.

{\sloppy\printbibliography[heading=subbibliography,notkeyword=this]}
\end{document}
