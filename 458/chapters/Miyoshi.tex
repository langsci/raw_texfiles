\documentclass[output=paper,colorlinks,citecolor=brown,arabicfont,chinesefont]{langscibook}
\ChapterDOI{10.5281/zenodo.15394477}
\author{Kusujiro Miyoshi\affiliation{Soka Women's College in Tokyo, Japan}}
\title[English high-frequency verbs in the \textit{Promptorium Parvulorum}]
      {The treatment of English high-frequency verbs in the \textit{Promptorium Parvulorum} (1440)}

\abstract{This dictionary has long attracted interest as the first English-Latin dictionary, but, as far as I am aware, its treatment of English high-frequency verbs has been little investigated. My intention is to fill this void. My analytic procedure will be to adopt a sampling method, selecting 128 entries related to 11 verbs, namely, \textit{cast}, \textit{come}, fall, \textit{get}, \textit{go}, \textit{make}, \textit{put}, \textit{set}, \textit{take}, \textit{throw} and \textit{turn}. I will then analyze how this dictionary treats individual verbs, phrasal verbs and other relevant phrases. If a dictionary published more than three hundred years before Johnson’s Dictionary does in fact treat English high-frequency verbs in detail, one generally accepted notion about the history of English lexicography may have to be revised.
}

\usepackage{fontspec}
\usepackage{parskip}
\IfFileExists{../localcommands.tex}{
  \addbibresource{../localbibliography.bib}
  % add all extra packages you need to load to this file

\usepackage{tabularx,multicol}
\usepackage{url}
\urlstyle{same}

\usepackage{listings}
\lstset{basicstyle=\ttfamily,tabsize=2,breaklines=true}

\usepackage{langsci-basic}
\usepackage{langsci-optional}
\usepackage{langsci-lgr}
\usepackage{langsci-osl}
% \usepackage{./langsci/styles/langsci-lgr}
% \usepackage{./langsci/styles/langsci-osl}
% \usepackage{langsci-gb4e}

\usepackage{tikz}
\usetikzlibrary{patterns,calc}
\pgfdeclarepatternformonly{south east lines}{\pgfqpoint{-0pt}{-0pt}}{\pgfqpoint{3pt}{3pt}}{\pgfqpoint{3pt}{3pt}}{
    \pgfsetlinewidth{0.6pt}
    \pgfpathmoveto{\pgfqpoint{0pt}{3pt}}
    \pgfpathlineto{\pgfqpoint{3pt}{0pt}}
    \pgfpathmoveto{\pgfqpoint{.2pt}{-.2pt}}
    \pgfpathlineto{\pgfqpoint{-.2pt}{.2pt}}
    \pgfpathmoveto{\pgfqpoint{3.2pt}{2.8pt}}
    \pgfpathlineto{\pgfqpoint{2.8pt}{3.2pt}}
    \pgfusepath{stroke}}
    
\usepackage{stmaryrd}
\usepackage{wasysym}
\usepackage{multirow}
\usepackage{caption}
\usepackage{subcaption}
\usepackage{mathrsfs}
\usepackage{qtree}

\usepackage{linguex}


  %pminos do not split footnotes
% \interfootnotelinepenalty=10000 %Footnote in Laporte chapters has to be split SN


%\DeclareIndexNameFormat{default}{%
%\nameparts{#1}%
%\usebibmacro{index:name}%
%{\index[names]}%
%{\namepartfamily}%
%{\namepartgiveni}%
% {}% L1
% {}% L2
%{\namepartprefix}% generates spurious space L3
%{\namepartsuffix}% generates spurious space L4
%}

%  {\DeclareIndexNameFormat{default}{%
%     \usebibmacro{index:name}{\index[names]}{#1}{#3}{#5}{#7}}}

%\DeclareIndexNameFormat{default}{%
%  \usebibmacro{index:name}{\sindex[nom]}{#1}{#3}{#5}{#7}}

%\DeclareIndexNameFormat{default}{%
%  \usebibmacro{index:name}{\sindex[person]}{#1}{#3}{#5}{#7}}
%\DeclareIndexNameFormat{default}{%
%\nameparts{#1} \usebibmacro{index:name}{\sindex[person]]}{\namepartfamily}{‌​\namepartgiven}{\nam‌​epartprefix}{\namepa‌​rtsuffix}}

%\newcommand{\smiley}{:)}

%\renewbibmacro*{index:name}[5]{%
%\usebibmacro{index:entry}{#1}%
%{\iffieldundef{usera}{}{\thefield{usera}\actualoperator}\mkbibindexname{#2}{#3}{#4}{#5}}}

% \newcommand{\noop}[1]{}

%remove for final
%\overfullrule=1mm

\newcommand{\tobi}[2]}}
\renewcommand{\S}[1]{\tobi{#1}{\textsc{*}}}

% this volume references
% puts: [this volume]
% already defined: \citetv
%\newcommand{\citepv}[1]{(\citeauthor{#1} \citeyear*{#1} [this volume])}
\newcommand{\citealtv}[1]{\citeauthor{#1} \citeyear*{#1} [this volume]}

%parentheses around example number
\newcommand{\pref}[1]{(\ref{#1})}

% in-text examples

\newcommand{\lnex}[1]{\textit{#1}} %target lang word
\newcommand{\lnlit}[1]{(lit.: `#1')} %literal reading
\newcommand{\lnlat}[1]{(#1)} % latinization
\newcommand{\lntrans}[1]{`#1'} %translation
\newcommand{\lnexl}[2]%
{\lnex{#1}{} \lnlat{#2}} % ex with latinization
\newcommand{\lnexlat}[3]{\lnex{#1}{} \lnlat{#2}{} \lntrans{#3}} % ex with latinization and tranl.

%ch01
\newcommand{\co}[1]{\mbox{\textbf{#1}}}

%ch09

\newcommand{\cyrbulg}[1]{\begin{otherlanguage*}{bulgarian}#1\end{otherlanguage*}}


%ch10
\newcommand{\nlp}{{\small NLP}}
\newcommand{\mwe}{{\small MWE}}
\newcommand{\rae}{{\small RAE}}
\newcommand{\lvc}{{\small LVC}}
\newcommand{\pos}{{\small P}o{\small S}}
%\newcommand{\todo}[1]{ \textcolor{red}{#1} }

%\renewcommand{\labelenumi}{\theenumi}
%\ainamefmt{{vv}{ll}{, ff}{, jj}} % fullname

\newcommand{\biberror}[1]{{\color{red}#1}}

\newcommand{\osenovaitem}{--~}
  %% hyphenation points for line breaks
%% Normally, automatic hyphenation in LaTeX is very good
%% If a word is mis-hyphenated, add it to this file
%%
%% add information to TeX file before \begin{document} with:
%% %% hyphenation points for line breaks
%% Normally, automatic hyphenation in LaTeX is very good
%% If a word is mis-hyphenated, add it to this file
%%
%% add information to TeX file before \begin{document} with:
%% %% hyphenation points for line breaks
%% Normally, automatic hyphenation in LaTeX is very good
%% If a word is mis-hyphenated, add it to this file
%%
%% add information to TeX file before \begin{document} with:
%% \include{localhyphenation}
\hyphenation{
    Beck-man
    Ngu-yen
    back-chan-nel
    back-chan-nels
    mo-not-o-nous
    ste-reo-typ-i-cal
}

\hyphenation{
    Beck-man
    Ngu-yen
    back-chan-nel
    back-chan-nels
    mo-not-o-nous
    ste-reo-typ-i-cal
}

\hyphenation{
    Beck-man
    Ngu-yen
    back-chan-nel
    back-chan-nels
    mo-not-o-nous
    ste-reo-typ-i-cal
}

  \togglepaper[5]%%chapternumber
}{}

\begin{document} 
\maketitle

\section{Introduction}

I want to begin my paper by posing this question: “What is the first English dictionary that treated quite a few high-frequency verbs, distinguishing their senses and indicating their conjugated forms, as well as listing their relevant idiomatic phrases, especially phrasal verbs?”. To my understanding, it has generally been believed to be Samuel Johnson’s \textit{Dictionary of the 
English Language} of 1755. If so, is this answer in accord with the facts? Anne McDermott and Rosamund Moon once remarked that [Johnson] was the first to deal systematically with such fundamental features of the English language as phrasal verbs, common polysemous verbs such as take […] participial adjectives, verbal nouns, and many others. \citep[153]{McDermottMoon2005} If we disregard one word “systematically” in this account, a concept they do not expand on in their paper, this statement is quite off the mark.

In fact, there are two ways to define an English dictionary. One is to regard it as a monolingual dictionary, which may justify seeing Johnson’s Dictionary as the first that detailed the senses and usage of English high-frequency verbs.

However, there is another way. For one thing, we know Henry Wheatley’s pioneering survey of the history of English lexicography “Chronological notices of the dictionaries of the English language” (\citet{Wheatley1865}). discusses both English-Latin dictionaries and English monolingual dictionaries. And we also know Gabriele Stein’s monumental philological study on English lexicography entitled \textit{The English Dictionary Before Cawdrey} \citep{Stein1985}. Stein mainly treats historic bilingual and polyglot dictionaries published in England before 1604.

These examples show that, for Wheatley and Stein, the category of English dictionaries includes not only monolingual ones but also those which show how English words and phrases can be translated into and explained in other languages. Such a situation also applies to the \textit{Oxford History of English Lexicography} \citep{Cowie2009} edited by A. Cowie and the first and second volumes of the \textit{Ashgate Critical Essays on Early English Lexicographers} \citep{Franzen2012} edited by Christine Franzen.

Taking the second view, which is more inclusive, makes it far from correct to regard Johnson’s Dictionary as the answer to the initial question. According to my research of the available dictionaries today, it is the \textit{Promptorium Parvulorum} (1440) that includes all the features mentioned above. This first English-Latin dictionary was issued at Norfolk in the Middle English period by an anonymous author known as Galfredus Grammaticus, a monk and grammarian, according to R. C. Alston \citet{Alston1968}. The contents of the dictionary are available in its modern reproduced version, edited by A. Mayhew and published by the Early English Text Society in 1908.

The Promptorium, compiled more than three hundred years before Johnson’s Dictionary, has long attracted scholars as it is the first English-Latin dictionary, being discussed quite often in their various research papers and books, including \citet{Wheatley1865}, \citet{StarnesNoyes1946},  \citet{Wells1973}, \citet{Hullen2006}, \citet{Sauer2009} and \citet{Bately2009}, with the culmination of \citet{Stein1985} and \citet{Stein1997}.  However, it seems that research specializing in English high-frequency verbs has yet to be done. I hope to contribute to meeting this need.

My sampling method selects 128 entries representing 11 verbs. Specifically, the \textit{Promptorium} provides the number of entries noted in parentheses for each verb in my selection like this: \textit{cast} (10), \textit{come} (5), \textit{fall} (6), \textit{get} (4), \textit{go} (15), \textit{make} (47), \textit{put} (14), \textit{set} (10), \textit{take} (6), \textit{throw} (6) and \textit{turn} (5).

There are two further notes to be added. One concerns the spellings in the \textit{Promptorium} that often seem to be archaic and dialectal to our modern eye. The author of the dictionary thus often renders the infinitives of verbs according to contemporaneous morphology, such that the modern \textit{get}, \textit{put}, \textit{set} and \textit{take}, for instance, appear as \textit{getyn}, \textit{puttyn}, \textit{settyn} and \textit{takyn}, applying the infinitive endings of his region in the period. In this regard, we may also have to remember that the use of to was yet to be fully consolidated for indicating an infinitive at his time, except for the case where the notion of a purpose is expressed.

In relation to such features of Middle English, Gloria Mambelli’s paper “A dictionary of the languages of medieval England”, included in this book, is highly useful in determining how the language has been and will be dealt with in relevant dictionaries.

The other note concerns the lemmatization of relevant entries in the dictionary, to which I will attach considerable importance. Regarding this point, Stein, based on her investigation made from an overall viewpoint, remarks that: 

\begin{quote}
    It looks as if the author [of the \emph{Promptorium}] explains the English words before he gives the Latin translation. [...] This unusual lemmatization derives obviously from the practice of changing Latin-English entries into English-Latin ones, The English lemma is thus basically still an explanation of a former Latin-English dictionary.\hfill\citep[98--99]{Stein1985}\hbox{}
\end{quote}

Assuming that this is also applicable to the treatment of English high-frequency verbs in the dictionary, including distinguishing their senses and providing entries for their conjugated forms and their relevant idiomatic phrases, this will still not affect the fact that the \emph{Promptorium} is the earliest existent dictionary that lists relevant words and phrases, which I suggest should be given its due weight.

Given our discussion so far, I will first survey the treatment of the infinitival and conjugated forms of English high-frequency verbs in the \textit{Promptorium} (Section 2). This will constitute my preliminary analysis. I will then detail how the dictionary treats individual verbs (Section 3), phrasal verbs (Section 4) and other types of phrases (Section 5). This allocation of sections is based on three points which I regard as characterizing English high-frequency verbs: (1) having irregular conjugations; (2) having multiple senses; and (3) forming quite a few idiomatic phrases.

\section{\textit{The Promptorium}'s treatment of the infinitival and conjugated forms of English high-frequency verbs}

In analyzing how the \emph{Promptorium} handles the infinitival and conjugated forms of these verbs, and their related phrases, I will consider three viewpoints: (1) the treatment of their infinitival and gerundive forms; (2) that of their past-participial forms; and (3) the provision of an overview. I will also indicate the basic structure of entries in the dictionary related to the high-frequency verbs.

\subsection{The treatment of infinitival and gerundive forms}

The reason why I discuss the treatment of the infinitival and gerundive forms together here is that, within the range of my study, there are four cases where the \emph{Promptorium} provides two entries for the same verb, one for the infinitival form and one for the gerundive form, to which I also want to refer, recognizing that this fact will benefit my later analysis.

Two cases out of the four are as follows:

\begin{quote}
    \textbf{Puttyn}, or leyyn: \emph{Pono, -is, -sui, -re}; 3com., \emph{act.: Colloco, -as, -aui, -re; prime con., act.}
    
    \textbf{Puttynge}, or leyynge: \emph{Possicio, -nis; Collacio, -nis}; omnia \emph{fem.}, 3 \emph{Ded.}
    
    \rule{5cm}{.5pt}
    
    \textbf{Throwyn}, or castyn: \emph{Iacto, -as; prime} con., \emph{act; Projicio, -is, eci, -re}; 3 con., \emph{act.}
    
    \textbf{Throwynge}, or castynge: \emph{Iactura, -e}; fem., \emph{prime; Iactas, -tus; masc.}, 4.
\end{quote}

Here, Latin words corresponding to the entry-words ending with “-yn”, shows that they are indicative present forms, meaning that the verbs \emph{puttyn} and \emph{throwyn} are treated as infinitives. The head-words with “-ynge”, they are gerunds, not present participles, as their corresponding Latin words \emph{possicio} and \emph{iactura} are, respectively, nominal forms.

As to other two cases, they are the followings, which show the same situation as above:

\begin{quote}
    \textbf{Settyn} and plantyn: \emph{Planto, -as, aui, -re; prime con., neut.}
    
    \textbf{Settynge}, or plantynge: \emph{Plantacio, -nis; fem.} 3.

    \rule{5cm}{.5pt}

    \textbf{Takyn}, or resevyn: \emph{Accipio, -is, cepi, -re, -ceptum; Capio, -is, cepi, -re, captum [...].}
    
    \textbf{Takynge}, or resevynge: \emph{Accepcio, -nis: Suscepcio, -nis}; omnia \emph{fem.}, 3: \emph{Caplura, -re; fem., prime.}
\end{quote}

Of the 128 entries in my analysis, 42 concern words with “-yn” and 19 concern the words with “-ynge”. However, this does not mean that they are limited to 42 with “-yn”. When relevant Latin words are browsed through, it we find that the \emph{Promptorium} provides 102 entries for infinitival forms of verbs and verbal phrases, with the use of spellings like \emph{cum} for \emph{come}, \emph{goon/gon} for \textit{go} and \emph{make}, as well as \emph{makyn} for \emph{make}. The entries related number only 19. The \emph{Promptorium} seems not to treat present-participial forms , which may be confused with gerundive forms, at least within the  limits of my scope study,  , including when the Latin words listed are   considered.

\subsection{The treatment of past-participial forms}

For the treatment of English past-participial forms, the \textit{Promptorium} usually provides the perfect passive participle and adjectival forms of Latin words, as the following three entries show:

\begin{quote}
    \textbf{Cast}, or castyde: \emph{Iactus, -ta; Proiectus, -ta, -tum.}

    \rule{5cm}{.5pt}

    \emph{Sette}, or putte: \emph{Positus, -ta, -um; Collocatus, -a, -um.}

    \emph{Put togedyr}, or onyd: \emph{Contiguus}, -a, -um.
\end{quote}

The author of the \emph{Promptorium}, thus provides 7 entries on English past-participial words and phrases within my data. The Promptorian does not deal with preterit forms, which are equivalent to past-participial forms in our modern English, as in the cases of \emph{cast}, \emph{put} and \emph{set}.

\subsection{The provision of an overview}

How often does the \emph{Promptorium} provide entries on infinitival and conjugated forms of verbs and verbal phrases for each of the 11 verbs I cover? \tabref{tab:miyoshi:table1} illustrates this ;hereafter, the abbreviations “inf.”, “ger.” and “pp.” stand for, “infinitival”, “gerundive” and “past-participial”, respectively.

\begin{table}
\caption{The number of entries for infinitival and conjugated forms of verbs and verbal phrases in the \textit{Promptorium}}
\label{tab:miyoshi:table1}
\begin{tabularx}{.8\textwidth}{l YYYYYY Y}
 \lsptoprule
  & \multicolumn{2}{c}{inf.} & \multicolumn{2}{c}{ger.} & \multicolumn{2}{c}{pp.} & \\\cmidrule(lr){2-3}\cmidrule(lr){4-5}\cmidrule(lr){6-7}
               & $n$ & \% & $n$ & \% & $n$ & \% & total  \\\midrule
 \emph{cast}   & 7 & 70.0 & 1 & 10.0 & 2 & 20.0 & 10   \\        
 \emph{come}   & 5 & 100.0 & 0  & 0.0  & 0 &  0.0 & 5    \\        
 \emph{fall}   & 3 & 50.0& 2 & 33.3 & 1 & 17.7  & 6    \\        
 \emph{get}    & 2 & 50.0 & 2 & 50.0 & 0 & 0.0     & 4    \\        
 \emph{go}     & 15 & 100.0 & 0 & 0.0     & 0 & 0.0     & 15   \\        
 \emph{make}   & 47 & 100.0 & 0 & 0.0     & 0 & 0.0     & 47   \\        
 \emph{put}    & 9  & 64.3 & 2 & 14.3 & 3 & 21.4 & 14  \\ 
 \emph{set}    & 6  & 60.0 & 3 & 30.0 & 1 & 10.0 & 10  \\      
 \emph{take}   & 5 & 83.3 & 1 & 16.7 & 0  & 0.0  & 6   \\       
 \emph{throw}  & 3 & 50.0 & 3 & 50.0 & 0  & 0.0  & 6   \\      
 \emph{turn}   & 0 &  0.0 & 5 & 100.0 & 0 & 0.0  & 5   \\\midrule   
 total         & 102 & 79.7 & 19 & 14.8 & 7 & 5.5 & 128\\   
 \lspbottomrule
\end{tabularx}
\end{table}

The distribution of the three forms being quite uneven among entries on relevant verbs, it is hardly possible from this table alone to tell what principle(s) the author of the dictionary might have employed in providing such entries. However, it may at least be said that the table presents an overview of the historically most primitive treatment of English high-frequency verbs in lexicography.

\section{The \textit{Promptorium}'s treatment of individual verbs}

Having surveyed the treatment of the high-frequency verbs in the \emph{Promptorium} in the previous section, this section, aims to clarify how the dictionary deals with the relevant verbs individually from two points of view: (1) the provision of entries for their infinitival and conjugated forms; and (2) the discrimination and indication of their senses in the English language.

For this purpose, I will collate Tables~\ref{tab:miyoshi:specified} and \ref{tab:miyoshi:notspecified} which my investigation has produced; the former concerns entries whose head-words include the senses of the verbs and the latter entries whose head-words do not include them, in both of which I arrange the head-words in accordance to their infinitival, gerundive and past-participial forms, indicating, for reference, Latin word/words the author provides for each.

\begin{table}
\caption {Entries on verbs in the \textit{Promptorium} where their senses are “specified” in the head-words}
\label{tab:miyoshi:specified}
\small
\begin{tabularx}{\linewidth}{ QQQ }
\lsptoprule
\textbf{“inf.” (Latin)} &\textbf{“ger.” (Latin)} & \textbf{“pp.” (Latin)} \\
\midrule
“\textbf{Castyn}, or brakyn” (\emph{vomo}) & &  \\
\tablevspace
“\textbf{Castyn}, or trowyn” (\emph{iacto, jacio}) & &  \\
\tablevspace
“\textbf{ffallyn}, or ouer trowyn” (\emph{cado, ruo}) & &  \\
\tablevspace
“\textbf{ffalyn}, or happyn” (\emph{accidit, eacenit}) & &  \\
\tablevspace
“\textbf{Gettyn}, or wynnyn” (\emph{lucror, obtimeo}) & “\textbf{Gettyng}, or havying be wynnyng” (\emph{lucrum, adquisicio}) &  \\
\tablevspace
“\textbf{Puttyn}, or leyyn” (\emph{pono, colloco}) & “\textbf{Puttynge}, or leyynge” (\emph{posicio, collacio}) & “\textbf{Put}, or leyd” (\emph{positus}) \\
\tablevspace
 & “\textbf{Puttynge}, or schowynge” (\emph{pulsus}) &  \\
\tablevspace
“\textbf{Settyn} and plantyn” (\emph{planto}) & “\textbf{Settynge}, or plangynge” (\emph{plantacio}) &  \\
\tablevspace
“\textbf{Settyn} and ordeynyn” (\emph{statuo}) & & \\
\tablevspace
“\textbf{Settyn}, puttyn, or leyn” (\emph{pono, colloco}) & “\textbf{Settynge}, or puttynge” (\emph{plantacio}) & “\textbf{Sette}, or putte” (\emph{positus}) \\
\tablevspace
“\textbf{Takyn}, or resevyn” (\emph{accipio, capio, apprehendo, sumo, tollo}) & “\textbf{Takynge}, or reseyvynge” (\emph{accepcio, omnia, captura}) &  \\
\tablevspace
“\textbf{Throwyn}, or castyn” (\emph{iacto, projicio}) & “\textbf{Throwynge}, or castynge” (\emph{iactura, iactus}) &  \\
\lspbottomrule
\end{tabularx}
\end{table}

\begin{table}
\caption {Entries for verbs in the \textit{Promptorium} where their senses are “not specified” in the head-words}
\label{tab:miyoshi:notspecified}
\small
\begin{tabularx}{\linewidth}{ QQQ }
\lsptoprule
\textbf{“inf.” (Latin)} &\textbf{“ger.” (Latin)} & \textbf{“pp.” (Latin)} \\
\midrule
 & & “\textbf{Cast}, or castyde” (\emph{iactus}) \\
\tablevspace
“\textbf{Cum}, or come” (\emph{venio}) & &  \\
\tablevspace
 & “\textbf{ffallyng}” (\emph{defectus}) & “\textbf{ffalle}” (\emph{casus}) \\
\tablevspace
“\textbf{Goon}” (\emph{pergo, vado, eo, gradior}) & &  \\
\tablevspace
“\textbf{Makyn}, or make” (\emph{facio, compono, psalmo}) &  &  \\
\lspbottomrule
\end{tabularx}
\end{table}

\subsection{The provision for entries on infinitival and conjugated verb forms}

Out of the 26 entries provided for the 11 individual verbs, 14 are for infinitival forms, 8 are for gerundive and 4 are past-participial, the detail of which is as indicated in \tabref{tab:miyoshi:table4}.

\begin{table}
\caption{The number of entries for infinitival and conjugated forms of individual verbs in the \textit{Promptorium}}
\label{tab:miyoshi:table4}
\begin{tabularx}{.8\textwidth}{ lYYYYYr }
\lsptoprule
 & \emph{cast} & \emph{come} & \emph{fall} & \emph{get} & \emph{go} & \emph{make} \\\midrule
inf. & 2 & 1 & 2 & 1 & 1 & 1 \\
ger. & --- & --- & 1 & 1 & --- & --- \\
pp. & 1 & --- & 1 & --- & --- & --- \\
total & 3 & 1 & 4 & 2 & 1 & 1 \\\midrule
 & \emph{put} & \emph{set} & \emph{take} & \emph{throw} & \emph{turn} & total \\\midrule
inf. & 1 & 3 & 1 & 1 & --- & 14 \\
ger. & 2 & 2 & 1 & 1 & --- & 8 \\
pp. & 1 & 1 & --- & --- & --- & 4 \\
total & 4 & 6 & 2 & 2 & --- & 26 \\
\lspbottomrule
\end{tabularx}
\end{table}

\subsection{The discrimination and indication of the senses of the individual verbs and conjugated forms of verbs}

What is important here is why the \emph{Promptorium} provides such a number of entries for 11 individual verbs. Examining this more closely, we see that in 20 entries out of the 26 head-words, which make up 76.9\%, the dictionary indicates the senses of the verbs, as reflected in \tabref{tab:miyoshi:specified}. \ shows the situation numerically.

\begin{table}
\caption {the number of entries on infinitival and conjugated forms of individual verbs with the indication of their senses in the \textit{Promptorium}}
\label{tab:miyoshi:table5}
\begin{tabularx}{.8\textwidth}{ lYYYYYr }
\lsptoprule
 & {\emph{cast}} & {\emph{come}} & {\emph{fall}} & {\emph{get}} & {\emph{go}} & {\emph{make}} \\
\midrule
inf. & 2 & --- & 2 & 1 & --- & --- \\
ger. & --- & --- & --- & 1 & --- & --- \\
pp. & --- & --- & --- & --- & --- & --- \\
total & 2 & --- & 2 & 1 & --- & --- \\\midrule
 & {\emph{put}} & {\emph{set}} & {\emph{take}} & {\emph{throw}} & {\emph{turn}} & {total} \\\midrule
inf. & 1 & 3 & 1 & 1 & --- & 11 \\
ger. & 2 & 2 & 1 & 1 & --- & 7 \\
pp. & 1 & 1 & --- & --- & --- & 2 \\
total & 5 & 6 & 2 & 2 & --- & 20 \\
\lspbottomrule
\end{tabularx}
\end{table}

The author of the \emph{Promptorium} provides more than one entry of the infinitival forms of verbs in \emph{Table} \ref{tab:miyoshi:table5}, \emph{cast}, \emph{fall} and \emph{set}, indicating their senses in English. What we see here is the first discrimination of the senses of English high-frequency verbs in a dictionary.

Besides, considering the number of entries for conjugated verb forms in \emph{Table 5}, the \emph{Promptorium} provides 9 entries for forms whose head-words include the senses of verbs. Returning to \tabref{tab:miyoshi:specified}, this shows 2 cases where entries are provided for infinitival, gerundive and past-participial forms whose senses are the same, under the head-words:

\begin{quote}
    “\textbf{Puttyn}, or leyyn”, “\textbf{Puttynge}, or leyynge” and “\textbf{Put}, or leyd”.

    \rule{5cm}{.5pt}

    “\textbf{Settyn}, puttyn, or leyn”, “\textbf{Settynge}, or puttynge” and “\textbf{Sette}, or putte”.
\end{quote}

\tabref{tab:miyoshi:specified} also shows 4 cases where entries are provided for infinitival and gerundive forms whose senses are the same: “\textbf{Gettyn}, or wynnyn” and “\textbf{Gettyng}, or havying be wynnyng”; “\textbf{Settyn} and plantyn” and “\textbf{Settynge}, or plangynge”; “\textbf{Takyn}, or resevyn” and “\textbf{Takynge}, or reseyvynge”; and “\textbf{Throwyn}, or castyn” and “\textbf{Throwynge}, or castynge”.


\section{The \textit{Promptorium}'s treatment of phrasal verbs}

Having analyzed entries for the 11 verbs in the \emph{Promptorium} individually, I will now deal with those on idiomatic phrases related to these verbs in the following two sections, beginning with its treatment of phrasal verbs.

There are quite a few concepts of what a phrasal verb is. In this section, I define it as the following, lumping together what are called both literal meanings and figurative meanings

\begin{itemize}
    \item[(i)] that regarded as a transitive verb: an intransitive verb + a preposition(s)\slash a transitive verb + an adverbial particle(s).
    \item[(ii)] that regarded as an intransitive verb: an intransitive verb + an adverbial particle(s).
\end{itemize}

Based on this definition, \emph{Tables} \ref{tab:miyoshi:table6} and \ref{tab:miyoshi:table7}, show entries for phrasal verbs, like those in the analysis of entries for individual verbs in the previous section.

\begin{table}
\caption {Entries on phrasal verbs in the \textit{Promptorium} where their senses are “specified” in the head-word}
\label{tab:miyoshi:table6}
\begin{tabularx}{\linewidth}{QQQ}
\lsptoprule
\textbf{“inf.” (Latin)} &\textbf{“ger.” (Latin)} & \textbf{“pp.” (Latin)} \\
\midrule
“\textbf{Cum aftyr}, or folow” (\emph{succedo, sequor}) & &  \\
\tablevspace
“\textbf{Goyn abowten}, or wynlyllyn” (\emph{circumeo}) & &  \\
\tablevspace
“\textbf{Go behynd}, or folow” (\emph{sequor, descenao}) & &  \\
\tablevspace
“\textbf{Goo to}, or begyn a dede” (\emph{aggredior}) & &  \\
\tablevspace
“\textbf{Puttyn a-wey}, or refusyn” (\emph{repudio, refuto}) &  &  \\
\tablevspace
“\textbf{Puttyn forth}, as man doth his hand or oder lyke” (\emph{porrigo, extendo}) & & \\
 &  & “\textbf{Put togyder}, or onyd” (\emph{contiguus}) \\
\lspbottomrule
\end{tabularx}
\end{table}



\begin{table}
\caption {Entries for phrasal verbs in the \textit{Promptorium} where their senses are “not specified” in the head-words }
\label{tab:miyoshi:table7}
\small
\begin{tabularx}{\linewidth}{QQQ}
\lsptoprule
\textbf{“inf.” (Latin)} &\textbf{“ger.” (Latin)} & \textbf{“pp.” (Latin)} \\
\midrule
“\textbf{Castyn a-wey}” (\emph{abico, proicio}) & &  \\
 & “\textbf{Castyng downe} or a-wey” (\emph{proieccio}) & “\textbf{Cast downe}” (\emph{deiecio, prosterno}) \\
“\textbf{Cum dowyn}” (\emph{descendo}) & &  \\
“\textbf{Cum in}” (\emph{ingredior, introeo, ineo}) & &  \\
“\textbf{Cum to}” (\emph{advenio}) & &  \\
 & “\textbf{ffallyng dowyn}” (\emph{epilencia}) &  \\
“\textbf{Goone a-forne}” (\emph{precedo}) &  &  \\
“\textbf{Goon aftyr}” (\emph{succedo}) & & \\
“\textbf{Goon a-wey}” (\emph{recedo, descedo}) & & \\
“\textbf{Goo forthe}” (\emph{procedo}) &  &  \\
“\textbf{Gon owte}” (\emph{exeo, egredior}) & &  \\
“\textbf{Puttyn aftyr}” (\emph{propono}) & &  \\
“\textbf{Puttyn a-fore}” (\emph{prepono}) & &  \\
“\textbf{Puttyn a-wey}” (\emph{depono, expello, depello}) & &  \\
% \end{tabularx}
% \end{table}
%
% \begin{table}
% \begin{tabularx}{\linewidth}{QQQ}
\textbf{“inf.” (Latin)} &\textbf{“ger.” (Latin)} & \textbf{“pp.” (Latin)} \\
 & “\textbf{Puttyn owt}, or away” (\emph{eruo}) &  \\
 &  & “\textbf{Putt to} anythynge” (\emph{appono}) \\
“\textbf{Takyn a-wey} by strength or violens” (\emph{extorqueo}) & & \\
 & “\textbf{Throwynge downe} fro hey place” (\emph{precipicium}) & \\
 & “\textbf{Turnyng abowte}” (\emph{versio, giriversio}) &  \\
 \lspbottomrule
\end{tabularx}
\end{table}

\emph{Tables} \ref{tab:miyoshi:table8} and \ref{tab:miyoshi:table9}, on the number of entries have been produced in accordance with  \emph{Tables} \ref{tab:miyoshi:table6} and \ref{tab:miyoshi:table7},  since adopting this method reveals the treatment of phrasal verbs in the \emph{Promptorium}.

\begin{table}
\caption {The number of entries for phrasal verbs “with” the indication of their senses in the \textit{Promptorium}}
\label{tab:miyoshi:table8}
\begin{tabularx}{.8\textwidth}{ lYYYYYr }
\lsptoprule
 & {\emph{cast}} & {\emph{come}} & {\emph{fall}} & {\emph{get}} & {\emph{go}} & {\emph{make}} \\
\midrule
inf. & --- & 1 & --- & --- & 3 & --- \\
ger. & --- & --- & --- & --- & --- & --- \\
pp. & --- & --- & --- & --- & --- & --- \\
total & --- & 1 & --- & --- & 3 & --- \\\midrule
 & {\emph{put}} & {\emph{set}} & {\emph{take}} & {\emph{throw}} & {\emph{turn}} & {total} \\\midrule
inf. & 2 & --- & --- & --- & --- & 6 \\
ger. & --- & --- & --- & --- & --- & --- \\
pp. & 1 & --- & --- & --- & --- & 1 \\
total & 3 & --- & --- & --- & --- & 7 \\
\lspbottomrule
\end{tabularx}
\end{table} 

\begin{table}
\caption {The number of entries on phrasal verbs “without” the indication of their senses in the \textit{Promptorium}}
\label{tab:miyoshi:table9}
\begin{tabularx}{.8\textwidth}{ lYYYYYr }
\lsptoprule
 & {\emph{cast}} & {\emph{come}} & {\emph{fall}} & {\emph{get}} & {\emph{go}} & {\emph{make}} \\
\midrule
\textbf{inf.} & 1 & 3 & --- & --- & 5 & --- \\
\textbf{ger.} & 1 & --- & 1 & --- & --- & --- \\
\textbf{pp.} & 1 & --- & --- & --- & --- & --- \\
\textbf{total} & 3 & 3 & 1 & --- & 5 & --- \\
\midrule
 & {\emph{put}} & {\emph{set}} & {\emph{take}} & {\emph{throw}} & {\emph{turn}} & {total} \\
\midrule
\textbf{inf.} & 4 & --- & 1 & --- & --- & 14 \\
\textbf{ger.} & --- & --- & --- & 1 & 1 & 4 \\
\textbf{pp.} & 1 & --- & --- & --- & --- & 2 \\
\textbf{total} & 5 & --- & 1 & 1 & 1 & 20 \\
\lspbottomrule
\end{tabularx}
\end{table}

These numerical tables, show two notable facts; that is, the tendencies apparent in Tables  \ref{tab:miyoshi:table8} and \ref{tab:miyoshi:table9} are in stark contrast with those in the case of individual verbs in two respects. One is that the \textit{Promptorium} only provides a small number of entries for the conjugated forms of phrasal verbs; in fact, 7 (1+4+2) entries out of 27 (7+20), which make up 25.9\%, as compared to the entries for individual verbs where 12 entries out of 26 are given for their conjugated forms, accounting for 46.2\%. The other is that its provision of entries for phrasal verbs whose head-words include their senses are also small in number. The dictionary gives 20 entries for individual verbs whose head-words include their senses out of 26, accounting for 76.9\%, but provides only 7 entries for phrasal verbs whose head-words include their senses out of 27 (7+20), which make up 26.9\%.

These two facts may give the impression that the author’s treatment of phrasal verbs was quite slovenly and far from satisfactory. Still, he actually considers 27 such idioms related to the 11 verbs, which can hardly be thought small in number for the first dictionary that lists these idioms. How then can we interpret these contradictory situations? There are two answers to this question. One is the extension of Gabriele Stein’s analysis alluded to in Section 1, that is, the author of the \emph{Promptorium} changed the Latin-English entries of his sources into English-Latin ones, the English lemma still being basically an explanation of a former Latin-English dictionary, which applied to his treatment of individual verbs and phrasal verbs, as well. However, does this answer clarify the reason for the stark contrast between his treatment of individual verbs and that of phrasal verbs which I have shown, as well as the reason for his treatment of quite a few phrasal verbs?

Another is that the author of the \emph{Promptorium} basically treats usual phrasal verbs at his time whose senses are not necessary to be explained in head-words in the relevant entries. From this viewpoint, the reason why the indications of the senses in head-words, as seen in “\textbf{Cum aftyr}, or folow”, “\textbf{Goo to}, or begyn a dede” and “\textbf{Puttyn a-wey}, or refusyn”, signify that the relevant phrasal verbs concern those with figurative, not literal, meanings, seems to become understandable, the indications being a special type of annotation to avoid confusing readers. If this is the case, it is no wonder that the author intended to list as many phrasal verbs as possible in the dictionary.

\section{Its treatment of other notable phrases}

The term “notable phrases" in the title of this section, means phrases which are thought to be common at the time of the author of the \emph{Promptorium}. This means that the head-words of entries like the following, which only explain the senses of the Latin words provided, are totally irrelevant to such phrases:

\begin{quote}
    \textbf{Turnyng} fro bad to goode: \emph{Conversio, -is; fem.}, 3.
    
    \textbf{Turnyng} fro gode to bad: \emph{Peruersio, -is; fem.}, 3.

    \textbf{Turnyng} of divers ways: \emph{Diverticu lum, -i; neut.}, 2 [...]: \emph{Diversiclinicum, -ij; neut.}, 2 [...].
\end{quote}

From this point of view, entries concerning \emph{make}, which constitute 47 of the 128 entries related to the 11 verbs considered here, which make up 36.7\%, are very special. Of these 47, 45, most of which concern \emph{make} as a causative verb, seem to be provided for standard phrases at the time of the author of the dictionary. For this reason, I will mainly investigate entries for idiomatic phrases related to \emph{make}, pointing out remaining entries on such phrases by way of addendum at the end.

\subsection{Notable phrases related to \emph{make}}

The head-words of the 45 entries with Latin words provided are as indicated in \emph{Table} \ref{tab:miyoshi:table10}; the abbreviation “RAEI” in the parentheses standing for “reference to another entry indicated”.

\begin{table}
\caption {Notable head-words in entries concerning \textit{make} in the \textit{Promptorium}}
\label{tab:miyoshi:table10}
\small
\begin{tabularx}{\linewidth}{lQlQlQ}
\lsptoprule
 1 & “\textbf{Makyn able}” (\emph{abilito}) & 2 & “\textbf{Makyn a seth}” (\emph{satisfacio}) & 3 & “\textbf{Makyn better}” (\emph{melioro}) \\
 4 & “\textbf{Makyn bytter}” (\emph{exacerbo, similiter}) & 5 & “\textbf{Makyn blake}” (\emph{denigro}) & 6 & “\textbf{Makyn blunte}” (\emph{obtundo}) \\
 7 & “\textbf{Makyn clene}” (\emph{mundo, purifico}) & 8 & “\textbf{Makyn cumnawnte}” (\emph{pango}) & 9 & “\textbf{Makyn drunkyn}” (\emph{inebrio}) \\
 10 & “\textbf{Make dul}” (\emph{hebito, obtundo}) & 11 & “\textbf{Makyn evyn}” (\emph{equo}) & 12 & “\textbf{Make fette}, or fatte” (\emph{inpinguo, sagino}) \\
 13 & “\textbf{Make fowle}” (\emph{deturpo, sordido}) & 14 & “\textbf{Make gay}” (\emph{orno}) & 15 & “\textbf{Make fre}” (\emph{manumitto}) \\
 16 & “\textbf{Makyn harde}” (\emph{induro}) & 17 & “\textbf{Makyn hevy} in herte or sory” (\emph{contrusto, mestico}) & 18 & “\textbf{Makyn hevy } in wytte” (\emph{grauo}) \\
 19 & “\textbf{Make Ioy}” (RAEI) & 20 & “\textbf{Make knowyn}” (\emph{manifesto, natifico}) & 21 & “\textbf{Make large}” (\emph{amplio}) \\
 22 & “\textbf{Make lawful}” (\emph{legitimo}) & 23 & “\textbf{Make lene}” (\emph{macero}) & 24 & “\textbf{Make lesse}” (\emph{minoro}) \\
 25 & “\textbf{Make meynd}” (\emph{commemoro}) & 26 & “\textbf{Make wonderful}, or mervelyows” (\emph{mirifico}) & 27 & “\textbf{Make mery} and gladyn oder mene” (\emph{letifico}) \\
 28 & “\textbf{Make myry}, or be mery in hert or chere” (\emph{letor, iocundor}) & 29 & “\textbf{Makyn more}” (\emph{maioro}) & 30 & “\textbf{Make nesch}” (\emph{mollifico}) \\
 31 & “\textbf{Makyn perfyte}” (\emph{perficio}) & 32 & “\textbf{Make playne}” (\emph{plano}) & 33 & “\textbf{Make pleyne}” (\emph{conqueor}) \\
 34 & “\textbf{Make plentyows}” (\emph{ffecundo}) & 35 & “\textbf{Make qveynte}, or wonderful” (\emph{mirifico}) & 36 & “\textbf{Make redy}” (\emph{paro}) \\
 37 & “\textbf{Make rych}” (\emph{dito}) & 38 & “\textbf{Make paste}” (\emph{intero}) & 39 & “\textbf{Make sacrifyce}” (\emph{sacrifico}) \\
 40 & “\textbf{Make sekyr} in grawnte” (\emph{ratifico, confirmo}) & 41 & “\textbf{Make sythy}” (\emph{elucido}) & 42 & “\textbf{Make solemnyte}” (\emph{solemnyzo}) \\
 43 & “\textbf{Make tokyn} to a noder” (\emph{nuo, annuo}) & 44 & “\textbf{Make wery}” (\emph{ffatigo, lasso}) & 45 & “\textbf{Make with chylde}” (\emph{impregno}) \\
 \lspbottomrule
\end{tabularx}
\end{table}

This table, shows two notable facts. One is that all head-words of the 45 entries are infinitival forms and the other is that none of the head-words include the senses of relevant phrases, which may strengthen the possibility that the author took trouble of giving such indications only when necessary, far from simply inverting Latin head-words and English explanations in Latin-English dictionaries which may have been his sources, which I have suggested in the previous section.

Incidentally, it may be interesting to note that there are two cases in which the author distinguishes the use of phrases: (1) “\textbf{Makyn hevy} in herte or sory” (No. 17) and “\textbf{Makyn hevy} in wytte” (No. 18); and “\textbf{Make mery} and gladyn oder mene” (No. 27) and “\textbf{Make myry}, or be mery in hert or chere” (No. 28). The senses of \emph{makyn hevy} and \emph{make mery/myry} being not indicated, this may be regarded as the reflection of the author’s consciousness of the difference of semantic domains between the English phrases and Latin words provided for each. If this is the case, it may indicate his refined sense of language.

\subsection{Remaining notable phrases}

\begin{table}[b]
\caption {Remaining entries on notable phrases in the \textit{Promptorium}}
\label{tab:miyoshi:table11}
\begin{tabularx}{\linewidth}{ QQQ }
\lsptoprule
\textbf{“inf.” (Latin)} &\textbf{“ger.” (Latin)} & \textbf{“pp.” (Latin)} \\
\midrule
 &  & “\textbf{Cast werk}, or disposin”” (\emph{dispono}) \\
 & & “\textbf{Cast lott}” (\emph{sortior}) \\
“\textbf{Getyn by prayers}” (\emph{impetro}) & &  \\
“\textbf{Goon in-to a place}” (\emph{introeo, ingredior}) & &  \\
“\textbf{Goon on forte}” (\emph{pedito}) & &  \\
“\textbf{Go slowly}” (\emph{lento}) &  &  \\
“\textbf{Go wrong}” (\emph{devio, deliro}) &  &  \\
“\textbf{Settyn at noȝte}, or disposin” (\emph{vilipendo, ffloccipendo, nichilpendo}) & & \\
“\textbf{Settyn in yese and reste}” (\emph{quieto}) & & \\
 & “\textbf{Takyn on honde}” (\emph{manucapio}) &  \\
\lspbottomrule
\end{tabularx}
\end{table}

\tabref{tab:miyoshi:table11} shows the situation of the head-words in 10 remaining entries which are thought to concern usual phrases at the author’s time.



This table shows that the author only allocates senses to the phrases \emph{cast werk} and \emph{settyn at noȝt}, providing two head-words of conjugated forms, other head-words being infinitival forms without the indications of senses.

\section{Conclusion}
\largerpage
Having explored the treatment of infinitival and conjugated forms of English high-frequency verbs, as well as that of phrasal verbs and other idiomatic phrases, in the \emph{Promptorium}, we may now safely say that it is the earliest known dictionary that substantially exhibits the features I mentioned at the beginning of this chapter, so that “the \emph{Promptorium}, which was compiled as a ‘storehouse’ for children in the 1440’s, is the first dictionary that treated quite a few high-frequency verbs, distinguishing their senses and indicating their conjugated forms, with listing their relevant idiomatic phrases, especially phrasal verbs.

It is not that authorities thus far have not approached this point. In her meticulous analysis of her 1997 book, Gabriele Stein investigates its head-words from a comprehensive and overall perspective, dealing with their prepositional phrases, past participles, infinitives and infinitive clauses, \emph{etc.} and has clarified the whole picture of the dictionary. I consider that after a holistic analysis a specialized analysis should follow. From this viewpoint, I aimed to conduct a specific analysis of the treatment of English high-frequency verbs in the dictionary which Stein seems not to have done in relation to the history of lexicography.

One more point. If there is a grave misunderstanding that Samuel Johnson was the first lexicographer who detailed the high-frequency verbs, it may be a consequence of exaggerating the movement to ascertain the English language which arose at the beginning of the Late Modern English period. I do not deny that the movement was historically large in scale and that it actually induced Johnson to compile his \emph{Dictionary}. At the same time, however, we should not disregard the probability that there had been lexicographers of bilingual and polyglot dictionaries before him who had greatly troubled themselves over the treatment of the high-frequency verbs. I claim that  whoever issued the \emph{Promptorium} more than three hundred years before Johnson’s \emph{Dictionary}, or more than one hundred years before the birth of William Shakespeare, was one of such lexicographers.

\section*{Acknowledgments}
\largerpage
I want to express my sincere gratitude to Professor Emeritus John Considine (the University of Alberta in Canada) for his invaluable and inspiring suggestions in writing this paper. Remaining errors, if any, are my own. Since my presentation at ICHLL (International Conference for Historical Lexicography and Lexicology) in 2022, on which this chapter is based, Professor Emeritus Jean-Louis Duchet kindly informed me that the \emph{Promptorium} also lists verbs and verbal phrases which were specific at the time, such as \emph{delvyn owt off þe earth}, which may broaden the horizon of my future research; I also want to extend to him my sincere gratitude.

{\sloppy\printbibliography[heading=subbibliography,notkeyword=this]}
\end{document}
