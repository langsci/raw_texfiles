\documentclass[output=paper,colorlinks,citecolor=brown,arabicfont,chinesefont,booklanguage=french]{langscibook}
\ChapterDOI{10.5281/zenodo.15394489}
\author{Rosa Cetro\affiliation{Université de Pise / PLIDAM, INALCO}}
\title[De Félibien à Boutard]{De Félibien à Boutard: L’évolution du dictionnaire artistique entre le XVIIème et le début du XIXème siècle}

\abstract{Artistic terminology displays a hybrid and complex character, due to several factors (among others, the vastness of the field, the coexistence of erudite and ordinary technolects, the status of the arts over the centuries). The first lexicographic description of this terminology in the French language was written by the historiographer André Félibien. This chapter aims to show the evolution and the different variations of the artistic dictionary on a corpus of five monolingual French dictionaries covering a century and a half (1676--1826).}

\IfFileExists{../localcommands.tex}{
  \addbibresource{../localbibliography.bib}
  % add all extra packages you need to load to this file

\usepackage{tabularx,multicol}
\usepackage{url}
\urlstyle{same}

\usepackage{listings}
\lstset{basicstyle=\ttfamily,tabsize=2,breaklines=true}

\usepackage{langsci-basic}
\usepackage{langsci-optional}
\usepackage{langsci-lgr}
\usepackage{langsci-osl}
% \usepackage{./langsci/styles/langsci-lgr}
% \usepackage{./langsci/styles/langsci-osl}
% \usepackage{langsci-gb4e}

\usepackage{tikz}
\usetikzlibrary{patterns,calc}
\pgfdeclarepatternformonly{south east lines}{\pgfqpoint{-0pt}{-0pt}}{\pgfqpoint{3pt}{3pt}}{\pgfqpoint{3pt}{3pt}}{
    \pgfsetlinewidth{0.6pt}
    \pgfpathmoveto{\pgfqpoint{0pt}{3pt}}
    \pgfpathlineto{\pgfqpoint{3pt}{0pt}}
    \pgfpathmoveto{\pgfqpoint{.2pt}{-.2pt}}
    \pgfpathlineto{\pgfqpoint{-.2pt}{.2pt}}
    \pgfpathmoveto{\pgfqpoint{3.2pt}{2.8pt}}
    \pgfpathlineto{\pgfqpoint{2.8pt}{3.2pt}}
    \pgfusepath{stroke}}
    
\usepackage{stmaryrd}
\usepackage{wasysym}
\usepackage{multirow}
\usepackage{caption}
\usepackage{subcaption}
\usepackage{mathrsfs}
\usepackage{qtree}

\usepackage{linguex}


  %pminos do not split footnotes
% \interfootnotelinepenalty=10000 %Footnote in Laporte chapters has to be split SN


%\DeclareIndexNameFormat{default}{%
%\nameparts{#1}%
%\usebibmacro{index:name}%
%{\index[names]}%
%{\namepartfamily}%
%{\namepartgiveni}%
% {}% L1
% {}% L2
%{\namepartprefix}% generates spurious space L3
%{\namepartsuffix}% generates spurious space L4
%}

%  {\DeclareIndexNameFormat{default}{%
%     \usebibmacro{index:name}{\index[names]}{#1}{#3}{#5}{#7}}}

%\DeclareIndexNameFormat{default}{%
%  \usebibmacro{index:name}{\sindex[nom]}{#1}{#3}{#5}{#7}}

%\DeclareIndexNameFormat{default}{%
%  \usebibmacro{index:name}{\sindex[person]}{#1}{#3}{#5}{#7}}
%\DeclareIndexNameFormat{default}{%
%\nameparts{#1} \usebibmacro{index:name}{\sindex[person]]}{\namepartfamily}{‌​\namepartgiven}{\nam‌​epartprefix}{\namepa‌​rtsuffix}}

%\newcommand{\smiley}{:)}

%\renewbibmacro*{index:name}[5]{%
%\usebibmacro{index:entry}{#1}%
%{\iffieldundef{usera}{}{\thefield{usera}\actualoperator}\mkbibindexname{#2}{#3}{#4}{#5}}}

% \newcommand{\noop}[1]{}

%remove for final
%\overfullrule=1mm

\newcommand{\tobi}[2]}}
\renewcommand{\S}[1]{\tobi{#1}{\textsc{*}}}

% this volume references
% puts: [this volume]
% already defined: \citetv
%\newcommand{\citepv}[1]{(\citeauthor{#1} \citeyear*{#1} [this volume])}
\newcommand{\citealtv}[1]{\citeauthor{#1} \citeyear*{#1} [this volume]}

%parentheses around example number
\newcommand{\pref}[1]{(\ref{#1})}

% in-text examples

\newcommand{\lnex}[1]{\textit{#1}} %target lang word
\newcommand{\lnlit}[1]{(lit.: `#1')} %literal reading
\newcommand{\lnlat}[1]{(#1)} % latinization
\newcommand{\lntrans}[1]{`#1'} %translation
\newcommand{\lnexl}[2]%
{\lnex{#1}{} \lnlat{#2}} % ex with latinization
\newcommand{\lnexlat}[3]{\lnex{#1}{} \lnlat{#2}{} \lntrans{#3}} % ex with latinization and tranl.

%ch01
\newcommand{\co}[1]{\mbox{\textbf{#1}}}

%ch09

\newcommand{\cyrbulg}[1]{\begin{otherlanguage*}{bulgarian}#1\end{otherlanguage*}}


%ch10
\newcommand{\nlp}{{\small NLP}}
\newcommand{\mwe}{{\small MWE}}
\newcommand{\rae}{{\small RAE}}
\newcommand{\lvc}{{\small LVC}}
\newcommand{\pos}{{\small P}o{\small S}}
%\newcommand{\todo}[1]{ \textcolor{red}{#1} }

%\renewcommand{\labelenumi}{\theenumi}
%\ainamefmt{{vv}{ll}{, ff}{, jj}} % fullname

\newcommand{\biberror}[1]{{\color{red}#1}}

\newcommand{\osenovaitem}{--~}
  %% hyphenation points for line breaks
%% Normally, automatic hyphenation in LaTeX is very good
%% If a word is mis-hyphenated, add it to this file
%%
%% add information to TeX file before \begin{document} with:
%% %% hyphenation points for line breaks
%% Normally, automatic hyphenation in LaTeX is very good
%% If a word is mis-hyphenated, add it to this file
%%
%% add information to TeX file before \begin{document} with:
%% %% hyphenation points for line breaks
%% Normally, automatic hyphenation in LaTeX is very good
%% If a word is mis-hyphenated, add it to this file
%%
%% add information to TeX file before \begin{document} with:
%% \include{localhyphenation}
\hyphenation{
    Beck-man
    Ngu-yen
    back-chan-nel
    back-chan-nels
    mo-not-o-nous
    ste-reo-typ-i-cal
}

\hyphenation{
    Beck-man
    Ngu-yen
    back-chan-nel
    back-chan-nels
    mo-not-o-nous
    ste-reo-typ-i-cal
}

\hyphenation{
    Beck-man
    Ngu-yen
    back-chan-nel
    back-chan-nels
    mo-not-o-nous
    ste-reo-typ-i-cal
}

  \togglepaper[11]%%chapternumber
}{}

\begin{document}
\begin{otherlanguage}{french}
\maketitle

\section{Introduction}

Le dictionnaire monolingue français naît vers la fin du XVIIe siècle, avec la publication de trois ouvrages fondateurs, ce que \citet[463]{Pruvost2003} appelle la «~trilogie fondatrice~», à savoir les dictionnaires de \citet{Richelet1680}, de \citet{Furetière1690} et la première édition du \emph{Dictionnaire de l’Académie française} (1694). Cela est bien connu. Ce que l’on sait peut-être moins est que d’autres tentatives lexicographiques monolingues ont vu le jour avant ces ouvrages fondateurs, notamment dans des domaines spécialisés. C’est le cas, par exemple, du dictionnaire de termes d’art compilé par André Félibien (1619--1695) et annexé au traité \emph{Des Principes de l’Architecture, de la Sculpture et de la Peinture et des autres Arts qui en dépendent} (1676). Signé par Félibien, cet ouvrage est cependant le fruit du travail de toute une institution, celle de l’Académie royale d’Architecture, fondée en 1671 et dont Félibien était secrétaire.

Malgré ses limites, ce dictionnaire s’avère être un observatoire intéressant pour l’étude des dynamiques qui accompagnent la mise en place du «~chantier lexicographique en français~» du XVIIe siècle, ainsi qu’un modèle de référence dans la lexicographie spécialisée sur l’art. Dans ce chapitre, nous voudrions retracer l’évolution – et les déclinaisons – du genre dictionnaire artistique en langue française de la fin du XVIIe siècle jusqu’au début du XIXe siècle, en nous appuyant sur un corpus de cinq dictionnaires représentatifs. Il s’agira, d’un côté, de relever les questionnements qui sous-tendent la description lexicographique mise en place par les différents auteurs et qui sont influencés à la fois par leur conception du vaste domaine artistique et par les publics cibles auxquels ces ouvrages s’adressent. De l’autre côté, une confrontation constante sera menée entre ces ouvrages spécialisés et la lexicographie généraliste.

Après avoir présenté les dictionnaires composant le corpus ainsi que leurs auteurs, nous analyserons les objectifs poursuivis par chaque ouvrage et les destinataires auxquels ils s’adressent, avant de passer à une comparaison entre leurs macrostructures et microstructures.

\section{Présentation du corpus et méthodologie}
\largerpage
Le corpus est composé des cinq dictionnaires suivants~: le précité Dictionnaire annexé au traité \emph{Des Principes de l’Architecture, de la Sculpture et de la Peinture et des autres Arts qui en dépendent} (1676) d’André Félibien, le \emph{Dictionnaire abrégé de peinture et d’architecture} (1746) de François Marie de Marsy, le \emph{Dictionnaire portatif des beaux-arts} (1752) de Jacques Lacombe, le \emph{Dictionnaire portatif de peinture, sculpture et gravure, avec un traité pratique sur les differentes manieres de peindre} (1757) d’Antoine-Joseph Pernety et le \emph{Dictionnaire des arts du dessin, la peinture, la sculpture, la gravure et l’architecture} (1826) de Jean-Baptiste-Bon Boutard. 

Un rapide survol des titres nous montre la diversité du format de ces ouvrages et la variété des sous-domaines artistiques sur lesquels se focalisent les auteurs. Par ailleurs, il ne nous semble pas superflu de donner quelques informations biographiques concernant ceux-ci. 

André Félibien (1619--1695) a été un homme de pointe du Grand Siècle. Nommé Historiographe des Bâtiments du Roi Soleil par le ministre Colbert, il a fait partie de trois académies royales (Académie des Inscriptions et des belles lettres, Académie de Peinture et de Sculpture et Académie royale d’Architecture). Spécialiste reconnu des beaux-arts et écrivain prolifique, il est surtout connu pour son cycle des \emph{Entretiens sur les vies et les ouvrages des plus excellens Peintres} (1666--1688)\footnote{Pour plus d’informations sur André Félibien et son dictionnaire, nous nous permettons de renvoyer à notre ouvrage \citep{Cetro2022}.}.

Le Jésuite François Marie de Marsy (1710--1763), auteur du \emph{Dictionnaire abrégé de peinture et d’architecture} (1746) en deux tomes, a été professeur de grammaire, d’humanités et de rhétorique au collège Louis-le-Grand de Sommervogel \citep[690]{Sgard1999}.

Jacques Lacombe (1724--1811), avocat de formation, a été un important libraire puis imprimeur dans la France des Lumières, proche des Philosophes (notamment de Voltaire). Il s’est occupé de la compilation des dictionnaires de remplissage de l’\emph{Encyclopédie méthodique} pour Panckoucke \citep[188]{MartinChartier1984}.  Son \emph{Dictionnaire portatif des beaux-arts} (1752), un de ses premiers ouvrages et fruit de son intérêt pour la critique d’art, a connu un grand succès \citep[550--551]{Sgard1999}.

Antoine-Joseph Pernety (1716--1796) dit Dom Pernety a été un bénédictin mauriste défroqué, alchimiste et écrivain prolifique. Il est connu surtout pour avoir fondé le mouvement maçonnique des \emph{Illuminés d’Avignon}. Comme on peut le deviner du titre, son \emph{Dictionnaire portatif de peinture, sculpture et gravure, avec un traité pratique sur les differentes manieres de peindre} (1757) reprend la formule adoptée par Félibien d’un dictionnaire associé à un traité. Cependant, ici la focalisation est sur la peinture, dont l’auteur retrace le statut dans l’Antiquité d’après des sources anciennes (par ex. Pline) avant d’aborder les différentes techniques picturales. 

Pour finir, Jean-Baptiste-Bon Boutard (1771--1838), architecte, publie son \emph{Dictionnaire des arts du dessin, la peinture, la sculpture, la gravure et l’architecture} (1826)\footnote{Dorénavant, les dictionnaires du corpus seront indiqués par le nom de l’auteur suivi de l’année de publication entre parenthèses~: \citet{Félibien1676,Marsy1746,Lacombe1752,Pernety1757} et \citet{Boutard1826}.} après une longue collaboration (1800--1823) avec le \emph{Journal des Débats} en tant que publiciste sur les beaux-arts.

À l’exception de \citet{Felibien1676}, que nous avons numérisé pour des travaux précédents \citep{Cetro2022}, les dictionnaires du corpus ne sont pas exploitables de façon automatique, ce qui nous oblige à un dépouillement manuel. La comparaison entre les cinq ouvrages a été donc menée à partir d’un échantillon. En raison de la présence de termes-clés de ce domaine (comme \emph{colonne} et \emph{couleur}), le choix est tombé sur la lettre C.

\section{Buts et destinataires des ouvrages}

L’analyse des préfaces et des avertissements de ces ouvrages – là où ils sont présents – nous renseigne sur les objectifs poursuivis par leurs auteurs et sur les destinataires potentiels auxquels ils s’adressent. S’il est vrai que tous ont pour cible un public cultivé, appartenant aux élites françaises, il est tout aussi vrai que l’on peut assister à un élargissement du nombre de lecteurs touchés par ce genre de publication au fil des décennies, grâce aussi aux progrès de l’alphabétisation et à l’engouement pour les dictionnaires au XVIIIe siècle. Comme le soulignent \citet [190]{MartinChartier1984} 

\begin{quote}
    le dictionnaire devient une des formes canoniques, et la plus dynamique, de l’accumulation du savoir
\end{quote}

Ainsi passe-t-on de Félibien 1676, conçu pour un public très restreint – les commanditaires d’œuvres d’art \citep[178]{Maes2016} –, à un nombre de plus en plus vaste de lecteurs-amateurs qui désirent s’instruire sur ce sujet et faire usage des connaissances acquises dans la conversation mondaine. Les dictionnaires d’art deviennent ainsi de véritables «~vecteurs d[e] sociabilité~» \citep[182]{Maes2016}.

Le seul ouvrage du corpus à ne pas offrir une préface ou un avertissement est \citet{Marsy1746}. \citet{Felibien1676}, \citet{Pernety1757} et \citet{Boutard1826} offrent une préface, tandis que \citet{Lacombe1752} propose au lecteur un avertissement précédé d’une épitre aux beaux-arts. Dans le cas de \citep{Boutard1826}, la préface n’est pas rédigée par l’auteur du dictionnaire mais par ses éditeurs. L’objectif de ces pages est de présenter les contenus des ouvrages et de justifier les choix opérés par les auteurs.

On peut identifier quatre parties principales dans la préface de \citet{Felibien1676}. Dans le paragraphe d’ouverture, l’auteur explique que la terminologie artistique employée dans ses ouvrages précédents n’a pas été tout à fait comprise par ses lecteurs. Il a donc décidé d’en entreprendre la description pour faciliter l’accès au sens de ces mots. Puisqu’il est bien plus aisé de comprendre l’utilisation des termes lorsqu’ils sont insérés dans un discours, le projet initial de l’auteur de décrire ces termes à l’intérieur d’un dictionnaire se tourne vers la forme d’un traité suivi d’un dictionnaire. De ces deux premières parties ressort clairement la vocation pédagogique de l’ouvrage. Dans la troisième, l’accent est mis sur le service que cet ouvrage peut rendre au public visé dans la communication avec les artisans. Cette attention à la dimension communicationnelle justifie aussi la présence de nombreux termes appartenant aux technolectes ordinaires \citep{Messaoudi2013}, employés dans les ateliers. Pour finir, dans la quatrième se manifeste la volonté de l’auteur de fournir à la postérité un témoignage précieux sur les arts figuratifs pendant le Grand Siècle.

\citet{Lacombe1752} résume dans le premier paragraphe de son \emph{Avertissement} les raisons d’être de l’ouvrage~:

\begin{quote}
    Le goût que le Public témoigne pour les BEAUX-ARTS, l’empressement avec lequel il saisit tout ce qu’on lui présente sur cet objet, enfin l’utilité, je dirai même la nécessité d’un Livre qui renferme les Recherches \& les Connoissances d’un \emph{Amateur}~; tels sont les principaux motifs qui m’ont poussé à entreprendre cet Ouvrage. 
\end{quote}

Bien qu’il existe déjà des textes sur le même sujet, il estime que ceux-ci sont lacunaires à plusieurs égards, notamment dans le traitement de l’ensemble des beaux-arts. De plus, il choisit d’inclure aussi dans son dictionnaire – genre textuel qu’il qualifie de «~commode~» – des articles sur les plus grands artistes de tous les temps et de tous les pays ainsi que sur leurs chefs-d’œuvre, sans pour autant délaisser la description de la terminologie des beaux-arts. Comme nous le verrons dans la section suivante, sous cette dénomination Lacombe inclut aussi la musique et la poésie. Ces deux arts sont évoqués dans l’avertissement pour souligner la richesse des contenus du dictionnaire, ainsi que l’exactitude des informations fournies~: 

\begin{quote}
    […] En un mot, on n’a rien négligé pour s’assurer de l’exactitude des recherches~; \& on a tâché surtout de concilier la netteté avec la précision, \& l’utile avec l’agréable.  
\end{quote}

L’auteur passe ensuite à donner quelques-unes des références bibliographiques qu’il a consultées pour étayer ses recherches. L’avertissement se clôt sur la présentation de quelques musiciens illustres qui, faute de documentation suffisante, ne se voient pas attribuer un article à l’intérieur du dictionnaire. 

La préface de \citealt{Pernety1757} s’ouvre par une longue section de défense du genre dictionnaire, dans laquelle l’auteur fait référence au succès de ce genre d’ouvrages publiés au XVIIIe siècle et aux critiques de ceux qui considèrent qu’ils ne fournissent qu’une érudition superficielle~:

\begin{quote}
    Les Dictionnaires sont […] nécessaires \& plus aujourd’hui que jamais, parce que le goût des Sciences \& des Arts a gagné tous les États. On veut sçavoir tout, ou plutôt parler de tout, \& ne paroître ignorer de rien~: il faut donc se prêter à ce goût du siècle.
\end{quote}

Il passe ensuite à présenter les spécificités du lexique de la peinture, qui peut être de difficile compréhension car il emploie beaucoup de termes usuels de façon imagée. Dans la dernière partie de la préface, Pernety explique les raisons qui l’ont poussé à rédiger cet ouvrage~: d’un côté, un souci franchement pédagogique envers les amateurs et les élèves d’art – «~la volonté de me rendre utile~», comme il le dit –, qui justifie entre autres le choix de faire précéder d’un traité le dictionnaire et, de l’autre, l’envie de pallier aux manques des ouvrages précédents. 

La confrontation avec d’autres ouvrages du même genre est aussi un \emph{leitmotiv} de la préface des éditeurs de \citealt{Boutard1826} qui, contrairement aux autres auteurs, en font expressément mention. Après avoir souligné l’importance des arts comme sujet de conversation et l’engouement des «~gens du monde~» – destinataires cibles du texte – pour ceux-ci, ils rappellent la réticence de l’Académie française à insérer ces termes spécialisés dans son dictionnaire. Les autres ressources lexicographiques citées – l’\emph{Encyclopédie} de Diderot et d’Alembert, \emph{l’Encyclopédie méthodique}, le dictionnaire de Watelet et celui de Lunier –, bien qu’elles contiennent des articles consacrés aux termes artistiques, affichent néanmoins des défauts. En particulier, les préfaciers déplorent le fait que les arts soient traités parmi une multitude de sujets, par le biais d’abondantes dissertations pouvant fatiguer le lecteur. Ce qui plus est, dans quelques cas ces articles ne sont même pas originaux, mais se limitent à copier d’autres dictionnaires. D’une façon générale, toutes ces ressources sont datées pour les contemporains de Boutard, ne prenant pas en compte les avancées techniques survenues au XIXe siècle.

L’attention accordée aux néologismes constitue donc un atout de \citet{Boutard1826}. Ceux-ci peuvent désigner de nouvelles réalités dues aux avancées des procédés mécaniques ou encore de nouvelles idées. Cet argument n’est cependant pas le seul pilier sur lequel les préfaciers bâtissent leur discours promotionnel de l’ouvrage de Boutard. Ils en louent l’équilibre :

\begin{quote}
    [le] milieu entre l’aridité des lexicographes qui n’ont eu à rendre compte que de la signification des mots et la prolixité déclamatoire, jusqu’ici trop ordinaire aux littérateurs qui ont écrit sur les beaux-arts, 
\end{quote}

l’originalité et le caractère de nouveauté qu’il représente dans le panorama éditorial français de l’époque. L’autorité reconnue de l’auteur dans le domaine ne peut être qu’un gage de succès~: 

\begin{quote}
    […] Il nous a semblé qu’un écrivain habitué depuis longues années à manier la langue de l’art, dans un commerce en quelque sorte familier avec le public, qui souvent a dû éprouver par lui-même la difficulté de se faire comprendre de ses lecteurs, et qui a plus qu’aucun autre l’occasion d’observer sur quels mots, quelles expressions, quelles idées, porte particulièrement cette difficulté, seroit aussi plus qu’un autre habile à composer le dictionnaire de cette langue, tel que nous l’avions conçu. 
\end{quote}

\section{La macrostructure}

Examiner la macrostructure de ces dictionnaires revient à identifier à la fois la façon dont ces auteurs ont délimité le vaste domaine des arts et les choix opérés dans la description de la nomenclature. 

\subsection{Le découpage du domaine}

La terminologie artistique affiche un caractère hybride et complexe, dont les causes sont à rechercher dans le statut que ces arts ont recouvert au fil des siècles. Dans la hiérarchisation aristotélicienne des disciplines en théorétiques, pratiques et poïétiques, les arts figuratifs occupaient cette dernière catégorie \citep{Eusebi2013}, en tant qu’arts mécaniques et opposés aux arts libéraux comme la géométrie ou la musique. Maintenue tout au long du Moyen Âge, cette distinction ne sera remise en cause que pendant l’Humanisme et la Renaissance italiens, suite à la redécouverte de la culture classique et à la diffusion de sources textuelles jusque-là inconnues par le biais de traités spécialisés rédigés exclusivement en latin \citep{Biffi2007}. Cela provoque une révolution épistémologique et culturelle qui a pour résultat l’élévation de l’architecture, de la sculpture et de la peinture au rang des arts libéraux. Pour l’architecture, un travail remarquable de systématisation est accompli par l’architecte italien Leon Battista Alberti à partir de la refonte de la terminologie de Vitruve, source classique de référence pour cette discipline, et de l’alignement entre ce lexique classique et le lexique artisanal d’origine régionale. Faute de références de l’Antiquité, la terminologie de la peinture subit en revanche l’influence de la littérature en volgare \citep[8]{Motolese2012}, en particulier des \emph{Vite} du peintre arétin Giorgio Vasari. Les quatre composantes du lexique vasarien identifiées par \citet[12]{Le_molle1988} – théorique, critique, artistique et artisanale – témoignent de la coprésence de différents registres au sein de cette terminologie.

À partir du XVIe siècle, les ouvrages de Vasari, Alberti et d’autres artistes/experts d’art font l’objet de traductions et se diffusent ainsi dans d’autres pays, faisant entrer bon nombre de termes italiens dans d’autres langues européennes, dont le français.  

La compilation d’une ressource lexicographique spécialisée est réalisée à partir de l’identification préalable du secteur de la connaissance – le «~domaine~» selon \citet{Wuster1981} – qui fait l’objet de la description. Cette tâche peut ne pas être aisée, notamment pour les terminologies riches et hétérogènes comme celle de l’art, pour lesquelles la notion de domaine s’avère être rigide et peu fonctionnelle.

Or, la comparaison entre ces cinq dictionnaires met en exergue des différences dans le découpage du domaine décrit qui sont dues aussi bien aux connaissances de l’auteur qu’aux choix opérés en fonction des destinataires-cibles. Même les titres des ouvrages contiennent des informations importantes à cet égard. \citet{Felibien1676} s’intéresse aux trois arts principaux – architecture, peinture et sculpture – et aux arts qui en dépendent, à savoir les arts mécaniques comme la charpenterie, la menuiserie, la serrurerie, l’orfèvrerie, la maçonnerie, à travers la description des technolectes ordinaires \citep{Messaoudi2013}, typiques de la communication orale et liés aux objets-référents \emph{in situ}. En tant qu’expert du domaine, Félibien adopte une approche onomasiologique, qui va du sens aux mots.

\citet{Marsy1746} se limite aux sous-domaines de la peinture et de l’architecture, tandis que \citet{Lacombe1752} ajoute aux trois arts figuratifs principaux la musique et la poésie, se focalisant ainsi sur les beaux-arts, exprimés par les technolectes savants.

Si \citet{Pernety1757} ne prend en considération que la peinture, la sculpture et la gravure, \citet{Boutard1826} ajoute à celles-ci l’architecture, dont il est spécialiste, et indique ce regroupement de disciplines par l’étiquette «~arts du dessin~». Cependant, à la différence de Félibien, Boutard exclut de son dictionnaire les termes techniques employé dans les ateliers. Sans doute Boutard choisit-il sciemment de ne pas employer l’appellation «~beaux-arts~» qui inclut aussi la musique après la constitution officielle de l’Académie des Beaux-Arts en 1816, héritière des Académies royales de Sculpture et Peinture (1666), de Musique (1669) et d’Architecture (1671).

\subsection{La nomenclature}

Avant de rentrer dans les détails de la microstructure, voyons comment les données lexicographiques sont présentés au lecteur. Tous les auteurs optent pour l’agencement alphabétique des articles, qui sont disposés sur une seule colonne occupant la page dans \citet{Felibien1676}, \citet{Marsy1746} et \citet{Boutard1826} ou sur deux colonnes dans \citet{Lacombe1752} et \citet{Pernety1757}.

Dans tous les dictionnaires la vedette est affichée en lettres capitales ou en petites capitales. Comme dans les dictionnaires de langue, les verbes sont enregistrés à l’infinitif, les adjectifs au masculin singulier et les noms au singulier. Cependant, il n’est pas rare de trouver la forme au pluriel comme vedette si le nom est employé au pluriel et ce choix concerne tous les dictionnaires du corpus. \citet{Felibien1676} adopte le critère du dégroupement homonymique \citep{Pruvost2006} dans la description des différentes acceptions d’un même terme~: chaque acception occupe une ligne différente, comme ce sera aussi le cas pour les dictionnaires de \citet{Richelet1680} ou de l’Académie française (1694). Ce critère est repris aussi par les autres lexicographes avec quelques exceptions~: \citet{Lacombe1752} n’opte pour le dégroupement homonymique que lorsqu’un même terme est employé dans des sous-domaines différents (tel est le cas, par exemple, du terme \emph{camayeux} utilisé en peinture et en gravure), \citet{Boutard1826} n’utilise pas souvent les capitales ou petites capitales pour indiquer le terme-vedette dans une acception différente.

Il en va de même pour le traitement des collocations, qui sont disposées sur une ligne séparée à l’intérieur de l’article du nom-tête à partir duquel elles sont formées. Elles peuvent aussi constituer des entrées indépendantes, comme par exemple le terme \emph{amitié des couleurs} que \citet{Felibien1676} décrit séparément et non pas au sein de l’article \emph{couleur}, ou tous les termes composés à partir du nom-tête \emph{colle} dans le dictionnaire de \citet{Pernety1757}, auxquels l’auteur consacre des entrées individuelles. Dans ce dernier cas, il nous semble que ce choix est fonctionnel aux enjeux pédagogiques de l’ouvrage. De son côté, \citet{Boutard1826} décide d’adopter un critère morphologique, selon lequel les mots composés par trait d’union – comme \emph{clair-obscur} ou \emph{contre-allée} – constituent des entrées indépendantes, tandis que les collocations sont traitées à l’intérieur de l’article du nom-tête. Ainsi tous les types de colonnes sont-ils décrits sous l’article \emph{colonne}, conformément à la démarche encyclopédique de l’auteur. Contrairement aux autres lexicographes, \citealt{Lacombe1752} décrit souvent les collocations en faisant apparaitre le collocatif en position de vedette et le nom-tête entre parenthèses~: à titre d’exemple, nous citerons \emph{crayon (dessin au)} et \emph{chinoise (peinture)}.

Il peut arriver aussi que deux ou plusieurs vedettes ouvrent un même article~: il s’agit de mots formés à partir d’un même radical, comme \emph{conduire} et \emph{conduite} ou \emph{correct}, \emph{correction} et \emph{correctement}. C’est \citet{Marsy1746} qui opte pour ce regroupement par racines, caractérisant la première édition du \emph{Dictionnaire de l’Académie} \citep{AcadFr1694} et puis abandonné par les Académiciens dès la deuxième. Ce procédé permet à l’auteur de traiter ensemble des mots pouvant être employés pour exprimer un même concept, sans avoir pour cela à créer deux articles séparés. À titre d’exemple, nous reportons ici l’article \emph{conduire}, \emph{conduite}~:

\begin{quote}
    CONDUIRE, CONDUITE. 
    
    \emph{Conduire} signifie diriger, ménager, distribuer. 
    
    Des jours \& des ombres \emph{conduits} judicieusement. Félibien a dit~: un tableau bien \emph{conduit} de couleurs, c’est-à-dire, où les couleurs sont ménagées, \& distribuées avec Art. il y a beaucoup de \emph{conduite} dans les compositions de Poussin, c’est-à-dire, beaucoup d’entente \& d’ordonnance. 
\end{quote}

\section{La microstructure}

La comparaison entre les cinq dictionnaires du corpus sera approfondie dans cette section, dans laquelle nous nous focaliserons sur leur microstructure, à savoir les types d’informations contenues au sein des articles. Parmi celles-ci, notre attention portera en particulier sur les catégories syntaxiques et sémantiques faisant partie de la nomenclature, sur les types de définitions employées et sur les marques de domaine. La comparaison sera étayée par l’analyse d’un même article. 

\subsection{Catégories syntaxiques et sémantiques}

Comme on peut le déduire à partir de ce qui a été présenté dans les sections précédentes, ces dictionnaires ne contiennent pas les mêmes lemmes. Si une partie de la nomenclature est commune aux cinq ouvrages, force est de constater qu’ils affichent des différences aussi bien au niveau quantitatif – nombre de lemmes traités – que qualitatif. S’il est vrai que \citet{Felibien1676} contient le plus grand nombre d’entrées (372) pour la lettre C, il ne faut pourtant oublier qu’une partie de celles-ci ne constituent que des renvois au traité ou aux planches contenues dans celui-ci et sont donc dépourvues d’une définition.

Dans les ressources terminographiques, les noms constituent la catégorie syntaxique la plus représentée. Les cinq dictionnaires du corpus ne font pas exception dans ce sens, mais ils accueillent également des verbes, des adjectifs (et des participes passés employés comme adjectifs) et quelques adverbes. Parmi ces dictionnaires, le seul à fournir une indication de la catégorie grammaticale au sein des articles est \citet{Boutard1826}. Deux dictionnaires du XVIIIe siècle – \citet{Marsy1746} et \citet{Lacombe1752} – font la part belle aux noms propres, qui désignent surtout des artistes\footnote{Dans \citet{Marsy1746} bon nombre de noms propres désignent aussi des monuments.}, français ou internationaux. Il en résulte alors un genre hybride, combinant les caractéristiques du dictionnaire spécialisé sur l’art et celles du recueil de biographies d’artiste. Ce dernier est en effet très en vogue dans les mêmes années (1745--1755), notamment pour les vies des peintres \citep[171]{Maes2016}. À côté des artistes, les noms propres peuvent aussi désigner des divinités ou des figures de la mythologie représentées dans des œuvres célèbres. Dans les articles consacrés à ces noms propres on peut aussi trouver le récit de ces mythes. Le dictionnaire abrégé\slash portatif accomplit ainsi sa mission~:

\begin{quote}
    Aux côtés de l’\emph{Encyclopédie}, le portatif obéit à un besoin profond de l’époque~: celui de vulgariser le savoir, et aussi de le rendre accessible, à un prix relativement modique, à une clientèle cultivée mais élargie. […] Mais le portatif est aussi et surtout le domaine de la liberté, de la sollicitation permanente des goûts du public par les compilateurs de toute sorte. [...] Le portatif ressort dans ce contexte global comme sous-dictionnaire à tout faire, mais il est lui-même ouvert aux influences encyclopédiques. \citep[191]{MartinChartier1984}
\end{quote}

Voyons à présent les catégories sémantiques représentées dans la nomenclature des dictionnaires du corpus. Les verbes désignent principalement des techniques ou des actions, tandis que les adjectifs se réfèrent à des qualités ou à des caractéristiques. Pour les substantifs, il est possible d’identifier des catégories sémantiques communes à tous les dictionnaires du corpus – bien que les nomenclatures ne se recouvrent pas parfaitement – comme celles des ornements et des décorations, des styles, des techniques, des couleurs, des parties de la peinture, des objets. À l’exception de \citealt{Pernety1757}, tous les dictionnaires contiennent des noms désignant des types et des parties d’édifices. En raison de l’importance accordée aux technolectes ordinaires, les noms décrits par \citet{Felibien1676} désignent surtout des matériaux, des outils, des objets, comme le feront aussi \citet{Marsy1746} et \citealt{Pernety1757}, bien qu’en moindre mesure. \citet{Lacombe1752} et \citet{Boutard1826} sont davantage tournés vers le vocabulaire conceptuel de la peinture. Puisque \citet{Lacombe1752} s’intéresse aussi à la poésie et à la musique, bon nombre des noms répertoriés appartiennent aux catégories sémantiques des genres poétiques et musicaux et des instruments de musique.

\subsection{Les types de définitions}

Venons-en maintenant aux types de définitions que l’on peut observer dans ces dictionnaires à partir de l’ouvrage fondateur de \citet{Felibien1676}. Celui-ci se sert de plusieurs types de définitions, à commencer par les définitions logiques, d’origine aristotélicienne et très répandues dans la pratique lexicographique. Appelées aussi «~définitions en compréhension~», elles visent une catégorisation logique du monde \citep{Pruvost2006}. Il s’agit de procédés définitoires directs qui, à un premier niveau de classification, mettent en relation un hyponyme avec son hyperonyme par le biais de définisseurs génériques. À un deuxième niveau, les définisseurs spécifiques fournissent plus de détails sur l’objet décrit – notamment, sa fonction ou sa finalité. Suivant la quantité d’informations fournies, les définitions logiques peuvent être qualifiées de hypospécifiques, suffisantes ou hyperspécifiques. Au vu de la complémentarité entre le dictionnaire et le traité \emph{Des Principes}, la taille des définitions logiques fournies par Félibien est variable, pouvant aller de définitions très réduites pour les concepts déjà abordés dans le traité à de véritables développements encyclopédiques pour d’autres unités terminologiques.

À côté des définitions logiques de taille variable, \citet{Felibien1676} propose aussi des définitions plus courtes, comme les définitions synonymiques, typiques des dictionnaires de poche et consistant à définir un mot par le biais d’un synonyme\footnote{Les définitions synonymiques peuvent parfois indiquer des régionalismes.}, et des définitions par rattachement morphologique, employées surtout pour expliquer le sens de verbes spécialisés, comme par exemple

\begin{itemize}
    \item CALER, Mettre une cale.
    
    \item CONTRESCARPER, Faire une contrescarpe. 
\end{itemize}

Cependant, parmi les définitions utilisées par l’historiographe celles qui ont particulièrement retenu notre attention sont les définitions finalisées. Celles-ci sont des énoncés définitoires ayant la forme d’une phrase relative du type \emph{N qui} ou \emph{N que… / N dont} et qui sont employés pour exprimer les finalités sociales ou instrumentales d’un objet par le biais d’un complément circonstanciel. Relevées par  \citet{CollinotMazieret1990} dans le \emph{Dictionnaire Universel de Furetière} \citep{Furetière1690} et dans la première édition du \emph{Dictionnaire de l’Académie française} \citep{AcadFr1694}, les définitions finalisées semblent caractériser la lexicographie monolingue au XVIIe siècle. Les schémas définitoires de ce type employés par Félibien sont les suivants~:

\begin{enumerate}
    \item N qui sert à Nhum\footnote{Nhum: nom humain. Dans le cas de ces types de définitions, les Nhum désignent surtout des noms de métiers (ex. les serruriers, les charpentiers, etc.).} pour / à
    \item N servant à
    \item N qui est propre à / pour
    \item N dont on se sert pour / à 
    \item N dont les Nhum se servent pour / à. 
\end{enumerate}

Les prépositions \emph{à} et \emph{pour} peuvent être suivies par un verbe à l’infinitif ou par un complément d’objet, comme dans

\begin{itemize}
    \item CABLES, ou Chables, ce sont de grosses cordes servant à monter des fardeaux. 
    \item COYERS, ce sont des pieces de bois servant à la couverture d’un bastiment. 
    \item BEC D’ASNE croche, dont les Serruriers se servent pour ferrer les fiches dans le bois.
\end{itemize}

Comme nous avons déjà eu l’occasion de le remarquer ailleurs, 

\begin{quote}
    […] La présence des définitions finalisées dans le dictionnaire de Félibien est un indice incontestable du ferment du chantier lexicographique monolingue dans la seconde moitié du XVIIe siècle en France et de la circulation des savoirs entre les différentes académies royales. \citep[162]{Cetro2022}.
\end{quote}

Un dernier type de définitions relevées dans \citet{Felibien1676}, bien que rarement employées, sont les définitions en extension \citep{Bejoint1993}. Dans ces rares cas, une classification de l’objet décrit est fournie à la place de sa définition. D’habitude, il s’agit d’objets communs\footnote{C’est le cas par exemple des entrées \emph{balance}, \emph{gomme} et \emph{porte}.}, d’un usage très répandu, que le lecteur est censé connaitre~: Félibien exploite donc une connivence avec celui-ci pour donner des détails sur ces objets sans passer par une définition. Ce procédé définitoire est parfois employé aussi par \citealt{Marsy1746}\footnote{Voir par exemple l’article \emph{colle}.}  ou par \citealt{Pernety1757}. Pour revenir aux autres types de définitions passés en revue plus haut, tous les dictionnaires du corpus fournissent des définitions en compréhension, qui sont souvent hyperspécifiques, conformément à leur vocation encyclopédique. Les trois dictionnaires du XVIIIe siècle enregistrent aussi des définitions synonymiques aussi bien pour des raisons d’espace que pour signaler des variantes régionales\footnote{Voir pour cela l’article \emph{cocagne} dans \citet[131]{Marsy1746}~: Cocagne, c’est le Pastel du Languedoc. Voyez Pastel.}, tandis que \citet{Boutard1826} s’en sert très rarement. Sur l’échantillon analysé, ce procédé définitoire est utilisé pour indiquer un nécrologisme~:

\begin{itemize}
    \item COMMISSURE, s.f. Archit. Quelques vieux auteurs, comme Philibert de Lorme et le père Deran, appellent ainsi les joints de pierre. \citep[167]{Boutard1826}.
\end{itemize}

Cet exemple montre l’emploi d’un énoncé métadiscursif qui sert d’appui à la définition. Ce procédé, que \citealt{Boutard1826} utilise occasionnellement puisque ses définitions se basent la plupart des fois sur un hyperonyme, est en revanche très fréquent chez ses prédécesseurs, à partir de \citet{Felibien1676}, qui antépose à ses définitions un présentatif ou un énoncé métadiscursif formé sur un verbe du dire – \emph{appeler}, \emph{nommer}, \emph{dire}, etc. Dans le sillage de Félibien, \citet{Marsy1746} et \citet{Lacombe1752} exploitent souvent ces procédés, tandis que \citet{Pernety1757} élimine les présentatifs de ses énoncés définitoires, se limitant parfois à employer un énoncé métadiscursif.

\subsection{Les marques de domaine}
\largerpage
\citet{Furetière1690}, qui enregistre de nombreux termes spécialisés, exprime les marques de domaine de deux façons. Une première, au début de l’article, qui est \emph{terme de} suivie du domaine dans lequel le terme est employé (ex. \emph{terme de géographie})\footnote{Ce même procédé est employé dans la première édition du \emph{Dictionnaire de l’Académie française}.} et une deuxième qui passe par l’emploi d’un énoncé métadiscursif (ex. \emph{Les Medecins appellent isthme, cette partie estroite de la gorge qui est située entre les deux amygdales}). Il en va de même pour \citet{Felibien1676}, qui se sert de trois schémas récurrents pour indiquer l’appartenance d’un terme à un sous-domaine artistique. Leur but est donc de souligner la dimension fonctionnelle des termes. Il s’agit des schémas suivants~:

\begin{itemize}
    \item \emph{En terme de} + nom de domaine~: ex. en terme de fortification, de peinture, etc.~; 
    \item Un énoncé métadiscursif avec le pronom impersonnel \emph{on} suivi d’un verbe du dire, de la préposition \emph{en} et du nom de domaine~: ex. L’on dit en peinture […]~; 
    \item Un énoncé métadiscursif ayant pour sujet le nom désignant une catégorie professionnelle, suivi d’un verbe du dire et d’un complément d’objet direct~: ex. Les Joüaillers et les Lapidaires nomment Camayeus les Onices, Sardoines et autres pierres taillées de relief.
\end{itemize}

Si l’emploi des marques de domaine n’est pas systématique, il est néanmoins très fréquent chez Félibien. Il n’en est pas ainsi chez \citet{Marsy1746}, qui n’y recourt que rarement. Sur le corpus analysé, les quelques cas relevés font usage du schéma félibien \emph{en terme de} suivi d’un nom de domaine (ex. en terme d’architecture) ou simplement de la préposition \emph{en} suivie du nom de domaine (ex. en architecture) ou encore d’un énoncé métadiscursif calqué sur le troisième schéma employé déjà par Furetière et Félibien (ex. les peintres appellent X…). De son côté, \citet{Lacombe1752}, chez qui l’emploi des marques d’usage reste assez limité, privilégie la marque \emph{terme de} suivi du nom de domaine à l’emploi d’un énoncé métadiscursif impersonnel. Quant à  \citet{Pernety1757}, il reprend les deux premiers schémas employés par Félibien, auxquels il ajoute une autre formulation impersonnelle (ex. se dit en terme de peinture). Si un même terme est employé dans deux domaines différents, les noms de ceux-ci sont donnés entre parenthèses (ex. peinture et gravure). Avec \citealt{Boutard1826}, l’emploi des marques de domaines devient systématique et homogène. Les énoncés métadiscursifs disparaissent au profit d’abréviations, et cette convention est désormais un acquis de la lexicographie.

\subsection{Comparaison d’un même article}

Dans cette dernière section, nous nous appuierons sur le traitement d’un même mot au sein des cinq dictionnaires pour procéder à une comparaison entre leurs microstructures. Pour cela, nous avons reporté dans le tableau suivant (voir \REF{cetro:frequencies}) les articles relatifs au mot \emph{chargé} – \emph{charge} dans \citet{Lacombe1752} et \citet{Boutard1826}~:

\begin{exe}
\ex L’article chargé/charge dans les dictionnaires du corpus.
\label{cetro:frequencies}
\begin{xlist}
\ex \citet{Felibien1676}\\
CHARGÉ. Les Peintres appellent un portrait chargé, lorsqu’on représente un visage avec des traits marquez avec excès, et de telle manière qu’avec trois ou quatre coups de crayon ou autrement on connait une personne, quoy que ce ne soit pas un véritable portrait, mais plutôt des défauts marquez. Aussi quand une Figure est trop marquée on dit qu’elle est chargée.

\ex \citet{Marsy1746}\\
CHARGE, CHARGER. On appelle proprement \emph{charges} ou \emph{tableaux chargés}, des représentations où l’on exagère les choses en bien ou en mal, mais plutôt ordinairement en mal.

Un Peintre satyrique en trois ou quatre coups de pinceau, fait un portrait ridicule, mais fort ressemblant, quoiqu’en laid, en exagérant la difformité de la personne qu’il représente. 
On appelle encore \emph{charges} certains caprices \& certaines figures grotesques~: par exemple, des animaux avec une figure humaine, des hommes avec des pieds d’animaux, des femmes sous la figure d’un pot ou de quelqu’autre vase. C’est une espèce de genre burlesque que le mauvais goût, qui corrompt tous les Arts, a introduit dans la Peinture […].
 
\ex \citet{Lacombe1752}\\
CHARGE. On donne en général ce nom dans le Dessein, à tout ce qui est outré et hors de vraisemblance. Mais on appelle particuliérement ainsi, en terme de Perspective ou de Dessein, une exagération burlesque des parties les plus marquées du visage, de façon que la ressemblance soit conservée, \& que l’on puisse reconnoître la personne dont on fait la \emph{Charge. Voyez Carricature.} 

\ex \citet{Pernety1757}\\
CHARGÉ, se dit d'un portrait, dont les traits sont marqués avec excès ou exagérés. Ce terme est employé pour définir un contour qui mérite le blâme \& la critique, pour être dur et prononcé. Un portrait chargé, se fait en trois ou quatre coups de crayon, selon M. Félibien~; \& ils suffisent pour présenter la ressemblance d'une personne, quoique le portrait ne soit pas achevé. Les mauvais Peintres chargent les portraits par ignorance de leur art, \& les bons Artistes par la science qu'ils en ont, aidée d'une main hardie, ferme \& décidée.

\ex \citet{Boutard1826}\\
CHARGE, s.f. \emph{Peint}. Exagération dans les formes, dans l’inflexion des lignes, dans les traits du visage, les attitudes, le mouvement d’une figure. On appelle aussi charge l’ouvrage exécuté de cette manière. La charge diffère de la caricature avec laquelle on la confond quelquefois, en ce qu’elle n’admet point d’êtres fantasques, mais qu’elle a toujours son type dans la nature animée. La charge n’est pas une fiction, mais une exagération de la vérité. […] La charge est aussi l’artifice ou plutôt la ressource des peintres du portrait de second ordre […].
\end{xlist}
\end{exe}

Comme on peut l’observer, jusqu’à \citet{Pernety1757} la définition s’ouvre par un énoncé métadiscursif, tandis qu’avec \citet{Boutard1826} l’objet est immédiatement qualifié à l’aide du définisseur \emph{exagération}. Par rapport à ses prédécesseurs, ce dernier auteur affiche une tendance à la nominalisation, qui lui permet de condenser les informations et qui caractérisera la lexicographie contemporaine. Les informations données sur l’objet décrit diffèrent quelque peu selon le dictionnaire. Cependant, il est possible de relever des motifs récurrents et la reprise ou la reformulation de certaines informations du dictionnaire fondateur, celui de Félibien.

\section{Conclusion}

La terminologie artistique – et plus en général le monde des beaux-arts – a suscité un véritable engouement au sein de la société française de l’Ancien Régime, comme en témoignent les nombreuses publications consacrées à ce sujet, dont les dictionnaires. Ce genre textuel va particulièrement à l’encontre du «~goût du siècle~» pendant les Lumières et se poursuit même après la Révolution, se perfectionnant et anticipant les tendances de la lexicographie contemporaine. 

Rédigés par des amateurs, des connaisseurs ou des spécialistes et destinés à des publics cibles diversifiés, les ouvrages pris en examen dans ce chapitre mettent en avant les difficultés dans le découpage et la description d’un si vaste secteur de connaissance(s). Malgré leurs limites, ils constituent un témoignage inestimable des questionnements qui sous-tendent à la rédaction d’un dictionnaire spécialisé et des réponses-solutions apportées par les auteurs.

{\sloppy\printbibliography[heading=subbibliography,notkeyword=this]}
\end{otherlanguage}
\end{document}
