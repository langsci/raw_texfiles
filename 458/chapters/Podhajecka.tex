\documentclass[output=paper,colorlinks,citecolor=brown,arabicfont,chinesefont]{langscibook}
\ChapterDOI{10.5281/zenodo.15394499}
\author{Mirosława Podhajecka\affiliation{University of Opole}}
\title[The beginnings of English-Polish lexicography]
      {The bilingual dictionary as a mediator between West and East: The beginnings of English-Polish lexicography}

\abstract{This article looks at the beginnings of the English-Polish lexicographical tradition by focusing on the earliest dictionary for both languages compiled in 1799 and extant in manuscript form. It is something of a curiosity, inasmuch as English enjoyed very little recognition among speakers of Polish at the end of the eighteenth century. French and German, along with classical languages, would at that time have been the only foreign languages taught in the school system. We may assume that the dictionary was made by a bilingual author, but, since borrowing from predecessors was a widespread phenomenon in lexicography, other options also need to be considered. Indeed, a comparative analysis of the manuscript indicates that the compiler took into account an edition of Abel Boyer’s English-French dictionary published originally in 1699 \citep{Boyer1780}; Nathan Bailey’s 1721 monolingual dictionary expanded into an English-French-Latin-German work in 1771; and two English-French grammars. The area of overlap is traced through the spelling of the headwords, the choice of phraseology and illustrative citations, and French and German items added to the entry when Polish equivalents were apparently difficult to provide. The findings are illustrated on the basis of examples.}

\IfFileExists{../localcommands.tex}{
  \addbibresource{../localbibliography.bib}
  \usepackage{langsci-optional}
\usepackage{langsci-gb4e}
\usepackage{langsci-lgr}

\usepackage{listings}
\lstset{basicstyle=\ttfamily,tabsize=2,breaklines=true}

%added by author
% \usepackage{tipa}
\usepackage{multirow}
\graphicspath{{figures/}}
\usepackage{langsci-branding}

  
\newcommand{\sent}{\enumsentence}
\newcommand{\sents}{\eenumsentence}
\let\citeasnoun\citet

\renewcommand{\lsCoverTitleFont}[1]{\sffamily\addfontfeatures{Scale=MatchUppercase}\fontsize{44pt}{16mm}\selectfont #1}
  
  %% hyphenation points for line breaks
%% Normally, automatic hyphenation in LaTeX is very good
%% If a word is mis-hyphenated, add it to this file
%%
%% add information to TeX file before \begin{document} with:
%% %% hyphenation points for line breaks
%% Normally, automatic hyphenation in LaTeX is very good
%% If a word is mis-hyphenated, add it to this file
%%
%% add information to TeX file before \begin{document} with:
%% %% hyphenation points for line breaks
%% Normally, automatic hyphenation in LaTeX is very good
%% If a word is mis-hyphenated, add it to this file
%%
%% add information to TeX file before \begin{document} with:
%% \include{localhyphenation}
\hyphenation{
affri-ca-te
affri-ca-tes
an-no-tated
com-ple-ments
com-po-si-tio-na-li-ty
non-com-po-si-tio-na-li-ty
Gon-zá-lez
out-side
Ri-chárd
se-man-tics
STREU-SLE
Tie-de-mann
}
\hyphenation{
affri-ca-te
affri-ca-tes
an-no-tated
com-ple-ments
com-po-si-tio-na-li-ty
non-com-po-si-tio-na-li-ty
Gon-zá-lez
out-side
Ri-chárd
se-man-tics
STREU-SLE
Tie-de-mann
}
\hyphenation{
affri-ca-te
affri-ca-tes
an-no-tated
com-ple-ments
com-po-si-tio-na-li-ty
non-com-po-si-tio-na-li-ty
Gon-zá-lez
out-side
Ri-chárd
se-man-tics
STREU-SLE
Tie-de-mann
}
  \togglepaper[15]%%chapternumber
}{}

\begin{document}
\maketitle

\section{Introduction}

This paper is based on the premise that there is no better example of a linguistic mediator between West and East than the bilingual dictionary. The reason is simple: by describing the vocabulary of two different languages, the bilingual dictionary juxtaposes two different worlds \citep[19]{Podhajecka_miroslawa2016a}. The divide between West and East has been pervasive in a variety of discourses. While it may be drawn roughly along geographical lines, it has admittedly been more of an ideological dichotomy between the West, with its presumed superiority, and the East, with its presumed inferiority (see, e.g., \citealt{SpencerWollman2002}:  98). This ideological facet has been ignored here, allowing me to focus on linguistic and cultural perspectives.

In what follows, I look at the beginnings of the English-Polish lexicographical tradition by focusing on the earliest dictionary extant in manuscript form, \emph{Do Słownika Angielsko-Polskiego Dodatek ktorego układanie przedsięwzięte zostało  Roku 1799 Dnia 17 Czerwca} [\emph{An addendum to an English-Polish dictionary whose compilation was undertaken on 17 June 1799}] \citep{[SAP]}.\footnote{The manuscript was discovered by \citet[134--136]{Siekierska_krystyna1985} in the Vernadsky National Library of Ukraine in Kiev. Two other unpublished dictionaries were compiled in the mid-nineteenth century: Paweł Sobolewski’s in 1840 \citep{Podhajecka_miroslawa2016b} and Michał Wiszniewski’s around 1854 \citep{Jajdelski2000}.}  From the European perspective, English represents a Western and Polish an Eastern language. Printed reference works pairing both tongues appeared a few decades later: the first monoscopal dictionary, Krystyn Lach-Szyrma’s \textit{{Słownik angielsko-polski~...}} [An English-Polish dictionary~...], was published in 1828 \citep{Lach-szyrma_krystyn1828} and the first biscopal one, Erazm Rykaczewski’s \emph{A complete dictionary English and Polish and Polish and English~...} \citep{Rykaczewski1849}, in the mid-nineteenth century.

\section{The historical context}

Let me begin with brief historical explanations to put the analysis into the right context. Between the sixteenth and eighteenth centuries, Polish was the main language of the Great Polish-Lithuanian Commonwealth,\footnote{More exactly, Polish was the superstratum, whilst Ruthenian and Lithuanian were the substrata \citep[280--281]{Walczak_bogdan2018}.} one of Europe’s largest and most populous countries. The turbulent period that followed, however, resulted in its loss of independence. Annexed by Russia, Prussia, and Austria\hyp Hungary in the wake of three subsequent partitions, Poland was erased from the world’s map for 123 years and, as a consequence, Polish became a language without a nation state. English, by contrast, was the official language of the vast British Empire, whose effective administration, robust parliamentary system, and economic power, among other things, were held in high esteem. It is, therefore, unsurprising that there were attempts to achieve close political cooperation with Britain, a relationship that would, in the fullness of time, lead to the restoration of Poland, but the plans eventually fell through.\footnote{The aristocratic Czartoryski family attempted to introduce a set of reforms modelled on Britain and advocated a Polish-British alliance. Prince Adam Kazimierz Czartoryski’s British experiences earned him the nickname “the perfect Englishman”, but it was his son Adam Jerzy who became an ardent anglophile. He studied in London, travelled throughout Britain, and was fluent in English (\citealt[18--19, 28]{Liponski_wojciech1978}; cf. \citealt{Konopczynski_wladyslaw1947}).}

Both before and after the partitions, foreign languages were regarded as an integral part of the education of young Poles, so French, German, and, to a lesser extent, Italian were taught alongside classical languages. French, in particular, became the common language of the Polish nobility; what France represented was perceived as the height of civilisation. As \citet[31]{Brandes_george1903} observed,

\begin{quote}
    French is the auxiliary language of the Poles, the language which among the higher classes all know perfectly (…) the language which is spoken as fluently as the mother-tongue and even better than the Russians speak it. In aristocratic circles Poles frequently converse with one another in French, a state of things which from the beginning of the century was promoted not only by the continual intellectual intercourse with France, and the emigration thereto.
\end{quote}

Interest in English, formerly treated as a “barbarous tongue lacking in refinement” \citep[121]{Ciesla_michal1974}, emerged falteringly at the end of the eighteenth century. It may be worthwhile noting that, in the Age of Reason, English was not only the language of Shakespeare and Milton, but also that of British inventors and men of science. As Julian \citet{Antonowicz1788}, the author of the first grammar of English for Polish students, so eloquently expressed it,

\begin{quote}
    The glory of being the native Country of Creative Genius’s is granted to Great Britain by all Europe. It was this happy Kingdom that has brought forth the famous Bacons and Newtons, to whom, it seems, nature has opened its closet~(...). Then it is pity, that this Language is not still so universally known, as it deserves  to be, above all, in the Kingdom to which the blessing of liberty is common with Great Britain.
\end{quote}

Native speakers of English in Poland, a country soon to fall under foreign rule, were a rarity. John Lind (1737--1781), the governor of the Cadet School in Warsaw established by King Stanisław August, and Joseph Saunders (1778‒1836), an engraver affiliated with the University of Wilno, are among the few acknowledged in the literature. No wonder the early history of English-Polish language contact relied, out of necessity, on autodidacts such as Antonowicz (c. 1750-−1824), a Basilian monk and teacher in Włodzimierz (in the Wołyń province, now in Ukraine).\footnote {\citet[175]{Butterwick_richard1998} states that, at the request of King Stanisław August, Antonowicz offered his English grammar to Charles Whitworth, the British Envoy Extraordinary and Minister Plenipotentiary at the Polish court.} Benjamin Haustein (1778−1836), a professor of German and English at the University of Wilno (now in Lithuania),\footnote{The Romantic poet Juliusz Słowacki, a graduate of the university, noted that Haustein’s English was so awkward that “even native speakers would be unable to recognize what language he spoke” [M.P.] (\citealt[96]{Piechota_marek2016}).} as well as Gaspar Friese and a Hennig, who taught English to senior students of Toruń’s academic gymnasium \citep[408]{Lukaszewicz_jozef1849}, might have been eligible had they not been Germans.

The schools in Warsaw, Włodzimierz, and Toruń were not the only ones to promote the teaching of English. The Wołyń Gymnasium in Krzemieniec, initiated and headed by Tadeusz Czacki, was another. Staffed with distinguished teachers, it was the best Polish educational facility with ambitious aims: to educate students in line with the highest teaching standards in order to give them “a solid knowledge and comprehension of problems as well as [to] stimulate independent thinking” \citep[3]{Buczek2016a}. Teaching languages was a distinct priority; the school’s curriculum featured Polish and five foreign languages, i.e., French, German, Russian, Latin, and Greek. Lessons in English, recommended to Czacki by Antonowicz, were introduced in 1807 (\cites[125]{Ciesla_michal1974}[154]{Buczek2016b}). Originally an optional subject, English was taught by the Englishmen George Forster and William Rose, both of whom were brought from abroad specifically for that purpose,\footnote{According to \citet[154]{Buczek2016b}, Forster spoke neither French nor Polish. It is thus unclear how he taught English with the so-called Grammar-Translation Method, as has been suggested by \citet[214]{Piotrowski_wojciech2005}.} and Józef Mikulski, a teacher of Latin and Polish \citep[344]{Slowikowski1863}.\footnote{Jack William MacDonald, a Scot \citep[38]{Nicoll_josephine1923}, was a private teacher in Krzemieniec, but there is no evidence that he ever worked at the school.}

There seems to be a link between the Wołyń Gymnasium and the draft dictionary. After the collapse of the November Uprising (1830−1831), an armed rebellion against the Russian Empire, the school was closed and its entire library of approximately 34,000 volumes was transported to Kiev \citep[82]{Danilewicz-zielinska_maria1992}.\footnote{The school’s library included 15,580 volumes from the royal book collection purchased by Czacki in 1803 after the death of King Stanisław August \citep[8--9]{Buczek2016a}.}  \citet[18]{Buczek2016b} affirms that part of the school’s book collection rescued from the wars has now been deposited in the National Library of Ukraine. Since the manuscript comes from the holdings of that library, one might assume that it belonged to the Wołyń Gymnasium, having either been compiled in Krzemieniec or, perhaps more likely, donated to this centre of Polishness as a cultural artefact.

\section{The manuscript dictionary}

\subsection{The compiler}

The identity of the compiler remains undisclosed. The same handwriting used throughout, clearly not Antonowicz’s, suggests a single author, no doubt a man. This hypothesis stems from two factors: firstly, schooling was restricted to boys and, secondly, lexicography was a predominantly male business at that time.\footnote{Hester Lynch Piozzi, the author of \emph{British Synonymy~...} (1794) and “hostess, muse, amanuensis, intimate friend” of Samuel Johnson \citep[69]{Berglund_lisa2010}, was one of very few recognised female lexicographers in the eighteenth century.} The problem is that none of the Poles familiar with English either by dint of travel to Britain, including Krystyn Lach-Szyrma (1790‒1866), Karol Sienkiewicz (1793‒1860), and Michał Wiszniewski (1794−1865), or extensive self-study, including Hipolit Błotnicki (1792‒1886), Józef Korzeniowski (1797‒1863), and Konstanty Piotrowski (1790‒1863), was born sufficiently early to undertake the project.\footnote{The latter category included authors who translated English literature (e.g., Shakespeare, Milton, Byron, Goldsmith, and Stern) into Polish.} Jacek Idzi Przybylski (1756−1819), a professor, writer, and translator, was an exception, but his linguistic knowledge was far too sophisticated to be able to consider him the author. The manuscript thus appears to have been a spontaneous act of creativity on the part of an enthusiast of English who never achieved distinction.\footnote{At the same time, increasing numbers of Poles of high social rank such as Michał Mniszech (1748−1806) and Franciszek Bohusz (1746−1820) embarked on journeys abroad that included visits to Britain. While their peregrinations had intellectual purposes, other Poles such as Tadeusz Burzyński (1730−1773) and Franciszek Bukaty (1747−1797) remained in Britain for political and diplomatic reasons (see, e.g., \citealt{Libiszowska_zofia1972}).}

\subsection{The form}

The manuscript consists of white sheets of paper ruled by hand and bound into a book with the word \emph{Słownik} [\emph{Dictionary}] printed on the spine.\footnote{The binding is of half-leather, which means that the sides are covered with board paper, but the spine and corners are in leather.} Of the book’s 404 pages, around 280 are filled, rather haphazardly, with text. Although the order is alphabetical, ascertaining the chronological sequence in which the dictionary was compiled is a sheer impossibility. There is every indication that, after writing a batch of entries, the compiler would return to his sources only to find more headwords that deserved inclusion, and he would add them between other letters of the alphabet. Headwords from B, for instance, are given at the beginning, in the middle, and even at the very end of the book. Here is a small sample of how the headwords were placed in the subsequent letters of the alphabet:

\begin{itemize}
    \item[\textbf{A}]  4, 5‒7, 8‒29, 32‒43, 288, 317
    \item[\textbf{B}]  4, 7, 43‒61, 220‒224, 289, 318‒322, 397‒404
    \item[\textbf{C}]  7, 8‒12, 15‒16, 62‒66, 69‒89, 290, 323, 404
    \item[\textbf{D}]  89‒101, 291, 326‒328
    \item[\textbf{E}]  101‒115, 117‒119, 292, 329
\end{itemize}

As the title evinces, the material was an addition to another volume (see \figref{pod:fig1}). If we look at the title page more carefully, however, we will discern that the draft dictionary was originally titled \emph{Słownik Angielsko-Polski~...} [\emph{An English-Polish dictionary~...}], but the title was later modified by using a pen with black ink. The same ink was applied for the ornament attached below, which brings to mind the engravings on the title pages of Renaissance dictionaries such as Calepino’s.\footnote{With over 200 successive editions printed across Europe, Ambrogio Calepino’s dictionary was one of the best-selling endeavours in the history of lexicography.} The manuscript has neither the front matter, explaining the purposes of the project and its envisaged readership, nor the back matter. 

\begin{figure}
\includegraphics[width=\textwidth]{images/Podhajecka_figure1.jpg}
\caption{The title page of the manuscript dictionary}
\label{pod:fig1}
\end{figure}

Each page is divided into two columns with extensive margins left for the notes. Sometimes they include lexicographical information such as a missing headword (e.g., \emph{Laden} ‘ołowiany’; \emph{Swan} ‘łabędź’; \emph{Ungodliness} ‘niezbożność’), at other times providing what seem to be references to the sources consulted (e.g., I, 120, 134). From page 317, the prospective headwords were to be written in three columns. 

The last pages (381−404) are titled “Przydatek do dykcionarza” [An addendum to the dictionary]. This part, including single words, phrases, and citations, is arranged in no alphabetical order. Whereas the first 460 lines (entries?) are numbered, the other 64 numbers are provided in a way that cannot be explained logically. Generally speaking, a great deal of internal diversity, partly illegible handwriting, and numerous inconsistencies make the draft dictionary a challenge to research. 

The material quoted below has been kept in the original orthography. If French or Polish equivalents were recorded without diacritics, for example, this is how they appear here. I also stick to the compiler’s inconsistent spelling with upper and lower case.

\subsection{The contents}

What did it take to compile a bilingual dictionary of English and Polish at the end of the eighteenth century? One might venture to claim that fluency in both languages was a prerequisite, but it was not an overriding requirement in the past \citep[554]{Podhajecka_miroslawa2016a}. Many English-Polish dictionaries, including Lach-Szyrma’s and Rykaczewski’s, were made by authors whose knowledge of English was imperfect, but who spoke other foreign languages, mainly French and German. It is little wonder, then, that the compiler treated both languages as a conduit between English and Polish.

My analysis was carried out on Google Books, a multilingual diachronic archive.\footnote{In doing so, I adjusted Google Books functionalities to my research needs. I restricted the time frame to specific periods (for example, 1700−1800) and searched for whole constructions (e.g., “she is a little out of order"), also with wildcards (e.g., “she * a little out of order"), or various Boolean expressions (e.g., “she" “out of order"). Google Books is not a diachronic corpus (see, e.g., \citealt{Podhajecka_miroslawa2011}), but, with 40 million scanned books, it is currently the largest resource available for historical scrutiny.} The research methodology was twofold: firstly, I searched for single words, sequences of words, and longer stretches of text recorded in the manuscript dictionary in order to identify the compiler’s sources and, secondly, the rest of the searches were conducted specifically in the sources. The second part of the study was exceedingly labour-intensive.

The results suggest that at least two reference works were drawn on: an edition of Abel Boyer’s \emph{The royal dictionary~...} (1792?), considered  “one of the most important and innovative works of all time” \citep[479--480]{CormierFernandez2005};\footnote{The original edition of Boyer’s dictionary was published in 1699.} and Nathan Bailey’s great folio dictionary, originally monolingual, expanded into an English-German-French-Latin dictionary by Anton E. Klausing and retitled \emph{A compleat English dictionary: oder vollständiges englisch-deutsches Wörterbuch~...} \citep{Bailey1771}.\footnote{The dictionary must have been convenient to the compiler, who was undoubtedly familiar with German, French, and Latin, though possibly in different degrees.} Boyer’s \emph{The complete French master~...} \citep{Boyer1762}, a grammar-cum-phrasebook-cum-dictionary,\footnote{It is worthy of mention that \emph{The complete French master~...} was one of Antonowicz’s sources.} and Guy Miege and Boyer’s \emph{Grammaire angloise-francoise~...} \citep{MiegeBoyer1763}, also came into play. There is a clear rationale behind such a choice of materials: two different dictionaries allowed the compiler to decide, firstly, what headwords should be selected and, secondly, how to explain the source language (SL) meanings in the target language (TL).\footnote{In the light of the above, the SL refers to English, French, and German, so it would be more accurate to speak of three SLs.} The grammars were, among other things, a handy source of citations extracted from the sections titled “Familiar phrases”.\footnote{\citet{Siekierska_krystyna1985} sensibly, albeit wrongly, assumed that the compiler had drawn on Kunstmann’s \emph{Kurzes Wörterbuch in deutscher, französischer, italienischer, englischer und polnischer Sprache~...} (1794).}

A few words should be said about the compiler’s working practice. The headwords in upper case are recorded alphabetically. Every now and then, we encounter ostensible errors, but they reflect the type of arrangement employed in the sources. \emph{Alehouse}, for instance, is followed by \emph{Alecost}, but this is because \emph{Alehouse} was clustered in the entry for \emph{Ale} in both dictionary sources, whilst \emph{Alecost} was accorded main headword status. Under the letter C, by contrast, \emph{Camber}, \emph{Càmbrick}, and \emph{Càmber} were copied verbatim from Bailey and Klausing’s work together with their German equivalents. 

Different parts of speech are treated as different headwords. A proportion of nouns are followed by a bracketed article and verbs by the bracketed particle \emph{to}, e.g., \emph{Artichoke} [: the] ‘karczoch’; \emph{Pour} [: to:] ‘nalewać, wylewać’; and \emph{Shepherdess} [a] ‘owczarka’. Irregular word forms are also recorded (e.g., \emph{Begun} ‘part. od słowa \emph{to Begin}’; \emph{Dead} ‘imiesłow od tego słowa \emph{Die}’; \emph{Kept} ‘praet y part: od sł: \emph{keep}’; \emph{Lighted} ‘part y praet: od sł: \emph{light}’; \emph{Swolln} ‘Imiesł. od \emph{Swell}’; and \emph{Wrung} ‘praet y part: od sł: \emph{wring} wykręcać’). Apart from that, grammatical information is practically non-existent. 

In the first letters of the alphabet, the compiler provides variant spellings (e.g., \emph{Antelope} ob. \emph{Antilope}; \emph{Arsevercy} or \emph{Arsiversy}; \emph{Bandalier} ob. \emph{Bandaleer}; and \emph{Esq} zamiast \emph{Esquire} ‘szlachcic’), all of which come from Boyer’s or Bailey and Klausing’s works.\footnote{One also encounters dead references. In the case of \emph{Basin}, the user was told to look up \emph{Bason}, but it is not the word list.} In the remaining part of the manuscript, when it dawned on him how time-consuming the compilation would be, fewer variant forms were incorporated.\footnote{See, however, abbreviated forms such as \emph{Dr.} ‘zamiast \emph{Doctor} nauczyciel’; \emph{Excy} ‘zamiast \emph{Excellency}’; \emph{Knt} ‘zamiast \emph{knight} kawaler’; \emph{Lap} ‘zamiast \emph{Ladiship}’; \emph{O’er} ‘\emph{over} nad’; \emph{T’other} ‘zamiast \emph{the other}’; and \emph{Won’t} ‘zamiast \emph{will not}’. Where they come from remains undetermined due to formal inconsistencies.} 

For the word list, the author drew from two bulky dictionaries, from which he copied some headwords and ignored others; many of his choices seem debatable. Under A, for instance, he admitted \emph{Albacor} ‘see Dolphin’,\footnote{OED3 \citep{OED3} defines \emph{albacore} as ‘any of several small and medium-sized tunas~...’. }\emph{Alburin}, \emph{Alburn} ‘dark colour’, in its old-fashioned spellings (cf. OED3's \emph{auburn}); and \emph{Alet} ‘an Indian partridge’ [M.P.],\footnote{In his \emph{English dictionary …} (1676), the British lexicographer Elisha \citet{Coles_elisha1677} defines \emph{Alet} as ‘The true Peru-falcon’. This word has not been incorporated in OED3 \citep{OED3}.} at the same time leaving out more common words such as \emph{Alarming} ‘causing terror’, \emph{Alchymy} ‘a secret part of chymistry’, and \emph{Alcove} ‘a recess in a room’.\footnote{The headwords are defined with English glosses added to Boyer’s dictionary, except for \emph{Alet}, which stands for Bailey and Klausing’s German equivalent in my translation.} His motives may only be speculated upon, but the manuscript dictionary was probably intended as an aid in reading English literature, so, from this angle, the status of common and rare words was comparable. Here are a handful of other peripheral items or senses: \emph{Agramatist} ‘nieumiejętny, nieuczony’; \emph{Banstickle} ‘nazwisko pewney małey ryby’; \emph{Bargh-master} ‘dozorca kopalni’; \emph{Belagged} ‘zostawiony w tyle, zmoczony’; \emph{Fillamot} ‘feuville morte, naz.[wa] koloru’; \emph{Loll} ‘zgarbić się’; \emph{Scullion} ‘kuchta’; \emph{Squint-ey’d} ‘kosooki’; \emph{Tight} ‘tęgi’; and \emph{Unchew’d} ‘niepożuty’.

The manuscript includes some proper names, primarily place names (e.g., \emph{Constantinopol} ‘Konstantynopol’, \emph{Edinburgh} ‘Edinburg’, \emph{Muscow} ‘Moskwa’, \emph{Prague} ‘Praga’, \emph{Stock-holm} ‘Sztokholm’, \emph{Warsaw} ‘Warszawa’); names of regions (e.g., \emph{Brandenburg} ‘Brandenburgia’, \emph{Piedmont} ‘Pedemontana’, \emph{Tuscany} ‘Toskania’); and names of countries (e.g., \emph{Denmark} ‘Dania’, \emph{Hungary} ‘Wegry’, \emph{Norway} ‘Norwegia’, \emph{Persia} ‘Persya’, \emph{Scotland} ‘Szkocya’, \emph{Turkey} ‘Turcya’). Proper names were recorded in Bailey and Klausing’s trilingual “Catalogue of the most usual proper, Christian and nicknames of men and women~...”, located in the back matter, as well as in both grammars.

A number of headwords in the dictionary such as \emph{Man} ‘męszczyzna’, \emph{Manhood} ‘męski wiek’, and \emph{Mankind} ‘Rodzay ludzki’ are formally and semantically related. Once the compiler established the meaning of the main headword, meanings of derivatives or compounds should be easy to guess or find, but this is not always the case. This refers to, among others, \emph{Despair} ‘rospacz’ vs. \emph{To desper} ‘desesperer’; \emph{Foot} ‘noga, stopa, miara’ vs. \emph{Foot-stool} ‘un marche pied’; \emph{Herb} ‘ziele’ vs. \emph{Herb-woman} ‘herbiere’; \emph{Juice} ‘sok’ vs. \emph{Juicy} ‘succulente’; \emph{Kitchin} ‘kuchnia’ vs. \emph{Kitchin Tackling} ‘batterie de cuisine’; and \emph{Wine} ‘wino’ vs. \emph{Wine-press} ‘un pressoir’. Curiously, despite some variation in form, all the missing Polish equivalents may be found in the then contemporary reference works: ‘przywieść kogo do rospaczy’ (\emph{Desesperer} → \emph{To despair}); ‘podnożek’ (\emph{Marche-pied} → \emph{Foot-stool}); ‘zielenniczka, co rożne ziele przedaie, ziołniczka, zieleniarka’ (\emph{Kräuterfrau} → \emph{Herb-woman});\footnote{None of these have survived in Polish to date, being ousted by ‘zielarka’.} ‘soczysty, soku wiele w sobie maiący’ (\emph{Sucullent} → \emph{Juicy}); ‘statki kuchenne miedziane i mosiężne~...’ (\emph{Batterie} → \emph{Kitchin Tackling}); ‘and ‘prasa winna do tłoczenia wina y tym podobnych rzeczy’ (\emph{Pressoir} → \emph{Wine-press}).\footnote{The equivalents quoted above come from a dictionary by \citet{DanetKola1745} and two different dictionaries by \citet{Trotz_michel_a1764, Trotz_michel_a1772}.} 

\begin{figure}
\includegraphics[width=\textwidth]{images/Podhajecka_Figure2.png}
\caption{Part of a sample page of the manuscript dictionary}
\label{pod:fig2}
\end{figure}

The compiler’s decisions are occasionally baffling. \emph{Translate} ‘przetłumaczyć’ and \emph{Translation} ‘przetłumaczenie’, for instance, are followed by \emph{Translator} ‘un savetier szwiec który naprawia stare obuwie’ [a cobbler who repairs old shoes].\footnote{The same definition is paired with the headword \emph{Cobler}.} The reason behind neglecting the primary sense of the word, ‘one who renders anything into a different language’,\footnote{This English gloss comes from Boyer’s 1792 edition.} alluding to the compiler’s temporary status, is unknown. There are also headwords with no Polish counterparts at all, of which the entry for \emph{Break} is a good example (e.g.,\emph{To break a horse} ‘ein Pferd bändigen’; \emph{To break company} ‘die Gesellschaft verlassen’; \emph{To break silence} ‘anfängen zu reden’; \emph{To break a jest} ‘scherzen’; and \emph{To break out into tears} ‘häufige Thränen vergiessen’).

Clues as to which dictionary was perused should be sought in the selection of the phrases and/or their foreign equivalents. In the letter B, the compiler included \emph{Back of the hand} ‘Dłoń wierzchnia’, \emph{Back Bone} ‘Pacierzowa kość’, \emph{Back of the Chimney} ‘contre-coeur de cheminée, plague de fen’, \emph{Back-stairs} ‘un Page, \emph{ou} un Garçon de la chambre du lit’,\footnote{In another part of the manuscript, \emph{Back-stairs} is paired with Bailey and Klausing’s German equivalent (‘eine heimliche Treppe’).} and \emph{Back Gammon} ‘toutes tables’. As has turned out, only Boyer has all the English phrases and identical French explanations, with the exception of \emph{Back-stairs}, the description of which comes from one of the grammars. Sometimes the wording is harder to trace to a source, which may be exemplified by \emph{Balkers} → \emph{Conders} ‘Ci ktorzy z wysokiego mieysca ukazuią Rybakom, gdzie są śledzie’. In this case, the syntactic patterns of French (‘Ceux qui étant postés sur une hauteur montrent aux bateaux de péchceurs par où les harengs passent’) and German (‘Die von einem hohen Ort den Fischern zeigen, um welche Gegend die Heringe sind’) could be translated into Polish in largely the same ways. 

For most dictionaries, it would be challenging to give a precise count of the headwords, inasmuch as no single method of calculating them has gained universal acceptance among practical lexicographers and metalexicographers (see, e.g., \citealt[109--112]{Landau2001}). It is even more puzzling in the case of a manuscript wordbook like this, where each line could be perceived as one entry or a component of a broader entry. For this reason alone, I made no attempt to count the number of potential headwords and, likewise, decided to eschew any reference to \emph{macrostructure} and \emph{microstructure},\footnote{\citet[xii, 94]{HartmannJames2001} treat \textit{macrostructure} as “the overall access format of a dictionary”, usually alphabetical organisation, whereas \emph{microstructure} refers to the internal design of the entry.} two standard lexicographical terms, because the borderline between these two is blurred (see \figref{pod:fig2}).

\subsection{Seeking equivalence}

Although the nature of equivalence between English and Polish remains unaddressed, the compiler must have taken a somewhat naïve approach to it by assuming that the lexical systems of different languages describe the world in exactly the same ways.\footnote{This, as we know today, is entirely wrong. Different languages describe extralinguistic reality in quite diverse ways, which \citet[295]{Zgusta1971} terms \emph{anisomorphism of languages}. No wonder it has been argued that “perfect interlingual equivalence is an exception rather than the rule” \citep[201]{Adamska-salaciak2020}.} After all, a plethora of dictionaries which paired the vocabulary of two or more languages might attest that translatability is an undisputed fact. In any case, working with French and German was all the easier that German-Polish and French-Polish dictionaries had a long tradition in Poland. \citet[358]{Fraczek_agnieszka1999} lists ten German-Polish or Polish-German dictionaries issued in the eighteenth century, as well as those produced between the sixteenth and seventeenth centuries, whereas \citet{Jakubczyk_marcin2016} analyses five lesser known French-Polish or Polish-French volumes, published alongside Pierre Danet and Dymitr Franciszek Kola’s \emph{Nouveau grand dictionnaire~... francois, latin \& polonois} (1743−45) and other multilingual works of the eighteenth century. 

The manuscript is lexicographically unsophisticated; the English headword is usually paired with one single Polish equivalent (e.g., \emph{Law} ‘Prawo’; \emph{Perjury} ‘krzywoprzysiędztwo’; \emph{Spectacles} ‘okulary’; \emph{Triennial} ‘trzyletni’). Two or more alternatives appear less frequently, as in the case of \emph{Agree} ‘zezwolić, godzić się’; \emph{Barefacedly} ‘iawnie, bez wstydu’; \emph{Dragoon} ‘dragon, żołnierz’; \emph{Endeavour} ‘starać się, usiłować’; \emph{‘Pudding} ‘kiszka, jątrznica’; \emph{Sorrow} ‘boleść, nieukontentowanie, smutek’; \emph{Tutor} ‘Guwerner, Nauczyciel’; and \emph{Verily} ‘w samey rzeczy, prawdziwie’.

A great many English headwords are polysemous, but the compiler rarely differentiates between senses. One suitable example is the entry for \emph{Baggage} ‘Bagaże, podróżne rzeczy, Nierządnica woyskowa’, which recreates in Polish two different senses of Bailey and Klausing’s headword (‘das Reisegeräth, Wanderung; eine Soldatenhure’); another is the entry for \emph{Baste} ‘Polewać pieczyste, co zle szyć, bić’ (‘gebratenes am Spiess begiessen; etwas Schlecht nähen; abschmieren, abprügeln’); and still another is the entry for \emph{Deny} ‘zapierać, przeczyć, odmówić’ (‘läugnen, verneinen; abschlagen, verweigern; absagen, verläugnen’).\footnote{Importantly, different senses in Bailey and Klausing’s dictionary are separated with semi-colons.}

The need to negotiate meaning by making assumptions about what the lexicographers meant and correlating them with factual knowledge was an essential part of the compilation process.\footnote{In foreign-language teaching, meaning negotiation has been perceived as an interactive process, particularly in a language classroom (see, e.g., \citealt[17]{MacaroWooreGraham2015}). In the historical perspective, however, a student working with a foreign text was often bound to establish meaning solely on the basis of his or her dictionary.} This was intellectually demanding because the French and German glosses were often dissimilar. \emph{He looks as big as Bull-Beef}, for example, was rendered into French as ‘Il a le regard extremement fier’ [He looks extremely proud] and into German as ‘Er siehet aus, als ob er mich fressen wollte’ [He looks as if he wanted to eat me].\footnote{OED3 \citep{OED3} explains that, in such examples of usage as \emph{ to bluster like bull-beef} and \emph{as big as bull-beef}, \emph{bull-beef} is a term of abuse.} It was only to be anticipated that mediating between the SLs would result in misconceptions (e.g., \emph{Bed-rid} ‘łożkiem bawiący się’; \emph{Cellar} ‘sklep na trunki’; \emph{Charity} ‘miłość’; \emph{Dimple} ‘Dołek pod nozdrzami nad wargami’;\footnote{Cf. Bailey and Klausing’s ‘ein Grübchen im Kinn oder Backen’ and Trotz’s ‘Dołek iako w brodzie, śmiechowy w policzkach’ (1744).} \emph{Foot-men} ‘stołeczek do ściągania butów’; \emph{Humoursome} ‘dziwak, chymeryk’; \emph{Poem} ‘poema rym’; \emph{Speech} ‘Parole słowo’; \emph{Tempest} ‘Niepogoda’; \emph{Toast} ‘pieczyste’; \emph{Womb} ‘żywot niewieści’; and \emph{Ye} ‘zamiast the’).

The Polish part covers single-word equivalents (e.g., \emph{Experience} ‘doświadczenie’), descriptive equivalents (\emph{Outworks} ‘Dzieła zewnętrzne fortyfikacyi’), and translated citations (\emph{Is there any of you that will undertake it?} ‘Czy iest kto z was który by chciał to przedsięwziąć?’). Upon closer analysis, three categories of lexicographical treatment may be distinguished: 

\begin{itemize}
    \item Polish constructions documented in historical texts, including bilingual dictionaries
    \item Polish word-for-word translations from French or German unattested elsewhere
    \item missing Polish counterparts for citations, culture-bound terms, and other types of vocabulary
\end{itemize}

These categories will be discussed one by one and illustrated with examples. 

The fact that many Polish constructions are documented in historical sources may be indicative of the compiler’s native speaker competence, but they are more likely to derive from French-Polish or German-Polish dictionaries. As already remarked, French and German were taught extensively to Polish students, so a number of bilingual and multilingual resources were available on the market. Danet and Kola’s French-Latin-Polish dictionary \citep{DanetKola1743, DanetKola1745} and Trotz’s French-German-Polish dictionary \citep{Trotz_michel_a1747} were the largest and the best (see, e.g., \citealt{Kurylowicz_beata2021}), but there were also less comprehensive works such as Eberlein's German-Polish \citep{Eberlein_christian1763}, Trotz's German-Polish \citep{Trotz_michel_a1772},  and Kaliszewski's French-Polish dictionaries \citep{Kaliszewski1795}. It may be safely assumed that the compiler used them in his foreign language education. 

Let us look at a few examples, beginning with \emph{Ancle} ‘kostka u nogi’. Bailey and Klausing’s definitions ‘la cheville de pied’\slash ‘der Knöchel am Fuss’ have been traced to Danet and Kola’s ‘kostka u nogi~...’ \citep{DanetKola1743} and Trotz’s ‘kostka, nogi część~...’ \citep{Trotz1744}, so we could infer that the former served as the compiler’s direct source. The verb phrase ‘mieć płynienie krwi z nosa’ (\emph{Bleed at the nose}) was probably borrowed from Trotz’s 1772 dictionary, whereby it is the first of five stylistic variants (‘krwi płynienie z nosa, iście krwi z nosa, ciorkanie krwi z nosa, gdy kroplemi ciecze kroplami,\footnote{This wording is erroneous; it should have been replaced by the gloss ‘gdy krew ciecze kroplami’.} cieczenie krwi z nosa’). \emph{Falling-Sickness} is paired with ‘kaduk, wielka choroba’. In his \emph{Royal dictionary}, Boyer offers two equivalents: ‘le haut mal, le mal caduc’, both of which, unsurprisingly, lead us to Danet and Kola’s ‘kaduk, wielka choroba’. The same procedure was followed for \emph{Fob} ‘kieszonka w spodniach’, first looked up in Boyer (‘gousset’) and then in a French-Polish dictionary, this time Kaliszewski’s. More examples are exhibited in \tabref{tab:Podhajecka:table1}. The symbol ≠ signifies a marked difference between the structure of the SL and TL items.

\begin{table}[t]
\small
\caption{Polish constructions documented in historical texts.}
\label{tab:Podhajecka:table1}
\begin{tabularx}{\linewidth}{lQQ}
\lsptoprule
The headword & Background sources & The manuscript \\\midrule
Ablepsy & eine Blindheit der Augen oder des Gemüths & Ślepota oczów albo umysłu \\
Bantling & (kleines Kind) ein Bankert, ein Kind, welches vor der Hochzeit erzenget worden & Małe dziecię przed ślubem urodzone \\
To Blister & appliquer des vesicatoires & wezykatoria przykładać \\
Campeche & lignum Indicum sic dictum & drzewo Indyiskie tak nazwane \\
Cousin & cousin, cousine & ≠ Stryieczny cioteczny brat, krewny \\
Dizziness & vertige, tournement de tête & Zawrót głowy \\
Embryo & eine unzeitige Geburth, unvollkommene Frucht in Mutterleibe & Płód w żywocie macierzyńskim \\
Epiphany & le jour \emph{ou} la fête des Rois & Święto trzech królów \\
Fish-day & jour maigre & Postny dzień \\
Loyn of Veal & la langue de veau & ≠ Pieczenia nerkowa cielęca \\
Make mouths & faire la mouë & Wykrzywiać się wargami komu \\
Parlour & der Sprachsaal & izba gościnna do rozmowy \\
Peace, peace there & paix, paix-là & ≠ nakazując milczenie \\
Sixpence & ein Sechs Pence, oder halber Schilling & ≠ pieniędzy Gatunek \\
Unbloody victory & une Victoire obtenue sans effusion de sang  & Zwycięstwo bez wylania krwi \\
Wrist & das Gelenke an der Hand & ≠ Przegub ręki między dłonią y łokciem  \\
\lspbottomrule
\end{tabularx}
\end{table}


Whenever literal translation would produce constructions atypical of, or entirely alien to, the compiler’s mother tongue, the Polish equivalents diverged from the SL patterns. ‘Pieczenia nerkowa cielęca’ for \emph{Loyn of Veal} (‘la langue de veau’) and ‘przegub ręki między dłonią y łokciem’ for \emph{Wrist} (‘das Gelenke an der Hand’) are suitable examples. The linguistic tradition was also a factor. Taking \emph{Cousin} into consideration, it has been customary for Poles to refer to a male cousin in a twofold way as ‘brat stryjeczny’, i.e., the son of one’s father’s sibling, and ‘brat cioteczny’, i.e., the son of one’s mother’s sibling.\footnote{Female cousins are called ‘siostra stryjeczna’ and ‘siostra cioteczna’.} As for \emph{Sixpence} ‘Pieniędzy Gatunek’, culture-bound terms tend to be ambiguous to the target user; hence, a descriptive equivalent.\footnote{Cf. \emph{Penny} ‘un sou gatunek pieniędzy’. In his \emph{Anglia i Szkocja: przypomnienia z podróży roku 1823−24 odbytey} \citep{Lach-szyrma1828}, Lach-Szyrma preferred to borrow \emph{penny} (\emph{pense}) and \emph{shilling} (\emph{szyling}) rather than describe them.} Although the idiomatic expression \emph{Make mouths} was rendered word for word (‘Wykrzywiać się wargami komu’), the compiler should not be reproached for this, as Danet and Kola’s definition of \emph{Moue} was as follows: ‘Wystawienie wargi na kogo wydrzeźniaiąc mu się wykrzywianie się na kogo’ (1745). ‘Nakazując milczenie’ (\emph{Peace, peace there}) looks more like an explanatory gloss than a functional equivalent. The descriptive equivalent for \emph{Campeche} suggests that Latin might have been another SL.

The second category sometimes sees the compiler go astray by wording the Polish equivalents in a way unattested in eighteenth-century materials. For example, \emph{Alectromancy} denotes ‘divination using a cock’. Bailey and Klausing’s ‘das Wahrsagen durch einen Hahn’ was translated word for word into ‘wróżenie przez koguta’, but the preposition was incorrect; \emph{wróżenie z koguta} \citep[117]{Osinski_alojzy1806} sounded more natural.\footnote{Cf. ‘wróżenie z ręki’ [divination from the palm], ‘wróżenie z rzeczy’ [divination from things], and ‘wróżenie z roślin’ [divination from plants].} ‘Pasztet z węgorzy’ (\emph{An eel pie}), probably a rendition of Boyer’s ‘un pâté d’anguilles’, is the earliest occurrence in Polish, but it is Lach-Szyrma’s ‘pasztet z węgorza’ that became the standard form. \emph{Empirick} ‘lekarz z doświadczenia’ denotes (positively) a doctor by experience,\footnote{\emph{Empirick} must have been borrowed from Bailey and Klausing, as Boyer has it in the spelling \emph{Empiric}.} which ought to be treated as a mistake. The traditional Polish equivalent found in eighteenth-century bilingual dictionaries, ‘lekarz jarmarkowy’, had pejorative connotations and there are reasons for thinking that ‘une empirique’ was a term of contempt in French, just like ‘Quacksalber’ and ‘Marktschreier’ in German. ‘Mieysce do zdybania się’ (\emph{Rendevous}) must have been a translation of Bailey and Klausing’s ‘ein bestimmter Ort der Zusammenkunft, Sammelplatz’. The Polish deverbal noun ‘zdybanie się’ was less outdated than it is today, but ‘schadzka’ would, nonetheless, be a better option (cf. Trotz’s \emph{Schadzka, Zchadzka, Schacka} ‘Ort der Zusammenkunft’, 1764). 

\tabref{tab:Podhajecka:table2}   includes additional Polish constructions, translated word for word on the basis of the background sources, which are undocumented in historical materials.

\begin{table}[t]
\small
\caption {Word-for-word Polish translations unattested in texts.}
\label{tab:Podhajecka:table2}
\label{tab:Podhajecka:2}
\begin{tabularx}{\linewidth}{ lQQ }
\lsptoprule
The headword & Background sources  & The manuscript\\
\midrule
An Alder-bed & ein Ort, wo viele Erlenbäume wachsen & Miejsce gdzie wiele Drzew olchowych rośnie \\
Anticor & eine Geschwulst an der Brust der Pferde & puchlina na piersiach konia\\
Bestead & se montrer ami de quelqu'un, le servir & Pokazać się komu przyiacielem, iemu usłużyć \\
Bilboes & eine Art der Strafe unter den Seeleuten & pewny Gatunek kary między Żeglarzami \\
Calambo & in wohlriechendes Holz, so aus der Insel Java wächset & Gatunek pachnącego drzewa na wyspie Jawa \\
Ferret & ein Iltis, geringe Seide → faire sortir les lapins de leur terrier par le moyen des furets
 & łasica zwierząt do wypędzania krolikow z iam zdatna \\
Instrument-keeper & garde des instrumens & Instrumentów muzycznych dozorca \\
Fall of the leaf & la chute des feuilles, l'automne & opadanie liści albo Jesień \\
Megrim & migraine, douleur de tête, mal de tête & Migrena, albo bol części głowy \\
Pedlar & petit mercier, qui porte sa boutique sur soi\slash einer der sein Krämlein am Halse herum trägt  & torbiarz ten co towary swoje nosząc przedaie \\
Reassume & wieder annehmen & znowu wziąć co przed się \\
Scull & die Hirnschale, der Schedel & czaszka na głowie \\
Silk-weaver & ouvrier en soie & Ten co w iedwabiu wyrabia \\
Tyler & un couvreur en tuile & ten co nakrywa dachy dachówką \\
Usher & Unterschulmeister & Namiestnik mistrza szkoły \\
A Venison Pasty & pâté de venaison & pasztet ze zwierzyny \\
\lspbottomrule
\end{tabularx}
\end{table}

In the case of \emph{Ferret}, the original equivalents did not match the Polish gloss, so I used a back-translation, which points to the entry for \emph{Fureter} in a handful of works, including \emph{Table analytique et raisonnée des matieres} …  \citep{Noauthor1782}. The Polish equivalent in the entry for \emph{Pedlar} was less erudite than Rykaczewski’s ‘kramarz wędrujący’ (1849), whilst that for \emph{Scull} was imprudent (cf. Trotz’s ‘czaszka głowy’, 1764). It may be interesting to note that single-word equivalents such as ‘jedwabnik’ (\emph{Silk-weaver}) and ‘dachownik’ (\emph{Tyler}) were attested in a range of eighteenth-century dictionaries, including \citet{DanetKola1743}, \citet{Trotz1744}, and \citet{Eberlein_christian1763}. ‘Podnauczyciel’, recorded in Antonowicz’s Polish-English glossary included in his grammar \citep{Antonowicz1788}, was a far more natural equivalent for \emph{Usher}.

It needs to be explained that the constructions in \tabref{tab:Podhajecka:2} have not been found in their entirety, but components thereof were obviously known and used. ‘Kara’ [punishment], for instance, has been retrieved in contexts wholly unrelated to ‘żeglarze’ [sailors]; none of the uses of ‘instrumenty muzyczne’ [musical instruments] pertains to ‘dozorca’ [keeper, guard]; and nowhere are ‘opadanie liści’ [the falling of leaves] and ‘jesień’ [autumn] put side by side.\footnote{To arrive at credible results, I also consulted Samuel Bogumił Linde’s \emph{Słownik języka polskiego} \citep{Linde_samuel_b1807}, the first fully-fledged dictionary in the history of Polish lexicography.} Burdened with the task of expressing foreign concepts, especially when precise and compact Polish equivalents escaped him, the compiler was torn between the SL and TL patterns. In fact, this would be in line with modern research revealing that translators tend to introduce peculiarities of the SL into the TL. As \citet[222]{Teich_elke2003} puts it, the SL “shines through” the TL translation.




Let us consider the third category. Every now and then, the draft dictionary includes citations borrowed from one of the grammars; the author was evidently aware that language primarily serves communication. Regrettably, only a handful of contextual uses are translated into Polish (e.g., \emph{Let him be ever so rich}\footnote{This citation comes from another source, conceivably Peyton’s \emph{La nouvelle grammaire angloise~...} \citep{Peyton_vj1767} or Haussner’s \emph{Phraseologia anglo-germanica~...} \citep{Haussner_friedrich_w1798}. In the former, the SL version was ‘laissez-le être qu'il soit jamais si riche: quelque riche qu'il soit’ and, in the latter, ‘wenn er noch so reich wäre’.} ‘niech sobie będzie jakożkolwiek bogatym’), which undermines the quality of the English-Polish lexicographical project.

While some of the French citations in \tabref{tab:podhajecka:3} might have been a little problematic to render into Polish, others posed no special difficulty, but they, too, were left untranslated. This raises questions of the compiler’s proficiency in the SLs and of his method of work. One cannot exclude that, once the French or German counterparts had been put down, a search for the Polish items was to ensue. For some reason, however, this procedure failed to take place.

\begin{table}
\caption{English citations recorded only in French.}
\label{tab:podhajecka:3}
\begin{tabularx}{\linewidth}{ lQQ }
\lsptoprule
The headword & English citations & French versions \\\midrule
Able & I have eat so much that I shan’t be able to eat my Dinner. & J'ai tant mangé, que je ne pourrai pas diner. \\
Conspicuous & The English Valour has been Conspicuous on several occasions. & La Valeur angloise s'est fait voir avec éclat dans plusieurs Occasions. \\
Could & I could not write it. & Je n'ai pu l'écrire. \\
Favours & The wedding ring and Favours are already bought. & On a dejà acheté l'anneau nuptial, et les livrées. \\
Hop & Shall we hop with one leg. & Voulez-vous que nous fautions à cloche pied. \\
Order & She is a little out of order. & Elle est un peu indisposée. \\
Parted & Methinks ‘tis not long since we parted. & Il me semble qu'il n'y a pas long-tems que nous nous sommes séparez. \\
Races & Let us run Races. & Exerçons-nous à la course. \\
Tattle & She does nothing but prattle or tattle. & Elle ne fait que causer, ou caqueter. \\
View & Let’s view those Fields and Meadows. & Promenons notre vûë sur ces champs \& sur ces prairies. \\
Way & We are a great way from the River. & Nous sommes éloignez de la riviere. \\
Yonder & I suppose they are those that walk yonder. & Je m'imagine que ce sont ceux qui se promenent là-bas. \\
\lspbottomrule
\end{tabularx}
\end{table}

Other headwords lacking Polish equivalents are culture-bound terms, also taken from the two grammars. What the compiler must soon have realised is that they named realia typical of Britain (e.g., \emph{the Keeper of the Privy Purse}) and France (e.g., ‘le Trésorier des Menus’), whilst here equivalence was sought between English and Polish. As literal translation was not always the best strategy, more time and resources were needed to identify fitting Polish terms, so only French items were copied into the manuscript.\footnote{\emph{Dean of his Majesty’s Chappel} ‘Dziekan Kaplicy Królewskiej’ and \emph{The Master of the King’s Household} ‘Marszałek Nadw.[orny] Król.[ewski]’ may be viewed as exceptions to the rule.}

\begin{table}
\caption {English culture-bound terms recorded only in French.}
\label{tab:podhajecka:4}
\begin{tabularx}{\linewidth}{ lQQ }
\lsptoprule
The headword & English terms & French equivalents \\\midrule
Almoner & the Lord Almoner & Grand-Aumônier \\
Buttery & the Gentleman of the great Buttery & le chef de la grande depense \\
Groom & the Groom of the stole and first Gentleman of the Bed-Chamber & le premier gentilhomme de la chambre du lit du roi \\
House & the House of Commons & la chambre des communes, \emph{ou} la chambre basse \\
Master & Pay Master of his Majesty’s Forces & Trésorier de l'ordinaire \& de l'extraordinaire des Guerres \\
Picture & the Keeper of the King’s Pictures & Huissier du Cabinet des Peintures \\
Poultry & the Clerk of the Poultry & Le comptrolleur de l'office de la volaille \\
Privy & a Gentleman of the King’s Privy – chamber in ordinary & Gentilhomme ordinaire de la Chambre Privée du Roi \\
Seal & Lord Privy Seal & Garde du Petit-Sceau \\
State & Secretary of State & Un Secretaire d'Etat \\
Usher & A Gentleman Usher of the Presence Chamber & un Ecuyer de la chambre de présence \\
Work & the Surveyor-General of the works & Intendant des Bâtiments \\
Yeoman & Yeoman of the Pantry & l'Aide de la Paneterie \\
\lspbottomrule
\end{tabularx}
\end{table}

Even today, few of the terms in \tabref{tab:podhajecka:4} could be paired with Polish equivalents, of which \textit{Izba Gmin} (`The House of Commons') and \textit{sekretarz stanu} (`Secretary of State') are the easiest. Whilst \emph{the Keeper of the King’s Pictures} might potentially be expressed as ‘strażnik obrazów królewskich’,\footnote{Google Books has attestations for, among others, ‘strażnik lasów królewskich' (\emph{Keeper of the Royal Forests}) and 'strażnik grobów królewskich' (\emph{Keeper of the Royal Tombs}).} other renditions (e.g., ‘urzędnik do spraw drobiu’ for \emph{the Clerk of the Poultry}) would look idiosyncratic.\footnote{Even \emph{James} was juxtaposed with its French version ‘Jacques’ (cf. \emph{Ralph} ‘Rudolf’).} The fact that Rykaczewski succeeded in pairing \emph{Groom of the Stole} with ‘dozorca garderoby królewskiej’, \emph{Gentleman of the Bed-Chamber} with ‘kamerjunker’, \emph{the House of Commons} with ‘Izba niższa Parlamentu’, \emph{Pay Master General} with ‘jeneralny płatnik’, \emph{Keeper of the Privy Seal} with ‘podkanclerz’,\footnote{Diki (\url{https://www.diki.pl/}), an online dictionary, proposes the equivalent ‘Lord Strażnik Tajnej Pieczęci’ for \emph{Keeper of the Privy Seal}, which is richly documented in Google Books (\url{https://books.google.com}).}  and \emph{Yeoman of the Pantry} with ‘gwardzista piechotny’ (1849) attests to his being a highly skilled lexicographer.

There are further headwords without Polish counterparts. Explaining the verb \emph{to Bastard} (‘überführen, das einer ein Hurenkind ist, ein Hurkind zeugen’) must have been a pitfall because even Rykaczewski omitted it. ‘Ospa wietrzna’ is the modern equivalent of \emph{Chicken-pox}, which was previously termed 'ospice owcze', i.e., a disease affecting sheep.\footnote{Cf. Trotz’s \emph{verole volante} ‘Windpocken. Ospice wietrzne, łuskawe’ (1747).} As for the noun \emph{Cabbage}, the prototypical equivalent is self-evident (‘kapusta’), but we also come across another nominal meaning, ‘das Dicke und harte zu unterst am Hirschgeweihen’, and two verbal meanings, ‘pommer’ and ‘voler’, which turned out to be more troublesome.\footnote{Only the latter was admitted into Rykaczewski’s English-Polish volume (‘kraść okrawki sukna (o krawcach)’).} This notwithstanding, circumventing lexicographical problems by leaving foreign material untranslated into Polish is hardly surprising if we take into account the scope of the project, which sought to pair two languages that had never been paired before. The fact that the compilation was undertaken by a language learner justifies, at least to some extent, the shortcomings of the manuscript.

The compiler drew on bilingual dictionaries for different language pairs in order to ensure the feasibility of his endeavour, but the vocabulary of the SLs was not made up of single words that could be substituted with their TL equivalents in a neat and tidy way. Countless TL multi-word expressions such as collocations and idioms, moreover, were unlike their SL counterparts in terms of lexical and syntactic structures, and some had no equivalents at all. There is, therefore, no doubt that translatability had then, and still has today, inevitable limits. As \citet[17]{Baker_mona1992} explains,

\begin{quote}
    the nature of language is such that, in the majority of cases, words have ‘blurred edges’; their meanings are, to a large extent, negotiable and are only realized in specific contexts. The very notion of ‘types of meaning’ is theoretically suspect.
\end{quote}

The bilingual lexicographer’s aptitude for expressing the concepts of one culture in the language of another was often taken for granted, but, in reality, few, if any, reference works were compiled from scratch. This is the reason borrowing was notorious in lexicography, even though not always “performed on dictionaries of the same direct lineage” \citep[137]{Francoeur_aline2010}.\footnote{Instead of speaking of blatant plagiarism, \citet[97]{Dolezal_fredric_t2023} takes a less critical stance by considering borrowing [a] “time-honored lexicographic practice”.} Suffice it to say that, due to the dominant tradition of teaching foreign languages via the Grammar-Translation Method, literal translation in both the West and the East prevailed in the eighteenth century as much as in previous centuries.

\section{Conclusions}

As demonstrated above, the compiler of the manuscript dictionary faced a number of quandaries, all the more because he was presumably a tyro. His lack of expertise has been exhibited, on the one hand, by leaving a great many words, phrases, and citations untranslated into Polish and, on the other, by failing to employ the full linguistic repertoire typical of an educated native speaker of Polish. Unable to match the English headwords with their Polish counterparts, he used bilingual dictionaries as a bridge in the hope of arriving at helpful patterns in French or in German. The comparative analysis indicates that he followed a straightforward method of gathering data. In other words, he rendered the SL material, often resorting to word-for-word translation, unless his native speaker intuition or bilingual dictionaries with Polish guided him otherwise. Whether he undertook his task with “mixed enthusiasm and apprehension” \citep[133]{Cormier_monique2010}, we are unlikely ever to establish.

Time and money have been crucial factors in any lexicographical project, and this one is no exception. The pressure of time, in particular, may explain why the manuscript was left unfinished: looking up thousands of headwords in the various sources to hand would have been painfully slow and laborious. All in all, relying on equivalents culled from English-French and English-German works, some of which were translated with the help of French-Polish and German-Polish dictionaries, the anonymous compiler showed a great deal of ingenuity. This should be appreciated, but, unlike Lach-Szyrma and Rykaczewski who spent several years in Britain, he was insufficiently acquainted with the English language and culture. Consequently, even if the manuscript had been published, it would have stood no chance of becoming a genuine cross-linguistic and cross-cultural intermediary. 

One might wonder at the motives that prompted the author to take on such a taxing project. Browsing through the material, I discovered three citations from Antonowicz’s bilingual preface to his grammar, of which the following is worth quoting here: \emph{had I contributed to render English more common to my Countrimen} ‘gdybym się przyłożył do uczynienia powszechnieyszym Języka angielskiego’. Perhaps one of Antonowicz’s “few gentle boys” (1788) was so impressed with English that he decided to contribute, as did his mentor, to familiarising his compatriots with the English vocabulary. The names of three of Antonowicz’s students are found in the literature: Alojzy Feliński (1771--1820), a poet, translator, and literary critic; Count Jan Feliks Tarnowski (1777--1842), Czacki’s nephew, a historian, bibliophile, and senator of the Kingdom of Poland; and Rev. Antoni Bęczkowski (?−1858), a polyglot, teacher, and rector of the Piarist gymnasium in Międzyrzecze Koreckie (in the Wołyń province). The last of these is known to have encouraged his students to read English literature in the original \citep[236]{Kowalski_franciszek1859}, which makes him a most suitable candidate.

{\sloppy\printbibliography[heading=subbibliography,notkeyword=this]}

\end{document}

