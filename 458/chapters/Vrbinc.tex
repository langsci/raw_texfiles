\documentclass[output=paper,colorlinks,citecolor=brown,arabicfont,chinesefont]{langscibook}
\ChapterDOI{10.5281/zenodo.15394481}
\author{Alenka Vrbinc\affiliation{University of Ljubljana} and
        Donna Farina\affiliation{New Jersey City University} and
        Marjeta Vrbinc\affiliation{University of Ljubljana}}
\title[Project Cleveland]
      {Project Cleveland: Documenting the lexicographic output of 20th-century Slovenian immigrants in the US}

\abstract{At the beginning of the 20\textsuperscript{th} century, an enormous immigration resulted in the establishment of multiple European ethnic groups in the United States. The Slovenian immigration to the US presents a compelling story. From 1905  until World War I, the greatest mass of Slovenians arrived and settled mostly in the midwestern and northeastern US. A smaller Slovenian wave post-World War II had refugees from communism and fascism.

The present endeavor analyzes the macrostructure, with some glimpses at microstructure, of six Slovenian–English, English–Slovenian dictionaries and reference works targeted at the Slovenian immigrant group. This is part of a project investigating both microstructure and macrostructure of a greater number of books, as well as the lives of their authors. A research trip to Cleveland, Ohio as well as Joliet and Lemont, Illinois allowed for the discovery of some dictionaries and an examination of newspaper archives, genealogical and other relevant materials. Certainly, the works found have played a role in the acculturation of Slovenians in the US. Slovenian immigrants are said to have “acquired English with impressive speed and facility” \citep{Gobetznodate}; the ongoing project seeks in part to understand how dictionaries and reference books may have supported this effort.}

\IfFileExists{../localcommands.tex}{
  \addbibresource{../localbibliography.bib}
  % add all extra packages you need to load to this file

\usepackage{tabularx,multicol}
\usepackage{url}
\urlstyle{same}

\usepackage{listings}
\lstset{basicstyle=\ttfamily,tabsize=2,breaklines=true}

\usepackage{langsci-basic}
\usepackage{langsci-optional}
\usepackage{langsci-lgr}
\usepackage{langsci-osl}
% \usepackage{./langsci/styles/langsci-lgr}
% \usepackage{./langsci/styles/langsci-osl}
% \usepackage{langsci-gb4e}

\usepackage{tikz}
\usetikzlibrary{patterns,calc}
\pgfdeclarepatternformonly{south east lines}{\pgfqpoint{-0pt}{-0pt}}{\pgfqpoint{3pt}{3pt}}{\pgfqpoint{3pt}{3pt}}{
    \pgfsetlinewidth{0.6pt}
    \pgfpathmoveto{\pgfqpoint{0pt}{3pt}}
    \pgfpathlineto{\pgfqpoint{3pt}{0pt}}
    \pgfpathmoveto{\pgfqpoint{.2pt}{-.2pt}}
    \pgfpathlineto{\pgfqpoint{-.2pt}{.2pt}}
    \pgfpathmoveto{\pgfqpoint{3.2pt}{2.8pt}}
    \pgfpathlineto{\pgfqpoint{2.8pt}{3.2pt}}
    \pgfusepath{stroke}}
    
\usepackage{stmaryrd}
\usepackage{wasysym}
\usepackage{multirow}
\usepackage{caption}
\usepackage{subcaption}
\usepackage{mathrsfs}
\usepackage{qtree}

\usepackage{linguex}


  %pminos do not split footnotes
% \interfootnotelinepenalty=10000 %Footnote in Laporte chapters has to be split SN


%\DeclareIndexNameFormat{default}{%
%\nameparts{#1}%
%\usebibmacro{index:name}%
%{\index[names]}%
%{\namepartfamily}%
%{\namepartgiveni}%
% {}% L1
% {}% L2
%{\namepartprefix}% generates spurious space L3
%{\namepartsuffix}% generates spurious space L4
%}

%  {\DeclareIndexNameFormat{default}{%
%     \usebibmacro{index:name}{\index[names]}{#1}{#3}{#5}{#7}}}

%\DeclareIndexNameFormat{default}{%
%  \usebibmacro{index:name}{\sindex[nom]}{#1}{#3}{#5}{#7}}

%\DeclareIndexNameFormat{default}{%
%  \usebibmacro{index:name}{\sindex[person]}{#1}{#3}{#5}{#7}}
%\DeclareIndexNameFormat{default}{%
%\nameparts{#1} \usebibmacro{index:name}{\sindex[person]]}{\namepartfamily}{‌​\namepartgiven}{\nam‌​epartprefix}{\namepa‌​rtsuffix}}

%\newcommand{\smiley}{:)}

%\renewbibmacro*{index:name}[5]{%
%\usebibmacro{index:entry}{#1}%
%{\iffieldundef{usera}{}{\thefield{usera}\actualoperator}\mkbibindexname{#2}{#3}{#4}{#5}}}

% \newcommand{\noop}[1]{}

%remove for final
%\overfullrule=1mm

\newcommand{\tobi}[2]}}
\renewcommand{\S}[1]{\tobi{#1}{\textsc{*}}}

% this volume references
% puts: [this volume]
% already defined: \citetv
%\newcommand{\citepv}[1]{(\citeauthor{#1} \citeyear*{#1} [this volume])}
\newcommand{\citealtv}[1]{\citeauthor{#1} \citeyear*{#1} [this volume]}

%parentheses around example number
\newcommand{\pref}[1]{(\ref{#1})}

% in-text examples

\newcommand{\lnex}[1]{\textit{#1}} %target lang word
\newcommand{\lnlit}[1]{(lit.: `#1')} %literal reading
\newcommand{\lnlat}[1]{(#1)} % latinization
\newcommand{\lntrans}[1]{`#1'} %translation
\newcommand{\lnexl}[2]%
{\lnex{#1}{} \lnlat{#2}} % ex with latinization
\newcommand{\lnexlat}[3]{\lnex{#1}{} \lnlat{#2}{} \lntrans{#3}} % ex with latinization and tranl.

%ch01
\newcommand{\co}[1]{\mbox{\textbf{#1}}}

%ch09

\newcommand{\cyrbulg}[1]{\begin{otherlanguage*}{bulgarian}#1\end{otherlanguage*}}


%ch10
\newcommand{\nlp}{{\small NLP}}
\newcommand{\mwe}{{\small MWE}}
\newcommand{\rae}{{\small RAE}}
\newcommand{\lvc}{{\small LVC}}
\newcommand{\pos}{{\small P}o{\small S}}
%\newcommand{\todo}[1]{ \textcolor{red}{#1} }

%\renewcommand{\labelenumi}{\theenumi}
%\ainamefmt{{vv}{ll}{, ff}{, jj}} % fullname

\newcommand{\biberror}[1]{{\color{red}#1}}

\newcommand{\osenovaitem}{--~}
  %% hyphenation points for line breaks
%% Normally, automatic hyphenation in LaTeX is very good
%% If a word is mis-hyphenated, add it to this file
%%
%% add information to TeX file before \begin{document} with:
%% %% hyphenation points for line breaks
%% Normally, automatic hyphenation in LaTeX is very good
%% If a word is mis-hyphenated, add it to this file
%%
%% add information to TeX file before \begin{document} with:
%% %% hyphenation points for line breaks
%% Normally, automatic hyphenation in LaTeX is very good
%% If a word is mis-hyphenated, add it to this file
%%
%% add information to TeX file before \begin{document} with:
%% \include{localhyphenation}
\hyphenation{
    Beck-man
    Ngu-yen
    back-chan-nel
    back-chan-nels
    mo-not-o-nous
    ste-reo-typ-i-cal
}

\hyphenation{
    Beck-man
    Ngu-yen
    back-chan-nel
    back-chan-nels
    mo-not-o-nous
    ste-reo-typ-i-cal
}

\hyphenation{
    Beck-man
    Ngu-yen
    back-chan-nel
    back-chan-nels
    mo-not-o-nous
    ste-reo-typ-i-cal
}

  \togglepaper[7]%%chapternumber
}{}

\begin{document}
\maketitle

\section{Introduction}

\subsection{Project}

This paper does not present a finished product, but rather is part of an ongoing project that seeks to analyze the lexicographic output targeted toward Slovenian immigrants in the United States; it also seeks to understand the circumstances and context of this output. During Phase 1, the researchers identified, obtained, and began the analysis of numerous Slovenian–English, English–Slovenian dictionaries and reference works (phrasebooks, grammar books, school textbooks, etc.). These were published from the turn of the 19\textsuperscript{th} century to the 1940s. This time frame corresponds to the most active periods of Slovenian immigration to the US.

Phase 2 of the project was a research trip in Summer 2021 to Cleveland, Ohio; Joliet, Illinois; and Lemont, Illinois. These are just a few of the sites where Slovenian immigration has had a lasting effect. Newspaper archives, genealogical, and other available materials were examined; these provided information about dictionaries and reference books, as well as about their authors, many of whom were prominent Slovenian-Americans. During this trip, meetings with descendants of Slovenian immigrants were invaluable. We were unable to meet with Dr. Edward Gobetz, an immigrant himself and a lifelong researcher of Slovenian immigration; he died in his 90s from covid, six months before our trip. \citet{Gobetz2014} maintained that Slovenian immigrants learned English with “impressive speed and facility”; in part, this project seeks to understand how these dictionaries and reference books supported this effort.

Future work will examine the biographies of several of the Slovenian/American authors, among them one who was a Franciscan priest (Kasimir Zakrajšek), one who was a medical doctor (Fran Javh Kern), one who started a commune in California (Peter Jos Jeram), and one who had one of his dictionaries printed in the building that today is City Hall, New York (Viktor J. Kubelka).

In this paper we first present as background an overview of the context of Slovenian immigration to the US; while it is a given that dictionaries are cultural artifacts, products of their times, this background brings home the reality of the lexicographic work. Second, here we provide a general outline of the macrostructure of six of the reference works. While we incorporate some pages from the dictionaries and reference books, a presentation of the full microstructure is beyond the objectives of this paper. 

\subsection{Slovenia and Slovenian}

The Republic of Slovenia today (\figref{vrbinc:fig1}) is a country of more than two million people, located in central Europe. It was part of the former Yugoslavia and was one of the six republics within the  Socialist Federal Republic of Yugoslavia. Slovenia declared independence in 1991. Modern Slovenian territory was part of the Austro-Hungarian Empire from 1867 to 1918 and it was also under the Habsburg Monarchy, various parts of modern Slovenia having been annexed from 1282 to 1335. 

Slovenian, the most widely spoken language in the country, is classified genetically as a South Slavic language along with other languages spoken both within the former Yugoslavia and beyond it. Although Slovenian has a relatively small number of speakers, it nevertheless has a significant lexicographic tradition; this history, like that of many other traditions (cf. \citealt{Bejoint2016, Farina2009, Fontenelle2016}) began with needs arising from contact with other languages and cultures. For Slovenian, the main cultural and linguistic contact was with the German language within Central Europe; this contact deeply impacts Slovenian culture and lexicography to this day.

\begin{figure}
\includegraphics[width=.66\textwidth]{images/slovenia.pdf}
\caption{Map of present-day Slovenia, created by the UN Office for the Coordination of Humanitarian Affairs (OCHA)}
\label{vrbinc:fig1}
\end{figure}

\section{Part 1: History of Slovenian immigration to the US}

There is a rich history of Slovenian immigration around the world that has been well documented. One part of this is the immigration to North America, particularly to the US. Here, we present a chronological overview with an emphasis on the Slovenian presence in the US at the end of the 19\textsuperscript{th} to the mid-20\textsuperscript{th} century.

\subsection{The beginnings}

Up to the 1850s, when Slovenia was still a part of the Hapsburg monarchy, mobility of people was limited, and thus outmigration was constrained. Nevertheless, there were clusters of Slovenians that set out for other lands before the mid-19\textsuperscript{th} century. For example, following the Counter-Reformation in Europe, some Protestants from Slovenia settled in the US state of Georgia by way of Germany \citep{Zugel1987}. 

Beginning in 1680, a few Slovenian settlements were founded in the US, primarily associated with the activities of Catholic missionaries \citep{Gobetz2014}. Sporadic emigration continued into the 18\textsuperscript{th} century; there are Slovenian last names among the soldiers in the 1775 continental army of George Washington \citep{Prisland1950}. Until 1782,  Austria restricted the movement of its citizens, and those who managed to leave were mostly exiles and people avoiding conscription or deserting from the army.

Starting in 1830, the missionary work of the Slovenian bishop Frederic Baraga (1797–1868) among the Native Americans spurred Slovenian migration activity. His books were widely read in Central Europe and awakened a curiosity about America \citep{Mazi-leskovar2003}. In the preface to his Ojibwe dictionary, Baraga mentions Samuel Johnson (see Figures~\ref{vrbinc:fig2a} and~\ref{vrbinc:fig2b}). Baraga is currently moving toward canonization in the Catholic Church.

\begin{figure}
\includegraphics[width=.9\textwidth]{images/Vrbinc_figure2.jpg}
\caption{Title page of \citegen{Baraga1853} Ojibwe dictionary}
\label{vrbinc:fig2a}
\end{figure}

\begin{figure}
\includegraphics[width=.8\textwidth]{images/Vrbinc_figure2bis.jpg}
\caption{Preface of \citegen{Baraga1853} Ojibwe dictionary}
\label{vrbinc:fig2b}
\end{figure}

\subsection{Mass migration and mass immigration}

Slovenian emigration began to pick up as part of the mass global movement of peoples. The Industrial Revolution of the late 18\textsuperscript{th} and early 19\textsuperscript{th} centuries unleashed internal migration, a flow of rural migrants to European towns and cities. The internal migration of Slovenians was exacerbated by the abolition of serfdom, resulting in the disintegration of farms and a dire economic situation for many \citep{Zugel1987}. It is not surprising that the less developed Austrian south (see \figref{vrbinc:fig3}, area labeled as Carniola), including territories that are part of modern Slovenia, saw substantial out-migration \citep{Drnovse2001}.

\begin{figure}
\includegraphics[width=\textwidth]{images/austria-hungary.pdf}
\caption{The ethnic groups of Austria-Hungary in 1910 according to 
\emph{Distribution of races in Austria-Hungary} by William R. Shepherd, 1911. This work is in the public domain.}
\label{vrbinc:fig3}
\end{figure}

The increasing commercial viability of steam power after 1764 led to the development of rail and sea transport, making the whole world more accessible. The population of Europe increased to 300 million in 1900, from 128 million in 1750. This increase is especially significant in light of the huge emigration; from 1820 to World War I, 55 million Europeans left, about 39 million of them arriving in North America \citep{Ferrie2013}. 

In the 20\textsuperscript{th} century, the face of European emigration to the US began to change. As noted by \citet{Gerber2011}:

\begin{quote}
    Between 1871  and 1900, 11.7 million immigrants arrived [in the US]; between 1901 and 1920 alone 14.5 million did. The points of origin were changing dramatically; while in the nineteenth century, western and northern Europeans predominated, now southern, central, and eastern Europeans did. (p35)
\end{quote}

The years 1905 to 1913 saw the largest wave of Slovenian immigrants arrive in North America \citep{Gobetz2014}. According to US census data, over 183,000 people said that their native language was Slovenian in 1910; ten years later almost 209,000 people made this claim \citep{Drnovsek1994}. It is estimated that prior to World War I, about one-third of the entire Slovenian population left their home region and went abroad. The Slovenian arrivals in North America were by and large peasants, but very few were able to continue farming. Most began working in coal mines or quarries; on railroad tracks, as masons, longshoremen, and lumbermen or in iron- and steel-working (\citealt{Zugel1987, Prisland1950}). Among the non-economic reasons for immigration were the desire to evade military service or the avoidance of criminal prosecution \citep{Drnovse2001}.

A list of the early Slovenian settlements in the US (see \tabref{tab:slovenian-settlements}) indicates that the ethnic group’s presence was significant and widespread.

\begin{table}
\caption{Early Slovenian Settlements in the US (\citealt{Gobetz2014}, \citealt{Klemencic1986}, \citealt{Zugel1987}}
\label{tab:slovenian-settlements}
\fittable{
\begin{tabular}{rl}
\lsptoprule
1865 & Brockway, Minnesota \\
1868 & Omaha, Nebraska \\
1873 & Joliet, Illinois \\
1878 & New York City \\
1881 & Cleveland, Ohio \\
By 1914  & Slovenian settlements in 34 US states and three Canadian provinces \\
\lspbottomrule
\end{tabular}
}
\end{table}

The Midwestern US (Pennsylvania, Ohio, Illinois, Minnesota, Wisconsin, and Michigan) was home to the most Slovenians; there were also settlements in Colorado and California (\citealt{Gobetz2014}, \citealt{Klemencic1986}, \citealt{Zugel1987}.) 

\subsection{Backlash}

Americans did not look favorably on the immigrants of the pre-World War I period; many feared that they would not assimilate as the northern European and Protestant immigrants before them had done. Their religion, physical appearance (called their “race” at that time), clothing, and poverty fostered suspicion \citep{Gerber2011}. Slovenians were subjected to pejorative names, including "Grainers" or "Grenish" from "Krainers," people from the province of Krain, Kranjska, or Carniola \citep{Gobetz2014}. American concerns about this new wave of immigrants resulted in the Immigration Act of 1891 \citep{Hoyt2017} as well as the restrictive quota immigration laws of the 1920s.

After World War II, a more sympathetic attitude on the part of the American public allowed exceptions to or expansions in the strict quotas of earlier immigration laws. Many Slovenians who entered the US during this postwar period were displaced persons, refugees either from fascism or from communism (the Tito regime). Approximately 100,000 Slovenians left communist Yugoslavia, with the US among their top destinations \citep{Barkeretal2023}. These new immigrants layered onto the existing Slovenian–American communities, not without religious, political, and ideological conflicts among them. One author describes this situation as a “schism” and indicates that there were dueling organizations, separation of living areas within cities, and even avoidance of intermarriage between groups \citep{Milharcic-hladnik2007}; another writer refers to the situation as a “cleavage” \citep{Susel}.

After the mid-20\textsuperscript{th} century, there was little migration out of Slovenia, to the US or anywhere else \citep{Barkeretal2023}. In the US today, the era of the close-knit and active Slovenian communities has passed, but the descendants of immigrants whom we met during this project have been motivated and persistent in keeping the knowledge base of their culture alive, through the maintenance of continuing organizations and through connections to organizations in Slovenia. The enormous archive of Edward Gobetz will be transported to Slovenia for preservation.

\section{Part 2: Macrostructure}

To date we have identified 11 different Slovenian–English, English–Slovenian reference books and dictionaries targeted at the above-described Slovenian–American immigration. Most of the books were obtained as intact, full copies; in some cases, only excerpts are available. The analysis of the books is ongoing. Here we will discuss six of these books, intended for the Slovenian immigrants who arrived up to World War I. The discussion is focused mostly on the macrostructure, with some glimpses of microstructure. The patterns in macrostructure emerging from the six books examined lead to the expectation that subsequent analyses of the remaining books will not result in many (or any) new macrostructure discoveries.

\subsection{Jeram’s \emph{Slovensko-angleška slovnica} [Slovenian–English Grammar], 1895.}

Peter Josip Jeram (see \figref{vrbinc:fig4}) was born in 1857\footnote{Another source, \citet{Grafenauer2013}, gives a birthdate of around 1865, but this is unlikely.} in the village of Smoleva; he died in Covelo, California in 1897 (\citealt{Cesar-nedzbala1990, Sedmak2021a}). Jeram attended a seminary (St. Francis de Sales, today in the Milwaukee Archdiocese, Wisconsin); he became a priest in 1880. He first practiced as a priest in Minnesota and South Dakota, writing the \emph{Slovensko-angleška slovnica} [Slovenian–English grammar] while in South Dakota, from about 1889 to 1892 (\citealt{Cesar-nedzbala1990, Sedmak2021a, Sedmak2021b, Sedmak2021c}).

\begin{figure}
\includegraphics[width=.33\textwidth]{images/Vrbinc_figure4.jpg}
\caption{Peter Josip Jeram (Archives of the Archdiocese of St. Paul and Minneapolis).}
\label{vrbinc:fig4}
\end{figure}

\citegen{Jeram1895} \emph{Slovensko-angleška slovnica} of 1895 is a reference book rather than a dictionary; it contains some segments with similarities to dictionaries. It is the oldest of the books found for the Slovenian–American immigration. Its place of publication, Tower, Minnesota, is part of a larger area where early Slovenian immigrants settled. Brockway, Minnesota, the first Slovenian settlement (1865) is about a four-hour drive south from Tower (see \tabref{tab:slovenian-settlements}).

The title page of Jeram’s grammar is in Slovenian only with no English (\figref{vrbinc:fig5}), indicative of this early stage in the Slovenian immigration when the native language was still primary for most users. The grammar has all explanations in Slovenian. There is not a full foreword but just a short quote from a well-known Slovenian writer, Janez Trdina.

\begin{figure}
\includegraphics[width=.8\textwidth]{images/Vrbinc_figure5.jpg}
\caption{Title page from \citealt{Jeram1895}.}
\label{vrbinc:fig5}
\end{figure}

Pages three to 18 (16 pp.) contain a very detailed description of English pronunciation. \figref{vrbinc:fig6} shows the treatment of pronunciation irregularities. For example, A) §43.1 discusses the pronunciation of what is called the English vowel \emph{a}, provides examples of English words containing \emph{a} (for example, \emph{have}, \emph{scarce}, \emph{scarcely}, etc. with their translations into Slovenian), and compares the English pronunciation with words in Slovenian containing \emph{e} or \emph{a}. Overall, the explanations here and elsewhere in the pronunciation section were confusing to three trained linguists and most likely would have been incomprehensible to uneducated Slovenian–American users.


\begin{figure}
\includegraphics[width=.8\textwidth]{images/Vrbinc_figure6.jpg}\\
\includegraphics[width=.8\textwidth]{images/Vrbinc_figure6bis.jpg}
\caption{Irregularities of English pronunciation \citep[13--14]{Jeram1895}.}
\label{vrbinc:fig6}
\end{figure}

\largerpage
Part One of the grammar, pages 17–81 (65 pp.), consists of translation exercises; Exercise II is shown in \figref{vrbinc:fig7}. This exercise begins with some general information, presumably that would enable a reader to undertake a translation. The information provided is a list of Slovenian words and their English equivalents related to \emph{Družina} The Family (as the title of the exercise indicates); the verb \emph{imeti} ‘(to) have’ is also presented. The actual exercise that follows consists of English sentences related to the theme of family that use the English words previously presented. The sentences are related to each other by theme but could not be called a paragraph. The learner is apparently expected to translate the sentences into Slovenian, although no instructions are given.

\begin{figure}
\includegraphics[width=.8\textwidth]{images/Vrbinc_figure7.jpg}\\
\includegraphics[width=.8\textwidth]{images/Vrbinc_figure7bis.jpg}
\caption{Part One, Exercise II, thematically-based translation exercise \citep[18--19]{Jeram1895}.}
\label{vrbinc:fig7}
\end{figure}

The backdrop for this exercise format is that translation exercises for language-learning purposes date at least to the bilingual manuals of Ancient Greek and Latin used in third-century Rome; in 18\textsuperscript{th}- and 19\textsuperscript{th}-century Europe, the use of grammar–translation was widespread for language teaching (\citealt{Hernandez_jaramillo2019}; see also \citealt{Zhou2015}. It would not be surprising if Jeram based his book on the grammar–translation methods of his time. However, one reviewer of the \emph{Slovensko-angleška slovnica} \citep{Anonymous1895} linked Jeram’s approach to that of Heinrich Gottfried Ollendorff (also known as Henri Godefroy Ollendorf), whose own method was more conversationally based.

Underneath the English sentences to be translated (19), Jeram presents a rule (\emph{Pravilo}) explaining that English does not decline nouns in the nominative, accusative, singular, or plural. This is followed by a note (in Slovenian): “Look at every word carefully, how it is spelled, if you want to spell English words correctly and if you want to speak nicely and correctly.” The next part of Exercise II contains Slovenian sentences related to the theme of family; apparently the reader is supposed to translate them~-- though again, no instructions are given. 

The other exercises of Part One follow essentially the same pattern as Exercise II; a few of the themes covered by these exercises are listed in \tabref{tab:excercises}. Within each theme, grammar explanations are also provided; for example, modal verbs, tenses, and plural formation, to name a few. Sample correspondence (letters) related to the themes is also given.

\begin{table}
\caption{Part One exercises, a sampling of themes covered \citep{Jeram1895}.}
\label{tab:excercises}
\begin{tabular}{lll}
\lsptoprule
Meals            & The Garden & Minerals \\
Division of Time & The Flower-Garden & Geographical Names \\
The School & Vegetable-Garden & The Universe \\
The Church & The Orchard &  \\
\lspbottomrule
\end{tabular}
\end{table}

Part Two of the book, pages 82–173 (92 pp.) is called Elementary Grammar. Unlike Part One, this section is not organized by conceptual themes but by points of grammar. It is interesting that some titles are in English, others are in Slovenian, and some are in both languages. 

As an example, we can examine Exercise XLVII (p47), shown in \figref{vrbinc:fig8}. The grammar point presented is nouns with irregular plurals; the section begins with a list of such English nouns, usually separated from their Slovenian equivalent by a comma. English \emph{Man} is an exception in that two equivalents are given, separated from each other by a semi-colon due to semantic differences. The first equivalent has the meaning ‘male’ whereas the second one means ‘human being, person’. Similar to what we saw in Part One, the text for translation is not a paragraph, but just a string of sentences, usually with no apparent connection between them. For example, while it is possible to loosely connect: “Wine has drowned more men then [sic] water” with “The more women look into their glasses, the less they look to their homes,” these two sentences are not conceptually linked to: “Geese have very valuable feathers” or “The Dutch are renowned for their cleanliness.” After the sentences provided for translation, we see a short glossary designed to help the user carry out the translation~-- a list of English words and their Slovenian equivalents. Following the list there is a description of the grammatical topic, irregular plurals.

\begin{figure}
\includegraphics[width=.66\textwidth]{images/Vrbinc_figure8.jpg}\\
\includegraphics[width=.66\textwidth]{images/Vrbinc_figure8bis.jpg}
\caption{Part Two, Exercise XLVII, translation exercise linked to grammar \citep[87--88]{Jeram1895}.}
\label{vrbinc:fig8}
\end{figure}

While each section in Jeram’s Part Two usually begins immediately with the sentences to be translated, sometimes the ordering of elements can be altered. A section can sometimes begin with vocabulary or else with the grammatical point being described, with the translation exercises appearing afterwards. Some of the section headings in \citet{Jeram1895} reflect terms for grammatical categories that are not in line with modern usage. To take one example, while today the English term \emph{indefinite pronoun} is used, Jeram has “Undefined Pronouns”; the Slovenian term he used, \emph{nedoločni zaimek}, is the same as in modern terminology.

Following Part Two, there are two pages on weights and measures and two pages of errata. The existence of the errata pages might indicate that there was a printing before the book we have seen; however, we have found no evidence of a previous printing. It is also possible that during the process of preparation of the final book, Jeram or someone else noticed errors which there was no time to correct. At the end of the errata section, Jeram added an interesting final sentence \citep[177]{Jeram1895}: \emph{Druge manjše popravke si učenec lahko sam popravi} (`The learner himself can correct other minor mistakes'). The final two pages are the table of contents.

Jeram died in 1897 at the age of 40, two years after his grammar was published. While he apparently envisaged a dictionary as a companion to the grammar dictionary itself\citep[16]{Jeram1895}, none ever materialized.

\subsection{The dictionaries of Viktor Kubelka}

Viktor J. Kubelka was a New York City-based immigrant. While he was born in Ljubljana in 1879, his last name as well as genealogical sources indicate that he was partially of ethnic Czech origin. He died in Sullivan, New York in 1961.

\subsubsection{Kubelka’s Slovenian–English pocket dictionary to facilitate the study of both languages, Slovensko-angleški žepni rečnik v olajšavo naučenja obeh jezikov, 1904.}

This apparently is the first reference book Kubelka published. The first page is a bilingual Slovenian and English title page, listing Kubelka as editor. Next comes a full-page advertisement (in Slovenian) for Frank Zotti \& Co., headquartered in New York City, which offered banking services (including the possibility for immigrants to send money to the homeland), sold steamer tickets, and provided other services likely to be needed by immigrants. The next page states only (in English): “POCKET DICTIONARY”; the following page states (in English): “Copyright by V. J. KUBELKA. 1904 ALL RIGHTS RESERVED.” Finally, we have two additional pages that correspond to the normal formula for title pages. The first is in English: \emph{Slovenian–English Pocket Dictionary to Facilitate the Study of Both Languages}, and the second is in Slovenian: \emph{Slovensko-angleški žepni rečnik v olajšavo naučenja obeh jezikov}. Unlike \citet{Jeram1895}, where the sole title page is in Slovenian, nine years later in Kubelka’s publication, the English title page has first position, reflecting possibly a transition toward English already beginning as the Slovenian immigration to the US continues.

A preface of one page in English followed by a one-page preface in Slovenian come after the two title pages. The next component is one page with the English alphabet and the pronunciation of each letter, followed on the same page by the Slovenian alphabet. The front matter also contains an explanation of the pronunciation of English vowels, diphthongs, and consonants. 

The central part of this reference book is the dictionary itself (p24–122) (99 pp), in which Slovenian lemmas in column 1 are followed by English equivalents in column 2 and the pronunciation of the English equivalent in column 3. The dictionary’s outer access structure consists of two running heads per page. As seen in \figref{vrbinc:fig9}, the top left corner of the left-hand page contains the first lemma of the left-hand page and the top right corner of the left-hand page contains the last lemma of the left-hand page. 

\begin{sidewaysfigure}
\includegraphics[width=.5\textwidth]{images/Vrbinc_figure9.jpg}\hfill
\includegraphics[width=.5\textwidth]{images/Vrbinc_figure9bis.jpg}
\caption{Two pages from the dictionary section \citep[24,36]{Kubelka1904}.}
\label{vrbinc:fig9}
\end{sidewaysfigure}

The dictionary itself is followed by a section called “Easy dialogus,” written in English (122–131; see \figref{vrbinc:fig10}). It is arranged according to the following topics: Salutations; A Meeting; To make Inquiry; The Necesseries [sic] of Life; and Looking for work.

\begin{figure}
\includegraphics[width=.8\textwidth]{images/Vrbinc_figure10.jpg}
\caption{“Easy dialogus” \citep[127]{Kubelka1904}.}
\label{vrbinc:fig10}
\end{figure}

The next two chapters provide additional information about the US. For example, the names of US states are given along with their abbreviation and pronunciation~-- with the title in Slovenian rather than in English (see \figref{vrbinc:fig11}). 

\begin{figure}
\includegraphics[width=.8\textwidth]{images/Vrbinc_figure11.jpg}
\caption{Names of US states \citep[132]{Kubelka1904}.}
\label{vrbinc:fig11}
\end{figure}

Other information includes money in the US, with the name of a coin or bill, the pronunciation, how to write it, and the value in crowns. Under the topic “Names,” Slovenian first names are followed by their equivalents in English; the pronunciation of the English names is provided (see Figure 12). On the next page of the book is the table of contents (see Figure 13). 

\begin{figure}
\includegraphics[width=.8\textwidth]{images/Vrbinc_figure12.jpg}
\caption{Slovenian first names with English equivalents and pronunciation \citep[135]{Kubelka1904}.}
\label{Figure 12}
\end{figure}

\begin{figure}
\includegraphics[width=.8\textwidth]{images/Vrbinc_figure13.jpg}
\caption{Table of contents, \citep[136]{Kubelka1904}.}
\label{Figure 13}
\end{figure}

After the table of contents there are two pages with Slovenian-language advertisements. One page has an advertisement that contains the name and address of the official representative of different shipping companies as well as information for those wishing to send money back to their homeland. On the next page is an ad for the same Frank Zotti \& Co. advertised in the front matter, who provides tickets for “the best, the quickest, the safest and the cheapest steamers in the world.” This advertisement lists ports in Europe, the US, South Africa, Asia, and Australia; also included is the purchase of train tickets.

\subsubsection{Kubelka’s \emph{Slovensko-angleški razgovori}, Slovenian–English Interpreter, 1912a.}

\citet{Kubelka1912a} is a small phrasebook rather than a dictionary proper. The entire book has 136 pages, as compared to \citet{Kubelka1912b} (below) which has 431 pages including the ads. The phrasebook starts with a title page in Slovenian and English, a copyright page, and a table of contents in Slovenian. The copyright page bears the notation: “Press of Stettiner Bros., New York.” These Austrian printers and brothers worked out of the building that is now New York City Hall.

Most immigrants of this era were ignorant of the languages of the countries through which they traveled, and few spoke English. Instructions written in their native language were accordingly more than welcome. While this work is just a phrasebook, its importance for a new immigrant should not be underestimated. As was emphasized by \citet{Drnovsek2006}: “Even on Ellis Island (New York), many emigrants were given bilingual handbooks such as … Kubelka’s … (1912). These contained simple dialogues, which immigrants needed to know in order to communicate in their new environment” (p10).

There are 16 different topics identified in the table of contents; some examples are: \emph{The Voyage; Forwarding of Baggage; The Rail Road: At the Depot; On Ellis Island; Looking for Work; Taking Lodgings}. After the list of topics, the central part of the phrasebook follows, in which each topic is treated. As one would expect from a phrasebook of this type, each topic contains typical conversational phrases, short dialogs, or other texts considered useful for Slovenian immigrants. In Column 1, Slovenian phrases are listed; Column 2 provides translations of the Slovenian phrases into English, and Column 3 contains the pronunciation of the English phrases~-- this is the same organization we have already seen in \citet{Kubelka1904}. An example of a dialog that might have been used at Ellis Island is shown in Figure 14.

\begin{figure}
\includegraphics[width=\textwidth]{images/Vrbinc_figure14.jpg}
\caption{On Ellis Island \citep[9]{Kubelka1912a}.}
\label{Figure 14}
\end{figure}

The pronunciation presented is neither IPA nor a respelling, but a spelling geared toward a native speaker of Slovenian with no knowledge of the English phonological system. It is very simplified; the American English phonemes not present in Slovenian (e.g., /θ/, /ð/, /æ/, /ŋ/) are completely ignored. 

At the end of the phrasebook, there is a full-page advertisement for the Austro-Americana Line (with an office address in Trieste). On the back cover is a picture of a steamer ship and a list of steamers that sail between Trieste and New York, along with the deadweight tonnage (DWT) of each (see Figure 15).

\begin{figure}
\includegraphics[width=.9\textwidth]{images/Vrbinc_figure15.jpg}
\caption{Back cover \citep{Kubelka1912a}.}
\label{Figure 15}
\end{figure}

\subsubsection{Kubelka’s \emph{Slovensko-angleška slovnica, tolmač, spisovnik in navodilo za naturalizacijo. Angleško-slovenski in slovensko-angleški slovar}, Slovenian–English Grammar Interpreter, Letterwriter and Information on Naturalization. English–Slovenian and Slovenian-English Dictionary, 1912b.}

\citet{Kubelka1912b} is the most comprehensive of Kubelka’s three reference books both in terms of its structure and content. As was previously mentioned, the book has 431 pages including the ads. After the two title pages, first in Slovenian and then in English, there is a preface in Slovenian written in August 1912. In it, the author notes that because the demand for his various Slovenian–English books published in the previous eight years has been so great, he decided to write a new, improved, and more extensive book. The new book, Kubelka says, is the most complete book Slovenians have for learning English. Kubelka maintains that “The book is organized so that it meets every single demand and \textbf{\emph{everyone}} using it will most certainly be satisfied” (p9).

The book is divided into the following six parts: English Grammar, Every-Day Conversations, Slovenian–English Letterwriter, Information on Naturalization, English–Slovenian Dictionary, Slovenian–English Dictionary. The first part (English Grammar, 10–86) provides essentially the same type of information that we saw in \citet{Jeram1895}. While the format does not correspond exactly to that of Jeram, it is strikingly similar. Like Jeram, all of the explanations in Kubelka’s English Grammar section appear in Slovenian. This section treats: alphabet (containing English and Slovenian alphabets), the pronunciation of vowels, diphthongs and consonants, articles, nouns, verbs, adjectives, numerals, as well as many more topics. The internal organization of each grammar topic follows a similar pattern. For example, the topic dealing with articles (27) contains a brief discussion (in Slovenian) of both the definite and indefinite articles, followed by one translation exercise. The translation exercise begins with a short glossary with words that the learner can make use of in the sentences to be translated. The sentences to be translated follow a theoretical discussion and are divided into two paragraphs. The first paragraph has Slovenian sentences to be translated into English, and the second paragraph has English sentences to be translated into Slovenian. Like in Jeram, the sentences are semantically unrelated and form neither a true paragraph nor a narrative.

While, as mentioned, the internal organization of each grammar topic follows a similar pattern, not all of the topics have an identical structure and the same level of complexity. Unlike the Article section with no subsections, the Verb is a very long section (50–76) that contains four subsections: Auxiliary Verbs, Regular Verbs, Conjugation of Regular Verbs, and Irregular Verbs.

The second part of the book (p88–146) is entitled Everyday Conversations. The structure here resembles that of \citet{Kubelka1904} in terms of topics; however, \citet{Kubelka1912b} has a wider variety of topics. The content of this part of \citet{Kubelka1912b} is not identical to the earlier book, indicating a reworking of the 1904 material. In \citet{Kubelka1912b}, some phrases are identical to \citet{Kubelka1904}, others differ slightly, and some are new. The third part of the book, with the title Slovenian–English Letterwriter (147–191), contains guidance on how to write a letter in English, including beginnings, endings, salutations, what to write on an envelope, numerous sample letters, etc. Then follows Information on Naturalization (192–208), where the naturalization procedure is described, including questions that could be posed by a judge to a person applying for American citizenship.

As in \citet{Kubelka1904}, the central part (Parts 5 and 6) of this more comprehensive book is a bidirectional dictionary: English–Slovenian Dictionary, Slovenian–English Dictionary. Unlike \citet{Kubelka1904}, there is no outer access structure at all, no running heads to represent the first or the last lemmas on a page. While an outer access structure is not used, the new dictionary is more sophisticated than \citet{Kubelka1904}, which (as we have seen) is organized into three columns per page: lemma + equivalent(s) + pronunciation. \citet{Kubelka1912b} looks more like a familiar 20\textsuperscript{th}-century dictionary: Each page has two columns, each of which contains a list of entries.

The English–Slovenian portion of the dictionary contains 85 pages (210–294). The structure, as shown in Figure 16, is as follows: The English word (which does not appear in boldface) is followed by a comma, followed by Slovenian equivalents written in italics and separated by commas. 

\begin{figure}
\includegraphics[width=.8\textwidth]{images/Vrbinc_figure16.jpg}
\caption{{Part 5, English–Slovenian Dictionary \citep[254]{Kubelka1912b}.}}
\label{Figure 16}
\end{figure}

It is noteworthy that no semi-colons at all are used to separate the equivalents. The ordering of the equivalents appears to be by part of speech; however, it also appears that this ordering is not entirely consistent. Nouns and verbs are listed first and second, but there is fluctuation in the ordering of adjectives and adverbs. For example, \emph{yell} has three equivalents; the first two are nouns (\emph{krik, kričanje}) followed by one verb (\emph{kričati}). The ordering of adverb and adjective differs in the entries for \emph{round} and \emph{wrong}. \emph{Round} has four equivalents separated by commas in the following order: noun (\emph{krog}), verb (\emph{krožiti}), adverb and preposition (\emph{okolo}),\footnote{This is an obsolete Slovenian word used as an adverb and preposition. It is not included in any contemporary Slovenian reference books but appears in a 16\textsuperscript{th}-century Slovenian dictionary and in a late 19\textsuperscript{th}-century bilingual Slovenian–German dictionary.} and finally adjective (\emph{okrogel}). The word \emph{wrong} has four equivalents, again separated by commas, in the order: noun (\emph{krivica}), two adjectives (\emph{neprav, krivičen}) and adverb (\emph{narobe}).

The Slovenian–English part of the dictionary consists of 128 pages (p296–423). The Slovenian word appears first in the entry in boldface~-- unlike the English words in the English–Slovenian section which are not in boldface. After the Slovenian word, a comma follows and then the English equivalents (also not in boldface). Immediately following each English equivalent is the pronunciation of the English word in parentheses. Each pronunciation is followed by a comma, and if applicable, another English equivalent. The pronunciations follow the same system used in \citet{Kubelka1904}: neither IPA nor a respelling, but a spelling that Slovenian speakers are likely to understand. Just like in the English–Slovenian section of \citet{Kubelka1912b}, no semi-colons are used to separate equivalents.

The ordering of equivalents in the Slovenian–English section follows a different system as compared to the English–Slovenian section. This is due to the difference in the two languages. While a single English word can have Slovenian equivalents of several parts of speech, Slovenian words are clearly marked in their morphology for part of speech, so that English equivalents are all a single part of speech. Because of the lack of ambiguity in Slovenian words regarding part of speech, there are usually fewer English equivalents for each Slovenian word, as compared to the number of Slovenian equivalents for each English word. When, infrequently, there is more than one English equivalent listed for a Slovenian lemma, the ordering of the equivalents does not appear to follow any discernible rules~-- semantic, alphabetical, or otherwise. For example, the Slovenian verb \emph{zmenjati} has two English equivalents: \emph{to exchange for}, and \emph{to change}. Note that they are not listed in alphabetical order; it is not possible to understand whether there is a semantic motivation for the ordering.

After the Slovenian–English dictionary, there is a table of contents in Slovenian and six advertisements. The first advertisement is for the Austro-Americana Line, although this is not the same advertisement that appeared in \citet{Kubelka1912a} (and a New York address is now included). Another advertisement is for the main representative of the Austro-Americana Line in the west, with an address in Chicago. The third advertisement (see Figure 17) is for The National Herald [\emph{Narodni vestnik}] weekly newspaper and for its publisher, the Slovenian Print \& Publishing Company [Slovenska tiskovna družba] in Duluth, Minnesota. There is also an advertisement for another newspaper, The People’s Voice [\emph{Glas naroda}], which calls itself “the \textbf{\emph{only}} Slovenian newspaper in America” (despite the ad in the same publication for The National Herald). There is also an ad for Kubelka himself, publisher of Slovenian–English books. On the back cover is, again, as in \citet{Kubelka1912a}, a picture of a steamer and a list of steamers sailing between Trieste and New York.

\begin{figure}
\includegraphics[width=.9\textwidth]{images/Vrbinc_figure17.jpg}
\caption{Advertisement from the final pages \citep{Kubelka1912b}.}
\label{Figure 17}
\end{figure}

\subsubsection{Košutnik’s Ročni slovensko-angleški in angleško slovenski slovar: Zlasti namenjen izseljencem v Ameriko [A Pocket Slovenian–English and English–Slovenian Dictionary: Intended for Immigrants to America], 1912.}

Silvester Košutnik’s dictionary is the sole book being examined that was published outside of the US (Ljubljana: Anton Turk), although clearly it was intended for Slovenians planning to emigrate. The title page and the copyright page (Figure 18) are followed by a preface in Slovenian. According to the preface: “… in this dictionary, the most common words are collected, i.e., those that are used most frequently in everyday life” (p1). The preface contains an explanation of the pronunciation of English. 


\begin{figure}
\includegraphics[width=\textwidth]{images/Vrbinc_figure18.jpg}
\caption{{Title page \citep{Kosutnik1912}.}}
\label{Figure 18}
\end{figure}

The dictionary that follows is a bidirectional dictionary, first 59 pages of Slovenian–English (5–63), and then 85 pages of English–Slovenian (64–148). Page 149 contains a list of books published by Anton Turk, the publisher of this dictionary.

The Slovenian–English dictionary (Figure 19) has a very simple microstructure: lemma in Slovenian, dash, equivalent in English, comma if there is another equivalent, period at the end. The English–Slovenian dictionary (Figure 20) is about 25 pages longer than the Slovenian–English section. It has the following structure: English lemma, comma followed by a simplified pronunciation (with primary word stress provided as an apostrophe immediately after the stressed syllable), then a dash and the equivalent(s) separated by a comma, even if they are not synonyms. In both the Slovenian–English and the English–Slovenian parts, a semi-colon is used to separate different parts of speech. For example, in Figure 20, English \emph{holy} is followed by the Slovenian adjective \emph{svet}, which is separated by a comma from the adjective \emph{prazničen}, which means more ‘holiday’ than ‘holy’. These two adjectives are separated by a semi-colon from the noun \emph{praznik} meaning ‘holiday’. 

On the other hand, note the entry for \emph{home} (Figure 20), which diverges from the dictionary’s usual procedure with semi-colons. Here, \emph{dom} and \emph{domovje}, both nouns meaning ‘home’, are separated from the two following nouns~-- the same part of speech~-- meaning ‘apartment’ and ‘dwelling’. Uncharacteristically, the semi-colon is being used to indicate a semantic difference. The final three words in the entry are an adjective and two adverbs. \emph{Domač}, an adjective, means ‘domestic’; a semi-colon is not used to separate it from the adverb \emph{domu} ‘going toward home’, or from the adverb \emph{doma} ‘at home’.

\begin{figure}
\includegraphics[width=\textwidth]{images/Vrbinc_figure19.jpg}
\caption{Slovenian–English dictionary, \citep[56]{Košutnik1912}.}
\label{Figure 19}
\end{figure}

\begin{figure}
\includegraphics[width=\textwidth]{images/Vrbinc_figure20.jpg}
\caption{English–Slovenian dictionary, \citep[103]{Košutnik1912}.}
\label{Figure 20}
\end{figure}

\subsubsection{Kern’s A Complete Pronouncing Dictionary of the English and Slovene Languages for General Use, \emph{Popoln angleško-slovenski besednjak z angleško izgovarjavo, 1919Popoln angleško-slovensk}.}

Fran/Frank Javh Kern (Figure 21) was born in 1887 in Breznica and died in Cleveland, Ohio in 1977. He arrived in the US at the age of 16 (1903) to attend the Saint Paul Seminary in Saint Paul, Minnesota (established 1894). He brought with him a good gymnasium education from his home country. He later became a physician in Cleveland as well as a prominent community member.

\begin{figure}
\includegraphics[width=\textwidth]{images/Vrbinc_figure21.jpg}
\caption{Frank/Fran Javh Kern.}
\label{Figure 21}
\end{figure}

Kern’s A Complete Pronouncing Dictionary of the English and Slovene Languages for General Use, \emph{Popoln angleško-slovenski besednjak z angleško izgovarjavo} \citep{Kern1919}, was published in 1919, at a time when immigration to the US (and worldwide) was generally slowing due to WWI. At that time, the number of people of Slovenian origin in the US was quite high, especially in Cleveland, where Kern lived. 

Kern lists the English title first on the title page (Figure 22). 

\begin{figure}
\includegraphics[width=.8\textwidth]{images/Vrbinc_figure22.jpg}
\caption{Title page \citep{Kern1919}.}
\label{Figure 22}
\end{figure}

A preface in English (III) and another in Slovenian (IV–V) immediately follow the title page and the copyright page. Kern’s treatment of pronunciation is explained in the English preface:

\begin{quote}
    … I have followed the latest authorities, particularly the New Standard Dictionary,\footnote{The first edition of the Funk and Wagnalls dictionary with “new” in the title appeared in 1913.} which employs the phonetic method of indicating the pronunciation, and which is similar to the phonetic system of spelling and writing employed in the Slovene (Slovenian) language. (III)
\end{quote}

In the Slovenian preface, Kern explained how he selected Slovenian equivalents; he considered 

\begin{quote}
    … the general use and most contemporary books. In technical and scientific terms Slovenian terminology is not determined precisely enough for me to be able to get an appropriate term in each and every case. In more important words, I added several synonyms (similar expressions) for explanation. (1919, IV).
\end{quote}

Also noted in the Slovenian preface: 

\begin{quote}
    Spelling and pronunciation of English words follows the American tradition and is somehow different from continental English. 
\end{quote}

Most likely Kern is the first among the authors addressed here to notice that British and American spelling and pronunciation differ.

Following the two prefaces, on pages VI–VII, we find a user’s guide (\emph{Pojasnila}) which mostly treats pronunciation; there is a very brief clarification of how to use the dictionary (an explanation of running heads and alphabetical arrangement of entries) and a brief note on compounds. 

The Kern dictionary has two-column pages with running heads (i.e., an outer access structure). Each page has two running heads consisting of three letters (Figure 23). The left running head contains the first three letters of the first lemma on the page; as would be expected, the first three letters of the page’s last lemma appear in the running head on the right. Among all the works examined here, only \citet{Kubelka1904} and Kern have running heads. Kern’s English lemma is printed in boldface and is followed by a comma; next, pronunciation appears in parentheses followed by a comma; finally, we have the Slovenian equivalent(s).

\begin{figure}
\includegraphics[width=\textwidth]{images/Vrbinc_figure23.jpg}
\caption{\citealt[258]{Kern1919}.}
\label{Figure 23}
\end{figure}

The use of punctuation in Kern is more systematic than, for example, in \citet{Kosutnik1912}. There are two uses for semi-colons. First, they separate equivalents of the same part of speech when there are perceived semantic differences. For example (Figure 23), for English utter, four adjectives appear as equivalents: \emph{skrajen} ‘extreme’, \emph{prevelik} ‘too big’, \emph{popoln} ‘complete’, and \emph{cel} ‘complete, whole’. The first two, \emph{skrajen} and \emph{prevelik}, are separated by semi-colons due to their perceived semantic differences. Kern clearly considered the last two adjectives, \emph{popoln} and \emph{cel}, as closer semantically; so, they are separated by a comma. We could quibble over Kern’s choice of adjective equivalents or his judgments about semantic closeness, but it is clear that he is employing a systematic approach. 

While part of speech is not listed in Kern’s dictionary, the second use for the semi-colon is to separate equivalents of different parts of speech. For example, in Kern’s \emph{utter} (Figure 23), two equivalents~-- verbs~-- follow the adjective \emph{cel} ‘complete’: \emph{izreči} ‘say’ and \emph{govoriti} ‘talk, speak’. These are separated by a semi-colon from the preceding adjective \emph{cel}. It is interesting that a semi-colon is not used to separate the two verbs from each other; instead, Kern uses a comma~-- despite the fact that the two verbs are not synonyms. 

At the end of the dictionary, we can find a table of the irregular verbs spanning the final three pages (p271–273).

\section{Conclusion}

The dictionaries and other reference works examined here were designed to help the large population of Slovenian immigrants, “liberated from the Hapsburg yoke” (Kern, 1919, III) in the late 19\textsuperscript{th} and early 20\textsuperscript{th} centuries, to learn English and acculturate to their strange new American environment. The present effort has focused on six books; there are others to be analyzed and likely still others that have not yet been identified. While we have mainly examined macrostructure, there is much more to be said about the microstructure, beyond the parsimonious remarks above. The investigation so far leads us to the following reflections and preliminary conclusions.

Some of the books are relatively standard 20\textsuperscript{th}-century print dictionaries (\citealt{Kern1919} and \citealt{Kosutnik1912}) that look familiar to a modern eye. On the other hand, other books are more hybrid works \citep{Kubelka1904}, \citep{Kubelka1912a}, and \citep{Kubelka1912b} that aimed to meet the needs of a specific time and place and did not concern themselves with replicating traditional scholarly models (if they were even known). In a class by itself, \citet{Jeram1895} grammar, while apparently following prevailing language-learning models popular at the time, may have been rather mysterious to the mostly uneducated Slovenian immigrants who used it; it certainly does not contain exercises of the type that are given to language learners today.

However we may regard the quality of these reference works today, all indications are that their authors were motivated by a sense of duty to a population in need. Some of the books contained language in the form of dialogs on topics that would have been quite relevant in everyday conversations with American English native speakers. What is more, these books gave Slovenian immigrants instructions on how to become a naturalized citizen~-- not an easy thing to accomplish in any era of American history. As we have seen, these books were constructed quite simply for the most part (\citealt{Jeram1895} is perhaps the exception). But if you are uneducated or poorly educated and not in the habit of using any dictionary or reference work at all, isn’t simplicity better?

Finally, many of these books contained advertisements. Likely the ads themselves had the purpose of offsetting the books’ publishing costs, but it is also evident that the types of ads included would have been useful to the target audience. The ads for US–Slovenian-language newspapers pointed their public toward consequential sources of social cohesion for people thrust into a new milieu. The ads for steamer ships helped orient people planning to emigrate or hoping to bring relatives to the United States. Ads also helped people maintain contact with those back in Slovenia, because they provided help in sending money home.

The books examined here and others of this type targeted toward Slovenian immigrants were published in very small print runs. Today, they are dispersed in libraries and archives of varied size, sometimes in paper and sometimes in digital format. Not all (it seems) have been digitized and some lie in poor condition, in little-used, dusty collections. Their continued study and preservation is imperative, and this is why our voyage through Slovenian–American lexicography must continue.

\section*{Acknowledgements}
It is our pleasure to acknowledge the Slovenian Research Agency for providing funding for the grants \textit{Slovarji in izkušnje Slovencev–emigrantov v Združenih državah Amerike (konec 19. in začetek 20. stoletja)/Dictionaries and the Slovenian Immigrant Experience in the United States (Late 19th, early 20th Century)} [BI-US/22-24-042] and \textit{Stare besede, nove besede, novi svet: Življenje slovenskih leksikografov - priseljencev v ZDA/Old Words, New Words, New World: The Lives of Slovenian Immigrant Lexicographers in the U.S.} [BI-US/22-24-030]. 

{\sloppy\printbibliography[heading=subbibliography,notkeyword=this]}

\end{document}
