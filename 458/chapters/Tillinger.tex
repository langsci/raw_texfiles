\documentclass[output=paper,colorlinks,citecolor=brown,arabicfont,chinesefont,booklanguage=french]{langscibook}
\ChapterDOI{10.5281/zenodo.15394503}
\author{Gábor Tillinger\affiliation{Uppsala University, Sweden}}
\title[Les exemples dans les dictionnaires français–hongrois]
      {Les exemples dans les dictionnaires français–hongrois à travers les siècles}

\abstract{The aim of this chapter is to give an overview of the most important French–Hungarian general dictionaries, published since 1844, as well as to present a comparative study of their microstructure, especially from the point of view of the different types of examples provided in the entries. The study focuses on entries of frequent and semantically overloaded verbs, namely \emph{aller, avoir, devoir, être, faire, laisser, mettre, pouvoir, prendre, rendre} and \emph{venir}, the use of which is often illustrated by numerous examples – due to the polysemy and high frequency of these verbs in different contexts. Among other things, it can be noted that the evolution of the entries over time does not automatically mean improvement. The study also comments on the choice and the quantity of examples in the dictionaries, while analysing the functions and the purpose of the different examples. As for the quantity of examples, the analyses include a comparison with other bilingual dictionaries having French as the source language. Finally, some entries are analysed in more detail to illustrate the difficulties arising from the fact that Hungarian and French are not related languages.}

\IfFileExists{../localcommands.tex}{
  \addbibresource{../localbibliography.bib}
  % add all extra packages you need to load to this file

\usepackage{tabularx,multicol}
\usepackage{url}
\urlstyle{same}

\usepackage{listings}
\lstset{basicstyle=\ttfamily,tabsize=2,breaklines=true}

\usepackage{langsci-basic}
\usepackage{langsci-optional}
\usepackage{langsci-lgr}
\usepackage{langsci-osl}
% \usepackage{./langsci/styles/langsci-lgr}
% \usepackage{./langsci/styles/langsci-osl}
% \usepackage{langsci-gb4e}

\usepackage{tikz}
\usetikzlibrary{patterns,calc}
\pgfdeclarepatternformonly{south east lines}{\pgfqpoint{-0pt}{-0pt}}{\pgfqpoint{3pt}{3pt}}{\pgfqpoint{3pt}{3pt}}{
    \pgfsetlinewidth{0.6pt}
    \pgfpathmoveto{\pgfqpoint{0pt}{3pt}}
    \pgfpathlineto{\pgfqpoint{3pt}{0pt}}
    \pgfpathmoveto{\pgfqpoint{.2pt}{-.2pt}}
    \pgfpathlineto{\pgfqpoint{-.2pt}{.2pt}}
    \pgfpathmoveto{\pgfqpoint{3.2pt}{2.8pt}}
    \pgfpathlineto{\pgfqpoint{2.8pt}{3.2pt}}
    \pgfusepath{stroke}}
    
\usepackage{stmaryrd}
\usepackage{wasysym}
\usepackage{multirow}
\usepackage{caption}
\usepackage{subcaption}
\usepackage{mathrsfs}
\usepackage{qtree}

\usepackage{linguex}


  %pminos do not split footnotes
% \interfootnotelinepenalty=10000 %Footnote in Laporte chapters has to be split SN


%\DeclareIndexNameFormat{default}{%
%\nameparts{#1}%
%\usebibmacro{index:name}%
%{\index[names]}%
%{\namepartfamily}%
%{\namepartgiveni}%
% {}% L1
% {}% L2
%{\namepartprefix}% generates spurious space L3
%{\namepartsuffix}% generates spurious space L4
%}

%  {\DeclareIndexNameFormat{default}{%
%     \usebibmacro{index:name}{\index[names]}{#1}{#3}{#5}{#7}}}

%\DeclareIndexNameFormat{default}{%
%  \usebibmacro{index:name}{\sindex[nom]}{#1}{#3}{#5}{#7}}

%\DeclareIndexNameFormat{default}{%
%  \usebibmacro{index:name}{\sindex[person]}{#1}{#3}{#5}{#7}}
%\DeclareIndexNameFormat{default}{%
%\nameparts{#1} \usebibmacro{index:name}{\sindex[person]]}{\namepartfamily}{‌​\namepartgiven}{\nam‌​epartprefix}{\namepa‌​rtsuffix}}

%\newcommand{\smiley}{:)}

%\renewbibmacro*{index:name}[5]{%
%\usebibmacro{index:entry}{#1}%
%{\iffieldundef{usera}{}{\thefield{usera}\actualoperator}\mkbibindexname{#2}{#3}{#4}{#5}}}

% \newcommand{\noop}[1]{}

%remove for final
%\overfullrule=1mm

\newcommand{\tobi}[2]}}
\renewcommand{\S}[1]{\tobi{#1}{\textsc{*}}}

% this volume references
% puts: [this volume]
% already defined: \citetv
%\newcommand{\citepv}[1]{(\citeauthor{#1} \citeyear*{#1} [this volume])}
\newcommand{\citealtv}[1]{\citeauthor{#1} \citeyear*{#1} [this volume]}

%parentheses around example number
\newcommand{\pref}[1]{(\ref{#1})}

% in-text examples

\newcommand{\lnex}[1]{\textit{#1}} %target lang word
\newcommand{\lnlit}[1]{(lit.: `#1')} %literal reading
\newcommand{\lnlat}[1]{(#1)} % latinization
\newcommand{\lntrans}[1]{`#1'} %translation
\newcommand{\lnexl}[2]%
{\lnex{#1}{} \lnlat{#2}} % ex with latinization
\newcommand{\lnexlat}[3]{\lnex{#1}{} \lnlat{#2}{} \lntrans{#3}} % ex with latinization and tranl.

%ch01
\newcommand{\co}[1]{\mbox{\textbf{#1}}}

%ch09

\newcommand{\cyrbulg}[1]{\begin{otherlanguage*}{bulgarian}#1\end{otherlanguage*}}


%ch10
\newcommand{\nlp}{{\small NLP}}
\newcommand{\mwe}{{\small MWE}}
\newcommand{\rae}{{\small RAE}}
\newcommand{\lvc}{{\small LVC}}
\newcommand{\pos}{{\small P}o{\small S}}
%\newcommand{\todo}[1]{ \textcolor{red}{#1} }

%\renewcommand{\labelenumi}{\theenumi}
%\ainamefmt{{vv}{ll}{, ff}{, jj}} % fullname

\newcommand{\biberror}[1]{{\color{red}#1}}

\newcommand{\osenovaitem}{--~}
  %% hyphenation points for line breaks
%% Normally, automatic hyphenation in LaTeX is very good
%% If a word is mis-hyphenated, add it to this file
%%
%% add information to TeX file before \begin{document} with:
%% %% hyphenation points for line breaks
%% Normally, automatic hyphenation in LaTeX is very good
%% If a word is mis-hyphenated, add it to this file
%%
%% add information to TeX file before \begin{document} with:
%% %% hyphenation points for line breaks
%% Normally, automatic hyphenation in LaTeX is very good
%% If a word is mis-hyphenated, add it to this file
%%
%% add information to TeX file before \begin{document} with:
%% \include{localhyphenation}
\hyphenation{
    Beck-man
    Ngu-yen
    back-chan-nel
    back-chan-nels
    mo-not-o-nous
    ste-reo-typ-i-cal
}

\hyphenation{
    Beck-man
    Ngu-yen
    back-chan-nel
    back-chan-nels
    mo-not-o-nous
    ste-reo-typ-i-cal
}

\hyphenation{
    Beck-man
    Ngu-yen
    back-chan-nel
    back-chan-nels
    mo-not-o-nous
    ste-reo-typ-i-cal
}

  \togglepaper[19]%%chapternumber
}{}

\begin{document} 
\begin{otherlanguage}{french}
\maketitle

\renewcommand{\tabref}[1]{Tableau~\ref{#1}}
\section{Introduction}\label{sec:tillinger:1}

\subsection{L'arrière-plan}\label{sec:tillinger:1.1}

De nos jours, les Hongrois francophiles, ainsi que les Francophones intéressés par la langue hongroise, connaissent de manière générale trois dictionnaires français–hongrois volumineux (de grand format ou de taille moyenne)~: le grand dictionnaire \citet{EckhardtOlah1999}, le dictionnaire de \citet{Palfy1999} et celui de \citet{BardosiSzabo2007}. Les philologues et les générations plus âgées connaissent également au moins une des nombreuses éditions du grand dictionnaire d’Eckhardt, publié pour la première fois en 1953~– soit le prédécesseur du grand dictionnaire paru en 1999, une version complètement remaniée et complétée sous la direction de Tibor Oláh.\footnote{En Hongrie, le nom de famille précède le prénom. Puisque notre texte est écrit en français, nous avons changé l'ordre des noms de personne hongrois.} Le premier grand dictionnaire français–hongrois rédigé par Aurélien Sauvageot n’est connu désormais que par les érudits ou les francophiles les plus engagés. En ce qui concerne les dictionnaires plus anciens, publiés dans le XIXe siècle ou au début du XXe siècle, ils ont presque complètement disparu dans les oubliettes de l’Histoire. Bien évidemment, cela n’est pas un phénomène unique, en revanche, le fait que les dictionnaires français–hongrois les plus anciens soient quasi inconnus même parmi les philologues est tout de même dommage.

\subsection{Les objectifs}\label{sec:tillinger:1.2}

Nous avons pour objectifs de faire le panorama des dictionnaires français–hongrois les plus importants~– parus dans les XIXe, XXe et XXIe siècles, compte tenu seulement des dictionnaires généraux –, et de présenter notre étude comparative, en mettant l’accent sur la microstructure et surtout sur les différents types d’exemples fournis dans certaines entrées. Il est important de mettre en évidence que les dictionnaires hongrois–français ne font pas partie de cette étude. Notre but principal consiste donc à présenter les dictionnaires français–hongrois les plus significatifs – avec plus de détails pour ceux parus jusqu’au début du XXe siècle –, énumérer leurs caractéristiques, tout en décrivant les changements constatables au fil du temps. Ce faisant, nous pourrons aussi montrer comment certains traits typiques d’une langue de «~l’Ouest~» sont rendus par des équivalents dans une langue de «~l’Est~».

\section{Matériels et méthodes}\label{sec:tillinger:2}

\subsection{Matériels~: Les dictionnaires français–hongrois}\label{sec:tillinger:2.1}

Dans cette partie, nous présentons séparément tous les dictionnaires français–hongrois dont nous nous sommes servis dans nos analyses. Nous nous concentrons sur les dictionnaires généraux (c’est-à-dire les ouvrages n’ayant pas pour but de couvrir seulement un lexique spécialisé), et les plus importants du point de vue de leur volume, de leur contenu et de leur rôle dans l’histoire de la lexicographie franco-hongroise. En fin de cette partie, nous précisons aussi les types d’ouvrages exclus de notre étude.

\subsubsection{\citealt{Kiss1844}} \label{sec:tillinger:2.1.1}

Celui de Mihály Kiss (Michel Kiss sur la page de titre française) est le premier véritable dictionnaire bilingue avec le français en tant que langue source et le hongrois comme langue cible. Avant 1844, il existait quelques grammaires françaises destinées aux Hongrois, ainsi que d’autres ouvrages plurilingues, qui, entre autres, comprenaient des listes de vocabulaire avec des équivalents français et hongrois, mais c’est le dictionnaire de Kiss qui est le premier à inclure seulement le français et le hongrois. L’année 1844 est par ailleurs très importante dans l’histoire de la langue hongroise, étant donné que c’est cette année-là qu’elle est devenue la langue officielle unique en Hongrie.

Sur la page de titre, on trouve le titre «~Nouveau dictionnaire de poche français–hongrois et hongrois–français~». L’épithète «~nouveau~» peut paraître surprenant, compte tenu du fait que c’est le premier dictionnaire français–hongrois. Dans ce cas – tout comme dans le cas de plusieurs ouvrages publiés avant le XXe siècle –, le mot «~nouveau~» indique la nouveauté de l’ouvrage dans le sens où il s’agit soit d’une œuvre auparavant inexistante, soit d’une publication récente. Cette épithète ne s’utilisait pas forcément pour indiquer la nouvelle édition d’un livre, et, en quelque sorte, c’est aussi un moyen de la part de la maison d’édition d’attirer l’attention des clients. Ce qui de plus peut être fallacieux, d’après les informations de la page de titre, c’est que l’on s’attendrait à deux parties différentes (français–hongrois et hongrois–français) parues dans un même volume, ou bien au moins publiées parallèlement, en même temps, en la même année. En revanche, ce que l’on trouve dans l’ouvrage publié en 1844, c’est seulement la partie français–hongrois (rédigée donc par Kiss), alors que la partie hongrois–français a paru en 1848, rédigée par Ignácz Karády (Ignace Karády sur la page de titre française). 

Karády mentionne dans sa préface courte, écrite seulement en hongrois, que l’éditeur lui a confié la rédaction de la partie hongrois–français, puisque Kiss n’avait pas continué son travail – peut-être en raison de certains empêchements, ajoute-t-il – et l’éditeur souhaitait publier la seconde partie le plus tôt possible. On peut donc supposer que – selon l’intention originelle de l’éditeur – Kiss aurait aussi dû rédiger la partie hongrois–français, et la parution de la seconde partie n’aurait pas dû tarder quatre ans.

Dans la partie français–hongrois, on ne trouve aucune préface, aucune information sur le travail de la rédaction. On ne peut que supposer que Kiss s’est servi de dictionnaires français–allemand en tant que point de départ, mais cela n’est pas sûr du tout. D’ailleurs, on ne peut rien savoir sur les sources de Karády non plus. 

Il est aussi intéressant de remarquer qu’aucun exemplaire de la première édition n’est accessible en Hongrie, même la Bibliothèque nationale de Hongrie ne possède que la deuxième édition de 1852 – qui est d’ailleurs une simple réimpression, le contenu des deux éditions est identique. Nous avons travaillé avec une copie de la première édition issue du British Museum. 

Selon certaines sources (voir p.ex. \citealt{Perrot1992} ~: 95), il existerait une troisième édition aussi, parue en 1865, mais nous n’en avons trouvé aucune trace. Ce fait a été également souligné par \citet[80]{Szabo2011} qui se demandait s’il s’agissait du même dictionnaire que celui de Babos. Perrot en écrit ainsi~:

\begin{quote}
    II avait existé plusieurs dictionnaires hongrois-français et français-hongrois dont certains étaient très estimables, notamment celui de KISS et KARÁDY qu'avait publié l'éditeur Heckenast à Pest et dont il faut en particulier citer la troisième édition «~revue et considérablement augmentée~», publiée en 1865 par BABOS Kálmán sous le titre de \emph{Nouveau dictionnaire hongrois–français et français–hongrois d'après les ouvrages publiées} [sic] \emph{jusqu'à ce jour}. (\citealt{Perrot1992} ~: 95)
\end{quote}

D’après les données fournies par Perrot, nous supposons que Perrot a pris – incorrectement, à notre avis – le dictionnaire rédigé (et non seulement «~publié~») par Babos pour une troisième édition du dictionnaire de Kiss et de Karády. En revanche, si l’on lit attentivement la préface du dictionnaire de Babos, on peut mieux comprendre la situation – pour notre solution, voir \sectref{sec:tillinger:2.1.2}. Il faut aussi remarquer que Perrot ne cite pas correctement le titre du dictionnaire paru en 1865 – pour le titre correct, voir nos références bibliographiques ou \sectref{sec:tillinger:2.1.2}. Malheureusement, on ne peut pas savoir pour quelles raisons Perrot considérait le dictionnaire publié en 1865 le plus estimable parmi tous les dictionnaires parus avant celui de Sauvageot en 1932.

Si ce dictionnaire s’intitule «~dictionnaire de poche~», cela ne s’explique pas par le nombre des entrées (dont il y a plus de 20 mille), mais surtout par la simplicité (la pauvreté) des entrées – en ce qui concerne le nombre des exemples. Nous avons décidé d’inclure cet ouvrage dans nos analyses surtout parce que c'est le premier véritable dictionnaire français–hongrois.

\subsubsection{\citealt{Babos1865}}

Le dictionnaire qui apparait en 1865 est rédigé par Kálmán Babos, mais son nom ne figure pas sur la page de titre de la partie français–hongrois (bien que ce soit le «~tome premier~»), on ne le trouve que dans le «~tome second~» (ainsi que la forme francisée de son nom~: Coloman Babos), soit la partie hongrois–français. Curieusement, c’est aussi dans la partie hongrois–français que l’on peut trouver une préface, en langue hongroise seulement, écrite par Babos. C’est grâce à cette préface que l’on peut apprendre qu’il était le rédacteur des deux parties. La préface révèle également la solution du mystère autour de la troisième édition supposée du dictionnaire de Kiss. La page de titre du dictionnaire de Babos inclut l’inscription «~Troisième édition revue et considérablement augmentée~», ce qui est aussi une source de confusion, sachant que le dictionnaire en question est le premier par Babos. Dans sa préface, Babos explique que son œuvre n’a rien à voir avec les deux éditions du dictionnaire de Kiss (et de Karády, qui était le rédacteur de la partie hongrois–français, publiée d’abord en 1848, puis en 1852). Comme Babos l’explique, la formulation «~troisième édition~» est due au fait que l’éditeur de son dictionnaire (notamment Gustave Heckenast, en hongrois Heckenast Gusztáv) est le même que celui des deux éditions de Kiss. Selon cette explication, la troisième édition mystérieuse du dictionnaire de Kiss est en réalité la première édition du dictionnaire de Babos – soit la troisième édition d’un dictionnaire français–hongrois et hongrois–français chez Heckenast. Même au niveau du nombre des entrées, le dictionnaire de Babos avec ses plus de 30 mille entrées est un ouvrage plus volumineux que celui de Kiss.

Sur la page de titre de l’ouvrage de Babos, on trouve également l’information suivante, soit le titre du dictionnaire~: «~Nouveau dictionnaire français–hongrois et hongrois–français d’après les meilleurs ouvrages publiées [sic] jusqu’à ce jour et spécialement d’après le dictionnaire français de Molé~». Dans sa préface, Babos mentionne deux dictionnaires dont il s’est servi au cours de la rédaction de la partie hongrois–français~: la deuxième édition du dictionnaire hongrois et allemand de Mór Ballagi (nous précisons que c’est le dictionnaire hongrois–allemand et allemand–hongrois de Mór Ballagi, ouvrage publié en 1847–1848, les deux parties dans des volumes séparés, le nom de l’auteur dans le dictionnaire étant Móricz Bloch) et le dictionnaire allemand–français de Schuster (il s’agit probablement de l’édition de 1848). Babos explique qu'il a employé ces dictionnaires de façon presque exclusive au cours de la rédaction de la partie hongrois–français de son dictionnaire, et il a complètement réécrit le dictionnaire, indépendamment du dictionnaire de Kiss et de Karády. En revanche, Babos n’écrit rien sur les ouvrages utilisés au cours de la rédaction de la partie français–hongrois, ce qui est plus que surprenant si l’on considère l’information sur la page de titre~: «~spécialement d’après le dictionnaire français de Molé~». Ce que l’on ne peut pas donc savoir après avoir examiné les différentes parties informatives du dictionnaire (surtout la préface), c’est qu’à quel point Babos s’est servi de l’œuvre de Molé~-- mentionné seulement sur la page de titre. En même temps, d’autres questions surgissent~: De quel dictionnaire de Molé s’agit-il exactement~? Qui est-ce qui se cache derrière le nom de Molé~? On ne trouve aucune réponse directe à ces questions dans l’ouvrage de Babos. 

Si l’on considère la préface du dictionnaire français–hongrois de \citet{Theisz1902}~-- dans laquelle Theisz prétend que tous les dictionnaires français–hongrois précédant le sien, excepté le dictionnaire de Spanraft, ont été basés sur des dictionnaires allemands (voire même qu’ils étaient les copies de certains dictionnaires allemands)~-- «~le dictionnaire français de Molé~» doit être un dictionnaire français–allemand. Le nom «~Molé~» apparait parfois sous la forme «~A. Molé~» en tant que l’auteur de plusieurs dictionnaires français, dont le plus remarquable (en raison de son volume et de son contenu) du point du vue de l’allemand est le \emph{Nouveau dictionnaire français–allemand et allemand–français à l’usage de tous les états}, paru pour la première fois en 1841 (\citealt{Bray2000} ~: 175). Ce dictionnaire de Molé a eu 23 éditions entre 1841 et 1865 (la 42e édition a paru en 1900~!), il est donc impossible de savoir de quelle édition Babos s’est servi au cours de son travail~-- si c’est bien l’ouvrage de Molé que Babos a utilisé. En même temps, il faut aussi remarquer que les rééditions de ce dictionnaire (comme c’était souvent le cas dans la seconde moitié du XIXe siècle) était des réimpressions par stéréotypage (\citealt{Bray2000} ~: 168), le contenu n’ayant pas ainsi été altéré. Il existe une version abrégée de ce dictionnaire (\emph{Nouveau dictionnaire de poche français–allemand et allemand–français à l’usage des écoles}), dont la première édition a été publiée en 1844 (\citealt{Bray2000} ~: 175), mais, à notre avis, il est peu probable que ce soit l’ouvrage utilisé par Babos.

Le Babos~-- Molé n’a eu qu’une seule édition.

\subsubsection{\citealt{Martonffy1879}}\label{sec:tillinger:2.1.2}

Le dictionnaire de Frigyes Mártonffy (ou Frédéric Mártonffy dans la partie écrite en langue française de la page de titre) semble être un travail plus autonome. Au sujet de ses sources, Mártonffy n’écrit dans sa préface qu’une seule phrase que nous citons ici~:

\begin{quote}
    Notre ouvrage renferme dans ses deux parties la nomenclature de tous les mots usuels des deux langues, puisée pour la partie française dans les meilleurs ouvrages anciens et modernes, et surtout dans le \emph{Dictionnaire de l’Académie} et celui \emph{des difficultés de la langue française}~; et pour la partie hongroise dans les œuvres de M. Dr. MAURICE BALLAGI.
\end{quote}

Mártonffy écrit donc qu’il a utilisé les meilleurs ouvrages français et hongrois, mais on ne peut pas savoir comment et à quel point. De plus, outre la question des différentes éditions qu’il ne précise pas non plus, on ne peut pas être sûr de quel dictionnaire «~des difficultés de la langue française~» il s’agit. On ne peut que se demander si Mártonffy pense à l’ouvrage de Pierre Boiste (paru en 1828) ou à celui de Jean-Charles Laveaux (dont la première édition a été publiée en 1818, suivie par des éditions revues, corrigées et augmentées, la deuxième en 1822, la troisième en 1846 et la quatrième en 1873). On peut aussi constater qu’il ne fait mention d’aucun dictionnaire français–allemand employé.

En ce qui concerne le nombre des entrées, nous estimons que ce dictionnaire contient plus de 40 mille mots-vedettes.

Le dictionnaire de Mártonffy n’a eu qu’une seule édition. La partie hongrois–français a été rédigée par Babos~-- soit la 2e édition de la partie hongrois–français de son dictionnaire de 1865.

\subsubsection{\citealt{Pokorny1880} }\label{sec:tillinger:2.1.3}

L’auteur du dictionnaire français–hongrois paru en 1880 est Jenő Ármin Pokorny (Eugène Armand Pokorny selon la page de titre française). 

Dans sa préface (écrite seulement en hongrois), Pokorny écrit que les autres dictionnaires français–hongrois, parus avant le sien, n’offrent que de simples «~nomenclatures~», plus ou moins volumineuses. Comme il écrit, présenter des mots-vedettes accompagnés de simples équivalents peut rendre l’utilisation d’un dictionnaire confuse. Selon lui, c’est aussi la raison pour laquelle les dictionnaires français–allemand étaient tellement populaires en Hongrie à l’époque. Son but était donc de créer un dictionnaire français–hongrois et hongrois–français grâce auquel les dictionnaires allemands ne seraient plus indispensables. Pokorny met en relief la présence d'exemples d’usage (ayant pour but d’illustrer aussi bien des expressions figées que de différents styles) et des informations grammaticales dans son dictionnaire.

Pokorny mentionne également qu’il s’est servi des ouvrages français «~les plus récents et les plus renommés~», tels que «~le Dictionnaire de l’Académie, les dictionnaires de Littré, de Poitèvin et de Landais, mais surtout le Bescherelle~». Malheureusement, Pokorny n’indique pas les données exactes de ces ouvrages, même pas leurs titres. Il écrit aussi que c’était à l’aide du «~dictionnaire de Sachs (Toussaint-Langenscheit [sic]), tant recherché en Allemagne~», qu’il a structuré son dictionnaire, mais il ne précise pas à quel point et de quelle façon il l’a employé, et, tout comme dans le cas des autres ouvrages, il ne précise pas les données bibliographiques. On ne peut que deviner que Pokorny se réfère au \emph{Dictionnaire encyclopédique français–allemand et allemand–français} par Karl Sachs et Césaire Villatte.

Le nombre des entrées du dictionnaire de Pokorny dépasse les 50 mille.

Ce dictionnaire de Pokorny a été publié en trois volumes, et il a connu quatre éditions, celle de 1880, une deuxième en 1881, une troisième en 1885 et une quatrième en 1892 (à l’occasion de la parution des deux volumes de la partie hongrois–français en 1891–1892).

\subsubsection{\citealt{Theisz1902} }\label{sec:tillinger:2.1.4}

Gyula Theisz (Jules Theisz), le rédacteur du dictionnaire paru en 1902,\footnote{L’année de la parution n’est pas indiquée dans le dictionnaire de Theisz. Ce n’est qu’à la fin de la préface de Theisz que l’on trouve en hongrois «~au mois de décembre 1902~».} prétend dans sa préface (écrite seulement en langue hongroise) que son dictionnaire est le premier ouvrage qui se base sur le propre travail du rédacteur, c’est-à-dire lui-même. En même temps, Theisz ne mentionne même pas le nom de Pokorny (ni celui de Babos, ni celui de Mártonffy d’ailleurs) quand il parle des dictionnaires français–hongrois précédents, il fait seulement mention des noms des rédacteurs allemands et français dont les dictionnaires~-- selon lui~-- servaient de bases pour les dictionnaires français–hongrois. Theisz parle de Molé et de Sachs. Dans le cas de Molé, il est facile de deviner qu’il s’agit du dictionnaire de \citet{Babos1865}, mais dans le cas de Sachs, ce n’est pas si évident qu’il s’agirait du dictionnaire de Pokorny. Theisz donne quelques exemples des fautes que l’on peut trouver dans les dictionnaires français–hongrois précédents, mais les fautes d’équivalence qu’il présente pour le dictionnaire de Sachs, ne sont que partiellement retrouvables dans le dictionnaire de Pokorny.  De plus, en lisant la préface de Theisz, on ne peut identifier le dictionnaire de Mártonffy non plus.

Ce qui est sûr et certain, c’est que le dictionnaire de Theisz est le fruit d’un travail autonome se concentrant sur les particularités du hongrois et du français, et sur les particularités qui s’émergent si l’on va depuis le français directement vers le hongrois. En même temps, il faut remarquer que le dictionnaire de Theisz, quant au nombre des entrées~-- n’étant qu’environ 26 mille –, n’est pas parmi les ouvrages les plus riches. 

Ce dictionnaire de Theisz n’a eu qu’une seule édition.

\subsubsection{\citealt{Sauvageot1932} }\label{sec:tillinger:2.1.5}

C’est avec le grand dictionnaire d’Aurélien Sauvageot que la lexicographie franco-hongroise prendra une tout autre direction, notamment une bonne. Pour la première fois, le rédacteur est un véritable savant,\footnote{Sauvageot était le premier titulaire de la chaire des langues finno-ougriennes à l’École nationale des langues orientales vivantes, aujourd’hui appelée INALCO~-- Institut national des langues et civilisations orientales. Il a occupé ce poste depuis la création de la chaire en 1931 jusqu’à sa retraite en 1967 (avec une interruption entre 1941 et 1943). Sauvageot raconte l’histoire de son dictionnaire dans son livre intitulé \emph{Souvenirs de ma vie hongroise}. On y apprend~-- entre autres~-- qu’il était choqué de voir ses élèves hongrois utiliser un dictionnaire français–allemand, les dictionnaires français–hongrois étant jugés tellement insuffisants. Dans un premier temps, il n’était pas spécialement tenté d’assumer la tâche de la rédaction d’un dictionnaire, qui lui a été proposée par Marcell Benedek~-- avec la collaboration de qui, et avec celle de József Balassa, il a finalement rédigé son grand dictionnaire. Sauvageot raconte également qu’il avait plus d’une fois parlé de la question du dictionnaire avec Sándor Eckhardt (voir \sectref{sec:tillinger:2.1.7}) qui lui disait toujours qu’il envisageait de produire un jour un dictionnaire français–hongrois, mais Sauvageot comprenait au ton d’Eckhardt que ce n’était pas pour lui un projet très pressant.} et la taille du dictionnaire a considérablement augmenté par rapport aux dictionnaires précédents~-- non seulement quant à la nomenclature~-- le nombre des entrées y atteint les 70 mille –, mais aussi au niveau de la richesse des entrées en exemples. Même du point de vue lexicographique, il s’agit d’un grand pas en avant~: la microstructure des entrées a beaucoup évolué. Ce qui est très surprenant, c’est que ce dictionnaire n’a eu que deux éditions~: la première en 1932 et la seconde en 1942.

\subsubsection{\citealt{Eckhardt1953} }\label{sec:tillinger:2.1.6}

Le dictionnaire qui a pris la place de celui de Sauvageot, qui l’a complètement remplacé, c’est celui de Sándor Eckhardt (Alexandre Eckhardt), dont la première édition date de 1953. À notre avis, cela peut s’expliquer par la situation politique après la seconde guerre mondiale. Sous le régime communiste, les représentants de la vie académique cherchaient~-- ils étaient obligés de faire ainsi~-- à atteindre des résultats significatifs, à créer «~de grandes choses~» dans tous les domaines possibles, tout en rompant avec les réalisations et les réussites liées à des représentants de l’Ouest. Eckhardt avait l’intention d’écrire son propre dictionnaire depuis les années 1920 (voir p.ex. \citealt{Sauvageot1932} ~: 165). Il a finalement créé son dictionnaire énorme, avec 135 mille entrées, avec des entrées très-très riches en exemples, mais avec un grand inconvénient~: le manque total de collaborateurs français a automatiquement entraîné un certain manque de fiabilité dans le cas de certaines entrées.\footnote{Dans le cas de plusieurs entrées, Eckhardt n’avait pas le choix, quand il ne connaissait pas l’équivalent, il essayait de deviner et d’inventer des choses, ce qui est encore plus frappant dans le cas de son dictionnaire hongrois–français. Si l’on regarde par exemple l’équivalent du mot hongrois \emph{sütőpor}, on trouve \emph{poudre gazifère / gazogène} (quelque chose qui s’utilisait dans la deuxième moitié du XIXe siècle et au début du XXe pour faire de l’eau gazeuse, pétillante), alors qu’il s’agit de ce que l’on appelait à l’époque en France \emph{la levure chimique} (~!)~-- aujourd’hui on préfère \emph{poudre à lever}, peut-être parce que le mot \emph{chimique} n’a pas de bonnes connotations.}

Ce dictionnaire d’Eckhardt a eu une édition revue et augmentée en 1960, suivie par une troisième édition complétée par une annexe en 1973, réimprimée en 1981, en 1987, en 1990 et finalement en 1992.

\subsubsection{\citealt{EckhardtOlah1999} }\label{sec:tillinger:2.1.7}

À la fin des années 1990, de la part de la maison d’édition liée à l’Académie hongroise des sciences, une équipe d’une trentaine de chercheurs a eu la tâche de renouveler les dictionnaires d’Eckhardt sous la direction de Tibor Oláh. Le problème, c’est que les rédacteurs n’avaient que très peu de temps, un an seulement, à leur disposition. Ils ont donc enlevé des entrées, ils en ont rajouté d’autres. Malheureusement, ils ont aussi enlevé des mots-vedettes qu’ils n’auraient pas forcément dû enlever (surtout des termes techniques, des mots liés à la faune et à la flore), alors qu’ils n’ont pas réussi à corriger un grand nombre d’erreurs et de fautes au niveau des équivalents, et ils en ont même rajouté quelques-unes. Le nombre total des entrées y était à peu près conservé, soit toujours environ 135 mille. 

La première édition a été réimprimée en 2001.

\subsubsection{\citealt{Palfy1999}}\label{sec:tillinger:2.1.8}

1999 est l’année de la parution d’un autre dictionnaire français–hongrois, qui a vu le jour sous la direction de Miklós Pálfy, professeur à l’Université de Szeged en Hongrie à l’époque. Il s’agissait alors d’un dictionnaire tout nouveau, dont les travaux de rédaction sont partis de zéro. Le résultat en a été un dictionnaire très bien réussi, de taille moyenne, avec un bon choix de nomenclature comprenant 35 mille entrées assez riches en exemples. De plus, ce dictionnaire peut être considéré à la fois actif et passif, incluant des informations supplémentaires utiles destinées non seulement aux utilisateurs ayant le hongrois comme langue maternelle, mais aussi aux utilisateurs francophones. 

Des éditions revues ont paru à partir de 2001, la septième en 2009.

\subsubsection{\citealt{BardosiSzabo2007} }\label{sec:tillinger:2.1.9}

Le dictionnaire paru en 2007 est celui rédigé par Vilmos Bárdosi et Dávid Szabó. Il s’agit, bien évidemment, d’un dictionnaire moins riche que celui d’Eckhardt, mais plus riche que celui de Pálfy quant au nombre des entrées, soit à peu près 51 mille mots-vedettes. Au niveau du nombre des entrées, il s’agit également d’un dictionnaire de taille moyenne, mais avec un excellent choix de nomenclature, comprenant des mots et des expressions issues de toute la francophonie. Tout comme dans le dictionnaire de Sauvageot, on y trouve la prononciation de chaque mot-vedette~-- quelque chose qui manque dans les Eckhardt et le Pálfy. Les articles comprennent suffisamment d’exemples pour illustrer l’usage général des mots-vedettes. Il faut remarquer que Bárdosi est un grand expert de la phraséologie et de la parémiologie (il a publié plusieurs recueils de proverbes et des dictionnaires phraséologiques français–hongrois), Szabó est expert de l’argot, ce qui explique la quantité élevée d'expressions de tels types parmi les exemples. En même temps, il faut aussi préciser qu’il ne s’agit pas d’une quantité exagérée, les rédacteurs ont donc réussi à trouver le juste milieu. 

Ce dictionnaire a été réimprimé quatre fois jusqu’à présent, la dernière fois en 2018 (soit une cinquième édition).

\subsubsection{Les ouvrages exclus de nos analyses}\label{sec:tillinger:2.1.10}

Les ouvrages dont notre étude ne traite pas sont d’une part des dictionnaires français–hongrois que nous avons choisi d’exclure soit en raison de leur volume et de leur format (avec une quantité modérée d’entrées), soit parce qu’ils se concentrent sur un lexique spécialisé. D’autre part, nous avons également exclu les ouvrages qui contiennent seulement des listes de vocabulaire avec des lexèmes français et hongrois, comme c’est le cas, par exemple, de certaines grammaires. La raison en est simple~: nous avons choisi d’examiner les véritables dictionnaires en fonction de la microstructure de leurs entrées et les exemples fournis, alors que les simples listes de vocabulaire se caractérisent~-- entre autres~-- par l’absence des exemples d’usages. En ce qui concerne les dictionnaires spécialisés, étant donné qu’ils ne traitent que d’une certaine couche définie du lexique, leurs contenus et leurs structures seraient difficilement comparables à ceux des dictionnaires généraux. Finalement, les dictionnaires généraux de petite taille ont pour caractéristique d’être restreints non seulement au niveau de la taille de la nomenclature, mais aussi pour ce qui est de la richesse des entrées en exemples d’usage. (Comme nous l’avons déjà précisé dans \sectref{sec:tillinger:1.2}, les dictionnaires hongrois–français ne font pas non plus partie de cette étude.)

Pour une liste complète des dictionnaires français–hongrois consulter \citealt{Jezerniczky1933}  (pour les parutions jusqu’à 1848), \citealt{Sagi1922}  et \citealt{Magay2011} . Dans ces bibliographies, on trouve tous les dictionnaires français–hongrois et hongrois–français, et même d’autres ouvrages (p.ex. des grammaires) contenant des listes de vocabulaire, en revanche, ces derniers sont loin d’être inventoriés d’une manière exhaustive.

\subsection{Méthodes: À la recherche des exemples}\label{sec:tillinger:2.2}

\subsubsection{Qu’est-ce qu’un exemple~?}\label{sec:tillinger:2.2.1}

En ce qui concerne la définition du terme «~exemple~», on peut constater une grande variété d’interprétations, et beaucoup ont déjà remarqué qu’il n’existe pas de convention unanime. Cela a été souligné par Jacobsen et ses coauteurs (1991) aussi, qui expliquent en même temps pourquoi les exemples que l’on a l’habitude d’appeler ainsi ne sont pas des exemples. Selon eux, suivant les différentes fonctions des différentes informations fournies par les entrées, on peut catégoriser et nommer ces informations par certains termes, mais un «~exemple~» n’est qu’un supplément à un équivalent de traduction, fournissant des informations implicites sur l’équivalent ou le mot-vedette.

Nous continuons cependant à nous servir d’une terminologie plus classique, plus proche de celle suivi par \citet{Svensen2009}. Pour nous, il s’agit donc de tous les éléments supplémentaires dans la langue source qui comprennent une certaine forme du mot-vedette, ayant pour but d’illustrer l’usage du mot-vedette en question par rapport à un sens donné, ou bien de présenter une unité lexicale complexe, une collocation, une locution, etc., comprenant une certaine forme du mot-vedette.

Quels peuvent être les différents types d’exemples~?

\begin{itemize}
    \item phrases (propositions) / citations
    \item collocations / locutions / expressions
    \item unités phraséologiques / phrasèmes (expressions figées~: proverbes, dictons, devises, slogans)
    \item unités lexicales complexes
\end{itemize}

Dans plusieurs cas, Jacobsen et ses coauteurs (1991) préfèrent le terme «~sous-entrée~».

Dans nos analyses que nous présentons ici, nous nous concentrons sur certains verbes (voir \sectref{sec:tillinger:2.2.2}). Pour ce qui est de cette catégorie, on peut faire la distinction entre deux types principaux d’exemples dans les entrées de dictionnaire~:

\begin{figure}
% % % \includegraphics[scale=1]{images/Tillinger_Figure1.png}\\
\parbox{.8\textwidth}{\begin{description}
\item[Type 1:] Formes neutres à l'infinitif
\begin{itemize}[leftmargin=*]
\item Type 1a: \textit{aller en voiture} (non-figé)
\item Type 1b: \textit{s'en aller} (figé)
\end{itemize}

\item[Type 2:] Formes conjuguées / propositions (syntagmes verbaux)
\begin{itemize}[leftmargin=*]
\item Type 2a: \textit{je vais à Lyon} (non-figé)
\item Type 2b: \textit{ça va?} (figé)
\end{itemize}
\end{description}}
\caption{\emph{Schématisation des différents types d’exemples pour la catégorie des verbes.}}
\label{fig:tillinger:Figure 1}
\end{figure}

Pour d’autres aspects de la définition et du choix des exemples, voir p.ex. al-\citealt{Kasimi1983, Piotrowski2000, Nielsen2014, Karpinska2015}  et \citealt{Belleflamme2019}.

\subsubsection{Nos méthodes}\label{sec:tillinger:2.2.2}

Pour nos analyses, nous avons choisi d’examiner des entrées présentant des verbes fréquents et sémantiquement surchargés, notamment \emph{aller}, \emph{avoir}, \emph{devoir}, \emph{être}, \emph{faire}, \emph{laisser}, \emph{mettre}, \emph{pouvoir}, \emph{prendre}, \emph{rendre} et \emph{venir}, soit des mots-vedettes dont l’emploi est le plus souvent illustré par de nombreux exemples d’usage~-- aussi bien en raison de la polysémie qu’en raison de la fréquence élevée de ces verbes dans de différents contextes.

Nous avons compté les exemples dans les onze entrées mentionnées de tous les dix dictionnaires présentés dans \sectref{sec:tillinger:2.1}. Nous avons également compté combien d’exemples appartiennent au type T1 et combien au T2 par entrée et par dictionnaire. En tant que référence, nous avons aussi analysé le nombre des exemples dans certaines de nos entrées choisies dans d’autres dictionnaires bilingues. Les statistiques de nos résultats seront présentés dans la partie 3.

Nous analysons la microstructure des entrées par dictionnaire (voir \sectref{sec:tillinger:3.1}--\ref{sec:tillinger:3.1}0), décrivant surtout la structuration des sens et la présentation des exemples et des équivalents, mais nous tenons également compte de la prononciation éventuellement indiquée des mots-vedettes, ainsi que des autres types d’informations fournies par les entrées. Nous étudions également des exemples concrets surtout issus de l’entrée \emph{aller} (typiquement riche en exemples, ce verbe ayant de nombreux sens) de tous les dictionnaires analysés.

Finalement, nous présentons les problèmes des équivalents et des exemples dans les entrées \emph{avoir}, \emph{devoir} et \emph{pouvoir} de nos dictionnaires~-- des verbes français dont le contenu sémantique peut s’exprimer dans la langue hongroise par des constructions grammaticales très différentes de celles utilisées avec ces verbes français (3.11).

\section{Étude comparative des dictionnaires}\label{sec:tillinger:3}

Le \tabref{til:tab1} présente le nombre des exemples dans les entrées choisies dans les différents dictionnaires parus entre 1844 et 1902, alors que le \tabref{til:tab2} indique les chiffres pour les dictionnaires parus entre 1932 et 2007.

\begin{table}[p]
\small
\caption{Le nombre des exemples dans les entrées choisies des dictionnaires parus entre 1844 et 1902.}
\label{til:tab1}
\begin{tabular}{>{\itshape}l *5{c}}
\lsptoprule
 & \citeauthor{Kiss1844} & \citeauthor{Babos1865} & \citeauthor{Martonffy1879} & \citeauthor{Pokorny1880} & \citeauthor{Theisz1902} \\
 & (\citeyear{Kiss1844}) & (\citeyear{Babos1865}) & (\citeyear{Martonffy1879}) & (\citeyear{Pokorny1880}) & (\citeyear{Theisz1902}) \\
\midrule
aller & 4 & 32 & 39 & 99 & 74 \\
avoir & 1 & 9 & 13 & 33 & 20 \\
devoir & 0 & 0 & 5 & 15 & 6 \\
être & 0 & 0 & 6 & 77 & 54 \\
faire & 0 & 2 & 46 & 172 & 31 \\
laisser & 0 & 0 & 4 & 15 & 15 \\
mettre & 1 & 1 & 27 & 83 & 32 \\
pouvoir & 0 & 2 & 7 & 22 & 11 \\
prendre & 0 & 5 & 61 & 140 & 39 \\
rendre & 1 & 5 & 22 & 35 & 15 \\
venir & 0 & 15 & 17 & 65 & 29 \\
\lspbottomrule
\end{tabular}
\end{table}

\begin{table}[p]
\small
\caption{Le nombre des exemples dans les entrées choisies des dictionnaires parus entre 1932 et 2007.}
\label{til:tab2}
\begin{tabular}{>{\itshape}l *5{c}}
\lsptoprule
 & \citeauthor{Sauvageot1932} & \citeauthor{Eckhardt1953} &  \textsc{Eckhardt} \&                        & \citeauthor{Palfy1999} &  \textsc{Bárdosi} \& \\
 & (\citeyear{Sauvageot1932}) & (\citeyear{Eckhardt1953}) &  \textsc{Oláh} (\citeyear{EckhardtOlah1999}) & (\citeyear{Palfy1999}) &  \textsc{Szabó} (\citeyear{BardosiSzabo2007}) \\
\midrule
aller & 125 & 196 & 237 & 134 & 58 \\
avoir & 70 & 120 & 127 & 77 & 59 \\
devoir & 35 & 49 & 29 & 42 & 19 \\
être & 118 & 201 & 161 & 52 & 69 \\
faire & 282 & 240 & 357 & 136 & 131 \\
laisser & 74 & 98 & 117 & 52 & 44 \\
mettre & 135 & 300 & 328 & 206 & 60 \\
pouvoir & 18 & 41 & 40 & 23 & 19 \\
prendre & 169 & 234 & 197 & 138 & 91 \\
rendre & 69 & 103 & 85 & 44 & 31 \\
venir & 85 & 155 & 148 & 61 & 35 \\
\lspbottomrule
\end{tabular}
\end{table}

On peut voir que le nombre des exemples a augmenté avec le temps de manière constante dans les dictionnaires du XIXe siècle~-- en parallèle d’ailleurs avec le nombre des entrées. Cette tendance continue même au XXe siècle avec le dictionnaire de Sauvageot et celui d’Eckhardt~-- le dictionnaire de Theisz représentant une exception. La fin du XXe et le début du XXIe siècle se caractérisent par la parution de dictionnaires qui ne sont pas de grand format (Pálfy, Bárdosi~-- Szabó), mais qui sont très riches en exemples. Le dictionnaire de Pálfy représente un cas exceptionnel~: le nombre de ses entrées représente à peu près un quart du dictionnaire d’Eckhardt, mais cela ne se reflète pas du tout dans le nombre des exemples dans ses entrées.

Le \tabref{til:tab3} montre le nombre des exemples par types dans les dictionnaires de la période 1844–1902. Le premier chiffre indique la quantité des exemples de T1, alors que le second chiffre représente le T2 (voir aussi \figref{fig:tillinger:Figure 1} sous \sectref{sec:tillinger:2.2.1}). Nous avons ombré les cases avec du gris clair quand il y a plus d’exemples avec des infinitifs (T1) et avec du gris foncé quand il y a plus de propositions ou phrases avec des formes conjuguées (T2). Le blanc (sans ombrage) signifie qu’il n’y a pas de différence remarquable entre les deux types, ou bien il n’y a aucun type d’exemple pour le verbe donné. On peut constater que~-- de manière générale~-- les différents dictionnaires ont les mêmes tendances pour un verbe donné, c’est-à-dire la préférence pour un type varie en fonction des verbes. Si cette tendance est rompue quelques fois, cela se constate toujours dans les deux premiers dictionnaires qui montrent une préférence générale pour les exemples de T1~-- si des exemples sont fournis du tout. Ainsi, les dictionnaires préfèrent T1 pour les verbes \emph{faire}, \emph{laisser}, \emph{mettre}, \emph{prendre} et \emph{rendre}, alors qu’ils préfèrent le T2 pour \emph{aller}, \emph{avoir}, \emph{devoir}, \emph{être}, \emph{pouvoir} et \emph{venir}. Nous avons trouvé le même schéma dans les autres dictionnaires de la période 1932–2007 (sans donner les chiffres concrets ici).

Pour pouvoir comparer les chiffres des dictionnaires hongrois avec d’autres dictionnaires bilingues ayant le français comme langue source, nous avons fait le \tabref{til:tab4}, avec des dictionnaires qui sont parmi les plus complets pour les paires de langues en question. Pour le norvégien, nous avons pris en compte deux dictionnaires qui ont à peu près le même nombre d’entrées~: le premier chiffre correspond à \citealt{Elligers1995} et le second à \citealt{Jacobsen2014}.

\begin{table}[p]
\caption{Le nombre des exemples par types (T1 / T2) dans les entrées choisies des dictionnaires parus entre 1844 et 1902.}
\label{til:tab3}
\begin{tabular}{>{\itshape}l *5{c}}
\lsptoprule
& \citeauthor{Kiss1844} & \citeauthor{Babos1865} & \citeauthor{Martonffy1879} & \citeauthor{Pokorny1880} & \citeauthor{Theisz1902}\\
& (\citeyear{Kiss1844}) & (\citeyear{Babos1865}) & (\citeyear{Martonffy1879}) & (\citeyear{Pokorny1880}) & (\citeyear{Theisz1902})\\
\midrule
aller & \cellcolor{lightgray}4 / 0 & 16 / 16 & \cellcolor{gray}15 / 24 & \cellcolor{gray}48 / 51 & \cellcolor{gray}30 / 44 \\
avoir & \cellcolor{lightgray}1 / 0 & \cellcolor{lightgray}5 / 4 & \cellcolor{gray}4 / 9 & \cellcolor{gray}5 / 28 & \cellcolor{gray}0 / 20 \\
devoir & 0 / 0 & 0 / 0 & \cellcolor{gray}1 / 4 & \cellcolor{gray}4 / 11 & \cellcolor{gray}2 / 4 \\
être & 0 / 0 & 0 / 0 & \cellcolor{gray}1 / 5 & \cellcolor{gray}20 / 57 & \cellcolor{gray}8 / 46 \\
faire & 0 / 0 & 1 / 1 & \cellcolor{lightgray}31 / 15 & \cellcolor{lightgray}117 / 55 & \cellcolor{lightgray}20 / 11 \\
laisser & 0 / 0 & 0 / 0 & \cellcolor{lightgray}3 / 1 & \cellcolor{lightgray}10 / 5 & \cellcolor{lightgray}11 / 4 \\
mettre & \cellcolor{lightgray}1 / 0 & \cellcolor{lightgray}1 / 0 & \cellcolor{lightgray}27 / 0 & \cellcolor{lightgray}82 / 1 & \cellcolor{lightgray}32 / 0 \\
pouvoir & 0 / 0 & \cellcolor{gray}0 / 2 & \cellcolor{gray}1 / 6 & \cellcolor{gray}1 / 21 & \cellcolor{gray}0 / 11 \\
prendre & 0 / 0 & \cellcolor{lightgray}3 / 2 & \cellcolor{lightgray}37 / 24 & \cellcolor{lightgray}100 / 40 & \cellcolor{lightgray}32 / 7 \\
rendre & \cellcolor{lightgray}1 / 0 & \cellcolor{lightgray}4 / 1 & 11 / 11 & \cellcolor{lightgray}26 / 9 & \cellcolor{lightgray}14 / 1 \\
venir & 0 / 0 & \cellcolor{lightgray}8 / 7 & \cellcolor{gray}6 / 11 & \cellcolor{gray}20 / 45 & \cellcolor{gray}13 / 16 \\
\lspbottomrule
\end{tabular}
\end{table}

\begin{table}[p]
\caption{Le nombre des exemples dans certaines entrées choisies dans certains dictionnaires bilingues.}
\label{til:tab4}
\begin{tabular}{>{\itshape}lccccc}
\lsptoprule
& anglais & danois & catalan & norvégien & suédois\\
\midrule
aller   & 240 & 144 & 60  & 34/35 & 38 \\
avoir   & 148 & 102 & 35  & 20/7  & 23 \\
être    & 138 & 200 & 94  & 10/11 & 28 \\
faire   & 198 & 290 & 137 & 21/32 & 56 \\
mettre  & 179 & 67  & 92  & 45/25 & 35 \\
prendre & 206 & 221 & 145 & 57/29 & 54 \\
\lspbottomrule
\end{tabular}
\end{table}

Entre autres, on peut voir que les dictionnaires français–hongrois sont beaucoup plus riches en exemples que les dictionnaires français–norvégien ou français–suédois. Le cas du français–danois est une véritable exception pour les langues scandinaves~: dans ce cas il s’agit des chiffres du fameux dictionnaire d’Andreas Blinkenberg et Poul Høybye, qui est un des dictionnaires bilingues les plus volumineux pour le français. Quant au norvégien, au fait, nous avons choisi d’inclure les deux dictionnaires, puisqu’il peut y avoir des différences considérables au niveau de la richesse en exemples, et cela se montre parfaitement dans notre tableau p.ex. pour les entrées \emph{avoir}, \emph{mettre} ou \emph{prendre}.



\subsection{\citealt{Kiss1844}}\label{sec:tillinger:3.1}

Dans le dictionnaire de Kiss, les mots-vedettes français commencent par une majuscule, et ils sont suivis par une abréviation (française) en italique indiquant la catégorie grammaticale du mot-vedette. Le mot-vedette et l’abréviation se séparent en général par une virgule, mais l’ouvrage n’est pas conséquent, parfois on y trouve un point ou rien du tout. Ensuite viennent les équivalents hongrois, séparés par une virgule ou par un point-virgule~-- il est difficile de trouver la logique du choix entre ces deux. 

Les équivalents du mot-vedette sont éventuellement suivis par des exemples d’usages de différents types avec leur traduction hongroise. Chez Kiss, on ne trouve que des exemples de T1. 

En ce qui concerne la forme des équivalents, les verbes hongrois sont indiqués à l’indicatif présent (de la conjugaison appelée «~indéfinie~» ou «~générale~») de la 3e personne du singulier\footnote{Cette forme du verbe n’est marquée par aucun suffixe, c’est pourquoi elle sert de point de départ dans la conjugaison. L’utilisation de cette forme au lieu de l’infinitif peut ainsi aider les apprenants du hongrois.}~-- c’est ce qui est de norme dans les dictionnaires hongrois de nos jours depuis une centaine d’années, mais à l’époque, ce n’était pas ainsi. Dans les tout premiers dictionnaires hongrois, surtout dans les bilingues en paire avec le latin, on trouve les verbes hongrois à la 1ère personne du singulier, alors que dans les ouvrages où le hongrois est en paire avec l’allemand, les verbes hongrois se présentent d’habitude à l’infinitif. Ce sont aussi des infinitifs qui apparaissent dans les dictionnaires français–hongrois du XIXe siècle, mais non chez Kiss. En même temps, il y a une entrée, notamment \emph{avoir}, où Kiss donne un infinitif hongrois aussi (\emph{lenni} outre la forme conjuguée \emph{van})~-- mais tout cela en vain, car les équivalents restent défectifs, étant donné qu’ils signifient simplement \FrTrans{être} (voir \sectref{sec:tillinger:3.11}). D’ailleurs, justement dans le cas d’\emph{avoir}, Kiss donne aussi l’infinitif allemand \emph{haben} en tant qu’équivalent.

L’infinitif de plusieurs des verbes français de notre étude s’utilise également en tant que nom. Chez Kiss, on trouve parfois des entrées séparées pour les mots homonymiques (\emph{aller}, \emph{avoir} et \emph{pouvoir}), alors que d’autres fois les homonymes sont traités à l’intérieur d’une même entrée (\emph{devoir} ou \emph{être}). Justement dans le cas d’\emph{avoir} et de \emph{devoir}, il serait fort souhaitable de fournir des exemples de T2, puisque le hongrois exprime les sens principaux de ces verbes français d’une manière complètement différente (voir \sectref{sec:tillinger:3.11}).

Parmi les entrées étudiées, c’est \emph{aller} où l’on trouve le plus d’exemples chez Kiss, il y en a quatre. Ces exemples sont les suivants~: \emph{aller à cheval}~; \emph{aller en carosse} [sic]~; \emph{aller faire qc.}~; \emph{se laisser aller sur}…. Le choix des trois premiers est tout à fait compréhensible du point de vue de l’usage de l’époque, en revanche, le quatrième exemple nous semble trop spécifique~-- vu le nombre total des exemples. Si l’on étudie les équivalents hongrois de ces quatre exemples, les deux premiers sont facilement compréhensibles. Le troisième, qui exemplifie surtout l’usage d’\emph{aller} en tant qu’auxiliaire, est cependant expliqué par un équivalent hongrois qui signifie \FrTrans{commencer à faire qc}. Le quatrième exemple est le plus problématique même au niveau des équivalents~: on n’y trouve qu’un seul verbe hongrois «~\emph{roskad}~» qui peut avoir plusieurs sens, et nous avons du mal à en identifier un qui pourrait refléter l’exemple français. Dans ce cas, l’utilisateur du dictionnaire aurait vraiment besoin d’un exemple de T2, et une bonne traduction hongroise comme équivalent~-- qu’il s’agisse de quelqu’un apprenant le hongrois ou d’un(e) Hongrois(e) apprenant le français.

Le Kiss est un dictionnaire pauvre en exemples. Parmi les articles traitant des onze verbes de notre étude, il n’y a que quatre contenant des exemples d’usage, et trois sur ces quatre entrées ne présentent qu’un seul exemple.

L’insuffisance des équivalents reflétant les différents sens d’un mot-vedette donné, et l’absence presque totale d'exemples d’usage rendent l’utilisation de ce dictionnaire vraiment difficile, même aux Hongrois à qui cet ouvrage a probablement été destiné.

Finalement, il est intéressant de remarquer que le style typographique des mots français et celui des mots hongrois ne sont pas identiques, ce qui facilite l’usage du dictionnaire. La différence se constate surtout au niveau des espaces entre les lettres, les lettres étant plus éloignées les unes des autres dans les mots français.

\subsection{\citealt{Babos1865} }\label{sec:tillinger:3.2}

Tout comme dans le dictionnaire de Kiss, les mots-vedettes du dictionnaire commencent par une majuscule, et sont suivis par une virgule, ensuite par une abréviation (française) en italique indiquant la nature du lemme. Les équivalents hongrois sont tous en italique, séparés par de simples virgules s’il s’agit de synonymes (représentant à peu près le même sens), alors que l’équivalent appartenant à un autre champ sémantique se sépare des autres équivalents par un point-virgule. On trouve un point-virgule devant les exemples aussi. 

Chez Babos, l’homonymie éventuelle est toujours traitée à l’intérieur d’une même entrée.

Les verbes hongrois (servant d’équivalents) sont indiqués à l’infinitif, ce qui suit la norme des dictionnaires allemand–hongrois de l’époque.

En ce qui concerne les exemples, parmi les entrées étudiées, c’est le verbe \emph{aller} qui est le mieux décrit par le plus d’exemples dans le Babos aussi~-- on y trouve 32 exemples (il y en avait 4 dans le Kiss). En revanche, il n’est pas toujours clair comment il faut comprendre certains exemples, car Babos n’y ajoute aucune marque d’usage désignant le champ d’emploi de l’exemple. Par exemple, dans l’article \emph{aller}, on trouve \emph{de combien allez-vous} [~?], mais cette question est difficilement interprétable, si l’on ne sait pas qu’il faudrait penser à un jeu (auquel on joue en argent). Quant à \emph{avoir}, le sens principal \FrTrans{posséder} est aussi mal rendu par les simples équivalents insuffisants que dans le Kiss. De plus, les exemples \emph{avoir chaud} et \emph{avoir froid} sont mal traduits~-- les équivalents hongrois signifient \FrTrans{être chaud/froid}.

Dans le Babos, la proportion des exemples de T1 et de T2 est assez en balancée, mais T1 reste dominant.

Tout compte fait, du point de vue de la microstructure, le dictionnaire de Babos est beaucoup plus conséquent que celui de Kiss. On peut trouver plus d’exemples d’usage dans le Babos, mais leur nombre est loin d’être suffisant. Parmi les onze entrées de notre étude, il y a trois dans lesquelles on ne trouvera aucun exemple, et dans six articles le nombre des exemples n’atteint pas les dix. Les exemples choisis ne reflètent pas toujours les sens les plus fréquents ou les plus problématiques à expliquer du point de vue de la langue hongroise. Le Babos a été probablement destiné aux utilisateurs hongrois, mais même eux, ils devaient avoir du mal à «~décoder~» des sens fréquents en raison de l’insuffisance des équivalents et des exemples dans certains cas.

\subsection{\citealt{Martonffy1879} }\label{sec:tillinger:3.3}

Dans le Mártonffy, les mots-vedettes commencent par une majuscule~-- comme chez Kiss et chez Babos. Ce qui est novateur et fort appréciable dans ce dictionnaire, c’est que la prononciation des lemmes français est indiquée entre parenthèses après les mots-vedettes. Même s’il ne s’agit que d’une prononciation approximative rendue à l’aide d’une transcription proche de l’écriture de la langue hongroise, cela devait représenter une grande aide aux utilisateurs hongrois. Une abréviation (française, cette fois-ci non en italique) indique la catégorie grammaticale du lemme. Les équivalents hongrois sont tous en italique, comme dans le Babos. 

Une autre innovation du Mártonffy est que les différents sens et/ou les différents types de fonction du lemme sont numérotés, ce qui facilite l’orientation à l’intérieur de l’entrée. Par exemple, dans l’article \emph{aller}, on trouve 15 différents «~sens~» de ce verbe. Les sens et les exemples sont toujours précédés d’un point-virgule qui les sépare de l’information précédente, alors que les exemples et leurs traductions se séparent par de simples virgules. 

Les verbes hongrois (servant d’équivalents) sont indiqués à l’infinitif.

Chez Mártonffy, tout comme chez Babos, l’homonymie est traitée dans un même article. P.ex. après les 15 sens du verbe \emph{aller}, le sens du nom \emph{aller} a reçu le numéro 16. 

Pour ce qui est des exemples, ils sont beaucoup plus nombreux que dans le Babos. Pour le verbe \emph{aller}, on y en trouve 39, mais pour deux de nos verbes étudiés il y en a encore plus~: 46 pour \emph{faire} et 61 pour \emph{prendre}.

Dans le Mártonffy, la proportion des exemples de T1 et de T2 n’est pas homogène du tout, elle varie en fonction des entrées~-- ce qui caractérisera d’ailleurs les dictionnaires français–hongrois du XXe siècle aussi. Le T1 reste prédominant dans le cas de \emph{faire}, \emph{laisser}, \emph{mettre} et \emph{prendre}, tandis que les exemples de T2 sont plus fréquents dans les articles \emph{aller}, \emph{avoir}, \emph{devoir}, \emph{être}, \emph{pouvoir} et \emph{venir}. Chez Mártonffy, dans le cas de \emph{rendre}, le nombre des T1 et celui des T2 sont en égalité, mais ce verbe se caractérise plutôt par l’usage dominant du T1 dans les dictionnaires postérieurs.

Quant à l’innovation de numéroter les différents sens / types de fonction (que nous avons déjà mentionnée), sa réalisation n’est pas la mieux réussie, étant donné que sa logique reste souvent obscure. Il arrive que l’on tombe sur des exemples et des équivalents sémantiquement bien différents, mais présentés sous le même numéro. Pour ne citer que quelques cas, dans l’article \emph{aller} seul, on trouve trois «~sens~» problématiques. Sous le numéro 7, on trouve les exemples suivants (que nous allons marquer par des minuscules différentes ici~: a) \emph{il y va avec prudence}~; b) \emph{je vais lui écrire}~; c) \emph{voyons ce qu’il va dire}~; d) \emph{aller voir q.}~; e) \emph{n’allez pas croire}. On voit bien que ce sont seulement b et c qui exemplifient le même type de fonction du mot-vedette. Sous le numéro 14, on trouve trois exemples~: a) \emph{il s’en va 2 heures}~; b) \emph{comment vous va~?}~; c) \emph{cela va bien}. On constate alors que \emph{b} et \emph{c} sont liés l’un à l’autre, mais \emph{a} exemplifie quelque chose de très différent. Finalement, nous mentionnons les exemples sous le numéro 15~: a) \emph{aller par dessus le marché}~; b) \emph{aller selon le vent}~; c) \emph{allez-vous par terre ou par mer~?}. Dans ce cas, le verbe \emph{aller} s’utilise dans des expressions très différentes, et il peut être difficile de trouver ce qui est commun dans les différents usages. 

Nous voudrions attirer l’attention à l’expression \emph{c}, dont la traduction hongroise signifie \FrTrans{mangez-vous de la viande ou non~?}. En effet, on peut identifier le sens métaphorique de la partie «~par terre ou par mer~» de l’expression~: c’est comme demander à quelqu’un «~mangez-vous de la viande ou du poisson~?~». En revanche, cet exemple du dictionnaire présente une expression qui n’est pas du tout répandue en France, et, selon nos sources, elle ne l’a jamais été. En fait, nous n’avons réussi à retrouver aucune trace de cette «~pseudo-expression~» française, à notre avis, il s’agit d’un hapax intéressant.

\subsection{\citealt{Pokorny1880} }\label{sec:tillinger:3.4}

Dans le Pokorny, comme dans les dictionnaires antérieurs, les mots-vedettes commencent par une majuscule, et, ce qui est nouveau et très efficace du point de vue du visuel, ils sont composés en gras. Les mots-vedettes sont suivis par une prononciation approximative entre parenthèses comme dans le Mártonffy, et d’une abréviation indiquant la catégorie grammaticale.

Pokorny~-- tout comme Mártonffy~-- a numéroté les sens / les types de fonctions, p.ex. de 1 à 25 pour \emph{aller}. Les homonymes éventuels, ainsi que la forme pronominale des verbes, apparaissent comme des entrées à part. C’est la raison pour laquelle \emph{s’en aller} apparaît comme une entrée indépendante après \emph{aller}, avec ses différents sens numérotés de 1 à 9, et, ensuite, on trouve \emph{aller} en tant que nom. La typographie des verbes pronominaux est un peu bizarre~: ils apparaissent notamment avec le pronom \emph{se} en minuscules, mais le verbe de base commence toujours par une majuscule~: p.ex. \emph{s’en aller} apparaît sous la forme \textbf{s’en Aller}.

On peut aussi constater un nouveau phénomène lexicographique dans le Pokorny~: les équivalents hongrois sont précédés d’un signe d’égalité. On trouve ce signe toujours devant le numéro du premier sens, et après tous les exemples d’usage. Les équivalents hongrois qui apparaissent après les numéros des différents sens sont en italique, on peut alors se demander pourquoi Pokorny trouvait nécessaire de marquer le passage depuis le français vers le hongrois d’une telle manière, si la différence entre les mots français et les mots hongrois est conséquemment signalée par les différents styles typographiques. En revanche, si l’on contemple tous les types d’équivalents, on constate que ceux qui suivent les exemples d’usages ne sont pas en italique, et, dans ces cas, le texte hongrois a exactement le même style typographique que le texte français~-- c’est ce qui rend nécessaire l’usage des signes d’égalité. Si tous les équivalents hongrois avaient été composés en italique, il n’y aurait pas eu besoin de ces signes d’égalité, mais il paraît que Pokorny voulait séparer visuellement les deux types d’équivalents.

Dans certaines entrées, pour certains sens du mot-vedette français, on trouve non seulement des équivalents hongrois, mais aussi des explications courtes en hongrois, qui peuvent aider les utilisateurs hongrois du dictionnaire à mieux décoder un sens donné du mot-vedette. Ces explications en langue hongroise apparaissent cependant en italique, tout comme les équivalents des différents sens, et, ainsi, elles sont fort trompeuses du point de vue des utilisateurs francophones qui cherchent à se servir de cet ouvrage en tant que dictionnaire d’encodage. Voir p.ex. le sens numéro 8 dans l’entrée du verbe \emph{aller}~: on y trouve l’explication \emph{Állapot, helyzet} \FrTrans{état, situation} en hongrois, et rien ne montre que ce ne sont pas de simples équivalents. Dans le même article, on trouve des explications trompeuses de ce type sous les sens numéros 12, 16, 17, 19, 20, 23 et 24. En même temps, ces explications~-- qui manquaient dans les ouvrages antérieurs~-- jouent un rôle très important dans le décodage correcte des mots-vedettes ou des exemples d’usage et dans l’interprétation des équivalents. Voir p.ex. le sens numéro 20 sous \emph{aller}. On y rencontre l’explication \emph{Játék} \FrTrans{jeu(x)} en hongrois, qui est indispensable pour comprendre le contexte des exemples et de leurs équivalents~:

\begin{itemize}
    \item Va~! = marad~! tartom~! tartom a tételt.
    \item Va banque~! = tartom a bankot.
    \item Tout va~! = minden játszik. minden tételt elfogadunk.
    \item De combien allez-vous~? = mennyit tart ön~? meddig megy ön~?
\end{itemize}

Un autre élément nouveau et unique de certaines entrées est l’indication des antonymes éventuels entre parenthèses, tout de suite après la catégorie grammaticale ou le numéro du sens dont il s’agit. Par exemple, dans le cas du verbe \emph{aller}, Pokorny donne \emph{venir} en tant qu’antonyme. Pokorny donne parfois même des synonymes et des homonymes (homophones) éventuels, mais ceux-ci se trouvent toujours à la fin d’un article, précédés d’un signe de double ligne verticale.

Même si Pokorny est novateur à plusieurs égards, les verbes hongrois en tant qu’équivalents des mots-vedettes verbaux français apparaissent à l’infinitif~-- comme chez Babos et chez Mártonffy. 

Quant à la quantité des exemples, le Pokorny est beaucoup plus riche que ces prédécesseurs. Dans ce dictionnaire, on trouve bien plus d’exemples d’usage dans les entrées. P.ex. sous \emph{aller} on en trouve 99, mais pour d’autres verbes, ce chiffre dépasse les cent~-- il y en a 140 pour \emph{prendre} et 172 pour \emph{faire}, une quantité comparable à celle dans certains dictionnaires de grand format de nos jours. 

À l’égard des types d’exemples, le Pokorny présente les mêmes schémas que l’on rencontre déjà dans le Mártonffy~: la proportion de T1 et de T2 varie en fonction des entrées. Le T1 est prédominant dans le cas de \emph{faire}, \emph{laisser}, \emph{mettre}, \emph{prendre} et \emph{rendre}, alors que le T2 est plus fréquent dans les articles \emph{aller}, \emph{avoir}, \emph{devoir}, \emph{être}, \emph{pouvoir} et \emph{venir}.

Comme nous avons montré, la microstructure a beaucoup évolué, et les articles de ce dictionnaire fournissent beaucoup plus d’informations sur les mots-vedettes et leur usage que les prédécesseurs de Pokorny. P.ex. les verbes sémantiquement surchargés~-- tels que ceux dans notre étude~-- sont présentés et décrits d’une manière bien plus nuancée (voir l’augmentation du nombre des sens indiqués dans les entrées, ou la richesse en exemples). 

Les deux points les plus faibles de ce dictionnaire que nous avons pu identifier sont la structuration des sens (surtout quand ils sont nombreux) dans le cas de certaines entrées et la typographie trompeuse des explications.

\subsection{\citealt{Theisz1902} }\label{sec:tillinger:3.5}
\largerpage
Les entrées de Theisz se caractérisent par une microstructure extrêmement complexe qui paraît quelque fois impossible à déchiffrer. 

Dans le Theisz, les mots-vedettes sont doublement mis en relief~: ils apparaissent à la fois en majuscules et en gras. Les articles continuent par des abréviations françaises en italique indiquant la catégorie grammaticale. Parfois, on y rencontre d’autres abréviations (françaises ou hongroises~-- p.ex. (\emph{Rh}) = \emph{rendhagyó} \FrTrans{irrégulier}) fournissant des informations supplémentaires. Le mot-vedette peut même être précédé d’un chiffre qui indique que certaines entrées consécutives appartiennent l’une à l’autre. P.ex. dans le cas d’\emph{aller}, on trouve trois entrées liées («~verbe neutre~»~; «~substantif masculin~»~; verbe pronominal)~:

\begin{itemize}
    \item 1, ALLER (Rh) v/n.
    \item 2, ALLER s/m.
    \item 3, S’en ALLER
\end{itemize}

Theisz ne donne pas la prononciation du mot-vedette, ce qui représente un pas en arrière dans l’histoire des dictionnaires français–hongrois. D’autre part, la structuration des sens et des équivalents est devenue trop complexe, notamment à trois niveaux dans une même entrée~-- quatre si l’on y rajoute le numérotage devant les mots-vedettes –, les différents niveaux étant marqués par des chiffres romains, des chiffres arabes (en gras devant les mots-vedettes et en forme régulière après) et des lettres en minuscule. Parmi les équivalents, les verbes hongrois apparaissent à l’infinitif. 

Ce qui peut être considéré comme un pas en avant, c’est la typographie et son usage conséquent. Les exemples, tout comme le mot-vedette, sont en gras, tandis que les équivalents et les traductions en hongrois se présentent en caractères réguliers, et ce n’est que les explications et les informations supplémentaires qui apparaissent en italique.

Dans beaucoup d’entrées (plus souvent que chez Pokorny), on rencontre aussi des synonymes et des antonymes, présentés entre crochets. 

Quant au nombre des exemples, le Theisz est considérablement moins riche que le Pokorny, mais il contient en moyenne plus d’exemples que le Mártonffy~-- même s’il y a parmi nos entrées analysées quelques-unes (notamment trois sur onze) où les exemples sont moins nombreux dans le Theisz. 

Au sujet des types d’exemples, le Theisz suit les mêmes schémas que le Pokorny et le Mártonffy~: la proportion de T1 et de T2 varie en fonction des entrées.

Dans le cas des mots-vedettes qui sont difficiles à rendre par de simples équivalents hongrois (p.ex. \emph{avoir}), le Theisz recourt à des moyens d’explication inhabituels dans les dictionnaires bilingues et qui caractérisent plutôt les livres d’enseignement, tel que l’indication de structures / constructions similaires dans d’autres langues qui peuvent être connues par les utilisateurs de l’ouvrage. Dans l’entrée \emph{avoir}, entre crochets directement après le mot-vedette, on trouve les équivalents latin \emph{habēre} et allemand \emph{haben}. Même à l’intérieur de la même entrée, après l’exemple \emph{j’ai une lettre à écrire}, on peut aussi trouver la traduction allemande \emph{ich habe einen Brief zu schreiben}, outre la traduction hongroise. Vu les intérêts des utilisateurs, cela pouvait représenter une aide significative à l’époque.

\subsection{\citealt{Sauvageot1932} }\label{sec:tillinger:3.6}

La microstructure des entrées dans le Sauvageot ressemble déjà à celles dans les dictionnaires de grand format modernes. Les mots-vedettes apparaissent en minuscules en gras, suivis éventuellement par un chiffre arabe en exposant pour séparer les mots homonymiques qui constituent donc des entrées à part (p.ex. \textbf{aller\textsuperscript{1}} pour le verbe et \textbf{aller\textsuperscript{2}} pour le nom). 

La prononciation des mots-vedettes est indiquée entre crochets, suivie par des abréviations hongroises en italique entre parenthèses qui fournissent des informations sur la nature (surtout la catégorie grammaticale) des mots-vedettes. Si le mot-vedette est un verbe régulier, une lettre en gras entre parenthèses indique le type de conjugaison. 

Les équivalents hongrois sont composés en caractères réguliers. Les verbes hongrois apparaissent (pour la première fois depuis le dictionnaire de Kiss) à l’indicatif présent de la 3e personne du singulier, et non à l’infinitif. 

Les entrées plus complexes sont organisées à deux niveaux hiérarchiques~: les chiffres romains indiquent de différents types d’usage/de fonction du mot-vedette (p.ex. dans le cas des verbes, le dictionnaire fait ainsi différence entre «~verbe transitif~», «~verbe intransitif~», «~verbe impersonnel~», «~verbe pronominal~», etc.), alors que les différents sens sous un tel type d’usage sont indiqués par des chiffres arabes. P.ex. \emph{aller} est structuré de la manière suivante~: \textbf{aller\textsuperscript{1}} I. intransitif (16 sens), II. impersonnel (1 sens), III. \emph{s’en aller} (6 sens)~; \textbf{aller\textsuperscript{2}} nom masculin (1 sens).

Quant à la richesse en exemples, nous avons trouvé cinq entrées parmi les onze analysées dans lesquelles le nombre des exemples dépasse les cent~-- sous \emph{faire}, on en trouve 282. En même temps, il faut aussi remarquer que certains parmi les nombreux exemples sont superflus, puisque plusieurs reflètent le même sens ou le même type d’usage. Parmi les 125 exemples sous \emph{aller}, nous avons trouvé les suivants (nous avons mis une barre oblique entre les exemples de type identique)~:

\begin{itemize}
    \item I. 6.	je vais lui écrire / je vais vous le raconter
    \item I. 10. 	de combien y allez-vous~? / il y est allé de dix francs
    \item I. 16. 	le malade va mieux / elle va de mieux en mieux
    \item III. 1.	va-t’en~! / allez-vous-en~! / allons-nous-en~!
\end{itemize}

Surtout dans le cas de I. 6. et de I. 16., on peut se demander si plus d’un exemple est vraiment nécessaire.

\subsection{\citealt{Eckhardt1953} }\label{sec:tillinger:3.7}

La microstructure des entrées dans le Eckhardt ressemble beaucoup à celle dans le Sauvageot, mais les entrées sont plus détaillées, plus nuancées~-- avec plus de sens pour les mots-vedettes et avec plus d’exemples. Nous pouvons mettre en relief trois changements importants~: 1) la prononciation des mots-vedettes n’est pas indiquée~; 2) après les mots-vedettes, Eckhardt indique entre parenthèses une vingtaine de formes conjuguées des verbes irréguliers~; 3) il n’y a pas de numérotage en exposant après certains mots-vedettes pour différencier les mots homonymiques, car ils apparaissent à l’intérieur de la même entrée, différenciés par des chiffres romains~-- tout comme les différents types d’usage. P.ex. \emph{aller} est structuré de la manière suivante~: I.\footnote{Par erreur, le chiffre romain «~I.~» manque à sa place dans l’entrée \emph{aller}, mais il apparaît dans la deuxième édition de 1960.} (19 sens), II. auxiliaire (5 sens), III. impersonnel (1 sens), IV. transitif (1 sens), V. \emph{s’en aller} (9 sens), VI. nom masculin (1 sens).

L’Eckhardt est extrêmement riche en exemples (voir \tabref{til:tab2}), mais~-- tout comme dans le Sauvageot~-- il y en a des superflus. Nous en citons ici quelques-uns dans l’entrée \emph{aller}~:

\begin{itemize}
    \item (I.) 16. de combien y allez-vous~? / il y est allé de mille francs
    \item II. 1. je vais vous le dire / je vais faire ce qu’on veut / son imprudence va le perdre
    \item V. 1. va-t’en~! / allez-vous-en~! / allons-nous-en~!
\end{itemize}

Il arrive qu’un exemple de T2 est trop simple et/ou mal choisi, et ne fournit plus d’information qu’un exemple similaire de T1. Si le verbe conjugué de l’exemple (de T2a) est donné à l’indicatif présent de la 3e personne du singulier (soit également la forme de base pour les équivalents des verbes hongrois)~-- de plus, avec le pronom «~il~» en tant que sujet, qui ne se traduit pas en hongrois si le sujet n’est pas mis en relief~-- on aura la même traduction hongroise qu’avec l’infinitif en français. Voici deux exemples dans l’entrée \emph{aller}~:

\begin{itemize}
    \item (I.) 2. aller à cheval
    \item (I.) 2. il va à pied
\end{itemize}

Dans ces deux cas, \emph{aller} et \emph{il va} ont pour traduction hongroise la même forme, \emph{megy}.

Nous avons également identifié des exemples qui n’exemplifient pas du tout l’usage du mot-vedette, mais plutôt celui d’un autre constituant de l’exemple. Aussi dans l’entrée \emph{aller}, on rencontre l’exemple suivant qui~-- à notre avis~-- devrait figurer dans l’article \emph{avoir}~:

\begin{quote}
    (I.) 1. ils n’avaient qu’à ne pas y aller
\end{quote}

\subsection{\citealt{EckhardtOlah1999} }\label{sec:tillinger:3.8}

Du point de vue de la structuration des sens, la microstructure des entrées dans l’Eckhardt renouvelé est semblable à celle dans l’Eckhardt ancien. Les formes conjuguées des verbes placées derrière certains mots-vedettes ont été enlevées. À leur place, un chiffre arabe indique le type de conjugaison. Les exemples ne sont plus en italique mais en gras, ce qui rend la lecture des entrées beaucoup plus facile.

Une partie des exemples est identique dans les deux ouvrages, mais il y a eu tout de même de nombreux changements à cet égard~: certains exemples (surtout parmi les superflus, vieillis, trop spécifiques ou problématiques) ont été retirés, d’autres ont été rajoutés. Quant à la richesse des entrées étudiées, le nombre des exemples a considérablement augmenté dans certaines entrées (\emph{aller}, \emph{faire}, \emph{laisser}, \emph{mettre}), alors qu’il a baissé dans d’autres (\emph{devoir}, \emph{être}, \emph{prendre}, \emph{rendre}). 

Beaucoup d’exemples superflus ont été cependant retenus, p.ex. ceux que nous avons mentionnés dans \sectref{sec:tillinger:3.7}. De plus, un bon nombre de nouveaux exemples superflus ont été rajoutés, voir p.ex. dans l’entrée \emph{aller}~:

\begin{quote}
    I. 8. b)	ces vêtements vont dans le placard / cette housse va sur le matelas
\end{quote}

\subsection{\citealt{Palfy1999} }\label{sec:tillinger:3.9}

Visuellement, la microstructure du Pálfy ressemble à celle des entrées de l’Eckhardt \& Oláh, car les exemples y sont composés en caractères gras, et la structuration des sens suit une logique semblable. Ce qui est différent, entre autres, c’est le traitement des mots homonymiques~: les homonymes apparaissent dans des entrées à part, et l’on trouvera un chiffre arabe devant le mot-vedette (comme dans le Theisz). 

Les entrées du Pálfy comprennent beaucoup plus d’informations supplémentaires (en langue française, entre parenthèses) qui suivent les différents exemples et qui aident les utilisateurs à mieux comprendre de quelles manières et dans quels types de contexte les structures présentées par les exemples sont utilisées, fournissant ainsi des informations aussi bien sur les formalités grammaticales que sur la sémantique. Les équivalents hongrois sont aussi formulés d’une telle façon que cet ouvrage peut être considéré comme un dictionnaire de décodage et d’encodage à la fois.

Les exemples d’usage dans les entrées sont plutôt bien choisis et bien formulés, mais nous en avons aussi trouvés quelques-uns qui~-- pour des raisons diverses~-- sont problématiques. Si l’on regarde l’entrée \emph{aller}, on tombe sur l’exemple \emph{tout son argent s’en va en cassettes}. Dans ce cas, le choix du mot \emph{cassette} dans un tel exemple est discutable~-- les cassettes étant en voie de disparition en 1999. Toujours dans la même entrée, on trouve des exemples parfaitement choisis pour illustrer le futur proche, tels que \emph{elle va accoucher dans six mois} ou \emph{je vais y aller}, en revanche, le rajout des exemples «~ça va faire des histoires~!/ça va barder~! \emph{fam}~» [sic~-- le manque de l’espace devant les signes doubles est un problème général dans le Pálfy] n’est pas justifié. D’une part, ces deux expressions (familières) n’apportent rien de plus par rapport aux exemples précédents quant à l’illustration de la fonction grammaticale en question, d’autre part la nature familière des expressions n’est pas due à l’utilisation du verbe \emph{aller} (soit le mot-vedette dans ce cas), mais aux expressions \emph{faire des histoires} et \emph{barder}. 

Finalement, nous voudrions mentionner le problème des nouvelles éditions dans lesquelles les différents types de fautes peuvent être conservés, même s’il s’agit de plusieurs éditions revues/corrigées (la 7e édition revue et corrigée a paru en 2009). P.ex. dans l’entrée \emph{aller}, l’exemple «~on sonne, va ouvrir~» et même sa traduction hongroise figure deux fois. Ce qui est encore plus surprenant, c’est que de telles fautes ne sont corrigées dans la version en ligne du dictionnaire non plus~-- comme s’il s’agissait d’une simple réimpression.

\subsection{ \citealt{BardosiSzabo2007} }\label{sec:tillinger:3.10}

Du point de vue de la clarté de la microstructure des entrées, ainsi qu’au niveau du visuel et de la maniabilité, ce dictionnaire est celui qui nous paraît le mieux organisé.

Les mots-vedettes apparaissent en minuscules en gras (comme dans les autres dictionnaires du XXe siècle depuis Sauvageot), suivis éventuellement par un chiffre arabe en exposant pour séparer les mots homonymiques qui sont traités dans des entrées à part. Ce qui est une nouveauté remarquable et très utile, c’est la réapparition de la prononciation des mots-vedettes, cette fois-ci (pour la première fois~!) avec les symboles de l’alphabet phonétique international (API). La transcription phonétique des mots-vedettes est suivie par de différents types d’informations sur le mot-vedette (p.ex. type de conjugaison pour les verbes, classe grammaticale, etc.). 

Les équivalents hongrois sont composés en caractères réguliers, souvent complétés d’informations supplémentaires entre crochets.

Quant à la richesse en exemples (qui apparaissent en gras), leur nombre n’atteint pas la même quantité que l’on trouve dans le Pálfy. En même temps, les exemples sont assez nombreux pour pouvoir bien illustrer les différents sens présentés des mots-vedettes, et ils sont très bien choisis. Ce qui ne paraît pas toujours logique, c’est la structuration de la présentation de certains exemples~-- quelque chose qui peut être remarqué par des linguistes, mais qui ne gêne pas l’utilisateur moyen du dictionnaire, surtout qu’il s’agit de cas rares et sporadiques. Voyons un exemple parmi les exemples d’usage de l’entrée \emph{aller}~:

\begin{itemize}
    \item~ \textbf{en bus/tram/métro} busszal/villamossal/metróval megy
    \item~ \textbf{en avion} repülőgépen utazik
    \item~ \textbf{en voiture} kocsival megy
\end{itemize}

Il n’est pas évident pourquoi \emph{en voiture} n’apparaît pas ensemble avec \emph{bus/tram/métro}, surtout que la construction hongroise reflétée par les traductions est la même pour les deux. Bien sûr, on peut reconnaître que les éléments du trio «~bus/tram/métro~» représentent le même type de transport (transport en commun), alors que \emph{voiture} n’en fait pas partie, mais on peut aussi se demander si une telle différenciation sémantique est vraiment nécessaire. De plus, \emph{en avion} pourrait également être mentionné avec les transports en commun (soit des transports collectifs), si son équivalent hongrois suivait la même structure (avec le verbe \emph{megy} et le mot signifiant le mode de transport à l’instrumental-comitatif)~-- ce qui serait tout à fait possible en hongrois~: \emph{repülőgéppel megy} au lieu de \emph{repülőgépen utazik} (où le mot pour «~avion~» est au superessif avec le verbe \emph{utazik}). Il se peut que les auteurs aient eu l’intention de montrer que le verbe \emph{utazik} (\FrTrans{voyager}) s’utilise souvent en hongrois en cas de voyages en avion, mais ce verbe s’utilise aussi avec les autres modes de transport. Il serait peut-être préférable de les présenter ensemble, avec deux équivalents possibles (tous les deux étant souvent utilisés). L’équivalent comprenant le verbe \emph{megy} pourrait même être complété avec le verbe \emph{utazik} en tant que variante, puisque ce dernier s’emploie aussi bien avec l’instrumental-comitatif qu’avec le superessif en hongrois~: «~aller en avion/bus/tram/métro~» \emph{repülőgéppel/busszal/villamossal/metróval megy/utazik, repülőgépen/buszon/villamoson/metrón utazik}.

\subsection{«~L’Ouest rencontre l’Est~» : Ou le cas des entrées \emph{avoir}, \emph{devoir} et \emph{pouvoir}}\label{sec:tillinger:3.11}

Puisque le hongrois est une langue finno-ougrienne (et non indo-européenne comme le français), ce n’est pas surprenant que certaines constructions grammaticales s’y forment de manières différentes. On pourrait nommer de nombreux cas dans lesquels un sens ou une fonction s’exprime par un seul mot dans une langue, alors qu’il est impossible de définir le même sens par un simple équivalent dans une autre langue. Nous avons choisi de mettre en relief ici trois mots-vedettes français qui peuvent poser des problèmes à cet égard, et dont la présentation se montre insuffisante dans les premiers dictionnaires français–hongrois. Ce type de problème ne touche plus le dictionnaire de Sauvageot et les ouvrages parus depuis.

\subsubsection{\textit{avoir}}

Dans la langue hongroise, il n’y a pas de verbe \emph{avoir} pour exprimer la possession. C’est l’équivalent du verbe \emph{être} qui s’utilise en tant que prédicat, et le sujet sera l’objet directe (le possédé) de la construction française, marqué en hongrois par un suffixe possessif, tandis que le sujet du verbe français sera utilisé en hongrois au datif. Ce type de construction est d’ailleurs typique pour les langues finno-ougriennes. Voyons quelques phrases concrètes~:

\ea hongrois\\
\gll Van (nekem) (egy) szótár-am.\\
     est (\textsc{dat.1sg}) (un) dictionnaire-\textsc{1sg.poss}\\
\glt \FrTrans{J’ai un dictionnaire.} (litt. \FrTrans{Il y a (à moi) mon dictionnaire.})
        
\ex hongrois\\
\gll Péter-nek van (egy) szótár-a.\\
     Pierre-\textsc{dat} est (un) dictionnaire-\textsc{3sg.poss}\\
\glt \FrTrans{Pierre a un dictionnaire.} (litt. \FrTrans{À Pierre il y a son dictionnaire.})
\z

Une telle construction s’explique mieux par des exemples d’usage dans un dictionnaire, mais les premiers dictionnaires français–hongrois n’en donnent aucun pour ce sens de «~posséder~». En tant qu’équivalent hongrois, Kiss donne les mots \emph{lenni} (l’infinitif \FrTrans{être}) et \emph{van} (forme conjuguée~-- voir \sectref{sec:tillinger:3.1}), et il les complète avec l’infinitif allemand \emph{haben}. Outre \emph{lenni}, Babos donne aussi \emph{birni}, un verbe polysémique qui peut signifier \FrTrans{posséder}, mais sans exemples d’usage, il est difficile d’interpréter cet équivalent. Mártonffy donne aussi \emph{bírni}\footnote{Chez Mártonnfy, ce mot s’écrit déjà avec \emph{í} (et non avec \emph{i} comme chez Babos)~-- soit la forme utilisée aujourd’hui.} et \emph{lenni}, et il complète ce dernier entre parenthèses avec les suffixes \emph{-nak}, \emph{-nek} indiquant le datif, mais~-- comme il n’en écrit plus~-- il n’est pas évident quelle partie de l’expression devrait être mise au datif (et il existe d’autres constructions en hongrois avec \emph{lenni} et le datif). Pokorny donne seulement \emph{birni} (avec \emph{i}), et il le complète avec un seul exemple, un «~proverbe~» formellement assez complexe, ce qui ne facilite pas vraiment la tâche de l’utilisateur du dictionnaire~: \emph{Il n’est rien de tel que d’en avoir.} C’est d’abord chez Theisz que l’on trouve plus d’informations sur ce sens en question. Theisz ne donne que les infinitifs latin (\emph{habēre}) et allemand (\emph{haben}) comme équivalents, et il met un mot hongrois (\emph{birtok} \FrTrans{possession}) entre parenthèses en tant qu’information supplémentaire. Il continue ensuite avec des exemples qui sont déjà plus utiles~: \emph{j’ai une plume}~; \emph{il a des amis}~; \emph{je l’ai pour ami} et \emph{il a de quoi (vivre)}.

\subsubsection{\textit{devoir}}

Ce verbe français peut avoir plusieurs sens, parmi lesquels nous voudrions mentionner son usage avec un autre verbe à l’infinitif pour exprimer une sorte d’obligation~: «~quelqu’un doit faire quelque chose~». Dans la langue hongroise, il n’y a pas de verbe correspondant qui pourrait être conjugué en fonction de l’agent~-- soit la personne qui doit agir, identique avec le sujet de la construction française. Avec ce sens, c’est le verbe \emph{kell} qui s’utilise en hongrois, dont le sujet est l’infinitif (désignant l’activité à accomplir), et l’agent (comme une sorte de complément) se met au datif. Très souvent, l’infinitif utilisé dans une telle construction peut être marqué par un suffixe possessif indiquant la personne de l’agent~-- soit une forme personnelle conjuguée de l’infinitif. 

\ea hongrois\\
\gll Péter-nek vár-ni-a kell.\\
     Pierre-\textsc{dat} attendre-\textsc{inf-3sg.poss} falloir-\textsc{3sg.pres}\\
\glt \FrTrans{Pierre doit attendre.}

\ex hongrois\\
\gll (Nekem) vár-n-om kell.\\
     (\textsc{dat.1sg}) attendre-\textsc{inf-1sg.poss} falloir-\textsc{3sg.pres}\\
\glt \FrTrans{Je dois attendre.}
\z

Dans le Kiss, on ne trouve que trois équivalents simples qui seraient censés de couvrir tous les sens (principaux) du verbe \emph{devoir}~: \emph{adós}, \emph{kell} et \emph{tartozik}. Le premier est un adjectif (utilisé également comme substantif) signifiant \FrTrans{ayant une dette~; débiteur}, le deuxième (mentionné plus haut) est un verbe qui s’emploie d’une manière complètement différente au niveau de la syntaxe par rapport au verbe \emph{devoir}, tandis que le troisième~-- également un verbe~-- signifie \FrTrans{avoir une dette}. 

Babos donne les équivalents suivants, sans exemples~: \emph{adósnak lenni} (\FrTrans{être débiteur}), \emph{tartozni} (\FrTrans{avoir une dette}), \emph{köszönhetni} (\FrTrans{être redevable de qc. à qu.}), \emph{kelleni} (l’infinitif de \emph{kell}), et \emph{kötelesnek lenni} (\FrTrans{être obligé}). 

Mártonffy différencie cinq sens du verbe \emph{devoir}, parmi lesquels le sens que nous mettons ici en relief est représenté par l’équivalent \emph{kelleni}. Heureusement, l’auteur donne deux exemples, parmi lesquels c’est surtout le premier qui peut être utile avec sa traduction hongroise~: \emph{il doit parti} [sic], en hongrois \emph{el kell utaznia}. Malheureusement, la forme erronée de l’infinitif \emph{partir} peut tromper l’utilisateur hongrois du dictionnaire, car la forme «~parti~» existe en tant que participe passé. Le second exemple est moins utile, puisqu’il comprend une négation, ce qui peut changer la sémantique du verbe en fonction du contexte~: \emph{vous ne devez pas rester} dont la traduction dans le Mártonffy est \emph{nem szabad maradnia}. Cette traduction suggère justement un autre sens du verbe \emph{devoir}, elle signifie \FrTrans{Vous ne pouvez pas rester}. 

Chez Pokorny, on trouve deux équivalents à l’égard du sens en question~: \emph{kell} (curieusement, l’auteur donne la forme conjuguée du verbe, alors qu’ailleurs il donne l’infinitif des verbes hongrois) et \emph{köteles} (adjectif qui signifie \FrTrans{obligé}). Les exemples dans le Pokorny sont plutôt compliqués et moins utiles~: \emph{Les choses devaient se venir là}~; \emph{Dussais-je y périr}~; \emph{Fais ce que dois, advienne que pourra} (proverbe)~; \emph{Vas où tu peux, mourir ou} [sic] \emph{tu dois} (proverbe).

Dans le Theisz, une petite remarque d’usage éclaircit que le sens en question s’exprime si l’infinitif d’un autre verbe suit \emph{devoir}. Au lieu de simples équivalents, Theisz ne donne que deux exemples pour ce sens~: \emph{je dois servir…} et \emph{il doit savoir que…}. Le second exemple peut être considéré comme problématique, puisqu’il est sémantiquement ambigu.  

\subsubsection{\textit{pouvoir}}

En français, on a l’habitude de faire la différence entre \emph{pouvoir} (\emph{faire qc}) et \emph{savoir} (\emph{faire qc})~-- si quelqu’un est capable de faire quelque chose ou bien parce que les circonstances le rendent possible (\emph{pouvoir}), ou bien en raison de ses connaissances/compétences (\emph{savoir}). En hongrois, on ne fait pas cette distinction, ces deux verbes français peuvent être traduits par le même verbe hongrois~: \emph{tud}. On aura donc besoin de certains compléments en hongrois pour pouvoir faire la différence. Les phrases «~je ne peux pas nager~» et «~je ne sais pas nager~» peuvent donc se traduire en hongrois par une même phrase~: \emph{nem tudok úszni} (\textsc{neg} pouvoir/savoir-\textsc{1sg.prs} nager-\textsc{inf}). En revanche, en fonction de la nature des circonstances, \emph{pouvoir} peut être exprimé de manières différentes. Le verbe hongrois \emph{tud} représente donc l’une des possibilités. L’autre possibilité consiste à se servir d’un suffixe potentiel~-- une forme qui n’est pas considérée par les grammaires hongroises comme un mode verbal à part. (Le verbe \emph{pouvoir} peut être traduit en hongrois de plusieurs manières, ici nous ne mentionnons que les constructions les plus fréquentes pour le sens «~avoir la possibilité de faire qc~».) Ce potentiel est utilisé en hongrois surtout pour exprimer que l’acte (indiqué par l’infinitif qui suit le verbe \emph{pouvoir} en français) est permis, mais il peut aussi se référer à d’autres circonstances~-- voir les deux phrases suivantes.

\ea hongrois\\
\gll Nem utaz-hat-nak Párizs-ba.\\ 
     \textsc{neg} voyager-\textsc{pot-3pl} Paris-\textsc{ill}\\
\glt \FrTrans{Ils/Elles ne peuvent pas voyager à Paris.}
\ex hongrois\\
\gll Nem tud-nak Párizs-ba utaz-ni.\\
     \textsc{neg} pouvoir-\textsc{3pl.prs} Paris-\textsc{ill} voyager-\textsc{inf}\\
\glt \FrTrans{Ils/Elles ne peuvent pas voyager à Paris.}
\z

Dans le Kiss, on ne trouve qu’un seul équivalent hongrois~: \emph{lehet}. À l’origine, ce mot hongrois est le verbe \emph{van} \FrTrans{être} (\emph{lenni} à l’infinitif) au potentiel (au présent de la 3e personne du singulier). Cette forme signifie donc \FrTrans{pouvoir/peut être}, et, en effet, c’est aussi un des équivalents possibles de l’expression française «~peut-être~». Il faut aussi remarquer que \emph{lehet} a un troisième type d’usage~: il peut être utilisé avec d’autres verbes à l’infinitif. P.ex. \emph{lehet menni} (être-\textsc{pot.3sg.prs} aller-\textsc{inf}) \FrTrans{on peut / il est possible d’y aller}. Il est donc difficile d’interpréter cet équivalent hongrois sans aucun exemple d’usage.

\begin{sloppypar}
Babos donne les équivalents suivants~: \emph{…hatni}, \emph{…hetni}, \emph{tehetni}, \emph{megtehetni}, \emph{birni} et \emph{birhatni}. Les deux premiers représentent les deux variantes du suffixe potentiel avec la terminaison de l’infinitif, les trois points indiquant la place d’un verbe quelconque. Le troisième et le quatrième sont, en revanche, des verbes concrets au potentiel ayant le sens \FrTrans{pouvoir faire}. Le cinquième est le même verbe polysémique que nous avons déjà mentionné à propos du verbe \emph{avoir} (\FrTrans{posséder} et \FrTrans{pouvoir} étant seulement deux des traductions possibles)~-- difficile à interpréter sans exemples d’usage. Finalement, le sixième est le même que le précédent, mais au potentiel~-- une forme tautologique.
\end{sloppypar}

Les équivalents de Mártonffy sont moins nombreux et moins compréhensibles. Il donne \emph{hatni…}, \emph{hetni…}~-- on peut se demander pourquoi les trois points suivent les suffixes. Ensuite, il continue avec un exemple de T1~: \emph{pouvoir marcher} avec l’équivalent \emph{mehetni}~-- exemple simple, sans faute et assez efficace. Son exemple suivant se base en revanche sur la forme pronominale (impersonnelle) \emph{se pouvoir}, ce qui est très curieux, étant donné qu’il traite ce sens séparément dans l’entrée avec d’autres exemples semblables. Plus loin dans l’entrée, il donne également les équivalents \emph{tehetni} \FrTrans{pouvoir faire} et \emph{birni} (voir plus haut).

Pokorny donne les équivalents suivants~: \emph{hatni}, \emph{hetni}, \emph{birni} et \emph{tehetni}. Les deux derniers ont déjà été présentés plus hauts. Quant aux deux premiers, ils apparaissent comme s’ils étaient des infinitifs indépendants, mais, en réalité, ce ne sont que des suffixes~-- comme nous l’avons expliqué à propos des équivalents de Babos. Les exemples~-- comme c’est souvent le cas chez Pokorny~-- sont trop compliqués et peu efficaces~: \emph{On ne peut être plus mal servi}~; \emph{Je ne puis qu’y faire}.  

Chez Theisz aussi, on trouve \emph{-hatni} et \emph{-hetni}, mais grâce à l’usage des traits d’union (tout comme avec les trois points chez Babos), il est clair comment on doit interpréter ces équivalents. Theisz continue ensuite avec \emph{tudni} (pour la première fois~!), \emph{bírni} (complété par \emph{vmit tenni} \FrTrans{faire qc}) et \emph{képesnek lenni} \FrTrans{être capable}. L’auteur précise que ce sens se base sur l’usage de \emph{pouvoir} avec un autre verbe à l’infinitif, ensuite, il continue avec ses exemples (peu nombreux et assez simples)~: \emph{il peut écrire}~; \emph{si je peux parler ainsi}.

\section{Conclusion}\label{sec:tillinger:4}

Les mots-vedettes et les équivalents suivent et reflètent en général l’époque des différents ouvrages, mais ce n’est pas toujours le cas des exemples d’usage. Le choix et la présentation des exemples représentent un problème constant. Dans les dictionnaires du XIXe siècle, les exemples ont été choisis sans logique transparent, plutôt au hasard, pourrait-on dire~-- un peu comme la structure des entrées au sujet des sens présentés.

Le nombre des entrées ne reflète pas automatiquement la richesse des entrées en exemples. Dans le cas du \citeauthor{Palfy1999}, nous avons montré qu’un dictionnaire à 35 mille entrées peut fournir presque autant d’exemples que les dictionnaires de grand format ayant au moins deux ou trois fois plus d’entrées. 

En raison de ses erreurs, l’Eckhardt renouvelé de \citeyear{EckhardtOlah1999} (l’ouvrage \citeauthor{EckhardtOlah1999}) devrait être renouvelé encore une fois. Il est cependant fort douteux qu'un tel projet puisse avoir lieu dans un avenir proche. Dans un premier temps, il suffirait de corriger les erreurs de l’ouvrage et de publier une version numérisée (en ligne), qui serait plus facile à compléter selon les besoins. Cela est aussi valable pour les autres dictionnaires actuellement en vente.

Le grand dictionnaire de \citeauthor{Sauvageot1932} mériterait d’être numérisé en raison de ses qualités pour faciliter le travail des philologues. 

Les dictionnaires de \citeauthor{Palfy1999} et de \citeauthor{BardosiSzabo2007} représentent le plus haut niveau de la lexicographie. Les articles sont très clairement structurés, d’une façon très efficace, ce qui fait qu’il s’agit de deux dictionnaires facilement maniables. De plus, les équivalents dans ses dictionnaires sont les plus fiables~– même si l’on peut y trouver quelques erreurs. Il serait très important d’«~entretenir~» constamment ces dictionnaires précieux, de publier des versions en ligne dont le contenu serait de temps en temps revu, corrigé et augmenté.

{\sloppy\printbibliography[heading=subbibliography,notkeyword=this]}
\end{otherlanguage}

\renewcommand{\tabref}[1]{Table~\ref{#1}}
\end{document}
