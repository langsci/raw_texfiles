\author{Geoffrey Williams and Mathilde Le Meur and Andrés Echavarría Peláez} %use this field for editors as well
\title{West meets East}
\subtitle{Papers in historical lexicography and lexicology from across the globe}
\renewcommand{\lsISBNdigital}{978-3-96110-503-8}
\renewcommand{\lsISBNhardcover}{978-3-98554-132-4}
\BookDOI{10.5281/zenodo.15020225}


\typesetter{Sebastian Nordhoff}
\proofreader{Amir Ghorbanpour,
Brett Reynolds,
Christopher Straughn,
Conor Pyle,
David Carrasco Coquillat,
Diana Lewis,
Elliott Pearl,
Jeroen van de Weijer,
Katja Politt,
Mary Ann Walter,
Nicoletta Romeo,
Rebecca Madlener,
Sebastian Nordhoff,
Vladan Sutanovac}
\renewcommand{\lsSeries}{whll}
\renewcommand{\lsSeriesNumber}{1}

\renewcommand{\lsID}{458}
\BackBody{Lexicography, in its many forms, is a very old, practical discipline solving practical problems concerning word usage. The term “word” seems more appropriate than “language” in this context, as lexicography addresses more questions relating to what we now call lexicology. As with all areas of human endeavour, what developed gradually through trial and error has eventually been subjected to a theoretical framework. The role of historical lexicography is to look back on the development of these highly varied word lists to understand how we arrived at the tremendous variety that characterises practice throughout the world.

This volume is both a selection of expanded papers from one conference on historical lexicography and lexicology, held under the aegis of the International Society for Historical Lexicography and Lexicology (ISHLL) in Lorient, France, in May 2022, and also the first in a new book series dedicated to the field. The new series represents a collaboration between two sister associations, ISHLL and the Helsinki Society for Historical Lexicography (HSHL). The volume contains texts in both English and French that provide insights into dictionaries, their compilers and users using evidence from numerous languages across the globe. It is also diachronic, moving from topics on medieval usage to contemporary issues concerning open access and digital publishing in historical lexicography. The title reflects the global scope of its authors and content, encompassing Japan to the United States, Eastern Europe to the United Kingdom, and Portugal.}




