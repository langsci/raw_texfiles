\documentclass[output=paper]{langscibook}
\ChapterDOI{10.5281/zenodo.6762270}

\author{Rosita L. Rivera\affiliation{University of Puerto Rico-Mayagüez} and Eva Rodríguez-González\affiliation{University of New Mexico}}
\title{Integrated approaches to language assessment in language learning: Introduction and chapter synopsis}
\abstract{This introductory chapter establishes the rationale for the volume by situating assessment within the context of multilingual learners in diverse settings. This chapter also includes a brief definition of assessment terminology. We conclude this introduction by providing an overview of the chapters and contributions of scholars included in this edited volume.


Recent research in the field of applied linguistics has addressed the complex and contextual realities of multilingual language learners (\citealt{Larsen-Freeman2018, Larsen-Freeman2017}; \citealt{Ortega2017}).  Scholars have also discussed the dynamic nature of language and the need to address the different characteristics and experiences learners bring with them to their learning context (\citealt{GarcíaWei2014}; \citealt{García2009}; \citealt{LantolfThorne2020}). There is a need for adaptability and creation of multiple ways to assess linguistic knowledge to address the needs of multilingual citizens. It is within this context that the concept of “assessment” is gaining momentum. Educational institutions are constantly dealing with the design of “reliable metrics to ensure that individuals have sufficient linguistic competence to carry out job- or school-related tasks and also to compare the language capabilities of individuals” (\citealt[17]{MenkeMalovrh2021}). In recent decades, alternative views of language assessment have been proposed and advocated for language learning.~This volume includes examples of these alternative views in multiple contexts to illustrate how the learning environment determines the design of assessment that suits the specific needs of the learners.


%In this volume, we intend to show how language is connected to ecology and the need for linguistic sensitivity in multilingual contexts. An analogy that connects nature with language as an ecological perspective is the notion of “fractals”. A fractal is a pattern that repeats at different scales following laws of nature. Trees and other plants are examples of fractals, which are found everywhere. Authors such as Maurice Claypole make a direct connection between fractals in nature and language learning. In his book “The Fractal Approach to Teaching English as a Foreign Language: Dynamism and Change in English Language Teaching”, \citet{Claypole2010} claims that the fractal approach envisages a new paradigm of language based on forms found in nature and indicates a goal-oriented method of developing teaching materials incorporating a holistic view of language acquisition. It is precisely within this dynamic view of language learning where we introduce our readers to the rationale for a volume on alternative assessments in language learning.
}

\IfFileExists{../localcommands.tex}{%hack to check whether this is being compiled as part of a collection or standalone
  \addbibresource{../localbibliography.bib}
  \usepackage{langsci-optional}
\usepackage{langsci-gb4e}
\usepackage{langsci-lgr}

\usepackage{listings}
\lstset{basicstyle=\ttfamily,tabsize=2,breaklines=true}

%added by author
% \usepackage{tipa}
\usepackage{multirow}
\graphicspath{{figures/}}
\usepackage{langsci-branding}

  
\newcommand{\sent}{\enumsentence}
\newcommand{\sents}{\eenumsentence}
\let\citeasnoun\citet

\renewcommand{\lsCoverTitleFont}[1]{\sffamily\addfontfeatures{Scale=MatchUppercase}\fontsize{44pt}{16mm}\selectfont #1}
  
  %% hyphenation points for line breaks
%% Normally, automatic hyphenation in LaTeX is very good
%% If a word is mis-hyphenated, add it to this file
%%
%% add information to TeX file before \begin{document} with:
%% %% hyphenation points for line breaks
%% Normally, automatic hyphenation in LaTeX is very good
%% If a word is mis-hyphenated, add it to this file
%%
%% add information to TeX file before \begin{document} with:
%% %% hyphenation points for line breaks
%% Normally, automatic hyphenation in LaTeX is very good
%% If a word is mis-hyphenated, add it to this file
%%
%% add information to TeX file before \begin{document} with:
%% \include{localhyphenation}
\hyphenation{
affri-ca-te
affri-ca-tes
an-no-tated
com-ple-ments
com-po-si-tio-na-li-ty
non-com-po-si-tio-na-li-ty
Gon-zá-lez
out-side
Ri-chárd
se-man-tics
STREU-SLE
Tie-de-mann
}
\hyphenation{
affri-ca-te
affri-ca-tes
an-no-tated
com-ple-ments
com-po-si-tio-na-li-ty
non-com-po-si-tio-na-li-ty
Gon-zá-lez
out-side
Ri-chárd
se-man-tics
STREU-SLE
Tie-de-mann
}
\hyphenation{
affri-ca-te
affri-ca-tes
an-no-tated
com-ple-ments
com-po-si-tio-na-li-ty
non-com-po-si-tio-na-li-ty
Gon-zá-lez
out-side
Ri-chárd
se-man-tics
STREU-SLE
Tie-de-mann
}
  %\togglepaper[]
}{}

\epigram{Study the science of art. Study the art of science. Develop your senses -- especially learn how to see. Realize that everything connects to everything else.}
\epigramsource{Leonardo da Vinci}

\shorttitlerunninghead{Integrated approaches to language assessment in language learning}
\begin{document}
\shorttitlerunninghead{Integrated approaches to language assessment in language learning}

\maketitle









\section{Rationale and significance of the book}



This book grew out of our intention to provide the reader with examples of what it means to design and implement research from a context-based perspective and how this view differs from other approaches to assessment. We address the following questions related to language assessment: Is it only instructors who assess language?; What other options become available to language testing inside and outside of a language classroom setting?; How do we assess different language profiles based on language exposure and experience outside of classroom boundaries?



This volume~attempts to introduce and address these questions in order to promote equitable access for assessment and initiate a conversation among scholars about inclusive practices in language assessments. Rather than universal or applicable (“it describes everyone”) to everyone, we work with the notion of teaching in multiple contexts. However, each context has specific needs that the contributors in this volume address based on the linguistic experiences and realities of their specific context. Whether the student is a second language learner, a heritage language learner, a multilingual language speaker, a community member, the authors in the present volume provide examples of assessment that do not follow a single universal or standardized design but an applicable one based on the needs and context of a given community.



The contributors in this volume are scholars from different disciplines and contexts in Higher Education. They have created and proposed multiple lower-stakes assignments and accommodated learning by being flexible and open without assuming that learners know how to do specific tasks already. Each chapter provides different examples on Justice, Equity, Diversity, and Inclusion (JEDI) assessment practices based on observation, examination and integrative notions of diverse language scenarios.



This volume is relevant at this particular point in time due to the need of addressing and validating contexts in which language assessment goes beyond the standard testing and evaluation practices. It also serves to provide examples of what it means to assess learners as a grassroots movement rather than a top down approach. Our intent is not to privilege one approach to assessment or one context over another, but to argue that researchers and practitioners may choose what they deem valuable research theories and techniques in their particular setting. We provide a descriptive approach to research in assessment.



This volume may be of interest to researchers and practitioners in the fields of curriculum and instruction, language learning, and applied linguistics as well as those in the field of language teaching in general.~In the following sections, we define and problematize key terms. We also provide an overview of the chapters in the volume.



\section{Definition of key terms}



In order to investigate the role of assessment in language education, this section explores and defines key terminology as it relates to assessment in multilingual settings. These key terms include assessment, evaluation, and testing. We contextualize these definitions based on the studies included in this volume and language learning in diverse contexts.



\subsection{Assessment}



Assessment goes beyond a final product and how students are able to use language. It includes formative and summative components in which learners are evaluated based on their ongoing performance. Historically, most of the research in Second Language Acquisition serves as the basis for a great body of literature in the area of language assessment. Many of these studies are based on psychology and variables that require data elicitation (\citealt{Ortega2017}; \citealt{BrownBrown2017}). These data are also self-reported in many instances, but they are measured. Examples of these are surveys and questionnaires measuring aptitude and attitudes as well as motivation. Foreign language testing and proficiency tests were developed as a response to the knowledge of foreign languages increasing with World War II and US involvement with Korea, and the realization that soldiers in these contexts became bilingual or multilingual \citep{BrownBrown2017}. This generated the first proficiency exams taking into consideration language abilities, language skills, and communicative competence. Placement tests became a venue to assess language.



In this volume, we provide examples of what could also be considered an ecological approach to assessment. The ecological perspective on language learning (\citealt{Larsen-Freeman2017}; \citealt{VanLier2010}; \citealt{VanLier1997}) explores relationships of many kinds in and across settings and systems as a way to examine relations and processes between learning and the environment. An ecological approach to language assessment acknowledges that an individual is not only a measure of variables in traditional types of assessment such as aptitude, attitude, and motivation. The ecological orientation of language assessment acknowledges that the individual is the result of multiple factors and interactions with the historical, sociocultural, and sociopolitical context. This involves the study of language anxiety, emotions, language ideologies, language policies, and power relations. In the formal environment of the classroom, individuals carry with them all these experiences, values, and beliefs. These experiences challenge educators to design assessment practices that acknowledge these dynamics in the context of formal education. This approach focuses primarily on the quality of learning opportunities, of classroom interaction and of educational experience in general.



\subsection{Evaluation}



  Evaluation is concerned with revising curricula and programs in order to provide a better view of what works or what needs to be modified. It includes a process that may lead to changes to curriculum and to ways in which students are being taught and assessed. For evaluation to be effective, there has to be an understanding of the linguistic needs of the learners as well as the resources available to them. These include the learning environment and an inventory of resources available to practitioners in order to generate curricula adapted to the learners’ needs. Language Policies as well as curriculum development are essential components of these evaluation processes and are usually the outcome. Evaluation is cyclical and necessary in order to support language learning. Research is a fundamental part of the evaluation process. The key element that helps us to understand how evaluation works in multiple settings is community involvement and sensitivity to recognize the needs of diverse communities of language learners. It is within this perspective that the exemplary studies in this volume illustrate what it means to be inclusive through language evaluation processes and how evaluation leads to curriculum design that works for a specific community.



\subsection{Testing}



Although there has been a shift in testing and adapting assessment to more authentic situations, standardized testing is still at the core of foreign and second language instruction. Literature in the field of foreign language education (FL) and L2 learning argue that standardized testing is not a direct reflection of the communicative competence in multilingual contexts \citep{Brown2010}. However, even though we continue to explore the role of standardized testing, it is still unclear how standardized testing assesses students’ ability to show communicative competence and pragmatic knowledge of foreign language learners, heritage language learners, multilingual learners in a variety of contexts.



Standardized testing is the direct result of a positivist epistemological stance in which the aim is to form generalizations based on results. Reliability is considered in terms of numbers and exams are designed to measure knowledge from a prescriptivist perspective (\citealt{Ortega2017}; \citealt{MenkeMalovhr2021}). This dates back to IQ testing and the bell curve in which results are measured and learners are compared based on their results.



Testing in diverse and multilingual settings raises the question of validity when designing diverse types of assessment. As we introduce the authors and their contributions to this volume in the following section, we also discuss how testing is conceptualized in the particular contexts of the studies represented in the chapters.


\section{Overview of the chapters}


The challenge of assessment in multilingual communities requires that educators contest more traditional and prescriptive notions of assessment to better serve their communities of learners. Some of these diverse contexts in higher education include different languages and different learner profiles. We feel it is timely to place the local contexts in assessment at the forefront of research and practice in multilingualism and linguistics studies within the landscape of a diverse perspective. As such, the different chapters of the volume serve as a collection of studies on the sociological, pedagogical and linguistic characteristics of language learning and assessment in different institutions and learning environments. Each chapter addresses a variety of different approaches and methods in language assessment, ranging from self-efficacy assessment tools such as Can-Do assessments to integrative approaches that include multiple sources of data beyond language proficiency. One of the overarching questions that guide the volume, related to the pedagogical challenges and needs heritage language learners face in terms of assessment, is examined in multiple chapters of the volume. The results reported in the chapters are a starting point for discussion about curriculum design and strategies to monitor and assess language growth of a variety of language speakers and learners within the classroom and beyond. The role of the local communities is also mentioned and discussed in most of the chapters as a way to organically integrate and connect instruction, language proficiency and language development.



 In Chapter 2, Gregory Thompson proposes three alternative assessments to better understand students’ abilities in the target language (Spanish) as well as their overall proficiency in Higher Education classroom-based contexts, namely, assessment in community-based language learning, Integrated Performance Assessments (IPAs), and student portfolios to document language growth and development. Thompson makes a strong recommendation regarding the importance of assessing for learning (vs. “of” learning). In this regard, he advocates for curricular implementation of formative and ongoing assessments that allow both the instructor and the learner to make a more explicit connection between what is taught and what is learned and assessed through constant and detailed feedback.



In Chapter 3, Eva Rodríguez-González, María de los Angeles Giráldez-Elizo and Sarah Schulman explore student self-perceptions for curricular alignment across different language Programs of Spanish language learning in Higher Education. The authors have used an assessment tool, a Can-Do Statements survey that highlights learners’ perspective and individual reflection of learning growth and development as perceived during the duration of a specific course. Similar to Thompson’s chapter, the authors also focus on Spanish language learning instruction and make recommendations for the multiple profiles of Spanish language learners in a college setting in the US, namely, Spanish second language learners and heritage learners. They also propose the implementation of Can-Do statements as both a reflection activity “for” learning and ongoing assessment.



In Chapter 4, Todd Hernández provides an overview of current research on pragmatic competence when learning second languages. It also describes the nature of the methodology used to assess second language pragmatic development in a study abroad context. Hernández identifies multiple assessment methodologies used to measure pragmatic knowledge in both informal and formal contexts. Based on results from a substantial literature review on pragmatic learning, Hernández highlights the importance of the potential of pedagogical interventions before and during a given study abroad experience to enhance pragmatic learning.



In Chapter 5, Kendra Dickinson and Glenn Martínez present the case of an integrated assessment of Healthcare Interpreting Competencies among Spanish heritage language learners. The proposed assessment includes student development and career-readiness from a variety of different perspectives. The authors share the results from Spanish heritage language speakers’ participation from the IMPACT program (Interpreters for the Medical Profession through Articulated Curriculum and Training). The results indicate that language proficiency, language attitudes and career decision self-efficacy were key factors that were positively affected by the participation in the program. They also highlight the implementation of the Career Decision Self-Efficacy Scale Form as a useful measure of career-readiness and assessment tool to identify areas of need and specific intervention.



In Chapter 6, Gláucia Silva analyzes assessment in community-based heritage language schools and the case of Brazilian Portuguese in the United States and calls for an inclusive approach to assessment that includes families, learners and educators. Silva shares the results from a survey distributed to Brazilian Portuguese language teachers and school administrators in the U.S. The profile of learners in her study includes heritage of Portuguese and younger populations from previous chapters of the volume. The answers provided on the survey call for consideration of conversations to parents regarding the use of Portuguese at home as a key factor for linguistic and cultural growth. In addition to the importance of including family when maintaining and developing Portuguese as a heritage language, Silva recommends the consideration of organizing activities related to Brazilian traditions as a way to make explicit connections between the school and the community. She also addresses the complexity behind placement and provides multiple suggestions regarding assessing children’s progress in Brazilian Portuguese in community-based heritage language programs.



In Chapter 7, Rosa Vallejos, Fernando García and Haydée Rosales Alvarado discuss indigenous languages in Higher education in case studies from the Amazon of Peru. They focus on Kukama and Kichwa as heritage languages. Instruments for assessment include videos, picture cards and a survey addressing the nature of social and cultural factors that play a role in learners’ attitudes towards dialectal and generational varieties, self-assessment and motivation of heritage language abilities and language choices. The use of multiple assessment tools allows the authors to connect data and address assessment from a holistic point of view and as an inclusive practice that monitors individual linguistic and cultural growth and development. Given the degree of endangerment of the languages under study, the authors identify as an important finding of this chapter the fact that endangered languages can be relearned in well-structured instructional settings.



In Chapter 8, Rosita L. Rivera and Eva Rodríguez-González provide a concluding essay that discusses context-based approaches to assessment as an eclectic approach that requires a robust knowledge and understanding of the linguistic diversity of language learners. This chapter makes the case for a linguistic sensitivity and pedagogical training beyond prescriptive methodologies. It also calls for a more inclusive approach to the design and implementation of assessment in higher education. The chapter ends by drawing on the findings from previous chapters and posing the argument that an organic approach to assessment in education and linguistic diversity in multilingual contexts deserves a place in language research. Pedagogical implications and challenges posed by the need to assess in times of crisis and the impact of technology on assessment during recent events are also discussed.  Suggestions for further research in assessment in multilingual contexts are also made based on these implications.


\sloppy
\printbibliography[heading=subbibliography,notkeyword=this]
\end{document}
