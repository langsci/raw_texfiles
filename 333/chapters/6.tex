\documentclass[output=paper]{langscibook}
\ChapterDOI{10.5281/zenodo.6762280}

\author{Gláucia V. Silva\affiliation{University of Massachusetts-Dartmouth}}
\title[Assessment in community-based heritage language programs]
      {Assessment in community-based heritage language programs: The case of Brazilian Portuguese}
\abstract{In heritage language (HL) education, testing for administrative motives (i.e., student placement) has received special attention \citep{Fairclough2012b}, and with good reason. Given the heterogeneity that characterizes HL learners, it is important to be able to place them adequately. However, it is also important to be able to determine whether learning goals are being met. According to \citet{Carreira2012a}, it is essential that educators utilize formative assessment in HL classes, which would allow them to address issues of learner diversity. Most of the literature dealing with HL assessment, however, is based on university-level education. We know little about how community-based HL schools in the United States assess learner progress and determine student readiness, and even less, if anything, about the assessment of linguistic and cultural skills in less commonly taught HLs, such as Portuguese. This chapter aims to shed some light on issues of placement and of assessment of learning in community-based HL schools by presenting data from a survey distributed to Brazilian Portuguese language teachers and school administrators in the U.S. Results indicate that there is an array of behaviors in relation both to administrative and to instructional assessment in these schools, which range from grouping only by age to using tests to place learners and assess their progress. Based on the available literature and on the data analyzed, the chapter also presents suggestions regarding assessment in Brazilian Portuguese community-based HL programs and possibly others.
\keywords{assessment, heritage language, community-based programs, Brazilian Portuguese}
}

\begin{document}
\AffiliationsWithoutIndexing{}
\maketitle



\section{Introduction}

  For the past few decades, researchers have emphasized that heritage language (HL) learners present specific needs that differ from those of foreign language (FL) and of monolingual learners. According to \citet{Valdés1981}, HL teaching should provide opportunities for learners to develop their oral proficiency and their listening skills, which are normally at a higher level than that presented by FL learners, besides their reading and writing abilities. The expansion of learners’ bilingual range, according to \citet{Valdés1997}, should be one of the goals of a heritage language class. She argues that this expansion may be difficult to attain in courses designed for FL learners.

  Besides exhibiting different pedagogical needs from those of FL learners, HL learners also diverge from their FL counterparts in relation to the spectrum of linguistic abilities. While learners in a beginning FL class generally (though by no means always) start out knowing nothing or very little of the language, HL learners display a wide range of abilities (e.g., \citealt{WangGreen2001}; \citealt{BeaudriePotowski2014}). This spectrum of broad linguistic skills is captured in the definition of a heritage language learner proposed by \citet[38]{Valdés2001}, which refers to someone who speaks or at least understands the HL and is, therefore, bilingual to a certain extent. This range of abilities may be related to the special attention received by diagnostic assessment in HL education \citep{Fairclough2012b}. Given the variation in linguistic abilities exhibited by HL learners, placing students in the adequate courses is essential.

  Tests have ideological force, as pointed out by \citet[54]{Leeman2012}, and traditional language exams (i.e., those that emphasize grammar and spelling, for example) “devalue or erase the conversational, pragmatic, and cultural arenas where many SHL [Spanish as a heritage language] students excel.” Recognizing that the needs of HL learners merit revisiting the way language testing was often carried out, researchers and practitioners have proposed exams that include ways of assessing the abilities that HL learners already possess and what they need to develop. However, much of the discussion revolves around university-level courses. We know little at this point about what happens in relation to assessment in HL community-based schools, especially in the case of languages that are less commonly discussed in the literature.

\begin{sloppypar}
The \citet{TheCoalitionofCommunity-BasedHeritageLanguageSchools2018} explains that, typically, HL community-based schools in the United States are non-profit organizations founded and operated by members of the immigrant or heritage language community. The Coalition adds that the goal of these schools is to maintain and teach the language and culture of the immigrants’ heritage (but note that such schools may also teach and maintain indigenous languages/cultures), and that they may offer classes for learners from Pre-K to Grade 12. HL community-based schools often operate in rented spaces on weekends or after school during the week. Even if these spaces are located on public or private school premises, community-based schools are not connected to school systems, nor are they subject to the regulations of the U.S. education system. Regarding assessment, the Coalition states that community-based schools may choose to administer U.S. language tests or tests used in the home country when appropriate assessments are available.
\end{sloppypar}

\begin{sloppypar}
\citet{Lu2020} reports that there are 29 Portuguese language community-based schools in the U.S., a number based on a survey made available by the Coalition of Community-Based Heritage Language Schools (\url{https://www.surveymonkey.com/r/HLProgram}). However, many Portuguese language community school leaders have not completed the survey. The 29 programs reported by Lu include both European and Brazilian Portuguese community-based schools; this chapter reports on assessment practices in Brazilian schools only. A discussion of the differences between European and Brazilian Portuguese is outside the scope of this text. For our purposes, suffice it to say that, although there are schools that serve mainly the Brazilian community and others that serve mainly the Portuguese community, sometimes students of Brazilian heritage attend a community-based school of Portuguese heritage, or vice-versa.
\end{sloppypar}

\begin{sloppypar}
One example of Brazilian Portuguese community-based school in the United States is ABRACE (\textit{Associação Brasileira de Cultura e Educação}), which was founded by three immigrant Brazilian mothers \citep{CenterforAppliedLinguistcs2013,ABRACE2020}. ABRACE’s mission is “to preserve and promote Brazilian education and cultural and social integration in the Brazilian community with the aim of strengthening its identity within the United States” \citep{CenterforAppliedLinguistcs2013}. Funding for ABRACE comes from student tuition, occasional sponsorship by Brazilian companies that operate in the U.S., individual donors, and occasional support from the Brazilian government. Many other Brazilian Portuguese programs, however, are funded solely through student tuition. In relation to assessment, the information for ABRACE mentions that there are no standardized tests available for children and youth \citep{CenterforAppliedLinguistcs2013}.
\end{sloppypar}

This chapter illustrates what may happen in HL education of young learners by discussing assessment practices in several Brazilian Portuguese community-based schools. Starting from what is already in use, the chapter also outlines suggestions of assessment tools that can be utilized by community-based Brazilian Portuguese HL schools and possibly others.

After this brief introduction, this chapter is divided as follows. \sectref{sec:6:2} presents some of the previous research on HL assessment and is followed by a discussion of community-based schools (focusing on Brazilian Portuguese) in \sectref{sec:6:3}. \sectref{sec:6:4} introduces the study, including methodology and participants, while \sectref{sec:6:5} presents the research findings. A discussion of those results, as well as suggestions regarding assessment, are presented in \sectref{sec:6:6}. Final remarks are offered in \sectref{sec:6:7}.

\section{Assessment in heritage language education}\label{sec:6:2}

  The HL education field has been growing steadily since at least the early 1980s, when \citet{ValdésGarcía-Moya1981} published their seminal edited volume on teaching Spanish to heritage leaners (at that point, referred to simply as “bilinguals”, which they are). In that volume, the chapter by Janet \citet{Ziegler1981} discusses placement examinations, calling attention to which skills, according to her, should be tested when placing heritage speakers of Spanish into Spanish language courses, including issues related to morphology and syntax.

  The importance of assessment for placement purposes is evident in the HL education field, as pointed out by \citet{Fairclough2012b} and \citet{Carreira2012a}, due to the great diversity found among HL learners. \citet{MacGregor-Mendoza2012} calls for placement tests that are informed by current research and that tap into learners’ oral, aural, and pragmatic abilities, and include a background survey (which, in the case of children, may be completed by parents or caretakers) as well as face-to-face interviews. \citet[126]{Fairclough2012a} suggests that HL placement tests should measure three general areas: receptive skills (such as general vocabulary), productive skills (focusing on linguistic gaps, dialectal forms and language transfer), and creative skills (speaking and writing, if appropriate, that reflect a range of functions and contexts). \citet{Carreira2012b} argues that the data gathered from placement exams should be used to inform syllabus design on a regular basis (a point also defended by \citealt{IlievaClark-Gareca2016}).

In a review article focusing on assessment of HL learning at the university level, \citet{Son2017} shows that discrete-item tests are the most common type of placement exam for that level. However, these exams often utilize a combination of methods to better assess students’ abilities. \citet{Son2017} espouses (as do \citealt{BeaudrieDucar2012}) the notion that placement exams must address the needs of specific programs and students. The idea that each program needs its own exam may account for why HL educators may not be able to use or adapt placement tools made available by the National Heritage Language Resource Center (\url{https://nhlrc.ucla.edu}), as \citet{Carreira2014} notes.

In a discussion about assessing the language of young learners, \citet{Bailey2017} calls attention to complexities involving bilingual first language acquisition, including the fact that children who acquire more than one language simultaneously may become literate in only one of their languages. For diagnostic purposes, Bailey suggests that processing demands may be lessened with verbal scaffolding that would help elicit responses from young test takers, a strategy that would generate diagnostic information.

  Beyond placement tests, assessment is often categorized as either summative or formative. Summative assessment is an exam that “evaluates learning after instruction for purposes of assigning a grade or determining the efficacy of particular programs or interventions” \citep[100]{Carreira2012a}. \citet{Carreira2012a,Carreira2012b} asserts that formative assessment, conceptualized as assessment for learning (as opposed to assessment of learning), is ideal for HL education. She ties formative types of assessment with differentiated teaching, which, she argues, is an instructional approach that meets the pedagogical needs of HL learners, given the diversity found among these students \citep{Carreira2012b}. Among the activities that lend themselves to formative assessment, \citet{Carreira2012a} lists exit cards, journals, portfolios, surveys, oral interviews, and presentations. She maintains that these types of activities provide information about each learner, making them ideal for differentiated instruction: instructors can assess differences among learners’ linguistic abilities as well as attitudes and goals, and then attend to them. \citet[152]{Beaudrie2016} goes a step further and states that “[d]ifferentiated assessment complements differentiated instruction, seeking to provide all students with multiple opportunities to demonstrate their learning and progress.”

  \citet{Bailey2017} maintains that assessment \textit{for} learning is especially relevant in the case of young learners, since they are still acquiring the language. She adds that formative assessment “can capture a broad array of relevant language information for teachers that is closely tied to the young learners’ instructional needs” \citep[329]{Bailey2017}. A central focus of formative assessment, Bailey states, is teacher feedback to students; students may also self-assess their language learning.

Formative assessment can make use of performance-based tasks, as argued by \citet{IlievaClark-Gareca2016}, who highlight the need to take into consideration the abilities of HL learners in their totality. Following \citet{BrownAbeywickrama2010}, Ilieva and Clark-Gareca contend that simulations of real-world tasks lend themselves to testing of linguistic abilities in more authentic ways and that, through low-stakes performance tasks, “teachers can create an excellent learning environment that boosts student motivation and positive attitude toward learning the languages” (\citealt{IlievaClark-Gareca2016}: 227). The researchers recommend a model of assessment that reveals nuances of learners’ proficiency by incorporating multiple modality assessment strategies consistent with four principles: (1) centrality of authentic contexts; (2) multiplicity of measures; (3) diversity of feedback; and (4) reliance on research (\citealt{IlievaClark-Gareca2016}: 229--30). They maintain that this model offers valuable information to teachers on what kind of instruction, topics, and structures need to be practiced in the HL classroom.

\citet{Bailey2017} highlights that the content of the assessment tasks must be relevant to young learners both in terms of cognitive demands and of cultural specificity. According to Bailey, the younger the learner, the higher the need for contextualization in assessment: items in a test need to be “topically appropriate for the target age of the test taker, and the ability to answer the items should not require knowledge of information not already provided in the tasks or test items” \citep[332]{Bailey2017}. For the youngest learners, Bailey mentions that toys may be incorporated in questions and in response formats, since young children are more successful in production and comprehension tasks if objects rather than pictures are used. Still, Bailey recognizes that choosing age-appropriate content for tests for young learners is complex because language development is concurrent with developments in other areas.

\citet{Elder2005} considers the role of testing in HL education in selected Australian schools and discusses dilemmas faced by evaluators in implementing testing programs and interpreting test results. The programs discussed by Elder included two schools for primary school-age children and two for secondary school-age learners. As mentioned above, HL community-based schools may serve learners of all the ages included in \citegen{Elder2005} study.

Formal testing may receive more attention in educational systems such as those included in \citegen{Elder2005} study than in HL community-based schools. However, both types of schools may face similar challenges in relation to assessment. Some of the issues mentioned by Elder are: lack of expertise (teachers may not be experts in language assessment); limited opportunities for test piloting; appropriateness of level for each learner; difficulty devising instruments that reflect a program’s curriculum at a specific point and also serve as indicators of language achievement over a more extended period. Given these (and other) challenges, Elder outlines recommendations for effective use of heritage language tests in Australian schools. Elder highlights the importance of systematically documenting relevant aspects of HL learners’ home language background, including language exposure, a recommendation that may be useful for HL community-based schools as well. Elder argues that such sociolinguistic profiling is essential in the beginning and throughout the program, and should be taken into account in the interpretation of test results.

  This section has reviewed some of the relevant literature regarding general assessment practices in heritage languages. The next section discusses community-based heritage language programs, which strive to help children maintain their heritage language and culture, despite the many challenges faced, including the lack of adequate assessment instruments.

\section{Assessment in community-based heritage language programs}\label{sec:6:3}

  Research on the assessment of HL children tends to revolve around bilingual programs (e.g., \citealt{Lucero2018}) or how to assess their progress in mainstream education (e.g., \citealt{Gonzalez2012}). Practices in community-based schools have received less attention, even if, around the world, these programs support efforts to maintain and develop the linguistic and cultural skills of HL learners. Historically, these schools have been established and supported by groups interested in the maintenance and development of their languages and cultures (\citealt{BradunasTopping1988}), though they may also be sponsored by both the community and a local public school or community college \citep{Compton2001}. As \citet{Douglas2005} notes, in the North American context most pre-college HL instruction is provided by these schools. Establishing and maintaining such schools involves many challenges, as outlined, for example, in \citet{Compton2001} and \citet{LiuEtAl2011}. Calling attention to the wide range of linguistic skills among HL learners, \citet[155]{Compton2001} maintains that “new approaches to placement, testing, teaching and learning for heritage language students are crucial.” At that time, Compton states, instruments for assessing HL learner skills were still in developmental stages.

The diversity of students’ language skills and backgrounds is, according to \citet{LiuEtAl2011}, one of the challenges faced by community-based HL programs. Although \citeauthor{LiuEtAl2011} mention that educators “would like information about placing students when their proficiency levels and backgrounds differ” (\citealt{LiuEtAl2011}: 5), the suggestions offered to address this particular challenge focus on instructional approaches and materials, not on assessing learner level for placement reasons. Assessment of learning outcomes is not included in other challenges related to instruction either (such as teaching materials and instructional time), possibly because other issues are considered more pressing for those schools. However, community-based HL programs may not be seen as schools by teachers and administrators. In an analysis of the curriculum of a Brazilian Portuguese HL program, \citet[83]{Boruchowski2014} explains that the limited hours of contact with the children and the fact that teachers do not “grade or use measurement tests to evaluate their students” leads teachers and administrators not to classify their own program as a school. Thus, assessment (or lack thereof) may be directly related with how HL programs are perceived by those responsible for instructional decisions.

In an article that proposes a theoretical framework for curriculum development for Japanese HL schools, \citet[71]{Douglas2005} highlights that assessment methods must reflect the principles that were used as a basis for the curriculum. She calls attention to the elements of ideal assessment listed by \citet{GutierrezSlavin1992}, the principles of which can serve as a framework for developing assessment tools for young HL learners. The first element in the list seems especially applicable to HL learners, given the range of abilities found among them: “Children are evaluated in terms of their own achievement and potential, not by comparison to group norms. Expectations differ for different children” \citep[71]{Douglas2005}. The list also contains elements that relate to formative assessment, which, as \citet{Douglas2008} and \citet{Carreira2012a,Carreira2012b} maintain, is individualized in nature and, thus, necessary to address the needs of HL learners. \citet[72]{Douglas2005} argues that assessment tools for young HL learners should address academic language as well as basic communication skills, given the varied HL environment to which children are exposed.

\citet[256]{Douglas2008} specifies that “in order to assess oral language development, assessment is conducted whenever possible while students are engaging in authentic meaning making activities.” Assessment of oral skills can be recorded in different ways, such as checklists, teacher observations, reports, and student interviews, among others. \citet{Douglas2008} also discusses assessment of reading and writing abilities, providing examples of skills and categories that can be assessed. Results may reveal a mismatch between learners’ needs and the curriculum objectives. When this happens, \citet[259]{Douglas2008} recommends that instructional goals be adjusted, reestablished or repeated.

The evaluation of curriculum objectives based on assessment results is also proposed by \citet{Boruchowski2015}, who, following \citet{WigginsMcTighe2005}, maintains that assessment should be used to assess whether instructional activities did in fact help young HL learners reach learning goals. The importance of establishing curriculum goals for young HL learners is highlighted by \citet{SilvaBoruchowski2016}, who recommend that educators examine learners’ history of HL use and schooling in order to set specific objectives. Like \citet{Douglas2005,Douglas2008}, \citet{SilvaBoruchowski2016} also assert that educators use formative assessment for young HL learners, including checks for understanding as well as performance tasks and projects.

Though not exhaustive, this literature review provides a base for discussing assessment in Brazilian Portuguese HL community-based programs. The next section describes the present study and is followed by a presentation of the answers elicited by the survey used.

\section{The present study}\label{sec:6:4}


  Heritage language community-based schools play an important role in preserving linguistic and cultural ties \citep{Kondo-Brown2010} and, thus, enriching any society. However, we do not have much information about how these schools assess learner progress and determine student readiness, and even less, if any, about the assessment of linguistic and cultural skills in less commonly taught HLs, such as Brazilian Portuguese. This study aims to shed some light on assessment practices in these HL schools and was guided by the following questions:

\begin{itemize}
\item  \begin{sloppypar}How are students placed in Brazilian Portuguese community-based schools?\end{sloppypar}
\item  Do these schools assess learner progress during the academic year? If so, how?
\item  Do these schools assess learner progress at the end of the academic year? If so, how?
\end{itemize}

In order to answer these research questions, approximately 80 educators connected to HL schools that teach Brazilian Portuguese in the United States (specifically, in Massachusetts and in Florida) were invited, via email, to participate in an anonymous online survey. Many (perhaps most) of these schools were created by community members who may have identified a desire by Brazilian immigrants for their children to maintain and develop their language and culture. \citet{BlizzardBatalova2019} show that, according to American Community Survey data, approximately 450,000 Brazilian immigrants lived in the U.S. in 2017. The authors add that about 32\% of those immigrants resided in Florida and Massachusetts. These patterns of immigration may account for the creation of several community-based schools in these two states. Although I do not have specific information about the schools these educators were connected with, in my experience most Brazilian Portuguese HL community-based schools are funded with student tuition, though some may establish partnerships with Brazilian companies and/or obtain limited funding from the Brazilian government, as mentioned earlier in the case of ABRACE.

About two weeks after the first email, another email invitation to take part in the anonymous online survey was sent to the same list. The survey contained nine questions (see \hyperlink{appendixA}{Appendix A}), four of which required an answer, while the other five were dependent on other questions, that is, five questions would be answered depending on the answer provided in the required questions. Six questions were open-ended, which leads to a mostly qualitative analysis of the results. Questions addressed placement, assessment during the academic year, and summative assessment. A total of 19 respondents completed the survey, but answers by one participant were excluded because it was evident that s/he does not work at a community-based HL school, but rather at a regular school (possibly in a bilingual program). Answers were collected within the online survey tool used to design and distribute the survey. The results are presented in the next section.

\section{Survey results}\label{sec:6:5}

  This section presents the answers to the online survey provided by 18 participants. The first question (Q1) in the survey addressed the period before classes started, asking whether anyone at the school spoke with the parents or guardians of new students. All of the respondents answered that there were in fact people who spoke with parents prior to the child joining the program and proceeded to answer the second question (Q2), which depended on an affirmative answer for the first and focused on the goal of the conversation. Since Q2 was open-ended, a few participants mentioned more than one objective for the conversation with the parents of new students (and, in the case of one respondent, parents of returning students as well). The answers revealed some common themes, as well as topics that were unique among the participants. The list below summarizes the reasons for the conversation with the parents of new students:

\begin{itemize}
\item  to explain logistical issues about the school and classes, as well as their goals and methods (nine respondents);
\item  \begin{sloppypar}to elicit information about the child’s linguistic abilities (eight respondents);\end{sloppypar}
\item  to raise parents’ awareness of the importance of the Portuguese language in their children’s lives (four respondents);
\item  to discuss the role of the Brazilian Portuguese school in the community (one respondent);
\item  to have parents fill out a form with questions about the child (one respondent);
\item  to learn what the child expects from the classes (one respondent).
\end{itemize}

The third question in the survey (Q3) was one of the multiple-choice questions. It sought to elicit how new students were placed (and answer the first research question guiding this study). Possible answers were: (a) according to age only; (b) according to the result of an evaluation and/or interview; (c) according to age and the result of an evaluation and/or interview; and (d) other. No participant elected “other”. Most respondents (n = 10) chose (c): their schools adopt a mixed approach to placement, which is based on age as well as the result of some type of evaluation of the child’s abilities. Six participants reported that students are placed according to age only, while placement in two schools follows the result of a test and/or interview.

Participants who did not choose (a) in Q3 (n = 12) were invited to answer Q4, Q5, and Q6. Q4 asked respondents to describe the instruments used to place new students. A combination of interview and a written test was mentioned by five respondents, while three participants answered that their institutions placed new students after an interview only. Another two respondents stated that new students took a test (“a little evaluation”; “a vocabulary and reading test”). Two participants mentioned a combination of age and a survey filled out by parents, including one who explained that very young students (2--4 years old) were placed according to age only.

Next, Q5 sought to elicit who is in charge of assessing the new students, and how long that assessment takes (be it an interview, a test or a combination of both). Six respondents said that either the school director or a pedagogical coordinator assessed new students, while five participants answered that the teacher was responsible for that assessment (two of those respondents mentioned that such assessment was carried out in the first few classes). One participant wrote simply “according to each student’s needs.” One participant who had checked (a) in Q3 (new students are placed according to age only) did answer Q5 and explained that they “assess children’s progress every class”, and if they notice that a child is mature for his/her age and is more advanced than their class, they speak with the parents and, if everyone agrees, the child is moved to another level (provided there is not a significant age difference between the child in question and those in the more advanced level). As for how long the assessment to place new students takes, only six participants specified the duration, which ranged from 10 minutes to four classes (or one month, since classes take place on Saturdays), including 15 minutes, 30--45 minutes, 60 minutes on average, and one class.

Still following up on placement of new students, Q6 addressed when it took place and when and how parents were informed of the decision about placement. Participants indicated that the assessment for placement either took place before classes started (n = 3) or was carried out in the first or first few classes (n = 3). The other six respondents did not provide an answer regarding when placement happened. As for communicating a decision to parents, most respondents (n~=~9) indicated that parents were informed of their children’s placement, either in a meeting or conversation (n = 7), or through a copy of the evaluation (n~=~2). One participant did not specify how the parents were informed. One respondent stated that “(it) is information that pertains to the school only”, and it would only be shared if parents demanded it (though it is not clear what “it” may refer to). The other two respondents did not specify how parents were informed about placement decisions.

Results regarding placement procedures address the first research question in this study. Unsurprisingly, we find variation in approaches to placement among Brazilian Portuguese HL programs. Student placement may be done based on age alone, or may involve other elements, such as interviews with parents and/or with students, a type of test, a few classes, or a combination of elements (e.g. interview and test; age and survey filled by parents). When placement is based on more than a child’s age, the process (interview, test, survey, class) may take from 10 minutes to four classes and may be carried out by a program director, a pedagogical coordinator, or a teacher.

Q7 in the survey sought to address the second research question in this study. It required an answer, was open ended, and looked for information on assessment during the academic year: whether it was carried out, what instruments were used, what kind of feedback was provided to students and/or parents. Q7 also requested examples of assessment and/or feedback, if possible. Every respondent answered the question, although most did not address all the points in the question. Most participants (n = 12) mentioned that assessment was ongoing and was done by means of in-class or homework activities, which provided the teacher with information about student development. Two respondents (including one who had mentioned reports) mentioned tests: “a type of exam” and “written and oral tests”. As for feedback, four participants alluded to a report: three specified that the report was sent to parents, while one respondent wrote simply “a descriptive report”, without further details. Five respondents indicated that teachers and/or coordinators spoke with parents about their children’s progress. The answer provided by one participant did not shed light on assessment instruments or feedback: “[a] meeting with moments of integration”.  Importantly, one of the answers that mentioned that assessment was done through in-class activities also stated that the teacher and the coordinator analyze student performance, and “new strategies are implemented if students don’t reach the learning goals” established for the activities. This points to formative assessment, since teaching strategies are adapted depending on how students do in the activities proposed.

\begin{sloppypar}
  The results for Q7 address the second research question in this study: whether (and how) Brazilian Portuguese community-based HL schools assess learner progress during the academic term. For the most part, results of in-class activities appear to provide teachers with the information they need about learner development during the term. In some cases, there may be more formal ways of assessing learner progress, even if the tests used are not considered to carry the kind of formality normally associated with them. One participant indicated that lesson plans may respond to students’ needs as evidenced by in-class activities.
\end{sloppypar}

Like Q7, Q8 also required an answer. This multiple-choice question addressed summative assessment, asking whether students were evaluated at the end of the academic term. The options were (a) yes; (b) no, because we assess and provide feedback during the term; (c) no, because groups move together regardless of results; and (d) other. Ten participants answered (a), six chose (b), one chose (c) and one responded (d) (“The teacher evaluates students’ development during the whole year”). Participants who answered (a) and (d) were asked to also answer Q9, which asked respondents to explain how assessment is carried out at the end of the academic term. However, one of the participants who answered (a) in Q8 did not answer Q9. The results for Q9 show that, for the most part, assessment at the end of the term is not done formally: only three participants mention some kind of test. One respondent wrote the word “exam” in quotation marks, which suggests that s/he does not consider the end-of-semester exam to be the type of formal exam to which s/he may be used. Two participants mentioned that the children do a presentation at the end of the academic year. Another two referred to observing development through in-class activities, while one alluded to a “descriptive evaluation” of each student. Interestingly, two respondents indicated that classes generally do move together (even though they chose option (a) in Q8, not option (c)), with possible exceptions of children skipping a level or being held back.

The third research question that guided this study is addressed by the results of Q8 and Q9. We see that at least one program among those that participated in the survey chooses to move students together, regardless of possible achievement (incidentally, the answer Q7 did not suggest that there was any form of assessment during the term either). In other cases, programs either carry out some form of assessment (such as presentations or even tests) or rely on information gathered during the term only. The next section turns to a discussion of these results in light of the literature. It also presents suggestions of assessment approaches that can be utilized by Brazilian Portuguese community-based HL programs and possibly programs in other less commonly taught languages.

\section{Discussion and suggestions}\label{sec:6:6}

  The data gathered reveal that it seems to be common for Brazilian Portuguese community-based programs to invite parents of new students for a conversation, mostly to explain logistics and to have an idea about whether the child uses Portuguese, and if so, how much. However, other objectives were also revealed in the answers, suggesting that administrators and teachers may view the organization as more than a language school. Raising parents’ awareness of the importance of the Portuguese language for their children is an important goal that this conversation may serve \citep{Boruchowski2014}, as revealed in the data. The relevance of bilingualism/multilingualism cannot be overstated; however, in the United States, many still see bilingualism as a problem rather than an asset. As \citet[342]{Foster1982} puts it, “bilingualism is seen by many as evidence of insufficient assimilation.” Due to this ideology and the “monolingual bias [\ldots] that views bilingualism [\ldots] as something that should be eliminated” \citep[67]{Block2007}, parents may be unaware that they need to use the Portuguese language at home if they want their children to learn it. As \citet[224]{Lico2015} notes, the role of community-based schools is “not to make up for or ‘fix’ what is not done at home; after all, the basis of this process is to recognize and to value the [children’s] heritage, of which the parents are the source” (my translation). Thus, some school administrators and teachers feel the need to explain to parents that their use of Portuguese at home is essential for their children’s linguistic and cultural development.

Besides the role of the family in maintaining and developing children’s HL, the role of the school in the community may also be the topic of the conversation with parents, as mentioned by one of the participants. Community-based schools may organize activities around Brazilian traditions (\citealt{Quadros2017}; \citealt{Souza2017}), providing the diasporic community with an opportunity to gather, meet, and celebrate their traditions. These schools may also teach Portuguese as a foreign language to adults and teach the majority language to speakers of Portuguese (\citealt{GodoyLitran2017}), as well as invite the community at large to discuss bilingualism and its advantages \citep{Lira2017}. Thus, the role of community-based schools may go well beyond helping children maintain and develop their HL.

\begin{sloppypar}
  Although Brazilian Portuguese community-based programs may do much more than teach language, linguistic and cultural development is undoubtedly their main objective. As discussed in \sectref{sec:6:2} and \sectref{sec:6:3}, placing students in HL classes is a challenge, be it in university-level courses or in community-based schools. Placement methods elicited by this study vary: while one-third of respondents stated that their schools place new students only according to age, many schools adopt more than one criterion to determine what class a new student should join. Aside from age, interviews and/or tests may be used to place students, as well as surveys that parents fill out. The assessment related to placement may be conducted by a teacher or by someone who fulfills another role, such as the program director or a pedagogical coordinator.
\end{sloppypar}

  Differences in placement procedures are to be expected, as evidenced by the literature reviewed in \sectref{sec:6:2}. While age is a very important factor in grouping children, HL programs should also consider a child’s linguistic and cultural skills, as some already do. If a school has enough students to warrant more than one class per age group, then even children as young as four years old can be placed according to their ability in Portuguese. Some suggestions regarding university-level placement exams can be useful for community-based HL programs as well. \citet{Fairclough2012a} recommends that receptive, productive, and creative areas be measured. In the case of children, listening and speaking skills would be assessed; reading and writing may apply in the case of older children, to determine whether they have already developed some ability in those domains. However, it is important to focus on language use, not exclusively on metalanguage (i.e., names of linguistic structures or grammatical terminology). Furthermore, children who learn Brazilian Portuguese as a HL are bilinguals and may display linguistic behaviors that are common among that population, such as code-switching, which demonstrates bilingual competence \citep{Carvalho2012}. However, it is important to keep in mind that a bilingual is not two monolinguals in one and does not develop identical strengths in both languages \citep{Valdés2001}. Therefore, it is not realistic to assess HL learners as if they were monolingual speakers of Brazilian Portuguese; this fact needs to be considered in the development of placement tools. Instead, placement instruments should reflect the local linguistic context, as suggested by \citet{MacGregor-Mendoza2012}, so as to assess learners’ abilities in their local circumstances.

The instruments used for placement purposes should be developed by each program, taking into consideration the linguistic and cultural experiences of the group these programs serve (\citealt{BeaudrieDucar2012}; \citealt{MacGregor-Mendoza2012}; \citealt{VergaraWilson2012}; \citealt{Son2017}), as well as the mission of the program \citep{Fairclough2012a}. There should also be ongoing development of assessment tools to respond to learners’ needs and to improve the placement instruments (\citealt{BeaudrieDucar2012}). Furthermore, the data gathered should inform the curriculum of the program and the syllabi for each class (\citealt{Carreira2012b}; \citealt{IlievaClark-Gareca2016}). Most community-based HL programs may not have the means to implement electronic placement tests, which would make data compiling easier \citep{FaircloughBermejo2010}, so it is important to take care to develop placement tools that do not make data gathering a cumbersome process. While programs should consider their own context in devising these tools, each program should not need to “reinvent the wheel.” On the contrary: community-based schools should exchange best practices in order to find out if strategies adopted by other programs may be applicable to their own. Institutions such as universities and consulates may sponsor periodic events geared towards the exchange of best practices and invite representatives of community-based programs in their regions.

  Assessment of student progress during the academic term can also vary quite a bit among community-based schools, as attested in the literature. Some of the schools included in \citegen{BradunasTopping1988} report tested their students regularly, whereas others left it to the teachers to monitor pupils’ progress, and others had informal types of assessment (such as spelling contests). The survey conducted for the present study reveals that, today, Brazilian Portuguese community-based schools also adopt different strategies regarding assessment of progress during the academic term. Most participants in the survey stated that their schools assess student progress during the year, which is done with in-class or homework activities. However, there were no details provided regarding the types of activities that may elicit evidence of development. Two participants did mention tests but did not provide specific information about the structure of such tests. Only one participant mentioned that student performance is analyzed and that new strategies are adopted depending on whether learning goals were reached. This type of analysis suggests that at least some schools may adopt formative types of assessment that allow for instructors “to adapt their teaching so as to attend to the needs of all learners” \citep[115]{Carreira2012a}. As \citet{Carreira2012a} emphasizes, formative assessment helps instructors address issues of learner diversity, which are undoubtedly present in any HL classroom. Even though the examples provided by Carreira are from a college-level class, some of the activities may be used in community-based programs that meet only once a week. Exit cards can quickly provide information about the day’s lesson. For example, children who can already write may be asked to list new words learned, or to form a sentence with the structure practiced, or to give feedback on a game. Younger children may be asked to draw words they have practiced, or to color a certain number of squares, or to use certain colors -- always receiving instructions in Portuguese. Children who can write may be asked to keep journals, which they may share with the teacher either in class or electronically. These may be reading journals, as suggested by \citet{Carreira2012a}, or simply journals about what was interesting (or not) in a lesson. The teacher can then identify whether students perceive a given activity as effective, whether they can control linguistic structures that have been studied and/or whether they have grasped a certain cultural concept. Lessons would then be tailored to the needs of the learners as expressed in such assessment tools. This process results in differentiated teaching and learning, which, as \citet{Carreira2012a} and \citet{Beaudrie2016} note, is ideal for HL learners. It is also important to keep in mind that assessment should also be differentiated and children should be evaluated in relation to their own achievement \citep{Douglas2005}, not by comparison to monolingual norms.

  With respect to summative assessment, at least one respondent mentioned that students move together as a group and no formal end-of-term assessment is carried out. Other answers indicate, however, that some kind of assessment takes place at the end of the academic term in some schools. Nevertheless, end-of-term assessment is mostly done informally, not unlike what was reported regarding assessment during the term. Even when there is a test, little formality seems to be attached to it, as suggested by the use of quotation marks around the word “exam” by one of the participants. This informality is not a negative aspect: formal evaluations would suggest to children that the HL program is just like regular school, an idea that is certain to demotivate students. Given that student recruitment is one of the challenges faced by these programs \citep{LiuEtAl2011}, schools need to do what they can to keep students motivated so they will remain in the program.

Presentations by students, which was mentioned by several respondents, may constitute a formal type of summative assessment. At the end of the academic term, several Brazilian Portuguese community-based programs invite parents to a celebration that includes such presentations. This type of activity may be characterized as project-based learning if students are responsible for choosing a topic, researching it and putting the presentation together with suggestions and help from the teacher. Projects should be an opportunity for learners to develop different linguistic and cultural skills. Young children can also be engaged in projects: teachers may, for example, have learners work on linguistic and cultural elements of a given song during the semester, and present the song at the end. Older students might produce a video on the topic of their choice and/or in consultation with the teacher. Other possibilities for projects that learners can work on during the term include posters, a class magazine, and an art show (paintings, drawings, photos, etc.), among other possibilities that would revolve around the heritage language and culture. Project-based learning may increase motivation and help students learn \citep{BlumenfeldPalincsar1991}. Though possibly varied in nature, these projects would culminate in a product that can not only be shown to parents, but may also serve as a springboard for the following year’s curriculum for each class, even if the group moves together regardless of individual performance. During the project, teachers may take note of aspects that need to be reinforced and/or revisited as well as aspects that have been acquired and may no longer need special attention. Furthermore, projects may give learners an opportunity to engage in community-based activities while integrating their linguistic and cultural abilities \citep{Ilieva2007}. Teachers and pedagogical coordinators may find ideas on project-based learning, for example, in \citet{BeckettMiller2006} and in \citet{VacaTorresRodrígues2017}, although such ideas would need to be adapted to each community-based program context.

\begin{sloppypar}
Although each program must to be tailored to its own context, exchanges of best practices can provide new avenues to be explored. Events such as the Annual Community-Based Heritage Language Schools Conference, held by the Coalition of Community-Based Heritage Language Schools (\url{http://heritagelanguageschools.org/coalition}), gives HL educators an opportunity to listen to experts and to discuss relevant issues. However, many Brazilian Portuguese HL teachers cannot attend the annual conference in Washington, DC. Therefore, consulates and universities that have Portuguese language programs may sponsor events that would be more easily attended by HL educators in a particular region. Like the annual conference in DC, these local events would also serve to keep educators current on research that may inform their assessment practices (\citealt{IlievaClark-Gareca2016}). Another possibility would be to create an online portal with resources for educators, including a forum. These teachers would likely work in similar contexts, which would possibly allow them to more easily adapt strategies and approaches that have worked well in a given program.
\end{sloppypar}

\section{Final remarks}\label{sec:6:7}

This study has contributed to discussions of assessment by shedding light on what is practiced by Brazilian Portuguese community-based HL programs. Limitations of the study include the small number of participants and the lack of details about the forms of assessment practiced by these programs. The fact that respondents completed the survey anonymously, which was done in order not to discourage participation did not allow for follow up questions with respondents which might have clarified what their assessment practices consist of. Future research should include interviews with educators in order to elicit more details about assessment procedures, including the content of the interviews with parents and with students, as well as the content of tests, how they are administered, and at what point in the school year. Conversations with educators may also elicit their views on types of assessment and what they believe may be more helpful to learners in their particular context.

The suggestions offered here are only a starting point. Heritage language educators have proven to be very creative and capable of achieving a lot with limited resources. Given a chance to learn about what other programs do and exchange resources, these teachers and administrators may be able to help their pupils develop their HL abilities more effectively.

\section*{\hypertarget{appendixA}Appendix A}

Survey distributed to HL educators in community-based programs (translated from Brazilian Portuguese). The author wishes to thank Ana Lúcia Lico for her input on a previous version of this survey.

\begin{enumerate}
\item[1.]Before classes begin, does anyone in your school talk to parents/guardians of new students?
\begin{enumerate}
\item[a.]Yes (please answer question 2)
\item[b.]No (please skip to question 3)
\end{enumerate}\item[2.]The goal of the talk with parents/guardians before the beginning of classes is:
\item[3.]At your school, new students are placed in classes:
\begin{enumerate}
\item[a.]according to age only. (Please skip to question 7)
\item[b.]according to the result of an assessment and/or interview. (Please answer questions 4--6)
\item[c.]according to age and assessment/interview (Please answer questions 4--6)
\item[d.]Other
\end{enumerate}
\item[4.]If you answered (b), (c) or “Other” in question 3, please describe the instruments used in the assessment to place new students (e.g., interviews, checklists, written evaluation, etc.).
\item[5.]If you answered (b), (c) or “Other” in question 3, how long (on average) does it take to assess new students? Who carries out the assessment (i.e., what position does this person/do these people occupy)?
\item[6.]If you answered (b), (c) or “Other” in question 3, when is the assessment of new students done? Is the family informed of the assessment result? If so, how?
\item[7.]During the academic term, is students’ linguistic/cultural development assessed? If so, how is this assessment carried out (e.g., with activities done in class or at home, with a checklist, etc.)? What kind of feedback is provided to students and/or parents/guardians (e.g., grade, progress report, etc.)? If you can, please describe examples of assessment and/or feedback.
\item[8.]At the end of the academic term (trimester, semester, year or any period adopted at the institution), is learner development assessed?
\begin{enumerate}
\item[a.]Yes (please answer question 9).
\item[b.]No, because we assess and provide feedback during the academic term.
\item[c.]No, because classes move together, regardless of results.
\item[d.]Other
\end{enumerate}
\item[9.]If you answered “Yes” or “Other” in question 8, please explain how the assessment at the end of the term is carried out and whether it takes into account any assessment done at the beginning of the term.
\end{enumerate}
\sloppy
\printbibliography[heading=subbibliography,notkeyword=this]
\end{document}
