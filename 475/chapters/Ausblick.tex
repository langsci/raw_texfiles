\chapter{Ausblick}
\label{cha:Ausblick}
In der vorliegenden Untersuchung wurden metapragmatische Bewertungen der Genitiv- und Dativrektion ausgewählter Präpositionen sowie Produktionsdaten zu diesen Varianten erhoben und ausgewertet. 
Auf diese Weise war erstmals eine detaillierte Analyse der Indexikalitäten der Rektionsvarianten möglich. 
Es wurde gezeigt, dass Genitiv und Dativ über differenzierte Bedeutungspotenziale verfügen, die in engem Zusammenhang mit der Verwendung der Rektionsvarianten stehen. 
Dadurch ist deutlich geworden, dass die Variation und der Wandel der Rektion von Sekundärpräpositionen nicht allein grammatikalisierungstheoretisch, sondern nur unter der Berücksichtigung sprachideologischer Faktoren erklärt werden können. 

Am Schluss der Studie stehen nun einige Überlegungen dazu, welche Forschungsdesiderata sich aus den Ergebnissen ergeben. 
Im Mittelpunkt steht dabei die Frage, unter welchen Voraussetzungen welche sozialen Bedeutungen der Genitiv- und Dativrektion aktiviert werden. 
Diese zu beantworten ist relevant, um den Einfluss der Indexikalität auf die Verwendung der Rektionskasus noch besser zu verstehen.  
Zunächst wird ein Ausblick auf diesbezügliche Untersuchungen gegeben, die an den im Fragebogen bereits erhobenen Daten noch durchgeführt werden können. 
Anschließend wird darauf eingegangen, welche Daten nötig wären, um weitere Faktoren zu beleuchten, die für die soziale Bedeutung von Genitiv- und Dativrektion relevant sind. 

Die soziodemografischen Informationen zu den Teilnehmer:innen ermöglichen es, die Daten der Onlinebefragung dafür zu nutzen, die soziale Stratifikation der indexikalischen Aufladung der Varianten genauer zu analysieren. 
% Mit den vorhandenen Daten: 
So wurden in der vorliegenden Studie bereits verschiedene soziale Gruppen in Bezug auf ihre Bewertung und Verwendung der Rektionsvarianten miteinander verglichen. 
Bspw. wurden Personen unterschiedlicher regionaler Herkunft, unterschiedlicher Altersgruppen und und mit unterschiedlichem Bildungsstand betrachtet. 
Die vorhandenen Daten können außerdem für einen Vergleich zwischen verschiedenen Berufsgruppen genutzt werden. 
Hierfür müssten die freien Berufsangaben der Befragten in Gruppen zusammengefasst werden, sodass bspw. Lehrer:innen mit anderen Berufsgruppen verglichen werden können. 

Die freien Assoziationen wurden inhaltsanalytisch ausgewertet, um die Indexikalität der abgefragten Varianten herauszuarbeiten. 
An einzelnen Stellen der Studie wurden bereits Unterschiede und Gemeinsamkeiten zwischen den Assoziationen verschiedener Befragtengruppen erwähnt.  
Ein umfangreicherer Vergleich der Assoziationen nach verschiedenen Befragtengruppen wäre für die Offenlegung von sozialen Unterschieden in der Bewertung aufschlussreich. 
Interessant wäre etwa, inwiefern sich die Assoziationen älterer und jüngerer Befragter unterscheiden oder die Assoziationen von variationstoleranten und weniger variationstoleranten Befragten. 

In \autoref{sec:DiskVariationundWandel} wurde bereits herausgearbeitet, inwiefern die Verwendung der Genitivrektion im Produktionsexperiment eine Positionierung der Befragten darstellt. 
Es würde sich darüber hinaus anbieten, die Positionierung einzelner Befragter über die gesamte Befragung hinweg zu analysieren. 
Dies würde einen detaillierteren Blick auf die Zusammenhänge zwischen dem soziodemografischen Profil einer Person, ihren Assoziationen zu einer Variante, ihren Angaben im Akzeptabilitätstest und ihrer Verwendung dieser Variante ermöglichen. 

%Inwiefern sich soziale Gruppen in ihrer Bewertung der untersuchten Varianten unterscheiden, konnte anhand der Daten aus dem Akzeptabilitätsteil lediglich explorativ untersucht werden (\autoref{sec:ErgAkzCTrees}). 
%Bspw. hat sich angedeutet, dass süddeutsche Sprachbenutzer:innen die Dativrektion bei \wegen{} und \waehrend{} positiver beurteilen als norddeutsche Sprachbenutzer:innen. 
%Diese Ergebnisse können als Ausgangspunkt für eine konfirmatorische statistische Analyse an einer größeren Stichprobe genutzt werden. 

% Mit neuer Datenerhebung: 
Die bisher angesprochenen Forschungsfragen beschäftigen sich mit der Frage, inwiefern die Aktivierung sozialer Bedeutungen damit zusammenhängt, welcher Gruppe sich Personen in einer Interaktionssituation zugehörig fühlen.  
Daneben sind weitere Aspekte der Situation entscheidend dafür, welche indexikalische Bedeutung zum Tragen kommt. 
Die Kommentare der Befragten bieten einige Anhaltspunkte dafür, welche Kontextfaktoren für die Indexikalität der Dativ- und Genitivrektion wesentlich sind. 
So ist etwa die Relevanz des Äußerungsmediums und der Vertrautheit der Interaktionspartner:innen deutlich geworden. 
Viele Faktoren, die für die Interpretation eine Rolle spielen, fehlen bei der Präsentation von Varianten in einem Fragebogen jedoch.
So haben Befragte keine Informationen über die Intonation oder die kommunikative Absicht der Person, die eine Variante äußert. 
Einen genaueren Einblick in den Zusammenhang von Indexikalität und Kontext könnten daher Untersuchungen liefern, in denen solche Kontextfaktoren gezielt manipuliert werden. % Änderung Anfang
Von besonderem Interesse wäre dabei mit Sicherheit der Einbezug modal mündlicher Items und Situationen, etwa in Form von Tonbeispielen. % Änderung Ende

Die Ergebnisse der Studie haben bereits gezeigt, dass die soziale Bedeutung nicht allein am Rektionskasus festgemacht wird, sondern erst durch die Kombination von Präposition und Kasus aktiviert wird: 
Bei der kaum schwankenden Präposition \gegenueber{} sind die Rektionsvarianten weit weniger stark indexikalisch aufgeladen als bei \wegen, \waehrend{} und \dank. 
In Bezug auf die Primärpräposition \object{seit} ist anzunehmen, dass die Genitivrektion ein ähnliches indexikalisches Bedeutungspotenzial aufweist wie bei \gegenueber: 
Da \object{seit} beinahe ausschließlich mit dem Dativ vorkommt, wird die Genitivrektion wahrscheinlich auch hier vor allem als inkorrekt und auffällig empfunden. 
Dies kann anhand der in dieser Studie erhobenen Daten jedoch nur vermutet werden.
Freie Assoziationen zu \object{seit} sowie zu weiteren Primärpräpositionen sollten elizitiert werden, um zu untersuchen, ob sich die Ergebnisse zur Indexikalität der Rektionsvarianten von Sekundärpräpositionen auf Primärpräpositionen übertragen lassen. 

Abseits von der Präposition selbst könnte für die Indexikalität von Dativ- und Genitivrektion die Form der von der Präposition regierten Nominalphrase relevant sein. 
Die in der vorliegenden Studie untersuchten Formen entsprechen alle dem Muster Präposition plus Definitartikel plus Substantiv im Singular und wurden damit bewusst vergleichbar gehalten. 
Eine offene Frage bleibt daher, welche Rolle die Form der Nominalphrase für die Indexikalität spielt. 
\citet[78]{Lindqvist1994} vermutet, dass nicht jede Genitivform gleicherma{\ss}en indexikalisch aufgeladen ist.
Nur Formen mit Genitv-\textit{s}, wie etwa in \textit{wegen des Umzugs},\textit{ }fungierten als indexikalische Verweise, w{\"a}hrend Formen mit \textit{-er}, wie etwa in \textit{wegen kleiner Abweichungen},\textit{ }nicht sozialsymbolisch aufgeladen seien:
\begin{quote}Demnach lie{\ss}e sich bei manchen Schreibern, wom{\"o}glich wegen des stilistischen Werts, ein bewu{\ss}tes Einsetzen des Genitivs nur da beobachten, wo er mit dem Genitiv-\textit{(e)s} markiert werden kann. Im Plural, wo diese Stilmarkierung nicht m{\"o}glich ist, vermag sich die eindeutige Genitivmarkierung gegen den Dativ nicht zu behaupten.~\citep[78]{Lindqvist1994}\end{quote}
Diese Hypothese müsste mithilfe entsprechender Assoziations- und Produktionsdaten überprüft werden. 

Die hier aufgeführten offenen Forschungsfragen machen deutlich, dass die Ergebnisse der vorliegenden Studie eine gute Grundlage für weitere Forschung bieten: 
Die differenzierten Indexikalitäten der Genitiv- und Dativrektion spielen eine entscheidende Rolle bei der Verwendung von Sekundärpräpositionen und damit auch für ihren Wandel. 
Zukünftige Arbeiten können an diese Ergebnisse anknüpfen, um die Zusammenhänge zwischen der sozialen Bedeutung von Genitiv und Dativ und der synchronen und diachronen Variation im Bereich der Präpositionen noch intensiver zu beleuchten. 
