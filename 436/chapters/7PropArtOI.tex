\documentclass[output=paper,colorlinks,citecolor=brown]{langscibook}
\ChapterDOI{10.5281/zenodo.10641195}
\author{Hannah Booth\orcid{}\affiliation{Ghent University}}
%\ORCIDs{}

\title{Beyond given versus new: The proprial article in Old Icelandic}

\abstract{The proprial article (\textit{hann Jón} `he John') is attested across North Germanic and has attracted recent interest for Icelandic in particular (\citealp{Sigurdsson2006}; \citealp{Wood2009}; \citealp{sigurdhsson2020we}). Previous considerations of its pragmatics have focused on the given/new dimension, with the claim that it marks familiarity/givenness (\citealp{Sigurdsson2006}; \citealp{Johnsen2016}). Yet a large and growing body of work shows the need to go beyond given versus new for a full understanding of the morphosyntax--information structure interface (e.g.~\citealp{frascarelli2007types}; \citealp{cook2013identifying}). 
I examine the proprial article in Old Icelandic in a wider information-structural context which recognises different types of topic transition.  I show that the proprial article at this early stage is a topic management device which signals various types of topic-shift. Additionally, I confirm an early claim \citep{heusler1962altislandisches} that a special variant of the proprial article (\textit{þeir Jón} `they John') serves two functions in Old Icelandic as (i) an associative plural and (ii) a strategy for coordinating referents which are asymmetrically topical, discussing this in the context of recent theoretical research on associativity, coordination and information structure.}


\IfFileExists{../localcommands.tex}{
   \addbibresource{../localbibliography.bib}
   % add all extra packages you need to load to this file

\usepackage{tabularx,multicol}
\usepackage{url}
\urlstyle{same}

\usepackage{listings}
\lstset{basicstyle=\ttfamily,tabsize=2,breaklines=true}

\usepackage{langsci-basic}
\usepackage{langsci-optional}
\usepackage{langsci-lgr}
\usepackage{langsci-osl}
% \usepackage{./langsci/styles/langsci-lgr}
% \usepackage{./langsci/styles/langsci-osl}
% \usepackage{langsci-gb4e}

\usepackage{tikz}
\usetikzlibrary{patterns,calc}
\pgfdeclarepatternformonly{south east lines}{\pgfqpoint{-0pt}{-0pt}}{\pgfqpoint{3pt}{3pt}}{\pgfqpoint{3pt}{3pt}}{
    \pgfsetlinewidth{0.6pt}
    \pgfpathmoveto{\pgfqpoint{0pt}{3pt}}
    \pgfpathlineto{\pgfqpoint{3pt}{0pt}}
    \pgfpathmoveto{\pgfqpoint{.2pt}{-.2pt}}
    \pgfpathlineto{\pgfqpoint{-.2pt}{.2pt}}
    \pgfpathmoveto{\pgfqpoint{3.2pt}{2.8pt}}
    \pgfpathlineto{\pgfqpoint{2.8pt}{3.2pt}}
    \pgfusepath{stroke}}
    
\usepackage{stmaryrd}
\usepackage{wasysym}
\usepackage{multirow}
\usepackage{caption}
\usepackage{subcaption}
\usepackage{mathrsfs}
\usepackage{qtree}

\usepackage{linguex}


   %pminos do not split footnotes
% \interfootnotelinepenalty=10000 %Footnote in Laporte chapters has to be split SN


%\DeclareIndexNameFormat{default}{%
%\nameparts{#1}%
%\usebibmacro{index:name}%
%{\index[names]}%
%{\namepartfamily}%
%{\namepartgiveni}%
% {}% L1
% {}% L2
%{\namepartprefix}% generates spurious space L3
%{\namepartsuffix}% generates spurious space L4
%}

%  {\DeclareIndexNameFormat{default}{%
%     \usebibmacro{index:name}{\index[names]}{#1}{#3}{#5}{#7}}}

%\DeclareIndexNameFormat{default}{%
%  \usebibmacro{index:name}{\sindex[nom]}{#1}{#3}{#5}{#7}}

%\DeclareIndexNameFormat{default}{%
%  \usebibmacro{index:name}{\sindex[person]}{#1}{#3}{#5}{#7}}
%\DeclareIndexNameFormat{default}{%
%\nameparts{#1} \usebibmacro{index:name}{\sindex[person]]}{\namepartfamily}{‌​\namepartgiven}{\nam‌​epartprefix}{\namepa‌​rtsuffix}}

%\newcommand{\smiley}{:)}

%\renewbibmacro*{index:name}[5]{%
%\usebibmacro{index:entry}{#1}%
%{\iffieldundef{usera}{}{\thefield{usera}\actualoperator}\mkbibindexname{#2}{#3}{#4}{#5}}}

% \newcommand{\noop}[1]{}

%remove for final
%\overfullrule=1mm

\newcommand{\tobi}[2]}}
\renewcommand{\S}[1]{\tobi{#1}{\textsc{*}}}

% this volume references
% puts: [this volume]
% already defined: \citetv
%\newcommand{\citepv}[1]{(\citeauthor{#1} \citeyear*{#1} [this volume])}
\newcommand{\citealtv}[1]{\citeauthor{#1} \citeyear*{#1} [this volume]}

%parentheses around example number
\newcommand{\pref}[1]{(\ref{#1})}

% in-text examples

\newcommand{\lnex}[1]{\textit{#1}} %target lang word
\newcommand{\lnlit}[1]{(lit.: `#1')} %literal reading
\newcommand{\lnlat}[1]{(#1)} % latinization
\newcommand{\lntrans}[1]{`#1'} %translation
\newcommand{\lnexl}[2]%
{\lnex{#1}{} \lnlat{#2}} % ex with latinization
\newcommand{\lnexlat}[3]{\lnex{#1}{} \lnlat{#2}{} \lntrans{#3}} % ex with latinization and tranl.

%ch01
\newcommand{\co}[1]{\mbox{\textbf{#1}}}

%ch09

\newcommand{\cyrbulg}[1]{\begin{otherlanguage*}{bulgarian}#1\end{otherlanguage*}}


%ch10
\newcommand{\nlp}{{\small NLP}}
\newcommand{\mwe}{{\small MWE}}
\newcommand{\rae}{{\small RAE}}
\newcommand{\lvc}{{\small LVC}}
\newcommand{\pos}{{\small P}o{\small S}}
%\newcommand{\todo}[1]{ \textcolor{red}{#1} }

%\renewcommand{\labelenumi}{\theenumi}
%\ainamefmt{{vv}{ll}{, ff}{, jj}} % fullname

\newcommand{\biberror}[1]{{\color{red}#1}}

\newcommand{\osenovaitem}{--~}
   %% hyphenation points for line breaks
%% Normally, automatic hyphenation in LaTeX is very good
%% If a word is mis-hyphenated, add it to this file
%%
%% add information to TeX file before \begin{document} with:
%% %% hyphenation points for line breaks
%% Normally, automatic hyphenation in LaTeX is very good
%% If a word is mis-hyphenated, add it to this file
%%
%% add information to TeX file before \begin{document} with:
%% %% hyphenation points for line breaks
%% Normally, automatic hyphenation in LaTeX is very good
%% If a word is mis-hyphenated, add it to this file
%%
%% add information to TeX file before \begin{document} with:
%% \include{localhyphenation}
\hyphenation{
    Beck-man
    Ngu-yen
    back-chan-nel
    back-chan-nels
    mo-not-o-nous
    ste-reo-typ-i-cal
}

\hyphenation{
    Beck-man
    Ngu-yen
    back-chan-nel
    back-chan-nels
    mo-not-o-nous
    ste-reo-typ-i-cal
}

\hyphenation{
    Beck-man
    Ngu-yen
    back-chan-nel
    back-chan-nels
    mo-not-o-nous
    ste-reo-typ-i-cal
}

   \boolfalse{bookcompile}
   \togglepaper[7]%%chapternumber
}{}

\begin{document}
\maketitle

%%%%%%%%%%%%%%%%%%%%%%%%%%%%%%%%%%%
\section{Introduction} 

A number of present-day \ili{Germanic} languages have a functional element which can combine with proper nouns, in particular personal names, commonly referred to as a ``proprial \isi{article}'' (PA) (see e.g.~\citealp{delsing1993internal}; \citealp{Langendonck2009}; \citealp{dahl2015grammaticalization}; \citealp{Kokkelmans2018}; \citealp{Munoz2019}).\footnote{An alternative label for the proprial \isi{article} is ``onymic \isi{article}'', as commonly used in the literature on \ili{German} (e.g.~\citealp{nubling2017growing}; \citealp{schmuck2020grammaticalisation,schmuck2020rise}; \citealp{ackermann2021pre}).}
In West \ili{Germanic}, the PA is formally identical to the definite \isi{article}, e.g.~(\ref{wgmc}); in North \ili{Germanic}, it is formally identical to the personal \isi{pronoun}, e.g.~(\ref{ngmc}).\footnote{I gloss instances of the proprial \isi{article} as \textsc{pa} throughout.}

\ea \label{wgmc} 
 \ea[]{\label{germ-pa}(non-standard) High \ili{German} \\
 \gll [\textbf{der} Peter] hat dem Kind einen Ball geschenkt\\
      \textsc{pa.m.nom.sg} Peter has \textsc{def.dat} child \textsc{indef.acc} ball gifted\\
 \glt `Peter gave the child a ball as a present.' \citep[33]{altmann1993fokus}
}
\ex[]{\ili{Dutch}, Brabantisch\\
\gll Wette gelle nog da we [\textbf{de} Jan] op de met emme gezien?\\
know you.\textsc{pl} still \textsc{comp} we \textsc{pa.comm} Jan on the market have seen\\
\glt `Do you know that we saw Jan at the market?'
\citep[164]{schmuck2020grammaticalisation}
}
\z 
\z
        
\ea \label{ngmc}
 \ea[]{ \label{intro-ex} (modern) \ili{Icelandic}  \\
 \gll [\textbf{Hún} Þuríður] gat ekki tekið þessum tíðindum\\
     \textsc{pa.f.nom.3sg} Þuríður.\textsc{nom} could \textsc{neg} take \textsc{dem.dat} news.\textsc{dat}\\
 \glt `Þuríður could not take this news.' (IcePaHC: 2008, Ofsi.1163)
}
\ex[]{ \ili{Norwegian}, Inner Østfold \\
\gll På tjueårsdagen heldt [\textbf{han} Torbjørn] og [\textbf{ho} Eline] ein fest for [\textbf{ho} Sissel]\\
on twenty-year-day.\textsc{def} held \textsc{pa.m.3sg} Torbjørn and \textsc{pa.f.3sg} Eline a party for \textsc{pa.f.3sg} Sissel\\
\glt `On her 20th anniversary, Torbjørn and Eline held a party for Sissel.'
\citep[194]{Johnsen2016} 
}
\z 
\z\largerpage[2]

The function of the PA has been examined for a range of \ili{Germanic} varieties, with various functions attributed to it. In West \ili{Germanic}, where the PA takes the form of the definite \isi{article}, cf.~(\ref{wgmc}), it is assumed that the PA does not mark \isi{definiteness}, since personal names refer to an entity conceptualised as unique and are thus inherently definite  (\citealp{longobardi1994reference}; \citealp{nubling2017growing, nubling2020capital}; \citealp{schmuck2020rise}). 
In Southern \ili{German} varieties, the PA is obligatory and does not appear to have any pragmatic effect (\citealp{nubling2020capital}; \citealp{schmuck2020grammaticalisation}), whereas in Northern \ili{German} varieties it is optional and has been claimed to fulfil various pragmatic functions. \citet{hartmann1982deixis}, for instance, has argued that the PA as exhibited in Rhineland dialects can indicate that the individual referred to is known to speaker and hearer, and can also convey a speaker's annoyance with an individual. Similarly, the PA in (Flemish) \ili{Dutch} dialects has been claimed to express \isi{familiarity}  \citep[158]{Langendonck2007theory}. \citet{werth2014funktionen}, meanwhile, has argued that in Northern \ili{German} dialects, the PA can act as a focus marker, or as a marker of social distance. 

In North \ili{Germanic}, a similarly wide-ranging list of pragmatic functions has been attributed to the PA, with a specific focus on the given/new dimension. For modern \ili{Icelandic}, it has been argued that the PA marks ``\isi{familiarity} or \isi{givenness}'', as evidenced by the fact that the PA is only felicitous if the referent is known to both speaker and listener \citep[220]{Sigurdsson2006}. Similar claims with respect to \isi{familiarity} have been made for the North \ili{Germanic} PA elsewhere, notably by \citet{Haberg2010} for certain \ili{Norwegian} dialects and by \citet[97]{dahl2015grammaticalization}, who claims the same for ``most colloquial varieties'' of \ili{Swedish}. At the same time, others have claimed that the PA in present-day Mainland \ili{Scandinavian} plays a role in discourse activation (\citealp{Teleman1999}; \citealp{Strahan2007}; \citealp{Lie2008,lie2010om}).
\citet{Johannessen2008,johannessen2020psychologically}, meanwhile, claims that a superficially similar element which occurs in Present-day \ili{Norwegian} signals ``\isi{psychological distance}'', occurring in contexts where the speaker/addressee does not know the person referred to, or in contexts where the speaker expresses a negative attitude towards the individual.

Furthermore, the pragmatic status of the PA in Early North \ili{Germanic} is disputed. While \citet{Johnsen2016} claims that the PA marks \isi{familiarity}/\isi{givenness} in Old \ili{Norwegian}/Icelandic, as claimed for modern \ili{Icelandic} \citep{Sigurdsson2006}, \citet{Kinn2016-thesis} claims the contrary for Old \ili{Norwegian}, namely that 3rd person pronouns which occur before proper names (i.e.~potential PAs) do not have any semantic or pragmatic effects.
As such, the precise pragmatic status of the PA in \ili{Old Icelandic} remains unclear. Moreover, discussion of its (potential) pragmatic effects have, as for modern \ili{Icelandic}, been limited to a consideration of the given/new dimension. 
There is, however, a large and growing body of work on the interaction between morphosyntax and \isi{information structure} from various perspectives which shows that one needs to go beyond a simple given versus new distinction in order to fully understand phenomena at this interface (e.g.~\citealp{ariel1990accessing,ariel2001accessibility}; \citealp{vallduvi92}; \citealp{groszetal95}; \citealp{vallduvi1996linguistic}; \citealp{walker1998centering}; \citealp{erteschik-shir_information_2007}; \citealp{frascarelli2007types}; \citealp{krifka07}; \citealp{bianchi2010topic}; \citealp{cook2013identifying}).\largerpage

In this chapter, I examine the proprial \isi{article} in \ili{Old Icelandic} in this wider information-structural context which recognises different types of ``topic transition'' (e.g.~\citealp{danes1974functional}; \citealp{groszetal95}; \citealp{walker1998centering}; \citealp{frascarelli2007types}; \citealp{bianchi2010topic}). On the basis of \ili{Old Icelandic} \isi{corpus} data from IcePaHC \citep{IcePaHC}
and further supplementary data from a related \isi{corpus}, MIcePaHC \citep{MIcePaHC}, I show that the PA in \ili{Old Icelandic} is more than a straightforward \isi{givenness} marker, as previously claimed (\citealt{Sigurdsson2006}; \citealt{Johnsen2016}). While the PA is indeed restricted to discourse-given referents, it has a more nuanced motivation, marking a referent which is known from the discourse but which represents a new \isi{aboutness topic} \citep{reinhart1981pragmatics}, i.e.~``shift-topic''. In this respect, the \ili{Old Icelandic} PA functions as a specialised topic management device which signals the (re-)establishment of a familiar referent as topic.
In addition, I confirm an early claim by \citet{heusler1962altislandisches} that a special variant of the proprial \isi{article} serves two functions in \ili{Old Icelandic} as (i) an \isi{associative plural} and (ii) a strategy for coordinating referents which are asymmetrically topical, discussing this in the context of recent theoretical research on \isi{associativity}, \isi{coordination} and \isi{information structure}.

The chapter proceeds as follows. I discuss the current understanding of the proprial \isi{article} in \ili{Icelandic} and other North \ili{Germanic} varieties in Section \ref{sect:bg}, and discuss the diachrony of proprial articles in Section \ref{sect:hist-context}. Section \ref{sect:method} outlines the methodology for the \isi{corpus} study which is the focus of this chapter, including the collection and tagging of the data, and the relevant information-structural concepts.  Section \ref{sect:plain} examines the so-called ``plain'' proprial \isi{article} \citep{Sigurdsson2006} in the broader context of topic management devices, and Section \ref{sect:gapped} discusses the pragmatic properties of a special variant of the proprial \isi{article}. Section \ref{sect:conc} concludes the chapter.

%%%%%%%%%%%%%%%%%%%%%%%%%%%%%%%%%%%
\section{The proprial article in modern North Germanic}\label{sect:bg}

%%%%%%%
\subsection{The proprial article in modern Icelandic}\largerpage

\citet{Sigurdsson2006} provides a detailed overview of the properties of the proprial \isi{article} in modern \ili{Icelandic}. The PA can occur, generally optionally, with simple personal names and short forms of kinship terms, but is highly questionable or ruled out with common nouns beyond these kinship terms, cf.~(\ref{noun-types}).{\footnote{The proprial \isi{article} is generally ruled out with full names, presumably due to the fact that the referent must be familiar/given. \citet[225]{Sigurdsson2006} notes it is possible if the referent is commonly known by their full name, as with, for instance, a famous politician.
The PA is also possible with simple personal names modified by adjectives, though optional, as elsewhere \citep[134]{delsing1993internal}: 

\ea \label{name-adj} modern \ili{Icelandic}
\ea[]{ 
 \gll Svo fæddist [\textbf{hann} Siggi litli]\\
 so was-born \textsc{pa.m.nom.3sg} Siggi.\textsc{nom} little.\textsc{nom}\\
 \glt `So little Siggi was born.' (IcePaHC: 2008, Mamma.1122)
 }
 \ex[]{
 \gll Og [Lancelot litli] rak óðara upp glaðlegt gelt\\
 and Lancelot.\textsc{nom} little.\textsc{nom} drove madly up cheerful.\textsc{acc} bark.\textsc{acc}\\
 \glt `And little Lancelot madly drove up a cheerful bark.' \\(IcePaHC: 2008, Mamma.1809)} 
 \z 
\z}} Note that such examples do not involve an intonational break between the PA and the referent it combines with, which distinguishes the PA from straightforward apposition (\citealp{Sigurdsson2006}; \citealp{Wood2009}).


\ea \label{noun-types} modern \ili{Icelandic} \judgewidth{??}
 \ea[]{\gll \textbf{hann} Jón / \textbf{hún} María\\
    \textsc{pa.m.nom.3sg} Jón.\textsc{nom} {} \textsc{pa.f.nom.3sg} María.\textsc{nom}\\}
\ex[]{\gll \textbf{hann} pabbi / \textbf{hún} amma\\
\textsc{pa.m.nom.3sg} dad.\textsc{nom} {} \textsc{pa.f.nom.3sg} grandma.\textsc{nom}\\}
\ex[??]{ \gll \textbf{hann} kennari / * \textbf{hún} bók\\
\textsc{pa.m.nom.3sg} teacher.\textsc{nom} {} {} \textsc{pa.f.nom.3sg} book.\textsc{nom}\\ 
\glt {} \hspace{.5ex} \citep[224--225]{Sigurdsson2006}}
\z
\z

The PA can (optionally) occur on a range of grammatical functions, e.g.~subject, object, prepositional complement and \isi{predicative} complement, cf.~(\ref{pa-gfs}).


\ea \label{pa-gfs} modern \ili{Icelandic}
\judgewidth{??}
 \ea[]{ Subject:\\
 \gll [\textbf{Hún} \textbf{María}] kom {í gær}\\
    \textsc{pa.f.nom.3sg} María.\textsc{nom} came yesterday \\
\glt `Maria came yesterday.'
}
\ex[]{ Object:\\
\gll Við sáum [\textbf{hana} \textbf{Maríu}] {í gær}\\
we.\textsc{nom} saw \textsc{pa.f.acc.3sg} María.\textsc{acc} yesterday\\
\glt `We saw Maria yesterday.'
}
\ex[]{ Prepositional complement:\\
\gll Bréfið er frá [\textbf{henni} \textbf{Maríu}]\\
letter.\textsc{nom.def} is from \textsc{pa.f.dat.3sg} María.\textsc{dat}\\
\glt `The letter is from Maria.'
}
\ex[]{ Predicative complement:\\
\gll Er þetta ekki [\textbf{hún} \textbf{María}]?\\
is \textsc{dem.nom} \textsc{neg} \textsc{pa.f.nom.3sg} María.\textsc{nom}\\
\glt `Is that not Maria?' \\
 \citep[225]{Sigurdsson2006}
 }
\z 
\z

The PA shows case, person and number \isi{agreement} with the personal name it combines with, and is always \isi{prenominal} (\textit{\textbf{hún} María}); \isi{postnominal} distribution (*\textit{María \textbf{hún}}) is ruled out according to \citet{Sigurdsson2006}. Additionally, the PA can occur on possessors in what \citet[216]{Sigurdsson2006} refers to as the ``Name-Genitive Construction'', e.g.~(\ref{noun-gen}). 


\ea \label{noun-gen} modern \ili{Icelandic}\\
\gll Allir bílarnir [\textbf{hans} Jóns] eru gulir \\
    all.\textsc{nom} cars.\textsc{nom.def} \textsc{pa.m.gen.3sg} Jón.\textsc{gen} are yellow\\
\glt `All Jón's cars are yellow.' \citep[213]{Sigurdsson2006}
\z

In this particular context, a genitive-marked PA is obligatory if the head \isi{noun} bears the suffixed definite \isi{article}  \citep{Sigurdsson2006}, cf.~the \isi{contrast} in (\ref{gen-obl}).

\ea \label{gen-obl} modern \ili{Icelandic}
 \ea[]{
 \gll bókin [\textbf{hennar} Maríu] \\
book.\textsc{nom.def} \textsc{pa.f.gen.3sg} Maria.\textsc{gen}\\
}
\ex[*]{ 
\gll bókin [Maríu]\\
book.\textsc{nom.def} Maria.\textsc{gen}\\
\glt  \citep[224]{Sigurdsson2006} }
\z 
\z

In terms of its pragmatic properties, \citet[220]{Sigurdsson2006} claims that the PA is a ``marker of \isi{familiarity} or \isi{givenness}'', on the basis that it is only felicitious if both speaker and addressee know and can identify the referent in question.  \citet[226]{Sigurdsson2006} further claims that the \isi{familiarity} signalled by the PA is a ``deictic feature'' which speakers use to signal that both they and the addressee are familiar with the referent.

Crucially, the examples of the \ili{Icelandic} PA discussed so far must be considered as just one variant of the PA, specifically what \citet{Sigurdsson2006} calls the ``Plain Proprial Article Construction''. Another example of this ``plain'' type, this time occurring with a coordinated referent (`Jón and María'), is provided in (\ref{plain-sig}) below. This ``plain'' PA is distinct from what \citet{Sigurdsson2006} calls the ``Gapped Proprial Article Construction'', illustrated in (\ref{gap-sig}).

\ea modern \ili{Icelandic}
 \ea[]{ \label{plain-sig}
 \gll [(\textbf{Þau}) Jón og María] eru vinir \\
    \textsc{pa.n.nom.3pl} Jón and María are.\textsc{pl} friends\\
    \glt `Jón and María are friends.'
}
\ex[]{ \label{gap-sig}
\gll María fór út. [\textbf{Þau} Jón] ætla að hittast \\
María went out. \textsc{pa.n.nom.3pl} Jón.\textsc{nom} intend.\textsc{pl} to meet\\
\glt `María went out. She and Jón are going to meet.'\\
 \citep[227--228]{Sigurdsson2006}
} 
\z 
\z


In the ``plain'' type in (\ref{plain-sig}), the PA can be omitted and the sentence is still grammatical -- indeed, as already mentioned, this variant of the PA is optional. By \isi{contrast}, in the ``gapped'' type in (\ref{gap-sig}) the PA cannot be omitted, since this would result in a mismatch in number \isi{agreement} between the verb and subject; \textit{Þau Jón} in this context functions as a plural for subject-verb \isi{agreement}, denoting a set consisting of María and Jón.\footnote{As \citet{Sigurdsson2006} observes, in some contexts the ``gapped'' PA can combine with more than one name in a \isi{coordination} structure: 

\ea \label{gapped-ambig} modern \ili{Icelandic}\\
\gll Anna kemur líka. [\textbf{Þau} Jón og María] eru vinir \\
    Anna comes too \textsc{pa.n.nom.3pl} Jón.\textsc{nom} and María.\textsc{nom} are.\textsc{pl} friends\\
\glt `Anna is coming too. She, John and Mary are friends.'  \citep[229]{Sigurdsson2006}
\z

\noindent In such cases, the ``gapped'' type is identical in form to a ``plain'' PA construction, cf.~(\ref{plain-sig}). 
}

The ``gapped'' PA in modern \ili{Icelandic} has been discussed under a number of different labels in line with different analyses. As mentioned, \citet{Sigurdsson2006} discusses it as a special ``gapped'' variety of the proprial \isi{article}, in line with the fact that he analyses it as involving a \isi{coordination} structure and deletion; see \citet{Wood2009} for a similar analysis in the context of ``imposters'' \citep{collins2012imposters}, i.e.~elements which exhibit a mismatch between grammatical person and notional person.
\citet{sigurdhsson2020we}, meanwhile, develop a different analysis for the construction, which they instead refer to as the ``Pro[NP]'' construction, as distinct from the (plain) proprial \isi{article}.\footnote{In the context of \ili{Old Icelandic}, the ``gapped'' PA has also been discussed as an ``\isi{associative plural}'' construction in modern theoretical and typological work (\citealp{denbesten1996associative}; \citealp{moravcsik2003semantic}; \citealp{wals-36}; \citealp{sigurdhsson2020we}), as I discuss in detail in Section \ref{sect:hist-context}.}

The construction in question in fact appears to qualify as what is often labelled as 
an ``inclusory'' construction in a diverse range of languages, in particular Austronesian and Australian languages (cf.~\citealp{lichtenberk2000inclusory}; \citealp{singer2001inclusory}; \citealp{bhat2004conjunction}; \citealp{gaby2005some}; \citealp{haspelmath2007coordination}; \citealp{bril2011noun}; \citealp{schultze2013kriol}); it consists of a non-singular \isi{pronoun} (``superset'') plus a personal name (``subset'') whose referent is included in the reference of the non-singular \isi{pronoun}, cf.~(\ref{super-sub}) and the example repeated in (\ref{gap-sig-rep}).

\ea \label{super-sub}
 $\underbrace{\textsc{pa.du/pl}}_{\textsc{superset}}$ --  $\underbrace{\textrm{name}}_{\textsc{subset}}$
 \z 

\ea  \label{gap-sig-rep}
\gll María fór út. [\textbf{Þau} Jón] ætla að hittast \\
María went out. \textsc{pa.n.nom.3pl} Jón.\textsc{nom} intend.\textsc{pl} to meet\\
\glt `María went out. She and Jón (=they including Jón) are going to meet.'
\z 

Compare similar inclusory constructions from a range of languages in (\ref{typ-incl}), where the non-singular \isi{pronoun} (superset) is shown in bold and the personal name (subset) in italics.

\ea \label{typ-incl}
\ea \ili{Yapese}\\
\gll Ka \textbf{ra} bow \textit{Tamag}\\
\textsc{pst} 3\textsc{du/pl} come+\textsc{du} tamag\\
\glt `He and Tamag came.' (\citealp[270]{jensen1977yapese}, as cited in \citealp[519]{aissen1989agreement})
\ex \ili{Toqabaqita} \\
\gll \textbf{Keeroqa} \textbf{tha} \textit{Bita} kero sifo naqa\\
3\textsc{du} \textsc{pers.mkr} Bita 3.\textsc{du.nonfut} descend \textsc{perf}\\
\glt `He/she and Bita have gone down.' \citep[10]{lichtenberk2000inclusory}
\ex \ili{Kriol} \\
\gll \textbf{Mindubala} \textit{Namij} kol-im dardaga\\
1.\textsc{du.excl} Namij call-\textsc{tr} bloodwood.apple\\
\glt `Me and Namij call it dardaga.' \citep[243]{schultze2013kriol}
\ex M{\=a}ori \\
\gll {Kei te} aha \textbf{k{\=o}rua} ko \textit{Tame}?\\
\textsc{tam} what 2\textsc{du} \textsc{spec} Tame\\
\glt `What are you and Tame doing'? \\ 
(\citealp[548]{bauer1997reed}, as cited in \citealp[246]{bril2011noun})
\z 
\z 


On this basis, I will refer to examples like (\ref{gap-sig-rep}) as the ``inclusory PA'', as distinct from the ``plain PA'' already discussed. With this term, I commit to no more than the observation that the \isi{pronoun} and the name are in a superset-subset relation as in (\ref{super-sub}).\footnote{\citet[2]{sigurdhsson2020we} also briefly acknowledge that the construction is similar to inclusory constructions as discussed for other languages, but say that the term is not ``entirely satisfactory'' and that they ``use the term `inclusive' in a different sense'', namely in relation to whether the \isi{pronoun} refers to a subset that is included in the set denoted by the \isi{noun} it combines with. On this particular definition, they note that constructions such as \textit{við Ólafur} (I and Ólafur) are ``non-inclusive'', since the \isi{pronoun} is not included in the reference set of \textit{Ólafur}. However, this seems to be the reverse of the standard understanding of ``inclusory'' constructions (e.g.~\citealp{lichtenberk2000inclusory}; \citealp{bhat2004conjunction}; \citealp{gaby2005some}; \citealp{haspelmath2007coordination}; \citealp{gil2009associative}; \citealp{bril2011noun}; \citealp{schultze2013kriol}), whereby the \isi{pronoun} is the superset, whose reference includes the referent expressed by the \isi{noun} (subset), cf.~(\ref{super-sub}) above. In line with this wider typological body of work, I opt for the term ``inclusory PA'' for the \ili{Icelandic} construction.}

Just like the plain PA, the inclusory PA can occur in various grammatical functions and agrees in case with the personal name, e.g.~(\ref{gap-gf}).


\ea \label{gap-gf} modern \ili{Icelandic}
 \ea[]{ Quirky subject:\\
 \gll [\textbf{Okkur} Ólafi] leiddist \\
  \textsc{pa.1pl.dat} Olaf.\textsc{dat} bored\\
  \glt     `Olaf and I were bored.'\\
}
\ex[]{ Direct object:\\
\gll Hún sá [\textbf{okkur} Ólaf] \\
   she.\textsc{nom} saw \textsc{pa.1pl.acc} Olaf.\textsc{acc}\\
   \glt     `She saw Olaf and me.' \\
}
\ex[]{ Possessor:\\
\gll Hún er vinur [\textbf{okkar} Ólafs] \\
      she.\textsc{nom} is friend \textsc{pa.1pl.gen} Olaf.\textsc{gen}\\
   \glt   `She is a friend of Olaf and me.'\\
   \citep[5]{sigurdhsson2020we}
} 
\z 
\z

Also similar to the plain PA, \citet{Sigurdsson2006} claims for modern \ili{Icelandic} that the inclusory PA marks \isi{familiarity}/\isi{givenness}, i.e.~indicates that the addressee knows and can identify the PA-marked referent, in other words the same function as that attributed to the plain PA \citep{Sigurdsson2006}.  

At the same time, there are some differences between the plain and inclusory PA, as extensively discussed by \citet{sigurdhsson2020we}. Firstly, unlike the plain PA, which cannot occur with common nouns \citep{Sigurdsson2006}, \citet{sigurdhsson2020we} observe that the inclusory PA can occur with animate common nouns often denoting occupations (e.g.~\textit{\textbf{við} kennarinn} `the teacher and I'). Secondly, they show that the \isi{pronoun} in the inclusory PA shows head properties, controlling person and number \isi{agreement} on the finite verb, e.g.~(\ref{verb-agr}), and number, gender and case \isi{agreement} on adjectival and participial predicates, e.g.~(\ref{pred-agr}).\largerpage

\ea \label{verb-agr} modern \ili{Icelandic}
 \ea[]{ 
 \gll [\textbf{Við} María] fórum \\
 \textsc{pa.nom.1sg} Mary.\textsc{nom} went.\textsc{1pl}\\
  \glt  `Mary and I went/left.'
}
\ex[]{ 
\gll [\textbf{Þið} María] fóruð. \\
  \textsc{pa.nom.2pl} Mary.\textsc{nom} went.\textsc{2pl}\\
   \glt `Mary and you went/left.'
}
\ex[]{ 
\gll [\textbf{Þær} María] fóru \\
      \textsc{pa.f.nom.3pl} Mary.\textsc{nom} went.\textsc{3pl}\\
   \glt `Mary and she went/left.'\\
   \citep[4]{sigurdhsson2020we}
} 
\z 
\z

\ea  \label{pred-agr} modern \ili{Icelandic} 
 \ea[]{ A male + Olaf:\\
 \gll [\textbf{Þeir} Ólafur] eru sterk\textbf{ir} \\
    \textsc{pa.m.nom.3pl} Olaf.\textsc{nom} are.\textsc{3pl} strong.\textsc{m.nom.pl}\\
  \glt  `Olaf and he are strong.'\\
}
\ex[]{ A female + Mary:\\
\gll [\textbf{Þær} María] eru sterk\textbf{ar} \\
  \textsc{pa.f.nom.3pl} Mary.\textsc{nom} are.\textsc{3pl} strong.\textsc{f.nom.pl}\\
   \glt `Mary and she are strong.'\\ 
}
\ex[]{ A non-male + Olaf:\\
\gll [\textbf{Þau} Ólafur] eru sterk \\
      \textsc{pa.n.nom.3pl} Olaf.\textsc{nom} are.\textsc{3pl} strong.\textsc{n.nom.pl}\\
   \glt `Olaf and she/it are strong.'\\
}
\ex[]{ A non-female + Mary:\\
\gll [\textbf{Þau} María] eru sterk.\\
      \textsc{pa.n.nom.3pl} Mary.\textsc{nom} are.\textsc{3pl} strong.\textsc{n.nom.pl}\\
   \glt `Mary and he/it are strong.' \\
   \citep[6]{sigurdhsson2020we}
} 
\z 
\z

%%%%%%%
\subsection{The proprial article in modern Mainland Scandinavian}


As \citet{sigurdhsson2020we} note, the inclusory PA construction is only present in modern Insular \ili{Scandinavian}, and in Faroese it is less robust than in \ili{Icelandic}. In modern Mainland \ili{Scandinavian}, the inclusory PA construction has been lost altogether. The plain PA, however, is present in some \ili{Norwegian} and \ili{Swedish} varieties, e.g.~(\ref{mainland-plain}). 

\ea \label{mainland-plain}
 \ea[]{ \ili{Norwegian}, Voss\\
 \gll  Men [\textbf{ho} Inger] se kkje e så mykkje te\\
 but \textsc{pa.f.3sg} Inger see \textsc{neg} I so much to\\
 \glt `But Inger, I don't see much.' \citep[90]{Haberg2010}\\
}
\ex[]{ \ili{Northern Swedish} \\
\gll [\textbf{En} Bjürström] ha affärn\\
   \textsc{pa.m.3sg} Bjürström has shop.\textsc{def}\\
 \glt  `Bjürström has the shop.' \\
 (\citealp[21]{delsing2003syntaktisk}, as cited in \citealp[10]{johannessen2014proprial})
} 
\z 
\z

According to \citet{jorgensen2000studien}, there is no PA in \ili{Danish}, contrary to suggestions in earlier work \citep{hulthen1944studier}.


In terms of the function of the PA in modern Mainland \ili{Scandinavian}, there seems to be a wide range of claims, varying across individual varieties. \citet[97]{dahl2015grammaticalization}, for instance, notes that in ``most colloquial varieties of \ili{Swedish}'' the PA has a clear pragmatic effect which he illustrates with \textit{han Erik} meaning `that person Erik that you know', i.e.~signalling \isi{familiarity} (see also \citealp{delsing2003syntaktisk}). In varieties where the PA is obligatory with given names and name-like uses of kinship terms, no such effect is found according to Dahl. Others, meanwhile, have claimed that the PA signals a new person in the discourse, thus playing a role in discourse activation (\citealp{Teleman1999}; \citealp{Strahan2007}; \citealp{Lie2008,lie2010om}). \citet{lie2010om}, for instance, argues that the PA in Present-day \ili{Norwegian} does not refer to a previously mentioned referent or a referent available in the situational context, but rather serves to activate specific, shared knowledge. Similarly, \citet{Teleman1999} state for \ili{Swedish} that the PA functions to actualise referents that are not present in the current discourse but which are present in the shared knowledge of speaker and hearer. 

\citet{Johannessen2008,johannessen2020psychologically}, meanwhile, claims in the context of Present-day \ili{Norwegian} that the PA must be distinguished from what she refers to as the ``psychologically distal \isi{demonstrative}'' (PDD). The PDD can occur with any kind of human \isi{noun} as well as proper names, and typically occurs in contexts where the speaker/addressee does not know the person referred to, or in contexts where the speaker expresses a negative attitude to the person referred to, e.g.~(\ref{pdd}).

\ea \label{pdd} \ili{Norwegian}  \\
\gll jeg og Magne vi sykla jo og [\textbf{han} Mikkel] da\\
I and Magne we cycled then and he Mikkel then\\
\glt `I and Magne and that guy Mikkel we cycled then.'\\
\citep[164]{Johannessen2008}
\z

This leads \citet{Johannessen2008,johannessen2020psychologically} to claim that the PDD signals ``\isi{psychological distance}''.  Due to the fact that the earliest written examples noted by \citet{Johannessen2008} are from the beginning of the 20\textsuperscript{th} century, and that the use of the PDD is found to have increased between 1970 and 2005 \citep{Johannessen2008}, \citet{kinn2017null} suggests that the PDD is a relatively recent development.

%%%%%%%%%%%%%%%%%%%%%%%%%%%%%%%%%%%
\section{Historical context}\label{sect:hist-context}

%%%%%
\subsection{Proprial articles, case and the grammaticalisation of definiteness}

\begin{sloppypar}
The diachronic development of the PA in \ili{Germanic} has generally been neglected in modern theoretical research, although its emergence in the history of (High) \ili{German} has attracted some recent interest (\citealp{Schmuck2014}; \citealp{Schmuck2020-hex,schmuck2020grammaticalisation,schmuck2020rise}). 
Here, the rise of the PA has been characterised as representing a relatively late stage in the overall grammaticalisation of the definite \isi{article} (e.g.~\citealp{Schmuck2014}; \citealp{schmuck2020grammaticalisation}), in line with the fact that the \ili{German} PA is formally identical to the definite \isi{article}, cf.~(\ref{germ-pa}) above. The grammaticalisation of the definite \isi{article} with common nouns is virtually complete by the end of the Old High \ili{German} period (c.~750--1050), where it occurs even with unique common nouns, e.g.~\textit{thiu sunna}, `the sun' (9\textsuperscript{th} century, Otfrid, as discussed in \citealp[103]{Schmuck2014}). The establishment of the \isi{article} with unique common nouns -- which, like personal names, are inherently definite -- is seen as a crucial step which in turn facilitated the \isi{article}'s further grammaticalisation to proper nouns, including personal names, as of the Early New High \ili{German} period (c.~1350--1650) (\citealp{Schmuck2014}; \citealp{schmuck2020grammaticalisation}). In this context, \citet{Schmuck2014} propose  (\ref{schmuck-scale}) as the grammaticalisation pathway of the definite \isi{article}, which is an adapted version of that in \citet[337]{lyons1999definiteness}, also taking into account uniques.
\end{sloppypar}

\ea \label{schmuck-scale} simple definite > generic > unique > proper \isi{noun}
\z 

In addition, many have attributed the rise of the PA in \ili{German} to the loss of case-marking on proper nouns (e.g.~\citealp[52--55]{behaghel1923deutsche}; \citealp{Schmuck2014}; \citealp{ackermann2018grammatik}; \citealp{schmuck2020rise}). In Old High \ili{German}, proper nouns inflected like common nouns, but case-marking was lost as of the Middle High \ili{German} period (1050--1350). The assumption is that, as case-marking was lost on proper nouns, including on personal names, this was compensated for by the newly emerging PA which became the exclusive exponent of case. This is supported, for instance, by data in \citet[153--154, 188--189]{ackermann2018grammatik}, who observes that use of the PA increases as case \isi{inflection} is lost.\largerpage

\begin{sloppypar}
In \isi{contrast} to West \ili{Germanic}, the PA is attested comparatively early in North \ili{Germanic}, and is exhibited already in the earliest attested stage, Old \ili{Norse}/Icelandic (c.~1150--1450), e.g.~(\ref{ice-top}). As in the modern language, the \ili{Old Icelandic} (plain) PA takes the form of the 3\textsc{sg} personal \isi{pronoun} (masc.~\textit{hann}, fem.~\textit{hún}).\footnote{In the \isi{corpus} data which this chapter makes use of (outlined in Section \ref{sect:method}), all texts are normalised to modern \ili{Icelandic} orthography, regardless of their date. For sake of consistency, I retain this normalised orthography here.}
\end{sloppypar}


 \ea[]{\label{ice-top} \ili{Old Icelandic}\\
 \gll  Og er Túta kemur fyrir Halla þá réttir [\textbf{hann} Halli] hendur {í móti} grísinum\\
    and when Túta.\textsc{nom} comes before Halli.\textsc{acc} then outstretches \textsc{pa.m.nom.3sg} Halli.\textsc{nom} hands.\textsc{acc} towards pig.\textsc{dat.def}\\
 \glt  `And when Túta comes before Halli, then Halli stretches out his hands towards the pig.' (IcePaHC: 1275, Morkin.1156)
}
\z 

In light of the development of the PA in \ili{German}, where it is generally assumed that the loss of case and the grammaticalisation of the definite \isi{article} to unique common nouns were crucial factors, the early attestation of the PA in \ili{Old Icelandic} is interesting; \ili{Old Icelandic} has rich morphological \isi{inflection} on nominals, including personal names -- as indeed the modern language still does~-- and it is well known that the definite \isi{article} was not yet fully grammaticalised with common nouns at this stage (\citealp{leiss2000artikel, leiss2007covert}; \citealp{Borjarsetal16}).
As \citet{Borjarsetal16} discuss, \isi{noun} phrases in Old \ili{Norse}/Icelandic do not require an explicit marker of (in)\isi{definiteness} in order to receive a definite/\isi{indefinite} interpretation.  In this context, the thorough investigations by \citet{leiss2000artikel,leiss2007covert} show that \ili{Old Icelandic} can be considered a ``hypodeterminating language'', whereby \isi{definiteness} is overtly marked on definite expressions where \isi{definiteness} cannot be presupposed, i.e.~rhemes, but is not marked on expressions which are inherently definite, i.e.~themes and proper nouns. 
Moreover, evidence from \ili{Old Swedish} suggests that the definite \isi{article} was not grammaticalised to unique and generic contexts at this early stage of North \ili{Germanic} \citep{skrzypek2012grammaticalization}.\largerpage


As such, the early attestation of the PA in North \ili{Germanic} cannot be related to the establishment of a highly grammaticalised definite \isi{article} in the way that the rise of the \ili{German} PA is often accounted for. Rather, as its formal
identity with personal pronouns suggests, the North \ili{Germanic} PA should be considered on its own terms, separate to the ongoing grammaticalisation of \isi{definiteness}.\footnote{In this sense, the label ``proprial \textit{article}'' is perhaps misleading. Nevertheless, as it is by far the most common term in the literature on North \ili{Germanic}, I retain the term here.} As further support of this, \citet[98]{dahl2015grammaticalization}, citing synchronic data from \citet{delsing1993internal}, points out that, although there is significant overlap in the distribution of the proprial \isi{article} and ``extended'' use of definite forms (e.g.~with generics) in present-day North \ili{Germanic} varieties, there are also dialects which have the PA and no extended use of definite forms, %, as in most \ili{Norwegian} dialects,
as well as those which have extended use of definite forms and no PA. % such as Ovansiljan dialects.
This synchronic distribution leads Dahl to suggest that the PA and extended uses of definite forms have separate histories of origin, in other words cannot be considered as part of the same grammaticalisation process as they are for historical \ili{German}.


%%%%%
\subsection{The disputed status of the plain proprial article}

As already mentioned, the status of the plain PA in early North \ili{Germanic} is disputed in the literature (\citealp{Faarlund04}; \citealp{Kinn2016-thesis}; \citealp{Johnsen2016}).
\citet[89]{Faarlund04}, for instance, notes that the plain PA in Old \ili{Norse}/Icelandic in the singular (i.e.~\textit{hann, hún}) is  ``rather unusual'' and ``mostly confined to a colloquial style''. \citet[165]{Kinn2016-thesis} observes that a \textsc{3sg} personal \isi{pronoun} \textit{hann} ``sporadically co-occurs with proper names'' in Old \ili{Norwegian}, providing, for instance, the example in (\ref{kinn-pa}).

\ea \label{kinn-pa}  Old \ili{Norwegian} \\
\gll Oc i þuí k{\oe}mr [\textbf{hann} asbiorn] i stovuna \\
   and in that comes he Ásbj\c{o}rn in dining.room.\textsc{def} \\
\glt  `And in that moment, Ásbj\c{o}rn entered the dining room.' \citep[165]{Kinn2016-thesis}\\
\z

\citet[165]{Kinn2016-thesis} claims that personal pronouns in contexts like (\ref{kinn-pa}) in Old \ili{Norwegian} do not seem to have any semantic or pragmatic effect. She instead assumes that such instances, while superficially similar to the modern PA construction, are in fact cases of straightforward apposition. This is in line with the theory of null subjects which is the main component of her thesis.

\citet{kinn2017null} revisits the status of the (plain) PA in Old \ili{Norwegian}. She notes that in her dataset taken from two texts (\textit{The Legendary Saga of St. Óláfr} and \textit{The Old \ili{Norwegian} Homily Book}), the appearance of a plain PA-like element (\textit{hann, hon}) before a personal name does not appear to be systematic; there are only four such instances, and the great majority of personal names appear without any accompanying \isi{pronoun}. She contrasts this finding with the study by \citet[98]{dahl2015grammaticalization}, which found a more systematic use of \textit{hann/hon} before personal names in a short \ili{Norwegian} charter in the \textit{\ili{Norwegian} Diplomatarium} from 1430. On the basis of this, \citet{kinn2017null} suggests that the PA emerged in some dialects around that particular time, i.e.~later than the Old \ili{Norwegian} data she herself examined. 
 
 
\citet{Johnsen2016}, however, takes issue with Kinn's claim for Old \ili{Norwegian} and provides early examples from \ili{Old Icelandic} which he claims exhibit proprial articles, e.g.~(\ref{john-skalla}), which is taken from an episode in which King Harald Fairhair meets Skalla-Grímr, from an Icelandic manuscript from c.~1320--1350.
 
 \ea \label{john-skalla} \ili{Old Icelandic} \\
\gll \c{O}lvir tók til máls: ``Nú er Grímr hér kominn, sonr Kveld-Ulfs.'' [\dots] Konungr litaðist um. Hann sá, at maðr stóð at baki \c{O}lvi [\dots]. ``Er þetta [\textbf{hann} Skalla-Grímr]'', sagði konungr, ``in mikli maðr?'' Grímr sagði, at hann kenndi rétt. ``Ek vil þá'', sagði konungr, ``ef þú beiðast bóta fyrir Þórolf, [\dots] veita þér [\dots] s\'{ø}md, eigi minni en ek veitta [\textbf{honum} Þórolfi], bróður þínum''\\
\c{O}lvir took to speech now is Grímr here come son Kveld-Ulfr's [\dots] king looked around he saw that man stood at back \c{O}lvir [\dots]. is this he Skalla-Grímr said king the great man Grímr said that he knew right I will then said king if you request compensation for Þórolfr [\dots] give you [\dots] honor not smaller than I gave him Þórolfr brother yours\\
\glt  `\c{O}lvir began speaking: ``Now Grímr has arrived, the son of Kveld-Ulf''. [\dots] The king looked around. He saw a man standing behind \c{O}lvir [\dots]. ``Is this Skalla-Grímr'', said the king, ``the great man?'' Grímr said that he was right. ``Then I wish'', said the king, ``if you request compensation for Þórolfr, [\dots] to honor you no less than I honored Þórolfr, your brother.'' '
\citep[197]{Johnsen2016}
\z
 
\citet{Johnsen2016} provides a number of convincing arguments that these examples can be considered instances of the (plain) PA and that they do not merit analysis as apposition, including the fact that the \isi{pronoun} cannot stand on its own with its referent retrievable from context, as in the example in (\ref{john-ambig}). Since Ketill Auðunarson has not been mentioned earlier in this chapter, nor the fact that anyone is going to receive rafters, without the proper name the referent is impossible to identify.



\ea \label{john-ambig} \ili{Old Icelandic}\\
\gll Halli á Hakavíkinni borgaði fyrir Loðini á Holtum uppá eitt hundrað sperna [\textbf{honum} Katli Auðunarsyni] \\
   Halli on Hakavika bailed for Loðinn on Holtar upon one hundred rafters him Ketill Auðunarson\\
\glt  `Halli from Hakavika guaranteed one hundred rafters to Ketill Auðunarson on behalf of Loðinn from Holtar.' \citep[200]{Johnsen2016}
\z

\noindent In terms of the pragmatic properties of the PA in early \ili{Norwegian}/Icelandic, \citet{Johnsen2016} observes that the PA-marked referent is known and given information in the context, while personal names which refer to individuals who are not familiar from the context are not accompanied by a PA. Johnsen's claims, however, are made on the basis of a relatively small dataset, and so merit testing on a larger scale.

Finally, \citet[89]{Faarlund04} specifically comments on the plain PA in the plural with a coordinated referent in Old \ili{Norse}/Icelandic (`they X and Y'), saying that this is much more common than the singular plain PA with a single name (`he X'). He provides the example in (\ref{faarlund-pl}).

\ea \label{faarlund-pl} Old \ili{Norse}/Icelandic\\
\gll með hverjum skildaga [\textbf{þeir} Einarr ok Brúsi br{\oe}ðr] h\c{o}fðu félag sitt g\c{o}rt\\
with what \isi{agreement}.\textsc{dat} \textsc{pa.m.nom.3pl} Einar.\textsc{nom} and Brúsi.\textsc{nom} brothers.\textsc{nom} had partnership.\textsc{acc} their.\textsc{refl.acc} made\\
\glt `with what \isi{agreement} the brothers Einar and Brusi had formed a partnership' (Hkr II.206.15, \citealp[89]{Faarlund04})
\z 

%%%%%
\subsection{The inclusory proprial article, number and associativity}

Besides the plain PA, \ili{Old Icelandic} also exhibits the inclusory PA, as discussed in early philological work on Early \ili{Germanic}, notably \citet[220]{bergmann1838poemes}, \citet[350--351]{grimm1898deutsche} and \citet[§395--396, §404--405]{heusler1962altislandisches}, as well as more recently by \citet[90]{Faarlund04}. Both Grimm and Heusler point out that the inclusory PA can be a \isi{dual} \isi{pronoun} in the 1st and 2nd person, which in \ili{Old Icelandic} retain a distinction between \isi{dual} (\textsc{du}) and plural (\textsc{pl}) in the personal \isi{pronoun} paradigm, cf.~Table \ref{tab:du-pl}. Thus, in the first and second persons, one finds both \isi{dual} PAs and plural (inclusory) PAs, as in (\ref{dual-1st}) for the first person and (\ref{dual-2nd}) for the second person.

\begin{table}
    \caption{Distinction between dual and plural in the 1st and 2nd person personal pronouns in Old Icelandic \citep[61]{Barnes08} }
    \label{tab:du-pl}
 \begin{tabular}{l llll}
  \lsptoprule
  & \multicolumn{2}{c}{1st person} & \multicolumn{2}{c}{2nd person}\\\cmidrule(lr){2-3}\cmidrule(lr){4-5}
  &  \textsc{du} &  \textsc{pl}   &  \textsc{du} &  \textsc{pl}\\
  \midrule
    \textsc{nom} & \textit{vit} & \textit{vér} & \textit{(þ)it} & \textit{(þ)ér}\\
    \textsc{acc} & \textit{ok(k)r} & \textit{oss} & \textit{yk(k)r} & \textit{yðr}\\
    \textsc{dat} & \textit{ok(k)r} & \textit{oss} & \textit{yk(k)r} & \textit{yðr}\\
    \textsc{gen} & \textit{okkar} & \textit{vár} & \textit{ykkar} & \textit{yð(v)ar}\\
  \lspbottomrule
 \end{tabular}
\end{table}


\ea \label{dual-1st} Old \ili{Norse}/Icelandic
\ea \gll erom [\textbf{vit} Gunnarr] nú sátter\\
are \textsc{pa.nom.1du} Gunnarr.\textsc{nom} now  reconciled\\
\glt `Gunnar and I are now reconciled'
\ex \gll sætt, þeire er konungr gørþe mille [\textbf{vár} Brúsa]\\
\isi{agreement} \textsc{dem} \textsc{rel} king.\textsc{nom} made between \textsc{pa.gen.1pl} Brúsi\\
\glt `the \isi{agreement} which the king made between us and Brúsi' \\ 
\citep[p.~124, §395]{heusler1962altislandisches}
\z 
\z 

\ea \label{dual-2nd} 
\ea Old \ili{Norse}/Icelandic \\
\gll þó at [\textbf{it} Egell] talezk viþ\\
though \textsc{comp} \textsc{pa.nom.2du} Egill.\textsc{nom} speak.\textsc{recp} with\\
\glt `although you(sg) and Egill speak with each other'\\
\citep[p.~124, §395]{heusler1962altislandisches}
\ex  \gll og hefir þetta mikið um spillt, er [\textbf{þér} Eyvindur] fundust við Jótland."\\
and has \textsc{dem} much \textsc{ptcl} worsened \textsc{rel} \textsc{pa.nom.2pl} Eyvindur.\textsc{nom} met.\textsc{recp} by Jutland\\
\glt `and this has greatly worsened since you(pl) and Eyvindur met each other by Jutland' (IcePaHC: 1250, Thetubrot.73)
\z
\z

\citet[350]{grimm1898deutsche} provides similar examples of inclusory PAs in the \isi{dual} from both Old \ili{English} and Old High \ili{German} poetry, e.g.~(\ref{grimm-wgmc}), which indicates that this particular construction is a broader Early \ili{Germanic} phenomenon, although examples in West \ili{Germanic} seem to be much rarer than in North \ili{Germanic}.

\ea \label{grimm-wgmc}
\ea Old \ili{English} \\
\gll \textbf{vit} Scilling \\
\textsc{pa.nom.1du} Scilling.\textsc{nom}\\
\glt `Scilling and I' (Traveller's Song, line 103, \citealp[350]{grimm1898deutsche})
\ex Old \ili{English} \\
\gll \textbf{uncer} Grendles\\
\textsc{pa.gen.1du} Grendel.\textsc{gen}\\
\glt `mine and Grendel's' (Beowulf, line 2002, \citealp[350]{grimm1898deutsche})
\ex Old High \ili{German}\\
\textbf{wiz} Hiltiprant \\
`Hildebrandt and I' \citep[350]{grimm1898deutsche}\footnote{The specific text is not provided by Grimm. I have not been able to find the precise example, and it has been claimed that Grimm himself constructed this example \citep[236, fn.~2]{krause1924entwickelung}; thanks to Nelson Goering and Svetlana Petrova for pointing this out.}
\z 
\z 


In the third person, which lacks a \isi{dual} in Old \ili{Norse}/Icelandic, inclusory PAs in the plural are also attested and have been discussed in modern theoretical work as an ``\isi{associative plural}'' construction  (\citealp{denbesten1996associative}; \citealp{moravcsik2003semantic}; \citealp{wals-36}; \citealp{sigurdhsson2020we}), i.e.~a construction which refers to a focal, typically human referent, plus their (unnamed) associates. Such discussions refer to examples like (\ref{gizorr}).


\ea \label{gizorr} Old \ili{Norse}/Icelandic\\
\gll \textbf{þeir} Gizorr\\
\textsc{pa.m.nom.3pl} Gizorr.\textsc{nom}\\
\glt `Gizorr and his associates' \citep[ex. (4)]{wals-36}
\z

However, \citet[§404]{heusler1962altislandisches}, who discusses the example in (\ref{gizorr}) in detail, notes that it can have two different meanings: (i) `Gizorr and his people' and (ii) `Gizorr plus another named individual'.\footnote{Original: ``G.~und die Seinen'' and ``ein genannter nebst G.'' respectively \citep[§404]{heusler1962altislandisches}.}  According to Heusler, in the first case Gizorr is the main person, around which one or several unnamed (or not to be named again) people are grouped, i.e.~an \isi{associative plural}.\footnote{Original: ``Im ersten Falle ist G.~die Hauptperson, an die sich ein oder mehrere ungennante (oder nicht wieder zu nennende) anreihen, ``G.~und die um ihn'' '' \citep[§404]{heusler1962altislandisches}.} In the second case, the already named individual is in the ``consciousness'' of the speaker, to which Gizorr is added as a second person.\footnote{\label{heusler-fn}Original: ``Im zweiten Falle liegt der andere, schon genannte (\dots) im Bewußtsein des Sprechenden, so daß nur Gizorr als Ergänzung, als 2.~Person hinzugefügt werden muß.''} 
These two meanings are also reflected in the translations of the examples provided by \citet[90]{Faarlund04}, e.g.~(\ref{faarlund-pas}), though he does not discuss the construction in detail.\footnote{\citet[9]{sigurdhsson2020we} also acknowledge this second function of the inclusory PA in modern \ili{Icelandic}, and in fact state that the \isi{associative plural} usage of the construction, as seen in \ili{Old Icelandic}, is ``obsolete'' in the modern language.} 

\ea \label{faarlund-pas} Old \ili{Norse}/Icelandic
\ea \gll [\textbf{þeir} Ásbj{\k{o}}rn] lendu útan at eyjunni\\
\textsc{pa.m.nom.3pl} Ásbj{\k{o}}rn landed from.out at island.\textsc{dat.def} \\
\glt `Ásbj{\k{o}}rn and his men landed on the outside of the island'\\ 
(Hkr II.250.18, \citealp[90]{Faarlund04})
\ex \gll hvat [\textbf{þau} dróttning] tala jafnan\\
what \textsc{pa.n.nom.3pl} queen.\textsc{nom} talk constantly \\
\glt `what he and the queen are always talking about'\\
 (Hkr I.293.5, \citealp[90]{Faarlund04})
\z 
\z 


Strikingly, the inclusory PA is also attested in the Poetic Edda, as pointed out by \citet[220]{bergmann1838poemes}, who provides the examples in (\ref{edda-pa}).

\ea \label{edda-pa} Old \ili{Norse}/Icelandic
\ea \gll \textbf{þau} Högni\\
\textsc{pa.n.nom.3pl} Högni.\textsc{nom}\\
\glt `she and Högni' (Atlamál, verse 10, \citealp[220]{bergmann1838poemes})
\ex  \gll \textbf{við} Freyr\\
\textsc{pa.nom.1du} Freyr.\textsc{nom}\\
\glt `Freyr and I' (Skírnismál, verse 20, \citealp[220]{bergmann1838poemes})
\ex \gll \textbf{ið} Gymir\\
\textsc{pa.nom.2du} Gyrmir.\textsc{nom}\\
\glt `you(sg) and Gymir' (Skírnismál, verse 24, \citealp[220]{bergmann1838poemes})
\z 
\z

Such examples in the Poetic Edda, which preserves poems likely composed in c.~800-1100 CE, show that the inclusory PA has a long history in North \ili{Germanic}.

%%%%%%%%%%%%%%%%%%%%%%%%%%%%%%%%%%%
\section{Methodology}\label{sect:method}

%%%%
\subsection{Data collection}

The claims in this chapter are based on data from two parsed corpora of historical Icelandic, IcePaHC \citep{IcePaHC} and MIcePaHC \citep{MIcePaHC}, henceforth referred to collectively as ``(M)IcePaHC''. IcePaHC spans the whole \ili{Icelandic} diachrony from 1150-2008 CE, with 61 text excerpts from varying genres, which altogether contain around 1 million words. MIcePaHC is an extended \isi{corpus} of \ili{Old Icelandic} saga texts currently under development, and I use this resource to complement the IcePaHC data, since the PA is a relatively low-frequency phenomenon, at least in the extant written texts which are available to us from the \ili{Old Icelandic} period. 

I restrict the study to texts dated up to 1450 ($\approx$ \ili{Old Icelandic}), according to the dating provided by the corpora themselves.\footnote{1450 is relatively late to be considered ``\ili{Old Icelandic}'', but since the PA is a relatively low-frequency phenomenon, I stretch the period to gather as much data as possible.} Both corpora are syntactically  annotated according to the Penn Treebank format established for historical \ili{English} \citep{Santorini2010annotation}, which allows for the extraction and quantitative investigation of specific hierarchical structures and linear orders via the CorpusSearch query language \citep{randall2005corpussearch2}. Each sentence from (M)IcePaHC is equipped with a unique sentence ID which provides information about the year and name of the text, the text \isi{genre} and the number of the token in the text. When citing an example from (M)IcePaHC, I specify the particular \isi{corpus} and provide the year, text name and token number, allowing for identification of the example in the relevant \isi{corpus}. 

In the (M)IcePaHC annotation, the plain and the inclusory PA are treated identically as a \isi{pronoun} which combines with an appositive \isi{noun} phrase (\texttt{NP-PRN}), headed by a proper \isi{noun} (\texttt{NPR-*}). An example of an annotated plain PA is provided in (\ref{icepahc-pa}) and of an inclusory PA in (\ref{icepahc-gap}).\footnote{See the official annotation policy at \url{https://linguist.is/icelandic_treebank/NP-PRN}.}

\ea 
 \ea[]{ \label{icepahc-pa} 
\texttt{
(NP-SBJ (PRO-N hann-hann)\\
\hspace*{9ex}     (NP-PRN (NPR-N Ófeigur-ófeigur)))}
}
\ex[]{\label{icepahc-gap} 
 \texttt{ 
(NP-SBJ (PRO-N þeir-hann)\\
\hspace*{9ex} 			(NP-PRN (NPR-N Þorleifur-þorleifur))) }
}
\z 
\z



I extract all third person PAs as the basis of the study via CorpusSearch queries \citep{randall2005corpussearch2};  as mentioned in Section \ref{sect:hist-context}, the inclusory PA also occurs in the first and second person but I leave such examples for further research. 

\begin{sloppypar}
As outlined in Section \ref{sect:bg}, in contexts where the inclusory PA combines with more than one personal name in a \isi{coordination} structure, it will be identical in form to a plain PA construction.
There are many such examples in the (M)IcePaHC data where, without contextual information, the construction could in principle be an instance of either the plain or inclusory PA, e.g.~(\ref{old-ambig}).
\end{sloppypar}

\ea \label{old-ambig} \ili{Old Icelandic}
 \ea[]{ 
 \gll  Síðan fara [\textbf{þeir} Arinbjörn og Egill] á fund Bjarnar\\
then go \textsc{pa.m.nom.3pl} Arinbjörn.\textsc{nom} and Egill.\textsc{nom} to meeting.\textsc{acc} Björn.\textsc{gen}\\
 \glt `Then Arinbjörn and Egill (at least)  go to a meeting with Björn.'\\
 (IcePaHC: 1250, Thetubrot.60)
}


\ex[]{ \gll [\textbf{Þau} Rannveig og Gamli] tóku allvel við Gretti\\
  \textsc{pa.n.nom.3pl} Rannveig.\textsc{nom} and Gamli.\textsc{nom} received very.well with Grettir.\textsc{dat}\\
\glt  `Rannveig and Gamli (at least) gave Grettir a very good welcome.'\\
(IcePaHC: 1310, Grettir.1635)
}

\z 
\z


Examples like (\ref{old-ambig}) can only be categorised as plain or inclusory via close manual examination of the example in context; it is not possible to categorise them automatically via the (M)IcePaHC annotation. Thus, I set them aside as a third ``mixed'' group, alongside a set of examples where the PA is identical in form to the 3\textsc{sg} \isi{pronoun} \textit{hann/hún} and which are straightforwardly all of the plain type, cf.~(\ref{icepahc-pa}), and  a set which are straightforwardly all of the inclusory type, i.e.~examples where the PA is identical in form to the 3\textsc{pl} \isi{pronoun} \textit{þeir/þær/þau} and where the PA combines with only one personal name, cf.~(\ref{icepahc-gap}).



The inclusory and mixed types are attested more frequently in the corpora than the plain type, which is less frequent. Thus, while I rely only on IcePaHC data for the inclusory and mixed types, for the plain type I also include relevant examples from MIcePaHC to supplement the small number in IcePaHC; for the inclusory and the mixed type, including all examples from MIcePaHC would yield too many examples to allow manual qualitative checks. After manual checking of the search outputs to exclude misannotations and erroneous examples, this yields the three datasets outlined in Table \ref{tab:all}.\footnote{In order to make the study of manageable scope, I restrict the study to third person instances of the PA, and exclude any examples which include nouns tagged as ``proper nouns'' which are not personal names, e.g.~\textit{goði} `chieftain', \textit{jarl} `earl'.
}


\begin{table}
    \caption{PAs in (M)IcePaHC (1150–1450)}
    \label{tab:all}
 \begin{tabular}{lccc}
  \lsptoprule
    Corpus   &  Plain &  Inclusory & Mixed\\
  \midrule
    IcePaHC   & 38 & 169 & 107 \\
    MIcePaHC   & 46 & -- &  --\\
    \midrule 
        Total  & 84 & 169 & 107 \\
  \lspbottomrule
 \end{tabular}
\end{table}


The plain and inclusory subsets in Table \ref{tab:all} are manually tagged for properties relevant to the investigation, specifically (i) the grammatical function of the PA-marked expression, (ii) whether the referent marked by the PA is discourse-given or discourse-new, and (iii) whether the referent marked by the PA represents a topic, and if so, what type of topic transition is relevant as per the definitions in (\ref{topic-types}) below.


In terms of the distribution of the PA across different text types, one can look at the IcePaHC data in isolation to gain at least an impression, since that \isi{corpus} spans a range of genres (narrative, religious, biographical, scientific and legal texts), while MIcePaHC consists of solely saga texts. The generalisation for all three subsets of data in Table \ref{tab:all} is that the PA is virtually restricted to narrative texts in \ili{Old Icelandic}, i.e.~sagas. For the plain subset, all 38 examples from IcePaHC occur in sagas. For the inclusory subset, all but one of the 169 examples occur in sagas and all but one of the 142 examples from the mixed subset occur in sagas. The two examples of the PA found in non-sagas occur in  the religious texts \textit{Homiliubok} and \textit{Judit} and are provided in (\ref{bib-pa}).

\ea \label{bib-pa} \ili{Old Icelandic}
 \ea[]{ \gll að af træenu kom það epli, er [\textbf{þau} \textbf{Eva} \textbf{og} \textbf{Adamur}] átu fyrirboðið \\
\textsc{comp} of tree.\textsc{dat.def} came \textsc{dem.nom} apple.\textsc{nom} \textsc{rel} \textsc{pa.n.nom.3pl} Eve.\textsc{nom} and Adam.\textsc{nom} ate forbidden\\
 \glt `that from the tree came that apple, which Eve and Adam ate \\(and it was) forbidden' (IcePaHC: 1150,  Homiliubok.2082)
}
\ex[]{ \gll og hugði að \textbf{þau} \textbf{Júdit} mundu sofa bæði saman\\
  and thought \textsc{comp} \textsc{pa.n.nom.3pl} Judith.\textsc{nom} would sleep both together\\
\glt `and thought that he and Judith would both sleep together'\\
(IcePaHC: 1450, Judit.434)
} 
\z 
\z

\textit{Homiliubok} is a collection of sermons featuring extensive quoted passages from the bible, and \textit{Judit} is a bible translation of the Book of Judith, and it is clear from the examples in (\ref{bib-pa}) that they occur in narrative passages. Thus, one can claim on the basis of the IcePaHC data that, at least within the written language, the PA in \ili{Old Icelandic} appears to be a narrative-specific device.

%%%%%
\subsection{Topicality and topic transitions}

Any study of the morphosyntax--\isi{information structure} interface must first outline one's terminology and understanding of key information-structural concepts. In particular, terms such as ``topic'' and ``focus'' subsume a range of notions depending on author and approach, and the definition of topichood in particular is a slippery customer (e.g.~\citealp{chafe1976givenness}; \citealp{reinhart1981pragmatics}; \citealp{givon1983topic}; \citealp{jacobs2001dimensions}; \citealp{frascarelli2007types}; \citealp{krifka07}; \citealp{neeleman2009syntactic}; \citealp{bianchi2010topic}). In this chapter, ``topic'' will be understood as roughly equivalent to ``\isi{aboutness topic}'', i.e.~the entity about which information is expressed (cf.~``sentence topic'', \citealp{reinhart1981pragmatics}).
In this context, the diagnostic tests provided by \citet[165]{gotze2007information} can be used to identify the \isi{aboutness topic} of an utterance, cf.~(\ref{aboutness-test}).

\ea \label{aboutness-test}  An NP X is the \isi{aboutness topic} of a sentence S containing X if:\\
 \ea[]{ S would be a natural continuation to the announcement\\
         \textit{Let me tell you something about X}
}
\ex[]{ S would be a good answer to the question\\
         \textit{What about X?}
}
\ex[]{ S could be naturally transformed into the sentence\\
         \textit{Concerning X, S'}\\
         where S' differs from S only insofar as X has been replaced by a suitable \isi{pronoun}
}
\z 
\z
 
\begin{sloppypar}
As already mentioned, there is good reason to assume that studies of information-structural phenomena should go  beyond a distinction between given and new and this is no less the case with topic and focus; whereas topics are prototypically given and foci prototypically new, there are many non-trivial exceptions to these general correspondences (see e.g.~\citealp{cook2013identifying}). With respect to \isi{topicality}, one way of distinguishing between different subtypes of topic is to go beyond whether a referent is given or new and instead consider specifically the relation between a current topic and the topic of the immediately preceding utterance, i.e.~the topic transition (see e.g.~\citealp{danes1974functional} and ``Centering Theory'' in \citealp{groszetal95}). In this chapter, I recognise four types of topic transition, as defined in (\ref{topic-types}) (cf.~\citealp{NapolesRiester2021} for a similar typology).
\end{sloppypar}

\ea \label{topic-types}
\begin{enumerate}[noitemsep]
    \item \textsc{topic continuity}: current topic is co-referential with topic of immediately preceding utterance
        \item \textsc{topic promotion}: current topic is co-referential with focus of previous utterance
    \item \textsc{topic resumption}: current topic is co-referential with an earlier topic which was not the topic of the immediately preceding utterance
     \item \textsc{subsectional topic selection}: current topic is an element of a previously introduced set of entities
\end{enumerate}
\z 

Type 1 in (\ref{topic-types}), \isi{topic continuity}, equates to notions defined elsewhere as ``familiar topics'' \citep{frascarelli2007types} or ``continuous topics'' (\citealp{bianchi2010topic}; cf.~also \citealp{givon1983topic}), whereas types 2--4 represent various types of what are standardly labelled ``shift-topics'' (\citealp{frascarelli2007types}; \citealp{bianchi2010topic}). Note however that types 2--4 each involve a shift of topic to a referent which, although not the topic of the preceding sentence, is already present in the discourse in some way, i.e.~given/familiar. This will become particularly relevant in the discussion of the plain proprial \isi{article} alongside other topic management devices in \ili{Old Icelandic} in Section \ref{sect:plain}.

%%%%%%%%%%%%%%%%%%%%%%%%%%%%%%%%%%%
\section{Topic management and the plain proprial article}\label{sect:plain}

Recent years have seen a surge of interest in the morphosyntax--\isi{information structure} interface in Early \ili{Germanic}, especially within Early West \ili{Germanic}, (e.g.~\citealp{Trips2009syntax}; \citealp{HinterholzlPetrova09, hinterholzl2010v1}; \citealp{petrova2010evidence}; \citealp{Epstein2011}; \citealp{Breban12}; \citealp{MeurmanSolinetal12}; \citealp{vangelderen2013diachrony}; \citealp{BechEide14}; \citealp{los-Kem2018}; \citealp{catasso2021he}). Particular attention has been centred on the various devices which are employed for the management of discourse participants, and different types of topic transition (e.g.~\citealp{Epstein2011}; \citealp{Breban12}; \citealp{vangelderen2013diachrony}; \citealp{los-Kem2018}; \citealp{catasso2021he}). By comparison, relatively little has been said about Early North \ili{Germanic} in this context.\footnote{Relevant exceptions include \citet{kossuth1980-narrative-syntax}, \citet{leiss2007covert}, and \citet{booth-beck2021jhs}.} In this section, I examine various morphosyntactic devices in \ili{Old Icelandic}, including the proprial \isi{article}, in terms of how they contribute to topic management. The discussion in this section is limited to the plain PA; I examine the inclusory PA in Section \ref{sect:gapped}.

%%%%%%%
\subsection{Narratives and information structure}

As outlined in Section \ref{sect:method}, the evidence from IcePaHC indicates that the PA is a narrative-specific phenomenon, at least in the exclusively written language which is available to us from the period. As many authors have noted (e.g.~\citealp{carroll2003information}; \citealp{dimroth2010given}; \citealp{riester2015analyzing}), narratives as a \isi{genre} bring their own specific characteristics which interact with the expression of \isi{information structure}. \citet{riester2015analyzing} notes, for instance, that narratives are primarily structured on the temporal dimension, and that the ``question-under-discussion'' (\citealp{vonStutterheim1989referential}; \citealp{vankuppelt1995discourse}) is typically a global one (e.g.~\textit{What happened? What happened next?}). \citet{dimroth2010given} note that this global question-under-discussion which underlies so much of narrative texts results in a prototypical narrative structure where the time talked about (```topic time''', \citealp{klein1994time}) shifts from one utterance to the text, while the protagonist is maintained, and the predicate that holds for the protagonist constantly changes. 


Furthermore, medieval Icelandic sagas must be considered on their own terms as a particular type of narrative with their own saga-specific linguistic traits, which may be to some extent linked to their (at least partly) oral origins (e.g.~\citealp{byock1984saga}; \citealp{quinn2000orality}; \citealp{sigurdhsson2007orality}). Various literary studies of the sagas have pointed out the rather unique style of saga narrative. \citet{clover1974scene}, for instance, refers to sagas as exhibiting a ``narrative of parataxis'', where a series of relatively independent units or ``scenes'' occur in paratactic sequence, without connecting narrative of any kind. As she also notes, the narrative in the sagas is often ``stranded'', with the scene shifting back and forth between accounts, involving rhetorical devices of scene-setting. Similarly, \citet{byock1994narrating} observes that the basic building blocks of saga structure are small, discrete particles of action and that they have a characteristic sense of homogeneity, with repeated presentation of incident after incident, in an economic style which the sagas have become famous for. 

Given their rather unique style, it is unsurprising that certain authors have highlighted various morphosyntactic phenomena with special pragmatic properties which are particularly characteristic of saga texts, such as the ``\isi{narrative inversion}'' V1 \isi{pattern}  (\citealp{kossuth1980-narrative-syntax}; \citealp{Platzack1985}; \citealp{hopper1987}; \citealp{booth-beck2021jhs}), discussed below in Section \ref{subsect:cont-ni}, tense switching \citep{richardson1995tense} and certain formulae which signal a shift in scene and/or temporal backtracking \citep{clover1974scene}. In this section, I claim that the PA is another such device employed for a specific type of topic management.

%%%%%%%
\subsection{Givenness, topic continuity and narrative inversion}\label{subsect:cont-ni}

As outlined in Section \ref{sect:bg}, previous claims regarding the pragmatics of the \ili{Icelandic} PA have focused on the given/new dimension, with the standard view that it is a \isi{familiarity}/\isi{givenness} marker, both in the plain variety and the inclusory type \citep{Sigurdsson2006}. Moreover, this claim has been extended to early \ili{Norwegian}/Icelandic by \citet{Johnsen2016}, as also discussed in Section \ref{sect:bg}. However, on closer inspection it is clear that the plain PA is not motivated in prototypical \isi{givenness} contexts, for instance, where a single referent is maintained as the topic (cf.~``\isi{topic continuity}'' in (\ref{topic-types}) above) and where no other referents are active in the discourse. Rather, in such contexts, the referent is expressed via straightforward personal pronouns, in line with the expression of \isi{topic continuity} in Early \ili{Germanic} more generally (e.g.~\citealt{vangelderen2013diachrony}; \citealt{los-Kem2018}). An example is provided in (\ref{ex-cont-top}), which represents a continuous discourse segment from the opening of a new chapter, where the character of Hafliði Höskuldsson is introduced and maintained as the topic throughout.\pagebreak

\ea \label{ex-cont-top} \ili{Old Icelandic}\\
 \ea[]{ 
 \gll  Maður hét Hafliði Höskuldsson bróðir Sighvats auðga \\
    man.\textsc{nom} was.called Hafliði Höskuldsson brother.\textsc{nom} Sighvatur.\textsc{gen} wealthy.\textsc{gen}\\
 \glt  `There was a man called  Hafliði Höskuldsson, brother of Sighvatur the wealthy.' 
}
\ex[]{ 
\gll \textbf{Hann} dreymdi um veturinn eftir jól {þá er} Melaför var að hann var úti staddur á Kolbeinsstöðum \\
    he.\textsc{acc} dreamt in winter.\textsc{def} after Christmas when Melaför was \textsc{comp} he.\textsc{nom} was outside stood at Kolbeinsstaðir\\
\glt  `He dreamt in the winter after Yule, when Melaför was, that he was standing outside at Kolbeinsstaðir.'
}
\ex[]{ 
\gll Þar átti \textbf{hann} heima í Haugatungu \\
    there had he.\textsc{nom} home in Haugatunga\\
\glt   `He had his home there in Haugatunga.'
}
\ex[]{ 
\gll \textbf{Hann} sá að leikur var sleginn þar skammt frá garði\\
    he.\textsc{nom} saw \textsc{comp} game.\textsc{nom} was struck there not.far from farmstead\\
\glt  `He saw that a game was struck there not far from the farmstead.'\\
(IcePaHC: 1250, Sturlunga.389.28--32)
} 
\z 
\z

A similar example, from the opening of \textit{Finnboga saga ramma}, is shown in (\ref{finnbogi}).

\ea \label{finnbogi} \ili{Old Icelandic}
 \ea[]{ 
 \gll  Ásbjörn hét maður \\
    Ásbjörn.\textsc{nom} was.called man.\textsc{nom}\\
 \glt  `There was a man called Ásbjörn.' 
}
\ex[]{ 
\gll \textbf{Hann} var kallaður dettiás \\
    he.\textsc{nom} was called Dettiás\\
\glt  `He was called Dettiás.'
}
\ex[]{ 
\gll \textbf{Hann} var Gunnbjarnarson Ingjaldssonar \\
   he.\textsc{nom} was Gunnbjörnson.\textsc{nom} Ingjaldurson.\textsc{gen}\\
\glt   `He was the son of Gunnbjörn, son of Ingjaldur.'
}
\ex[]{ 
\gll Mikill maður var \textbf{hann} og sterkur og vænn að áliti \\
   great.\textsc{nom} man.\textsc{nom} was he.\textsc{nom} and strong.\textsc{nom} and handsome.\textsc{nom} to appearance.\textsc{dat}\\
\glt  `He was a great man, strong, and handsome in appearance.'
}
\ex[]{ 
\gll \textbf{Hann} bjó í Flateyjardal á bæ þeim er heitir á Eyri \\
    he.\textsc{nom} lived in Flateyjardalur on farmstead.\textsc{dat} \textsc{dem.dat} \textsc{rel} is.called á Eyri\\
\glt  `He lived in Flateyjardalur on the farmstead which is called á Eyri.'\\
(IcePaHC: 1350, Finnbogi.625.1--5)
} 
\z 
\z


All of the sentences in (\ref{ex-cont-top}) and (\ref{finnbogi}) exhibit verb-second (V2) \isi{word order} but a particular type of verb-first (V1) order has also been claimed to signal \isi{topic continuity}, namely the ``\isi{narrative inversion}'' construction \citep{Platzack1985}, where a clause-initial finite verb is followed by a topical subject, typically realised as a personal \isi{pronoun}, e.g.~(\ref{ni-v1}) (\citealp{kossuth1980-narrative-syntax}; \citealp{booth-beck2021jhs}).

\ea[]{ \label{ni-v1} \ili{Old Icelandic}  \\
 \gll Sat \textbf{hún} hjá fótum hans\\
 sat she.\textsc{nom} by feet.\textsc{dat} he.\textsc{gen}\\
\glt `She sat by his feet.' (IcePaHC: 1150, Homiliubok.1875) 
}
\z


\begin{sloppypar}
The construction is particularly common in narrative texts, especially the sagas \citep{Platzack1985} and cannot initiate a new discourse \citep{sigurdhsson2018icelandic}, instead typically appearing in the reporting of sequenced temporal events with no change in participants \citealp{Platzack1985}; \citealp{hopper1987}; \citealp{kossuth1980-narrative-syntax}). 
\citet{booth-beck2021jhs} discuss the construction at length as an exception to V2, on the basis of \isi{corpus} data from IcePaHC, and claim that the construction signals a clause with a subject which is an ``anaphoric topic'', i.e.~a topic with a direct antecedent in the immediately preceding context in the same narrative section. They provide the example in (\ref{jhs-v1}), which represents a series of temporally sequenced clauses, and where V2 coincides with topic-shift and \isi{narrative inversion} V1 with \isi{topic continuity}.
\end{sloppypar}
\pagebreak
\ea \label{jhs-v1} \ili{Old Icelandic}\\
\ea[]{
\gll \textbf{Gissur} kom í Reykjaholt um nóttina eftir Máritíusmessu \\
Gissur.\textsc{nom} came to Reykjaholt in night.\textsc{acc.def} after Máritíusmass.\textsc{acc}\\
\glt `Gissur came to Reykjaholt in the night after Máritíusmass.' 
}
\ex[]{
\gll Brutu \textbf{þeir} upp skemmuna er Snorri svaf í \\
broke they.\textsc{nom} up storehouse.\textsc{acc.def} \textsc{rel} Snorri.\textsc{nom} slept in\\
\glt `They (=Gissur and his men) broke open the storehouse where Snorri was sleeping.'
}
\ex[]{
\gll En \textbf{hann} hljóp upp og úr skemmunni og í hin litlu húsin er voru við skemmuna \\
but he.\textsc{nom} leapt up and out storehouse.\textsc{dat.def} and in \textsc{dem.acc} little.\textsc{acc} houses.\textsc{acc.def} \textsc{rel} were by storehouse.\textsc{acc.def}\\
\glt `But he (=Snorri) leapt up and out of the storehouse and into those little houses which were next to the storehouse.'
}
\ex[]{\gll Fann \textbf{hann} þar Arnbjörn prest og talaði við hann\\
found he.\textsc{nom} there Arnbjörn.\textsc{acc} priest.\textsc{acc} and spoke with he.\textsc{acc}\\
\glt `He (=Snorri) found there Arnbjörn the priest and spoke with him.' 
}
\ex[]{\gll Réðu \textbf{þeir} það að Snorri gekk í kjallarann er var undir loftinu þar í húsunum \\
planned they.\textsc{nom} \textsc{dem.acc} \textsc{comp} Snorri.\textsc{nom} went in cellar.\textsc{acc.def} \textsc{rel} was under loft.\textsc{dat.def} there in houses.\textsc{dat.def}\\
\glt `They (=Arnbjörn and Snorri) planned that Snorri would go into the cellar which was under the loft there in the houses.'
}
\ex[]{\gll \textbf{Þeir} \textbf{Gissur} fóru að leita Snorra um húsin\\
they.\textsc{nom} Gissur.\textsc{nom} began to seek Snorri.\textsc{gen} around house.\textsc{acc.def}\\
\glt `Gissur and his men began to search for Snorri around the house.'
 (IcePaHC: 1250, Sturlunga.439.1765--1772,  \citealp[21]{booth-beck2021jhs})
 }
\z
\z

Of the 83 examples of the (singular) plain PA in the (M)IcePaHC data (see Section \ref{sect:method}), there is only one instance where the PA occurs on the subject of a \isi{narrative inversion} V1 clause. On the standard assumption that the PA is a \isi{familiarity}/\isi{givenness} marker, this is unexpected, since \isi{narrative inversion} V1 by definition involves a topical subject which is discourse-given. Rather, it suggests that the function of the (plain) PA in \ili{Old Icelandic} should be more closely examined. The one example where the (plain) PA coincides with the subject of a \isi{narrative inversion} clause is shown in (\ref{pa-v1}), together with the relevant preceding context. 

\ea \label{pa-v1} \ili{Old Icelandic}
\ea[]{\gll Þuríður gengur þá innar og leggur sitt stykki fyrir \textit{hvern} \textit{þeirra} \textit{bræðra}\\
%og var þar þá yxinsbógurinn og brytjaður í þrennt.\\
Þuríður.\textsc{nom} goes then in and places her.\textsc{refl.acc} piece.\textsc{acc} before each.\textsc{acc} \textsc{dem.gen} brothers.\textsc{dem}\\ %and was there then \textcolor{red}{X} and \textcolor{red}{X} in threefold\\
\glt `Þuríður goes in then and places her piece before each of the brothers.'
} 
\ex[]{\gll Tekur [\textbf{hann} \textbf{Steingrímur}] til orða og mælti:\\
takes \textsc{pa.m.nom.3sg} Steingrímur.\textsc{nom} to word and said\\
\glt `Steingrímur takes up the word and said:'\\
(MIcePaHC: 1300, Heidarviga.1450--1454)
} 
\z 
\z 

Sentence B in (\ref{pa-v1}) involves a \isi{topic shift} from Þuríður (=topic of sentence A) to Steingrímur and is thus an atypical use of \isi{narrative inversion}, which typically marks \isi{topic continuity}. Note, however, that Steingrímur is referenced in sentence A as one of the brothers, i.e.~that sentence A involves \isi{topic shift} via subsectional topic selection (cf.~(\ref{topic-types}) above). In the next section, I show that marking this type of topic-shift is overall a common function of the plain PA in \ili{Old Icelandic}. 

%%%%%%%
\subsection{The plain proprial article, subjecthood and topic-shift}

As in modern \ili{Icelandic} (Section \ref{sect:bg}), the plain PA in \ili{Old Icelandic}  can occur on a range of grammatical functions, as evidenced by the (M)IcePaHC data which provide examples on subjects, possessors, objects and prepositional complements, cf.~Table \ref{tab:plain-gf}. For each grammatical function, I compare the number of PA-marked personal names against the number of simple personal names which occur without the PA. This reveals that the presence of the PA is in fact incredibly rare across all functions, cf.~Table \ref{tab:plain-gf}.
In this section, I focus on the plain PA as it occurs on subjects, which is the most common in the dataset (65/84 examples). 

\begin{table}
\caption{Frequency of the plain PA across grammatical functions in (M)IcePaHC (1150–1450)}
\label{tab:plain-gf}
 \begin{tabular}{l rrr}
  \lsptoprule
       Grammatical function  &  PA &  no PA & \%PA\\
  \midrule
      Subject & 65 & 28\,391 & 0.23\\
      Possessor & 13 & 3\,961 & 0.33\\
      Object & 3 & 3\,048 & 0.10\\
      Prepositional complement\footnote{I do not make this comparison for the PA on prepositional complements as unlike proper nouns (\texttt{NPR-*}) which occur as subjects, possessors and objects, which are virtually all personal names, proper nouns which occur as prepositional complements are very often place names, which cannot be disambiguated from personal names in the \isi{corpus} annotation.} & 3 & -- & -- \\
    \midrule 
         Total & 84 &  \\
  \lspbottomrule
 \end{tabular}
\end{table}


As already shown in Section \ref{subsect:cont-ni}, the (plain) PA does not occur in prototypical \isi{givenness} contexts, i.e.~those which involve \isi{topic continuity}. On first sight, this observation appears to cast doubt on the standard assumption that it functions as a \isi{familiarity}/\isi{givenness} marker. At the same time, the \isi{familiarity}/givenness association with the PA is not in fact inaccurate; the (M)IcePaHC data for the plain PA, once tagged as described in Section \ref{sect:method}, confirm that the PA-marked referent in \ili{Old Icelandic} is always discourse-given. Specifically, in all 84 instances of the plain PA in (M)IcePaHC, the PA marks a referent which is referred to in the previous discourse. However, as I will show in this section, the (M)IcePaHC data indicate that the (plain) PA is more than just a straightforward \isi{familiarity}/\isi{givenness} marker. In particular, it occurs on the subject in contexts involving a particular type of topic-shift, where a discourse-given referent is promoted to, resumed, or subsectionally selected as the topic (cf.~the topic transitions in (\ref{topic-types}) above). Crucially, such an account relies on a more complex understanding of the interaction between  morphosyntax and \isi{information structure}, beyond a simple given/new distinction.
 

Of the topic-shift contexts in which the plain PA appears, one can distinguish three sub-contexts that involve particular types of topic transition as outlined above in (\ref{topic-types}): (i) \textsc{topic promotion}, (ii) \textsc{topic resumption} and (iii) \textsc{subsectional topic selection}. Firstly, the plain PA can mark instances of topic-shift involving topic promotion, i.e.~where a referent construed as non-topical in the previous context (e.g.~as a focused element) is ``promoted'' to topic, e.g.~(\ref{topic-prom}).

\ea \label{topic-prom} \ili{Old Icelandic}
 \ea[]{  \gll  Og er Túta kemur fyrir \textit{Halla} þá réttir [\textbf{hann} Halli] hendur {í móti} grísinum\dots\\
    and when Túta.\textsc{nom} came before Halli.\textsc{acc} then outstretched \textsc{pa.m.nom.3sg} Halli.\textsc{nom} hands.\textsc{acc} towards pig.\textsc{dat.def}\\
 \glt  `And when Túta came before Halli, then Halli stretched out his hands towards the pig.' (IcePaHC: 1275, Morkin.1156)
}
\ex[]{ \gll Svo er sagt að þeir kæmu að máli við \textit{Þórodd} \textit{goða} \textit{Eyvindarson} frænda sinn synir Þóris flatnefs. Hét annar þeirra Þórður illugi en annar Björn.
Þeir báðu hann ráðagerðar til að drepa Skútu Áskelsson {því að} hann hafði drepið föður þeirra og bróður. [\textbf{Hann} {Þóroddur}] vill nú þreifa um þá\\
so is said \textsc{comp} they.\textsc{nom} came to talk.\textsc{dat} with Þóroddur.\textsc{acc} chief.\textsc{acc} Eyvindarson.\textsc{acc} kinsman.\textsc{acc} their.\textsc{acc} sons.\textsc{nom} Þórir.\textsc{gen} flat-nose.\textsc{gen} was.called other.\textsc{nom} they.\textsc{gen} Þórður illugi and other Björn they.\textsc{acc} asked he.\textsc{acc} plan.\textsc{gen} to to kill Skúta Áskelsson because he.\textsc{nom} had killed father.\textsc{acc} they.\textsc{gen} and brother.\textsc{acc} \textsc{pa.m.nom.3g} Þóroddur.\textsc{nom}  will now consider about they.\textsc{acc}\\
\glt `So it is said that they, the sons of Þórir Flat-nose, came to speak with Chief Þóroddur Eyvindarson, their kinsman. One of them was called Þórður illugi and the other Björn. They asked him for a plan to kill Skúta Áskelsson  because he had killed their father and brother. Þóroddur now wishes to consider them.'\\
(MIcePaHC: 1400, Reykdæla.2035--2038)
} 
\z 
\z


Secondly, the plain PA signals topic-shift in contexts where a referent who was a previous topic, but was not the topic in the immediately preceding context, can be re-established or resumed as the topic (``topic resumption'', cf.~\citealp{gast_contrastive_2010}). A very common context here is extended dialogues which alternate back and forth between at least two speakers. An example is shown in (\ref{topic-dia}), which is a continuous piece of discourse where the conversation alternates between Ófeigur and Gellir, and where the PA is used to signal turn-taking.

\ea \label{topic-dia}  \ili{Old Icelandic}   \\
    \ea[]{\gll ``Hví sætir það?" segir Ófeigur \\
    why amounted \textsc{dem.nom} says  Ófeigur.\textsc{nom}\\
    \glt `{``}How did that come about?", says  Ófeigur.'
    } 
    \ex[]{\gll``Því," kvað [\textbf{hann} Gellir], ``að eigi hafa þeir menn til orðið er bæði séu vel ættaðir og fémiklir og hefðu staðfestur góðar''\\ 
    because said \textsc{pa.m.nom.3sg} Gellir.\textsc{nom} \textsc{comp} \textsc{neg} have \textsc{dem.nom} men.\textsc{nom} \textsc{ptcl} become \textsc{rel} both would.be well born and moneyed and would.have residences good\\ 
    \glt ` ``Because", said Gellir ``these men have not come forth, who were both well-born and (well-)moneyed and who have good residences." '
    }
    \ex[]{\gll ``Já," kvað [\textbf{hann} Ófeigur], ``þar er gott mannval''\\
    yes said \textsc{pa.m.nom.3sg} Ófeigur.\textsc{nom} there is good.\textsc{nom} choice.people.\textsc{nom}\\
    \glt ` ``Yes", said Ófeigur, ``there is a good choice of people there." '\\
   (IcePaHC: 1450, Bandamenn.39.717-721) 
   } 
\z
\z 

Thirdly, the plain PA occurs in another type of environment involving topic-shift, specifically where a discourse-old referent, which was previously explicitly mentioned/understood as the member of a set of referents, is picked out from the set as a new topic (``subsectional topic'', cf.~\citealp{vandeemter1992towards}; \citealp{dekker1996links}; \citealp{krahmer1998anaphoric}), e.g.~(\ref{topic-subset}).

\ea \label{topic-subset} \ili{Old Icelandic}
    \ea{}{\label{topic-subset-a} \gll og þar koma til móts við þá \textit{Egill} \textit{og} \textit{Gellir} [...] Einn dag um þingið er á leið gengur Ófeigur frá búð og kemur til Mýramannabúðar og var [\textbf{hann} Egill] úti í virkinu og talar við mann einn\\
     and there come to meeting.\textsc{gen} with they.\textsc{acc} Egill.\textsc{nom} and Gellir.\textsc{nom} {} one.\textsc{acc} day.\textsc{acc} at assembly.\textsc{acc.def} when on way.\textsc{acc} goes Ófeigur.\textsc{nom} from booth and comes to Mýramenn's.booth.\textsc{gen} and was \textsc{pa.m.nom.3sg} Egill.\textsc{nom} out in work.\textsc{def} and speaks with man.\textsc{acc} one.\textsc{acc}\\
    \glt `and Egill and Gellir come there to meet with them [...] One day at the assembly, when it is underway, Ófeigur leaves the booth and comes to the booth of the Mýramenn and Egill was out working and he speaks with a certain man.'
    } (IcePaHC: 1450, Bandamenn.36.599)
    \ex {\gll En þau voru í akri \textit{Vigdís} \textit{og} \textit{Sigmundur}.
Og er [\textbf{hún} Vigdís] sá hann gekk hún {í mót} honum\\
and they.\textsc{nom} were in field.\textsc{dat} Vigdís.\textsc{nom} and Sigmundur.\textsc{nom} and when \textsc{pa.f.nom.3sg} Vigdís.\textsc{nom} saw he.\textsc{acc} went she.\textsc{nom} towards he.\textsc{dat}\\
\glt `And they were in the field, Vigdís and Sigmundur. And when Vigdís saw him, she went towards him.' (MIcePaHC: 1350, Viga.505)
}
    \ex[]{ \gll Þá mælti Glúmur við Ingólf: [``\dots"] Og nú gengu \textit{þeir} \textit{báðir} saman og nú víkur [\textbf{hann} Glúmur] í hlöðu\\
    then spoke Glúmur.\textsc{nom} with Ingólfur.\textsc{acc} {} and now go they.\textsc{nom} both.\textsc{nom} together and now turns \textsc{pa.m.nom.3sg} Glúmur.\textsc{nom} into barn.\textsc{acc}\\
   \glt  `Then Glúmur spoke with Ingólfur: [``\dots"] and now they both go together and now Glúmur turns into the barn.'\\
   (MIcePaHC: 1350, Viga.887)
    } 
\z 
\z 

\largerpage
In sum, the plain PA -- at least on subjects -- signals a specific type of \isi{topic shift} involving the (re-)establishment of a discourse-given referent as topic. As such, the standard assumption that the (plain) PA signals \isi{givenness} is not incorrect, but it is only part of the story. A final observation which is relevant in this context is that the order of the PA and the PA-marked referent in the (M)IcePaHC data is fixed; the PA is always \isi{prenominal}. This fixed ordering is striking, given that \isi{word order} in the nominal domain is known to be relatively free in early North \ili{Germanic} (e.g.~\citealp{Borjarsetal16}), where e.g.~demonstratives, adjectives and possessors can occur before or after the head \isi{noun}. However, as \citet{Borjarsetal16} point out, \isi{word order} in the Old \ili{Norse}/Icelandic \isi{noun} phrase is not completely free; there is a structurally defined, discourse-prominent position at the left edge which
hosts information-structurally privileged elements. On the assumption that the (plain) PA serves a special information-structural function in marking topic-shift, its restriction to this information-structurally privileged position is thus expected. 


%%%%%%%%%%%%%%%%%%%%%%%%%%%%%%%%%%%
\section{The inclusory proprial article}\label{sect:gapped}

 %%%%%%%
\subsection{Associativity, givenness and topicality}

As discussed in Section \ref{sect:bg}, previous accounts of the pragmatics of the inclusory PA in modern Icelandic have been restricted to the given/new dimension, with the claim that, like the plain PA, the gapped PA marks \isi{familiarity}/\isi{givenness} \citep{Sigurdsson2006}. At the same time, the gapped PA in \ili{Old Icelandic} has been discussed, often in passing, as an ``\isi{associative plural}'' construction (\citealp{denbesten1996associative}; \citealp{moravcsik2003semantic}; \citealp{wals-36}; \citealp{sigurdhsson2020we}), although, as mentioned in Section \ref{sect:hist-context}, \citet{heusler1962altislandisches} points out that this is only one function.
%although \citet{sigurdhsson2020we} claim that use of the gapped PA as an \isi{associative plural} is no longer possible in modern Icelandic. 
As Heusler states, the PA can also express two individuals, one of whom is already in the ``consciousness'' of the speaker, i.e.~in the common ground \citep{stalnaker2002common}, and thus not explicitly named, and one who is explicitly named and ``added'' as a second person (cf.~footnote \ref{heusler-fn} above). To my knowledge, the precise properties of the inclusory PA in \ili{Old Icelandic} have not been examined since the early descriptive work by \citet{heusler1962altislandisches}. In this section, I examine to what extent the two different functions of the inclusory PA are exhibited in the (M)IcePaHC data. I focus specifically on examples in the third person, which in principle allow for both functions.
%In this section, I examine the ``Gapped'' dataset from Table \ref{tab:all} which consists of 169 examples from IcePaHC. \textcolor{blue}{Possibly look at the mixed ones too later... I show that only some instances qualify as an \isi{associative plural} in the standard sense; others serve to mark a topic which consists of a set featuring a continuing topic and a shift-topic.}
%Due to the high level of qualitative analysis required for the examination of information-structural properties which are not annotated in the \isi{corpus} data, I do not consider examples from the ``Mixed'' dataset shown in Table \ref{tab:all} and leave that for future research.


With respect to associative plurals, they are typically defined both in terms of form and meaning. \citet[1]{corbett1996associative}, for instance, define them as consisting of a nominal plus some sort of marker, which denote a set comprised of the referent of the nominal and one or more associated members (for similar definitions cf.~\citealp{moravcsik2003semantic}; \citealp{lewis2022associative}). %\citet{wals-36}, meanwhile, identify associative plurals in terms of two semantic properties: (i) they denote a referentially heterogeneous set (unlike additive plurals which denote a referentially homogenous set) and (ii) they refer to a close-knit group of individuals, rather than to sets without any internal cohesion. 
In terms of pragmatics, the set denoted by an \isi{associative plural} is ranked, with the referent around which the associate(s) is/are centred being ``focal'' \citep{moravcsik2003semantic} or ``pragmatically dominant'' \citep{wals-36}. Although such constructions generally have a restricted distribution within individual languages, typologically they are relatively common; \citet{wals-36}, for instance, found \isi{associative plural} constructions to be present in 201/238 sample languages. They are particularly common throughout Australia, Asia and Africa, although rare in Western Europe, found only in Icelandic, \ili{Norwegian}, Frisian, \ili{German}, Northern Saami and Basque.\footnote{Note that the associative plurals for \ili{Norwegian}, Frisian and \ili{German} are rather different to the \ili{Icelandic} construction discussed here:

\ea 
\ea \ili{Norwegian}\\
\gll moren og di\\
mother and they\\
\glt `mother and the rest of the family' \citep[Sentence igt-1209]{wals-36}
\ex Frisian\\
\gll heit en hjar\\
father and them\\
\glt `father and them' \citep[Sentence igt-3403]{wals-36}
\newpage
\ex \ili{German}\\
\gll Anna und die\\
Anna and \textsc{pl.def.art}\\
\glt `Anna and her group' \citep[Sentence igt-3235]{wals-36}
\z 
\z

}

\newpage
In order to investigate to what extent the inclusory PA in \ili{Old Icelandic} functions as an \isi{associative plural} on the terms just outlined, I conducted a manual investigation of two texts which provide particularly abundant examples of the construction and for which reliable published \ili{English} translations are available: (i) \textit{Grettir} \citep{faulkes2001three} and (ii) \textit{Jomsvikingar} \citep{finlay2018jomsviking}.  As with the investigation of the plain PA in Section \ref{sect:plain}, I focus here on the inclusory PA as it occurs on the subject, which constitutes the vast majority of the examples in the two texts (\textit{n}=19). 11 of the 19 examples are translated with an \isi{associative plural} meaning (`X and his associates'), where the PA-marked expression refers to a group of unidentifiable human individuals centred around the PA-marked referent (`X'), e.g.~(\ref{old-assoc}). 
In each instance the PA-marked referent is discourse-given, but is not present in the immediately preceding context. %\textcolor{blue}{What does the literature on assoc plurals say about other features which we can test here...?}
Since many of the examples involve long passages of text, I do not provide glossing but simply the accompanying published translations from  \citet{faulkes2001three} and \citet{finlay2018jomsviking}.

\begin{exe}
    \ex \label{old-assoc} \ili{Old Icelandic} % (exe environment!!)
    \begin{xlist}
        \ex Þorgils frétti að [\textbf{þeir} Þorsteinn] fjölmenntu mjög til alþingis og sátu í Ljárskógum. Því frestaði hann heiman að ríða að hann vildi að [\textbf{þeir} Þorsteinn] væru undan suður riðnir þá er hann kæmi vestan og svo varð. \\
        `Thorgils heard that \textbf{Thorstein's party} was assembling a great following for the Althing and was waiting in Liarskogar. He delayed his own departure because he wanted \textbf{Thorstein and his party} to have ridden away south by the time he came from the west, and so it turned out.' (IcePaHC: 1310, Grettir.1381--1383) 
        
        \ex Þau Rannveig og Gamli tóku allvel við Gretti og buðu honum með sér að vera en hann vildi heim ríða. Þá frétti Grettir að [\textbf{þeir} Kormákur] voru sunnan komnir og höfðu gist í Tungu um nóttina.\\ 
        `Rannveig and Gamli welcomed Grettir warmly and invited him to stay on with them, but he wanted to ride home. Then Grettir found out that \textbf{Kormak's party} had come back from the south and had lodged at Tunga for the night.' (IcePaHC: 1310, Grettir.1635--1638) 
        
        \ex Og nú er það sagt, að Haraldur konungur gráfeldur fellur þar í bardaganum og mestur hluti liðs hans, og lauk svo um hans æfi. [5]  
        Og þegar er Hákon jarl veit þessi tíðendi, þá gerir hann atróður harðan, þá er [\textbf{þeir} Gull-Haraldur] voru sízt viður búnir.\\
        `And now it is said that King Haraldr gráfeldr fell there in battle with the greater part of his company, and thus his life ended. And as soon as Jarl Hákon learned this news, he makes a hard rowing attack when \textbf{Gull-Haraldr and his men} were least prepared for it.' \\
        (IcePaHC: 1260, Jomsvikingar.490--492)
        
        \ex En um daginn eftir, þá berjast þeir allan dag til nætur, og þá eru hroðin tíu skip Haralds konungs, en tólf af Sveini, og lifir enn hvortveggi þeirra, og leggur Sveinn nú skip sín inn í vogsbotninn um kveldið. En [\textbf{þeir} Haraldur konungur] tengja saman skip sín um þveran voginn fyrir utan og leggja stafn við stafn, og búa svo umb, að Sveinn væri inni tepptur í voginum, og ætla að hann skyldi eigi út koma skipunum, þótt hann vildi við það leita.\\ 
        `But the following day they fight all day until night, and then ten of King Haraldr's ships are stripped, and twelve of Sveinn's, and both of them are still alive, and now Sveinn berths his ships in at the head of the bay in the evening. But \textbf{King Haraldr and his men} link their ships together across the outside of the bay, setting stem to stem and arranging things so that Sveinn would be trapped in the bay, and intended that he would not be able to get his ships out if he wanted to try it.' (IcePaHC: 1260, Jomsvikingar.1283--1290)

    \end{xlist}
\end{exe}


%\ea \label{old-assoc} %\textbf{Gapped PA translated with an \isi{associative plural} meaning}
%\ili{Old Icelandic}
%\ea[]{
%Þorgils frétti að [\textbf{þeir} Þorsteinn] fjölmenntu mjög til alþingis og sátu í Ljárskógum. Því frestaði hann heiman að ríða að hann vildi að [\textbf{þeir} Þorsteinn] væru undan suður riðnir þá er hann kæmi vestan og svo varð\\
%`Thorgils heard that \textbf{Thorstein's party} was assembling a great following for the Althing and was waiting in Liarskogar. He delayed his own departure because he wanted \textbf{Thorstein and his party} to have ridden away south by the time he came from the west, and so it turned out.' (IcePaHC: 1310, Grettir.1381--1383)
%}

%\ex[]{
%Þau Rannveig og Gamli tóku allvel við Gretti og buðu honum með sér að vera en hann vildi heim ríða. Þá frétti Grettir að [\textbf{þeir} Kormákur] voru sunnan komnir og höfðu gist í Tungu um nóttina.\\
%`Rannveig and Gamli welcomed Grettir warmly and invited him to stay on with them, but he wanted to ride home. Then Grettir found out that \textbf{Kormak's party} had come back from the south and had lodged at Tunga for the night.' (IcePaHC: 1310, Grettir.1635--1638)
%}
%ch30 mid
% \b. Grettir bjóst snemma frá Melum. Gamli bauð honum menn til fylgdar. Grímur hét bróðir Gamla. Hann var allra manna hvatastur. Hann reið með Gretti við annan mann. Þeir voru fimm saman, riðu uns þar til er þeir komu á Hrútafjarðarháls vestur frá Búrfelli. Þar stendur steinn mikill er kallaður er Grettishaf. Hann fékkst við lengi um daginn að hefja steininn og dvaldi svo þar til er [\textbf{þeir} Kormákur] komu.\\
% Grettir set out early from Melar. Gamli offered him men to go with him. Gamli’s brother was called Grim. He was the boldest of men. He rode with Grettir as well as one other person. They were five in all; they rode until they got to Hrutafiord ridge, west of Burfell. A great stone stands there which is called Grettir’s Lift. He strove for a long time during the day to lift the stone and thus stayed until \textbf{Kormak’s party} came up 1310.GRETTIR.NAR-SAG,.1648 %ch 30 mid
% \b. Grettir var að Bjargi er hann frétti að Barði var suður riðinn. Hann brást við reiður er honum voru engin orð ger og kvað þá eigi svo búið skilja skyldu. Hafði hann þá spurn af nær þeirra væri sunnan von og reið hann þá ofan til Þóreyjargnúps og ætlaði að sitja þar fyrir [\textbf{þeim} Barða] þá þeir riðu sunnan. Hann fór frá bænum í hlíðina og beið þar. Þenna sama dag riðu [\textbf{þeir} Barði] sunnan af Tvídægru frá Heiðarvígum. Þeir voru sex saman og allir sárir mjög.\\
% Grettir was at Biarg when he found out that Bardi had ridden south. He was angry at this, that no word had been sent him, and said that was not going to be the end of the matter. So he found out when they were expected to be back from the south, and he then rode down to Thoreyiargnup and prepared to lie in wait for \textbf{Bardi’s party} as they rode from the south. He went from the farm onto the hillside and waited there. That same day \textbf{Bardi’s party} rode from the south off Two-days moor from the Battle of the Heath 1310.GRETTIR.NAR-SAG,.1714
% %ch31 mid
% \b. Þeir segja það allráðlegt og svo gerðu þeir. Síðan riðu [\textbf{þeir} Barði] veg sinn. \\
% They said that was very sensible, and that is what they did. Then \textbf{Bardi and his men} rode on their way. 1310.GRETTIR.NAR-SAG,.1726
% %ch31 mid
% \b. Nú leggur Haraldur að þeim öllum megin, og lýstur þegar í bardaga með þeim bræðrum. En það var jafnt jólaaftan sjálfan er þeir börðust. En svo lýkur bardaganum að Knútur fellur þar og allt lið hans, eða nær því, þvíað Haraldur neytti þess er hann hatði lið miklu meira. Eftir þessi tíðendi fara [\textbf{þeir} Haraldur] þar til er þeir komu í stöðvar Gorms konungs síð um aftan\\
% Now Haraldr advances against them with all his might, and battle breaks out at once between the brothers. And it was the eve of Yule itself when they fought. But the battle ends so that Knútr falls there with all his company, or almost, for Haraldr made the most of the fact that he had a much larger force. After this event \textbf{Haraldr and his company} travel until they came to King Gormr’s harbor late in the evening, and they went fully armed to the king’s estate.
% 1260.JOMSVIKINGAR.NAR-SAG,.342 %ch3
%\ex[]{Og nú er það sagt, að Haraldur konungur gráfeldur fellur þar í bardaganum og mestur hluti liðs hans, og lauk svo um hans æfi. [5] 
%   Og þegar er Hákon jarl veit þessi tíðendi, þá gerir hann atróður harðan, þá er [\textbf{þeir} Gull-Haraldur] voru sízt viður búnir.\\
%   `And now it is said that King Haraldr gráfeldr fell there in battle with the greater part of his company, and thus his life ended. And as soon as Jarl Hákon learned this news, he makes a hard rowing attack when \textbf{Gull-Haraldr and his men} were least prepared for it.' (IcePaHC: 1260, Jomsvikingar.490--492)
%   }
 %1260.JOMSVIKINGAR.NAR-SAG,.492  %ch5 mid
% \b. Ótta keisari var á hesti um daginn, og er nú sagt að þeir sækja ofan að skipunum, og keisarinn ríður fram að sjónum og hefir í hendi spjót eitt mikið, gullrekið og alblóðugt; og síðan stingur hann spjótinu í sæinn og nefnir síðan guð almáttkan í vitni og mælti síðan: "Í annað sinni þá er eg kem til Danmerkur, þá skal vera annað hvort að eg skal kristnað fá Danmörk eða ella láta hér lífið." 
%   Eftir þetta ganga [\textbf{þeir} Ótta] keisari á skip sín og fer hann nú heim til Saxlands.\\
%   Emperor Ótta was on horseback during the day, and it is now said that they advance down to the ships, and the emperor rides forward to the sea and has a large spear in his hand, gold-inlaid and covered with blood; and then he stabs the spear into the sea and then calls on God Almighty in witness and then declared: “Next time I come to Denmark it shall come about either that I succeed in making Denmark Christian, or else lose my life here.”
% After that \textbf{Emperor Ótta and his men} go on board their ships and he now sails home to Saxland.
% 1260.JOMSVIKINGAR.NAR-SAG,.591 %ch6 mid
% \b. Og er nú svo frá sagt, þá er Áki kömur við Sjóland í Danmörku, að [\textbf{þeir} Áki] hafa tjöld á landi og ugga þá ekki að sér.\\
% And it is now told that when Áki comes past Sjóland in Denmark, that \textbf{Áki and his men} had a tent pitched on land and then have no fear for themselves. 1260.JOMSVIKINGAR.NAR-SAG,.927 %ch 8 mid
%\ex[]{En um daginn eftir, þá berjast þeir allan dag til nætur, og þá eru hroðin tíu skip Haralds konungs, en tólf af Sveini, og lifir enn hvortveggi þeirra, og leggur Sveinn nú skip sín inn í vogsbotninn um kveldið. En [\textbf{þeir} Haraldur konungur] tengja saman skip sín um þveran voginn fyrir utan og leggja stafn við stafn, og búa svo umb, að Sveinn væri inni tepptur í voginum, og ætla að hann skyldi eigi út koma skipunum, þótt hann vildi við það leita.\\
%`But the following day they fight all day until night, and then ten of King Haraldr's ships are stripped, and twelve of Sveinn's, and both of them are still alive, and now Sveinn berths his ships in at the head of the bay in the evening. But \textbf{King Haraldr and his men} link their ships together across the outside of the bay, setting stem to stem and arranging things so that Sveinn would be trapped in the bay, and intended that he would not be able to get his ships out if he wanted to try it.' (IcePaHC: 1260, Jomsvikingar.1283--1290)
%}
%\z 
%\z 
%1260.JOMSVIKINGAR.NAR-SAG,.1287 %ch 11 mid

% Not all of the examples of the gapped PA in these two texts are like the examples in (\ref{old-assoc}) in exhibiting the traits associated with associative plurals; 

\largerpage
The remaining eight examples of the inclusory PA in these two texts are translated instead as `he and X' and as such do not appear to qualify as associative plurals on the understanding of the term here. Some examples from this group are provided in (\ref{old-non-assoc}).



\begin{exe}
    \ex \label{old-non-assoc} \ili{Old Icelandic} % (exe environment !)
    \begin{xlist}
        \ex Um vorið fór Grettir norður í Voga með byrðingsmönnum. Skildu [\textbf{þeir} Þorkell] með vináttu en Björn fór vestur til Englands.\\ 
        `In the spring Grettir went north to Vågan with merchants; \textbf{he and Thorkel} parted on friendly terms.'  
        (IcePaHC: 1310, Grettir.1040--1042)
       \begin{sloppypar}

        \ex Þá var til jarls kominn Bersi Skáld-Torfuson, félagi Grettis og vin. Gengu [\textbf{þeir} Þorfinnur] fyrir jarl\\ 
        `By this time Grettir's comrade and friend Bersi Poet-Torfa's son had arrived at the earl's. \textbf{He and Thorfinn} approached the earl.' \\ 
        (IcePaHC: 1310, Grettir.1147--1148)
        \end{sloppypar} 

        \ex Fór Grettir með Þorfinni. Skildust [\textbf{þeir} Þorsteinn bróðir hans] með vináttu. \\ 
        `Grettir went with Thorfinn. \textbf{He and his brother Thorstein} parted in friendship.' (IcePaHC: 1310, Grettir.1263--1264) 

        \ex Og nú tekur jarl upp þetta fé allt að herfangi og geldur Haraldi konungi af því fé þriggja vetra skatt fyrir fram, og kveðst eigi mundu í öðru sinni betur til fær en nú. Haraldur konungur tekur því vel, og skiljast [\textbf{þeir} Hákon] nú, og fer hann í braut úr Danmörku\\ 
        `And now the jarl takes all that money as booty and pays King Haraldr from that money three years’ tribute in advance, and said he would not another time have a better opportunity than now. King Haraldr accepts that gladly, and \textbf{he and Hákon} part now, and he goes away from Denmark until he comes to Norway.' 
        (IcePaHC: Jomsvikingar.507--511)

        \ex Þess er nú við getið að Pálnatóki á son við konu sinni Ólöfu, og er hann fæddur litlu síðar en konungur fór í braut af veizlunni; sá sveinn var kallaðut Áki. Hann var þar upp fæddur heima með feður sínum, og várust [\textbf{þeir} Sveinn Haraldsson] fóstbræður.\\ 
        `It is now told further that Pálnatóki has a son with his wife Ólǫf, and he is born shortly after the king went away from the feast; this boy was called Áki. He was brought up there at home with his father, and \textbf{he and Sveinn Haraldsson} were foster-brothers.' \\ 
        (IcePaHC: 1260, Jomsvikingar.1128--1133) 
    \end{xlist}
\end{exe}

%\ea \label{old-non-assoc} %\textbf{Gapped PA without an \isi{associative plural} meaning}
%\ili{Old Icelandic}
%\ea[]{ %top ch22 
%Um vorið fór Grettir norður í Voga með byrðingsmönnum. Skildu [\textbf{þeir} Þorkell] með vináttu en Björn fór vestur til Englands.\\
%`In the spring Grettir went north to Vågan with merchants; \textbf{he and Thorkel} parted on friendly terms.' 
%\\ (IcePaHC: 1310, Grettir.1040--1042)
% }
% \b. Sveinn jarl sat inn í Þrándheimi að Steinkerum þá er hann spurði víg Bjarnar. Þá var þar með honum Hjarrandi, Bjarnar bróðir. Hann var hirðmaður jarls. Hann varð við reiður mjög er hann spurði víg Bjarnar og beiddi jarl liðveislu til málsins. Jarl hét honum því. Sendi hann þá mann til Þorfinns og stefndi [\textbf{þeim} Gretti báðum] á sinn fund. Bjóst hann þegar og [þeir \textbf{Grettir} báðir] eftir boði jarls og fóru inn í Þrándheim á hans fund.\\
% Earl Svein was at Stenkjer in Trøndelag when he heard of Biorn’s killing. At that time Biorn’s brother Hiarrandi was there with him; he was one of the earl’s men. He was very angry when he heard of Biorn’s killing and he asked for the earl’s support in prosecuting the offender. The earl promised him this; then he sent someone to Thorfinn and summoned \textbf{both him and Grettir} to come before him. He set out immediately, and \textbf{Grettir with him}, in accordance with the earl’s sum- mons, and they went inland to Trøndelag to meet him. (IcePaHC: 1310, Grettir.1088--1096) %ch22 end
%\ex[]{Þá var til jarls kominn Bersi Skáld-Torfuson, félagi Grettis og vin. Gengu [\textbf{þeir} Þorfinnur] fyrir jarl\dots \\
%`By this time Grettir's comrade and friend Bersi Poet-Torfa's son had arrived at the earl's. \textbf{He and Thorfinn} approached the earl\dots'
% (IcePaHC: 1310, Grettir.1147--1148) 
% }
%ch23 end
%\ex[]{ %ch24 end, above verse
%Fór Grettir með Þorfinni. Skildust [\textbf{þeir} Þorsteinn bróðir hans] með vináttu. \\
%`Grettir went with Thorfinn. \textbf{He and his brother Thorstein} parted in friendship.' (IcePaHC: 1310, Grettir.1263--1264) 
%}
% \b. Nú er þar til að taka að Grettir Ásmundarson sat heima að Bjargi um haustið síðan [\textbf{þeir} Víga-Barði] skildu á Þóreyjargnúpi. Og er mjög var komið að veturnóttum reið Grettir heiman norður yfir hálsa til Víðidals og gisti á Auðunarstöðum. Sættust [\textbf{þeir} Auðunn] til fulls og gaf Grettir honum öxi góða og mæltu til vináttu með sér.\\
% Now the story is to be taken up, that Grettir was at home at Biarg in the autumn after his parting from Viga-Bardi at Thoreyiargnup. And when it came close to the Winter Nights Grettir rode out from home northwards over the ridges to Vididal and stayed at Audunarstadir. \textbf{He and Audun} were fully reconciled and Grettir gave him a fine axe and they confirmed their friendship together. 
% 1310.GRETTIR.NAR-SAG,.2000--2003
% %ch34 top
%  \b.    Og nú fer Hákon jarl á fund Haralds Gormssonar og selur honum einsköpun fyrir þetta, er hann hafði drepið Gull-Harald frænda hans, en þó var þetta hégómi einn, fyrir því að þetta var beggja þeirra ráð reyndar. En Haraldur konungur gerir það á hendur Hákoni, að hann skal skyldur til að fara til Danmerkur um sinn og bjóða út leiðangri um allan Noreg til liðs við sig, þá er hann þykist liðs þarfi  vera, en fara jafnan sjálfur þá er hann sendir honum orð og hann vildi hans ráðuneyti hafa. Hann skyldi og gjalda skatta þá alla er fyrr var í frá sagt. 
%   Og áður en [\textbf{þeir} Hákon] skiljast, þá tekur hann gull það er átt hafði Gull-Haraldur\\
%   And now Jarl Hákon goes to see Haraldr Gormsson and offers him the right of judgment for the fact that he had killed his kinsman Gull-Haraldr, but that was only a trick, because that was really what they had both planned. But King Haraldr stipulates against Hákon that he must go to Denmark for the present and call out the levy throughout Norway to support Haraldr when he thought he had need of support, and always come himself when he sends him word and wanted to have his counsel. He must also pay all the tribute that was mentioned before.
% And before \textbf{he and Hákon} part, he takes all the gold that Gull-Haraldr had owned
% 1260.JOMSVIKINGAR.NAR-SAG,.503 %ch5 mid
%\ex[]{
% Og nú tekur jarl upp þetta fé allt að herfangi og geldur Haraldi konungi af því fé þriggja vetra skatt fyrir fram, og kveðst eigi mundu í öðru sinni betur til fær en nú. Haraldur konungur tekur því vel, og skiljast [\textbf{þeir} Hákon] nú, og fer hann í braut úr Danmörku\dots\\
% `And now the jarl takes all that money as booty and pays King Haraldr from that money three years’ tribute in advance, and said he would not another time have a better opportunity than now. King Haraldr accepts that gladly, and \textbf{he and Hákon} part now, and he goes away from Denmark until he comes to Norway\dots'
%\\ (IcePaHC: Jomsvikingar.507--511)
% }
%1260.JOMSVIKINGAR.NAR-SAG,.511 %ch5 mid
%\ex[]{
%Þess er nú við getið að Pálnatóki á son við konu sinni Ólöfu, og er hann fæddur litlu síðar en konungur fór í braut af veizlunni; sá sveinn var kallaðut Áki. Hann var þar upp fæddur heima með feður sínum, og várust [\textbf{þeir} Sveinn Haraldsson] fóstbræður.\\
%`It is now told further that Pálnatóki has a son with his wife Ólǫf, and he is born shortly after the king went away from the feast; this boy was called Áki. He was brought up there at home with his father, and \textbf{he and Sveinn Haraldsson} were foster-brothers.'
% (IcePaHC: 1260, Jomsvikingar.1128--1133)
% }
%\z 
%\z 
%1260.JOMSVIKINGAR.NAR-SAG,.1133 %ch8 end

This second group, as exemplified in (\ref{old-non-assoc}), appears to qualify as the second function of \citet{heusler1962altislandisches}. The PA-marked expression denotes a set comprising two individuals, one of whom is already in the common ground (Heusler's ``consciousness'') and represents a continuing topic in the present utterance and is not explicitly named. Close inspection reveals that the second referent is consistently discourse-given, but never has the status of continuing topic; rather it is typically a newly promoted or resumed topic (cf.~(\ref{topic-types}) above). I discuss this issue of asymmetry in \isi{topicality} further in Section \ref{subsect:incl-coord}.

%%%%%%%
\subsection{Inclusory constructions and noun--pronoun coordination}\label{subsect:incl-coord}

Various authors have discussed inclusory constructions in the wider context of linguistic devices which indicate the involvement of two or more persons in a particular semantic role, including standard \isi{coordination} (e.g.~\citealp{bhat2004conjunction}; \citealp{gaby2005some}; \citealp{haspelmath2007coordination}; \citealp{bril2011noun}). 
\citet{bril2011noun}, for instance, in her discussion of conjoining strategies in Austronesian languages, observes that inclusory constructions often (though not always) occur in languages which ban (standard) noun--\isi{pronoun} conjunction. She further notes that, if inclusory constructions are available in a language which permits (standard) noun--\isi{pronoun} conjunction, the choice between standard \isi{coordination} and the \isi{inclusory construction} typically correlates with discourse effects, whereby standard \isi{coordination} expresses equal \isi{topicality}, salience, or \isi{emphasis} between conjuncts, and inclusory constructions involve pragmatic asymmetry between conjuncts.

Searches in (M)IcePaHC show that standard \isi{coordination} of a 3\textsc{sg} \isi{pronoun} and a personal name  (`he and X') is attested in \ili{Old Icelandic}, though very rare, and certainly much rarer than the inclusory PA. I have found only one continuous example, i.e.~where the coordinated \isi{pronoun} and personal name are directly adjacent, shown here in (\ref{coord-han}).  
The example occurs on a possessor and is from a late text (\textit{Ectorssaga}, 1450).

\ea \label{coord-han}
\ili{Old Icelandic} \\ 
\gll En gamli maður segir: [``\dots''] Gekk hann þá út og kom aftur skjótliga leiðandi eftir sér einn þræl stórran að ekki var {í milli} um vöxt \textbf{hans} \textbf{og} \textbf{Aprívals}\\
but old.\textsc{nom} man.\textsc{nom} says {} went he.\textsc{nom} then out and came back quickly leading after \textsc{refl.dat} one.\textsc{acc} slave.\textsc{acc} big.\textsc{acc} \textsc{comp} nothing was between about size he.\textsc{gen} and Apríval.\textsc{gen}\\
\glt `But the old man says [``\dots''] Then he went out and came back quickly, leading after him a big slave such that there was nothing between his and Apríval's size.' (IcePaHC: 1450, Ectorssaga.1515)
\z 


Besides the continuous example in (\ref{coord-han}), I have also found one example where the 3\textsc{sg} \isi{pronoun} and coordinated personal name are discontinuous (`he\dots and X'), shown here in (\ref{coord-discont}).

\ea \label{coord-discont}\ili{Old Icelandic}\\
\gll Var Þorleifur á húsum þeim er eru í útnorður frá kirkju. Hafði \textbf{hann} þar hanboga \textbf{og} \textbf{Jósteinn} \textbf{glenna} \textbf{austmaður} \textbf{hans}\\
was Þorleifur.\textsc{nom} at buildings.\textsc{dat} \textsc{dat.dem} \textsc{rel} are in northwest.\textsc{acc} from church.\textsc{dat} had he.\textsc{nom} there handbow.\textsc{acc} and Jósteinn.\textsc{nom} glenna.\textsc{nom} east.man.\textsc{nom} he.\textsc{gen}\\
\glt `Þorleifur was at those buildings which were north-west of the church. He and Jósteinn Glenna, his man from the east, had there a handbow.' (IcePaHC: 1250, Sturlunga.391.102) %first mention of Josteinne
\z


The difference between (\ref{coord-han}) and (\ref{coord-discont}) is that the named referent \textit{Apríval} in (\ref{coord-han}) is known from the previous discourse, whereas in (\ref{coord-discont}) \textit{Jósteinn} is a first mention, and occurs with other identifying material (`his man from the East'). Like inclusory constructions, discontinuous nominals crosslinguistically have been observed to often coincide with information-structurally asymmetric conjuncts, especially in languages where \isi{word order} is sensitive to \isi{information structure} (e.g.~\citealp{mcgregor1997functions}; \citealp{deKuthy2002discontinuous}; \citealp{fanselow2006prosodic}; \citealp{skopeteas2020discontinuous}). On the basis of the very limited data available for Icelandic, one can suggest that discontinuous \isi{coordination} of a \isi{pronoun} and a name is used when the \isi{pronoun} is a continuing topic, and the name is discourse-new. When the name is familiar, but not a continuing topic, i.e.~when the referents differ not in \isi{givenness} but in \isi{topicality}, the M(IcePaHC) data indicate that \ili{Old Icelandic} by far favours the inclusory PA construction compared to standard \isi{coordination}, which in such contexts appears to be very rare.\footnote{Relatedly, \citet[230]{Sigurdsson2006} states for modern \ili{Icelandic} that the inclusory PA \textit{við Jón} (`we John') is ``often or usually preferred'' to the standard pronoun-\isi{noun} \isi{coordination} structure \textit{ég og John} (`John and I'). }
Besides functioning as an \isi{associative plural}, the inclusory PA (at least on subjects) thus serves an additional function in expressing a topic which comprises a continuing topic and an additional discourse-given referent which is re-established as topical (shift-topic), in line with the general trend for inclusory constructions to involve conjuncts which differ in \isi{topicality}  \citep{bril2011noun}.

%%%%%%%%%%%%%%%%%%%%%%%%%%%%%%%%%%%
\section{Conclusion}\label{sect:conc}

\begin{sloppypar}
In this chapter, I have demonstrated that investigations of linguistic features at the morphosyntax--\isi{information structure} interface must go beyond the given/new dimension in order to a achieve a full understanding of such phenomena. By considering different types of \isi{aboutness topic} in terms of types of topic transition,
I have shown that the proprial \isi{article} in \ili{Old Icelandic} is more than a straightforward \isi{givenness} marker, as previously claimed by \citet{Sigurdsson2006} and \citet{Johnsen2016}.
Rather, the (M)IcePaHC \isi{corpus} data indicate that the proprial \isi{article} is often employed in \ili{Old Icelandic} saga narratives as a topic management device. 
The plain PA was shown to occur optionally as a topic-shift marker, employed specifically when a discourse-given referent is (re)established as a topic via topic promotion or resumption, or via subsectional topic selection. The (M)IcePaHC data also confirmed an early claim by \citet{heusler1962altislandisches} that the inclusory PA serves two different functions in \ili{Old Icelandic}: as (i) an \isi{associative plural} and (ii) a strategy for coordinating (at least) two human referents which are both discourse-given but differ in \isi{topicality} (continuing topic versus shift-topic). 
More broadly, the \ili{Old Icelandic} facts emphasise the different nature of the diachrony of the proprial \isi{article} in North \ili{Germanic} compared to Continental West \ili{Germanic} (e.g.~\citealp{Schmuck2014}; \citealp{Schmuck2020-hex,schmuck2020grammaticalisation,schmuck2020rise}), and in particular that, in the former, topic management rather than the grammaticalisation of \isi{definiteness} and loss of case is a key factor.
\end{sloppypar}

 
%%%%%%%%%%%%%%%%%%%%%%%%%%%%%%%%%%%
\section*{Abbreviations}
\begin{tabularx}{.5\textwidth}{@{}lQ@{}}
\textsc{acc} & accusative\\
\textsc{comm} & common\\
\textsc{comp} & complementiser\\
\textsc{dat} & {dative} \\
\textsc{def} & definite\\
\textsc{dem} & {demonstrative}\\
\textsc{du} & {dual}\\
\textsc{excl} & exclusive\\
\textsc{f} & feminine\\
\textsc{gen} & {genitive}\\
\textsc{indef} & {indefinite}\\
\textsc{inf} & infinitive\\
\textsc{m} & masculine\\
\textsc{mkr} & marker\\
\textsc{n} & neuter\\
\end{tabularx}%
\begin{tabularx}{.5\textwidth}{@{}lQ@{}}
\textsc{neg} & negation\\
\textsc{nom} & {nominative}\\
\textsc{nonfut} & non-future\\
\textsc{pa} & proprial {article} \\
\textsc{perf} & perfect\\
\textsc{pers} & person\\
\textsc{pl} & plural\\
\textsc{pst} & past\\
\textsc{ptcl} & particle\\
\textsc{refl} & reflexive\\
\textsc{rel} & relativiser\\
\textsc{sg} & singular\\
\textsc{spec} & specifying preposition\\
\textsc{tam} & tense-aspect-mood\\
\textsc{tr} & transitive\\
\end{tabularx}



%%%%%%%%%%%%%%%%%%%%%%%%%%%%%%%%%%%
\section*{Acknowledgements}

\begin{sloppypar}
The research for this chapter was funded by a Postdoctoral Fellowship awarded to the author by the Research Foundation -- Flanders (FWO) [project no.~12ZL522N, 2021--2024]. I thank the editors of this volume as well as two reviewers for their valuable comments which have significantly improved this work.
\end{sloppypar}

%%%%%%%%%%%%%%%%%%%%%%%%%%%%%%%%%%%
%\section*{Contributions}
%John Doe contributed to conceptualization, methodology, and validation.
%Jane Doe contributed to the writing of the original draft, review, and editing.

%%%%%%%%%%%%%%%%%%%%%%%%%%%%%%%%%%%
{\sloppy\printbibliography[heading=subbibliography,notkeyword=this]}
\end{document}
