\documentclass[output=paper,colorlinks,citecolor=brown]{langscibook}
\ChapterDOI{10.5281/zenodo.10641197}
\author{Juliane Tiemann\orcid{}\affiliation{University of Bergen}}
%\ORCIDs{}

\title{Modifying variation: Adjective position in Old Norwegian}

\abstract{In this chapter I analyze the positional variation of adnominal adjectives in Old Norwegian. Even though a syntactic development towards a fixed prenominal appearance of adjectives is already well underway in the period studied here, the corpus material still shows cases of postnominal adjectives and cases where the adjectives flank the head noun. For other Germanic languages, positional variation of adjectives relative to the noun that they are modifying has been addressed within discussions of the developing article system and of functional differences such as ``attribution versus predication'' or ``restrictive versus non-restrictive modification''. I will build on these discussions, and further focus on information-structural influence on word order variation, including a left periphery to the Old Norwegian NP with designated positions for \emph{topic}, \emph{focus} and \emph{contrast} in accordance with the split DP hypothesis. I argue that information-structural constraints play an important role for the observed variation within the nominal projection in Old Norwegian.}


\IfFileExists{../localcommands.tex}{
   \addbibresource{../localbibliography.bib}
   \usepackage{langsci-optional}
\usepackage{langsci-gb4e}
\usepackage{langsci-lgr}

\usepackage{listings}
\lstset{basicstyle=\ttfamily,tabsize=2,breaklines=true}

%added by author
% \usepackage{tipa}
\usepackage{multirow}
\graphicspath{{figures/}}
\usepackage{langsci-branding}

   
\newcommand{\sent}{\enumsentence}
\newcommand{\sents}{\eenumsentence}
\let\citeasnoun\citet

\renewcommand{\lsCoverTitleFont}[1]{\sffamily\addfontfeatures{Scale=MatchUppercase}\fontsize{44pt}{16mm}\selectfont #1}
  
   %% hyphenation points for line breaks
%% Normally, automatic hyphenation in LaTeX is very good
%% If a word is mis-hyphenated, add it to this file
%%
%% add information to TeX file before \begin{document} with:
%% %% hyphenation points for line breaks
%% Normally, automatic hyphenation in LaTeX is very good
%% If a word is mis-hyphenated, add it to this file
%%
%% add information to TeX file before \begin{document} with:
%% %% hyphenation points for line breaks
%% Normally, automatic hyphenation in LaTeX is very good
%% If a word is mis-hyphenated, add it to this file
%%
%% add information to TeX file before \begin{document} with:
%% \include{localhyphenation}
\hyphenation{
affri-ca-te
affri-ca-tes
an-no-tated
com-ple-ments
com-po-si-tio-na-li-ty
non-com-po-si-tio-na-li-ty
Gon-zá-lez
out-side
Ri-chárd
se-man-tics
STREU-SLE
Tie-de-mann
}
\hyphenation{
affri-ca-te
affri-ca-tes
an-no-tated
com-ple-ments
com-po-si-tio-na-li-ty
non-com-po-si-tio-na-li-ty
Gon-zá-lez
out-side
Ri-chárd
se-man-tics
STREU-SLE
Tie-de-mann
}
\hyphenation{
affri-ca-te
affri-ca-tes
an-no-tated
com-ple-ments
com-po-si-tio-na-li-ty
non-com-po-si-tio-na-li-ty
Gon-zá-lez
out-side
Ri-chárd
se-man-tics
STREU-SLE
Tie-de-mann
}
   \boolfalse{bookcompile}
   \togglepaper[8]%%chapternumber
}{}

\begin{document}
\maketitle

\section{Introduction} \label{ch8s1}
In Old \ili{Norse} (Old \ili{Norwegian} and \ili{Old Icelandic}), we can observe
considerable syntactic \isi{variation} of various elements within modified NPs
in the surface structure. 
For instance, possessives, demonstratives and adjectives can appear either before or after the \isi{noun} they modify (cf. \citealp[55]{Faarlund04}; \citealp[e12]{Borjarsetal16}). In this chapter, I focus on
\isi{variation} within NPs in Old \ili{Norwegian} that are modified by \isi{attributive}
adjectives, embedded in nominal expressions through direct
\isi{modification} (occurring in adnominal position, expressing inherent
or enduring properties; individual-level reading).\footnote{Cf. \citet[17]{Pfaff2015}, referring to \citet{Cinque2010}, 
  who addresses indirect and direct \isi{modification}: ``indirect
  modifiers are syntactic predicates in a {[}reduced \isi{relative clause}{]},
  whereas direct modifiers are APs merged in dedicated functional
  projections''.} Excluded from the 
analysis are adjectives that occur in a \isi{predicative} context, i.e.
structures where the \isi{adjective} functions as a predicate to the subject
(expressing some kind of accidental or temporary property of the nominal
expression; stage-level reading; see e.g. \citealp[192ff]{SadlerArnold1994}; \citealp[94f]{Cinque1994}; \citealp[6ff]{Cinque2010}; \citealp[274f]{LarsonMarušič2004}; \citealp[7ff]{LarsonTakahashi2004}). Discontinuous
phrases\footnote{The only linearly non-adjacent cases considered here
  are those where the \isi{adjective} \isi{article} \emph{hinn} (ART) appears
  between the \isi{noun} and the \isi{adjective}. However, as this element is
  interpreted as an element of the adjectival constituent (see Section \ref{ch8s3.1.1}), I do not analyze these cases as actually discontinuous (cf.
  also \citealp{Skrzypek2009,Skrzypek2010}; \citealp{StrohWollin2009,StrohWollin2015}; \citealp{Borjarsetal16}; \citealp{Pfaff2019}).} are excluded from the analysis as well.\largerpage[2]

In the extended NP in Old \ili{Norse}, modifiers can occur either before or
after the \isi{noun} (I here assume that this reflects information-structurally
motivated \isi{variation}, see Section \ref{ch8s3.3}). However, the order  \isi{adjective} +
\isi{noun} is already the predominant order in the material (contra \citealp[68]{Faarlund04}; see also \citealp[394]{Mørck2016} who repeats the statement made in
Faarlund, referring also to \citealp[19ff]{Ringdal1918}), and I assume, in
opposition to \citet{vanGelderenLohndal2008}, that this is the base
order at this stage of the language (see also \citetv{chapters/3Modifiers}, who
show that all early \ili{Germanic} languages had the order \isi{adjective} + \isi{noun}).
The development away from the possibility of \isi{postnominal} appearance of
the \isi{adjective} and towards a strict \isi{adjective} + \isi{noun} order in \ili{Germanic}
languages is said to correlate with two factors: 1) the emergence of a
\isi{determiner} system, entailing overt marking of definite contexts (cf. \citealp{vanGelderenLohndal2008}; \citealp{Pfaff2019}), and 2) the general fixation of
\isi{word order} with less influence of information-structural constraints
and \isi{prosodic weight} in the \isi{syntax} (cf. \citealp{Fischer06,Fischer12}; \citealp{Tiemann2022}). This development thus implies a change from
information-structurally marked positions to canonical positions (i.e.
from pragmatics to grammar, see \citealp{SankoffBrown1976}; \citealp{Givón1979}).

For Old \ili{Norwegian}, three general surface patterns\footnote{The notion
  \emph{pattern} is used descriptively and refers to the linear orders
  in the surface structure.} are found in the \isi{corpus} material:
adjectives may precede (\ref{ch8ex1a}), follow (\ref{ch8ex1b}) or flank (\ref{ch8ex1c})--(\ref{ch8ex1d}) (I will refer
to this as the \emph{split construction})\footnote{Only examples with two
  adjectives modifying the same referent (strict identity) were
  considered under the \isi{split construction} (see Sections \ref{ch8s4.1} and \ref{ch8s4.2}). I
  excluded constructions containing two adjectives referring to two
  different referents, as in \emph{\textbf{gamla} {menn} ok
  \textbf{unga}}, `old and young men' (taken from \citealp[7]{Bech17}). Note that square brackets used in examples illustrating a \isi{split construction}, e.g. (\ref{ch8ex1c}), do not refer to an underlying syntactic structure. In these instances, they are used simply to clarify that the adjectives refer to one common referent.} the
\isi{noun} they modify.\footnote{Examples are taken from the main text of the \isi{corpus} material studied here, \textit{Konungs skuggsjá} in AM 243 bα fol. The references are given according to the manuscript page (r/v=recto/verso), the column (a/b), and the line number on the manuscript page. In all the examples, the adjectives are marked in bold, while the head \isi{noun} is marked by italics. Additional elements of interest are marked by a \isi{combination} of bold and italics.}

\begin{exe}
\ex\label{ch8ex1}\begin{xlist}
\ex\label{ch8ex1a} Adjective – Noun\\
\gll þeir hafa \textbf{storar} \textit{vaker} þar \\
they have large.\textsc{acc.pl.str} opening.\textsc{acc.pl} there\\
\glt `they have large openings there' (10v, col.b:21–22)
\ex\label{ch8ex1b} Noun – Adjective\\
\gll komi  i \textit{skola} \textbf{goðan}\\
come.\textsc{sbjv} in school.\textsc{acc.sg} good.\textsc{acc.sg.str}\\
\glt ‘would come in/enter a good school’ (17v, col.b:15)
\ex\label{ch8ex1c} Adjective – Noun – and – Adjective\\
\gll sæm byriar [\textbf{lyðnum} \textit{syni} oc \textbf{litillatom}] at finna [\textbf{astsamlegan} \textit{foður} oc \textbf{gofgan}]\\
as behooves humble.\textsc{dag.sg.str} son.\textsc{dat.sg} and obedient.\textsc{dat.sg.str} to find loving.\textsc{acc.sg.str} father and renowned.\textsc{acc.sg.str}\\
\glt `as it behooves a humble and obedient son to approach a loving and renowned father’ (1r, col.a:22–26)
\ex\label{ch8ex1d} Adjective – Noun – Adjective\\
\gll annat~hvart mæð [\textbf{longu} \textit{hafi} \textbf{rasta}~\textbf{fullu}] \\
whether with wide.\textsc{dat.sg.str} sea.\textsc{dat.sg} full.of.strong.current.\textsc{dat.sg.str}\\
\glt `whether with a wide sea full of strong currents’ (15v, col.a:12–13)

\end{xlist}
\end{exe}

This kind of syntactic \isi{variation} has been discussed extensively for Old
\ili{English}, mainly in correlation with phenomena of \isi{definiteness},
declension, and linear iconicity (see especially \citealp{Fischer00,Fischer06,Fischer12}; \citealp{Haumann03,Haumann10}; \citealp{Bech19}). For Old \ili{Norse}, however, orders
differing from the assumed base order A--N (see e.g. \citealp{Nygaard05}; \citealp{Ringdal1918}; \citealp[68]{Faarlund04}; \citealp[394]{Mørck2016}) have not been studied
in detail. \Citet{vanGelderenLohndal2008} and \citet{Bech17} touch upon
this topic, concentrating on Old \ili{Norwegian}, but do not analyze possible
triggers for the observable \isi{variation} in greater detail. I argue here
that in many ways syntactic \isi{variation} is a choice by the user, and thus
due to information-structural constraints. To examine how and to what
degree these constraints influence \isi{variation} in the Old \ili{Norwegian} NP,
the central point of the discussion concentrates on an examination of the following factors and their possible interplay: i) the \isi{definiteness} of the
NP, ii) the conveyed information status of the elements involved, and iii)
\isi{prosodic weight}. Note that this study is intended to propose an initial 
unified analysis of the positional alternation of adjectives in Old
\ili{Norwegian}, thus there are some distinctions that have not been made and
lie outside the scope of the study (e.g. a systematic analysis of the
semantics/classes of adjectives; cf. e.g. \citealp{Cinque1994}; \citealp{Dimitrova-Vulchanova2003}; \citealp{LarsonMarušič2004}; \citealp{Laenzlinger2005}; \citealp{AlexiadouHaegemonStavrou2007}).

The present chapter has two main objectives. The first is to study
the syntactic \isi{variation} observed within the Old \ili{Norwegian} NP separately from
\ili{Old Icelandic}, focusing on adjectives directly modifying a \isi{noun}. In
syntactic studies, these two languages are most often treated under one
common notion: ``Old \ili{Norse}''. However, Icelandic and \ili{Norwegian} show
distinct developments towards their modern counterparts, and thus may
show syntactic differences already relatively early in their 
histories (cf. also \citealp{Tiemann2022}). The second objective is to study the
influence of various factors and constraints triggering \isi{variation} within
the extended NP. The structural analysis builds on Pfaff's \citeyearpar{Pfaff2015,Pfaff2019}
analysis of Icelandic; however, I extend the structure for the NP in Old
\ili{Norwegian} through the inclusion of the split DP hypothesis.

The chapter is structured as follows: in Section \ref{ch8s2}, I present the \isi{corpus}
material used here and lay out the parameters examined in this study. In
Section \ref{ch8s3}, I discuss the different factors assumed to be responsible for
syntactic \isi{variation} and the theoretical background for syntactic
movement operations within the extended NP. After that, Section \ref{ch8s4}
presents a discussion of the derivation of various surface patterns,
focusing on the \isi{split construction} in Sections \ref{ch8s4.1}. and \ref{ch8s4.2}, before I
conclude this chapter with a summary and remarks in Section \ref{ch8s5}.



\section{Corpus material and parameters}\label{ch8s2}
The data for the analysis presented here is gathered from a \isi{corpus}
compiled by the author at the University of Bergen, \nocite{KoNoKs} \textit{Korpus over den norske Konungs skuggsjá} (KoNoKs). This \isi{corpus} contains the Old \ili{Norwegian} text of
\emph{Konungs skuggsjá} `The king's mirror' in the \ili{Norwegian} main manuscript, AM 243 bα fol.\footnote{\url{https://handrit.is/manuscript/view/da/AM02-0243-b-alpha/0\#mode/2up}}
from the 1270s. The text is annotated for \isi{syntax} and information
structure, following the work collated in ANNIS,\footnote{ANNotation of
  Information Structure, which was originally designed in the \ili{German} collaborative research centre (Sonderforschungsbereich) 632
  (see \citealp{KrauseZelde16}).} and in accordance with the annotations
done within two large projects on information-structural analyses of
older languages.\footnote{These were two projects funded by the \ili{German} Research Foundation:
  \textit{Informationsstruktur in komplexen Sätzen -- synchron und diachron}
  \url{https://gepris.dfg.de/gepris/projekt/199843560?context=projekt\&task=showDetail\&id=199843560\&}
  (2011--2017), and \textit{Informations-struktur in älteren indogermanischen
  Sprachen} \url{https://gepris.dfg.de/gepris/projekt/109055449}
  (2009--2016).} KoNoKs is a \isi{corpus} under development and at the time of
this analysis it consists of 36,861 words. Even though this is still a
relatively small \isi{corpus}, it is sufficiently large to be able to make statements about the
\isi{adjective} position in Old \ili{Norwegian}, since NPs containing adjectives
directly modifying a \isi{noun} are rather frequent. Additionally, I
cross-checked my findings and the patterns given in \citet{Bech17} with
four other Old \ili{Norwegian} texts in five manuscripts: the \emph{Old
\ili{Norwegian} homily book} (in AM 619 4to) from ca. 1200--1225, \emph{Óláfs
saga ins helga} (in Upps DG 8 II) from ca. 1225--1250, \emph{Landslǫg
Magnúss Hákonarsonar} (in HolmPerg 34 4to and in Upps DG 8 I) from ca.
1275 and 1300--1350, and \emph{Strengleikar} (in Upps DG 4--7to) from ca.
1270. These texts were examined through the PROIEL
web application;\footnote{\url{http://foni.uio.no/proiel}} however,
they had to be checked manually due to incomplete
annotations and/or missing annotation review. Moreover, since these texts do not
follow the same annotation practice for phrase structure and information
structure as the text in KoNoKs, the analysis of these four texts was
limited to cross-checking for examples and the existence of patterns.
Thus, the results are mainly presented in a qualitative--descriptive
way, and a detailed analysis of adjectives in these texts is
left for a later study.

To extract the data from KoNoKs, the ANNIS query system was used. The
first query was a request for all adjectives in KoNoKs (Corpus A in
\figref{ch8ft1}). In a second query, I narrowed down the search to all NPs where
the head \isi{noun} is directly modified by one or more APs on which it is
dependent. I then studied these findings in detail and removed
predicate constructions (copula constructions and constructions showing
semantic temporality), an example of which is given in (\ref{ch8ex2}).

\begin{exe}\label{ch8ex2} 
\ex Predicative construction\\
\gll þar sæm \textit{haf-it} er \textbf{diupt} oc þo \textbf{salltr} \textit{sær-inn}\\
there as ocean-\textsc{def.nom.sg} is deep.\textsc{nom.sg.str} and yet salty.\textsc{nom.sg.str} sea-\textsc{def.nom.sg}\\
\glt `there where the ocean is deep and yet the sea salty' (12r, col.b:18--19)
\end{exe}



% \begin{table}
% \todo[inline]{I find it hard to understand this table. Maybe some other representation, like a list or a tree could be found. I'll ask the author, KB}
% \begin{tabularx}{ \textwidth}{Qrl}
% \lsptoprule
% \textbf{Corpus A}: Total number of adjectives & 1117 &  \\
%  in KoNoKs &  &  \\
% \textbf{Corpus B}: Total number of adnominal  & 878 & 100\% (all adnom. APs) \\
% APs in Corpus A & &\\
% Strong adjectives in Corpus B & 823& 100\% (all strong APs)\\
% Weak adjectives in Corpus B &55 & 100\% (all weak APs)\\
% \textbf{Corpus B1}: Total number of APs with at least one postposed \isi{adjective} &
% 63 & 7.2\% of all adnom. APs  \\
% B1.a: Total number of strong adjectives in B1 & 59 & 7.2\% of all strong APs \\
% B1.b: Total number of weak adjectives & 4 & 7.3\% of all weak APs in B1  \\
% \textbf{Corpus B2}: Total number of APs with at least one preposed \isi{adjective}& 777 & 88.5\% of all adnom. APs  \\
% B2.a: Total number of strong adjectives in B2 & 726 & 88.2\% of all strong APs\\
% B2.b: Total number of weak adjectives in B2& 51 & 92.7\% of all weak APs\\
% \textbf{Corpus B3}: Split construction (with and without coordinator) & 38 & 4.3\% of all adnom. APs \\
% \lspbottomrule
% \end{tabularx}
% \caption{Number of adjectives and APs found in the Old Norwegian corpus}\label{ch8t1}
% \end{table}

\begin{figure}
\caption{Number of adjectives and APs found in the Old Norwegian corpus}\label{ch8ft1}
\fittable{
\begin{forest}
[\begin{tabularx}{5cm}{|Q|}
   \hline
   \textbf{\strut Corpus A:\newline Adjectives in KoNoKs}\\
   \hline
   \multicolumn{1}{|c|}{1\,117}\\
   \hline
   \end{tabularx}
  [\begin{tabularx}{5cm}{|C|C|}
      \hline
     \multicolumn{2}{|p{4.5cm}|}{\textbf{\strut Corpus  B:\newline Adnominal APs in Corpus A}}\\
     \hline
     \multicolumn{2}{|c|}{878}\\
      \hline
      Strong  adjectives&Weak   adjectives\\
      \hline
       823& 55\\
      \hline
    \end{tabularx}
      [\begin{tabularx}{3.9cm}{|C|C|}
      \hline
     \multicolumn{2}{|p{3.4cm}|}{\textbf{\strut Corpus  B1:\newline APs with
     at least one postposed adjective}}\\
     \hline
     \multicolumn{2}{|c|}{63/878 (7.2\%)}\\
      \hline
      B1.a: Strong   &B1.b: Weak   \\
      \hline
       59/823 (7.2\%)& 4/55 (7.3\%)\\
      \hline
    \end{tabularx}]
      [\begin{tabularx}{3.9cm}{|C|C|}
      \hline
     \multicolumn{2}{|p{3.4cm}|}{\textbf{\strut Corpus  B2:\newline APs with
     at least one preposed adjective}}\\
     \hline
     \multicolumn{2}{|c|}{777/878 (88.5\%)}\\
      \hline
      B2.a: Strong   &B2.b: Weak   \\
      \hline
       726/823 (88.2\%)& 51/55 (92.7\%)\\
      \hline
    \end{tabularx}]
      [\begin{tabular}{|Z{4.3cm}|}
      \hline
     \parbox{4.2cm}{\textbf{\strut Corpus  B3:\newline Split construction (with\newline and without coordinator)}}\\
     \hline
     38/878 (4.3\%)\\
     \hline
    \end{tabular}, anchor=south]
  ]
]
\end{forest}
}
\end{figure}


I also excluded adjectives in the \isi{comparative} form, as they only occur
with reduced weak \isi{inflection} that might even be treated as an \isi{inflection}
class separate from strong/weak. The result of the second query, after
these exclusions, is Corpus B in Figure \ref{ch8ft1}. From Corpus B, I
extracted all adjectives that precede an NP and all adjectives that
follow an NP. The results constitute the subcorpora B1 and B2. Examples
that show adjectives both to the left and to the right of one \isi{noun} that
they both modify are given in Corpus B3. I paid attention to possible
overlapping results in Corpus B1, B2 and Corpus B3 -- instances of
the \isi{split construction} were subtracted from Corpus B1 and B2. Finally, I distinguished between strong and weak adjectives (B1.a, B1.b, B2.a,
and B2.b). All the instances of the \isi{split construction} in Corpus B3 display strong adjectives.

KoNoKs contains a total of 1,117 adjectives. Of these, 878 adjectives
appear as direct modifiers in a nominal projection. The majority of
these display the order A--N (88.5\%), while there are considerably
fewer examples showing the order N--A (7.2\%). Even fewer adjectives
occur in a \isi{split construction} (4.3\%). As mentioned above, I do not give
any frequencies for adjectives and their positions in the other texts
considered here. All numbers are restricted to KoNoKs.

As optionality in \isi{word order} is often a complex phenomenon and the
result of several parameters that are interlinked, I consider the
influence of various assumed triggers for \isi{variation} (see Figure \ref{ch8tf2}), but
focus is especially on the discussion of information-structural
constraints (see in this context \citealp{Gundel1988}; \citealp{Bech2001}; \citealp{Petrova2009, Petrova2012}; \citealp{TaylorPintzuk2012}; \citealp{StruikvanKemenade2018}).

It has been noted in the literature that information-structural
features of adjectives are difficult to determine (cf. \citealp[13]{vanGelderenLohndal2008}; \citealp[259f]{Allen12}). I therefore translate these
features into a division of \emph{non}-\emph{essential} versus
\emph{essential}, providing grounds for clearer assignments of \isi{emphasis}
on adjectives on the basis of an analysis of the immediate surrounding
context (see Section \ref{ch8s3.4} for a detailed discussion). Prosodic \isi{weight}
was measured by a syllable count of the \isi{adjective}(s), counted from nucleus
to nucleus, and grouped into light (1--3 syllables) and \isi{heavy} (4--6
syllables) adjectives. If the NP includes two adjectives, their combined
syllable number was considered. Additional elements, such as the
\isi{adjectival article}, were left out of the count.

\begin{figure}[t]
% \begin{tabularx}{.999\textwidth}{c|c|c|c}
% \lsptoprule
% Morphology & Syntax & Information structure & Prosodic \isit{weight}          \\
% \hline
%  weak & \isit{prenominal}  & & \\
%  vs. strong &  vs. \isit{postnominal}  &           non-\isit{essential}         &        light vs. \isit{heavy}            \\
%  \cline{1-2}
% positive, & co-occurrence  &        vs. \isit{essential}           &                   \\
% \isit{superlative} &  with ART, DEF &           (emphasised)         &                   \\
% & DEM, INDEF              &                   \\
% \lspbottomrule
% \end{tabularx}
\caption{Parameters for adjectives directly modifying the head noun}

\begin{forest} for tree={forked edges, grow'=east, anchor=west}
  [Parameters
    [Morphology
      [weak vs. strong]
      [{positive, superlative}]
    ]
    [Syntax
      [prenominal vs. postnominal]
      [{co-occurrence with ART, DEF, DEM, INDEF}]
    ]
    [Information structure
      [non-essential vs. essential\\(emphasized),align=left]
    ]
    [Prosodic weight
      [light vs. heavy]
    ]
  ]
\end{forest}

\label{ch8tf2}
\end{figure}




\section{Factors of variation and movement within the NP}
\label{ch8s3}

\subsection{Morphological and syntactic definiteness}
\label{ch8s3.1}

Definiteness is, according to \citet[14]{Heltoft10}, cited in \citet[e15]{Borjarsetal16}, ``a paradigmatic \isi{contrast} in adjectives and thus in NPs, but
not in nouns''. It is generally held that weak versus strong \isi{inflection}
compensated for the lack of a definite and \isi{indefinite} \isi{article} in older
language stages (see \citealp[vol I, 51]{Mitchell85}; \citealp[171ff]{Traugott1992}; \citealp[159ff]{Fischer00}; \citealp[249ff]{Fischer01}; \citealp[256ff]{Fischer06}). The two declensions are thus
dependent on syntactic and semantic functions (see \citealp[37]{Faarlund04};
see also \citealp[122ff]{Abbott2008} for a discussion of definite and \isi{indefinite}
NPs), where weakly inflected adjectives are mainly used in semantically
definite NPs and strongly inflected adjectives in semantically
\isi{indefinite} NPs. This distinction can be translated into informational
features. The strong (\isi{indefinite}) \isi{adjectival inflection} may indicate
that the feature presented by the \isi{adjective} is new in the context, while
the weak \isi{adjectival inflection}, syntactically supported by an overt
\isi{definiteness} marker, points towards a given feature within the context.
Syntactically, the emergence of the definite (and \isi{indefinite}) \isi{article}
starts to mark the NP overtly for \isi{definiteness} and contextually for
\isi{givenness}. Adjectival \isi{inflection} together with these overt markers can
create narrow semantic content, e.g. in constructions showing a strong
\isi{adjective} in \isi{combination} with an overtly marked definite \isi{noun} (cf. e.g.
\citealp[3]{Thráinsson2007} for modern Icelandic), implying that the \isi{noun} is
known in the context, while the adjectival property describes a new
feature of this known referent (this, however, is only possible with the
occurrence of the nominal \isi{article} -\emph{inn} (DEF), as the adjectival
\isi{article} is exclusively bound to the weak \isi{inflection} in the oldest
attestations; cf. \citealp{Pfaff2019}, see also Section \ref{ch8s3.1.1}).

\largerpage
The distinction between weak/strong \isi{adjectival inflection} and semantically
definite/\isi{indefinite} NPs has often been brought into correlation with a
distinction between (\isi{prenominal}) \isi{attributive} versus (\isi{postnominal}) \isi{predicative}
use of adjectives (see e.g. \citealp[256]{Fischer12} for Old \ili{English}).
Attributive adjectives occur inside a \isi{noun} phrase, modifying the head
\isi{noun}, while \isi{predicative} adjectives form a separate constituent and do
not function as a modifier governed by the head \isi{noun}. However, in the
analysis presented here, I do not define strong adjectives as solely
functionally \isi{predicative}. Prenominal strong adjectives are thus not
unexpected and are patterned with \isi{prenominal} weak \isi{adjective} readings
regarding their semantic and functional properties, in line with \citet[66ff]{Haumann10}, unless explicitly stated otherwise.


\subsubsection{Definiteness}\label{ch8s3.1.1}
\begin{sloppypar}
The prototypical way of marking a context for \isi{definiteness} is by using
the definite articles ({[}+definite{]}; they can have the feature
{[}+specific{]}), or by using demonstratives (which have a
{[}+deictic{]} feature), which clearly show distinct reference and
anaphoricity within the discourse (see e.g. \citealp{Schwarz2009}). Accounts of \isi{definiteness} phenomena have described the ability to identify a referent and refer to a totality, i.e. unique
referents, uncountable nouns and plurals (cf. e.g. \citealp{Lyons1999}; \citealp{Rampazzo2012}). All \ili{Germanic} languages developed a definite
\isi{article} system as they developed towards their modern counterparts to encode this kind of
information. In Old \ili{Norwegian} (and \ili{Old Icelandic}), one of two definite
\isi{article} items was used: a free morpheme (ART; \isi{adjectival article}; cf.
\citealp[e15]{Borjarsetal16}) and a bound morpheme (DEF; nominal suffix
\isi{article}), as shown in (\ref{ch8ex3}).
\end{sloppypar}

\begin{exe}
\ex\label{ch8ex3}Definite articles in Old \ili{Norse}\begin{xlist}
\ex\label{ch8ex3a} Adjective \isi{article} (ART)\\
\gll \textbf{\textit{hinn}} \textbf{fyrsta} \textit{dag} \\
\textsc{art.acc.sg} first.\textsc{acc.sg.wk} day.\textsc{acc.sg} \\
\glt `the first day' (7v, col.b:16) \\
 
\ex\label{ch8ex3b} Nominal \isi{article} (DEF)\\
\gll \emph{dag}-\textbf{\textit{inn}} \\
day-\textsc{def.acc.sg} \\
\glt `the day' (40r, col.a:8)
\end{xlist}
\end{exe}

The \isi{adjective} \isi{article} (ART) complements the weak \isi{adjective} (cf.
e.g. \citealt{StrohWollin2009}, \citetv{chapters/9AdjArtOG}) and is illicit
with a bare \isi{noun} (*\emph{hinn dag}),\footnote{However, cases of double
  \isi{definiteness} which display both of these elements can be found in Old
  \ili{Norwegian}, as in `\textbf{\textit{hinir}} \textbf{bæzto} \textit{mænn}-\textbf{\textit{iner}}' `\textsc{art} best men-\textsc{def}' (26v,
  col.b:20--21).} in which case simple
\isi{definiteness} may be expressed through the element DEF as in (\ref{ch8ex3b}). The
only element obligatorily marked for (in)\isi{definiteness} within the NP is
the \isi{adjective}, meaning that the definite \isi{article} in
semantic\slash discourse--pragmatic definite NPs (identificatory and
contextually given in the discourse) is often still missing in Old
\ili{Norwegian}.\footnote{Note that Old \ili{Norse} does not yet have a fully
  grammaticalized \isi{article} system (see \citealp[27f]{Nygaard05}; \citealp[56, 74]{Faarlund04}; \citealp[225]{CrismaPintzuk2019}).} In overtly marked definite NPs
modified by an \isi{adjective}, the unbound \isi{article} ART triggers the definite
(i.e. weak) form of the \isi{adjective}, which may be considered an \isi{agreement}
relation between the features {[}DEFINITE{]} and {[}WEAK{]} (cf. \citealp[118]{Vangsnes1997}; \citealp[54]{Pfaff2015}, who translates this into a c-command
relation).\footnote{The weak form of the \isi{adjective} is not found outside
  of definite contexts with an overt definite marker/trigger (an
  exception is the word \emph{samr} `same' whose degree of adjectivity,
  however, can be discussed; see also \citealp[12]{Bech17}).} Describing two
different definite articles as shown in (\ref{ch8ex3}), I follow \citet{Pfaff2019} and
assume that DEF is present in a position below \emph{n}P and closest to N,
while I assume ART to be merged as the head of weak APs (note that
strong APs are illicit with ART). Strongly inflected adjectives are found in semantically \isi{indefinite} NPs; however, they can also occur in definite contexts like those shown in (\ref{ch8ex4}) and (\ref{ch8ex5}) when ART is absent (see (\ref{ch8ex12})). I therefore consider the strong inflection as the default form in all contexts.
Above the merging site for (all) adjectives is a CardP hosting numerals
or cardinal quantifiers in its specifier position, and above this a
projection for demonstratives (layered DP, see \citealp{Julien2002,Julien2005}; \citealp{Adger2013}). Note that a DP in this sense is a \isi{demonstrative} phrase headed
either by a \isi{demonstrative} or a \isi{pronoun} (cf. also \citealp{LanderHaegeman2014}). Based on the analysis presented in \citet{Pfaff2019} for Icelandic
(see also \citealp{Hardarson2017}), as well as the proposed universal by \citet[87]{Greenberg63}\footnote{Universal 20: ``When any or all of the
  items (\isi{demonstrative}, \isi{numeral} and descriptive \isi{adjective}) precede the
  \isi{noun}, they are always found in that order. If they follow, the order
  is either the same or its exact opposite.''} regarding the order of
demonstratives, numerals, adjectives, and nouns, I assume the base
structure for the extended Old \ili{Norwegian} NP to be the one given in (A).
For the purpose of this chapter, I will present a relatively simple
structure, ignoring aspects that are not at the center of the
discussion.

\begin{exe}\label{ch8exa}
 \exi{(A)} {[}\textsubscript{DemP} \emph{sá} ... {[}\textsubscript{PossP} pronouns ... {[}\textsubscript{CardP} ... {[}\textsubscript{αP} AP...{[}\textsubscript{\emph{n}P} DEF N {]}{]}{]}{]}{]}
\end{exe}

For weak adjectives, the AP consists of two elements, forming one
constituent ({[}ART A\textsubscript{WK}{]}). ART can also co-occur
with additional elements that may render the NP definite, such as the
\isi{demonstrative} \emph{sá}, exemplified in (\ref{ch8ex4})\footnote{It has also been
  noted that ``{[}d{]}emonstratives do not necessarily give the NP a
  unique or specific reference'' \citep[85f]{Faarlund04}, so that the
  \isi{indefinite} form of the \isi{adjective} may co-occur with demonstratives.}
or a \isi{possessive pronoun} like \emph{MINN}, as in (\ref{ch8ex5}). This in turn
implies that these elements are not on a par with ART, neither categorically, nor functionally, nor structurally (see \citealp[24, 31f]{Pfaff2019}; cf. also \citealp{Faarlund04,Faarlund2009}). These elements are merged in a separate
position above the adjectival projection.

\begin{exe} 
\ex \label{ch8ex4}\emph{sá} ART A.WK\\
\gll Kona \textbf{\textit{þærs}} \textbf{\textit{hins}} \textbf{rika} \textit{mannz} \\
wife \textsc{dem.gen.sg} \textsc{art.gen.sg} rich.\textsc{gen.sg.wk} man.\textsc{gen.sg}\\
\glt `wife of this rich/mighty man' (35v, col.a:14--15)
\end{exe}

\begin{exe}
\ex\label{ch8ex5}  POSS ART A.WK\\
\gll \textit{\textbf{mina}} \textit{\textbf{hina}} \textbf{liotligo} \textit{asion} \\
my \textsc{art.acc.sg} horrible.\textsc{acc.sg.wk} appearance.\textsc{acc.sg}\\
\glt `my terrible appearance' (43v, col.a:12--13)
\end{exe}

The bound \isi{article} DEF is less frequent in structures involving an
\isi{adjective}. In structures that only contain DEF as an overt \isi{definiteness}
marker, the default form of the \isi{adjective} is used (=strong declension).
However, in the \isi{corpus} material DEF may also co-occur with ART (see
also \citealp[18]{Pfaff2019} for \ili{Old Icelandic}), as in the examples given in (\ref{ch8ex6}).
In this case the weak form of the \isi{adjective} is triggered. These examples
also show that these two morphemes cannot be the same element and occupy
different syntactic positions (contra \citealp{Faarlund04}).\footnote{For Old
  Icelandic, \citet[18]{Pfaff2019} even shows examples of direct adjacency of
  these two elements. However, constructions showing some kind of double
  \isi{definiteness} are still quite rare in the \isi{corpus} material (see also
  \citealp{Lundeby1965}). Double \isi{definiteness} was generally rare in Old \ili{Norse}
  \citep[58]{Faarlund04}. According to \citet{Lundeby1965}, double \isi{definiteness}
  in \ili{Norwegian} developed around 1200 and was established as a
  structure before 1400 (see also \citealp[290]{Lohndal2007}; \citealp{vanGelderenLohndal2008}). Note, however, that the type of double \isi{definiteness}
  shown in the examples in (\ref{ch8ex6}) is of a different kind from the one found in
  modern \ili{Norwegian} (cf. \citealp[292]{LanderHaegeman2014}), since ART
  disappeared from the language by the end of the Old Norwegian period %towards modern \ili{Norwegian} 
  while its function was taken over
  by the \isi{demonstrative} (developing into a \isi{determiner}). 
  While NPs modified by an \isi{adjective} obligatorily display double \isi{definiteness} in modern \ili{Norwegian}, in Icelandic this ``is consistently attested
  as a marked \isi{pattern} from the 12\textsuperscript{th} century onwards and disappeared in
  the early 20\textsuperscript{th} century'' (see \citealp[19]{Pfaff2019}).}

\begin{exe}
\ex\label{ch8ex6}Double \isi{definiteness} in Old \ili{Norwegian}\begin{xlist}
\ex\label{ch8ex6a} Co-occurrence (ART+DEF)\\
\gll \textit{\textbf{hinn}} \textbf{heiti} \textit{vægr}-\textit{\textbf{inn}} \\
\textsc{art.nom.sg} hot.\textsc{nom.sg.wk} way-\textsc{def.nom.sg}\\
\glt `the hot zone' (12v, col.b:29--30)
\ex\label{ch8ex6b}Co-occurrence (DEM+ART+DEF)\footnote{The
    occurrence of the \isi{demonstrative} in this example points towards a
    contrastive reading of this phrase, as it immediately follows the
    phrase given in (\ref{ch8ex6a}) within the discourse context.} \\
\gll \textit{\textbf{þeir}} \textbf{\textit{hiner}} \textbf{kalldu} \textit{vægir}-\textbf{\textit{nir}} \\
 \textsc{dem.nom.pl} \textsc{art.nom.pl} cold.\textsc{nom.pl.wk} way-\textsc{def.nom.pl}\\
\glt `the cold zones' (13r, col.a:3)
\end{xlist}
\end{exe}

With these general observations in mind, we can now take a closer look
at the surface patterns displaying one \isi{adjective} modifying a head \isi{noun}
found in the \isi{corpus} material. I will follow Pfaff's \citeyearpar{Pfaff2019} listed
patterns for \ili{Old Icelandic},\footnote{The three patterns (V), (VI) and (VII) are
  not described by \citet{Pfaff2019}. However, they are patterns which are also found in Icelandic.
  I decided to add these to the description here, even though these and other patterns are represented only by very few examples in KoNoKs. Also \isi{pattern}
  III, for instance, is only represented by one example, but it is a
  verified \isi{pattern} in other Old \ili{Norwegian} texts, cf. e.g. `Crist
  \textit{stol} hinn \textbf{dýri}'  `the \textbf{valuable} \textit{chair} of Christ' (HómNo 2.33,8), `firir \textit{nott}-ena \textbf{hælgu}' `for the \textbf{holy} \textit{night}' (MLL 7,3), or
  `\textit{cross}-en \textbf{helga}' `the holy cross' (HómNo 3.3,66). Two additional patterns show cases of double
  \isi{definiteness}: DD--a and DD--b. Pattern (VII) is also found with the
  proximal \isi{demonstrative}, \textit{sjá}/\textit{þessi} (two distinct types of
  demonstratives), as in `\textit{Ormr} þæsse hinn \textbf{orðslœgi}' `That \textbf{articulate} \textit{worm}' (41r, col.b:8).} starting here with
\isi{pattern} (II) (see Table \ref{ch8t3}), as there are no instances found of what Pfaff labelled
\isi{pattern} (I) for Icelandic (A.WK N--DEF). 

The last column in the table shows the number of examples for the specific patterns found in KoNoKs. Only those
adjectives are represented in Table \ref{ch8t3} that appear with an overt
\isi{definiteness} marker (DEF, ART and/or \emph{sá}).\footnote{Only basic patterns are presented in Table \ref{ch8t3}. These structures may show additional elements, such as a \isi{possessive pronoun}. A quick search in the other Old \ili{Norwegian} texts considered showed the same patterns. In total, the \isi{corpus} presents 55 examples of weak adjectives. The reason for the total count of 51 adjectives in Table \ref{ch8t3} is that four examples did not appear with a definite marker (with the word \textit{samr} and some adjectives in the \isi{superlative}).}


\begin{table}\small
\begin{tabular}{llllllr}
\lsptoprule
(II)\footnote{This pattern is especially used with superlatives or in enumerations, e.g. `Hinn \textbf{þriðe} \textit{lutr}' ‘the \textbf{third} \textit{thing}’ (11r, col.b:26–27); `hit \textbf{þriðia} \textit{sæla kyn}' ‘the \textbf{third} \textit{kind of seal}’ (10v, col.a:26).} & ART & WK &
prenom. & post-articular & \textit{\textbf{hina}} \textbf{bæztu} \textit{mænn} & 40  \\ 
 & &  & &  &  (2v, col.b:20--21)  \\ 
 & &  & &  &  `the best man'   \\ 
(III) & DEF \textgreater{}  & WK & postnom. & post-articular & \textit{haf}-\textbf{\textit{et}} \textbf{mykla} &1 \\ 
  &   ART\footnote{According to Pfaff \citeyearpar[18f, 31]{Pfaff2019}, the
  adjectival article ART can occur as a free or a bound element in Old
  Icelandic. He further notes that nominal and adjectival
  articles are two distinct elements, as cases of double definiteness
  including both ART and DEF suggest against treating these as one. The
  Old Norwegian data support this statement (cf. ex. 6).}
  &  & &  &    (13r, col.a:17)  &  \\ 
  &   &  & &  &  'the great ocean' &  \\ 
(IV) & DEF & STR & prenom. & pre-articular & \textbf{visan} \textit{mæistar}-\textit{\textbf{ann}}  & 2 \\ 
 &  &  & &  & (4r, col.b:1)  &  \\ 
 &  &  & &  &  `the wise master' &  \\ 
(DD--a) & ART +  & WK & prenom. & pre- and  & \textbf{\textit{hinum}} \textbf{heita} \textit{væg}- &5  \\ 
 & DEF &  &  & post-articular &\textbf{\textit{inum }} (14v, col.b:1)  &   \\ 
 &  &  &  &  &`the hot way/zone' &  \\ 
(DD--b) & \emph{sá} +  & WK & prenom. & pre- and & \textbf{\textit{þeim}} \textbf{heita} \textit{væg}- & 2 \\ 
&  DEF &  &  &  post-articular& \textit{\textbf{inum}} (14v, col.b:9)  &  \\ 
&   &  &  &  &  `the hot way/zone'  &  \\ 
(V)\footnote{One example of pattern (V) displays the word \emph{sialfr} `self'.
  It is questionable whether this is a true example of this pattern.} & DEF &  STR& postnom. &  post-articular& \textit{lannd}-\textit{\textbf{et} } \textbf{\textit{þitt}} &  4\\ 
  & & &  &  &  (12r, col.a:1) &  \\ 
  & & &  &  & `the unfrozen soil' &  \\ 
 (VI) & \emph{sá} & STR & prenom. & post-articular & \textit{\textbf{þeim}} \textbf{hælgum} & 6 \\
 &  &  &  &  & \textit{manne} (8r, col.b:15) \\
  &  & & &  &  `this holy man'  &  \\  
(VII) & \emph{sá} +  & WK & postnom. & post-articular & \textit{Tre} \textit{\textbf{þat hit}} \textbf{fagra} & 3 \\ 
 & ART & & &  &  (40r, col.b:16)  &  \\ 
 &  & & &  &  `This beautiful tree' &  \\
 \lspbottomrule
\end{tabular}
\caption{Possible word order patterns connected to overt definiteness. Pre-/post-articular refers to the adjectival position relative to DEF or ART}
\label{ch8t3}
\end{table}
% % % % % \begin{table}
% % % % % % \todo[inline]{maybe ``nom'' and ``articular'' can be removed and only pre/post can be written, to save horizontal space. Is it important to save horizontal space here? If nom and articular are removed, how do we tell the difference between prenominal and pre-articular? In a row at the top? KB }
% % % % % \small
% % % % % \begin{tabularx}{\textwidth}{llXXllr}
% % % % % \lsptoprule
% % % % %  & & & nom. & articular & & \\
% % % % %  \midrule
% % % % % (II)\footnote{This \isi{pattern} is especially used with superlatives or in enumerations, e.g. `Hinn \textbf{þriðe} \textit{lutr}' ‘the \textbf{third} \textit{thing}’ (11r, col.b:26–27); `hit \textbf{þriðia} \textit{sæla kyn}' ‘the \textbf{third} \textit{kind of seal}’ (10v, col.a:26).} & ART & WK &
% % % % % pre & post & \textit{\textbf{hina}} \textbf{bæztu} \textit{mænn} & 40  \\
% % % % %  & &  & &  &  (2v, col.b:20--21)  \\ 
% % % % %  & &  & &  &  `the best man'   \\ 
% % % % % \tablevspace
% % % % % (III) & DEF >  ART\footnote{According to Pfaff \citeyearpar[18f, 31]{Pfaff2019}, the
% % % % %   \isi{adjectival article} ART can occur as a free or a bound element in Old
% % % % %   Icelandic. He further notes that nominal and adjectival
% % % % %   articles are two distinct elements, as cases of double \isi{definiteness}
% % % % %   including both ART and DEF suggest not treating these as one. The
% % % % %   Old \ili{Norwegian} data support this statement (cf. ex. 6).} & WK & post & post & \textit{haf}-\textbf{\textit{et}} \textbf{mykla} &1 \\
% % % % %   &
% % % % %   &  & &  &    (13r, col.a:17)  &  \\ 
% % % % %   &   &  & &  &  'the great ocean' &  \\ 
% % % % % \tablevspace
% % % % % (IV) & DEF & STR & pre & pre & \textbf{visan} \textit{mæistar}-\textit{\textbf{ann}}  & 2 \\
% % % % %  &  &  & &  & (4r, col.b:1)  &  \\ 
% % % % %  &  &  & &  &  `the wise master' &  \\ 
% % % % % \tablevspace
% % % % % (DD--a) & ART + DEF & WK & pre & pre/post  & \textbf{\textit{hinum}} \textbf{heita} \textit{væg}- &5  \\
% % % % %  &  &  &  &  &\textbf{\textit{inum }} (14v, col.b:1)  &   \\
% % % % %  &  &  &  &  &`the hot way/zone' &  \\ 
% % % % % \tablevspace
% % % % % (DD--b) & \emph{sá} + DEF & WK & pre & pre/post & \textbf{\textit{þeim}} \textbf{heita} \textit{væg}- & 2 \\
% % % % % &   &  &  &   & \textit{\textbf{inum}} (14v, col.b:9)  &  \\
% % % % % &   &  &  &  &  `the hot way/zone'  &  \\ 
% % % % % \tablevspace
% % % % % (V)\footnote{One example of \isi{pattern} (V) involves the word \emph{sialfr} `self'.
% % % % %   It is questionable whether this is a true example of this \isi{pattern}.} & DEF &  STR& post &  post& \textit{lannd}-\textit{\textbf{et} } \textbf{\textit{þitt}} &  4\\
% % % % %   & & &  &  &  (12r, col.a:1) &  \\ 
% % % % %   & & &  &  & `the unfrozen soil' &  \\ 
% % % % % \tablevspace
% % % % %  (VI) & \emph{sá} & STR & pre & post & \textit{\textbf{þeim}} \textbf{hælgum} & 6 \\
% % % % %  &  &  &  &  & \textit{manne} (8r, col.b:15) \\
% % % % %   &  & & &  &  `this holy man'  &  \\  
% % % % % \tablevspace
% % % % % (VII) & \emph{sá} + ART & WK & post & post & \textit{Tre} \textit{\textbf{þat hit}} \textbf{fagra} & 3 \\
% % % % %  &  & & &  &  (40r, col.b:16)  &  \\
% % % % %  &  & & &  &  `This beautiful tree' &  \\
% % % % %  \lspbottomrule
% % % % % \end{tabularx}
% % % % % \caption{Possible word order patterns connected to overt definiteness. Pre-/post-articular refers to the adjectival position relative to DEF or ART}
% % % % % \label{ch8t3}
% % % % % \end{table}

In \isi{contrast} to the \ili{Old Icelandic} data (cf. \citealp[14]{Pfaff2019}), \isi{pattern}
(DD--b) is already a possible surface \isi{pattern} in the 13\textsuperscript{th} century in Old
\ili{Norwegian}, showing that the replacement of ART by the distal
\isi{demonstrative} \emph{sá} started relatively early in the language
history of \ili{Norwegian} (see also \citealp{StrohWollin2009,StrohWollin2015}). However, an
additional \isi{definiteness} marker is still needed to support the
replacement of ART. The element \emph{sá} slowly developed into a
\isi{definite determiner} and the universal \isi{adjectival article} in the Mainland
\ili{Scandinavian} languages. The fact that a competition between ART and
\emph{sá} is still going on in Old \ili{Norwegian} is also supported by the
appearance of \isi{pattern} (VII) showing both elements next to each other.
The later exchange/retention of the element ART in the \isi{syntax} leads to a
split betweeen the Mainland \ili{Scandinavian} languages and Icelandic. As
predicted, no examples of a co-occurrence of ART with
strongly inflected adjectives are found in the \isi{corpus} material.


\largerpage
\subsubsection{Indefiniteness}\label{ch8s3.1.2}
Apart from Icelandic, all \ili{Germanic} languages have also developed an
\isi{article} system to mark \isi{indefiniteness}. In Old \ili{Norse}, the element
\emph{einn}, if used as an \isi{indefinite} marker, may mark
specificity\footnote{Note that the \isi{adjective} position might also be
  sensitive to the specific or non-specific reading of the NP in which
  it appears (see \citealp[72]{Jacob2005}; see also \citealp{Bosque1996}). A detailed
  discussion of this, however, is put aside for a later analysis of the
  material, as KoNoKs does not entail an annotation for
  \emph{specificity}.} but is not an obligatory element within
\isi{indefinite} structures (see also Heine 1997: 72f, 2002 in \citealp[51, 53]{Skrzypek2012}; cf. \citealp[142]{Crisma2015} for the three stages of the development of
the \isi{indefinite} \isi{article}). \citet[232]{CrismaPintzuk2019} refer to Old
\ili{Swedish} and \citeauthor{Skrzypek2012}'s \citeyearpar[76, 158]{Skrzypek2012} analysis, stating that
``\emph{en} is used exclusively as a numerical {[}...{]} at least until
1225. Skrzypek found the earliest attestation of non-numerical
\emph{en} in \emph{Bur} (dated 1276--1307)'', which falls into the same
period analyzed for Old \ili{Norwegian} in this study. \citet[387]{Mørck2016}
further notes for Old \ili{Norwegian} that ``{[}a{]}llerede på 1200-tallet
fins det {[}...{]} bruk av \textit{einn} som minner om den ubestemte artikkelen i
moderne norsk {[}...{]}'' (`Already in the 13\textsuperscript{th} century, there are
instances of the usage of \emph{einn} that resemble the \isi{indefinite}
\isi{article} in modern \ili{Norwegian}'). In the \isi{corpus} material analyzed
here, some examples of \emph{einn} already displaying a specific marker
were found as well. However, the function as a non-\isi{numeral} still
reflects an earlier stage as a presentative marker to introduce new and
salient referents with an anaphoric chain following its introduction
into the discourse (see \citealp[52]{Skrzypek2012}; \citealp[33]{Skrzypek2013}). Examples of the
non-numerical usage of \emph{einn} in the \isi{corpus} material are given in
(\ref{ch8ex7}).

\begin{exe}
\ex\label{ch8ex7}Indefinitely marked modified NP
\begin{xlist}
\ex\label{ch8ex7a}
\gll \textbf{heilagr} \textit{maðr} \textit{\textbf{einn}}  \\
holy.\textsc{nom.sg.str} man.\textsc{nom.sg} \textsc{indef}\\
\glt `a holy man' (7r, col.b:25)
\ex\label{ch8ex7b}
\gll \textbf{\textit{æinn}} \textbf{heilagr} \textit{maðr}  \\
\textsc{indef} holy.\textsc{nom.sg.str} man.\textsc{nom.sg}\\
\glt `a holy man' (7r, col.b:14)
\ex\label{ch8ex7c}
\gll \textit{holme} \textit{\textbf{æinn}} \textbf{litell }\\
islet.\textsc{nom.sg} \textsc{indef} small.\textsc{nom.sg.str}\\
\glt `a small islet' (6r, col.b:19--20)
\end{xlist}
\end{exe}

These examples reflect the three surface patterns including INDEF found
in the \isi{corpus} material, here given in Table \ref{ch8t4} (again, the number of
examples found is given in the last column). As expected, weak adjectives
do not appear in overtly marked \isi{indefinite} extended NPs.

\begin{table}[t]
\begin{tabularx}{.8\textwidth}{Xllllr}
\lsptoprule
(I--b) & INDEF & STR & \isi{prenominal} & pre-articular & 2 \\ 
(II--b) & INDEF & STR & \isi{prenominal} & post-articular & 13 \\ 
(III--b) & INDEF & STR & \isi{postnominal} & post-articular & 8  \\ 
\lspbottomrule
\end{tabularx}\caption{Possible word order patterns connected to overt indefiniteness. Pre-/post-articular refers to the adjectival position relative to INDEF}\label{ch8t4}
\end{table}

But, whatever the `exact' stage of \emph{einn} is in Old \ili{Norwegian}, I
have here only considered examples that are already semantically
different from the \isi{numeral} use of \emph{einn}, i.e. introducing new
referents and starting to mark \isi{indefiniteness} by these means.\footnote{There
  are clear examples in which \emph{einn} functions as a \isi{numeral},
  especially in constructions including \emph{sjá/þessi}, e.g. `Þæssa \textit{\textbf{æina}} \textit{grein}' `this \textbf{one} branch' (43v, col.b:25), or
  \emph{sá}, e.g. `Ða er þar ænn \textit{\textbf{æinn}} sa \textit{lutr}' `There is yet \textbf{one} such (one other) thing' (8r, col.b:4--5).}
However, strong
adjectives are not in need of an overt marker (INDEF) in the same sense
as weak adjectives are dependent on ART. In Old \ili{Norwegian}, structures
with an \isi{indefinite} interpretation and without any overt \isi{indefinite}
marker are still the norm, as shown in (\ref{ch8ex8}).

\begin{exe}
\ex\label{ch8ex8}Indefinite modified NP
\begin{xlist}
\ex\label{ch8ex8a}
\gll  \textit{Nalar} \textbf{margar} oc \textit{þræðr} \textbf{œrna}. eða \textit{sviptingar} \\
nail.\textsc{acc.pl} many.\textsc{acc.pl.str} and thread.\textsc{acc.pl} strong.\textsc{acc.pl.str} or cord.\textsc{acc.pl}\\
\glt `many nails, and strong thread or cords' (3v, col.a:10--11)
\ex\label{ch8ex8b}
\gll sænnder \textbf{varmar} \textit{vingiafer} norðanvinnde \\
sends warm.\textsc{acc.pl.str} friendship.gift.\textsc{acc.pl} northwind.\textsc{dat.sg}\\
\glt `sends warm gifts of friendship to the north wind' (4v, col.a:16)
\end{xlist}
\end{exe}

Only 23 examples displaying non-numerical \emph{einn} used as a marker to introduce a new referent within a directly
modified nominal projection could be identified in KoNoKs. Of these, 15
examples show a surface \isi{pattern} with \isi{prenominal} adjectives and 8
examples show \isi{postnominal} adjectives. Pattern (II--b) in Table \ref{ch8t4} with
both a \isi{prenominal} \isi{adjective} and a \isi{prenominal} \isi{article} is the predominant
\isi{pattern} in these contexts. It is also the only grammatical \isi{pattern}
possible in modern \ili{Norwegian}, where \emph{einn} is grammaticalized as the
\isi{indefinite} \isi{article}. However, being first of all a \isi{quantifier} in Old
\ili{Norwegian}, \emph{einn} is assumed to be merged as the specifier of CardP
above the AP, as shown in (B).\footnote{See e.g. the following example:
  þætta \textit{\textbf{æitt}} satt \textit{upphaf} `this one
  true source' (1v, col.b:8--9).}

\begin{exe}
\exi{(B)}\label{ch8exb}   {[}\textsubscript{DemP} ... {[}\textsubscript{CardP} \emph{einn} {[}\textsubscript{αP} AP\textsubscript{STR} ... {[}\textsubscript{\emph{n}P} N {]}{]}{]}{]}
\end{exe}

Patterns deviating from INDEF--A--N (i.e. A--N--INDEF and
N--INDEF--A) can be explained through NP-movement with the option
of pied-piping the \isi{adjective} (see Section \ref{ch8s4}).

From the discussion of adjectives in definite and \isi{indefinite} contexts in
Old \ili{Norwegian}, it seems that concerning the positioning of adjectives
relative to N, both weak and strong adjectives can appear in pre- and
\isi{postnominal} position.\footnote{The \isi{prenominal} position for adjectives
  is, however, already preferred in Old \ili{Norwegian} with 88.2\% of all
  strong APs (726/823) and 92.7\% of all weak APs (51/55)
  appearing in this position in the \isi{corpus} material.} For weak
adjectives in \isi{postnominal} position, \citet[265f.]{Fischer01} states for Old
\ili{English} that these adjectives are weak because they do not convey new
information, thus connecting \isi{inflection} to \isi{givenness}. However, adjectives
that convey given information are not exclusively weak, neither in Old
\ili{English} (see \citealp{Bech19}) nor in Old \ili{Norwegian} (e.g. the adjectives in the
examples given for patterns (III) and (VII) in Table \ref{ch8t3} have not been
mentioned in the previous discourse and are not inferable from that discourse). Thus, neither the form of the \isi{adjective} nor the
additional (in)\isi{definiteness} markers seem to be decisive factors
for the ordering of adjectives within the modified NP in Old \ili{Norwegian}.
Optionality in \isi{word order} is a complex phenomenon and the result of
several interlinked parameters.


\subsection{Context and referentiality}\label{ch8s3.2}
After this closer look at the extended NP,  the immediate context of
a phrase also needs to be taken into account, and with this the distinction
between attributively and predicatively used adjectives. \citet{Fischer00,Fischer01} argues for Old \ili{English} that the weak
\isi{adjectival inflection} has an identifying and \isi{attributive} function
(inherent or enduring property of the \isi{noun} it modifies), iconically
relates to `old information', and appears in \isi{prenominal} position. Strong
adjectives, on the other hand, relate iconically to `new information'
and to \isi{predication} (e.g. not an inherent property of the \isi{noun} it
modifies; a one-time occurrence). These adjectives are not incorporated
into the \isi{noun} and may appear postnominally. However, in the following
examples, I will show that the two generalisations: i)
\isi{attributive}=\isi{prenominal}, and ii) weak=\isi{attributive} and strong=\isi{predicative}
cannot be transferred to Old \ili{Norwegian} (see also \citealp[8]{Bech17}). For Old
\ili{English}, too, Fischer's strict distribution has been discussed as
problematic (see \citealp{Bech19}). The examples given in (\ref{ch8ex9}) show weak
adjectives in \isi{postnominal} position (see also \citealp[14]{Pfaff2019} for Old
Icelandic; he finds 212 examples of weak adjectives in \isi{postnominal}
position and writes that this is a marked, but stable \isi{pattern}). These
adjectives are \isi{attributive}, despite their placement in relation to N
(see also \citealp[62]{Haumann10} and \citealp[vol. I, 75]{Mitchell85}), and do not
necessarily need to be given information within the discourse, but can
be new mentions in the given context.\largerpage[-1]

\begin{exe}
\ex\label{ch8ex9}
\begin{xlist}
\ex\label{ch8ex9a}
Strengleikar\\
\gll  hia \textit{havi} hinu \textbf{mykla} \\
at ocean.\textsc{dat.sg} \textsc{art.dat.sg} great.\textsc{dat.sg.wk}\\
\glt `at the great ocean' (Streng 7,3)
\ex\label{ch8ex9b}
Old \ili{Norwegian} homily hook\\
\gll \textit{vitni} hinu \textbf{sanna } \\
witness.\textsc{dat.sg} \textsc{dat.sg.art} true.\textsc{dat.sg.wk}\\
\glt `the true witness' (HómNo 2.8,24)
\end{xlist}
\end{exe}

Moreover, predicatehood seems not to be inherent to strong adjectives in
Old \ili{Norwegian}. See the following examples in (\ref{ch8ex10}).

\begin{exe}
\ex\label{ch8ex10}
\begin{xlist}
\ex\label{ch8ex10a}
\gll  þvi at þeir hafa heilhugaðer værit við alla {[}\textbf{goða} \textit{mænn} oc \textbf{hælga}{]} \\
because that they have sincere/kind been towards all good.\textsc{acc.pl.str} man.\textsc{acc.pl} and holy.\textsc{acc.pl.str} \\
\glt `because they have been kind towards all good and holy men \\ (6r, col.a:18--19)'
\ex\label{ch8ex10b}
\gll  engan \textbf{visan} \textit{mæistar-ann}\\
no wise.\textsc{acc.sg.str} master-\textsc{def.acc.sg}\\
\glt `no wise master'  (4r, col.b:1)
\end{xlist}
\end{exe}

In (\ref{ch8ex10a}) both adjectives identify the referent and modify the \isi{noun} \emph{mænn} (i.e. they do not show any signs of semantic
temporariness or stage-level reading). Note also that I analyze both
adjectives as \isi{prenominal} adjectives (see Section \ref{ch8s4.2}). Example (\ref{ch8ex10b})
also shows a \isi{prenominal} strong \isi{adjective} that modifies the referent
directly. Being \isi{prenominal} and \isi{attributive}, these examples show that inherent predicatehood for strong adjectives seems not
to be strictly applicable to Old \ili{Norwegian}. However, I considered further
arguments made by \citet{Fischer00, Fischer01} for Old \ili{English} for a transfer discussion of
Old \ili{Norwegian} data, as she provides an extensive discussion on syntactic
\isi{variation} focusing on the \isi{adjective} position. Fischer also correlates
the predicatehood of strong adjectives to the observation that Old
\ili{English} adjectives are non-recursive, and due to this, not hierarchically
ordered in a correlating relationship (see \citealp{vanGelderenLohndal2008}
repeating the statement made by Fischer for Old \ili{Norwegian}; see however \citealp{Bech17} for examples of stacked adjectives in Old \ili{English} and \citealp{Bech19} for further discussions). It is true that stacked adjectives are
rare in the Old \ili{Norwegian} material. However, they do occur, as shown in
(\ref{ch8ex11}).\footnote{As \citet[15]{Bech17} notes in her study, however, the
  majority of examples found displaying this \isi{pattern} include
  \emph{margr} `many, numerous' as the first of the two adjectives.
  KoNoKs only displays one example of stacked adjectives, also including
  \emph{margr}, which is annotated as an \isi{adjective} in the \isi{corpus}
  material (following \citealp[142]{Haugen2001}; \citealp[71]{Nedoma2010}; \citealp{Cleasby1957}; 
\citealp{Zoëga1910}). However, \textit{margr} might be discussed further
  concerning its degree of adjectivity. The overlap of \textit{margr} with the category of
  quantifiers and its possible semantic and syntactic integration in this
  word class is likely to influence its strong tendency to appear as the
  first of two adjectives in stacked \isi{adjective} constructions. A further
  discussion of the membership of \emph{margr} in the \isi{adjective} or
  \isi{quantifier} class is an interesting topic, but will not be discussed further in this chapter.}

\begin{exe}
\ex\label{ch8ex11}
\begin{xlist}
\ex\label{ch8ex11a}Stacked weak adjectives\\
\gll Sa hinn \textbf{riki} \textbf{gamle} \textit{maðr}\\
\textsc{dem} \textsc{art.nom.sg} rich.\textsc{nom.sg.wk} old.\textsc{nom.sg.wk} man.\textsc{nom.sg}\\
\glt `the rich, old man' (Streng 2,282)
\ex\label{ch8ex11b}Stacked strong adjectives\\
\gll   hafðe [...] kallað \textbf{margha} \textbf{goða} \textit{hufðingia} \\
had [...] called many.\textsc{acc.pl.str} good.\textsc{acc.pl.str} leader.\textsc{acc.pl}\\
\glt `had [...] invited many good leaders' (36r, col.b:3)
\end{xlist}
\end{exe}

Moreover, Fischer's discussion (cf. \citeyear[257ff]{Fischer01}; see also \cite[260f]{Haumann03}) points out a mismatch between \isi{definiteness} and \isi{indefiniteness} in structures with a definite nominal expression and a strong \isi{adjective}, which indicates a \isi{predicative} status of strong adjectives. This  mismatch, showing a strong \isi{adjective}
and a \isi{possessive pronoun}, is exemplified for Old \ili{Norwegian} in (\ref{ch8ex12}).

\begin{exe}
\ex\label{ch8ex12}
\gll \textbf{sœmilect} \textit{nafn} \textbf{\textit{sitt}}. ~\textbf{gott} \textit{yfirlæti} oc \textbf{fagra} \textit{þionosto}.\\
honourable.\textsc{acc.sg.str} title.\textsc{acc.sg} his good.\textsc{acc.sg.str} repute.\textsc{acc.sg} and fair.\textsc{acc.sg.str} service.\textsc{acc.sg}\\
\glt `his honourable title, (his) good repute, and (his) fair service'\\(21r, col.b:6--7)
\end{exe}

Here, the nominal expressions are semantically/pragmatically definite
(by vir- tue of containing a \isi{possessive pronoun} anchoring them in the
discourse as defined entities); however, the adjectives signal that they are \isi{indefinite} (by virtue of the strong morphology of the \isi{adjective}) at the same time. Contextually, neither the nominal expressions nor the
properties of the adjectives of this example convey new information.
Note also that possessive pronouns arguably have definiteness-like
features but do not carry the feature {[}DEFINITE{]} yet (see also \citealp{Borjarsetal16}), and thus do not yet render the NP syntactically
definite. They are rather interpreted as anaphoric or cataphoric
deictics (see \citealp{Tiemann2023}). Furthermore, adjectives might add a
new property to an already given referent. \citet[257ff, 265ff]{Fischer01} argues that the strong \isi{adjective} in constructions like these
cannot be analyzed as a modifier of the head \isi{noun}, but must be analyzed
as a secondary predicate, e.g. a reduced relative. Structurally, the
example in (\ref{ch8ex12}), however, shows that the strong \isi{adjective} and the \isi{noun}
moved together in front of the possessive (pied-piping of the
\isi{adjective}, see also the structure given in (B)). This movement indicates a
stronger connection between \isi{adjective} and \isi{noun} than that given by
\isi{predication}.\footnote{Note that example (\ref{ch8ex12}) displays an enumeration,
  which might be a decisive factor for this word ordering.} In addition,
\citet[26]{Pfaff2019} notes for \ili{Old Icelandic}/Old \ili{Norse} that adjectives
following a \isi{possessive pronoun} generally seem to be strongly inflected, which is also true for the Old
\ili{Norwegian} material examined in this study (unlike modern \ili{Norwegian}).\footnote{The \isi{corpus} material
  also showed two examples of an alternative \isi{pattern} where a possessive
  precedes the sequence ART + A.WK, as in `nema ec skryði \textit{\textbf{mina}}
  hina \textbf{liotligo} \textit{asion}' `unless I adorn \textbf{\textit{my}} \textbf{terrible} \textit{appearance}') (43v, col.a:12--13).} Here, several factors seem to
influence the \isi{inflection} and position of the \isi{adjective}; the structure,
however, seems not to be of a \isi{predicative} nature.

Turning lastly back to Fischer's \citeyearpar{Fischer00, Fischer01} analysis for Old \ili{English} concerning the positioning of
adjectives which, according to her, is directly attributed to their
function as \isi{attributive} (=\isi{prenominal}) or \isi{predicative} (=\isi{postnominal}),
several examples have already demonstrated that strong adjectives in Old \ili{Norwegian}
occur in \isi{postnominal} position in an \isi{attributive} function. This seems to
be the general case for listings, as shown in (\ref{ch8ex13b}).

\begin{exe}
\ex\label{ch8ex13}Attributive use
\begin{xlist}
\ex\label{ch8ex13a}
\gll eða skilningar laus komi  i \textit{skola} \textbf{goðan }\\
or wit less come.\textsc{sbjv} in school.\textsc{acc.sg} good.\textsc{acc.sg.str}\\
\glt `or a simple-minded (person) would come/enter a good school'\\(17v, col.b:14--15)

\ex\label{ch8ex13b}
\gll \textit{Nalar} \textbf{margar} oc \textit{þræðr} \textbf{œrna}. eða \textit{sviptingar} \\
nail.\textsc{acc.pl} many.\textsc{acc.pl.str} and thread.\textsc{acc.pl} strong.\textsc{acc.pl.str} or cord.\textsc{acc.pl}\\
\glt `many nails, and strong threads or cords' (3v, col.a:10--11)
\end{xlist}
\end{exe}

What can be determined, however, is that while strong adjectives may
appear as \isi{predicative} adjectives (cf. also ex. (\ref{ch8ex14b}), with an example of a
\isi{prenominal} \isi{predicative} \isi{adjective} in a coordinated structure), weak
adjectives do not act in this function in any of the positions available
to them.
\pagebreak
\begin{exe}
\ex\label{ch8ex14}Predicative use
\begin{xlist}
\ex\label{ch8ex14a}
\gll þvi at af iðrottum væ{[}r{]}ðr \textit{maðr} \textbf{froðr} \\
this that of arts becomes man.\textsc{nom.sg} wise.\textsc{nom.sg.str}\\
\glt `because a man becomes wise through (the/a) arts (crafts/procession)'\\
(1v, col.b:20--21)
\ex\label{ch8ex14b}
\gll  þar sæm \textit{haf-it} er \textbf{diupt} oc þo \textbf{salltr} \textit{sær-inn} \\
there as ocean-\textsc{def.nom.sg} is deep.\textsc{nom.sg.str} and yet salty.\textsc{nom.sg.str} sea-\textsc{def.nom.sg}\\
\glt `there where the ocean is deep and yet the sea salty' (12r, col.b:17--19)
\end{xlist}
\end{exe}

\begin{sloppypar}
According to these findings, no clear generalisation about the position
of strong adjectives correlating with their function can be made.
Other factors might be more decisive when it comes to the syntactic
\isi{variation} seen within the Old \ili{Norwegian} extended nominal projection. The
weak \isi{inflection}, on the other hand, has a very restricted distribution:
weakly inflected adjectives only occur attributively in overtly definite
marked NPs (\emph{Elsewhere Principle}: the strong \isi{inflection} appears
when the weak \isi{inflection} is not triggered by a c-commanding definite marker, see \cite[13]{Pfaff2019}; cf. also \citealp{Pfaff2015} for modern Icelandic). The morphology of the
\isi{adjective} thus restricts the possible functions, but it does not
determine the function in a strict 1:1 ratio (recall that the strong
form is the default form in all contexts). The examination so
far can be summarized as follows:
\end{sloppypar}

\begin{enumerate}
  \item  Old \ili{Norwegian} does not yet have a dedicated (in)definite element,
neither free nor bound. More specifically, the feature {[}DEFINITE{]}
existed in Old \ili{Norwegian} but did not have obligatory exponence.

\item The opposition of strong versus weak adjectives and their position
relative to N does not seem to be a strict one in Old \ili{Norwegian}.
(However, the appearance of weak adjectives in \isi{postnominal} position is
more restricted than for strong adjectives.)

\item The \isi{article} ART acts to license the weak AP (cf. \citealp{Pfaff2019}; see also \citealp[8]{PerridonSleeman2011}; \citealp[7f]{StrohWollin2009}) as the head of
(exactly one) AP.

\item Neither the morphology of the \isi{adjective} nor the presence or absence
of overt (in)\isi{definiteness} markers seems to solely determine the position
of the \isi{adjective} in the surface structure.
\end{enumerate}




\subsection{Information structure}\label{ch8s3.3}
I will now turn to pragmatic influence on word ordering, and with this
to an information-structural approach. This postulates that utterances
are structured according to the transmission goals of a communicative
situation, allowing for \isi{variation} on various linguistic levels to reach
an optimal informational exchange (cf. \citealp{Halliday1967}; \citealp{chafe1976givenness}; \citealp{Lambrecht1994}; \citealp{Büring2005}; \citealp{Caruso2016}). Assuming one underlying base
structure (\emph{Universal Base Hypothesis}), the positioning of
constituents then reflects their informational content in the given
structure (cf. \citeauthor{Rizzi1997}'s \citeyear{Rizzi1997} split CP hypothesis). Within the clause,
positions for \emph{topic}, \emph{focus} and \emph{contrast} are
generated in the left periphery of CP (see among others, \citealp{Petrova2009}; \citealp{Hróarsdóttir2009}). Scholars such as \citet{Giusti1996} and \citet{IsacKirk2008} have further suggested that the nominal domain, too, encodes
discourse-related notions, mirroring the structure of the CP (in the
same hierarchical order: Top\textgreater Foc; see \citealp[31]{Caruso2016} for
arguments on the order of TopP and FocP in the nominal projection; cf.
also \citealp{AbohCorverDyakonovaKoppen2010} for a summary of work on \isi{information structure}
within the NP). \citet[5]{Caruso2015} further notes that ``{[}t{]}he most
prominent discourse-related notions associated with \isi{noun} phrases,
namely (in)\isi{definiteness} and specificity, are assumed to be realized
within the nominal left periphery'' (\emph{split
DP} \emph{hypothesis}; cf. also \citealp{IhsanePuskás2001}; \citealp{Laenzlinger2005}; \citealp{Giusti2005,Giusti2012}; \citealp{Haegeman2004}).\footnote{For an account arguing against \emph{topic}
  and \emph{focus} inside DP, see \cite{Szendröi2010}.} These encode
(non)\isi{familiarity} through the choice of determiners that mark the \isi{noun} as
either identifiable or non-identifiable for the addressee (see \citealp[783]{AbohCorverDyakonovaKoppen2010}). With this, NP-internal movement operations that
correlate e.g. with \emph{focus} readings are explained in the same
manner on a phrasal level as on a clausal level (cf. \citealp{Giusti2006}:
AP--to--SpecDP versus A--to--D movement). \citet[28]{Caruso2016} summarizes
the various domains of the \isi{noun} phrase (parallel to the clause) under
the following domains: 1) NP-shells; 2) an inflectional domain; and 3)
the left periphery. The initial position is associated with
information-prominent and contrastive elements.

Discussions within this approach often focus on the nominal constituent
(e.g. \citealp[142]{IsacKirk2008}). However, as on the clausal level, any
constituent can be targeted by information-structural interpretations
(cf. \citealp{Truswell2004} who argues that ``standard theories'' of \emph{focus}
should be extended to adjectives; see also \citealp[103f]{Hardarson2017} for
information-structurally triggered movement of adjectives within the
extended NP for (modern) Icelandic). \citet[92]{Harries2014} notes for Old
\ili{Norse} that elements in a fronted position (FOC in her framework) carry
information ``which is more prominent discoursally than the \isi{noun} itself,
{[}and{]} that the information which follows the \isi{noun} carries
information which is discoursally less significant (backgrounded)'',
following the same basic assumptions for the NP in Old \ili{Norse} as assumed
in this study. Furthermore, with a split of \emph{focus} into
\emph{presentational} and \emph{contrastive focus} (see \citealp{chafe1976givenness}; \citealp{KatzSelkirk2011}), \emph{focus} and activation status are considered
distinct concepts. However, the constituent in \emph{focus} is
universally marked by prosodic prominence, i.e. by carrying main stress
(or pitch accent). Moving elements into a designated fronted slot above
the \isi{noun} (in KontrP: \emph{contrastive focus}) is a strategy exploited
in languages that show syntactic \isi{variation}, and a strategy that works
like focalizing an element by adding pitch accent (cf. \citealp{CorvervanKoppen2009}). Adopting this, I expand the Old \ili{Norwegian} NP with a
full-fledged left periphery and designated slots for \emph{topic},
\emph{focus} and \emph{contrast} (note that a DP in Old \ili{Norwegian} is a
\isi{demonstrative} phrase, as stated in Section \ref{ch8s3.1.1}). It has been mentioned, however,
that the coding of adjectives for e.g. saliency and whether they are
presupposed or not, or whether such an element carries relational
\emph{focus} (more informative with respect to the \isi{noun}), is
unfortunately difficult (see \citealp[13]{vanGelderenLohndal2008}; \citealp[259f]{Allen12}). This is a more general issue of assigning
information-structural features to the word class of adjectives, and 
I tackle this problem by introducing the concept of
\emph{essentiality}, focusing on information flow within a given discourse.

\subsection{Information status and essentiality}\label{ch8s3.4}
Information status can be analyzed as a binary \emph{given} (unfocused)
-- \emph{new} (focused) distinction (see e.g. \citealp{Prince1981}; \citealp{GundelHedbergZacharski1993}; \citealp{Birner2006}). \citet[256]{Fischer06} uses the terms
\emph{given}/\emph{new} in the sense of ``\emph{salience}, i.e. which
elements add least and most to the advancing process of communication''
(\citealp[25]{Bech19}; see also \citealp[122]{FischervanderWurff2006}). Through
the concept of \emph{essentiality}, the information status of adjectives
can be assigned in a more effective way, following a strict annotative
evaluation based on the immediately preceding and continuing context.


\subsubsection{Non-essential}\label{ch8s3.4.1}
I start with what I term \emph{non}-\emph{essential} adjectives. In
relation to the preceding context, these adjectives convey
information that is known in the discourse situation (=old/given
information; directly mentioned in the preceding discourse or
contextually known/active in the knowledge stock of the interlocutors
due to world/situational knowledge that can be assumed for the specific
cultural sphere of the text,\footnote{\textit{Konungs skuggsjá} `The king's mirror' is written in
  a courtly context in a Christianized society. As such,  general
  knowledge about e.g. church order and masses can be assumed to be
  present in the knowledge stock of the interlocutors. Such mentions
  are annotated as \emph{accessible} in the \isi{corpus} material of KoNoKs,
  either through common ground or situational knowledge (cf. \citealp[94ff]{Tiemann2023}).} as in \textit{\textbf{holy}} \textit{Mary}, \textit{\textbf{almighty}} \textit{God}, and are not
necessary to identify the \isi{noun}'s referent). In relation to the continuing
context, non-\isi{essential} adjectives do not carry information necessary for the interpretation of the following sequences. Thus, an omission of the
\isi{adjective} does not lead to a change in the reference interpretation,
nor to difficulties interpreting subsequent information.\footnote{This
  comes close to what has been discussed in the literature on adjectives
  under the notion (non-)restrictivity (cf. e.g. \citealp{Bolinger1967}; \citealp{LarsonMarušič2004}; \citealp{Truswell2005}; \citealp{Umbach2006}; \citealp{Demonte2008}; \citealp{Cinque2010}; \citealp{Pfaff2015}). Furthermore, (non-)\isi{referentiality} relates directly to the
  referent (referential=needed to understand the reference;
  non-referential=additional information for the encoding of the
  reference), while \emph{essentiality} additionally relates to the
  discourse development (\isi{essential}=needed for the understanding of the
  developing discourse; non-\isi{essential}=information that does not add any
  informational value to the proceeding discourse).} The developing
discourse is in these cases not based on the property carried by the
\isi{adjective}, as exemplified by the modern examples in (\ref{ch8ex15}).

\begin{exe}
\ex\label{ch8ex15}Non-\isi{essential}
\begin{xlist}
\ex\label{ch8ex15a} My cat may seem arrogant from time to time. But this is not surprising, as cats are known to have this \textbf{arrogant} \textit{nature} in general. I still treat her as a queen.
\ex\label{ch8ex15b} I was eating a lot of candy at Christmas. The \textbf{sweet} \textit{treats} are just for this time.
\end{xlist}
\end{exe}

The second mentioning of the \isi{adjective} \emph{arrogant} in (\ref{ch8ex15a}) is a
repetition of the immediately preceding utterance and can thus be
omitted from the NP without changing the meaning of the utterance
containing the \isi{noun} \emph{nature} in any direction or to any degree. The
reference of \emph{nature} still refers to the arrogant character of
cats mentioned here. Also, the following clause, referring back to the
property given by the \isi{adjective}, can be understood in this context. The
\isi{adjective} \emph{sweet} in (\ref{ch8ex15b}) can be inferred by the earlier mentioned
\emph{candy} and the general knowledge of candy being sweet and is thus
not necessary for the correct encoding of the sentence it appears in and
for the interpretation of the reference of \emph{treats}. Omitting this
\isi{adjective} does not lead to a change in meaning. In other words,
\emph{non}-\emph{essential} adjectives carry {active} information not
needed for the understanding/interpretation of the \isi{noun}'s referent, the
immediate phrase or the further development of the discourse referring
to the specific entity. In \isi{contrast} to the notion of
\emph{non}-\emph{restrictivity}, which often is described as adding
some kind of (unnecessary) ``extra" information with no difference in the
denotation of the \isi{noun} alone, \emph{non}-\emph{essentiality} does not
describe ``extra" information, but active information through discourse
development. Informationally speaking, the \isi{adjective} gives old
information to a new or given referent. An Old \ili{Norwegian} example is
given in (\ref{ch8ex16}).

\begin{exe}
\ex\label{ch8ex16}Non-\isi{essential}\\
En þo er sa æinn lutr ænn æptir er geta ma æf syniz firir \textbf{gamans} \textit{saker}
oc skemtanar. \textit{Gamans maðr} æinn var í lande þvi mioc longu oc þo var
hann cristinn oc var sa maðr kallaðr Klefsan at nafni Ðat var mælt um
þænn mann at ængi maðr munnde sa væʀa er hann sæ at hann munnde æi lægia
gera mæð sinum \textbf{gamansamlegum} \textit{orðum} oc þo \textbf{lygiligum}. oc þo at
maðr væri ryggr í hug sinum þa er þat sagt at maðr munnde æi latrs
binndaz æf þeir han heyrðe þæssa mannz rœðu.\\

\noindent `But there is yet one thing that one can learn, if you wish, for the
\textit{sake of amusement} and entertainment. A (certain) \textit{funny man} was in
this land very long/for a long time and also, he was a Christian, and
this man was called Klefsan by name. It was told of this man that
(there) would be no man, when he saw (Klefsan), that he would not be made to laugh at
his \textbf{amusing }and yet \textbf{fantasized} \textit{words}/\textit{speech}/\textit{stories}.
Even though a man would be \isi{heavy} in his mind, then it is said that a man
could not restrain (his laughter) when he heard that man talk.' (9r, col.a:8--19)
\end{exe}

\begin{sloppypar}
The \isi{adjective} \emph{gamansamlegum} `amusing' can easily be omitted from the phrase
without creating any problems for the hearer in interpreting the
words/stories spoken by Klefsan as `funny'.
\emph{Gamansamlegum} is a direct repetition of a property introduced
through the first mention of \emph{gamans} in connection with the following
discussion and the referent Klefsan. \emph{Gamans} thus sets the
frame\footnote{The theory of schemes and frames deals with the
  processing of entities that are in a firm relation to each other.
  Elements that a scheme contains can open a scheme by simply being
  mentioned. As soon as the scheme is active, the other elements
  contained are treated like slots that want to be filled. If a slot is
  not saturated, the reader will fill it by inference (the typical
  information will be supposed). As such, the mention of “joke”, for example, sets the scene for the interpretation of possible following information, such as “laughter”, “tears”, “funny”, etc. Though this information is not mentioned in the preceding
context, its status is not \textit{new}, nor is it \textit{given}, but rather a relation of its own, i.e. \textit{bridging} (within a binary division, analyzed as \textit{given}).} for the following
discourse, while \emph{gamansamlegum}
presents neither a new property nor necessary information for the
correct interpretation of the \isi{noun} \emph{orðum} in the context of
laughing people\emph{.} The second \isi{adjective}, \emph{lygiligum} `fantasized', is
annotated as part of the same scheme as \emph{gamansamlegum} (annotated
under frames) in the \isi{corpus} material and with this as an active part of
the connotation (=non-\isi{essential}). The \emph{non}-\emph{essential}
nature of an \isi{adjective} is thus evaluated on the basis of the preceding
discourse (given/known features carried by the \isi{adjective}), and on its
informational value for the continuation of the discourse (referent
identification needed for the interpretation of the utterance or not;
see \cite[94ff]{Tiemann2023} for a more detailed account of the \isi{corpus} annotation).
Consequently, the \isi{adjective} is assumed not to carry any (prosodic)
\isi{emphasis} and \emph{non}-\emph{essential} adjectives are thus generally
de-focalized (the same is true for the following \isi{adjective}
\emph{lygiligum}).
\end{sloppypar}



\subsubsection{Essential by context}\label{ch8s3.4.2}
\emph{Essential} adjectives, on the other hand, are those which are
assumed to be contextually emphasized, used in contexts where the
\isi{adjective} cannot be omitted without a change in the interpretation of
the modified \isi{noun}'s referent or without causing encoding problems for
what follows. They carry information needed for the identification of
the modified entity and the contextual interpretation of the developing
discourse. This makes the \isi{adjective} a key element of the informational
flow. In this sense, it is more prominent within the given discourse
than the \isi{noun}, might carry \isi{emphasis}, and is by these means focalized. This is
the case when ``the \isi{noun} represents information which does not
differ from the presupposition'' \citep[98]{Harries2014} -- then it is the
\isi{adjective} that is more informative. The \isi{adjective} is then
\emph{essential} \emph{by context} and displays information that cannot
be assumed to be active in the knowledge stock of the interlocutors (it
has not been mentioned in the immediate previous context, nor does it
belong to/is it annotated as part of an active scheme under the frames tag). A modern example is
given in (\ref{ch8ex17}).

\begin{exe}
\ex\label{ch8ex17}Essential by context (\textasciitilde presentational focus)\\
The \textbf{old} \textit{man} had difficulties doing squats. I am impressed, though, that he started working out again so late in his life.
\end{exe}

The core information in this utterance is given through the \isi{adjective}
\emph{old}. It identifies a specific referent; however, its property
also describes a condition that becomes relevant information for the
developing discourse and signals how the common ground develops. If the
\isi{adjective} \emph{old} were omitted from the phrase, the
information in the first clause would change in its meaning (to the
general reference of \emph{man} and with no clue as to why he appears to have 
difficulties doing squats, which here is connected to the advanced age), and
the later phrase \emph{so late in his life} would be difficult to
comprehend cognitively. The semantic component expressed by the
\isi{adjective} is thus the crucial element of the assertion and opens a new
scheme (`an old man'). It can then be identified as the focused
component of the phrase. These adjectives are tagged as \emph{new} and
are within the focus domain in the \isi{corpus} material of KoNoKs.

Stating that \emph{focus} is expressed at a designated position in the
left periphery of the NP (cf. Section \ref{ch8s3.3}) entails movement inside the
NP, making the moved constituent the part of the phrase that carries the main information or assertion. Both stress and movement to a designated position can put focus on a constituent (highlighting system; see also \citealp{Truswell2005} for the syntactic
parallel between the clausal left-periphery and the DP, and for focus
movement inside the DP). For Old \ili{Norwegian}, the assumed unmarked
(de-focalized) position for the \isi{adjective} is \isi{prenominal}, thus I assume
that postpositioning puts \isi{emphasis} on the \isi{adjective}
(\emph{presentational focus} mentioned above; following the hierarchy
Top\textgreater Foc), as shown in (\ref{ch8ex18}).

\begin{exe}
\ex\label{ch8ex18} Essential by context -- \isi{postnominal} position\\
\gll at hvæʀ þæiʀa systra hafa fullan rett i domi æpter \textit{tali}
\textbf{retto}. \\
that each \textsc{dem} sisters have full right in decision after \isi{weight}.\textsc{dat.sg} right.\textsc{dat.sg.str}\\
\glt `that each of the sisters has full right in the decision according to their respective position (concerning the particular case discussed)'\\ (42r, col.a:8--9)
\end{exe}

However, positional \isi{variation} to signal a focused \isi{adjective} seems to
be a weakened strategy already in Old \ili{Norwegian}, as the \isi{prenominal}
position is the predominant order in all contexts. We thus most often
find adjectives that are \emph{\isi{essential} by context} (focused) already in
\isi{prenominal} position, as shown in (\ref{ch8ex19}).

\begin{exe}
\ex\label{ch8ex19}Essential by context -- \isi{prenominal} position\\
Ða er þat ænn æitt sæla kyn ænn smæst er skemmingr heiter oc ero
þeir æigi længri at væxti en tvæggia alna oc er þat mæð
\textbf{unndarlegre} \textit{natturo} þviat sva er fra sagt at hann fæʀr unnder
þa isa er flater ero 
\glt `Then there is still one kind of seal yet the smallest (kind), which is
called the ``shori seal" and they are in growth/length no longer than
two ells; and it has a \textbf{marvellous} \textit{nature}; for it is said that
he (the seal) goes under that ice (masses) which are flat.' (10v, col.b:11--18)
\end{exe}

The prominence of the \isi{adjective} \emph{unndarlegre} is relatively easy to
spot due to the fact that the continuing subclause adds additional
information to the property of the \isi{adjective}. The adjectival feature
presents key information for the developing discourse, needed to
interpret the reference in context of what follows.


\subsubsection{Essential by contrast}\label{ch8s3.4.3}
Putting \isi{essential} adjectives against non-\isi{essential} adjectives, I
also consider adjectives that display known information (tagged as
\emph{anchoring} and within an already existing scheme), but are needed
to clearly identify the \isi{noun}'s referent within the discourse or to
clearly identify the transferred core information under the notion of
\emph{essential} (this overlaps with the common definitions of
restrictive adjectives). These adjectives show effects of \isi{contrastivity}
-- something that is less problematic to assign to adjectives than
\emph{presentational focus}. Within the field of \isi{information structure},
\emph{contrast} has been assumed to be an autonomous notion (see \citealp{Molnár2002}), but it often co-occurs with other information-structural
categories, i.e. \emph{topic} and \emph{focus} (see e.g. \citealp{Repp2010}). In
modern languages, \emph{contrast} is, in addition to or instead of \isi{word order} \isi{variation},
connected to phonological rules (e.g. a pause between the contrasted
modifier and the rest of the NP; cf. \citealp[267f]{Rijkhoff2002} for \isi{adjective}
displacement in \ili{Turkish} and \ili{Hungarian}). It is not part of this chapter to
enter into a discussion on the concept of \emph{contrast} in much detail.
However, as \emph{contrast} puts \isi{emphasis} on an element, these elements
are not omissible, even though they convey known information within the
discourse (the domain of \isi{contrast} is defined as given). With \isi{contrastivity}, an
entity needs to be distinct/unambiguous.\footnote{Note that \isi{prenominal}
  adjectives characterized by focalization through \emph{contrast} seem
  to have a high pragmatic affinity with the {[}SPECIFIC{]} usages of
  NPs.} The \isi{adjective} in these constructions is thus most often
accompanied by the \isi{definiteness} marker \emph{hinn}, marking
\isi{referentiality} for the \isi{adjective} (in \isi{contrast} to e.g. \emph{sá} which
marks deictic entities). Thus, we expect to find mainly weak adjectives
in these structures in Old \ili{Norwegian}. I assume then that these
emphasized adjectival elements in A--N order are assigned a feature
{[}CONTRAST{]}, marking the \isi{adjective} \emph{essential} (focused) for the
identification of the referent within the given discourse. A modern
example is presented in (\ref{ch8ex20}).

\begin{exe}
\ex\label{ch8ex20}Essential by \isi{contrast} (\textasciitilde contrastive focus)\\
They had a lot of bikes at the store in various colours (including gray, yellow, blue, and brown). I decided to buy the \textbf{yellow} \textit{bike} since you can see it better in the dark.
\end{exe}

The \isi{adjective} \emph{yellow} is in this context of another nature than
the adjectives described as \emph{\isi{essential} by context}; however, it is
still important for the correct encoding of the information transferred,
as it presents the relevant property to correctly identify the referent.
This example shows that the \isi{adjective} can either be annotated as
\emph{given} by a direct previous mention (if the part in brackets is
included) or by its activation status, triggered by the scheme opened by
the \isi{noun} \emph{colours}. For the actual analysis of the information
status of adjectives, textual context is thus most important. An Old
\ili{Norwegian} example is given in (\ref{ch8ex21}).

\begin{exe}
\ex\label{ch8ex21}Essential by \isi{contrast}\\
Biorn er þar oc a þvi lannde oc er hvitr oc ætla mænn at hann fœðez a
þvi lannde þvi at hann hæfir alt aðra natturu en \textbf{svarter}
\textit{birnir} er i skogum ganga þeir væiða at ser ross oc naut oc annat bu oc
fœðaz við þat En \textbf{\textit{hinn}} \textbf{hviti} \textit{biorninn} er a Grœnalannde er þa
fæʀr hann mæst í hafi ut a ísum oc væiðer þar at ser bæðe sæla oc hvala
oc lifir við þat \\

\noindent `(A) bear is there, too, in that land, and it is white, and men/people
think that he is born in this land, for he has a completely different
nature than \textbf{black} \textit{bears} that roam in the forests. They hunt
horse(s), and cattle, and other beasts and feed on it. But \textit{\textbf{the}}
\textbf{white} \textit{bear} which is in Greenland, he goes/wanders mostly out
on the ice in the sea, and hunts there himself both seals and whales and
lives on it.' (11v, col.b:7--16)
\end{exe}

In this example, the \isi{adjective} \emph{hviti} is needed to correctly
identify the current referent. The mentioning of \emph{bear} allows for easy
processing of entities that are in firm relation to each other, and the
entities  \emph{black} and \emph{white bears} are active in this
sense. A correct encoding of the utterances they appear in is, however,
dependent on the \isi{adjective}, putting \isi{emphasis} on these elements by
means of \isi{contrast}. The feature {[}CONTRAST{]} is often connected to the
movement of elements in Old \ili{Norwegian} 
(most often fronting; cf. e.g. \citealp{Demonte2008} for
movement of adjectives to a \isi{prenominal} position in \ili{Spanish}); however, the \isi{prenominal} position is considered the base
position of adjectives. Movement of the \isi{adjective} to a position
hierarchically higher up in the structure within the left periphery of
the nominal projection, triggered by the feature {[}CONTRAST{]}, thus
does not lead to a visible reordering in the surface structure.
Nevertheless, the prominence of the \isi{adjective} in the \isi{prenominal} position
can be structurally signalled by multiple \isi{definiteness} markers, as
contrastive readings naturally have an identificational function
(associated with \isi{definiteness}). These markers are not yet obligatory to
specifically single out a certain entity in Old \ili{Norwegian}, and thus,
their appearance puts special \isi{emphasis} on the phrase. Additionally, the
appearance of definite elements in a fronted position makes it clear
that \emph{focus} is a more complex phenomenon than  expressing
newsworthy information. It is better described as being more of a
highlighting device within the phrase.\footnote{See \citet[25]{Giusti2005},
  who proposes an association between ``multiple occurrences of determiners''
  and the split DP hypothesis.} It is important to note that
a contrastive reading in Old \ili{Norwegian} seems to be supported mainly by
the definite markers ART and DEF (also in double \isi{definiteness} constructions as in (\ref{ch8ex22a})), while demonstratives generally might
not point towards a \isi{contrast}, but appear as deictic elements in double
\isi{definiteness} constructions, cf. (\ref{ch8ex22b}).\footnote{Note that the example in
  (\ref{ch8ex22a}) also shows a \isi{pattern} containing both a \isi{demonstrative} and ART+DEF
  (`\textit{\textbf{þeir hiner}} \textbf{kalldu} \textit{vægir-\textbf{nir}}' `\textsc{dem} \textsc{art} cold zone-\textsc{def}'). In cases like these,
  the \isi{adjective} is still considered \emph{\isi{essential} by contrast} due to
  the appearance of the \isi{article} ART in the double \isi{definiteness}
  structure. In example (\ref{ch8ex22b}), the same reference occurs without ART (`\textit{\textbf{þeim}} \textbf{kallda} \textit{vægi-\textbf{num}}' `\textsc{dem} cold zone-\textsc{def}')
  here in a deictic function and with a \emph{non}-\emph{essential}
  reading of the \isi{adjective}.}

\begin{exe}
\ex\label{ch8ex22}Double \isi{definiteness}
\begin{xlist}
\ex\label{ch8ex22a}ART+DEF (\textasciitilde Essential reading)\footnote{The definite
  markers (double \isi{definiteness}; see Section \ref{ch8s3.1.1}, see also \citealp{Schwarzschild1999}; \citealp{Wagner2006}) in (\ref{ch8ex22a}) mark off
  diametrically opposed constituents (specific and definite, cf. \citealp[784]{AbohCorverDyakonovaKoppen2010}).}\\
\noindent Nu er þar allt byggiannda unnder þeim vægum er millum ero kulðans oc
brunans. ... en æpter hugþocca minum at ætla þa þyckir mer þat licazt at
\textbf{\textit{hinn}} \textbf{heiti} \textit{vægrinn} ligr or austri oc i væstr. mæð biugum ring
brænnanda vægar um kringðum allum iarðar bollum. En þat er þa licazt
moti þvi at þeir \textit{\textbf{hiner}} \textbf{kalldu} \textit{vægirnir} liggia á hinum yztum
siðum heimsens til norðrs oc suðrs \\
\noindent `Now is all built/occupied under these ways/zones which are between the
cold and the burning heat. {[}...{]} but in my opinion it seems likely
to me that \textit{\textbf{the}} \textbf{hot} \textit{zone} lies from east to west with a curved
ring (like) a burning way around the entire globe. And it is then likely
in return that \textbf{\textit{the}} \textbf{cold}\textit{ zones} lie on the outer edges of the
world to the north and south.' (12v/13r, col.b/a:20--4)
\ex\label{ch8ex22b}DEM+DEF (\textasciitilde Non-\isi{essential} reading)\\
\noindent En þat er þa licazt moti þvi at þeir hiner kalldu vægirnir
liggia á hinum yztum siðum heimsens til norðrs oc suðrs oc æf ec hæfi
þætta ætlat æptir rettri skipan þa er þat æi ulict at grœna land liggi
unnder \textit{\textbf{þeim}} \textbf{kallda} \textit{væginum} \\
\noindent `And it is then likely in return that the cold zones lie on
the outer edges of the world to the north and south; and if I have
thought this out correctly, then it is not unlikely that Greenland lies
under \textit{\textbf{this}} \textbf{cold} \textit{zone}.' (13r, col.a:2--8)
\end{xlist}
\end{exe}

As mentioned above, \emph{essentiality} \emph{by} \emph{contrast} is
expected to occur mainly with weak adjectives, as they are triggered by
the occurrence of the definite marker ART, and thus signal a specific
referent by default. However, strong adjectives
may also appear in this function within contrasted pairs in elliptic parallel
constructions, as shown in (\ref{ch8ex23}).\footnote{\citet[209]{Umbach2005} writes that
  ``{[}f{]}ollowing Krifka {[}(1999){]} contrastive topics must comply
  with a ``distinctiveness condition" requiring that they are subject to
  different {[}comment{]} predications.'' Another type of
  \emph{contrastive topic} is formed through parallel structures.
  According to \citet[1343]{Repp2010}, ``they are found for instance in
  coordinations with ellipsis, {[}and{]} ... display exactly those
  characteristics that have been suggested to be typical for \isi{contrast}:
  there is a restricted set of explicit, identifiable alternatives,
  given in the two conjuncts, which serve as the immediate context for
  each other''.}

\begin{exe}
\ex\label{ch8ex23}
\gll millim \textbf{illra} \textit{luta} oc \textbf{goðra} \\
between bad.\textsc{gen.pl.str} thing.\textsc{gen.pl} and good.\textsc{gen.pl.str}\\
\glt `between bad and good things' (40v, col.a:28--29)
\end{exe}

Adjectives analyzed as an \isi{essential} part of the phrase can thus be
divided into two subcategories: a) they convey information that goes
beyond the linguistic information of the referent \isi{noun}, making it the
prominent element of the phrase within the given context, and b) their
property narrows down the interpretation of the \isi{noun} in \isi{contrast} to
entities that are in a tight relation to each other within the given
discourse, preventing a misinterpretation (to various degrees) of the
information. If an \isi{adjective} is the \emph{focus}-candidate of the
phrase, the form of the \isi{adjective} correlates with the two types of
\emph{essentiality}. The vast majority of adjectives that are
\emph{\isi{essential} by contrast} appear in the weak form (with the exception
of occurrences of strong adjectives in elliptic constructions);
adjectives \emph{\isi{essential} by context} are usually strong (correlating
with a new feature connected to their \isi{indefinite} form). Assuming further
an unmarked \isi{prenominal} position for the \isi{adjective} in Old \ili{Norwegian}, this
position leaves room for the \emph{non-essential} function of the
\isi{adjective}, thus permitting the \isi{adjective} to be used for other pragmatic
and/or semantic purposes. The generalisation concerning an unmarked \isi{prenominal} position is, however, challenged by
a) the feature {[}CONTRAST{]} (see e.g. \citealp{Rizzi1997}), leading to
emphasized \isi{prenominal} adjectives as mentioned by \citet{Faarlund04}, as
well as by b) the beginning of fixed \isi{word order} (towards a strict
\isi{prenominal} position of \isi{attributive} adjectives).



\subsection{Prosodic weight}\label{ch8s3.5}
Another factor often considered within studies on syntactic \isi{variation} is
\isi{prosodic weight} (see e.g. \citealp{Hróarsdóttir2009}; \citealp{Hinterhölzl2009}; \citealp{HinterhölzlPetrova2018}), often connected to element length and
complex structures. Thus, \citet[45]{Bech19}, referring to \citet{Grabski17},
notes for Old \ili{English} ``that A--N--\emph{and}--A is the default \isi{pattern}
for {[}...{]} complex constructions, and {[}that this{]} relates
{[}...{]} to the tendency to avoid \isi{heavy} clusters of elements, as noted
by \citet{Mitchell85}''. This seems to be the case for Old \ili{Norwegian}, too, which disfavours a) stacked \isi{adjective} constructions, and b) a \isi{heavy}
constituent in a fronted position. Prosodic \isi{weight} has not been  the
centre of attention in studies on NP-internal \isi{syntax} in Old \ili{Norse}, but
its impact has been noticed. \citeauthor{Faarlund04} (\citeyear[71]{Faarlund04}; see also \citealp[e13f]{Borjarsetal16}) points out that an \isi{adjective} followed by a complement
always has to follow its head \isi{noun}, as in (\ref{ch8ex24}), marked by a \isi{combination} of bold and italics.

\begin{exe}
\ex\label{ch8ex24}
\gll þar fylgði \textit{segl} \textbf{stafat} \textit{\textbf{með}} \textit{\textbf{vendi}} \\
there followed sail.\textsc{nom.sg} striped.\textsc{nom.sg.str} with stripe.\textsc{dat.sg}\\
\glt `A sail striped with stripes came with it.' (Hkr II.244.9, \citealp[71]{Faarlund04})
\end{exe}

This clearly shows an effect of \isi{prosodic weight}. A brief search in
the \isi{corpus} material of KoNoKs revealed the same tendency  described by \citet[71]{Faarlund04}. For the following analysis of \isi{prosodic weight} and
its correlation with other factors, I left out split constructions (Corpus B3, cf. Figure \ref{ch8ft1}) from the examination. For the analysis of \isi{prosodic weight}, I
first examined the relation between the syllable count for APs (measured
from nucleus to nucleus) and their position relative to N
(pre-/\isi{postnominal}). I only considered the number of syllables of the actual \isi{adjective}, leaving ART out of the calculation, as it is an
additional structural element. The adjectives were divided into two
groups: light APs (with 1--3 syllables) and \isi{heavy} APs (with 4--6
syllables), as shown in (\ref{ch8ex25}). Table \ref{ch8t5} shows the results of this analysis.

\begin{exe}
\ex\label{ch8ex25}Syllable division
\begin{xlist}
\ex\label{ch8ex25a}Light adjectives\\
\gll i \textbf{heiðnum} \textit{lonndum} \\
in heathen.\textsc{dat.pl.str} land.\textsc{dat.pl}\\
\glt `in heathen lands' (2r, col.a:20)
\ex\label{ch8ex25b}Heavy adjectives\\
\gll unnder \textbf{þyccskqvaðum} \textit{hialmi} \\
under cloudy.\textsc{dat.sg.str} helmet.\textsc{dat.sg}\\
\glt `under (the/a) cloudy helmet' (15v, col.b:19--20)
\end{xlist}
\end{exe}

 

\begin{table}[t]
% \begin{tabular}{cc|cc|cc}
% \lsptoprule
% \multicolumn{2}{c|}{Light APs (1--3 syllables)}  & \multicolumn{2}{c|}{Heavy APs (4--6 syllables)}  &  \multicolumn{1}{c|}{Total AN} &   Total NA \\ \hline
% \multicolumn{1}{c|}{AN} &  \multicolumn{1}{c|}{NA}& \multicolumn{1}{c|}{AN} & \multicolumn{1}{c|}{NA} & \multicolumn{1}{c|}{\multirow{2}{*}{777 (92.5\%) }} & \multirow{2}{*}{63 (7.5\%)} \\
% \multicolumn{1}{c|}{712 (92.1\%)} & \multicolumn{1}{c|}{61 (7.9\%)} & \multicolumn{1}{c|}{65 (97\%)} & \multicolumn{1}{c|}{2 (2.9\%)} & \multicolumn{1}{l|}{}    &    \\ \hline
% \multicolumn{2}{c|}{Total: 773}    & \multicolumn{2}{c|}{Total: 67}    & \multicolumn{2}{c}{Total: 840}  \\
% \lspbottomrule
% \end{tabular}
\caption{Order distribution of adjectives with regard to weight}\label{ch8t5}

\begin{tabular}{l rr  rr r}
\lsptoprule
& \multicolumn{2}{c}{A--N} & \multicolumn{2}{c}{N--A}\\
\cmidrule(lr){2-3}\cmidrule(lr){4-5}
                             &  n  & \%   &  n  & \% & total\\
\midrule
  Light APs (1--3 syllables) & 712 & 92.1 & 61 & 7.9 & 773\\
  Heavy APs (4--6 syllables)~~ & 65  & 97.0 & 2  & 2.9 & 67\\
  \midrule
                             & 777 &      & 63 &     & 840\\
\lspbottomrule
\end{tabular}

\end{table}

As already shown in Figure \ref{ch8ft1}, there are considerably fewer examples of
\isi{postnominal} than \isi{prenominal} adjectives in the
\isi{corpus} material. For both light and \isi{heavy} APs, the results summarized
in Table \ref{ch8t5} show that a \isi{prenominal} position of the \isi{adjective} is clearly
preferred. Furthermore, there are fewer \isi{heavy} than light
adjectives in \isi{postnominal} position. However, as there are considerably
fewer \isi{heavy} adjectives in the \isi{corpus} material in general, no conclusive
statements can be drawn from this observation. An example of a \isi{heavy} AP
in \isi{postnominal} position showing the expected correlation between
\isi{prosodic weight} and element positioning within the phrase is presented
in (\ref{ch8ex26}). However, following the predominant order, \isi{heavy} adjectives are
most likely already placed in a \isi{prenominal} position, as in (\ref{ch8ex27}).

\begin{exe}
\ex\label{ch8ex26}Heavy \isi{postnominal} modifier\\
\gll þrir \textit{vægir} \textbf{torfœrileger}~ \\
three way.\textsc{nom.pl} difficult.to.cross.\textsc{nom.pl.str}\\
\glt `three ways that are difficult to cross' (14v, col.a:24--25)

\ex\label{ch8ex27}Heavy \isi{prenominal} modifier\\
\gll sæm mæð \textbf{ottasamlegre} \textit{vorn} \\
as with terrifying.\textsc{dat.sg.str} defence.\textsc{dat.sg.str}\\
\glt `as with terrifying defence' (15v, col.b:14--15)
\end{exe}

It thus seems that there are no \isi{weight} effects, or at least that
syllable count does not play a role (anymore?). In a second step, I
analyzed the correlation between \isi{prosodic weight} and the form of the
\isi{adjective} (i.e. weak/strong). The results are shown in Table \ref{ch8t6}. As
mentioned above, ART is not part of the calculation, and neither is the
\isi{split construction}.

\begin{table}[t]
% \begin{tabularx}{.999\textwidth}{l|cc|cc|cc}
% \lsptoprule
% \multirow{2}{*}{} & \multicolumn{2}{c|}{Light APs} & \multicolumn{2}{c|}{Heavy APs}& \multicolumn{1}{c|}{Total AN} & \multicolumn{1}{c}{Total NA} \\
% & \multicolumn{2}{c|}{(1--3 syllables)} & \multicolumn{2}{c|}{(4--6 syllables)} & \multicolumn{1}{c|}{}  & \\ \hline
%  & \multicolumn{1}{c|}{AN} &NA& \multicolumn{1}{c|}{AN} & NA & \multicolumn{1}{c|}{} &  \\ \hline
% weak& \multicolumn{1}{c|}{51 (6.6\%)} & 4 (0.5\%) & \multicolumn{1}{c|}{0 (--)} & 0 (--) & \multicolumn{1}{c|}{51 \textbar \space 55 } &  4 \textbar \space 55 \\
% & \multicolumn{1}{c|}{} &  & \multicolumn{1}{c|}{} &  & \multicolumn{1}{c|}{ (92.7\%)} &   (7.3\%)\\
% strong & \multicolumn{1}{c|}{661 (85.5\%)} & 57 (7.4\%) & \multicolumn{1}{c|}{65 (97\%)} & 2 (2.9\%) & \multicolumn{1}{c|}{726 \textbar \space 785  } & 59 \textbar \space 785  \\
%  & \multicolumn{1}{c|}{} &  & \multicolumn{1}{c|}{} &  & \multicolumn{1}{c|}{  (92.5\%)} &  (7.5\%) \\ \hline
% & \multicolumn{2}{c|}{Total: 773}    & \multicolumn{2}{c|}{Total: 67}    & \multicolumn{2}{c}{ Total: 840}    \\
% \lspbottomrule
% \end{tabularx}

\begin{tabularx}{\textwidth}{Q rrrr rrrr}
\lsptoprule
& \multicolumn{4}{c}{weak} & \multicolumn{4}{c}{strong}\\\cmidrule(lr){2-5}\cmidrule(lr){6-9}
& \multicolumn{2}{c}{A--N} & \multicolumn{2}{c}{N--A}& \multicolumn{2}{c}{A--N} & \multicolumn{2}{c}{N--A}\\
\cmidrule(lr){2-3}\cmidrule(lr){4-5}\cmidrule(lr){6-7}\cmidrule(lr){8-9}
                             &  n  & \%   &  n  & \% &  n  & \%   &  n  & \% \\
\midrule
  Light APs\newline \mbox{(1--3 syllables)} & 51 & 6.6 & 4 & 0.5& 661 & 85.5 & 57 & 7.4  \\
  \tablevspace
  Heavy APs\newline \mbox{(4--6 syllables)} &  0 & 0   & 0  & 0 &  65 & 97.0 & 2  & 2.9  \\
  \midrule
  Total                     &51/55& 92.7&4/55&7.3&726/785&92.5&59/785&7.5\\
\lspbottomrule
\end{tabularx}

\caption{Order distribution of adjectives with regard to adjective form and weight (syllable division)}\label{ch8t6}
\end{table}

In addition to the general preference for a \isi{prenominal} position for all
adjectives, the division into weak and strong adjectives in correlation
with \isi{prosodic weight} shows that if an \isi{adjective} appears postnominally,
it is most likely strong (an effect of morphology correlating with
positioning is visible; cf. Section \ref{ch8s3.1}). The analyzed \isi{corpus} material
did not show any examples of \isi{heavy} weak adjectives. Therefore, no
further statements about the distribution of \isi{heavy} adjectives can be
made. It seems that the only cases where a clear \isi{weight} effect can be
described are those in which the language turns to parallel structures
instead of stacking adjectives or where the \isi{adjective} itself is further
modified (see also Bech's \citeyear[4]{Bech17} general overview of ordering
possibilities for adjectives in the Old \ili{Norwegian} NP; cf. (\ref{ch8ex11}) above).
However, in these constructions as well, \isi{prenominal} adjectives (\isi{prenominal}
position/pre-\emph{pro}, see Section \ref{ch8s4}) are relatively
\isi{heavy}.\footnote{See e.g. (\ref{ch8ex1c}). Note also that the \isi{adjective}
  \emph{\textbf{astsamlegan}} in this example (`\textbf{astsamlegan} \textit{foður} oc \textbf{gofgan}', `a \textbf{loving }and \textbf{renowned} \textit{father}') is already a relatively
  \isi{heavy} \isi{adjective}, appearing in \isi{prenominal} position.} Stacking is still
largely dispreferred in Old \ili{Norwegian}, but \isi{prenominal} position of two
adjectives including a coordinator (elliptic case of the first conjunct:
{[} {[}AP\textsubscript{1} \emph{pro}\textsubscript{k}{]} {[}\&
{[}AP\textsubscript{2} NP\textsubscript{k}{]}{]} {]}, see Section \ref{ch8s4}) already seems more acceptable, as exemplified in (\ref{ch8ex28}).

\begin{exe}
\ex\label{ch8ex28}
\begin{xlist}
\ex\label{ch8ex28a}
\gll \emph{mæð} {[}\textbf{varmum} oc \textbf{biartum} \textit{geislum}{]} \\
with warm.\textsc{dat.pl.str} and bright.\textsc{dat.pl.str} light.ray.\textsc{dat.pl}\\
\glt `with warm and bright beams' (4r, col.b:28--29)
\ex\label{ch8ex28b}
\gll \emph{mæð} {[}\textbf{goðum} oc \textbf{gnogum} \textit{svorum}{]} \\
with good.\textsc{dat.pl.str} and sufficient.\textsc{dat.pl.str} answer.\textsc{dat.pl}\\
\glt `with good and sufficient answers' (20r, col.b:3--4)
\end{xlist}
\end{exe}

Adjectives that do appear in \isi{postnominal} position might react to both informa- tion-structural constraints and \isi{prosodic weight}. It seems, however, that the information status of the \isi{adjective} (=\emph{\isi{essential}
by context}) is the decisive factor in these cases, as most of the
\isi{postnominal} adjectives are relatively light (see, however, (\ref{ch8ex26}) for a
\isi{heavy} \isi{postnominal} \isi{adjective}).



\section{Structure and movement}\label{ch8s4}\largerpage
As seen in our discussion on \emph{essentiality}, \isi{word order} \isi{variation}
inside the NP is explained with reference to discourse-relations (see
\citealp{Truswell2005}; \citealp{Laenzlinger2005}). The different \isi{word order} patterns are
then the result of movement inside the extended NP including a complex
left periphery that sorts out the landing sites for the moved elements
(cf. \citealp{Giusti2005} for \ili{Romance}). The movement of elements into the left
periphery is triggered by the interpretive features {[}TOPIC{]},
{[}FOCUS{]} and {[}CONTRAST{]}. TopP hosts information that has been
pre-es\-tab\-lished in the discourse, such as nominal elements marked as
{[}SPECIFIC{]}.\footnote{Note that this feature does not collapse into
  one property with the feature {[}DEFINITE{]}.} Below TopP is a
projection FocP for focused (\emph{presentational focus}) elements,
and above TopP there is a projection KontrP, hosting contrasted
elements, mirroring the structure of the CP, as shown in (C).

\begin{exe}
\exi{(C)}\label{ch8exc}
KontrP {$>>$}  TopP $>>$ FocP $>>$  ...
\end{exe}

As for specificity (which has not been discussed in any depth in this
chapter), it is assumed that specific nouns move into the NP-internal
topic position, while non-specific nouns may move into a focus
position. \citet[61f]{Harries2014} further notes that specificity in Old
\ili{Norse} was marked on the \isi{adjective}, and ``the cognitive status of
discourse referents was within the remit of the
\isi{demonstrative}''.\footnote{NPs modified by adjectives can be classified as identifiable even though the referent of the NP is not established by previous mention in the given discourse. This is similar to other modifying structures, such as possessive-marked NPs. The interpretation of the referent as identifiable, although the referent has not been established in the given discourse, happens ``on the basis of their inclusive relation to an established set'' (\citealp[595]{Schroeder2006}; see also
  \citealp[67]{Nilsson1985} for specificity-marked objects in \ili{Turkish}). } And
\citet[93]{Schroeder1999} aptly writes that ``the \isi{modification} of a referent
forces a subset-reading of this referent, because a particular
(qualitative) specification of a referent usually implies a delimitation
of the specified referent from other possible (qualitative)
specifications''. As such, nouns modified by adjectives are specific and
assumed to move to Top\textsuperscript{0} in all cases presented here (cf. also \citealp{Rizzi1997}; \citealp{Haegeman2000}). The following movement operations within the Old
\ili{Norwegian} extended NP are assumed (see also Table \ref{ch8t7}):

\begin{enumerate}[label=\alph*.]
\item Neutral, known adjectival feature (no \isi{emphasis}, maybe repetition of
the immediate context), structure with one \isi{adjective}: the \isi{noun} moves to
Top\textsuperscript{0}, the \isi{adjective} is carried along (pied-piping case, phrasal
movement; see e.g. \citealp{Cinque2010}) resulting in the surface \isi{pattern} A--N.
An example is given in (\ref{ch8ex29}).

\begin{exe}
\ex\label{ch8ex29}
\gll hinn \textbf{heiti} \textit{vægr} bœygiz or austri oc i væstr \\
\textsc{art.nom.sg} hot.\textsc{nom.sg.wk} way.\textsc{nom.sg.} bends from east and in west\\
\glt `the cold way/zone bends from east to west' (2v, col.b:20--21)
\end{exe}

\item Focused structure with one \isi{adjective}: the \isi{noun} moves to Top\textsuperscript{0} while
the focused \isi{adjective} moves to Foc\textsuperscript{0}, resulting in the surface \isi{pattern}
N--A (end-focus). Focus on adjectives is analyzed through
\emph{essentiality}. An example is given in (\ref{ch8ex30}).

\begin{exe}
\ex\label{ch8ex30}
\gll  skilningar laus komi i {\textit{skola}} \textbf{goðan } [...] æf hann kæmr fra skola þa hygz hann þægar væra goðr \emph{klærcr} \\
wit less come.\textsc{sbjv} in school.\textsc{acc.sg} good.\textsc{acc.sg.str} {} if he comes from school so thinks he then be good.\textsc{nom.sg.str} educated.man.\textsc{nom.sg}\\
\glt `(if) a simple-minded (person) would come/enter into a good school [...]  if he comes from school then (he) believes (himself) to be a well-educated man'  (17v col.b:14--20)
\end{exe}

The further development of the clause given in (\ref{ch8ex30}), describing the
attitude of a person, is dependent on the property transferred by the
\isi{adjective} `good' in the first phrase (the referent `a good school' sets
the scene that the following sequence elaborates on).\footnote{See
  the full context of the utterance: Ðvi er lict æf skynlauss maðr fæʀ
  til hirðar sæm ufroðr ... fari til Iorsala eða skilningar laus komi i
  skola goðan. æf ufrodr maðr fæʀr til Iorsala þa truir hann sialfr at
  hann se froðr oc sægir ífra sinni færð oc þat flæst er froðum manni
  þycki ænskis vært nema gabs oc haðs. Sva gerir oc hinn skilningarlausi
  æf hann kœmr fra skola þa hygz hann þægar væra goðr klærcr oc værðr
  fæginn oc gœrir af miket spott æf hann finnr þann eʀ æcki kann mæð
  ollu. En æf hann finnr noccorn þann er klærcr er þa væit hann sialfr
  æcki.

  \noindent (`This is like if a dull man goes to court, as (when) an
  unknowledgeable (man) goes to Jerusalem, or a simple-minded (person)
  would enter a good school. If an unknowledgeable man goes to
  Jerusalem, then he believes himself that he would be knowledgeable and
  tells much of his journey; but most seems worthless to a knowledgeable man, (all) but mockery and foolery. As
  such is also the simpleton if he comes from school then he believes (himself) to be a well-educated man and rejoices and shows much mockery if he
  meets one who knows nothing. But if he meets someone who is a scholar,
  he himself knows naught.')} In the annotation, the \isi{adjective} is marked as \emph{new} within the nominal context (the \isi{adjective} in
this context has not been mentioned before in the discourse), and is connected to the following sequence either in a separate comment
level or through a scheme annotation under frames. 


\item Emphasis through direct \emph{contrast} with one \isi{adjective}: the \isi{noun}
moves to Top\textsuperscript{0} while the contrasted \isi{adjective} moves to a position above
Top\textsuperscript{0} (see e.g. \citealp[226]{Molnár2006}) due to the feature {[}CONTRAST{]},
resulting in the surface \isi{pattern} A--N and a contrasted topic reading.
The moved constituent can then mark its sister as the domain of
\emph{contrast} and \emph{given} at the same time (cf. \citealp{Schwarzschild1999}; \citealp{NeelemanTitovVandeKootVermeulen2009}; \citealp{Wagner2006, Wagner2010}; see also \citealp{Krifka1998, Krifka1999}). An example is given in (\ref{ch8ex31}) (see ex. (\ref{ch8ex21}) for the context).

\begin{exe}
\ex\label{ch8ex31}
\gll  En hinn \textbf{hviti} \textit{biorn-inn} er a Grœnalannde er\\
but \textsc{art.nom.sg} white.\textsc{nom.sg.wk} bear-\textsc{def.nom.sg} which on Greenland is\\
\glt `but the white bear which is in Greenland' (11v, col.b:13)
\end{exe}
\end{enumerate}

Depending on certain conditions, movement can affect just the phrase
bearing the feature triggering the movement, or alternatively, it can
affect a larger entity containing the phrase bearing the relevant
feature (pied-piping case). Positioning and movement of elements within
the NP may, however, also be affected by other factors. It is thus
important to consider the interplay of different parameters/factors. It
is also important to note that already in the 13\textsuperscript{th} century, Old
\ili{Norwegian} started to grammaticalize a fixed \isi{word order} (shown by
e.g. the slowly developing possibility of \isi{adjective} stacking), where
morphological restrictions, information-structural constraints and
\isi{prosodic weight} play a less significant role in word ordering and might
not trigger movement in all contexts where it would be expected.
According to the patterns identified in Table \ref{ch8t3} and following the
structure given in (A--C), Table \ref{ch8t7} summarizes the assumed movement
operations.\largerpage[-1]

\begin{sidewaystable}[p]\small
\begin{tabularx}{\textwidth}{lll}
\lsptoprule
Pattern &Reading  & Movement \\

\midrule
Pattern (II):  &base structure  &  the \isi{noun} moves to Top\textsuperscript{0}, pied-piping the \isi{adjective}\\
ART A.WK N &neutral reading  & {[}\ldots{} {[}\textsubscript{TopP} hina bæztu
mænn{]}\textsubscript{j} ...  {[}\textsubscript{αP} ART
A\textsubscript{WK} {[}NP{]}{]}t\textsubscript{j} {]}\\

\tablevspace
Pattern (III):  &emphasized  & the \isi{noun} moves to Top\textsuperscript{0}, while the  \isi{adjective} moves to Foc\textsuperscript{0} \\
N ART A.WK & \isi{adjective} &  {[}\ldots{} {[}\textsubscript{TopP} stol{]}\textsubscript{i}
{[}\textsubscript{FocP} hinn dýri{]}\textsubscript{j} ...  {[}\textsubscript{αP} {[}ART A\textsubscript{WK}{]}t\textsubscript{j} {[}NP{]}t\textsubscript{i} {]}{]}\\

\tablevspace
Pattern (IV):  & emphasized  &  the \isi{noun} moves to Top\textsuperscript{0}, while the  \isi{adjective} moves to Kontr\textsuperscript{0} \\
A.STR N–DEF &\isi{adjective}  &  {[}\ldots{} {[}\textsubscript{KontrP} visan{]}\textsubscript{j}
{[}\textsubscript{TopP} mæistarann{]}\textsubscript{i} ... {[}\textsubscript{αP} {[}A\textsubscript{STR}{]}t\textsubscript{j} {[}NP{]}t\textsubscript{i}
{]}{]}\footnote{Note that the simple structure presented here does
  not show the movement of the \isi{noun} to combine with the bound \isi{article}
  DEF, an element which is not assumed to be part of the base position
  of N.}\\

\tablevspace
Pattern (DD–a):  & emphasized   & the \isi{noun} moves to Top\textsuperscript{0}, while the  \isi{adjective} moves to Kontr\textsuperscript{0} \\
ART A.WK N–DEF &  \isi{adjective} &  {[}\ldots{} {[}\textsubscript{KontrP} hinn hviti{]}\textsubscript{i}
{[}\textsubscript{TopP} biorninn{]}\textsubscript{j} ...  {[}\textsubscript{αP} {[}ART A\textsubscript{WK}{]}t\textsubscript{i} {[}NP{]}t\textsubscript{j}  {]}{]}\\

\tablevspace
Pattern (DD–b):  &neutral reading   & the \isi{noun} moves to Top\textsuperscript{0}, pied-piping  the \isi{adjective} (the \isi{demonstrative} appears above \\
\emph{sá} A.WK N–DEF & (emphasized  & Top in Kontr\textsuperscript{0})\\
&  \isi{demonstrative}) &  {[}... {[}\textsubscript{KontrP} þeim{]} {[}\textsubscript{TopP} heita
væginum{]}\textsubscript{i} ... {[}\textsubscript{αP} ART
A\textsubscript{WK} {[}NP{]}{]}t\textsubscript{i} {]}\\

\tablevspace
 Pattern (V): & emphasized & the \isi{noun} moves to Top\textsuperscript{0}, while the  \isi{adjective} moves to Foc\textsuperscript{0} \\
 N–DEF A.STR& \isi{adjective} &  {[}\ldots{} {[}\textsubscript{TopP} lanndet{]}\textsubscript{j}
{[}\textsubscript{FocP} þitt{]}\textsubscript{j} ... {[}\textsubscript{αP} {[}A\textsubscript{STR}{]}t\textsubscript{j} {[}NP{]}t\textsubscript{i}
{]}{]} \\

\tablevspace
 Pattern (VI): &  neutral reading & the \isi{noun} moves to Top\textsuperscript{0},   pied-piping the \isi{adjective} (the  \isi{demonstrative} \\
\emph{sá} A.STR N  & (emphasized   & appears in Kontr\textsuperscript{0}) \\
&  \isi{demonstrative}) & {[}\textsubscript{KontrP} þeim {[}\textsubscript{TopP} hælgum
manne{]}\textsubscript{i} ...   {[}\textsubscript{αP} A
{[}NP{]}{]} t\textsubscript{i} {]}\\

\tablevspace
 Pattern (VII): & neutral reading  & the \isi{noun} moves to Top\textsuperscript{0}, while the \isi{adjective} is stranded (the \isi{demonstrative}   moves to \\
N DEM ART A.WK & (emphasized  & Foc\textsuperscript{0}) \\
&  \isi{demonstrative}) &  {[}... {[}\textsubscript{TopP} Tre{]}\textsubscript{i}
{[}\textsubscript{FocP} þat{]} \ldots{}  {[}\textsubscript{αP} hit fagra {[}NP{]}t\textsubscript{i} {]}{]}\\
\lspbottomrule
\end{tabularx}\caption{Patterns and their derivation}\label{ch8t7}
\end{sidewaystable}

% \begin{table}[p]\small
% \begin{tabularx}{\textwidth}{X}
% \lsptoprule
%   \parbox{.95\textwidth}{
% \begin{description}
%   \item[\textbf{Pattern (II): ART A.WK N}]  base structure, neutral reading\\
% the \isi{noun} moves to Top\textsuperscript{0}, pied-piping  the \isi{adjective} \\
%  {[}\ldots{} {[}\textsubscript{TopP} hina bæztu mænn{]}\textsubscript{j} ... \\
%   {[}\textsubscript{αP} ART
% A\textsubscript{WK} {[}NP{]}{]}t\textsubscript{j} {]}\item[\textbf{Pattern (III): N ART A.WK}]  emphasised \isi{adjective}\\
%  the \isi{noun} moves to Top\textsuperscript{0}, while the  \isi{adjective} moves to Foc\textsuperscript{0} \\
% {[}\ldots{} {[}\textsubscript{TopP} stol{]}\textsubscript{i} {[}\textsubscript{FocP} hinn dýri{]}\textsubscript{j} ...\\
% {[}\textsubscript{αP} {[}ART A\textsubscript{WK}{]}t\textsubscript{j} {[}NP{]}t\textsubscript{i} {]}{]}\item[\textbf{Pattern (IV): A.STR N–DEF}]  emphasised \isi{adjective} \\
% the \isi{noun} moves to Top\textsuperscript{0}, while the  \isi{adjective} moves to Kontr\textsuperscript{0} \\
%   {[}\ldots{} {[}\textsubscript{KontrP} visan{]}\textsubscript{j} {[}\textsubscript{TopP} mæistarann{]}\textsubscript{i} ... \\
%  {[}\textsubscript{αP} {[}A\textsubscript{STR}{]}t\textsubscript{j} {[}NP{]}t\textsubscript{i} {]}{]}\footnote{Note
%   that the simple structure presented here does   not show the movement of the \isi{noun} to combine with the bound \isi{article}   DEF, an element which is not assumed to be part of the base position   of N.
%   }
% \item[\textbf{Pattern (DD–a): ART A.WK N–DEF}]  emphasised \isi{adjective}\\
% the \isi{noun} moves to Top\textsuperscript{0}, while the  \isi{adjective} moves to Kontr\textsuperscript{0} \\
% {[}\ldots{} {[}\textsubscript{KontrP} hinn hviti{]}\textsubscript{i} {[}\textsubscript{TopP} biorninn{]}\textsubscript{j} ...  \\
% {[}\textsubscript{αP} {[}ART A\textsubscript{WK}{]}t\textsubscript{i} {[}NP{]}t\textsubscript{j}  {]}{]}\item[\textbf{Pattern (DD–b): \emph{sá} A.WK N–DEF}] neutral reading  (emphasised  \isi{demonstrative})\\
%  the \isi{noun} moves to Top\textsuperscript{0}, pied-piping   the \isi{adjective} (the \isi{demonstrative}  appears above Top in Kontr\textsuperscript{0}) \\
% {[}... {[}\textsubscript{KontrP} þeim{]} {[}\textsubscript{TopP} heita  væginum{]}\textsubscript{i}\\
% ... {[}\textsubscript{αP} ART A\textsubscript{WK} {[}NP{]}{]}t\textsubscript{i} {]}\item[\textbf{Pattern (V): N–DEF A.STR}] emphasised \isi{adjective}\\
% the \isi{noun} moves to Top\textsuperscript{0}, while the \isi{adjective} moves to Foc\textsuperscript{0} \\
% {[}\ldots{} {[}\textsubscript{TopP} lanndet{]}\textsubscript{j} {[}\textsubscript{FocP} þitt{]}\textsubscript{j} ...\\
% {[}\textsubscript{αP} {[}A\textsubscript{STR}{]}t\textsubscript{j} {[}NP{]}t\textsubscript{i} {]}{]} \item[\textbf{Pattern (VI): \emph{sá} A.STR N}]  neutral reading (emphasised \isi{demonstrative}) \\
% the \isi{noun} moves to Top\textsuperscript{0},  pied-piping the \isi{adjective} (the  \isi{demonstrative} appears in Kontr\textsuperscript{0})\\
%  {[}\textsubscript{KontrP} þeim {[}\textsubscript{TopP} hælgum manne{]}\textsubscript{i} ... \\
%  {[}\textsubscript{αP} A {[}NP{]}{]} t\textsubscript{i} {]}\item[\textbf{Pattern (VII):  N DEM ART A.WK}] neutral reading  (emphasised \isi{demonstrative}) \\
% the \isi{noun} moves to Top\textsuperscript{0}, while the   \isi{adjective} is stranded (the \isi{demonstrative}   moves to Foc\textsuperscript{0})\\
% {[}... {[}\textsubscript{TopP} Tre{]}\textsubscript{i} {[}\textsubscript{FocP} þat{]} \ldots{}  \\
% {[}\textsubscript{αP} hit fagra {[}NP{]}t\textsubscript{i} {]}{]}
% \end{description}
% }\\
% \lspbottomrule
% \end{tabularx}
% \caption{Patterns and their derivation}\label{ch8lt7}
% \end{table}
% \clearpage


However, there are also examples in the \isi{corpus} material that do not
quite fit the approach taken here. These are cases of \isi{postnominal}
adjectives that form a fixed compound-like expression with the head
\isi{noun} (one informational unit) and carry information that is active in
the hearer's knowledge stock, as in (\ref{ch8ex32}). Additionally, in (\ref{ch8ex32b}) the
\isi{adjective} is classified as light.

\begin{exe}
\ex\label{ch8ex32}Postnominal adjectives (active information)
\begin{xlist}
\ex\label{ch8ex32a}
\gll  Iafnan skaltu \textit{guð} \textbf{almatkan} oc hina \textbf{hælgu} \textit{Maʀiu} lata æiga noccot í felage mæð þer \\
always should.you god.\textsc{acc.sg} almighty.\textsc{acc.sg.str} and the holy.\textsc{acc.sg.wk} Mary.\textsc{acc.sg} let own something in fellowship with you\\
\glt `Always let God Almighty and the holy Mary own something together with
you in fellowship.' (3v, col.b:13--16)
\ex\label{ch8ex32b}
\gll ok merkir þat í þvi at fyr \textit{cross-en} \textbf{helga} ok fyr holld~tekio Crists er friðr settr á miðli himnescra luta ok iarðnescra. \\
and mark this in this that through/in.front.of cross-\textsc{def.acc.sg} holy.\textsc{acc.sg.wk} and through/in.front.of incarnation Christ is peace settled on between heavenly.\textsc{gen.pl.str} thing.\textsc{gen.pl} and earthly.\textsc{gen.pl.str}\\
\glt `and marked through/in front of the holy cross and through/in front of the incarnation of Christ, peace is settled between heavenly and earthly things' (HómNo 3.3,66)
\end{xlist}
\end{exe}

We would expect such situationally/contextually known entities to appear in \isi{prenominal} position, as the \isi{postnominal} appearance suggests (according
to the analysis presented here) information-structural \isi{emphasis}
(\emph{presentational focus}). The compound-like nature of these
expressions would also suggest that the \isi{adjective} should get pied-piped
when the \isi{noun} moves to the topic position. A quick search in the Old
\ili{Norwegian} \isi{corpus} material also reveals that the combinations \textit{almáttigr
guð}/\textit{heilagr kross} are more common than \textit{guð almáttigr}/\textit{kross heilagr}, so that we can exclude a fixed \isi{postnominal} order for these
expressions. Within the given approach, these examples might be
explained by stating that the adjectives `almighty' and `holy' are
actually the locus of information within these contexts
(\emph{\isi{essential} by context}), important for the development of the
discourse. Structurally, the adjectives seem simply to be left stranded while
the \isi{noun} moves into the topic position. This could be explained through
\isi{prosodic weight}; however, the adjectives are not classified as \isi{heavy} within
the approach given here. One aspect that might be of importance is the
parallel structure in which these phrases appear. Considering a
stylistic point of view, the choice of the \isi{postnominal} position of the
\isi{adjective} in the first conjunct becomes clearer. Example (\ref{ch8ex32a}) shows a
case of assonance, in which \emph{Maʀiu} in the second conjunct is bound
together through the second syllable of \emph{almatkan} with stress on
-\emph{mat}- (as is the case in e.g. modern \ili{German} or
\ili{English}). The structure is then comprised of two times two syllables
(\emph{guð al}- \emph{\textbar{} mat}-\emph{kan} and
\emph{hæl}-\emph{gu \textbar{} Maʀ}-\emph{iu}) with stress on the
first syllable of the second part, respectively. Example (\ref{ch8ex32b}) shows a
chiasmus of the onsets \textit{\textbf{cr}ossen} and \textit{\textbf{Cr}ists}, and \textit{\textbf{h}elga} and \textit{\textbf{h}olld tekio}. These analyses are part of the
annotation within KoNoKs; however, a more detailed discussion of cases
like these is put aside for a later study.


\subsection{Split construction – Type I}\label{ch8s4.1}\largerpage
Turning now to two or more adjectives within one NP, these are generally
rare in the Old \ili{Norwegian} \isi{corpus} material (cf. also \citealp[5]{Bech17}). Here,
I only consider structures under the \isi{split construction} in which the
quality expressed by the \isi{adjective} in the \isi{postnominal} position
is attributed to the same referent (strict identity) as the \isi{prenominal}
\isi{adjective}. In structures without a coordinator, the \isi{noun} may pass
multiple adjectives on its way up the tree (to Top\textsuperscript{0}), leaving both in a
\isi{postnominal} position, as in (\ref{ch8ex33}) -- with no example in KoNoKs -- or a
\isi{split construction} occurs with one \isi{adjective} in \isi{prenominal} and one
\isi{adjective} in \isi{postnominal} position (stranded) as the result of phrasal NP
movement (pied-piping movement of the lowest \isi{adjective}), as in (\ref{ch8ex34}). In
total, four examples of this are found in KoNoKs. The movement is
illustrated in (D).\footnote{Note that \emph{samfeðra} is an indeclinable \isi{adjective} and thus is not  glossed.} 
 

\begin{exe}
\ex\label{ch8ex33}
\gll \textit{faður~systir} \textbf{skilgen} \textbf{samfædra} \\
father.sister.\textsc{nom.sg} trueborn.\textsc{nom.sg.str} same.father.Ø\\
\glt `aunt trueborn of the same father' (DG 8 5.284)

\ex\label{ch8ex34}
\gll mæð {[}\textbf{longu} \textit{hafi} \textbf{rasta} \textbf{fullu}{]} \\
with wide.\textsc{dat.sg.str} ocean.\textsc{dat.sg} of.strong.currents full.\textsc{dat.sg.str}\\
\glt `with the wide sea full of strong currents' (15v, col.a:12)

\exi{(D)} \label{ch8exd}
\begin{xlist}
\ex\label{ch8exda}{[}AP\textsubscript{1} AP\textsubscript{2} NP{]} \rightarrow{}  {[} {[}NP{]}\textsubscript{i} {[}AP\textsubscript{1} AP\textsubscript{2} t\textsubscript{i} {]} {]} \hfill (\isi{postnominal} position)
\ex\label{ch8exdb}{[}AP\textsubscript{1} AP\textsubscript{2} NP{]} \rightarrow{} {[} {[}AP\textsubscript{2} NP{]}\textsubscript{i} {[}AP\textsubscript{1} t\textsubscript{i} {]} {]} \hfill (\isi{split construction} I)
\end{xlist}
\end{exe}

The \isi{postnominal} \isi{adjective} in these patterns is structurally merged in a
higher position than the \isi{prenominal} one (reversed ordering of the
adjectives on the surface after movement). \citet[12]{Pfaff2019} notes for
the surface structure (linear) \isi{postnominal} \isi{adjective} that the higher
merging position ``has semantic effects. Put informally, the \isi{adjective}
provides some comment or evaluation on the referent denoted by the lower
\isi{noun} phrase''. Possible reasons for the movement resulting in a \isi{pattern}
with only one of the two adjectives being pied-piped could either be
due to the merging zone of the higher \isi{adjective}, preventing it from
being pied-piped together with the \isi{noun}, or because of factors of
\isi{prosodic weight} and the avoidance of \isi{heavy} elements in the left
periphery (serving the end-\isi{weight} principle). Properties that are
decisive for the emergence of split constructions in general are free
\isi{word order}, flexible intonation, and no obligatory articles (cf. \citealp{FéryPaslawskaFanselow2007} for Ukrainian). The  movement is here assumed to be
triggered by information-structural constraints parallel to movement
within the clause, as discussed in Section \ref{ch8s3.3}. All examples of the type I
construction found in the \isi{corpus} material in KoNoKs show the strong
(\isi{indefinite}) form of the \isi{adjective}. \citet[16]{Bech17} further notes that
adjectives in these constructions often show restrictions concerning
their type.

\subsection{Split construction – Type II}\label{ch8s4.2}
If two adjectives are involved, they most often occur in a parallel
\isi{split construction} (including a \isi{coordination}; placing the adjectives
equally next to each other), rather than in a hierarchically ordered
stacking construction or in a \isi{split construction} of type I, as shown
in (\ref{ch8ex35}). I term this construction a \isi{split construction} of type II. Here too, both adjectives are analyzed as \isi{prenominal}. This \isi{pattern}, too, is
found only rarely in the \isi{corpus} material, with 34 examples in total (see
also notes from \citealp[57--60]{Ringdal1918}; \citealp[72]{Faarlund04}).

\begin{exe}
\ex\label{ch8ex35}
\begin{xlist}
\ex\label{ch8ex35a}
\gll sæm byriar {[}\textbf{lyðnum} \textit{syni} oc \textbf{litillatom}{]} at finna {[}\textbf{astsamlegan} \textit{foður} oc \textbf{gofgan}{]} \\
as behooves humble.\textsc{dat.sg.str} son.\textsc{dat.sg} and obedient.\textsc{dat.sg.str} to find {loving.\textsc{acc.sg.str}} father.\textsc{acc.sg} and renowned.\textsc{acc.sg.str}\\
\glt `as it behooves a humble and obedient son to approach a loving and renowned father' (1r, col.a.:22--26)
\ex\label{ch8ex35b}
\gll en aðr hirti hann {[}\textbf{gott} \textit{korn} oc \textbf{reinnt}{]} \\
but before gathered he good.\textsc{acc.sg.str} grain.\textsc{acc.sg} and clean.\textsc{acc.sg.str}\\
\glt `but before he gathered good and clean grain' (24v, col.a:25--26)
\end{xlist}
\end{exe}

\citet[72]{Faarlund04} states that this \isi{pattern} shows an alternative to a
very common type of extraposition (with coordinated adjectives at the
end of the NP), where the first \isi{adjective} may remain to the left of the
\isi{noun}, while the other one is extraposed.\footnote{In host-internal
  extraposition, the extraposed material is base-generated internal to
  its host (non-movement approach, see e.g. \citealp[ch.7]{deVries2002}; see
  also \citealp{Overfelt2015}).} Here, I will not analyze the two adjectives as
ambilateral adjectives or as extraposition, but as instances of NP
\isi{coordination} with an empty nominal element \emph{pro} in the second
conjunct of the type {[}A--N--\emph{and}--A--nonDP \emph{pro}{]} and
with co-reference of the two nouns in an empty copy (again, note that
DP is used in a theory-neutral manner in this study; see \citealp{Lobeck1995} for a broad
discussion of ellipsis and nonDP \emph{pro}; also \citealp{Haumann03} for Old
\ili{English}). The second \isi{adjective} is then in a \isi{prenominal} position to a
phonetically empty head.

As in type I, the adjectives involved in the type II \isi{split construction}
found in KoNoKs are all strong. For other languages it has been argued
that the second, seemingly \isi{postnominal} \isi{adjective} functions as a
\isi{predicative} \isi{adjective} (cf. \citealp{Spamer1979}; \citealp[171, 176]{Fischer00}). However,
\citet[64f]{Haumann03} argues that examples of Old \ili{English} showing a
\isi{demonstrative} \isi{pronoun} repeated in an `\emph{and} \isi{adjective}' construction
account for the fact that the second \isi{adjective} cannot be \isi{predicative}.
She writes that ``{[}t{]}he presence of a \isi{demonstrative} or
\isi{possessive pronoun} is indicative of \isi{definiteness} and \isi{definiteness} does
not go hand in hand with predicatehood. Moreover, the presence of the
\isi{demonstrative} \isi{pronoun} is a clear indicator of the nominal status of what
follows \emph{and}'' (\citealp[65]{Haumann03}, supporting an ellipsis analysis).\footnote{\citet[176]{Fischer00} accounts for this fact by analyzing the weak adjectives as
  substantivized, i.e. nominalized adjectives.} Also for Old \ili{Norwegian} I
assume that the two positions (pre- vs. \isi{postnominal}) are not
automatically assigned to two different functions (see discussion
above). I then follow Haumann and assume a nonDP \emph{pro} (elided
category as base-generated empty category; see \citealp{Lobeck1995}) or in other
words a reduced copy in multiple spell-out, following an economy-based
analysis (\emph{Economy of Pronunciation}; cf. \citealp{vanUrk2018}) predicting
``that additional copies in multiple spell-out must be minimal in form,
much like a linearization-based approach'' \citep[965]{vanUrk2018}. The
reconstruction of the elided nominal within the second DP and its
semantic content must, according to \citet[76]{Haumann03}, referring to
\citet[786f.]{Lobeck1993} ``be recovered under sense identity {[}...{]} with
the logical representation of the antecedent'' -- the lexical NP in the
first conjunct (see also \citealp[966]{vanUrk2018}).\footnote{Strong adjectival
  \isi{agreement} features help recover grammatical information about nonDP
  \emph{pro}, whereas the semantic content of nonDP \emph{pro} ``is
  recovered through dependency on a lexical antecedent'' (\citealp[74]{Haumann03}, citing \citealp[193]{Kester1996thesis}). Under the strict identity
  interpretation, the \isi{adjective} contained in the \isi{postnominal}
  ``\emph{and}-\isi{adjective}" construction is attributed to the same
  referent as the \isi{prenominal} \isi{adjective}. Whether a given nonDP \emph{pro}
  is interpreted as strictly identical or as sloppily identical with its
  antecedent is essentially triggered by the linguistic context and/or
  world knowledge.} \citet[66]{Haumann03}, referring to \citet[187ff]{Kester1996thesis}, 
further notes ``that \emph{pro} is licensed in the vicinity of overt
adjectival number and gender \isi{agreement} morphology'', something that also
holds for Old \ili{Norwegian}, as it has rich inflectional paradigms for both
weak and strong adjectives. Turning back to the examples in (\ref{ch8ex35}), these
show that the strong \isi{adjective} in the \isi{postnominal} ``\emph{and}-\isi{adjective}"
position functions as an \isi{attributive} \isi{adjective} in \isi{prenominal} position,
i.e. in pre-\emph{pro} position (there is no sign of them functioning
as \isi{predicative} adjectives and assigning an additional property to the
\isi{noun} or occurring in a \isi{predicative} context). This is stated, however,
not with respect to the preceding \isi{noun}, but within a second nominal
expression of the same referent whose head is phonetically empty (see
also \citealp[244]{Spamer1979}; \citealp[71f]{Haumann03}), as simplified in (E). The
relation between nonDP \emph{pro} and its lexical antecedent is given by
co-indexation.

\begin{exe}
\exi{(E)}\label{ch8exe}
\begin{xlist}
\ex\label{ch8exea}{[} {[}AP\textsubscript{1} NP\textsubscript{j}{]} {[}\& {[}AP\textsubscript{2} \emph{pro}\textsubscript{j}{]}{]} {]} \hfill (\isi{split construction} II)
\ex\label{ch8exeb}{[} {[}\textbf{goða} mænn\textsubscript{j} {[}\& {[}AP \textbf{hælga}{]} {[}NP \emph{pro}\textsubscript{j}{]}{]}{]} {]}
\end{xlist}
\end{exe}

\begin{sloppypar}
The \isi{noun} in the first conjunct then functions as the lexical antecedent
of nonDP \emph{pro} (whence the impression that the \isi{attributive} \isi{adjective}
modifying nonDP \emph{pro} modifies the antecedent of nonDP
\emph{pro}).\footnote{Note, however, that nonDP \emph{pro} is not a
  referential expression itself (referential properties are determined
  elsewhere; cf. \citealp[76]{Haumann03}).} Assignment of stress within the
second conjunct falls on the only constituent left that can get
intonational stress within the elliptic phrase (the second \isi{adjective}
shows what is recognized as \isi{emphasis} because it is a separate
phonological/intonational phrase).
\end{sloppypar}

\section{Summary and concluding remarks}\label{ch8s5}
 This chapter has given an overview of positional \isi{variation} of \isi{attributive}
\isi{adjective}s in the Old \ili{Norwegian} extended NP. The examination shows
that \isi{attributive} adjectives in Old \ili{Norwegian} may be found in \isi{prenominal} or
\isi{postnominal} position, or in a \isi{split construction} flanking the modified
\isi{noun}. In total, seven patterns connected to overt \isi{definiteness}, three
connected to overt \isi{indefiniteness} and two types of \isi{split construction}
are described within this study and are briefly compared to patterns
found in the history of Icelandic. The discussion of pre- and
\isi{postnominal} position of adjectives focuses on the underlying base
structure and the factors responsible for the \isi{variation} in the surface
structure. This \isi{variation} involves NP-internal movement that can still
be observed in the Old \ili{Norwegian} \isi{corpus} material, although the data
suggests that a fixed \isi{prenominal} position of the \isi{adjective} is already
the most common case (see also \citetv{chapters/3Modifiers}). For the analysis of the remaining cases of
structural \isi{variation}, I suggest extending the split DP hypothesis with
a full-fledged left periphery to the Old \ili{Norwegian} NP, where the
various orders are mainly triggered by information-structural
constraints. It has been shown that phenomena of morphology or
\isi{definiteness} alone do not play a decisive role for constituent
ordering within the Old \ili{Norwegian} NP. The \isi{adjective} morphology seems to
group \isi{attributive} adjectives according to their ability to appear in
\isi{postnominal} position (cf. esp. Table \ref{ch8t3}); however, for both weak and
strong adjectives, cases of postpositioning are found. The nouns in the
structures analyzed in this chapter are considered to move to Top\textsuperscript{0} in
all cases, while the adjectives may either move to Foc\textsuperscript{0} or Kontr\textsuperscript{0}, or are pied-piped or are left stranded, resulting in the various surface patterns
that have been described. To determine if an \isi{adjective} is emphasized
within the phrase, I have introduced the concept of \emph{essentiality}, based on the appearance of feature descriptions previously mentioned in
the discourse, and on the further development of the
discourse/informational flow, i.e. if a following sequence is dependent
on the feature described by the \isi{adjective}. This approach might be used
in further studies on adjectives cross-linguistically. Additionally, I
have analyzed two structural types of split constructions, one
involving NP-internal movement, while the other one shows
\isi{coordination} with an empty head in the second conjunct.

For the observable \isi{variation} including an \isi{attributive} \isi{adjective} (or an
\isi{adjective} group) in Old \ili{Norwegian}, the following statements have been
made in this chapter:

\begin{enumerate}
\def\labelenumi{\roman{enumi})}
\item
  \isi{adjective}s occur in \isi{postnominal} position as the result of either
  solely in\-for\-ma\-tion-struc\-tur\-al constraints or as a result of a
  \isi{combination} of \isi{information structure} and \isi{prosodic weight};
\item
  \isi{adjective}s appear in \isi{prenominal} position as the result of different
  factors:

  \begin{enumerate}
  \def\labelenumii{\alph{enumii}.}
  \item
    the \isi{prenominal} position is the result of pied-piping within a
    neutral reading (no \isi{emphasis} assigned; might ignore prosodic
    \isi{weight});
  \item
    the \isi{prenominal} position is the result of \isi{contrast} with movement of
    the \isi{adjective} into a position above Top\textsuperscript{0}. However, the influence of
    \isi{prosodic weight} can still be observed through flanking (split
    construction) to avoid \isi{heavy} elements in \isi{prenominal} position
    (stacking of two adjectives);
  \item
    the \isi{adjective}(s) no longer react to information-structural
    movement triggers (no movement into the lower Foc\textsuperscript{0} position;
    incipient \isi{grammaticalization} of the fixed order A--N).
  \end{enumerate}
\end{enumerate}

The last point takes the development towards a strict \isi{word order} into
account (as well as the rise of a proper \isi{determiner} system that helps
to indicate whether the information conveyed by the \isi{adjective} presents
focused or backgrounded information). Because of this development, many examples from as early as the 13\textsuperscript{th} century challenge the
statements given in i) and ii), showing that the factors
involved in \isi{word order} \isi{variation} had already weakened to a high degree.
Thus, the  effects and movement operations triggered by
information-structural constraints do not apply to all cases found in
the \isi{corpus} material. On the contrary, many examples are not affected by
these constraints anymore. We therefore find both information-structurally
highlighted and ``neutral" constituents, as well as both \isi{heavy} and light
constituents in \isi{prenominal} position.

Further research is still needed to get a more detailed picture of
factors that may have influenced the internal order of elements within
the NP in the history of \ili{Norwegian}. Even though several Old \ili{Norwegian}
texts were consulted for this study, only one text was analyzed in
greater detail. A detailed analysis of other Old \ili{Norwegian} texts could
provide stronger evidence for the approach presented here, and clarify
further the factors responsible for \isi{word order} \isi{variation} within the
extended NP.


 
\section*{Abbreviations}
\begin{tabularx}{.45\textwidth}{lQ}
αP & alpha phrase\\
A & {adjective}\\
\textsc{acc} & accusative \\
AP & {adjective} phrase\\
ART/\textsc{art} & {adjective} {article}\\
CardP & cardinal phrase\\
CP & complementizer phrase\\
\textsc{dat} & {dative} \\
DEF/\textsc{def} & nominal suffix {article}\\
DEM/\textsc{dem} & {demonstrative}\\
DemP & {demonstrative} phrase\\
DP & {determiner} phrase\\
Foc & Focus\\
FocP & Focus phrase\\
INDEF/\textsc{indef} & {indefinite}\\
\end{tabularx}
\begin{tabularx}{.5\textwidth}{lQ}
\textsc{gen} & {genitive} \\
KontrP & contrastive phrase\\
N & {noun}\\
\textsc{nom} & {nominative} \\
nonDP \textit{pro} & instances of \textit{pro} licensed by overt adjectival {agreement} morphology\\
NP & nominal phrase\\
\textit{n}P & little NP\\
POSS & possessive\\
PossP & possessive phrase\\
SpecDP & specifier of DP\\
\textsc{str} & strong\\
t & trace\\
Top & topic\\
TopP & topic phrase\\
\textsc{wk} & weak\\
\\
\end{tabularx}

\section*{Acknowledgements}
I would like to thank Kristin Bech and Alexander Pfaff for giving me the opportunity to present my work on \isi{adjective} position in Old \ili{Norwegian} at one of the workshops held by the project \textit{Constraints on syntactic \isi{variation}: \isi{Noun} phrases in early \ili{Germanic} languages} in October 2020, and for the opportunity to contribute to this volume. I especially owe many thanks to Alexander for fruitful discussions, as well as for helpful comments and suggestions when working on this paper. Additionally, I would like to thank both for their reviews and constructive feedback that helped improve this paper and get it into its present shape. My contribution was written while I held a research grant from the Department of Linguistic, Literary and Aesthetic Studies, Faculty of Humanities, University of Bergen.




{\sloppy\printbibliography[heading=subbibliography,notkeyword=this]}
\end{document}
