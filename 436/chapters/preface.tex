\addchap{Preface}
\begin{refsection}

The present volume is one of two major outputs of the project \textit{Constraints on syntactic \isi{variation}: \isi{Noun} phrases in early \ili{Germanic} languages}, funded by the Research Council of Norway (grant no. 261847). The other major output is the \isi{noun} phrase \isi{database} NPEGL, which is also presented in this volume. 

As suggested by the title, the overall aim of the project was to achieve a better understanding of syntactic \isi{variation} between languages that are closely related to each other, and to model linguistic change in the light of constraints on \isi{variation}. Our basic assumption was that languages seldom display truly free \isi{variation}. Hence, we wanted to find out which types of constraints are at work, and what the motivation behind these constraints might be. Furthermore, \isi{variation} is often considered to be a corollary of ongoing change, making synchronic \isi{variation} a window on diachronic developments. We were therefore interested in the way in which synchronic cross-varietal \isi{variation} provides information about similarities and divergences in changes between languages that have a common ancestor. 

One aspect of \isi{noun} phrases in early \ili{Germanic} languages that interested us from the outset was \isi{word order}. We noticed that in textbooks on Old \ili{Norse} it was often claimed that the order noun--modifier is the default order in Old \ili{Norse} \isi{noun} phrases, as opposed to other early \ili{Germanic} languages such as Old \ili{English} and Old High \ili{German}, where the order modifier--\isi{noun} is the default. We wondered whether this was actually the case, considering the common ancestry of these languages. In addition, we realized that although much work has been carried out on language \isi{variation} and change in the domain of the clause, less had been done on the structure of the \isi{noun} phrase from a cross-varietal and diachronic perspective, although \isi{noun} phrases, too, display flexibility with respect to \isi{word order} in their early stages. 
  
\section*{Germanic} 

\ili{Germanic} is a branch of the \ili{Indo-European} language family; a Proto-\ili{Germanic} language is not attested, but can be reconstructed on the basis of the attested \ili{Germanic} languages. There are several properties that set the \ili{Germanic} languages apart from the other \ili{Indo-European} languages, such as the set of sound shifts referred to as Grimm’s law, e.g. [p] $\rightarrow$ [f] (cf. \ili{Latin} \textit{pater} – Old High \ili{German} \textit{fater}), and the formation of a ``weak past tense'' involving a dental suffix (\ili{English} \textit{laugh-ed}; \ili{German} \textit{lach-te}). Of particular relevance for our purposes, we also observe the formation of a weak nominal \isi{inflection} on the basis of the \ili{Indo-European} n-stems. In other \ili{Indo-European} languages, the n-stems were one subgroup of the consonantal stems, but in \ili{Germanic}, the weak \isi{inflection} forms an opposition to the strong \isi{inflection} (= the vocalic stems). This dichotomy is particularly pronounced in the emergence of two adjectival inflections, where the weak \isi{adjectival inflection} has often been thought to be related to \isi{definiteness} marking. 

\ili{Germanic} is traditionally divided into three branches: East \ili{Germanic}, North \ili{Germanic} (= \ili{Norse}) and West \ili{Germanic}. \ili{Gothic} is the most prominent representative of East \ili{Germanic}, with the oldest attestation of a substantial text body of a \ili{Germanic} language, i.e. the New Testament translation by Wulfila (4\textsuperscript{th} century). At the same time, there is not much attested material apart from the Wulfila Bible, and East \ili{Germanic} becomes extinct. Proto-\ili{Norse} is attested in the form of runic inscriptions mainly from the 3\textsuperscript{rd} century onwards. During the Viking period (ca. 800--1100), a dialect split between Old East \ili{Norse} (\ili{Old Swedish}, Old \ili{Danish}) and Old West \ili{Norse} (Old \ili{Norwegian}, \ili{Old Icelandic}) becomes discernible. From the 12\textsuperscript{th} century onwards, the North \ili{Germanic} languages are attested in manuscripts (in the \ili{Latin} alphabet). Proto-West \ili{Germanic} is not attested in the same way as Proto-\ili{Norse}, but from the 8\textsuperscript{th} century onwards, the main representatives Old \ili{English}, Old High \ili{German}, and \ili{Old Saxon} (= Old Low \ili{German}) are attested in manuscripts.  

In the context of \isi{noun} phrases and \isi{noun} phrase-internal \isi{variation}, some properties are of particular relevance: 



\begin{itemize}
    \item The strong/weak \isi{adjectival inflection}: \\ 
    This is a \ili{Germanic} specialty, which touches upon syntactic, morphosyntactic and semantic issues. Even though the weak \isi{inflection} and the  strong/ weak \isi{contrast} have been extensively studied from various different angles, there are still several open issues.  
    \item The \isi{grammaticalization} of (definite/\isi{indefinite}) articles: \\
    Proto-\ili{Germanic} did not have any articles (in the conventional sense). Due to continuous documentation for more than a thousand years, the \ili{Germanic} languages (except \ili{Gothic}) offer rich material through which to study the development of definite articles from (distal) demonstratives, and \isi{indefinite} articles from the \isi{numeral} `one'.
    \item The placement of modifiers: \\  
    In the modern \ili{Germanic} languages, (inflecting) adnominal modifiers and determiners usually occur prenominally (some exceptions are the suffixed definite \isi{article} in \ili{Scandinavian}, and \isi{postnominal} possessives in \ili{Norwegian} and Icelandic). In \isi{contrast}, in the early \ili{Germanic} languages, modifiers occur both prenominally and postnominally. This opens up for several perspectives that can be explored, e.g.: 1) synchronic-\isi{comparative} studies; 2) diachronic-\isi{comparative} studies; 3) internal differences (e.g. \isi{prenominal} vs. \isi{postnominal} position). 
\end{itemize}



\section*{Noun phrases and variation}

During the past 40 years, there has been an increasing interest in \isi{noun} phrases (e.g. the DP-hypothesis, parallelism between the nominal and the verbal projection, possessives as subjects, adjectival ordering, the interpretation of adjectives). Notably, cartographic approaches have closely examined the internal constituency and the overall architecture of \isi{noun} phrases, drawing very fine-grained \isi{noun} phrase ``maps''.  
 
At the same time, during the past 20 years or so, there has been another strand of research that pays attention to syntactic \isi{variation}. The term “\isi{variation}” itself can be given various interpretations. It can be used to simply make reference to (surface) diversity, e.g. the number of possible constellations. One example would be pre- vs. \isi{postnominal} adjectives in the \ili{Romance} languages, where it can often be shown that that two varieties are not semantically and/or functionally equivalent. For instance, in the \ili{Spanish} minimal pair (i) vs. (ii), the surface position of the \isi{adjective} correlates with interpretation.

\begin{exe}
    \exi{(i)} \gll un ladrón bueno \\
                  a thief good  \\ 
              \glt  `X is a thief and (a) good (person)' \hfill                 (\isi{postnominal}: intersective)     
    \exi{(ii)} \gll un buen ladrón \\
                    a good thief  \\ 
            \glt `X is good as a thief (= good at stealing)' \hfill   (\isi{prenominal}: subsective)
\end{exe}

But ``\isi{variation}'' can also mean ``deviation from a given standard form'', where seemingly corresponding varieties differ from that standard with respect to some parameter (\isi{word order}, \isi{inflection}, case marking, ...) without it being obvious whether that deviation correlates with a different interpretation. This applies, for instance, to dialectal \isi{variation} and cases where a deviant form indicates a difference in register.    
 
In both senses, the early \ili{Germanic} languages offer an ideal ``playground''. They display a greater range of surface diversity and possible constellations compared to the modern \ili{Germanic} varieties, e.g. optional(?) \isi{postnominal} occurrences of modifiers, optionality(?) of determiners, distribution of strong/weak \isi{inflection}(?), etc. They are maybe no longer at the level of dialectal varieties, but they are still closely related, and there is a reasonably large body of extant written material to draw upon. A relevant instance of diversity/\isi{variation} is illustrated in (iii) and (iv), which are examples frequently found in the Old \ili{Norse} saga literature (in the same textual environment).  

\begin{exe}
    \exi{(iii)} \gll hann var mikill maðr \\ 
                  he was great man  \\ 
    \exi{(iv)} \gll hann var maðr mikill \\
                 he was man great \\ 
\end{exe}


	 
Variation of this kind can easily be observed in the old manuscripts, but differently from (i) vs. (ii), it is not immediately obvious whether the pre- vs. \isi{postnominal} occurrence of the \isi{adjective} correlates with a systematic difference in function or interpretation. 
 
\section*{Corpora and databases }

One significant advantage for the study of old (= ``dead'') languages is the increasing availability of annotated text corpora (as opposed to labouring through manuscripts or edited volumes manually). These corpora not only make it possible to browse for individual items or categories, but also for constellations, which facilitates reliable quantitative studies and comparisons, e.g. of \isi{prenominal} vs. \isi{postnominal} adjectives. All the contributors to this volume have relied heavily on various corpora and databases of early \ili{Germanic} languages. 

Corpora and databases are usually built for a specific purpose, often within a (more or less) specific framework, and to various degrees of granularity. Thus, it is not always easy to compare the results of queries in two different corpora; they may differ in terms of segmentation and categorization, in the amount of morphological, syntactic, or semantic information annotated, etc.  

Therefore, a second major output of the project has been the creation of a \isi{database} specifically dedicated to \isi{noun} phrases in early \ili{Germanic} languages and suitable to the study of \ili{Germanic} internal \isi{variation} (NPEGL). The advantages of such a \isi{database} are: 1) unified annotation for the languages to be compared, and 2) the possibility to annotate \isi{noun} phrase-internal subtleties that would be difficult or impossible to annotate in a general text-based \isi{corpus}. 


\section*{The chapters of this volume }  

In Chapter 1, Alexander Pfaff and Gerlof Bouma present the NPEGL \isi{noun} phrase \isi{database}, which they created. The authors describe the overall purpose and design of the NPEGL \isi{database}, and address the motivation for a specialized \isi{noun} phrase \isi{database} and its advantages, as well as technicalities pertaining to the processing of the source corpora, automatic conversion, and the annotation process. Furthermore, the chapter illustrates how the NPEGL \isi{database} can be used for research.

In Chapter 2, Alexander Pfaff introduces a method for measuring syntactic diversity, called \textit{Patternization}. In accordance with the project title (“constraints on syntactic \isi{variation}”), and on the basis of the NPEGL annotation scheme, Pfaff develops a mathematical method that can be used to process, quantify and visualize syntactic \isi{variation}. Even though largely illustrated with the NPEGL annotation, the method is intended to be applicable, in principle, to any text sample that has, at least, part-of-speech annotation.   

Chapter 3 is a pilot study carried out by several members of the project group: Kristin Bech, Hannah Booth, Kersti Börjars, Tine Breban, Svetlana Petrova, and George Walkden. They compare various aspects of modifier position in Old \ili{English}, Old High \ili{German}, \ili{Old Icelandic} and \ili{Old Saxon}, focusing on similarities and differences, and possible reasons for the observed distribution, such as texts and genres, \isi{weight}, and lexical factors. The chapter shows that the default position is modifier--\isi{noun} in all the early \ili{Germanic} languages, and that modifiers in \isi{postnominal} position are the result of specific factors. 

In Chapter 4, Kristin Bech studies two Old \ili{English} quantifiers meaning ‘many’: \textit{fela} and \textit{manig}. \textit{Fela} either occurs as the head of the \isi{noun} phrase, taking a \isi{noun} complement in the \isi{genitive} case, or it occurs in \isi{agreement} constructions, with \textit{fela} as the modifier of a nominal head, the latter representing a newer development. \textit{Manig}, on the other hand, consistently occurs with \isi{agreement}. However, \textit{fela} with \isi{agreement} is almost only found in the texts by Ælfric and in the \textit{Peterborough Chronicle}, where there is \isi{variation}, while all other Old \ili{English} texts consistently use \textit{fela} with \isi{genitive}. Bech suggests that the usage in Ælfric and the chronicle is caused by semantic factors, and that it points ahead to later developments in the \isi{noun} phrase. In the lexical competition between \textit{fela} and \textit{manig}, \textit{manig} eventually emerges victorious. 

Alexandra Rehn’s point of departure in Chapter 5 is the \isi{inflection} of stacked adjectives. With reference to both modern \ili{German} and earlier stages of \ili{German}, Rehn considers the \isi{combination} of zero-\isi{inflection} and overt \isi{adjective} \isi{inflection} in Old High \ili{German}, and of uninflected and inflected adjectives in the modern \ili{German} dialect group \ili{Alemannic}. It emerges that Old High \ili{German}, though it has zero-inflected adjectives, does not allow them in stacking, unlike e.g. modern \ili{Scandinavian} languages and \ili{Old Saxon}. Uninflected adjectives in \ili{Alemannic} are only possible in DPs with one \isi{adjective}. Rehn uses the \isi{Obligatory Contour Principle} to account for the distribution, and suggests that \isi{adjective} \isi{inflection} has a double function, both marking features and serving as a linker in stacking.   

Inflectional patterns, specifically those of \isi{attributive} adjectives in Old High \ili{German}, are also the topic of Svetlana Petrova’s study in Chapter 6. Petrova uses two datasets, i.e. bare DPs and DPs containing a determiner-like marker of \isi{definiteness} and \isi{indefiniteness}. Contrary to previous research, she finds that the choice of inflectional \isi{pattern} is not driven by the interpretation of the DP in terms of (in)\isi{definiteness}, but rather that the strong \isi{inflection} occurs with any semantic type of DP, while the weak \isi{inflection} is due to certain grammatical and constructional factors. In addition, Petrova considers how position within the \isi{noun} phrase correlates with \isi{adjectival inflection}, and she ends by showing that the distribution of inflectional patterns in modern \ili{German} started to develop already in Old High \ili{German}. 

In Chapter 7, Hannah Booth takes us to \ili{Old Icelandic} and the proprial \isi{article}, attested across North \ili{Germanic}. Booth shows that focusing on the given/new dimension with respect to the pragmatic function of the \isi{article}, as has been done in previous research, can only provide a partial picture of its precise function. Booth considers the wider information-structural context and different types of topic transition, finding that the proprial \isi{article} in \ili{Old Icelandic} is in fact used as a topic management device to signal different types of \isi{topic shift}. She also observes that a special variant of the proprial \isi{article} functions as a strategy for coordinating referents which differ in their \isi{topicality} status. 

Most of the research on Old \ili{Norse} focuses on \ili{Old Icelandic}, but in Chapter 8, Juliane Tiemann carries out a study of \isi{adjective} position in Old \ili{Norwegian} specifically. Although Old \ili{Norwegian} is quite far advanced in the direction of a fixed \isi{prenominal} position for adjectives, \isi{postnominal} adjectives still occur, as well as \isi{adjective} flanking. Tiemann builds on previous research on positional \isi{variation} within the \isi{noun} phrase, and focuses on how \isi{word order} is mainly determined by \isi{information structure}, suggesting a left periphery in the Old \ili{Norwegian} \isi{noun} phrase, with positions for topic, focus and \isi{contrast}. 

Adjectival articles in early \ili{Germanic} is the topic of Chapter 9 by Alexander Pfaff and George Walkden. The authors discuss the idea that certain items that appear to be definite articles are, in fact, narrow components of an adjectival phrase. For North \ili{Germanic}, this is easily illustrated because the \isi{grammaticalization} path from \isi{demonstrative} took two distinct paths: the nominal (definite) \isi{article} came to be realized as a suffix on the \isi{noun}, whereas an \isi{article} specifically for (weak) adjectives remained a freestanding element. In West and East \ili{Germanic}, no such visible distinction exists, yet the authors show that certain \isi{article} uses of distal demonstratives are not definite articles of nouns that merely happen to be accompanied by an \isi{adjective}, but genuine adjectival articles.  
  
Finally, in Chapter 10, Alexander Pfaff addresses a peculiar class of adjectival modifiers, termed “positional predicates”, that deviate in various ways from “regular” adjectives. The deviation applies to syntactic, semantic and morphosyntactic properties. Syntactically, positional predicates are deviant because they precede determiners and may combine with pronouns and proper names. Semantically, they do not merely modify the \isi{noun}, but express a part--whole relation. In addition, they display the strong \isi{inflection} in an unexpected environment, namely in definite \isi{noun} phrases.\bigskip\\
\noindent Oslo, October 2023\hfill Kristin Bech and Alexander Pfaff\hbox{}\\




% {\sloppy\printbibliography[heading=subbibliography]}
\end{refsection}

