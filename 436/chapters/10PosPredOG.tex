\documentclass[output=paper,colorlinks,citecolor=brown]{langscibook}
\ChapterDOI{10.5281/zenodo.10641201}
\author{Alexander Pfaff\orcid{}\affiliation{University of Stuttgart}}


\title{Positional predicates in early Germanic }

\abstract{This chapter addresses a class of adjectival modifiers that has received relatively little attention in the literature. Those modifiers, referred to here as \textit{positional predicates}, differ from ``regular'' adjectives semantically, syntactically, and, at least in Germanic, morpho-syntactically. Their most outstanding syntactic property is that they precede determiners (prenominally) and combine with pronouns and proper names. On the semantic side, they do not simply modify the noun description, but denote a part--whole relation with respect to the NP referent.   
Starting out from modern Icelandic, Latin and Greek, I will show that items displaying the same deviant behaviour can also be identified in the early Germanic languages to varying degrees. The evidence across the Germanic languages, however, is not equally strong and we find variation, but the discussion suggests that the extant examples are remnants of a system (a class of modifiers/a special adjectival syntax) that must have been more widespread and productive in older stages of Germanic.  }


\IfFileExists{../localcommands.tex}{
   \addbibresource{../localbibliography.bib}
   % add all extra packages you need to load to this file

\usepackage{tabularx,multicol}
\usepackage{url}
\urlstyle{same}

\usepackage{listings}
\lstset{basicstyle=\ttfamily,tabsize=2,breaklines=true}

\usepackage{langsci-basic}
\usepackage{langsci-optional}
\usepackage{langsci-lgr}
\usepackage{langsci-osl}
% \usepackage{./langsci/styles/langsci-lgr}
% \usepackage{./langsci/styles/langsci-osl}
% \usepackage{langsci-gb4e}

\usepackage{tikz}
\usetikzlibrary{patterns,calc}
\pgfdeclarepatternformonly{south east lines}{\pgfqpoint{-0pt}{-0pt}}{\pgfqpoint{3pt}{3pt}}{\pgfqpoint{3pt}{3pt}}{
    \pgfsetlinewidth{0.6pt}
    \pgfpathmoveto{\pgfqpoint{0pt}{3pt}}
    \pgfpathlineto{\pgfqpoint{3pt}{0pt}}
    \pgfpathmoveto{\pgfqpoint{.2pt}{-.2pt}}
    \pgfpathlineto{\pgfqpoint{-.2pt}{.2pt}}
    \pgfpathmoveto{\pgfqpoint{3.2pt}{2.8pt}}
    \pgfpathlineto{\pgfqpoint{2.8pt}{3.2pt}}
    \pgfusepath{stroke}}
    
\usepackage{stmaryrd}
\usepackage{wasysym}
\usepackage{multirow}
\usepackage{caption}
\usepackage{subcaption}
\usepackage{mathrsfs}
\usepackage{qtree}

\usepackage{linguex}


   %pminos do not split footnotes
% \interfootnotelinepenalty=10000 %Footnote in Laporte chapters has to be split SN


%\DeclareIndexNameFormat{default}{%
%\nameparts{#1}%
%\usebibmacro{index:name}%
%{\index[names]}%
%{\namepartfamily}%
%{\namepartgiveni}%
% {}% L1
% {}% L2
%{\namepartprefix}% generates spurious space L3
%{\namepartsuffix}% generates spurious space L4
%}

%  {\DeclareIndexNameFormat{default}{%
%     \usebibmacro{index:name}{\index[names]}{#1}{#3}{#5}{#7}}}

%\DeclareIndexNameFormat{default}{%
%  \usebibmacro{index:name}{\sindex[nom]}{#1}{#3}{#5}{#7}}

%\DeclareIndexNameFormat{default}{%
%  \usebibmacro{index:name}{\sindex[person]}{#1}{#3}{#5}{#7}}
%\DeclareIndexNameFormat{default}{%
%\nameparts{#1} \usebibmacro{index:name}{\sindex[person]]}{\namepartfamily}{‌​\namepartgiven}{\nam‌​epartprefix}{\namepa‌​rtsuffix}}

%\newcommand{\smiley}{:)}

%\renewbibmacro*{index:name}[5]{%
%\usebibmacro{index:entry}{#1}%
%{\iffieldundef{usera}{}{\thefield{usera}\actualoperator}\mkbibindexname{#2}{#3}{#4}{#5}}}

% \newcommand{\noop}[1]{}

%remove for final
%\overfullrule=1mm

\newcommand{\tobi}[2]}}
\renewcommand{\S}[1]{\tobi{#1}{\textsc{*}}}

% this volume references
% puts: [this volume]
% already defined: \citetv
%\newcommand{\citepv}[1]{(\citeauthor{#1} \citeyear*{#1} [this volume])}
\newcommand{\citealtv}[1]{\citeauthor{#1} \citeyear*{#1} [this volume]}

%parentheses around example number
\newcommand{\pref}[1]{(\ref{#1})}

% in-text examples

\newcommand{\lnex}[1]{\textit{#1}} %target lang word
\newcommand{\lnlit}[1]{(lit.: `#1')} %literal reading
\newcommand{\lnlat}[1]{(#1)} % latinization
\newcommand{\lntrans}[1]{`#1'} %translation
\newcommand{\lnexl}[2]%
{\lnex{#1}{} \lnlat{#2}} % ex with latinization
\newcommand{\lnexlat}[3]{\lnex{#1}{} \lnlat{#2}{} \lntrans{#3}} % ex with latinization and tranl.

%ch01
\newcommand{\co}[1]{\mbox{\textbf{#1}}}

%ch09

\newcommand{\cyrbulg}[1]{\begin{otherlanguage*}{bulgarian}#1\end{otherlanguage*}}


%ch10
\newcommand{\nlp}{{\small NLP}}
\newcommand{\mwe}{{\small MWE}}
\newcommand{\rae}{{\small RAE}}
\newcommand{\lvc}{{\small LVC}}
\newcommand{\pos}{{\small P}o{\small S}}
%\newcommand{\todo}[1]{ \textcolor{red}{#1} }

%\renewcommand{\labelenumi}{\theenumi}
%\ainamefmt{{vv}{ll}{, ff}{, jj}} % fullname

\newcommand{\biberror}[1]{{\color{red}#1}}

\newcommand{\osenovaitem}{--~}
   %% hyphenation points for line breaks
%% Normally, automatic hyphenation in LaTeX is very good
%% If a word is mis-hyphenated, add it to this file
%%
%% add information to TeX file before \begin{document} with:
%% %% hyphenation points for line breaks
%% Normally, automatic hyphenation in LaTeX is very good
%% If a word is mis-hyphenated, add it to this file
%%
%% add information to TeX file before \begin{document} with:
%% %% hyphenation points for line breaks
%% Normally, automatic hyphenation in LaTeX is very good
%% If a word is mis-hyphenated, add it to this file
%%
%% add information to TeX file before \begin{document} with:
%% \include{localhyphenation}
\hyphenation{
    Beck-man
    Ngu-yen
    back-chan-nel
    back-chan-nels
    mo-not-o-nous
    ste-reo-typ-i-cal
}

\hyphenation{
    Beck-man
    Ngu-yen
    back-chan-nel
    back-chan-nels
    mo-not-o-nous
    ste-reo-typ-i-cal
}

\hyphenation{
    Beck-man
    Ngu-yen
    back-chan-nel
    back-chan-nels
    mo-not-o-nous
    ste-reo-typ-i-cal
}

   \boolfalse{bookcompile}
   \togglepaper[10]%%chapternumber
}{}

\begin{document}
\maketitle

\section{Introduction} 

In this chapter, I will discuss a class of adjectival modifiers that has received relatively little attention in the literature, and that will be referred to here 
as \textit{positional predicates}.\footnote{Notable exceptions are \citet{Marib} for \ili{Latin}, and \citet{Pfaff2015,Pfaff2017} for \ili{Icelandic}; some relevant discussion is also found in \citet{Fischer01} and   \citet{Grabski17} in the context of Old \ili{English} \isi{adjective} placement. The term ``\isi{positional predicate}'' is adopted from \citet{Pfaff2015,Pfaff2017}. }  
Two examples from \ili{Icelandic} are given in (\ref{PosIce0}).


\begin{exe}
   \ex \label{PosIce0}
     \begin{xlist}  
         \ex \gll á \textbf{norðanverðu} nesinu \\
           on  northern   peninsula.\textsc{def}  \\
       \glt `on the northern part of the peninsula'  
         \ex \gll á \textbf{ofanverðri} þessari öld \\
           on   upper  \textsc{dem}  century  \\
         \glt `in the latter part of this century'  
    \end{xlist}
\end{exe}


As the discussion will show, positional predicates are adjectival elements even though they display a number of peculiarities that clearly set them apart from ``regular'' adjectives; as illustrated in (\ref{PosIce0}), they denote a part--whole relationship and they may precede (definite) determiners. 

This chapter primarily provides an overview and tries to establish the phenomenon by showing that positional predicates are a deviant class of adjectival modifiers and constitute a worthwhile object of investigation in their own right. Moreover, I will show that it is a topic relevant to the study of (\isi{comparative}) early \ili{Germanic} \isi{syntax}. The discussion itself will draw on data from modern \ili{Icelandic} and early \ili{Germanic} languages, but also from \ili{Latin} and \ili{Classical Greek}.  
A secondary, but related purpose is to motivate a separate annotation label for positional predicates in the NPEGL \isi{database}, as will be explained in Section \ref{sec:10:npegl}.


The structure of this chapter is as follows: In Section \ref{sec:10:PosPrr}, I discuss the phenomenon and the prototypical characteristics following the exposition in \citet{Pfaff2015,Pfaff2017} on positional predicates in modern \ili{Icelandic}. Section \ref{sec:10:issu} is concerned with a number of general issues of interest. First, I summarize the account of ``\isi{partitive} adjectives'' in \ili{Latin} by \citet{Marib} and discuss the so-called ``\isi{predicative} position'' in \ili{Ancient Greek}. It will transpire that there is a significant overlap between those \isi{partitive} adjectives occurring in the \isi{predicative} position (in \ili{Latin} and \ili{Greek}) and positional predicates, and that, to a significant extent, they can be treated as the same phenomenon.
I then compare \isi{agreement} vs. \isi{genitive} constructions, the latter representing an alternative strategy and, presumably, a later development. Section \ref{sec:10:POG} discusses relevant data from various early \ili{Germanic} languages. I will illustrate their prototypical behaviour, and  point out some language-specific deviations. Section \ref{sec:10:sum} concludes.




\subsection{Annotated corpora and the NPEGL database}
\label{sec:10:npegl}

\begin{sloppypar}
One practical purpose of the project \emph{Constraints on syntactic \isi{variation}: \isi{Noun} phrases in early \ili{Germanic} languages}\footnote{Funded by the Research Council of Norway (grant no. 261847).} has been the creation of an annotated \isi{noun} phrase \isi{database} (NPEGL). While many annotated corpora (notably those that fed into NPEGL) use the label ``Adjective'' for a broad class of adjectival elements, the NPEGL annotation (see \citetv{chapters/1Database}) divides the class of modifiers into (i) adjectives (in a narrow sense),  (ii) cardinal elements, and (iii) positional predicates. Since positional predicates are not an established class, this label needs to be motivated,\footnote{Initially, this chapter was meant to be an appendix to \textcitetv{chapters/1Database}, precisely for the purpose of elaborating on and motivating this label. } which raises the following questions:
\end{sloppypar}
\begin{enumerate}
  \item[1.] What are positional predicates in the first place?
%  \item[2.] Do they constitute a phenomenon of interest?
  \item[2.] Did they exist in in the early \ili{Germanic} languages? and if so:
  \item[3.] Are they a relevant topic to the study of syntactic \isi{variation}?
\end{enumerate}
%\noindent Thus a corollary purpose of this \isi{article} is to answer these questions (in the affirmative). \\

Originally, the term  was used to describe a small class of modifiers in modern \ili{Icelandic} that deviate from regular adjectives, syntactically and semantically, see Section \ref{sec:10:PosPrr}. In the initial phase of the project, there was some evidence that we might also find items with a similar deviant behaviour in early \ili{Germanic} languages. Since this is a phenomenon of potential interest to the project, a closer look  at the issue was warranted. As a consequence, the annotation itself has been a part of the investigation into positional predicates in order to determine how widespread/frequent the phenomenon is %and which modifiers of this type we would find in the relevant contexts %in the extant texts
in the first place. Since annotation is still in progress at the time of writing, no final results or definite numbers can be provided here. However, even though we may not find too many attestations in the extant texts, there are indications that it was a native phenomenon, not imported via scholarly translations from \ili{Greek} or \ili{Latin}, and, by extension, that it must have been a component of early \ili{Germanic} \isi{syntax}. Formulated more carefully, in all early \ili{Germanic} languages, we find remnants of a presumably older system that must have been productive in  Proto-\ili{Germanic} and has survived through Old \ili{Norse} into modern \ili{Icelandic}.

As we will see, positional predicates are more versatile than regular adjectives in that they may occur in non-canonical \isi{adjective} positions. For instance, they precede determiners, combine with pronouns and proper names, and, at least in Old \ili{English}, they occur much more frequently and easily in \isi{postnominal} position. Thus, not making a distinction amounts to missing out on potentially relevant insights.  Notably, when examining \isi{adjective} ordering/placement or the distribution of \isi{adjectival inflection}, the results are, in all probability, more precise if positional predicates are treated as a separate class. At the same time, there are noticeable differences among the individual languages, and thus, positional predicates are clearly a topic relevant to the study of (\isi{word order}) \isi{variation} in early \ili{Germanic}, and a separate annotation label is warranted.  \\

\noindent In addition to NPEGL, the following corpora/sources have been consulted for examples:

\begin{enumerate} 
    \item \textit{Perseus} (\ili{Classical Greek}) \hfill = \textit{Perseus} \\
    \url{http://www.perseus.tufts.edu/hopper/collections}
    \item \textit{Project Wulfila} (\ili{Gothic}/\ili{Biblical Greek}) \hfill = \textit{Wulfila} \\ 
    \url{http://www.wulfila.be/gothic/}
    \item \textit{Referenzkorpus Altdeutsch} 1.1 (Old High \ili{German}) \hfill = \textit{ReA} \\ 
    \url{https://korpling.german.hu-berlin.de/annis3/ddd}
    \item \textit{Saga Corpus} (\ili{Old Icelandic}) \hfill = \textit{Saga} \\ 
    \url{https://malheildir.arnastofnun.is/?mode=fornrit}
    \item \textit{Bosworth-Toller Anglo-Saxon Dictionary online} (Old \ili{English})  \\ \url{https://bosworthtoller.com}
\end{enumerate}





\section{Characteristics and notable features}
\label{sec:10:PosPrr}

I will start out by looking at the properties of positional predicates in modern \ili{Icelandic}, largely summarizing the exposition in \citet{Pfaff2015,Pfaff2017}. Next I will show that, based on the same criteria, cognates with rather similar properties can also be identified in the early \ili{Germanic} languages.


\subsection{Positional predicates in modern Icelandic}
\label{sec:10:posprICE}

\citet{Pfaff2015,Pfaff2017} characterizes positional predicates as expressing a temporal/ spatial part--whole relation relative to the nominal referent; the respective \isi{noun} denotes a temporal or spatial extension or a plurality, cf. (\ref{PosIce1}).\footnote{\label{fnPos}Notice that the regular definite \isi{article} in \ili{Icelandic} is a bound morpheme occurring suffixed to the \isi{noun}, and will be glossed as \textsc{def}.  Note also that the glosses for the positional predicates themselves will be an approximation since there are no direct (lexical) equivalents in \ili{English}. }


\begin{exe}
    \ex\label{PosIce1}
       \begin{xlist} %  
            \ex \gll á \textbf{norðanverðri} eyjunni \\
               on northern island.\textsc{def}  \\
      \glt `on the northern part of the island'  

    \ex \gll á \textbf{ofanverðu} tímabilinu \\
       on    latter    period.\textsc{def} \\
    \glt `in the latter part of  the period'   

    \ex \gll í \textbf{miðri} borginni     \\ 
       in  middle  city.\textsc{def}  \\
    \glt `in the middle part of the city'  

    \ex \gll á \textbf{öndverðri}  öldinni \\
       on  former century.\textsc{def}   \\
     \glt `in the early part of the century'   
  \end{xlist}
\end{exe} 

Paraphrases involving the component  ``x-part of the N'' are a useful first  approximation, but in some cases, a more elaborate translation may be called for. Consider the examples in (\ref{position}).


\begin{exe}
   \ex\label{position}
     \begin{xlist}  
       \ex \gll í \textbf{miðjum} áhorfendum     \\ 
          in  middle  spectators  \\
         \glt `amidst/among/between the spectators'  
       \ex \gll um \textbf{þveran} heiminn \\
          about  across  world.\textsc{def} \\
       \glt  `around/across the world'     
      \ex \gll eftir \textbf{endilöngu} landinu \\
         after  along  land.\textsc{def} \\
      \glt `from one part of the country to the other'
   \end{xlist}
\end{exe} 

The paraphrases may often give the impression that positional predicates are simply elements of complex adverbial or  prepositional expressions.  
This impression may be compounded by the fact that, in most cases, they do occur as part of an actual PP.\footnote{However, in principle, they can occur in \isi{noun} phrases not embedded under a preposition, e.g. as part of a subject or where the \isi{noun} phrase itself is used as an adverbial expression: 

\begin{exe}
  \ex\label{positionB}
   \begin{xlist} 
    \ex \gll \textbf{öndverður}  veturinn  \  var kaldur  \\
         beginning winter.\textsc{def}.\textsc{nom} \  was cold   \\
      \glt `the  beginning of the winter (was cold)' \ $\sim$ `the winter in its early part (was cold)'
    \ex \gll \textbf{öndverðan}  veturinn \  ({kom hann heim}) \\
       beginning   winter.\textsc{def}.\textsc{acc} \ ({came he home})  \\
      \glt `at the beginning  of the winter (he returned)'  \ \ (adverbial accusative) 
  \end{xlist}
\end{exe} }  
Also the glosses themselves may be misleading insofar as they involve adverbs (\textit{across, along}), nouns (\textit{beginning}), and adjectives (\textit{northern}) that, by themselves, not always fully convey the appropriate meaning; see  fn. \ref{fnPos}.

Crucially, however, like regular adjectives, positional predicates agree in case, number, and gender with their respective \isi{noun}, cf. (\ref{positionII}).

\begin{exe}
  \ex\label{positionII}
   \begin{xlist}  
    \ex \gll í miðj \textbf{-um} bænum   \\ 
     in  middle  -\textsc{m.dat.sg} town.\textsc{def}.\textsc{dat.sg (m)} \\
    \ex \gll í mið \textbf{-ri} borginni   \\ 
     in middle -\textsc{f.dat.sg}  city.\textsc{def}.\textsc{dat.sg (f)} \\
    \ex \gll á miðj \textbf{-u} sumrinu   \\ 
     in  middle -\textsc{n.dat.sg} summer.\textsc{def}.\textsc{dat.sg (n)} \\
    \ex \gll um miðj -\textbf{an} mánuðinn   \\ 
      about  middle -\textsc{m.acc.sg}  month.\textsc{def}.\textsc{acc.sg (m)} \\
    \ex \gll um mit -\textbf{t} hverfið   \\ 
      around middle -\textsc{n.acc.sg} neighbourhood.\textsc{def}.\textsc{acc.sg  (n)} \\
  \end{xlist}
\end{exe} 

One striking feature of positional predicates is that they usually occur in  definite \isi{noun} phrases, and even if the \isi{noun} is  not overtly marked for \isi{definiteness}, the interpretation is definite nonetheless (\ref{pos3}).


\begin{multicols}{2}{
\begin{exe}
  \ex\label{pos3}
   \begin{xlist}
      \ex \gll í miðjum bæ \textbf{-num} \\
        in middle town \textsc{-def} \\  
        \glt `in the middle of the town' \\ \ \\

     \ex \gll í miðjum bæ \\
       in middle town  \\ 
       \glt `in the middle of the town' \\
       \glt $\#$  `in the middle of a town'  \\ 
       \glt $\#\#$`in a middle of the  town'
  \end{xlist}
\end{exe} }
\end{multicols}

In this context, it must be pointed out that ``regular'' adjectives in definite \isi{noun} phrases occur in the so-called weak \isi{inflection} (\ref{stora}).\footnote{While the strong \isi{inflection} is largely a PIE heritage, the weak \isi{inflection} is a \ili{Germanic} innovation/phenomenon; the strong/weak distinction has survived into most modern \ili{Germanic} languages. Traditionally, it has been associated with (the semantic expression/morphological marking of) \isi{definiteness}, even though this is a simplification, both diachronically and e.g. for modern \ili{German} (\citealt{ratkus2011, Pfaff2017,Pfaff2019, Rehn2019}, \citetv{chapters/6StrWeakOHG}). }  

{\small 
\begin{multicols}{2}{
\begin{exe}
  \ex \label{stora}
     \begin{xlist}
        \ex \gll í stór-\textbf{a} bæ\textbf{-num} \\
          in big-\textsc{wk} town\textsc{-def} \\ 
           \glt `in the big town' 
        \ex \gll í falleg-\textbf{a} bæ\textbf{-num} \\
           in beautiful-\textsc{wk} town\textsc{-def}  \\ 
           \glt `in the beautiful town'  
  \end{xlist}
\end{exe}}
\end{multicols}}

Positional predicates, in \isi{contrast}, consistently have strong \isi{inflection} (\ref{pos2a}).

\begin{multicols}{2}{
\begin{exe}
  \ex\label{pos2a}
   \begin{xlist}
    \ex \gll í miðj-\textbf{um} bæ\textbf{-num}  \\
        in middle-\textsc{str} town-\textsc{def} \\ 
    \ex \gll *í miðj-\textbf{a} bæ\textbf{-num}  \\
      \phantom{*}in middle-\textsc{wk}  town-\textsc{def} \\    
  \end{xlist}
\end{exe} }
\end{multicols}

As an extension of the above observation that positional predicates combine with definite \isi{noun} phrases, we find that positional predicates can also modify proper names and personal pronouns, as in (\ref{position4}).

\begin{exe}
  \ex \label{position4}
    \begin{xlist}
     \ex \gll á sunnanverðri  \textbf{{Ítalíu}} \\
       on southern.\textsc{str} Italy  \\
       \glt `in the southern part of Italy' 
    \ex  \label{position4b} \gll  {Við sáum á} {og brú} í \textbf{{henni}} miðri  \\
         {we saw river} {and bridge} in \textsc{prn.f.dat.sg} middle.\textsc{f.dat.sg.str}  \\ 
         \glt  `We saw a river and a bridge in the middle of it' % \\
       %  \glt ($^*$\textit{á} `river' is feminine)

   \end{xlist}
\end{exe} 
 


Notice that positional predicates follow the \isi{pronoun}, as in (\ref{position4b}),\footnote{Occasionally, positional predicates may be found preceding a \isi{pronoun}.    
However, Einar Freyr Sigurðsson (p.c.) points out that the post-pronominal position is more natural (or the default).  } whereas they \textit{precede the noun} in the other examples discussed so far. This is the default situation in modern \ili{Icelandic} -- even though we  may find \isi{postnominal} occurrences as well, with no apparent difference in meaning (\ref{pospr3}).

\begin{multicols}{2}{ 
\begin{exe}
  \ex\label{pospr3}
   \begin{xlist}
    \ex \gll  að aftanverðu  húsinu  \\
        to back house.\textsc{def} \\ 
        \glt `to/at the back of the house'
    \ex \gll  að húsinu  aftanverðu  \\
        to house.\textsc{def}  back   \\ 
        \glt `to/at the back of the house'
  \end{xlist}
\end{exe} }
\end{multicols} 


\begin{sloppypar}
Finally,  positional predicates precede determiners such as demonstratives, the freestanding \isi{article},  pronominal possessives and quantifiers. This differs significantly from the position of ``regular'' adjectives (between \isi{determiner} and \isi{noun}), cf. (\ref{position5}).
\end{sloppypar}


\begin{exe}
   \ex \label{position5}
    \begin{xlist}
      	\ex \label{position5a} \gll á ofanverðri \textbf{þessari}  öld   \\
      	 on latter \textsc{dem}  century  \\
      	 \glt `in the latter part of this century' 
      	\ex \label{position5b} \gll í miðri \textbf{{{hinni}}} alþjóðlegu fjármálakreppu \\
      	 in middle.\textsc{str} \textsc{art} international.\textsc{wk} financial.crisis \\
         \glt  `halfway through the international financial crisis' 
      	\ex \label{position5c} \gll meðan hún var í miðri \textbf{sinni} ræðu \\
        	while she was in middle her speech \\
      	\glt `while she was giving  her speech'
      	\ex \label{position5d} \gll  í miðri  \textbf{allri} \textbf{þeirri} pólitísku óróleika \\  
          in  middle.\textsc{str} all \textsc{dem}  political.\textsc{wk} unrest  \\ 
          \glt `in the midst of all that political turmoil'
      	\ex \label{position5e} \gll   í  miðjum \textbf{öllum} öðrum leikmönnum \\
               in middle all other players   \\ 
               \glt `amidst all other players'
    \end{xlist}
\end{exe}

This position is not merely an option: they \textit{cannot} follow a \isi{determiner} (\ref{cannotfollow}). 

\begin{exe}
      \ex \label{cannotfollow}\gll $*$á \textbf{þessari} ofanverð-ri / ofanverð-u  öld   \\
       \phantom{$*$}on \textsc{dem}  latter-\textsc{str} /  latter-\textsc{wk} century \\ 
\end{exe} 

In (both Old and modern) \ili{Icelandic}, positional predicates appear to form a closed class; i.e. there is only a small class of elements displaying the set of properties described above. The list in (\ref{ICE1}) is essentially an exhaustive(?) list.\footnote{If it were not for the elements \textit{miður, þver, endilangur}, they could also be construed as one morphological class; notice that the other elements are morphologically complex dividing into a locational component plus the suffix \textit{-verð-}; cf. Engl. \textit{(back)-ward(s)}; Germ. \textit{(rück)-wärts}.} 



{\small
\begin{exe}
  \ex \label{ICE1} Positional predicates in (Old and modern) Icelandic: inventory
\begin{multicols}{3}{
     \begin{xlist}
	\ex     \textbf{miður}     \\    {middle} 

	\ex     \textbf{þver}     \\    {across, transverse} 

	\ex     \textbf{endilangur}     \\    {along} 

	\ex     \textbf{öndverður}     \\    {former part} 

	\ex     \textbf{ofanverður}     \\    {latter/upper part} 

	\ex     \textbf{neðanverður}     \\    {lower part} 

	\ex     \textbf{framanverður}     \\    {front part} 

	\ex     \textbf{aftanverður}     \\    {back part} 

	\ex     \textbf{utanverður}     \\    {outer part} 

	\ex     \textbf{innanverður}     \\    {interior part}  

	\ex     \textbf{vestanverður}     \\    {western part} 

	\ex     \textbf{austanverður}     \\    {eastern part} 

	\ex     \textbf{norðanverður}     \\    {northern part} 

	\ex     \textbf{sunnanverður}     \\    {southern part}  \\ 

     \end{xlist}  }
\end{multicols}  
\end{exe} }


This brief summary shows that the behaviour of this class of modifiers differs considerably from the behaviour of regular adjectives in terms of \isi{syntax}, semantics and, at least partially, morphosyntax, which justifies treating them as a separate group. 

\subsection{Positional predicates in early Germanic}
\label{sec:10:posprdEG}

In Section \ref{sec:10:posprICE}, the following characteristics of positional predicates for modern Icelandic were identified, see (\ref{IceCHAR}).

\begin{exe}
\ex \label{IceCHAR} 
\begin{enumerate}\sloppy
     \item[(i)] \label{enum:IceCHARi} temporal/spatial part--whole relation (\isi{noun} denotes temporal/spatial extension or plurality),
    \item[(ii)]  \label{enum:IceCHARii} \isi{agreement} in case, number and gender with the rest of the \isi{noun} phrase 
    (like ``regular'' adnominal \isi{adjective}), 
    \item[(iii)] strong \isi{adjectival inflection} 
    (in spite of occurring in definite contexts),   
    \item[(iv)] combining with definite \isi{noun} phrases, including pronouns and proper names (definite interpretation even when not overtly marked as definite),   
    \item[(v)] \label{IceCHARv} preceding (definite) determiners,
    \item[(vi)] (default position: \isi{prenominal} and post-pronominal),
    \item[(vii)] (paraphrase by PP/adverbial expression).
\end{enumerate}
\end{exe} 


Outside \ili{Icelandic}, positional predicates are not found in the modern \ili{Germanic} languages. However, based on the criteria discussed in Section \ref{sec:10:posprICE}, we can diagnose items in the early \ili{Germanic} languages that appear to display the same properties. A brief illustration using the item `middle (part-of)' is given in (\ref{position6a})--(\ref{position6Sx}).

\begin{exe}
   \ex \label{position6a} \ili{Old Icelandic}
    \begin{xlist}
      \ex\gll  í \textbf{miðju}  héraði-nu    \\     
       in middle.\textsc{str}   district-\textsc{def} \\ 
       \glt  `in the middle of the district' (Saga, Vopnfirðinga saga) 

     \ex \gll að \textbf{miðjum} Noregi   \\ 
        towards  middle.\textsc{str} Norway \\ 
       \glt  `towards Mid-Norway' (Saga, Egils saga Skallagrímssonar)
    \end{xlist} 

   \ex   Old \ili{English}
    \begin{xlist}
    \ex \gll of \textbf{middre} þære bremelþyrnan   \\  
        from  middle.\textsc{str} \textsc{dem}  {bramble.bush} \\ 
       \glt  `from the middle of the bramble bush' (NPEGL, OEng.458.602)

      \ex \label{position6d} \gll on \textbf{middum} ðinum temple  \hspace{2mm}(Lat. in medio templo tui) \\
        in middle.\textsc{str} your temple \\ 
       \glt  `in the middle of your temple' (\url{https://bosworthtoller.com/22789})
    \end{xlist} 

   \ex   \ili{Gothic}
    \begin{xlist}
      \ex \gll  in \textbf{midj{aim}}  laisarjam    \\ 
         in middle.\textsc{str} teachers  \\ 
       \glt  `in the midst of the doctors' (Wulfila, Luke 2:46)               

     \ex \gll þairh \textbf{midja}  {Samarian jah Galeilaian}  \\ 
        through middle.\textsc{str} {Samaria and Galilee} \\  
        \glt `through the midst of Samaria and Galilee' (Wulfila, Luke 17:11)
    \end{xlist} 

   \ex   Old High \ili{German} 
    \begin{xlist}
    \ex \gll in \textbf{míttemo} iro rínge   \\  
       in middle.\textsc{str} their circle    \\ % 
       \glt  `in their midst' (ReA, O\_Otfr.Ev.4.19)

      \ex \gll Untar \textbf{mitten}  íu      \\  
        among middle.\textsc{str} you  (\textsc{dat.pl}) \\ 
       \glt  `among your midst' (ReA, T\_Tat13)
    \end{xlist} 

   \ex \label{position6Sx} \ili{Old Saxon}
    \begin{xlist}

    \ex \gll an \textbf{middian} dag     \\ 
      on middle.\textsc{str} day  \\ 
        \glt  `in the middle of the day' (NPEGL, OSax.444.216)

     \ex \gll under iu \textbf{middeon}   \\ 
       among you middle.\textsc{str} (\textsc{dat.pl})   \\  
       \glt  `among your midst' (NPEGL, OSax.367.476)   

     \end{xlist}  
\end{exe}

Apart from the fact that the items in question are etymologically related, these examples also display the syntactic peculiarities of positional predicates in modern \ili{Icelandic} (preceding determiners, strong \isi{inflection} in definite contexts, combining with pronouns, etc.). Thus they are suggestive evidence of the idea that positional predicates and/or something akin to a positional-predicate \isi{syntax} may be found in early \ili{Germanic}.  
Individual occurrences may not be overly frequent in the extant texts, and for some languages  merely a handful of attestations have been identified (so far). But various examples suggest that the pecularities are not merely the result of adaptation in the process of translation.\footnote{See e.g. (\ref{position6d}), where the position of the possessive is \isi{postnominal} in the \ili{Latin} source (in parentheses), but \isi{prenominal} in the Old \ili{English} rendering; the noteworthy observation is here that the item \textit{middum} precedes the possessive in the latter. } 
The deeper implication of this observation is that we are looking at a phenomenon native to the early \ili{Germanic} languages, and that, even where we only find few attestations, the respective examples can be viewed as remnants of an older system that must have been productive in Proto-\ili{Germanic}. 

However, before examining the data from early \ili{Germanic} in more detail in Section \ref{sec:10:POG},  I will take a look at \ili{Latin} and \ili{Ancient Greek}, and discuss the \isi{syntax} of positional predicates and alternatives to the \isi{agreement} construction.


\section{Positions and agreement}
\label{sec:10:issu}

\subsection{Latin and  Greek: The ``predicative'' position}
\label{sec:10:predAtt}

\citet{Marib} examines the ``\isi{partitive} reading'' of certain adjectives in \ili{Latin} which she contrasts with a ``restrictive reading''. Some examples are given in \tabref{LATtabfig1}.

\begin{table}
\caption{\label{LATtabfig1} Restrictive and partitive readings of Latin adjectives; adapted from \citet[361]{Marib}}
    \begin{tabular}{l  l   l}
    \lsptoprule
       	 			              & Restrictive reading   & Partitive reading \\
       \textit{summus mons}       & {the highest mountain}		   &	{the top of the mountain} \\   
       (title of the \isi{article})	  &  					           &	 (= `the highest part of') \\
       \midrule
       \textit{primo vere}	      & {the first spring}			   &	{at the beginning of the spring} \\
       \textit{in ultima platea}  & {on the last square}		   &	{at the end of the square} \\
       \textit{in imo dolio}	  & {in the deep jar}			   &	{in the bottom of the jar} \\
       \textit{in media insula}	  & {in the middle island}		   &	{in the middle of the island} \\
       \lspbottomrule
    \end{tabular}
\end{table}



As with the \ili{Icelandic} examples in Section \ref{sec:10:posprICE}, the interpretation is definite, even though \ili{Latin} does not, for the most part,  give any overt cues in terms of \isi{definiteness} marking or configuration. %\footnote{\citet[363]{Marib} shows that pre- vs. \isi{postnominal} position cannot be unambiguously associated with either \isi{definiteness} or \isi{indefiniteness}. }
But we do find occurrences with proper names and pronouns; the examples in (\ref{lat2}) are from \citet[364]{Marib}. % (abbreviated). 

\begin{multicols}{2}{
\begin{exe} 
   \ex \label{lat2} 
     \begin{xlist}  
         \ex \gll  in \textbf{ultima} Phrygia  \\  
            in final Phrygia    \\ 
            \glt `at the end of Phrygia'  
         \ex \gll  in \textbf{medios} eos  \\ 
            in middle them   \\ 
            \glt `in their midst'  
     \end{xlist}
\end{exe} }
 \end{multicols}  


\ili{Classical Greek}, on the other hand, overtly distinguishes between two constellations, traditionally referred to as ``\isi{attributive}'' and  ``\isi{predicative}'' position. It must be stressed that ``\isi{predicative} position'' here is close to a technical term defined with reference to the \isi{article} position.\footnote{So it should not be confused with what may otherwise also qualify as a ``\isi{predicative} position'' (e.g. the complement of a copula verb). }  
An \isi{adjective} occurs in the \isi{attributive} position if it is preceded by the definite \isi{article} (either pre-  or postnominally), but in the \isi{predicative} position if it either precedes the \isi{article} prenominally or occurs without \isi{article} in \isi{postnominal} position. With  ``regular'' adjectives, this terminology is straightforward, distinguishing attributes from (secondary) predicates (\ref{Gr1}).\footnote{Examples (\ref{Gr1}) and (\ref{latGr}) are taken from \citet[169--171]{bornegriech}, but comparable examples can be found in most grammars or textbooks of Ancient (= Classical or Biblical) \ili{Greek}. } 


\begin{exe}
   \ex \label{Gr1}
     \begin{xlist}
         \ex \label{Gr1a}  Attributive position  \\ 
             \gll ho agathos an\={e}r\hspace{14mm}or\hspace{10mm}   ho an\={e}r h{o} agathos  \\ 
               the good man the man the good   \\ 
               \glt `the good man'   
         \ex \label{Gr1b}  Predicative position  \\ 
             \gll  agathos h{o}  an\={e}r\hspace{14mm}or\hspace{10mm}  h{o} an\={e}r agathos   \\  
               good the  man  the  man  good \\
               \glt `the man is good' \ \ (or: `being good, the man \ {\color{gray} ... did this  or that}')
     \end{xlist}
\end{exe} 


With certain other modifiers, however, this distinction directly correlates with the restrictive and \isi{partitive} readings illustrated in Table \ref{LATtabfig1}, cf. (\ref{latGr}).

\begin{exe}
   \ex \label{latGr}
     \begin{xlist}
         \ex \label{latGra}  Attributive position  \\
             \gll h\={e} eschat\={e} n\={e}sos\hspace{14mm}or\hspace{10mm}  h\={e}  n\={e}sos  h\={e} eschat\={e}  \\ 
               the outermost  island  the   island   the outermost   \\ 
               \glt `the outermost island (out of several islands)'  \hfill $\rightarrow$ restrictive reading
         \ex \label{latGrb}  Predicative position   \\ 
             \gll  eschat\={e} h\={e} n\={e}sos\hspace{14mm}or\hspace{10mm}  h\={e}  n\={e}sos    eschat\={e}  \\ 
               outermost the  island  the  island  outermost    \\
               \glt `the outermost \textit{part of} the  island' \hfill $\rightarrow$ \isi{partitive} reading
     \end{xlist}
\end{exe} 

Thus \ili{Greek} overtly marks a structural distinction between the restrictive and \isi{partitive} readings that is not normally visible in \ili{Latin}. I will follow Romero in assuming that the underlying structure is the same: even though \ili{Latin} does not have articles, adjectives with the \isi{partitive} reading occur in the \isi{predicative} position, but adjectives with the restrictive reading occur in the \isi{attributive} position. 


\subsection{``Attributive'' vs. ``predicative'' position in (Old) Icelandic?} 
\label{sec:10:attrPIC}

It transpires that there is a significant overlap between ``adjectives with a \isi{partitive} reading''/``adjectives in the \isi{predicative} position''  and ``positional predicates''.  Some further clarification is in order, though. Notice that, although the class of adjectives that can occur with a \isi{partitive} reading in \ili{Latin} and \ili{Greek} is subject to semantic restrictions -- ordering adjectives (ordinal numerals or superlatives), adjectives that express a temporal/spatial dimension (\textit{deep, high, low ...}) etc., it is not  immediately clear that they constitute a closed class in the strict sense. %; after all, the same lexical item can occur in different configurations and  have other readings, as illustrated in (\ref{LAT1})/(\ref{latGr}).  


On the other hand, we have seen that positional predicates in \ili{Icelandic} do form a closed class, and, differently from \ili{Latin} and \ili{Greek}, they do not give rise to a restrictive/\isi{partitive} ambiguity themselves.  In order to produce such a \isi{contrast}, different lexical items will have to be used (\ref{posRest}).

\begin{exe}
   \ex \label{posRest}
     \begin{xlist}
         \ex  \label{posResta} Ordering \isi{adjective} \ (\isi{superlative}) \hfill $\rightarrow$ restrictive reading  \\ 
            \gll á nyrst-u  eyju-nni   \\ 
              on northern.\textsc{supl-wk}  island-\textsc{def}    \\
              \glt `on the northernmost island (out of several islands)'
         \ex  \label{posRestb} Positional predicate \hfill $\rightarrow$ \isi{partitive} reading  \\  
         \gll á norðanverð-ri eyju-nni   \\ 
           on  northern-\textsc{str}  island-\textsc{def}   \\
           \glt `on the northernmost part of the  island'  
     \end{xlist}
\end{exe} 

Due to the suffixal nature of the definite \isi{article} in \ili{Icelandic}, the two readings in (\ref{posRest}) do not seem to visibly correlate with a  structural distinction; the only apparent difference stems from  the choice of a different lexical item.   
There is, however, a visible morphological distinction: the (restrictive) ordering \isi{adjective} carries the weak \isi{inflection}, cf. (\ref{posResta}), whereas the \isi{positional predicate} is strongly inflected, cf. (\ref{posRestb}), see (\ref{pos2a}).\footnote{Notice the following example also involving an ordering \isi{adjective} (\isi{comparative} form):  

\begin{exe}
    \ex   \gll á nyrð-ri eyjunni   \\ 
       on northern-\textsc{cmpr}  island.\textsc{def}   \\
       \glt `on the northern island (out of two islands)'; `the island to the north'  
\end{exe}

Here, the ending \textit{-ri} is the \isi{comparative} morpheme and should not be confused with the formally identical feminine \isi{dative} singular strong ending  \textit{-ri} in (\ref{posRestb}).  
The \isi{comparative} \isi{inflection} in \ili{Icelandic} is even more impoverished than the weak \isi{inflection}, expressing no case distinctions and no gender/number distinctions other than neuter singular: \textit{-ra} vs. the rest: \textit{-ri}. Crucially, it does not alternate between two sets of endings, and in this sense, the distinction strong vs. weak cannot be meaningfully applied in the first place.  Compare the positioning of the two modifiers ending in \textit{-ri} in (\ref{posRest2a}): pre- vs. post-\isi{article} (= \isi{predicative} vs. \isi{attributive}).    }    \citet{Pfaff2015,Pfaff2017} shows that this morphological difference in \isi{inflection} does, in fact, correlate with a structural difference arguing that ``\isi{adjectival inflection} is a diagnostic for structural position'': weakly inflected adjectives occur in the c-command domain of the \isi{article}, whereas the strongly inflected \isi{adjective} is merged outside the projection comprising \isi{noun} and definite \isi{article}. This external position, in turn, can essentially be equated with the \isi{predicative} position, cf. (\ref{Gr1b}) and (\ref{latGrb}), which also makes reference to the \isi{article} position.  
An example illustrating this \isi{contrast} even better is the following where a descriptive, \isi{predicative} \isi{adjective} occurs with a definite \isi{noun}. Here, the weakly inflected \isi{adjective} receives a restrictive interpretation, but the strongly inflected version of the same \isi{adjective} receives an appositive interpretation (\ref{AppoRest}).  

\begin{exe}
   \ex \label{AppoRest}
     \begin{xlist}
         \ex  Predicative \isi{adjective} \ (weakly inflected) \hfill $\rightarrow$ restrictive reading  \\ 
        \gll  full-i  strákur-inn   \\  
         drunk-\textsc{wk}  boy-\textsc{def}     \\
         \glt `the drunk boy' 
    
     \ex   Predicative \isi{adjective} (strongly inflected) \hfill $\rightarrow$ appositive reading   \\ 
      \gll  full-ur strákur-inn   \\ 
       drunk-\textsc{str}  boy-\textsc{def}     \\
       \glt `the boy, who happens to be drunk' (\citealt[300]{Pfaff2017})
     \end{xlist}
\end{exe} 

Even though not entirely identical, this \isi{contrast} is comparable to the one observed with ``regular'' adjectives in \isi{attributive} vs. \isi{predicative} position in (\ref{Gr1}). The upshot is that the weak vs. strong \isi{inflection} in these examples is indicative of a structural difference akin to the \isi{attributive} vs. \isi{predicative} position in \ili{Greek}. 

Of course, this structural difference is made visible if a freestanding \isi{determiner} is present, as was already illustrated in (\ref{position5}): positional predicates precede determiners and are strongly inflected, while regular adjectives follow the \isi{determiner} and are weakly inflected if the \isi{determiner} is definite, cf. (\ref{position5b}) and (\ref{position5d}). Consider also the examples in (\ref{posRest2}) from \ili{Old Icelandic} and Old \ili{English}.

\begin{exe}
   \ex \label{posRest2}
     \begin{xlist}
         \ex \label{posRest2a} \gll  nær mið-ri hinni vestri byggð  \\ 
          near middle-\textsc{str} \textsc{art} \ western.\textsc{cmpr} settlement  \\
          \glt `near the middle part of the western settlement' (Saga, Landnámabók)
        \ex \label{posRest2b} \gll in midd-re þære micl-an cirican \\ 
         in middle.\textsc{str} \textsc{dem} great-\textsc{wk} church     \\
       \glt  `in the middle of the great church' (NPEGL, OEng.803.266)        
     \end{xlist}
\end{exe} 

In (\ref{posRest2a}), the \isi{adjective} preceding the freestanding \isi{article}  has a \isi{partitive} reading (``middle part of''),  while the one following the \isi{article} has a restrictive reading (= ``not the eastern settlement''). The same goes for (\ref{posRest2b}) where we see, once more, how strong vs. weak \isi{adjectival inflection} correlates with the pre- vs. post-\isi{article} position.  

\newpage
In short, even though positional predicates may be a closed class in \ili{Icelandic} (and in the extant early \ili{Germanic} languages), it can be shown that they have the same structural properties as adjectives occurring in the \isi{predicative} position in \ili{Greek}. \citet{Marib} argues for \ili{Latin} and \ili{Greek} that this \isi{predicative} position is a DP-external position, and \citet{Pfaff2015,Pfaff2017} independently arrives at the same conclusion on the basis of Icelandic data, but largely for the same reasons.  
Technical details notwithstanding, we can state that occurring in this position is the single most important structural property of positional predicates, from which most other properties derive, and which sets them apart from ``regular'' adjectives.  


For clarification, I point out that the term ``\isi{positional predicate}'' as introduced in Section \ref{sec:10:posprICE} strictly speaking conflates three distinct aspects:\footnote{\label{famousFootnote}In addition, the \isi{predicative} position is associated with the strong \isi{inflection} in \ili{Germanic}, a phenomenon not applicable to \ili{Latin} and \ili{Greek}. \citet{Pfaff2015,Pfaff2017} argues that the weak \isi{inflection} is essentially \isi{definiteness} concord indicating that the \isi{adjective} is merged in the c-command domain of a \isi{definiteness} feature in D$^0$, which corresponds to the \isi{attributive} position. With adjectives merged outside the \isi{definiteness} domain (= \isi{predicative} position), on the other hand, the weak \isi{inflection} cannot be triggered, and by default, the \isi{adjective} is strongly inflected.  }

\begin{enumerate}
    \item[(i)] modifier with certain semantic properties that
    \item[(ii)] occurs in the \isi{predicative} position and (as a consequence)
    \item[(iii)] has a \isi{partitive} reading. 
\end{enumerate}

For the most part, I will look at these aspects in conjunction,\footnote{Point (i) expresses merely a semantic restriction for \ili{Latin} and \ili{Greek}, but for \ili{Germanic}, the qualifier ``with certain semantic properties'' is tantamount to belonging to a closed class. It could be worthwhile studying that class as such, notably, the etymology of the items based on Proto-\ili{Germanic} *-\textit{verþ-}, as was suggested by a reviewer; see Section \ref{sec:10:vert}. These originally had a directional meaning and were adjectival in nature, but have developed into adverbs in most \ili{Germanic} languages (Germ.: \textit{rück-wärt-s}; Engl. \textit{back-ward-s}), except for \ili{Icelandic}. } but in Section \ref{sec:10:self}, I will discuss the idea that the \isi{partitive} interpretation may be one possibility of a larger spectrum of readings.



\subsection{Agreement vs. dependent case} 
\label{sec:10:AGR}

As already mentioned, cf. (\ref{positionII}), one configurational key property of positional predicates is that, like regular adjectives, they agree in case, number and gender with the semantic head \isi{noun} denoting the ``source location''.  
 But there is an obvious similarity to constructions involving a corresponding \isi{noun} and \isi{dependency marking} on the semantic head \isi{noun} (cf. \ili{English} \textit{the middle of the city}).\footnote{Likewise, certain adverbs modifying a PP could be mentioned in this context; cf.  \ili{German}:
\begin{multicols}{2}{
\begin{exe}
 \ex
   \begin{xlist}
     \ex \gll mitten \phantom{[$_{_{PP}}$ }in der Stadt    \\ 
        middle  {[$_{_{PP}}$  in} the city ] \\ 
        \glt  `in the middle of the city'
     \exi{(ii)} \gll oben \phantom{[$_{_{PP}}$ }auf dem Turm   \\
         up  {[$_{_{PP}}$ }on the tower ] \\ 
        \glt `at the top (part) of the tower'
   \end{xlist}
\end{exe} }
\end{multicols}
}  We can distinguish as in (\ref{cons}).

\begin{exe}
  \ex \label{cons}
     \begin{xlist}
        \ex  in  [$_{DP}$ \textit{middle}-\textbf{\textsc{agr}} [ the city ]-\textbf{\textsc{agr}} \ ] \hfill (\isi{positional predicate}) 
        \ex  in [$_{DP}$ \textbf{(the)} \textit{middle}  [$_{DP}$ \textbf{the} city]-\textbf{\textsc{gen}} \ ] \hfill  (corresponding \isi{noun}) 

        \ex  in [$_{DP}$ \textbf{(the)} \textit{middle} [$_{PP}$ \textit{\textbf{of}} \ \textbf{the} city ] \hfill  (corresponding \isi{noun})
    \end{xlist}
\end{exe}


Differently from a \isi{positional predicate}, a corresponding \isi{noun} does not constitute  an \isi{agreement} construction with the rest of the \isi{noun} phrase, but instead establishes a second \isi{agreement} domain.  In particular, it takes the semantic head \isi{noun}  as a --  PP or  \isi{genitive} DP -- dependent, and may have its own \isi{article}.    
Moreover, in languages with morphological gender marking, the \isi{noun} may have a gender value different from the semantic head \isi{noun}. These  points are illustrated with the following Old High \ili{German} examples: the item \textit{mitti} can either be an \isi{adjective} (displaying \isi{adjectival inflection}) or a feminine \isi{noun} (displaying nominal \isi{inflection}). In the former case, it agrees with the head \isi{noun}, while in the latter case, it occurs with its own \isi{article} and takes the semantic head \isi{noun} as a \isi{genitive} complement (\ref{mitti}).

\begin{exe}
  \ex \label{mitti}
     \begin{xlist}
      \ex  \textit{mitti} + \isi{adjectival inflection} \\ 
      \gll in \textbf{mitt-an}  { thén}  uueizi   \\ 
          in  middle-\textsc{m.acc.sg.str}  [\textsc{dem}  wheat]-\textsc{m.acc.sg}  \\  
          \glt `amidst the wheat'    (ReA T\_Tat72)  
        \ex \textit{mitti} + nominal \isi{inflection} \\ 
        \gll  {die} \textbf{mítti-n\^{a}}  { der-o b\'oum-o}  \\ 
          the middle-\textsc{f.nom.pl}  [{\textsc{dem} \ \ \ tree}]-\textsc{m.gen.pl}   \\ 
          \glt `the middle part(s) of the trees'    (ReA, N\_Mart\_Cap.I.14-37)
     \end{xlist}
\end{exe}

Obviously, it is useful to keep these points in mind in order to distinguish positional predicates from etymologically related nouns, but it also allows  us to pay attention to subtler distinctions. Compare the examples in (\ref{grexAgrGen}) from \ili{Classical Greek}.

\begin{exe}
   \ex \label{grexAgrGen}
      \begin{xlist}
         \ex \label{grexAgr}
          \gll en \textbf{mes\=e} \textbf{{t\=e}} \textbf{{polei}}    \\ 
             in middle.\textsc{f.dat.sg}  the.\textsc{f.dat.sg}  city.\textsc{dat.sg}  (f)  \\ 
           \glt (Perseus, Isokrates; To Philip, speech 5, Section 48)    
         \ex \label{grexGen}
           \gll   en \textbf{mes\=o} \textbf{t\=es} \textbf{pole\=os}    \\ 
             in  middle.\textsc{n.dat.sg} the.\textsc{f.gen.sg}  city.\textsc{gen.sg} (f) \\
              \glt (Perseus, Plutarch, Sertorius, chap. 18)       %  , from Perseus 
     \end{xlist}
     both: `in the middle of the city'
\end{exe}


Example (\ref{grexAgr}) shows a straightforward use of the \isi{positional predicate} \textit{mesos} `middle'  displaying \isi{agreement} in (feminine) gender, case and number. Example (\ref{grexGen}), on the other hand, involves the neuter singular form taking the semantic head \isi{noun} as a \isi{genitive} complement. In this latter case, it is not immediately clear whether \textit{meson} should be construed as a genuine \isi{noun} or a nominal use/nominalized version of the adjectival form.\footnote{Differently from \ili{Germanic}, nominal and \isi{adjectival inflection} are form-identical in \ili{Greek}.}  There is some \isi{variation} between authors/genres; most notably, in the \ili{Greek} of the New Testament, the use of the \isi{genitive} construction appears to dominate, and at least the item \textit{mesos} `middle' is only found in the \isi{genitive} construction.  This will be of particular relevance for the discussion of \ili{Gothic}.




\section{Positional predicates in early Germanic}
\label{sec:10:POG}


In Section \ref{sec:10:posprdEG}, we saw that (etymologically related) items displaying (some of) the same syntactic peculiarities as in modern \ili{Icelandic}, see Section \ref{sec:10:posprICE} and (\ref{IceCHAR}i--v), can be found in all early \ili{Germanic} languages. This is a strong indication that positional predicates  and their properties really belong to the inventory of early \ili{Germanic} \isi{syntax}. At the same time, we also find various deviations and interesting variations among the attested languages. In this section, I will point out and discuss the most noticeable features/deviations for each language.


\subsection{Old Icelandic}
\label{sec:10:nors}

As illustrated in (\ref{ICE1}), we find the same items occurring as positional predicates in \ili{Old Icelandic} and modern \ili{Icelandic}. Some examples are given in (\ref{OIce1}).\footnote{\label{fnoic}The \textit{Saga Corpus} contains a bit more than 500 relevant examples (queries based on the items in (\ref{ICE1}) together produce 637 hits, but among them, we find  a small number of PPs without an overt \isi{noun}).   NPEGL contains 69 annotated instances (at the time of writing). }

\begin{exe}
   \ex \label{OIce1} 
      \begin{xlist}
        \ex \gll í öndverðu liði-nu    \\ 
          in front.part.\textsc{str} troops-\textsc{def}   \\   
           \glt `in the foremost part of the army' (NPEGL, OIce.803.935)

         \ex \gll  þú situr á austanverðu landi en vér á vestanverðu landi \\ 
            you sit on eastern.\textsc{str} land but we on western.\textsc{str} land   \\  
             \glt `You are (based) in the eastern part of the country, but we in the  western part of the country' \\ (Saga, Hrafnkels saga Freysgoða)
 
         \ex \gll eftir endilöngum setaskála-num   \\ 
           after along.\textsc{str} building-\textsc{def}  \\   
          \glt  `from one end of the building to the other' (Saga, Eyrbyggja saga)

      \ex \gll ofanverðan þenna vetur   \\ 
         latter.\textsc{str} this winter \\  
         \glt `in the latter part of that winter' (NPEGL, OIce.548.527)
    \end{xlist}
\end{exe} 

We find both pre- and \isi{postnominal} occurrences, even though the \isi{prenominal} position seems to be dominant, cf. (\ref{OIce2}).\footnote{\textit{Saga Corpus}: 415 \isi{prenominal} vs. 67 \isi{postnominal} occurrences; NPEGL: 61 vs. 4.}\footnote{Notice
  that positional predicates are consistently strongly inflected even when following a definite \isi{noun}, i.e. a \isi{noun} carrying a suffixed definite \isi{article}, cf. (\ref{OIce2}b). In \isi{contrast}, ``regular''  adjectives are normally weakly inflected in this constellation:

  \begin{multicols}{2}{
  \begin{exe}
    \ex
    \begin{xlist}
      \ex\gll í á-nni helg-u \\
        in river-\textsc{def} holy-\textsc{wk} \\
        \glt  `in the holy river' \\ (Saga, Heimskringla)
      \ex\gll sverð-ið góð-a \\
        sword-\textsc{def} good-\textsc{wk} \\
        \glt  `the good sword' \\ (Saga, Gull-Þóris saga)
    \end{xlist}
  \end{exe}}
  \end{multicols}
  Thus, in \ili{Old Icelandic}, \isi{inflection} can be used as a diagnostic also in the \isi{postnominal} position: weak \isi{inflection} $\sim$ \isi{attributive} position, strong \isi{inflection} $\sim$ \isi{predicative} position.
}


\begin{exe}
   \ex \label{OIce2}  
    \begin{xlist}
      \ex \gll   of þvera  götu-na     \\
        over across.\textsc{str}  road-\textsc{def}  \\ 
         \glt `across the road' (NPEGL, OIce.902.814)

      \ex \gll   um á-na  þvera     \\
        about  river-\textsc{def}  across.\textsc{str}  \\ 
        \glt `across the river' (Saga, Vatnsdæla saga)

       \ex \gll  á ofanverðum dögum {\phantom{[}Haralds Sigurðarsonar}   \\  
          on latter.\textsc{str} days  {[Haraldur Sigurðarson]-\textsc{gen} }   \\ 
          \glt  `in the latter days of Harald Sigurðarson' 
          (Saga, Heimskringla)

      \ex \gll  á dögum {\phantom{[}Hákonar {hins ríka}}  öndverðum   \\   
        on days  {[Hákon the mighty]-\textsc{gen}} beginning.\textsc{str}  \\
            \glt `in the early days of {H\'akon} the mighty' \\(Saga, Egils saga Skallagrímssonar)
    \end{xlist}
\end{exe} 


In the context of names and pronouns, there appear to be certain restrictions.   
We find both pre- and \isi{postnominal} occurrences with place names,  cf.  (\ref{OIce3a}) and (\ref{OIce3b}), but only \isi{postnominal} occurrences with names of persons, cf. (\ref{OIce3c}).  

\begin{exe}
   \ex \label{OIce3} 
    \begin{xlist}
       \ex \label{OIce3a} \gll    yfir  Borgarfjörð  þveran   \\
         over Borgarfjörður across.\textsc{str}  \\ 
   \glt `across Borgarfjörður' (Saga, Laxdæla saga) 

      \ex \label{OIce3b} \gll  á framanverðu  Reykjanesi  \\ 
        on front.part.\textsc{str} Reykjanes    \\ 
        \glt  `at the front part of the Reykjanes peninsula' (Saga, Gull-Þóris saga) 

      \ex \label{OIce3c} \gll  Hann tvíhenti spjótið á Þóri miðjum \\  
        he hurled spear.\textsc{def} on Þór middle.\textsc{str} \\  
        \glt  `He hurled the spear right at Þór' (Saga, OIce.822.459)

    \end{xlist}
\end{exe} 

Likewise, only post-pronominal occurrences are found (\ref{OIce3BB}).

\begin{exe}
   \ex \label{OIce3BB}  
    \begin{xlist}
     \ex \gll bóndinn féll um  hann þveran  \\
       yeoman.\textsc{def} fell about him across.\textsc{str}   \\ 
        \glt  `the yeoman fell over him' (Saga, Brennu-Njáls saga)

       \ex \gll  Bolli skýtur að  honum spjóti og kemur á hann miðjan  \\
         Bolli shoots at him spear and comes on him middle.\textsc{str} \\  
        \glt `Bolli shoots a spear at him and it hits him squarely' (Saga, Íslendingaþættir)
    \end{xlist}
\end{exe} 

Beyond that, positional predicates are rather versatile and may occur in unexpected constellations. For instance, %in  (\ref{OIce2}d), the item \textit{öndverður} follows a \isi{postnominal} \isi{genitive} phrase where we otherwise do not expect to find adjectives.
in (\ref{st2}), the \isi{positional predicate} appears to have been stranded, while the lower part of the \isi{noun} phrase has been fronted to the clause-initial position.


\begin{exe}
    \ex \label{st2} \gll  \textbf{þessa nótt hina sömu} kom Mörður [\textbf{ofanverða}   t ]   \\   
      {this \ \ night \ the \ same} came Mörður  latter-part.\textsc{str } \\ 
	   \glt `Later that very same night, Mörður showed up' (Saga, Brennu-Nj\'als saga)
\end{exe}


In some cases, we find neuter forms of positional predicates, de facto acting as the head \isi{noun}, in PPs without an overt \isi{noun}, cf. (\ref{OIce4}).


\begin{exe}
   \ex \label{OIce4}  
%    \begin{xlist}
%       \ex \gll  hann  hjó hann {í sundur} í  \textbf{miðju}  \\ 
%         he chopped  him asunder in middle.\textsc{str} \\ 
%         \glt `he chopped him asunder' \\ 
%          \glt (Saga: Brennu-Nj\'als saga)   
       
       \gll   frá \textbf{öndverðu} til \textbf{ofanverðs} \\ 
     from  former/lower.part.\textsc{str} to latter/upper-part.\textsc{str} \\ 
     \glt `from top to bottom' or `from beginning to end' 
%  \end{xlist}
\end{exe} 

 In other words, the (singular) neuter forms have nominal uses, in addition to their more frequent ad-nominal use. However, this nominal use only appears to occur in the absence of a semantic head \isi{noun}. Whenever there is a constituent denoting the source location, it is realized as the (semantic and) syntactic head \isi{noun}, and the \isi{positional predicate} agrees with that head \isi{noun} in case, number and gender. In this respect, \ili{Old Icelandic} behaves differently from the neuter forms of \textit{meson} in \ili{Ancient Greek}, cf. (\ref{grexGen}), which may take the semantic head \isi{noun} as a \isi{genitive} complement. Judging from the examples examined here, \ili{Old Icelandic} never takes genitival dependents. %\footnote{There is at least one such example in the Old \ili{Norwegian} Homily book:
%\begin{exe}
%  \ex \gll    fyrir öndverðu heims þessa  \\ 
%   before beginning  [world this]-\textsc{gen} \\
% \end{exe} }





\subsection{Old English}
\label{sec:10:oeng}

In Old \ili{English}, we find largely the same inventory of positional predicates as in (Old) \ili{Icelandic}; some examples are given in (\ref{Oeng}).


\begin{exe}
   \ex \label{Oeng}  
    \begin{xlist}
     \ex \gll  on middre   ðære sæ     \\ 
       in middle.\textsc{str} \textsc{dem} sea \\ 
       \glt `in the middle of the sea' (NPEGL, OEng.436.568)      

      \ex \label{Oengb} \gll on middum ðinum temple  \hspace{2mm}(Lat. in medio templo tui)  \\ 
       in middle.\textsc{str} your temple  \\ 
       \glt `in the middle of your temple' (\url{https://bosworthtoller.com/22789})


      \ex \gll   þæt heafod   foreweard   \\ 
       \textsc{dem} head  front.part.\textsc{str} \\  
       \glt `the front part of the head' (NPEGL, OEng.349.012)          
%    \ex \gll \textbf{inneweardre} heortan \\ \textit{from} \textit{inner-part}.\textsc{str} \textit{heart} \\  OEng.716.688          

    \ex \gll genim hamorwyrt \& efenlastan nyðowearde   \\ 
      take wall.pellitory \& herb.mercury nether.part.\textsc{str} \\ 
       \glt `take the lower part of pellitory-of-the-wall and herb mercury \\ (= plant names)' (NPEGL, OEng.241.262)              

      \ex \gll on  þam  lande norþweardum   \\ 
       on  \textsc{dem} land northern.part.\textsc{str} \\  
       \glt `in the northern part of the land' (NPEGL: OEng.097.051)
   \end{xlist}
\end{exe} 

Besides the item ``middle (part of)'', we find a large class of complex items consisting of a locational component plus a morpheme \textit{-weard-} (plus \isi{inflection}) like \textit{norþ-weard-}, cf. \ili{Icelandic} \textit{norðan-verð-}. Previous research on the position of adjectives in Old \ili{English} has noted that those items in \textit{-weard-} have some ``adverbial interpretation'' and occur more frequently in \isi{postnominal} position than regular adjectives (especially \citealt{Fischer01}; \citealt{Grabski17,Grabski20}). Thus even in Old \ili{English}, which otherwise displays a relatively rigid modifier ordering in general (see \citetv{chapters/3Modifiers}), positional predicates are much more versatile than regular adjectives. 
At the time of writing, 213 positional predicates have been identified in the NPEGL \isi{database} (annotation still in progress). This is a comparatively large number, and therefore, it is noteworthy that, so far, no occurrences with pronouns have been identified. 

%%%%%%%%%%%%%however, position midd vs. - verþ-

%%%%%%%%%%%%%%weird cases . . .   Lǣcebōc


%     \ex  \gll on \textbf{foreweardre} þa adle    \hspace{12mm}$\leftrightarrow$\hspace{12mm}    on \dh{}a adle \textbf{foreweardre}   \\  \textit{on} \textit{front-part}.\textsc{str} \textit{the} \textit{disease}  { }  \textit{on} \textit{the} \textit{disease} \textit{front-part}.\textsc{str}    \\       (OEng.218.374)  \hspace{42mm}  (OEng.806.931) \\ 
 %           (both: $\sim$ ``in the initial phase of the disease'')       

 
Also notice (\ref{Oengb}), taken from Bosworth Toller's \textit{Anglo-Saxon Dictionary online}, which in addition gives the \ili{Latin} original that the Old \ili{English} phrase is supposed to translate. The possessive occurs postnominally in the \ili{Latin}, but prenominally in \ili{English}; this is perhaps not very surprising given that possessives in Old \ili{English} almost exclusively occur prenominally (see \citetv{chapters/3Modifiers}). Yet it is noteworthy that, in accordance with our expectations, the \isi{adjective} precedes that possessive.   

Other deviations from \ili{Latin} are even more revealing, for instance cases where the \ili{Latin} text has a \isi{genitive} dependent while the \ili{English} translation uses an \isi{agreement} construction. The examples in (\ref{OengA}) (likewise taken from  Bosworth-Toller's dictionary entry: \textit{midd}) illustrate some such  mismatches between Old \ili{English} and the \ili{Latin} source (bracketing indicates \isi{agreement} in case, number, gender).

\begin{exe}
   \ex \label{OengA}   
    \begin{xlist}

     \ex \gll  in   {\ middum}  wulfum   \\  
      in  [middle.\textsc{str}   wolf]-\textsc{dat.pl}  \\  
      \glt `amidst the wolves' \\ 
      \glt Lat. in medio luporum  ($\rightarrow$ wolf.\textsc{gen.pl}) 

     \ex \gll  þurh  {\ midde}  ða ceastre   \\  
      through  [middle.\textsc{str} \textsc{dem} camp]-\textsc{acc.sg}  \\  
      \glt `through the middle of the camp' \\ 
      \glt Lat. per medium castrorum  ($\rightarrow$ camp.\textsc{gen.pl}) 
          
     \ex \gll  On {\ middum}  ð\'æm úrum wícum   \\ 
     in [middle.\textsc{str} \textsc{dem} our camp]-\textsc{dat.pl}   \\  
     \glt `in the middle of our camps' \\ 
      \glt Lat. in media castrorum  ($\rightarrow$ camp.\textsc{gen.pl}) 
  
     \ex \label{OengAd} \gll  On {\ middan}  ða wic   \\  
     in [middle.\textsc{wk} \textsc{dem} camp]-\textsc{acc.sg}   \\  
     \glt `in the middle of the camp' \\ 
      \glt  Lat. in medio castrorum ($\rightarrow$ camp.\textsc{gen.pl})

   \end{xlist}
\end{exe} 

These apparently systematic deviations are an indication that the construction is precisely not a scholarly translation from \ili{Latin}, but a native phenomenon. As already seen in the previous subsection on \ili{Old Icelandic}, Old \ili{English} seems to prefer the \isi{agreement} construction. However, in \isi{contrast} to \ili{Icelandic}, we find a handful of examples instantiating the \isi{genitive} construction, as in (\ref{OengB}).

\begin{exe}
   \ex \label{OengB}   
    \begin{xlist}
     \ex \label{OengBa} \gll  on  westeweardum   { þisses} middangeardes   \\  
     in western.part.\textsc{str}   [\textsc{dem} world]-\textsc{gen}  \\
     \glt `in the western part of this world' (NPEGL, OEng.078.130)

     \ex \label{OengBb}  \gll  wið  middan {\ þæs} suðwages   \\  
       at middle.\textsc{wk}  [\textsc{dem} south.wall]-\textsc{gen}  \\ 
     \glt `at the middle of the south wall' (NPEGL, OEng.540.709)

   \end{xlist}
\end{exe} 


%At the time of writing, 213 positional predicates have been identified in the NPEGL \isi{database} (annotation still in progress);
Except for the examples in (\ref{OengB}), all positional predicates annotated in NPEGL occur in an \isi{agreement} construction, which indicates that, albeit attested, the \isi{genitive} construction seems to be dispreferred. 

There is a more noticeable feature of positional predicates in Old \ili{English} concerning \isi{adjectival inflection}. As illustrated by (\ref{OengAd}) and (\ref{OengBb}), we find weak \isi{inflection} where we otherwise expect the strong \isi{inflection} according to (\ref{IceCHAR}iii).  
Currently, we have 35 (out of 213) such weakly inflected positional predicates in the NPEGL \isi{database} of Old \ili{English}. This aspect has been noted before.  \citet[vol. I, 70]{Mitchell85} discusses exceptions regarding the distribution of \isi{adjectival inflection} and the deviant behaviour of \textit{midd} and elements ending in \textit{-weard-}. Of course, positional predicates are deviant only from the point of view of ``regular'' adjectives, generally speaking, but Mitchell points out certain cases that are unexpected also from the perspective of positional predicates. We can distinguish three constellations (\ref{positioX1})--(\ref{positioX3}). \\

\noindent (I) \ \ \  Predicative position -- weak  \isi{inflection} 
\begin{exe}
   \ex \label{positioX1}  
    \begin{xlist}
      \ex \gll on ufeweard-an þam geate \\  
      on upper.part-\textsc{wk} \textsc{dem} gate \\ 
      \glt `in the upper part of the gate' (NPEGL, OEng.010.465)        
      \ex \gll  betwux þam eorode midd-an  \\ 
       among \textsc{dem} troop middle-\textsc{wk}  \\ 
      \glt `among the middle of the troop' (NPEGL, OEng.340.258)
   \end{xlist}
\end{exe} 

\noindent (II)  \ \  Attributive position -- strong  \isi{inflection} 
\begin{exe}
   \ex \label{positioX2}  
    \begin{xlist}
     \ex \gll þære midd-re nihte \\
       \textsc{dem} middle-\textsc{str} night  \\ 
      \glt `the mid-night' (NPEGL, OEng.429.571)  
     \ex \gll þone mid-ne sumor    \\ 
       \textsc{dem} middle-\textsc{str}  summer     \\ 
      \glt `the mid-summer' (NPEGL,  OEng.175.907) 
   \end{xlist}
\end{exe} 

% þeſſe miðri nꜵt

\noindent  (III)  \  Attributive position -- weak  \isi{inflection} 
\begin{exe}
   \ex \label{positioX3}  
      \gll þam midd-an wintra \\ 
        \textsc{dem} middle-\textsc{wk}  winter   \\ 
      \glt `the mid-winter' (NPEGL, OEng.697.340, OEng.685.076) 
\end{exe} 

Constellations (I) and (II) are unexpected with respect to both regular adjectives and positional predicates; with a handful of relatively systematic exceptions, weak adjectives are usually restricted  in their occurrence to (formally) definite contexts, which normally means when following a \isi{definite determiner}.  

Thus, while the elements  in constellation (I) display the expected  \isi{syntax} (= the \isi{predicative} position), the pre-\isi{determiner} weak \isi{inflection} is unexpected. Conversely, the post-\isi{determiner} strong \isi{inflection} is unaccounted for in constellation  (II). Moreover,  the \isi{attributive} position is unexpected given that the elements in (\ref{positioX2}) still produce a \isi{partitive} reading, not a restrictive one (see Section \ref{sec:10:predAtt}).

The latter  issue can possibly be addressed by analyzing (II) as a mere surface phenomenon derived via \isi{determiner} raising to a pre-adjectival position while the \isi{adjective} itself  occupies the \isi{predicative} position all along (\ref{anal}).\footnote{In other words, rather than the relative \isi{article} position, here the strong \isi{inflection} could be taken as a diagnostic for  the \isi{predicative} position of the respective modifier. %This perspective is compatible with \citet{Pfaff2017} who argues that \isi{adjectival inflection} is indeed a diagnostic for adjectival position: inside vs. outside the core DP (in Icelandic); see fn. \ref{famousFootnote}. 
Still, this raises the question what motivates the \isi{determiner} movement.   }

\begin{exe}
   \ex \label{anal}  $[$ \  þære \  middre  [$_{DP}$  \sout{\color{gray}þære}  nihte  ]$]$
\end{exe}

An analysis along those lines can thus account for the \isi{partitive} reading with (II). However, constellation (III), which is  what is expected for regular adjectives, poses a more serious problem -- precisely because of the weak \isi{inflection}, an analysis like (\ref{anal}) does not work here. All formal criteria indicate that  \textit{middan} in (\ref{positioX3}) genuinely occupies an \isi{attributive} position. We should therefore expect a restrictive reading ($\sim$ `the middle one in a sequence of winters'), but we get a \isi{partitive} reading (`the middle part of the winter'). 

Thus while Old \ili{English} provides ample evidence for positional predicates, we also find ``deviations'' from the prototypical behaviour as characterized in (\ref{IceCHAR}), notably in terms of \isi{adjectival inflection}.  
Obviously, more research is called for, but, in all likelihood, such deviations are part of (later) \ili{English}-internal developments. For one thing, the inflectional system shows first signs of disintegration already towards the end of the Old \ili{English} period.\footnote{\label{thx2}As a result, there is an increase of syncretism  and a decrease in distinctions made between cases, but also between strong vs. weak \isi{inflection}; thus it cannot always be unambiguously decided whether a given \isi{adjective} is  strongly or weakly inflected. Incidentally, this also applies to \ili{Old Saxon}, see fn. \ref{thxGeorge}. Thanks to George Walkden (p.c.) for pointing this out to me.} But also more broad syntactic changes in the transition to Middle \ili{English}, e.g. the emergence of the \isi{determiner} system and an increasingly fixed \isi{word order},  had an impact on \isi{adjective} \isi{syntax} in general, cf. \citet{Fischer2004,Fischer06}, and presumably on the behaviour of positional predicates. 



\subsection{Gothic} 
\label{sec:10:got}


In \ili{Gothic}, we find six relevant instances of the item \textit{midjis} `{middle}', all of which are given in (\ref{got}) ((\ref{gotd}) represents two occurrences).

\begin{exe}
   \ex \label{got}   
     \begin{xlist}
       \ex \label{gota} \gll  in midjaim  laisarjam  \\   
        in middle.\textsc{str} teachers \\  
       \glt  `in the midst of the doctors' (Wulfila, Luke 2:46)

       \ex \label{gotb} \gll ana midjai dulþ  \\   
         at middle.\textsc{str} feast  \\ 
         \glt `about the midst of the feast' (Wulfila, John 7:14)

       \ex \label{gotc} \gll þairh midja {Samarian jah Galeilaian} \\ 
         through   middle.\textsc{str}  {Samaria and Galilee} \\ 
         \glt `through the midst of Samaria and Galilee' (Wulfila, Luke 17:11)

       \ex \label{gotd} \gll þairh midjans ins  \\  
          through middle.\textsc{str} them  \\ 
          \glt `through the midst of them' (Wulfila, Luke 4:30; John, 8:59)

       \ex \label{gote} \gll in midjaim im \\ 
         in middle.\textsc{str} them \\
          \glt `in the midst of them'; `amongst them' (Wulfila, Mark 9:36)

    \end{xlist}
\end{exe} 

Even though none of these examples involves a \isi{determiner}, they illustrate the characteristics of positional predicates in \isi{predicative} position: the modifier is strongly inflected, it combines with proper names and pronouns  and they fully agree in case, number (and gender),  the \isi{noun} denotes a temporal or spatial extension or plurality, and we get a \isi{partitive} interpretation. Of course, based on only six ``well-behaved'' examples, not much can be said about \isi{variation} and language-specific peculiarities, but it is worthwhile pointing out  two obervations of interest. 

Firstly, out of three co-occurrences with a \isi{pronoun}, the \isi{positional predicate} precedes the \isi{pronoun} three times,  (\ref{gotd}) and (\ref{gote}); that is 100\%.   
Recall that, in \ili{Old Icelandic}, positional predicates \textit{never} occur pre-pronominally, and as will be seen in the following section(s), the same applies to Old High \ili{German} and \ili{Old Saxon} (with one counterexample).  Thus if the post-pronominal position is otherwise the default across \ili{Germanic}, even three instances might be sufficient to indicate that \ili{Gothic} differs from the other \ili{Germanic} languages, at least in that respect. 

However, one permanent problem with \ili{Gothic}  is the question to what degree it reflects  the \ili{Greek}  rather than the native \isi{syntax} (see \citealt{ratkus2011} for a thorough discussion); the pre-pronominal position could, in principle, be such a reflection.   
 It is therefore revealing to take a look at the \ili{Greek} source text; (\ref{gri}) illustrates the relevant passages underlying the \ili{Gothic} translations in (\ref{got}).\footnote{The \ili{Greek} text is from taken from \textit{Project Wulfila} (\url{http://www.wulfila.be/gothic/}), which relies on the Streitberg edition of the \ili{Gothic}/\ili{Greek} New Testament. }

 
\begin{exe}
   \ex \label{gri}  
    \begin{xlist}
         \ex \gll en mes\={o} { t\=on didaskal\=on}    \\ 
           in middle.\textsc{n.dat.sg} [{the teacher}]-\textsc{m.gen.pl}  \\
    {\color{gray} \ex \label{grib}  \gll  t\=es heort\=es mesous\={e}s   \quad{(\textsc{f.gen.sg})}  \\  
      the feast  in.middle.being      \\  }
          \ex \gll dia meson samarei\=as     \\ 
           through  middle.\textsc{n.acc.sg} Samaria.\textsc{f.gen.sg}  \\ 
          \ex \label{grid}  \gll dia mesou aut\=on   \\  
             through  middle.\textsc{n.gen.sg} they.\textsc{gen.pl} \\
         \ex \label{grie}  \gll en mes\=o aut\=on \\  
          in middle.\textsc{n.dat.sg} they.\textsc{gen.pl} \\
    \end{xlist}
\end{exe} 

\begin{sloppypar}
Strictly speaking, the \ili{Greek} examples show a pre-pronominal position, cf. (\ref{grid}) and (\ref{grie}), but upon closer inspection, we discern a systematic mismatch between \ili{Greek} and \ili{Gothic}.   
Even though  \ili{Classical Greek} does have positional predicates occurring in an \isi{agreement} construction/the \isi{predicative} position as was discussed in Section \ref{sec:10:predAtt}, cf. (\ref{latGrb}) and (\ref{grexAgr}), \ili{Biblical Greek} seems to prefer a \isi{genitive} construction, as in  (\ref{grexGen}). With the exception of (\ref{grib}),\footnote{Note that this example is different at any rate; it actually involves a participle form of a verb `be-in-the-middle' and the whole phrase is a so-called \textit{genitivus absolutus}, a small clause construction with an adverbial function.} the \ili{Greek} examples in (\ref{gri})  involve a nominalized \isi{adjective} (based on the neuter singular) that takes the \isi{noun}/\isi{pronoun} as a \isi{genitive} dependent. In spite of this, the \ili{Gothic} translations in (\ref{got}) all use the \isi{agreement} construction. This, in turn, is a strong indication that the \isi{partitive} \isi{agreement} construction found with positional predicates is a native phenomenon and a productive \isi{pattern} of the \ili{Gothic} \isi{syntax}, and precisely not a borrowing from \ili{Greek} -- which would, in principle, be a plausible source.
\end{sloppypar}

Similarly to \ili{Old Icelandic}, cf. (\ref{OIce4}), in \ili{Gothic}, we find four cases without an overt \isi{noun} where the modifier itself is used nominally (\ref{mi}). 

\begin{exe}
   \ex \label{mi} \gll   in \textbf{midj{aim}}  \\ 
      in middle.\textsc{dat.pl}  \\ 
       \glt `in(to) the middle/midst' (Wulfila, Luke 2:35, 5:9, 6:8; Mark 14:60) 
\end{exe}  

There are no instances of the adjectival form taking a genitival dependent. However, differently from \ili{Old Icelandic}, \ili{Gothic} has a morphologically distinct (feminine) \isi{noun} \textit{miduma} that occurs six times, of which four times with a \isi{genitive} dependent (\ref{midumai}).

\begin{exe}
    \ex \label{midumai}\gll in \textbf{midumai}  wulfe\hspace{20mm} (en  mes\=o  luk\=on) \\ 
       in middle wolf.\textsc{gen.pl}  { in} middle.\textsc{neut.dat.sg} wolf.\textsc{gen.pl}    \\
      \glt `amidst wolves' (Wulfila,  Luke 10:3)
\end{exe} 

Setting aside nominal uses as in (\ref{mi}), the adjectival form \textit{midjis} only occurs in \isi{agreement} constructions, while the \isi{noun} \textit{miduma} can  take  the (semantic) head \isi{noun} only as a genitival dependent. In \ili{Greek}, on the other hand, the adjectival form occurs both in \isi{agreement} constructions and with genitival dependents; notice that both \textit{midjis} in (\ref{mi}) and \textit{miduma}  translate the  neuter adjectival forms (\textit{meson}) in the \ili{Greek} text. 



%In other words, while the pre-pronominal position may be influenced by the \ili{Greek}, the fact that \ili{Gothic} uses the agrement construction in spite of the \ili{Greek} \isi{genitive} construction would seem to suggest that positional predicates do instantiate a productive \isi{pattern} in \ili{Gothic}. 






\subsection{Old High German}


A query  in ANNIS  yields 79 matches for the lemma \textit{mitti} `middle'. In 49 cases, these can straightforwardly be diagnosed as positional predicates; some examples are given in (\ref{OHG1}).

\begin{exe}
   \ex \label{OHG1}   
    \begin{xlist}
       \ex \gll in mittemo seuue  \\ 
        in middle.\textsc{str} sea  \\ 
        \glt `in the midst of the sea' (ReAT\_Tat81) 
         
        \ex \gll únder mítten díen planetis   \\ 
           under middle.\textsc{str} \textsc{dem} planets \\
        \glt `amidst the planets' (ReA, N\_Mart\_Cap.I.14-37) 

        \ex \gll in mittan thén uueizi   \\ 
         in middle.\textsc{str} \textsc{dem} wheat  \\ 
          \glt `amidst the wheat' (ReAT\_Tat72)   

       \ex \gll in míttemo iro rínge \\ 
         in middle.\textsc{str} their circle   \\ 
       \glt  `in their midst' (ReAO\_Otfr.Ev.4.19)

       \ex \label{pWG} \gll duruh den Fredthantes uuingarton mittan \\  
          through \textsc{dem} Fredant's vineyard  middle.\textsc{m.acc.sg.str}  \\
        \glt `(right) through the middle of Fredant's vineyard' \\ 
           \glt (ReA, WM2\_Wuerzburger\_Markbeschreibung\_2)

       \ex \label{prM} \gll  in mittan  Moin    \\  
         in  middle.\textsc{str} Main  \\ 
        \glt `in the middle (part) of the (river) Main' \\ 
          \glt (ReA, WM2\_Wuerzburger\_Markbeschreibung\_2)

    \end{xlist}
\end{exe} 

In 36 of these cases, \textit{mitti} occurs prenominally,  and we find four \isi{postnominal} occurrences, e.g. (\ref{pWG}). However,  with nouns denoting (place) names, only \isi{prenominal} occurrences are found, e.g. (\ref{prM}).  In addition, we find nine occurrences with pronouns, as in (\ref{OHG2}), one of which in pre-pronominal position, cf. (\ref{OHG2c}).

\begin{exe}
   \ex \label{OHG2} 
    \begin{xlist}
       \ex \gll untar sie mitte  \\ 
         among them  middle.\textsc{m.acc.pl.str}    \\
        \glt `into/between their midst' (ReA, T\_Tat120)
        
      \ex \gll in dhir mitteru   \\  
        in you.\textsc{sg} middle.\textsc{f.dat.sg.str}    \\ 
        \glt `right inside you' (ReA, I\_DeFide\_3) 
         
      \ex \label{OHG2c} \gll Untar mitten  íu   \\ 
          among middle.\textsc{dat.pl.str} you.\textsc{pl} \\ 
        \glt `amongst you' (ReA, T\_Tat13)
    \end{xlist}
\end{exe} 


As was already shown for \ili{Old Icelandic} and  \ili{Gothic}, we also find nominal uses of the item \textit{mitti}, i.e. without a semantic head \isi{noun}, in prepositional phrases (\ref{OHG3}).

\begin{exe}
   \ex \label{OHG3} 
      \gll arstant inti gistant in  mitten  \\ 
         raise and stand in  middle.\textsc{str} \\ 
        \glt `raise and stand in the middle' (ReA, T\_Tat69)
\end{exe} 

 Moreover, however, we find examples where \textit{mitti} occurs with a \isi{genitive} dependent (\ref{OHG4}), and here we have to  distinguish between two cases: in (\ref{OHG4a}) and (\ref{OHG4b}), \textit{mitti} occurs with a strong adjectival ending, whereas in (\ref{OHG4c}), it occurs with a nominal ending; the latter has to be construed as  an instance of a feminine (\={\i}n-stem) \isi{noun} \textit{mitti}; the additional feminine \isi{article} \textit{die} is another indication of nounhood of \textit{mitti} in this example (cf. Section \ref{sec:10:AGR}). Here, the \isi{inflection} definitively disambiguates and distinguishes the \textit{nominal use} of an \isi{adjective} from an actual \textit{noun}, even though the two happen to have the same \isi{nominative} singular form: \textit{mitti}.   % (but compare (\ref{OHG4}) with (\ref{OHG3}b)):


\begin{exe}
   \ex \label{OHG4} 
    \begin{xlist} 
       \ex \label{OHG4a}  \gll untar mítten { thes} sélben dages   \\  
          under middle.\textsc{str} [\textsc{dem} same day]-\textsc{gen}   \\  
        \glt `during the same day' (ReA, O\_Otfr.Ev.5.11)

       \ex \label{OHG4b}  \gll thar bin ih in mítten  iro \\ 
          there am  I  in  middle.\textsc{str} they.\textsc{gen}   \\ 
        \glt `there I am in their midst' (ReA, T\_Tat98)

        \ex \label{OHG4c}  \gll   die mítti-n\^{a}  { der-o bóum-o}  \\ 
          \textsc{dem} middle-\textsc{f.nom.pl}  [{\textsc{dem} \ \ \ tree}]-\textsc{m.gen.pl}   \\ 
        \glt `the middle part(s) of the trees' (ReA, N\_Mart\_Cap.I.14-37)
   \end{xlist}
\end{exe}

 In \ili{Gothic}, the two can be distinguished more easily: \textit{midjis} vs. \textit{miduma}. Differently from \ili{Gothic}, however, where only the latter takes a \isi{genitive} dependent, in Old High \ili{German}, also the adjectival   forms \textit{can} take a \isi{genitive} dependent, cf. (\ref{OHG4a}) and (\ref{OHG4b}), besides occurring in the \isi{agreement} construction as in (\ref{OHG1}) and (\ref{OHG2}).   

%Note that the Old High \ili{German} adjectival  \isi{inflection} poses an additional twist in that adjectives can also have a zero ending. Thus we potentially have to distinguish more than one set of endings:   
%\begin{exe}
%  \ex  
%    \begin{tabular}{l | l |  l | l}
%       				& \textbf{masc}			&	\textbf{fem} 			&	\textbf{neut} \\ \hline
%       	\textit{strong}		& {mitt-er}			&	{mitt-iu}		 	&	{mitt-az} \\
%       	\textit{zero}		& {mitti{\color{gray}-\O}}	&	{mitti{\color{gray}-\O}} 	&	{mitti{\color{gray}-\O}}  \\  
%       	\textit{weak}		& {mitt-o}			&	{mitt-a} 			&	{mitt-a}  \\
%    \end{tabular}   
%\end{exe}
%Two instances of zero-inflected \textit{mitti} are given below:
%\begin{exe}
%   \ex \label{OHG5} 
%    \begin{xlist}
%      \ex \gll   \textbf{mitti} tak    \\ 
%        middle.\textsc{0}  day     \\
%          \glt (ANNIS: {\footnotesize DDD-AD-Murbacher\_Hymnen\_1.1 > MH\_Murb.H.XII})
%
%      \ex \gll thuruh \textbf{mitti} Samariun inti	Galileam  \\ 
%         through middle.\textsc{0}  Samaria and Galilee   \\ 
%          \glt (ANNIS: {\footnotesize DDD-AD-Tatian\_1.1 > T\_Tat111})      
%   \end{xlist}
%\end{exe} 
%In traditional grammars, the zero \isi{inflection} is often subsumed under the strong paradigm; thus the fact that we find (\ref{OHG5}a) alongside strongly inflected forms such as \textit{mitt-en dag / mitt-er tag} etc. may not be remarkable.   The same goes for (\ref{OHG5}b), which, incidentally, translates the same sentence as (\ref{got}c).\footnote{However, this latter example is annotated as (neuter) \isi{noun} in ANNIS; since the \isi{noun} \textit{Samariun} is annotated as accusative -- as expected, given that the preposition \textit{thuruh} governs the accusative -- this construal is problematic because then we have a juxtaposition of two nouns without morphological expression of the dependence relationship. Notice that, with weak feminine (\=on-stem) nouns, the ending \textit{-un} could technically also be construed as \isi{genitive} (rendering \textit{{mitti} Samari-un} structurally parallel to (\ref{OHG4}c)); but again, this does not work for \textit{Galileam}, which can only be an accusative ending. It seems that  (\ref{OHG5}b), where \textit{mitti} is analyzed as adjectival element with zero \isi{inflection}, is a more straightforward construal. } 
%Finally, I want to point out two problematic cases. Both occur in consecutive clauses of the same sentence (the annotations ``\textsc{0}'' and ``\textsc{wk}'' are due to ANNIS):
%{\small 
%\begin{exe}
%   \ex \label{OHG6} \gll Diu fri\^{u} diu pizeichinet die chindiska \\  \textit{the}  \textit{earliness}  \textit{it}  \textit{signifies}  \textit{the}   \textit{childhood} \\
%    \begin{xlist}
%      \ex \gll  der \textbf{mitt\textit{i}} morgen die i\'ugent   \\    \textit{the} \textit{middle\textsc{.0}} \textit{morning}  \textit{the}   \textit{youth}   \\  
%      \ex \gll der \textbf{mitt\textit{e}} tac die tugent \\ \textit{the}  \textit{middle\textsc{.wk}} \textit{day}  \textit{the}   \textit{virtue}       \\     
%   \end{xlist} 
%    {\scriptsize(DDD-AD-Kleinere\_Althochdeutsche\_Denkmäler\_1.1 > APB\_PredigtsammlungB)}
%\end{exe} }
%First of all, since the relevant sequences are structured in such a parallel or identical fashion, it is reasonable to expect that both instances of \textit{mitti} are identical as well, i.e. both weak or both zero -- in spite of the orthographic difference \textit{-i} vs. \textit{-e}.  The actual  problem here is that the modifier follows the \isi{determiner} and thus appears to occur in the \isi{attributive} position,\footnote{The suggested interpretation is ``middle of the day'',  not `` middle day between two other days''; cf. (\ref{positioX3}). } a situation we have already encountered for Old \ili{English}. But as long as the inflectional status of \textit{mitti} is not clear, it is not obvious whether (\ref{OHG6}) should be compared to constellation (II) or (III) discussed in Section \ref{sec:10:oeng}, or whether it instantiates yet something else (which may depend on  which status we assign the zero \isi{inflection}).
%On the other hand, (\ref{OHG6}b) could be construed as a  compound \isi{noun}; at least  \textit{mittitac} (``mid-day''), written as one word,  is already attested elsewhere  in Old High \ili{German} texts according to the \textit{Althochdeutsches Wörterbuch} (\citealt{koblerwb}). It is thus conceivable that this construal applies to ``middle.0 + day'' more generally. If so, \textit{mitti morgen}  could be seen as an analogical ad-hoc compound formation in (\ref{OHG6}a) (motivated by parallelism, metrics etc.), even though a compounded form does not seem to be attested elsewhere (no entry in \citealt{koblerwb}).  On this account, the issue of \textit{mitti} in the \isi{attributive} position does no longer arise in (\ref{OHG6}).  
%It is less clear, however, that the compound analysis can be applied to all instances of zero-inflected \textit{mitti}; it is certainly not a plausible analysis for examples like   (\ref{OHG5}b).



\subsection{Old Saxon }

%\cite{HeliPaD}
    
In \ili{Old Saxon}, we find the examples presented in (\ref{Osax}). 

\begin{exe}
   \ex \label{Osax}   
    \begin{xlist}
     \ex \gll an middian dag      \\ 
       on  middle.\textsc{str} day  \\ 
        \glt `in the middle of the day' (NPEGL, OSax.444.216,  OSax.075.303)

       \ex \gll middi dag     \\  
         middle.\textsc{str} day   \\
        \glt `(the) middle of the day'; `mid-day' (NPEGL, OSax.869.882)  

       \ex \gll under iu middeon      \quad{(\textsc{dat.pl})}   \\  
        among you  middle.\textsc{str}   \\  
        \glt `amongst you' (NPEGL, OSax.367.476)  

     \ex \label{Osaxd} \gll an herdan sten ovanwardan \\ 
       on hard.\textsc{str} stone upper.part.\textsc{str}   \\ 
        \glt `on the upper part of the hard stone' (NPEGL, OSax.914.974)   
   \end{xlist}
\end{exe} 

These few examples do not convey much that has not already been addressed. Note that we  only have \isi{agreement} constructions, no genitival dependents. It is, however, worthwhile dwelling for a moment on the item \textit{ovan-verd} in  (\ref{Osaxd}).

\subsection{The component *\textit{-werþ-}}
\label{sec:10:vert}

We have already seen several cognates of the type  \textsc{location} + *-\textit{werþ}- + \textsc{str}, cf. Old \ili{Norse} \textit{ofan-verð-an}  and Old \ili{English} \textit{ufe-weard-an}. In  Old High \ili{German}, we also find etymologically corresponding forms/items comprising the component -\textit{vert}-, cf. (\ref{wart}), but it is not clear that they are relevant in the present discussion.\footnote{Notice that [v] is often spelled 〈uu〉 in Old High German manuscripts, cf. (\ref{wart}).}

\begin{exe}
  \ex \label{wart}
    \begin{xlist}
        \ex \label{warta} \gll inuúertes sint sie ráze uúolua   \\ 
          inwardly are they furious wolves    \\ 
        \glt `inwardly, they are furious wolves' (ReA, T\_Tat41)  
        \ex \label{wartb} \gll  ci thesemo antuuerden libe   \\ 
           to  \textsc{dem} present.\textsc{wk}  life     \\ 
        \glt `to this present life' (ReA, WK\_Weissenburger\_Katechismus)   
    \end{xlist}
\end{exe}

Example (\ref{warta}) involves a fossilized \isi{genitive} -\textit{es} and is used adverbially (cf. modern \ili{German} items in \textit{-wärt-s}). The item \textit{antwert} in (\ref{wartb}),\footnote{Etymologically, it corresponds to \ili{Icelandic} \textit{önd-verður} `former/front-part', `beginning'. } meaning `current, present', is weakly inflected and occurs in the \isi{attributive} position. In all likelihood, it has to be construed as a ``regular'' (non-subsective) \isi{adjective}, rather than a \isi{positional predicate}. Thus it is not a counterexample or problematic case in the same way as constellation (III) is for the examples discussed for Old \ili{English}, cf. (\ref{positioX3}).  At the same time, it does not support anything. More generally, we do not seem to have positive evidence that items in \textit{-vert-} were used as positional predicates in Old High \ili{German}.


Therefore, \textit{ovan-verd-an} in  (\ref{Osaxd}) is a valuable hint that positional predicates of this type also existed in continental West \ili{Germanic}, even though attestations are much scarcer than in Old \ili{Norse} and Old \ili{English}. ANNIS annotates Old Saxon  \textit{ovanverdan} as adverb; when viewed as an isolated case, this decision may be justified, but when viewed in the context of comparable examples from Old \ili{Norse} and Old \ili{English} discussed in previous subsections, even this single example can be seen as part of a larger \isi{pattern}, complying with the \isi{syntax} of positional predicates as characterized here.\footnote{\label{thxGeorge}In this vein, the ending \textit{-an} can be analyzed as strong, masculine, accusative singular (compare the \isi{prenominal} \isi{adjective} \textit{herd-an}). However, it should also be mentioned that certain inflectional endings -- especially \textit{-an} -- are ambiguous/syncretic. It is not even always clear whether \textit{-an} stands for strong or weak \isi{inflection}, or whether it is rather some sort of general-purpose or default \isi{inflection}. Thanks to George Walkden (p.c.) for pointing this out to me; see fn. \ref{thx2}.   } 




\subsection{Beyond partitivity:  \textit{self}}
\label{sec:10:self}

The discussion so far has shown that all the early \ili{Germanic} languages provide evidence for the existence of positional predicates as described in Section \ref{sec:10:posprICE} to varying degrees. More precisely, we have looked at cognates of ``middle'' and compound adjectives in *\textit{-verþ-}. Of course, the mere attestation of these items is not decisive; what matters most is that they manifest the (``deviant'') syntactic properties (\ref{IceCHAR}i--v), notably, occurrence in the \isi{predicative} position and \isi{partitive} interpretation. The \isi{partitive} interpretation had been independently argued for by \citet{Marib} concerning \ili{Latin} and \ili{Classical Greek}. Still, we might ask the question whether the \isi{partitive} reading is the primary or canonical interpretation of the \isi{predicative} position, or just one special case. For one thing, positional predicates in Old \ili{English} can be viewed as a subclass of a large group of adjectives with an ``adverbial reading,''\footnote{I thank Olga Fischer (p.c.) for pointing this out to me.} cf. \citet{Fischer01,Grabski17,Grabski20}. In the same vein, the (Old) \ili{Icelandic} items \textit{þver} `across' and \textit{endilangur} `along'; `from part to the other', cf. (\ref{position}), do not immediately strike one as \isi{partitive} elements even though, morphosyntactically, they \isi{pattern} like all the other positional predicates. In either case, this could be part of a language-specific development, e.g. as an instance of broadening or narrowing the range of interpretations; evidence from the other early \ili{Germanic} languages is too scarce to be helpful in that matter.

\begin{sloppypar}
Apart from that, however, there is another observation of interest, which should be mentioned since I have made reference to evidence from \ili{Greek}.  
Practically every textbook or grammar on \ili{Ancient Greek} uses the example in (\ref{selbo}) when illustrating the two positions of adjectives.
\end{sloppypar}

\begin{multicols}{2}{
\begin{exe}
  \ex  \label{selbo}
     \begin{xlist}
       \ex  Attributive position \\
        \gll ho  autos basileus \\  
            the \textsc{self}   king   \\ 
            \glt `the same king'             \\
        \ex  Predicative position \\ 
         \gll   autos ho basileus  \\ 
         \textsc{self}  the  king  \\
         \glt `the king himself / in person', `even the king'             
     \end{xlist}
\end{exe}}
\end{multicols} 

When occuring in the \isi{attributive} position, the item \textit{autos}, here simply glossed as \textsc{self}, expresses an identity/sameness relation corresponding to \ili{English} \textit{(the) same} (= `same'-reading). However, when occurring in the \isi{predicative} position, it rather acts as a focus modifier emphasizing the referent in some sense and largely overlaps in usage with \ili{English} \textit{him-/herself} (= `self'-reading). When viewed in isolation, this ambiguity could be seen as a quirk of (Ancient) \ili{Greek}. However, when we take into account the bigger cross-\ili{Germanic} picture, we find the same distinction involving the same item \textit{self} (\ref{selbesjalfur}). % -- with a \ili{Germanic} twist:


\begin{multicols}{2}{
\begin{exe}
  \ex  \label{selbesjalfur}
     \begin{xlist}
       \ex  Attributive position  \hfill (\ili{German}) \\
       \gll der selb-e König  \\ 
           the   \textsc{self-wk} king     \\ 
           \glt `the same king'             
       \ex  Predicative position  \hfill (\ili{Icelandic}) \\
        \gll  sjálf-ur konungur-inn  \\  
         \textsc{self-str}  king-\textsc{def} \\ 
        \glt  `the king himself', `even the king'              
     \end{xlist}
\end{exe}}
\end{multicols}


In modern \ili{German}, we visibly only find the `same'-reading  of \textsc{self} (weakly inflected), while in Old \ili{Norse} and modern \ili{Icelandic}, only the `self'-reading is found (strongly inflected).\footnote{The `self'-reading of \textit{self} as such is found in modern \ili{German}, in which case, however, the item \textit{selbst/selber} is not inflected. 
In North \ili{Germanic}, the lexical item \textit{sam-} = `same' has been in use since early on, and replaced the use of \textit{self} in the `same'-reading. } However, we do find subtle remnants of the same systematic alternation, also within one and the same language, at least in early West \ili{Germanic}; compare the a- vs. b-examples in (\ref{seohg})--(\ref{seoe}).

\begin{exe}
  \ex  \label{seohg}  Old High \ili{German}
     \begin{xlist}
    \ex  \gll   demu selb-in tage \\ 
       \textsc{dem} \textsc{self-wk}   day      \ \    \quad{(\isi{attributive}:  `the same')}  \\
        \glt `the same day' (ReA, B\_14)       
    \ex \gll selb-emu dhemu {gotes sune} \\  
      \textsc{self-str}  \textsc{dem}   {God's son}    \ \  \quad{(\isi{predicative}: `himself')}    \\   
        \glt `the son of God himself' (ReA, I\_DeFide\_4)   
     \end{xlist}

  \ex  \label{seos} \ili{Old Saxon}
     \begin{xlist}
        \ex \gll  thia  selv-un   tid   \\  
        \textsc{dem}  \textsc{self-wk}   time     \ \    \quad{(\isi{attributive}:  `the same')}  \\ 
        \glt `the same time' (NPEGL, OSax.522.758)              
    \ex \gll thie  heland   self \\  
     \textsc{dem}   saviour   \textsc{self.str}      \ \  \quad{(\isi{predicative}:  `himself')} \\
        \glt `the saviour himself' (NPEGL, OSax.048.265)
     \end{xlist}

  \ex  \label{seoe}  Old \ili{English}
     \begin{xlist}
    \ex  \gll  þæt sylf-e land  \\  %{ðe wit ær coman}
      \textsc{dem}   \textsc{self-wk} land  \ \ \quad{(\isi{attributive}:  `the same')}    \\ % {that we.\textsc{dual} before came}
        \glt `the same land' (NPEGL, OEng.614.076)
     \ex \gll  þone hælend  silf-ne  \\ 
       \textsc{dem} saviour  \textsc{self-str} \ \  \quad{(\isi{predicative}  `himself')}  \\ 
        \glt `the saviour himself' (NPEGL, OEng.527.762)
     \end{xlist}
\end{exe}



At any rate, on the `self'-reading, the item \textit{self}  behaves like a \isi{positional predicate} with respect to points (ii)--(v) above (i.e. modulo \isi{partitive} interpretation) in several early \ili{Germanic} languages:  it occurs in the \isi{predicative} position (DP-externally), and is strongly inflected.\footnote{\ili{Gothic} is an exception insofar as all occurrences of the item \textit{self}, regardless of use or meaning, appear to be weakly inflected.  } 

In other words, in spite of being a relatively small class compared to regular adjectives, positional predicates may still be part of a larger phenomenon involving other modifiers in non-standard positions with a nonstandard interpretation. The non-standard position in all cases is the \isi{predicative} position, but the non-standard interpretation is not always \isi{partitive}. The commonality observable is thus primarily a syntactic property. Even though attestations are scarce in several cases, the big picture that emerges from the discussion in this section is that this syntactic property is likely to have been a feature of early  \ili{Germanic}. 





%Gull-Þóris saga:   Svo er sagt að 	Ketilbjörn einn vildi 	fara með Þóri \ \\ \ \\ 
%  Surlunga saga:   Í þann tíma kom biskup að söðlabúrinu og hljóp þegar upp á mæninn og fló grjótið á hvorutveggju hlið honum og yfir höfuðið sem í drífu sæi. En er menn kenndu vildu öngvir honum mein gera og stöðvaðist þá bardaginn. ......... Gengu þeir Sturla þá til Órækju. Var þá talað um sættir og var á það sæst sem Órækja bauð að biskup einn skyldi gera um öll óskoruð mál \\ \ \\ \ 
% Guð einn veit, hverju fólki datt í hug að ljúga upp á mig,







\section{Summary and outlook}
\label{sec:10:sum}

The discussion has shown that positional predicates are a class of modifiers with a number of peculiar properties that set them apart from ``regular'' adjectives. One goal has been to establish this class, i.e. to show that they constitute a worthwhile object of investigation in their own right, and that the phenomenon is relevant to early \ili{Germanic} \isi{syntax}. % A corollary  purpose has been to motivate a separate annotation label for positional predicates (distinct from ``regular'' adjectives) in the NPEGL \isi{database}.  
We have established the following prototypical properties:  % semantic and syntactic


\begin{enumerate}[(i)]
     \item  Positional predicates express a temporal/spatial part--whole relation, and they typically combine with nouns denoting temporal/spatial extensions or pluralities. 
    \item They agree with their head \isi{noun}/the rest of the \isi{noun} phrase  in case, number, and gender.  
    \item Especially relevant for \ili{Germanic}: they display the strong \isi{adjectival inflection}, even though they occur in contexts where, at least at a surface glance, the strong \isi{inflection} is unexpected. % (see next point).     
    \item They occur in definite \isi{noun} phrases, and combine with pronouns and proper names;  even when not overtly marked as definite, there is an underlying definite interpretation.   %, the interpretation of the micro/macro \isi{noun} phrase is still definite), 
    \item They precede determiners (when present), such as demonstratives, articles, and possessives. %, which essentially means that they occur in the \isi{predicative} position.
\end{enumerate} 

Point (v) is indicative of the ``\isi{predicative} position'' in \ili{Ancient Greek}, where the same phenomenon (modulo \isi{adjectival inflection}) is found. The construction is also found in \ili{Latin} even though it is not equally visible due to the lack of an \isi{article}. We have seen for \ili{Greek}, \ili{Latin} and some \ili{Germanic} languages that, occasionally, a \isi{genitive} construction is used. However, we have likewise observed instances where a (\ili{Gothic}, Old \ili{English}) translation uses the \isi{agreement} construction instead of a \isi{genitive} construction used in the (\ili{Greek}, \ili{Latin}) original. This is a subtle, but important hint that the \isi{agreement} construction/\isi{positional predicate} in the \isi{predicative} position is part of the native \ili{Germanic} \isi{syntax}, and not imported via scholarly translations. By extension, we may infer that the \isi{syntax} of positional predicates is older than the extant texts.

%Formally, this constellation can be analyzed as a position outside the core \isi{noun} phrase (DP), but still as a part of the same macro \isi{noun} phrase (or \isi{agreement} domain), as suggested by \citet{Marib,Pfaff2015,Pfaff2017}. 

Besides all the commonalities among the early \ili{Germanic} languages, we have also observed some \isi{variation} and deviation from the expected behaviour, presumably as a result of language-specific developments. There is for instance some \isi{variation} in relative positions and co-occurrences; positional predicates occur pre-pronominally in \ili{Gothic}, but post-pronominally in virtually all  other attested cases, while they do not appear to co-occur with pronouns in Old \ili{English} at all. We have also seen various degrees to which a \isi{genitive} construction is used as an alternative to the \isi{agreement} construction. 

In all likelihood, there are more details and questions that remain to be addressed, and, at a more general level, we can add the following questions: 


\begin{enumerate}[(i)]
     \item On the assumption that the syntactic peculiarities pertaining to positional predicates/the \isi{predicative} are native to \ili{Germanic},  \ili{Greek} and \ili{Latin}, is this a syntactic property inherited from a common source (PIE), or did it develop independently? 
     \item How widespread is this phenomenon outside \ili{Germanic},  \ili{Greek} and \ili{Latin} -- or, for that matter, outside \ili{Indo-European}? 
\end{enumerate} 

I leave these issues to further investigation. 




\section*{Abbreviations}
\begin{tabularx}{.45\textwidth}{lQ}
\textsc{acc} & accusative \\
\textsc{art} & freestanding {article}  \\
\textsc{cmpr} & {comparative} \\
\textsc{dat} & {dative} \\
\textsc{def} & suffixed definite {article} \\
\textsc{f} & feminine \\
\textsc{gen} & {genitive} \\
Lat. & Latin \\
\textsc{m} & masculine \\
\end{tabularx}
\begin{tabularx}{.45\textwidth}{lQ}
\textsc{n} & neuter\\
\textsc{nom} & {nominative} \\
PIE & Proto-\ili{Indo-European} \\
\textsc{pl} & plural \\
\textsc{prn} & {pronoun} \\
\textsc{sg} & singular \\
\textsc{str} & strong {inflection} \\
\textsc{supl} & {superlative} \\
\textsc{wk} & weak   {inflection} \\
\end{tabularx}


\section*{Acknowledgements}


I would like to thank the members and associated researchers of the project \textit{Constraints on syntactic \isi{variation}: \isi{Noun} phrases in early \ili{Germanic} languages} for discussion.  
I am especially grateful to two reviewers for various comments and constructive feedback, which helped improve the quality of this chapter considerably. All remaining shortcomings are mine.\largerpage

{\sloppy\printbibliography[heading=subbibliography,notkeyword=this]}
\end{document}
