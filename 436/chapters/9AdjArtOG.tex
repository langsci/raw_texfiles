\documentclass[output=paper,colorlinks,citecolor=brown]{langscibook}
\ChapterDOI{10.5281/zenodo.10641199}
\author{Alexander Pfaff\orcid{}\affiliation{University of Stuttgart} and George Walkden\orcid{0000-0001-5950-9686}\affiliation{University of Konstanz}}
%\ORCIDs{}

\title{Adjectival articles in early Germanic}

\abstract{The grammaticalization of demonstratives into definite articles is a well-known phenomenon and has received a lot of attention in the literature. Less attention has been paid to the observation that there is another outcome where the demonstrative develops into an article element -- not of the nominal projection, but narrowly of the adjectival phrase. In North Germanic, the nominal definite article came to be realized as a suffix, which is why specific uses of the former demonstrative as an adjectival article are clearly identifiable. In the other Germanic languages, however, adjectival articles are not as easily identified. Article uses of a demonstrative can simply be construed as a nominal definite article, which, in certain cases, merely happens to be accompanied by an adjective. 
In this chapter, we will first illustrate the properties of the Old Icelandic adjectival article, based on distributional evidence, but also in comparison to modern Icelandic. Next, we will argue that, upon a closer look, evidence can be adduced for an adjectival article in West Germanic and Gothic, as well.}


\IfFileExists{../localcommands.tex}{
   \addbibresource{../localbibliography.bib}
   \usepackage{langsci-optional}
\usepackage{langsci-gb4e}
\usepackage{langsci-lgr}

\usepackage{listings}
\lstset{basicstyle=\ttfamily,tabsize=2,breaklines=true}

%added by author
% \usepackage{tipa}
\usepackage{multirow}
\graphicspath{{figures/}}
\usepackage{langsci-branding}

   
\newcommand{\sent}{\enumsentence}
\newcommand{\sents}{\eenumsentence}
\let\citeasnoun\citet

\renewcommand{\lsCoverTitleFont}[1]{\sffamily\addfontfeatures{Scale=MatchUppercase}\fontsize{44pt}{16mm}\selectfont #1}
  
   %% hyphenation points for line breaks
%% Normally, automatic hyphenation in LaTeX is very good
%% If a word is mis-hyphenated, add it to this file
%%
%% add information to TeX file before \begin{document} with:
%% %% hyphenation points for line breaks
%% Normally, automatic hyphenation in LaTeX is very good
%% If a word is mis-hyphenated, add it to this file
%%
%% add information to TeX file before \begin{document} with:
%% %% hyphenation points for line breaks
%% Normally, automatic hyphenation in LaTeX is very good
%% If a word is mis-hyphenated, add it to this file
%%
%% add information to TeX file before \begin{document} with:
%% \include{localhyphenation}
\hyphenation{
affri-ca-te
affri-ca-tes
an-no-tated
com-ple-ments
com-po-si-tio-na-li-ty
non-com-po-si-tio-na-li-ty
Gon-zá-lez
out-side
Ri-chárd
se-man-tics
STREU-SLE
Tie-de-mann
}
\hyphenation{
affri-ca-te
affri-ca-tes
an-no-tated
com-ple-ments
com-po-si-tio-na-li-ty
non-com-po-si-tio-na-li-ty
Gon-zá-lez
out-side
Ri-chárd
se-man-tics
STREU-SLE
Tie-de-mann
}
\hyphenation{
affri-ca-te
affri-ca-tes
an-no-tated
com-ple-ments
com-po-si-tio-na-li-ty
non-com-po-si-tio-na-li-ty
Gon-zá-lez
out-side
Ri-chárd
se-man-tics
STREU-SLE
Tie-de-mann
}
   \boolfalse{bookcompile}
   \togglepaper[9]%%chapternumber
}{}

\begin{document}
\maketitle

\section{Introduction} 

Definite articles have received their fair share of attention in the literature (DP hypothesis, \isi{definiteness}, \isi{grammaticalization} etc.).     
The definite \isi{article} in question is a nominal \isi{article} that occupies a position in the nominal extended projection (D\textsuperscript{0}) and marks the \isi{noun} phrase/DP as ``definite'', and diachronically usually derives from a \isi{demonstrative}. There is, however, an \isi{article} use that has not been the centre of attention the same way, even though it has not gone unnoticed either. Consider the bold-print elements in examples (\ref{adarza})--(\ref{adarzd}).  % (for simplicity merely glossed as ``the''): 

\begin{exe}
\ex  \label{adarz}
    \begin{xlist}
        \ex  \label{adarza} \ili{Greek} \\ 
        \gll \textbf{to}  megalo \textbf{to} kokkino to vivlio     \\  
        the big the red the book    \\    
        \glt `the big red book'  (\citealt{alexiadouWilder})
        \ex \label{adarzb} \ili{Slovenian} \\ 
        \gll   tá  \textbf{ta} zelen svinčnik   \\   
        this the green pencil \\
        \glt `this green pencil' (\citealt{LankoZaucer})
        \ex \ili{Hebrew} \\ 
        \gll ha-yeladim \textbf{ha}-nexmadim   \\
         the-children the-nice  \\
         \glt `the nice children'  (\citealt{Ritter91})
        \ex  \label{adarzd} \ili{Swedish}, \ili{Norwegian}, \ili{Danish} \\ 
        \gll \textbf{den} store mann-en / \textbf{den} store mand   \\ 
        the big man-the / the big man  \\
      \glt  `the big/tall man'   (cf. mann-\textbf{en} `the man')
\end{xlist} 
\end{exe} 

These \isi{article} elements only occur if the \isi{noun} is modified by an \isi{adjective}; hence we will refer to them as \textit{adjectival articles}. Adjectival articles are formally often identical to the respective regular (= nominal) definite \isi{article} or to a \isi{demonstrative};  they often occur {in addition} to the nominal \isi{article}/a \isi{demonstrative} -- or, as is the case in \ili{Danish}, {instead} of the nominal \isi{article} if the \isi{noun} is modified. 

In this chapter, we will have a close look at adjectival articles in the early \ili{Germanic} languages, which broadly consists of two tasks: 
 firstly, we will give a characterization of the \ili{Old Icelandic} \isi{adjectival article} \textit{(h)inn}.\footnote{The notation \textit{(h)inn} indicates that, in \ili{Old Icelandic} manuscripts, we find instances both with and without an initial <h>.}
%\begin{exe}
%\ex  \label{ONAA}
%    \begin{xlist}
%        \ex \gll  \textbf{hinn} mikli maður\\
%         \textsc{art} tall.\textsc{wk} man\\
%        \ex \gll  þenna \textbf{hinn} sama morgun \\
%         this \textsc{art} same.\textsc{wk} morning \\
%         \glt `that very same morning' (Saga: Brennu-Njáls saga)         
%        \ex \gll  karl \textbf{hinn} skegglausi \\
%          geezer  \textsc{art} beardless.\textsc{wk}\\
%        \glt `the old geezer' (Saga: Brennu-Njáls saga)
%    \ex \gll  \textbf{hinn} mesti dreki-nn\\
%      \textsc{art} biggest.\textsc{wk} dragon-\textsc{def}\\
%\end{xlist} 
%\end{exe}   
Following \citet{Pfaff2019,Pfaff2020,Pfaff2023}, we will argue that \textit{(h)inn} is, in fact, a component of the adjectival constituent (AP) rather than a \isi{determiner} in the extended nominal projection (DP) -- differently from modern \ili{Icelandic} where \textit{hinn} can be argued to occupy the D$^0$ position. This idea can be supported by various observations, the gist of which can be summarized as follows: 1) there is an intimate relationship between  \textit{(h)inn} and precisely one weakly inflected \isi{adjective}; 2) the sequence \textit{(h)inn} + A.\textsc{wk} has the same distribution as strongly inflected adjectival phrases; 3) \textit{(h)inn} (+ A.\textsc{wk}) co-occurs with various other determiners, including demonstratives and the (suffixed) nominal \isi{article}. 
    
Next, we will consider the other early \ili{Germanic} languages, primarily addressing the question whether they even have a designated \isi{adjectival article}, i.e. an element comparable to \textit{(h)inn}. Upon careful examination, it turns out that the formally distal \isi{demonstrative} in West \ili{Germanic}, halfway through the \isi{grammaticalization} path towards a (nominal) definite \isi{article}, does indeed have uses/occur-rences that are on a par with \textit{(h)inn}, but not with what we would expect from a regular definite \isi{article} (or a \isi{demonstrative} for that matter). Similarly, for \ili{Gothic}, it can be shown that the distal \isi{demonstrative} in many cases behaves like an \isi{adjectival article}, notably in cases where the \ili{Gothic} translation deviates from the \ili{Greek} source text.

Notice that the assumed adjectival articles in West \ili{Germanic} and \ili{Gothic}, and a fortiori the definite articles (i.e. the d-determiners), historically derive from the distal \isi{demonstrative} \textit{s\={a}} (with a stem in \textit{þ-}).\footnote{For \ili{Gothic}, of course, we cannot speak of a definite \isi{article} proper because it disappeared from the record before the \isi{article} could fully grammaticalize. As a matter of interest, the adjectival/definite articles in \ili{Greek} and \ili{Slovenian}, see (\ref{adarza}) and (\ref{adarzb}), are also etymologically related to the same \isi{demonstrative}, and so is the \isi{adjectival article} in Mainland \ili{Scandinavian} (\ref{adarzd}), cf.  fn. \ref{demS}.} On the other hand, the \ili{Old Icelandic} \isi{adjectival article} \textit{(h)inn}  derives from a Proto-\ili{Norse} \isi{demonstrative} \textit{hinn} (PIE: *ke + *eno), which in turn is also the source of the suffixed definite (= nominal) \isi{article} in the \ili{Scandinavian} languages.\footnote{\label{demS}The \isi{demonstrative} use of \textit{hinn} (meaning `the other one') is found in \ili{Old Icelandic} and has survived into modern \ili{Icelandic}, whereas it has essentially disappeared from the other \ili{Scandinavian} languages, see \citet{Pfaff2019}.  
In Old East \ili{Norse} (\ili{Old Swedish}, Old \ili{Danish}), \textit{(h)inn} is still found in use as an \isi{adjectival article}, but enters into competition with \textit{sá} (oblique form \textit{þ{æn}}) and is replaced as \isi{adjectival article} early on (see \citealp{Ulla2009,Ulla2015,Ulla2020}; \citealp{Pfaff2019}); eventually the same happens in \ili{Norwegian} and Faroese. \ili{Icelandic} is the only language where \textit{hinn} has survived as \isi{adjectival article}; see Section  \ref{sec:9:prelud}. Demonstrative uses of \textit{hinn} will not be addressed here.

In addition, \textit{{h}inn} has an appositive use, which  \citet{Pfaff2020,Pfaff2023} argues  to represent an intermediate stage, diachronically, between genuine \isi{demonstrative} and \isi{adjectival article}. } 
Etymological difference aside, due to the fact that ``articulization'' has taken two formally/visibly distinct paths in North \ili{Germanic}, but not in West \ili{Germanic}, the \isi{adjectival article} in the former is plainly visible since it only occurs with adjectives, whereas in the latter, it is ``hiding in plain sight'' insofar as it appears to be a regular \isi{article} that merely happens to be accompanied by an \isi{adjective}. 

In addition, the evidence adduced is (partly due to the extant textual material) not always of the same sort, and we cannot always test all properties in all languages.  
Nonetheless, the conclusion will be that, for all early \ili{Germanic} languages, we can identify an element that acts an an \isi{adjectival article},  a formal element that is grammaticalized from a \isi{demonstrative} and that forms a constituent with a weak \isi{adjective}.

The structure of the chapter is as follows: in Section \ref{sec:9:prelud}, we will give a brief characterization of the element \textit{hinn} in modern \ili{Icelandic}. Even though it may be considered an \isi{adjectival article} in the sense that the presence of an \isi{adjective} is a necessary precondition for its occurrence, there are good reasons to assume that it really is a \isi{determiner} in the extended nominal projection. The purpose is to have a \isi{contrast} foil for the different behaviour of the same element in \ili{Old Icelandic}. Following this, in Section \ref{sec:9:sources} we give a brief overview of the sources we draw upon in the rest of the paper.

In Section \ref{sec:9:adjArtON} we discuss the \isi{adjectival article} in \ili{Old Icelandic}. The cumulative evidence from a wide range of observations -- as such and in comparison to modern \ili{Icelandic} -- suggests that \ili{Old Icelandic} \textit{hinn} is a narrow  component of the adjectival phrase. Section \ref{sec:9:wgmc} turns to West \ili{Germanic} and the languages Old \ili{English}, Old High \ili{German}, and \ili{Old Saxon}. Here we argue that three types of evidence -- from possessive + \isi{demonstrative} constructions (Section \ref{sec:9:poss-dem}), \isi{postnominal} adjectives (Section \ref{sec:9:postnom}), and vocatives (Section \ref{sec:9:voc}) -- suggest that what is formally identical to the distal \isi{demonstrative} also has an \isi{adjectival article} function in these languages. In Section \ref{sec:9:gothic} we turn briefly to \ili{Gothic} as a representative of the East \ili{Germanic} branch of the family, showing that here, too, the case can be made for an \isi{adjectival article}. Section \ref{sec:9:conclusion} then summarizes and concludes.



\subsection{Prelude: the adjectival article \textit{hinn} in modern \ili{Icelandic}} 
\label{sec:9:prelud}

Apart from using an \isi{article} suffix with simple definite \isi{noun} phrases (\textit{bil-en} `car-the'), the modern \ili{Scandinavian} nominal system is famous for employing a freestanding \isi{article} that is mandatory if a definite \isi{noun} phrase is modified by an \isi{adjective}; the respective \isi{adjective} occurs in the so-called weak \isi{inflection}, cf. (\ref{gulbil}). 


\begin{exe}
  \ex \label{gulbil}
    \begin{xlist}
      \ex \ili{Swedish} \\
      \gll  \textbf{den} *(gul-a) bilen   \\ 
      \textsc{det}  \phantom{*}yellow-\textsc{wk} car.\textsc{def}   \\ 
      \ex \ili{Danish} \\ 
      \gll \textbf{den} *(gul-e) bil     \\
      \textsc{det}  \phantom{*}yellow-\textsc{wk}   {car}   \\ 
      \ex \ili{Norwegian} \\ 
      \gll  \textbf{den} *(gul-e) bilen    \\ 
      \textsc{det} \phantom{*}yellow-\textsc{wk}  car.\textsc{def}   \\ 
    \end{xlist}  
    \glt all: `the yellow car'
\end{exe}  

\begin{sloppypar}
Modern \ili{Icelandic} also has a freestanding \isi{article} element, \textit{hinn}, and even though it has a different etymology, it behaves alike in several respects, most notably, in that it also requires the presence of an adjectival modifier, as in (\ref{free1}).\footnote{\label{standP}But not vice versa; the standard \isi{pattern} of \isi{definiteness} marking with modified \isi{noun} phrases employs the suffixed \isi{article}: \textit{gul-i bíll-inn} `yellow-\textsc{wk} car-\textsc{def}', not the freestanding \isi{article}. All non-sourced \ili{Icelandic} examples are from fieldwork by Alexander Pfaff. } 
\end{sloppypar}

\begin{exe}
  \ex  \label{free1}  
    \begin{xlist}
      \ex\gll \textbf{hinn} *(meint-i) njósnari  \\ 
      \textsc{art}  \phantom{*}alleged-\textsc{wk}  spy   \\ 
      \glt `the alleged thief' 
      \ex\gll  \textbf{hin} *(fræg-a) leikkona  \\ 
      \textsc{art}  \phantom{*}famous-\textsc{wk}  actress   \\
      \glt `the famous actress' 
      \ex \gll \textbf{hið} *(fullkomn-a) fyrirtæki  \\ 
      \textsc{art} \phantom{*}perfect-\textsc{wk} enterprise   \\
      \glt `the perfect enterprise/company' 
    \end{xlist}  
\end{exe}

There are certain semantic and stylistic restrictions on the use of \textit{hinn}, and it mostly occurs in written language. Nonetheless, it displays a number of interesting syntactic properties, as will be illustrated below.  
For one thing, in principle any number of adjectives can occur between \textit{hinn} and the \isi{noun} (\ref{free2a})--(\ref{free2c}). The observation that adjectives can be modified by an adverbial/measure phrase illustrates that \textit{hinn} combines with an adjectival projection/AP rather than simply with an \isi{adjective} item (\ref{free2c})--(\ref{free2d}).\footnote{\label{strange}Notice that (\ref{free2c}) constitutes an  intermediate case; on the one hand, the ordinal \isi{numeral} occurs as a separate  \isi{adjective}; on the other hand, it  strictly speaking modifies the following \isi{superlative} \isi{adjective}, not the \isi{noun}. Crucially, both are weakly inflected.  We will return to this kind  of construction in Section  \ref{sec:9:mest}.} Likewise, cardinal quantifiers and numerals can occur between \textit{hinn} and \isi{adjective} (\ref{free2fa})--(\ref{free2fb}); notably, we find cases  with a \isi{numeral} as the only modifier -- without an \isi{adjective} (\ref{free2h}). Moreover, we even find cases where a measure \isi{genitive} phrase appears to be the sole modifier (\ref{free2i}):\footnote{(\ref{free2ia}) could potentially be construed in analogy to (\ref{free2e}), but involving a deleted \isi{adjective}; after all, the alternative \textit{hin tveggja tíma \textbf{langa} sýning} `the two-hours long.\textsc{wk} show' is a possibility. However, it is hard to see which kind of \isi{adjective} could have been deleted in (\ref{free2j}). At least for this latter example, it would seem as though the \isi{genitive} phrase is a modifier of the \isi{noun}, rather than of an (invisible) \isi{adjective}.   }

\begin{exe}
     \ex \label{free2a}
     \begin{xlist}         
        \ex  \gll  {hinn} umdeild-i bresk-i aktívisti   \\ 
      \textsc{art} controversial-\textsc{wk} British-\textsc{wk} activist  \\
        \glt `the controversial British activist'        
      \ex \label{free2b} \gll  hinn svokallað-a klassísk-a fransk-a  arkitektúr  \\  
      \textsc{art}  so-called-\textsc{wk} classical-\textsc{wk} French-\textsc{wk} architecture \\   
        \glt `the so-called classical French architecture'             
     \end{xlist}
     \ex \label{free2c}
     \begin{xlist}
      \ex  \gll hið þriðj-a mest-a bankagjaldþrot {í sögunni}  \\
      \textsc{art} third-\textsc{wk} greatest-\textsc{wk} bankruptcy {in history.\textsc{def} }  \\  
        \glt `the third-greatest bankruptcy in history'        
      \ex \label{free2cc} \gll hin fjórða-a stærst-a borg heimsins  \\
      \textsc{art} fourth-\textsc{wk} biggest-\textsc{wk} city world.\textsc{def}.\textsc{gen}  \\  
        \glt `the fourth-biggest city in the world'        
     \end{xlist}
    \ex \label{free2d}
    \begin{xlist}
      \ex  \gll  hin nýlega frosna  tjörn  \\ 
      \textsc{art} recently frozen-\textsc{wk}  pond \\
        \glt `the recently frozen pond'  (\citealt[3]{Sigurdsson2013resPass})   
      \ex \label{free2e} \gll Hin { 51 árs} gaml-a  einhleyp-a {Lorrea Carr} \\ 
      \textsc{art} [{51 year.\textsc{gen}}]  old-\textsc{wk} single-\textsc{wk}  {Lorrea Carr} \\ 
        \glt `the 51-year-old single Lorrea Carr'        
    \end{xlist}
    \ex \label{free2f}
    \begin{xlist}
    \ex \label{free2fa} \gll hinar mörg-u alþjóðleg-u skuldbindingar okkar  \\ 
      \textsc{art}  many-\textsc{wk} international-\textsc{wk}  obligations our  \\ 
     \glt `our many international obligations'        

      \ex \label{free2fb} \label{free2g} \gll hinar fjórar fræg-u kenningar  \\ 
      \textsc{art}  four famous-\textsc{wk} theories  \\ 
        \glt `the four famous theories'        

      \ex \label{free2h} \gll hin þrjú  lögmál Newtons um  hreyfingu  \\  
      \textsc{art}  three laws Newton.\textsc{gen} about motion    \\ 
        \glt `Newton's three laws of motion'        
    \end{xlist}
    \ex  \label{free2i} 
    \begin{xlist}
        \ex \label{free2ia} \gll hin { tveggja}  tíma  sýning  \\  
      \textsc{art}   [two.\textsc{gen} hours.\textsc{gen}]  show \\  
        \glt `the two-hour show'        
      \ex \label{free2j} \gll  hin { tveggja} barna móðir \\ 
      \textsc{art}  [two.\textsc{gen}  children.\textsc{gen}] mother \\
        \glt `the mother of two children'        
    \end{xlist}    
\end{exe}

Even though \textit{hinn} requires the presence of some (\isi{prenominal}) modificational material, it is not strictly dependent on precisely one weakly inflected \isi{adjective}.  
Setting aside a number of peculiarities, it essentially behaves like a \isi{determiner} element in a high position (above numerals) that triggers the weak \isi{inflection} on (adjectival) modifiers in its c-command domain.  
This is broadly the view that has emerged during the past 30 years or so (e.g. \citealt{Magnusson84}; \citealt{Sigurdsson93NP,Sigurdsson2006}; \citealt{Pfaff2009,Pfaff2014,Pfaff2015,Pfaff2017}; \citealt{Hardarson2016,Hardarson2017}; \citealt{Ingason2016}). A rather simplified schematic can be rendered as in (\ref{tree9}):

\begin{exe}
  \ex \label{tree9} \Tree [.DP [.D$^0$ \textit{\textbf{hinn}} ] [. \textsc{numeral} 
    [.  {\color{gray}\textsc{measure}}\\{\color{gray}\textsc{genitive}}   [. \textbf{ AP$_n$ }\\\textsc{[weak]}   
              [.  \textbf{ AP$_2$ }\\\textsc{[weak]}  [. \textbf{ AP$_1$ }\\\textsc{[weak]}     NP  ]  ] ] ] ] ] 
\end{exe}


In \ili{Old Icelandic}, we find  examples involving \textit{(h)inn} + weak \isi{adjective} that superficially look like the ones found in modern \ili{Icelandic}, e.g. (\ref{free2bb}).

\begin{multicols}{2}{
\begin{exe} 
  \ex  \label{free2bb}  
    \begin{xlist}
%      \ex\gll  hinn blindi maður \\ 
%      \textsc{art}  \textit{blind-\textsc{wk}} \textit{man} \\
%      (Saga: Heimskringla)
%      \ex \gll  hins banna\dh{}a eplis  \\ 
%      \textsc{art}  \textit{forbidden-\textsc{wk}} \textit{apple} \\ 
%      (OIce.289.438)
      
      \ex\gll  hinir íslensk-u menn \\ 
      \textsc{art}  Icelandic-\textsc{wk} men \\
      \glt `the Icelanders' \\  (Saga: Eyrbyggja saga)
      \ex \gll  hið röskvast-a fólk  \\ 
      \textsc{art} bravest-\textsc{wk} people \\ 
      \glt `the most brave people' \\ (OIce.715.541)
    \end{xlist}  
\end{exe} }
\end{multicols}


However, it has been argued that the \isi{syntax} of \ili{Old Icelandic} \textit{(h)inn} is considerably different in that it is not a \isi{determiner} in the extended nominal projection, but forms a narrow constituent with the weak \isi{adjective} to the exclusion of the \isi{noun}, cf. (\ref{ex:Pfafftree}).

\begin{exe}
 \ex\label{ex:Pfafftree} \Tree [.DP {\ \ D$^0$ \ \ } [. [.AP    [.\textsc{Art} {\textit{hinn}} ]  [.weakP  A.\textsc{wk} ] ]  [.{\ \ NP \ \ } ] ] ]   \ \quad{ (\citealt[198]{Pfaff2019}) } \hfill  { } 
\end{exe}

In the following, we will make the case for this latter idea, providing evidence from various early \ili{Germanic} languages for an \isi{article} element that forms a unit with a weak \isi{adjective}.  We will first take a detailed look at \ili{Old Icelandic} establishing the idea descriptively. After that we will examine the West \ili{Germanic} languages, and finally, take a brief look at \ili{Gothic}. 



\subsection{Sources}\label{sec:9:sources}

\begin{sloppypar}
Unless otherwise stated, for all languages, example IDs are drawn from the \textit{Noun Phrases in Early \ili{Germanic} Languages} (NPEGL) \isi{database}; for a comprehensive overview of NPEGL as a \isi{corpus} resource, see \textcitetv{chapters/1Database}. NPEGL IDs are in the format Language.Number.Number, e.g. OIce.681.656 for an \ili{Old Icelandic} example. The Old \ili{English} (OE) portion of NPEGL consists of all nominals drawn from the \textit{York--Toronto--Helsinki Parsed Corpus of Old \ili{English} Prose} (YCOE, \cite{YCOE}), of which at the time of writing circa 3,500 had been more richly annotated according to NPEGL guidelines. \ili{Old Saxon} (OS) is represented in NPEGL by an exhaustive sample of nominals from the C manuscript of the \emph{Heliand}, a 9\textsuperscript{th}-century gospel harmony (see \citealp{Walkden16}). The \ili{Old Icelandic} portion in NPEGL contains the texts in the \textit{Icelandic Parsed Historical Corpus} (IcePaHC, \cite{IcePaHC}), 1150--1350. 
\end{sloppypar}


Additional material for \ili{Old Icelandic} is drawn from  the \textit{Saga Corpus}.\footnote{\url{https://malheildir.arnastofnun.is/?mode=forn}. }  
For OE and Old High \ili{German} (OHG) it was also necessary to supplement the material in NPEGL with other sources. NPEGL does not contain OE poetic sources, and these were investigated using the \textit{York--Helsinki Parsed Corpus of Old \ili{English} Poetry} (YCOEP, \citealt{YCOEP}). For OHG, examples are drawn from the \textit{Referenzkorpus Altdeutsch} 1.1 (ReA, \cite{RefKorpAltD}), part of the \emph{Deutsch Diachron Digital} (DDD) megaproject; the ANNIS search interface\footnote{\url{https://korpling.german.hu-berlin.de/annis3/ddd}.} was used to retrieve them.




\section{The adjectival article \textit{(h)inn} in Old Icelandic}
\label{sec:9:adjArtON}

Contrary to first appearances, there are good reasons to assume that  \textit{(h)inn} is not a regular \isi{article} element in \ili{Old Icelandic}, i.e. a   \isi{determiner} occupying an immediate position in the extended nominal projection (pace 
\citealt{RoehrsSapp2004}; \citealt{Faarlund04,Faarlund2007,Faarlund2009}; \citealt{Laake2007}; \citealt{Lohndal2007}). Instead, it has long since been suggested that it  actually is an element of the adjectival constituent\footnote{Of course, the older authors did not talk about ``constituents'' or ``AP'', etc., but they clearly express the general idea, e.g. \citet[48]{Nygaard05}: ``Den foranstillede artikel er adjektivisk'' (`The preposed \isi{article} is adjectival'). } with \textit{(h)inn} + A forming a unit to the exclusion of the \isi{noun}   
  (for instance \citealt{Nygaard05}; \citealt{Lundeby1965}; \citealt{Perridon96}; \citealt{Skrzypek2009,Skrzypek2010}; \citealt{PerridonSleeman2011}; \citealt{Ulla2009,Ulla2015}; \citealt{BorjarsPayne2016}; \citealt{Borjarsetal16};   \textit{Gelenkartikel} (`linking \isi{article}') in \citealt{Heinrichs54}; \citealt{Himmelmann97}; \textit{\isi{attributive} article} in  \citealt{Riessler2016};  \textit{adjectival complementizer} in \citealt{Pfaff2019}).   
In this subsection, we will summarize some arguments in support of the view that it is  a narrow component of the adjectival phrase, and show that it is \textit{(h)inn} together with a weakly inflected \isi{adjective} that  constitutes an AP.  





\subsection{ ``Bare'' weak adjectives in Old Icelandic}
\label{sec:9:bare}

The first relevant observation is that there is an intimate relationship between \textit{(h)inn} and weak adjectives. \citet[7]{Ulla2009} notes that ``\textit{(h)inn} seems to be just a formal element preceding adjectives with so called weak \isi{inflection}'', and \citet[3]{BorjarsPayne2016} state that ``\textit{(h)inn} allows the weak A to function as an  \textsc{adj}''. In other words, \textit{(h)inn} only occurs when immediately followed by exactly one weakly inflected \isi{adjective}, which could not act as a ``proper'' \isi{adjective} on its own.   
This goes hand in hand with the observation that  bare weak adjectives  are virtually absent in \ili{Old Icelandic}, or at least highly exceptional. Here the qualifier ``bare'' can, in principle, be understood to indicate that the \isi{adjective} is not preceded by anything;  but we will use it in the sense ``weak \isi{adjective} specifically not preceded by \textit{(h)inn}'' (thus weak adjectives only preceded by a \isi{demonstrative} will also count as ``bare''). 


Notice that some bare weak adjectives  are attested; those constitute a closed class and may be referred to as ``functional'' adjectives: determiner-like adjectives, ordinal numerals, and certain superlatives.\footnote{E.g. \textit{sami} `same', \textit{fyrsti} `first', \textit{þriði} `third', \textit{næsti} `next', etc. Note, however, that even these usually occur with preceding \textit{(h)inn}. Thus the generalization is not that functional adjectives are (always) ``bare'', but that they \textit{can} more easily occur without preceding \textit{(h)inn}.}
  This is also the case at earlier stages: \citet{Perridon96} identifies five attestations of bare weak adjectives in the runic \isi{corpus}, and they all qualify as functional under the characterization just given.\footnote{Those five adjectives are: \textit{æningi/æninga} `only-one', \textit{bæzti} `best', \textit{fyrsta} `first', \textit{þriðia} `third'. In addition, he mentions \textit{ungu} (\textbf{uku}) `young'. However, \citet{Ulla2012}  argues against interpreting \textbf{uku} as weak \isi{adjective}, and suggests instead that it has to be read as a name.}    On the other hand, bare weak ``lexical'' adjectives (with descriptive content) are basically non-existent in the oldest texts. Thus the big picture that emerges if we abstract away from the ``noise'' is that bare weak adjectives without  preceding \textit{(h)inn}  are essentially non-existent. 

Traditionally, weak \isi{inflection} is associated with \isi{definiteness}, but, as will be shown in the next subsection, adjectives do not automatically occur weakly inflected when accompanied by  definite elements (like demonstratives and possessives), and they are not found in vocatives, names and name-like expressions -- differently from modern \ili{Icelandic}. Likewise, the modern \ili{Icelandic} standard \isi{pattern} (\textsc{A.wk N-def}), see fn. \ref{standP}, is virtually absent from the older Icelandic. \citet[179--184]{Pfaff2019} shows that in the \textit{Saga Corpus} this \isi{pattern} occurs 11 times, and in IcePaHC (texts from 12\textsuperscript{th}--16\textsuperscript{th} centuries) we find 10 occurrences at most; two examples are shown in (\ref{Imim}) (from \citealt[180]{Pfaff2019}).

\begin{multicols}{2}{
\begin{exe}
\ex \label{Imim}   
    \begin{xlist}
%      \ex\gll  fyrri athlaup-ið \label{Imimd}  \\ 
%       earlier assault-\textsc{def} \\  
%       \glt `the earlier/first attack' (IcePaHC: 1275 morkin)
      \ex\gll  þriðja nótt-in   \\ 
       third night-\textsc{def} \\  
       \glt `the third night' \\ (IcePaHC: 1475 aevintyri)      
%      \ex\gll  dýra blóð-ið   \\ 
%       dear blood-\textsc{def} \\ 
%       \glt (IcePaHC: 1593 eintal)

%      \ex\gll  þriðji   maður-inn    \\ 
%       third man-\textsc{def}   \\
%       \glt (Saga: Sturlunga saga)  			
%      \ex\gll    mesta  styrk-inn   \\
%       greatest strength-\textsc{def}   \\
%       \glt (Saga: Sturlunga saga)    
      \ex\gll   rauðflekkótta  uxa-nn     \\
      red-speckled ox-\textsc{def}   \\
      \glt `the red-speckled ox' \\ (Saga: Vopnfirðinga saga) 
    \end{xlist}  
\end{exe} }
\end{multicols}
 
Fifteen out of this small set of 21 attestations involve functional adjectives in the sense above. At the same time, there are 140 cases where the weak \isi{adjective} is, in addition, preceded by \textit{(h)inn}  (\textsc{art A.wk N-def}),\footnote{\label{DD}Incidentally, the fact that various constellations of ``double \isi{definiteness}'' (where the nominal and the \isi{adjectival article} co-occur) are considerably more frequent than examples like (\ref{Imim}) also suggests that weak adjectives are dependent on \textit{(h)inn}, but do not necessarily interact with the nominal \isi{article} (\textsc{-def}), as in (\ref{BaIsPo}). 
\begin{multicols}{2}
\begin{exe}
 \ex  \label{BaIsPo}
  \begin{xlist}
      \ex\gll  \textbf{hin} litlu hús-\textbf{in}    \\ 
      \textsc{art} little houses-\textsc{def}  \\ 
      \glt `the little houses' \\(OIce.681.656) \\ 

      \ex\gll  tré-\textbf{ð} \textbf{hið} mikla   \\ 
       tree-\textsc{def} \textsc{art} big \\ 
       \glt  `the big tree' \\ (Saga: Gunnlaugs saga)

%      \ex\gll  hest-\textbf{inn} \textbf{hinn} fífilbleika   \\ 
%       horse-\textsc{def} \textsc{art} dandelion.yellow \\ 
%       \glt  `the dandelion yellow horse' (Saga: Víglundar saga)
  \end{xlist}
\end{exe}
\end{multicols}  }  
and several thousand cases where the weak \isi{adjective} is only preceded by \textit{(h)inn}   without the suffixed \isi{article} present  (\textsc{art A.wk N}). 
This staggering numerical discrepancy between \ili{Old Icelandic} and modern \ili{Icelandic} indicates a number of syntactic differences concerning the status of weak adjectives, the \isi{adjectival article} and the nominal (suffixed) \isi{article}. Relevantly, we see once more that bare weak adjectives are extremely rare in \ili{Old Icelandic} even if a potential source of \isi{definiteness} marking is present, unless the \isi{adjective} is also preceded by \textit{(h)inn}.  

Taking \citet{BorjarsPayne2016} one step further,  \citet{Pfaff2019} therefore suggests  that weak adjectives are ``defective'', or ``incomplete'' APs, as it were, and that \textit{(h)inn} is an ``adjectival complementizer'' that, by merging with a weak \isi{adjective}, produces a ``complete'' adjectival phrase:  [$_{xAP}$ \textit{(h)inn} [$_{weakP}$ A.\textsc{wk} ] ]. In the following, we will refer to the unit of these two elements \textit{(h)inn} + A.\textsc{wk} as \textit{weak sequence}.






\subsection{Weak sequences and strong adjectives}


\subsubsection{Adnominal contexts}
\label{sec:9:ADN}

Upon a closer look at the occurrence of adjectival elements in \ili{Old Icelandic}, we discern a recurrent distributional \isi{pattern}: there are slots or (syntactic) contexts where we either find a weak sequence or a strongly inflected \isi{adjective} -- but negligibly rarely (or not at all) a bare weak \isi{adjective}. Below, some prominent such contexts are given: adjectives following a  pronominal possessive (\ref{pos})--(\ref{posB}), adjectives following a \isi{demonstrative} (\ref{dem})--(\ref{demB}),  and adjectives in \isi{noun} phrases used as a direct address ($\sim$ \isi{vocative}) (\ref{voc})--(\ref{vocB}).


\begin{exe}
\ex \label{pos}  \textbf{\textsc{Poss + Art + A.wk:}}  
    \begin{xlist}
         \ex\gll  minn \textbf{hinn} best-\textbf{i} vin \\ 
         my \textsc{art} best-\textsc{wk} friend \\ 
         \glt `my best friend' (Saga: Íslendinga þættir)
         \ex\gll    hans \textbf{ina} björt-\textbf{u} frægð \\
         his \textsc{art} illustrious-\textsc{wk} fame \\
         \glt `his illustrious fame'  (OIce.100.538)
         \ex\gll   gullhring sínum \textbf{hinum} góð-\textbf{a} \\ 
         goldring \textsc{poss} \textsc{art} good-\textsc{wk}   \\ 
         \glt `his good gold ring' (Saga: Harðar saga) 
         \ex\gll   karfi hans \textbf{hinn} stór-\textbf{i} \\ 
         ship his \textsc{art} big-\textsc{wk}   \\  
         \glt `his big ship' (OIce.488.876) 
%\clearpage %%%%%% delete again!
    \end{xlist}
    \ex \label{posB} \textbf{\textsc{Poss +  A.str:}}
    \begin{xlist} 
         \ex\gll    sinni  fullkomin-\textbf{ni} vináttu \\ 
         \textsc{poss} complete-\textsc{str} friendship \\ 
         \glt `his complete friendship' (Saga: Sturlunga saga)
         \ex\gll  vors  heilag-\textbf{s} föður \\ 
         our holy-\textsc{str} father \\ 
          \glt `(of) our holy father' (OIce.558.908)
         \ex \gll öxi sína  forn-\textbf{a}   \\ 
         axe \textsc{poss} old-\textsc{str}  \\ 
         \glt `his old axe' (Saga: Sturlunga saga)
         \ex\gll    brauð vort  yfirveranleg-\textbf{t}  \\ 
          bread our spiritual-\textsc{str}   \\ 
          \glt `out spirital bread' (OIce.923.674)
     \end{xlist} 
\end{exe} 


 
%\clearpage

\begin{exe}
\ex \label{dem} \textbf{\textsc{Dem + Art + A.wk:}} %   
    \begin{xlist}
         \ex\gll þann  \textbf{hinn} digr-\textbf{a} mann  \\  
         \textsc{dem} \textsc{art} stout-\textsc{wk} man  \\ 
         \glt `that stout man' (Saga: Heimskringla)
         \ex\gll  þessi \textbf{hin} söm-\textbf{u} orð \\ 
         \textsc{dem} \textsc{art} same-\textsc{wk} words  \\ 
         \glt `these (very) same words' (Saga: Fljótsdæla saga)
         \ex\gll  þau \textbf{hin} spakleg-\textbf{u} fræði \\
         \textsc{dem} \textsc{art} sagacious-\textsc{wk} lore \\
         \glt `that sagacious lore' (OIce.239.056)
    \end{xlist}
    \ex \label{demB} \textbf{\textsc{Dem +  A.str:}}
    \begin{xlist} 
         \ex\gll þann helg-\textbf{an} dóm \\ 
         \textsc{dem} holy-\textsc{str} relic  \\ 
         \glt `that holy relic' (OIce.729.539)
         \ex\gll  þessi vond-\textbf{ur} svikari   \\ 
         \textsc{dem} evil-\textsc{str} traitor  \\ 
         \glt `this evil traitor' (Saga: Íslendinga þættir)
         \ex\gll þeim norræn-\textbf{um} manni   \\ 
         \textsc{dem} Nordic-\textsc{str} man   \\ 
         \glt `that \ili{Norse} man' (Saga: Fóstbræðra saga)
     \end{xlist} 
\end{exe}

%\clearpage %%%%%% delete again!

\begin{exe}
\ex \label{voc} \textbf{\textsc{Vocative: Art +  A.wk:}} %{AP} in vocatives   
    \begin{xlist}
         \ex\gll   hann beiðist svo oft friðar af yður, \textbf{inn} mildast-\textbf{i} konungur \\ 
         he demands so often peace of you \textsc{art} mildest-\textsc{wk} king   \\ 
         \glt `he asked you for peace so often, mildest king' (OIce.657.127)
         \ex \gll Heyr þú, \textbf{hinn} ung-\textbf{i} maður, rís upp \\  
         listen you \textsc{art} young-\textsc{wk} man stand up  \\ 
         \glt `Listen (to me), young man, stand up!' (OIce.707.561)

         \newpage
    \end{xlist}
    \ex \label{vocB} \textbf{\textsc{Vocative +  A.str:}}
    \begin{xlist} 
         \ex\gll Ér, góð-\textbf{ar} konur, bölvið eigi  \\
          ye good-\textsc{str} women curse not \\ 
          \glt `Don't curse, (you) good women.'   (OIce.358.860) 
         \ex \gll Minn virðugleg-\textbf{ur} herra Jón erkibiskup, eg  kæri fyrir yður {upp á} {Sighvat Hálfdanarson}   \\
          my gracious-\textsc{str} lord Jón archbishop I charge  before you against  {Sighvat Hálfdanarson} \\ 
          \glt `My gracious lord archbishop Jón, I bring (these) charges against Sighvat Hálfdanarson before you.' (OIce.339.778)
\end{xlist} 
\end{exe} 


Observations like these suggest two things: firstly, the fact that strong adjectives and weak sequences essentially occur in the same environments can be taken to mean that both instantiate the same syntactic object (category), viz. AP. In particular, it strongly corroborates the notion that \textit{(h)inn} really is a component of the AP, rather than simply a definite (nominal) \isi{article}.  
Secondly, even though both \textit{(h)inn} and the weak \isi{inflection} are somehow related to semantic \isi{definiteness}, the distribution cannot (entirely) be governed by semantics, otherwise the occurrence of strongly inflected adjectives  in these contexts would be completely unexpected. Based on the extant material, it is not obvious how to determine whether there is a (systematic) semantic difference in use between the two in examples such as the above, or to what extent a difference would be related to \isi{definiteness}. 

In \isi{contrast}, in modern Icelandic (and modern \ili{Scandinavian} more generally), the distribution of \isi{adjectival inflection} is rather rigidly governed by \isi{definiteness}: weak adjectives (not weak sequences) occur in definite contexts, strong adjectives elsewhere (see esp. \citealt{Pfaff2017}). Thus contexts such as the above simply  involve a bare weak \isi{adjective} in modern Icelandic, cf. (\ref{pos2})--(\ref{voc2}).\footnote{Two further contexts  could be mentioned: (i) adjectives occurring in (fixed) name-like expressions, e.g. `the holy spirit' (both constellations below are attested several times in IcePaHC: 1150.HOMILIUBOK): 
\begin{multicols}{2}
\begin{exe}
\ex \label{nam}    
    \begin{xlist}
         \ex\gll    \textbf{hinn} heilag-\textbf{i} andi  \\ 
         \textsc{art} holy-\textsc{wk} spirit   \\ 
         \ex\gll heilag-\textbf{ur} andi \\ 
         holy-\textsc{str} spirit  \\  
         %(IcePaHC: 1150.\textsc{homiliubok})
     \end{xlist} 
\end{exe} %heilagur andi: 15    hinn heilagi andi:       9      andi heilagur: 3  
\end{multicols}  
Here modern Icelandic uses the standard \isi{pattern}: \textit{heilag-\textbf{i} andi-nn}. 

(ii) a reviewer points out that our claim on distribution is backed up by a rare \isi{pattern} in  \ili{Old Swedish} where a strong \isi{adjective} precedes a definite \isi{noun} (\textit{luct hæræznæmpð-in} `closed.\textsc{str} jury-\textsc{def}'). This alternates with ``double \isi{definiteness}'' cases, see fn. \ref{DD}, and is in \isi{contrast} to the modern Icelandic standard \isi{pattern}. The \isi{pattern} A.\textsc{str N-def} is also found in \ili{Old Icelandic}; see \citet{Pfaff2019} for discussion and a quantitative comparison of those patterns. 
}     

\begin{exe}
\ex  \label{pos2}
    \begin{xlist}
       \ex\gll  minn góð-\textbf{i} / *-ur vinur     \\ 
        my good-\textsc{wk} / \phantom{*}-\textsc{str} friend  \\    
        \ex\gll minn (*hinn) best-\textbf{i}  vinur   \\ 
        my \phantom{(*}\textsc{art}  best\textsc{-wk} friend  \\     
\end{xlist} 
\end{exe} 
\begin{exe}
\ex  \label{dem2}  
    \begin{xlist}
         \ex\gll þessi vondi-\textbf{i} / *-ur svikari     \\   
         \textsc{dem} evil-\textsc{wk} / \phantom{*}-\textsc{str} traitor \\ 
         \ex\gll þann  (*hinn) digr-\textbf{a} mann  \\ 
          \textsc{dem} \phantom{(*}\textsc{art} stout-\textsc{wk} man \\     
\end{xlist} 
\end{exe} 
\begin{exe}
\ex   \label{voc2}
    \begin{xlist}
         \ex  \label{voc2a} \gll kær-\textbf{u} / ágæt-\textbf{u} gestir  \\ 
         dear-\textsc{wk} / good-\textsc{wk} guests  \\    
         \ex  \label{voc2b} \gll   Quo vadis, laus-\textbf{i}  greinir?  \\ 
         \textit{quo} \textit{vadis} free-\textsc{wk} \isi{article} \\   
         \glt `Whither goest thou, free \isi{article}?'
\end{xlist} 
\end{exe}  


Pronominal possessives ``trigger'' the weak \isi{inflection} on a following \isi{adjective}; the strong \isi{inflection} is ruled out in this context, and so is the occurrence of \textit{hinn}  in a post-possessive position, cf. (\ref{pos2}).  The same can be said about demonstratives, cf. (\ref{dem2}). 
Bare weak adjectives also occur in direct addresses and can thus be said to have a \isi{vocative} function in these contexts; this applies not merely to adjectives conventionally used in addressings, cf. (\ref{voc2a}),  but to any \isi{adjective} occurring in an address \isi{noun} phrase (cf. (\ref{voc2b}), which is the title of a talk given in 2012).





\subsubsection{Predicative  contexts}
\label{sec:9:PRD}

As an initial observation, notice that, typically (but not exclusively) in \isi{predicative} position,  adnominal weak sequences involving a \isi{superlative} \isi{adjective} yield a so-called absolute/\isi{indefinite} \isi{superlative} (also known as ``elative'') interpretation: no actual comparison is involved and the \isi{superlative} does not indicate the unique extreme, but merely a high degree on a scale established by the property denoted by the \isi{adjective}, cf. (\ref{IL}).
 
\begin{exe}
\ex \label{IL}   
    \begin{xlist}
       \ex \label{ILa} \gll  Þúríður var hin vitrasta kona  \\
       Þúríður was \textsc{art} wise.\textsc{supl.wk} woman  \\ 
       \glt `Þúríður was a very wise  woman.' (Saga: Fljótsdæla saga) \\ 
       \glt (NOT: `the wisest woman  among all women out of a given group')  \\ 
%       \ex\gll   hann var hinn vinsælsti maður  \\ 
%         he was  \textsc{art} popular.\textsc{supl.wk} man \\
%         \glt `he was  a very popular man'  (Saga: Heimskringla)
      \ex \label{ILb} \gll  Hann  var hið mesta illmenni \\ 
       he  was \textsc{art} big.\textsc{supl.wk} villain \\ 
       \glt `He was the greatest villain (i.e.  a very bad person).'
       \glt {(Saga: Brennu Njáls saga)}
      \ex \label{ILc} \gll  Skildu þeir með  hinni mestu vináttu  \\
       departed they with \textsc{art} great.\textsc{supl.wk} friendship \\ 
       \glt `They departed with great friendship.'  (OIce.260.119) %{(IcePaHC: 1350.\textsc{finnbogi})}
\end{xlist} 
\end{exe}  


Weak sequences of that kind also occur in \isi{predicative} position on their own (\ref{SL}), which includes \isi{coordination} structures, cf. (\ref{SLe}).\footnote{Notice also (\ref{SLd}), which would violate the uniqueness condition if the \isi{superlative} did have its \isi{comparative} meaning/use in this example.  } 


\begin{exe}
\ex \label{SL}   
    \begin{xlist}
         \ex \label{SLa} \gll   Gunnar var \textbf{hinn} \textbf{reiðasti}  \\ 
           Gunnar was \textsc{art} angry.\textsc{supl.wk}  \\ 
           \glt `Gunnar was  very   angry.'   (Saga: Brennu Njáls saga)
         \ex\gll  Trausti  var \textbf{hinn} \textbf{kátasti} \\ 
          Trausti was \textsc{art} cheerful.\textsc{supl.wk} \\ 
          \glt `Trausti was  very   cheerful.'  (Saga: Víglunda saga)
          \ex\gll   konungur var \textbf{hinn}  \textbf{glaðasti} \\ 
          king was \textsc{art} glad.\textsc{supl.wk}  \\ 
          \glt`the king was  very   glad'   (Saga: Bárðar saga Snæfellsáss)
         \ex \label{SLd}  \gll   hvorirtveggju voru \textbf{hinir} \textbf{óðustu}  \\ 
         each.of.two were \textsc{art} frantic.\textsc{supl.wk} \\ 
         \glt `both were extremely furious'    (Saga: Eyrbyggja saga)
          \ex \label{SLe}  \gll  Jarl var   \textbf{hinn}  \textbf{reiðasti} og \textbf{hinn}  \textbf{erfiðasti} lengi   \\ 
           jarl was \textsc{art} angry.\textsc{supl.wk} and \textsc{art} difficult.\textsc{supl.wk} long \\
           \glt `The Jarl was  very   angry and (very) irritable for a long time.' \\ 
           \glt {(Saga: Brennu Njáls saga)} \\  \
         \ex \label{SLf} \gll {compare with (\ref{SLa}): \ \ } Bolli [...] var \textbf{mjög} \textbf{reiður}   \\ 
         { } Bolli {} was very angry.\textsc{str} \\
         \glt  `Bolli was very angry.' (Saga: Laxdæla saga)
\end{xlist} 
\end{exe}  


The crucial point to observe here, other than the elative interpretation itself, is that the weak sequence merely denotes a property,  just like any \isi{adjective} in \isi{predicative} position -- such as the strong AP in (\ref{SLf}).  In other words, examples like these suggest that weak sequences (can) have a simple adjectival interpretation. So in addition to the distributional evidence discussed in Section  \ref{sec:9:ADN}, we also have a semantic perspective corroborating the view that weak sequences are, in fact, APs.

Comparing (\ref{IL}) and  (\ref{SL}), it would seem as though weak sequences involving individual level (IL) properties (\textit{wise}, \textit{popular}, ...) occur adnominally, and weak sequences involving stage level (SL) properties ({\textit{angry}, \textit{glad}, \textit{furious}, ...}) can occur on their own. While we have not found any examples involving SL adjectives in adnominal contexts like (\ref{IL}) so far, weak sequences involving  IL predicates are occasionally found  in \isi{predicative} position, especially, so it seems, when coordinated with a strong AP, cf. (\ref{cood}). Maybe a better clue is given by  (\ref{ILb}) and (\ref{ILc}) where the \isi{adjective} involved is  non-intersective, i.e. it does not denote a separate property, but its interpretation is dependent on the property denoted by the \isi{noun}. In other words, here \textit{great} denotes a degree on a scale indicating \textit{villain-ness} or \textit{friendly-ness}, rather than an independent property \textit{great-ness}.  So the weaker generalization might be that, minimally, the \isi{adjective} involved in cases like (\ref{SL}) must be  \isi{predicative} (i.e. of type 〈e,t〉). 

Finally, it should be mentioned that weak sequences  can be coordinated with ``proper'' APs headed by a strongly inflected \isi{adjective} in \isi{predicative} position, as in (\ref{cood}).

\begin{exe}
\ex \label{cood}   
    \begin{xlist}
         \ex\gll   Var það lið  $[_{AP}$ hið  fríðasta$]$  {\textbf{og}}  $[_{AP}$ vopnað allvel$]$ \\ 
           was that army  { } \textsc{art} fine.\textsc{supl.wk} and { } armed.\textsc{str} all.well \\ 
           \glt `That army was very fine (= consisting of fine men) and extremely well armed.'   (Saga: Egils saga Skallagrímssonar)
          \ex\gll   hann  var   $[_{AP}$ hinn  vasklegasti$]$  {\textbf{og}}  $[_{AP}$ fullur     af  ofurkappi$]$   \\
           he was   {} \textsc{art} brave.\textsc{supl.wk} and {} full.\textsc{str} of over-eagerness  \\ 
           \glt `he was very brave and full of over-eagerness' \\ 
           \glt (Saga: Þórðar saga hreðu) 
\end{xlist} 
\end{exe} 

Under the premise that only like categories can be coordinated, this would be the definite argument in favour of the idea that weak sequences are APs rather then (possibly elliptical) NPs (or DPs). However, \isi{coordination} is not an absolutely perfect criterion; notably, \isi{predicative} categories are more flexible in that respect -- after all, \isi{predicative}  NP \& AP coordinations are well-known (``she is [a linguist] and [proud of it]''). Therefore, examples like these should rather be seen in conjunction with the other observations, as cumulative evidence. But there are further observations from \isi{coordination} to be discussed in Section \ref{subsect:multadj}.




\subsection{Multiple adjectives (and adjectival〈-like〉 modifiers)}
\label{subsect:multadj}

Various observations from the distribution of bare weak adjectives, weak sequences, and strong adjectives  suggest that weak sequences  in many crucial respects behave like strong adjectives, and hence should be treated alike, viz. as APs. The particular suggestion is that \textit{(h)inn} should be construed as an \isi{adjectival article} in a narrow sense, i.e. as a component of AP rather than a definite \isi{article} in the (extended) nominal projection (DP). A natural expectation following  from that conclusion is that, in cases of \isi{adjectival stacking}, i.e. in \isi{noun} phrases comprising more than one \isi{adjective}, we should find an \isi{adjectival article} with every individual \isi{adjective}, similarly to the phenomenon known as \textit{Determiner Spreading}, e.g. in Modern \ili{Greek} (\ref{greek}).


\begin{exe}
\ex       \label{greek}
    \begin{xlist}
         \ex\gll \textbf{to}  {megalo}  \textbf{to} {kokkino} to vivlio    \\  
         \textsc{art}  big \textsc{art} red the book  \\
         \glt `The big red book' (\citealt[303]{alexiadouWilder})
         \ex    to   vivlio  \textbf{to} {kokkino}  \textbf{to} {megalo}  
         \ex  \textbf{to} {megalo} to vivlio  \textbf{to} {kokkino}  
         \ex  \textbf{to} {kokkino} to vivlio  \textbf{to} {megalo} 
\end{xlist} 
\end{exe} 

However, \isi{adjectival stacking} in the narrow sense is extremely rare in \ili{Old Icelandic}, also with strong adjectives. What at first glance may look like multiple adjectives can usually be broken down into a quantificational element (such as \textit{many}) or some kind of functional \isi{adjective} (e.g. \textit{other}), or both, alongside the actual (lexical) \isi{adjective} (see also \cite{Bech17}), as in (\ref{gofugir}).\footnote{One strategy occasionally used to accommodate multiple (strong) adjectives is for the \isi{noun} to be ``flanked'' on both sides (which actually may involve functional adjectives and cardinal quantifiers), as in (\ref{flanked}) (see also \citetv{chapters/3Modifiers}). 
\begin{multicols}{2}{
\begin{exe} 
\ex \label{flanked}   
    \begin{xlist}
         \ex\gll  einn ungur maður fátækur  \\ 
          one young man poor  \\ 
          \glt (OIce.008.041)
          \ex\gll  svörtu  merhrossi góðu  \\ 
            black mare good\\ 
            \glt (Saga: Eyrbyggja saga) 
          \ex\gll góðir  menn margir  \\ 
             good men many   \\
             \glt (Saga: Ljósvetninga  saga) 
         \ex \gll mörgum  manni öðrum  \\ 
          many.a man other \\ 
          \glt (Saga: Sturlunga saga)
\end{xlist} 
\end{exe}  }
\end{multicols}}


\begin{exe} 
   \ex \label{gofugir}
   \gll margir aðrir göfugir   menn  \\ 
     many other noble men \\
   \glt `many other noble men' (Saga: Svarfdæla saga)
\end{exe} 

\subsubsection{Adjectival coordination: Occurrence per adjective}

Usually, when the \isi{noun} phrase comprises more than one \isi{adjective}, those are coordinated.   Most commonly, this is a matter of two (or more) strong adjectives, but there are also cases of strong adjectives and weak sequences being coordinated, in both orders, which ties in with the observations made in the previous subsection. Most relevantly, there are also several cases of two weak sequences being coordinated. Some examples are given  in (\ref{xx}) (see also \cite[192--193; fn. 36]{Pfaff2019}). %\footnote{NB: There are three general constellations of adjectival \isi{coordination} in Old \ili{Norse}, pre-, cross-, and \isi{postnominal}: AP \& AP N;  AP N \& AP;   N AP \& AP.  Note furthermore that we do not find constellations of two bare weak adjectives or strong and weak adjectives being coordinated. There are, however, two instances of  [\textsc{art A.wk}]  \& [\textsc{A.wk}] to be found in \textit{Saga}; note however, that in both cases both adjectives are superlatives (see. Section  \ref{sec:9:bare};  see also \citet[192--193; fn. 36]{Pfaff2019}). }

\begin{exe}
\ex \label{xx}   
    \begin{xlist}
       \ex\gll \textbf{hinn} hraustasti og  \textbf{hinn} vaskasti drengur \\
       \textsc{art} energetic.\textsc{supl} og \textsc{art} brave.\textsc{supl}  lad  \\ 
       \glt `the most energetic and bravest young man'  
       \glt (Saga: Gunnlaugs saga ormstungu)
       \ex \gll  \textbf{inn} sanni og \textbf{inn}  eilífi drottinsdagur \\
       \textsc{art} true and \textsc{art} eternal  {Lord's.day}  \\ 
       \glt `the true and eternal day of the Lord'  (OIce.704.345)
%        \ex\gll \textbf{hina}  efnilegustu  menn  og  \textbf{hina}  hraustustu  \\ 
%        \textsc{art} promising.\textsc{supl.wk} men and \textsc{art} valiant.\textsc{supl.wk}  \\
%        \glt ``the most promising and (most) valiant men'' \\ 
%        \glt (Saga: Finnboga saga ramma)
        \ex\gll \textbf{hinir}  bestu  menn og \textbf{hinir} vitrustu  \\ 
        \textsc{art} best.\textsc{supl} men and \textsc{art} wise.\textsc{supl} \\  
        \glt `the best and wisest men'   (Saga: Heimskringla)
        \ex\gll  Þú  \textbf{hið}  arga og \textbf{hið} illa kvikindi \\ 
         you \textsc{art} vile and \textsc{art} evil creature \\
         \glt `you vile and evil creature'   (Saga: Flóamanna saga) 
        \ex\gll  Hálfdan \textbf{hinn} mildi og \textbf{hinn} matarilli \\ 
          Hálfdan \textsc{art} mild and \textsc{art} meat.stingy \\ 
          \glt `Hálfdan the Mild and Meat-stingy'   (Saga: Heimskringla)
\end{xlist} 
\end{exe}

Many instances of this structure are found in \isi{predicative} \isi{noun} phrases, but also in argumental \isi{noun} phrases, \isi{vocative} \isi{noun} phrases, and even  appositive nominals constituting an epithet with a proper name.  One crucial observation linking back to the issue raised in the introduction of this subsection is that  the \isi{adjectival article} \textit{(h)inn}  is repeated with every \isi{adjective}/adjectival conjunct.  % Moreover, the \isi{adjectival article} does not establish a separate (possibly elliptical) \isi{noun} phrase or referent; both adjectival conjuncts modify one and the same referent in the above examples. 
In other words, examples like these are another strong  indication  that \textit{(h)inn} really belongs with the AP, but the same point can be made more clearly, cf. (\ref{demCo}).

\begin{exe}
\ex \label{demCo}   
    \begin{xlist}
        \ex \gll fé það \textbf{hið} mikla og \textbf{hið}  góða \\
         money \textsc{dem} \textsc{art} big/much and \textsc{art} good \\
         \glt `that handsome amount of money'   (Saga: Brennu Njáls saga) 
         
        \ex\gll    skaða  þann \textbf{hinn}  mikla og \textbf{hinn} illa \\ 
         damage \textsc{dem} \textsc{art} extensive and \textsc{art} bad  \\ 
         \glt `that extensive and bad damage'   (Saga: Brennu Njáls saga)  
         
        \ex \label{demCoc} \gll   þeim  \textbf{hinum}  smám og \textbf{hinum}  fám skipum \\ 
         \textsc{dem} \textsc{art} small and \textsc{art} few ships  \\
         \glt `those few small ships'   (Saga: Sturlunga saga)  
         
\end{xlist} 
\end{exe} 

 As was already shown in (\ref{dem}), weak sequences often occur in \isi{noun} phrases headed by a \isi{demonstrative}. In cases involving adjectival \isi{coordination} structures, we find that the \isi{demonstrative} occurs once per \isi{noun} phrase, but \textit{(h)inn} occurs once per \isi{adjective}.\footnote{For comparison: in modern \ili{Icelandic}, both demonstratives and \textsc{art} occur once per \isi{noun} phrase, and what is coordinated are bare weak adjectives: 
\begin{exe}
\ex \label{xx2}   
    \begin{xlist}
        \ex \gll  hinn einfaldi og augljósi sannleikur \\  
        \textsc{art} simple and obvious truth    \\
        \glt `the simple and obvious truth'
         \ex\gll   þessi mikilhæfi  og fjölgáfaði strákur \\
         \textsc{dem} talented and highly.intelligent boy  \\
         \glt`this talented and highly intelligent boy'
\end{xlist} 
\end{exe} }   This  iterates the point that \textsc{art} belongs more closely with the \isi{adjective} rather than with the \isi{noun} phrase, and once more corroborates  the idea  that \textsc{art} forms a constituent with a weak  \isi{adjective} (= AP) to the exclusion of the \isi{noun}. 




\subsubsection{Numerals}
\label{sec:9:numi}

Cardinal  quantifiers (\textit{many, few, ...}) are treated as adjectives in \ili{Old Icelandic} insofar as they display a strong/weak alternation, they can be coordinated with regular adjectives  as is  illustrated in  (\ref{demCoc}), and consequently, they occur with \textit{(h)inn}. In \isi{contrast}, cardinal  numerals, which otherwise behave rather similarly in terms of semantics and \isi{syntax}, cannot be construed as adjectival elements in the same way as shown by \citet[192--193]{Pfaff2019}. Apart from the fact that numerals do not inflect weakly and are not attested in adjectival \isi{coordination} structures, there are no attestations of \textsc{art} + \isi{numeral}, either;  %\footnote{\citet[193]{Pfaff2019} actually reports one instance of \textit{hinn} + \isi{numeral} (see boldprint below), but a closer look  at the wider context reveals that the item \textit{hinir} in that example has to be interpreted as the etymologically related  \isi{demonstrative} meaning  ``the-other (out of two alternatives)'' -- not as \textsc{art}: 
%\begin{exe}
%    \ex\gll eyxn [...] aðrir þrír eyxn og voru þeir rauðir [...] \textbf{hinir} \textbf{þrír} \textbf{eyxn} [...] {svartir að  lit} \\
%     oxen {} other three oxen and were they red {} \textsc{dem} three oxen {}  {black in colour}   \\	
%     \glt (IcePaHC: 1260.\textsc{jomsvikingar.nar-sag},.152)
%\end{exe}
%There are two groups of three oxen each, and the relevant phrase can only mean ``the other three oxen''.  }   
not in isolation, and especially not as an intervening element between \textit{(h)inn} and A.\textsc{wk}. The latter would be a natural expectation on the assumption that \textit{(h)inn} were a regular (\isi{noun} phrase) \isi{determiner}  (cf.  \citealt{Cinque2005U20}), and, as already illustrated in the introduction, this is exactly what we find in modern \ili{Icelandic}, cf. (\ref{free2fb}) and (\ref{free2h}). 
  


\subsubsection{The ``third-greatest'' piece of evidence }
\label{sec:9:mest}

It is not only numerals that cannot intervene between \textit{(h)inn} and A.\textsc{wk}; in fact, nothing can occur in that intermediate position, not even adverbial/degree elements -- again, which is what we do find in modern \ili{Icelandic}, cf. (\ref{free2d}) and (\ref{free2i}). This is yet another indication that there is an intimate relationship between \textit{(h)inn} and a weakly inflected \isi{adjective}. We will finally have a brief look at a handful of rather peculiar cases that make the same point from a slightly different angle. First, recall the examples in (\ref{free2c}), one of which is repeated in (\ref{free2crep}).

\begin{exe}
      \ex \label{free2crep}
      \gll hið þriðj\textbf{-a} mest\textbf{-a} bankagjaldþrot {í sögunni}  \\ 
      \textsc{art} third-\textsc{wk} greatest-\textsc{wk} bankruptcy {in history.\textsc{def} }  \\  
\end{exe} 


As already mentioned, this cannot be considered \isi{adjectival stacking} in the proper sense because the first element (\textit{þriðja}) modifies the following \isi{adjective}, rather than the \isi{noun}, cf. fn. \ref{strange}. But what matters in the present context is that all adjectives following \textsc{art} are weakly inflected. Now consider a corresponding example from \ili{Old Icelandic} in (\ref{mestur}).

\begin{exe}
\ex \label{mestur}   
  %  \begin{xlist}
       \gll  hann var \textbf{hinn} þriðj\textbf{-i} mest\textbf{-ur} lögmaður {á Íslandi} \\ 
         he was \textsc{art} third-\textsc{wk} greatest-\textsc{str} lawyer {on Iceland} \\ 
         \glt `he was the third-greatest lawyer in Iceland' (Saga: Brennu-Njáls saga)
 %       \ex \gll   hann var \textbf{hinn} \th{}ri\dh{}j\textbf{-i} mest\textbf{-ur} lagama\dh{}ur {\'a \'Islandi} \\  \textit{he} \textit{was} \textsc{art}  \textit{third-{\sc wk}}  \textit{greatest-{\sc str}} \textit{lawyer} \textit{on Iceland}  \\  (M\'IM: Brennu-Nj\'als saga) 
%        \ex \gll   {Herra Birgir} var \textbf{hinn} þrið-\textbf{i} mest-\textbf{r}  ráðamaðr {í landinu}  \\ 
%         {Master Birgir} was \textsc{art} third-\textsc{wk} greatest-\textsc{str} counsellor {in country.\textsc{def}} \\  
%         \glt (Fornmannasögur: Saga Hákonar Hákonarsonar)   
% \end{xlist} 
\end{exe} 

Here, the following \isi{superlative} \isi{adjective} occurs strongly inflected; notice that this is not a defect on behalf of that element, which regularly occurs weakly inflected when immediately preceded by \textit{(h)inn} (e.g. \textit{inn mest-i höfðingi} `\textsc{art} greatest-\textsc{wk} chieftain'). Consider furthermore that we do not find strongly inflected adjectives immediately preceded by \textit{(h)inn}, nor do we find more than one weakly inflected \isi{adjective} following \textit{(h)inn}. On the rare occasion that another \isi{adjective} follows \textit{(h)inn} + A.\textsc{wk}, it is strongly inflected: 

\begin{exe}
  \ex \gll hinn  þriðj-\textbf{i} sek-\textbf{ur} maður \\ 
   \textsc{art} third-\textsc{wk} guilty/condemned-\textsc{str} man \\ 
   \glt `the third guilty man; the third one of those guilty' (Saga: Sturlunga saga)
  \ex \gll ins himnesk-\textbf{a} vors heilag-\textbf{s} föður \\
   \textsc{art} heavenly-\textsc{wk} our holy-\textsc{str} father \\   
   \glt `our heavenly holy father' (OIce.558.908)
\end{exe} 

On the reasonable assumption that \textsc{art} (featurally)  interacts with the weak \isi{inflection} one way or another (see \citealt{Pfaff2017}, \citeyear[198]{Pfaff2019}), we can infer that  \textsc{art}  has scope only over one \isi{adjective} in \ili{Old Icelandic}, cf. (\ref{tra1}), but over all adjectives between it and the \isi{noun} in modern \ili{Icelandic}, cf. (\ref{tra2}). 
 
\begin{multicols}{2}{
\begin{exe} 
    \ex \label{tra1} {\small\Tree [.DP {\ \ D^0 \ \ } [. [.AP    [.\textsc{Art} {\textbf{\textit{(h)inn}}} ]  [.weakP  A.\textsc{wk} ] ]  [.{\ \ NP \ \ } ] ] ] }
    \ex \label{tra2} {\small \Tree [.DP [.{\ \ D^0 \ \ } {\textbf{\textit{hinn}}}  ]  [. [.AP   A.\textsc{wk}  ]  [.  [.AP A.\textsc{wk}  ]   [.{\ \ NP \ \ } ] ] ] ]    }
\end{exe} }
\end{multicols} 



\subsection{Summary}
In this section, we have provided various pieces of evidence to the claim that \textsc{art}  is narrowly associated with the \isi{adjective} (= is a part of AP) to the exclusion of the \isi{noun} in \ili{Old Icelandic}. We have pointed out distributional, semantic, and morphological properties all supporting that claim. Also  by comparison, we have seen that \textsc{art} has a rather different status in \ili{Old Icelandic} and modern \ili{Icelandic}.
















\section{An adjectival article in West Germanic}\label{sec:9:wgmc}

Cognates of \textit{(h)inn} with the status of an \isi{article} are not found outside the North \ili{Germanic} languages. The early West \ili{Germanic} languages do exhibit reflexes of Proto-\ili{Germanic} *jainaz (as does \ili{Gothic}), but these have retained the semantics of a distal deictic \isi{demonstrative} up until the present day. Perhaps because of this fact, it is not generally thought that an \isi{adjectival article} can be found in West \ili{Germanic}. \citet[30--37]{Heinrichs54} proposes that the \isi{demonstrative} can function as a \textit{Gelenkpartikel} (`linking particle'), and adduces examples from early West \ili{Germanic} languages, but his treatment is not systematic, and has had little influence on subsequent work.\footnote{A notable exception is \citet{Allen2006}, which we discuss in Section \ref{sec:9:poss-dem}.}

In this section, we make the case that West \ili{Germanic} indeed shows evidence of an \isi{adjectival article}. We begin in Section \ref{sec:9:adjart-eWGmc} with a brief discussion of the literature on articles in the early West \ili{Germanic} languages, as the dating of the emergence of definite and \isi{indefinite} articles, and DP structure in general, is disputed. Subsequently we discuss the different strands of evidence that lead us to suggest that the early West \ili{Germanic} languages might have had adjectival articles after all. Our empirical focus is on the three West \ili{Germanic} languages attested in the first millennium CE: Old \ili{English} (OE), Old High \ili{German} (OHG), and \ili{Old Saxon} (OS).

\subsection{Evidence for an adjectival article in early West Germanic}\label{sec:9:adjart-eWGmc}

Recent research on OE and OHG suggests that \isi{grammaticalization} of demonstratives as definite articles was more advanced at this earlier stage than previously thought (\citealt{Wood2007}; \citealt{Crisma2011}; \citealt{Sommerer2018}; \citealt{Allen2019}; \citealt{Flick2020}). In the most extensive study of \isi{article} emergence in the history of \ili{English} to date, using prose evidence, \citet[312]{Sommerer2018} concludes that ``the form \emph{se} takes up \isi{article} function from early Old \ili{English} onwards'', and increases dramatically in frequency during the period. \citet{Crisma2011} shows that the use of historically \isi{demonstrative} forms is higher in prose than in (putatively early) poetic texts, and proposes that the definite \isi{article} in \ili{English} was already established by the time of the ``Alfredian" prose of the second half of the 9\textsuperscript{th} century. Similarly, \citet[207]{Flick2020} reaches the conclusion that the development of the definite \isi{article} has already progressed substantially by early OHG.\footnote{``Die Funktionsanalyse von \emph{dër} hat gezeigt, dass die Entwicklung des Definitartikels schon im frühen Althochdeutschen weit fortgeschritten ist.''} This raises a problem for any proposal suggesting that demonstratives also grammaticalized as adjectival articles: how are we to distinguish definite articles from adjectival articles?

We will henceforth refer to what was historically the distal \isi{demonstrative} as \textsc{dem}, without prejudice as to its categorical status, except where more specificity is required in particular contexts. The distributional diagnostics presented in the following subsections are intended to isolate contexts in which \textsc{dem} can be neither a definite \isi{article} nor a \isi{demonstrative}. 



In Section \ref{sec:9:poss-dem} we discuss patterns of co-occurrence of \textsc{dem} and possessives. Section \ref{sec:9:postnom} discusses the use of \textsc{dem} postnominally, and Section \ref{sec:9:voc} presents its use in \isi{vocative} contexts.

\subsubsection{Possessives and \textsc{dem}}\label{sec:9:poss-dem}

In Present-day \ili{English} (PDE), and in many other languages for which it is widely assumed that the \isi{article} is the head of DP, \isi{prenominal} possessives may not co-occur with articles: *\emph{the my book}, *\emph{my the book}, *\emph{Mary's the book}, *\emph{the Mary's book}. Evidence for the co-occurrence of \textsc{dem} and possessives has therefore played a role in the debate around DP status in OE: \citet[§4]{Wood2007} summarizes the findings. Evidence of \textsc{dem} preceding the possessive is not particularly striking or problematic for the DP hypothesis, since similar structures are attested for PDE (e.g. \emph{this(,) my book}), and are usually analysed as close apposition. Another potential approach sketched by \citet{Wood2007} is to view such structures as involving adjectival possessors in the sense of \citet{Lyons1986,Lyons1999}, parallel to \ili{Italian} \textit{il mio libro} `the my book'.\footnote{See also \citet{Demske01} on OE and OHG examples.} Either way, such examples are of no relevance to the \isi{adjectival article} hypothesis.

The opposite order, in which the possessive precedes \textsc{dem}, as in (\ref{ex:possdem}) -- henceforth the \textsc{poss} \textsc{dem} construction (cf. \citealp{Sommerer2018}) -- is more interesting. 

\begin{exe}     
 \ex\label{ex:possdem}
 \begin{xlist}
     \ex\gll\textbf{his} \textbf{þam} ecan Fæder\\
     his \textsc{dem.dat.sg} eternal.\textsc{dat.sg.wk} father.\textsc{dat.sg}\\
     \glt `his eternal Father'   (OEng.813.633; Ælfric's Homilies Supplemental)

     \ex\gll\textbf{his} \textbf{þæs} clænan lifes\\
     his \textsc{dem.gen.sg} clean.\textsc{gen.sg.wk} life.\textsc{gen.sg}\\
     \glt `his clean life'   (OEng.269.358; Gregory's Dialogues, C)

     \ex\label{ex:possdemc}\gll\textbf{min} \textbf{se} swetesta sunnan scima\\
     my.\textsc{nom.sg.str} \textsc{dem.nom.sg} sweet.\textsc{supl.nom.sg.wk} sun shine.\textsc{nom.sg}\\
     \glt `my sweetest sunshine'   (YCOEP; cocynew,117.164.1165)
 \end{xlist}
\end{exe}

Such examples, which occur relatively frequently depending on the text (see the figures in Table 1 of \citealp[153]{Allen2006}), share two features which are of particular importance for the \isi{adjectival article} hypothesis. First, they are only found with the historically distal \textsc{dem} and not with the proximal \citep[158]{Allen2006}, suggesting that we are dealing with a grammaticalized form. Secondly, and crucially, they always occur with an \isi{adjective}: that is, there are no examples of \textsc{poss} \textsc{dem} followed immediately by the \isi{noun}.\footnote{For discussion and dismissal of potential counterexamples see \citet{Wood2007demposs}. \citet[156]{Allen2006} identifies just two apparent counterexamples to this generalization, observing that both are from very late manuscripts, which casts doubt on their authenticity.}

\citet[182]{Wood2007} claims that, in examples such as (\ref{ex:possdem}), \textsc{dem} occupies D and the possessive is in Spec,DP. While such a structure allows co-occurrence of \textsc{poss} \textsc{dem}, it fails to predict the exclusive co-occurrence of this construction with a weak \isi{adjective}. An alternative analysis is presented by \citet[158--159]{Allen2006}, who suggests that ``\isi{adjective} phrases, like \isi{noun} phrases, have a slot for a \isi{determiner}''. The tree structure she proposes is given in (\ref{ex:Allentree}) (with glosses added).

\begin{exe}
     \ex\label{ex:Allentree}\Tree [.DP {\ \ Det \ \ } [.NP [.DP    [.Det {\textit{þam}\\\textsc{dem.dat.sg}} ]  [.AP  {\textit{leofan}\\dear.\textsc{dat.sg.wk}} ] ]  [.{\ \ N \ \ } {\textit{þegne}\\thane.\textsc{dat.sg}} ] ] ]   
\end{exe}

What is striking about this tree is that -- modulo labels -- the structure is exactly the same as the one proposed by \citet{Pfaff2019} for \ili{Old Icelandic} on similar but independent grounds (see (\ref{ex:Pfafftree}) above).

The construction is found both in prose and in poetry (see (\ref{ex:possdemc})). To be sure, there is \isi{variation} across and within the early West \ili{Germanic} languages as to the occurrence of the \textsc{poss} \textsc{dem} construction. Starting with \ili{English} itself, it is essentially restricted to the OE period: by the early 12\textsuperscript{th} century it was no longer a productively-used possibility \citep[161--164]{Allen2006}. Within OE, too, there was \isi{variation}, and in this context it is interesting to compare the C text of Gregory's Dialogues -- which plausibly dates to the 9\textsuperscript{th} century\footnote{The manuscript itself is from the second half of the 11\textsuperscript{th} century, but the translation it contains has been associated with Bishop Wærferth of Worcester, working during the reign of King Alfred in the second half of the 9\textsuperscript{th} century.} -- with the revised H text of the 10\textsuperscript{th}--11\textsuperscript{th} centuries (\citealp[§10]{Yerkes1982}; \citealp[164]{Allen2006}; \citealp[180--181]{Wood2007}). There are sixteen examples of the \textsc{poss} \textsc{dem} construction in C where the relevant DP is also found in H (the manuscripts do not overlap in their entirety). In all sixteen cases, the reviser has made changes, and in eleven of them the \textsc{dem} has been deleted. Regardless of whether this is evidence of a diachronic change in progress or simply of inter-individual \isi{variation}, it is clear that the construction was not consistently found across OE texts, hence not consistently preferred by writers of OE.

Turning to OS, the \textsc{poss} \textsc{dem} construction is not found at all in \isi{prenominal} position. In fact, the only place that the construction shows up in NPEGL is in the set phrase in (\ref{ex:fromin}).

\begin{exe}     
     \ex\gll\label{ex:fromin}fro \textbf{min} \textbf{thie} guodo\\
     lord.\textsc{nom.sg} my.\textsc{nom.sg.str} \textsc{dem.nom.sg} good.\textsc{nom.sg.wk}\\
     \glt `my good lord' (OSax.115.210)
\end{exe}     
     
In all, this phrase occurs a further six times in manuscript C of the \emph{Heliand}, each time with exactly the same wording. It also occurs twice in the OS \textit{Genesis} (not included in NPEGL). It could moreover be analysed as a case of apposition. Thus the evidence from the \textsc{poss} \textsc{dem} construction for an \isi{adjectival article} in OS is hardly overwhelming -- though other sources of evidence point in the same direction.

OHG shows a similar lack of evidence for this construction. Searching the OHG texts in the \textit{Referenzkorpus Altdeutsch} (ReA) for a possessive immediately followed by \textsc{dem} only yields two relevant examples, (\ref{ex:OHG-possdem}), both from Otfrid's 9\textsuperscript{th}-century \textit{Evangelienbuch}.\footnote{The following ANNIS query was used: \textit{posLemma = "DPOS" \& posLemma = "DD" \& \#1.\#2}. Examples were then filtered manually.}

\begin{exe}
 \ex\label{ex:OHG-possdem}
 \begin{xlist}
  \ex \label{ex:OHG-possdema} \gll Drúhtin \textbf{min} \textbf{ther} gúato\\
  lord.\textsc{nom.sg} my.\textsc{nom.sg.str} \textsc{dem.nom.sg} good.\textsc{nom.sg.wk}\\
  \glt `my good lord' (ReA; O\_Otfr.Ev.3.7)
  \ex \label{ex:OHG-possdemb} \gll Múater \textbf{sin} \textbf{thiu} gúata\\
  mother.\textsc{nom.sg} his.\textsc{refl.nom.sg.str} \textsc{dem.nom.sg} good.\textsc{nom.sg.wk}\\
  \glt `his good mother' (ReA; O\_Otfr.Ev.4.32)
 \end{xlist}
\end{exe}

The first example differs only by one word from the OS example in (\ref{ex:fromin}). Moreover, in both examples the final \isi{adjective} is part of a rhyming couplet (rhyming with \textit{gimúato} `benevolently' in (\ref{ex:OHG-possdema}) and with \textit{scówota} `viewed' in (\ref{ex:OHG-possdemb})), so one might suspect that the choice of this construction may have been motivated primarily by metrical considerations. Still, insofar as this construction is not simply a case of apposition, the commonalities between OE, OS and OHG may be taken to indicate that the construction was an inherited one, even if it was formulaic and unproductive for the authors of the OHG and OS texts that we have at our disposal.

\subsubsection{Postnominal adjectives}\label{sec:9:postnom}

For OS, element order in nominals is also an indication that we are dealing with an \isi{adjectival article}. There is some flexibility with regard to the position of elements within OS nominals, but when it comes to \textsc{dem} elements -- our focus here -- the possibilities are extremely restricted. By far the most common \isi{pattern} has \textsc{dem} initial within the nominal phrase, as in modern West \ili{Germanic} languages: NPEGL has well over two thousand examples of this type. There are, however, a minority of instances in which \textsc{dem} follows a \isi{common noun}: 33 in total in the NPEGL \isi{database}.\footnote{This includes the seven instances of `lord my the good' discussed in Section \ref{sec:9:poss-dem}.} In every one of them, \textsc{dem} is formally distal, and immediately followed by a weak \isi{adjective}. Examples are given in (\ref{ex:n-dem}).

\begin{exe}     
 \ex\label{ex:n-dem}
 \begin{xlist}
     \ex\gll suerdu \textbf{thiu} \textbf{scarpon}\\
     sword.\textsc{ins.sg} \textsc{dem.ins.sg} sharp.\textsc{ins.sg.wk}\\
     \glt `(the) sharp sword' (OSax.622.918)
     
     \ex\gll himile \textbf{them} \textbf{hohon}\\
     heaven.\textsc{dat.sg} \textsc{dem.dat.sg} high.\textsc{dat.sg.wk}\\
     \glt `(the) high heaven' (OSax.471.220)

     \ex\gll nadra \textbf{thiu} \textbf{feha}\\
     snake.\textsc{nom.sg} \textsc{dem.nom.sg} colourful.\textsc{nom.sg.wk}\\
     \glt `(the) colourful snake' (OSax.429.338)
 \end{xlist}
\end{exe}     

This construction is not restricted to set phrases, but occurs with a variety of nouns and adjectives, as the examples in (\ref{ex:n-dem}) show.\footnote{There are also twelve examples of a proper \isi{noun} followed by an \isi{adjective}; since this is possible in PDE titles such as \emph{Alfred the Great}, it is less comparatively striking.} The fact that \isi{postnominal} \textsc{dem} in OS is restricted to this construction strongly suggests that we are not dealing with a normal \isi{demonstrative} or \isi{article} here.

A related observation is that the converse also holds: just as \isi{postnominal} \textsc{dem} is only possible when immediately followed by a weak \isi{adjective}, so too are \isi{postnominal} weak adjectives in OS only possible when immediately preceded by \textsc{dem}. To all intents and purposes, the two words function as a unit.\footnote{There are in fact a handful of exceptions to this, all involving the elements \textit{selƀo} `self' (an intensifier, as in `God himself') and \textit{eno} `only/alone'. The -\emph{o} ending on these functional elements is formally (masculine) weak. However, \textit{selƀo} and \textit{eno} in this context seem to serve as focus particles rather than prototypical adjectives, and are found interchangeably with the strong forms \emph{self} and \emph{en}. For these reasons we do not consider them counterexamples to our general claim.} Adjectives that follow the \isi{noun} (regardless of whether there is a \isi{prenominal} \textsc{dem} or not) otherwise must be strong. The sequence \textsc{dem} plus weak \isi{adjective} thus appears to have the same distributional properties as strong adjectives on their own, as argued in Section \ref{sec:9:ADN} for \ili{Old Icelandic}.

In OE prose, too, the generalization seems to hold that \textsc{dem} is never \isi{postnominal} unless followed by a weak \isi{adjective}.\footnote{The following search was used: \textit{(NP* iDoms N|N$^*$) AND (NP* iDoms D*) AND (N|N$^*$ precedes D*)}. Two examples were retrieved, one of which (conicodA,Nic\_[A]:15.2.4.313) is a misannotation. The other is \emph{garsecg ðone} `ocean \textsc{dem}' (coalex,Alex:31.3.393), for which it is possible to analyse \emph{ðone} as a variant form of the temporal adverb \emph{ðonne}. By \isi{contrast}, the reversed search with \textit{(D* precedes N|N$^*$)} returns over 80,000 hits. A problem with using the YCOE to search for the \isi{postnominal} \textsc{dem} plus weak \isi{adjective} construction, however, is that all instances of it have been annotated as involving NP-internal apposition. This makes it difficult to distinguish between the construction we are interested in and other, more prototypical cases of apposition (e.g. those in which there is manuscript punctuation between the two phrases). The query \textit{(NP* iDoms NP*PRN*) AND (NP*PRN* iDoms D*)} retrieves all instances, but very many irrelevant examples besides, even when it is further specified that only examples containing adjectives should be included.} Examples of the construction are given in (\ref{ex:oe-prose-n-dem}).

\begin{exe}
 \ex\label{ex:oe-prose-n-dem}
 \begin{xlist}
  \ex\gll geallancoðe \textbf{þa} \textbf{readan}\\
  gall-disease.\textsc{acc.sg} \textsc{dem.acc.sg} red.\textsc{acc.sg.wk}\\
  \glt `(the) red gall disease' (OEng.284.604; Leechbook)
  
  \ex\gll wermod \textbf{se} \textbf{hara}\\
  wormwood.\textsc{acc.sg} \textsc{dem.nom.sg} old.\textsc{nom.sg.wk}\\
  \glt `(the) old wormwood' (OEng.550.650; Lacnunga)
  
  \ex\gll hælend \textbf{se} \textbf{Nadzarenisca}\\
  saviour.\textsc{nom.sg} \textsc{dem.nom.sg} Nazarene.\textsc{wk}\\
  \glt `the Nazarene saviour' (OEng.278.039; Vercelli Homilies)
 \end{xlist}
\end{exe}

Such examples are not hugely common, but then again \isi{postnominal} adjectives are extremely uncommon in OE prose in general: over 96\% of unmodified adjectives in the YCOE are \isi{prenominal} (\citetv{chapters/3Modifiers}), with the majority of the rare \isi{postnominal} adjectives involving specific collocations or structures; one such is the phrase \emph{God ælmihtig} `God almighty' and variants on it, which \citet{Crisma1999} argues involves movement of N to D. 

The constraint operative in OS that \isi{postnominal} adjectives must either be strong or be immediately preceded by \textsc{dem} seems to hold in OE too. In the more richly annotated NPEGL subsample of OE prose, there is only a single \isi{postnominal} weak \isi{adjective}, found in example (\ref{ex:godalmighty}). All of the other 58 examples of \isi{postnominal} adjectives in this sample are strong, including five more instances of \emph{God ælmihtig}.

\begin{exe}
  \ex\gll\label{ex:godalmighty}God elmihtiga\\
  God.\textsc{nom.sg} almighty.\textsc{nom.sg.wk}\\
  \glt `God almighty' (OEng.448.299; Chronicle, E)
\end{exe}

Evidently little can be concluded from this example, especially since it is attested very late, in the Chronicle entry for 1085, by which point the distinction between strong and weak adjectives may already have been starting to blur.\footnote{We are grateful to a reviewer for pointing this out.}

More examples of \isi{postnominal} \textsc{dem} are found in OE poetry, but here the picture is not as clear as in OS. Examples of \isi{postnominal} \textsc{dem} plus weak \isi{adjective} sequences from the YCOEP \citep{YCOEP} are given in (\ref{ex:oe-n-dem}).

\begin{exe}
 \ex\label{ex:oe-n-dem}
 \begin{xlist}
  \ex\gll sele \textbf{þam} \textbf{hean}\\
  hall.\textsc{dat.sg} \textsc{dem.dat.sg} high.\textsc{dat.sg.wk}\\
  \glt `(the) high hall' (YCOEP; cobeowul,23.710.598; \\there are three more identical examples)
  
  \ex\gll beorh \textbf{þone} \textbf{hean}\\
  mountain.\textsc{acc.sg} \textsc{dem.acc.sg} high.\textsc{acc.sg.wk}\\
  \glt `(the) high mountain' (YCOEP; cobeowul,95.3093.249)
  
  \ex\gll mægðhad \textbf{se} \textbf{micla}\\
  maidenhood.\textsc{nom.sg} \textsc{dem.nom.sg} great.\textsc{nom.sg.wk}\\
  \glt `great maidenhood' (YCOEP; cochrist,5.82.56)
  
  \ex\gll wyrd \textbf{seo} \textbf{mære}\\
  fate.\textsc{nom.sg} \textsc{dem.nom.sg} great.\textsc{nom.sg.wk}\\
  \glt `great Fate' (YCOEP; coexeter,136.99.92)
  
  \ex\gll salwonges bearm \textbf{þone} \textbf{bradan}\\
  field.\textsc{gen.sg} bosom.\textsc{acc.sg} \textsc{dem.acc.sg} broad.\textsc{acc.sg.wk}\\
  \glt `the field's broad bosom' (YCOEP; coriddle,181.1.29)
 \end{xlist}
\end{exe}

However, there are also a handful of other \isi{postnominal} demonstratives without weak adjectives, including proximal demonstratives.\footnote{Concretely, there are fourteen such examples. A search for \textit{(NP* iDoms N|N\^{}*) AND (NP* iDoms D*) AND (N|N\^{}* precedes D*)} in the YCOEP yields 32 examples in total. 18 of these involve \textsc{dem} plus weak \isi{adjective}, the expected type. The other 14 include six examples of a distal \isi{determiner} alone, all of them \textit{þone} and all from Beowulf. Seven involve \isi{postnominal} proximal demonstratives alone from various texts, and there is one misannotation. A further search for \textit{(NP* iDoms NP*PRN*)  AND (NP*PRN* iDoms D*)} in the YCOEP yields a handful of other potentially relevant examples of \textsc{dem} plus weak \isi{adjective}.} Moreover, \isi{postnominal} weak adjectives in poetry do not need to be immediately preceded by \textsc{dem}, as examples like (\ref{ex:oe-postnom-weak}) show.

\begin{exe}
  \ex\gll\label{ex:oe-postnom-weak}se maga geonga\\
  \textsc{dem.nom.sg} kinsman.\textsc{nom.sg} young.\textsc{nom.sg.wk}\\
  \glt `the young kinsman' (YCOEP; cobeowul,83.2673.2189)
\end{exe}

The evidence from \isi{postnominal} ordering in OE provides further evidence for \isi{adjectival article} behaviour of \textsc{dem}, then, though occasional problematic examples are also found.

In OHG, the evidence is variable. Among the larger OHG texts, the \isi{postnominal} \textsc{dem} plus weak \isi{adjective} construction is only robustly attested in Otfrid's \emph{Evangelienbuch}.\footnote{The following query was used: \textit{posLemma = "DD" \& posLemma = "NA" \& posLemma = "ADJ" \& \#2.\#1 \& \#1.\#3}; examples were then filtered manually.} Examples are given in (\ref{ex:ohg-n-dem}).

\begin{exe}
 \ex\label{ex:ohg-n-dem}
 \begin{xlist}
  \ex\gll Múater \textbf{thiu} \textbf{gúata}\\
  mother.\textsc{nom.sg} \textsc{dem.nom.sg} good.\textsc{nom.sg.wk}\\
  \glt `(the) good mother' (ReA; O\_Otfr.Ev.1.15)
  
  \ex\gll kúningin \textbf{thia} \textbf{ríchun}\\
  queen.\textsc{nom.sg} \textsc{dem.nom.sg} rich.\textsc{nom.sg.wk}\\
  \glt `(the) rich queen' (ReA; O\_Otfr.Ev.1.3)
  
  \ex\label{ex:ohg-n-demc}\gll Gímma \textbf{thiu} \textbf{wíza}\\
  gem.\textsc{nom.sg} \textsc{dem.nom.sg} white.\textsc{nom.sg.wk}\\
  \glt `(the) white gem' (ReA; O\_Otfr.Ev.1.5)
  
  \ex\gll gótes drut \textbf{ther} \textbf{máro}\\
  God.\textsc{gen.sg} friend.\textsc{nom.sg} \textsc{dem.nom.sg} great.\textsc{nom.sg.wk}\\
  \glt `God's great friend' (ReA; O\_Otfr.Ev.2.7)
 \end{xlist}
\end{exe}

Caution is needed here, since, as with the examples from Otfrid in Section \ref{sec:9:poss-dem}, the \isi{adjective} very often participates in a rhyming couplet. However, there are numerous such examples, and it is unlikely that Otfrid is drawing on an ungrammatical construction, even if it was marginal outside poetic usage. Examples are also found in other, smaller OHG texts, as in (\ref{ex:ohg-n-dem-small}).

\begin{exe}
 \ex\label{ex:ohg-n-dem-small}
 \begin{xlist}
  \ex\gll uuiroh \textbf{daz} \textbf{rota}\\
  incense.\textsc{nom.sg} \textsc{dem.nom.sg} red.\textsc{nom.sg.wk}\\
  \glt `(the) red incense' (ReA; BR1\_BaslerRezept1)
  
  \ex\gll uuiroh \textbf{daz} \textbf{uuizza}\\
  incense.\textsc{nom.sg} \textsc{dem.nom.sg} white.\textsc{nom.sg.wk}\\
  \glt `(the) white incense' (ReA; BR1\_BaslerRezept1)
  
  \ex\gll engila \textbf{dê} \textbf{skônun}\\
  angels.\textsc{nom.pl} \textsc{dem.nom.pl} beautiful.\textsc{nom.pl.wk}\\
  \glt `(the) beautiful angels' (ReA; G\_Georgslied\_Tschirch)
 \end{xlist}
\end{exe}

Since these examples are found in early (8\textsuperscript{th}- and 9\textsuperscript{th}-century) texts written in different OHG scribal dialects, the most plausible hypothesis is that we are dealing with something that is a relic feature, if not synchronically fully productive.

\subsubsection{Vocatives}\label{sec:9:voc}

In prototypical DP languages, such as PDE and \ili{Italian}, vocatives are a context in which the DP layer may be absent, with vocatives surfacing as bare NPs (e.g. \citealp[626--627, note 20]{longobardi1994reference}). This stance receives support from the fact that, in \ili{English} and \ili{Italian}, both definite and \isi{indefinite} articles are ungrammatical in vocatives, cf. (\ref{ex:vocs}).

\begin{exe}     
 \ex\label{ex:vocs}
 \begin{xlist}
     \ex\gll ?*I ragazzi, venite qui!\\
     \phantom{?*}the boys come here\\
     \glt \phantom{?*}`Come here, (the) boys!' (\ili{Italian}; \citealp[626]{longobardi1994reference})
     
     \ex\gll *Un/Qualche ragazzo, vieni qui!\\
     \phantom{*}a/some boy come here\\
     \glt \phantom{*}`Come here, (a/some) boy!' (\ili{Italian}; \citealp[627]{longobardi1994reference})
     
     \ex *The boys, come here!
     
     \ex *A/some boy, come here!
     
     \ex \label{ex:vocse} *This/that boy, come here!
 \end{xlist}
\end{exe}     

In PDE, demonstratives are also excluded from vocatives: see (\ref{ex:vocse}).\footnote{\citet[note 20]{longobardi1994reference} observes that demonstratives are permitted in vocatives in ``literary \ili{Italian}''. Similarly, Cindy Allen (p.c.) points out that definite articles are possible -- if dispreferred -- in appositions to vocatives in Present-day \ili{English}, such as ``O Lord, the maker of all things". Due to the nature of our evidence it is not always possible to rule out appositive status, especially for \isi{postnominal} sequences of \textsc{dem} plus \isi{adjective}, and especially since it is difficult to define and delimit what apposition actually is. As a result, the diagnostic discussed in this section is perhaps not as strong as those laid out in the previous sections.} This suggests, in fact, the stronger hypothesis that DP \textit{must} be absent in vocatives -- though \citet[626--627, note 20]{longobardi1994reference} is cautious about this, noting that there are varieties in which at least definite articles seem to be acceptable in vocatives. He therefore suggests that the ungrammaticality of (certain) D elements in vocatives may be due to a semantic incompatibility. Under either theory, it is instructive to consider the predictions for adjectival articles. Under the \isi{adjectival article} theory, \textsc{dem} is not part of the DP layer of the nominal, and its function is as a pure categorizer. Thus, under both the semantic theory and the no-DP theory, the prediction is that \textsc{dem} \emph{qua} \isi{adjectival article} should be unproblematic in vocatives.

For OS and OE, this prediction is borne out. Starting with OS, the examples in (\ref{ex:os-voc}) illustrate.

\begin{exe}     
 \ex\label{ex:os-voc}
 \begin{xlist}
     \ex \label{ex:os-voca} \gll Herro \textbf{thie} \textbf{guodo}\\
     lord.\textsc{nom.sg} \textsc{dem.nom.sg} good.\textsc{nom.sg.wk}\\
     \glt `good lord' (OSax.811.792)
     
     \ex\gll fro min \textbf{thie} \textbf{guodo}\\
     lord.\textsc{nom.sg} my.\textsc{nom.sg.str} \textsc{dem.nom.sg} good.\textsc{nom.sg.wk}\\
     \glt `my good lord' (OSax.115.210 = (\ref{ex:fromin}) above)
 \end{xlist}
\end{exe}     

There are sixteen such examples in total in the OS portion of NPEGL.\footnote{\citet[160]{Allen2006} discusses (\ref{ex:os-voca}) based on its inclusion in \citet{Heinrichs54}, and suggests that its interpretation is appositional: `The Lord, the good one'. However, neither \citet{Heinrichs54} nor \citet{Allen2006} mentions that the example is \isi{vocative}, which makes this interpretation implausible.} In all cases, \textsc{dem} is formally distal and immediately followed by a weak \isi{adjective}, as in the other examples of putative adjectival articles provided so far.\footnote{The \isi{noun} varies, but the \isi{adjective} in all sixteen examples is `\textsc{dem} good', suggesting we may be dealing with a fossilized construction (though this inference is not on solid ground). Even if so, however, fossilized constructions by definition tell us something about a possibility that was once productive.} In all cases, \textsc{dem} is also \isi{postnominal}, which as argued in Section \ref{sec:9:postnom} is a strong indication of \isi{adjectival article} status in OS.

In OE, this construction is also very widespread: examples include (\ref{ex:oe-voca})--(\ref{ex:oe-voce}).

\begin{exe}     
 \ex\label{ex:oe-voc}
 \begin{xlist}
     \ex \label{ex:oe-voca} \gll Men \textbf{ða} \textbf{leofestan}\\
     men.\textsc{nom.pl} \textsc{dem.nom.pl} dear.\textsc{supl.nom.pl.wk}\\
     \glt `Dearest men' (OEng.586.608; Ælfric's Lives of Saints, Christmas Sermon)
     
     \ex\gll min \textbf{se} \textbf{leofeste} sune\\
     my.\textsc{nom.sg.str} \textsc{dem.nom.sg} dear.\textsc{supl.nom.sg.wk} son.\textsc{nom.sg}\\
     \glt `my dearest son' (OEng.708.922; Alcuin)
     
     \ex\gll min \textbf{se} \textbf{leofa} magister\\
     my.\textsc{nom.sg.str} \textsc{dem.nom.sg} dear.\textsc{nom.sg.wk} magister\\
     \glt `my dear magister' (OEng.640.906; Alexander's Letter)
     
     \ex\gll \textbf{þa} \textbf{leofestan} broðor\\
     \textsc{dem.nom.pl} dear.\textsc{supl.nom.pl.wk} brothers.\textsc{nom.pl}\\
     \glt `dearest brothers' (OEng.934.199; Bede)
     
     \ex \label{ex:oe-voce} \gll min \textbf{se} \textbf{halga} Petrus\\
     my.\textsc{nom.sg.str} \textsc{dem.nom.sg} holy.\textsc{nom.sg.wk} Peter.\textsc{nom.sg}\\
     \glt `my holy Peter' (OEng.496.724; Blickling Homilies)
 \end{xlist}
\end{exe}     

The example in (\ref{ex:oe-voca}) is a formula that is incredibly widespread, especially in sermons; according to \citet{Porck2020}, it is attested more than 200 times in OE homilies across a variety of manuscripts, and is the most common way for priests to begin their sermons. However, the construction is attested with a variety of adjectives and nouns. Both \isi{prenominal} and \isi{postnominal} sequences of \textsc{dem} plus weak \isi{adjective} are found. In OE poetry, we find eight additional examples, four of which are \isi{prenominal} and four of which are \isi{postnominal}.\footnote{The queries used for the YCOEP \citep{YCOEP} were \textit{NP*-VOC* iDoms D*} and \textit{(NP*-VOC* iDoms NP*PRN*) AND (NP*PRN* iDoms D*)}.}

Once more, all examples in both poetry and prose are formally distal, and all examples occur with an immediately following weak \isi{adjective}. This suggests that in OE, too, the \textsc{dem} element that shows up in vocatives is always an \isi{adjectival article} (rather than a definite \isi{article} or \isi{demonstrative}).

A full investigation of vocatives in OHG is a desideratum for future work: existing resources such as the \textit{Deutsch Diachron Digital} \isi{corpus} do not make it possible to extract expressions with \isi{vocative} function straightforwardly. Nevertheless, OHG also exhibits \textsc{dem} plus weak \isi{adjective} sequences in vocatives, e.g. (\ref{ex:ohg-n-demc}) in the previous section, repeated here as (\ref{ex:ohg-n-voc}).

\begin{exe}     
  \ex \label{ex:ohg-n-voc} \gll Gímma thiu wíza\\
  gem.\textsc{nom.sg} \textsc{dem.nom.sg} white.\textsc{nom.sg.wk}\\
  \glt `(the) white gem' (ReA; O\_Otfr.Ev.1.5)
\end{exe}

\subsection{Summary of evidence}

Table \ref{tab:wgmc} summarizes the different pieces of evidence presented so far for adjectival articles in the early West \ili{Germanic} languages.

\begin{table}
    \begin{tabular}{lccccc}
    \lsptoprule
        Language & \multicolumn{2}{c}{OE} & OS & \multicolumn{2}{c}{OHG}\\\cmidrule(lr){2-3}\cmidrule(lr){5-6}
                 & prose & poetry &            & Otfrid & other\\
        \midrule
        \textsc{poss dem} (Sect. \ref{sec:9:poss-dem}) & + & + & + & + & – \\
        Postnominal \textsc{dem} (Sect. \ref{sec:9:postnom}) & + & + & + & + & + \\
        Vocative \textsc{dem} (Sect. \ref{sec:9:voc}) & + & + & + & + & ??\\
        \lspbottomrule
    \end{tabular}
    \caption{Evidence for adjectival articles in early West Germanic}
    \label{tab:wgmc}
\end{table}

In each case, crucially, it is the formally distal \isi{demonstrative} that is found in these configurations, not the formally proximal \isi{demonstrative}, and in each case there is a close connection between the \textsc{dem} element and an immediately following weak \isi{adjective}. We take this to indicate that the early West \ili{Germanic} languages had an \isi{adjectival article} derived from the distal \isi{demonstrative}.\footnote{Once a regular definite \isi{article} has grammaticalized, one might expect the \isi{adjectival article} to co-occur with it, simply on the grounds that nothing rules this out: the two ``articles'' are not in complementary distribution with each other, syntactically or semantically. For that matter, we might expect to see an \isi{indefinite} \isi{article} co-occurring with an \isi{adjectival article}. There are a few scattered examples of this kind: see for instance (\ref{ex:berg}), from OS.

    \begin{exe}     
        \ex\gll\label{ex:berg}enon berage \textbf{them} \textbf{hohon}\\
         a.\textsc{dat.sg} mountain.\textsc{dat.sg} \textsc{dem.dat.sg} high.\textsc{dat.sg.wk}\\
         \glt `a high mountain' (OSax.406.580)
    \end{exe}     

\citet{Heinrichs54} also remarks upon this example. However, we have not been able to find any comparable examples of co-occurrence in any of the early West \ili{Germanic} texts we have looked at. We leave this mystery to future research.}

More tentatively, we posit that this is a shared inheritance from Proto-West \ili{Germanic} (at least). Strikingly, the evidence for adjectival articles is found in all three of the West \ili{Germanic} languages attested in the first millennium; however, it is not distributed equally across texts. The empirical picture we have so far seems to suggest that it is found in early texts, such as the plausibly 9\textsuperscript{th}-century OE C text of Gregory's Dialogues and the 9\textsuperscript{th}-century OHG \emph{Evangelienbuch} of Otfrid, more than it is found in later texts. This is consistent with an interpretation in which the \isi{adjectival article} is an inherited West \ili{Germanic} feature that becomes archaic and dies out in the individual histories of the West \ili{Germanic} languages.





\section{An adjectival article in Gothic}\label{sec:9:gothic}

\ili{Gothic}, the \ili{Germanic} language with the earliest substantial textual attestation, presents well-known problems when trying to draw inferences about its \isi{syntax}: the main text that we have at our disposal is a partial Bible translation, mostly of the New Testament, which remains very close to its \ili{Greek} original (see \citealp[21--39]{ratkus2011}; \citealp[11--13]{Walkden2014book}; \citealp[8--20]{Miller2019} and references cited there).\footnote{The other major \ili{Gothic} text, the \emph{Skeireins}, is probably also a translation from \ili{Greek} (\citealt{Bennett1960}; \citealt{Schaeferdiek1981}).} For any syntactic feature observed in the \ili{Gothic} Bible, the challenge to the analyst is therefore to establish whether it is truly a feature of \ili{Gothic} or rather reflects a \ili{Greek} original. Beyond this, moderately extensive \ili{Latin} influence is also found in \ili{Gothic} (see \citealp[chapter 5]{Falluomini2015} and references cited there).

When the \ili{Gothic} and the \ili{Greek} original are in \isi{agreement}, any conclusion about the syntactic properties of \ili{Gothic} must be viewed with some scepticism. This is true, for instance, for any statement about null subjects: it can be shown that whether the subject in \ili{Greek} is overt or null is by far the best predictor of whether the subject in \ili{Gothic} is overt or null (\citealp{Fertig2000}; \citealp{Ferraresi2005}; \citealp[chapter 5]{Walkden2014book}). In the case of \isi{article} use, however, \ili{Gothic} on the whole does not follow \ili{Greek} usage. New Testament \ili{Greek}, like Modern \ili{Greek}, exhibits polydefiniteness (see \citealp{Ramaglia2008} and \citealp{Leu2007greek}), as in (\ref{ex:Greek-holy-spirit}).

\begin{exe}     
    \ex\gll\label{ex:Greek-holy-spirit}hupo [\textbf{tou} pneumatos \textbf{tou} hagiou]\\
    by the.\textsc{n.gen.sg} spirit.\textsc{n.gen.sg} the.\textsc{n.gen.sg} holy.\textsc{n.gen.sg}\\
    \glt `by the holy Spirit'  (Luke 2:26; \citealp[139]{ratkus2011})
\end{exe}

When rendering polydefinite constructions, the translator(s) of the \ili{Gothic} Bible did not translate every \ili{Greek} \isi{article} using a distal \isi{demonstrative}. Instead, ``the translator, faced with the choice of eliminating one of the two determiners, chooses to delete the one before the \isi{noun} while keeping the one preceding the \isi{adjective}'' \citep[140]{ratkus2011}, as in (\ref{ex:Gothic-holy-spirit}).\footnote{There are a handful of examples where both articles are rendered in \ili{Gothic}, e.g. Mark 1:27. These, however, are comparatively so rare that, in light of the fact that such examples follow the structure of the \ili{Greek}, \citet[140]{ratkus2011} goes so far as to call this structure ungrammatical in \ili{Gothic}.}

\begin{exe}     
    \ex\gll\label{ex:Gothic-holy-spirit}fram [ahmin \textbf{þamma} weihin]\\
    from spirit.\textsc{m.dat.sg} that.\textsc{m.dat.sg} holy.\textsc{m.dat.sg}\\
    \glt `by the holy Spirit'  (Luke 2:26; \citealp[139]{ratkus2011})
\end{exe}

In general, where the \ili{Gothic} systematically deviates from the \ili{Greek}, it is plausible that what is found in \ili{Gothic} is a genuinely autochthonous construction. This is Ratkus's conclusion for the rendering of polydefiniteness. It is then striking that the single \isi{demonstrative} form that is translated overtly is not the one adjacent to the \isi{noun}, but the one adjacent to the \isi{adjective}. Of 151 examples containing a \isi{demonstrative}, a weak \isi{adjective} and a \isi{noun}, 100 have the order \textsc{dem} Adj.\textsc{wk} N, 47 have the order N \textsc{dem} Adj.\textsc{wk}, and only four have the order \textsc{dem} N Adj.\textsc{wk} \citep[141]{ratkus2011}. In 97\% of examples, then, the \isi{demonstrative} immediately precedes the weak \isi{adjective}. Ratkus concludes that ``[f]rom a philological point of view, the \isi{definite determiner} and the \isi{adjective} can perhaps be seen to form an indivisible unit'' \citeyearpar[141]{ratkus2011}, noting that this has implications for the reconstruction of \ili{Germanic} nominal \isi{syntax}. Further examples are given in (\ref{ex:Gothic-adjart}).

\begin{exe}     
 \ex\label{ex:Gothic-adjart}
 \begin{xlist}
    \ex\gll hairdeis \textbf{sa} goda\\
    shepherd.\textsc{m.nom.sg} \textsc{dem.m.nom.sg} good.\textsc{m.nom.sg.wk}\\
    \glt `the good shepherd'\\ %~\hfill 
    (John 10:11; \ili{Greek}: \emph{\textbf{ho} poimēn \textbf{ho} kalos} `the shepherd the good')

    \ex\gll in fon \textbf{þata} unhvapnando\\
    into fire.\textsc{n.acc.sg} \textsc{dem.n.acc.sg} unquenchable.\textsc{n.acc.sg.wk}\\
    \glt `into the fire that shall never be quenched'\\
    %~\hfill 
    (Mark 9:43; \ili{Greek}: \emph{eis \textbf{to} pur \textbf{to} asbeston} `into the fire the unquenchable')
    \end{xlist}
\end{exe}

\citet[chapter 5]{ratkus2011} goes on to develop an account in which an `artroid' element, historically derived from the \isi{demonstrative}, precedes weak adjectives in \ili{Gothic}.\footnote{The term ``artroid'' is taken from the work of Albertas Rosinas on the Baltic languages; see \citet[85--93]{Rosinas2009} and references cited there.} Ratkus's artroid is a {``}`fake' \isi{determiner}'', distinct in function from either a prototypical \isi{demonstrative} or a prototypical \isi{article}. This notion of artroid -- in the context of its co-occurrence specifically with weak adjectives -- is effectively the same as the notion of \isi{adjectival article} that has been laid out in detail in \citet{Pfaff2019} and in this chapter.

The \ili{Gothic} data pose additional challenges in that weak adjectives need not be accompanied by an artroid/\isi{adjectival article}, and occur without it in non-trivial numbers: \citet[141]{ratkus2011} counts 63 weak adjectives without a preceding \textsc{dem}. This is unlike the situation in the other early \ili{Germanic} languages, where it is normal for the two elements to occur together, as discussed in Section \ref{sec:9:adjart-eWGmc}. \ili{Gothic} weak adjectives also need not be definite: see in particular \citet{ratkus2018weak}. However, that an \isi{adjectival article} existed in \ili{Gothic} -- even if its use was not quite obligatory -- seems to be a safe conclusion.

\section{Summary and conclusion}\label{sec:9:conclusion}

We have shown in this paper that all five of the substantially attested early \ili{Germanic} languages -- \ili{Old Icelandic}, Old \ili{English}, Old High \ili{German}, \ili{Old Saxon}, and \ili{Gothic} -- display evidence for an \isi{adjectival article}. In all five languages this element is grammaticalized from a \isi{demonstrative}, forms a constituent with the weak \isi{adjective}, and does not serve to mark \isi{definiteness}.

From the perspective of \isi{comparative} reconstruction, the obvious next step is to project these properties back to Proto-\ili{Germanic} itself.\footnote{This is also the stance taken by \citet[249--250]{ratkus2011} with respect to his ``artroid'' element; for Ratkus, the emergence of the artroid in fact precedes the emergence of the strong-weak \isi{adjective} distinction in \ili{Germanic}.} The major difference between North \ili{Germanic} and the other branches, of course, is that the \isi{adjectival article} \emph{(h)inn} in North \ili{Germanic} is a reflex of Proto-\ili{Germanic} *jainaz, whereas in East and West \ili{Germanic} it is a reflex of Proto-\ili{Germanic} *sa (and its paradigm). This need not be fatal for a reconstruction of the \isi{adjectival article} as a common Proto-\ili{Germanic} feature, however. Rather, we are plausibly dealing with a single functional element whose morphophonological realization varies and changes within the \ili{Germanic} family. In support of this, we know from the attested histories of the North \ili{Germanic} languages that reflexes of *jainaz and *sa are in competition for other linking functions too (\citealt{Ulla2009,Ulla2015,Ulla2020}; \citealt{Pfaff2019}), with the latter also appearing variably in a \isi{relative clause} context (\citealt{wagenerRC}; \citealt{sapp2019RC}). Thus we can reconstruct the underlying \isi{syntax} of an \isi{adjectival article} without committing ourselves to a particular morphophonological form. More needs to be said about the precise diachronic developments involved, of course, but this shared behaviour observed across all branches of \ili{Germanic} makes continuity a more appealing scenario than independent innovation, on grounds of parsimony.


\section*{Abbreviations}

\begin{multicols}{2}
\begin{tabbing}
MMM \= accusative\kill
\textsc{acc} \> accusative \\
\textsc{art} \> {adjectival article} (Old Icelandic) \\
\textsc{dat} \> {dative} \\
\textsc{def} \> suffixed definite {article}  \\
\textsc{dem} \> {demonstrative} \\
\textsc{det} \> {determiner} \\
\textsc{gen} \> {genitive} \\
IL \> individual level \\
\textsc{ins} \> {instrumental} \\
\textsc{m} \> masculine \\
\textsc{n} \> neuter \\
\textsc{nom} \> {nominative} \\
OE      \> Old English \\ 
OHG     \> Old High German \\ 
OS      \>  Old Saxon \\ 
PDE     \> Present-day English \\ 
\textsc{pl} \> plural \\
\textsc{poss} \> possessive \\
\textsc{refl} \> reflexive \\
\textsc{sg} \> singular \\
SL \> stage level \\
\textsc{str} \> strong {adjectival inflection}  \\
\textsc{supl} \> {superlative} \\
\textsc{wk} \> weak {adjectival inflection}
\end{tabbing}
\end{multicols}

\section*{Acknowledgements}

We would like to thank members of the project \textit{Constraints on syntactic \isi{variation}: \isi{Noun} phrases in early \ili{Germanic} languages} for their useful feedback on this research at early stages of its development, and to two  reviewers for their comments on the first draft of this chapter, which led to substantial improvements. We would also like to thank audiences at the Jena-Göttingen \ili{Germanic} Linguistics Colloquium (November 2021) and the DGfS Annual Meeting (February 2022), where this work was presented. We gratefully acknowledge the financial support of the Research Council of Norway (RCN), grant number 261847. In addition, George Walkden acknowledges the support of the \ili{German} Research Council (DFG), grant number 429663384, \textit{\ili{Germanic} dispersion beyond trees and waves}.

%\citet{Nordhoff2018} is useful for compiling bibliographies.
%\section*{Contributions}
%John Doe contributed to conceptualization, methodology, and validation.
%Jane Doe contributed to the writing of the original draft, review, and editing.

{\sloppy\printbibliography[heading=subbibliography,notkeyword=this]}
\end{document}
