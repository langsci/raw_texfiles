\documentclass[output=paper,colorlinks,citecolor=brown]{langscibook}
\ChapterDOI{10.5281/zenodo.10641189}
\author{Kristin Bech\orcid{}\affiliation{University of Oslo}}
%\ORCIDs{}

\title[The Old English quantifiers \textnormal{fela} ‘many’ and \textnormal{manig} ‘many’]{The Old English quantifiers \textit{fela} ‘many’ and \textit{manig} ‘many’, and Ælfric as a linguistic innovator}

\abstract{This chapter explores the Old English quantifiers \textit{fela} and \textit{manig}, both meaning ‘many',  with special focus on \textit{fela}. It is shown that the works of Ælfric and the \textit{Peterborough Chronicle}, both from the late Old English period, stand out with respect to which constructions \textit{fela} enters into. In those texts, \textit{fela} can occur in agreement constructions or with a partitive genitive, whereas in the rest of the Old English texts, the genitive is used consistently. Thus, \textit{fela} shows clear signs of moving from being the head of the noun phrase, taking a genitive complement, to becoming a modifier of a head noun. \textit{Manig}, on the other hand, has always been a modifier of a nominal head. I show that the variation in the use of \textit{fela} in Ælfric’s texts and the chronicle is determined by semantic factors, and that the trajectory of change is what we would expect for a word of this kind. As the construction with \textit{fela} changed, it was in many cases no longer structurally distinguishable from constructions with \textit{manig}. In addition, as inflections were levelled, the genitive plural case marking was no longer there to support \textit{fela}. Hence, \textit{fela} lost the lexical competition, since \textit{manig} in any case was the most frequent quantifier meaning ‘many’, and did not have to undergo any radical structural changes.}


\IfFileExists{../localcommands.tex}{
   \addbibresource{../localbibliography.bib}
   \usepackage{langsci-optional}
\usepackage{langsci-gb4e}
\usepackage{langsci-lgr}

\usepackage{listings}
\lstset{basicstyle=\ttfamily,tabsize=2,breaklines=true}

%added by author
% \usepackage{tipa}
\usepackage{multirow}
\graphicspath{{figures/}}
\usepackage{langsci-branding}

   
\newcommand{\sent}{\enumsentence}
\newcommand{\sents}{\eenumsentence}
\let\citeasnoun\citet

\renewcommand{\lsCoverTitleFont}[1]{\sffamily\addfontfeatures{Scale=MatchUppercase}\fontsize{44pt}{16mm}\selectfont #1}
  
   %% hyphenation points for line breaks
%% Normally, automatic hyphenation in LaTeX is very good
%% If a word is mis-hyphenated, add it to this file
%%
%% add information to TeX file before \begin{document} with:
%% %% hyphenation points for line breaks
%% Normally, automatic hyphenation in LaTeX is very good
%% If a word is mis-hyphenated, add it to this file
%%
%% add information to TeX file before \begin{document} with:
%% %% hyphenation points for line breaks
%% Normally, automatic hyphenation in LaTeX is very good
%% If a word is mis-hyphenated, add it to this file
%%
%% add information to TeX file before \begin{document} with:
%% \include{localhyphenation}
\hyphenation{
affri-ca-te
affri-ca-tes
an-no-tated
com-ple-ments
com-po-si-tio-na-li-ty
non-com-po-si-tio-na-li-ty
Gon-zá-lez
out-side
Ri-chárd
se-man-tics
STREU-SLE
Tie-de-mann
}
\hyphenation{
affri-ca-te
affri-ca-tes
an-no-tated
com-ple-ments
com-po-si-tio-na-li-ty
non-com-po-si-tio-na-li-ty
Gon-zá-lez
out-side
Ri-chárd
se-man-tics
STREU-SLE
Tie-de-mann
}
\hyphenation{
affri-ca-te
affri-ca-tes
an-no-tated
com-ple-ments
com-po-si-tio-na-li-ty
non-com-po-si-tio-na-li-ty
Gon-zá-lez
out-side
Ri-chárd
se-man-tics
STREU-SLE
Tie-de-mann
}
   \boolfalse{bookcompile}
   \togglepaper[4]%%chapternumber
}{}

\begin{document}
\maketitle

\section{Introduction}

\citet[vol. I, 172--174]{Mitchell85} groups \textit{fela} and \textit{manig} under ``indefinites", and lists three different uses: i) dependent (attributively in \isi{agreement} constructions) (\ref{felaattr}); ii) independent with a \isi{partitive} \isi{genitive} (\ref{felapart}), or iii) alone (\ref{felaalone}).\footnote{I only gloss according to what is necessary for the purposes of this study. Hence, I gloss the \isi{noun} phrases for case and number, and in some of the longer sentences presented later in this study, I provide some glosses that are necessary in order to understand the structure of the sentence.} \textit{Fela} is indeclinable and cannot be compared, but it can be accompanied by an intensifying adverb.\footnote{The \citet{DOE} lists three exceptions, where \textit{fela} in fact is declined.} Note that when \textit{fela} stands alone, it often means ‘much’ rather than ‘many’, as in (\ref{felaalone}).

\ea\label{felaattr}
\gll fela þing\\
	many thing.\ACC.\PL{}\\
\glt ‘many things’ (OEng.562.416; ÆHS)
\z

\ea\label{felapart}
\gll fela oðra wundra \\
	many other.\GEN.\PL{} wonder.\GEN.\PL{}\\
\glt‘many (of) other wonders’ (OEng.507.515; ÆHS)
\z

\ea\label{felaalone}
\gll Fela ic hæbbe eow to secganne\\
	much I have you to say\\
\glt ‘I have much to say to you.’ (OEng.834.824; Cura)
\z

 In \isi{contrast} to \textit{fela}, \textit{manig} can be declined weak or strong. If an \isi{adjective} occurs between \textit{manig} and a \isi{noun} head, the \isi{adjective} is declined strong unless a possessive or \isi{demonstrative} intervenes \citep[vol. I, 61]{Mitchell85}. It is unclear whether \textit{manig} can be compared; \citet[vol. I, 174 fn. 112]{Mitchell85} seems to suggest that \textit{mæstra} may be a \isi{superlative} of \textit{manig} rather than of \textit{micel}. Like \textit{fela}, \textit{manig} can be used dependently (\ref{manigdep}), independently with a \isi{partitive} \isi{genitive} (\ref{manigindep}) or alone (\ref{manigalone}).


\ea\label{manigdep}
\gll wel monige godcunde lareowas\\
	well many.\ACC.\PL{} religious.\ACC.\PL{} teacher.\ACC.\PL{}\\
\glt ‘very many religious teachers’ (OEng.970.662; ASC(A))
\z

\ea\label{manigindep}
\gll hiora monigne\\
	them.\GEN{} many.\ACC.\SG{}\\ 
\glt ‘many of them’ (OEng.908.724; Bede)
\z

\ea\label{manigalone}
\gll \& eac monige cwomon to bicgenne þa þing\\
	 and also many.\NOM.\PL{} came to buy \DEF{} things\\
\glt ‘and many also came to buy the things’	(OEng.376.864; Bede)
\z

Apart from what is found in \citet{Mitchell85}, it seems that very little has been written about \textit{fela}. 
 \citet{RoehrsSapp18} deal with complex quantifiers, so they specifically do not consider \textit{fela} \citeyearpar[389]{RoehrsSapp18}, but they mention in a footnote that \textit{fela} is probably a head-type rather than a \isi{phrase-type quantifier} \citeyearpar[389, fn. 6]{RoehrsSapp18}. Wright’s \citeyearpar{WrightWright25} and Campbell’s \citeyearpar{Campbell62} Old \ili{English} grammars mostly provide phonological information about \textit{fela}. \textit{Fela} has not been deemed worthy of attention in the Old \ili{English} \citep{Hogg92} or Middle \ili{English} \citep{Blake92} volume of the \textit{Cambridge history of the \ili{English} language} either. The time has therefore come to give \textit{fela} its fifteen minutes of fame.  

Section \ref{sect:matmeth} provides information about the texts and corpora used. Section \ref{sect:results} presents the results with respect to which constructions \textit{fela} and \textit{manig} enter into. Section \ref{sect:disc} contains the discussion, focusing on \textit{fela} in Ælfric's texts and the \textit{Peterborough Chronicle}. The latter is of interest because it shows the transition from Old to Middle \ili{English}. Reference is especially made to \citet{RoehrsSapp18} on complex quantifiers, as it is highly relevant for the present study. Section \ref{sect:concl} concludes the study.

\section{Material and method}\label{sect:matmeth}

\begin{sloppypar}
For the purposes of this study, I used both the \textit{Noun Phrases in Early \ili{Germanic} Languages} \isi{database} (NPEGL, see \citetv{chapters/1Database}) and the \textit{York--Toronto--Helsinki Corpus of Old \ili{English} Prose} \citep[YCOE,][]{YCOE}. I first searched for \textit{fela}, with the spelling variants \textit{fela}, \textit{feola}, \textit{feala}, and \textit{fæla},\footnote{The \citet{DOE} lists a few other spelling variants, and I searched for those as well, but they either occur in poetry, or in texts that are not considered in this study, so I do not list them here.}  including with capital letters, in the NPEGL \isi{database},\footnote{Unless otherwise marked, all the examples are from the NPEGL \isi{database}, and can be found by entering the unique ID provided, in the format Language.number.number.} and I did the same in the YCOE \isi{corpus}. NPEGL is a \isi{noun} phrase \isi{database} that has been created on the basis of the material in the YCOE \isi{corpus}. The advantage of using both is that the NPEGL \isi{database} provides the textual context for the examples, while the YCOE provides a syntactic analysis. I extracted all examples of \textit{fela} from all the Old \ili{English} texts that contain ten or more instances of \textit{fela}, see Table \ref{texts}.\footnote{The YCOE corpus contains three versions of the \textit{Anglo-Saxon Chronicle}, in addition to the \textit{Peterborough Chronicle}. I included two of them, i.e. the text of the A manuscript (Cambridge, Corpus Christi College, 173), which is the oldest of the extant manuscripts, and the D manuscript (British Library, Cotton Tiberius B. iv), which contains a relatively high number of \textit{fela}. The C manuscript is also included in the YCOE corpus and has more than ten instances of \textit{fela}, but it was not included here, as I did not want too much data from what is essentially the same text. Likewise, there are two versions of \textit{Gregory’s Dialogues} in the YCOE. I included the H manuscript (Oxford, Bodleian, Hatton 76), which, though having fewer words, contains more instances of \textit{fela} than the C manuscript (Cambridge, Corpus Christi College, 322).} I then wrote all the examples down manually in a file, in order to sort them into the categories presented in Table \ref{tab:distrfela}, and to gain an understanding of the usage through studying each example in context.\end{sloppypar}


\begin{table}
\caption{The texts used in the study. The Old English texts are those that contain ten or more instances of \textit{fela}}\label{texts}
\label{tabCorpora4}
\begin{tabular}{lllr}
\lsptoprule
                & Corpus   & Abbre-  & No. of \\
  Text          & filename & viation & words\\\midrule                                                                                
  \textit{The Anglo-Saxon Chronicle} (A ms) & cochronA & \textit{ASC(A)} &          14\,583 \\ 
  \textit{The Anglo-Saxon Chronicle} (D ms)	& cochronD & \textit{ASC(D)} &          26\,691 \\ 
  \textit{Bede’s Ecclesiastical History} & cobede & \textit{Bede} &                 80\,767 \\ 
  \textit{Cura Pastoralis}	& cocura & \textit{Cura} &                              68\,556 \\ 
  \textit{Orosius}	& coorosiu & \textit{Oros} &                                    51\,020 \\ 
  \textit{Bald's Leechbook}	& colaece & \textit{Leech} &                            34\,727 \\ 
  \textit{Vercelli Homilies} & coverhom & \textit{Verc} &                           45\,674 \\ 
  \textit{Gregory's Dialogues} & cogregdH & \textit{Greg} &                         25\,593 \\ 
  \textit{The Gospel of Nichodemus} & conicodA & \textit{Nich} &                    8\,197  \\ 
  \textit{Heptateuch} (Old Testament) & cootest & \textit{Hept} &                   59\,524 \\ 
  \textit{The West-Saxon Gospels} & cowsgosp & \textit{WSG} &                       71\,104 \\ 
  \textit{The Homilies of Wulfstan} & cowulf & \textit{Wulf} &                      28\,768 \\ 
  \textit{Ælfric’s Lives of Saints} & coaelive & \textit{ÆLS} &                     100\,193\\ 
  \textit{Ælfric’s Catholic Homilies 1} & cocathom1 & \textit{ÆCH1} &               106\,173\\ 
  \textit{Ælfric’s Catholic Homilies 2} & cocathom2 & \textit{ÆCH2} &               98\,583 \\ 
  \textit{Ælfric’s Homilies Supplemental} & coaelhom & \textit{ÆHS} &               62\,669 \\ 
  \textit{Ælfric’s Letter to Sigeweard} & colsigewZ & \textit{Sigew} &              10\,420 \\ 
  \textit{Old \ili{English} Peterborough Chronicle}  & cochronE & \textit{OE Pet} & 40\,641 \\ 
  \textit{Middle \ili{English} Peterborough Chronicle} & cmpeterb & \textit{ME Pet} & 7\,333  \\ 
\lspbottomrule
\end{tabular}
\end{table}


As regards \textit{manig}, I limited the extraction to seven texts from Old \ili{English} (see Table \ref{tab:distrmanig}), and the spellings were \textit{manig}, \textit{monig}, \textit{mænig}, \textit{maneg}, \textit{moneg}, and \textit{mæneg}, including with capital letters and all possible case forms. In this search, I only used the NPEGL \isi{database}, as it gives easy access to all forms through the query interface. The reason why I did not analyze \textit{manig} for all the texts is that it soon became apparent that it is very consistent in usage throughout (see Table \ref{tab:distrmanig}). The possible added value in analyzing all the instances of \textit{manig} for all the texts would therefore be disproportional to the work involved.  

I have only used one text from the early Middle \ili{English} period, namely the first and second continuations of the \textit{Peterborough Chronicle}.\footnote{The First Continuation covers the years 1122–1132, and the Second Continuation the years 1132–1154. See Section \ref{ssect:peterb} for further information about the \textit{Peterborough Chronicle}.}  I searched for the word forms \textit{fela}, \textit{fele}, \textit{feola}, \textit{feole}, \textit{feala}, \textit{feale} and the forms \textit{manig}, \textit{mani}, \textit{manie}, \textit{monig}, \textit{moni}, \textit{monie}, \textit{mænig}, \textit{mæni}, \textit{mænie}, \textit{mane}, \textit{manege} in the \textit{Penn--Helsinki Parsed Corpus of Middle English} \citep[PPCME2,][]{PPCME2}. The aim was to trace the development of the use of \textit{fela} within the chronicle. \textit{The Peterborough Chronicle} will be discussed in Section \ref{ssect:peterb}.



\section{Results}\label{sect:results}

\subsection{Results for \textit{fela} in the individual texts}\label{sect:felatexts}
In Table \ref{tab:distrfela}, I distinguish between the following constructions: \textit{fela} with \isi{agreement}, \textit{fela} with \isi{genitive}, \textit{fela} standing alone, \textit{fela} in constructions with numerals, and a ``miscellaneous" category for occurrences that could not be placed in the previous categories. For the sake of consistency in the table, I have given raw numbers and percentages for each text, but keep in mind that percentages do not say much if the total number of occurrences in a text is low. 



Examples (\ref{felaagr1})--(\ref{felaagr3}) show \textit{fela} with a following \isi{noun} that is not in the \isi{genitive} case. For lack of a better term, I call this type ``\textit{fela} with \isi{agreement}", even though \textit{fela} is indeclinable. 

\ea\label{felaagr1}
\gll fela untrume men\\
many sick.\NOM.\PL{} man.\NOM.\PL\\
\glt ‘many sick men’ (OEng.663.860; ÆLS)
\z

\ea\label{felaagr2}
\gll fela wytegan \& ryhtwise men\\
many wiseman.\NOM.\PL{} and righteous man.\NOM.\PL\\
\glt ‘many wise men and righteous men’ (OEng.278.158; ÆCH1)
\z

\ea\label{felaagr3}
\gll fela wintrum\\
many winter.\DAT.\PL\\
\glt ‘many winters’ (OEng.807.991; OE Pet)
\z

Examples (\ref{felagen1})--(\ref{felagen3}) show \textit{fela} followed by a \isi{noun} in the \isi{genitive} case, a \isi{partitive} \isi{genitive}. 

\ea\label{felagen1}
\gll fela wundra\\
	many wonder.\GEN.\PL\\
\glt ‘many wonders’	 (OEng.254.309; ÆCH2)
\z

\ea\label{felagen2}
\gll fela geara\\
	many year.\GEN.\PL\\
\glt ‘many years’ (NPEGL, OEng.275.716; Bede)
\z

\ea\label{felagen3}
\gll fela manna\\
	many man.\GEN.\PL\\
\glt ‘many men’ (OEng.677.479; Greg)
\z

\begin{table}
\caption{The distribution of \textit{fela}}\label{tab:distrfela}
 \begin{tabularx}{\textwidth}{l@{~}r Y@{~~}r Y@{~~}r Y@{~~}r Y@{~~}r Y@{~~}r }
  \lsptoprule
  & & \multicolumn{2}{c}{{Agreement}} & \multicolumn{2}{c}{{Genitive}} & \multicolumn{2}{c}{{Numeral}} & \multicolumn{2}{c}{{Alone}} & \multicolumn{2}{c}{{Misc.}}\\
  \cmidrule(lr){3-4}\cmidrule(lr){5-6}\cmidrule(lr){7-8}\cmidrule(lr){9-10}\cmidrule(lr){11-12}
  Texts & {\textit{fela} total} & $n$ & \% & $n$ & \% & $n$ & \% & $n$ & \% & $n$ & \% \\
  \midrule
  \textit{ASC(A)} & 12 & 1 & 8.3 & 8 & 66.7 & 2 & 16.7 & 1 & 8.3 & 0 & 0\\
\textit{ASC(D)} & 37 & 4 & 10.8 & 19 & 51.4 & 7 & 18.9 & 6 & 16.2 & 1 & 2.7\\
\textit{Bede}	& 13	& 1	& 7.7	& 10	& 76.9	& 0	 & 0	& 2	& 15.4	& 0	& 0\\
\textit{Cura}	& 19	& 0	& 0	& 12	& 63.2	& 0	& 0	& 7	& 36.8	& 0	& 0\\
\textit{Oros} & 46	& 2	& 4.4	& 35	& 76.1	& 6	& 13.0	& 1	& 2.2	& 2	& 4.4\\
\textit{Leech}	& 14	& 1	& 7.1	& 9	& 64.3	& 0	& 0	& 3	& 21.4	& 1	& 7.1\\
\textit{Verc}	& 13	& 0	& 0	& 10	& 76.9	& 0	& 0	& 3	& 23.1	& 0	& 0\\
\textit{Greg} & 21	& 3	& 14.3	& 16	& 76.2	& 0	& 0	& 1	& 4.8	& 1	& 4.8\\
\textit{Nich}	& 11	& 1	& 9.1	& 6	& 54.6	& 0	& 0	& 3	& 27.3	& 1	& 9.1\\
\textit{Hept}	& 18	& 0	& 0	& 12	& 66.7	& 1	& 5.6	& 5	& 27.8	& 0	& 0\\
\textit{WSG}	& 34	& 0	& 0	& 20	& 58.8	& 0	& 0	& 14	& 41.2	& 0	& 0\\
\textit{Wulf}	& 70	& 3	& 4.3	& 53	& 75.7	& 1	& 1.4	& 12	& 17.1	& 1	& 1.4\\
\textit{ÆLS}	& 110	& 42	& 38.2	& 44	& 40.0	& 2	& 1.8	& 15	& 13.6	& 7	& 6.4\\
\textit{ÆCH1}	& 63	& 13	& 20.6	& 22	& 34.9	& 4	& 6.4	& 16	& 25.4	& 8	& 12.7\\
\textit{ÆCH2}& 81	& 32	& 39.5	& 26	& 32.1	& 3	& 3.7	& 13	& 16.1	& 7	& 8.6\\
\textit{ÆHS}	& 47	& 10	& 21.3	& 24	& 51.1	& 1	& 2.1	& 9	& 19.1	& 3	& 6.4\\
\textit{Sigew}	& 12	& 2	& 16.7	& 4	& 33.3	& 4	& 33.3	& 1	& 8.3	& 1	& 8.3\\
\textit{OE Pet}	& 51	& 16	& 31.3	& 15	& 29.4	& 12	& 23.5	& 2	& 3.9	& 6	& 11.8\\
\textit{ME Pet}	& 13	& 10	& 76.9	& 0	& 0	& 0	& 0	& 0	& 0	& 3	& 23.1\\
 \lspbottomrule
 \end{tabularx}
\end{table}



The \isi{genitive} category also includes those instances in which the \isi{noun} is definite and preceded by a \isi{demonstrative}, as in (\ref{partdem1}) and (\ref{partdem2}), a few instances of possessives (\ref{partposs}), and some examples of pronouns, in which case the \isi{pronoun} often precedes \textit{fela} (\ref{partpron}). \citet[386--388]{RoehrsSapp18} call the quantified constituents in (\ref{felaagr1})–(\ref{felagen3}) “non-DP(-size) dependents", and the ones in (\ref{partdem1})--(\ref{partpron}) “DP(-size) dependents" (see Section \ref{ssect:RoehrsSapp}). DP dependents are always in the \isi{genitive} case, and they will therefore be disregarded from Table \ref{tab:felaagrpart} onward, since I want to focus on the possible \isi{variation} here. There are 54 such instances in total, and many of them occur in \textit{Orosius} and in Wulfstan's homilies.\footnote{Note that the -\textit{um} ending in \textit{senatorum} in example (\ref{partdem2}) is the \ili{Latin} \isi{genitive} plural \isi{inflection}. \ili{Latin} words sometimes kept their \ili{Latin} endings.}

\ea\label{partdem1}
\gll fela þære hæðenra\\
	many \DEF.\GEN.\PL{} heathen.\GEN.\PL\\
\glt ‘many of the heathens’ (OEng.411.534; ÆLS)
\z

\ea\label{partdem2}
\gll fela þara senatorum\\
many \DEF.\GEN.\PL{} senator.\GEN.\PL\\
\glt ‘many of the senators (OEng.394.441; Oros)
\z

\ea\label{partposs}
\gll feola his gersuma\\
many his treasure.\GEN.\PL\\
\glt ‘many of his treasures’ (OEng.569.541; OE Pet)
\z

\ea\label{partpron}
\gll and heora feala þær adruncon\\
and them.\GEN.\PL{} many there drowned\\
\glt ‘and many of them drowned there’ (OEng.490.467; OE Pet)
\z

I kept the numerals in a separate category, although these are also \isi{partitive} genitives. The reason for keeping them apart is that sometimes the \isi{numeral} itself is in the \isi{genitive} case, as in (\ref{numgen}), while sometimes it is the complement of the \isi{numeral} that is in the \isi{genitive} (\ref{numcompgen}). I did not want the numerals, which might also be idiomatic expressions, to interfere with the data, since I was interested in the possible choice between \isi{agreement} constructions and genitives.

\ea\label{numgen}
\gll fela þusenda\\
	many thousand.\GEN.\PL\\
\glt ‘many thousand’ (OEng.147.776; Sigew)
 \z

\ea\label{numcompgen}
\gll fela hund wintra\\
	many hundred winter.\GEN.\PL\\
\glt ‘many hundred winters’ (OEng.533.562; Wulf)
\z


\textit{Fela} may occur on its own, as in (\ref{felaalone2}) and (\ref{felaalone3}). As shown in (\ref{felaalone}), \textit{fela} can also mean ‘much’, and this is especially the case when \textit{fela} stands alone. In other words, \textit{fela} can sometimes be singular in meaning. 

\ea\label{felaalone2}
\gll Fela sind gelaðode and feawa gecorene\\	
	many are invited and few chosen\\
\glt ‘Many are invited and few are chosen.’ (OEng.021.630; ÆCH2)
\z

\ea\label{felaalone3}
\gll and fela þær wurdon ofslægen\\
	and many there became killed\\
\glt ‘and many were killed there’ (OEng.037.151; ASC(D))
\z

A few occurrences of \textit{fela} did not fit into the previous categories, so I collected them into a “miscellaneous" category, cf. Table \ref{tab:distrfela}. Some examples are given in (\ref{misc1})--(\ref{misc5}). In (\ref{misc1}), the case endings do not match, as we would expect either \textit{oðra} if it is a \isi{genitive}, or \textit{tacn} if it is \isi{agreement}. In (\ref{misc2}), it is not possible to be certain about the case, since \textit{ðrowung} is a feminine \isi{noun} and thus can have an \textit{a}-ending in the \isi{nominative}, accusative and \isi{genitive} plural.\footnote{The YCOE \isi{corpus} has tagged it as a \isi{genitive}.} Example (\ref{misc3}) is unusual in the sense that there is a \isi{demonstrative} before \textit{fela}. There are in addition two instances of \textit{fela} in \isi{combination} with the preposition \textit{of}. In (\ref{misc4}), from the \textit{Homilies of Wulfstan}, there is clear case marking on the \isi{adjective} and \isi{noun},\footnote{The case is either \isi{genitive} or \isi{dative} here; the YCOE \isi{corpus} analyzes it as \isi{dative}, governed by the preposition \textit{of}.} while in (\ref{misc5}), from the \textit{Peterborough Chronicle} year 1070, the case marking is opaque. 

\ea\label{misc1}
\gll fela oðre tacna\\
	many other tokens\\
\glt ‘many other signs’ (OEng.652.573; ÆCH2)
\z

\ea\label{misc2}
\gll hu fela ðrowunga\\
how many sufferings\\
\glt ‘how many sufferings’ (OEng.664.564; ÆCH1)
\z

\ea\label{misc3}
\gll þa fela rican\\
\DEF{} many rich\\
\glt ‘the many rich (people)’ (OEng.094.050; ÆCH1)
\z

\ea\label{misc4}
\gll to fela […] of godcundre heorde\\
too many {} of religious.\textsc{gen./dat.sg} flock.\textsc{gen./dat.sg}\\
\glt ‘too many […] of the religious flock’ (OEng.965.861; Wulf)
\z

\ea\label{misc5}
\gll fela of þa oðre gærsume\\
many of \DEF{} other treasures\\
\glt ‘many of the other treasures’ (OEng.771.849; OE Pet)
\z

For the sake of illustration, the examples provided so far are quite straightforward, with \textit{fela} followed by a \isi{noun} phrase, except for a couple of examples of a preceding \isi{pronoun}. However, language is seldom completely straightforward, so (\ref{compl1})–(\ref{compl3}) serve to illustrate some \isi{variation} in constructions with \textit{fela}. 

\ea\label{compl1}
\gll \& se cyng ofsloh heora swa feala swa he offaran mihte\\
	and \DEF{} king killed them.\GEN.\PL{} as many as he overtake could\\
\glt ‘and the king killed as many of them as he was able to reach and attack’ (OEng.901.366; OE Pet)
\z


\ea\label{compl2}
\gll \& hi him þar foregislas sealdon swa feala swa he habban wolde\\
	and they him there hostage.\ACC.\PL{} gave as many as he have would\\
\glt‘and there they gave him as many hostages as he wanted’ \\ (OEng.134.533; OE Pet)
\z

\ea\label{compl3}
\gll wundra on þyssere worulde fela\\
wonder.\GEN.\PL{} in \textsc{dem} world many\\
\glt ‘many of the wonders in this world’ (OEng.571.901; ÆLS)
\z

\subsection{Agreement versus genitive with \textit{fela}}\label{sect:agrvspart}
Table \ref{tab:distrfela} gives an overview of the entire distribution of \textit{fela}, but I am particularly interested in the \isi{variation} between \isi{agreement} and \isi{genitive}. Therefore, in Table \ref{tab:felaagrpart}, I disregard the instances of \textit{fela} standing alone, the instances of \textit{fela} with a \isi{numeral}, and the ``miscellaneous" instances. I also exclude the ``DP dependents", i.e. constructions with a \isi{pronoun} (\ref{partpron}), or with a \isi{demonstrative} (\ref{partdem1}) or possessive (\ref{partposs}) preceding the \isi{noun}, since these are always in the \isi{genitive} case, as well as the two instances of constructions with the preposition \textit{of}. 

\begin{table}
\caption{The distribution of \textit{fela} used with agreement vs. genitive in Ælfric’s texts and the \textit{Peterborough Chronicle} (Old English parts) vs. the rest of the Old English texts}\label{tab:felaagrpart}
 \begin{tabularx}{\textwidth}{l@{}rYrYrYr}
  \lsptoprule
  Texts &   & \multicolumn{2}{c}{Agreement} & \multicolumn{2}{c}{Genitive}\\
  \cmidrule(lr){3-4}\cmidrule(lr){5-6}
  &\textit{fela} total agr + gen & $n$ & \% & $n$ & \%\\
  \midrule
  Ælfric’s texts	& 202 & 99 & 49.0 & 103 & 51.0\\
  The OE \textit{Peterb. Chron}. & 26 & 16 & 61.5 & 10 & 38.5\\
  All other OE texts	& 191 & 16 & 8.4 & 175 & 91.6\\
 \lspbottomrule
 \end{tabularx}
\end{table}

Table \ref{tab:felaagrpart} is thus meant to show the distribution when the writer in principle had a choice between \isi{agreement} and \isi{genitive}. With DP dependents, there is no choice, because the grammar dictates that they always occur in the \isi{genitive}. But with non-DP dependents, there was apparently a choice for Ælfric and for the writers of the first part of the chronicle.

In Table \ref{tab:felaagrpart} we can clearly see the difference between Ælfric’s texts and the \textit{Peterborough Chronicle} on the one hand, and the rest of Old \ili{English} on the other. Ælfric’s texts and the chronicle are quite similar, but the chronicle is even more ``modern" than Ælfric, in the sense that \isi{agreement} is used more than the \isi{genitive}. The distribution seen in Table \ref{tab:felaagrpart} will be further discussed in Section \ref{sect:disc}.

\subsection{Results for \textit{manig}}\label{sect:resultmanig}
Table \ref{tab:distrmanig} shows the distribution of \textit{manig} in the seven Old \ili{English} texts studied here. I distinguish between \textit{manig} with \isi{agreement}, \textit{manig} with \isi{genitive}, \textit{manig} standing alone, and miscellaneous cases. Examples are given below. 

\begin{table}
\caption{The distribution of \textit{manig} in the texts}\label{tab:distrmanig}
 \begin{tabularx}{\textwidth}{l@{}r Yr Yr Yr Yr}
  \lsptoprule
  & & \multicolumn{2}{c}{Agreement} & \multicolumn{2}{c}{Genitive} & \multicolumn{2}{c}{Alone} & \multicolumn{2}{c}{Misc.}\\
  \cmidrule(lr){3-4}\cmidrule(lr){5-6}\cmidrule(lr){7-8}\cmidrule(lr){9-10}
  Texts & {\textit{manig} total} & $n$ & \% & $n$ & \% & $n$ & \% & $n$ & \% \\
  \midrule
  \textit{ASC(A)} & 19 & 11 & 57.9 & 2 & 10.5 & 4 & 21.1 & 2 & 10.5 \\
\textit{Bede} & 195 & 122 & 62.6 & 17 & 8.7 & 34 & 17.4 & 22 & 11.3\\
\textit{Oros}	& 101 & 79 & 78.2 & 1 & 1.0 & 5 & 5.0 & 16 & 15.8\\
\textit{ÆLS}	& 95 & 69 & 72.6 & 3 & 3.2 & 15 & 15.8 & 8 & 8.4\\
\textit{ÆCH1}	& 57 & 36 & 63.2 & 0 & 0 & 14 & 24.6 & 7 & 12.3\\
\textit{ÆCH2}	& 55 & 33 & 60.0 & 1 & 1.8 & 12 & 21.8 & 9 & 16.4\\
\textit{OE Pet}	& 71 & 54 & 76.1 & 1 & 1.5 & 7 & 9.9 & 9 & 12.7\\
 \lspbottomrule
 \end{tabularx}
\end{table}

Examples (\ref{manigagr1})–(\ref{manigagr3}) show \textit{manig} with \isi{agreement}, while (\ref{manigpart1})–(\ref{manigpart3}) are examples with a \isi{genitive}. 

\ea\label{manigagr1}
\gll manegum ðeowracum\\
	many.\DAT.\PL{} threat.\DAT.\PL\\
\glt ‘many threats’ (OEng.393.842; ÆLS)
\ex\label{manigagr2}
\gll swa manege gersumas\\
	so many.\ACC.\PL{} treasure.\ACC.\PL\\
\glt ‘so many treasures’ (OEng.407.002; OE Pet)
\ex\label{manigagr3}
\gll hu monega gefeoht\\
	how many.\ACC.\PL{} battle.\ACC.\PL\\
\glt ‘how many battles’ (OEng.777.881; Oros)
\ex\label{manigpart1}
\gll monige […] lifigendra manna\\
	many.\NOM.\PL{} {} living.\GEN.\PL{} man.\GEN.\PL\\
\glt ‘many […] living men’ (OEng.773.105; Bede)
\ex\label{manigpart2}
\gll mænigo þara wergra gasta\\
	many.\ACC.\PL{} \DEF.\GEN.\PL{} evil.\GEN.\PL{} spirit.\GEN.\PL\\
\glt ‘many of the evil spirits’ (OEng.847.366; Bede)
\ex\label{manigpart3}
\gll Manega tacna\\
	many.\NOM.\PL{} sign.\GEN.\PL\\
\glt ‘many signs’ (OEng.941.407; ÆCH2)
\z

\textit{Manig} can also stand alone, as in (\ref{manigalone1})--(\ref{manigalone2}).

\ea\label{manigalone1}
\gll þæt manega cumað fram eastdæle\\
	that many.\NOM.\PL{} come from eastpart\\
\glt ‘that many come from the east’ (OEng.086.173; ÆCH1)
\ex\label{manigalone2}
\gll \& mænige gewundedon þærinne\\
	and many.\ACC.\PL{} wounded therein\\
\glt ‘and wounded many there’ (OEng.749.054; OE Pet)
\z

As Table \ref{tab:distrmanig} shows, there were more instances of \textit{manig} in the miscellaneous category than of \textit{fela} in the same category. I show a few of them here. Quite commonly, the construction with \textit{manig} is the complement of a \isi{noun}, so that both \textit{manig} and its \isi{noun} are in the \isi{genitive} case, cf. (\ref{manigcompl}). Hence, this is not a relevant construction for my purposes.

\ea\label{manigcompl}
\gll manegra ðeoda fæder\\
	many.\GEN.\PL{} people.\GEN.\PL{} father.\NOM.\SG\\
\glt ‘the father of many peoples’ (OEng.270.759; ÆCH1)
\z

Quite a few of the examples sorted into the miscellaneous category contained a feminine \isi{noun}, so that it is strictly speaking not possible to determine case on the basis of the form alone. In (\ref{manigfem}), \textit{leoda} could in principle be either accusative or \isi{genitive}; the ending would be the same. The YCOE \isi{corpus} annotates such cases as agreeing with \textit{manig}, so that \textit{leoda} in (\ref{manigfem}) would be an accusative plural. This is of course the most likely analysis, since \textit{manig} is very consistent in occurring with \isi{agreement}. I have, however, chosen to keep such instances apart.

\ea\label{manigfem}
\gll manega leoda\\
	many.\ACC.\PL{} peoples\\
\glt ‘many peoples’ (OEng.206.233; ÆCH1)
\z

In (\ref{manighus}), \textit{huses} has an unexpected ending for a neuter, plural \isi{noun}: it should be \textit{hus}. But this example is from the \textit{Peterborough Chronicle} year 1117, so clearly the generic plural form in -(\textit{es}) is starting to develop. I could have analyzed this as \isi{agreement}, but chose to place this example in the miscellaneous category.

\ea\label{manighus}
\gll manige mynstras \& turas \& huses\\
	many.\NOM.\PL{} minster.\NOM.\PL{} and tower.\NOM.\PL{} and houses\\
\glt ‘many minsters and towers and houses’ (OEng.042.102; OE Pet)
\z

With \textit{fela} there were two examples of an \textit{of}-construction. With \textit{manig}, there were eight in the texts under consideration here. Two of them are shown in (\ref{manigof1}) and (\ref{manigof2}).

\ea\label{manigof1}
\gll monige of his folce\\
many.\NOM.\PL{} of his people.\DAT.\SG\\
\glt ‘many of his people’ (OEng.608.943; Bede)
\z

\ea\label{manigof2}
\gll swyðe manega of þæs cynges hired\\
very many.\NOM.\PL{} of \DEF.\GEN.\SG{} king.\GEN.\SG{} court.\ACC.\SG\\
\glt ‘very many of the king’s court’ (OEng.908.344; OE Pet)
\z

\subsection{Agreement versus genitive with \textit{manig}}\label{sect:manigagrpart}

In the same way as for \textit{fela}, I also made a table for \textit{manig} comparing the distribution of \isi{agreement} and \isi{genitive} constructions. I excluded the instances of \textit{manig} standing alone and the ``miscellaneous" instances, as well as constructions with a \isi{demonstrative} or a \isi{pronoun}, and the eight instances with the preposition \textit{of}.

\begin{table}
\caption{The distribution of \textit{manig} used with agreement vs. genitive in the texts}\label{tab:manigagrpart}
 \begin{tabularx}{.8\textwidth}{l@{}r Yr Yr}
  \lsptoprule
   && \multicolumn{2}{c}{Agreement} & \multicolumn{2}{c}{Genitive}\\
   \cmidrule(lr){3-4}\cmidrule(lr){5-6}
  Texts & \textit{manig} total  agr + gen & $n$ & \% & $n$ & \%\\
  \midrule
  \textit{ASC(A)} & 11 & 11 & 100.0 & 0 & 0\\
  \textit{Bede} & 126 & 122 & 96.8 & 4 & 3.2\\
  \textit{Oros} & 72 & 72 & 100.0 & 0 & 0\\
  \textit{ÆLS} & 70 & 68 & 97.1 & 2 & 2.8\\
  \textit{ÆCH1} & 36 & 36 & 100.0 & 0 & 0\\
  \textit{ÆCH2} & 34 & 33 & 97.1 & 1 & 3.0\\
  \textit{OE Pet} & 54 & 53 & 98.1 & 1 & 1.9\\
 \lspbottomrule
 \end{tabularx}
\end{table}

As Table \ref{tab:manigagrpart} shows, \textit{manig} overwhelmingly occurs with \isi{agreement}. While Ælfric’s texts show \isi{variation} between \isi{agreement} and \isi{partitive} as concerns \textit{fela}, they are very consistent with respect to \textit{manig}, like the other Old \ili{English} texts.


\section{Discussion}\label{sect:disc}
In this section I first give an outline of a study \citep{RoehrsSapp18} that has been useful for this chapter, before I go on to a discussion of the findings of the texts under consideration here. The assumption is that \textit{fela} originally occurred with the \isi{genitive}, and that there was a development away from this, before \textit{fela} eventually disappeared. Ælfric seems to have been ahead of the field in this respect, and the \isi{variation} is also evident in the \textit{Peterborough Chronicle}. I propose that the \isi{variation} is not random, but is conditioned by the following factors:

\begin{itemize}
\begin{sloppypar}
   \item Cardinal vs. \isi{proportional reading}. \textit{Fela} + \isi{agreement}, i.e. the newer construction, can only have a \isi{cardinal reading}. \textit{Fela} + \isi{genitive}, i.e. the older construction, mostly has a \isi{proportional reading}, but can have a \isi{cardinal reading} in some cases. In earlier times, when \textit{fela} + \isi{genitive} presumably was the common construction, it was used to convey both cardinal and proportional meaning, hence we would expect to see remnants of this \isi{variation} in the old  construction, whereas the new construction with \isi{agreement} would be consistent. Cf. \citet[404]{Drinka17}: “Innovations virtually never completely occlude previous categories, but build on them.”

   \item Concrete and countable nouns vs. abstract nouns. \textit{Fela} + \isi{agreement} is mostly used with concrete, countable nouns, while \textit{fela} + \isi{genitive} is mostly used with abstract nouns. 
   
   \item Constructions with \textit{fela} + \isi{genitive} are frequently objects and prepositional complements rather than subjects. If \textit{fela} + \isi{genitive} functions as subject, it is usually in existential/presentative constructions, or in passive constructions, which testifies to their non-agentive nature, as opposed to \textit{fela} + \isi{agreement} constructions, which are more likely to be agentive.    \end{sloppypar}
   

 \end{itemize}
\subsection{Roehrs and Sapp (2018)}\label{ssect:RoehrsSapp}
Of particular relevance for this chapter is Roehrs and Sapp's \citeyearpar{RoehrsSapp18} study of complex quantifiers in Old \ili{English}, with \ili{Old Icelandic} and Old High \ili{German} playing supporting roles. They propose a distinction between head-type quantifiers and phrase-type quantifiers. Head-type quantifiers are not inflected and are not modified by degree words \citeyearpar[389]{RoehrsSapp18}. Examples are \textit{awiht} ‘some/any (thing)’, \textit{nanþing} ‘no(thing)’,\textit{ (ge)hwa }‘some/any (one)’. As regards Old \ili{English}, the dependents of such quantifiers are, with a few exceptions, in the \isi{genitive} \citeyearpar[390]{RoehrsSapp18}. Phrase-type quantifiers, on the other hand, are adjective-like \citeyearpar[398]{RoehrsSapp18} and take dependents that are either genitives or in \isi{agreement} (they call it concord) with the \isi{quantifier} \citeyearpar[399--401]{RoehrsSapp18}. Examples are \textit{ælc} ‘each’, \textit{(ge)hwæðer} ‘either (of two)’ and \textit{nænig} ‘no/none’. If the dependent of a \isi{phrase-type quantifier} is what they call a “DP-size dependent", i.e. pronouns, and nominals with an overt \isi{determiner} \citeyearpar[388]{RoehrsSapp18}, it is in the \isi{genitive} \citeyearpar[399]{RoehrsSapp18}. If the dependent is a “\isi{non-DP dependent}", i.e. dependent nouns and constructions with an \isi{adjective} plus a \isi{noun} \citeyearpar[388]{RoehrsSapp18}, it is overwhelmingly in \isi{agreement} with the \isi{quantifier} \citeyearpar[399--401]{RoehrsSapp18}. Of the three languages, Old \ili{English} shows the most \isi{variation}, as Old High \ili{German} has \isi{genitive} dependents regardless of the type of \isi{quantifier}, while \ili{Old Icelandic} mostly has \isi{agreement}.  

On the basis of their empirical findings, Roehrs and Sapp propose a syntactic analysis of the \isi{variation}, couched within the generative framework. Head-type quantifiers are, as the name suggests, in a head position (in the syntactic structure), whereas phrase-type quantifiers are in a specifier position. Furthermore, DP-size dependents are always the complement of N, whereas non-DP \isi{genitive} dependents are in a specifier position, and non-DP \isi{agreement} dependents are in the nominal projection line \citeyearpar[381, 396, 398, 404]{RoehrsSapp18}. I will not enter into a detailed discussion about this proposal, but merely point out that if this is meant to be valid for quantifiers in general, \textit{fela} does not quite fit in, as we shall see.

A few more relevant points from Roehrs and Sapp’s work is that they do not find that semantics plays a role in the choice between \isi{genitive} and \isi{agreement} \citeyearpar[417]{RoehrsSapp18}. They also mention diachronic change \citeyearpar[416]{RoehrsSapp18}, and propose that Old High \ili{German} is the “oldest" language, since it may be assumed that \isi{genitive} dependents represent the older stage, while \ili{Old Icelandic} is the “youngest", since quantifiers occur in \isi{agreement} constructions. As usual with Old \ili{English}, it is somewhere in between. But Roehrs and Sapp \citeyearpar[416]{RoehrsSapp18} make the interesting point that a change is taking place with some Old \ili{English} writers, since there are instances of head-type quantifiers that have non-DP dependents that are not in the \isi{genitive} case. 

As mentioned, \citet{RoehrsSapp18} specifically study complex quantifiers, so \textit{fela} is not included, apart from a mention in a footnote where they say that \textit{fela} is probably a \isi{head-type quantifier} \citeyearpar[389]{RoehrsSapp18}, since according to \citet[vol. I, 172]{Mitchell85}, \textit{fela} mostly occurs with the \isi{genitive}. But now that we have seen the data for \textit{fela} and the \isi{variation} that exists, the questions that arise are: what caused the \isi{variation}, and what type of \isi{quantifier} is \textit{fela }in this terminology – head-type or phrase-type? \textit{Fela} is indeclinable, i.e. not adjective-like, so in that sense it is like a \isi{head-type quantifier}.\footnote{Note also that Roehrs and Sapp \citeyearpar{RoehrsSapp16} demonstrate that the Old High \ili{German} cognate \textit{filu }is a \isi{head-type quantifier}, being indeclinable and occurring exclusively with \isi{genitive} nouns.}  But it can be modified by a degree adverb, \textit{swiðe} ‘very’, though admittedly this is rare. Furthermore, as we have seen, in Ælfric’s texts and the \textit{Peterborough Chronicle}, \textit{fela} commonly occurs with \isi{agreement}, which we would not expect with head-type quantifiers. 

\subsection{Ælfric's texts}\label{ssect:Ælfric}

If we assume that \textit{fela} + \isi{genitive} was the original construction, as indicated both by other \ili{Germanic} languages (cf. \cite{RoehrsSapp16}), and by the great majority of Old \ili{English} texts, Ælfric’s usage was clearly unusual with respect to \textit{fela}. His use anticipates what we see in the \textit{Peterborough Chronicle}, and this change would be as expected in light of the general developments of \ili{English} and the way in which \isi{noun} phrases are structured in Present-day \ili{English}, i.e. with quantifiers modifying a nominal head, rather than the \isi{noun} being the complement of the \isi{quantifier}. Note that Wulfstan, Ælfric’s contemporary, and even a little younger, is much more conservative in the use of \textit{fela }(cf. Table \ref{tab:distrfela}).\footnote{\citet[vol. I, 174]{Mitchell85} comments that in Ælfric's texts the verb is usually plural after \textit{fela} + \isi{genitive}, whereas \textit{fela} + \isi{genitive} is followed by a singular verb in Wulfstan. In my data from Wulfstan’s homilies, there are only 12 cases of a \textit{fela} construction that functions as the subject of a verb, and of those, nine have a singular verb, whereas plural verbs are the most common in Ælfric's texts. This is an interesting difference between the contemporaries, because it supports the impression that for Ælfric, \textit{fela} was becoming a \isi{quantifier}, with the \isi{noun} governing the verbal concord, while for Wulfstan, it was a \isi{partitive}, with \textit{fela} governing the verbal concord.} The question is: can we discern any patterns of usage when it comes to Ælfric’s use of \textit{fela}? Roehrs and Sapp find that the choice between \isi{agreement} and \isi{genitive} is not semantically motivated for the complex quantifiers they study \citeyearpar[417]{RoehrsSapp18}, but I will argue that it conditioned the use of \textit{fela }in the texts that show \isi{variation}. Language change has to start somewhere, and if an individual shows signs of it in his language, it would not be unlikely that the \isi{variation} arises due to different shades of meaning in certain constructions. Furthermore, Ælfric was known as a great and conscious stylist (\cite{Gatch1977}; \cite{Godden2004}; \cite{Harris2006}), and my point of departure is therefore that the distribution with respect to the use of \textit{fela} in Ælfric's texts is a result of linguistic choice.

\citet[417]{RoehrsSapp18} comment that for Present-day \ili{English}, there is, for non-DP dependents (e.g. Old \ili{English} \textit{fela men}, \textit{fela manna} ‘many men', \textit{fela gode men} ‘many good men'), a distinction between a \isi{cardinal reading}, denoting members of a set, and a \isi{proportional reading}, denoting members of a pre-established set. For example, \textit{many men fought the battle} can mean that the number of men that fought the battle was large (\isi{cardinal reading}), or it can mean that a large proportion of the men fought the battle (\isi{proportional reading}). DP dependents (e.g. Old \ili{English} \textit{fela þara manna}) only have a \isi{proportional reading} (cf. Present-day \ili{English} \textit{many of the men fought the battle}). According to \citet[417]{RoehrsSapp18}, this interpretative distinction likely held in the older languages as well, since if it did not, the question arises as to when and why that distinction arose later. I follow Roehrs and Sapp in this assumption, also because there are so few instances of \textit{fela} + a \isi{noun} preceded by a \isi{demonstrative}. We would expect more constructions with a \isi{demonstrative} if that was the only way of indicating proportionality. 

As concerns the complex quantifiers that Roehrs and Sapp study, they find that DP dependents are always in the \isi{genitive}, but that non-DP dependents are in \isi{agreement} with phrase-type quantifiers in Old \ili{English} and with all quantifiers in \ili{Old Icelandic}. If non-DP dependents can also have a \isi{proportional reading} in the older languages, we might expect to see more genitives for non-DP dependents, on a par with DP-dependents. Since DP-dependents are always in the \isi{genitive} and always have a \isi{proportional reading}, proportionality and the \isi{genitive} case seem to be associated. But Roehrs and Sapp find that non-DP dependents agree with the \isi{quantifier}. There are only a few cases of \isi{genitive}, and they are mostly idiomatic expressions. Hence, they conclude that although the distinction between cardinal and proportional readings existed in earlier language stages, the distribution they see for the complex quantifiers is better explained structurally rather than semantically \citeyearpar[417]{RoehrsSapp18}.

However, as we have seen in the present study, \textit{fela} can occur with non-DP dependents either in \isi{agreement} or with \isi{genitive} case in Ælfric and the \textit{Peterborough Chronicle}. This means that the status of \textit{fela} was probably vacillating between head and specifier, and Ælfric and the chronicle thus represent both an older and a newer stage with respect to this construction. Considering semantic factors might therefore provide some insight, so in the following sections I have consequently studied some aspects of these texts in more detail, with the purpose of unearthing possible patterns.

\subsubsection{The type of noun in agreement and genitive constructions in Ælfric’s texts}\label{sssect:nountype}

One question was whether the type of \isi{noun} plays a role with respect to whether fela would occur with \isi{agreement} or with \isi{genitive}. In (\ref{felanounsagr}), the nouns found with \textit{fela} + \isi{agreement} in Æfric’s texts are listed alphabetically (a total of 53), and in (\ref{felanounsgen}) those with \textit{fela} + \isi{genitive} (a total of 26). 


\begin{exe}
    \ex \label{felanounsagr}
\textit{ælmyssan} 'alms', \textit{ærendracan} 'messengers', \textit{bec} ‘books’, \textit{bedredan}\footnote{Nominalized adjectives are always in \isi{agreement}.} ‘bedridden (people)’, \textit{bisceopas} ‘bishops’, \textit{blinde} ‘blind (people)’, \textit{cnapan} ‘knaves’, \textit{cnottan} ‘knots’, \textit{corn} ‘grains’, \textit{cristene} ‘Christians’, \textit{cynincgas} ‘kings’, \textit{cyrcan} ‘churches’, \textit{dæda} ‘deeds’, \textit{deade} ‘dead (people)’, \textit{earfoþnyssum} ‘difficulties’, \textit{englas} ‘angels’,\textit{ estmettum} ‘delicate meats’, \textit{fugolcynn} ‘fowl-kind’, \textit{gearum} ‘years’, \textit{gerefan} ‘stewards’, \textit{gereord} ‘languages’, \textit{gesetnyssa} ‘decrees’, \textit{geþoh- tas} ‘thoughts’, \textit{gewinn} ‘battles’, \textit{gewissungum} ‘instructions’, \textit{gewitan} ‘witnesses’, \textit{gleda} ‘coals’, \textit{god} ‘good deeds/things’, \textit{godspel} ‘gospels’, \textit{goldhordas} ‘gold hoards’, \textit{halgan} ‘saints’, \textit{heahfæderas} ‘patriarchs’, \textit{herereaf} ‘plunders’, \textit{hundas} ‘dogs’, \textit{lande }‘lands’, \textit{mædenu} ‘maidens’, \textit{menn} ‘men’, \textit{næddran} ‘adders’, \textit{oðre} ‘others’, \textit{reoflige} ‘leprous (people)’, \textit{sceoccan} ‘demons’, \textit{þearfan} ‘poor (people)’, \textit{þing} ‘things’, \textit{tunnan} ‘barrels’, \textit{unlybban} ‘poisons’, \textit{unþeawas} ‘vices’, \textit{untrume} ‘sick (people)’, \textit{werod} ‘bands (of angels)’, \textit{witan} ‘wise men’, \textit{wode} ‘mad (people)’,\textit{ wyrta }‘plants’, \textit{wytegan }‘wise men’, \textit{yfelu} ‘evils’. 
\end{exe}

\begin{exe}
    \ex \label{felanounsgen}
\textit{byrðena} ‘loads (of earth)’, \textit{cnihta} ‘boys’, \textit{daga} ‘days’, \textit{engla} ‘angels’, \textit{gasta} ‘spirits’, \textit{gereorda} ‘languages’, \textit{gewitnyssa} ‘testimonies’, \textit{goda} ‘good deeds/ things’, \textit{laca} ‘offerings’, \textit{læca} ‘physicians’, \textit{leorningcnihta} ‘disciples’, \textit{manna} ‘men’, \textit{muneca} ‘monks’, \textit{munuclifa} ‘monasteries’, \textit{musa} ‘mice’, \textit{searacræfta} ‘treacherous arts’, \textit{þinga}\footnote{There was only one example of the \isi{noun} \textit{þing} with \isi{genitive}; this \isi{noun}, which occurs quite frequently, is categorically in \isi{agreement}.} ‘things’, \textit{tacna} ‘signs’, \textit{templa} ‘temples’, \textit{tida} ‘time periods’, \textit{ungelimpa} ‘misfortunes’, \textit{wildeora} ‘wild animals’, \textit{winboga} ‘vine branches’, \textit{wundra}\footnote{\textit{Wundra }occurs frequently, and always in the \isi{genitive}.} ‘wonders’, \textit{yfela} ‘evils’, \textit{yrmða} ‘calamities’.
\end{exe}


We may note several things here. First, the number of distinct nouns occurring with \textit{fela} + \isi{agreement} is double the number of nouns occurring with \textit{fela} + \isi{genitive}. Second, the majority of the nouns in (\ref{felanounsagr}) are animate nouns denoting people or groups of people, or human-like spirits of various kinds, or they are tangible nouns denoting objects or substances. There are some such nouns in (\ref{felanounsgen})  as well, but here we see a larger proportion of abstract nouns, such as \textit{gewitnyssa} ‘testimonies’, \textit{searacræfta} ‘treacherous arts’, etc. 

If we take Ælfric’s usage of \textit{fela }+ \isi{agreement} to be of the new type, the fact that it occurs with so many different nouns indicates that his usage was perhaps even more advanced than the data in Table \ref{tab:felaagrpart} indicate. The numbers there show an even distribution between \isi{agreement} and \isi{genitive} with \textit{fela}, but here we see that the distribution is uneven with respect to \isi{noun} types, which points towards the \textit{fela} + \isi{agreement} construction being the more productive \isi{pattern} for Ælfric. Moreover, it might indicate that the change in the use of \textit{fela }towards a construction with \textit{fela} as specifier of a \isi{noun} head started with concrete, countable nouns, which would not be unexpected with a word meaning ‘many’. 

Furthermore, if we look into some of the animate nouns in (\ref{felanounsgen}), it becomes apparent that they mostly get a \isi{proportional reading}. Compare (\ref{felaenglas}) and (\ref{felaengla}). \textit{Fela englas} (with \isi{agreement}) in (\ref{felaenglas}) has a \isi{cardinal reading} and denotes angels arriving, armed for fight. It is many angels, not many angels out of a pre-established set. The context is that the Roman general (and later saint) Gallicanus relates how he was converted to God. He was besieged in a town, along with a small army, and tried sacrifices to the gods to get out of this predicament. This did not help, but he was told that if he would bow to the God of heaven, he would be victorious. He did so, and immediately an angel came with a cross, and thereafter many splendidly armed angels. Only a \isi{cardinal reading} is possible here. 

In (\ref{felaengla}), \textit{fela engla} (with \isi{genitive}) are also arriving, but in the company of the Lord, so here they are a part of the entourage, and it is possible to give (\ref{felaengla}) a \isi{proportional reading}, meaning not all the angels, but a sizable proportion of the heavenly host, a presupposed set of angels. Note that this is a possible reading. It is not impossible to give this example a \isi{cardinal reading}. The point is that \textit{fela} + \isi{agreement} must get a \isi{cardinal reading}, while \textit{fela} + \isi{genitive} can, and in most cases does, have a \isi{proportional reading}. The newer construction, i.e. \textit{fela} + \isi{agreement}, is the marked alternative. It marks a certain nuance, and it is consistent. The older construction, i.e. \textit{fela} with \isi{genitive}, retains the possibility of both meanings. However, I argue that the \isi{proportional reading} is the most likely one in most cases, and that the cardinal–proportional distinction was in fact a conditioning factor in Ælfric’s usage.

\ea\label{felaenglas}
\gll Ic him fyligde ða, and fela englas coman on manna gelicnyssum, mærlice gewæpnode\\
I him followed then and many angel.\NOM.\PL{} came in man.\GEN.\PL{} likenesses splendidly armed\\
\glt ‘I followed him then, and many angels came in the likeness of men, splendidly armed.’ (OEng.837.589; ÆLS)
\z

\ea\label{felaengla}
\gll Þær com eac se hælend mid þam heofonlican leohte, and fela engla mid him\\
there came also \DEF{} Lord with \DEF{} heavenly light and many angel.\GEN.\PL{} with him\\
\glt ‘There the Lord also came with the heavenly light, and many angels with him.’ (OEng.938.505; ÆLS)
\z

In (\ref{muneca}), \textit{fæla muneca} can also get a \isi{proportional reading}. The context is that (saint) Julian established one monastery for himself and one for (saint) Basilissa; hence Julian became the spiritual father of many monks (\textit{fæla muneca}), and Basilissa the spiritual mother of many nuns (\textit{manega mynecena}, which is in fact a very rare example of the \isi{genitive} after \textit{manig}). A possible reading here is that these monks are members of a pre-established set of monks, since the existence of monasteries implies monks.\footnote{A reviewer points out that \citet[vol. I, 172--173]{Mitchell85} is sceptical with regard to a \isi{proportional reading} of \textit{fela} + \isi{non-DP dependent}. Mitchell says that \textit{fela oðerra muneca} ‘many other monks’ cannot be proportional because there is no \isi{demonstrative} \textit{þara}, giving \textit{fela þara oðerra muneca}. But this reasoning is somewhat circular: a reading is proportional when a \isi{demonstrative} is present, and a \isi{demonstrative} is present because the reading is proportional. In addition, the example \textit{fela oðerra muneca} does not exist. Mitchell refers to it, but he has it from another source, and he comments that he was not able to find it anywhere. I have not found it either. It is therefore not possible to check the context for it. If it is a real example, there are two possibilities: either it is from a non-Ælfrician text, in which case the \isi{genitive} would be used in any case, or it is from a text by Ælfric, in which case it might have a \isi{proportional reading}, but we cannot check it. In any case, I do not agree with Mitchell here, and the main reason is that \isi{demonstrative} determiners are in fact rare in these constructions, except in\textit{ Orosius}. In Ælfric’s texts, the type with \isi{demonstrative} only occurs 12 times, e.g. \textit{fela þæra læca} ‘many \DEF.\GEN.\PL physician.\GEN.\PL', and of those 12, five are singulars with the \isi{noun} \textit{folc} ‘people’, e.g. \textit{fela þæs folces} ‘many \DEF.\GEN.\SG people.\GEN.\SG'. I therefore think it likely that the type without \isi{demonstrative} could also express proportional meaning. }

\ea\label{muneca}
\gll He wearð þa fæder ofer fæla muneca\\
he became then father over many monk.\GEN.\PL{}\\
\glt ‘He then became the [spiritual] father of many monks.’ \\(OEng.939.611; ÆLS)
\z

In (\ref{musa}) we have \textit{fela} + \isi{genitive} as well, but here a \isi{proportional reading} is not possible – it is not many mice out of a pre-established set of mice. It is rather a mass of mice, for which it is probably not possible to count individuals, that happens to pour out of the idol. The description continues by saying that the mice were \textit{floccmælum yrnende geond þa widgillan flor} ‘flockwise running across the wide floor’ so men might know that this was the abode of mice, and certainly not of anything divine. It may be that the mass meaning of the \isi{noun} pushes it towards \isi{genitive} here, since \textit{fela} + \isi{agreement} is mostly used with concrete, countable, agentive nouns. 

\ea\label{musa}
\gll Þar wearð þa micel gamen þæt feala musa scutan of þære anlicnysse\\
there happened then much mirth that many mouse.\GEN.\PL{} shot from \DEF{} idol\\
\glt ‘Then the amusing thing happened that many mice poured out of the idol.' (OEng.019.729; ÆHS)
\z

\subsubsection{\textit{Fela men} (agreement) vs. \textit{fela manna }(genitive) in Ælfric’s texts}\label{sssect:felamenfelamanna}

As a final exercise in trying to disentangle Ælfric’s use of \isi{agreement} vs. \isi{genitive} with \textit{fela}, I consider the use of \textit{fela} with the \isi{noun} \textit{man}. This \isi{noun} occurs with both \isi{agreement} and \isi{genitive}, even within the same text, but the \isi{variation} is particularly apparent in the \textit{Lives of Saints}. Table \ref{tab:felaman} shows the distribution, including whether there is also an \isi{adjective} present, as in (\ref{adligemenn})–(\ref{ricramanna}).\footnote{There were no instances in Ælfric’s letter to Sigeweard.}

\begin{table}
\caption{The distribution of \textit{fela} with the noun \textit{man} in Ælfric's texts}\label{tab:felaman}
 \begin{tabularx}{.8\textwidth}{l YYYY}
  \lsptoprule
    & \multicolumn{2}{c}{Agreement} & \multicolumn{2}{c}{Genitive}\\
    \cmidrule(lr){2-3}\cmidrule(lr){4-5}
  Texts& +\isi{adjective} & -\isi{adjective} & +\isi{adjective} & -\isi{adjective}\\
  \midrule
  \textit{ÆLS} & 7 & 2 & 1 & 5\\
  \textit{ÆCH1} & 1 & 0 & 4 & 0\\
  \textit{ÆCH2} & 0 & 0 & 2 & 5\\
  \textit{ÆHS} & 1 & 0 & 2 & 1\\
 \lspbottomrule
 \end{tabularx}
\end{table}

With two exceptions, in all the instances of \textit{fela} with \textit{man} in \isi{agreement} in Ælfric’s texts, there is also an \isi{adjective}, as in (\ref{adligemenn}) and (\ref{cristenemenn}). \textit{Fela} with \textit{man} in the \isi{genitive} may contain an \isi{adjective}, cf. (\ref{ricramanna}).

\ea\label{adligemenn}
\gll fela adlige menn\\
many sick.\NOM.\PL{} man.\NOM.\PL{}\\
\glt ‘many sick men’ (OEng.530.902; ÆLS)
\z

\ea\label{cristenemenn}
\gll fela cristene menn\\
many Christian.\NOM.\PL{} man.\NOM.\PL{}\\
\glt ‘many Christian men’ (OEng.553.207; ÆLS)
\z

\ea\label{ricramanna}
\gll fela ricra manna\\
many rich.\GEN.\PL{} man.\GEN.\PL{}\\
\glt ‘many rich men’ (OEng.524.280; ÆCH1)
\z

The presence of adjectives lends \isi{weight} to an analysis of \textit{fela} in a specifier rather than a head position (see \cite[403]{RoehrsSapp18}). Furthermore, it seems that this change – if it was indeed a change from head to specifier – was taking place in Ælfric’s grammar in particular, because in the other Old \ili{English} texts, adjectives rarely occur with \textit{fela}, though there are examples scattered here and there, often with the \isi{adjective} \textit{god} ‘good’ (see ex. (\ref{godramanna})).\footnote{The adjective-like word \textit{oðer} ‘other’ also often occurs with \textit{fela}.} As mentioned, \citet[398]{RoehrsSapp18} find that with complex phrase-type quantifiers and non-DP dependents, there is almost always \isi{agreement}. \textit{Fela} is not quite like that, since its non-DP dependents can also be in the \isi{genitive}. But the fact that Ælfric in his late texts chooses \isi{agreement} when the \isi{noun} is modified by an \isi{adjective} shows that \textit{fela} is not in a head position. The one example in the \textit{Lives of Saints} of \textit{fela} with \isi{adjective} + \textit{man} in the \isi{genitive} is a special case, because a participle intervenes between \textit{fela} and the \isi{noun} phrase complement (\ref{untrumramanna}). The participle \textit{gehælde} has a \isi{nominative} plural ending, so it agrees with the meaning of \textit{fela} rather than its indeclinable form.\footnote{A reviewer points out that \textit{gehælde} could be a \isi{predicative} \isi{adjective}. It is possible, since it can be difficult to determine whether a participle is \isi{predicative} or verbal \citep[vol. I, 649]{Mitchell85}, but considering that there is an expressed ``agent", i.e. the cross, it seems that a verbal reading is more likely here.} The reading here is thus that many were healed, of both people and animals. The focus is on ‘many’ and ‘healed’, and it is then specified who the ‘many’ are. 

\ea\label{untrumramanna}
\gll and wurdon fela gehælde untrumra manna and eac swilce nytena þurh ða ylcan rode\\
and became many healed.\NOM.\PL{} sick.\GEN.\PL{} man.\GEN.\PL{} and also too animal.\GEN.\PL{} through \DEF{} same cross\\
\glt ‘and many sick men and also animals were healed through the same cross’ (OEng.401.711; ÆLS)
\z

Let us now dig a little deeper and look at the constructions where \textit{man} is not modified. In the \textit{Lives of Saints}, Ælfric gives us two examples of \textit{fela} with \textit{man} in \isi{agreement} (\ref{felamen1})–(\ref{felamen2}) and five of \textit{fela} with \textit{man} in the \isi{genitive} (see Table \ref{tab:felaman}). Two of the latter are shown in (\ref{felamanna1}) and (\ref{felamanna2}).

\ea\label{felamen1}
\gll Oft wurdon eac gehælede fela untrume men þurh his reafes fnæda, þe fela men of atugon, and bundon on þa seocan, and him wæs bet sona\\
often became also healed many sick.\NOM.\PL{} man.\NOM.\PL{} through his garment.\GEN{} hem that many man.\NOM.\PL{} out pulled and bound on \DEF{} sick and them was better immediately\\
\glt ‘Many sick men were also often healed through the hem of his garment, from which many men pulled out [threads] and bound on the sick, and they immediately recovered.’ (OEng.551.536; ÆLS)
\z


\ea\label{felamen2}
\gll Wurdon þa on fyrste fela men gebigde þurh heora drohtnunge fram deofles biggengum to Cristes geleafan and to clænum life\\
became then in time many man.\NOM.\PL{} turned through their conversation from devil.\GEN{} worships to Christ.\GEN{} faith and to clean life\\
\glt ‘In time, through their conversation, many men turned from worship of the devil to faith in Christ and to a clean life.’ (OEng.275.096; ÆLS)
\z


\ea\label{felamanna1}
\gll and fela manna þa gehyrdon on his forðsiðe singendra engla swiðe hlude stemna\\
and many man.\GEN.\PL{} then heard on his death singing.\GEN.\PL{} angel.\GEN.\PL{} very loud.\ACC.\PL{} voice.\ACC.\PL{}\\
\glt ‘and upon his death many men heard very loud voices of singing angels’ (OEng.320.345; ÆLS)
\z


The question is why Ælfric uses different constructions like this. It could of course be free \isi{variation}; when you have access to parallel constructions in your grammar, you may want some \isi{variation} for \isi{variation}’s sake. But if we consider that Ælfric was a conscious language user, we want to look for clues that might explain the \isi{variation}, and this is what I will briefly attempt here. 

As mentioned, my proposal is that if the \isi{noun}, in this case \textit{man}, has a \isi{cardinal reading}, is concrete, and refers to agentive individuals, Ælfric would use \isi{agreement}, whereas if the \isi{noun} is abstract, non-agentive, or the reading is proportional, Ælfric would use the \isi{genitive}.

In (\ref{felamen1}), the hem in question is St. Martin’s hem, and we can think of the \textit{fela men} as individuals that one by one come and take threads from the hem in order to use them for healing. The reading is obligatorily cardinal, as there are many such men. In (\ref{felamen2}), the reference is to the saints Chrysantus and Daria, and the \textit{fela men} who became Christians through conversing with them. Again the reference is to many individual men, and not a proportion of a pre-established set of men, so the only possibility is a \isi{cardinal reading}. For (\ref{felamen1}) and (\ref{felamen2}), we would therefore expect \isi{agreement}.

Example (\ref{felamanna1}), on the other hand, is clearly proportional, since these are the men surrounding St. Martin when he dies. A possible, and likely, reading is thus ‘many of the men who were there’, and a \isi{genitive} would be as expected. I also checked the remaining three examples of \textit{fela manna} in the \textit{Lives of Saints}, and in those as well, the (hypothesized) criteria for the \isi{genitive} are fulfilled. 

In (\ref{felamanna2}), however, with \isi{genitive}, we are faced with a counterexample. A \isi{proportional reading} of \textit{fela manna} is not possible, since it is a part of a presentative construction that introduces a new section of the story; hence the men are not members of any pre-established set. Recall that the \isi{genitive} is the older construction, which would retain the possibility of both old and new readings in the event of a change. In other words, while we would expect the new, marked, construction to be consistent, the possibility for \isi{variation} would be kept with the old construction. Hence it would be as expected to come across examples like (\ref{felamanna2}).  

\ea\label{felamanna2}
\gll Auitianus hatte sum hetol ealdorman, wælhreow on his weorcum, se gewrað fela manna, and on racenteagum gebrohte to þære byrig Turonia\\
Avitianus was.called a.certain evil alderman cruel in his actions \DEM{} tied many man.\GEN.\PL{} and in chains brought to \DEF{} city Tours\\
\glt ‘There was a certain evil alderman called Avitianus, cruel in his actions, who put many men in chains and brought them to the city of Tours.’ (OEng.890.917; ÆLS)
\z

To sum up concerning Ælfric: When it comes to \textit{fela}, Ælfric uses \textit{fela} both with \isi{agreement} and \isi{genitive}, and it is not done randomly. If we assume that \textit{fela} goes from being a head to being a specifier, we can, through studying Ælfric in some detail, see that this process follows an expected trajectory of change for a \isi{quantifier}, with the \isi{agreement} construction appearing with nouns that are concrete, countable, or get a \isi{cardinal reading}. The \isi{genitive} remains longer with nouns that are abstract and invite a \isi{proportional reading}.

As we have seen, Ælfric is a linguistic innovator when it comes to the \isi{variation} in the use of \textit{fela}. The other Old \ili{English} texts do not show this, with the exception of the \textit{Peterborough Chronicle}, to which we now turn.


\subsection{The \textit{Peterborough Chronicle}}\label{ssect:peterb}

The \textit{Peterborough Chronicle} is a fascinating text, as it shows the transition from Old to Middle \ili{English}. It is one of seven surviving manuscripts of the Anglo-Saxon Chronicle, i.e. the ‘E’ manuscript (Bodleian MS Laud Misc. 636). After the Norman invasion of 1066, \ili{English} book production largely ceased, but at Peterborough, chronicle writing continued into the post-conquest era as well. However, there was a fire at Peterborough in 1116, which destroyed the original manuscript, so the first part of the chronicle, the annals up until 1121, is copied from other sources, and by the same hand. The \textit{First Continuation} of the Peterborough Chronicle covers the years 1122 to 1131, and the \textit{Second} or \textit{Final Continuation} the years from 1132 to 1154, with the year 1154 marking the end of the \ili{English} chronicle tradition. The continuations are regarded as Early Middle \ili{English}, with the Second Continuation being even more solidly so than the First Continuation. We may also note that interpolations occur in the copied part of the chronicle; these are additions made by the copyist, and they contain information that would only be evident in retrospect. The language of the interpolations is quite different from regular Old \ili{English}. (See \cite[5--12]{BergsSkaffari07} for further information about the chronicle.)

This information about the provenance of the \textit{Peterborough Chronicle} is necessary in order to understand the distribution of \textit{fela} in the text. Below I show that the copied part differs from the interpolations with respect to how \textit{fela} is used, and that the continuations in their turn show further developments of \textit{fela}. In other words, I propose that the change that we see the beginnings of in Ælfric’s texts continues in the chronicle. Table \ref{tab:felapet} shows the distribution of \textit{fela} (with the spellings \textit{fela, feola, feala, feale, feole}) in the different parts of the \textit{Peterborough Chronicle}. Recall that we still, as in Table \ref{tab:felaagrpart}, disregard \textit{fela} standing alone or with a \isi{numeral}, genitives with demonstratives, \isi{genitive} pronouns, instances of \textit{of}, and cases where the construction is opaque.

In the copied part of the chronicle, i.e. the oldest part, the distribution of \textit{fela} with \isi{agreement} or with \isi{genitive} is quite even; (\ref{scipugislas}) and (\ref{godramanna}) are two examples of \isi{agreement} and \isi{genitive}, respectively.

\begin{table}
\caption{The distribution of \textit{fela} in the \textit{Peterborough Chronicle}}\label{tab:felapet}
 \begin{tabularx}{.8\textwidth}{l YY}
  \lsptoprule
  Text parts  & {Agreement} & {Genitive}\\
  \midrule
  Copied part & 10 & 9\\
  Interpolations & 6 & 1\\
  First continuation & 10 & 0\\
  Second continuation & 0 & 0\\
 \lspbottomrule
 \end{tabularx}
\end{table}


\ea\label{scipugislas}
\gll scipu \& gislas swa fela swa hi woldon\\
ship.\ACC.\PL{} and hostage.\ACC.\PL{} as many as they wanted\\
\glt ‘as many ships and hostages as they wanted’ (OEng.642.022, OE Pet)
\z

\ea\label{godramanna}
\gll feala godra manna\\
many good.\GEN.\PL{} man.\GEN.\PL{}\\
\glt ‘many good men’ (OEng.481.782; OE Pet)
\z

As was the case in Ælfric, the nouns occurring with \isi{agreement} in the chronicle are largely concrete and countable nouns. They are: \textit{Bryttas} ‘Britons’, \textit{foregislas} ‘foremost hostages’, \textit{hreowlice} \textit{\&} \textit{hungerbitende} ‘miserable and hunger-bitten (people)’, \textit{lande} ‘lands’, \textit{sceattas} ‘treasures’, \textit{scipe} ‘ships’, \textit{scipu} ‘ships’, \textit{scipu} \textit{\&} \textit{gislas} ‘ships and hostages’, \textit{þeodan} ‘peoples’, \textit{þingan} ‘things’, \textit{wintrum} ‘winters’. They also have a \isi{cardinal reading}. Out of the nine occurrences with \textit{fela} and a \isi{genitive}, five contain the \isi{noun} \textit{manna}. In all of those cases, \textit{manna} has a \isi{proportional reading} ‘many of the men’, as in (\ref{willelm}), which is about King William fighting a battle in which his son William is wounded and many of his men (alternatively many of the men fighting the battle) were killed. 

\ea\label{willelm}
\gll \& eac his sunu Willelm wearð þær gewundod. \& fela manna ofslagene\\
and also his son William became there wounded and many man.\GEN.\PL{} killed\\
\glt ‘and his son William was also wounded there, and many men were killed’ (OEng.433.102; OE Pet)
\z

The remaining four are: \textit{þegna} ‘thanes’, \textit{þinga} ‘things’, \textit{þunra} ‘thunderstorms’, \textit{tuna} ‘towns’. Except for \textit{tuna}, these either have a \isi{proportional reading} (\textit{þegna} and \textit{þinga}) or denote an uncountable mass (\textit{þunra}). The exception is \textit{feala tuna} ‘many towns’, which occurs in a description of a flood (\textit{sæflod} `tide') immersing many towns. Here we cannot justify a \isi{proportional reading}, unless we construe it as `many of the towns that were near the sea'. However, as mentioned above, we would not expect the distribution to be completely consistent for the old variety, and we also have to remember that the copied part of the chronicle was originally written by several scribes over many years. 

In the interpolations, which, recall, were inserted by the scribe that copied the chronicle after the fire, there is only one instance of a \isi{genitive}, namely (\ref{minstra}), so here the scribe is presumably using his own grammar.\footnote{Odd Einar Haugen (p.c.) informs me that in scholarship on Old \ili{Norse}, the relation between the scribe’s own linguistic norm and the manuscript being copied is often discussed (see e.g. \cite{Maartensson13}), and it would be as expected to see the scribe using his own norm in the interpolations. See also \citet[15, Section 3.3.2]{BenskinLaing86} on how the scribe moves from copying visually to copying via ``the mind's ear", and \citet[500--501]{Thaisen2014} on how scribes introduced their own spelling when copying.}

\ea\label{minstra}
\gll fela minstra\\
many minster.\GEN.\PL{}\\
\glt ‘many minsters’ (OEng.800.699; OE Pet (interpolation))
\z

The rest are \isi{agreement} constructions, as in e.g. (\ref{ricemen}). However, at this point the case system is becoming blurred, so it might be that what we see in (\ref{ricemen}) is levelling of inflections rather than true case inflections.

\ea\label{ricemen}
\gll feola oðre rice men\\
many other.\NOM.\PL{} rich.\NOM.\PL{} man.\NOM.\PL{}\\
\glt ‘many other rich men’ (OEng.869.650; OE Pet (interpolation))
\z

When we arrive at the First Continuation, the \isi{genitive} is gone, as Table \ref{tab:felapet} shows, and by the Second Continuation, \textit{fela} itself has disappeared.\footnote{Obviously, this does not mean that \textit{fela} abruptly disappeared from the language altogether. The \citet{MED} provides attestations of \textit{fele}, but the word is now used in more restricted contexts and with more idiomatic meanings. There are no attestations in the \citet{OED} after 1598.} In the First Continuation we see examples like (\ref{shipmen}) and (\ref{tunes}). \textit{Tunes} is the new -\textit{(e)s} plural, which we have in Present-day \ili{English} as well. Note that the scribe who copied the chronicle up until 1121 was probably also responsible for the First Continuation (\cite[6--7]{BergsSkaffari07}), hence the similarity between the use of \textit{fela} in the interpolations and in the First Continuation.

\ea\label{shipmen}
feole shipmen\\
‘many shipmen’ (PPCME2, CMPETERB,42.16; ME Pet)
\z

\ea\label{tunes}
feola tunes\\
‘many towns’ (PPCME2, CMPETERB,47.172; ME Pet)
\z

The First Continuation also contains an example like (\ref{cnihte}), which was placed in the ``miscellaneous`" category, since it shows traces of \isi{genitive} case, but with the wrong endings; in Old \ili{English} it would have been \textit{fela oðra godra cnihta} in the \isi{genitive}, or \textit{fela oðre gode cnihtas} in the \isi{nominative} or accusative. So here there is clearly no steady case system in the scribe’s grammar. 

\ea\label{cnihte}
fela oðre godre cnihte\\
‘many other good knights’ (PPCME2, CMPETERB,45.110; ME Pet)
\z

In the Second Continuation there are no examples of \textit{fela}, but some of \textit{manig}, one of which is given in (\ref{munekes}).

\ea\label{munekes}
manie munekes\\
‘many monks’ (PPCME2, CMPETERB,57.494; ME Pet)
\z

What we see with the development of \textit{fela} in the \textit{Peterborough Chronicle} is language change in progress, and it can be argued that \textit{fela} shows the stages of the change that we would expect. In the copied part, there is \isi{variation} in the use of \isi{agreement} versus \isi{genitive} with \textit{fela}. In the interpolations to the Old \ili{English} part, which were inserted by the scribe that copied the chronicle at the beginning of the 12\textsuperscript{th} century, \textit{fela} occurs with \isi{agreement}, with one exception, so it probably reflects the scribe’s own grammar. The same scribe is at work in the First Continuation, where the \isi{genitive} disappears with \textit{fela}, and in the Second Continuation, \textit{fela} itself disappears. There was no longer any good reason to keep \textit{fela}, since the language already had the more frequent word \textit{manig}, and the two were no longer used in structurally different constructions. \textit{Fela} was changing from head to modifier, while \textit{manig} had always been a modifier.  

\section{Conclusion}\label{sect:concl}

This chapter is a study of the quantifiers \textit{fela} ‘many’ and \textit{manig} ‘many’, with particular focus on \textit{fela}. I have shown that \textit{fela} quite consistently occurs with a \isi{partitive} \isi{genitive} in Old \ili{English} rather than with a complement in \isi{agreement}, and can thus be argued to be a \isi{head-type quantifier} in Roehrs and Sapp’s \citeyearpar{RoehrsSapp18} terminology. The notable exceptions are Ælfric’s texts and the \textit{Peterborough Chronicle}, and the question was what conditioned the \isi{variation} in these texts. When Ælfric’s texts were studied in some detail, it emerged that the \isi{variation} is not random, but rather a result of semantic factors, with \textit{fela} occurring with \isi{agreement} when the construction has a \isi{cardinal reading} and the \isi{noun} is concrete, countable and agentive (though not necessarily all of these factors at the same time). The tendency for \textit{fela} with \isi{genitive} is to occur when the \isi{noun} is more abstract, non-agentive and has a \isi{proportional reading} (or sometimes possibly a mass reading). There are some exceptions, which is not surprising, considering that it is the older construction. The newer construction, i.e. \textit{fela} + \isi{agreement}, behaves in a consistent manner, while the older construction to some extent retains the possibility of \isi{variation}. In terms of general patterns of language change, the development of \textit{fela} that we see in Ælfric’s texts and the \textit{Peterborough Chronicle} is in line with the trajectory of change that we would expect. \textit{Fela} changes from being a head to becoming a \isi{quantifier} modifying a nominal head, and as such the expectation is that this change would happen first with concrete, countable, agentive nouns with a \isi{cardinal reading}.  

The only surprise is perhaps that this should be so evident in Ælfric’s texts in particular, and not in the other Old \ili{English} texts apart from the chronicle. However, as mentioned in Section \ref{ssect:RoehrsSapp}, \citet[417]{RoehrsSapp18} notice a change with some Old \ili{English} writers from \isi{genitive} to \isi{agreement} with respect to the complements of certain complex quantifiers. It is thus not inconceivable that individual writers can be trailblazers in this respect.

\textit{Fela} has, however, disappeared from \ili{English}, while its semantic competitor \textit{manig} survived. In the chronicle, \textit{fela} disappears completely towards the middle of the 12\textsuperscript{th} century. Attestations are found throughout the Middle \ili{English} period, but with a much more limited use. If we assume that \textit{fela} was changing from head to modifier, as Ælfric’s texts and the chronicle indicate, it was on its way to becoming structurally identical to \textit{manig}, which has always been a modifier. As inflections levelled and the case system disappeared, there were no longer distinct \isi{genitive} plural case inflections that could mark constructions with \textit{fela} as structurally different from constructions with \textit{manig}. Hence, the language had two words meaning the same thing and that were no longer in complementary distribution. One of them was destined to become superfluous, and that was \textit{fela}, since \textit{manig} was the more frequent word. 

\section*{Abbreviations}
\begin{tabularx}{.5\textwidth}{@{}lQ}
\textsc{acc} & accusative\\
\textsc{dat} &  {dative}\\
\textsc{def} & definite\\
\textsc{dem} &  {demonstrative}\\
\end{tabularx}%
\begin{tabularx}{.5\textwidth}{lQ@{}}
\textsc{gen} &  {genitive}\\
\textsc{nom} &  {nominative}\\
\textsc{pl} & plural\\
\textsc{sg} & singular\\
\end{tabularx}


\section*{Acknowledgements}
I am very grateful to two reviewers for valuable comments which improved the chapter substantially. I also thank members of the audience at the 21\textsuperscript{st} \textit{International Conference on \ili{English} Historical Linguistics} in Leiden (2021) for useful comments at an early stage. I have also presented this work for colleagues in the \ili{English} Language and Corpus Linguistics Research seminar at the University of Oslo, and thank them for their constructive feedback.  


%\section*{Contributions}
%John Doe contributed to conceptualization, methodology, and validation.
%Jane Doe contributed to the writing of the original draft, review, and editing.

{\sloppy\printbibliography[heading=subbibliography,notkeyword=this]}
\end{document}
