\documentclass[output=paper,colorlinks,citecolor=brown]{langscibook}
\ChapterDOI{10.5281/zenodo.10641193}
\author{Svetlana Petrova\orcid{}\affiliation{Bergische Universität Wuppertal}}
%\ORCIDs{}

\title[Strong and weak adjectival inflection in Old High German]{On the distribution of the strong and weak adjectival inflection in Old High German: A corpus investigation}

\abstract{Analyzing the evidence in the \textit{Referenzkorpus Altdeutsch}, the present chapter investigates the distribution of strong (incl. zero) and weak inflectional patterns of attributive adjectives in Old High German. Two types of datasets are considered, namely DPs containing a determiner-like marker of definiteness and indefiniteness, and bare DPs. The study revises previous accounts according to which the choice of the inflectional pattern of the adjective is driven by the interpretation of the DP in terms of (in)definiteness. It is shown that, in both datasets, the strong inflection occurs with any semantic type of DP. The weak inflection, on the other hand, one correlates with some grammatical and constructional factors, such as gradation and the adverbial use of nominalized adjectives derived from proper names by means of the morpheme -\textit{isk}. In addition, the analysis shows that the choice of strong patterns in definite DPs increases if the adjective is postnominal, supporting previous observations reported by \citet{Demske01}. Finally, it is shown that the modern German standard distribution according to which the choice of inflectional pattern depends on the presence or absence of overt inflection on the determiner begins to be established already in Old High German, especially in the domain of DPs headed by a possessive determiner.}


\IfFileExists{../localcommands.tex}{
   \addbibresource{../localbibliography.bib}
   \usepackage{langsci-optional}
\usepackage{langsci-gb4e}
\usepackage{langsci-lgr}

\usepackage{listings}
\lstset{basicstyle=\ttfamily,tabsize=2,breaklines=true}

%added by author
% \usepackage{tipa}
\usepackage{multirow}
\graphicspath{{figures/}}
\usepackage{langsci-branding}

   
\newcommand{\sent}{\enumsentence}
\newcommand{\sents}{\eenumsentence}
\let\citeasnoun\citet

\renewcommand{\lsCoverTitleFont}[1]{\sffamily\addfontfeatures{Scale=MatchUppercase}\fontsize{44pt}{16mm}\selectfont #1}
  
   %% hyphenation points for line breaks
%% Normally, automatic hyphenation in LaTeX is very good
%% If a word is mis-hyphenated, add it to this file
%%
%% add information to TeX file before \begin{document} with:
%% %% hyphenation points for line breaks
%% Normally, automatic hyphenation in LaTeX is very good
%% If a word is mis-hyphenated, add it to this file
%%
%% add information to TeX file before \begin{document} with:
%% %% hyphenation points for line breaks
%% Normally, automatic hyphenation in LaTeX is very good
%% If a word is mis-hyphenated, add it to this file
%%
%% add information to TeX file before \begin{document} with:
%% \include{localhyphenation}
\hyphenation{
affri-ca-te
affri-ca-tes
an-no-tated
com-ple-ments
com-po-si-tio-na-li-ty
non-com-po-si-tio-na-li-ty
Gon-zá-lez
out-side
Ri-chárd
se-man-tics
STREU-SLE
Tie-de-mann
}
\hyphenation{
affri-ca-te
affri-ca-tes
an-no-tated
com-ple-ments
com-po-si-tio-na-li-ty
non-com-po-si-tio-na-li-ty
Gon-zá-lez
out-side
Ri-chárd
se-man-tics
STREU-SLE
Tie-de-mann
}
\hyphenation{
affri-ca-te
affri-ca-tes
an-no-tated
com-ple-ments
com-po-si-tio-na-li-ty
non-com-po-si-tio-na-li-ty
Gon-zá-lez
out-side
Ri-chárd
se-man-tics
STREU-SLE
Tie-de-mann
}
   \boolfalse{bookcompile}
   \togglepaper[6]%%chapternumber
}{}

\begin{document}
\maketitle

\section{Introduction}\label{sec:intro}
\begin{sloppypar}
Adjectives in Old High \ili{German} (OHG, c. 750–1050) display two inflectional paradigms traditionally termed strong and weak, with the zero \isi{inflection} considered a subtype of the strong inflectional \isi{pattern} (\cite[170--171]{behaghel1923deutsche}; \cite[298--299]{Braune2018AHD}). This formal distinction is also known as \isi{dual} \isi{adjective} \isi{inflection}, a phenomenon shared by all early \ili{Germanic} varieties, with the strong inflectional \isi{pattern} being inherited from \ili{Indo-European} (IE) and the weak one representing a common \ili{Germanic} innovation.
\end{sloppypar}

The emergence of two inflectional paradigms of adjectives in \ili{Germanic} and the understanding of the principles underlying their distribution in the individual varieties are some of the most intriguing questions in \ili{Germanic} philology and historical linguistics (\cite[230]{Bammesberger90}, see also the overview in \cite[60--66]{Rehn2019}). Researchers investigating the rise of the weak paradigm have established a relation between the origins of this \isi{pattern} and a class of nominal expressions conveying a special meaning, namely, that they denote a referent identifiable by virtue of some characteristic property (\cite[119--121]{Osthoff1876}; \cite[191--192]{Delbruck09}; \cite[171]{behaghel1923deutsche}; \cite[297]{Braune2018AHD}). This observation gave rise to the assumption that the weak variant is associated with the identifiability of the referent and therefore with the \isi{definiteness} of the DP used to denote it. By \isi{contrast}, the strong inflectional \isi{pattern} was considered to be irrelevant regarding the semantic interpretation of the DP in early \ili{Germanic}, being found both in \isi{indefinite} as well as definite environments. \citet[189--190]{Delbruck09}, who presents and discusses \isi{comparative} evidence for modified bare nouns in Old \ili{English} and Old \ili{Norse}, states:

\begin{quotation}\relax
[E]in Substantivum, welches mit einem nach indogerm[anischer] Weise flektierten (starken) Adjektivum verbunden ist, kann unbestimmt und bestimmt gebraucht werden
\glt `A \isi{noun} which is combined with an \isi{adjective} inflecting in the IE (strong) \isi{pattern} can be used both as definite and \isi{indefinite}'. (\cite[189]{Delbruck09})
\end{quotation}

\citet[196]{klein2007semantischen}, providing additional references and summarizing the state of the art in the literature, concludes:

\begin{quotation}
Das starke Adjektiv war […] in der älteren Zeit hinsichtlich der Definitheit offenbar noch nicht festgelegt. Das ergibt sich aus seiner resthaften Verwendung auch in definiten NPs in den altgerm[anischen] Sprachen
\glt `Obviously, in the earliest period, the strong \isi{adjective} was not restricted regarding \isi{definiteness}. This follows from its residual use in definite NPs as well, in the early \ili{Germanic} languages'.
\end{quotation}

Evidence supporting the original semantic underspecification of the strong inflectional \isi{pattern} is also found in \ili{Gothic} \citep[143--144, 167]{ratkus2011} and continues to exist as late as in the system of \ili{Old Swedish} \citep{StrohWollinSimke14}.

Against the original situation found in \ili{Germanic}, OHG is assumed to have established a kind of complementary distribution of the two paradigms, depending on the interpretation of the DP in terms of (in)\isi{definiteness} (see \cite{Demske01}; \cite[297]{Braune2018AHD}). According to this view, the weak paradigm was associated with the \isi{definiteness} of the DP, while the strong one was restricted to \isi{indefinite} contexts. \citet{Hotzenkocherle68} shapes the term \textit{Sinnregel} (‘sense rule’) to account for this situation in OHG, as opposed to the so-called \textit{Formregel} (‘formal rule’) applying to Present-day \ili{German} (PDG), in which the type of \isi{inflection} of the \isi{adjective} depends on the morphological form of the \isi{determiner}. On the basis of these considerations, it is commonly assumed that \ili{German} underwent a change from a semantically driven distribution of \isi{adjectival inflection} in the earliest attestation to a morphologically driven one in the present-day stage of the language, although the precise time span during which this change must have taken place remains unclear.\footnote{\citet{Demske01}  suggests that this process must have taken place during the Early New High \ili{German} period (c. 1350--1650). On analyzing newly retrieved \isi{corpus} data, \citet{klein2007semantischen} convincingly shows that the PDG standard distribution in \isi{indefinite} contexts is in place much earlier, already in Middle High \ili{German} (c. 1050–1350) documents of the Upper \ili{German} area. Finally, \citet{Sahel2022}  shows that some additional principles underlying the present-day standard distribution (see Section \ref{sec:morphdrivenPDG}) are established much later, during the New High \ili{German} period (after 1650).}

However, there is data contradicting the strict applicability of the semantic principle of distribution of \isi{adjectival inflection} in the earliest vernacular attestation. The literature cites examples of strong adjectives found in definite environments in OHG (\cite[750]{Wilmanns09}; \cite[185]{behaghel1923deutsche}; \cite[68--69]{Heinrichs54}; \cite[68--70]{Dal14}; \cite[298]{Braune2018AHD}), suggesting that the original semantic underspecification of the strong \isi{pattern} in \ili{Germanic} continues to exist in this variety as well. In addition, \citet[70]{Demske01} observes that adjectives preceding their head \isi{noun} are more consistent with the semantic principle of distribution of \isi{adjectival inflection} than those following their head \isi{noun}. Finally, \citet{klein2007semantischen} considers an additional factor leading to cases of strong adjectives in definite DPs. He accounts for differences in the organization and spread of adjectival paradigms in Central and Upper \ili{German} in the Middle High \ili{German} period, showing that a levelling of the original paradigms in \ili{Central German} leads to a preference for the strong forms and a partial loss of the weak ones. According to him, the resulting presence of strong adjectives in definite environments in \ili{Central German} texts can already be found in late OHG documents of the respective dialectal area, see \citet[200]{klein2007semantischen}.

These observations suggest that there is a degree of variability in the spread of inflectional patterns of adjectives in OHG, which has not been addressed on a large scale by using the functionalities of \isi{corpus} search. The aim of this study is to uncover the degree of variability in the distribution of inflectional patterns of adjectives in OHG by evaluating the evidence retrievable from the \textit{Referenzkorpus Altdeutsch} (ReA 1.1, \cite{RefKorpAltD}).

The chapter is structured as follows. Section \ref{sec:morphdrivenPDG} discusses the principles of the morphologically driven distribution of \isi{adjectival inflection} in PDG, focusing on the situation in the standard variety, but also accounting for some deviations attested in non-standard, colloquial style. Section \ref{sec:distrEGOHG} describes the basic facts underlying the notion of a semantically driven alternation of adjectival inflectional in \ili{Germanic} and the respective situation in OHG, summarizing the statements of the previous literature. Section \ref{sec:corp} presents the methods and results of the \isi{corpus} study. Two basic types of datasets are distinguished: one involving demonstratives, possessive and \isi{indefinite} pronouns used as determiners, and one involving bare DPs, allowing to investigate the distribution of the inflectional patterns of adjectives independently of the semantic type and the morphological properties of a \isi{determiner}. Section \ref{sec:dataset} provides details on the various datasets, which are analyzed in Sections \ref{sec:DPDet} and \ref{sec:bareDP}. Section \ref{sec:conc} summarizes the results of the \isi{corpus} study.

\section{The principles of distribution of adjective paradigms in Present-day German and in early Germanic}

\subsection{The morphologically driven system of adjectival inflection in Present-day German}\label{sec:morphdrivenPDG}
With some well-known exceptions,\footnote{The \isi{inflection} is missing on
  adjectives in some idiomatic expressions of the type \emph{auf gut}-∅
  \emph{Glück} `randomly', but also on some loan adjectives like
  \emph{prima} `great', \emph{extra} `additional', and those denoting
  colours, such as \emph{lila} `purple', \emph{rosa} `rose', \emph{pink}
  `pink' etc. (see \emph{rosa}-∅ \emph{Brille} `pink spectacles'). Also,
  so-called toponymic formations ending in -\emph{er} such as
  \emph{Kieler}-∅ \emph{Bucht} `Bay of Kiel' are considered as a special
  class of adjectives which remain uninflected (see \citealp[347--349]{Duden16}; \citealp{Fuhrhop01}). On the lack of \isi{inflection} in the \ili{Alemannic} variety of \ili{German}, see \citeauthor{Rehn2019} (\citeyear{Rehn2019}, \citeyear{chapters/5Stacking} [this volume]).} adnominal adjectives in PDG obligatorily
inflect, agreeing in case, number and grammatical gender with the
respective head \isi{noun}. The distribution of the strong and the weak
paradigm is considered morphologically driven because the choice of the
respective variant is determined by the morphological form of the
accompanying \isi{determiner}, more precisely by the presence or absence of
overtly realized case, number and gender features on it. This is
illustrated in (\ref{ch6ex1})--(\ref{ch6ex3}) adapted from \citet{Rehn2019}, see also \citet[369--370]{Duden16}. The strong \isi{adjective} variant appears whenever no distinct
morphological features are realized on the \isi{determiner}, either because
the \isi{determiner} is missing (\ref{ch6ex1})\footnote{Forms of the \isi{genitive} singular
  masculine and neuter are exceptional in that they display weak
  \isi{inflection} although the \isi{determiner} is missing, as in \emph{gut-en
  Mut-es} instead of \emph{gut-es Mut-es} `in a good temper'. Note that
  until the beginning of the New High \ili{German} period, the strong
  \isi{inflection} was present here as well, see \citet[27--32]{Sahel2022} and the
  references therein.} or because it carries no such features itself
(\ref{ch6ex2}).\footnote{This pertains to the forms of the \isi{indefinite} \isi{article}
  \emph{ein} `a(n)', its negative variant \emph{kein} and the possessive
  \isi{determiner} series \emph{mein} `my', etc., in the \isi{nominative} singular
  masculine and the \isi{nominative}/accusative singular feminine and neuter \citep[369]{Duden16}. Some grammars consider the paradigm of adjectives
  following these determiners a mixed paradigm because it combines both
  weak and strong patterns. This is in \isi{contrast} to the \isi{inflection} of
  adjectives in determinerless (bare) environments in which the
  adjectives consistently display strong \isi{inflection}, as well as to
  adjectives in overtly definite environments where only the weak
  \isi{pattern} (ending in -\emph{e} and -\emph{en}) appears.} In the presence
of an overtly inflected \isi{determiner} of any type, the \isi{adjective} appears in
its weak and morphologically indistinctive variant, ending in -\emph{e}
in the \isi{nominative} singular of all genders as well as in the accusative
singular feminine and neuter, and in -\emph{en} in all remaining cases,
see (\ref{ch6ex3}).

\begin{exe}
\ex\label{ch6ex1}
\gll gut-\textbf{er} Wein\\
good-\textsc{m.nom.sg.str} wine.\textsc{m.nom.sg}\\
\glt `good wine'
\end{exe}

\begin{exe}
\ex\label{ch6ex2}
\gll ein gut-\textbf{er} Wein\\
\textsc{indef} good-\textsc{m.nom.sg.str} wine.\textsc{m.nom.sg}\\
\glt `a good wine'
\end{exe}

\begin{exe}
\ex\label{ch6ex3}
\gll ein-\textbf{es}/d-\textbf{es}/dies-\textbf{es} gut-\textbf{en} Wein-s\\
\textsc{indef/def/dem}--\textsc{m.gen.sg} good-\textsc{m.gen.sg.wk} wine-\textsc{m.gen.sg}\\
\glt `of a/the/this good wine'
\end{exe}

In contexts involving some kind of \isi{determiner}, a relevant property
concerning the spread of distinct morphological features in the DP in
PDG is observable, namely, that such features are coded only once, either
on the \isi{determiner}, or on the \isi{adjective}, in case the \isi{determiner} is
uninflected as in (\ref{ch6ex2}).\footnote{Again, exceptions to this \isi{pattern} are
  cases such as the \isi{genitive} singular masculine and neuter presented in
  footnote 3, where the \isi{adjective} has weak \isi{inflection} although there is
  no \isi{determiner}.} The notion underlying this kind of division of labour
between the \isi{determiner} and the adnominal \isi{adjective} is termed
\emph{single inflection} or \emph{monoinflection} (\emph{Monoflexion})
(see also \citealp[954]{Duden16}). At the same time, in the absence of a
\isi{determiner}, the features of the strong inflectional \isi{pattern} are equally
spread on each of the adjectives included in the DP, a phenomenon
traditionally termed \emph{parallel inflection} (\emph{Parallelflexion})
and illustrated in (\ref{ch6ex4}) (see also \citealp{Bildhauer2019}).

\begin{exe}
\ex\label{ch6ex4}
\gll mit gut-\textbf{em} spanisch-\textbf{em} Wein\\
with good-\textsc{m.dat.sg.str} \ili{Spanish}-\textsc{m.dat.sg.str} wine.\textsc{m.dat.sg}\\
\glt `with good \ili{Spanish} wine'
\end{exe}

However, there are well-known violations of both principles in informal
varieties of PDG. For example, the principle of monoinflection is
violated in the way exemplified in (\ref{ch6ex5}), in that an inflected \isi{determiner}
is followed by an \isi{adjective} displaying an ending of the strong paradigm,
thereby instantiating a case of \emph{double inflection}
(\emph{Doppelflexion}).\footnote{See \cite{Niebuhr21} for a corpus-based
  investigation of double \isi{inflection} in overtly \isi{indefinite} DPs from the
  end of the 15\textsuperscript{th} century to PDG.} In addition, the principle of
parallel \isi{inflection} exemplified in (\ref{ch6ex4}) is suspended in favour of the 
so-called \emph{variable inflection} (\emph{Wechselflexion}) (see
\citealp{Bildhauer2019}; \citealp{MunzbergHansen20}) in the way illustrated
in (\ref{ch6ex6}), whereby the strong inflectional ending required on all modifiers
in determinerless contexts is realized only once, on the leftmost one
of several coordinated adjectives, while the subsequent ones bear weak
\isi{inflection}.\footnote{The preposition \emph{trotz}, originally selecting
  the \isi{dative} case, is nowadays used both with the \isi{genitive} and the
  \isi{dative}. The latter, as in the example at issue, is considered more
  colloquial (see \citealp{Vieregge19}).}

\begin{exe}
\ex\label{ch6ex5}
\gll mit ein-\textbf{em} sachkundig-\textbf{em} Referenten\\
with \textsc{indef-m.dat.sg} professional-\textsc{m.dat.sg.str} guide.\textsc{m.dat.sg}\\
\glt `with a professional guide' \\ \emph{Metallsenioren besuchen Museum, Wochenspiegel online}, September
28th, 2021,
https://www.wochenspiegelonline.de/news/detail/metallsenioren-besuchen-museum
{[}visited November 19th, 2021{]}.
\end{exe}

\begin{exe}
\ex\label{ch6ex6}
\gll trotz fehlend-\textbf{em} direkt-\textbf{en} Beweis\\
despite lacking-\textsc{m.dat.sg.str} direct-\textsc{m.dat.sg.wk} proof.\textsc{m.dat.sg}\\
\glt `despite the lack of direct proof' (\cite[296, ex. (2)]{Bildhauer2019})
\end{exe}

As the examples suggest, the most common cases in which the principles
of monoinflection and parallel \isi{inflection} are violated in PDG are cases
involving the \isi{dative} singular of masculine and neuter nouns, i.e. those
cases in which the strong ending -\emph{em} alternates with the weak one
-\emph{en}. But other cases are not excluded, although they are less
frequent in corpora (see \citealp{Niebuhr21}). However, all these instances
concern language use and not the underlying system determining the
distribution of the \isi{adjectival inflection} in PDG.

\subsection{The distribution in early Germanic and in Old High German: The state of the art}\label{sec:distrEGOHG}
As pointed out in the introduction, OHG displays two paradigms of
adnominal adjectives: the strong one, including a subtype of uninflected
(zero) forms, and the weak one. The endings of the strong paradigm were
originally identical to those of the masculine and neuter nouns of the
\emph{a}-stems and of the feminine nouns of the \emph{ô}-stems,
including their \emph{ja-/jô}- and \emph{wa-/wô}- variants, with some
exceptions in which adjectives inflected like nouns of the \emph{i}- and
\emph{u}-stems \citep[289]{Braune2018AHD}. However, novel endings stemming from
the pronominal paradigm entered the system and replaced the nominal
ones, a process which was especially resilient in OHG in \isi{contrast} to the
remaining \ili{Germanic} varieties \citep[194--195]{klein2007semantischen}. The nominal
paradigm only survived in the \isi{nominative} singular of all genders, the
accusative singular, as well as the \isi{nominative} and accusative plural of
the neuter gender \citep[170]{behaghel1923deutsche}, where the original endings were
lost due to phonological reduction, giving rise to uninflected
(zero-inflected) forms, co-occurring with the new, pronominal ones (see
also \citealp[441, 733]{Wilmanns09}).

The weak paradigm of adjectives, in turn, shares the inflectional
behaviour of the nouns of the \emph{n}-stems of all genders, a fact that
plays a crucial role in explaining the emergence and the status of the
weak \isi{pattern} in \ili{Germanic}. Already in \ili{Indo-European}, the \emph{n}-suffix
was used to derive nouns with a special function, namely to refer to
persons by assigning them a characteristic property
expressed by the respective base word. Standard textbook examples are
formations using the \emph{n}-suffix in \ili{Greek} \emph{strábō\textbf{n}}
`squinter' derived from \emph{strabós} `squinting' or \ili{Latin}
\emph{cato\textbf{n}is}, the \isi{genitive} singular of \emph{cato} `the
shrewd one', derived from \emph{catus} `shrewd' \citep[298]{Braune2018AHD}.
\citet[46--47]{Osthoff1876}, \citet[196]{Delbruck09} and \citet[171]{behaghel1923deutsche}
provide many more examples of this type from \ili{Latin} and \ili{Greek} (see also
the extended discussion in \citealp[6--12]{Trutmann72}). Crucially, it is
assumed that the same word formation \isi{pattern} was also used in \ili{Germanic},
i.e., \ili{Germanic} also employed the \emph{n}-suffix to derive nouns
referring to individuals, making these individuals distinguishable by
virtue of some characteristic property. A significant part of these
formations were nominalized adjectives, often used as by-names of
persons or as parts of proper names referring to places, and attested in
all early \ili{Germanic} varieties (\citealp[746]{Wilmanns09}; \citealp{Kogel89}). This distributional observation gave rise to the
following way of reasoning: Because individuation was a core function of
appositive nouns derived by way of \emph{n}-suffixation, and because the
identifiability of the referent is linked to the \isi{definiteness} of the DP
used to denote it, adjectives sharing the inflectional behaviour of the
nouns of the \emph{n}-stems became associated with \isi{definiteness} as their
inherent property. Notably, this process is assumed to have taken place
prior to the emergence of a system of determiners and independently of
the presence of demonstratives as overt markers of \isi{definiteness}. The
association of appositional adjectives with individuation and
\isi{definiteness}, and the subsequent spread of their inflectional behaviour
to adjectives in definite environments is taken to represent the turning
point in the process of the emergence of the weak inflectional \isi{pattern}
of adjectives in \ili{Germanic}, and of \isi{dual} \isi{adjectival inflection} as a
whole.\footnote{But see \citet{Trutmann72} and \citet{ratkus2011} on alternative
  scenarios regarding the rise of \isi{dual} \isi{inflection} in \ili{Germanic}. The more
  recent literature on the rise of the weak \isi{adjectival inflection} is
  given and summarized in \citet[footnote 1]{ratkus2011}. See also \citet{ratkus2018weak} who argues in favour of a more general semantics of weak
  adjectives in bare DPs in \ili{Gothic} and in early \ili{Germanic}. According to
  him, only weak adjectives in determined DPs are firmly associated with
  \isi{definiteness}.}

According to standard textbooks, the use of the weak paradigm of
adjectives is already strongly associated with the presence of some
overt marker of \isi{definiteness} in OHG (see \citealp[183--184]{behaghel1923deutsche}; \citealp[68]{Dal14}; \citealp[297, 309]{Braune2018AHD}). Some sporadic instances of weak
adjectives in determinerless DPs are still found in formulaic
expressions involving proper names, e.g. \emph{druhtîn nerrend-\textbf{o} Christ} `Lord, the
saving Christ' (Is. 17, 15. 11, cit. in \citealp[748]{Wilmanns09}), being
considered as remnants of the original use of weak adjectives in bare
definite contexts in \ili{Germanic}.\footnote{This use of the weak paradigm of
  adjectives is preserved, e.g. in modern \ili{Danish} \citep{HaberlandHeltoft2008}.} The literature on OHG also suggests that, once the weak
\isi{inflection} was associated with \isi{definiteness}, it was extended to
adjectives in DPs introduced by \isi{demonstrative} (or possessive) pronouns
as markers of \isi{definiteness}, while the strong \isi{pattern} became restricted
to \isi{indefinite} contexts. In the course of this process, the use of the
strong and weak \isi{pattern} established a complementary distribution,
depending on the semantic class of the accompanying \isi{determiner}, irrespective of its morphological form.

Studying the diachronic development of the DP in the history of \ili{German},
\citet{Demske01} also adopts this view. She describes the distribution of
\isi{adjectival inflection} in OHG as a semantically driven one, as according
to her, the type of \isi{inflection} depends on the semantic interpretation of
the DP in terms of (in)\isi{definiteness}, rather than on the morphological
form of the accompanying \isi{determiner} (see \citealp[68]{Demske01}). A basic
consideration is that, in an example like (\ref{ch6ex7}), where the possessive
\isi{determiner} is considered a marker of \isi{definiteness} but carries no
morphological features on its own, the \isi{adjective} nevertheless displays
weak \isi{inflection}, contrary to the distribution in PDG.\footnote{All examples are cited according to ReA 1.1, including those taken from the previous literature.}

\begin{exe}
\ex\label{ch6ex7}
\gll mîn liob-\textbf{o} sun\\
my.\textsc{m.nom.sg} dear-\textsc{m.nom.sg.wk} son.\textsc{m.nom.sg}\\
\glt `my dear son' (T 14.5, cit. in \cite[67, ex. (39a)]{Demske01}) 
\end{exe}

To illustrate that the distribution of \isi{adjectival inflection} in OHG is
semantically driven, \citet{Demske01} first provides data from contexts
involving overt adnominal pronouns used as determiners. She demonstrates
that weak adjectives appear in overtly definite environments like those
headed by possessive or \isi{demonstrative} determiners, see (\ref{ch6ex7}) above, (\ref{ch6ex8a}) and (\ref{ch6ex8b}), whereas the strong inflectional \isi{pattern}, including its zero
variant, occurs in overtly \isi{indefinite} environments such as those shown
in (\ref{ch6ex9a}) and (\ref{ch6ex9b}). \citet{Demske01} also refers to the fact that in inflected
\isi{indefinite} contexts as in (\ref{ch6ex10}), adjectives in OHG display strong
inflectional endings, violating the principle of monoinflection typical
of the system of standard PDG.\footnote{\citet{Sahel2022} shows that the
  principle of monoinflection becomes the dominant \isi{pattern} as late as in
  the 18\textsuperscript{th} century.}

\begin{exe}
\ex\label{ch6ex8}\begin{xlist}
\ex\label{ch6ex8a}
\gll thes-\textbf{er} firntatig-\textbf{o} mán\\
\textsc{dem}-\textsc{m.nom.sg} sinful-\textsc{m.nom.sg.wk} man.\textsc{m.nom.sg}\\
\glt `this sinful man' (T 118.2, cit. in \cite[67, ex. (38b)]{Demske01})
\ex\label{ch6ex8b} 
\gll [in] thi-\textbf{z} írthisg-\textbf{a} dál\\
[in] \textsc{dem}-\textsc{n.acc.sg} earthy-\textsc{n.acc.sg.wk} valley.\textsc{n.acc.sg}\\
\glt `into this valley on Earth' \\ (O V.23.102, cit. in \cite[67, ex. (38c)]{Demske01})
\end{xlist}
\end{exe}

\begin{exe}
\ex\label{ch6ex9}\begin{xlist}
\ex\label{ch6ex9a}
\gll ein arm-\textbf{az} wíb\\
\textsc{indef.n.acc.sg} poor-\textsc{n.acc.sg.str} woman.\textsc{n.acc.sg}\\
\glt `a poor woman' (O II.14.84, cit. in \cite[67, ex. (42a)]{Demske01})
\ex\label{ch6ex9b} 
\gll sum árm betalari\\
a.certain.\textsc{m.nom.sg} poor.\textsc{m.nom.sg.∅} beggar.\textsc{m.nom.sg}\\
\glt `a certain poor beggar' (T 107.1, cit. in \cite[67, ex. (42c)]{Demske01})
\end{xlist}
\end{exe}

\begin{exe}
\ex\label{ch6ex10}
\gll mít éin-\textbf{emo} rôt-\textbf{emo} tûoch-e\\
with \textsc{indef-n.dat.sg} red-\textsc{n.dat.sg.str} scarf-\textsc{n.dat.sg}\\
\glt `with a red scarf' (N MC 56.15, cit. in \cite[67, ex. (67b)]{Demske01})
\end{exe}

In addition, \citet{Demske01} demonstrates that the above shown correlation
between the semantic interpretation of the DP and the inflectional type
of the \isi{adjective} also applies in determinerless contexts. Especially
important for her analysis is the use of the weak inflectional \isi{pattern}
in vocatives (\ref{ch6ex11}), superlatives (\ref{ch6ex12}) and nouns with unique reference
(\ref{ch6ex13}). In the seminal typology of definite expressions proposed by \citet{Lobner85}, these classes of DPs represent the type of semantic
\isi{definiteness}, i.e. of expressions denoting referents which are
identifiable on the basis of uniqueness and world knowledge. The
opposite category is that of pragmatic \isi{definiteness}, i.e. of DPs which
acquire definite interpretation on the basis of previous mention.
According to \citet{Demske01}, pragmatic and semantic \isi{definiteness} behave
differently in the process of \isi{grammaticalization} of the definite
\isi{determiner} in \ili{German}. While anaphoric DPs systematically appear with a
\isi{determiner} already in OHG, representatives of the semantic type of
\isi{definiteness} reject the \isi{determiner} until the end of this period.
Therefore, \citet{Demske01} concludes that the weak \isi{adjectival inflection}
in bare DPs expressing the semantic type of \isi{definiteness} acts as a
substitute of the \isi{definite determiner} during the OHG period:

\begin{exe}
\ex\label{ch6ex11}
\gll líob-\textbf{o} man\\
dear-\textsc{m.nom.sg.wk} man.\textsc{m.nom.sg}\\
\glt `dear man' (O II.7.27, cit. in \cite[67, ex. (40a)]{Demske01})
\end{exe}

\begin{exe}
\ex\label{ch6ex12}
\gll in ira bárm si sazta barn-o bézist-\textbf{a}\\
in her lap she set child-\textsc{n.gen.pl} best-\textsc{n.acc.sg.wk}\\
\glt `onto her lap, she put the loveliest one of all children' \\ (O I.13.10, cit. in \cite[67, ex. (44)]{Demske01})
\end{exe}

\begin{exe}
\ex\label{ch6ex13}
\gll fon hímilisg-\textbf{en} líaht-e\\
from heavenly-\textsc{n.dat.sg.wk} light-\textsc{n.dat.sg}\\
\glt `by heavenly light' (O I.12.4, cit. in \cite[67, ex. (45a)]{Demske01})
\end{exe}

But at the same time, there is variability in the data, which challenges
the strict applicability of the semantic principle in OHG. \citet[70]{Demske01} accounts for some inconsistencies by taking into account differences
in the positional realization of adjectives relative to their head
nouns. She observes that the correlation between the semantic
interpretation of the DP and the type of \isi{inflection} on the \isi{adjective} is
more systematically established in DPs displaying \isi{prenominal} modifiers
than in those displaying \isi{postnominal} ones. This is illustrated by the
minimal pair in (\ref{ch6ex14})=(\ref{ch6ex7}) and (\ref{ch6ex15}). In both cases, the DP is headed by
the same type of \isi{determiner}, namely the possessive one. However, the
\isi{inflection} of the adjectives differs. Only the \isi{prenominal} one displays
the weak inflectional \isi{pattern}, whereas the corresponding \isi{postnominal} one
bears strong \isi{inflection}, therefore violating the semantic principles of
distribution of \isi{adjectival inflection}:

\begin{exe}
\ex\label{ch6ex14}
\gll mîn liob-\textbf{o} sun = (\ref{ch6ex7})\\
 my.\textsc{m.nom.sg} dear-\textsc{m.nom.sg.wk} son.\textsc{m.nom.sg}\\
\glt  `my dear son' (T 14.5, cit. in \cite[67, ex. (39a)]{Demske01})
\end{exe}

\begin{exe}
\ex\label{ch6ex15}
\gll min sun leob-\textbf{ar}\\
 my.\textsc{m.nom.sg} son.\textsc{m.nom.sg} dear-\textsc{m.nom.sg.str}\\
\glt  `my dear son' (T 91.3, cit. in \cite[67, ex. (46a)]{Demske01})
\end{exe}

This explanation, however, fails to account for examples involving
\isi{prenominal} strong adjectives in definite environments, as those cited in
the philological literature (see \citealp[750]{Wilmanns09}; \citealp[185]{behaghel1923deutsche};
\citealp[68--69]{Heinrichs54}; \citealp[68--70]{Dal14}; \citealp[298]{Braune2018AHD}). A
representative example is given in (\ref{ch6ex16}). Note that the \isi{adjective}
modifies a \isi{noun} with unique reference (\emph{sunna} `the sun'), a
representative of the semantic type of \isi{definiteness}.

\begin{exe}
\ex\label{ch6ex16}
\gll thiu éwinig-\textbf{u} súnna\\
\textsc{def.f.nom.sg} eternal-\textsc{f.nom.sg.str} sun.\textsc{f.nom.sg} \\
\glt `the eternal sun' (O IV.35.43, cit. in \cite[69]{Heinrichs54})
\end{exe}

Additional evidence challenging the semantic principle of distribution
of \isi{adjectival inflection} comes from \isi{variation} in multiple \isi{modification}.
The examples cited in (\ref{ch6ex17})--(\ref{ch6ex19}) and found by way of \isi{corpus} search show
that adjectives varying regarding their inflectional features may
alternate within one and the same DP, i.e. following the same semantic
type of \isi{determiner}. Note that this alternation equally applies to
adjectives appearing both before (\ref{ch6ex17}) and after (\ref{ch6ex18}) the head \isi{noun}. The
\isi{variation} increases if we take into account cases of possessive
adjectives following a \isi{definite determiner}, see (\ref{ch6ex19}).\footnote{One might
  assume that the weak \isi{inflection} of \emph{einag} `single' in (\ref{ch6ex17})
  results from analogy with the \isi{numeral} \emph{ein}, which displays the
  weak \isi{pattern} exclusively if used in the meaning `single, alone'
  (\citealp[322]{Braune2018AHD}). But note that this does not
  apply to the derivational forms \emph{einag, einig} or \emph{eining}
  (see \citealp[347]{Braune2018AHD}).}\footnote{I follow ReA 1.1 in interpreting
  the forms \emph{libhafte} and \emph{redohafte} in (\ref{ch6ex18}) as inflected,
  displaying the weak ending of adjectives sharing the paradigm of
  \emph{jung} `young' (see \citealp[305]{Braune2018AHD} on adjectives derived by the
  suffix-like element -\emph{haft}(\emph{ig}) in OHG), contra \citet{klein2007semantischen}, who lists this example as one involving zero inflected
  adjectives, see the appendix sec. A 3.1.1. in \citet[217]{klein2007semantischen}.}

\begin{exe}
\ex\label{ch6ex17}
\gll thin-\textbf{an} uuar-\textbf{an} einag-\textbf{un} sun\\
your-\textsc{m.acc.sg} true-\textsc{m.acc.sg.str} only-\textsc{m.acc.sg.wk} son.\textsc{m.acc.sg}\\
\glt  `your true and single son' (MH\_Murb.H.XXVI (edition 66--76))
\end{exe}

\begin{exe}
\ex\label{ch6ex18}
\gll Ter mennisco ist ein ding libhaft-\textbf{e}, redohaft-\textbf{e}, totig lachenn-\textbf{es} mahtig\\
 \textsc{def} human is \textsc{indef.n.nom.sg} thing.\textsc{n.nom.sg} vivid-\textsc{n.nom.sg.wk} reasonable-\textsc{n.nom.sg.wk} mortal.\textsc{n.nom.sg.∅} laughing-\textsc{n.gen.sg} capable.\textsc{n.nom.sg.∅}\\
\glt  `The human being is something vivid, reasonable, mortal, capable of
laughing.' (DD\_DeDefinitione (edition 168--180))
\end{exe}

\begin{exe}
\ex\label{ch6ex19}
\gll th-\textbf{az} mín-\textbf{az} heil-\textbf{a} múat\\
 \textsc{dem-n.nom.sg} my-\textsc{n.nom.sg} joyful-\textsc{n.nom.sg.wk} temper.\textsc{n.nom.sg}\\
\glt  `this joyful temper of mine' (O\_Otfr.Ev.2.13 (edition 189--191))
\end{exe}

Examples of this kind suggest that there are violations of the semantic
principle of distribution of \isi{adjectival inflection} in OHG which go
beyond the ones accounted for in the previous literature. The present
\isi{corpus} study aims to re-examine the validity of the semantic principle
of distribution of strong and weak adjectives in OHG, searching the OHG
data in the \emph{Referenzkorpus Altdeutsch} and using the
functionalities of the searching platform ANNIS.





\section{Corpus study}\label{sec:corp}

\subsection{The datasets}\label{sec:dataset}
The present study distinguishes two types of datasets, differing
regarding the presence or absence of an overt \isi{determiner}.\footnote{It is
  controversial whether OHG displayed a system of definite and
  \isi{indefinite} determiners comparable to the PDG one (see the most recent
  investigation by \cite{Flick2020} on the rise of the \isi{definite determiner}, and \cite{Petrova2015}, who argues that \textit{ein} was determiner-like and clearly distinguishable from the \isi{numeral} `one'),
  but it is well-known that different types of \isi{demonstrative} and
  \isi{indefinite} pronouns were used as markers of the semantic properties of
  the respective DP. In the face of the latter observation, the question
  is how the semantic class and the morphological properties of the
  accompanying adnominal \isi{pronoun} influenced the type of \isi{inflection}
  realized on the \isi{adjective}. This means that, for the time being, the
  structural interpretation of the \isi{pronoun} in terms of a representative
  of some class of functional element (e.g. D) heading the DP and
  taking an NP as its complement, will be ignored.} The first one
involves DPs displaying some kind of \isi{determiner}, thus allowing an investigation of how the use of the various inflectional patterns depends on
the semantic class of the \isi{determiner} on the one hand, and on the
presence of \isi{inflection} on it on the other. The second dataset involves
bare DPs in which the morphological features on the modifier are not
influenced by any property of the \isi{determiner}.

Both datasets include \isi{prenominal} and \isi{postnominal} modifiers as well as
instances of nominalized adjectives. In addition, not only canonical
adjectives are tested but also past and present participles used as
modifiers of nominal heads, or in \isi{nominalization} constructions. For each
dataset, the frequency of inflectional patterns of modifiers is
determined and related to the semantic interpretation of the DP. The
results of the \isi{corpus} search and the semantic analysis are presented and
discussed in the subsections below. For the sake of consistency, the
\isi{database} is restricted to DPs involving single \isi{modification}.
Modification by way of two or more coordinated categories, as
exemplified in (\ref{ch6ex17})--(\ref{ch6ex19}), is left aside for further research.

\subsection{DPs containing a determiner}\label{sec:DPDet}
\begin{sloppypar}
The following semantic classes of determiners distinguished in ReA 1.1
and tagged at the level of part of speech (pos) have been considered in
the present analysis: i) the \isi{indefinite} \isi{determiner} \emph{ein} `a(n)'
tagged as DIA (\isi{indefinite} \isi{determiner}), as well as its negative counterpart \emph{nihein, nohein,
niheinig} etc. `no one' tagged as DINEG (negative \isi{indefinite} \isi{determiner}); ii) the \isi{definite determiner} of
the series of the simple \isi{demonstrative} \isi{pronoun} \emph{der} `the' tagged
as DDA (\isi{demonstrative} \isi{determiner}), and iii) the possessive pronouns of the series \emph{min} `my',
etc., interpreted as possessive determiners and tagged as DPOS (possessive \isi{determiner}). In
addition, the class of \isi{indefinite} DPs was extended to the adnominal
\isi{indefinite} pronouns \emph{sum/sumalih} `a certain one' used as markers
of indefinitess of the DP.
\end{sloppypar}

Table \ref{ch6t1} gives an overview of the occurrences of the inflectional
patterns of strong, zero and weak adjectives in DPs headed by the three
types of determiners distinguished above. The figures in Table~\ref{ch6t1} show that the strong \isi{pattern}, both in its zero and pronominal variant, is widely preferred in \isi{indefinite} DPs (94.2\%), whereas the weak \isi{pattern} predominates in the remaining types of DPs, amounting to 87.7\% in definite DPs and 69.4\% in possessive DPs. This
distribution confirms the standard opinion according to which in OHG,
the type of \isi{inflection} of adjectives depends on the semantic type of the
\isi{determiner}.

\begin{table}
\begin{tabular}{ll *3{rr}}
\lsptoprule
 & & \multicolumn{2}{c}{\textsc{indef}} & \multicolumn{2}{c}{\textsc{def}} & \multicolumn{2}{c}{\textsc{poss}}\\\cmidrule(lr){3-4}\cmidrule(lr){5-6}\cmidrule(lr){7-8}
\multicolumn{2}{l}{Inflection} & \multicolumn{1}{c}{$n$} & \multicolumn{1}{c}{\%}     & \multicolumn{1}{c}{$n$}   & \multicolumn{1}{c}{\%}     & \multicolumn{1}{c}{$n$} & \multicolumn{1}{c}{\%}\\\midrule
\multicolumn{2}{l}{strong}          & 113 & 94.2 &   226 & 12.3 &  74 & 30.6\\
              & zero                &  59 & 49.2 &    16 &  0.9 &  13 &  5.4\\
              & pronominal          &  54 & 45.0 &   210 & 11.4 &  61 & 25.2\\
\multicolumn{2}{l}{weak}            &   7 &  5.8 & 1\,608 & 87.7 & 168 & 69.4\\\addlinespace
\multicolumn{2}{l}{Total}           & 120 &        & 1\,834 &        & 242 & \\
\lspbottomrule
\end{tabular}
\caption{Strong (zero and pronominal) and weak adjectival inflection in
DPs headed by an \isi{indefinite}, definite, or possessive \isi{determiner} in ReA 1.1
($n=2{,}196$)}\label{ch6t1}
\end{table}




But at the same time, the figures in Table \ref{ch6t1} suggest that there are
examples violating the semantic principle of adjectival distribution. On
the one hand, there is evidence for weak adjectives in \isi{indefinite}
contexts, which is surprising, given the previous knowledge about the
distribution of this inflectional \isi{pattern} in early \ili{Germanic}. On the
other hand, there is evidence for strong adjectives in definite
contexts, which is as expected in the face of the previous literature,
but which demands an explanation, given that the frequencies of the individual
patterns in definite and possessive DPs differ considerably. At first
glance, it seems that definite DPs are more consistent with the semantic
principle of distribution of \isi{adjectival inflection} in OHG than
possessive DPs because the former correlate with the weak \isi{inflection}
more strictly than the latter ones. This observation, however, must be
corroborated by looking into the effect of the presence of \isi{inflection} on
the possessive \isi{determiner} and the \isi{linear order} in the DP in both
datasets, see Sections \ref{sec:detinfl} and \ref{sec:linorder} below.

The following subsections will take a closer look at the patterns
attested in the individual classes of DPs, focusing on those cases which
contradict the semantic rule of distribution of \isi{adjectival inflection} in
OHG. In addition, some factors potentially explaining these
inconsistencies will be addressed, such as the presence of \isi{inflection} on
the \isi{determiner} and the positional realization of the modifier relative
to the respective head \isi{noun}.

\subsubsection{The indefinite contexts}\label{sec:indef}
According to the numbers in Table \ref{ch6t1}, modifiers in DPs introduced by an
in\isi{definite determiner} most often display a type of the strong
inflectional \isi{pattern}. But in addition, the \isi{corpus} search reveals that
there are cases of weak adjectives in \isi{indefinite} contexts as well. Let
us examine the properties of these examples in more detail.

There are seven instances of weak adjectives in \isi{indefinite} DPs in the
data. All share the property that they occur in DPs in the masculine or neuter
singular. Two of the examples, given in (\ref{ch6ex20a}) and (\ref{ch6ex20b}), involve DPs in the masculine
\isi{nominative} singular, i.e., the weak \isi{adjective} follows an uninflected
\isi{determiner}.

\begin{exe}
\ex\label{ch6ex20}\begin{xlist}
\ex\label{ch6ex20a}
\gll Sum iung-\textbf{o} folgeta imo\\
 a.certain.\textsc{m.nom.sg} young-\textsc{m.nom.sg.wk} followed him\\
\glt `A certain young one followed him.' (T\_Tat185 (edition 251--262))
\ex\label{ch6ex20b} 
\gll da saz ein plint-\textbf{e}\\
there sat \textsc{indef.n.nom.sg} blind-\textsc{m.nom.sg.wk}\\
\glt `A blind man was sitting there.' \\ (APB\_PredigtsammlungB (edition 1883--1894))
\end{xlist}
\end{exe}

The example in (\ref{ch6ex20a}) is ambiguous because \emph{iungo} can be
interpreted both as a \isi{noun} of the masculine \emph{n}-stems meaning
`young man, boy', also accounted for in standard dictionaries of OHG
(e.g. \citealp[170]{Schutzeichel12})\footnote{See also the entry for
  \emph{jungo} in the online version of the OHG dictionary:
  \url{http://awb.saw-leipzig.de/cgi/WBNetz/wbgui\_py?sigle=AWB\&lemma=jungo},
  visited on May 7\textsuperscript{th}, 2022.} and a nominalized variant
of the \isi{adjective} \emph{jung} `young'. In ReA, \emph{iungo} is tagged
three times as a \isi{noun} and once as an \isi{adjective}, i.e. in the example in (\ref{ch6ex20a}), but it is very likely that (\ref{ch6ex20a}) involves the \isi{noun} \emph{iungo}. In (\ref{ch6ex20b}),
however, the form is unambiguous because the lemma \emph{blind} is
attested only as an \isi{adjective} in the dictionaries, and never as a \isi{noun}
as well, differently from \emph{iungo}. The \isi{nominalization} of this
\isi{adjective} results in a \isi{pattern} that is exceptional not only because it
contradicts the semantic principle of distribution of adjectival
\isi{inflection}, but also because it is also incompatible within the
morphologically driven one in PDG. Note that in PDG, weak adjectives
following an uninflected \isi{indefinite} \isi{determiner} are
ungrammatical.\footnote{The respective form would be \emph{ein}
  *\emph{Blind-e} `a blind man' instead of \emph{ein Blind-er},
  requiring the strong \isi{inflection} on the nominalized \isi{adjective}.} At the
same time, the figures represented in \citet[202]{klein2007semantischen} suggest that
this \isi{pattern} is not exceptional in the historical stages of \ili{German}, as
some additional instances of weak adjectives following uninflected
\emph{ein} can be found in Upper and \ili{Central German} texts of the Middle
High \ili{German} period.

Consider that the property unifying the examples in (\ref{ch6ex20a}) and (\ref{ch6ex20b}) is the
individualizing function of the DPs involved, i.e., both cases involve
secondary formations which describe an individual as distinguishable by
virtue of the property expressed by the base word. Recall that word
formations of this type share the inflectional behaviour of the nouns of
the \emph{n}-stems, and that it is assumed that the weak adjectival
paradigm evolved out of nominalizations of this type, displaying
\isi{definiteness} as its inherent property. Note, however, that while the two
examples fit perfectly well into the \isi{nominalization} \isi{pattern}, they are
overtly \isi{indefinite}, suggesting that the respective word formation
\isi{pattern} was not restricted to DPs which were inherently definite.

\begin{sloppypar}
In the remaining five instances, the weak \isi{adjective} follows an inflected
in\isi{definite determiner}. In four of these, the DP is in the masculine accusative
singular, as shown in (\ref{ch6ex21a}), and in one it is in the neuter \isi{dative} singular,
see (\ref{ch6ex21b}).
\end{sloppypar}

\begin{exe}
\ex\label{ch6ex21}\begin{xlist}
\ex\label{ch6ex21a}
\gll Án dero uuínsterun trûog er éin-\textbf{en} rôt-\textbf{en} skílt\\
on \textsc{def} right wore he \textsc{indef-m.acc.sg} red-\textsc{m.acc.sg.wk} buckler.\textsc{m.acc.sg}\\
\glt `He wore a red buckler on his right arm.'\\(N\_Mart\_Cap.I.64-72 (edition 1805--1816))
\ex\label{ch6ex21b} 
\gll ûfen éin-\textbf{emo} blánch-\textbf{en} róss-e\\
on \textsc{indef-n.dat.sg} white-\textsc{n.dat.sg.wk} horse-\textsc{n.dat.sg}\\
\glt `on a white horse' (N\_DeCon\_II\_63--66 (edition 508--519))
\end{xlist}
\end{exe}

All examples are found in texts of the late OHG writer Notker. The
\isi{corpus} search reveals that in Notker's writings, -\emph{en} is the
default inflectional ending of adjectives in the accusative singular
masculine, appearing in 202 of the total of 207 instances of this
form.\footnote{See also Klein's \citeyearpar[291]{klein2007semantischen} remark on forms of the
  accusative singular masculine in Notker's work: ``Bei Notker sind
  starke und schwache Flexion nicht mehr unterscheidbar''(=`Strong and
  weak \isi{inflection} is undistinguishable in work by Notker').
  Unfortunately, the annotation in the \isi{corpus} is inconclusive, tagging
  57 of these cases as weak and 145 cases as strong. Needless to say,
  all hit lists that the \isi{corpus} produced were checked manually while
  compiling the data and statistics of this chapter.} Very probably,
this ending results from formal overlapping of the original strong
ending -\emph{an} and the weak one -\emph{un}/-\emph{in}\footnote{In the
  texts written by Notker and included in ReA, the adjectival ending
  -\emph{un} in the masculine accusative singular occurs once, annotated
  as weak, and the ending -\emph{in} is found four times, all annotated as
  strong.} in the course of phonological reduction of vowels in
unaccented syllables to \emph{schwa,} taking place toward the end of the
OHG period and leading to the loss of formal distinctions in large parts
of the inflectional system of the language. Consequently, the forms of
the masculine accusative singular ending in -\emph{en} are ambiguous,
and we cannot tell whether the \isi{adjectival inflection} is strong or weak
in the respective examples. But in the case of the neuter \isi{dative} singular in (\ref{ch6ex21b}), we observe a weak form ending in -\emph{en} that is
sufficiently distinguishable from the strong one ending in
-\emph{em(o)}, still present in texts by Notker. This means that
by virtue of this example, we find conclusive evidence suggesting that
the weak paradigm of adjectives starts to spread after inflected
\isi{indefinite} determiners in the late OHG period. However, strong and weak
forms after inflected \isi{indefinite} determiners continue to compete for
centuries. \citet{Demske01} shows that this \isi{variation} is present as late as 
the Early New High \ili{German} period. According to \citet{Sahel2022}, multiple \isi{inflection} is still present until the 18\textsuperscript{th} century.

\subsubsection{The definite and possessive environments}\label{sewc:defposs}
The numbers in Table \ref{ch6t1} show that weak adjectives represent the most
common category in DPs headed by a definite and a possessive \isi{determiner}.
However, at the same time, strong adjectives, both pronominal and zero ones,
are also possible in these two classes of DPs. In addition, the figures reveal
significant differences regarding the frequency of strong and weak
adjectives in definite and possessive DPs. This raises the question of
whether the semantic class of the \isi{determiner} is the single factor
determining the distribution of inflectional patterns in these domains.

Let us start with the interpretation of zero-inflected adjectives in
definite and possessive DPs. In both types of DPs, zero inflected
adjectives constitute the most infrequent option. But there are
quantitative and qualitative differences regarding the presence of
zero-inflected adjectives in definite and possessive contexts. First,
with a frequency of 0.9\%, zero-inflected adjectives are practically
non-existent in definite DPs, while their frequency in possessive DPs is
higher, amounting to 5.4\%. Second, there is a difference regarding the
lexical inventory of adjectives displaying zero-\isi{inflection} in these two
groups of DPs. In definite DPs, seven of the total of 16 occurrences are
cases of the \isi{adjective} \emph{frono} `divine, kingly', which is
indeclinable (see \citealp[285]{Braune2018AHD} with references). An example is
provided in (\ref{ch6ex22a}). In the remaining cases, the adjectives are
declinable. But they display uninflected forms in poetic texts, probably
due to metrical considerations or where they are used in a rhyme position,
as the examples in (\ref{ch6ex22b}) and (\ref{ch6ex22c}) suggest.\footnote{The \isi{adjective} \emph{sconi}
  `beautiful, good' in (\ref{ch6ex22b}), and also the majority of declinable
  zero-inflected adjectives in definite DPs, is a representative of the
  class of adjectives of the \emph{ja-/jo}-stem. The uninflected form
  ends in -\emph{i}, see \citet[289]{Braune2018AHD}. The respective weak form
  ends in -\emph{o} in the masculine \isi{nominative} singular as well as
  -\emph{a} in the feminine \isi{nominative} singular and the neuter \isi{nominative} and
  accusative. An example is given in (i). 
  
\begin{exe}
\exi{(i)}\label{ch6exi}\gll th-\textbf{az} scon-\textbf{a} séltsani \\
\textsc{def-n.acc.sg} good-\textsc{n.acc.sg.wk} wonder.\textsc{n.acc.sg}\\
\glt `the good wonder' (O\_Otfr.Ev.1.9 (edition 350--361))
\end{exe}}


\begin{exe}
\ex\label{ch6ex22}\begin{xlist}
\ex\label{ch6ex22a}
\gll d-\textbf{es} fraono capet-es\\
\textsc{def-n.gen.sg} divine.\textsc{n.gen.sg.∅} prayer-\textsc{n.gen.sg}\\
\glt `of the Lord's prayer' (E\_Exhortatio (edition 129--139))
\ex\label{ch6ex22b} 
\gll ni was imo ánawani th-a\textbf{z} árunti sconi\\
\textsc{neg} was him believable \textsc{def-n.nom.sg} message.\textsc{n.nom.sg} good.\textsc{n.nom.sg.∅}\\
\glt `he did not trust the good news' (O\_Otfr.Ev.1.4 (edition 404--416)
\ex\label{ch6ex22c} 
\gll Zéinot ouh thio dáti th-\textbf{az} púrpurin giwáti\\
denotes also \textsc{dem} acts \textsc{def-n.nom.sg} crimson.\textsc{n.nom.sg.∅} cloak.\textsc{n.nom.sg}\\
\glt `The crimson cloak also denotes these acts.'\\ (O\_Otfr.Ev.4.25 (edition 89--100))
\end{xlist}
\end{exe}

In possessive DPs, in \isi{contrast}, none of the zero-inflected adjectives are
indeclinable, and the \isi{pattern} is well-attested in prose as well, see
(\ref{ch6ex23a}) and (\ref{ch6ex23b}). This suggests that there must be independent reasons
responsible for the higher percentage of zero-inflected adjectives in
possessive DPs, rooted in the morphological form of the \isi{determiner}, or in
the fact that possessive determiners do not assign the same kind of
definite interpretation to the DP as definite determiners do.

\begin{exe}
\ex\label{ch6ex23}\begin{xlist}
\ex\label{ch6ex23a}
\gll únser héilig sáng ze\_lóbenn-e\\
our.\textsc{n.acc.sg} holy.\textsc{n.acc.sg.∅} song.\textsc{n.acc.sg} to-praise-\textsc{inf.dat.sg}\\
\glt `to praise our holy song' (N\_Mart\_Cap.II.106-110\_J (edition 932--943))
\ex\label{ch6ex23b} 
\gll Dîn guôt uuíllo . ist uns skérm\\
your.\textsc{m.nom.sg} good-\textsc{m.nom.sg.∅} will.\textsc{m.nom.sg} {} is us shelter\\
\glt `Your good will is our shelter.' (N\_Ps\_5\_16--19 (edition 789--800))
\end{xlist}
\end{exe}

Let us turn to the instances of the pronominal variant of strong
adjectives in definite contexts. In the introduction, it was outlined
that such examples are cited in the literature (see \citealp[185--188]{behaghel1923deutsche}) and that they occur in \ili{Germanic} as a whole. Note that the
explanations put forward in the literature fail to explain the presence
of these patterns in the data. First, recall Klein's \citeyearpar[200]{klein2007semantischen}
observation that the replacement of the weak \isi{inflection} by the strong
one in some parts of the paradigm, taking place in \ili{Central German}
dialects, is already present in the late phase of OHG. But this
consideration cannot account for the presence of strong adjectives in
definite contexts in the cases under investigation, because the examples
are found outside the \ili{Central German} dialectal area. Second, the
previous literature has ascribed the use of strong adjectives in
definite environments to Otfrid's \emph{Gospel Book} (see \citealp[298]{Braune2018AHD}). But this argument must be rejected as well, because the \isi{corpus}
search reveals that the respective instances are attested in virtually
all texts included in ReA. This suggests that the strong inflectional
\isi{pattern} is compatible with definite determiners throughout the OHG
attestation.\footnote{It might be assumed that a factor favouring the
  occurrence of strong adjectives in definite DPs in Otfrid's
  \emph{Gospel Book} is the rhyming structure of this poem, as shown in
  the example in (\ref{ch6exii}). Here, the strong \isi{adjective} \emph{guater} at the
  end of the first half-line rhymes with the \isi{noun} \emph{múater} in the
  end of the second half-line:
  
\begin{exe}
\ex\label{ch6exii}
\gll Tho fuar ther sún guat-er //  thar ínan zoh sin múater \\
then went \textsc{def.m.nom.sg} son.\textsc{m.nom.sg} good-\textsc{m.nom.sg.str} {}  where him led his mother\\
\glt `Then the good son followed his mother everywhere.' (O\_Otfr.Ev.2.11 (edition 5--17))
\end{exe}

However, note that the frequency of \isi{prenominal} and \isi{postnominal} strong adjectives in definite DPs in Otfrid's \emph{Gospel Book} is 31 and 17, respectively; i.e., the \isi{pattern} is not strictly attributed to \isi{postnominal} occurrences of adjectives in rhyming positions.} Third, according to
\citet[750]{Wilmanns09}, strong adjectives in definite environments are due
to a phonological resemblance of the respective endings of the weak
paradigm, most obvious in the accusative singular of the masculine
gender as in (\ref{ch6ex24a}) and (\ref{ch6ex25a}), where the strong ending -\emph{an} is
phonologically similar to the weak ending -\emph{on}. But in the \isi{corpus},
strong forms of adjectives are well-represented in virtually all
paradigm positions. The examples in (\ref{ch6ex24b}) and (\ref{ch6ex25b}) illustrate strong
forms in definite and possessive contexts in the \isi{genitive} plural and the
\isi{dative} singular, respectively.

\begin{exe}
\ex\label{ch6ex24}\begin{xlist}
\ex\label{ch6ex24a}
\gll th-\textbf{en} líob-\textbf{an} man\\
\textsc{def-m.acc.sg} beloved-\textsc{m.acc.sg.str} man.\textsc{m.acc.sg}\\
\glt `the beloved man' (O\_Otfr.Ev.1.22 (edition 413--423))
\ex\label{ch6ex24b} 
\gll thie heroston the-\textbf{ro} heithafte-\textbf{ro} mann-o\\
\textsc{def} first.ones \textsc{def-m.gen.pl} serving-\textsc{m.gen.pl.str} man-\textsc{m.gen.pl}\\
\glt `the first ones among the priests' (T\_Tat124 (edition 253--264))
\end{xlist}
\end{exe}

\begin{exe}
\ex\label{ch6ex25}\begin{xlist}
\ex\label{ch6ex25a}
\gll thuruh sin-\textbf{an } éineg-\textbf{an} sun\\
through his-\textsc{m.acc.sg} single-\textsc{m.acc.sg.str} son.\textsc{m.acc.sg}\\
\glt `through his only son' (O\_Otfr.Ev.2.1 (edition 385--396))
\ex\label{ch6ex25b} 
\gll fona sine-\textbf{mu} uuihe-\textbf{mu } liham-in\\
from his-\textsc{n.dat.sg} holy-\textsc{n.dat.sg.str} body-\textsc{n.dat.sg}\\
\glt `from his holy body' (MH\_Murb.H.XVII (edition 12--22))
\end{xlist}
\end{exe}

Finally, \citet[289, Anm. 1]{Braune2018AHD} suggests that there is a difference in the
interpretation of strong and weak forms of adjectives in definite
contexts, in that the strong ones refer to a temporary property of the
object or individual denoted by the DP, whereas the weak ones apply to a
permanent property. This opinion cannot be maintained in the face of
examples like (\ref{ch6ex24a}) referring to Mary's persistent love of her son
Jesus, or (\ref{ch6ex25a}) referring to a permanent property (namely, that Jesus is
the only son of God, see also the argumentation in \cite[750--751]{Wilmanns09}).

Importantly, the strong \isi{inflection} of adjectives occurs in overtly
definite environments representing various subtypes of semantic
\isi{definiteness}. It is attested in DPs expressing uniqueness such as (\ref{ch6ex26a})
referring to doomsday (see also (\ref{ch6ex16}) referring to the sun), or in DPs
referring to common knowledge, e.g. the old laws in (\ref{ch6ex26b}) or the names
of customs and feasts in (\ref{ch6ex26c}).

\begin{exe}
\ex\label{ch6ex26}\begin{xlist}
\ex\label{ch6ex26a}
\gll an de-\textbf{mo} giunstie-\textbf{mo} tag-a  \\
on \textsc{def-m.dat.sg} youngest-\textsc{supl.m.dat.sg.str} day-\textsc{m.dat.sg}\\
\glt `at doomsday' (KB\_KlosterneuburgerGebet (edition 48--58))
\ex\label{ch6ex26b} 
\gll in th-\textbf{en} ált-\textbf{en } éw-on\\
in \textsc{def-f.dat.pl} old-\textsc{f.dat.pl.str} law-\textsc{f.dat.pl}\\
\glt `in the old laws' (O\_Otfr.Ev.1.20 (edition 238--249))
\ex\label{ch6ex26c} 
\gll zi th-\textbf{en} óstrig-\textbf{en} gizít-in\\
to \textsc{def-f.dat.pl} Easter-\textsc{f.dat.pl.str} holiday-\textsc{f.dat.pl}\\
\glt `to the Easter holidays' (O\_Otfr.Ev.2.11 (edition 611--621))
\end{xlist}
\end{exe}

This is similar in the possessive environments. Strong adjectives may
occur in DPs denoting entities which are inferable in the context, as
the dead body of Jesus in (\ref{ch6ex27a}), or the uniqueness of the son of God,
see (\ref{ch6ex27b}). Note that zero-inflected adjectives may also denote unique
referents, see (\ref{ch6ex27c}).

\begin{exe}
\ex\label{ch6ex27}\begin{xlist}
\ex\label{ch6ex27a}
\gll fona  sine-\textbf{mu } uuihe-\textbf{mu} liham-in\\
from his-\textsc{n.dat.sg} holy-\textsc{n.dat.sg.str} body-\textsc{n.dat.sg}\\
\glt `from his holy body' (MH\_Murb.H.XVII (edition 12--22))
\ex\label{ch6ex27b} 
\gll thuruh sin-\textbf{an} éineg-\textbf{an} sun\\
through his-\textsc{m.acc.sg} single-\textsc{m.acc.sg.str} son.\textsc{m.acc.sg}\\
\glt `through his only son' (O\_Otfr.Ev.2.1 (edition 385--396))
\ex\label{ch6ex27c} 
\gll Ich geloube an sin-\textbf{in} aininborn sun\\
I believe in his-\textsc{m.acc.sg} only-begotten.\textsc{m.acc.sg.∅} son.\textsc{m.acc.sg}\\
\glt `I believe in his only-begotten son.'\\(GGB3\_SangallerGlaubenBeichteIII (edition 29--40))
\end{xlist}
\end{exe}

This data suggests that the definite interpretation of the DP does not
categorically trigger weak \isi{inflection} on the \isi{adjective}.

In addition, the question regarding the differences in the frequencies
of strong, zero and weak adjectives in definite and possessive DPs
remains unresolved. In Sections \ref{sec:detinfl} and \ref{sec:linorder}, two potential factors
explaining this difference will be addressed, i.e. the morphological
form of the \isi{determiner} on the one hand, and the positional realization of
the modifier relative to the head \isi{noun} on the other.

\subsubsection{The role of determiner inflection}\label{sec:detinfl}
\citet{Demske01} treats possessive determiners on a par with definite ones,
arguing that they trigger weak \isi{inflection} on adjectives included
in such DPs. Table \ref{ch6t1} reveals a frequency of 69.4\% weak adjectives in DPs
headed by a possessive \isi{determiner}, which is the most frequent \isi{pattern} in
this class of DPs, but nevertheless lower than the frequency of weak
adjectives in definite DPs, which is 87.7\%.

Recall that in PDG, uninflected determiners require strong adjectival
\isi{inflection}, while inflected ones require weak \isi{inflection} (Section \ref{sec:morphdrivenPDG}). In
OHG, the paradigm of the possessive \isi{determiner} also displays uninflected
forms, as does the paradigm of the \isi{indefinite} \isi{determiner}. By \isi{contrast},
the \isi{definite determiner} displays morphologically distinctive forms in
its entire paradigm. Therefore, it might be suggested that the higher
frequency of pronominal and zero-inflected adjectives in possessive DPs is
due to the lack of \isi{inflection} on the \isi{determiner}, similarly to the
situation in PDG. If this is true, the conclusion would be that at least
within the class of possessive DPs, the morphological principle holding
for PDG must have applied in OHG as well.

In order to investigate the relation between the lack of \isi{inflection} on
the \isi{determiner} and the choice of the strong \isi{inflection} on the \isi{adjective},
possessive DPs will be compared with \isi{indefinite} DPs, as they also
display uninflected \isi{determiner} forms.

Let us look at the distribution of \isi{adjectival inflection} in \isi{indefinite}
contexts first. In OHG, bare forms of the \isi{indefinite} \isi{determiner}
\emph{ein}, its negative variant \emph{nihein} and the markers of
in\isi{definiteness} \emph{sum} and \emph{sumalih} are present in the
\isi{nominative} singular of all genders, including the feminine, and the
accusative singular of the neuter gender.

Table \ref{ch6t2} summarizes the occurrences of the strong (both zero and
pronominal) and weak \isi{inflection} in \isi{indefinite} DPs, depending on the
presence of \isi{inflection} on the \isi{determiner}.

\begin{table}
\begin{tabular}{l r *3{r@{~}r}}
\lsptoprule 
                              &     & \multicolumn{4}{c}{strong}                                  &  \\\cmidrule(lr){3-6}
                              & $n$ & \multicolumn{2}{c}{zero}   & \multicolumn{2}{c}{pronominal} & \multicolumn{2}{c}{weak} \\\midrule  
Uninflected indef. determiner & 71  & 56 & (94.9\%)             & 13 & (24.1\%)                  & 2 & (28.6\%)\\
Inflected indef. determiner   & 49  & 3  & (5.1\%)              & 41 & (76.9\%)                  & 5 & (71.4\%) \\ 
Total                         & 120 & 59 & (100.0\%)            & 54 & (100.0\%)                 & 7 & (100.0\%) \\ \lspbottomrule
\end{tabular}
\caption{Strong (zero and pronominal) and weak adjectives in DPs headed
by an uninflected or inflected \isi{indefinite} \isi{determiner} in ReA 1.1}\label{ch6t2}
\end{table}

We will abstract away from the figures gained for weak adjectives in
\isi{indefinite} DPs because of the low number of instances and the special
conditions under which they apply (see Section \ref{sec:DPDet}). If we look at the
distribution of the remaining inflectional patterns, the figures in
Table \ref{ch6t2} suggest that there is a strong tendency for zero-inflected
adjectives to occur with DPs headed by a bare \isi{indefinite} \isi{determiner}
(94.9\%), a fact that has also been noticed in the previous literature
(see \citealp[205]{klein2007semantischen}). An example is presented in (\ref{ch6ex28a}). The exceptional
\isi{pattern} involving a zero-\isi{adjective} after an inflected \isi{determiner} is
given in (\ref{ch6ex28b}) and only involves forms of the inflected \isi{indefinite}
marker \emph{sum} `a certain'. In \isi{contrast}, the pronominal variant of
the strong adjectival declension is not as strictly linked to any form
of the \isi{determiner}. It is more frequently attested after an inflected
in\isi{definite determiner} (76.9\%), as in (\ref{ch6ex29a}), but it is also common in
\isi{indefinite} DPs displaying a bare \isi{determiner} (24.1\%), see (\ref{ch6ex29b}),
especially in work by Notker, as also observed by \citet[205]{klein2007semantischen}.

\begin{exe}
\ex\label{ch6ex28}\begin{xlist}
\ex\label{ch6ex28a}
\gll Chám óuh éin hálz smíd\\
came also \textsc{indef.m.nom.sg} lame.\textsc{m.nom.sg.∅} blacksmith.\textsc{m.nom.sg}\\
\glt `Also, a lame blacksmith arrived.' \\ (N\_Mart\_Cap.I.75--79\_J (edition 408--419))
\ex\label{ch6ex28b} 
\gll Súm-\textbf{ez} réht zímilîh\\
a.certain-\textsc{n.nom.sg} proper.thing.\textsc{n.nom.sg} approved\\
\glt `A certain proper thing is approved.' (N\_Syl\_8 (edition 238--249))\\
(Lat. Quoddam iustum honestum)
\end{xlist}
\end{exe}

\begin{exe}
\ex\label{ch6ex29}\begin{xlist}
\ex\label{ch6ex29a}
\gll in éin-\textbf{ero} chúrz-\textbf{ero} uuîl-o \\
in \textsc{indef-f.dat.sg} short-\textsc{f.dat.sg.str} while-\textsc{f.dat.sg}\\
\glt `within a short period of time' \\ (N\_DeCon\_II\_45--48 (edition 1042--1053))
\ex\label{ch6ex29b} 
\gll éin fáleuu-\textbf{er} stéin\\
\textsc{indef.m.nom.sg} yellow-\textsc{m.nom.sg.str} stone.\textsc{m.nom.sg}\\
\glt `a yellow stone' (N\_Mart\_Cap.I.64--72 (edition 290--301))
\end{xlist}
\end{exe}

Let us compare this picture to the one gained for DPs headed by a
possessive \isi{determiner}. In OHG, the paradigm of the possessive \isi{determiner}
displays bare forms in the \isi{nominative} singular and plural of the first
and second person of all genders (\emph{min} `my', \emph{din} `your',
\emph{unser} `our' and \emph{iuwer} `your'), as well as in the
\isi{nominative} singular masculine and the \isi{nominative} and accusative singular
neuter of the third person (\emph{sin} `his'). In the previous
literature, it has been argued that the uninflected forms of the
possessive determiners trigger weak \isi{inflection} on the \isi{adjective}, just
like definite determiners. But the \isi{corpus} data shows that next to weak
forms as in (\ref{ch6ex30a}), both variants of the strong \isi{pattern} may occur after
an uninflected possessive \isi{determiner}, see (\ref{ch6ex30b}) and (\ref{ch6ex30c}). The same applies to
DPs involving an inflected possessive \isi{determiner}. Next to the weak form
as shown in (\ref{ch6ex30a}), we find both zero and pronominal forms of the strong
\isi{inflection}, see (\ref{ch6ex31b}) and (\ref{ch6ex31c}).

\begin{exe}
\ex\label{ch6ex30}\begin{xlist}
\ex\label{ch6ex30a}
\gll  únser liob-\textbf{o} drúhtin\\
our.\textsc{m.nom.sg} beloved-\textsc{m.nom.sg.wk} God.\textsc{m.nom.sg}\\
\glt `our beloved Lord' (O\_Otfr.Ev.3.21 (edition 7--18))
\ex\label{ch6ex30b} 
\gll hábe in geuuónehéite . únser héilig sáng ze\_lóbenn-e\\
have.\textsc{imp.2sg} in custom {} our holy-\textsc{n.acc.sg.∅} song.\textsc{n.acc.sg} to-praise-\textsc{inf.dat.sg}\\
\glt `be accustomed to praising our holy song'\\(N\_Mart\_Cap.II.106-110\_J (edition 932--943)) \\
 (Lat. suesce probaresacros cantus)
\ex\label{ch6ex30c} 
\gll químit uns thiz gúat in unser ármilich-\textbf{az} múat\\
comes us \textsc{dem} goodness in our.\textsc{n.acc.sg} poor-\textsc{n.acc.sg.str} mind.\textsc{n.acc.sg}\\
\glt `This goodness will enter our poor mind.' \\ (O\_Otfr.Ev.3.3 (edition 18--29))
\end{xlist}
\end{exe}

\begin{exe}
\ex\label{ch6ex31}\begin{xlist}
\ex\label{ch6ex31a}
\gll  mít sîne-\textbf{mo} scôn-\textbf{en} suért-e\\
with his-\textsc{m.dat.sg} beautiful-\textsc{m.dat.sg.wk} sword-\textsc{m.dat.sg}\\
\glt `with his beautiful sword' (N\_Mart\_Cap.I.85--89\_J (edition 314--326))
\ex\label{ch6ex31b} 
\gll Ich geloube an sin-\textbf{in} aininborn sun\\
I believe in his-\textsc{m.acc.sg} only-begotten.\textsc{m.acc.sg.∅} son.\textsc{m.acc.sg}\\
\glt `I believe in his only-begotten son.'\\(GGB3\_SangallerGlaubenBeichteIII (edition 29--40))
\ex\label{ch6ex31c} 
\gll fona sine-\textbf{mu} uuihe-\textbf{mu} liham-in\\
from his-\textsc{n.dat.sg} holy-\textsc{n.dat.sg.str} body-\textsc{n.dat.sg}\\
\glt `by his holy body' (MH\_Murb.H.XVII (edition 9--21))
\end{xlist}
\end{exe}

The quantitative distribution of the various inflectional patterns of
adjectives in DPs introduced by uninflected and inflected possessive
determiners is shown in Table \ref{ch6t3}.

\begin{table}[t]
\begin{tabular}{l r *3{r@{~}r}}
\lsptoprule
&     & \multicolumn{4}{c}{strong}     \\\cmidrule(lr){3-6}
& $n$ & \multicolumn{2}{c}{zero} & \multicolumn{2}{c}{pronominal} &\multicolumn{2}{c}{weak}\\
\midrule
Uninflected poss. det. & 41 & 12  & (92.3\% ) &  10  &(16.4\%)  &19   &(11.3\%)\\
Inflected poss. det. & 201&  1  & (7.7\%  ) &  51  &(83.6\%)  &149  &(88.7\%) \\
Total                                  & 242&  13 & (100.0\%) &  61  &(100.0\%)  & 168 &(100.0\%)  \\
\lspbottomrule
\end{tabular}
\caption{Strong (zero and pronominal) and weak adjectives in DPs headed by an uninflected
or inflected possessive \isi{determiner} in ReA 1.1}\label{ch6t3}
\end{table}

The numbers in Table \ref{ch6t3} show that, similarly to the \isi{indefinite} contexts,
zero-inflected adjectives display a strong preference for DPs headed by
an uninflected possessive \isi{determiner}, applying in 12 out of 13 attested
cases (92.3\%). In \isi{contrast}, the pronominal variant is less restricted
with respect to the morphological form of the \isi{determiner}. It is more
common after an inflected \isi{determiner} (83.6\%) but is also present after
an uninflected one (16.4\%). Surprisingly, a similar frequency applies
for weak adjectives in possessive DPs. The weak variant is much more
common after an inflected possessive \isi{determiner} (88.7\%) than after a
bare one (11.3\%). Taking the two variants of the strong \isi{pattern}
together and performing a standard chi-square test reveals a
statistically significant relation between the presence of \isi{inflection} on
the possessive \isi{determiner} and the selection of the inflectional \isi{pattern}
on the \isi{adjective}, see Table \ref{ch6t4}.

\begin{table}[t]
\begin{tabular}{l rrr}
\lsptoprule
& $n$ & strong\footnote{Zero and pronominal} & weak  \\
\midrule
Uninflected possessive determiner & 41 & {22} & 19   \\
Inflected possessive determiner   &201 & {52} &149  \\
\midrule
Total & 242 & 74 &  168 \\
\lspbottomrule
\end{tabular}
\caption{The presence of inflection on the possessive determiner as a
factor influencing strong or weak \isi{adjective} inflection. $\chi^2 (1, n=242)=12.387$, $p=0.000432$, significant at $p <0.01$.}\label{ch6t4}
\end{table}

Given these figures, it can be concluded that in the domain of
possessive DPs, the choice of the weak \isi{inflection} is favoured by the
presence of overt morphological features on the \isi{determiner}. This, in
turn, suggests that in the domain of possessive DPs, the association of
the weak declensional \isi{pattern} with the overt realization of
morphological features on the \isi{determiner} that is constitutive of the
morphological principle of distribution of \isi{adjectival inflection} in PDG
already starts to evolve.

\subsubsection{The role of the linear order in the DP}\label{sec:linorder}
Recall that \citet{Demske01} observes that \isi{prenominal} adjectives are more
faithful to the semantic principle of distribution of adjectival
\isi{inflection} than \isi{postnominal} ones are (see (\ref{ch6ex14}) vs. (\ref{ch6ex15}) in Section \ref{sec:distrEGOHG}).
Therefore, the positional realization of the modifier relative to the
head \isi{noun} will be examined as a potential factor determining the
variability in the frequency of inflectional patterns in various types
of DPs in the data.

Table \ref{ch6t5} provides the absolute number of \isi{prenominal} and \isi{postnominal}
modifiers of the various inflectional types in \isi{indefinite}, definite and
possessive DPs in ReA.\footnote{The numbers for nominalized adjectives
  in each type of DP are excluded because the property of the linear
  order relative to a head \isi{noun} does not apply in these cases.} In
addition, it provides the frequency of \isi{postnominal} modifiers (as opposed
to \isi{prenominal} ones) of the respective inflectional class of the
\isi{adjective} for each class of DPs included in the dataset.

\begin{table}[t]
\begin{tabular}{ll *3{rr@{~}r}}
\lsptoprule
\multicolumn{2}{l}{} & \multicolumn{3}{c}{\scshape indef}    & \multicolumn{3}{c}{\scshape def}    & \multicolumn{3}{c}{\scshape poss}    \\\cmidrule(lr){3-5}  \cmidrule(lr){6-8}\cmidrule(lr){9-11}
\multicolumn{2}{l}{Inflection}    & \multicolumn{1}{c}{A--N} & \multicolumn{2}{c}{N--A} & \multicolumn{1}{c}{A--N} & \multicolumn{2}{c}{N--A} & \multicolumn{1}{c}{A--N} &  \multicolumn{2}{c}{N--A} \\\midrule
\multicolumn{2}{l}{strong}  & \\
                    & zero  & 26 & 2  &  (7.1\%) &  9 & 7  & (43.7\%) & 10 & 3  & (23.1\%)\\ 
                    & pron. & 35 & 15 & (30.0\%) & 85 & 35 & (28.9\%) & 30 & 21 & (41.2\%)\\
\multicolumn{2}{l}{weak}    & 5 & 0 & (0.0\%) & 1\,000 &  30 & (2.9\%) & 112 & 5 & (4.3\%) \\ 
\lspbottomrule
\end{tabular}
\caption{Frequency of strong (zero and pronominal) and weak adjectives in
\isi{postnominal} position in \isi{indefinite}, definite and possessive DPs in ReA
1.1}\label{ch6t5}
\end{table}

The figures for \isi{indefinite} DPs are not very reliable because after
leaving aside the cases of \isi{nominalization}, the number of weak adjectives
is very low, amounting to only five examples in total in adnominal use.
None of the weak adjectives in \isi{indefinite} contexts appear in
\isi{postnominal} position.

With the remaining types of DPs, the figures are more telling. In
definite and possessive DPs, the frequency of weak adjectives in
\isi{postnominal} position is almost equally low, amounting to 2.9\% and
4.3\%, respectively. At the same time, in definite and possessive DPs,
the frequency of strong adjectives, both zero-inflected and pronominal
ones, increases whenever the \isi{adjective} is \isi{postnominal}. In other words,
as already observed by \citet{Demske01}, the strong \isi{pattern} of adjectives
is more likely to occur in \isi{postnominal} position in definite and
possessive DPs if it follows the head \isi{noun}.


\subsubsection{Interim conclusion}\label{sec:interim conc}\largerpage
The \isi{corpus} search revealed that the previously assumed correlation
between the type of \isi{adjectival inflection} and the semantic class of the
\isi{determiner} is only partly confirmed by the data. Crucially, there is
variability in the distribution of the various inflectional patterns in
each type of DP, suggesting that the semantic principle of distribution
is subject to violations.

On the one hand, there are sporadic instances of weak adjectives in
\isi{indefinite} contexts. In the \isi{nominalization} construction, these
adjectives are used to introduce novel referents to the discourse;  i.e.
the semantics of the weak declension cannot be regarded as inherently
definite. In addition, we find early instances of weak adjectives
following inflected \isi{indefinite} determiners, suggesting that the modern
\ili{German} \isi{pattern} of monoinflection starts to spread already in this
period.

On the other hand, strong adjectives, both zero-inflected and pronominal
ones, are attested in definite and possessive DPs alike. This result is
explainable as a continuation of the original \ili{Germanic} situation in
which the strong \isi{pattern} is neutral with respect to the semantic
interpretation of the DP. At the same time, there are differences in the
frequencies of the various inflectional patterns of adjectives in definite and
possessive DPs, although they are both considered as definite.

Two factors explaining these differences were tested. The first one was
the morphological distinctiveness of the \isi{determiner}. Definite DPs
displaying determiners that are sufficiently distinguishable regarding
case, number and gender in the entire paradigm also display the highest
frequency of weak adjectives. Also, in possessive DPs, which display both
bare and inflected determiners, the lack of \isi{inflection} on the \isi{determiner}
results in higher frequencies of strong adjectives in the dataset, while the
presence of \isi{inflection} on the possessive \isi{determiner} correlates with the
choice of the weak \isi{inflection} in a statistically significant way. This
suggests that properties constitutive of the morphological principle of
distribution of \isi{adjectival inflection} governing the situation in PDG
start to emerge already in the system of OHG.

The second factor was the positional realization of the \isi{adjective}
relative to the head \isi{noun}. Weak adjectives are not attested in
\isi{postnominal} position in \isi{indefinite} contexts, and appear in definite and
possessive DPs in very low frequencies. At the same time, the frequency of
strong adjectives in definite and possessive DPs increases when the
\isi{adjective} follows the head \isi{noun}. This suggests that the weak \isi{inflection}
is strongly associated with the \isi{prenominal} position of the modifier in
these types of DPs, while the strong one is present on adjectives in
both positions.


\subsection{Distribution of adjectival inflection in bare DPs}\label{sec:bareDP}
This section investigates the principles underlying the distribution of
\isi{adjectival inflection} in bare DPs containing modifying or nominalized
adjectives (the latter referred to by A\textsubscript{\textsc{nom}} in the
tables). Both \isi{attributive} adjectives and participles are considered, as
well as the same categories used as heads of NPs in nominalizations.

Table \ref{ch6t6} represents the quantitative distribution of inflectional
patterns of adjectives in bare DPs found in ReA 1.1. The figures in Table \ref{ch6t6} show that in the absence of a \isi{determiner}, the
strong \isi{pattern} represents the predominant option, found at an average
frequency of 87.0\% in the entire sample, ranging between 77.5\% and
93.8\% in the individual types of DPs. This is in sharp \isi{contrast} to the
distribution of the strong \isi{pattern} in DPs involving some class of
\isi{determiner} and analyzed in Section \ref{sec:DPDet} (see Table \ref{ch6t1} in Section \ref{sec:DPDet}), where
the strong \isi{pattern} was infrequent as a whole (18.8\%) but highly
frequent in one class of DPs, namely those introduced by an \isi{indefinite}
\isi{determiner} (94.2\%). Consider also that the high percentage of strong
adjectives clearly goes back to the pronominal \isi{inflection} which
dominates in all types of bare DPs, while the zero one is
underrepresented, obtaining its highest score in those cases in which
the \isi{adjective} is \isi{postnominal}.

\begin{table}[t]
\begin{tabular}{ll *4{rr}}
\lsptoprule
&  & \multicolumn{2}{c}{A–N} & \multicolumn{2}{c}{N–A} & \multicolumn{2}{c}{A\textsubscript{\textsc{nom}}} & \multicolumn{2}{c}{All}\\
\cmidrule(lr){3-4}\cmidrule(lr){5-6}\cmidrule(lr){7-8}\cmidrule(lr){9-10}
\multicolumn{2}{l}{Inflection} & \multicolumn{1}{c}{$n$}   & \multicolumn{1}{c}{\%}     & \multicolumn{1}{c}{$n$}   & \multicolumn{1}{c}{\%}& \multicolumn{1}{c}{$n$}   & \multicolumn{1}{c}{\%}& \multicolumn{1}{c}{$n$}   & \multicolumn{1}{c}{\%}\\\midrule 
\multicolumn{2}{l}{strong}  & 1\,814 & 93.8 & 356 & 81.5 & 881 & 77.5 & 3\,051 & 87.0\\
& zero                      &   232 & 12.0 & 64  & 14.6 & 21  &  1.8 & 317   &  9.0\\
& pron.                     & 1\,582 & 81.8 & 292 & 66.8 & 860 & 75.7 & 2\,734 & 78.0\\
\multicolumn{2}{l}{weak}    &   120 &  6.2 &  81 & 18.5 & 255 & 22.4 & 456   & 13.0\\\addlinespace
\multicolumn{2}{l}{Total}   & 1\,934 &        & 437 &        & 1\,136  &     & 3\,507 &\\
\lspbottomrule
\end{tabular}
\caption{Strong (zero and pronominal) and weak adjectival inflection in
bare DPs in ReA 1.1}\label{ch6t6}
\end{table}


At the same time, weak adjectives in bare DPs are infrequent as a whole
(13.0\%) as well as across the individual types of DPs (between 6.2\%
and 22.4\%). This is in \isi{contrast} to their distribution in DPs containing
a \isi{determiner} (see Table \ref{ch6t1} in Section \ref{sec:DPDet}), where they were found in 81.2\%
in the entire sample, with a strong preference for DPs introduced by a
definite or possessive \isi{determiner} (87.7\% and 69.4\%, respectively).

These quantitative aspects of the distribution of \isi{adjectival inflection}
in bare DPs suggest that in the absence of a \isi{determiner}, the \isi{adjective}
hosts the information specifying the morphosyntactic features of the
entire DP. Note that the most frequently attested \isi{pattern}, the
pronominal type of the strong \isi{inflection}, is the most distinctive one on
formal grounds. This is compatible with the morphologically driven
system of distribution of \isi{adjectival inflection} as it applies to PDG.

Let us consider the qualitative distribution of the inflectional
patterns of adjectives in bare DPs attested in the \isi{corpus}. According to
the previous literature, the weak \isi{pattern} is associated with the
\isi{definiteness} of the DP already prior to the establishment of the
\isi{definite determiner}, as exemplified by weak adjectives as part of proper
names in \ili{Germanic} \citep[191--196]{Delbruck09}, e.g. in compound formations
with an initial adjectival element like \emph{Lutzilindorf}, etc. (\cite[310]{Braune2018AHD}\footnote{But see also compound names of places like
  \emph{Altheim}, etc., referred to in \citet[299]{Braune2018AHD}, in which the
  adjectival component bears zero \isi{inflection}.}, see also \citealp{Kogel89}), or
formulaic expressions referring to God, e.g. \emph{druhtîn
nerrend-\textbf{o} Christ} `Lord, the saving Christ' (Is. 17, 15. 11,
cit. in \citealp[748]{Wilmanns09}). In addition, the domain of weak adjectives
in bare DPs is associated with vocatives and DPs denoting situationally
inferable or unique referents, including superlatives (\citealp{Demske01}, see
also Section \ref{sec:distrEGOHG}).

The results of the \isi{corpus} search reveal, however, that the distribution
of inflectional patterns of adjectives in bare DPs in OHG cannot be
explained on the basis of the semantic principle only. Examples
explainable along the lines of the semantic principle are found
sporadically in the \isi{corpus}, as e.g. the minimal pair in (\ref{ch6ex32}). Here, the
\isi{adjective} \emph{tôter} `a dead one' introducing a novel entity bears
strong \isi{inflection}, while on its second mentioning, when it resumes a
notion already activated in the context, the same \isi{adjective} bears weak
\isi{inflection}, namely \emph{tôto}.

\begin{exe}
\ex\label{ch6ex32}
\gll ámoso tôt-\textbf{er} {[}\ldots{]} . daz chit . also tôt-\textbf{o} bestôzener . unde ioh uzer hérzen\\
like dead-\textsc{m.nom.sg.str} {} {} \textsc{dem} says {} like dead-\textsc{m.nom.sg.wk} banished {} and also without heart\\
\glt `like some dead one, this means, like the dead one {[}who is{]} banished and heartless' (N\_Ps\_30\_93 (edition 107--117))
\end{exe}

However, as a whole, we discover \isi{variation} between strong and weak adjectives
in various domains considered inherently definite in previous
research.

Consider adjectives in DPs used as appositions to proper names. As the
examples in (\ref{ch6ex33}) and (\ref{ch6ex34}) suggest, both weak and strong adjectives may
occur in these domains:

\begin{exe}
\ex\label{ch6ex33}
\gll umbi christ-\textbf{an} himilisch-\textbf{un} druhtin\\
about Christ-\textsc{m.acc.sg} heavenly-\textsc{m.acc.sg.wk} God.\textsc{m.acc.sg}\\
\glt `about Christ, the heavenly Lord' (I\_DeFide\_7 (edition 38--50)) \\ 
 (Lat. \emph{christum deum cęli})
\end{exe}

\begin{exe}
\ex\label{ch6ex34}
\gll fona Mari-\textbf{un} macad-\textbf{i} euuik-\textbf{eru}\\
from Mary-\textsc{f.dat.sg} virgin-\textsc{f.dat.sg} eternal-\textsc{f.dat.sg.str}\\
\glt `by Mary, the eternal virgin' (GC\_SangalerCredo (edition 32--44))
\end{exe}

The same alternation applies in DPs acting as proper names; i.e., in
those displaying the property of monoreferentiality or direct
\isi{referentiality} characteristic of proper names as rigid designators in
the sense of \citet{Kripke80}, see \citet[29]{Nublingetal15}. In DPs
referring to God, Jesus or the Holy Spirit, both weak and strong
adjectives appear, see (\ref{ch6ex35}) versus (\ref{ch6ex36a}) and (\ref{ch6ex36b}). Note that in (\ref{ch6ex36b}), the
nominalized strong \isi{adjective} in the prepositional phrase \emph{in uuihêmu} refers to
Christ, translating the proper name contained in the prepositional phrase \emph{in}
\emph{Christo} in the \ili{Latin} original.

\begin{exe}
\ex\label{ch6ex35}
\gll suueri bi himilisch-\textbf{in} got-e\\
swear.\textsc{imp.2sg} by heavely-\textsc{m.dat.sg.wk} God-\textsc{m.dat.sg}\\
\glt`Swear by the heavenly God.' (I\_DeFide\_7 (edition 27--39))
\end{exe}

\begin{exe}
\ex\label{ch6ex36}\begin{xlist}
\ex\label{ch6ex36a}
\gll  Ther infanganer ist fona heileg-\textbf{emo} geist-e\\
who created is from holy-\textsc{m.dat.sg.str} ghost-\textsc{m.dat.sg}\\
\glt `who is created by the Holy Ghost'\\(WK\_Weissenburger\_Katechismus (edition 546--558))
\ex\label{ch6ex36b} 
\gll alle in uuihe-\textbf{mu} ein piru-mes\\
all in holy-\textsc{m/n.dat.sg.str} one be-\textsc{1pres.ind.pl}\\
\glt `we are all united in the name of Christ' (B\_2 (edition 414--424))\\
(Lat. omnes in Christo unum sumus)
\end{xlist}
\end{exe}

Furthermore, weak and strong adjectives alternate in DPs denoting
situationally inferable entities or generally accessible notions. In
(\ref{ch6ex37}), a weak \isi{adjective} appears in a DP referring to a situationally
accessible entity, the lectures of the holy text during church mass. In
(\ref{ch6ex38}), a strong and a weak \isi{adjective} alternate in the same semantic
context. In (\ref{ch6ex39a})--(\ref{ch6ex39d}), strong adjectives appear in DPs referring to
well-known entities of Christian life and belief, such as the
Scriptures, eternal life, the Jewish people, Passover, or the
protagonists of the parable of the ten virgins going to meet their
bridegrooms (Matthew 25:1--5), which are familiar to the assumed
audience.

\begin{exe}
\ex\label{ch6ex37}
\gll danna uurdun gilesan heileg-\textbf{o} lection in dero chirihun\\
 when were read holy-\textsc{f.nom.pl.str} lecture.\textsc{f.nom.pl} in \textsc{def} church\\
\glt `when the holy texts were read aloud in church'\\(WB\_Wzb.Beichte (edition 134--146))
\end{exe}

\begin{exe}
\ex\label{ch6ex38}
\gll heilag-\textbf{a} messa enti heilag-\textbf{on} uuizzod nierita\\
 holy-\textsc{f.acc.sg.str} mass.\textsc{f.acc.sg} and holy-\textsc{m.acc.sg.wk} supper.\textsc{m.acc.sg} \textsc{neg}.respected\\
\glt `{[}I confess that I{]} failed to respect the holy mass and the holy supper.'\\(FB\_Fuldaer\_Beichte (edition 137--149))
\end{exe}

\begin{exe}
\ex\label{ch6ex39}\begin{xlist}
\ex\label{ch6ex39a}
\gll  minneont eouuesant-\textbf{an} lip\\
love.\textsc{3pl.pres.sbjv} eternal-\textsc{m.acc.sg.str} life.\textsc{m.acc.sg}\\
\glt `{[}They should{]} love the eternal life.' (MF\_5\_FH.XLI (edition 163--175))
\ex\label{ch6ex39b} 
\gll ist kúning er githíuto júdisg-\textbf{ero} líut-\textbf{o}\\
is king he obviously Jewish-\textsc{m.gen.pl.str} people-\textsc{m.gen.pl}\\
\glt `he is obviously the king of the Jewish people'\\(O\_Otfr.Ev.4.27 (edition 273--285))
\ex\label{ch6ex39c} 
\gll fuorun sine eldiron giiaro in Hierusalem in itmal-\textbf{emo} tag-e ôstr-ono\\
went his parents every.year to Jerusalem in festive-\textsc{m.dat.sg.str} day-\textsc{m.dat.sg} Passover-\textsc{f.gen.pl}\\
\glt `His parents went every year to Jerusalem to spend the festive period of
Passover.' (T\_Tat12 (edition 19--31))
\ex\label{ch6ex39d} 
\gll louffant uuih-\textbf{o} magadi {[}\ldots{]} tragante heitariu liotfaz tulisc-\textbf{o} auur pilibant  \\
go holy-\textsc{f.nom.pl.str} virgin.\textsc{f.nom.pl} {} carrying bright lamps foolish-\textsc{f.nom.pl.str} however stay.back\\
\glt `The holy virgins go forth {[}to meet their bridegrooms{]}, while the
foolish ones stay behind.' (MH\_Murb.H.I (edition 112--123))
\end{xlist}
\end{exe}

Finally, strong adjectives can also be found in vocatives, see (\ref{ch6ex40}):

\begin{exe}
\ex\label{ch6ex40}
\gll du hoh-\textbf{er} truhtin\\
you supreme-\textsc{m.nom.sg.str} God.\textsc{m.nom.sg}\\
\glt `you, supreme Lord' (MH\_Murb.H.XIV (edition 34--44))
\end{exe}

To illustrate the \isi{variation} of strong and weak adjectives in one and the
same semantic domain, I provide the respective figures for bare DPs in
vocatives. Table \ref{ch6t7} gives the absolute numbers of pronominal, zero and
weak patterns of adjectives in \isi{vocative} bare DPs, including the
frequency of the weak \isi{pattern}. The numbers are provided individually for
\isi{prenominal} and \isi{postnominal} modifiers as well as for nominalized
adjectives.

\begin{table}[t]
\centering
\begin{tabular}{ll *4{r@{~}r}}
\lsptoprule
\multicolumn{2}{l}{Inflection} & \multicolumn{2}{c}{A–N}       & \multicolumn{2}{c}{N–A}       & \multicolumn{2}{c}{A\textsubscript{\textsc{nom}}}     & \multicolumn{2}{c}{All}\\\midrule
\multicolumn{2}{l}{strong}       & 31   &          & 16   &          & 11   &          & 58\\
                   & zero        & 22   &          & 7    &          & 0    &          & 27   &      \\ 
                   & pronominal  & 9    &          & 9    &          & 11   &          & 29   &      \\ 
\multicolumn{2}{l}{weak}         & 28   & (47.4\%) & 18   & (52.9\%) & 9    & (45.0\%) & 55   & (48.7\%)\\\addlinespace
\multicolumn{2}{l}{Total}        & 59   &          & 34   &          & 20   &          & 113\\
\lspbottomrule
\end{tabular}
\caption{Distribution of strong (zero and pronominal) and weak
\isi{inflection} of adjectives in \isi{vocative} DPs in ReA 1.1}\label{ch6t7}
\end{table}

The figures in Table \ref{ch6t7} show that the proportion of weak adjectives in
\isi{vocative} DPs is around half of the instances per dataset, with a
slightly higher frequency of weak adjectives than strong ones in
\isi{postnominal} position. However, the standard statistical test shows no significant
correlation between the position of the \isi{adjective} and its inflectional
behaviour in \isi{vocative} DPs.\footnote{Considering the occurrences of the
  strong (both pronominal and zero) and the weak \isi{inflection} in
  \isi{prenominal} (A--N) and \isi{postnominal} (N--A) use, the chi-square
  result is as follows: $\chi^2 (2, N=94) = 0.1843$,
  $p=0.667692$. The result is not significant at $p < 0.05$.}

Analyzing the results of the \isi{corpus} search, two domains can be
identified in which the adjectives invariantly display weak \isi{inflection},
without alternating with strong ones. The first one is the adverbial use
of nominalized adjectives as shown in (\ref{ch6ex41a}) and (\ref{ch6ex41b}); the second one is
\isi{gradation}, see (\ref{ch6ex42}).

\begin{exe}
\ex\label{ch6ex41}\begin{xlist}
\ex\label{ch6ex41a}
\gll uuas giscriban in ebraisg-\textbf{on} inti in criehisg-\textbf{on} inti in latinisg-\textbf{on}\\
was written in \ili{Hebrew}-\textsc{n.dat.sg.wk} and in \ili{Greek}-\textsc{n.dat.sg.wk} and in \ili{Latin}-\textsc{n.dat.sg.wk}\\
\glt `was written in \ili{Hebrew} and in \ili{Greek} and in \ili{Latin}'\\(T\_Tat204 (edition 43--54))
\ex\label{ch6ex41b} 
\gll táz in únrûo-chesk-\textbf{un} únbedénchit stat\\
which in disregardful-\textsc{f.acc.sg.wk} neglected stays\\
\glt `which is neglected in a disregardful way' \\ (N\_Syl\_14 (edition 289--299))
\end{xlist}
\end{exe}

\begin{exe}
\ex\label{ch6ex42}
\gll Sie minnont furist-\textbf{a} sedal {[}\ldots{]} inti furist-\textbf{on} stoola\\
 they love front-\textsc{supl.n.acc.sg.wk} seat.\textsc{n.acc.sg} {} and first-\textsc{supl.m.acc.pl.wk} chair.\textsc{m.acc.pl}\\
\glt `They love the uppermost place {[}at feasts{]} and the chief seats
  {[}in the synagogues{].}' (Matthew 23,6) (T\_Tat141 (edition 89--99))
\end{exe}

The \isi{pattern} exemplified in (\ref{ch6ex41a}) and (\ref{ch6ex41b}) involves adverbial uses of
nominalized adjectives displaying the derivational morpheme -\emph{isk},
attested 23 times in the \isi{corpus}. It is well-known that the suffix
-\emph{isk} is used to derive adjectives expressing provenience or
affiliation to a well-established group (\cite[304]{Braune2018AHD}). The
respective base words refer to ethnic groups, names of places,
geographic regions or theological spheres (e.g. \emph{Heaven}). The
example in (\ref{ch6ex41b}) is exceptional, but it is found in late OHG, probably
suggesting that the suffix -\emph{isk} starts to attach to base words
outside the original domain of words denoting provenience. The invariant
use of the weak \isi{pattern} in this sample can be taken to suggest that
there is indeed a relation between the weak inflectional \isi{pattern} and the
\isi{familiarity} with the notion denoted by the DP.

Gradation is the second domain in which adjectives consistently display
weak \isi{inflection} in bare DPs.\footnote{DPs with
  graded adjectives may also involve determiners, contra \citet[69--70]{Demske01}; see (i) for a
  \isi{comparative} and (ii) for a \isi{superlative}:
  
\begin{exe}
\exi{(i)}\gll  th-\textbf{er} iung-\textbf{oro} sun elilentes fuor\\
\textsc{def-m.nom.sg} young-\textsc{cmpr.m.nom.sg.wk} son.\textsc{m.nom.sg} abroad went\\
\glt `the younger son went into foreign countries' (T\_Tat97 (edition 37--48))

\exi{(ii)}\gll scouuuonti uuio sie thiu furist-\textbf{un} sedal gicurun\\
seeing how they \textsc{def.m.acc.pl} high-\textsc{supl.m.acc.pl.wk} seat.\textsc{m.acc.pl} chose\\
\glt `observing how they chose the uppermost seats' (T\_Tat110 (edition 111--121)
\end{exe}

  \noindent The frequency of bare DPs including graded adjectives in ReA is 52.9\%
  (99 out of 187) for comparatives and 32.4\% (107 out of 330) for
  superlatives; i.e., bare DPs with adjectives in the \isi{superlative} are
  even lower in frequency than those with comparatives. See also the discussion on the inflectional properties of the
  \isi{superlative} in \ili{Germanic} in \citet[173--175]{behaghel1923deutsche}.} This
is expected because it is well-known that comparatives inflect weak in
\ili{Germanic} as a whole and in OHG specifically (see \citealp[172, \emph{inter alia}]{behaghel1923deutsche}), and because with some exceptions, superlatives in
OHG also share this property (see \citealp[315]{Braune2018AHD}).\footnote{See (\ref{ch6ex26a}) for an example of a strong \isi{adjective} in the \isi{superlative}, preceded by an inflected \isi{determiner}.}

The use of the weak \isi{inflection} in comparatives and superlatives in OHG
is explained on semantic grounds, see the argumentation put forward in \citet[314]{Braune2018AHD}:

\begin{quote}
Die schwache Flexion der Steigerungsgrade (Komparativ und Superlativ)
erklärt sich aus ihrer individualisierenden Bedeutung
\glt `The weak \isi{inflection} of the degrees of comparison (\isi{comparative} 
and \isi{superlative}) is explainable on the basis of their individualizing meaning'.
\end{quote}

Recall that \citet[69--70]{Demske01} also explains the use of the weak
\isi{inflection} in superlatives on semantic grounds, arguing that DPs
involving an \isi{adjective} in the \isi{superlative} grade display unique
reference, i.e. one of the subtypes of semantic \isi{definiteness}. In
addition, the invariant weak \isi{inflection} of adjectives in \isi{gradation}
occurs independently of the presence or absence of an \isi{article}.

Another observation regarding the \isi{inflection} of graded adjectives is
important, however. Note that we find examples like (\ref{ch6ex43}) showing that
comparatives bearing the weak \isi{inflection} may display \isi{indefinite}
interpretation as well. Note that the DP containing the \isi{adjective} in the
\isi{comparative} grade is in the scope of negative operators.

\begin{exe}
\ex\label{ch6ex43}
\gll Ni wárd io {[}\ldots{]} giwíssar-\textbf{a} thing\\
\textsc{neg} became ever {} certain-\textsc{cmpr.n.nom.sg.wk} thing.\textsc{n.nom.sg}\\
\glt `Never has there been a more certain issue.'\\(O\_Otfr.Ev.2.3 (edition 444--456))
\end{exe}

This data suggests that the use of the weak \isi{inflection} in comparison is
not strictly linked to the semantic interpretation of the DP, but rather
appears as a formal property specifying the inflectional behaviour of this
class of adjectives.


\section{Conclusion}\label{sec:conc}
The present chapter addressed the distribution of inflectional patterns
of adnominal adjectives in OHG by examining the evidence provided in the
reference \isi{corpus} ReA 1.1. Two datasets were considered, i.e. DPs
displaying some kind of \isi{determiner}, as well as determinerless DPs. The
results challenge previous generalizations according to which the spread
of the various inflectional patterns of adnominal adjectives in OHG is
determined by the interpretation of the respective DP in terms of
(in)\isi{definiteness}. This so-called semantically driven distribution of
\isi{adjectival inflection} can be detected in a part of the data, most
importantly in DPs displaying a definite or \isi{indefinite} \isi{determiner},
although there is \isi{variation} in this domain as well. However, weak
adjectives are not excluded in \isi{indefinite} contexts, while strong ones
are found in all kinds of definite contexts, suggesting that the strong
\isi{pattern} represents the unmarked, or default variant, as also described
for early \ili{Germanic} as a whole.

At the same time, properties of the PDG morphologically driven
distribution were detected in OHG as well, most importantly in the
domain of bare and possessive DPs, the latter displaying determiners
which can be both inflected and non-inflected. It was shown that the
lack of a \isi{determiner} of any kind strongly correlates with explicit
morphosyntactic marking on the \isi{adjective}, which also holds for PDG. In
addition, in possessive DPs, the distribution of \isi{adjectival inflection}
depends on the presence of \isi{inflection} on the \isi{determiner}. The lack of
morphosyntactic features on the \isi{determiner} favours the strong \isi{inflection}
on the \isi{adjective}, while the presence of \isi{inflection} on the \isi{determiner}
triggers the weak, and less distinctive variant.

\begin{sloppypar}
In the face of this observation, a scenario regarding the later
development of \isi{adjectival inflection} in the history of \ili{German} can be
sketched. In the process of reduction of vowels in inflectional
syllables, the distinction between strong and weak adjectives is
blurred, leading to formal overlapping of the two paradigms. At the same
time, the morphological distinctiveness of the determiners is
strengthened, in that the inventory of indistinctive forms of \isi{indefinite}
and possessive determiners is reduced, e.g. in the \isi{nominative} and
accusative singular of the feminine gender. In this way, the \isi{determiner}
system provides a transparent system of expressing the formal properties
of the DP. As a consequence, the double realization of features in the
DPs is suspended in favour of the more economical principle of
monoinflection, exploiting the invariant version of the weak \isi{pattern} in
all cases in which the morphological properties of the DP are
transparently assigned by the accompanying \isi{determiner}. Basically, the
main properties of this principle are already present in the system of
OHG, although its full establishment lasted for centuries.
\end{sloppypar}

 
\section*{Abbreviations}
\begin{tabularx}{.5\textwidth}{@{}lQ}
\textsc{acc} & accusative\\
A & {adjective} \\
A\textsubscript{\textsc{nom}} & nominalized {adjective} \\
\textsc{cmpr} & {comparative} \\
\textsc{dat} & {dative} \\
\textsc{def} & definite \\
\textsc{dem} & definite \\
DP & {determiner} phrase \\
\textsc{f} & feminine\\
\textsc{gen} & {genitive}\\
IE & {Indo-European} \\
\textsc{ind} & indicative \\
\textsc{indef} & {indefinite} \\
\textsc{inf} & infinitive\\
\textsc{imp} & imperative\\
Lat. & {Latin} \\
\end{tabularx}%
\begin{tabularx}{.5\textwidth}{lQ@{}}
N & {noun} \\
\textsc{m} & masculine \\
\textsc{n} & neuter\\
\textsc{neg} & negative particle \\
\textsc{nom} & {nominative}\\
NP & {noun} phrase\\
OHG & Old High {German} \\
PDG & Present-day {German} \\
\textsc{pl} & plural \\
\textsc{pres} & present \\
\textsc{sbjv} & subjunctive \\
\textsc{sg} & singular \\
\textsc{str} & strong\\
\textsc{supl} & {superlative} \\
\textsc{wk} & weak \\
& \\
\end{tabularx}

\section*{Acknowledgements}
I am grateful to the audience of the workshop “Partition and Individuation in \ili{Germanic} and Slavic”, 15--17 June 2022 in Stuttgart, and to the Annual Conference of the \ili{German} Society for \ili{Germanic} Diachronic Linguistics (GGSG), 28 September--1 October 2022 in Dresden. I am also indebted to the editors of the volume and to two reviewers for careful readings and valuable comments.

{\sloppy\printbibliography[heading=subbibliography,notkeyword=this]}
\end{document}
