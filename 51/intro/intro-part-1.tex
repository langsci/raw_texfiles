\section*{Introduction}

In the first part of this book, I will introduce the semantic and
syntactic templates of various language strategies that have been
introduced in \sectref{s:strats-for-colour}. For each of these
strategies, I will implement a predefined language system that will
allow me to evaluate the performance of the proposed semantic and
syntactic templates. The strategies will be compared to human language
systems using \textsc{naming benchmarks}, and these strategies will be
compared to each other in a \textsc{baseline experiment}.

A compositional approach is applied to both
semantic\is{semantic template} and syntactic
  templates\is{syntactic template}. This implies that parts of the
templates will be shared by various strategies. The semantic templates
will be represented by semantic constraint networks (see \sectref{s:semantic-constraint-network}). These networks consist of
semantic primitives which will be explained in more detail for each
strategy. The syntactic templates will be presented as a general
approach to expressing these semantic networks in language. These
templates will be illustrated by giving examples of coupled feature
structures (see \sectref{s:coupled-feature-structures}), which
allow to express the semantic constraint networks in language. Example
constructions are described that can build up these coupled feature
structures \citep{bleys06next, steels07emergence, bleys08expressing}.

Natural language systems are implemented by providing actual
ontologies and linguistic inventories to express these ontologies in
language. Concerning the basic colour strategy, this boils down to providing
agents with an ontology of categories that are reported in literature,
the lexical rules for that ontology, and the grammatical rules to
express the instantiated semantic template.

The implementation of natural language systems will allow for the
evaluation of the proposed templates. A first evaluation method is to
name a set of colour samples whose names have been reported in
literature in a naming benchmark\is{naming benchmark}. 
Comparing the names produced by the strategy to the
expected names gives a rough idea of how well the templates reflect
the natural system. A second evaluation method is a baseline
  experiment\is{baseline experiment} in which agents, equipped
with the implementation of the natural language system, involve in
colour naming games (see \sectref{s:language-games-for-colour}). This allows for a comparison of
the performance of the templates across language strategies.

\newpage
\thispagestyle{empty}
