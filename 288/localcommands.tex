\newcommand*{\orcid}{}
%herbig et al
\makeatletter
\let\thetitle\@title
\let\theauthor\@author 
\makeatother

\renewcommand{\sectref}[1]{Section~\ref{#1}}

\newcommand{\togglepaper}[1][0]{ 
%   \bibliography{../localbibliography}
  \papernote{\scriptsize\normalfont
    \theauthor.
    \thetitle. 
    To appear in: 
    2nd International Congress on Translation, Interpreting and Cognition
    ``Interdisciplinarity: the Way out of the Box'',
    Berlin: Language Science Press. [preliminary page numbering]
  }
  \pagenumbering{roman}
  \setcounter{chapter}{#1}
  \addtocounter{chapter}{-1}
}

\DeclareMathOperator{\SD}{SD}
\DeclareMathOperator{\SE}{SE}
\DeclareMathOperator{\df}{df}

%%%% Models

%% Baselines
% Cl
\newcommand{\baselineCLmean}{$\mathrm{SubjCL_{avg}}$}


%%%% Features

%% Subjective
\newcommand{\subjCL}{$\mathrm{SubjCL}$}

%% Time
\newcommand{\petime}{$\mathrm{PeTime}$}
\newcommand{\lnpetime}{$\mathrm{LNPeTime}$}

%% Text
\newcommand{\bleu}{$\mathrm{BLEU}$}
\newcommand{\ter}{$\mathrm{TER}$}
\newcommand{\hbleu}{$\mathrm{HBLEU}$}
\newcommand{\hter}{$\mathrm{HTER}$}
\newcommand{\sentencelength}{$\mathrm{SL}$}

%% Typing
\newcommand{\apr}{$\mathrm{APR}$}
\newcommand{\pwr}{$\mathrm{PWR}$}

%% Eyes
% Tobii
\newcommand{\blinkamount}{$\mathrm{BlinkAmount}$}
\newcommand{\normblinkamount}{$\mathrm{NormBlinkAmount}$}
\newcommand{\fixamount}{$\mathrm{FixAmount}$}
\newcommand{\normfixamount}{$\mathrm{NormFixAmount}$}
\newcommand{\fixdur}[1]{$\mathrm{FixDur_{{#1}}}$}
\newcommand{\saccdur}[1]{$\mathrm{SaccDur_{{#1}}}$}
\newcommand{\searchprob}[1]{$\mathrm{SearchProb_{{#1}}}$}
\newcommand{\rawpupil}[1]{$\mathrm{PupilDiameter_{{#1}}}$}
\newcommand{\ica}[2]{$\smash{\mathrm{ICA^{{#1}}_{{#2}}}}$}
\newcommand{\hilbert}[1]{$\mathrm{Hilbert_{{#1}}}$}
% Webcam
\newcommand{\ear}[1]{$\mathrm{EAR_{{#1}}}$}

%% Heart
% Polar and Garmin
\newcommand{\hr}[2]{$\smash{\mathrm{HR^{{#1}}_{{#2}}}}$}
% Polar and Empatica
\newcommand{\rr}[2]{$\smash{\mathrm{RR^{{#1}}_{{#2}}}}$}
\newcommand{\rmssd}[2]{$\smash{\mathrm{RMSSD^{{#1}}_{{#2}}}}$}
\newcommand{\sdnn}[2]{$\smash{\mathrm{SDNN^{{#1}}_{{#2}}}}$}
\newcommand{\pnn}[2]{$\smash{\mathrm{pNN50^{{#1}}_{{#2}}}}$}
\newcommand{\nn}[2]{$\smash{\mathrm{NN50^{{#1}}_{{#2}}}}$}
% Empatica
\newcommand{\bvpempatica}[1]{$\mathrm{BVP_{{#1}}}$}
\newcommand{\bvpamplitudeempatica}[1]{$\mathrm{BVPAmp_{{#1}}}$}
\newcommand{\bvpmeanadempatica}[1]{$\mathrm{BVPMeanAbsDiff_{{#1}}}$}
\newcommand{\bvpmedadempatica}[1]{$\mathrm{BVPMedAbsDev_{{#1}}}$}

%% Skin
% MSBand and Empatica
\newcommand{\gsr}[2]{$\smash{\mathrm{GSR^{{#1}}_{{#2}}}}$}
\newcommand{\freqgsr}[2]{$\smash{\mathrm{FreqGSR^{{#1}}_{{#2}}}}$}
\newcommand{\freqframegsr}[2]{$\smash{\mathrm{FreqFrameGSR^{{#1}}_{{#2}}}}$}
% Empatica
\newcommand{\skintemp}[2]{$\smash{\mathrm{SkinTemp^{{#1}}_{{#2}}}}$}
% Ledalab
\newcommand{\ledalab}{$\mathrm{Ledalab}$}
% Ledalab Global Measures 
\newcommand{\ledalabGlMean}{$\mathrm{Leda_{avg}}$} %Global.Mean 	Mean SC value within response window (wrw)
\newcommand{\ledalabGlMaxDefl}{$\mathrm{Leda_{MaxDefl}}$} %Global.MaxDeflection 	Maximum positive deflection wrw
% Ledalab Standard trough-to-peak (TTP) or min-max analysis
\newcommand{\ledalabttpnscr}{$\mathrm{Leda_{TTP.nSCR}}$} % TTP.nSCR 	Number of significant (= above-threshold) SCRs within response window (wrw)
\newcommand{\ledalabttpampsum}{$\mathrm{Leda_{TTP.AmpSum}}$} %TTP.AmpSum 	Sum of SCR-amplitudes of significant SCRs wrw [muS]
\newcommand{\ledalabttplatency}{$\mathrm{Leda_{TTP.Lat}}$} %TTP.Latency 	Response latency of first significant SCR wrw [s]
\newcommand{\ledalabcdanscr}{$\mathrm{Leda_{CDA.nSCR}}$} %CDA.nSCR 	Number of significant (= above-threshold) SCRs within response window (wrw)
\newcommand{\ledalabcdalatency}{$\mathrm{Leda_{CDA.Lat}}$} %CDA.Latency 	Response latency of first significant SCR wrw [s]
\newcommand{\ledalabcdaampsum}{$\mathrm{Leda_{CDA.AmpSum}}$} %CDA.AmpSum 	Sum of SCR-amplitudes of significant SCRs wrw (reconvolved from corresponding phasic driver-peaks) [muS]
\newcommand{\ledalabcdascr}{$\mathrm{Leda_{CDA.SCR}}$} %CDA.SCR 	Average phasic driver wrw. This score represents phasic activity wrw most accurately, but does not fall back on classic SCR amplitudes [muS]
\newcommand{\ledalabcdaiscr}{$\mathrm{Leda_{CDA.ISCR}}$} %CDA.ISCR 	Area (i.e. time integral) of phasic driver wrw. It equals SCR multiplied by size of response window [muS*s]
\newcommand{\ledalabcdaPhasicMax}{$\mathrm{Leda_{CDA.PhasMax}}$}  %CDA.PhasicMax 	Maximum value of phasic activity wrw [muS]
\newcommand{\ledalabcdaTonic}{$\mathrm{Leda_{CDA.Ton}}$} %CDA.Tonic 	Mean tonic activity wrw (of decomposed tonic component) 

% Posture
\newcommand{\headdist}[1]{$\mathrm{HeadDist_{{#1}}}$}

%shaimaa
\newcommand{\smiley}{:)}

\renewbibmacro*{index:name}[5]{%
  \usebibmacro{index:entry}{#1}
    {\iffieldundef{usera}{}{\thefield{usera}\actualoperator}\mkbibindexname{#2}{#3}{#4}{#5}}}

\newcommand{\herbigemph}[1]{\textsc{#1}}
\newcommand{\tablevspaceherbig}{\\[-.9em]}
\newcommand{\footurl}[1]{\footnote{\url{#1}}}

\DeclareNewSectionCommand
  [
    counterwithin = chapter,
    afterskip = 2.3ex plus .2ex,
    beforeskip = -3.5ex plus -1ex minus -.2ex,
    indent = 0pt,
    font = \usekomafont{section},
    level = 1,
    tocindent = 1.5em,
    toclevel = 1,
    tocnumwidth = 2.3em,
    tocstyle = section,
    style = section
  ]
  {appendixsection}

\DeclareNewSectionCommand
  [
    counterwithin = appendixsection,
    beforeskip=-10pt,
    afterskip=1sp,
    indent = 0pt,
    font = \usekomafont{subsection},
    level = 2,
    tocindent = 3.8em,
    toclevel = 2,
    tocnumwidth = 3.2em,
    tocstyle = section,
    style = section
  ]
  {appendixsubsection}
  
\renewcommand*\theappendixsection{\Alph{appendixsection}}
\renewcommand*{\appendixsectionformat}{\appendixname~\theappendixsection\autodot\enskip}
\renewcommand*{\appendixsectionmarkformat}{\appendixname~\theappendixsection\autodot\enskip}
