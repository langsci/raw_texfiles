\addchap{Preface (1996)}

The goal of this book is to demonstrate that the absence of methodology of grammaticality judgments in linguistics constitutes a serious obstacle to meaningful research, and to begin to propose suitable remedies for this problem. Throughout much of the history of linguistics, judgments of the grammaticality/acceptability of sentences (and other linguistic intuitions) have been the major source of evidence in constructing grammars. While this seems to have been an exceedingly fruitful approach, some skeptics have worried that theoretical linguists are in fact constructing grammars of intuition, which might not have much to do with the competence that underlies everyday production or comprehension of language. Also, in the pseudoexperimental procedure of judgment elicitation there is typically no attempt to impose any of the standard experimental controls, and often the only subject is the theorist himself or herself. Should we linguists be worried? I think so. I survey the way grammaticality judgments are currently used in theoretical syntax, and argue that such uses, together with the problems of intuition and experimental design, demand a careful examination of judgments, not as pure sources of data, but as instances of metalinguistic performance.

Several important issues arise when this view of grammaticality judgments is pursued, including which tasks one should use to elicit them, what people are doing when they give them, and what they can really tell us about linguistic competence. On the assumption that grammaticality judgments result from interactions among primary language faculties of the mind and general cognitive processes, I try to understand the process by identifying and analyzing its component parts. I review the psycholinguistic research that has examined ways in which the judgment process can vary with differences among subjects, experimental manipulations, and spurious features of the stimulus. Parallels with other cognitive behaviors are pointed out. After drawing together the substantive and methodological findings into a schematic picture of what the overall process of giving linguistic intuitions might look like, I propose strategies for collecting these intuitions that avoid the pitfalls of previous work and take account of the conditions that have been shown to influence such judgments. I suggest that we can actually strengthen the case for linguistic universals by giving empirical arguments that much of the variability in judgments can be explained without appealing to differences in Universal Grammar. Finally, I discuss how mainstream linguistic theory might be affected by the growing body of research in this area. I think we will increasingly feel not just a need but also a desire to tackle difficult data questions, particularly as theoretically sophisticated psycholinguistic research increases and we come to understand more about the ways in which linguistic competence is put to use in the mind.
