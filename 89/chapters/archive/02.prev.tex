\chapter{Definitions and Historical Background}\label{sec:2}

\epigraph{\textit{I dislike reliance on intuition as much as anyone. ... We should substitute rigorous criteria just as soon as possible, instead of clinging to intuition.\\[-2\baselineskip]}}{\citep{Chomsky1962}}

\section{Introduction} \label{sec:2.1}

The history of the concepts of grammaticality and acceptability and the notations used to denote them is a long and controversial one. In this chapter I review the important theoretical developments that have defined the use of these concepts in generative grammar, because an investigation into the nature of grammaticality judgments demands a description of precisely what is meant by the term \textit{grammaticality} and an idea of what could constitute a judgment of it. I then turn to the actual practice of linguists working within the generative paradigm today, to see how they apply these notions. The chapter is divided as follows. \sectref{sec:2} is devoted to a brief history of the issues surrounding the notion of grammaticality, the associated terminology, and diverse views on its role in linguistic theory. (See \citet[ch. 7]{Harris1993} for more on this topic.) In \sectref{sec:3}, I survey the varied ways in which grammaticality judgments are used in the linguistic literature, considering the types of data collected and the manner in which these are employed to argue theoretical points. \sectref{sec:4} raises the more fundamental question of whether judgments ought to be used at all, in the light of another bit of social science history\schdash{}namely, the downfall of introspectionism in psychology. \sectref{sec:5} summarizes what I perceive as the major problems with linguists' use of judgments, identifying those that we can tackle with the results of the empirical investigations discussed in subsequent chapters.

\section{A Short History of Grammaticality} \label{sec:2.2}

\epigraph{\textit{The obvious danger exists, however, that Chomsky's\ia{Chomsky, Noam} ``ideal speaker-hearer''\is{ideal speaker-listener} may be itself an artifact, a mere woolen outergarment worn in the attempt to achieve respectability by slipping  the wolf of an arbitrary and artifactual grammar into an unsuspecting flock  of linguistically naive psychologists.\\[-2\baselineskip]}}{\citep{Derwing1979}}
% 19

\noindent Since my goal is to scrutinize the use of grammaticality judgments in generative grammar, I adopt the assumptions of that framework without further comment, although much of this investigation has theory-independent implications.\is{acceptability!versus grammaticality}\is{grammaticality!versus acceptability} Thus, for the relevant definitions I turn to Chomsky,\ia{Chomsky, Noam} and in particular to the familiar \isi{competence/performance distinction}. Chomsky's\ia{Chomsky, Noam} basic point is that we must distinguish what speakers of a language know (subconsciously) about the structure of the language from their actual use of the language. The goal of linguistic theory, under this view, is to describe the knowledge, independent of (and logically prior to) any attempt to describe the role that this knowledge plays in the production, understanding, or judgment of language.\footnote{There has been considerable criticism of this view. See \citet{Greenbaum1976b} for a list of dissenting opinions, and \citet{Derwing1973} for the claim that competence versus performance is not an instance of the kind of idealization usually made in the natural sciences.
}
%\textsuperscript{1}
 Whether a sentence is \textit{grammatical} is a question about competence.\is{acceptability!versus grammaticality}\is{grammaticality!versus acceptability} A grammatical sentence is generated by the speaker's grammar; it is part of the language as delineated by his or her competence. I will assume for the purpose of discussion that whether a sentence is grammatical is determinate in all cases; i.e., that whatever form the competence takes in the mind, it implicitly ascribes (perhaps some degree of) grammaticality or ungrammaticality to each string of words.\footnote{It is conceivable, however, that competence in this sense of statically represented knowledge does not exist. It could be that a given string is generated or its status computed only when necessary, and that the demands of the particular situation determine how the computation is carried out, e.g., by some sort of comparison to prototypical sentence structures stored in memory. Since such a scenario would demand a major rethinking of the goals of the field of linguistics, I will not deal with it further.}
%\textsuperscript{2}
 (See below for a refinement of this assumption.) Whether a sentence is \textit{acceptable} is a question about performance.\is{competence/performance distinction} An acceptable sentence is consciously accepted by a speaker as part of his or her language upon hearing it. This apparently simple distinction is often muddied by the fact that the word \textit{performance} has been used in different ways at different times, by Chomsky\ia{Chomsky, Noam} and others. It is sometimes used to refer to specific instances of behavior, or patterns of behavior in general, as opposed to static knowledge that guides behavior. In other contexts it is used to refer to anything outside of the grammar. including static knowledge of things like discourse structure or mechanisms for using language. In this book, I use the term \textit{performance} to refer to behavior, as opposed to knowledge, including both people's behavior on specific
occasions and their general patterns of behavior. (The importance of the latter distinction will be discussed below.)

Concerning the relationship between performance and competence and the approach to linguistic research, Chomsky states that

\begin{quote}
linguistic theory is concerned primarily with an \isi{ideal speaker-listener}, in a completely homogeneous speech-community, who knows its language perfectly and is unaffected by such grammatically irrelevant conditions as memory limitations, distractions, shifts of attention and interest, and errors (random or characteristic) in applying his knowledge of the language in actual performance. ...

We thus make a fundamental distinction between \textit{competence} (the speaker-hearer's knowledge of his language) and \textit{performance} (the actual use of language in concrete situations).\is{acceptability!versus grammaticality}\is{grammaticality!versus acceptability} Only under the idealization set forth in the preceding paragraph is performance a direct reflection of competence. In actual fact it obviously could not directly reflect competence. \citep[3\textendash{}4]{Chomsky1965}
\end{quote}

Noam Chomsky (personal communication)\ia{Chomsky, Noam} views the \isi{competence/performance distinction} as a simple truism: what we know and what we do are different things.\is{acceptability!versus grammaticality}\is{grammaticality!versus acceptability} The trick is how to learn about the former on the basis of evidence from the latter. In the mid-1960s, Chomsky\ia{Chomsky, Noam} was already mentioning some of the ways in which performance data, including judgment data, might be assessed: ``It is not that these introspective judgments are sacrosanct and beyond any conceivable doubt. On the contrary, their correctness can be challenged and supported in many ways, some quite indirect. Consistency among speakers of similar backgrounds, as well as for a particular speaker on different occasions, is relevant information'' (\citet[56]{Chomsky1964}; I return to these suggestions in Chapters \ref{sec:4} and \ref{sec:5}). Nothing in the above passages implies that variation among speakers is either uninteresting for linguistic theory or necessarily an indication that incorrect judgments are involved. In practice, however, there might be a temptation to take one of these easy outs in order to avoid confronting variation; I return to this matter in \sectref{sec:2.3.2}. Given that linguistic competence is only one contributing factor in any observable behavior of a speaker, it is reasonable to ask how we can hope to gather behavioral evidence about grammaticality.\footnote{See  \citet{OllerEtAl1970} for commentary on the empirical status of generative grammars and the nature of the competence/performance distinction.}
%\textsuperscript{3}
 In \textit{The Logical Structure of Linguistic Theory} (\citet{Chomsky1975}, hereafter \textit{LSLT})\footnote{All pages numbers refer to the 1975 edition.} and \textit{Syntactic Structures}
(\citet{Chomsky1957}, hereafter \textit{SS}) we find the following remarks\footnote{I cite many of Chomsky's\ia{Chomsky, Noam} passages verbatim, because the wording is often subtly nuanced and easily misparaphrased. The reader is then free to disagree with my interpretation. See \citet[ch. 4]{Matthews1993} for a detailed review of Chomsky's\ia{Chomsky, Noam} early writings on these and related topics.
}
%\textsuperscript{5}
 (I have emphasized some hedges that indicate that no particular operational  test can be expected to yield the right result in all cases):\is{acceptability!operational tests for}

\begin{quote}
We know that a speaker of the language can select, among sequences that he has never heard,  \textit{certain} grammatical sentences, and that he will do this in much the same way as other speakers. We might test this by a direct determination of some sort of ``bizarreness reaction,'' or in various indirect ways. (\textit{LSLT},\ia{Chomsky, Noam} p. 95)\footnote{The emphasis in all quoted passages in this chapter is my own, unless otherwise indicated.
}
%\textsuperscript{6}



Yet (1) [\textit{Colorless green ideas sleep furiously}], though nonsensical, is grammatical, while (2) [\textit{Furiously sleep ideas green colorless}] is not.\is{acceptability!versus grammaticality}\is{grammaticality!versus acceptability}  Presented with these sentences, a speaker of English will read (1) with a normal sentence intonation, but he will read (2)  with a falling intonation on each word: in fact, with just the intonation pattern given to any sequence of unrelated words.\is{prosodic features, as evidence of ungrammaticality} He treats each word in (2)  as a separate phrase. Similarly, he will be able to recall\is{memory for sentence, as evidence of (un)grammaticality} (1) much more easily than (2), to learn it much more quickly, etc. (\textit{SS}, p. 16)

Such sentences with conjunction crossing constituent boundaries are also, in general, marked by special phonemic features such as extra long pauses, ... contrastive stress and intonation, failure to reduce vowels and drop final consonants in rapid speech, etc.\is{acceptability!versus grammaticality}\is{grammaticality!versus acceptability} Such features \textit{normally} mark the reading of non-grammatical strings.\ia{Chomsky, Noam} (\textit{SS}, pp. 35\textendash{}36, fn. 2)\footnote{Regarding this passage, \citet{Chomsky1961} states that he was careful not to suggest \textit{general} criteria for grammaticality.
}

 \end{quote}

At the same time, however, it was apparent that behavioral evidence was not \textit{always} the last word.\is{acceptability!versus grammaticality}\is{grammaticality!versus acceptability} Theoretical considerations could also dictate that some sentences \textit{must} be grammatical, regardless of how speakers might react to them. With regard to sentences containing embedded \textit{if-then} and \textit{either-or} pairs, Chomsky\ia{Chomsky, Noam} states, ``Note that many of the sentences ... will be quite strange and unusual. ... But they are all  sentences, formed by processes of sentence construction so simple and elementary that even the most rudimentary English grammar would contain them'' (\textit{SS}, p. 23). The reasoning seems to be this: given that
certain sentences are uncontroversially part of the language, our intuitions about what grammars can look like tell us that certain other sentences must also be part of it, although our judgments of these latter sentences are not so clear-cut. Such reasoning is not, in and of itself, sufficient to conclude that a sentence is grammatical.\footnote{Noam Chomsky (personal communication)\ia{Chomsky, Noam} believes that ``everyone in the entire history of the subject would agree'' with my claim, and that the quoted passage from \textit{SS} implies nothing to the contrary. If that is true then the point of the passage is unclear to me; perhaps Chomsky\ia{Chomsky, Noam} was presupposing other kinds of evidence as well.
} It must be shown that the factors that make it less than acceptable are extragrammatical, i.e., that they exist independent of language structure.

Originally, Chomsky suggested that the evidence used to construct theories be limited to clear cases,\is{clear cases!use of} that is, ones where our intuitions leave no doubt as to whether the sentence is acceptable or unacceptable, as the following passages state explicitly:\footnote{\citet{Carden1973} argues that in the 1960s, data on which there were disagreements were indeed considered outside the domain of the theory.
}
\begin{quote}
Our purpose is to construct an integrated and systematic theory, which, when applied rigorously to linguistic material, gives the correct analysis \textit{for the cases where intuition} (or experiment, under more desirable circumstances) \textit{makes a clear decision.} (\textit{LSLT}, p. 415)

We may assume for this discussion that certain sequences of phonemes are definitely sentences, and that certain other sequences are definitely non-sentences. In many intermediate cases we shall be prepared to let the grammar itself decide, when the grammar is set up in the simplest way so that it includes the clear sentence and excludes the clear non-sentences. (\textit{SS}, p. 14)
 \end{quote}
 
\noindent 
What is left unsaid here is \textit{why} it is sometimes legitimate to let the grammar decide, and how we are to know when such a move is allowed. The justification seems to be a parsimony argument: Why say that a marginal sentence is ungrammatical if that would complicate the theory? If we had access to no information about the sentence in question beyond a judgment that it is marginal, this would be a reasonable principle to follow, but I argue in subsequent chapters that we can do better by looking for additional evidence that could bear on the status of the sentence. We can thus hope to avoid missing true complications in the grammar. In answer to the second question raised above, we can then also avoid the need to know when it is safe to let the grammar decide, since it is actually never safe but
always potentially erroneous. Fortunately, in practice it seems to be done fairly infrequently anyway: see \citet{Labov1975} for specific instances in which Chomsky\ia{Chomsky, Noam} admits that data are unclear but proceeds to construct his theory solely on the basis of his own intuitions. (This in turn is problematic for other reasons.)

Let me make a brief aside about terminology in reference to Chomsky's\ia{Chomsky, Noam} writings\is{acceptability!factors contributing to}.\is{acceptability!versus grammaticality}\is{grammaticality!versus acceptability} Note that Chomsky generally does \textit{not} use the terms \textit{intuition} and \textit{judgment} interchangeably. It is my understanding that judgment is a product of performance and intuition is part of competence.\is{competence/performance distinction}\is{competence (grammatical)!versus performance} When Chomsky\ia{Chomsky, Noam} says that ``the speaker has an `intuitive sense of grammaticalness' '' (\textit{LSLT}, p. 95),\footnote{In Chomsky's\ia{Chomsky, Noam} early writings, one often finds the term \textit{grammaticalness}\is{grammaticalness, as technical term@\textit{grammaticalness}, as technical term} used instead of \textit{grammaticality}. Most people treat these as synonymous, but this was apparently not Chomsky's\ia{Chomsky, Noam} original intent: ``About the term `grammaticalness,' ... I purposely chose a neologism in the hope that it would be understood that the term was to be regarded as a technical term, with exactly the meaning that was given to it, and not assimilated to some term of ordinary discourse with a sense and connotations not to the point in this context'' (Noam Chomsky,\ia{Chomsky, Noam} quoted in \citet[14]{Paikeday1985}).
}
this does not translate into the ability to \textit{judge} grammaticalness. But the very close semantics are probably responsible for the misapprehension alluded to in \chapref{sec:1}, whereby the two are equated and thus judgments are seen as directly reflecting competence, since competence consists of intuitions. This was not Chomsky's\ia{Chomsky, Noam} intent (see \citet[21]{Chomsky1965}), although passages like the following could easily serve to mislead the unwary reader: ``The theory is refuted if the \textit{judgments} are not in accord with the predictions of the grammar'' \citep[36]{Chomsky1985}. In the accompanying footnote, however, we find a qualification: ``Note that there is a further idealization here, in that we abstract away from other factors that may interact with knowledge of language to determine judgments.'' As for acceptability, which speakers \textit{can} judge, the Introduction to \citet{Chomsky1985} describes the concept as follows: ``Sentences are acceptable (or perhaps acceptable under particular circumstances) if they are suitable, appropriate, adequate to the purpose at hand, etc. The competence grammar contributes to determining acceptability, but the latter concept involves many other factors'' \citep[8]{Chomsky1985}.\footnote{I have included passages from the 1973 Introduction of \citet{Chomsky1985} as representative of his early work, since it provides background to \textit{LSLT}. Of course, the intervening years might have brought a change in perspective.}
It seems that with linguistic training one can learn to abstract away from some of these factors: linguists certainly assume that they can judge a sentence's wellformedness independently of its appropriateness to any actual discourse situation, for example.
 
% Definitions and Historical Background  25

In \textit{Aspects of the Theory of Syntax} (\citet{Chomsky1965}, hereafter \textit{Aspects)}, the relationship between acceptability and grammaticality was more explicitly discussed in the following well-known passage\is{acceptability!versus grammaticality}:\is{grammaticality!versus acceptability}

\begin{quote}
For the purposes of this discussion, let us use the term ``acceptable'' to refer to utterances that are perfectly natural and immediately comprehensible without paper-and-pencil analysis, and in no way bizarre or outlandish. Obviously, acceptability will be a matter of degree\is{acceptability!degrees of}, along various dimensions. One could go on to propose various operational tests to specify the notion more precisely (for example, rapidity, correctness, and uniformity of recall and recognition, normalcy of intonation). ... The more acceptable sentences are those that are more likely to be produced, more easily understood, less clumsy, and in some sense more natural. The unacceptable sentences one would tend to avoid and replace by more acceptable variants, wherever possible, in actual discourse.

The\ia{Chomsky, Noam} notion ``acceptable'' is not to be confused with ``grammatical.'' Acceptability is a concept that belongs to the study of performance, whereas grammaticalness belongs to the study of competence\is{competence/performance distinction}. ... Like acceptability, grammaticalness is, no doubt, a matter of degree ... but the scales of grammaticalness and acceptability do not coincide. Grammaticalness\is{grammaticalness, as technical term@\textit{grammaticalness}, as technical term} is only one of many factors that interact to determine acceptability. Correspondingly, although one might propose various operational tests for acceptability, \textit{it is unlikely that a necessary and sufficient operational criterion\is{criterion versus evidence (for grammaticality)}\is{evidence versus criterion (for grammaticality)}\is{grammaticality!operational evidence versus criterion for} might be invented for the much more abstract and far more important notion of grammaticalness}. \citet[10--11]{Aspects}
\end{quote}

As Jim McCawley (personal communication)\ia{McCawley, {James D.}} points out, the above definition of \textit{acceptable} is dispositional\is{acceptability!dispositional nature of}\is{dispositional nature of acceptability}, in that it refers to a tendency across multiple actual encounters with a sentence, rather than to a speaker's reaction on any one occasion. Therefore, whether a sentence is \textit{acceptable} in general cannot in principle be determined on the basis of a single situation wherein it is \textit{accepted}. However, a \textit{judgment} about acceptability is presumably an attempt on the part of speakers to assess what their reactions to a sentence \textit{would be} across a range of situations. Thus, to elaborate on the definition of acceptability I gave above, whether a sentence is acceptable is a question about performance in three senses: any particular instance of a speaker accepting or rejecting a sentence is an act of performance, i.e., behavior. Any sort of generalization across many such instances is a generalization about performance. A judgment of one's disposition\is{acceptability!dispositional nature of}\is{dispositional nature of acceptability} towards accepting or rejecting a sentence is itself a type of performance.

The absence of an operational criterion\is{criterion versus evidence (for grammaticality)}\is{evidence versus criterion (for grammaticality)}\is{grammaticality!operational evidence versus criterion for} for grammaticality has evoked negative reactions from many quarters.\is{acceptability!versus grammaticality}\is{grammaticality!versus acceptability}  As \citet{Reich1969} puts it, ``when confronted by adverse data, Chomsky\ia{Chomsky, Noam} retreated from his empirical position of 1957, to a theory that he himself admits cannot be tested empirically''; \citet{Marks1967} comments on what he perceives as the incoherence of this position; \citet{Householder1973} calls it ``the paradox of linguistics: the only possible way of determining whether or not a grammar is correct is by consulting the speaker's intuitions, but they are inaccessible'' (p. 365, fn. 1); \citet{GleitmanEtAl1970} have the following reaction: ``[Acceptability] tests cannot invalidate, they can hardly bear on, a theory which when pushed will withdraw into successively deeper reaches of the mind'' (p. 25). Chomsky (personal communication)\ia{Chomsky, Noam} believes that such comments belie a confusion between evidence and criterion:\is{criterion versus evidence (for grammaticality)}\is{evidence versus criterion (for grammaticality)}\is{grammaticality!operational evidence versus criterion for} while speakers' behavior on some task\schdash{}for instance, the intonation they use in reading a string of words\schdash{}might constitute a piece of \textit{evidence} concerning its grammaticality, one cannot \textit{define} the notion \textit{grammatical} in terms of one or several of such tests; it is defined in terms of some notion of the contents of the mind. (To borrow a favorite analogy of Chomsky's,\ia{Chomsky, Noam} while a litmus test might provide some evidence about whether a substance is an acid or a base, chemistry does not define the concept \textit{acid} in terms of this test.) It seems to me that, in principle, there might someday be an operational criterion for grammaticality, but it would have to be based on direct study of the brain,\is{brain, as source of grammaticality data} not on human behavior, if it turns out to be possible to discern properties of the mind (e.g., the precise features of the grammar) from physical properties of the brain.

It does not make any sense to speak of grammaticality judgments given Chomsky's\ia{Chomsky, Noam} definitions, because people are incapable of judging grammaticality\schdash{}It is not accessible to their intuitions (\citet[51]{Newmeyer1983}; \citet{Gombert1992}). Linguists might construct arguments about the grammaticality of a sentence, but all that a linguistically naive subject can do is judge its acceptability.\is{acceptability!versus grammaticality}\is{grammaticality!versus acceptability} Nevertheless, in the remainder of this book I will follow the existing literature in treating \textit{grammaticality judgment} and \textit{acceptability judgment} as synonyms,\footnote{It is possible that researchers who have defined grammaticality and/or acceptability in other ways might make a principled distinction between two types of judgment (this appears to be the case for \citet{Langendoen1973}), but since I follow Chomsky's\ia{Chomsky, Noam} definitions I collapse the terms.}
%\textsuperscript{12}
 with the understanding that the former is unquestionably a misnomer,\is{grammaticality judgment@\textit{grammaticality judgment}!as misnomer} and only the latter is a
% Definitions and Historical Background  27
sensible notion. I do so because this usage occurs in many of the passages I quote, and because \textit{acceptability judgment} can have misleading connotations of its own in certain contexts. For instance, the term can indicate that one is discussing contextual appropriateness rather than structural wellformedness. I will continue to follow Chomsky's\ia{Chomsky, Noam} definitions in other contexts when the distinction is important, for example, in \textit{acceptable sentence} versus \textit{grammatical sentence}.

Given that grammaticality is what Chomsky\ia{Chomsky, Noam} seeks to investigate, it would not be surprising if he saw no useful purpose in the systematization of linguistic data collection.\is{acceptability!versus grammaticality}\is{grammaticality!versus acceptability} In the end, no single empirical fact can be crucial to the issues at hand. At the time of \citet{Aspects} this seems to have been his view, but he allowed for the possibility that circumstances might change, that there might be room at some future time for a methodology more systematic than reliance on everyday common sense:

\begin{quote}
There are, in other words, very few reliable experimental or data-processing procedures for obtaining significant information concerning the linguistic intuition of the native speaker. It is important to bear in mind that when an operational procedure is proposed, it must be tested for adequacy ... by measuring it against the standard provided by the tacit knowledge that it attempts to specify and describe. ... If operational procedures were available that met this test, we might be justified in relying on their results in unclear and difficult cases. This remains a hope for the future rather than a present reality, however. ... There is no reason to expect that reliable operational criteria for the deeper and more important theoretical notions of linguistics (such as ``grammaticalness'' and ``paraphrase'') will ever be forthcoming. ... The critical problem for grammatical theory today is not a paucity of evidence but rather the inadequacy of present theories of language to account for masses of evidence that are hardly open to serious question. ... It seems to me that sharpening of the data by more objective tests is a matter of small importance for the problems at hand. ... \textit{Perhaps the day will come when the kinds of data that we now can obtain in abundance will be insufficient to resolve deeper questions concerning the structure of language.}(\citet[19--21]{Aspects})
\end{quote}

\noindent
I argue that this day has come, some 30 years later. I devote \sectref{sec:2.3} to demonstrating that the questions linguists are now addressing rely crucially on facts that are indeed ``open to serious question.''


Note also that Chomsky\ia{Chomsky, Noam} again assumes that there is a core of\is{acceptability!versus grammaticality}\is{grammaticality!versus acceptability} ``unquestionable data concerning the linguistic intuition of the native speaker'' (\citet[20]{Aspects}), which would presumably include judgments of some sort, and that these ``obvious'' facts would keep linguists busy for a long time, thus postponing the need for reliable tests applicable to ``less obvious'' cases.\footnote{An argument against this position is that the large masses of unquestionable data, if indeed they exist, might still be of insufficient quality for linguistic theory, if they do not bear on the crucial issues that it must address (\citealt{Botha1973}; \citealt{Labov1972a}). \sectref{sec:2.3} shows this to be true.}
%\textsuperscript{13}
 That is, for some sentences, acceptability judgments provide transparent evidence about grammaticality, while still not constituting judgments  \textit{of} grammaticality in the literal sense. But how is it determined whether a given datum constitutes such a clear case?\is{clear cases!difficulty of identifying} For whom must it be clear \citep{Ringen1979}? The problem, of course, is that each investigator is free to pick and choose these ``unquestionable'' cases to suit the theory. (\citet{McCawley1976}
 argues that, by Chomsky's own definition of grammaticality, \textit{all} sentences must be considered unclear cases, because we never have direct information about grammaticality.) That there is no standard way to make this decision is argued in detail by \citet{Botha1973}: ``The level of rationality at which grammatical inquiry and general-linguistic inquiry are conducted would be raised if it were clear ... under what circumstances an intuitive evidential statement may be properly regarded as being evident'' (p. 188); ``Transformational grammar lacks a set of conditions ... governing the evidentness or obviousness of intuitive evidential statements'' (p. 193). Furthermore, even at an intuitive level, clear-case\is{clear cases!use of} judgments are not necessarily windows into competence:\is{competence/performance distinction} ``In terms of the notion `clear case' spurious linguistic intuitions could, despite their spuriousness, qualify for membership of the evidential corpus; in terms of the notion `unclear case' linguistic intuitions which were both genuine and correct could, despite their genuineness and correctness, be denied membership of this corpus'' (p. 206).\is{clear cases!use of} \citet{Botha1981} defines a spurious judgment, in contrast to a genuine one, as one that does not reflect competence, because it has been influenced by extralinguistic factors. In this book, I propose more reliable means to make these discriminations.

As our concept of the theory is refined, the status of any given sentence can change from ungrammatical to grammatical, or vice versa.\is{acceptability!versus grammaticality}\is{grammaticality!versus acceptability} (See \citet{McCawley1979} for a variety of definitions of grammaticality that have been used by different theoretical factions at different times.) For instance, in \textit{LSLT} Chomsky\ia{Chomsky, Noam} examines the naturalness of particle movement as a function of the complexity of the intervening NP and concludes, ``This is systematic behavior, and we might expect that a grammar should be able to state it'' (p. 477). But in \citet{Aspects} he says of the same sentences (and, more celebratedly, of multiply center-embedded ones), ``It would be quite impossible to characterize the unacceptable sentences in grammatical terms. For example, we cannot formulate particular rules of the grammar in such a way as to exclude them'' (pp. 11\textendash{}12). The latter claim seems to be an exaggeration, at least with regard to center embedding. In fact, \citet{KatzEtAl1976} elaborate Chomsky's\ia{Chomsky, Noam} position by saying that a recursion counter would do the job, but this is the only part of the grammar of English where it would be used, which should make us suspect that this is not the right analysis of the phenomenon. Noam Chomsky (personal communication)\ia{Chomsky, Noam} finds the idea of a recursion counter ``ridiculous, since it is obvious that whatever is involved in constraining recursion is quite different in character from the devices made available in UG\is{Universal Grammar} ... and may not even involve language.'' If the implicit empirical argument here were spelled out and supported, it would constitute a reasonable basis for not attributing limits on recursion to the grammar.

In general, there seem to be three lines of defense by which a theoretical claim is protected from potentially falsifying data,\is{falsifying a theory} before any change in the theory can be incited.\is{acceptability!versus grammaticality}\is{grammaticality!versus acceptability} This book is intended to help objectify two of these three procedures. For instance, when faced with a sentence that apparently contradicts a claim, e.g., an acceptable sentence that a grammatical rule excludes, the first line
of defense would be to argue that the data are invalid, that the sentence is not really acceptable as claimed. While it is trivial to make this statement for one's own intuitions, such arguments ought to be supported by empirical investigations of others as well. To the extent that we can standardize this process, we can reduce data disputes. The second defense is to claim that the data are not relevant to the theoretical issue at hand, that the sentence is good because it is allowed by some other part of the grammar and is not under the jurisdiction of the disputed construct. This approach generally relies on logical reasoning and might be subject to differing opinion but not to factual dispute; therefore, I do not discuss it further. The third defense, typically indicative of the least understanding on the part of the theory's proponent, is to say that the sentence \textit{is} prohibited by the grammar, but non-grammatical factors are causing judgments not to reflect this fact. Until we have an explicit understanding of such factors, such a claim is unfalsifiable. (For a much more detailed examination of the roles of argumentation and evidence in generative theory, see \citet{Botha1973}.) This state of affairs is presumably what prompts \citet{Postal1988} to quip:

\begin{quote}
Great strides are being made in linguistic rhetoric, whose progress puts the stasis in mere description and theorizing to shame. In the great rhetoric laboratories of the north-eastern United States, defensive shields are being perfected that can render any theory virtually impervious to factual corrosion. (p. 129)
\end{quote}

As I see it, this is precisely why we \textit{should} strive for a better understanding of acceptability judgments.\is{acceptability!versus grammaticality}\is{grammaticality!versus acceptability} It would allow us a \textit{principled} way to establish to what extent any such piece of evidence should be considered to bear on the grammar. We will still not be able to draw direct conclusions from such data, but it will at least be a matter of objective fact what the relevant data are. We clearly do sometimes use speaker judgments as evidence about grammaticality, so we should rigorously define when and why they can and cannot be used as such, and then try to expand the range of cases where they can be used. As \citet{Birdsong1989} states, until we do so, if we do not agree on what our data represent, we cannot hope to agree on an analysis. If we can understand what factors intervene between the grammar and performance, we can circumscribe the cases where these factors might cause (un)acceptability not to reflect (un)grammaticality, and exclude these cases as evidence. Chomsky\ia{Chomsky, Noam} seems to agree with this approach:

\begin{quote}
In actual practice, linguistics as a discipline is characterized by attention to certain kinds of evidence that are, for the moment, readily accessible and informative: largely, the judgments of native speakers. Each such judgment is, in fact, the result of an experiment, one that is poorly designed but rich in the evidence it provides. In practice, we tend to operate on the assumption, or pretense, that these informant judgments give us ``direct evidence'' as to the structure of the language, but, of course, this is only a tentative and inexact working hypothesis, and \textit{any skilled practitioner has at his or her disposal an armory of techniques to help compensate for the errors introduced}. In general, informant judgments do not reflect the structure of the language directly; judgments of acceptability, for example, may fail to provide direct evidence as to grammatical status because of the intrusion of numerous other factors. \citep[36]{Chomsky1986}
\end{quote}

\noindent
If we could go a (large) step further and deduce a general reverse mapping from acceptability to grammaticality, we could in principle determine the grammatical status of any sentence operationally, and thus fully specify the range of facts over
which grammars must have scope. ``We require a science of linguistic introspection to provide a theoretical and empirical basis for including some acceptability judgments as syntactically relevant and excluding others'' \citep[195]{Bever1974}.

Under Bever's proposed approach, only those unacceptable sentences whose badness cannot be explained by any plausible extragrammatical aspect of speech behavior are ungrammatical.\is{acceptability!versus grammaticality}\is{grammaticality!versus acceptability} (Bever's\ia{Bever, {Thomas G.}} example of such a case is \textit{I hope it for to be stopping raining when I am having leaving}; see also \citet{Bever1971}.)\footnote{Bever\ia{Bever, {Thomas G.}} goes on to make some preliminary suggestions about sentential properties that will likely affect acceptability but are outside the realm of the grammar. These include sentence length, absurdity, difficulty of comprehension and difficulty of pronunciation\schdash{}these will be the subject of \sectref{sec:5.3} (see also \citealt{KatzEtAl1976}).}
%\textsuperscript{14}
 The argument is that ``if a constraint may be adequately treated by independently motivated systems outside the grammar, its inclusion in the grammar is unwarranted and obscures the descriptive and explanatory power of the grammar'' \citep[150]{BeverEtAl1976a}. Of course, the paradigmatic example in this category would be the short-term memory limitations that are said to result in the unacceptability of multiply center-embedded sentences.\is{center-embedding, multiple} More generally, the parser may be misled (garden-pathed) into an incorrect parse that makes a well-formed sentence seem bad,\is{garden path sentence!effect on grammaticality judgments} as in the following example from \citet{Ellis1991}, where the
parser presumably first tries to attach \textit{which book} as the object of \textit{believes}.

\ea \label{ex:2:1}
    Which book did you say John believes offended many people?
\z

	   

We must add to this proposal the converse case, in which acceptable sentences that have been ameliorated by extragrammatical factors are not the responsibility of the grammar.\is{acceptability!versus grammaticality}\is{grammaticality!versus acceptability} Again, multiple center embedding\is{center-embedding, multiple} provides a plausible example. Sentences like (\ref{ex:2:2}a) below are clearly ungrammatical, lacking the necessary number of verb phrases, but because participants apparently cannot keep track of the number of subject noun phrases that have accumulated, they often judge such sentences to be acceptable, in contrast to the grammatical (\ref{ex:2:2}b), which is judged unacceptable \citep{Frazier1985}.

\ea \label{ex:2:2}
\ea
The patient the nurse the clinic had hired met Jack.
\ex
The patient the nurse the clinic had hired admitted met Jack.
\z
\z

\noindent
Another class of examples in this category involves the use of \isi{resumptive pronouns}, as in (\ref{ex:2:3}a), which is clearly better than the version without a pronoun in (\ref{ex:2:3}b), even though nonlinguists may generally reject both utterance types:


\ea\label{ex:2:3}
\ea[ ]{the guy who they don't know whether he wants to come or not}
\ex[*]{the guy who they don't know whether wants to come or not}
\z
\z


\noindent
\citet{Kroch1981} argues that possible grammar-based accounts of the distribution of such \isi{resumptive pronouns} are unappealing, and takes this as a reason to consider a processing account in terms of on-line production, which he argues is more straightforward and hence is to be preferred. On this account, resumptive pronouns are the result of the generator's inability to look far enough ahead to determine that the trace of an already uttered \textit{wh}-expression is within an island.\is{island, syntactic} Thus, the production mechanism is forced to produce ungrammatical sentences in some cases, and this fact somehow renders them more acceptable than other ungrammatical sentences, for reasons that are not entirely obvious. (Examples like this motivate a link between the generator and the judgment routine in the model to be presented in \sectref{sec:6.2}.) While this argument for attributing judgments to performance rather than to competence\is{competence/performance distinction} is theory-internal, it makes empirical predictions, for instance that carefully planned speech or writing will not contain sentences like (\ref{ex:2:3}a) at all; the astute reader will find a counterexample to this claim in Footnote \ref{fn:2:17}.

(See \citealt{LangendoenEtAl1973} and \citealt{Bever1974} for more instances of supposedly ungrammatical  acceptable sentences.)

A third possibility we must consider when deciding the grammaticality status of a sentence is that it might have no status at all.\is{acceptability!versus grammaticality}\is{grammaticality!versus acceptability} This possibility is raised by \citet{Morgan1972}, who presents unquantified data on interspeaker judgment differences to show that certain pathological cases of subject-verb agreement{agreement, subject-verb} in English display a large degree of variation and uncertainty:

\ea\label{ex:2:4}
\ea[??]{Are/??Is John or his parents here?}
\judgewidth{??} % Solves the problem of multi-line examples with a ??-judgment
\ex[\hspaceThis{??}]{I, who the FBI thinks *am/*is an anarchist, will doubtless be here.}
\z
\z

%\todo{Can we get rid of the space between "??" and "Are"? Previous note from Felix I think: Apparently, gb4e does not think indenting gram. judg. would is a good idea.}

\noindent
Morgan speculates that this  variation might exist because speakers' grammars simply do not contain a rule elaborate enough to apply in such cases, so when forced to make judgments they may apply ad hoc strategies. It is hard to see what the general criteria for reaching this conclusion of grammatical indeterminacy\is{indeterminacy in grammar} might be. Morgan's argument is based on the inability to find plausible grammatical principles that cover all the judgments that are firm, plus the fact that many speakers find \textit{neither} variant of the problematic sentences grammatical. Presumably, indeterminacy is distinguished in principle from mere interspeaker grammar differences if individual speakers seem not to have systematic judgment patterns, but what constitutes a systematic pattern is not always obvious a priori.


Naturally, there might not be total agreement on the status of a particular effect.\footnote{\citet{Newmeyer1983} believes that discrepancies on this point account for many cases of what appear to be \isi{data disagreements} among theorists.
} For instance, \citet{Reich1969} and \citet{Spencer1973} question why limitations on center embeddings should not  be taken to reflect the grammar.\is{acceptability!versus grammaticality}\is{grammaticality!versus acceptability} (See \citet{KatzEtAl1976} for discussion of how such decisions can be argued for.) More broadly, the role of meaning vis-à-vis the grammar has been a point of great debate over the history of generative linguistics. The distinction between syntactic and semantic intuitions is liable to be based on one's own theory \citep{Cohen1981}. McCawley, in his introduction to \citet{Postal1976}, expresses the view that such arguments are pointless. Once you know why a sentence is bad, it does not matter whether that cause is considered inside or outside the grammar.

\begin{quote}
If a distinction between ``ungrammatical'' and ``unacceptable'' is to be made, the ungrammatical items are those unacceptable items whose unacceptability is for a reason that the linguist takes to be in his province. That means that in order to tell whether an unacceptable item is ungrammatical, one must identify why it is unacceptable. But if one can identify why it is unacceptable, nothing is gained by in addition classing it as grammatical or ungrammatical. A grammar that specifies what is grammatical and what is ungrammatical but does not enable one to pinpoint what is wrong with all unacceptable sentences (or better, unacceptable uses of sentences) is of questionable value; and if a grammar performs the latter task, there is no obvious reason why one should care whether it performs the former. (p. 202)
\end{quote}

\noindent
I do not accept this argument. First, if one can identify why a sentence is unacceptable, then by definition one has also determined its grammatical status. It is grammatical if the cause of its unacceptability is behavioral (i.e., outside the grammar), and ungrammatical otherwise. No further arbitrary decision is required. Second, if one accepts that the nature of explanation differs for different kinds of unacceptability (e.g., that some kinds are due to UG\is{Universal Grammar} and some are due to working memory constraints), then it is important to establish which kind of explanation a particular datum calls for. A grammar of the type that McCawley finds to be of questionable value is useful if it provides a theory of a well-defined module of the mind. It will be less useful if it simultaneously tries to provide a theory of some unrelated module.

In fact, as McCawley has also pointed out, linguists are not really interested
in judgments about strings of words.\is{acceptability!versus grammaticality}\is{grammaticality!versus acceptability} Rather, they seek to know whether a given string is a grammatical expression of a particular meaning. In this sense, semantics is crucially involved.

\begin{quote}
The alleged ability of speakers of a language to distinguish between ``grammatical'' and ``ungrammatical'' strings of words is about as rare and as perverse as the ability to construct puns, an ability to which I believe it is closely related. Anyone who has taught an introductory syntax course has had the experience of presenting an ``ungrammatical'' example only to be told by some smart-aleck about an unsuspected interpretation on which the sentence is quite normal. ... Such interventions are usually greeted with the sort of groans that are the accepted form of expressing appreciation of puns, and they provide the same sort of comic relief that puns do in the midst of what is at times a boring enterprise. However, the extent to which comical virtuosity is required for a person to notice the existence of such an interpretation, as contrasted with the ease with which one recognizes its acceptability once it has been pointed out, shows that the strings of words on which grammaticality judgments are allegedly made exist only as typographical or acoustic objects, not as perceptual or cognitive objects, just as the Necker cube exists only as a graphic object and not as a perceptual object, as contrasted with its two interpretations, which do exist as perceptual objects. These points are in fact implicitly recognized  by virtually  all members of the generative grammar community, as is evidenced by the fact that the ``grammaticality judgments'' on which they base arguments systematically ignore interpretations other than those relevant to the points at issue, and the ``sentences'' are often exhibited with supplementary information as to the intended interpretation, for example, subscripts to indicate purported coreference of items, much in the same way that one might present the Necker cube with shading or context that picked out one of the interpretations. Generative grammarians speak as if they were doing linguistics in terms of something like sense data, when they in
fact are doing it in terms of something more like the perceptual data of gestalt psychologists.  \citep[78\textendash{}79]{McCawley1982}
\end{quote}

It is interesting to examine what other theoretical linguists today believe about the types of evidence that are available to them. The following unusually explicit passage (whence the epigraph of \chapref{sec:1} is drawn) confirms that judgment data are still considered the primary source of linguistic evidence, and thus
underscores the importance of studying their properties. It also reaffirms that we continue to lack a principled criterion\is{criterion versus evidence (for grammaticality)} \is{evidence versus criterion (for grammaticality)} for choosing data.

\begin{quote}
No kind of data is excluded in principle, only as a matter of practice \schdash{} judicious practice, we think, but not irrefutable. ... Grammarians use data like ``such and such a string of words is a sentence in such and such a language'' or ``such and such a string of words means such and such,'' where such facts are determined by native speakers of the languages in question. Data of this kind vary enormously in quality\schdash{}ranging from the clear fact that \textit{He are sick} is not grammatical in English to the rather subtle judgments involved in determining whether \textit{John} and \textit{his} can refer to the same person in \textit{His mother likes John}. Despite this variation in quality and despite the fact that linguists have not formulated a ``methodology of sentence judgments,'' such data remain the principal source of information about grammar, again, not as a matter of principle, but because they have so far provided successful insights.

Thus, the study of grammar is not the study of sentence judgments; rather, sentence judgments are our best current avenue to the study of grammar. In other words, the grammar is a real thing, not an artifact erected on top of an arbitrarily demarcated set of facts or types of facts. Therefore, it is often difficult to determine whether a given fact bears on grammar or not; this is not an arbitrary decision, but ultimately an empirical question about how the world divides up.''   \citep[2]{vanRiemsdijk1986}
\end{quote}

\noindent
It is at least possible a priori that the reason judgments seem to work well for linguists is that they can be manipulated and distorted to suit the purpose of an analysis. In this connection, \citep[82]{Birdsong1989} suggests that ``linguistics is a potentially fraudulent enterprise when elicitation data can be manipulated to instantiate pet theoretical analyses. It would be hard to imagine a more powerful argument for understanding the psychology of metalinguistic performance.'' Indeed. Most linguists profess ignorance of the reasons why judgment data are useful. For example, Baker notes:

\begin{quote}
We focus on those linguistic behaviors which \textit{for some reason} are most likely to reveal the mental structures in their true light. The situation can be likened to the physicist who tries to determine the force of gravity. ... Unfortunately there is every indication that much of the linguistic behavior we have record of is like the autumn leaf\schdash{}complicated by many other external factors. ... I do not claim to have the wisdom to reliably discern which linguistic behaviors are like autumn leaves and which are like steel ball bearings. \citep[29]{Baker1988}
\end{quote}

\noindent
Despite his lack of ``wisdom,'' Baker implicitly chooses to continue the tradition of making primary use of judgment data.

Before closing this section, I wish to dispel one possible misinterpretation that has been an all too common result of linguists' admission that judgments are their primary data. That is the view that grammaticality judgments are the objects of study in linguistics, that its purpose is to describe them, and that they therefore constitute the only kind of data the field requires (see \citet{Sampson1975} for views along these lines). That this is not the case has been stated repeatedly and unequivocally by Chomsky:

\begin{quote}
To say that linguistics is the study of introspective judgments would be like saying that physics is the study of meter readings, photographs, and so on, but nobody says that. ...

It just seems absurd to restrict linguistics to the study of introspective judgments, as is very commonly done. ... Many textbooks that concentrate on linguistic argumentation for example are more or less guided by that view. \citep[33\textendash{}34]{Chomsky1982}
\end{quote}

\section{The Use of Judgment Data in Linguistic Theory}  \label{sec:2.3}
\subsection{Introduction} \label{sec:2.3.1}

\epigraph{\itshape I cannot make native speakers behave the way Mr. Chomsky\ia{Chomsky, Noam} says they do, and if they insist on breaking his rules, it might be hard to pin down the grammatical and the ungrammatical.\\[-2\baselineskip]}{\citep{Sledd1962}}

\noindent The purpose of this section is twofold. First, I wish to demonstrate the claim, made in \sectref{sec:2.2}, that current issues in linguistic theory require nonobvious data for their resolution. Second, and relatedly, I wish to illustrate that the use of judgments in theoretical work has moved far beyond making distinctions of good versus bad, or even graded goodness and badness decisions. The situation is characterized most poignantly by Levelt:

\begin{quote}
In the early years of the transformational grammar [the low reliability of absolute grammaticality judgments] was not an important issue, since the `clear cases, i.e., the highly uncontroversial cases of grammaticality and ungrammaticality, were sufficient for constructing and testing linguistic theory. It was expected that, in its turn, the theory constructed in such a way would decide on the `unclear cases.''\is{clear cases!use of} This hope has vanished. \citep[88]{LeveltEtAl1977}

It has slowly but surely become clear that it is not possible, on the basis of incontrovertible, directly evident data, to construct a theory so extensive that all less obvious cases can be decided upon by the grammar itself.'\is{clear cases!insufficiency of} It is becoming more and more apparent that decisions on very important areas of theory are dependent on very unreliable observations. ... There is a tendency toward preoccupation with extremely subtle distinctions, not the importance, but rather the direct observability of which can seriously be called into question. \citep[vol. 2: 6]{Levelt1974}

\end{quote}

The same complaint has been made throughout much of the history of generative grammar, e.g., by \citet[348]{Bever1970a}, \citet[191]{Labov1972a}, \citet{Langendoen1972}, \citet{Coppieters1987}, and \citet[81]{Birdsong1989}. It has come to be generally acknowledged that not all speakers of ``the same language'' might have the same competence, but that does not justify basing the theory only on sentences for which there is universal agreement, and extrapolating by some means to dictate the status of the remainder. In cases where people disagree, that fact cannot be ignored; the theory must be able to describe \textit{every} speaker's competence, and thus must allow for variation wherever it occurs. This is why establishing the extent of interspeaker agreement is important. Theories are now being based on sentences whose status turns out not to be unanimous, as I discuss in \sectref{sec:2.3.2}. See \sectref{sec:6.3.3}
for further discussion of the implications of individual differences in grammaticality judgments  and in linguistic competence.

Unfortunately, the vast majority of work in linguistics today, unlike that reviewed in \sectref{sec:1.3}, has paid little heed to the issues involved in using judgment data. What follows is surely not a random sample of the theoretical syntax literature, but it includes some very influential and widely cited papers. My particular interest will be not just the types of judgment data that are employed, and hence the judgment abilities attributed to native speakers, but also the importance of these judgments to the theoretical arguments, i.e., to what extent the arguments would be weakened if the fine-grained judgments were  unavailable.  In  many cases I will not mention the details of the theoretical issues, since they are irrelevant to my purpose.
 
\subsection{The Dangers of Unsystematic Data Collection} \label{sec:2.3.2}

\epigraph{\itshape All too often the data in linguistic books and articles are dubious, which in turn casts doubts on the analyses. Since analyses usually build on the results of previous analyses, the effect of one set of dubious data left unquestioned can have far-reaching repercussions that are not easily perceived.\\[-2\baselineskip]}{\citep{Greenbaum1977c}}

\noindent Let us begin with an important case where interspeaker variation in judgments has been ignored, to the detriment of the theory, before examining the ways in which judgments are used by theoreticians. The belief seems to be widely held among theoreticians that the majority of the data on which their theories are based is indisputable. But one cannot assume that what is a clear-cut judgment for oneself is obvious to all, or even to a large majority of, speakers. A case in point is the widely cited article by \citet{LasnikEtAl1984}. One of the major proposals in this work is a substantial revision of the mechanisms of Proper Government to allow sentences like \REF{ex:2:5}:

\ea\label{ex:2:5}
Why do you think that he left?  [= Lasnik \& Saito's (99)]
\z

\noindent
The authors assume that such sentences are ambiguous,\is{ambiguous sentence!judging grammaticality of} i.e., that \textit{why} can be taken as questioning the reason for the thinking or the reason for the leaving. In general, they assume that adjunct \textit{wh}-words do not show \textit{that}-trace effects, so that all sentences of this form should also be ambiguous when \textit{why} is replaced by \textit{where, when}, or \textit{how}. On the basis of these assumptions, they propose various complications to the operation of the \isi{Empty Category Principle} (ECP) and a process of \textit{that}-deletion at Logical Form so that the sentences will not violate the ECP, as they did in earlier theories. But are the crucial readings grammatical?

In a subsequent paper, \citet{AounEtAl1987} propose an alternative theory in the same domain, this time with the goal of accounting for the \textit{ungrammaticality} of some of the very same sentences that Lasnik \& Saito went out of their way to include in the grammar, namely those containing \textit{why} and \textit{how} as the \textit{wh}-words. Anticipating reaction to the apparent data disagreement,\is{data disagreements} they make the following comments:

\begin{quote}
Some speakers claim to get a lower-clause interpretation for \textit{why} in (51a) [\textit{Why did she say that there are men outside}] even if a complementizer is present. However, we have found that when asked to repeat the sentence, those speakers omit \textit{that}, as if it were not perceived.


English speakers who accept (51a) may be able to use \textit{why} referentially, in the sense of `for what reason'. But the \textit{acceptability} of (51a) for such speakers does not seem to us to indicate \textit{grammaticality}, unless they also accept (26)\textendash{}(28) [e.g., \textit{*Who remembers what we bought why?}] and the like; rather, an analogical process is involved. (pp. 563\textendash{}564; emphasis in original)
\end{quote}

\noindent
Aoun et al. seem to be proposing  two different explanations. On the one hand, they suggest that the judgment data are inaccurate, that people really cannot get the relevant reading of this kind of sentence. On the other hand, they propose that this reading \textit{is} acceptable for some speakers, and then attempt to argue on theory-internal grounds that it still must not be generated by their grammars, since we would then expect certain other sentence to be acceptable as well.\is{acceptability!versus grammaticality}\is{grammaticality!versus acceptability}

This area provides a striking demonstration of why linguists must improve their data-gathering techniques. In the first case, Lasnik\ia{Lasnik, Howard} \& Saito\ia{Saito, Mamoru} show no evidence of being aware that the only sentences that prompt their major revision of the theory are not universally accepted. If their proposal had been uncritically adopted, it would have constituted a major step in the wrong direction, in the absence of any explanation for why many speakers should find the sentences bad. In the second case, Aoun et al.\ia{Aoun, Joseph}\ia{Hornstein, Norbert}\ia{Lightfoot, David}\ia{Weinberg, Amy} conclude on the basis of a less than rigorous survey that only a small number of speakers claim the sentences to be acceptable, and that some or all of these judgments are incorrect, i.e., that subjects did not consider the crucial presence of \textit{that}. My own preliminary survey suggests that this is not the case either. About two-thirds of the people I surveyed accepted the sentences after the presence of \textit{that} was explicitly pointed out to them, again calling the analysis into question. Given the extent to which judgments are divided, I suspect that syntacticians would not want to base any conclusions about \isi{Universal Grammar} (UG) on these sentences. But until the detailed judgment facts are known, there is no way to assess the situation accurately.\footnote{\citet{Newmeyer1983} attempts to play down the significance of a similar situation from the early 1970s, describing it as a case of ``letting the theory decide'' on a marginal case, as Chomsky\ia{Chomsky, Noam} proposed in \textit{SS}. In the situation under discussion, however, both camps went out of their way to account for the judgments they perceived, which were not predicted by existing versions of the theory.}
%\textsuperscript{16}


A similar case is made by \citet{Sobin1987} with regard to \textit{wh}-extraction across \textit{that} versus \textit{whether}.\is{Comp-trace effect} He points out that most theories assume that these two kinds of extraction, exemplified in sentences \REF{ex:2:6} and \REF{ex:2:7} below, are equally bad\schdash{}categorically  ungrammatical.

\ea[*]{Who did you say that kissed Harriet? [= Sobin's (1)]}\label{ex:2:6}\z
%7

\ea[*]{Who did you ask whether loves Mary? [= Sobin's (4)]}\label{ex:2:7}\z

\noindent
But the results of Sobin's questionnaire survey (corroborated by various anecdotal observations by others) present a distinctly different picture. He asked 42 nonlinguists to classify sentences into one of three groups, representing active acceptance, passive acceptance, and rejection.\footnote{His descriptions to subjects were worded as follows: ``(a) it sounds like a sentence that you
... (the informant) ... might say in the right context or situation; (b) it sounds like a possible English sentence, one that even if you don't say it that way, you would not be particularly surprised to hear someone else say it to you that way or to see it written; (c) it sounds odd, so that you doubt that people say it that way.'' One might raise some questions about these instructions; for instance, they require subjects to have some sense of what a ``possible English sentence'' is, and to be aware of how other people might speak. \label{fn:2:17}
}
%\textsuperscript{17}
 He found the average rejection rate for sentences like \REF{ex:2:6} to be 17.5\%, whereas for sentences like \REF{ex:2:7} it was 97.6\%. The active acceptance rate for \REF{ex:2:6} was 45.2\%. These differences certainly imply that we cannot rely on identical grammatical constraints to rule out both sentence types. While the \textit{whether} sentences are almost unanimously rejected, the \textit{that} sentences are quite widely accepted. Once again, several theories had been constructed on the assumption that \REF{ex:2:6} is bad for everyone, and furthermore these analyses predicted that it should be just as bad as \REF{ex:2:7}, since it violates precisely the same constraint.\footnote{Such an analysis might be salvageable if something like the process of \textit{that}-deletion proposed by Aoun et al.\ia{Aoun, Joseph}\ia{Hornstein, Norbert}\ia{Lightfoot, David}\ia{Weinberg, Amy} is going on in sentences like \REF{ex:2:6} to yield a grammatical sentence. Note that there is no closely related grammatical version of \REF{ex:2:7} (Elan Dresher, personal communication).}
%\textsuperscript{18}
 As before, these are core data for the formulation of the ECP\is{Empty Category Principle} and associated constraints. Sobin proposes that structures like \REF{ex:2:7} be ruled out universally, whereas a parametrized rule could determine whether \REF{ex:2:6} would be allowed. Whatever the eventual analysis, it is clear that ignoring variation led the theory astray in this case too.

\citet{BleyVromanEtAl1988} also gathered data on \isi{Comp-trace effect}s. While their primary purpose was to look at the
influence of UG\is{Universal Grammar} on second-language acquisition, they needed native English speakers as controls. Their 34 subjects made three-way judgments (accept/reject/not sure) on 32 sentences for which the authors already had a set of normative judgments in mind. On the average, subjects agreed with these norms 92\% of the time (where ``not sure'' was never counted as agreement).\footnote{Unfortunately, the article does not report the number of ``not sure'' responses, making it difficult to know how clear-cut the disagreements are.}
%\textsuperscript{19}
 On 6 of the 32 sentences, however, their mean
accuracy fell below 90\% (the authors' arbitrarily chosen cutoff for good agreement), and two of these involved Comp-trace\is{Comp-trace effect} violations. Only 48\% of subjects rejected  \textit{What did John say that would fall  on the floor,  if we're not careful?} and
only 74\% rejected \textit{What does Mary want to know whether John has already sold?} This study also found that supposedly good sentences are not always judged very good. Only 68\% of the subjects accepted \textit{Which information would it be possible for Mary to persuade  Susan to tell the reporters?}

\subsection{A Case Study in the Use of Subtle Judgments}\label{sec:2.3.3} 

\epigraph{\itshape The kinds of arguments that seem to bear very crucially on the nature and operation of syntactic systems involve [today's grammarian] in grammaticality decisions that are extremely difficult to make.\\[-2\baselineskip]}{\citep{Fillmore1972}}

\noindent Let us now consider various ways in which judgments are used in theoretical argumentation, beginning with \citegen{BellettiEtAl1988} influential work on psych-verbs in \isi{Italian}. They wish to argue that (some) Experiencer subjects are underlying internal arguments, so they rely heavily on the ability to diagnose derived subjects. One criterion they use is the inability of derived subjects to bind anaphors, which generally holds for the Experiencers in question. However, they admit that with nonclitic anaphors the resulting sentences are not entirely bad; they rate them as ``*?'' or ``(?),'' where by the latter they seem to mean ``very close to fully acceptable.'' They clearly consider the lack of total badness an important problem, since they propose an analysis to explain it. Parallel constructions in English, they point out, ``are judged deviant to some extent,'' which they apparently consider to be support for the \isi{Italian} data by implicitly assuming the same explanation in both cases. Another correlate of derived subjecthood is the impossibility of arbitrary \textit{pro}, but again there are generic contexts where the contrast with underlying subjects is ``weaker, but still detectable'' as compared to specific event contexts, the predicted bad sentences being marked ``?'' or ``??.'' (No definition is given for ``??'' in relation to ``*?,'' but in general, people seem to assume that any rating containing a star is worse than one containing only question marks.) Here again, an explanation is proposed by Belletti \& Rizzi. In both of these cases, in the absence of an explanation the marginality facts would undermine main arguments for their analysis. On the very next page, however, a ``??'' sentence is treated as bad with no further comment, and the same is true for a
sentence marked ``(*)'' later on.\footnote{Belletti\ia{Belletti, Adriana} \& Rizzi\ia{Rizzi, Luigi} do not exhaust the possible annotations. In addition to the five already mentioned, they employ the standard ``*'' and show grammatical sentences as unmarked, for a total of seven distinctions. But there are others in the literature. For instance, there are occasional uses of ``**'' that mean ``much worse than a sentence that is already pretty bad.'' As \citet{Hagege1981} puts it, ``The stars have also been eked out with further signs'' (p. 137). Jim McCawley (personal communication)\ia{McCawley, {James D.}} points out that the real problem with all these annotations is that linguists often do not use them consistently. For example, linguists are not clear on whether ``*'' represents the worst point on a two-point versus a five-point scale or some intermediate point.}
%\textsuperscript{20}
 Why is it that some instances of marginality demand comment whereas others do not? Most likely because the authors have no explanations for some marginal cases, but know that readers will not be upset if certain marginal data are left unaccounted for. Thus, we have another case of selective treatment of judgment  data.

Later in Belletti\ia{Belletti, Adriana} \& Rizzi's\ia{Rizzi, Luigi} paper we find an instance where (citing \citet{Burzio1981}) they equate two uses of the question mark, claiming that the marginal status of one sentence is unchanged by the application of a passivization rule, i.e., that the original sentence and its passive counterpart are equally marginal:

%\setcounter{itemize}{7}
\begin{exe} 
\ex \label{ex:2:8}
\begin{xlista}
	\ex[?]{John gave pictures of each other to the kids.\newline [= Belletti \& Rizzi's (69)]}
	\ex[?]{Pictures of each other were given to the kids.}
\end{xlista}
\end{exe}

\noindent
Their argument is that binding\is{Binding Theory} requirements may be satisfied at D-structure, before passive movement, so no change in grammaticality is predicted. However, in the corresponding \isi{Italian} cases the apparent surface binding\is{Binding Theory} violations (parallel to (\ref{ex:2:8}b))
are ``slightly more awkward, ... but the contrast is much weaker than cases involving violations of the \isi{Binding Theory}'' (p. 316). In the abstract, their argument takes the following form:

\begin{quote}
The difference in grammaticality between sentences A and B is significantly less than that between sentences C and D.\medskip

D constitutes a binding\is{Binding Theory} violation, but A and C are fine.\medskip

Therefore, B does not constitute a binding\is{Binding Theory} violation because it is not bad enough.
\end{quote}


\noindent
The assumption is that all (Principle A) binding\is{Binding Theory!severity of violations of} violations cause exactly the same change in grammaticality rating, independent of any properties of the sentences themselves. This is an assumption we ought to test.
 
It is interesting to look at the prose descriptions that Belletti\ia{Belletti, Adriana} \& Rizzi\ia{Rizzi, Luigi} use to accompany the various annotations of sentences. Their sentence (75), marked ``?,'' is ``more or less acceptable''\is{acceptability!degrees of} but such sentences with one question mark ``still produce a weak violation of the chain condition.'' That is, the sentence is bad enough that it must violate something, and just bad enough to be a violation of this condition, but must not be violating anything else or it would have to be worse. Examples labelled ``??'' are variously described as ``quite strange'' and ``weakly deviant''; does this mean that the notation underdifferentiates, or is the descriptive prose merely being stylistically varied?

Despite their distinction of no fewer than seven degrees of grammaticality, Belletti\ia{Belletti, Adriana} \& Rizzi\ia{Rizzi, Luigi} remark about some other sentences that ``these judgments are extremely subtle, and the usual OK vs. * notation is perhaps not appropriate for characterizing such contrasts. In fact, examples like (79a)\textendash{}(80b) are already quite
marked; still, there seems to be a detectable systematic difference in the indicated direction'' (p. 322). Now, for some reason, any detectable pattern warrants an explanation. Furthermore, a sentence that is ``quite marked'' receives no stars or question marks at all. After proposing an account of the difference between the starred and unstarred items, the authors then claim independent support for it by suggesting that another subtle contrast seems ``exactly on a par with'' this one. That is, we must allow for judging strict equality, as well as inequality, in contrasting pairs of judgments.

In this paper we also find the paradigmatic case of comparison of degrees of badness, ECP\is{Empty Category Principle!versus Subjacency} versus Subjacency violations:\is{Subjacency violation!versus ECP violation} ``The relatively mild ill-formedness of (94b) [which they mark ``??''] suggests that the empty category left after extraction is properly governed within NP, otherwise these examples would violate the ECP, and a stronger unacceptability should result'' (p. 328, fn. 22). Thus, people supposedly have an absolute sense of how bad ECP violations are and the sentence in question is not bad enough to be an ECP violation. Furthermore, since Subjacency violations\is{Subjacency violation} are usually not as bad as ECP\is{Empty Category Principle} violations, this is probably a \isi{Subjacency violation}. But there is no general theory of which principles \textit{should} cause worse violations. The theory makes no prediction about the relative badness of, say, \isi{theta@θ-Criterion} versus \isi{Case Filter} violations, let alone about how bad each one is in some absolute sense. The notion of relative and absolute badness of particular violations is ad hoc, and is used in just those cases where it is convenient.

\subsection{The Interpretation of the Annotations and Degrees of Badness}\label{sec:2.3.4}
\begin{flushright}
\itshape
Twinkle twinkle sentence starred,\\
The asterisk means that you are marred. \\
Because of you the linguists try \\
Inventing rules which you defy.\\
\end{flushright}
%\todo{try not to have a page break within a stanza}
\begin{flushright}
\noindent\itshape Twinkle twinkle sentence starred.\\
From linguistics you should be barred.\\
The truth is rarely all or none,\\
Not even with an N of one.\\

\noindent\upshape (Peter Reich, 1980)\\
\end{flushright}
\ia{Reich, Peter A.}

\citet{Householder1973} accepts both the credit and the blame for originating, around 1958, the use of asterisks to mark ungrammatical sentences.\is{annotations!of (un)acceptability} \is{asterisk!original use of} He complains that what was once a simple and unequivocal notation has come to be used in highly ambiguous ways, many of which are detailed below.\footnote{According to Householder, the asterisk originally marked forms that were ``quite obviously ungrammatical by anyone's standards.''}
%\textsuperscript{21}
 At the time of his writing, the asterisk seemed to have at least three possible interpretations:

\begin{enumerate}
\item ``I would never say X'';

\item ``I have never seen or heard a sentence of the type of X and hereby wager you can't find an example''; and

\item ``This is quite comprehensible, and I have heard people say it, but they were all K's [foreigners, Southerners, etc.]; in \textit{my} dialect we would say Y instead.'' \citep[370\textendash{}372]{Householder1973}\is{dialects (idiolects)}

\end{enumerate}


\noindent
As we shall see, this is only the tip of the iceberg.

Another kind of ambiguity with judgment notation\is{annotations!of (un)acceptability} involves two distinct uses of marginality markings, chiefly the question mark, in the literature. One use denotes variable interspeaker ratings, i.e., a sentence that is good for some people, bad for others.\is{annotations!of interspeaker variation} The second meaning is that (most) individuals rate the sentence as marginal. One could imagine the combination of these situations as well. The same is true of disjunctive notations like ``/'' or ``\{\}'': Are both alternatives acceptable to all speakers, or are there two groups, each of which accepts only one? This situation is at best a notational inaccuracy that could be easily corrected by adopting new symbols.\footnote{One does occasionally find ``\%'' used to mark acceptance by some speakers but not others.\is{annotations!of interspeaker variation} \citet{Neubauer1976} states that he and Larry Horn introduced this symbol to indicate important dialect\is{dialects (idiolects)} variation. It should be noted that \textit{dialect} here is not necessarily meant in the sense of regionally or 
socially based group speech, but simply in the sense of a number of speakers who share the same judgments.
}
%\textsuperscript{22}
 Unfortunately, there are cases where the surrounding
prose description does not make clear which meaning is intended. An example of the second kind of ambiguity appears in verb agreement with nominative objects in \isi{Icelandic}. \citet{Thrainsson1979} gives the following datum and description:

\ea\label{ex:2:9}
\gll Mér {líkar/líka} {\th}essir bílar.\\
{Me(D)} {likes(3rd sg.)/like(3rd pl.)} these cars(N{ }pl.)\\
\glt [= Thráinsson's (3), p. 466]\\
 \z

\noindent
``There are some idiolectal\is{dialects (idiolects)} differences as to the preference of verb forms, but the fact that some speakers prefer the 3rd sg. form here indicates that the nominative NP is not perceived as the subject'' (p. 466). One interpretation of these comments might be that there is between-speaker variation across degrees of preference for each form.\footnote{\citet{Newmeyer1983} suggests that there is an even more basic inconsistency in the use of stars, question marks, and other symbols\schdash{}namely, their indication of ungrammaticality versus unacceptability.\is{annotations!of (un)acceptability} My impression is that the authors of works reviewed here, like most current authors, intend the latter interpretation. \citet{McCawley1985} explicitly states that he uses asterisks to mark ``whatever kind of oddity of a sentence ... I am at the moment concerned with; thus I use it to report data, not conclusions as to `grammaticality\textquoteright{}'' (p. 673). As noted in \sectref{sec:2.2}, McCawley rejects the notion of grammaticality, as opposed to acceptability, anyway, feeling that there is nothing to be gained by classifying sources of unacceptability as being inside or outside the grammar.
}
%\textsuperscript{23}


\citet{Andrews1990} examined this and a number of other subtle agreement phenomena in \isi{Icelandic}.. He is an unusual theoretician in that his work actually reports the results of grammaticality questionnaires that he administered.\is{annotations!of (un)acceptability} In one, Andrews elicited ratings on a 6-point scale, characterized as follows (this is one of the rare instances in which an explicit meaning is given for the symbols):\footnote{\citet{Labov1972b} gives the following definitions: ? = questionably ungrammatical, * = ungrammatical, ** = outstandingly ungrammatical. See \citet{Householder1973} for another version.}
%\textsuperscript{24}

\begin{itemize}
 \item[{\checkmark}] Completely acceptable and natural
 \item[?] Acceptable, but perhaps somewhat unnatural
 \item[??]  Doubtful, but perhaps acceptable
 \item[?*] Worse, but not totally unacceptable
 \item[*] Thoroughly unacceptable
 \item[**] Horrible (p. 203)
\end{itemize}

\noindent
Andrews reports results for 20 sentences, with between 12 and 17 subjects responding. Of these sentences, only three were rated uniformly ``{\checkmark}''; none were rated uniformly ``*'' or uniformly ``**'', and only two were rated as either ``*'' or
%\originalpage{46} %  Chapter Two
``**'' by everyone. But despite having access to such detailed information, Andrews fails to clarify the status of Thráinsson's variability, stating that ``either of the above [agreement variants] seems to be acceptable (on the basis of questionnaires returned by seven informants)'' (p. 212), which is still ambiguous. The implications of interspeaker differences versus intraspeaker marginality should be clear. The former, if not reflective of extragrammatical factors, demand different grammars for the two groups, whereas the latter demands a single grammar with a less severe constraint.

Another use of ``?'' is\is{annotations!of (un)acceptability} illustrated by \citet{Pollock1989}
in his widely cited paper on the structure of IP (Inflection Phrase), with regard to French sentences like the following:

%\setcounter{itemize}{7}
\ea[?]{Je pensais ne pouvoir pas dormir dans cette chambre.\newline [= Pollock's (20b)]}\label{ex:2:10}\z

\noindent According to Pollock, ``the question mark is meant here as an indication that [sentences such as the one in \REF{ex:2:10}] have a very literary ring to them, not that they are unacceptable'' (p. 375). Such data are in serious danger of being misinterpreted out of context, especially since on the very next page ``?'' is used to indicate marginal acceptability. On this next page we also find ``(?)'' indicating that some speakers find another sentence better than the question-marked one, although Pollock admits to having found other speakers who hold the opposite opinion. One could see this either as again confounding marginality with interspeaker variability, or as an indication that ratings may reflect arbitrarily chosen subgroups of speakers. The contrast between ``?'' and ``(?)'' does not affect his arguments, but the difference between these marginal sentences and certain starred ones is crucial to several arguments in Pollock's paper. Later on, ``(?)'' is used not to mean ``slightly better than ?'' in a relative sense, but ``perfect, with at worst a slightly literary ring,'' which might or might not correspond in an absolute sense to the prior usage. Indeed, literariness and marginality might be separate dimensions of ratings altogether, which cannot be meaningfully compared on the same  ordinal scale, much as height and weight as integers cannot.

Finally, let us consider a paper by \citet{Browning1987} in which more than half of the example sentences are marked with some number of question marks\is{annotations!of (un)acceptability} or stars.\footnote{\citet{Hagege1981} finds this trend unappealing: ``In extreme cases, the pollution of starred, forbidden forms brings us to what has been called `a linguistics of the impossible'{\textquotedbl} (p. 158, fn. 16).}
%\textsuperscript{25}
 Since the paper is concerned with (among other things) the definition of Subjacency,\is{Subjacency violation} one of the few Government-Binding (GB) principles that has graded
% Definitions and Historical Background  47
behavior in its very definition, it is not surprising to find extensive reliance on relative judgments. One of Browning's proposals is to account for the marginality of parasitic gaps\is{parasitic gap} as in \REF{ex:2:11} and \REF{ex:2:12} by the same mechanism that accounts for paradigm Subjacency violations\is{Subjacency violation} such as those in the sentences in \REF{ex:2:13}:


\ea[?]{Which paper did you read before filing? [= Browning's (1)]}\label{ex:2:11}
   \z

    
\ea[?]{an artist that close friends of admire [= Browning's (2)]}\label{ex:2:12}
\z

\ea \label{ex:2:13}
\ea
Which car is it time for John to wash? [= Browning's (40)]
\ex Who did John buy a suit to impress?
\ex What did John wonder how to fix?
\ex Who did they leave before meeting?\\
\z
\z

\begin{quote}
Consider first the degree of ungrammaticality which results from one barrier intervening between two points in a chain. Several examples are given above in [\REF{ex:2:13}]. I have been assuming the standard judgment for parasitic gaps\is{parasitic gap} such as [\REF{ex:2:11}] and [\REF{ex:2:12}], namely, a mild marginality.\is{annotations!of (un)acceptability} If this marginality is due to the intervening barrier, then the severity of the violation is clearly in the ball park represented by [\REF{ex:2:13}]. (pp. 68\textendash{}69)
\end{quote}

\noindent
Two things are noteworthy in Browning's account. First, as we have seen before, there is a standard rating for constructions of a particular type, independent of the sentences themselves. This is surely a huge idealization\schdash{}there are experiments showing that identical structural violations are given different grammaticality ratings depending on their particular lexical content (see Sections \ref{sec:5.3.5} and \ref{sec:5.3.6}).


To the extent that \textit{linguists} give uniform ratings for all such sentences, it is much more likely to be because they recognize them (perhaps subconsciously) as instances of parasitic gaps\is{parasitic gap} than because of any pretheoretic goodness rating.\footnote{This hypothesis is supported by the fact that people's absolute judgments are highly unstable, as I discuss in later chapters. Identical ratings of the kind linguists claim to have could not arise on the basis of acceptability judgments alone.}
%\textsuperscript{26}
 That is, they are judging conformity to sentence patterns or templates and then reporting the standard rating for that pattern.\footnote{I assume that sentence processing itself does \textit{not} work primarily by templates.} Second, Browning's analysis is another example of equating the badness of sentences and taking this as support for a common violation. But is it not possible that two different principles could yield the same degree of ungrammaticality when violated? Just because two sentences are equally bad does not mean they violate exactly the same constraint(s), especially since most of GB's constraints have yet to be rated according to seriousness. Browning goes on to draw support for her arguments from alleged differences in goodness between sentences with no question marks or stars whatsoever.

\section{Introspection, Intuition, and Judgment} \label{sec:2.4}

\epigraph{\textit{Being a native speaker doesn't confer papal infallibility on one's intuitive judgments.\\[-2\baselineskip]}}{(Raven McDavid,\ia{McDavid, Raven} quoted in \citealt{Paikeday1985})}

\noindent Over the history of generative grammar, much has been made of its heavy reliance on introspective judgments and their nonequivalence to production and comprehension. Interestingly, this same feature has been used both to criticize and to defend the generative practice. In fact, \citet{Ringen1977} notes Chomsky's\ia{Chomsky, Noam} implicit suggestion that introspective psychology or psychophysics could serve as a model for the use of linguistic intuitions. But, as Ringen argues, the first of those analogies would be counterproductive to the field, and the second would be inappropriate (but see \citet[6\textendash{}7]{Cowart1997} for a different view). Let
us consider how these paradigms operate, and why linguistics should not be striving to emulate them.

In the introspectionist paradigm, established by Wilhelm Wundt in 1879 in the first experimental psychology laboratory, trained subjects were asked to describe their impressions of a wide variety of physical objects and experiences (\citet{Wundt1896}; \citet{Boring1953}). The idea was to describe internal experience in terms of elementary sensation. That is, rather than saying that one sees a book, one should relate the colors, shapes, etc., that are perceived. There were several problems with this approach. One was that the elements of most experiences simply cannot be discerned by reflecting on them, just as one cannot discern the elements of water by looking at it \citep{Dellarosa1988}. Another was the fact that Wundt's subjects were far from naive with regard to the experimental procedure. They had to undergo at least 10,000 supervised practice trials before they could be used in an experiment, during which time they were taught special terminology in which to describe their sensations. One cannot help but suspect that Wundt's own ideas on what experience was like affected subjects during this training period, although at the time it was thought that subjects were merely unlearning bad perception habits. Each of these problems is applicable to the linguist's situation today to some degree (\citet[vol. 3]{Levelt1974}), but perhaps the most significant drawback, which led to the
% Definitions and Historical Background  49
demise of introspectionist psychology, is strikingly evident among modern-day linguists. \citet{Dellarosa1988} describes it as follows:

\begin{quote}
Despite the careful training that observers received, agreement among introspective reports was the exception rather than the rule. It was not unusual to obtain markedly different reports from two observers who were exposed to the same stimulus. Such disagreements could not be settled in any scientific fashion owing to the inherently private nature of internal events. In more technical terms, introspection failed as a bona fide scientific method because it violated a fundamental rule concerning scientific investigation: that of independent access to both causes and effects. Although the cause (i.e., stimulus) was open to public observation, the effect (i.e., internal sensation) was not. Without such independent observation of the internal sensation, it was impossible to tell which of two conflicting introspective reports was the correct one. The conflicting reports could have arisen because (a) Subject A was truly experiencing a different sensation than Subject B, or (b) Subject A was experiencing the same sensation as Subject B but was misreporting it, or (c) Subject A was simply lying. ... There was no scientific way to determine which of these three conditions was true. (p. 5)
\end{quote}

\noindent
See \citet{Householder1965} for the view that there is no way to evaluate conflicting claims about linguistic intuitions either, and \citet{Levelt1972} for some more general background on the historical relationship between psychology and linguistics. I should stress that I do not believe that appealing to linguistic intuitions about sentences \textit{is} a form of introspection, for reasons that are detailed immediately below. However, the methodologies seem to have some disturbing commonalities.

Now, it is true that there has been somewhat of a resurgence in the use of introspectionist-style protocols in psychology recently, specifically in the use of ``talk aloud'' and ``think aloud'' methods.\is{protocol analysis} In their comprehensive review of this approach, however, \citet{EricssonEtAl1984} point out a fundamental difference in how such reports are used today. In a nutshell, they are not taken as facts about the inner workings of subjects' minds, but are studied from the point of view of understanding how it was that subjects were able to say what they did about a mental process, independent of its veracity. This distinction is too easily overlooked, but it is crucial, and many linguists have not taken heed of it yet. Except in those cases where they fail to suit the linguist's purpose, subjects' intuitions \textit{are} taken to reflect their true linguistic knowledge. We do not ask, ``What must be in subjects' minds in order for them to react this way to a sentence?'' but, rather, ``What must be in subjects' minds in order for this sentence to have the status they claim it has?'' This is what \citet{BeverEtAl1981} mean when they say that ``intuitions are not empirical primitives but complex behavioral performances in their own right'' (p. 232). Thus, current psychological practice does not redeem linguistic practice.

\citet{Ringen1977} summarizes the problem with a Wundtian approach thus: if subjects are reporting on truly private mental states (to which only they have access), then their reports are \textit{in principle} uncheckable; therefore, if intuitions are to be any good as data, they must differ in some way from this sort of introspection. Fortunately, I believe they do. Linguists are not asking subjects to describe an internal mental process when they encounter a sentence\schdash{}not even thinking aloud is involved, only reporting a reaction \citep{Cohen1981}. Introspection is reflection, analysis, or careful thought applied to accessible contents of the mind, which do not include grammatical knowledge. Thus, linguists are not introspectionists.

Ringen\ia{Ringen, {Jon D.}} goes on to consider Chomsky's\ia{Chomsky, Noam} second analogy, to \isi{psychophysics}. Like linguistic intuition gathering, psychophysics experiments involve first-person utterances given in response to stimuli (e.g., responses to questions like ``Which weight feels heavier, the one in your right or your left hand?''). But in psychophysics we can ask not only if the subjects' reports correctly describe their judgments (as opposed to being instances of lying, misspeaking, or carelessness), but also how \textit{accurate} the judgments are (e.g., which of the two weights really \textit{is} heavier). Psychophysics concerns the relationship between objective features of stimuli and subjective judgments of them, but grammaticality judgments have no objective standard to be measured against.\footnote{Wayne Cowart (personal communication)\ia{Cowart, Wayne} suggests that the analogy \textit{is} useful because the external standards against which judgments can be measured are the linguistic norms of the community where speakers learned their native languages. Beyond the obvious practical problems in applying this idea, there is still the conceptual problem of distinguishing inaccurate judgments from real differences in the grammars of people native to the same speech community.
}
%\textsuperscript{28}
 One is the final authority on what one's psychophysical judgments \textit{are}, but not on whether they are correct. Thus, psychophysics also fails as an analogy to grammaticality judgment.

There is one type of report that does seem to parallel linguistic intuitions, as Ringen points out: reports about mental states, e.g., whether one is in pain. These certainly \textit{can} be shown to be false\schdash{}we can show that a person is lying about being in pain, but once we show that the report is sincere, no other evidence can
% Definitions and Historical Background  51
disprove the assertion (it makes no sense to talk about the correctness of a sensation), and if two people's reports conflict in response to the same stimulus, that does not undermine the veracity of either report. And so it seems to be with linguistic intuitions. \citet{Pateman1987} makes the case most eloquently:

\begin{quote}
[Core grammatical rules] can be studied by means of \textit{intuition}, which is a quite distinct method or process from introspection. In my view, the exercise of intuition\schdash{}for example, offering intuitions as to the grammaticality or ungrammaticality of token sentences-provides us with indexical, that is, causally related, symptomatic evidence for the character of underlying mental representations. An intuition of grammaticality is not a social judgment, but the output of a computational process which the speaker as subject registers and reports, just as in the case where the viewer as subject registers and reports visual phenomena which may well be resistant to judgment proper (so, for example the \isi{Müller-Lyer illusion}\ia{Müller-Lyer, Franz Carl} persists even when we know it is an illusion). Of course, this is not always true, and it is clear and admitted that intuitions of grammaticality are liable to all kinds of interference `on the way up' to the level at which they are given as responses to questions. In particular, they are liable to interference from social judgments of linguistic acceptability. Ontologically, the difference may be expressed this way: mentally represented grammars \textit{compute} grammaticality categorizations, but they do not care about them any more than the perceptual system cares about perceptual illusion. In contrast, speakers as subjects can and do care about grammaticality (and the veridicality of perception), and can and do convert the categorization they register into social judgments. This can sometimes lead them into error. (p. 100)

I use the term `intuition' to designate that which gives us causally related indexical or symptomatic evidence for the character of underlying psycholinguistic (or, more generally, psychological) processes. Intuitions are not exercises of judgment which claim certainty or any kind of objectivity for the content of the judgment and hence which claim the assent of all those implicated by the judgment. Rather, intuitions are reports of appearances, hence subjective expressions which make no judgment about how others will or should respond. In intuition we tell how something strikes us, how it appears to us and thereby provide causal evidence about our minds. In Wittgenstein's\ia{Wittgenstein, Ludwig} terms, intuitions provide \textit{symptoms} rather than \textit{criteria} of what underlies them. ... Intuition is used to establish claims of form (1), not form (2), which involves the exercise of judgment (introspection in my terms).

\begin{exe}
\exi{(1)}
Sentence \textit{P} seems grammatical to subject \textit{S}.
\end{exe}

\begin{exe}
\exi{(2)}   
	  Sentence \textit{P} is grammatical in language \textit{L}, according to subject
\textit{S}. (p. 135)
\end{exe}

\end{quote}

\noindent
See \citet[117]{Carr1990} for a somewhat different conception of intuition versus introspection, as well as \citet{Katz1981} and \citet{Lyons1986} on the same topics.

Thus, much of the common terminology surrounding the use of grammaticality judgments is somewhat misleading. They are certainly not introspective. They are not judgments or intuitions in the everyday sense, in that they cannot be verified or resolved by observation or calculation. They also do not have an evaluative component as do social judgments. Perhaps more accurate terms for grammaticality judgments would be \textit{grammaticality sensations} and \textit{linguistic reactions}. Nonetheless, for the sake of familiarity I shall continue using traditional terminology, on the understanding that it must not be taken literally.\is{grammaticality judgment@\textit{grammaticality judgment}!as misnomer} As \citet{Pateman1987}
points out, treating grammaticality judgments like sensations does not preclude our looking for ways in which they might be inaccurate, just as our primary senses can be. Even staunch supporters of the generative enterprise such as \citet{Newmeyer1983} admit that the theory might well be skewed by artifacts of the introspection/intuition/sensation process, but they usually resign themselves to this as being part of the early stages of the field of linguistic investigation. I suggest that a more useful approach would be to try to learn more about the creature rather than simply to accept its influence over us. If indeed metalinguistic judgment behavior can tell us anything at all about normal language use, how can we extract that information? That is the question I begin to address in the next chapter.

\section{Conclusion}

In general, it is clear that subtle judgment data have become important to theoretical argumentation. If they were not crucial, surely they would be ignored\schdash{}clear-cut data make a much more impressive case. I have identified three kinds of problems in the use of these data, one of which will be the main focus for the remainder of this book. The first is that judgment data are not systematically reported or notationally identified. The second is that they are sometimes used or discarded as it suits the linguist's fancy. The third is that their use attributes various sophisticated abilities to native speakers without any evidence that speakers are actually capable of reliably making the discriminations in question, and without any attempt to systematically control the process of obtaining these judgments. The first two problems are more properly examined under the rubric of philosophy of science, and I will have little more to say about them. The third falls in the domain of psychology, which has the means to determine what people can actually do and provide a method for collecting the data when they do it. Subsequent chapters address these two goals and relate the results to the needs of the generative  enterprise.