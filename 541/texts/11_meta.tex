\begin{figure}[H]
    \centering
    \includegraphics[width=.8\textwidth]{\imgpath/contextual_images/swamp2.JPG}
    \caption{Young men punting a dinghy through the swamp near Parade (Coconut\is{flora!\textit{Cocos nucifera} (Coconut)} Island) in 2017}
    \label{fig:swamp}
\end{figure}

 \textit{Ngasinga wutamu} is a song that likely originated as a work song, sung while clearing the plants and mud from the canoe ways through the swamp waters to get from Limol to Parade (Coconut\is{flora!\textit{Cocos nucifera} (Coconut)} Island). This is an important passageway because it clears the route for children to access Upiara Secondary School and for those traveling up the Bituri River to reach the Fly River, which leads to many villages and towns, including the regional capital, Daru. Nowadays, this song has been set with choreography and is regularly performed by the Limol culture and dance group (see \figref{fig:limol-dance}). This song was contributed to the Ende language corpus in 2016 by \name{Biku (Madura)}{Kangge$^\dagger$}, and you can hear him sing it in the archive \citep{Kangge2016a}.