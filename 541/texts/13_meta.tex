\begin{figure}[H]
    \centering
    \includegraphics[width=.8\textwidth]{\imgpath/contextual_images/baby in baby basket.JPG}
    \caption{An infant sleeping in a \textit{ddäma} `baby basket,' a basket that can be hung to rock on a house post or suspended from someone's forehead}
    \label{fig:sleeping}
\end{figure}

\textit{Bandra bebi bälle}, literally `A song for a baby,' is a lullaby sung to help babies go to sleep. In the song, the baby is asked not to cry and to fall asleep, reassured that their mother is nearby squeezing sago\is{flora!\textit{Metroxylon sagu} (Sago)} (see \figref{fig:squeezing-sago}). The word \textit{bebi} can be replaced with the baby's name. Notably, this song contains all three vocative clitics =\textit{a}, =\textit{e}, and =\textit{o}, which are lengthened to carry the melody. This lullaby was contributed by \name{Sam}{Karao} in 2016. You can listen to S. Karao singing this song in the Ende language corpus \citep{Karao2016b}.