This story is a fable about two thieves -- Frog and Lobster -- who go out to \textit{zanggae}, a word that means `to stroll' or `saunter', that is to wander around the village or bush without any aim and possibly get into trouble. When the two thieves came across a bountiful banana\is{flora!\textit{Musa spp.} (Banana)} tree, laden with ripe fruit, they decided to climb the tree and steal the bananas. When the owner of the banana tree sees what has happened, he is angry and plans to catch the thieves the next morning. When Frog and Lobster return, the gardener is ready and asks the thieves who climbed the tree to knock the fruit. Frog and Lobster go back and forth -- “Frog did!", “Lobster did!" When the storyteller says this part, the phrase \textit{Pollgo a} `Frog did' is humorous because the words sound like a croaking frog. The two friends go back and forth until Frog admits to being the one that climbed the tree. The gardener chases them but only finds Lobster. This ending explains why humans eat lobster and why the two are not friends.

Stealing is looked down upon in Ende culture, especially because Ende people are very generous. When asked to define the word \textit{moko} `love', the Ende say that \textit{moko} is when you give me some food. When people talk about their loved ones, they describe them as \textit{ttonggag} `giving, kind'. If someone does not have food or objects to give when seeing someone on the road, they will give them kind words instead.

Some relevant pictures include the local white-lipped tree frog (\figref{fig:frog}), a bountiful banana\is{flora!\textit{Musa spp.} (Banana)} tree (\figref{fig:banana}), a communal garden fence (\figref{fig:fence}), and a personal garden fence (\figref{fig:personal-fence}).