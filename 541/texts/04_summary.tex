This text is a remembered narrative about a memorable hunting\is{hunting} trip undertaken by \JBD{} and his son. In the tale, \JBD{} is prepared to join the village men solo to bring home an animal, but his son begs to come with him, “as young boys are wont to do." \JBD{} and his son do get close to some deer, but the hunting is thwarted because the father is distracted worrying about his son. This story provides some examples of Ende hunting practices and a glimpse into Ende family dynamics.

Hunting\is{hunting} is an important activity in Ende society. Men hunt with longbows and unfletched arrows for deer (\textit{Rusa timorensis};\is{fauna!\textit{Rusa timorensis} (Rusa deer)} \figref{fig:deer}), boar, cassowary, birds, wallabies (\figref{fig:wallaby}), and bandicoots, among other animals. Hunting bows and arrows are crafted out of bamboo and are typically longer than the hunter is tall (\figref{fig:bow-spears}--\figref{fig:spears}). The kind of hunting described in this text is called \textit{mamoe}, which is a type of group hunting where part of the hunting party chases the animals, often with dogs (\figref{fig:dogs}), and the other members wait at a known bottleneck or path that they know the animals will take. In this story, \JBD{} is waiting at the swamp in a canoe for the deer that the hunters will chase towards him (see \figref{fig:canoe} and \figref{fig:huntingboat}).

One structure of interest in this text is the verb \textit{kam} `to start'. \textit{Kam} may appear in an intransitive verbal template, \textit{e.g.}, \textit{gongkam} `I started' in Line \ref{ex:text04-1-1}, or a transitive verbal template, \textit{e.g.}, \textit{dängkam} `I started it' in Line \ref{ex:text04-3}. In both cases, the verb inflects to agree with the valency of the infinitival verb in the lower clause, \textit{i.e. ibi} `to walk' and \textit{llädäd} `to grab', intransitive and transitive, respectively. As the infinitival verbs do not host any morphological inflection, such as person or number \isi{agreement}, \isi{pluractionality}, or \isi{associated motion}, the verb \textit{kam} will also inflect to mark these characteristics of the embedded event.