In this fishing\is{fishing} story, \DKSfull{} details an unexpected encounter with a \is{fauna!\textit{Crocodylus halli} (crocodile)}{crocodile} while fishing. The type of fishing described here is one where fishers stake long vertical nets across a river, leaving them there for an extended period of time.\footnote{Other types of \isi{fishing} include: shooting arrows with a bow, throwing harpoons from above, throwing harpoons while diving with goggles, dangling bait on a hook with a line, poisoning the water, setting bamboo traps, net casting (see \figref{fig:net-casting}) and funneling the water with circular nets and baskets (see \figref{fig:fishing-net}).} Later, someone will return to lift the nets out of the water to see if any fish are caught in the holes of the nets. Any fish that are caught in the nets are perfect bait for large predators, such as crocodiles (\textit{Crocodylus halli}\is{fauna!\textit{Crocodylus halli} (Hall's New Guinea Crocodile)}; see \figref{fig:crocodile}). When \DKS{} and \SSS{} go to check the nets, they find the caught fish half-eaten and suspect that a crocodile might be nearby. When they discover the crocodile, \DKS{} bravely kills it, pulls it into their boat, and carries it back to the village. This would be an impressive feat for any Ende villager -- crocodiles are dangerous and heavy -- but especially so for \DKS{}, who was in her late 50s or 60s when the event occurred. This fact is less surprising if we consider that women in this society are burdened with the responsibility of carrying heavy loads (see \figref{fig:donae-carrying}). For example, women are responsible for fetching water from the wells multiple times daily. Young girls carry buckets of various sizes until they can balance the standard 40-pound (20-kilogram) jerry can on their heads and backs. Women also transport sago\is{flora!\textit{Metroxylon sagu} (Sago)} bundles, which are even heavier at around 60--80 pounds (30--40 kilograms). Water and sago\is{flora!\textit{Metroxylon sagu} (Sago)} both need to be carried regularly uphill from the swamps to the village. Men do carry some loads, such as felled trees and hunted\is{hunting} animals.

You will also notice a reference to Christian prayer in this text (Line \ref{ex:text07-1-5}). The Ende community experienced a rapid conversion to Christianity in the 1950s and 60s when Christian missionaries came to the nearby towns of Goroka and Upiara. Ende representatives were sent out to these towns to learn what the missionaries had to say, and they returned with information about literacy, basic Western healthcare, and the Bible. This experience is captured in the documentary \textit{Ende tän e indrang} `Light into Ende tribe' (\cite{Warama2018b}). Although the Ende representatives needed to learn Gogodala\il{Gogodala-Suki!Gogodala}, Tok Pisin\il{English Creole!Tok Pisin}, and \ili{English} to communicate with the missionaries, the local conversion took place mostly in Ende. Because of this, the domain of religion is primarily in Ende, not in a \textit{lingua franca}, such as Tok Pisin\il{English Creole!Tok Pisin} or \ili{English}. \ili{English}, however, is used frequently to quote from the Bible as the text has not been completely translated into Ende. 

This text has a lovely quadruplet of verbs marked with ventive \isi{associated motion} in Lines \ref{ex:text07-2-5}--\ref{ex:text07-2-11}. They show how the ventive can mark motion into view, motion towards the speaker, or actions with intended motion towards a prominent place (\textit{e.g.}, the village). First, the crocodile's head \textit{ada digne} `goes like this \textsc{ven}=[into view]' and bites \DKS{} in the hand. She \textit{ttang de [...] gänyeri ada dige} `goes like this this way with her hand \textsc{ven}=[back into view/towards the speaker]' and sees that the crocodile only scraped the skin. She then \textit{giri de dipirngän} `draws her knife \textsc{ven}=[into view]' and hits him in the head. They \textit{gall ik e dizeneya} `pull him into the canoe \textsc{ven}=towards the speaker' and \textit{net [de...] dingädneg} `fold the nets \textsc{ven}=[in order to bring them back to the village]'.