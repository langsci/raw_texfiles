\ea\label{ex:text06-1}
Ttoen a ngasnges atta gänyan Llamda Kurupel bäne.\\
\gll ttoen=a	ngas{\textasciitilde}nges=att=a	gänya=n	llamda	Kurupel=bäne\\
     story=\textsc{nom}	\textsc{inf}{\textasciitilde}happen=\textsc{abl}=\textsc{nom}	here=\textsc{cop}.\textsc{prs}.\textsc{sg}	old\_man	\textsc{pn}=3\textsc{sg}.\textsc{abl}\_\textsc{poss}\\
\glt `This story is about what Old Man Kurupel did.'
\z

\ea\label{ex:text06-2-1}
Pazi da 1940 diba ngata me, ttongo lla da kuddäll gogon Malläm ttängäm me.\\
\gll pazi=da	1940\footnotemark{}	diba	ngata=me	ttongo	lla=da	kuddäll	g-o-g-on	Malläm	ttängäm=me\\
     year=\textsc{nom}	1940	that	spot=\textsc{loc}	a	man=\textsc{nom}	die	\textsc{rem}-\textsc{rt}\_\textsc{ext}-\textsc{aux}-\textsc{rem}.3\textsc{sg}S	\textsc{pn}	village=\textsc{loc}\\
\glt `Sometime around 1940, a man died in Malam village.'\footnotetext{from \ili{English} \textit{1940}}
\z

\ea\label{ex:text06-2-2}
Obo bin a Galo.\\
\gll obo	bin=a	Galo\\
     3\textsc{sg}.\textsc{poss}	name=\textsc{nom}	\textsc{pn}\\
\glt `His name was Galo.'
\z

\ea\label{ex:text06-2-3}
Galo oba baba daeya Erme a Baewa wa Zakae.\\
\gll Galo	oba	baba	da=eya	Erme	a	Baewa	wa	Zakae\\
     \textsc{pn}	3\textsc{nsg}.\textsc{poss}	father	\textsc{med}.\textsc{dem}=\textsc{cop}.\textsc{pst}.\textsc{sg}	\textsc{pn}	and	\textsc{pn}	and	\textsc{pn}\\
\glt `Galo was Erme, Baewa, and Zakae's father.'
\z

\ea\label{ex:text06-2-4}
Yawenang lla deya.\\
\gll Yawen=ang	lla	da=eya\\
     \textsc{pn}=\textsc{atr}	man	\textsc{med}.\textsc{dem}=\textsc{cop}.\textsc{pst}.\textsc{sg}\\
\glt `He was a Yawenang man.'
\z

\ea\label{ex:text06-2-5}
Kuddäll gogon Malläm me.\\
\gll kuddäll	g-o-g-on	Malläm=me\\
     die	\textsc{rem}-\textsc{rt}\_\textsc{ext}-\textsc{aux}-\textsc{rem}.3\textsc{sg}S	\textsc{pn}=\textsc{loc}\\
\glt `He died in Malam.'
\z

\newpage
\ea\label{ex:text06-3-1}
Tämamae lla da Llimoll atta gobällnän do Malläm.\\
\gll tämamae	lla=da	\textsc{pn}=att=a	g-o-bäll-n-än	do	Malläm\\
     all	man=\textsc{nom}	\textsc{pn}=\textsc{abl}=\textsc{nom}	\textsc{rem}-\textsc{rt}\_\textsc{ext}-go-\textsc{ipfv}-\textsc{rem}.3\textsc{pl}S	there	\textsc{pn}\\
\glt `Everyone from Limol went to Malam.'
\z

\ea\label{ex:text06-3-2}
Ge kuddällag me alla ingollang kolloenen allan naeka we.\\
\gll ge	kuddäll=ag=me	alla=ingoll=ang	kolloe-nen=allan	naeka=we\\
     this	death=\textsc{atr}=\textsc{loc}	how=like=\textsc{atr}	mix-I.\textsc{pl}=\textsc{aux}.\textsc{prs}.1\textsc{pl}S	cry=\textsc{all}\\
\glt `During times of death, we gather together to mourn.'
\z

\ea\label{ex:text06-3-3}
Ubi goklloeyän Malläm ttängäm e, naeka gognegnän ge, gottamänän.\\
\gll ubi	g-o-klloe-yän	Malläm	ttängäm=e	naeka	g-o-g-neg-n-än	ge	g-o-ttam-än-än\\
     3\textsc{nsg}.\textsc{nom}	\textsc{rem}-\textsc{rt}\_\textsc{ext}-gather-\textsc{rem}.3\textsc{sg}S	\textsc{pn}	place=\textsc{all}	cry	\textsc{rem}-\textsc{rt}\_\textsc{ext}-\textsc{aux}-3\textsc{pl}S-\textsc{ipfv}-\textsc{rem}.3\textsc{pl}S	this	\textsc{rem}-\textsc{rt}\_\textsc{ext}-finish-\textsc{ii}.\textsc{npl}-\textsc{rem}.3\textsc{sg}S\\
\glt `They gathered in Malam, mourned together, and finished.'
\z

\ea\label{ex:text06-4-1}
Oke, au we abo dibagaeya, auma däkälleyo, Galo bo pätt wanseg e, auma källkäll a gottamänän.\\
\gll oke	au=we	abo	diba=gaeya	au=ma	d-ä-käll-eyo	Galo=bo	pätt	wanse-g=e	au=ma	käll{\textasciitilde}käll=a	g-o-ttam-än-än\\
     okay	burial=\textsc{all}	then	that=\textsc{cop}.\textsc{pst}.\textsc{pl}	bury=\textsc{char}	\textsc{rem}-3\textsc{ndu}P-dig-\textsc{rem}.3\textsc{nsg}A	\textsc{pn}=3\textsc{sg}.\textsc{poss}	body	put-\textsc{iii}.\textsc{npl}=\textsc{all}	bury=\textsc{char}	\textsc{inf}{\textasciitilde}dig=\textsc{nom}	\textsc{rem}-\textsc{rt}\_\textsc{ext}-finish-\textsc{ii}.\textsc{npl}-\textsc{rem}.3\textsc{sg}S\\
\glt `Okay, then there was the burial. They dug the grave to leave Galo's body. They finished the digging.'
\z

\ea\label{ex:text06-4-2}
Pätt de dowansegeyo auma me.\\
\gll pätt=de	d-o-wanse-g-eyo	au=ma=me\\
     body=\textsc{acc}	\textsc{rem}-3\textsc{ndu}P-put-\textsc{iii}.\textsc{npl}-\textsc{rem}.3\textsc{nsg}A	bury=\textsc{char}=\textsc{loc}\\
\glt `They put the body in the grave.'
\z

\newpage
\ea\label{ex:text06-4-3}
Dawundeyo, gottamänän.\\
\gll d-a-wund-eyo	g-o-ttam-än-än\\
     \textsc{rem}-\textsc{rt}\_\textsc{ext}-cover\_pit-\textsc{rem}.3\textsc{nsg}A	\textsc{rem}-\textsc{rt}\_\textsc{ext}-finish-\textsc{ii}.\textsc{npl}-\textsc{rem}.3\textsc{sg}S\\
\glt `They covered it up and finished.'
\z

\ea\label{ex:text06-5-1}
Ge angde iddob gogon, lla ulleulle da, tämamae llamäg a gotäbanegän.\\
\gll ge	angde	iddob	g-o-g-on	lla	ulle{\textasciitilde}ulle=da	tämamae	llamäg=a	g-o-täba-neg-än\\
     this	when	night	\textsc{rem}-\textsc{rt}\_\textsc{ext}-\textsc{aux}-\textsc{rem}.3\textsc{sg}S	man	\textsc{nsg}{\textasciitilde}big=\textsc{nom}	all	old\_man=\textsc{nom}	\textsc{rem}-\textsc{rt}\_\textsc{ext}-plan-3\textsc{pl}S-\textsc{rem}.3\textsc{pl}S\\
\glt `When night fell, the big men and old men got together and made a plan.'
\z

\ea\label{ex:text06-5-2}
Ada, ``Ibi sisri tongoe de ngasnges e dan, gänya kuddäll täräp me.\\
\gll ada	ibi	sisri	tongoe=de	ngas{\textasciitilde}nges=e	da=n	gänya	kuddäll	täräp=me\\
     like\_this	1\textsc{nsg}.\textsc{incl}.\textsc{nom}	now	play=\textsc{acc}	\textsc{inf}{\textasciitilde}do=\textsc{all}	\textsc{med}.\textsc{dem}=\textsc{cop}.\textsc{prs}.\textsc{sg}	this	dead	time=\textsc{loc}\\
\glt `They said, ``Tonight, during this time of death, we will play a game.'''
\z

\ea\label{ex:text06-5-3}
Ngaskäma imomdae lla da llokttang a ainin, iddob e ibi wi do auma.\\
\gll ngaskäma	imomdae	lla=da	llokott=ang=a	ain=in	iddob=e	ibi=wi	do	au=ma\\
     \textsc{pot}	truth	man=\textsc{nom}	strong=\textsc{atr}=\textsc{nom}	who.\textsc{sg}=\textsc{cop}.\textsc{prs}.\textsc{sg}	night=\textsc{all}	go=\textsc{all}	there	bury=\textsc{char}\\
\glt `We will see who is truly brave by going to the grave at night.'
\z

\ea\label{ex:text06-5-4}
Ibi sisri yu bäng de onymäll e dan.''\\
\gll ibi	sisri	yu	bäng=de	ony-mäll=e	da=n\\
     1\textsc{nsg}.\textsc{incl}.\textsc{nom}	now	fire	firestick=\textsc{acc}	carry-\textsc{pl}=\textsc{all}	\textsc{med}.\textsc{dem}=\textsc{cop}.\textsc{prs}.\textsc{sg}\\
\glt `We will bring a fire-stick back-and-forth to the grave.'
\z

\ea\label{ex:text06-6-1}
Ttongda donyalle do auma dowansegalle, ttongda ako dallalle ngäs e.\\
\gll ttong=da	d-o-ny-alle	do	au=ma	d-o-wanse-g-alle	ttongo=da	ako	d-a-ll-alle	ngäs=e\\
     one=\textsc{nom}	\textsc{rem}-3\textsc{ndu}P-carry-\textsc{irr}.\textsc{sg}A	there	bury=\textsc{char}	\textsc{rem}-3\textsc{ndu}P-put-\textsc{iii}.\textsc{npl}-\textsc{irr}.\textsc{sg}A	one=\textsc{nom}	again	\textsc{rem}-\textsc{rt}\_\textsc{ext}-go-\textsc{hab}.\textsc{sg}S	return=\textsc{all}\\
\glt `One many must carry the fire-stick to the grave and leave it there; then another will go and fetch it.'
\z

\ea\label{ex:text06-6-2}
Auma watta diwenyalle.\\
\gll au=ma=watt=a	d-i-weny-alle\\
     bury=\textsc{char}=\textsc{abl}=\textsc{nom}	\textsc{rem}-\textsc{ven}-carry-\textsc{irr}.\textsc{sg}A\\
\glt `He will carry it back from the grave.'
\z

\ea\label{ex:text06-6-3}
Ttongda ako donyalle do auma.\\
\gll ttongo=da	ako	d-o-ny-alle	do	au=ma\\
     one=\textsc{nom}	again	\textsc{rem}-3\textsc{ndu}P-carry-\textsc{irr}.\textsc{sg}A	there	bury=\textsc{char}\\
\glt `Then another will carry it to the grave.'
\z

\ea\label{ex:text06-6-4}
Tämamae lla da ada gognegän, ``Ai dan.\\
\gll tämamae	lla=da	ada	g-o-g-neg-än	ai	da=n\\
     all	man=\textsc{nom}	like\_this	\textsc{rem}-\textsc{rt}\_\textsc{ext}-\textsc{aux}-3\textsc{pl}S-\textsc{rem}.3\textsc{pl}S	good	\textsc{med}.\textsc{dem}=\textsc{cop}.\textsc{prs}.\textsc{sg}\\
\glt `All of the men said, ``Okay.'
\z

\ea\label{ex:text06-6-5}
Ibi sisri mɨnyi däbe tongoe de bängesnalla iddob e.\\
\gll ibi	sisri	mɨnyi	däbe	tongoe=de	b-ä-nges-n-alla	iddob=e\\
     1\textsc{nsg}.\textsc{incl}.\textsc{nom}	now	\textsc{fut}	that	play=\textsc{acc}	\textsc{fut}.1A-3\textsc{ndu}P-do-\textsc{ipfv}-\textsc{fut}.1\textsc{nsg}A	night=\textsc{all}\\
\glt `We will play this game tonight.'
\z

\ea\label{ex:text06-6-6}
Ngaskäma aya lelmeny bogon auma we ibi wi.''\\
\gll ngaskäma	aya	lel=meny	b-o-g-on	au=ma=we	ibi=wi\\
     \textsc{pot}	who.\textsc{sg}	fear=\textsc{priv}	\textsc{fut}.3S-\textsc{rt}\_\textsc{ext}-\textsc{aux}-\textsc{rem}.3\textsc{sg}S	bury=\textsc{char}=\textsc{all}	go=\textsc{all}\\
\glt `Maybe someone without fear will go to the grave.'''
\z

\ea\label{ex:text06-7-1}
Ngämo baba Kurupel ada gogon ada, ``Ngäna bony.\\
\gll ngämo	baba	Kurupel	ada	g-o-g-on	ada	ngäna	b-ony\\
     1\textsc{sg}.\textsc{poss}	father	\textsc{pn}	like\_this	\textsc{rem}-\textsc{rt}\_\textsc{ext}-\textsc{aux}-\textsc{rem}.3\textsc{sg}S	like\_this	1\textsc{sg}.\textsc{nom}	\textsc{fut}.1A-carry\\
\glt `My father Kurupel went like this, ``I'll take it.'
\z

\newpage
\ea\label{ex:text06-7-2}
Ngäna yu di bony do auma do bowanseg.\\
\gll ngäna	yu=di	b-ony	do	au=ma	do	b-o-wanse-g\\
     1\textsc{sg}.\textsc{nom}	fire=\textsc{acc}	\textsc{fut}.1A-carry	there	bury=\textsc{char}	there	\textsc{fut}.1A-3\textsc{ndu}P-put-\textsc{iii}.\textsc{npl}\\
\glt `I'll take the fire to the grave and leave it there.'
\z

\ea\label{ex:text06-7-3}
Ttongda abo ngäs e ballän.''\\
\gll ttongo=da	abo	ngäs=e	b-a-ll-än\\
     one=\textsc{nom}	\textsc{nec}	return=\textsc{all}	\textsc{fut}.3S-\textsc{rt}\_\textsc{ext}-go-\textsc{rem}.3\textsc{sg}S\\
\glt `Someone else will have to go and fetch it.'''
\z

\ea\label{ex:text06-7-4}
Awayang Idugoe, Tizag lla daeya.\\
\gll Awayang	Idugoe	Tizag	lla	da=eya\\
     \textsc{pn}	\textsc{pn}	\textsc{pn}	man	\textsc{med}.\textsc{dem}=\textsc{cop}.\textsc{pst}.\textsc{sg}\\
\glt `Awayang Idugoe was a Tizag man.'
\z

\ea\label{ex:text06-7-5}
Obo ttäle da apte gagäll deya, malla mermer ibi ag deya.\\
\gll obo	ttäle=da	apte	gagäll	da=eya	malla	mer{\textasciitilde}mer	ibi=ag	da=eya\\
     3\textsc{sg}.\textsc{poss}	leg=\textsc{nom}	one\_side	bad	\textsc{med}.\textsc{dem}=\textsc{cop}.\textsc{pst}.\textsc{sg}	\textsc{neg}	\textsc{adv}{\textasciitilde}good	go=\textsc{atr}	\textsc{med}.\textsc{dem}=\textsc{cop}.\textsc{pst}.\textsc{sg}\\
\glt `His legs were bad on one side, so he did not walk properly.'
\z

\ea\label{ex:text06-7-6}
Bogo ada eka gogon ada, ``Abo ngänawa ngäs e balle.\\
\gll bogo	ada	eka	g-o-g-on	ada	abo	ngäna=wa	ngäs=e	b-a-lle\\
     3\textsc{sg}.\textsc{nom}	like\_this	speak	\textsc{rem}-\textsc{rt}\_\textsc{ext}-\textsc{aux}-\textsc{rem}.3\textsc{sg}S	like\_this	then	1\textsc{sg}.\textsc{nom}=\textsc{emph}	return=\textsc{all}	\textsc{fut}.1S-\textsc{rt}\_\textsc{ext}-go\\
\glt `He said, ``I will return to the grave.'
\z

\ea\label{ex:text06-7-7}
Auma watta ako ngäna yu bäng de beyangäs iddob säremang me.''\\
\gll au=ma=watt=a	ako	ngäna	yu	bäng=de	b-ey-a-ngäs	iddob	särem=ang=me\\
     bury=\textsc{char}=\textsc{abl}=\textsc{nom}	then	1\textsc{sg}.\textsc{nom}	fire	firestick=\textsc{acc}	\textsc{fut}.1A-\textsc{ven}-\textsc{rt}\_\textsc{ext}-return	night	darkness=\textsc{atr}=\textsc{loc}\\
\glt `And I will bring the fire-stick back during the night.'''
\z

\newpage
\ea\label{ex:text06-8-1}
Däbaeya, Kurupel donyän yu di do auma.\\
\gll däba=aeya	Kurupel	d-ony-än	yu=di	do	au=ma\\
     that=\textsc{cop}.\textsc{pst}.\textsc{sg}	\textsc{pn}	\textsc{rem}-carry-\textsc{rem}.3\textsc{sg}A	fire=\textsc{acc}	there	bury=\textsc{char}\\
\glt `Then, Kurupel carried the fire there to the grave.'
\z

\ea\label{ex:text06-8-2}
Auma me dowansegän yu bäng de.\\
\gll au=ma=me	d-o-wanse-g-än	yu	bäng=de\\
     bury=\textsc{char}=\textsc{loc}	\textsc{rem}-3\textsc{ndu}P-put-\textsc{iii}.\textsc{npl}-\textsc{rem}.3\textsc{sg}A	fire	firestick=\textsc{acc}\\
\glt `He left the fire-stick on the grave.'
\z

\ea\label{ex:text06-8-3}
Awayang ako ngäs e.\\
\gll Awayang	ako	ngäs=e\\
     \textsc{pn}	then	return=\textsc{all}\\
\glt `Then Awayang was going to return.'
\z

\ea\label{ex:text06-8-4}
Angde Awayang yu ngäs e ibi wi ada gogon, Kurupel däbe mängalae gungmaeyän ttongo nyongo dae dallän do auma.\\
\gll angde	Awayang	yu	ngäs=e	ibi=wi	ada	g-o-g-on	Kurupel	däbe	mängal=ae	g-u-ngmae-yän	ttongo	nyongo=dae	d-a-ll-än	do	au=ma\\
     when	\textsc{pn}	fire	return=\textsc{all}	go=\textsc{all}	like\_this	\textsc{rem}-\textsc{rt}\_\textsc{ext}-\textsc{aux}-\textsc{rem}.3\textsc{sg}S	\textsc{pn}	that quick=\textsc{adv}	\textsc{rem}-\textsc{rt}\_\textsc{ext}-go\_around-\textsc{rem}.3\textsc{sg}S	another	road=\textsc{perl}	\textsc{rem}-\textsc{rt}\_\textsc{ext}-go-\textsc{rem}.3\textsc{sg}S	there	bury=\textsc{char}\\
\glt `When Awayang went to go to fetch the fire, Kurupel quickly went to the grave by taking a shortcut.'
\z

\ea\label{ex:text06-8-5}
Galo bo auma sisor a eraeya.\\
\gll Galo=bo	au=ma	sisor=a	era=eya\\
     \textsc{pn}=3\textsc{sg}.\textsc{poss}	bury=\textsc{char}	new=\textsc{nom}	which=\textsc{cop}.\textsc{pst}.\textsc{sg}\\
\glt `Galo had a new grave.'
\z

\ea\label{ex:text06-8-6}
Do ngattong auma toko we gokakalän.\\
\gll do	ngattong	au=ma	toko=we	g-o-kak-al-än\\
     there	first	bury=\textsc{char}	top=\textsc{all}	\textsc{rem}-\textsc{rt}\_\textsc{ext}-enter-\textsc{ii}.\textsc{npl}-\textsc{rem}.3\textsc{sg}S\\
\glt `The first graves you entered from above.'
\z

\newpage
\ea\label{ex:text06-8-7}
Auma da pollepolle alle kättnan att dagaeya.\\
\gll au=ma=da	polle{\textasciitilde}polle=alle	kätt-nan=att	da=gaeya\\
     bury=\textsc{char}=\textsc{nom}	\textsc{dim}{\textasciitilde}fence=\textsc{ins}	fence-I.\textsc{pl}=\textsc{abl}	\textsc{med}.\textsc{dem}=\textsc{cop}.\textsc{pst}.\textsc{pl}\\
\glt `This grave was fenced in with a small fence.'
\z

\ea\label{ex:text06-9-1}
Daeya do llandär gognän, gontämonän Awayang bälle ada ngaskäma Awayang ibi allan.\\
\gll da=eya	do	llandär	g-o-g-n-än	g-o-tomon-n-än	Awayang=bälle	ada	ngaskäma	Awayang	ibi=allan\\
     \textsc{med}.\textsc{dem}=\textsc{cop}.\textsc{pst}.\textsc{sg}	there	listen	\textsc{rem}-\textsc{rt}\_\textsc{ext}-\textsc{aux}-\textsc{ipfv}-\textsc{rem}.3\textsc{sg}S	\textsc{rem}-\textsc{rt}\_\textsc{ext}-wait-\textsc{ipfv}-\textsc{rem}.3\textsc{sg}S	\textsc{pn}=3\textsc{sg}.\textsc{dat}	like\_this	\textsc{pot}	\textsc{pn}	go=\textsc{aux}.\textsc{prs}.3\textsc{sg}S\\
\glt `He waited there and listened for Awayang to know when Awayang was coming.'
\z

\ea\label{ex:text06-9-2}
Llamda Awayang daolle obo ibi da malla mer daeya, be ttimattimang ibi ag daeya.\\
\gll llamda	Awayang	da=olle	obo	ibi=da	malla	mer	da=eya	be	ttima{\textasciitilde}ttima=ang	ibi=ag	da=eya\\
     old\_man	\textsc{pn}	\textsc{med}.\textsc{dem}=\textsc{all}	3\textsc{sg}.\textsc{poss}	walk=\textsc{nom}	\textsc{neg}	good	\textsc{med}.\textsc{dem}=\textsc{cop}.\textsc{pst}.\textsc{sg}	but	\textsc{inf}{\textasciitilde}limp=\textsc{atr}	go=\textsc{atr}	\textsc{med}.\textsc{dem}=\textsc{cop}.\textsc{pst}.\textsc{sg}\\
\glt `Old man Awayang's gait was not good; he walked with a limp toward the grave.'
\z

\ea\label{ex:text06-9-3}
Daolle dallän abo tatraem de yaya Kurupel deyandärän.\\
\gll da=olle	d-a-ll-än	abo	tatäraem=de	yaya	Kurupel	d-ey-a-ndär-än\\
     \textsc{med}.\textsc{dem}=\textsc{all}	\textsc{rem}-\textsc{rt}\_\textsc{ext}-go-\textsc{rem}.3\textsc{sg}S	then	noise=\textsc{acc}	father	\textsc{pn}	\textsc{rem}-\textsc{ven}-\textsc{rt}\_\textsc{ext}-hear-\textsc{rem}.3\textsc{sg}A\\
\glt `He was walking that way and then father Kurupel heard his footsteps coming toward him.'
\z

\ea\label{ex:text06-9-4}
Käsre do auma me bogo do ddoddollem dägnegän ge wattällang a endagaeya do däbem ddoddollem dägnegän ge.\\
\gll käsre	do	au=ma=me	bogo	do	ddo{\textasciitilde}ddollem	d-ä-g-neg-än	ge	wattäll=ang=a	enda=gaeya	do	däbe-m	ddo{\textasciitilde}ddollem	d-ä-g-neg-än	ge\\
     then	there	bury=\textsc{char}=\textsc{loc}	3\textsc{sg}.\textsc{nom}	there	\textsc{inf}{\textasciitilde}make\_noise	\textsc{rem}-3\textsc{ndu}P-\textsc{aux}-\textsc{sg}>\textsc{pl}-\textsc{rem}.3\textsc{sg}A	this	put=\textsc{atr}=\textsc{nom}	what=\textsc{cop}.\textsc{pst}.\textsc{pl}	there	that-\textsc{acc}	\textsc{inf}{\textasciitilde}make\_noise	\textsc{rem}-3\textsc{ndu}P-\textsc{aux}-\textsc{sg}>\textsc{pl}-\textsc{rem}.3\textsc{sg}A	this\\
\glt `Then Kurupel made a lot of noise at the grave, he started banging the things that were left on the grave.'
\z

\ea\label{ex:text06-10-1}
Wiowa Llamda Awayang!\\
\gll wiowa	llamda	Awayang\\
     wow	old\_man	\textsc{pn}\\
\glt `Oh Old man Awayang!'
\z


\ea\label{ex:text06-10-2}
Dinduag ada gogon!\\
\gll dindu=ag	ada	g-o-g-on\\
     run=\textsc{atr}	like\_this	\textsc{rem}-\textsc{rt}\_\textsc{ext}-\textsc{aux}-\textsc{rem}.3\textsc{sg}S\\
\glt `Oh how he ran!'
\z

\ea\label{ex:text06-10-3}
Ede Galo eraeya, obo päzäg daeya, ge aya kuddäll gogon.\\
\gll ede	Galo	era=eya	obo	päzäg	da=eya	ge	aya	kuddäll	g-o-g-on\\
     so	\textsc{pn}	which=\textsc{cop}.\textsc{pst}.\textsc{sg}	3\textsc{sg}.\textsc{poss}	brother\_in\_law	\textsc{med}.\textsc{dem}=\textsc{cop}.\textsc{pst}.\textsc{sg}	this	who.\textsc{sg}	dead	\textsc{rem}-\textsc{rt}\_\textsc{ext}-\textsc{aux}-\textsc{rem}.3\textsc{sg}S\\
\glt `Galo was this man's brother-in-law, the one that died.'
\z

\ea\label{ex:text06-10-4}
Awayang ada gogon, ``Päzäg gädo!\\
\gll Awayang	ada	g-o-g-on	päzäg\footnotemark{}	gädo\footnotemark{}\\
     \textsc{pn}	like\_this	\textsc{rem}-\textsc{rt}\_\textsc{ext}-\textsc{aux}-\textsc{rem}.3\textsc{sg}S	brother\_in\_law	is\_here\\
\glt `Awayang was going, ``My brother-in-law is here!'''\footnotetext{The word \textit{päzäg} means `brother-in-law' in both Idi and Ende.}\footnotetext{from Idi \textit{gädo} `is here'}
\z

\ea\label{ex:text06-10-5}
Päzäg gädo!\\
\gll päzäg	gädo\\
     brother\_in\_law	is\_here\\
\glt `My brother-in-law is here!'
\z

\ea\label{ex:text06-10-6}
Päzäg gädo!''\\
\gll päzäg	gädo\\
     brother\_in\_law	is\_here\\
\glt `My brother-in-law is here!'''
\z

\ea\label{ex:text06-10-7}
Ada ekaekong dindugmällnän lel me, dowe de lla da ero dagaeya.\\
\gll ada	eka{\textasciitilde}eka=ong	d-indug-mäll-n-än	lel=me	do=we=de	lla=da	ero	da=gaeya\\
     like\_this	\textsc{adv}{\textasciitilde}speak=\textsc{atr}	\textsc{rem}-run-\textsc{pl}-\textsc{ipfv}-\textsc{rem}.3\textsc{sg}S	fear=\textsc{loc}	there=\textsc{all}=\textsc{acc}	man=\textsc{nom}	where	\textsc{med}.\textsc{dem}=\textsc{cop}.\textsc{pst}.\textsc{pl}\\
\glt `He was running and yelling in fear towards where the men were.'
\z

\ea\label{ex:text06-10-8}
A dindugalle, a gompedägalle, wup pätt me gobäddalle.\\
\gll a	d-indug-alle	a	g-o-mpedä-g-alle	up	pätt=me	g-o-bädd-alle\\
     and	\textsc{rem}-run-\textsc{hab}.\textsc{sg}S	and	\textsc{rem}-\textsc{rt}\_\textsc{ext}-trip-\textsc{iii}.\textsc{npl}-\textsc{hab}.\textsc{sg}S	banana	body=\textsc{loc}	\textsc{rem}-\textsc{rt}\_\textsc{ext}-hit-\textsc{hab}.\textsc{sg}S\\
\glt `He was running, and tripping, and knocking down bananas\is{flora!\textit{Musa spp.} (Banana)}.'
\z

\ea\label{ex:text06-10-9}
Ako gopällttänalle.\\
\gll ako	g-o-pällttän-alle\\
     again	\textsc{rem}-\textsc{rt}\_\textsc{ext}-start\_walking-\textsc{hab}.\textsc{sg}S\\
\glt `Then he took off again.'
\z

\ea\label{ex:text06-10-10}
Ako dindugalle, mameat endageya ge däbe dudunän, dindu mi daeya, ``Päzäg gädo!\\
\gll ako	d-indug-alle	mameat	enda=geya	ge	däbe	d-u-du-n-än	dindu=mi	da=eya	päzäg	gädo{}\\
     again	\textsc{rem}-run-\textsc{hab}.\textsc{sg}S	pawpaw	what=\textsc{cop}.\textsc{pst}.\textsc{pl}	this	that	\textsc{rem}-\textsc{rt}\_\textsc{ext}-blow\_down-\textsc{ipfv}-\textsc{rem}.3\textsc{sg}A	run=\textsc{loc}	\textsc{med}.\textsc{dem}=\textsc{cop}.\textsc{pst}.\textsc{sg}	brother-in-law	is\_here\\
\glt `He was running, knocking down pawpaws while running. ``My brother-in-law is here!'''
\z

\ea\label{ex:text06-10-11}
Päzäg gädo!'',\\
\gll päzäg	gädo\\
     brother\_in\_law	is\_here\\
\glt `{``}My brother-in-law is here!'''
\z

\ea\label{ex:text06-10-12}
ada ekaekong dindugmällnän.\\
\gll ada	eka{\textasciitilde}eka=ong	d-indug-mäll-n-än\\
     like\_this	\textsc{adv}{\textasciitilde}speak=\textsc{atr}	\textsc{rem}-run-\textsc{pl}-\textsc{ipfv}-\textsc{rem}.3\textsc{sg}S\\
\glt `he was yelling while running.'
\z

\ea\label{ex:text06-11-1}
Idi eka walle ada gogon, ``Päzäg gädo!\\
\gll Idi	eka=walle	ada	g-o-g-on	päzäg	gädo\\
     language\_name	language=\textsc{ins}	like\_this	\textsc{rem}-\textsc{rt}\_\textsc{ext}-\textsc{aux}-\textsc{rem}.3\textsc{sg}S	brother-in-law	is\_here\\
\glt `He was yelling in Idi, ``My brother-in-law is here!'''
\z

\ea\label{ex:text06-11-2}
Päzäg gädo!''\\
\gll päzäg	gädo\\
     brother\_in\_law	is\_here\\
\glt `{``}My brother-in-law is here!'''
\z

\ea\label{ex:text06-11-3}
Kuddäll lla de däbe Galo bom ada dägänän, ``Galo gänyan!\\
\gll kuddäll	lla=de	däbe	Galo=bom	ada	d-ä-g-n-än	Galo	gänya=n\\
     dead	man=\textsc{acc}	that	\textsc{pn}=3\textsc{sg}.\textsc{acc}	like\_this	\textsc{rem}-3\textsc{ndu}P-\textsc{aux}-\textsc{ipfv}-\textsc{rem}.3\textsc{sg}A	\textsc{pn}	here=\textsc{cop}.\textsc{prs}.\textsc{sg}\\
\glt `He was yelling at the dead man, ``Galo is here!'''
\z

\ea\label{ex:text06-11-4}
Galo gänyan!\\
\gll Galo	gänya=n\\
     \textsc{pn}	here=\textsc{cop}.\textsc{prs}.\textsc{sg}\\
\glt `{``}Galo is here!'''
\z

\ea\label{ex:text06-11-5}
Galo bo anyke da gänyan ttam agan!''\\
\gll Galo=bo	anyke=da	gänya=n	ttam	a-g-an\\
     \textsc{pn}=3\textsc{sg}.\textsc{poss}	spirit=\textsc{nom}	here=\textsc{cop}.\textsc{prs}.\textsc{sg}	alive	\textsc{rec}-\textsc{aux}-\textsc{rec}.3\textsc{sg}S\\
\glt `{``}Galo's spirit is alive and here!'''
\z

\ea\label{ex:text06-11-6}
Lla da angde dandärmällneyo ge, lla gulag a, ubi ako dinduag a duduli Awayang bom danttäkämälleyo, däbe lla kuddäll anyke de ada ka koenmäll erallo, ``Ya!\\
\gll lla=da	angde	d-a-ndär-mäll-n-eyo	ge	lla	gulag=a	ubi	ako	dindu=ag=a	du{\textasciitilde}duli	Awayang=bom	d-a-nttäkämäll-eyo	däbe	lla	kuddäll	anyke=de	ada	ka	koenmäll=erallo	ya\\
     man=\textsc{nom}	when	\textsc{rem}-\textsc{rt}\_\textsc{ext}-hear-\textsc{pl}-\textsc{ipfv}-\textsc{rem}.3\textsc{nsg}A	this	man	crowd=\textsc{nom}	3\textsc{nsg}.\textsc{nom}	again	run=\textsc{atr}=\textsc{nom}	\textsc{adv}{\textasciitilde}there	\textsc{pn}=3\textsc{sg}.\textsc{acc}	\textsc{rem}-\textsc{rt}\_\textsc{ext}-meet-\textsc{rem}.3\textsc{nsg}A	that	man	dead	spirit=\textsc{acc}	like\_this	\textsc{cntf}	chase=\textsc{aux}.\textsc{prs}.3\textsc{nsg}>3\textsc{sg}	go\_away\\
\glt `When the crowd of men heard this, they ran to meet Awayang and started chasing the dead spirit away, yelling, ``Ya!'
\z

\ea\label{ex:text06-11-7}
Ya!\\
\gll ya\\
     go\_away\\
\glt `Ya!'
\z

\ea\label{ex:text06-11-8}
Ya!''\\
\gll ya\\
     go\_away\\
\glt `Ya!'''
\z

\ea\label{ex:text06-11-9}
Adawede oba anyke da bowansegän Awayang bom, oba duli ballän.\\
\gll adawede	oba	anyke=da	b-o-wanse-g-än	Awayang=bom	oba	duli	b-a-ll-än\\
     so\_that	\textsc{cons}	spirit=\textsc{nom}	\textsc{fut}.3A-3\textsc{ndu}P-put-\textsc{iii}.\textsc{npl}-\textsc{rem}.3\textsc{sg}A	\textsc{pn}=3\textsc{sg}.\textsc{acc}	\textsc{cons}	away	\textsc{fut}.3S-\textsc{rt}\_\textsc{ext}-go-\textsc{rem}.3\textsc{sg}S\\
\glt `So that the spirit would leave Awayang along and go far away.'
\z

\ea\label{ex:text06-11-10}
Dibaeya ge Awayang daeya llayaba pate gogon, ada ``Alla?''\\
\gll diba=aeya	ge	Awayang	da=eya	lla=yaba=pate	g-o-g-on	ada	alla\\
     that=\textsc{cop}.\textsc{pst}.\textsc{sg}	this	\textsc{pn}	\textsc{med}.\textsc{dem}=\textsc{cop}.\textsc{pst}.\textsc{sg}	person=3\textsc{nsg}.\textsc{poss}=\textsc{an}.\textsc{all}	\textsc{rem}-\textsc{rt}\_\textsc{ext}-\textsc{aux}-\textsc{rem}.3\textsc{sg}S	like\_this	how\\
\glt `When Awayang reached the men, they said, ``What happened?'''
\z

\ea\label{ex:text06-11-11}
Ada, ``Ngäna mäse nallan do auma, dowae agan do auma de ddoddllem nägnegan ge adame ngäna gänyaeya indugan lelang a.''\\
\gll ada	ngäna	mäse	nallan	do	au=ma	dowae	a-g-an	do	au=ma=de	ddo{\textasciitilde}ddollem	n-ä-g-neg-an	ge	adame	ngäna	gänya=aeya	indug-an	lel=ang=a\\
     like\_this	1\textsc{sg}.\textsc{nom}	\textsc{imn}	\textsc{aux}.\textsc{prs}.1\textsc{sg}>2\textsc{sg}	there	bury=\textsc{char}	proximity	\textsc{rec}-\textsc{aux}-\textsc{rec}.1\textsc{sg}S	there	bury=\textsc{char}=\textsc{acc}	\textsc{inf}{\textasciitilde}make\_noise	\textsc{rec}.3\textsc{pl}P-3\textsc{ndu}P-\textsc{aux}-\textsc{sg}>\textsc{pl}-\textsc{rec}.3\textsc{sg}A	this	this\_is\_why	1\textsc{sg}.\textsc{nom}	here=\textsc{cop}.\textsc{pst}.\textsc{sg}	run-\textsc{rec}.1\textsc{sg}S	fear=\textsc{atr}=\textsc{nom}\\
\glt `Awayang replied, ``I was just going to the grave, when there was a lot of noise, which is why I ran back in fear!"'
\z

\ea\label{ex:text06-12}
Däbe angde lla de umllang dägnegän, Kurupel daollemae ngättägngättäg gogän oba pate.\\
\gll däbe	angde	lla=de	umllang	d-ä-g-neg-än	Kurupel	da=olle=mae	ngättäg{\textasciitilde}ngättäg	g-o-g-än	oba=pate\\
     that	when	man=\textsc{acc}	tell	\textsc{rem}-3\textsc{ndu}P-\textsc{aux}-\textsc{sg}>\textsc{pl}-\textsc{rem}.3\textsc{sg}A	\textsc{pn}	\textsc{med}.\textsc{dem}=\textsc{all}=\textsc{perl}	\textsc{adv}{\textasciitilde}arrive	\textsc{rem}-\textsc{rt}\_\textsc{ext}-\textsc{aux}-\textsc{rem}.3\textsc{sg}S	3\textsc{nsg}.\textsc{poss}=\textsc{an}.\textsc{all}\\
\glt `After Awayang said this to the men, Kurupel showed up and came close to them.'
\z

\ea\label{ex:text06-13-1}
Ada, ``Bibi ende eka tameny eralla?''\\
\gll ada	bibi	ende	eka	tameny=eralla\\
     like\_this	2\textsc{nsg}.\textsc{nom}	what.\textsc{acc}	speak	discuss=\textsc{aux}.\textsc{prs}.2\textsc{nsg}>3\textsc{sg}\\
\glt `Kurupel said, ``What are you talking about?'''
\z

\ea\label{ex:text06-13-2}
Ada, ``Ingollang ttoen a angesan.''\\
\gll ada	ingoll=ang	ttoen=a	a-nges-an\\
     like\_this	like=\textsc{atr}	story=\textsc{nom}	\textsc{rec}-happen-\textsc{rec}.3\textsc{sg}S\\
\glt `They said, ``This is what happened.'''
\z

\ea\label{ex:text06-14-1}
Kurupel Llamda Awayang bom dägmar täräll dägnegän.\\
\gll Kurupel	llamda	Awayang=bom	dägmar	täräll	d-ä-g-neg-än\\
     \textsc{pn}	old\_man	\textsc{pn}=3\textsc{sg}.\textsc{acc}	tongue	stick\_out	\textsc{rem}-3\textsc{ndu}P-\textsc{aux}-\textsc{sg}>\textsc{pl}-\textsc{rem}.3\textsc{sg}A\\
\glt `Then, Kurupel stuck his tongue out at Old Man Awayang.'
\z

\ea\label{ex:text06-14-2}
Llayabira a ddob mällayabira dällɨtneyo, ddone tongoe daeya.\\
\gll lla=yabira	a	ddob	mälla=yabira	d-ä-llɨt-n-eyo	ddone	tongoe	da=eya\\
     person=3\textsc{nsg}.\textsc{dat}	and	some	woman=3\textsc{nsg}.\textsc{dat}	\textsc{rem}-3\textsc{ndu}P-tell-\textsc{ipfv}-\textsc{rem}.3\textsc{nsg}A	\textsc{neg}	laugh	\textsc{med}.\textsc{dem}=\textsc{cop}.\textsc{pst}.\textsc{sg}\\
\glt `They told other men and woman, and oh they laughed a lot.'
\z

\ea\label{ex:text06-14-3}
Ddobaeddobae gotongoenegnän gänya ttoen me.\\
\gll ddobae{\textasciitilde}ddobae	g-o-tongoe-neg-n-än	gänya	ttoen=me\\
     \textsc{adv}{\textasciitilde}very	\textsc{rem}-\textsc{rt}\_\textsc{ext}-laugh-3\textsc{pl}S-\textsc{ipfv}-\textsc{rem}.3\textsc{pl}S	this	story=\textsc{loc}\\
\glt `People laughed so much at this story.'
\z

\ea\label{ex:text06-14-4}
Dibaeya story da.\\
\gll diba=aeya	story=da\\
     that=\textsc{cop}.\textsc{pst}.\textsc{sg}	story=\textsc{nom}\\
\glt `That is the story.'
\z

\ea\label{ex:text06-15-1}
Dägmar täräll bo midd a ada dan, ``Umllang agalle?\\
\gll dägmar	täräll=bo	midd=a	ada	da=n	umllang	a-g-alle\\
     tongue	stick\_out=3\textsc{sg}.\textsc{poss}	meaning=\textsc{nom}	like\_this	\textsc{med}.\textsc{dem}=\textsc{cop}.\textsc{prs}.\textsc{sg}	know	\textsc{rec}-\textsc{aux}-\textsc{rec}.2\textsc{sg}A\\
\glt `Sticking out your tongue means, ``Do you feel it?'
\z

\ea\label{ex:text06-15-2}
Bongo ddone llokttang dan käde auma ibi wi.''\\
\gll bongo	ddone	llokott=ang	da=n	käde	au=ma	ibi=wi\\
     2\textsc{sg}.\textsc{nom}	\textsc{neg}	strong=\textsc{atr}	\textsc{med}.\textsc{dem}=\textsc{cop}.\textsc{prs}.\textsc{sg}	when	bury=\textsc{char}	go=\textsc{all}\\
\glt `You're not brave enough to go to the grave.'''
\z

\ea\label{ex:text06-15-3}
Kurupel bäne ngasnges atta ada Idugoe a lla da.\\
\gll Kurupel=bäne	ngas{\textasciitilde}nges=att=a	ada	Idugoe	a	lla=da\\
     \textsc{pn}=3\textsc{sg}.\textsc{abl}\_\textsc{poss}	\textsc{inf}{\textasciitilde}do=\textsc{abl}=\textsc{nom}	like\_this	\textsc{pn}	and	man=\textsc{nom}\\
\glt `This is what happened to Kurupel and Idugoe.'
\z

\ea\label{ex:text06-15-4}
Ede ada ingoll ttoen da gongesän.\\
\gll ede	ada=ingoll	ttoen=da	g-o-nges-än\\
     so	like\_this=like	story=\textsc{nom}	\textsc{rem}-\textsc{rt}\_\textsc{ext}-happen-\textsc{rem}.3\textsc{sg}S\\
\glt `This is what happened.'
\z

\ea\label{ex:text06-15-5}
Däbe adawede ada dangesneyo ada ngaskäma ainin lelmeny a, auma we ngaska aya iddob e ballän llokttang.\\
\gll däbe	adawede	ada	d-a-nges-n-eyo	ada	ngaskäma	ain=in	lel=meny=a	au=ma=we	ngaska	aya	iddob=e	b-a-ll-än	llokott=ang\\
     that	so\_that	like\_this	\textsc{rem}-\textsc{rt}\_\textsc{ext}-make-\textsc{ipfv}-\textsc{rem}.3\textsc{nsg}A	like\_this	\textsc{pot}	who.\textsc{sg}=\textsc{cop}.\textsc{prs}.\textsc{sg}	fear=\textsc{priv}=\textsc{nom}	bury=\textsc{char}=\textsc{all}	\textsc{pot}	who.\textsc{sg}	night=\textsc{all}	\textsc{fut}.3S-\textsc{rt}\_\textsc{ext}-go-\textsc{rem}.3\textsc{sg}S	strong=\textsc{atr}\\
\glt `They did this to know who was brave enough to go to the grave at night.'
\z

\ea\label{ex:text06-16}
Eso.\\
\gll eso\footnotemark{}\\
     thank\_you\\
\glt `Thank you.'\footnotetext{\textit{Eso} is a regional word meaning `thank you' observed in many unrelated languages, including Southern Kiwai\il{Kiwaian!Kiwai}, Kalau Lagau Ya\il{Pama-Nyungan!Kalau Kawau Ya} and Torres Strait Creole\il{English Creole!Torres Strait Creole}.}
\z