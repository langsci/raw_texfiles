This is a fantastical story about an old woman and a small boy whose island gets flooded by an enormous wave. The old woman saves the boy by putting him in a small floating basket (see \figref{fig:baby-basket2}), then saves herself by turning into a fish. The boy floats in his basket and arrives at a new village. There, he is raised by a childless widow. The young boy grows up strong and has many wives---a happy ending.

In this story, we get a peek into Ende mythology. Many \textit{mabun eka} or origin stories involve shape-shifting, specifically, the ability for people to transform into one of the symbols of their clan, such as an animal or plant.\is{clan} Sometimes, this transformation has other fantastic properties, such as invisibility.

Though this story is the only one in the Ende collection that references a flood, the flood is a regional archetype that explains the movement of people to different places. For example, \citet[35]{Doehler2024} includes a \ili{Yam!Komnzo} text that describes how the killing of a mythological creature sparks a great flood that forces people to move to the north and south to escape its waters.

Adoption is a widespread practice in Ende culture and the region. While children are often raised in multiple households to ease the burden of child-rearing, adoption also takes place to balance families with too many or too few children, to satisfy sister-exchange traditions where women and girls are traded between clan groups to facilitate marriages, and to protect children born to unmarried parents.\is{clan}

Two lexical items of note in this text are \textit{ttongdae} `only one' and \textit{apte ttang lläpät} `five' in Lines \ref{ex:text10-2-2} and \ref{ex:text10-4-4}. These two lexical items represent two of the five distinct counting systems used in Ende. \textit{Ttongdae} is an inflected form of \textit{ttongo} `one,' which belongs to the set of Ende basic numerals that includes the numbers \textit{ttongo} `one,' \textit{komlla} `two,' and \textit{kumuddäga} `three.' These numbers take nominal clitics and can be combined to create larger numbers, such as four, five, and six. \textit{Apte ttang lläpät}, on the other hand, is one of two phrasal compounds to refer to the numbers five and ten: \textit{apte ttang lläpät} `lit: half of the digits of the hands' and \textit{komlla ttang lläpät} `lit: the digits of two hands.' Other counting systems include the Ende body part counting system which goes from one pinky to the other and counts to 19 (\textit{e.g.}, \textit{mända} `thumb, five'), the senary yam counting system, which is borrowed from the \ili{Yam} languages to the west and uses a base 6 to count yams, and the borrowed \ili{English} numerals, which can be found in other texts in this collection.


\begin{xltabular}{\textwidth}{llll}
    \caption{Three numeral systems in use in Limol}\label{tab:numeralsystems}\\
    
    \lsptoprule
        & \textsc{ende} (1-6) & \textsc{body-part} (1-19) & \textsc{yam}\il{Yam} (senary)\\
    \midrule
    \endfirsthead

    \multicolumn{4}{c}%
    {\tablename\ \thetable{} -- continued from previous page}\\
    \lsptoprule
        & \textsc{ende} (1-6) & \textsc{body-part} (1-19) & \textsc{yam}\il{Yam} (senary)\\

	\midrule
    \endhead

    \hline \multicolumn{4}{r}{{Continued on next page}}\\
    \endfoot

    
    \endlastfoot

   1 & \textit{ttongo} & \textit{tɨrangesa} ‘pinky’ & \textit{ttongo}  \\
   2 & \textit{komlla} & \textit{nɨtkin} ‘ring finger' & \textit{komlla} \\
   3 & \textit{kumuddäga}& \textit{kllatollma} & \textit{komlla a} \\
   & &  ‘middle finger' & \textit{ttongo duma} \\
   4 & \textit{komlla komlla}  & \textit{tupi} ‘pointer’ & \textit{komlla komlla} \\
   & ‘two-two’ & & \\
   5 & \textit{komlla komlla} & \textit{mända} ‘thumb’ & \textit{komlla komlla a}  \\
    &  \textit{a ttongo duma} & & \textit{ttongo duma} \\
   6 & \textit{kumuddäga}  & \textit{gabɨn} ‘wrist’ & \textit{putt}  \\ 
   & \textit{kumuddäga} ‘three-three’ & & \\
   & or \textit{komllaebme} & & \\
   & \textit{komllaebme komllaebme} & & \\
   & ‘two-two-two’ & & \\
   7 & & \textit{ttangkum} ‘elbow’ & \\
   8 & & \textit{matta} ‘shoulder’ &  \\
   9 & & \textit{ngam} ‘breast’ &  \\
   10 & & \textit{ddɨll} ‘chest’ &  \\
   11 & & \textit{apte ngam} &  \\
   & & ‘other breast’ & \\
   12 & & \textit{apte matta} & \textit{komlla putt}  \\
   & & & ‘two-six’ \\
   13 & & \textit{apte ttangkum} & \\
   14 & & \textit{apte gabɨn} & \\
   15 & & \textit{apte mända} & \\
   16 & & \textit{apte tupi} & \\
   17 & & \textit{apte kllatollma} & \\
   18 & & \textit{apte nɨtkin} & \\
   19 & & \textit{apte tɨrangesa} & \\
   36 & & & \textit{pärta} \\
   216 & & & \textit{taromba}  \\
   1,296 & & & \textit{damona} \\
   7,776 & & & \textit{waramakae} \\
   \lspbottomrule
\end{xltabular}