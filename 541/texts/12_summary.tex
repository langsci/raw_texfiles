\textit{Bundae bo pepeb} is a parable about a mythical man named Bundae. Bundae -- literally: `only a head' -- lives deep in the forest and has a distinctively large head. In this story, some rowdy children encounter Bundae in the forest while hunting\is{hunting} for birds. Bundae is sleeping and the children decide to crawl into his cavernous nostrils to clean and cook their birds. When Bundae awakes to find his nostrils full of debris, he plans to trick the children next time they invade his personal space. The next day, when the children crawl back into his nose to make a mess, Bundae sneezes them out, beats them, and sends them crying home to their parents. Some examples of the types of birds found in the area can be seen in Figures \ref{fig:bird-1}--\ref{fig:bird-9}.

Although Bundae is not a giant \textit{per se} -- only his head is large -- there are folk stories about giants in the Ende community and in the region (including in Idi\il{Pahoturi River!Idi} p.c. \name{Dineke}{Schokkin}). For instance, there are Ende stories about one man, who was 12 feet tall and would carry an axe or \textit{buitubuitu} `round stone axe'. If he killed a woman, he would say it was a cassowary. If he killed a man, he would say it was a pig. He had huge armbands (see \figref{fig:dance1}) and would put them on his aggressors. If they were loose, he would say, you are too weak to fight me. He was allegedly from \textit{Minkudd}, which is an origin place or \textit{mitma}. The origin place of the Ende tribe is said to be \textit{Dumoll}, while \textit{Minkudd} is the origin of the \il{Pahoturi River!Idi}{Idi}, \il{Pahoturi River!Agob}{Agob}, \il{Pahoturi River!Taeme}{Taeme}, and \ili{Gundme} tribes (p.c. \name{Nugan Paal}{Gurel}.)

One structure of note in this text is nonsingular \isi{reduplication}. Typically, nominals are unmarked for number. For example, in Line \ref{ex:text12-2-2} the nominals \textit{llɨg} `boy,' \textit{bägäl} `bow,' and \textit{täbäll} `arrow' are identical in form in singular contexts, although the context here is plural: `the boys got their bows and arrows.' However, a subclass of modifiers -- adjectives -- reduplicate in nonsingular conditions (cf. \textit{ulle} `big [head]' and \textit{ulleulle} `big [nostrils]' in Lines \ref{ex:text12-1-3} and \ref{ex:text12-1-4} or \textit{kälsre} `small [body]' and \textit{kälekäle} `small [boys]' in Lines \ref{ex:text01-1-5} and \ref{ex:text12-2-1}). A subclass of nominals -- \isi{kinship} nouns -- also undergo nonsingular reduplication (cf. \textit{nag} `friend' and \textit{nagnag} `friends').