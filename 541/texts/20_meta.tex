\begin{figure}[H]
    \centering
    \includegraphics[width=.8\textwidth]{\imgpath/contextual_images/image21.jpg}
    \caption{Bringing a \textit{käza} `crocodile' back to Limol village}
    \label{fig:crocodile2}
\end{figure}

\textit{Käza misima saima} is an ancient song for the \textit{Tizag} (\textit{Bobeag}) clan, which has the \textit{käza} `crocodile' as their primary totem. When a crocodile has been killed, the hunters sing this song to let the \textit{Tizag} clan know that one of their totem animals has been slain.\is{clan} Traditionally, a gift was also offered to the elders of the clan, out of respect for the loss of their totem. Unlike the other texts in this collection, this song is made up of untranslatable words. Some say that the words belong to a lost dialect, and others say that the lyrics are sacred words that belong to the \textit{Tizag} clan. There are other such untranslatable songs and phrases in the Ende language corpus. The Crocodile Song can be sung for hours, as indeed it was for the journey depicted in the photo above (\figref{fig:crocodile2}), when a 600-pound crocodile was carried from \textit{Karama} swamp up the hill to Limol village. Eight men took turns carrying the crocodile, which was bound to a wooden stretcher. You can hear \DKSfull{} sing a version of this song in the Ende language corpus \citep{Kurupel2016}.