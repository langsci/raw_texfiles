`Turtle's story' is a fable that explains how the \textit{kottllam} `turtle' got its cracked shell and home in the sea. In the beginning of the tale, Turtle is living a difficult life in a drought-stricken region with very little water. He is found by a giant bird, who recruits a friend to take him to a nicer village by the ocean. The birds transport Turtle by having him bite onto a long stick, warning him not to open his mouth during the flight. Turtle breaks this rule when he marvels at the beauty of the ocean and falls to the village, where his shell breaks into pieces. The villagers put Turtle back together and release him into the sea.

There are eight names for different species of turtles in the Ende dictionary \citep{Lindsey2017d}. The generic word for turtle, and the one used in this text, is \textit{kottllam}. More specific names include: \textit{atata kottllam} (see \figref{fig:atata-turtle}), \textit{gamo kottllam} (see \figref{fig:gamo-turtle}), \textit{gogo kottllam}, \textit{pall kottllam} (see \figref{fig:pall-turtle}), \textit{paro kottllam}, \textit{ulle kottllam} and \textit{uwo kottllam} (see \figref{fig:uwo-turtle}). The \textit{gamo} and \textit{pall} `red' turtles can be classified as the Pig-nosed turtle (\textit{Carettochelys insculpta})\is{fauna!\textit{Carettochelys insculpta} (Pig-nosed turtle)} and the New Guinea painted turtle (\textit{Emydura subglobosa}),\is{fauna!\textit{Emydura subglobosa} (New Guinea painted turtle)} respectively. 
%Photographs of these two turtles clearly show the red underbelly and long nose that distinguish these two species. 
Photographs of the \textit{atata} and the \textit{uwo} match the physical characteristics of the Northern snake-necked turtle (\textit{Chelodina rugosa})\is{fauna!\textit{Chelodina rugosa} (Northern snake-necked turtle)} and the New Guinea snapping turtle (\textit{Elseya branderhorsti})\is{fauna!\textit{Elseya branderhorsti} (New Guinea snapping turtle)} best.\footnote{\textit{Uwo} is also used to refer to the magnificent riflebird, a bird-of-paradise native to the region.} The \textit{ulle} `big' turtle may be another name for the Pig-nosed turtle, as that is the largest turtle in this region (\cite{Georges2006}), or it may be used for the New Guinea giant softshell (\textit{Pelochelys bibroni}),\is{fauna!\textit{Pelochelys bibroni} (New Guinea giant softshell)} the second largest.\footnote{I do not have photographs of the \textit{gogo}, \textit{ulle}, or \textit{paro} turtles. However, it is possible that \textit{gogo} refers to the other \textit{Emydura} turtle species as \textit{gogo} and \textit{pall} are often used in conjunction to name similar species where one is green and the other red. These words are also names for green and red sago\is{flora!\textit{Metroxylon sagu} (Sago)} palm trees.}

\citet{Georges2006} identified nine freshwater turtle species that inhabit the South Fly area in New Guinea -- the highest number of turtle species in the Australasian region. Curiously, none of the turtles in this list have very prominent scutes, which give many turtles and tortoises the appearance of a “cracked" shell. All nine species have very smooth shells, though the snake-necked and the snapping turtles have natural grooves or patterns that may resemble cracks. Perhaps one of these latter four was the inspiration for this story. Or, perhaps the story was borrowed from a nearby community, where turtles with “cracked" shells are more common.

Though the idea of a turtle with a “cracked" shell may have extra-local origins, the effects of drought and food scarcity are unfortunately well-known in the region. The yearly drought lasts three seasons, roughly corresponding to August, September, and October. These three seasons are called \textit{yäbäd} `dry season,' \textit{yäbäd bäng} `dry and hot season,' and \textit{yäbäd ttänttämang} `dry, hot, and burning season.' During these months, the residents of Limol rely on \textit{sana} `sago'\is{flora!\textit{Metroxylon sagu} (Sago)} (see \figref{fig:sago}), which can be harvested year-round, and floating swamp gardens for sustenance.

Some structures of note in the following text include the malefactive use of the dative. For example, in Line \ref{ex:text02-1-5}, the turtle thinks to himself “\textit{ngämlle giddollma da llokttang abal agan}" `living has become impossible for me', where \textit{ngämlle} is the first person dative form and the malefactive party in this event. Another interesting structure is the ventive \isi{associated motion} prefix \textit{i-}, which is found in verbs where the motion of the event is returning to a previous location or to the actual location of the speaker (\cite{Reed2021}). Ventive \textit{i-} is used in Line \ref{ex:text02-3-4} in the verb \textit{dingällbänän} `he got him and returned.'
