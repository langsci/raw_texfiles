\begin{Parallel}{0.47\textwidth}{0.47\textwidth}
    \ParallelLText{\textit{Keiti ttängäm me lla da eka de dätäbeyo Wur bun tawa mamoe e, ngämo bägäl a wa toboll a dänglläb, ngäna ibi di gongkam. Angde gopällttän, ngämo llɨg kälsre da, obo bin a Immanuel, mitmit dagän ngänäm, ge ddob llɨg a alla medädabim mitmit amallo. Da medäda ero we gopällttänalle, llɨg da mɨnyi mitmit bäganän ngänaeka pe\-yang. Ngäna obom dätram, ngämi llame deyareya mamoema.}}
    \ParallelRText{In KT village, some men were discussing plans to go hunting\is{hunting} in Wur Swamp. I got my bow and arrows, and I started walking. When I set off, my small boy, Immanuel, clung to me, the way young children cling to their fathers. If fathers set off, children will long for them with tears. I carried him, and we went hunting together.}
\end{Parallel}

\medskip

\begin{figure}
    \includegraphics[width=\textwidth]{\imgpath/illustrations/storyimage14}
    \caption{Hunting\is{hunting} with dogs}\label{fig:dogs}
\end{figure}

\begin{Parallel}{0.47\textwidth}{0.47\textwidth}
    \ParallelLText{\textit{Nyongo dae ngämi deyareya dowe Eramang gall tapma, gall e ngämi godagaleya, ngäna imne we, ede Immanuel ngattong e godmenän gall guwo me. Bobag daeya dedam Karama walle da adawatta yogoll ulle da dämanän. Ngämi tawa bo menae dae dagllaenalla ako ngämi gall me deyagirnalla, ddia we gongllaenalla. Mamoeya maenenang lla da däräng pe\-yang tawa de daddällgoeneyo, Wur bun tonton abal me.}}
    \ParallelRText{We walked down the road to Eramang canoe place, and we got in the canoe. I boarded in the back so Immanuel could sit in the front, in the heart of the canoe. Karama River was flooded because of heavy rain. We paddled through the swamp, stayed in the boat, and looked for deer. Other hunters with dogs went into the swamp, directly towards Wur.}
\end{Parallel}

\medskip

\begin{Parallel}{0.47\textwidth}{0.47\textwidth}
    \ParallelLText{\textit{Komlla ddia de deyangkoenmällneyo däräng a ngäsengäse dowe Taolang gall tapma. Ngäna gall de mängamängall dagllae ddia koenmäll e, gall alle gumbiebmeny. Komlla dagwaeya ddia da, ada gullbe täkäll peyang a ako mäg da ulle da. Ngäna gagäll kabag dae gall de dony ddia ngämenmäll e, Immanuel gall guwo me känyärtto gogän.}}
    \ParallelRText{The dogs chased two deer down into Taolang Canoe Place. I paddled the canoe quickly to chase the deer with the boat. There were two deer, one bull with horns and one doe. I did not direct the boat well through the flood grass to reach the deer. Immanuel stayed silent in the heart of the canoe.}
\end{Parallel}

\medskip
\largerpage[2]
\begin{Parallel}{0.47\textwidth}{0.47\textwidth}
    \ParallelLText{\textit{Angde ngämi ddia komlla de deyang\-meneya, ngäna llɨg kälsre de Immanuel dangnoe, ``Bongo ddia komlla de ikop yaralle?'' Bogo eka mu dagän ngänäm, ``Baba, ao gänyageyo ddia da.'' Ako ada eka dagän, ``Baba, ttongo nazu pakos alle.'' Ngäna umllang däga, ``Immanuel, bongo gall guwo me, ngäna walle we gäbän allan ddia llädäd e.''}}
    \ParallelRText{When we reached the two deer, I asked my small boy Immanuel, ``Do you see the two deer?'' He answered, ``Father, yes, those two deer are there.'' Then he said, ``Father, shoot one with the arrow.'' I told him, ``Immanuel, you stay in the canoe. I am jumping in the water to grab the deer.''}
\end{Parallel}
\clearpage

\begin{figure}
    \includegraphics[width=\textwidth]{\imgpath/illustrations/storyimage15}    
    \caption{Hunting\is{hunting} by boat}\label{fig:huntingboat}
\end{figure}

\begin{Parallel}{0.47\textwidth}{0.47\textwidth}
    \ParallelLText{\textit{Ngäna gogäbän walle we ddia dowae me, mäse llädäd e dängkam, ngämo llɨg kälsre de, Immanuel bom gongnongg, gall me daeya bogo, poper gognän, be ddone ngänaeka gogän näkäp alle ngäna dämbäl, ada Immanuel zäme walle we aspunan adawatta ngäna ddänddängeny me gogäbän gall atta.}}
    \ParallelRText{I jumped into the water close to the deer and tried to grab it. But in my mind, I thought of my small boy and how he was sitting in the boat, scared, but not crying. My mind raced and I thought Immanuel had fallen into the water when I jumped out of the canoe.}
\end{Parallel}

\medskip

\begin{Parallel}{0.47\textwidth}{0.47\textwidth}
    \ParallelLText{\textit{Be bogo dadewaeya gall guwo me dämenang dagernän, ngäna angde gongllae gall e, ttongo ddia de ikop däga. Immanuel däbe ttongo ddia de ikop alle dätraemänän, obo gall dowae dae dänggllanän. Immanuel känyärtto abal gogän. Ngäna gongäs gall e godagal, ngänäm Immanuel dangnoyän, ``Baba, ddia da eraya?'' Ngäna umllang däga obom ada, ``Ttam agan, adawatta ngäna bam ada ka bongo walle we aspunalle. Bam ngäna ullowae angnonggan, bongo llɨg kälsäre da yaralle ngämo peyang.''}}
    \ParallelRText{But he was still there sitting in the canoe. When I looked to the boat, I saw one deer. Immanuel was looking at that deer, he was swimming close to the boat. Immanuel stayed very quiet. I returned to the boat, got on, and Immanuel asked me, ``Father, where is the deer?'' I told him, ``He is alive because I thought that you fell into the water. I was thinking of you, you small boy who came with me.''}
\end{Parallel}

\medskip

\begin{Parallel}{0.47\textwidth}{0.47\textwidth}
    \ParallelLText{\textit{Ngämi mängamängall gogllaeya gall tapma we, gall de dowansegeya ede ma we ada deyareya Immanuel ngämaene ddia koenmällatt eka de ma me dɨllɨtnän mägda bälle, oblle kiliang daeya ngämaene ttoen llɨtɨt a. Iba umllang da llɨg kälekäle aba ttoen a, ende mäde da walle bangesneyo, mägda bälle mɨnyi ma me bɨllɨtnän obaene ttoen de.}}
    \ParallelRText{We quickly paddled back to the canoe place and went home. Immanuel told our story to his mother. She was happy when our story finished. We know that young children like to tell stories to their mothers about what they do with their fathers.}
\end{Parallel}
