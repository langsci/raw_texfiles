\textit{Baet bo llan a allame de tubutubu gogon} is an example of a \textit{pourquoi} story that gives an explanation of how something in nature came to be. This text explains why the cuscus (\is{fauna!\textit{Spilocuscus maculatus} (Cuscus)}{\textit{Spilocuscus maculatus}}; see \figref{fig:cuscus}) has short ears, particularly when compared with the dog. The tale goes that Cuscus and Dog used to be great friends. But one day, Dog played a trick on Cuscus, which resulted in Cuscus' ears being cut short. In retaliation, Cuscus condemned Dog to a life dependent on humans to survive. 

There are some parallels between this story and the parable \textit{Yu ingong} (\textref{text:yu}). Both stories revolve around a main character who imitates someone and consequently undergoes a terrible fate. 

\figref{fig:cuscus} displays the common spotted cuscus typically found in the Limol village region, while \figref{fig:huntingdogs} shows the type of dogs often raised for hunting\is{hunting} in the area. These dogs are likely a cross-breed between the New Guinea singing dog (\is{fauna!\textit{Canis lupus hallstromi} (New Guinea singing dog)}{\textit{Canis lupus hallstromi}}) and the Australian dingo (\is{fauna!\textit{Canis lupus dingo} (Australian dingo)}{\textit{Canis lupus dingo}}). Like the New Guinea singing dog, these dogs howl in groups.

One linguistic structure of note used in this story is diminutive \isi{reduplication}. While playing hide-and-seek, Dog hides himself in a \textit{mama} `palm leaf enclosure' or literally `small house' (see \figref{fig:mamahouse} and Line \ref{ex:text01-4-6}). A \textit{mama} does not look much like a typical \textit{ma} `house' in form (see \figref{fig:house}), but they serve a similar function and are made out of similar materials. Reduplication is used productively in Ende to derive \isi{diminutives}, adverbs, nonsingular \isi{kinship} nouns, and nonsingular adjectives.

Another morphological structure of interest is the \is{possession} possessive clitic \textit{=da}, which appears many times in this text. Possessive \textit{=da} differs from other possessive pronouns and clitics in that it can only refer to a third-person possessor (cf. \textit{obo} `his/her/their') and is only found after a closed class of \isi{kinship} nominals, such as \textit{nag} `friend' (see Line \ref{ex:text01-2-4}). In the interlinear text, possessive \textit{=da} is glossed as \textsc{cl\_poss.kin}, to indicate a close possessive \isi{kinship} relationship.

One additional type of possession featured in this story is \is{possession, ablative} ablative possession, which refers to a type of possession relationship where the possessor is the spatial or figurative source of the possessum. For example, \textit{Baet bäne dinduatt} `Cuscus's footsteps' (Line \ref{ex:text01-4-1}) uses the ablative possessive pronominal clitic \textit{bäne} to indicate that the footsteps are originating from Cuscus. Similarly, \textit{llaeyabaene tot} `people's rubbish' (Line \ref{ex:text01-7-11}) uses the ablative possessive to indicate that the rubbish has been discarded and is no longer considered to be in possession by the people who created it. In this example, the people are a temporal or figurative source of the rubbish.