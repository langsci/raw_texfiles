The \textit{Walking to the Grave} story is a comedic narrative told by \WKSfull{} about a trick that his father, \KKSfull{}, played during some funeral events in Malam in the 1940s (see \figref{fig:map} for the location of Malam). The narrative takes place after the burial of a \textit{Yawenang} man named \iai{Galo}, an Ende man whose grandchildren, great-grandchildren, and great-great-grandchildren lived in Limol in 2017, when the story was told. In the story, a group of men decide to play a game to test their bravery. The night after the burial, the men were challenged to carry a \textit{yu bäng} `fire stick' one-by-one to the grave and return without being frightened. Figure \ref{fig:firestick} shows a fire stick or firebrand. This dried cone may be the dried seed pod of \textit{Banksia dentata}\is{flora!\textit{Banksia dentata} (Fire stick)}, which can stay lit for hours and is used in northern Australia and southern New Guinea to transport fire.

The fear of graves and the recently deceased may be related to the widespread belief that the spirits of the dead may be present in these spaces. In the story, \KKS{} takes advantage of this fear to play a trick on \iai{Awayang}. When \iai{Awayang} takes his turn to visit the grave, \KKS{} surprises him by making a lot of noise, which sends \iai{Awayang} running back to the group of men in fear. As he runs, he shouts \textit{päzäg gädo}, which means `my brother-in-law is here' in Idi. This part of the story reveals several aspects of Ende culture. First, \iai{Awayang}'s belief that it was \iai{Galo}, his brother-in-law, that made the noise supports the hypothesis that the fear of graves is related to the fear of encountering the spirit of the recently deceased. Second, his shouts in Idi reflect the multilingual landscape of the community in the 1940s and the present. \WKS{} repeats this line seven times before translating it into Ende, but leaves it untranslated during the climax of the story, which is what solicits all the laughter from his audience. When he does translate the phrase into Ende, it is likely that he does so for the sake of the Ende-speaking linguist, not the usual audience who has basic knowledge of many local languages, including Idi. Finally, \iai{Awayang}'s shouts remind us of the in-law name taboo practices upheld in the Ende community. It is forbidden to call your in-laws by name; instead, they are referred to by their relationship. In this case, \iai{Awayang} respectfully refers to his dead brother-in-law as \textit{päzäg} `brother-in-law'.

At the end of the story, \KKS{} sticks out his tongue at \iai{Awayang}, indicating that he had tricked him. Showing one's tongue is just one of many paralinguistic gestures used by the Ende community. One linguistic particle observed in speech that contains trickery or multiple states of belief is \textit{ka}, the counterfactual (see Line \ref{ex:text06-11-6}). \textit{Ka} appears in diverse contexts such as counterfactuals, hypotheticals, potentials, rhetorical questions, and corrective negation (\cite{Tighe2022}). In counterfactual and hypothetical constructions, \textit{ka} marks scenarios as contrary to known reality or imagined possibilities, often framing situations of mistaken belief or unrealized expectations. The mistaken belief can involve new information, or it can indicate a separate reality from the one held in the common ground (\textit{e.g.}, in Line \ref{ex:text06-11-6}, \textit{ka} marks the event of chasing a prankster who is known to be pretending to be a ghost).

A photo of a typical grave site is shown in \figref{fig:grave}. This photo shows the graves of twin girls Kate and Grace, who died at two months of age. The signs above the graves provide the names and dates of birth and death for the deceased. Three crocodiles decorate the signs, indicating the clan totem of the girls: the \textit{käza} `crocodile' clan.\is{clan} The names Kate and Grace were given to the babies as they were born during \KLLfull{} and \GMMfull{}'s first visit to Limol in 2015. \figref{fig:grace-kate} shows \GMM{} and \KLL{} planting a coconut\is{flora!\textit{Cocos nucifera} (Coconut)} tree in honor of the opening of the Limol health center.