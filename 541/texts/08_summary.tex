This is a journey tale told by \TDDfull{}, about the time she and a party traveled to Kurunti and encountered a \is{fauna!\textit{Crocodylus halli} (crocodile)}{crocodile} on the way. Like \DKS{} in the previous story, \TDD{} bravely kills the crocodile, protecting the fishing\is{fishing} group, and creating an opportunity to make some money in the city by selling the skin. \TDD{} also describes how the group camped and hunted\is{hunting} along the way, a beloved Ende practice.

One notable linguistic feature in this text is an \isi{experiencer-object construction} in Line \ref{ex:text08-1-4}. In this construction, the word order is object-agent-verb (OAV), differing from the language’s typical agent-object-verb (AOV) order. Here, the experiencer (the entity feeling the emotion) is in the object case, the agent is the stimulus causing the experience, and the verb is often an auxiliary. For example, the sentence \textit{ngämim ddäddäg abal da deyagnegnän} describes the group’s hunger for meat, literally translated as `meat really got us.' This reflects how the meat (agent) affects the group (experiencer). 

Another significant feature is the \isi{inclusory construction}, exemplified in Line \ref{ex:text08-2-13}. This type of construction includes two noun phrases that reference participants, where the first refers to the entire group and the second just a subset of that group. The first noun phrase often includes a non-singular pronoun like \textit{we} or \textit{they}, while the second noun phrase is marked by the instrumental clitic =(\textit{w})\textit{alle}. For example, Line \ref{ex:text08-2-13} references two participants -- the author, \TDD{}, and her friend Tim -- who are continuing to fish in a crocodile-infested pond. The author uses the pronoun \textit{ngämi} `we' to refer to the two of them, the modifier phrase \textit{Tim alle} `with Tim' to name the participant besides herself, and the verb \textit{däpamnalla} `we fished,' which agrees with both of the participants. We can contrast the inclusory construction with a more typical \isi{comitative construction} like \textit{bongo ... yaralle ngämo peyang} `you came with me' from Text \ref{text:Tawa}, Line \ref{ex:text04-4-6}. In the comitative construction, the verb \textit{yaralle} `you (\textsc{sg}) came' only agrees with the pronoun \textit{bongo} `you (\textsc{sg})' and does not index the modifying noun phrase \textit{ngämo peyang} `with me.'