\begin{figure}[H]
    \centering
    \includegraphics[width=.8\textwidth]{\imgpath/contextual_images/Ende_20160730_LIZCAM_072.JPG}
    \caption{\name{Sali}{Goge (Wik)} and \name{Rhoda}{Kukuwang} performing the \textit{Kakayam bo tongoe} song and dance with their dance ensemble in 2016}
    \label{fig:kakayam}
\end{figure}

The Bird of Paradise song and dance was written by \name{Sali}{Goge (Wik)} to be performed by the Malam Culture and Dance Group (see \figref{fig:dance2}). When the Bird of Paradise song is performed, the dancers wear grass skirts, cassowary feather armbands, legbands, and headdresses, and carry \textit{abor}, a tool used to pound sago\is{flora!\textit{Metroxylon sagu} (Sago)} (see Figure \ref{fig:kakayam}). The dancers sing and are accompanied by the kundu drum (played by the decorated man in the foreground of Figure \ref{fig:kakayam}). The choreography for this song emulates the movements of the bird of paradise with flapping elbows, jumps, and turns. The lyrics repeat the lines `The Bird-of-Paradise is singing in the garden of Eden. The people, we see how he plays. Kwa-o, kwa-o, ke-ke-ke.' Here, the surroundings of the Ende villages are compared to the Garden of Eden from Christian mythology. This type of comparison is common among Ende-based Christian rhetoric, where the stories in the Bible have direct equivalence to local places, customs, and even ancestry. The Bird-of-Paradise has a special cultural importance in Ende culture and in Papua New Guinea. The Bird-of-Paradise is the national bird of Papua New Guinea and the totem of many local clan groups.\is{clan} The Bird-of-Paradise is renowned for its long tail feathers and elaborate dances.

The song, dance, and story behind the song can be heard and viewed in the Ende language corpus \citep{Goge2015,Goge2016,Goge2016b,Goge2016a,Goge2016c}.