\section{The language and its speakers}\label{sec:language}

This text collection features stories and songs in the Ende language. Ende (Glottocode: ende1235)\footnote{Not to be mistaken for the Ende language of Indonesia (Glottocode: ende1246; ISO-639: end).} is a \ili{Pahoturi River} language spoken in the South Fly region of Western Province, Papua New Guinea. Ende is sometimes classified as a dialect of \il{Pahoturi River!Agob}{Agob} (ISO-639: kit) due to similar lexical inventories (\cite{Eberhard2019}), but preliminary documentation indicates substantial grammatical differences between the two varieties. The \ili{Pahoturi River} language family (see \figref{fig:map}) cannot be convincingly grouped with the \ili{Trans-New Guinea} languages to its north, the \ili{Yam} languages to its west, the \ili{Pama-Nyungan} languages to its south, or the Kiwaian\il{Kiwaian!Kiwai} and \ili{Eastern Trans-Fly} languages to its east (\cite{Evans2018}), though extensive contact with these families has been used as evidence for Proto-Pahoturi reconstruction efforts \citep{Chon2025}. For more on the \ili{Pahoturi River} family, see the family portrait by Lindsey, Schokkin, and Wu \citeyearpar{Lindsey2022}.

\begin{figure}
    \centering
    \includegraphics[width=0.9\linewidth]{figures/contextual_images/PRmap.pdf}
    \caption{Map of villages in Western Province (Papua New Guinea) where \ili{Pahoturi River} languages are primarily spoken. The shading indicates which villages are described locally as having that language as a primary language of the village. The village language and GPS data were collected by \KLL{} and D. Schokkin in 2017.\ia{Schokkin, Dineke} \\ $\copyright$ Australian National University CC BY SA 4.0 CartoGIS CAP 18-356\_KP}
    \label{fig:map}\il{Eastern Trans-Fly!Bine}\il{Yam!Len}\il{Yam!Nen}\il{Trans-New Guinea!Abom}\il{Trans-New Guinea!Bitur}\il{Pama-Nyungan!Kalau Kawau Ya}\il{Pahoturi River!Idi}\il{Pahoturi River!Taeme}\il{Pahoturi River!Kawam}\il{Pahoturi River!Em}\il{Pahoturi River!Agob}\il{Trans-New Guinea!Makayam}\il{Gogodala-Suki!Gogodala}\il{Trans-New Guinea!Kiunum}\il{Gogodala-Suki!Ari/Waruna}\il{Eastern Trans-Fly!Wipi}\il{Eastern Trans-Fly!Gizrra}
\end{figure}

Historically, the South Fly region has experienced limited contact with colonizing forces and outside influences. However, \il{English!English (Australian)}{(Australian) English} and Tok Pisin\il{English Creole!Tok Pisin} were introduced during the colonial period in domains like education, religion, and governance. The introduction of these \textit{lingua franca}s, along with a colonial-era mandate for small kin groups to band together in named villages, disrupted a long tradition of egalitarian multilingualism in which locals lived in clan-size\is{clan} hamlets and had at least a passive understanding of the languages of nearby groups. Older Ende community members recall distinct dialects that have been lost since the community converged as a village in the first part of the 20\textsuperscript{th} century (\cite{Zakae2018}).

Nowadays, Ende is spoken by between 600 (\cite{Eberhard2019}) and 1000 (\cite{Dareda2016b}) speakers, primarily in the South Fly villages of Limol, Malam, and Kinkin (see \figref{fig:map}), with some speakers residing in larger towns such as Daru and Port Moresby. In Limol and Malam, Ende is spoken by all generations and in all domains except primary education, where \ili{English} is mandated. Multilingualism is common among Ende speakers. In addition to Ende, speakers use other \ili{Pahoturi River} languages, such as \il{Pahoturi River!Taeme}{Taeme} (Glottocode: tame1238), \il{Pahoturi River!Kawam}{Kawam} (Glottocode: kawa1281), \il{Pahoturi River!Em}{Em} (no Glottocode), and \il{Pahoturi River!Agob}{Agob} (Glottocode: agob1244). They also speak regional languages like \il{Trans-New Guinea!Bitur}{Bitur} (Glottocode: bitu1242) and \il{Gogodala-Suki!Gogodala}{Gogodala} (Glottocode: gogo1265), and regional \textit{lingua francas}, such as \il{Kiwaian!Kiwai}{Kiwai} (Glottocode: sout2949), \il{English Creole!Tok Pisin}{Tok Pisin} (Glottocode: tokp1240), and \il{English!English (Australian)}{(Australian) English} (Glottocode: aust1314). In terms of language endangerment, I consider Ende to be vigorous/safe (\cite[EGIDS 6a;][]{Lewis2010}). However, ongoing efforts by \ia{The Ende Language Committee}{the Ende Language Committee}, established in 2003, to develop a writing system, promote literacy, and utilize the language in local elementary schools show that the community hopes to safeguard the language further. The writing system in use in this collection is part of these language-strengthening efforts (see \sectref{orthography}).\is{orthography}
%For more on the Ende language context in Limol and Malam, see \KLL{} and Munsiff (forthcoming).

\subsection{Clan and kinship}\is{clan}
A foundational part of social organization in Limol is clan, or \textit{tän}, a \isi{kinship} system inherited patrilineally. Members of the Ende community are divided into two clans: the \textit{Ddɨliag} and the \textit{Yamkong}, each associated with a symbolic color and bird. The \textit{Ddɨliag} clan is represented by the color \textit{pällämpälläm} ‘white’ and the \textit{kakayam} ‘bird of paradise.’ The \textit{Yamkong} clan is represented by the color \textit{mamam} ‘red’ and the \textit{inpiak} ‘eagle’ (K. Dobola, p.c., 2018). Within the large clans are several subclans and sub-sub-clans that further organize families. Each subclan has three additional symbols: a \textit{mabun} ‘totem,’ a \textit{pa} ‘bird,’ and a \textit{tawar} ‘mark.’ These symbols may be shared across subgroups. Often, \textit{tän} that share the same totem, bird, or mark are more closely related than other groups.

Clan,\is{clan} as a social construct, has been identified as a variable with the potential to explain sociolinguistic variation within a community \citep{Stanford2009}. Quantitative studies of sociophonetic variation within Ende, such as /n/-deletion and \isi{retroflex affrication}, have not found significant correlation between clan membership and these patterns \citep{Lindsey2021b,Strong2022}. Still, some speakers suggest that dialectal differences exist between clan groups (\cite{Dobola2018}; \cite[253]{Kurupelsuwede2018a}). Curiously, there is some indication that the notion of clan may be becoming less important in modern society; in interviews, some community members did not know their mother’s clan \citep[160]{Kurupelsuwede2018d}, their spouse’s clan \citep[212]{Sowati2018}, or even sometimes their own \citep[103]{Geoff2018}. This drop in awareness could indicate a shift in the importance or usefulness of \textit{tän} groups within the community.

\subsection{Marriage}
Clan\is{clan} also influences which marriages are allowed, such that there is a strong preference for cross-clan marriages. Of forty-seven marriages surveyed in 2018, only 15\% were between two members of the \textit{Yamkong} clan. Unions between two \textit{Ddɨliag} members are even rarer, at 6\%. Most marriages (79\%) are between \textit{Ddɨliag} and \textit{Yamkong} people. Regardless of large clan affiliation, marrying into one’s sub-clan is prohibited. This appears to be a fixed rule, likely to maintain biological exogamy within the small population.
The traditional marriage model entails the practice of sister-exchange marriages called \textit{erang}.\footnote{\textit{Erang} is also the \isi{kinship} term that the four members of the exchange will call one another. Kinship terms also exist for one’s parents’ exchange sister (\textit{erngazmäg}), one’s parent’s exchange brother (\textit{erngazenda}), and one’s exchange cousins (\textit{erngazeg}).} This is when two (or more) clans swap daughters to marry the other family’s son. Clan affiliation, however, does not change for the women; they maintain affiliation with their father’s clan group.\is{clan}\is{kinship}

\subsection{Religion}
The Ende community is generally reluctant to discuss the spiritual and religious practices that predate their mass conversion to \isi{Christianity} in the 1960s. Knowledge of this sort is typically passed down in private initiation ceremonies and family-owned stories called \textit{mabun eka} that are not shared publicly.

Some former practices, including multiple forms of magic, are referenced at the end of \textref{text:yu}, when an older man brings another man back to life, in \textref{text:Auma}, when an older woman turns into a fish, and in the documentary \textit{Ende Tän e Indrang} ‘Light into Ende Tribe’ \citep{Warama2018b}. This short film tells the story of two couples -- brothers Dipa\ia{Nägäm, Dipa} and \name{Diwa}{Nägäm} and their wives \name{Mangkol}{Sobam} and \name{Wäli}{Wäziag} -- who traveled on behalf of the Ende community to Balimo to meet with Gogodala\il{Gogodala-Suki!Gogodala} and foreign missionaries. There, they learned about Western religion (Christianity), Western medicine, and Western education, bringing this knowledge back to the Ende community. This news was celebrated by the community, who exchanged their former spiritual practices, medicines, and educational traditions for these new Western manners.

Nowadays, \isi{Christianity} plays a vital role in the Ende community. In 2018, two Protestant denominations were practiced in Limol and Malam: a Lutheran church body that meets on Sundays and a Seventh-Day Adventist congregation that meets on Saturdays. In both churches, I observed sermons and songs primarily in Ende, and sometimes \ili{English}, and multiple community groups within the church that meet throughout the week to discuss special interests.

\subsection{Education}
Traditional methods of instruction and knowledge development are practiced in both informal and formal settings within the Ende community. Informally, novices are encouraged to observe daily activities, such as resource gathering and community building, from a very young age and to begin participating as soon as their abilities allow. Children are observed swinging machetes, climbing coconut palms\is{flora!\textit{Cocos nucifera} (Coconut)}, carrying small sago\is{flora!\textit{Metroxylon sagu} (Sago)} bundles, and tagging along on hunting trips as soon as, and sometimes before, their dexterity allows them to do so without hurting themselves or others. More formal instruction came in the form of gender-segregated initiation ceremonies that were once offered as a cohort of children entered puberty. The youngest people surveyed to have participated in such ceremonies were over 40 years old. There is some interest in reintroducing these ceremonies and/or recording them for posterity. On a daily basis, formal instruction is offered in the form of \textit{kawa}, public speeches delivered early in the morning or late at night by an orator who walks the paths of the village, sometimes stopping in front of an individual house for extra effect. These messages often convey community values, such as respecting others' property, waking up early to care for one’s family, and contributing to community-led projects.

Formal Western education was introduced in the South Fly area in 1965 with the establishment of the Upiara Primary School. Since then, primary education has been compulsory for all Ende children. Everyone born after 1960 traveled to Upiara to attend the \ili{English}-medium primary school in their youth. In 1995, an elementary school was established in Limol that was attended by all those born after 1989. In 2007, the Ende Language Committee\ia{The Ende Language Committee} submitted an official \isi{orthography} and texts in Ende to the Department of Education of Papua New Guinea, establishing Ende as a sanctioned language for teaching. This measure allows the elementary school teachers to teach in Ende and \ili{English} (\citealt{Mado2018a}; \citealt[54]{Jowanang2018}).

\section{Typological overview}\label{sec:typological}

Based on our modest understanding of the South Fly linguistic landscape, Ende is typologically consistent with other \ili{Pahoturi River} languages, which share some areal similarities to other local non-Trans-New Guinea language families (\cite{Evans2018}).

Regarding the \isi{phoneme inventory}, Ende shows phonemic contrasts between the consonants and vowels in Tables \ref{tab:consonants} and \ref{tab:vowels} (see \cite{Lindsey2021} for more details). Researchers have noted several key features: a sizable \isi{liquid inventory} (\cite{Evans2019b}), variable affrication of retroflex obstruents (\cite{Strong2022}),\is{retroflex affrication} an irregular \isi{vowel harmony} system (\cite{Kohut2021}), and limited use of stress or intonational prosody to mark word meaning or sentence type (\cite{Dailey2023}).

\begin{table}[!ht]
    \caption{Consonant inventory}
    \label{tab:consonants}
    \begin{tabularx}{\textwidth}{lcccccc}
        \lsptoprule
        & Labial&Alveolar&Retroflex&Palatal&Velar&Labiovelar\\
        \midrule
        Voiceless stop&p&t&ʈ͡͡ʂ <tt> &&k&\\
        Voiced stop&b&d&ɖ͡ʐ <dd> &&ɡ <g>&\\
        Nasal&m&n&&ɲ <ny>&ŋ <ng>&\\
        Tap/flap&&ɾ <r>&ɽ <ll>&&&\\
        Voiceless fric.&&s&&&&\\
        Voiced fric.&&z&&&&\\
        Approx.&&&&j <y, e>&&w\\
        Lat. approx.&&l&&&&\\
        \lspbottomrule
    \end{tabularx}
\end{table}


\begin{table}[!ht]
    \caption{Vowel inventory}
    \label{tab:vowels}
    \begin{tabularx}{.6\textwidth}{lccc}
        \lsptoprule
         &Front&Central&Back\\
         \midrule
        High&i&&u\\
        Mid-high&ɪ <ɨ>&&\\
        Mid&e&&o\\
        Mid-low&&ə <ä>&\\
        Low&&ɐ <a>&\\
        \lspbottomrule
    \end{tabularx}
\end{table}

About morphology, Ende exhibits an extensive set of case clitics, which attach to nominal and verbal phrases, limited nominal inflection, and complex \isi{verbal inflection}. The case clitics and verbal templates are described in detail by \KLL{}, Schokkin, and Wu (\citeyear{Lindsey2022}). These templates feature multiple and distributed exponence, such that the meaning of any given category, such as tense or argument number, is distributed across numerous morphemes, and the morphemes themselves may be associated with multiple meanings. Reduplication\is{reduplication} has inflectional and derivational functions and is observed in nominals and verbs (\cite{Scanlon2018a}). Some inflectional categories of interest include a ventive/allative system of \isi{associated motion} (\cite{Reed2021}) and verbal number.

As for syntax, sentences have regular subject-object-verb (SOV) word order (\cite{Brown2020}) and nominative-accusative alignment in both argument flagging and indexing (\ref{ex2}, from \textref{text:Ngaemaene}). 

\ea
{\relax}[\textbf{Mareyas}]\textsubscript{S} nyongo meae [\textbf{dirom gullbe de}]\textsubscript{O} [\textbf{paya dägagän}]\textsubscript{V} gabma bägäl alle.\\
\gll Mareyas	nyongo=me=ae	dirom	gullbe=de	{paya\footnotemark}	d-ä-gag-än	gabma	bägäl=alle\\
     \textsc{pn}	road=\textsc{loc}=\textsc{rst}	cassowary	huge=\textsc{acc}	shoot	\textsc{rem}-3\textsc{ndu}P-\textsc{aux}-\textsc{rem}.3\textsc{sg}A	white\_person	bow=\textsc{ins}\\
\glt `As we went, Mareas shot a huge cassowary with a gun [lit., white person's bow].'
\Corpus{WE\_SN024:3}{Dobola2016e}\label{ex2}
\z\footnotetext{from \ili{English} \textit{fire}}

Ende follows other trends for object-verb languages (\cite{Dryer1991,Dryer1992}), including exhibiting postpositions, postnominal adjectives, main verbs before auxiliaries, and predicates before copulas. However, some aberrant word orders have been observed, including phrase-initial determiners, one preposition, and some prenominal adjectives. Adnominal property words even occur discontinuously. For example, in (\ref{8.ripe}) the modifier phrase \textit{wo abal} `very ripe' and the nominal \textit{up} `banana'\is{flora!\textit{Musa spp.} (Banana)} occur on different sides of the verb phrase \textit{ikop dägaeyo} `they saw it.'

\ea
{\textbf{Up} de}	adade	ikop	dägaeyo	\textbf{wo	abal}.\\
\gll up=de	adade	ikop	d-ä-ga-eyo	wo	abal\\
banana=\textsc{acc}	like\_this	see	\textsc{rem}-3\textsc{ndu}P-\textsc{aux}.3\textsc{sg}P-3\textsc{nsg}A	ripe	very\\
\glt `They saw very ripe bananas\is{flora!\textit{Musa spp.} (Banana)}.'\\
\Corpus{RE\_EN025:8}{Warama2016d}\label{8.ripe}
\z

One notable exception to SOV word order is \isi{experiencer-object construction}s, which are typically OSV (\ref{ex1}, from \textref{text:Ngaemaene}). In a phrase like `the boy is hungry', hunger is the stimulus in the nominative case. The subject is preceded by the boy, the experiencer in the accusative case. The verbs in these constructions are typically auxiliaries.

\ea
Ngämi ddone ada kili gogaebne, adawatta [\textbf{ngämim}]\textsubscript{O} [\textbf{ddäddäg abal da}]\textsubscript{S} [\textbf{deyagnegnän}]\textsubscript{V}.\\
\gll ngämi	ddone	ada	kili	g-o-g-aeb-ne	adawatta	ngämim	ddäddäg	abal=da	d-ey-a-g-neg-n-än\\
     1\textsc{nsg}.\textsc{excl}.\textsc{nom}	a\_lot	like\_this	happy	\textsc{rem}-\textsc{rt}\_\textsc{ext}-\textsc{aux}-\textsc{pl}S-\textsc{ipfv}	because	1\textsc{nsg}.\textsc{excl}.\textsc{acc}	edible\_animal	very=\textsc{nom}	\textsc{rem}-1\textsc{pl}P-\textsc{rt}\_\textsc{ext}-\textsc{aux}-\textsc{sg}>\textsc{pl}-\textsc{ipfv}-\textsc{rem}.3\textsc{sg}A\\
\glt `We were so happy because we were very hungry for meat.'
\Corpus{WE\_SN024:4}{Dobola2016e}\label{ex1}
\z

Some semantic categories of note include an inclusivity distinction in the first-person pronouns, dual and nondual argument \isi{agreement}, verbal suppletion for participant number, three distance contrasts in demonstratives, and a genitive/ablative distinction in the \is{possession} possessive paradigm (see the discussion in the summary of \textref{text:Baet}).

\subsection{Related work on Ende}
Work on the language includes a translation of the Book of Mark into Ende by \WKS{}, \TTW{}, \WGG{}, and the Ende Language Committee in collaboration with the Lewada Bible Translation Centre (\citeyear{Kurupelsuwede2009a}), a PhD dissertation on Ende phonology with an appended sketch grammar by \KLL{} (\citeyear{Lindsey2019}), and a Master's thesis by \CAS{} on Ende \isi{reduplication} (\citeyear{Scanlon2021}). An illustration of Ende's phonology was published in the Journal of the International Phonetic Association (\cite{Lindsey2021}), and its phonetic variation was discussed in multiple articles by Lindsey (\citeyear{Lindsey2021b}) and Strong (\citeyear{Strong2022}). Two archival collections feature Ende recordings in PARADISEC: The Language Corpus of Ende and other Pahoturi River Languages (LSNG08; \cite{Lindsey2015b}) and Ende Recordings (CS3; \cite{Scanlon2018a}). Most recently, a collection of Ende Material Knowledge has been archived with the Endangered Material Knowledge Project at the British Museum (\cite{Scanlon2025}).

\section{Storytelling and data collection}\label{sec:source}

Storytelling is a cultural practice that plays an important role in Ende life. It is a source of entertainment and a way to transmit cultural knowledge, values, and norms across generations. There are multiple words for `story': \textit{pepeb}, \textit{ttoen}, \textit{eka} and \textit{mabun eka}. A \textit{pepeb} is a tale about unseen people or spirits that is told to young children only by the older generation, especially by older women. A \textit{ttoen} or \textit{eka} is a recollection of real events in the teller's life, or as was told to the storyteller. A \textit{mabun eka} is a sacred story that belongs to a clan group and is passed down from clan elders to new members during initiation ceremonies.\is{clan} It is generally agreed that these stories should not be shared outside the clan; if they are, they should not be retold. The stories in this collection are of the first two types: \textit{pepeb} and \textit{ttoen}.

\subsection{Sources of the data}
The Ende Language Committee, composed of interested community members and visiting linguists, collected the texts presented in this collection in Limol village between 2015 and 2017. We recorded these stories to create books of Ende stories in the Ende language. The village announced a call requesting storytellers and illustrators to submit their stories or drawings of stories to the Ende Language Committee, who assembled the submissions into physical books for the local elementary school (\cite{Johnson2016,Johnson2016a,Karao2016,Reed2017,Reed2017a}). The Ende Language Committee transcribed and edited the recorded stories. The stories were also translated into \ili{English} and analyzed at the morpheme, word, and phrase-level by various committee members, including \JBD{}, \JJD{}, \WKS{}, and \TTW{}, and visiting linguists, including \KLL{}, \CAS{}, and \LWR{}. We did not include these translations in the original publications, but I have added them to this collection in the form of interlinear glossing and phrase translations for an \ili{English}-speaking audience.

The stories presented here are written texts, meaning that the narratives either originated in written form or were initially spoken but edited into a written form after the recordings were transcribed. Although storytelling in Ende is almost exclusively an oral tradition, the Ende Language Committee\ia{The Ende Language Committee} has strong intuitions about how the stories should be written down. Thus, though the texts in this collection differ from their oral counterparts, they represent a new and evolving cultural tradition in the community. 

\subsection{Orthography development}\label{orthography}\is{orthography}
To write the stories down, the Ende Language Committee decided on a standardized orthography to use for representing all Ende texts. The orthography development began when the Ende Language Committee was established in 2003. At that time, the Ende tribe received an invitation from the Lewada Bible Translation Center in the nearby village of Lewada on the south bank of the Fly River. The village was asked to send several volunteers to spend several years in Lewada developing a writing system and translating the Book of Mark into Ende \citep{WaramaKurupel2007}. While at the Lewada Bible Translation Center, the volunteers, including \WKS{}, \WGG{}, \JBD{}, and \TTW{} (contributors introduced below), worked with \name{Shim}{Jae-Wook} (SIL) to establish an orthography for Ende and produce a short Ende reader to test it. In 2007, the Ende Language Committee completed translating the Book of Mark into Ende, which has since become a source of great pride among the Ende tribe. The volunteers then returned to Limol village to teach others how to read and write in Ende.

The resulting Ende orthography has a direct mapping of phonemic sounds to written characters, as shown in \tabref{tab:characters}. Digraphs  <tt>, <dd>, <ny>, and <ng> for the retroflex affricates, palatal nasal, and velar nasal are also in use in the writing systems for other Pahoturi River languages, such as \il{Pahoturi River!Kawam}{Kawam}, \il{Pahoturi River!Idi}{Idi}, and \il{Pahoturi River!Taeme}{Taeme}. The use of the digraph <ll> for the retroflex flap is unique to Ende but shares a pattern with the other retroflex sounds. The vowel symbols <ä> and <ɨ> are used in many Pahoturi River languages, but for different vowels. In Ende, <ä> is used for the mid-central schwa and <ɨ> for the mid-high front vowel. These vowels are phonologically reduced and are not written in all \ili{Pahoturi River} languages. The palatal approximant is written with a <y> in syllabic onsets and with an <e> in codas. Though all the digraphs are made up of characters used for single graph phonemes, there is very little ambiguity between the digraphs and sequences of similar consonants (\textit{e.g.}, \textit{ny} /nj/ and \textit{ny} /ɲ/). This is because Ende does not have geminates or long consonants, has limited consonant clusters, and nasals tend to assimilate in place before stops and fricatives. 

%Characters that differ from the International Phonetic Alphabet are organized in \tabref{tab:characters} (see also: \cite{Lindsey2021}).

%\begin{table}[h]
%    \caption{Characters in the Ende orthography whose mapped sounds correspond to different IPA symbols}
%    \label{tab:characters}
%    \begin{tabularx}{.55\textwidth}{ll}
%    \lsptoprule
%        \textsc{character} & \textsc{phoneme} (\cite{Lindsey2021})\\
%    \midrule
%        a&ɐ\\
%        ä&ʌ$\sim$ə\\
%        e&e, j (post-vocalic)\\
%        ɨ&ɘ$\sim$ɪ\\
%        dd&ɖ$\sim$ɖ͡ʐ\\
%        ll&ɽ\\
%        ng&ŋ\\
%        ny&ɲ\\
%        r&ɾ\\
%        tt&ʈ$\sim$ʈ͡͡ʂ\\
%        y&j\\
%        z&z$\sim$d͡ʒ\\
%    \lspbottomrule
%    \end{tabularx}
%\end{table}

\begin{xltabular}{\textwidth}{lllll}
    \caption{Pahoturi River orthographic conventions \citep[49]{Lindsey2022}}\label{tab:characters} \\
    \lsptoprule
    Phoneme & Ende & Kawam & Idi & Taeme \\\midrule
    \endfirsthead

    \multicolumn{5}{c}%
    {\tablename\ \thetable{} -- continued from previous page}\\
    \lsptoprule
    Phoneme & Ende & Kawam & Idi & Taeme \\\midrule
    \endhead
    \hline \multicolumn{5}{r}{{Continued on next page}}\\
    \endfoot
    \endlastfoot
    
    p & \textit{p} & \textit{p} & \textit{p} & \textit{p}\\
    {b} & \textit{b} & \textit{b} & \textit{b} & \textit{b}\\
    {t} & \textit{t} & \textit{t} & \textit{t} & \textit{t}\\
    {d} & \textit{d} & \textit{d} & \textit{d} & \textit{d}\\
    {k} & \textit{k} & \textit{k} & \textit{k} & \textit{k}\\
    {ɡ} & \textit{g} & \textit{g} & \textit{g} & \textit{g}\\
    {\t{kp}ʷ} & \textit{--} & \textit{--} & \textit{q} & \textit{kw}\\
    {\t{ɡbʷ}} & \textit{--} & \textit{--} & \textit{ḡ} & \textit{gw}\\
    {\t{ʈʂ}} & \textit{tt} & \textit{--} & \textit{th} & \textit{tt}\\
    {\t{tʃ}} & \textit{--} & \textit{ch} & \textit{--} & \textit{--}\\
    {\t{ɖʐ}} & \textit{dd} & \textit{--} & \textit{dh} & \textit{dd}\\
    {\t{dʒ}} & \textit{--} & \textit{jh} & \textit{--} & \textit{--}\\
    {s} & \textit{s} & \textit{s} & \textit{s} & \textit{s}\\
    {z} & \textit{z} & \textit{z} & \textit{z} & \textit{z}\\
    {m} & \textit{m} & \textit{m} & \textit{m} & \textit{m}\\
    {n} & \textit{n} & \textit{n} & \textit{n} & \textit{n}\\
    {ɲ} & \textit{ny} & \textit{ny} & \textit{ny} & \textit{ny}\\
    {ŋ} & \textit{ng} & \textit{ng} & \textit{ng} & \textit{ng}\\
    {l} & \textit{l} & \textit{l} & \textit{l} & \textit{l}\\
    {ɹ} & \textit{r} & \textit{r} & \textit{r} & \textit{r}\\
    {ɽ} & \textit{ll} & \textit{--} & \textit{--} & \textit{--}\\
    {ʎ} & \textit{--} & \textit{--} & \textit{ly} & \textit{ly}\\
    {j} & \textit{y, e} & \textit{y} & \textit{y} & \textit{j}\\
    {w} & \textit{w} & \textit{w} & \textit{w} & \textit{w}\\
    {i} & \textit{i} & \textit{i} & \textit{i} & \textit{i}\\
    {u} & \textit{u} & \textit{u} & \textit{u} & \textit{u}\\
    {ɪ} & \textit{ɨ} & \textit{not written} & \textit{é} & \textit{--}\\
    {e} & \textit{e} & \textit{e} & \textit{e} & \textit{e}\\
    {ə} & \textit{ä} & \textit{ɨ} & \textit{not written} & \textit{é}\\
    {o} & \textit{o} & \textit{o} & \textit{o} & \textit{o}\\
    {æ} & \textit{--} & \textit{ä} & \textit{ä} & \textit{ä}\\
    {ɐ} & \textit{a} & \textit{a} & \textit{a} & \textit{a}\\
    \lspbottomrule
\end{xltabular}

The Ende Language Committee also had to make choices regarding word breaks, punctuation, and capitalization. Because many of the Ende Language Committee members only knew how to write in one other language, \ili{English}, many common English conventions were borrowed into Ende writing, including: capitalizing proper nouns, ending sentences with a full stop (.), an exclamation mark (!), or a question mark (?), using opening (``) and closing (") quotation marks for quotations, and placing commas (,) between phrases or at intonation breaks. Hyphens (-) often precede vocative clitics that come at the ends of phrases (\textit{e.g.}, \textit{Wagiba-o!} `Hey, Wagiba!'). Some clitics, such as nominative \textit{da}, are written with a space between the clitic and the phrasal host, though they are analyzed as being a single phonological word. Other clitics, such as attributive \textit{ang} are written without a space, like a suffix. Present tense auxiliary verbs, such as \textit{allan}, are sometimes written with the preceding word, especially if it is a coverb and especially if it ends in a vowel, like \textit{ekallan} `he is speaking', and sometimes as two separate words \textit{eka allan} `he is speaking.' For this reason, present tense auxiliary verbs are analyzed as clitics in the following texts.

\subsection{Transcription choices}
After the texts were written down, the Ende Language Committee\ia{The Ende Language Committee} elected to edit the texts and remove certain discourse features. For example, we modified instances of loanwords or code-switching so that the story was written using what the committee calls \textit{Ende abäl} or `pure Ende.' We omitted other discourse features, such as word repetition or vowel lengthening (except in songs), which often indicate intensity, duration, or emphasis, because the committee did not consider these features appropriate in the written medium. Anyone interested in listening to or reading the transcriptions of the original spoken texts is invited to access these recordings in the Ende language corpus (\cite{Lindsey2015b}).

\subsection{Translation and analysis}
Every text was translated at the phrase-, word-, and morpheme-level into \ili{English}. This process was most often completed by a pair of an English-speaking linguist, such as \KLL{}, \CAS{}, or \LWR{}, and an Ende speaker. Some texts were translated unaccompanied at the phrase-level by a proficient bilingual Ende-English speaker, such as \WKS{} or \TTW{}. Some texts were translated unaccompanied at the word- and morpheme-level by a proficient Ende-speaking linguist, such as \KLL{} or \CAS{}, often with the help of the automated parser included in the Fieldworks Language Explorer (FLEx) software program, which uses an Ende-English lexicon and previously analyzed Ende texts to suggest word- and morpheme-level glosses \citep{Black2006}.


\subsection{Contributor acknowledgement}
The success of this collection is due to the dedication and contributions of many community members, including the Ende Language Committee, local storytellers, and illustrators, and visiting linguists. The committee played a pivotal role in advising the linguists on organizing the submissions, transcribing, translating, and editing the stories for publication. This collection would not have been possible without their expertise and passion for language preservation. Limol and Malam community leaders also supported the recording and collection of stories, ensuring that the narratives accurately reflected local traditions.

To better represent the range of contributors and their roles in the storytelling process, the contributors are introduced below. This group represents a diverse sample of men and women of multiple generations who contributed to the Ende language corpus (\cite{Lindsey2015b}). Of note, \JBD{}, \WGG{}, \WKS{}, and \TTW{} also assisted in the translation of the Book of Mark into Ende \citep{Kurupel2007}. They are joined by \JJD{}, M. Kidarga, \SKS{}, and \JSS{} as some of the most confident writers of Ende in the village.\footnote{Ende people have many names. Their first names and nicknames are given by their parents, godparents, and friends. Their last names are typically the name of their father (biological or adopted), and sometimes the names of their grandfathers. Secondary first or last names are written in parentheses. The names used in this book and bibliography were the preferred names of the speakers in 2018. Names of the deceased are followed by a cross symbol ($\dagger$).}

\subsection{Contributor biographies}
\renewcommand\tabularxcolumn[1]{m{#1}}

\begin{xltabular}{\textwidth}{XX}
   \includegraphics[width=\linewidth]{\imgpath/contributors/image1-1.jpg} &  \JBDfull{}: male, 46 (1969), born in Tenadra, lives in Malam. Speaks Ende, \ili{English}, Motu\il{Austronesian!Motu}, \il{Pahoturi River!Taeme}{Taeme}, \il{Pahoturi River!Idi}{Idi}, \il{Pahoturi River!Kawam}{Kawam}, and Tok Pisin\il{English Creole!Tok Pisin}. Brother of \TDD{}, uncle of J. Kaoga (Dobola). Contributed and translated \textref{text:Tawa}, a story about going hunting with his son. \\
    %testing something & testing something\\
   \includegraphics[width=\linewidth]{\imgpath/contributors/image2}&\WBBfull{}: female, 34 (1983), lives in Daru. Speaks Ende and \ili{English}. Contributed \textref{text:iram}, an ode to the crystal clear bathing waters in Limol village.\\
    \includegraphics[width=\linewidth]{\imgpath/contributors/image3-1} & \JJDfull{}: male, 42 (1975), born and lives in Limol as the village chief. Speaks Ende, \ili{English}, Tok Pisin\il{English Creole!Tok Pisin}, and Motu\il{Austronesian!Motu}. Father of S. Jerry. Contributed \textref{text:Bundae}, a story about a mythical man named \textit{Bundae}, and assisted in translation and morphological analysis for many other texts in the collection.\\
   \includegraphics[width=\linewidth]{\imgpath/contributors/image4} &\TDDfull{}: female, 33 (1984), born in Malam, lives in Daru. Speaks Ende, \ili{English}, and Tok Pisin\il{English Creole!Tok Pisin}. Sister of \JBD{}, aunt of J. Kaoga (Dobola). Contributed and translated \textref{text:Ngaemaene}, a travelling story in which she bravely traps a crocodile.\\
   \includegraphics[width=\linewidth]{\imgpath/contributors/image5-1} & \WGGfull{}: female, 54 (1963), born in Kinkin, lives in Limol, chairwoman of Women's Fellowship. Speaks Ende, \il{Pahoturi River!Taeme}{Taeme}, and \ili{English}, wife of \WKS{}, mother of \RWW{} and \TTW{}, grandmother of I. Kenny. Contributed \textref{text:yu}, assisted in translation, and explained much of the cultural and contextual background to many of the stories in this collection.\\
   \includegraphics[width=\linewidth]{\imgpath/contributors/Sali.jpg}&\name{Sali}{Goge (Wik)}: male, 76 (1940), lives in Malam, leader of the Malam Culture and Dance group (see \figref{fig:dance2}). Speaks Ende, Idi\il{Pahoturi River!Idi}, Motu\il{Austronesian!Motu}, and Tok Pisin\il{English Creole!Tok Pisin}. Wrote, contributed, and performed \textref{song:paradise}, a song about the Bird-of-Paradise, as well as many other songs in the Ende language corpus.\\
   \includegraphics[width=\linewidth]{\imgpath/contributors/samuel and family.JPG}&\name{Samuel}{Jerry}: male, 10 (2007), pictured here with his parents and sister. Speaks Ende and \ili{English}. Lives in Limol and attends Limol Primary School. Son of \JJD{}. Contributed and sung \textref{song:children}, a children's song about \textit{Ause Ur}, an old woman who used to live in the village and was generous with children.\\
   \includegraphics[width=\linewidth]{\imgpath/contributors/Jordan.png}&\name{Jordan}{Kaoga (Dobola)}: male, 11 (2006). Speaks Ende and \ili{English}. Lives in Limol and attends Limol Primary School. Nephew of \JBD{} and \TDD{}. Contributed and sung \textref{song:iramine}, a children's song about dancing at the washing place.\\
    \includegraphics[width=\linewidth]{\imgpath/contributors/Biku.jpg}&\name{Biku (Madura)}{Kangge$^\dagger$}: male, 75 (1941--2023), lived in Limol and was the oldest male of the \textit{Limollang} clan, which owns the Limol land.\is{clan} Spoke Ende and Motu\il{Austronesian!Motu}. Contributed and sung \textref{song:dance}, an encouraging work song to clear the Karama swamp for safe passage of canoes.\\
    \includegraphics[width=\linewidth]{\imgpath/contributors/Sam2.png}&\name{Sam}{Karao}: male, 26 (1991), lives in Limol. Speaks Ende and \ili{English}. Contributed and sung \textref{song:lullaby}, a beautiful lullaby sung while mothers are pounding sago\is{flora!\textit{Metroxylon sagu} (Sago)}. Also illustrated many texts in the Ende language corpus, including the Ende Alphabet Book \citep{Karao2016}. \\
    \includegraphics[width=\linewidth]{\imgpath/contributors/Ibetty2.png}&\name{Ibetty}{Kenny}: female, 10 (2007), lives in Limol and attends Limol Primary School. Speaks Ende and \ili{English}. Granddaughter of \WKS{} and \WGG{}, niece of \RWW{} and \TTW{}. Contributed and sung \textref{song:iramine}, a children's song about dancing at the washing place. Moreover, served the Ende Language Committee tremendously by assisting in cooking, cleaning, and entertaining.\\
    \includegraphics[width=\linewidth]{\imgpath/contributors/image6}&\name{Minong}{Kidarga}: male, 20s (1990s), lives in Limol, grandson of old man Kidarga Nakllae$^\dagger$\ia{Nakllae$^\dagger$, Kidarga} of the Crocodile clan. Contributed \textref{text:Baet}, a story about how Cuscus got his short ears. In this photo, he is working with his grandfather, noting down stories about old ways and the crocodile clan.\\
    \includegraphics[width=\linewidth]{\imgpath/contributors/image7-1} & \DKSfull{}: female, 65 (1952), born and lives in Limol. Speaks Ende and \il{Pahoturi River!Taeme}{Taeme}. Sister of \SKS{} and \WKS{}. Contributed \textref{text:Donae}, in which she kills and carries back a crocodile to share with the village.\\    
    \includegraphics[width=\linewidth]{\imgpath/contributors/image9.jpg}&\SKSfull{}: male, 57 (1960), born and lives in Limol, village recorder. Speaks Ende, \ili{English}, \il{Pahoturi River!Agob}{Agob}, Motu\il{Austronesian!Motu}, and Tok Pisin\il{English Creole!Tok Pisin}. Brother of \DKS{} and \WKS{}, father of \JSS{}. Contributed \textref{text:Iddob}, a story about hunting at night. Assisted with plant and animal identification to provide local context for many of the texts.\\
    \includegraphics[width=\linewidth]{\imgpath/contributors/image10-1} & \WKSfull{}: male, 60 (1957), born and lives in Limol, pastor of the local PNG Evangelical Church and coordinator of the Ende Language Project. Speaks Ende, \il{Pahoturi River!Taeme}{Taeme}, Motu\il{Austronesian!Motu}, \ili{English}, and Tok Pisin\il{English Creole!Tok Pisin}. Husband of \WGG{}, brother of \DKS{} and \SKS{}, father of \RWW{} and \TTW{}, grandfather of I. Kenny. Contributed \textref{text:Auma}, a humorous tale about his father, and \textref{text:karama}, an ode to Karama swamp. Assisted in the editing, translating, and morphological analysis of Texts \ref{text:Baet}, \ref{text:Kottllam}, \ref{text:Auma}, \ref{text:Donae}, \ref{text:Ause}, \ref{text:gamallang}, \ref{text:iram}, and \ref{text:karama}. Illustrated \textref{text:Bundae}.\\
    \includegraphics[width=\linewidth]{\imgpath/contributors/image11-1} &\JSSfull{}: male, 26 (1991), primary school teacher in Upiara. Speaks Ende, \ili{English}, \il{Pahoturi River!Kawam}{Kawam}, \il{Pahoturi River!Taeme}{Taeme}, and Tok Pisin\il{English Creole!Tok Pisin}. Son of \SKS{}. Contributed \textref{text:Ause}, a fantastical story about an old woman and a young boy.\\
    \includegraphics[width=\linewidth]{\imgpath/contributors/julia3.png}&\name{Julia}{Tätän (Delema)}: female, 10 (2007), lives in Limol and attends Limol Primary School. Speaks Ende and \ili{English}. Contributed and sung \textref{song:fishing}, a children's song about a fish who breaks a woman's fishing line.\\
    \includegraphics[width=\linewidth]{\imgpath/contributors/image12} & \RWWfull{}: female, 20s (1990s), born in Limol. Speaks Ende and \ili{English}. Daughter of \WKS{} and \WGG{}, sister of \TTW{}, aunt of I. Kenny. Contributed \textref{text:gamallang}, a hilarious tale about two friends: Lobster and Frog.\\
    \includegraphics[width=\linewidth]{\imgpath/contributors/image13-1} &\TTWfull{}: male, 33 (1984), born in Limol, vice-chairperson for the Ende Language Project. Speaks Ende, \ili{English}, Tok Pisin\il{English Creole!Tok Pisin}, and \il{Pahoturi River!Taeme}{Taeme}. Son of \WKS{} and \WGG{}, brother of \RWW{}, uncle of I. Kenny. Contributed \textref{text:Kottllam}, the Turtle's origin story. Assisted in the editing, translating, and morphological analysis of Texts \ref{text:Baet}, \ref{text:Auma}, \ref{text:Donae}, \ref{text:Ause}, \ref{text:Bundae}, \ref{text:yu}, and \ref{text:gamallang}.\\
\end{xltabular}

%In the next section, we discuss the orthography used for the Ende language, which was applied to the texts presented in this collection.

\section{Text presentation}
\largerpage[2]
The texts in this collection are presented in two forms: parallel running text and interlinear glossing. The parallel text aligns the Ende text on the left with an \il{English!English (American)}{English} translation on the right. The Ende text is written using the Ende \isi{orthography} (see \sectref{orthography}) while the translation in English uses standard \il{English!English (American)}{American English} spelling conventions.

Following the parallel text, each story is presented in an interlinear gloss format, see (\ref{ex3}) from \textref{text:Ngaemaene}. In this format, each sentence is presented in the standard orthography on the first line in italics {\textcircled{\raisebox{-.9pt} {1}}}. Then, each word is annotated for morphological structure on the second line {\textcircled{\raisebox{-.9pt} {2}}}. Here, hyphens (-) represent affix boundaries, equals signs (=) represent clitic boundaries, and tildes (\textasciitilde) represent reduplication boundaries.\footnote{Affixes attach to roots or stems, while clitics attach to phrases. Both exhibit properties of being part of the phonological word, such as \isi{vowel harmony}, but many speakers prefer to write some clitics as separate words on the orthographic line. Reduplication boundaries mark instances of phonological duplication and morphological doubling \citep{Inkelas2008}.} 

\ea
\raisebox{.5pt}{\textcircled{\raisebox{-.9pt} {1}}} Däbe dirom de Mareyas kapu dägagän.\\
\gll \raisebox{.5pt}{\textcircled{\raisebox{-.9pt} {2}}} däbe	dirom=de	Mareyas	kapu	d-ä-gag-än\\
     \raisebox{.5pt}{\textcircled{\raisebox{-.9pt} {3}}} that	cassowary=\textsc{acc}	\textsc{pn}	carry	\textsc{rem}-3\textsc{ndu}P-\textsc{aux}-\textsc{rem}.3\textsc{sg}A\\
\glt \raisebox{.5pt}{\textcircled{\raisebox{-.9pt} {4}}} `Mareyas carried that cassowary [to Dum river source].'
\Corpus{WE\_SN024:5}{Dobola2016e}\label{ex3}
\z

On the third line {\textcircled{\raisebox{-.9pt} {3}}}, glosses of each morphological element are aligned with the word above. This gloss is written using the Leipzig glossing rules (\cite{Comrie2008}) and represents the intended meaning of each morpheme for each use, but it does not necessarily represent all possible meanings of the morphological element. For example, a verb suffix that can be used to mean remote past or future will only be glossed as \textsc{rem} or \textsc{fut} depending on the context. The abbreviations used for the glosses are listed on page \pageref{chap:abbreviations}. The conventions used for the abbreviations are based on typological and areal trends (\textit{e.g.}, \citealt{Doehler2018}).

Finally, the translation of the sentence into English is presented in the fourth line in single quotes {\textcircled{\raisebox{-.9pt} {4}}}. This translation is typically a free translation, where the intended meaning of the Ende phrase is translated into a semantically equivalent English phrase in collaboration with Ende-speaking consultants and an English-speaking linguist. Sometimes, a literal translation of the words will follow the free translation in parentheses. Parts of the meaning that are retrievable from context or are assumed but not explicitly stated in the example are added to the translation in square brackets.

Examples (\ref{ex2})--(\ref{ex3}) are followed by an in-line citation that references the source text in the bibliography, the text ID, and the line number within the text.

\section{Special information}
The stories in this collection are organized into five parts. Part \ref{part:animal}, Animal Tales and Origin Stories, includes two \textit{pourquoi} stories, which are etiological narratives formulated for children that discuss how certain things in nature came to be, such as how the turtle got its patterned shell. These origin narratives differ from \textit{mabun eka}, sacred clan origin stories, which are more serious, passed down within a clan, and cannot be shared openly.\is{clan}

Parts \ref{part:hunting} and \ref{part:heroic} include two first-hand narratives and one folk/remembered story each. First-hand narratives describe personal experiences, while folk/remembered stories are traditional tales passed down through generations and involve known people, places, or clan groups, often with magical or supernatural elements.

Part \ref{part:hunting}, Tales of Hunting\is{hunting} and Survival, includes three stories that describe the real and perceived dangers of leaving the village's safety alone. These stories emphasize the importance of vigilance in your surroundings and how the community protects one another.

Part \ref{part:heroic} is called Heroic or Legendary Stories and includes two chance occurrences with crocodiles, some of the largest predators in the region, and one fantastical encounter with a flood.

Part \ref{part:misbehave} is called Tales of Misbehavior and Consequence and contains three parables that serve as instructions for how to behave in the Ende community.

Lastly, Part \ref{part:odes}, Odes and Reflections on the Natural World, brings together two odes as an example of more artistic narratives. Both are dedicated to important local water places: \textit{Karama} swamp and \textit{Iräm} washing place.

Within each part, each story is also organized into several sections. Section 1 of each story describes the origination of the text. Section 2 provides a summary and contextual background for the story. Section 3 provides the running Ende text and English translation in parallel, with the original illustrations, if available. Finally, Section 4 provides the text with interlinear glossing.

Just as storytelling sessions are interwoven with playfulness and song, the stories in this collection also alternate with selected songs from the rich inventory of sung texts in the Ende language corpus (\cite{Lindsey2015b}). The song texts have three sections: Section 1 describes the origins and provides a summary of the song, Section 2 provides the lyrics in Ende, and Section 3 shows the interlinear glossing. All songs may be downloaded for listening from the Ende language corpus archive \citep{Lindsey2015b}.

\section{Overview of the texts}

\tabref{tab:texts} provides an overview of the texts included in this collection.

\begin{xltabular}{\textwidth}{lXXr}
    \caption{The texts and songs in this collection.} \label{tab:texts}\\

    \lsptoprule
        \textsc{text} & \textsc{title} & \textsc{translation} & \textsc{words}\\
    \midrule
    \endfirsthead

    \multicolumn{4}{c}%
    {\tablename\ \thetable{} -- continued from previous page}\\
    \lsptoprule
        \textsc{text} & \textsc{title} & \textsc{translation} & \textsc{words}\\

	\midrule
    \endhead

    \hline \multicolumn{4}{r}{{Continued on next page}}\\
    \endfoot

    
    \endlastfoot

    %\ref{chap02}&Clan history&&\textbf{2576}\\
	\textref{text:Baet}&\textit{Baet bo llan a allame de tubutubu gogon}&How Cuscus Got His Short Ears&686\\
    \textref{text:Kottllam}&\textit{Kottllam bo pallall ttoenttoen}&Turtle's Story&350\\
   \textref{song:paradise}&\textit{Kakayam bo tongoe}&{The Bird of Paradise Song}&{81}\\
    \textref{text:Tawa}&\textit{Tawa mamoeatt ttoen}&Swamp Hunting Story&404\\
    \textref{text:Iddob}&\textit{Iddob käbama ibiatt}&Night Hunting&365\\
    \textref{text:Auma}&\textit{Auma we ibiatt ttoen}&Walking to the Grave Story&595\\
    \textref{text:Donae}&\textit{Donae Kurupel bäne käza gäzatt ttoen}&Donae Kurupel's Crocodile Killing Story&522\\
     \textref{song:crocodile}&\textit{Käza misima saima}&The Crocodile Song&18\\
    \textref{text:Ngaemaene}&\textit{Ngämaene ibiatt ttoenttoen}&Our Traveling Story&376\\
    {\textref{song:fishing}}&{\textit{Ttongo mälla da}}&Children’s song about fishing&{42}\\
    \textref{text:Ause}&\textit{Ause da llɨg kälsre peyang}&The Old Woman and the Small Boy&192\\
    {\textref{song:lullaby}}&{\textit{Bandra bebi bälle}}&{Sago Lullaby}&32\\
    \textref{text:Bundae}&\textit{Bundae bo pepeb}&The Tale of Bundae&275\\
    {\textref{song:children}}&{\textit{Ause Ur}}&{Children’s song about \textit{Ause Ur}}&{54}\\
    \textref{text:yu}&\textit{Yu ingong}&Fire Dance&473\\
    {\textref{song:dance}}&{\textit{Ngasinga wutamu}}&{Swamp Work Song}&{135}\\
    \textref{text:gamallang}&\textit{Gämällang komlla}&A Tale of Two Thieves&463\\
    \textref{text:karama}&\textit{Eramang Karama walle}&Ode to Karama Swamp&119\\
    \textref{text:iram}&\textit{Iräm ine, Iräm ine}&Ode to Iräm Waters&139\\
    {\textref{song:iramine}}&{\textit{Ttongo toto me yäbäd a goklanän}}&{Children’s song about \textit{Iräm ine}}&{41}\\
    \midrule    
        \textbf{Total}&&&\textbf{5362}\\
    \lspbottomrule
\end{xltabular}

\section{Contextual images}

In this section, I have included many photographs taken during our visits to Limol to provide examples of places, animals, traditions, structures, and tools mentioned in the stories. The Figures are referenced in the texts as they are mentioned.

\subsection{Flora}

\begin{multicols}{2}
    \begin{Contextfigure}
    \captionsetup{width=\linewidth,hypcap=false}
    \includegraphics[width=\linewidth]{\imgpath/contextual_images/sago.jpg}
    \captionof{figure}{\textit{Sana pätt} - Sago tree}
    \label{fig:sago}\is{flora!\textit{Metroxylon sagu} (Sago)}
\end{Contextfigure}
\begin{Contextfigure}
    \captionsetup{width=\linewidth,hypcap=false}
        \includegraphics[width=\linewidth]{\imgpath/contextual_images/image39.jpg}\captionof{figure}{\textit{Yu bäng} ‘fire stick’}\label{fig:firestick}\is{flora!\textit{Banksia dentata} (Fire stick)}
    \end{Contextfigure}
    \begin{Contextfigure}
    \captionsetup{width=\linewidth,hypcap=false}
        \includegraphics[width=\linewidth]{\imgpath/contextual_images/grace-kate.jpg}\captionof{figure}{\GMMfull{} and \KLLfull{} planting a \textit{nge pätt} `coconut tree'\is{flora!\textit{Cocos nucifera} (Coconut)} in honor of the opening of the Limol Health Center in 2018}\label{fig:grace-kate}
    \end{Contextfigure}
    \begin{Contextfigure}
    \captionsetup{width=\linewidth,hypcap=false}
        \includegraphics[width=\linewidth]{\imgpath/contextual_images/banana-tree.jpg}\captionof{figure}{\textit{Up pätt} `banana tree'}\label{fig:banana}\is{flora!\textit{Musa spp.} (Banana)}
    \end{Contextfigure}
    \begin{Contextfigure}
    \captionsetup{width=\linewidth,hypcap=false}
        \includegraphics[width=\linewidth]{\imgpath/contextual_images/image41.jpg}\captionof{figure}{\textit{Karama} swamp}\label{fig:karama1}
    \end{Contextfigure}

    \begin{Contextfigure}
    \captionsetup{width=\linewidth,hypcap=false}
        \includegraphics[width=\linewidth]{\imgpath/contextual_images/image42.jpg}\captionof{figure}{\textit{Karama} swamp}\label{fig:karama2}
    \end{Contextfigure}

    \begin{Contextfigure}
    \captionsetup{width=\linewidth,hypcap=false}
        \includegraphics[width=\linewidth]{\imgpath/contextual_images/image43.jpg}\captionof{figure}{\textit{Karama} swamp}\label{fig:karama3}
    \end{Contextfigure}

    \begin{Contextfigure}
    \captionsetup{width=\linewidth,hypcap=false}
        \includegraphics[width=\linewidth]{\imgpath/contextual_images/image44.jpg}\captionof{figure}{\textit{Karama} swamp}\label{fig:karama4}
    \end{Contextfigure}

    \begin{Contextfigure}
    \captionsetup{width=\linewidth,hypcap=false}
        \includegraphics[width=\linewidth]{\imgpath/contextual_images/image45.jpg}\captionof{figure}{\textit{Karama} swamp}\label{fig:karama5}
    \end{Contextfigure}
\end{multicols}

\subsection{Fauna}

\begin{multicols}{2}
    \begin{Contextfigure}
    \captionsetup{width=\linewidth,hypcap=false}
        \includegraphics[width=\linewidth]{\imgpath/contextual_images/image17.jpg}\captionof{figure}{\textit{Baet} - Common Spotted Cuscus (\textit{Spilocuscus maculatus})\is{fauna!\textit{Spilocuscus maculatus} (Common Spotted Cuscus)}}\label{fig:cuscus}
    \end{Contextfigure}
    \begin{Contextfigure}
    \captionsetup{width=\linewidth,hypcap=false}
        \includegraphics[width=\linewidth]{\imgpath/contextual_images/image19.jpg}\captionof{figure}{\textit{Däräng} - Hunting\is{hunting} dogs (\textit{Canis lupus hallstromi}\is{fauna!\textit{Canis lupus hallstromi} (New Guinea singing dog)} and \textit{Canis lupus dingo})\is{fauna!\textit{Canis lupus dingo} (Australian dingo)}}\label{fig:huntingdogs}
    \end{Contextfigure}
    \begin{Contextfigure}
    \captionsetup{width=\linewidth,hypcap=false}
        \includegraphics[width=\linewidth]{\imgpath/contextual_images/atata kottllam.jpg}
        \captionof{figure}{\textit{Atata kottllam} - Northern snake-necked turtle (\textit{Chelodina rugosa})\is{fauna!\textit{Chelodina rugosa} (Northern snake-necked turtle)}}
        \label{fig:atata-turtle}
    \end{Contextfigure}
    \begin{Contextfigure}
    \captionsetup{width=\linewidth,hypcap=false}
    \includegraphics[width=\linewidth]{\imgpath/contextual_images/image23.jpg}
    \captionof{figure}{\textit{Gamo kottllam} - Pig-nosed turtle (\textit{Carettochelys insculpta})\is{fauna!\textit{Carettochelys insculpta} (Pig-nosed turtle)}}
    \label{fig:gamo-turtle}
    \end{Contextfigure}
    \begin{Contextfigure}
    \captionsetup{width=\linewidth,hypcap=false}
    \includegraphics[width=\linewidth]{\imgpath/contextual_images/pall kottllam.jpg}
    \captionof{figure}{\textit{Pall kottllam} - New Guinea painted turtle (\textit{Emydura subglobosa})\is{fauna!\textit{Emydura subglobosa} (New Guinea painted turtle)}}
    \label{fig:pall-turtle}
\end{Contextfigure}

\begin{Contextfigure}
    \captionsetup{width=\linewidth,hypcap=false}
    \includegraphics[width=\linewidth]{\imgpath/contextual_images/uwo kottllam.jpg}
    \captionof{figure}{\textit{Uwo kottllam} - New Guinea snapping turtle (\textit{Elseya branderhorsti})\is{fauna!\textit{Elseya branderhorsti} (New Guinea snapping turtle)}}
    \label{fig:uwo-turtle}
\end{Contextfigure}
\begin{Contextfigure}
    \captionsetup{width=\linewidth,hypcap=false}
        \includegraphics[width=\linewidth]{\imgpath/contextual_images/image26.jpg}\captionof{figure}{\textit{Ddia} - Rusa deer (\textit{Rusa timorensis})\is{fauna!\textit{Rusa timorensis} (Rusa deer)}}\label{fig:deer}
    \end{Contextfigure}

    \begin{Contextfigure}
    \captionsetup{width=\linewidth,hypcap=false}
        \includegraphics[width=\linewidth]{\imgpath/contextual_images/bush-wallaby.jpg}\captionof{figure}{\textit{Kubull} `bush wallaby'}\label{fig:wallaby}
    \end{Contextfigure}
    \begin{Contextfigure}
    \captionsetup{width=\linewidth,hypcap=false}
        \includegraphics[width=\linewidth]{\imgpath/contextual_images/hunting-deer.jpg}\captionof{figure}{A group of hunters\is{hunting} gathering around a slain deer}\label{fig:hunting-deer}
    \end{Contextfigure}
    \begin{Contextfigure}
    \captionsetup{width=\linewidth,hypcap=false}
        \includegraphics[width=\linewidth]{\imgpath/contextual_images/image20.jpg}\captionof{figure}{\textit{Käza} - Hall's New Guinea crocodile (\textit{Crocodylus halli})\is{fauna!\textit{Crocodylus halli} (Hall's New Guinea Crocodile)}}\label{fig:crocodile}
    \end{Contextfigure}
    \begin{Contextfigure}
    \captionsetup{width=\linewidth,hypcap=false}
        \includegraphics[width=\linewidth]{\imgpath/contextual_images/image14.jpg}\captionof{figure}{\textit{Welwele} - Coroneted fruit dove (\textit{Ptilinopus coronulatus})\is{fauna!\textit{Ptilinopus coronulatus} (Coroneted fruit dove)}}\label{fig:bird-1}
    \end{Contextfigure}

    \begin{Contextfigure}
    \captionsetup{width=\linewidth,hypcap=false}
        \includegraphics[width=\linewidth]{\imgpath/contextual_images/bird-2.jpg}\captionof{figure}{\textit{Kär pipiem} - Purple-tailed imperial pigeon
 (\textit{Ducula rufigaster})\is{fauna!\textit{Ducula rufigaster} (Purple-tailed imperial pigeon)}}\label{fig:bird-2}
    \end{Contextfigure}

    \begin{Contextfigure}
    \captionsetup{width=\linewidth,hypcap=false}
        \includegraphics[width=\linewidth]{\imgpath/contextual_images/bird-3.jpg}\captionof{figure}{\textit{Kättekätte} - Red-cheeked parrot
 (\textit{Geoffroyus geoffroyi aruensis})\is{fauna!\textit{Geoffroyus geoffroyi aruensis} (Red-cheeked parrot)}}\label{fig:bird-3}
    \end{Contextfigure}

    \begin{Contextfigure}
    \captionsetup{width=\linewidth,hypcap=false}
        \includegraphics[width=\linewidth]{\imgpath/contextual_images/bird-4.jpg}\captionof{figure}{\textit{Mise} - Common cicadabird
 (\textit{Edolisoma tenuirostre})\is{fauna!\textit{Edolisoma tenuirostre} (Common cicadabird)}}\label{fig:bird-4}
    \end{Contextfigure}

    \begin{Contextfigure}
    \captionsetup{width=\linewidth,hypcap=false}
        \includegraphics[width=\linewidth]{\imgpath/contextual_images/bird-5.jpg}\captionof{figure}{\textit{Ngallngall} - Catbird
 (\textit{Aliruoedus maculosus})\is{fauna!\textit{Aliruoedus maculosus} (Catbird)}}\label{fig:bird-5}
    \end{Contextfigure}

    \begin{Contextfigure}
    \captionsetup{width=\linewidth,hypcap=false}
        \includegraphics[width=\linewidth]{\imgpath/contextual_images/bird-6.jpg}\captionof{figure}{\textit{Giwe} - Fruit dove
 (\textit{Ptilinopus coronulatus})\is{fauna!\textit{Ptilinopus coronulatus} (Fruit dove)}}\label{fig:bird-6}
    \end{Contextfigure}

    \begin{Contextfigure}
    \captionsetup{width=\linewidth,hypcap=false}
        \includegraphics[width=\linewidth]{\imgpath/contextual_images/bird-7.jpg}\captionof{figure}{\textit{Yal} - Yellow-billed kingfisher
 (\textit{Syma torotoro})\is{fauna!\textit{Syma torotoro} (Yellow-billed kingfisher)}}\label{fig:bird-7}
    \end{Contextfigure}

    \begin{Contextfigure}
    \captionsetup{width=\linewidth,hypcap=false}
        \includegraphics[width=\linewidth]{\imgpath/contextual_images/bird-8.jpg}\captionof{figure}{\textit{Mok} - Friarbird
 (\textit{Philemon corniculatus})\is{fauna!\textit{Philemon corniculatus} (Friarbird)}}\label{fig:bird-8}
    \end{Contextfigure}

    \begin{Contextfigure}
    \captionsetup{width=\linewidth,hypcap=false}
        \includegraphics[width=\linewidth]{\imgpath/contextual_images/bird-9.jpg}\captionof{figure}{\textit{Tarambobo} - Hooded butcherbird
 (\textit{Cracticus cassicus})}\label{fig:bird-9}
    \end{Contextfigure}
    \begin{Contextfigure}
    \captionsetup{width=\linewidth,hypcap=false}
        \includegraphics[width=\linewidth]{\imgpath/contextual_images/image18.jpg}\captionof{figure}{\textit{Pollgo} - White-lipped tree frog (\textit{Litoria infrafrenata})\is{fauna!\textit{Litoria infrafrenata} (White-lipped tree frog)}}\label{fig:frog}
    \end{Contextfigure}
\end{multicols}

\subsection{Material culture}

\begin{multicols}{2}
    \begin{Contextfigure}
    \captionsetup{width=\linewidth,hypcap=false}
        \includegraphics[width=\linewidth]{\imgpath/contextual_images/image32.jpg}\captionof{figure}{\textit{Mama} - palm leaf enclosure}\label{fig:mamahouse}
    \end{Contextfigure}
    \begin{Contextfigure}
    \captionsetup{width=\linewidth,hypcap=false}
        \includegraphics[width=\linewidth]{\imgpath/contextual_images/house.jpg}\captionof{figure}{\textit{Ma} - typical stilt house in Limol}\label{fig:house}
    \end{Contextfigure}
    \begin{Contextfigure}
    \captionsetup{width=\linewidth,hypcap=false}
        \includegraphics[width=\linewidth]{\imgpath/contextual_images/image36.jpg}\captionof{figure}{\textit{Bägäl a wa tobäll a} - Bow and arrows}\label{fig:bow-spears}
    \end{Contextfigure}

    \begin{Contextfigure}
    \captionsetup{width=\linewidth,hypcap=false}
        \includegraphics[width=\linewidth]{\imgpath/contextual_images/image38.jpg}\captionof{figure}{\name{Matthew}{Warama} with longbow and arrow}\label{fig:matthew}
    \end{Contextfigure}

\begin{Contextfigure}
    \captionsetup{width=\linewidth,hypcap=false}
        \includegraphics[width=\linewidth]{\imgpath/contextual_images/Spears.jpg}\captionof{figure}{\textit{Tobäll sapasapang} - Types of arrows, drawn by \AKDfull{}}\label{fig:spears}
    \end{Contextfigure}

\begin{Contextfigure}
    \captionsetup{width=\linewidth,hypcap=false}
        \includegraphics[width=\linewidth]{\imgpath/contextual_images/canoe.jpg}\captionof{figure}{An illustration of a \textit{gall} `canoe' for the letter G in the Ende Alphabet Book (\cite{Karao2016})}\label{fig:canoe}
    \end{Contextfigure}
    \begin{Contextfigure}
    \captionsetup{width=\linewidth,hypcap=false}
        \includegraphics[width=\linewidth]{\imgpath/contextual_images/grave.jpg}\captionof{figure}{\textit{Auma} ‘grave’ - the grave site of twin baby girls Grace and Kate}\label{fig:grave}
    \end{Contextfigure}
    \begin{Contextfigure}
    \captionsetup{width=\linewidth,hypcap=false}
        \includegraphics[width=\linewidth]{\imgpath/contextual_images/image40.jpg}\captionof{figure}{\WGGfull{} fishing\is{fishing} with a long net}\label{fig:net-casting}
    \end{Contextfigure}
    
    \begin{Contextfigure}
    \captionsetup{width=\linewidth,hypcap=false}
        \includegraphics[width=\linewidth]{\imgpath/contextual_images/fishing-net.jpg}\captionof{figure}{\textit{Gull} `fishing net'}\label{fig:fishing-net}
    \end{Contextfigure}
    \begin{Contextfigure}
    \captionsetup{width=\linewidth,hypcap=false}
        \includegraphics[width=\linewidth]{\imgpath/contextual_images/Donae-carrying.jpg}\captionof{figure}{\DKSfull{} carrying a load of \textit{yu} `firewood'}\label{fig:donae-carrying}
    \end{Contextfigure}

\begin{Contextfigure}
    \captionsetup{width=\linewidth,hypcap=false}
        \includegraphics[width=\linewidth]{\imgpath/contextual_images/baby-basket2.png}\captionof{figure}{A woman carrying a small child on her shoulders and an infant in a \textit{ddäma} `baby basket,' suspended from her forehead}\label{fig:baby-basket2}
    \end{Contextfigure}

   \begin{Contextfigure}
    \captionsetup{width=\linewidth,hypcap=false}
        \includegraphics[width=\linewidth]{\imgpath/contextual_images/squeezing sago.JPG}\captionof{figure}{A woman squeezing \textit{sana} `sago'}\label{fig:squeezing-sago}\is{flora!\textit{Metroxylon sagu} (Sago)}
    \end{Contextfigure} 
    \begin{Contextfigure}
    \captionsetup{width=\linewidth,hypcap=false}
        \includegraphics[width=\linewidth]{\imgpath/contextual_images/image33.jpg}\captionof{figure}{\name{Sali}{Goge (Wik)} in traditional dance attire}\label{fig:dance1}
    \end{Contextfigure}

    \begin{Contextfigure}
    \captionsetup{width=\linewidth,hypcap=false}
        \includegraphics[width=\linewidth]{\imgpath/contextual_images/image34.jpg}\captionof{figure}{The Malam Culture and Dance group}\label{fig:dance2}
    \end{Contextfigure}

\begin{Contextfigure}
    \captionsetup{width=\linewidth,hypcap=false}
        \includegraphics[width=\linewidth]{\imgpath/contextual_images/image35.jpg}\captionof{figure}{Warani Pewe$^\dagger$ (center) with an \textit{alläp} `Kundu drum'}\label{fig:dance3}
    \end{Contextfigure}

\begin{Contextfigure}
    \captionsetup{width=\linewidth,hypcap=false}
        \includegraphics[width=\linewidth]{\imgpath/contextual_images/limol dance group.JPG}\captionof{figure}{The Limol Culture and Dance group}\label{fig:limol-dance}
    \end{Contextfigure}
    \begin{Contextfigure}
    \captionsetup{width=\linewidth,hypcap=false}
        \includegraphics[width=\linewidth]{\imgpath/contextual_images/image29.jpg}\captionof{figure}{\textit{Polle} - Communal garden fence}\label{fig:fence}
    \end{Contextfigure}

    \begin{Contextfigure}
    \captionsetup{width=\linewidth,hypcap=false}
        \includegraphics[width=\linewidth]{\imgpath/contextual_images/image30.jpg}\captionof{figure}{\textit{Pollepolle} - Personal garden fence}\label{fig:personal-fence}
    \end{Contextfigure}
    \begin{Contextfigure}
    \captionsetup{width=\linewidth,hypcap=false}
        \includegraphics[width=\linewidth]{\imgpath/contextual_images/image47.jpg}\captionof{figure}{Meeting friends at the \textit{Iräm ine} washing place}\label{fig:iram2}
    \end{Contextfigure}
    \begin{Contextfigure}
    \captionsetup{width=\linewidth,hypcap=false}
        \includegraphics[width=\linewidth]{\imgpath/contextual_images/image46.jpg}\captionof{figure}{Washing clothes at the \textit{Iräm ine} washing place}\label{fig:iram1}
    \end{Contextfigure}


\end{multicols}
