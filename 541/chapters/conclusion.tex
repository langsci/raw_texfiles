During my most recent visit to Limol village, I was repeatedly asked what people in Australia and America thought of their stories and the Ende way of life. For many Ende people, working with a linguist to write down their oral practices was not just an exercise in linguistic analysis, but an opportunity to say to the outside world, “Hello! We are here and this is who we are." I think this act is significant because many Ende people feel isolated from the rest of the world and believe that if more people knew about them, they might want to visit and help raise their quality of life. 

In this way, the Ende Language Committee's effort to share their culture and literature with us can be considered an attempt to enter into a partnership via exchange, one of the primary means of relationship building used in broader Melanesia \citep[308]{Dobrin2008}. I discussed this theme in a podcast called ``Sweet Potato Love" after my first visit to Limol in 2015 \citep{Lindsey2016a}. This podcast compiled more than 50 interviews with Ende women discussing their views on life, community, and culture. On the topic of love and relationships, the importance of material exchange was paramount. \textit{Love}, as I came to understand it, was giving someone a sweet potato and receiving a fish in return. As \citet[309]{Dobrin2008} describes in her interactions with the Arapesh community in the Sepik coastal region of Papua New Guinea, exchanging goods with outsiders is even more empowering because it demonstrates a society's capacity for influence and authority on their cultural terms. Thus, this offering is a vulnerable one as there is an expectation, but no precedent, for a response in kind.

As the primary steward of the Ende language corpus outside of Papua New Guinea, I have endeavored to share information about the Ende people and culture through talks, books, papers, social media, and now in this collection of Ende texts. In return, I provided the community with material goods that are hard to get in Limol, which is a long list of almost anything you can think of (salt, fishing hooks, buckets, medicine, books, nails, eyeglasses, machetes, basketballs, shoes, mosquito nets...), and messages from abroad. 

I hope they know how proud I am of everyone who contributed to this work as storytellers, illustrators, editors, translators, or compilers. Each unique perspective helped bring life to these pages. 
%I wish I had the financial flexibility, the bravery, and the selflessness to engage in this international relationship more regularly.

Thematically, this collection reveals some fundamental aspects about the Ende way of life. Through harrowing tales about danger, hunger, drought, and conflict, we also witness accounts of bravery, resourcefulness, and kindness. Values of friendship, generosity, and mutual respect weave through the stories, just as they do in lived experiences, as Ende people navigate their relationships with the natural world, one another, and the mythical realm.

While storytelling has long been an oral practice, the Ende Language Committee's efforts to self-publish illustrated storybooks locally and to publish collections like this internationally safeguard the community's cultural legacy and history. I hope this collection inspires readers to appreciate the richness of the Ende language and culture and recognize the importance of supporting language strengthening efforts worldwide.