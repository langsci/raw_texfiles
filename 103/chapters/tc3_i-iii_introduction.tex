%\documentclass[output=paper]{LSP/langsci} 
\addchap{Preface to the new edition}
%\author{Oliver Čulo
% \and Silvia Hansen-Schirra\affiliation{University of %Mainz, Germersheim} 
% \lastand Stella Neumann\affiliation{RWTH Aachen 
%University}
%}
%\title{Preface to the new edition} 
%\abstract{Goodbye TC3, welcome TMNLP! A welcome address from the previous TC3 editors.}
%\maketitle

%\begin{refsection}

%\section{Introduction} 

\lehead{Oliver Čulo, Silvia Hansen-Schirra, Stella Neumann}

This is the first of three volumes which are made up of previously published volumes of the open-access journal ``Translation: Computation, Corpora, Cognition'' (TC3). Digitalisation has had an immense impact on the way we share our knowledge, including on the way how researchers publish their work. TC3 was one of the very first endeavours to make open-access online publication viable in Translation Studies. 

OpenAccess is still being met with quite some scepticism, but we, the former editors of TC3, Silvia Hansen-Schirra, Stella Neumann and Oliver Čulo, believe that the open access to knowledge is the right way to publish scientific results: Research, both for the community and the society at large, often funded by the public and consequently made accessible to the public. The acceptability of research findings is in part determined by how the community as well as the public is informed about this research (both its aims and its achievements). 

It can, however, not be taken for granted that the results of up-to-date research are easily and freely accessible to the community or a lay audience. Another problem is to keep pace with the speed of progress in the sciences and an increasing specialization, which widen the gap between the current state of research and the accessibility to published findings. As a counter model to traditional publishing, OpenAccess straightforwardly offers a solution to this problem providing free and online access to cutting-edge, innovative research.

Did we know what we were doing when we started? Well, partially so. OpenAccess has had a lot of positive effects on the availability of results and the impact of researchers' work, but in its current, often community-based form, it also poses a challenge for researchers who engage in organising an OpenAccess journal or book series: It is they who are responsible not only for the quality of the contents (which should not and we believe will not diminish in OpenAccess), but for much of the or even the whole appearance, including the design of the publication and the quality of the type setting.

After three years with a special issue every year, the journal TC3 was transformed into the book series now called ``Translation and Multilingual Natural Language Processing'' (TMNLP) under the roof of LangSci Press. This move reflects in some sense the currently fast-changing publication landscape in both sciences and humanities. Becoming a book series at LangSci has resulted in a boost of the quality of the published volumes. Also, a stringent proofreading process has helped ensure higher consistency within and across the contributions.

The idea to re-publish the TC3 volumes as TMNLP volumes came up very early, with two goals in mind:
\begin{itemize}
\item making the works contributed to TC3 available in the long run, beyond just by archiving them somewhere;
\item honouring the work which was put into the contributions by re-publishing them under higher quality standards.
\end{itemize}

The three volumes 3, 4 and 5 are thus not mere re-prints, but the contributions were re-edited according to LangSci guidelines and quality standards. Each volume is introduced by a dedicated introduction from the original volumes. The TC3 contributions are still available in their original format for documentary purposes under \url{http://www.t-c3.org} at the time of publication of the corresponding TMNLP volumes. Nevertheless, we believe that re-publication within LangSci will ensure enhanced impact and long-time availability, and on top of that it is a further step into the new world of open-access publishing for Translation Studies.

\bigskip
\hfill Germersheim and Aachen, January 2017 \\
\bigskip
\hfill Oliver Čulo, Silvia Hansen-Schirra, Stella Neumann

\printbibliography[heading=subbibliography,notkeyword=this]

%\end{refsection}