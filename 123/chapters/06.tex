







\chapter{Attributive constructions in the Christian dialect of Qaraqosh} \label{ch:Qaraqosh}

\renewcommand{\defaultDialect}{\Qar}

The data of the Qaraqosh dialect is based on \citet{KhanQaraqosh}. Note that the town name Qaraqosh is referred to as \transc{Baġdedə} in the dialect itself, as will be apparent in some of the examples below.\footnote{The examples are cited referring to the page in the grammar in which they are treated. Examples which are part of the texts collected by Khan are furthermore indicated by a  textual reference, using Khan's notation system in square brackets. The single letters refer to informants' free speech, while the labels \textit{Proverbs}, \textit{Play}, \textit{Poetry} and \textit{Gospel} refer to recordings of a collection of proverbs, a theatre play, poetry recitation, and a Gospel translation. Note that the recordings of informant K (including the proverbs) are publicly available (see link under \cite{KhanQaraqosh} in the Bibliography).}

 Compared to \JZax, \Qar presents a more conservative system of ACs, as we shall see below. With respect to the AC system, the dialect is quite similar to the neighbouring dialect of Alqosh (=\Alq), described by \citet{CoghillAlqosh}, but still somewhat more conservative. 

The main ACs in \Qar are the  suffixed CSC (see \sref{ss:Qar_CSC}) and the ALC (see \sref{ss:Qar_Lnk}). Due to the frequent \isi{resyllabification} of the \phonemic{d} segment, however, it is not always easy to distinguish between the two. This fuzzy situation may be related to the conservative nature of the dialect. Moreover, in contrast to \JZax, there is no regular \gen* \d marking, although some cases may resemble it (see \example{576}). In Gospel translations one finds also the DAC (see \sref{ss:Qar_DAC}). 

The chapter discusses also some other constructions found in \Qar: the use of the possessive pronominal suffixes, in which \Qar has some particularities, is discussed in the next section. Double marking due to hesitation is discussed in \sref{ss:Qar_double_1}. Juxtaposition constructions, on the periphery of the AC system of \Qar, are discussed in \sref{ss:Qar_juxt}.

\section{Possessive pronominal suffixes (X-y.\poss)}
\label{ss:Qar_Poss}

\Qar, like all \ili{NENA} dialects, has a series of pronominal possessive suffixes, that can attach freely to nouns. The possessive pronouns replace the last vowel of the noun, which  corresponds to the \isi{free state} and number marking. As a consequence, when the number distinction is expressed solely by this vowel, it is lost. Such is the case of the noun \foreign{tora}{ox} with the \pl* form \transc{torə}:

\acex{Noun}{Pronoun}{466}
{tór-əḥ}
{ox(en)-\poss.3\masc{}}
{his ox(en)}
{KhanQaraqosh}{76}

\Qar, however, exhibits a special feature, in that the \pl* possessive pronominal suffixes  retain the number distinction of the noun to which they attach: these suffixes transform a \pl* /-ə/ suffix to an /-e/ suffix, instead of suppressing it. As \citet[77]{KhanQaraqosh} notes, this is an archaism of the dialect, which conserves the reflex of an original \transc{*ay}:

\acex{Noun}{Pronoun}{467}
{tór-\zero-hən}
{ox(en)-\sg-\poss.3\pl}
{their ox}
{KhanQaraqosh}{77}

\acex{Noun}{Pronoun}{468}
{tor-é-hən}
{ox(en)-\pl-\poss.3\pl}
{their oxen}
{KhanQaraqosh}{77}

The \isi{possessive suffix}  attaches strictly to the \prim noun. A modifying adjective appears after this complex (contrast with \example{519}):

\acex{Noun Phrase}{Pronoun}{554}
{sús-əḥ kóma}
{horse-\poss.3\masc{} black}
{his black horse}
{KhanQaraqosh}{280}	

An interesting phenomenon particular to this dialect is the insertion of an \transc{-ətt} suffix (glossed below as \fem) before the \isi{possessive pronoun} in some feminine nouns, such as \foreign{ʾarnúwa}{rabbit}:

\acex{Noun}{Pronoun}{469}
{ʾarnuw-ə́tt-əḥ}
{rabbit-\fem-\poss.3\masc}
{his rabbit}
{KhanQaraqosh}{204}

As most of the nouns which behave in this way are of \Arab origin, \citet[206]{KhanQaraqosh} relates this phenomenon to the retention of the \ili{Arabic} \transc{tāʾ marbūṭa}, which is an \transc{-ət} suffix appearing in \cst* \fem* nouns.\footnote{\citet[204]{KhanQaraqosh} notes that the final \phonemic{a} of \ili{Arabic} \isi{loanwords}, which corresponds to \transc{tāʾ marbūṭa} (in its \free* form), is often pronounced as \phonetic[ə].}  In this account, the gemination of the \phonemic{t} segment may be explained by a merger with the Aramaic \fem* suffix \transc{-ta}, yielding \transc{-ət} + \transc{-ta} = \transc{-ətta}, noting that the final \transc{-a} vowel is dropped before the possessive suffixes. The gemination could also be explained on phonological grounds, as a mean to conserve the short \phonetic[ə] vowel in a closed syllable. Either way, in contrast to the \Arab \transc{tāʾ marbūṭa}, the \Qar \transc{-ətt} is not a generalized \fem* \cst* marker, as it  appears only  before possessive pronouns, and not before full nominal \secns. 

Another possibility which Khan raises is that the \transc{-ətt} segment may be related to the \lnk* \transc{did}, akin to the \ili{NWNA} \concept{heavy possessive suffixes} which contain the \lnk* \d,\footnote{See \vref{ft:Midn_ayd} for an example.} 
but this seems less plausible due to the restricted distribution of this suffix with feminine nouns only. 

Infinitives, as well as particles, take the same possessive suffixes as nouns. These suffixes mark then the complements of the verbal lexeme (see also \sref{ss:Qar_CSC_inf}):

\acex{Infinitive}{Pronoun}{568}
{(xálṣa) lyàš-əḥˈ}
{finish.3\fem{} knead.\inf-\poss.3\masc}
{(She finishes) kneading it.}
{KhanQaraqosh}{364 {[B:132]}}

\acex{Participle}{Pronoun}{573}
{k-ína šqíl-əḥ Maṣlàyəˈ}
{\ind-\cop.3\pl{} taken.\resl-\poss.3\masc{} M.\_inhabitants}
{The people of Mosul have taken it.}
{KhanQaraqosh}{363 {[S:49]}}

Some prepositions may also take the possessive pronominal suffixes:

\acex{Preposition}{Pronoun}{651}
{txíθ-əḥ}
{under-\poss.3\masc{}}
{under it}
{KhanQaraqosh}{80}

 Other prepositions, which cannot take this suffix, have a suppletive form which appears only with the possessive suffixes. Such is the case of \foreign{ʾeka\~gib-}{at} or \foreign{da\~ġdal-}{for}:

\acex{Preposition}{Pronoun}{578}
{gíb-an}
{at-\poss.1\pl}
{at our place}
{KhanQaraqosh}{234 {[B:138]}}

\acex{Preposition}{Pronoun}{633}
{ġdál-əḥ}
{for-\poss.3\masc}
{for him}
{KhanQaraqosh}{233}

As for the plural possessive suffixes, some prepositions take the plain variant, while others take the variant used after {\pl*} \prims, probably due to their nominal origin.

\acex{Preposition}{Pronoun}{656}
{txəθ-hən}
{under-\poss.3\pl}
{under them}
{KhanQaraqosh}{80}

\acex{Preposition}{Pronoun}{657}
{baθr-e-hən}
{after-\enquote*{\pl}-\poss.3\pl}
{after them}
{KhanQaraqosh}{80}

\section{The construct state construction (X.\textsc{cst} Y)} \label{ss:Qar_CSC}


The marking of the \prim by the \cst* suffix \ed  is the most common type of AC in \Qar, but its identification is not always easy, due to phonological considerations. Indeed, very often the \cst* suffix syllabifies with the subsequent \secn, rendering it similar to the \lnk* construction, discussed in \sref{ss:Qar_Lnk}. At the same time, the historical apocopate \cst* is only retained in a handful of expressions, discussed below.

\subsection{The historical construct state marking} \label{ss:Qar_hist_cst}

The historical \cst* marking, characterized synchronically by an \isi{apocope} of the \prim noun  (minimally the removal of the \free*  suffix),
is found only in a handful of \enquote{closely knit-phrases} \citep[209]{KhanQaraqosh}, i.e.\ proper nouns or fixed expressions (compounds), with either opaque semantics (see \example{697}) or transparent semantics (see \example{487}). Additionally, one finds the \prim \foreign{bi}{house.\cst} in the meaning of \transl{family/house of} used productively with a referential \secn (see \example{504}).\footnote{For a discussion of the various types of compounds in \ili{NENA}, see \citet{GutmanCompounds}.} Note that in example \ref{ex:487} the word \foreign{bắxət}{wife.\cst} is formed from \foreign{baxta}{wife.\free} by the removal of the \free*  suffix \transc{-a} and the insertion of an \isi{epenthetic} \transc{ə}.

\acex{Noun}{Noun (opaque semantics)}{697}
{{bi}\cb{} guba; {bi}\cb{} yəlda; bar\cb{} zarʾa}
{house.\cst\cb{} hole house.\cst\cb{} birth; son.\cst\cb{} field}
{tunnel', `Christmas', `seed}
{KhanQaraqosh}{209}

\acex{Noun}{Noun (transparent semantics)}{487}
{bắxət bába; syam\cb{} íḏa}
{wife.\cst{} father laying.\cst\cb{} hand}
{step-mother, ordination}
{KhanQaraqosh}{209}

\acex{Noun}{Noun (productive expressions)}{504}
{{bí} xə́θna; {bi}\cb{} Šə̀rḥa; {bi}\cb{} ʿàmm-i}
{house.\cst{} groom house.\cst\cb{} S. house.\cst\cb{} uncle-\poss.1\sg }
{family of the groom, of S., of my paternal uncle}
{KhanQaraqosh}{209 {[S:93]}}

In this respect, \Qar differs from \JZax, in which the apocopate \cst* is entirely productive (see \example{317} among others). 

\subsection{The suffixed construct state formation}

The productive formation of the \cst* is made with the help of the \mbox{\transc{-(ə)d}} suffix, which originates in the \isi{encliticization} of the \lnk* \d.\footnote{\citeauthor{KhanQaraqosh} treats the \cst* marker and the \lnk* as two manifestations of the same \isi{annexation} particle \transc{d}; see \cref{ch:synchrony} for similar opinions of other scholars. For the development of the \cst* suffix see discussion in \sref{ss:neo-CSC}.} The suffix replaces the final vowel of the \prim (if it ends in a vowel), leading often to a neutralisation of the number distinction of the \prim noun:




\acex{Noun}{Noun}{531}
{kθáw-əd qášɑ}
{book-\cst{} priest}
{the book of the priest}
{KhanQaraqosh}{276}

In some cases, especially after liquid consonants following a vowel, the \phonetic[ə] segment is not present, alluding  to its \isi{epenthetic} status. 

\acex{Noun}{Noun}{484}
{gúr-d máθa}
{men-\cst{} town}
{the men of the town}
{KhanQaraqosh}{208 {[F:96]}}

Some specific nouns seem to combine a historical apocopate \cst* form with an \ed suffix. Such is the case of the nouns \foreign{ʾəbra}{son} and \foreign{brata}{daughter}: 



\acex{Noun}{Noun}{474}
{bə́rd axón-i}
{son.\cst{} brother-\poss.1\sg{}}
{the son of my brother}
{KhanQaraqosh}{207}

\acex{Noun}{Noun}{475}
{bə́rtəd ʿàmma}
{daughter.\cst{} paternal\_uncle}
{the daughter of a paternal uncle}
{KhanQaraqosh}{207 {[S:40]}}

As \citet[208]{KhanQaraqosh} mentions, in some cases the \phonemic{d} segment is phonetically syllabified with the \secn. This happens predominantly when the \secn starts with a vowel (often preceded by an \isi{epenthetic} \isi{glottal stop}) or a consonant cluster, in which case an \isi{epenthetic} \phonetic[ə] is added after the \phonemic{d}. These cases are still in principle differentiable from the \lnk* \d (treated in \sref{ss:Qar_Lnk}), thanks to the replacement of the final (\free*) vowel of the \prim (typically an \transc{-a} for singular nouns) by an \phonetic[ə] (glossed in such cases as \isi{schwa}), or its complete elision in some cases (typically when the last consonant of the \prim is a liquid).\footnote{The elision of a final \isi{schwa} in a CVCə CV environment seems to be a regular phonological process in \Qar and the neighbouring \Alq dialect \citep[cf.][73]{CoghillAlqosh}. Since only a \isi{schwa} is thus elided (and not the vowel \phonetic[a]), the lack of a final vowel is a clear indication of an underlying \ed suffix. On the other hand, when a \isi{schwa} is present,  it is often difficult in normally paced speech to tell it apart from an \phonemic{-a} suffix, which is often realised as \phonetic[æ]. When the \prim noun ends in the \free* with an \phonetic[ə] (such as some plural nouns or infinitives) it is impossible to tell the difference between the two constructions in normal speech.  Indeed, \citet[298]{CoghillAlqosh}, describing the neighbouring \Alq dialect, notices that it is often impossible to tell whether the \phonemic{d} segment is associated with the \prim or the \secn, as in \foreign{nāšə-d-jéš}{men of the army}. }


\acex{Noun}{Noun}{483}
{yál -d\cb{} axòna}
{child(ren) -\cst\cb{} brother}
{children of the brother}
{KhanQaraqosh}{208 {[F:3]}}

\acex{Noun}{Noun}{480}
{ʾít-ə -də\cb{} Šmòni\footnotemark}
{church-\\isi{schwa}{} -\cst\cb{} S.}
{the church of Shmoni}
{KhanQaraqosh}{208 {[K:21]}}

\footnotetext{This is the text as it appears in the corpus. In the grammar, it is cited erroneously as \transc{ʾitə d-Ašmoni}.}

The syllabification of the \cst* suffix with a vowel-initial \secn is not automatic however, as \example{474} and the following example show (contrast with \example{483}):

\acex{Noun}{Noun}{698}
{báb-əd ʾaxón-əd ʾə́mm-i}
{father-\cst{} brother-\cst{} mother-\poss.1\sg{}}
{the father of the brother of my mother}
{KhanQaraqosh}{276 {[B:25]}}

Conversely, the suffix may syllabify with the \secn even when the above mentioned phonological conditions are not fulfilled:

\acex{Noun}{Noun}{477}
{kθayáθ-ə -də\cb{} Baġdèdə}
{chickens-\\isi{schwa}{} -\cst\cb{} B.}
{the chickens of Qaraqosh.}
{KhanQaraqosh}{208 {[B:105]}}


In some cases, the \phonemic{d} segment is assimilated to the following segment. In Khan's transcription the assimilated segment is written as a \isi{proclitic} of the \secn:







\acex{Noun}{Noun}{703}
{ʾilan-ə -z\cb{} záʾθa}
{tree-\\isi{schwa}{} -\cst\cb{} olive}
{olive tree}
{KhanQaraqosh}{208 {[K:56]}}


In cases where the resulting geminate is de-geminated, the only indicator of the \cst* is the lack of the final \free* vowel on the \prim, and its replacement by an \phonetic[ə] (in which cases it fully assumes the role of \cst* marking, and is thus glossed \cst):


\acex{Noun}{Noun}{486}
{ʾár-ə Baġdèdə}
{land-\cst{} B.}
{the land of Qaraqosh}
{KhanQaraqosh}{208 {[S:48]}}











When the \prim consists of conjoined nouns, one \cst* suffix is sufficient for the entire phrase, as in the following example (=\example{535bis}; compare to the other examples there):\footnote{Note that  \transc{wanat} is the  \pl* form of \foreign{ʾuwana, wana}{female sheep} and not an apocopate form \citep[727]{KhanQaraqosh}.}



\acex{Conjoined nouns}{Noun}{535}
{[wánat\cb{} u toráθ]-əd Baġdèdə}
{sheep\cb{} and cows-\cst{} B.}
{the sheep and cows of Qaraqosh}
{KhanQaraqosh}{276 {[F:1]}}










\subsection{Adverbial \prims}

Recall that the term \concept{adverbial} is used here as a cover term for both conjunctions and prepositions, as these often assume an \isi{adverbial} function (see \vref{ft:adverbial}). However, the CSC of \Qar admits only prepositions (taking nominal complements) as \prims: 


\acex{Preposition}{Noun}{515}
{ríš-əd kàlθa}
{on-\cst{} bride}
{on the bride}
{KhanQaraqosh}{239 {[K:41]}}

\acex{Preposition}{Noun Phrase}{649}
{txíθ-əd [də́nw-əd ḥaywā̀n]}
{under-\cst{} tail-\cst{} animal}
{under the tail of the animal}
{KhanQaraqosh}{240 {[B:75]}}

In some cases, only the \phonemic{-ə} segment remains of the \cst* suffix (compare with \example{486}):

\acex{Preposition}{Noun}{517}
{txíθ-ə sùdyaˈ ʾu\cb{} txíθ-ə làḥma}
	{under-\cst{} warp and\cb{} under-\cst{} weft}
{under the warp and under the weft}
{KhanQaraqosh}{240 {[B:22]}}

\acex{Preposition}{Noun}{632}
{báθr-ə báb-əḥ}
{behind-\cst{} father-\poss.3\masc}
{behind his father}
{KhanQaraqosh}{233}

Conversely, when the \secn is vowel-initial the \phonemic{d} segment tends to syllabify with the \secn (compare with \example{483}). In such cases, the \phonemic{ə} segment tends to fall.

\acex{Preposition}{Noun}{650}
{txiθ -d\cb{} àql-əḥ}
{under -\cst\cb{} foot-\poss.3\masc}
{under his foot}
{KhanQaraqosh}{240 {[F:38]}}

\newpage 
The same analysis holds for the prepositions \foreign{mən}{from} and \foreign{max}{like}, though their \cst* forms appear only before vowel-initial \secns:\footnote{The simple forms of the preposition (without a \phonemic{d} segment) are used preceding consonant-initial \secns, giving rise to the \isi{juxtaposition} construction (see \sref{ss:Qar_jux_adv} and examples \ref{ex:640} and \ref{ex:514} there). This restriction does not hold in other dialects, where the \cst* form \transc{mənn-əd} is found preceding such \secns, including \Nrt (see \example{1913}), \JUrm (see \example{191}) and \Boh, as illustrated by the following example:

\acexfn[\Boh]
{Preposition}{Noun}{1494}
{mənn-ət karačüke}
{from-\cst{} gypsies}
{from the Gypsies}
{FoxBohtan}{99}

} 

\acex{Preposition}{Noun}{512}
{mən -d\cb{} áḏa gdíša}
{from -\cst\cb{} \dem.\near.\masc{} pile}
{from this pile}
{KhanQaraqosh}{238 {[B:94]}}

\acex{Preposition}{Adverb}{513}
{mən -d\cb{} àxa}
{from -\cst\cb{} here}
{from here}
{KhanQaraqosh}{238 {[F:11]}}


\acex{Preposition}{Noun}{639}
{max -d\cb{} aḏa gora}
{like \cst\cb{} \dem.\near.\masc{} man}
{like this man}
{KhanQaraqosh}{238}


As the above examples show, in many of these cases the vowel-initial element opening the \secn is a \isi{demonstrative pronoun}. The attachment of the \phonemic{d} segment to it may be the first step in the path of its reanalysis as a \gen* \isi{case marker} conditioned morphologically by the presence of a \dem* (see \sref{ss:genitive_development}). This reanalysis has not taken place in \Qar (or at least, not completely), as the attachment of the  \phonemic{d} segment to the \secn is still conditioned by a vowel-initial phonological environment (as \example{650} shows), rather than being morphologically conditioned by the presence of certain demonstratives.  Moreover, as \example{637} shows, the \phonemic{d} segment does not appear consistently before demonstrative pronouns where a \isi{genitive marking} would be expected. See, however, \example{576} for a case where positing a \gen* prefix seems to be the best analytical possibility.


 
\subsection{Adjectival \prims}

There are two distinct types of cases in which adjectives appear in the \prim slot of the \cst* construction. In the first case, akin to the \concept{impure annexation} in \ili{Arabic}, the adjectival lexeme is modified syntactically by a \secn noun (which semantically is qualified by the adjective). The resulting AC is an adjectival phrase which modifies another noun (compare with the ALC in \example{558}):\footnote{Compare this with the \CArab example \foreign{ħasanu l-waǧh-i}{beautiful.\cst{} \definite-face-\gen} \citep[277]{GoldenbergSemitic}. See  \textcites[204ff.]{GoldenbergAdjectivization} for a discussion.}

\acex{Adjective}{Noun}{557}
{(góra) xwár-əd kósa}
{man white-\cst{} hair}
{a white-haired (man)}
{KhanQaraqosh}{281}

The second case represents an \concept{emotive genitive}, in which the noun posing as a \secn is in fact the semantic head, and the use of the adjective as a \prim (and subsequently as a syntactic head) adds emotional value to the phrase (compare with the \JZax \example{337}  and see \ref{ft:emotive_genitive} there.) 

\acex{Adjective}{Noun}{559}
{mḥúsy-əd xəmyàn-iˈ}
{absolved.\masc-\cst{} father\_in\_law-\poss.1\sg}
{my late father-in-law}
{KhanQaraqosh}{281 {[Play 13]}}

\acex{Adjective}{Noun}{560}
{b\cb{} áḏa həjím-əd màʿraḏ̣ˈ}
{in\cb{} \dem.\near.\masc{} collapsed-\cst{} showroom}
{in this accursed showroom}
{KhanQaraqosh}{281 {[Play 107]}}

\subsection{The \prim \foreign{nafs}{the same}} \label{ss:Qar_nafs}

An interesting  example of borrowing a \cst* construction together with 
its \prim (a matter-cum-\isi{pattern replication} in the sense of \cite{SakelTypes}), is the borrowing of the \ili{Arabic} function word \transc{\textarabic{نفس} nafs}. This word, originally meaning \transl{soul}, has been grammaticalised in \Arab into a reflexive pronoun, and as head of the CSC into a determiner meaning \transl{the same}:

\arabex{\textit{nafs}}{noun}{Arab_nafs}
{نفس الاكل}
{nafs-u l-ʾakl-i}
{soul-\nom{} \defi-food-\gen}
{the same food}
{}

In \Qar, the reflexive pronoun is \transc{roxa}, a native Aramaic word meaning \transl{soul} \citep[see][84]{KhanQaraqosh}.\footnote{This need not be a \isi{pattern replication}, as this meaning is a common source for reflexive pronouns cross-linguistically. A single example of \isi{matter replication} of this sense is attested in \Qar as  \foreign{nafə̀ssə}{themselves} \citep[739]{KhanQaraqosh}.} The morpheme \transc{nafs} has been borrowed, however, in its determinative function with the meaning \transc{the same}. Moreover, its syntactic position as the head of a CSC is replicated, albeit being marked with \ili{NENA} morphology, namely the \cst* \ed suffix.

\acex{\textit{nafs}}{Noun}{522}
{náfs-əd ʾəxàla}
{same-\cst{} food}
{the same food}
{KhanQaraqosh}{642 {[F:77]}}

\subsection{Clausal \secns}

The suffixal CSC can also introduce a clausal \secn:

\acex{Noun}{Clause}{500}
{sókə-d k-maθéhə}
{sprigs-\cst{} \ind-bring.3\pl}
{the sprigs that they bring}
{KhanQaraqosh}{209 {[K:56]}}

\acex{Noun}{Clause}{588}
{(nā́dir xáz-ət) béθ-əd lé-bə tawə̀rta.ˈ}
{rarely see-2\masc{} house-\cst{} \neg.\exist-in.3\masc{} cow}
{(You rarely see) a house that doesn't have a cow in it.}
{KhanQaraqosh}{477 (4) {[B:100]}}

\newpage 
In the following examples, the \cst* suffix syllabifies with the clause, which starts either with a vowel-initial \isi{copula}, or a consonant cluster (contrast with \example{500}):

\acex{Noun}{Clause}{502}
{maθwáθ -d\cb{} ina\cb{} xə́ḏran Baġdèdə}
{villages -\cst\cb{} \cop.\pl\cb{} around B.}
{the villages that are around Qaraqosh}
{KhanQaraqosh}{209 {[F:22]}}


\acex{Pronoun}{Clause}{605}
{kúll mán də\cb{} g-nápəl b\cb{} idàθ-θəˈ}
{all who -\cst\cb{} \ind-fall in\cb{} hands-\poss.3\pl}
{anybody who falls into their hands}
{KhanQaraqosh}{480 (4) {[Play 135]}}

Note that in the last example the pronoun \foreign{mani}{who} loses its final \phonemic{i} vowel in presence of the \cst* suffix. 

\subsection{Infinitives in the construct state construction} \label{ss:Qar_CSC_inf}

Infinitives of transitive verbs can appear as \prims of the CSC, having their objective complement as the \secn. Infinitives of intransitive verbs are not found in this position, in contrast to \JZax, where  infinitives can take their verbal subject as \secns (see \examples{328}{329}).

\acex{Infinitive}{Noun (object)}{541}
{(ʾəm-mólpi susawáθa) štáy-əd ʿəràqˈ u\cb{}  xál-əd bšála d\cb{}  nàšə.ˈ}
{\fut-teach.3\pl{} horses drink.\inf-\cst{} araq and\cb{}  eat.\inf-\cst{} cooked\_food \lnk\cb{}  people}
{(They would teach the horses) to drink arak and eat the food of people.}
{KhanQaraqosh}{369 (1) {[F:66]}}

\acex{Infinitive}{Pronoun (object)}{529}
{qṭál -də\cb{} ġḏáḏə}
{fight.\inf.\cst{} -\cst\cb{} each\_other}
{fighting each other}
{KhanQaraqosh}{275 (21)}

\acex{Infinitive}{Noun (object)}{532}
{(dúkθ-əd) ḥfáḏ̣-əd làxma}
{place-\cst{} store.\inf-\cst{} bread}
{(the place of) the storage of the bread}
{KhanQaraqosh}{276 {[B:15]}}

As the last example shows, an infinitive can also appear as a \secn of a CSC. In this case, it is akin to a regular noun, as the  following example shows:

\acex{Noun}{Infinitive}{479}
{yóm -də\cb{} gwàra}
{day.\cst{} -\cst\cb{} marry.\inf}
{wedding day}
{KhanQaraqosh}{208 {[K:40]}}







\section{The analytic linker construction (X \textsc{lnk} Y)} \label{ss:Qar_Lnk}
\subsection{Introduction}
\Qar has retained the usage of the \lnk* as an important element of its AC system. However, due to the phonological considerations explained above regarding the \cst* suffix, it is not always clear whether a specific occurrence of a \isi{proclitic} \d should be analysed as a \lnk* or rather as re-syllabified \cst* suffix. As a general rule, if a \free* suffix can be identified on the \prim (\sg: \transc{-a}; \pl: \transc{-ə}), I assume a \isi{proclitic} \d is indeed the \lnk*. Note, however, that the \pl* \free*  suffix is ambiguous since an \phonetic[ə] is also part of the \cst* suffix. 

In most cases, the \lnk* can be analysed as being pronominal, i.e.\ representing a noun. In some cases, however, this analysis is not tenable, as we shall see below. 

\largerpage
In Khan's  transcription, the \lnk* can occur in a variety of phonological shapes, be it \transc{d-}, \transc{də-}, \transc{ʾəd-}. In many cases, this variation can be explained by the realisation of an \isi{epenthetic} \phonetic[ə] which breaks up a consonant cluster \citep[cf.][64--65]{KhanQaraqosh}. A truly different morphemic shape is found in the rare form \transc{dəd-}.\footnote{Khan analyses the form \transc{dəd-} as a repetition of the \enquote{\isi{annexation} particle} \d, and classifies its occurrence together with \examples{494}{495}.}

\acex{Noun}{Noun}{470}
{bšála d\cb{} nàšə}
{cooked\_food \lnk\cb{} people}
{the food of people}
{KhanQaraqosh}{207 {[F:66]}}\antipar 
\newpage 

\acex{Noun}{Noun}{471}
{xəlxálə də\cb{} sə̀hma}
{bangles \lnk\cb{} silver}
{bangles of silver}
{KhanQaraqosh}{207 {[Poetry 29]}} 

\acex{Noun}{Noun}{497}
{zònaˈ dəd\cb{} roxáθa xəškanəˈ}
{time \lnk\cb{} souls dark.\pl}
{the time of dark souls}
{KhanQaraqosh}{209 {[Poetry 8]}} 

In principle the \lnk* may assimilate to the first consonant of the \secn. One such example may be the following:

\acex
{Noun}{Noun}{1921}
{ʾén-ə n\cb{} náša}
{eye-\free.\pl{} \lnk\cb{} man}
{the eyes of a man}
{KhanQaraqosh}{208 {[B:127]}}

This analysis relies on the possible identification of  the suffix \transc{-ə} as a \pl* \isi{free state} ending. Yet as this kind of assimilation normally happens in \enquote{fast speech} \citep[208]{KhanQaraqosh}, it is rather more probable that such examples should be understood as cases of the CSC, with a re-syllabified suffix \ed, similarly to \example{703} .

\largerpage
When the \secn is a pronoun realised as a \isi{pronominal suffix}, the \lnk* takes the form \transc{did-\~dəd-}.\footnote{The vowels \phonetic[i] and \phonetic[ə] are in allophonic \isi{complementary distribution}, depending on the syllable structure. The latter is used in closed syllables.}
These forms are mostly used with those \ili{Arabic} \isi{loanwords} which cannot be inflected directly with the possessive pronominal suffixes (for which see \sref{ss:Qar_Poss}):\footnote{From a typological point of view, the \lnk* qualifies as being a \concept{possessive noun} in the sense of \citet{BickelNicholsWals58}, which is an \enquote{abstract or generic noun [that] is put in \isi{apposition} to the semantically possessed non-possessible [\prim noun]}. Thus, one could add \Qar to the very short list of languages having exactly one {possessive noun}. It would moreover be the first identified {possessive noun} in Eurasia. \label{ft:possessive_noun} }



\acex{Noun}{Pronoun}{523}
{tarkī́z dìd-əḥˈ}
{concentration \lnk-\poss.3\masc{}}
{his concentration}
{KhanQaraqosh}{271 {[F:47]}}\antipar

\newpage 

\acex{Noun}{Pronoun}{524}
{ʾaṭfāl\cb{}  dìd-əḥ}
{children\cb{}  \lnk-\poss.3\masc}
{his children}
{KhanQaraqosh}{271 {[B:22]}}

In one case, the \lnk* is doubled before a \isi{possessive suffix}. This may be due to an ad-hoc re-analysis of the \lnk+\poss\ construction as a genitive pronoun, which is compatible with a preceding \lnk* (see general discussion in \sref{ss:pron_base}):

\acex{Noun}{Pronoun}{521}
{šúrṭa də\cb{}  də̀d-xuˈ}
{police \lnk\cb{}  \lnk-\poss.2\pl{}}
{your police}
{KhanQaraqosh}{271 {[F:77]}}


As expected from a \textit{pronominal} \lnk*, the \lnk* phrase has a certain prosodic and syntactic autonomy vis-à-vis the \prim. The \prim and \secn  can be separated prosodically  (see \example{497}) and  by intervening material: 

\acex{Noun}{Noun}{546}
{(ʾu\cb{}  ʾə́t-lə) bàlṭat k-ămí-hə,ˈ ʾəd\cb{}  xàšabˈ}
{and\cb{}  \exist-3\masc{} knives \ind-call.\agent3\pl-\patient3\masc{} \lnk\cb{}  wood}
{It has \enquote{knives} -- so they call it -- of wood.}
{KhanQaraqosh}{278 (14) {[K:28]}}

Of interest is also the possibility to negate the \lnk*, thus denying the existence of the specified \prim. Note also in this case the repetition of the \lnk* before each conjoint \secn: 

\acex{Negated noun}{Conjoined nouns}{540}
{(lá pəšlí\cb{} b-ə) qàtta,ˈ lá d\cb{} xəṭṭíθa\cb{} w lá d\cb{} sʾə̀rta.ˈ}
{\neg{} remain\cb{} in-3\masc{} stick \neg{} \lnk\cb{} wheat\cb{} and \neg{} \lnk\cb{} barley}
{No stick remains in it, neither of wheat nor of barley.}
{KhanQaraqosh}{277 (4) {[B:62]}}

In the following poetic example, two \lnk* phrases are topicalised before their \prims. For clarity, the \secns and \prims are marked with corresponding subscripts in the glosses, while the translation follows the normal \ili{English} word-order:

\acex{Noun}{Noun}{544}
{də\cb{} rəxmúθa k-óḏa ʾìḏa,ˈ màθ-an,ˈ ʾu\cb{} də\cb{} šláma k-péša yawə̀ntaˈ}
{\lnk\cb{} love\textsubscript{Y} \ind-do festival\textsubscript{X} village-\poss.1\pl{} and\cb{} \lnk\cb{} peace\textsubscript{Y} \ind-become dove\textsubscript{X}}
{It makes a festival of love, our town, and becomes a dove of peace}
{KhanQaraqosh}{278 (11) {[Poetry 18]}}

Finally, whenever the \prim consists of a noun qualified by an adjective, the ALC is regularly used. This can be contrasted with the situation in \JZax, in which the CSC is available in such cases; see \example{338}.





\acex{Noun Phrase}{Noun}{542}
{[qála ṣə́pya] də\cb{} zamìraˈ}
{sound pure \lnk\cb{} pipe}
{with the pure sound of the pipe}
{KhanQaraqosh}{278 (9) {[Poetry 2]}}

\acex{Noun Phrase}{Pronoun}{519}
{[béθa rába] díd-əḥ}
{house big \lnk-\poss.3\masc}
{his big house}
{KhanQaraqosh}{271}

With a pronominal \secn, an alternative formulation is possible, with the adjective following the \isi{possessive suffix}; see \example{554}.\footnote{In \CArd one finds a similar alternative construction, making use additionally of the \lnk*:

\acexfn[\CArd]
{Noun Phrase}{Adjective}{1793}
{qalp-e dīy-e aw zȧra}
{hull-\poss.3\masc{} \lnk-\poss.3\masc{} \defi{} yellow}
{Its yellow hull}
{KrotkoffAradhin}{22 {[19]}}


}

\subsection{Verbal nouns as \prims}

Recall that verbal nouns are nouns which can form a verbal construction, typically infinitives and \isi{participles}. 

As the following examples show, infinitives as well as active \isi{participles} can be complemented by their object by means of the ALC:

\acex{Infinitive}{Noun (object)}{567}
{(ʾu\cb{} k-óḏi) palóʾə d\cb{} əxàlaˈ}
{and\cb{} \ind-do.3\pl{} share.\inf{} \lnk\cb{} food}
{They make a division of food.\footnotemark}
{KhanQaraqosh}{369 (10) {[K:10]}} 

\footnotetext{This example appears as such in the transcribed corpus, but in the recording the speaker is saying \foreign{muqāsam-əd əxala}{division.\cst{} food}, using the \Arab loanword \transc{muqāsama}. Indeed, the apparent usage in the transcription of an Aramaic infinitive as a complement of a light verb is quite odd.}

\acex{Participle}{Noun (object)}{574}
{mšadr-an-íθa də\cb{} šlìḥəˈ}
{send-\ptcp-\fem{} \lnk\cb{} apostles}
{the sender of the apostles}
{KhanQaraqosh}{368 (1) {[Play 111]}}


\subsection{Clausal \secns}
\label{ss:Qar_Clausal_Attr}

The ALC is also compatible with clausal \secns:


\acex{Noun}{Clause}{652}
{léša də\cb{} k-óḏ-ax áxni làxma mə́nn-əḥ}
{dough \lnk\cb{} \ind-do-1\pl{} 1\pl{} bread from-3\masc}
{dough from which we make bread}
{KhanQaraqosh}{209 {[K:22]}}

\acex{Noun}{Clause}{585}
{ʾəb\cb{} máya ʾəd\cb{} iyéwa sə̀ryəˈ}
	{in\cb{} water \lnk\cb{} \cop.\pst{} dirty}
{in water that was dirty}
{KhanQaraqosh}{476 (11) {[K:12]}}

As these examples show, the \lnk* is often followed directly by a verb or a \isi{copula}.

In case of conjoined attributive clauses, the \lnk*  can be repeated:

\acex{Noun}{Conjoined Clauses}{582}
{súsə d\cb{} àxəlˈ gúrgur\cb{} u maràqa \cb{}wˈ d\cb{} sátə ʿəràqˈ}
{horse \lnk\cb{} eat burghul\cb{} and soup \cb{}and \lnk\cb{} drink araq}
{a horse that eats burghul and soup and drinks arak}
{KhanQaraqosh}{475 (8) {[F:79]}}
	
As we have seen above in the context of nominal \secns, the \lnk* phrase can be separated from its \prim, in line with the pronominal nature of the \lnk*.



\acex{Noun}{Clause}{584}
{ʾu\cb{} nášə makìx\cb{} iyewa d\cb{} g-ʿéši bgáw-aḥˈ}
{and\cb{} people simple\cb{} \cop.\pst{} \lnk\cb{} ind-live.3\pl{} in-3\fem}
{and the people were simple, those who lived in it}
{KhanQaraqosh}{475}

In some cases a pronoun, either interrogative or demonstrative, is inserted between the \prim noun and the \lnk*. This pronoun should be understood as standing in \isi{apposition} to both elements.

\acex{Noun Phrase}{Clause}{611}
{fa\cb{}  ʾáwwal mə́ndi ma\cb{}  də\cb{}  k-òḏiˈ}
	{and\cb{}  first thing what\cb{}  \lnk\cb{}  \ind-do.3\pl}
{the first thing they do}
{KhanQaraqosh}{482 (21) {[S:52]}}

\acex{Noun}{Clause}{612}
{há dúkθa ʾéka d\cb{}  íwa də́rya gàw-aḥ.ˈ}
{behold place where \lnk\cb{}  \cop.\pst{} laid.\resl{} in-3\fem}
{Behold the place where he was laid.}
{KhanQaraqosh}{482 (22) {[Gospel 25 = Mark 16:6]}}

When a \isi{demonstrative pronoun} is used, it usually follows a prosodic break:

\acex{Noun}{Clause}{572}
{ʾu\cb{}  ʾə́tlan biráθa Baġdédə,ˈ ʾán də\cb{}  k-maʾtəmdì-waˈ ġdedàyəˈ}
{and\cb{}  \exist-1\pl{} wells B. \dem.\pl{} \lnk\cb{}  \ind-depend.3\pl-\pst{} B.\_inhabitants}
{And we have wells in Qaraqosh, upon which the inhabitants depended.}
{KhanQaraqosh}{369 (5) {[K:87]}}


As in \Syr (see \sref{ss:syr_ALC_clausal}), the particle \d is used also as a \isi{complementizer}. While such a usage may resemble formally an AC without a \prim (see \sref{ss:Qar_ALC_noPrim}), the syntactic function of this \d is different, as it introduces a clausal complement of a verb (but see \vref{ft:Wertheimer_complement}).

\acex{Verb}{Clause}{626}
{ʾáxni g-báʾax d\cb{}  šákxax l\cb{}  àneˈ}
{1\pl{} \ind-want.1\pl{} \comp\cb{}  complain.1\pl{} about\cb{}  \dem.\far.\pl}
{We want to complain concerning those people.}
{KhanQaraqosh}{505 (1) {[F:73]}}

As a \isi{complementizer}, \d can also introduce an \isi{adverbial} purpose clause.

\acex{Verb}{Clause}{625a}
{daré-wa-l tàma,ˈ  ʾəd\cb{}  yàwəšˈ}
{put.\agent3\pl-\pst-\patient3\masc{} there \comp\cb{}  dry.3\masc}
{They used to put it there to dry.}
{KhanQaraqosh}{494 (8) {[K:52]}}


\subsection{Adverbial \prims}





The \d morpheme is used following adverbials serving as conjunctions, i.e.\ complemented by a clause. Superficially, these constructions may be assimilated to the ALC with clausal \secns discussed above. Yet as the \d in such cases cannot be said to represent pronominally the conjunction, I prefer to analyse the \d morpheme in this position as a \isi{complementizer}, as in \examples{626}{625a}.



\acex{Conjunction}{Clause}{592}
{hál də\cb{} mṭíhə lə́\cb{} ġḏa dúka (ʿamùqt\cb{} ela)ˈ}
{until \comp\cb{} arrived.3\pl{} to\cb{} \indef.\fem{} place deep\cb{} \cop}
{until they arrived at a place (that was deep).}
{KhanQaraqosh}{477 (10) {[S:26]}}

\acex{Conjunction}{Clause}{625}
{hál dəd\cb{} ʾáθə ʾìḏaˈ}
{until \comp\cb{} \sbjv.come festival}
{until the festival came}
{KhanQaraqosh}{494 (8) {[K:52]}}



The principal difference between a \lnk* and a \comp* in the current framework is that a \lnk* stands in \isi{apposition} to its \prim, while the \comp* is governed by the conjunction. This is represented schematically in \Vref{tb:Comp_lnk}.


\begin{table}[h!]
\centering
\begin{tabular}{ccccc}
\toprule
Conj. & $\mapsto$ & [\comp\ & $\mapsto$ & Clause] \\
Noun & $\leftrightarrow$ & [\lnk\ & $\mapsto$ & Clause] \\
\bottomrule
\end{tabular}
\caption{Complementizer and \lnk* constructions} \label{tb:Comp_lnk}
\end{table}



  
In some respects, however, the \comp* \d is quite similar to the \lnk* \d. For instance, with conjoined complement clauses, it can be repeated before each conjoint (compare the structure to \example{582}). Note also the  variation between the forms \transc{dəd} and \d.

\acex{Conjunction}{Conjoined Clauses}{616}
{ʾə́mma dəd\cb{} k-oḏí-la b\cb{} idàθθəˈ wa\cb{} də\cb{} g-dáre tùma bgáw-aḥˈ}
{when \comp\cb{} \ind-do.\agent3\pl-\patient3\fem{} in\cb{} hands and\cb{} \comp\cb{} ind-put garlic in-3\fem}
{when they make it with their hands and put garlic in it}
{KhanQaraqosh}{488 (5) {[S:77]}}


Other data also support the idea that the difference between the \comp* \d and the \lnk* \d is not so great. This is the case when the \isi{interrogative pronoun} \foreign{ma}{what} is inserted between the conjunction and the \d morpheme.\footnote{Functionally, \transc{ma} does not serve here as an \isi{interrogative pronoun}, but rather as a kind of indefinite pronoun, representing the event described in the complement clause.} In such cases, it is possible to analyse the \d as a \lnk* whose antecedent is the \isi{interrogative pronoun}:




\acex{Pronoun}{Clause}{622}
{ʾəm\cb{} qámə má d\cb{} yadə̀ʾ-laˈ}
{from\cb{} before what \lnk\cb{} know-\patient3\fem}
{before he knew her}
{KhanQaraqosh}{492 (3) {[Gospel 1 = Matthew 1:18]}}

\largerpage
In such cases the conjunction governs the \isi{interrogative pronoun}, which in turn stands in \isi{apposition} to the \lnk*. The \isi{apposition} between the \isi{interrogative pronoun} and the \lnk* can be deduced from cases in which the \lnk* is absent:


\acex{Pronoun}{Clause}{623}
{m\cb{} qámə ma\cb{} máṭax ʾəl\cb{}\footnotemark{} Már Qurṭàya}
{from\cb{} before what\cb{} reach.1\pl{} to\cb{} M. Q.}
{before we reach (the monastery) of Saint Q.}
{KhanQaraqosh}{492 (2) {[K:79]}}

\footnotetext{The preposition \transc{ʾəl}, absent in the original transcript, can be heard in the recording. This is immaterial to the current discussion.}

\newpage 
The two possibilities are represented schematically in \ref{tb:conj_intrg}:.
\begin{table}[h!]
\centering
\begin{tabular}{lcr}
\toprule
Conj. $\mapsto$ Intrg. \hspace{1ex} $\leftrightarrow$  & [\lnk\ & $\mapsto$ Clause] \\
Conj. $\mapsto$ Intrg. &  $\mapsto$ &  Clause\hspace{0.8ex} \\
\bottomrule
\end{tabular}
\caption{Combinations of conjunctions and interrogative pronouns} \label{tb:conj_intrg}
\end{table}

 


In contrast to conjunctions, prepositions cannot in general be followed by a \d morpheme, neither as a \lnk* nor as a \comp*. Examples where this seems to be the case were analysed above as \cst-marked prepositions where the \mbox{\transc{-d}} suffix was resyllabified with the \secn; see \examples{650}{639}. There is, however, one case which cannot be explained in this way, due to the conservation of the final \phonemic{-a} vowel at the end of the preposition, which excludes the occurrence of an \ed suffix. This is the case of the preposition \foreign{eka}{at}:

\acex{Preposition}{Noun}{576}
{ʾeka\cb{} d\cb{} áwa xána xə̀nna}
{at\cb{} \gen(?)\cb{} \dem.\far.\masc{} square(\fem) other.\masc}
{in the other square}
{KhanQaraqosh}{234 {[B:186]}}
 
Such cases may arise by analogy to the above mentioned examples, and may represent the first signs of an emerging \gen* marker. 



















\subsection{Adjectival \prims}

In one instance, an adjectival \prim is modified by a noun with the aid of the \lnk*. The construction is similar to the \concept{impure annexation} of  \ili{Arabic}, except for the fact that \lnk* replaces the \cst* marking of the \prim present in \ili{Arabic} (contrast with \example{557}, and the \ili{Arabic} example given there).

\acex{Adjective}{Noun}{558}
{xlíθ-i smóqta də\cb{} pàθaˈ}
{sweet.\fem-\poss.1\sg{} red.\fem{} \lnk\cb{} face}
{my rosy-faced sweet-heart}
{KhanQaraqosh}{281 {[Poetry 3]}}\antipar

\subsection{Pronominal \prims}

There are some cases of pronominal \prims in the ALC. According to my survey, in such cases the \secn is always clausal. The analysis of these cases as representing the ALC rather than the CSC relies on the fact the final vowel of the \prim noun is not altered, and moreover the \phonemic{d} marker cannot be said to replace a \free* suffix. Yet from a prosodic point of view the distinction is quite fuzzy, as the \phonemic{d} segment is often syllabified with the pronominal \prim.

Different types of pronouns can act as \prims. Among these are personal pronouns, as in the following example: 

\acex{Pronoun}{Clause}{596}
{w\cb{} áhu d\cb{} là g-laqə́m-waˈ}
{and\cb{} 3\masc{} \lnk\cb{} \neg{} \ind-catch-\pst}
{anyone who did not catch it}
{KhanQaraqosh}{479 (1) {[B:172]}}

Interrogative pronouns are quite common:

\acex{Pronoun}{Clause}{604}
{ʾu\cb{} šqə́lhə ma\cb{} d\cb{} šqə́lhə m\cb{} Baġdèdə.ˈ}
{and\cb{} took.3\pl{} what\cb{} \lnk\cb{} took.3\pl{} from\cb{} B.}
{And they took what they took from Qaraqosh.}
{KhanQaraqosh}{480 (1) {[F:4]}}

\acex{Pronoun}{Clause}{606}
{w\cb{} éma d\cb{} g-laqə́m-wa ʾáḏi qátta zùrtaˈ}
{and\cb{} which \lnk\cb{} \ind-catch-\pst{} \dem{} stick small}
{whichever (person) caught the small stick}
{KhanQaraqosh}{480 (7) {[B:172]}}



When the initial pronoun is a \isi{demonstrative pronoun}, two analyses are possible. Either the demonstrative is seen as the \prim element, as in the preceding examples, or it is  analysed as determiner, in which case there is no \prim (its absence is marked by a \zero below). The latter analysis is possible since the \lnk* acts as a syntactic head, due to its pronominal nature. Recall that such an analysis was suggested for \JZax, in which phonologically reduced demonstratives serve as \isi{determiners} (see \sref{ss:JZax_lnk_zero_head}).

\acex{Pronoun/\zero}{Clause}{597}
{ʾáḏa \opt\zero{} d\cb{} ilə mənn-àḥˈ}
{\dem.\masc{} \opt\zero{} \lnk\cb{} \cop{} with-3\fem{}}
{The one who is with her}
{KhanQaraqosh}{479 (3) {[Play 17]}}\antipar 
\newpage

Interesting to note is a short version of the plural demonstrative \transc{anə}, which resembles the determiner of \JZax (see also \example{572}):\footnote{\citet[82]{KhanQaraqosh} explains that \enquote{[t]he final /\textit{ə}/ may be elided altogether when the pronoun is closely connected to what follows by the relative particle \textit{d-}}. See also \vref{ft:schwa_drop}.}

\acex{Pronoun/\zero}{Clause}{601}
{ʾan \opt\zero{} də\cb{} g-náṭri xə̀θnaˈ}
{\dem.\pl{} \opt\zero{} \lnk\cb{} \ind-guard.\pl{} groom}
{Those who look after the groom}
{KhanQaraqosh}{479 (7) {[K:44]}}

Similar analytical difficulty is present in the case of the element \transc{xa}, which can be interpreted either as an indefinite pronoun or an \isi{indefinite determiner}. Yet in the following example, the fact that \transc{xa} follows the quantifier \foreign{ay}{any} (borrowed from \ili{Arabic}) renders the pronominal interpretation more plausible (compare with \JZax \example{368}). Be it as it may, both possibilities are given in the glossing below:

\acex{Pronoun/\zero}{Clause}{602}
{ʾáy\cb{} xa \opt\zero{} d\cb{} ílə márʾa ḥabábəˈ}
{any\cb{} \indef.\pro/\textsc{det} \opt\zero{} \lnk\cb{} \cop{} ill pustules}
{anyone who is ill with pustules}
{KhanQaraqosh}{479 (1) {[S:16]}}

It should be noted, moreover, that a determiner category has not been posited by \citeauthor{KhanQaraqosh} for \Qar. The analysis required to establish the existence of such a category is outside the scope of the work, and therefore I leave the two possibilities open.


\subsection{Linkers without an explicit \prim} \label{ss:Qar_ALC_noPrim}


In subject position, cases of the ALC without a preceding \prim are quite rare, and seem to be restricted to formal genres. Such are \example{544}, where the (semantic) \prim appears after the construction, as well as the following example, where no \prim appears at all (=\example{551bis}). For convenience, the place of the absent primary is marked by the symbol \zero, without implying the existence of a \zero\ morpheme there.

\acex{\zero}{Noun}{551}
{kə-mzámri \zero{} də\cb{} ʾurxàθa.ˈ}
{\ind-sing.3\pl{} \zero{} \lnk\cb{} roads}
{(The people) of the roads sing.}
{KhanQaraqosh}{279 (21) {[Poetry 29]}}\antipar 

\newpage 

A similar example is attested in the Gospels translation, with a clausal \secn:

\acex{\zero}{Clause}{613}
{(əmbùrx \cb{}elə) \zero{} d\cb{} áte  b\cb{} šə́mm-əḥ əd\cb{} rább-i.ˈ}
{blessed \cb{}\cop{} \zero{} \lnk\cb{} come.3\masc{} in\cb{} name-\poss.3\masc{} \lnk\cb{} lord-\poss.1\sg}
{Blessed is he who comes in the name of the Lord.}
{KhanQaraqosh}{482 (1) {[Gospel 19 = Matthew 23:29]}}

Clausal \secns following a \lnk* without an immediate \prim occur rarely also in cleft sentences.

\acex{\zero}{Clause}{628}
{ʾu\cb{} ʾàḏe \cb{}l \zero{} d\cb{} k-oḏí-la da\cb{} gùpta.ˈ}
{and\cb{} \dem.\near.\masc{} \cb{}\cop{} \zero{} \lnk\cb{} \ind-make.\agent3\pl-\patient3\fem{} into\cb{} cheese}
{It is this that they make into cheese.}
{KhanQaraqosh}{508 (2) {[S:73]}}



As a predicate, on the other hand, a \lnk* phrase can easily appear without an immediate \prim (as the latter is typically mentioned in the subject position). The predicate position is easily recognized as it is normally marked by a following \isi{copula}.

\acex{\zero}{Noun}{548}
{fá ʾáḏa pàšmaʾ \zero{} ʾəs\cb{} sáw-i \cb{}iwa.ˈ}
{and \dem.\near.\masc{} garment \zero{} \lnk\cb{} grandfather-\poss.1\sg{} \cb{}\cop.\pst}
{This \textit{pašma} belonged to my grandfather.}
{KhanQaraqosh}{279 (16) {[F:112]}}

\acex{\zero}{Noun}{549}
{ʾáhi ʾiyáwa \zero{} d\cb{} ʾə̀zla.ˈ}
{3\fem{} 3\fem.\cop.\pst{} \zero{} \lnk\cb{} wool}
{It was of wool.}
{KhanQaraqosh}{279 (18) {[K:29]}}

\acex{\zero}{Pronoun}{537}
{ʾəxála lélə \zero{} dìd-ux.ˈ kása lélə \zero{} dìd-ux?ˈ}
{food \neg.\cop{} \zero{} \lnk-\poss.2\masc{} stomach \neg.\cop{} \zero{} \lnk-\poss.2\masc{}}
{The food is not yours, but the stomach is indeed yours!}
{KhanQaraqosh}{273 (10) {[Proverbs 11]}}

\largerpage
Occasionally a nominal subject acting as \prim is lacking altogether in the sentence, though it appears in the textual context. Such are the following examples, the first one being poetic:

\acex{\zero}{Noun}{550}
{bəd-péša \zero{} d\cb{} rəxmùθaˈ ʾu\cb{} \zero{} d\cb{} bəsmúθa ʾu\cb{} bənyànə.ˈ}
{\fut-become.\fem{}	\zero{} \lnk\cb{} love and\cb{} \zero{} \lnk\cb{} delight and\cb{} buildings}
{It will become (a town) of love, delight and buildings.}
{KhanQaraqosh}{279 (20) {[Poetry 4]}}

\acex{\zero}{Pronoun}{536}
{\zero{} dìd-an \cb{}inaˈ}
{\zero{} \lnk-\poss.1\pl{} \cb{}\cop.3\pl}
{They are ours.}
{KhanQaraqosh}{273 (9) {[S:49]}}\antipar




\subsection{Ordinal \secns}

Ordinals are regularly formed by placing cardinal numerals as \secns of the ALC. These agree with the \prim noun, possibly by analogy with adjectives (see \example{630}), though morphologically the agreement pattern is different:

\acex{Noun}{Ordinal}{509}
{góra d\cb{}  tré}
{man \lnk\cb{}  second.\masc{}}
{the second man}
{KhanQaraqosh}{225}

\acex{Noun}{Ordinal}{511}
{báxta d\cb{}  tə́ttə}
{woman \lnk\cb{}  second.\fem}
{the second woman}
{KhanQaraqosh}{225}

This construction, however, is often avoided in favour of the construction using borrowed \ili{Arabic} numerals,  as in \examples{695}{696}.




\section{Double-marking constructions}
\label{ss:Qar_double}

\subsection{Simple double-marking (X.\textsc{cst} \textsc{lnk} Y)}  \label{ss:Qar_double_1}

The occurrence of an AC marked both by a \cst* suffix and a \lnk* happens mostly due to hesitation according to \citet[208]{KhanQaraqosh}. The following examples illustrate this (the \ldots\ sign signals the hesitation):

\acex{Noun}{Noun}{491}
{šə́kl-əd \ldots{} d\cb{} fàrwa}
{form-\cst{} \ldots{} \lnk\cb{} fur}
{the form of a fur}
{KhanQaraqosh}{208 {[K:17]}}

\acex{Noun}{Noun}{493}
{ʾarút-əd ... də-ḥàša}
{Friday-\cst{} \ldots{} \lnk\cb{} suffering}
{Good Friday}
{KhanQaraqosh}{208 {[K:77]}}

\acex{Infinitive}{Noun (object)}{571}
{(ʾáhu kə-mšaġə́l-wa bə\cb{}) raqóʾ-əd ... ʾəd qòndaratˈ}
{3\masc{} \ind-work-\pst{} in\cb{}  sew.\inf{} \ldots{} \lnk{} shoes}
{He used to work at the making of shoes.}
{KhanQaraqosh}{369 (4) {[K:50]}}

\acex{Infinitive}{Noun (object)}{572b}
{ʾəl\cb{} ʾastòy-ədˈ ... də\cb{} [wánat də́t-te ʾu\cb{} nàšəˈ]}
{to\cb{} drink.\caus.\inf-\cst{} \ldots{} \lnk\cb{} sheep \lnk-\poss.3\pl{} and\cb{} people}
{to  provide drinks for their sheep and for people}
{KhanQaraqosh}{369 (5) {[K:87]}}

\acex{Pronoun}{Noun}{579}
{(ʾu\cb{} k-zálan əg-máṭax) hál ékəd ... ʾəd\cb{} ʾàrʾaˈ }
{and\cb{} \ind-go.1\pl{} \ind-reach.1\pl{} until where-\cst{} \ldots{} \lnk\cb{} land}
{We go until we reach the land.}
{KhanQaraqosh}{390 (19) {[K:78]}}

  
Indeed, even when there is no hesitation sign in the transcribed corpus, listening to the examples reveals a hesitation, which I have marked in the following examples:\footnote{Recall that only the recordings of informant K are publicly available.}

\acex{Noun}{Clause}{503}
{əl\cb{} ʾén-əd \ldots{} də\cb{} kə- \ldots{} masxé-wa-la qadə́šta Sàra}
{to\cb{} spring-\cst{} \ldots{} \lnk\cb{} \ind- \ldots{} make\_swim.3\pl-\pst-3\fem{} St. S.}
{to the spring in which they made Saint Sarah swim}
{KhanQaraqosh}{209 {[K:12]}}


\acex{Adverbial}{Pronoun}{603}
{(yəʾə́l-wa-la káθlaka ʾə̀ll-aḥˈ) \ldots{} ʿan\cb{} ṭarī̀q-əd\footnotemark{} \ldots{} ʿan\cb{} ʾùrx-əd \ldots{}ˈ də̀\cb{} xxaˈ (šə̀mm-əḥˈ \ldots{} Már Yoḥánna Dilèmi.ˈ)}
{entered-\pst-3\fem{} Catholicism to-3\fem{} \ldots{} by\cb{} way-\cst{} \ldots{} by\cb{} way-\cst{} \ldots{} \lnk\cb{} one name-\poss.3\masc{} \ldots{} M. Y. D.}
{(Catholicism entered it) by means of one (whose name is Mar Yoḥanna Dilemi.)}
{KhanQaraqosh}{479 (2) {[K:1]}}

\footnotetext{The words \transc{ʿan ṭarī̀q-əd} are not present in the transcript, but are clearly audible in the recording. Indeed, the speaker first uses the \Arab loan-expression \transc{ʿan ṭarīq} with an Aramaic \cst* suffix, before correcting to the native Aramaic word \transc{ʾurxa} (but still employing the \ili{Arabic} preposition \transc{ʿan}).}

Of special interest are cases where the \prim noun is the reflex of an historical \cst* form, i.e. without a \phonemic{-d} suffix. This shows that it is not a simple repetition of the \phonemic{d} segment:

\acex{Noun}{Noun}{704}
{ʾu\cb{} bí \ldots{} d\cb{} Aqlìmus.ˈ}
{and\cb{} house.\cst{} \ldots{} \lnk\cb{} A.}
{and the family of Aqlimus.}
{KhanQaraqosh}{544 {[K:18]}}

All these cases can be analysed as instances of two appositive ACs, in which the first one is lacking a \secn, due to a difficulty of the speaker.

\begin{table}[h!]
\centering
\begin{tabular}{ccc}
\toprule
\Prim & & \Secn \\
\midrule
 X 				& $ \mapsto $ 	& \ldots{} 				  \\
 $\updownarrow$ 	& 				& $\updownarrow$  \\
 \lnk			& $ \mapsto $	& Y					  \\ 
\bottomrule
\end{tabular}
\caption{Appositive repetition of an AC due to hesitation}
\end{table}

\largerpage
However, the same construction is used \enquote{sporadically}, according to \citeauthor{KhanQaraqosh}, also where no hesitation is present (I could not verify this by listening to these examples):

\acex{Noun}{Noun}{494}
{b\cb{} paqárt-əd d\cb{} áne ḥawāwī̀n}
{in\cb{} neck-\cst{} \lnk\cb{} \dem.\far.\pl{} animals}
{on the neck of those animals}
{KhanQaraqosh}{208 {[B:72]}}\antipar 
\newpage 



\acex{Infinitive}{Noun}{495}
{dyáq-t-ət də\cb{} ləbbawàθa}
{beat.\inf-\fem-\cst{} \lnk\cb{} hearts}
{the beating of hearts}
{KhanQaraqosh}{208 {[Poetry 14]}}

The first case may be  explained by the emergence of a \concept{genitive} marker before a \isi{demonstrative pronoun} (see \examples{512}{639} for a discussion, as well as \example{576}), while the second one is taken from a poetic text, which may explain its peculiar syntax.\footnote{It is also possible that the \lnk* is in fact a resyllabified \cst* suffix. If this is true, the doubling of the \phonemic{d} segment may be explained by the phonological constraint of conserving the short \phonetic[ə] vowel in a closed syllable. Note that the \isi{schwa} cannot be elided (as is sometimes the case), since it follows a consonant cluster in both cases.}  Two other cases which \citet[209]{KhanQaraqosh} mentions as repetition of the \phonemic{d} particle are simply cases where the \ph/dəd/ allomorph of the \lnk* is used (see \example{497}).

Another exceptional case of seemingly \isi{double marking} is the following example, in which the \prim is marked by a \phonemic{-ə} suffix, while the \secn is marked by the \phonemic{ʾəd} allomorph of the \lnk*. Although it resembles cases of a re-syllabified \cst* suffix (see \examples{483}{703}), it differs from them by the intervening \isi{glottal stop}. Again, the poetic origin of this example may explain its peculiar morpho-phonology.\footnote{The sources marked as Poetry are transcriptions of recitations of poems, which may obey specific metrical rules, on the one hand, or try to imitate a classical syntax, on the other hand.} 

\acex{Noun}{Noun}{478}
{máθ-ə ʾəd\cb{} dérə ʾu\cb{} ʾitàθa}
{town-\cst{} \lnk\cb{} monasteries and\cb{} churches}
{town of monasteries and churches}
{KhanQaraqosh}{208 {[Poetry 17]}}

To summarize, the cases of \isi{double marking} which are not motivated by difficulties in production (i.e. hesitation) are highly exceptional.























\subsection{The double annexation construction (X-y.\poss\ \textsc{lnk} Y)} \label{ss:Qar_DAC}



Another type of \isi{double marking}, which occurs only in the Gospel translations, arises from a direct translation of the \Syr \isi{double annexation} construction (see \sref{ss:syr_DAC}).

\acex{Noun}{Noun}{552}
{Yósəf ʾə́br-əḥ əd\cb{} Dawìdˈ}
{Y. son-\poss.3\masc{} \lnk\cb{} D.}
{Joseph, the son of David}
{KhanQaraqosh}{279 (22) [Gospel 3 = Matthew 1:20}\antipar









\section{Juxtaposition (X Y.\opt{\agr})} \label{ss:Qar_juxt}

Juxtaposition in principle does not serve as an AC in \Qar, as it cannot be used to modify a noun by a noun.\footnote{A possible exception is the phrase \foreign{ḥŭ́kum Qaraqòš}{the governance of Qaraqosh} \citep[3; 643 (83)]{KhanQaraqosh}, but the usage of the \ili{Arabic} name of Qaraqosh indicates that the entire expression is borrowed from \ili{Arabic}.} 
 However, it is used in some tangential cases:

\subsection{Adjectival modification: Juxtaposition-cum-agreement}

Adjectives follow their \prim, and -- if inflecting -- agree in number (and possibly in gender) with it, forming effectively a \isi{juxtaposition-cum-agreement} construction:

\acex{Noun}{Adjective}{506}
{ˈənšə surayə}
{women Christian.\pl{}}
{Christian women}
{KhanQaraqosh}{212}

\acex{Noun}{Adjective}{507}
{baxta bāš}
{woman good(\invar)}
{a good woman}
{KhanQaraqosh}{220}

The ordinal \transl{first} behaves as an adjective:

\acex{Noun}{Ordinal (first)}{630}
{báxta qaméθa}
{woman first.\fem}
{the first woman}
{KhanQaraqosh}{225}


\newpage 
In some cases an adjective can precede the \prim.\footnote{Recall that the titles of the examples  always follow the order \textbf{\Prim-\Secn}.} This is the case for the Kurdish-borrowed adjective \foreign{xoš}{good} which occurs regularly before the noun, as in the following example:\footnote{This adjective shows also exceptional order in Kurdish, which normally has post-nominal adjectives. Its irregular syntax is due to the fact that it originates in Turkic languages (\ili{Turkish} \transc{hoş}, \ili{Azeri} \transc{xoş}), which regularly have pre-nominal adjectives.}

\acex{Noun}{Adjective}{561}
{xóš ʾəmmàθaˈ}
{good(\invar) mothers}
{good mothers}
{KhanQaraqosh}{281 {[Play 120]}}

\ili{Arabic} \isi{ordinals}, used often instead of the Aramaic equivalents, are regularly placed before their nominal \prims, mirroring the \ili{Arabic} construction (contrast with \examples{509}{511} as well as \vref{ex:630}):\footnote{In \ili{Arabic}, the \isi{numeral} is formally in \cst*, but this is no longer apparent in the borrowed expression.}

\acex{Noun}{Ordinal (first)}{695}
{ʾawwal yoma}
{first day}
{the first day}
{KhanQaraqosh}{516}

\acex{Noun}{Ordinal}{696}
{ʾu\cb{} θáləθ yóma}
{and\cb{} third day}
{the third day}
{KhanQaraqosh}{640 {[F:72]}}

The word \foreign{xənna}{other}, although semantically not a typical adjective\footnote{Unlike a typical adjective, it does not refer to an attribute (i.e., quality) of the referent, but rather signals that it is different from a similar previously-mentioned referent. Indeed,  \citet[285]{KhanQaraqosh} classifies it as a \enquote{non-attributive modifier}.} behaves syntactically as one insofar it agrees with the modified noun in number and gender. Unlike typical adjectives, however, it may variably appear before or after the noun (compare to \example{522}).

\largerpage
\acex{Noun}{Adjective}{653}
{(ʾə́t-lan) xə́rta ṭaʿólta}
{\exist-1\pl{} other.\fem{} game}
{(We have) another game.}
{KhanQaraqosh}{285 {[K:35]}}\antipar

\newpage
Finally, \enquote{emotionally charged} adjectives may appear before their noun. This is possible related to the \isi{emotive genitive} construction shown in \examples{559}{560}:

\acex{Noun}{Adjective}{562}
{məskín-ə ġdedáy-ə}
{poor-\pl{} inhabitants-\pl}
{the poor inhabitants of Qaraqosh}
{KhanQaraqosh}{281 {[B:68]}}

\subsection{Adverbial \prims} \label{ss:Qar_jux_adv}

Most adverbials attach to their complements without any special marking. Compare the following example to \example{639}:

\acex{Preposition}{Noun}{640}
{máx [tóbba zùrta]}
{like ball(\fem) small.\fem}
{like a small ball {[K:31]}}
{KhanQaraqosh}{238}

Note also the following example, in which the preposition is directly followed by a vowel-initial \dem* (with an \isi{epenthetic} \isi{glottal stop}), without any special marking (unlike in \JZax, where it would take \isi{genitive marking}):

\acex{Preposition}{Noun}{637}
{(zə́l-le) hál ʾáya máθa}
{went-3\masc{} until \dem.\far.\fem{} village}
{He went as far as that village}
{KhanQaraqosh}{235}

\largerpage
A preposition may appear unstressed and cliticize to its complement:

\acex{Preposition}{Noun}{575}
{ʾeka\cb{}  nəšwàθa}
{at\cb{}  relatives}
{at the home of relatives}
{KhanQaraqosh}{234 {[K:5]}}

 

\acex{Preposition}{Adverb}{514}
{mən\cb{}  táma}
{from\cb{}  there}
{from there}
{KhanQaraqosh}{238 {[F:86]}}\antipar
\newpage

\acex{Preposition}{Adverb}{634}
{hal\cb{} daha}
{until\cb{} now}
{until now}
{KhanQaraqosh}{235 {[K:87]}}

Conjunctions can also precede their clausal complement without any special marking. Such cases may be assimilated to asyndetic attributive clauses (see \sref{ss:Qar_Asyndetic_Relative_Clause}), but have a wider distribution (contrast with \example{616}):

\acex{Conjunction}{Clause}{618}
{fa\cb{}  ʾə́mma zálh\cb{}  ắmi da\cb{}  sàw-iˈ}
{and\cb{}  when go.3\pl{}\cb{}  say.3\pl{} to\cb{}  grandfather-\poss.1\sg}
{and, when they go and tell my grandfather}
{KhanQaraqosh}{489 (12) {[F:25]}}\antipar 




		
\subsection{Infinitival \prims}

The complement of an infinitive may be attached to it without any marking (contrast with \examples{541}{532} showing the CSC and \example{567} exhibiting the ALC, as well as the doubly-marked \examples{571}{572b}):

\acex{Infinitive}{Noun (object)}{569}
{(xálṣi) wáḏa ay\cb{}  ràqqəˈ}
{finish.3\pl{} do.\inf{} \dem.\near\cb{}  dish}
{They finish making those \textit{raqqe}.}
{KhanQaraqosh}{369 (6) {[B:134]}}\antipar

\subsection{Clausal \secns}
\label{ss:Qar_Asyndetic_Relative_Clause}

Certain restrictive clausal \secns may follow their \prim without any marking. This occurs exclusively with indefinite \prim nouns. It may allude to \ili{Arabic} influence, which has a similar distribution of asyndetic relative clauses only after indefinite \prims.  Judging by the examples given by \citet[477]{KhanQaraqosh} such examples occur most frequently when the matrix-clause is an existential clause.

 
\acex{Noun}{Clause}{591}
{ʾə́t-lan ʾəxálta k-amáx-la harìsa.ˈ}
{\exist-1\pl{} dish \ind-say.\agent1\pl-\patient3\fem{} harisa}
{We have a dish called \textit{harisa}.}
{KhanQaraqosh}{477 (9) {[K:62]}}\antipar

\newpage 

One exceptional case is the following (the second half of \example{592}):

\acex{Preposition}{Clause}{592b}
{lə́\cb{} ġḏa dúka ʿamùqt\cb{} elaˈ}
{to\cb{} \indef.\fem{} place deep\cb{} \cop}
{to a place that is deep}
{KhanQaraqosh}{477 (10) {[S:26]}}

Cases of attributive nominal clauses which lack a \isi{copula} (similar to the \concept{non-clausal adjectival nexus} of \JZax \example{451}) also occur. Note that the subject of such attributive clauses is always definite by virtue of a \isi{possessive pronoun}.

\acex{Noun}{Clause}{594}
{ʾíθə náša lə́bb-əḥ níxa b\cb{} áḏa yòma?ˈ}
{\exist{} man heart-\poss.3\masc{} content in\cb{} \dem.\near.\masc{} day}
{Is there a person whose heart is content nowadays?}
{KhanQaraqosh}{478 (14) {[Play 96]}}

  
A similar example occurs with an indefinite pronominal \prim \transc{xa}  (=end of \example{603}):

\acex{Adverbial}{Pronoun}{603bis}
{xaˈ šə̀mm-əḥˈ Már Yoḥánna Dilèmi.ˈ}
{one name-\poss.3\masc{} M. Y. D.}
{one whose name is Mar Y.D.}
{KhanQaraqosh}{479 (2) {[K:1]}}

\section{Conclusions}

The dialect of \Qar has in principle two {loci}\is{locus} of marking: the \prim (by the \cst* \ed suffix) and the \secn (by the \d \lnk*). The two strategies are in principle mutually exclusive, but in some peripheral cases they are used simultaneously, as explained in  \sref{ss:Qar_double}. The distinction, however, between the two types of marking is not always so clear, due to the frequent re-syllabification of the \ph/d/ segment of the \cst* suffix with vowel-initial \secns. Since the emergence of the \cst* suffix is related (at least partly) to the  re-syllabification of the \lnk* with the \prim (see \sref{ss:neo-CSC}), the flux between the two constructions in \Qar may be related to its conservative nature \citep[10]{KhanQaraqosh}.




An important source of impact on \Qar is \ili{Arabic}. Geographically, the town of Qaraqosh is located near the \ili{Arabic} speaking regions of Iraq (only 32 km away from Mosul), and as such it is influenced by \ili{Arabic} more than other dialects.  While \citet[9]{KhanQaraqosh} claims that the \ili{Arabic} influence is relatively recent, judging by the lexical material, we see that the dialect possess many constructions which are similar to \ili{Arabic} ones. These include the replication of the \ili{Arabic} \transc{tāʾ marbūṭa} (\example{469}), the \ili{Arabic} adjectival AC (\concept{impure annexation}; \example{557} and \example{558}), the integration of the \ili{Arabic} nominal modifier \transc{nafs} (see \sref{ss:Qar_nafs}) and ordinal numbers (\examples{695}{696})  and possibly also asyndetic clausal \secns (see \sref{ss:Qar_Asyndetic_Relative_Clause}). On the other hand, direct Kurdish influence is harder to pinpoint, except for the pre-nominal use of the Kurdish (originally \ili{Azeri}) loan-adjective \foreign{xoš}{good} (\example{561}).







	