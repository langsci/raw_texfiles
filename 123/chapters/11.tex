





\chapter{Further developments of attributive constructions}

In the last chapter I have discussed the development of D-marked attributive constructions in \ili{NENA} dialects. In this chapter I shall discuss further changes of the AC system, of which only some are related to the D-markers. 

The first section deals with the apocopate \isi{construct state} marking, functionally a variant of the \ed suffix, whose development may reflect both the retention of the historical \cst* and innovative forms. 

\Sref{ss:double} discusses the re-introduction of various double-marked attributive constructions, following the loss of the DAC discussed in the previous chapter. 

\Sref{ss:loss_marking} presents a true innovation of \ili{NENA}, very possible motivated by the influence of contact languages: the introduction of a productive zero-marked \isi{juxtaposition} construction in the core of the AC system of various dialects. 

\Sref{ss:matter-replication} discusses further clear cases of contact influence, namely \isi{matter replication} (borrowing of morphemes) from contact languages, notably the \ili{Iranic} \ez* (\sref{ss:borr_ez}) and subordinator particle ({\sref{ss:borrow_ke}}).

\textit{In lieu} of conclusions, I present in \sref{ss:case-study} a case study concerning a sub-domain of the AC system, namely nominal modification by \isi{ordinals}. As we shall see, this sub-system is symptomatic of the entire AC system, as it shows both high variation and uniformity across dialects, model languages and anterior strata. 


\largerpage
\section{Apocopate construct state: Retention and renewal} \label{ss:apcopate}

Alongside the innovation of the Neo-\cst\ suffix \ed, discussed in \sref{ss:neo-CSC}, in many \ili{NENA} dialects one finds nominals marked exhibiting \cst* by \isi{apocope}, i.e.\ the omission of their \free* suffix, usually \transc{-a} or \transc{-e}.\footnote{This is of course only possible with nous which exhibit the \free* suffix; unadapted \isi{loanwords} are thus excluded.} The \isi{apocope} is occasionally extended to a preceding consonantal segment.

\textcites[3, fn.\ 15]{MutzafiBarzani}[92]{MutzafiBetanure}  makes a distinction between innovated apocopated \cst* nouns (which he terms \enquote{Neo-construct}) and retentions of the ancient \cst* forms. For example, in \Betn he differentiates between \foreign{beθ}{house of} and \foreign{be}{household, family of}, both derived from \foreign{beθa}{house} \citep[92]{MutzafiBetanure}. While some couplets can easily be distinguished due to irregular morpho-phonological processes which took place in the historical form (for instance, the free form \foreign{brona}{son} is related both to the innovated \cst* form \transc{bron} and the historical \cst* \transc{bər}\footnote{The latter is the \cst* of \foreign{brā}{son}, extant in \il{Aramaic!Classical}Classical Aramaic, but not used as such in \ili{NENA}. Diachronically, \transc{brona} was derived from \transc{brā} by addition of the diminutive suffix \transc{-ona}, but in \ili{NENA} it is the regular form of the noun.}), in general it is not always clear what the philological basis for this distinction it. Note, for instance, that \citet[3, fn.\ 15]{MutzafiBarzani} asserts that \foreign{nāš}{people of} is an innovated \cst* form of \transc{nāšā}, but \citet[92]{MutzafiBetanure} lists it as a \enquote{possible old construct form}.\footnote{The situation for this noun is in fact more complex, as the form \transc{nāš} appears sometimes in various dialects in contexts where a \cst* is not regularly expected. In these cases it may represent a retention of the historical \isi{absolute state} singular form \citep[173]{KhanArbel}, or possible a phonological process of final vowel elision, as it is found also in \pl* use: for instance, in \Arb one finds the expression \foreign{náš xriwé}{bad people} \citep[229 {[B:13]}]{KhanArbel}, and in \JZax \foreign{ʾan nāš ʾımme}{the people with him} \citep[117 (112)]{CohenZakho} (=\example{377}).} Synchronically, moreover, the source of \isi{apocope} plays no role: if a given dialect allows the regular usage of apocopate \cst* nouns, the innovated and historical forms are morpho-syntactically equivalent, with the exception that some historical \cst* nouns may only appear as part of lexicalised compounds (see \examples{1905}{1904}). Therefore, I prefer to subsume the two under one category of \concept{apocopate construct state} and reserve the term \concept{neo-construct} for the truly innovated suffix \ed discussed in the previous chapter.

The distribution of the apocopate CSC is given in \ref{tb:app_cst}, which should be compared to \vref{tb:suff_cst}, showing the distribution of the Neo-CSC.

\begin{table}[p!ht]
\centering
\begin{tabular}{l l | c c c | c c c}
\toprule
		&					& \multicolumn{3}{c}{\Prims}& \multicolumn{3}{|c}{\Secns} \\
Region 	& Dialect			& Noun	& Adj. & Inf.		& N/NP & Ordinal & Clause \\	
\midrule
{SE Turkey} & \Her 	& (-)	&		&			& +				&		&	\\
					& \Boh 	& 		&		&			&				&		&	\\
					& \Bes 	& 		&		&			&				&		&	\\
					& \Gaz 	& 		&		&			&				&		&	\\ 
					& \Baz  & 		&		&			&				&		&	\\ 
					& \Cal  & +		&		&			&	+			& 	+	&	\\
					& \Jil  & (-)	&		&			& 	+			&		&	\\
\midrule
NW Iraq		& \JZax & +		&	+	&	+		& +				& +		& + \\ 
					& \JArd & (-)	&	&				& +				&		&		\\
					& \CArd & 		&		&			&				&		&	\\
					& \Barw & +		& + &				& +				&		&		\\
					& \Betn & +		&	&				& +				&		&		\\
					& \Amd 	& + 	& 	& 				& +				&		&	+	\\
					& \Barz & +		&	&				& +				&		&		\\
					& \Alq 	& 		&		&			&				&		&	\\
					& \Qar  & (-)	&	&				& +				&		&		\\ 
\midrule
NW Iran		& \JUrm & +		&	&				& +			&		& 	\\ 
					& \Sar 	& +		&	&				& +		&			&\\
\midrule
NE Iraq 	& \Rus  & 		&		&			& 				&		&	\\ 
					& \DiyZ	& (-)	&	&				& +			&		&		\\ 
					& \Arb 	& (-)	&	&				& +			&		&		\\ 
					& \JKoy & 		&		&			&				&		&	\\
					& \JSul & (-)	&	&				&			&		&		\\
					
\midrule
W. Iran			& \JSan & (-)	&	&				& +				&		&\\ 
					& \CSan & +		&	&				& +				&		&\\

\bottomrule& 
\end{tabular}
\caption[Distribution of apocopate \isi{construct state}]{Distribution of apocopate \cst* excluding compounds. The entry (-) marks dialects where only a handful of nouns can act as \prims.} \label{tb:app_cst}
\end{table}


As noted in \sref{ss:syr_cst}, the Syriac CSC \textquote[{\cite[61]{MuraokaSyriac}}]{tends to be confined to standing phrases verging on compound nouns}. This is in general quite true also for the historical CSC in many \ili{NENA} dialects,  which  do not make productive use of the apocopate \cst*, but in which one finds rather  lexicalised compounds in which the \prim is  derived from an historical apocopate \cst*. For instance, in  \Qar and \Alq the historical \cst* form \foreign{bar}{son of} is only used as part of some lexicalised compounds \parencites[211]{KhanQaraqosh}[251]{CoghillAlqosh}.\footnote{In these dialects \transc{bar} does not represent a separate noun at all, and consequently it is not synchronically in \cst*. However, I gloss it as such for the clarity of exposition.} Note the varying degree of lexicalization as manifested by the \pl* marking: In example \ref{ex:1905}  it is  phrase final, while in example \ref{ex:1904} it is also marked on the \prim.

\lex{\Alq}{1905}
{bar- zar-a, bar- zar-ə}
{son.\cst- crop-\sg{} son.\cst- crop-\pl}
{seed, seeds}
{CoghillAlqosh}{251}

\lex{\Alq}{1904}
{bar- nāš-a, bne- nās-ə}
{son.\sg.\cst- man-\sg{} son.\pl.\cst- man-\pl{}}
{human, humans}
{CoghillAlqosh}{251}



Such compounds are fixed phrases forming one conceptual unit, and the \secn is normally non-referential. Therefore, they cannot be considered as true productive ACs, and they are excluded from \ref{tb:app_cst}.\footnote{For more information about compounds in Neo-Aramaic see \citet{GutmanCompounds}.}

Other dialects allow additionally a semi-lexicalised usage of the historical apocopate \cst*. In these cases only a handful of nouns can act as \prims (typically kinship nouns  such as \foreign{bər}{son of},\footnote{Contrast with the usage of \transc{bron}, which is also found as a grammatical head, but indicating age and not kinship (=\example{125}):

\acexfn[\JUrm]
{Noun}{Noun}{125bis}
{bron [ičči šinne]}
{son.\cst{} sixty years}
{a man 60 years old}
{GarbellUrmi}{86} 
For a kinship use of \transc{bron}, see \example{1358}.
} and the \cst* noun \foreign{be}{house of}), but the \secn is referential (\reex{7}).\footnote{\citet[27--34, \S 16.(ii)]{MacleanGrammar} gives an extensive list of such expressions, including fixed compounds.}

\largerpage
\acex[\CSan]
{Noun}{Noun}{677}
{yoḥana bər ʿam-ī}
{Y. son.\cst{} uncle-\poss.1\sg}
{my cousin John}
{PanoussiSenaya}{125 (7)}\antipar

\lex{\JSan}{7bis}
{be\cb{} kalda}
{house.\cst\cb{} bride}
{house/family of the bride}
{KhanSanandaj}{201}\antipar

The last usage is also attested  in Early J. Neo-Aramaic:

\hebacex[\Nrt]
{Noun}{Noun}{2013}
{בי ישראל}
{bē yiśrāʾẹl}
{house.\cst{} I.}
{the people of Israel}
{(\textit{Pəšaṭ Wayəhî Bəšallaḥ} 1:25, \cite[38; see there  fn.\ 16]{SabarNerwa})}


Dialects which allow only such cases are marked with (-) in \ref{tb:app_cst}.\footnote{Some dialects which are marked with + may in fact fall into this category, as the table is generally based on the explicit statement of such a limitation in the respective grammars.}

However, even dialects which permit a productive use of \isi{apocope} for marking \cst* \prims (+ marking in the \ref{tb:app_cst}) make in fact a quite limited use of this possibility, both in terms of frequency and in terms of categorial diversity of \prim and \secn. As  \ref{tb:app_cst} shows, most dialects allow only N+N combinations in this construction, which is less extensive than the Syriac usage, which allowed also adjectival and infinitival \prims. The extension to clausal \secns, a clear innovation, is attested only in \JZax and the closely related \Amd dialect. Indeed, it seems that only  \JZax generalized the usage of the apocopate \cst* marking to be completely equivalent to the suffixed \cst* marking. It is probably no coincidence that also the suffixed \cst* marking is the most extensive in this dialect, as shown in \vref{tb:suff_cst}.\footnote{The data may contain a certain bias towards the \JZax data, as the \isi{attributive relation} has been specifically investigated in this dialect by \citet[Ch.\ 2]{CohenNucleus, CohenZakho}. Yet, at least amongst the well documented dialects, it is clear that no dialect matches its extensive usage  of  \cst*-marking, both suffixed and apocopate.}	


How can the renaissance of apocopate \cst* marking in \JZax and neighbouring dialects be explained? Judging by \Nrt, the apocopate \cst* is hardly available in the Early Jewish Cis-Zab\il{Cis-Zab NENA dialects} Neo-Aramaic, \example{2013} being typical.\footnote{One may doubt whether \Nrt is truly representative in this respect. Yet,  since there are no other comparable J. \ili{NENA} sources of that time, and since in general the dialect is quite similar to an archaic form of \JZax, I regard  this corpus tentatively as an approximation of the precursor of \JZax. Note also that \citet{SabarDictionary} writes that \Nrt \enquote{may be considered \enquote{classical} JNA} [=J. Cis-Zab\il{Cis-Zab NENA dialects} Neo-Aramaic].}  As in \Nrt the suffixed \cst* marker is highly productive, one can tentatively conclude that the apocopate \cst* is a recent innovation, appearing after the innovation of the \ed suffix as marker of \cst*, rather than being a retention of the historical apocopate \cst*. Moreover, the occurrences of the historical \cst* cannot account for the renewal of this marking type, as they are constrained to specific expressions, and also since they often exhibit an irregular morphology, as the form \transc{be} (\cst* of \foreign{besa}{house}) shows. 

Additionally, one can relate the renewed apocopate marking to the innovation of the \gen* marking, discussed in \sref{ss:genitive_development}. Recall that the development of the \gen* prefix involved a \isi{resyllabification} of the \ed suffix with a vowel-initial determiner, sometimes leaving  behind a \concept{bare \prim}, without any suffix at all, as in the following \Barw example:

\lex{\Barw}{1358}
{brōn\cb{} d-o\cb{} naša}
{son.\cst{} \gen-\defi.\masc\cb{} man}
{the son of that man}
{KhanBarwar}{400 {[A9:2]}}

After the \d prefix was reanalysed as a \gen* marker, the bare form of the \prim could have been reanalysed as a \cst* form equivalent to the suffixed form \transc{bron-əd}, and could consequently also occur without a \gen* prefix following it. This hypothesis is corroborated by the fact that all those dialects that developed a productive apocopate \cst* marking have also developed  a \gen* prefix, as is clear from comparing \vref{tb:app_cst} and \vref{tb:d_gen_dem}.\footnote{All the dialects which have a productive apocopate CSC in my survey show the development of a \gen* prefix \d in at least one environment (namely, after prepositions). On average, they exhibit 2⅓ plus-signs in \ref{tb:d_gen_dem}. Conversely, those which have not developed a productive apocopate CSC may show no evidence at all of a \gen* prefix, and exhibit on average less than 1½ plus-signs in this table.}

  Moreover, this explains why most dialects restrict this development to N+N combinations.  

Whatever the source of the apocopate CSC, it is clear, however, that the dialect of \JZax, and possibly also \Amd, went one step further, as they extended the usage of the apocopate \cst* marking to more contexts, most notably clausal \secns. In the context of \JZax, the latter development may be explained by the innovation of a \d marked subordinate \isi{copula}, mirroring the \d marked \gen* \dem*s in the clausal domain (see \sref{ss:JZax_genitive_clauses}; also \cites[90]{CohenNucleus}[119ff.]{CohenZakho}). As a consequence, the domain of usage of the apocopate \cst* was extended and levelled by analogy with the usage domain of the suffixed \cst*. As no neighbouring language  shows a similar \cst* marking, one must conclude that this was an internal development specific to \JZax and possibly neighbouring J. Cis-Zab\il{Cis-Zab NENA dialects} dialects. 

\section{Re-development of double marking} \label{ss:double}

In \sref{ss:NeoCSC_Origin} I have asserted that the Neo-CSC was preferred to the DAC due to the force of \concept{economy}, since single head-marking is preferable to double-marking of head and dependent. Countering the force of \isi{economy} is the force of \concept{clarity}, which leads to a preference for more elaborate structures in order to ensure  the correct transfer of the linguistic message.
 As is well known, these two forces shape language and cause cyclic changes in which marking is reduced and then re-introduced in another shape.\footnote{See in this respect the seminal paper of \citet[186]{Slobin1977}, who posits four \enquote{charges} which shape language: \enquote{(1) Be clear. (2) Be humanly processable in ongoing time. (3) Be quick and easy. (4) Be expressive.} The first two fall under the the above notion of clarity, while the latter two relate to our notion of \isi{economy}. In his words: \blockquote{The first two charges -- clarity and processibility -- strive toward segementalization. The other two charges -- temporal compactness and expressiveness -- strive toward synthesis, however. As a result, Language constantly fluctuates  between the poles of analyticity and syntheticity, since none of the charges can be ignored.} \citep[192]{Slobin1977}} The \ili{NENA} AC domain is no exception, as it exhibits the re-emergence of double-marked constructions.\footnote{Structurally, however, the \ili{NENA} double-marking differs from the \il{Aramaic!Classical}Classical Aramaic construction. In the \ili{Syriac} DAC\is{double annexation construction}, the \prim was marked by a pronoun indexing the \secn, while in the \ili{NENA} double constructions, the \prim is marked by a \cst* marker.} \ref{tb:double} shows the possible ways of \isi{double marking} of (pro)nominal \secns following a \cst* marked \prim.

\begin{table}[p!]
\centering
\begin{tabular}{l lccccc}
\toprule
		&					& \multicolumn{5}{c}{\Secn marking after \cst\ \prims}    \\
Region 	& Dialect			&	\gen-\dem& \transc{did/y}-\poss	& \lnk & \lnk+\gen	& \rel \\
\midrule
{SE Turkey} & \Her 	& +					&						& & & \\
					& \Boh 	&	+	&					& & & \\
					& \Bes 	&		&					& (Cl)	& & \\
					& \Gaz 	&	+	& 					& & & \\
					& \Baz  & & \\
					& \Cal  & 		&	+ 				& (Cl) & & \\
					& \Jil  &	+	&					& & & \\
\midrule
NW Iraq		& \JZax &	+	&					& & & \\
					& \JArd & & \\
					& \CArd & & \\
					& \Barw &	+				&	+				&	(-) &	& \\
					& \Betn & 					&	+ 				& 		& + & \\
					& \Amd 	&	+				&	+				&	+	& &\\
					& \Barz & &	+	& & & \\
					& \Alq 	&					&					& 	(-)		& & \\
					& \Qar  & (-)				&					&	(-)		& & \\
\midrule
NW Iran		& \JUrm &	+				&	(-)				&	+	& +	& + \\
					& \Sar 	& +			&	+ 				& & & +\\
\midrule
NE Iraq 	& \Rus  &	& 					& 	& & \\
					& \DiyZ  &	+				& +					&		& & \\
					& \Arb 	&	+				&	+				&	(-)	& & \\
					& \JKoy &					&	+				&	+	& +	& +\\
					& \JSul & &	+				& & & (-)\\
					
\midrule
W. Iran			& \JSan & +			&	+ 				& & & (-) \\
					& \CSan &	+		&					& & & \\
\bottomrule& & 
\end{tabular}
\caption[Double marked ACs]{Double marked ACs. (-) = rare/doubtful occurrences; (Cl) = marginally occurring with clausal \secns.} \label{tb:double}  
\end{table}


One type of \isi{double marking}  is the \gen* case marking, which, by the very nature of case marking, marks the \isi{attributive relation} on the \secn independently of the marking of the \prim. However, in \ili{NENA} the \gen* case is normally not enough to instantiate an \isi{attributive relation}, and therefore it appears typically alongside a \cst* marked \prim. One class of such \gen* \isi{double marking} is the \d prefix that appears on \dem*s (pre-nominal and independent), whose distribution is shown in the first column of \ref{tb:double}. The development of this \isi{double marking} has been discussed in \sref{ss:genitive_development}.\footnote{Note that the first column of \ref{tb:double} is identical to the third column of \vref{tb:d_gen_dem}.} Here one example should suffice:

\acex[\Jil]
{Noun}{Noun}{1477}
{xabr-əd d-a sawa}
{word-\cst{} \gen-\dem{} old\_man}
{the word of this old man}
{FoxJilu}{60}

Another class of \gen* \isi{double marking} is the use of independent attributive pronouns (e.g.\ \transc{did-i}\~\transc{diy-i}) after \cst* nouns, as shown in the second column of \ref{tb:double}. As discussed in \sref{ss:pron_base}, such pronouns can be analysed either as \lnk+\poss\ or as \gen+\poss. In the latter case, they should follow \cst* nouns, which is indeed the case for many dialects. 

\acex[\Barz]
{Noun}{Pronoun}{820}
{yāl-\opt{əd} did-i}
{child(ren)-\cst{} \gen-\poss.1\sg}
{my children}
{MutzafiBarzani2002}{66f.}

It is worthwhile noting that, with the exception of two dialects (\Amd and \Betn), the base \transc{did-} is not used in such dialects as a \lnk* (see \vref{tb:alt_lnk}), which is coherent with the idea that it has been reanalysed as a \gen* base, bleached of a pronominal reference. 



The two exceptional dialects (\Amd and \Betn) are amongst those dialects which present a more profound structural change, namely they  allow the independent \lnk* itself (with nominal or pronominal \secns) to occur after a \cst* noun, against the logic of Classical \ili{Semitic} languages, in which the \lnk* is supposed to be in \isi{apposition} with a \isi{free state} noun (see \sref{ss:Analytic_AC}.) 

\acex[\Amd]
{Noun}{Noun}{1599}
{dar-ət dəd xaye}
{tree-\cst{} \lnk{} life}
{the Tree of Life}
{GreenblattAmidya}{73}

Consequently, the following example, while being formally similar to example \ref{ex:820}, is analysed differently, in that \transc{did-} is understood as a \lnk* and not as a \gen* marker.

\acex[\Amd]
{Noun}{Pronoun}{1620}
{beθ-əd did-i}
{house-\cst{} \lnk-\poss.1\sg}
{my house}
{GreenblattAmidya}{81}


The occurrence of this construction with nominal \secns is presented in the third and fourth columns of \ref{tb:double}. As the table shows, it occurs  regularly also in \JUrm (see \sref{ss:JUrm_cst_lnk}) and in \Koy (see \examples{1509}{1518}).


In these dialects it is difficult to analyse the \lnk* as a \gen* exponent, since it is sometimes followed by a separate \gen* morpheme, as the following example shows.

\acex[\Betn]
{Noun}{Noun}{1549}
{ṃāy-əd did d-é\cb{} kθeθa}
{water-\cst{} \lnk{} \gen-\dem\cb{} chicken}
{the broth of that chicken}
{MutzafiBetanure}{42 {[548]}}

It seems rather that these dialects  have generalized the usage of \cst* marking to occur also before the \lnk*, against the above mentioned logic of Classical \ili{Semitic} languages. Note that in the dialects in which this is especially prominent, namely \JUrm and \Koy, the \lnk* is very probably derived from an erstwhile construct-state-marked \dem* (see \sref{ss:od_lnk} and \sref{ss:JUrm_ay}) standing in \isi{apposition} to the \prim. Moreover, in \Amd, \Betn and \Koy  the \lnk* has a visible \cst* form, as it ends with \transc{-əd}\~\transc{-d}. It seems, therefore, that these dialects have developed an \concept{agreement in state} pattern, in which two nominal elements in \isi{apposition} can optionally agree in their \cst* marking. This is especially plausible in \JUrm in which  two asyndetically coordinated nouns agree in state (see \example{136}), considering that asyndetic coordination is formally similar to \isi{apposition}.




Sporadic evidence of this construction is found also in \Arb (see \example{1289}),  \Qar (see \sref{ss:Qar_double_1}), \Alq, and \Barw. In these dialects this construction can at least partly be explained as resulting from some kind of \emph{lapsus} or hesitation in speech, as in the following example.

\acex[\Barw]
{Noun}{Noun}{1378}
{qúww-ət \ldots{} t\cb{} ʾurusnàyeˈ}
{force-\cst{} \ldots{} \lnk\cb{} Russians}
{the force \ldots{} of the Russians}
{KhanBarwar}{399 {[B7:8]}}

Indeed, even in Syriac there are rare occurrences of a similar construction with apocopate \cst*, which are normally explained as errors (see \example{1045} and the accompanying discussion). 

Two dialects of Turkey, namely \Bes and \Cal restrict this construction to occur only with  clausal \secns (\Bes and \Cal). 

\acex[\Bes]
{Noun}{Clause}{1664}
{m\cb{} qam d-ayyá săhadut-əd d\cb{} wəd-la}
{from\cb{} before \gen-\dem{} testimony-\cst{} \lnk\cb{} made-3\fem}
{because of this sacrifice (lit. testimony) she made}
{SinhaBespen}{212 {[186]}}

\newpage 
In this case, we may tentatively relate it to the \Kur pattern, where a clause is introduced both by the \ez* and the \rel* (see \sref{ss:cst_ez_clausal}).\footnote{I am grateful for Eran Cohen for this idea.} Yet  the rarity of the construction in these dialects hinders a conclusive statement in this respect.   


However, in other dialects we find direct evidence of such a relation, as a borrowed \rel* is used on top of a \cst* marked \prim or a \lnk* (see for instance \Koy \example{1508} and \Sar \example{1766-7}). The distribution of this construction is shown in the fifth column of \vref{tb:double}. This development may be seen as a case of pattern-cum-\isi{matter replication} of the \rel*s, which in the source languages (\Kur, \Sor and possible \Per) appear regularly after an \ez* morpheme, paralleling the \lnk* or the \cst* marking (see \sref{ss:borrow_ke}). This development, moreover, may had an indirect influence on the development of \cst+\lnk\ construction discussed above, as the \rel* may have been perceived as the pre-clausal counterpart of the \lnk*. 

\section{Loss of all marking: Juxtaposition constructions} \label{ss:loss_marking}



In various dialects we find ACs that are not marked at all: The \prim and \secn are simply juxtaposed one after the other. We consider here cases of Noun + Noun and Noun + Clause.\footnote{We exclude adjectival \secns, since these are normally marked by agreement of the adjective with the \prim noun. While some loan-adjectives do not inflect, and thus attach to the noun in a pure \isi{juxtaposition} construction, this is better seen as lexical property of these adjectives rather than the emergence of a new AC strategy.} The distribution of these constructions is presented in \ref{tb:juxt}, which shows that these constructions are often limited to a certain grammatical domain.  In all cases, we must consider two main scenarios for the emergence of these constructions: the construction may originate in the loss of previous markers (for phonological or morphological reasons), or the construction as such may be innovated or borrowed into the language. 


\begin{table}[p!ht!]
\centering
\begin{tabular}{l l c c}
\toprule
		&					& \multicolumn{2}{c}{\Secns} \\
Region 	& Dialect 			&  Noun  &  Clause \\
\midrule
{South-East Turkey} & \Her 	& (+) 	& \indef \\
					& \Boh & 	+	&	+	\\
					& \Bes & 		&		\\
					& \Gaz & 		&	+	\\
					& \Baz  &		&		\\
					& \Cal  &	(-)	& \\
					& \Jil  &		&	+	\\
\midrule
North-West Iraq		& \JZax &		& \indef \\
					& \JArd &		&		\\
					& \CArd &		& \indef \\
					& \Barw &	\Q	& \indef \\
					& \Betn &		&		\\
					& \Amd	 &	& + \\ 
					& \Barz	&	&	\\ 
					& \Alq &	&	\\
					& \Qar &	& \indef \\
\midrule
North-West Iran		& \JUrm & \Q 	& \indef \\
					& \Sar \\
\midrule
North-East Iraq 
					& \Diy & (-)		& \\
					& \Arb & \Q\textsuperscript+		& \indef \\
					& \JKoy &		&		\\
					& \JSul & + 	& \indef \\


\midrule
West Iran			& \JSan & +		&	+	\\	
					& \CSan \\
\bottomrule
\end{tabular}
\caption[Distribution of \isi{juxtaposition} constructions]{Distribution of \isi{juxtaposition} constructions. \Q\ = quantification \prims; \indef\ = indefinite \prims.}
 \label{tb:juxt}
\end{table}


\subsection{Clausal \secns} \label{ss:asyndetic_relatives}

Clausal \secns juxtaposed to their \prims are typically called \concept{asyndetic relative clauses}. As \ref{tb:juxt} shows, in the majority of dialects this construction is restricted to cases where the antecedent noun is indefinite.\footnote{It should be noted that even in those dialects marked as \indef{} exceptions to the rule may occur.} The following is a typical example:

\acex[\Barw]
{Noun}{Clause}{1455}
{ʾiθ-waˈ xa\cb{} ràbban,ˈ tíwɛ-wa gu\cb{} xa\cb{} gəppìθa.ˈ}
{\exist-\pst{} \indef\cb{} monk sit-\pst{} in\cb{} \indef\cb{} cave}
{There was once a monk who lived in a cave.}
{KhanBarwar}{961 {[A15:1]}}

This example presents a typical usage situation an \isi{asyndetic relative clauses}: the antecedent is introduced by the particle of existence and an \isi{indefinite determiner}, and subsequently qualified by a clause. In such cases one may reasonably doubt the validity of the \isi{relative clause} analysis, as an alternative analysis in terms of two separate clauses is viable as well (\transl{There was once a monk. He lived in a cave.}). Thus, the relative wide distribution of this construction should be taken with a grain of salt, as in some cases the examples can be disputed.\footnote{Paradoxically, the less-disputable cases are those where the \secn is a reduced clause, as these cannot occur as matrix clauses; see \JZax \example{451} and \Qar \example{594}.}

\citet[138]{CohenZakho} raises the possibility that these constructions are a replication of an \ili{Arabic} pattern. Indeed, \ili{Arabic} innovated a pattern in which asyndetic relative clauses can occur after indefinite nouns \parencites[494]{BadawiCarter}[for the historical development see][]{PatElmorphosyntax}.  This idea is corroborated by the fact that most of the dialects which exhibit this construction are located in Iraq, at most 100km away from the \ili{Arabic}-speaking urban center of Mosul.\footnote{The Jewish \ili{NENA} speakers of the region had regular contacts with the  J. community of Mosul, which was predominantly \ili{Arabic} speaking but had also bilingual \ili{NENA}-speaking Jews originating in Kurdistan  \parencites[XXV]{SabarNerwa}{SabarArabic}[54]{SabarEuropean} 
Moreover, in the region we find also other \ili{Arabic} speaking J. communities; see the map of \citet[4]{JastrowAqra}}. We may similarly assume there were connections between the speakers of \JUrm and the J. communities in Iraq.\footnote{As anecdotal evidence I can mention the case of Rabbi Ḥaim Yeshurun, whom I interviewed in Israel, who moved from Nerwa to Urmi around 1940.}
As for the C. \Her dialect in Turkey, this was in close geographical proximity to an \ili{Arabic} vernacular. 



Yet  \ili{Arabic} is not the only possible source. An alternative source may be a \Sor construction, in which the \prim is marked by an \transc{-êk} suffix before a clausal \secn (this suffix is often reduced to \transc{ê}, but is clearly distinct from the \ez* \transc{-i}). The \secn may furthermore be marked by the \rel* \transc{ke}, but this is not always the case.

As discussed in \sref{ss:Sor_alt_clause}, the  \transc{-êk} suffix is identical to the \isi{indefinite suffix}, and may very well be historically related to it, although synchronically it does not convey an indefinite sense (i.e.\ the \prim may be definite). If we assume that in a prior stage of the language it was simply the indefinite marker, we get the following formal construction: N-\indef\ \opt\rel\ Clause. Since the \rel* is optional to some extent, it is easy to see how this construction could be the source of the \ili{NENA} construction. Furthermore, even if the \transc{-êk} was never functionally identical to the \isi{indefinite suffix} (or cognate with it), a speaker of \ili{NENA} with knowledge of \Sor might mistakenly analyse it as the \isi{indefinite suffix}, leading effectively to the same kind of contact influence.




A difficulty is presented by those dialects which do not restrict the use of asyndetic relative clauses to indefinite \prims, but allow it also with definite \prims, as in the following \Jil example:

\acex[\Jil]
{Noun}{Clause}{1486}
{ġzi-li o naša qem mexi-li təmmal}
{saw-1\sg{} \defi.\masc{} man \pst{} hit.\agent3\masc{}-\patient1\sg{} yesterday}
{I saw the man that hit me yesterday.}
{FoxJilu}{81}

 In \JSan (see \examples{43}{44}) this may be understood as part of the larger tendency to omit all AC markers in this dialect (see \sref{ss:Juxt_general_usage}). In the other dialects, mostly present in Turkey with the exception of \Amd, the source of this generalized construction is not clear.  It should be noted, moreover, that amongst these dialects, only \Boh makes use of the generalized \isi{juxtaposition} construction regularly and extensively. Since the speakers of \Boh lived through several stages of immigration, and came in contact with various languages \citep[3--5]{FoxBohtan}, the exact source (and time of appearance) of this construction is difficult to pinpoint.\footnote{For instance, currently \Boh speakers reside in Russia, and speak \ili{Russian} as well. As some varieties of Russian allow \isi{asyndetic relative clauses} \citep[397]{MurelliRelative}, this could theoretically be the source of the construction in \Boh. \label{ft:Boh_Russian}}

\subsection{Nominal \secns} \label{ss:juxt_nom}


\subsubsection{Quantification expressions} \label{ss:juxt_nom_quant}
\largerpage
In general, the \isi{juxtaposition} construction with nouns as \secns is less common in \ili{NENA}. Yet, h	ere again, we have to identify a special sub-type which reoccurs in several dialects, namely the case where the \prim quantifies the \secn, as in the following examples:

\acex[\Barw]
{Q. Noun Phrase}{Noun}{1391}
{[xa\cb{} reša] tuma}
{one\cb{} head garlic}
{one head of garlic}
{KhanBarwar}{494 {[B10:19]}}\antipar\antipar

\newpage 

\acex[\Arb]
{Q. Noun Phrase}{Noun}{1319}
{[tré tannakè]ˈ xiṭṭeˈ}
{two tins grain}
{two tins of grain}
{KhanArbel}{239 {[B:116]}}

These cases are in the borderline of the AC domain, as semantically the \secn is in fact the head of the expression (see discussion in \ref{ss:JUrm_quant}). Syntactically, we may prefer to analyse the \prim as a phrasal realisation of the quantifier slot, which we assume is generally available in the \ili{NENA} NP (see the \textsc{Quant} slot in  \vref{tb:NP_struc} and the preceding discussion). Given such an analysis, it is no surprise that no AC marking  is found. 

Another possibility is to relate it  to the \CArab \textarabic{تَمْيِيز} \transc{tamyīz} construction, in which a counted or measured noun appears in the \acc*, rather than \gen*, case \citep[157]{SchulzArabic}. In \CArab this construction is also used for the specification of material, and indeed, we find the last usage also in sporadic examples of \ili{NENA}:


\acex[\Arb]
{Noun}{Noun (material)}{1320}
{ṣĭqilyé dehwàˈ}
{rings gold}
{rings of gold}
{KhanArbel}{239 {[L:466]}}

While a direct influence of \CArab on \ili{NENA} might seem implausible,\footnote{One cannot  completely rule out an influence of written \ili{Arabic} on \ili{NENA}, as the Early J. \ili{NENA} homilies available to us (\Nrt) might have been redacted from Judeo-\ili{Arabic} sources \citep[201]{SabarArabic}. Be that as it may, \citet[202]{SabarArabic} notes that the \ili{Arabic} elements in these texts belong to the northern \Iraq \textit{qeltu} dialects, which are of course different from the standard \CArab.}
 the classical construction may  have been mediated through the vernacular \Iraq  dialect, which lost case markings but retained the general structure of this construction (though allowing optional marking of \cst* on the \prim).

\acex[\Iraq]
{Q. Noun}{Noun}{Iraqi1}
{ḥafna-\opt{t} timman}
{handful-\opt{\fem.\cst} rice}
{a handful of rice}
{ErwinIraqi}{375}

\acex[\Iraq]
{Noun}{Noun (material)}{Iraqi2}
{sāʿa ðahab}
{watch gold}
{a gold watch}
{ErwinIraqi}{375}


 Alternatively, we may postulate this as a common \ili{Semitic} feature, which is preserved in \ili{NENA} although not present in Syriac.\footnote{It is interesting to note that we find a similar construction in \MHeb. Thus, we find the ubiquitous colloquial example \foreign{\texthebrew{מנה פלאפל} mana falafel}{one portion of falafel}, in which no \cst* marking is present.}

Another alternative motivation is the fact that in some Kurdish dialects we also find  the \isi{juxtaposition} construction expressing quantification (the \concept{partitive relation} of \cite[63f.]{MacKenzie}). As discussed in \sref{ss:kurd_quant}, in \Kur dialects this usage seems to be quite limited, while in \Sor dialects it is widespread. The following example is from the dialect of Sulemaniyya (=\example{905}):




\acex[\KSul]
{Q. Noun Phrase}{Noun}{905bis}
{[yek hegbe] pare}
{one bag money}
{a bag of money}
{MacKenzie}{63 {[29]}}

Thus, this construction may reflect an areal phenomenon rather than a specifically \ili{Semitic} heritage. 

\subsubsection{General usage} \label{ss:Juxt_general_usage}






A more general usage of the \isi{juxtaposition} construction is found in two regions: Turkey and the Iraqi-Iranian border area (that is, outside the historical core of the \ili{NENA} dialects). In Turkey, only \Boh seems to use this construction regularly, in alternation with D-markers:

\acex[\Boh]
{Noun}{Noun}{1492}
{tara gumota}
{door stables}
{the door of the stables}
{FoxBohtan}{93}

\citet[92]{FoxBohtan} postulates that this construction \enquote{may be the result of complete assimilation followed by simplification of the resulting geminate cluster: *\textit{tarəd gumota} > *\textit{tarəg gumota} > \textit{tara gumota}.} The difficulty with this explanation is that it does not explain the restoration of the Aramaic \free* suffix \transc{-a}. One may try to save this explanation by suggesting that the \transc{d} marker assimilated to the following consonant when it was still procliticized to the \secn (i.e.\ before the emergence of the \cst* suffix \ed), yet it is unclear why this happened specifically in this dialect. Given the complex immigration history of the \Boh speakers (see end of \ref{ss:asyndetic_relatives} \vpageref{ss:juxt_nom}) we cannot preclude some unknown \isi{language contact} motivating this construction.

In the other \ili{NENA} dialects of Turkey, the construction is quite limited: In \Cal we find it in a few expressions (notably \foreign{yoma šapsa}{the Sabbath day}), and in \Her it occurs only when the \prim is a loan-noun without an Aramaic inflectional ending. In the latter dialect, given that many of the \Her examples given by \citet[26]{JastrowHertevin} have \prims of \ili{Arabic} origin, such as example \ref{ex:1735}, it may be more specifically a matter-cum-\isi{pattern replication} of the \ili{Arabic} CSC, in which the \prim is normally left unaltered (unless it is a feminine noun).

\acex[\Her]
{Noun}{Noun}{1735}
{šekl ḥa zalama}
{appearance \indef{} man}
{the appearance of a man}
{JastrowHertevin}{26 {[576]}}

\largerpage
A productive and extensive usage of the \isi{juxtaposition} construction is found in my survey only in \JSul and \JSan in the Iraqi-Iranian border area.\footnote{The same construction is also found in the Iranian J. Saqqəz dialect, which is however not included in my survey. \il{NENA!Saqqəz} \citet[11]{GoldenbergInnovative} brings the example \foreign{belá šultaná}{king's house} from this dialect.  
In Iraq we find this construction also in \Diy, but apparently  only with the \prim \foreign{šəmma}{name}: 
\acexfn[\Diy]
{Noun}{Noun}{1996}
{šəmma sawun-i}
{name grandfather-\poss.1\sg{}}
{my grandfather's name}
{NapiorkowskaDiyana}{315}\antipar
}  See \sref{ss:JSan_juxt} for examples of \JSan, and the following examples for \JSul (\ref{ex:1059}=\example{1059bis}). Note that, in these dialects, the \transc{-a} ending of native Aramaic nouns cannot be analysed any more as designating the \free*.

\acex[\JSul]
{Noun}{Noun}{1059}
{šəmma brona}
{name son}
{the name of the boy}
{KhanSulemaniyya}{192}\antipar 

\newpage 

\acex[\JSul]
{Noun}{Noun}{1910}
{brona mălək}
{son king}
{the son of the king}
{KhanGrammatical}{202}

It is not entirely clear how this construction developed. The limited geographical extent of this constructions points to a \isi{language contact} origin, possibly from \Sor. Indeed, \citet[202]{KhanGrammatical} suggests that this construction results from the identification of the compounding \ez* \transc{-e} (see \sref{ss:Comp_Ez}) with the Aramaic inflectional ending \transc{-a}.\footnote{Both are pronounced \phonetic[æ]. The difference in transcription results from my decision to use the Kurdish Latinized orthography for Kurdish dialects (see \vref{ft:Kur_transc}).}
Thus, he compares example \ref{ex:1910} with the following \KSul example:

\acex[\KSul]
{Noun}{Noun}{908}
{kuř-e- paşa}
{son-\ez- king}
{the king's son}
{MacKenzie}{64 {[25]}}

This idea, however, poses some difficulties. First, as discussed in \sref{ss:Comp_Ez}, the compounding \ez* creates a nominal compound consisting of the two members of the construction. Consequently, the bond between the two members  cannot be interrupted by other grammatical elements. For instance, a \dem* can envelop such a phrase, but not intervene in the middle, judging by the examples at my disposal:

\arabex[\Sor]
{Noun}{Noun}{825}
{ئام هۆتێلە باشە}
{em [hotêl-e- baş]-é}
{\dem.\near{} hotel-\ez- good-\dem}
{this good hotel}
{\citep[11]{ThackstonSorani}}

\largerpage
In \JSul, by contrast, we find intervening \dem*s and even complex \secns, as in the following example:

\acex[\JSul]
{Noun}{Noun Phrase}{1095}
{bába [ʾó\cb{} brona [ga\cb{} libl-á-le ḥajì]] \cb{}yele.}
	{father \dem\cb{} son \rel\cb{} took-\patient3\fem-\agent-3\masc{} haji \cb{}\cop.\pst}
{The father of the boy who took her away was a ḥāji.}
{KhanSulemaniyya}{261 {[R:146]}}\antipar 

\newpage 

Another difficulty relates to the fact that the compounding \ez* itself is not borrowed in \JSul, nor in \JSan. If it were borrowed, we would expect it to appear on non-Aramaic \isi{loanwords} (which normally lack an Aramaic inflectional ending) acting as \prims, but this is not the case. A scenario in which the Aramaic inflection ending is confounded with \ez* seems implausible without the borrowing of the actual morpheme. 



One could assume that the \ez* was quickly reanalysed as \zero\ marking in the  \ili{NENA} dialects concerned, and then extended to nouns not ending in \transc{-a}. Yet  in this case we would have to assume that the borrowed morpheme did not leave any trace. Indeed, in some neighbouring Kurdish dialects the compounding \ez* is omitted following nouns ending in certain vowels, such as \phonetic[a] or \phonetic[e] \citep[64]{MacKenzie}. This means that, even if it was borrowed and used following a native Aramaic noun, it would be realised as a \zero\ rather than confounded with the Aramaic inflectional ending.  This would provide an easy explanation for the \isi{juxtaposition} construction used with \pl* \prims ending in \transc{-e}:

\acex[\JSul]
{Noun}{Noun}{1062}
{bate Šlomo}
{houses Š.}
{the houses of Shlomo}
{KhanSulemaniyya}{192}

An alternative and arguably simpler hypothesis is  to directly relate the \ili{NENA} \isi{juxtaposition} construction to a similar unmarked  construction which is extant in some \Sor dialects, in particular \War, which is geographically close to Sulemaniyya.\footnote{The \War \isi{juxtaposition} construction seems to be most frequent with \prims marked by the \isi{indefinite suffix} \transc{-êk} \citep[62, fn.\ 2]{MacKenzie}, which may in some contexts replace the \ez* (see \ref{ss:Sor_alt_clause}, but it does not exclude unmarked definite \prims.} This is shown in the following  example (=\example{895}):

\acex[\War]{Noun}{Noun}{895bis}
{meł Ḥacî}
{house H.}
{the house of Haji}
{MacKenzie}{62, fn.\ 2 {[246]}}








Such cases, if frequent enough, may provide a seed for the \isi{juxtaposition} construction in \JSul, without positing a borrowing and reanalysis  of the compounding \ez* as equivalent to the Aramaic \free* ending, though such a possibility cannot be completely excluded.


On a broader view, as mentioned in \sref{ss:Kur_Juxt}, the \isi{juxtaposition} construction is possibly an areal phenomenon, as it also attested in \ili{Gorani} and in \Per. These languages may more easily account for the existence of the \isi{juxtaposition} construction in \JSan, due to their closer geographical proximity to the latter dialect.

 





\section{Matter replication}\label{ss:matter-replication}

The constructions discussed above all make use of native Aramaic morphemic material, irrespective of the question of whether the development of these structures was influenced by \isi{language contact}. In some dialects, however, we see clear \concept{borrowing} of morphemic material from \ili{Iranic} languages, termed by \citet{MatrasSakel} \concept{matter replication}. As \vref{tb:borrowed} shows, two types of morphemes are borrowed as grammatical markers of ACs: the \ez* marker (in two shapes: \transc{e} and \transc{i}) and a subordinating particle of the general form \phonemic{kV}. Moreover, the table makes clear that most of the matter replications took place in North-East Iraq and the Iranian regions. North-West Iraq is practically immune from this type of borrowing. This may indicate a more intensive contact situation in the former regions.

\begin{table}[p!ht!]
\centering
\begin{tabular}{l l c c}
\toprule

Region 	& Dialect			& Ezafe & Subordinator \\
\midrule
{South-East Turkey} & \Her 	& 		&	\\
					& \Boh 	& 		&	\\ 
					& \Bes 	&  		&	\\
					& \Gaz 	& (\transc{e}) & \\
					& \Baz  &  		&	\\
					& \Cal  & (\transc{e}, \transc{i}) & \\
					& \Jil  &  		&	\\
\midrule
North-West Iraq		& \JZax &  		&	\\
					& \JArd &  		&	\\
					& \CArd &  		&	\\
					& \Barw &  		&	\\
					& \Betn &  		&	\\
					& \Amd 	&  		&	\\
					& \Barz &  		&	\\
					& \Alq 	&  		&	\\
					& \Qar  &  		&	\\
\midrule
North-West Iran		& \JUrm & 				& \transc{ki-\opt{t}} \\
					& \Sar 	& 				& \transc{qäd}, \transc{či} \\
\midrule
North-East Iraq 	& \Rus  & \transc{i} 	& \\
					& \Diy	&				& \\
					& \Arb 	& (\transc{i}) 	& \\
					& \JKoy & 				& \transc{ka} \\
					& \JSul & (\transc{i}) 	& \transc{ga}\~\transc{ka} \\
					
\midrule
West Iran			& \JSan & \transc{e} & \transc{ke}, \transc{ya} \\
					& \CSan &  		&	\\
\bottomrule
\end{tabular}
\caption[Borrowed AC markers]{Borrowed AC markers. Parentheses indicate restricted or marginal use.} \label{tb:borrowed}
\end{table}

\subsection{Borrowing of the Ezafe} \label{ss:borr_ez}

Only in two dialects, \Rus and \JSan, is the \ez* truly generalized. It is probably no coincidence that these dialects have by and large lost the inherited D-marking of ACs. Judging moreover by the data of \JSan (\sref{ss:JSan_Ez}), it seems that the introduction of the \ez* is relatively recent, as its usage is still to some extent privileged with loan-expressions. Moreover, it is quite probable that the \ez*, which was first introduced into the language by way of loan-expressions, has expanded its usage domain due to the loss of the inherited D-marker.\footnote{For an elaboration of this idea in terms of \concept{forgetting} the \cst* marking, see \citet{GutmanForgetting}.} A case in point in \JSul, which shows an intermediate stage of development: It has abandoned to a large extent the D-markers in favour of a \isi{juxtaposition} construction (see \sref{ss:juxt_nom}), but the \ez* markers are still constrained to loan-expressions.

In the following subsections we survey the occurrence of the \ez* in the different geographical regions of the \ili{NENA}-speaking area. 

\subsubsection{North-East Iraq: \transc{i} Ezafe} \label{ss:i_ezafe}

Three out of five dialects in the North-East Iraq region exhibit usage of the \ez* particle \transc{-i}, in all probability borrowed from the co-territorial \Sor dialects \citep[cf.][408]{KhanRustaqa}.

\citet[169]{KhanArbel} speculates that the \transc{-i} morpheme is a reduction of the \transc{ay} demonstrative, which is used as a \lnk* in \JUrm, and as a \rel* in the Bible translation of \Ruw (see \sref{ss:JUrm_ay}). We maintain, however, that the two morphemes should not be confounded, not least since a reduction of \transc{ay} would normally yield an \transc{ē} vowel.\footnote{See in this respect also \vref{fn:Solduz_lnk}.} 

Another hypothesis given by \citet[168]{KhanArbel} is that the \transc{-i} segment results from the elision of the \transc{d} segment of the \cst* \ed suffix, following the assimilation of the latter to the initial consonant of the subsequent word (i.e.\ the \secn).\footnote{Ideally, an \phonemic{i} and an \phonemic{ə} should be distinct phonetically, but in lax pronunciation both may be produced as \phonetic[ɪ].} The fact that the \transc{i} morpheme can occur following an \ed suffix is counter-indicative of this idea. Yet the phonetic similarity between the vocalic nucleus of the \ed suffix (namely, the \phonemic{ə} segment) and the \phonemic{i} segment may have led bilingual speakers (of \Sor and \ili{NENA}) to perceive the \phonemic{ə} segment as being the \transc{-i} \ez*, thus enhancing the availability of the latter morpheme. Indeed, \citet[169]{KhanArbel} tentatively suggests such a link: \blockquote{It may be more than a coincidence, however, that \transc{-i} is also the \textit{izafe} particle in the Kurdish dialects of the region \citep[61--64]{MacKenzie} and this may have had an influence on the Neo-Aramaic form.} Weighing the (admittedly meagre) evidence, it seems that a purely internal phonological process cannot account for the distribution of the \transc{-i} suffix. Rather, it must have been initially borrowed from \Sor, and only subsequently could there be a reanalysis of a stranding \phonemic{ə} of the \ed suffix as the \ez*. 

\paragraph{\Rus} In this dialect the \ez* suffix has replaced the native D-markers, except following some interrogative pronouns discussed below \citep[408f.]{KhanRustaqa}. In most cases it is appended after the Aramaic \transc{-a} ending, but in some cases it replaces it, similarly to the native \ed suffix.\footnote{I have conducted some interviews with elderly \Rus speakers in Israel, November 2012. Due to the complexity of the material and time constraints I have not yet been able to transcribe it in full, so my data relies mostly on \citet{KhanRustaqa}. I could however verify the existence of the \ez* suffix.}  

\acex[\Rus]{Noun}{Noun}{1888} 
{ṣiwa-i xabuše}
{tree-\ez{} apples}
{an apple tree}
{}{(own fieldwork)}

\lex{\Rus}{1338}
{baxt-i Šlomo}
{wife-\ez{} Š.}
{the wife of Shlomo}
{KhanRustaqa}{409}

As in \Sor, the \ez* can appear before a clausal \secn:

\acex[\Rus]{Noun}{Clause}{1341}
{ʾo gora-y [timmal idye-le] (dost \cb{}e)}
{\defi{} man-\ez{} yesterday came-3\masc{} friend \cb{}\cop}
{The man who came yesterday (is my friend).}
{KhanRustaqa}{409}

\citet[409]{KhanRustaqa} does not mention cases of the \ez* mediating between a noun and an adjective. He does however give a case where the \ez* is used to nominalize an adjective:

\acex[\Rus]{\zero}{Adjective}{1343}
{(šáqil) i rabta}
{take.\imp{} \ez{} big.\fem{}}
{(Take) the big one!}
{KhanRustaqa}{409}

This usage is analogous to the independent \ez* in \Sor (see \example{922}).

The historical \d \lnk* is only conserved as apparent \cst* marking after some interrogative pronouns preceding clausal \secns, in which case it transforms them to indefinite pronouns: 


\acex[\Rus]{Pronoun}{Clause}{1344}
{manni-t [ade bel-an] (paṣix)}
{who-\cst{} come house-\poss.1\pl{} please.\subj.3\masc}
{Whoever comes to our house (will be pleased).}
{KhanRustaqa}{409}

\acex[\Rus]{Pronoun}{Clause}{1345}
{ma-t [kayf-ox made] (ʾol)}
{what-\cst{} pleasure-\poss.2\masc{} bring.3\masc{} do.\imp}
{(Do) what brings you pleasure!}
{KhanRustaqa}{409}

The fact that the erstwhile \cst* \transc{-d} suffix is conserved in this context hints that the \ed \cst* suffix was operative in a precursor of \Rus. The indefinite pronouns \foreign{mannit}{whoever} and \foreign{mat}{whatever} must have conserved this segment since they have been grammaticalised as such.\footnote{The same indefinite pronouns are found in other dialects as well. See \JZax \examples{303}{313} and \JUrm \examples{247}{248}.} Yet  the \phonemic{-d} segment does not operate any more as a true \cst* suffix, and indeed it can be followed by the \ez* suffix.

\acex[\Rus]{Pronoun}{Clause}{1346}
{manni-t-i abe (maṣe hade)}
{who-\cst-\ez{} \subj.want.3\masc{} can.3\masc{} come.\subj.3\masc{}}
{Whoever wants (can come).}
{KhanRustaqa}{409}

\acex[\Rus]
{Pronoun}{Clause}{1347}
{ma-t-i abet (ʾol)}
{what-\cst-\ez{} \subj.want.2\masc{} do.\imp}
{(Do) what you want!}
{KhanRustaqa}{409}

Additionally, we find the historical \transc{did-} pronominal base. In the following example it appears after the \MHeb\ loan-expression \foreign{\texthebrew{עולה חדש} ʿole ḥadaš}{new immigrant to Israel}.

\acex[\Rus]
{Noun Phrase}{Pronoun}{1889}
{hole hadaš did-an}
{immigrant new \gen-1\pl}
{our new immigrants}
{}{(own fieldwork)}

\paragraph{\Arb}

In \Arb, we find the \transc{i} \ez* virtually only in the speech of one informant of \citet{KhanArbel}, originating from the town of Batas (50 km north-east of Arbel). Occasionally, the \ez* appears after the native \ed suffix, as in the following example:

\acex[\Arb]{Noun}{Noun}
{1241}
{kullà mamlakát-it \cb{}i\footnotemark{} Kurdistán}
{all towns-\cst{} \ez\cb{} K.}
{all the towns of Kurdistan}
{KhanArbel}{169 {[B:146]}}

\footnotetext{Khan transcribes this example with the \ez* \transc{i} attached to \transc{Kurdistán}. Listening to the example, it sounds to me rather syllabified with the preceding word.}

More frequently, however, it replaces it:

 
\acex[\Arb]{Noun}{Noun}
{1237}
{kolā́n-ĭ mšilmāne}
{street-\ez{} Muslims}
{the streets of the Muslims}
{KhanArbel}{168 {[B:47]}}

\acex[\Arb]{Noun}{Pronoun}
{1269}
{ʾízl-ĭ díd-i}
{wool-\ez{} \gen-1\sg}
{my wool}
{KhanArbel}{219 {[B:127]}}

\acex[\Arb]{Noun}{Adjective}
{1234}
{b\cb{} ṣalm-ĭ́  komé}
{in\cb{} face(\pl)-\ez{} black.\pl}
{with a dark face}
{KhanArbel}{229 {[B:111]}}


\citet[168]{KhanArbel} attributes the latter occurrences to the elision of the \phonemic{d} segment of the \ed suffix, and transcribes the ending as <\transc{ĭ}> which phonetically should be understood as \phonetic[ə] or \phonetic[ɪ]. As discussed at the introduction of \sref{ss:i_ezafe} we prefer to treat all these cases as borrowing of the \ez*.\footnote{Listening to the available recordings, moreover, I could not hear a clear difference between the \transc{-ĭ} and \transc{-i} suffixes. Thus, in \citet[540 {[B:145]}]{KhanArbel} we find the expression \foreign{b-dáwr-ĭ Pā́ša-i Kòra}{In the time of P. K.}. To my ear, the two \transc{-i} suffixes sound identical, notwithstanding the fact that the first one replaces an \transc{-a} ending. Note the second one is clearly a Kurdish \ez* as it is part of a Kurdish proper noun.} The co-occurrence of the \ez* with an adjective (\ref{ex:1234} above) is typical of the \ili{Iranic} construction, but extant in \Arb also with the native \ed suffix (see \example{1230}).


Note also the following example, where Khan  analyses the \transc{-i} as the \ez*, most probably because the \transc{-i} suffix does not replace the \prim's \transc{-a} ending:\footnote{This example is exceptional also in that it is produced by another informant, resident of Girdmāla, 20 km south of Arbel.}


\acex[\Arb]{Pronoun}{Noun Phrase}
{1257}
{hemà-i [xà ṭpurt-it hulaʾà]}
{which-\ez{} \indef{} fingernail-\cst{} Jew}
{whatever fingernail of a Jew}
{KhanArbel}{170 {[Y:182]}}

\paragraph{\JSul} In \JSul the \ez* seems to occur only after \isi{loanwords}, both nouns and prepositions \citep[192--193]{KhanSulemaniyya}.\footnote{The preposition \foreign{báyn-}{between} appearing in \ref{ex:1172} is listed by \citet[598]{KhanSulemaniyya} as originating in Kurdish. Of course, it must be ultimately borrowed from \Arab \transc{\textarabic{بَيْنَ} bayna}. While Aramaic has a cognate preposition \transc{ben} the diphthong \phonemic{ay} seems to indicate a foreign origin, or at least a merger of the two.} 

\acex[\JSul]{Noun}{Noun}{1078}
{ḥukmát-i ʿIráq}
{government-\ez{} I.}
{the government of Iraq}
{KhanSulemaniyya}{192 {[A:5]}}

\acex[\JSul]{Noun}{Pronoun}{253}
{maktab-i did-an}
{school-\ez{} \gen-1\pl}
{our school}
{KhanSulemaniyya}{253}

\acex[\JSul]
{Adverbial}{Noun}{1172}
{ga-báyn-i ʾo\cb{} guḏá \cb{}w ʾo\cb{} loʾà}
{in-between-\ez{} \defi\cb{} wall \cb{}and \defi\cb{} room}
{between the wall and the room}
{KhanSulemaniyya}{214 {[V]}}

As discussed in \ref{ss:borr_ez}, the usage of the \ez* only with loan-expressions in these dialects may indicate that in general the \ez* found in \ili{NENA} dialects was imported through loan-expressions, and only subsequently its usage was extended to native \prims in some dialects, such as \JSan discussed in the next paragraph.\footnote{It is interesting to note that also in the Turkic languages of Iran the \ili{Persian} \ez* is normally borrowed only as part of \ili{Persian} expressions, and only rarely with native Turkic words. See \citet{Kiral} for a discussion.}

\subsubsection{West Iran: J. Sanandaj} \label{ss:borr_ez_JSan}

Among the sampled dialects of Iran, we find the \ez* only in \JSan, as detailed in \sref{ss:JSan_Ez}.\footnote{\citet[171, \S 2.32.12]{Garbell1965impact} mentions the usage of the \Sor \ez* \transc{i} in \Sol, but we have no further information on this. See \vref{fn:Solduz_lnk}.} The form of the \ez* \transc{-e} is indicative of its \Per origin, as well as its typical occurrence inside \Per phrases. Its usage, however, has been extended beyond the domain of fixed \Per phrases, as discussed there.

\subsubsection{South-East Turkey}

In South-East Turkey, the usage of the \ez* is sporadically attested, in very particular usage. Thus, in \Cal we find the following loan-phrase. While the nouns are of \ili{Arabic} origin, the usage of the \ez* indicates the expression must have been borrowed from \Kur.

\acex[\Cal]
{Noun}{Noun}{1704}
{ʾawlād-e rasū́l}
{children-\ez{} Messenger}
{descendant of the Messenger}
{FassbergChalla}{56}

A possible productive use can be found in the following example:

\acex[\Cal]
{Adverbial}{noun}{1705}
{tuxm-i xalwa la pāyəš go xədyawás did-u}
{kind-\ez{} milk \neg{} remain in breasts.\cst{} \gen-3\pl}
{No trace of milk remains in their breasts.}
{FassbergChalla}{56, fn.\ 46}

An interesting development is presented by the \ili{Judi-dialects}, where the \ez* ending has been grammaticalised to become a lexical ending meaning \transl{descendant of}.\footnote{I am grateful to Joseph Alichoran, for pointing this out for me.} This indicates that the \ez* was only borrowed as part of proper names in these dialects.

\acex[\Gaz]
{Noun}{Noun}{1830}
{Yaqo-ye Musa}
{Y.-\ez{} M.}
{Yaqo son of Musa}
{GutmanGaznax}{317 (25)}


\subsection{Borrowing of subordinating particles} \label{ss:borrow_ke}

\subsubsection{North-East Iraq}

In \JSul and \Koy we find the \Sor subordinator \transc{ka} borrowed. The \Sor particle can act both as a \rel* and as a \comp* (much like the \ili{NENA} \lnk* \transc{d-}\~\transc{did}), and it appears in the two roles also in the recipient \ili{NENA} dialects. In \Koy it is borrowed simply as \transc{ka}, while in \JSul it often appears as \transc{ga}.\footnote{The reason for this shift is not clear. It may stem from an analogy to the existing preposition \foreign{ga}{in}.}  In both dialects,  it can co-occur as a \isi{relativizer} with any other available AC marking  (such as the \Koy \lnk* \transc{od} or the \JSul \cst* ending).

\acex[\Koy]
{Noun}{Clause}{1508}
{ʾo ʿaqubrona ʾod ka xlíṣwāle}
{\dem.\far.\sg{} mouse \lnk{} \rel{} saved-\pst-3\masc}
{that mouse who had been saved}
{MutzafiKoySanjaq}{63 {[\N20]}}

\acex[\JSul]
{Noun}{Clause}{1188}
{bába ʾó\cb{} brona ga\cb{} libl-á-le ḥajì \cb{}yele.ˈ}
{father \dem\cb{} son \rel\cb{} took-\patient3\fem-\agent3\masc{} ḥaji \cb{}\cop.\pst.3\masc}
{The father of the boy who took her away was a ḥāji.}
{KhanSulemaniyya}{414 {[R:146]}}

\acex[\JSul]
{Pronoun}{Clause}{1197}
{ʾó-d ga\cb{} k-imr-án-wa zarandà \cb{}y}
{\dem-\cst{} \rel\cb{} \ind-say-1\sg-\pst{} tough \cb{}\cop}
{the one whom I was just saying was tough}
{KhanSulemaniyya}{418 {[R:135]}}

As a \comp* we find the following example:

\acex[\JSul]
{Verb}{Clause}{1211}
{kăyén-wa ga\cb{} ʾó bratá il\cb{} d-o\cb{} bróna gbà.ˈ}
{know.3\pl-\pst{} \comp\cb{} \dem{} girl to\cb{} \gen-\dem\cb{} boy want.3\fem}
{They would know that the girl loved the boy.}
{KhanSulemaniyya}{440 {[R:29]}}

\subsubsection{North-West Iran}

In North-West Iran we find various forms of the \rel* which may be borrowed from \Kur \transc{ḳu}\footnote{This is the form of standard \Kur, but the Urmia dialect may show some variation.}, from \Per \transc{\textarabic{كه} kæ} or from \Azr \transc{ki}. \citet[180]{YounansardaroudSardarid} relates the \Sar form \transc{či} to the \ili{Persian} \rel* \transc{kæ},\footnote{It is not clear  whether she relates the form \transc{qad} as well to a \ili{Persian} origin.} while \citet[172, \S 2.32.1.(5)]{Garbell1965impact} relates the \JUrm form \transc{ki} to the \Azr source. The only evidence for a \Kur influence may be the use of the uvular \phonemic{q} segment in \Sar \transc{qad}, which may correspond to the unaspirated \transc{ḳ} of the \Kur \rel*. Yet  it may also be related to preposition \foreign{qa, qāt}{for} present in the dialect.

The \phonemic{d} segment of \Sar \transc{qäd} is probably a retention of the Aramaic \d, and can be analysed as an explicit \cst* marking of the \rel*. This marking can be optionally found in \JUrm as well, in the form \transc{kit} (see \sref{ss:JUrm_rel_cst}).

Various examples of the \rel* in \JUrm are given in \sref{ss:JUrm_Rel}. The usage in \Sar is quite similar. Note that it can optionally co-occur with \cst* marking on the \prim:

\acex[\Sar]
{Noun}{Clause}{1766-7}
{⁺o čtav-\opt{əd} qad/či at zvin-ət len mačuḥa}
{\defi{} book-\opt\cst{} \rel{} 2\masc{} bought-2\masc{} \neg.\cop.1\sg{} found}
{The book that you bought, I cannot find (it).}
{YounansardaroudSardarid}{181}

\subsubsection{West-Iran}

In West-Iran we find the \rel* \transc{ke} only in \JSan, and not in adjacent \CSan. In this case, a \Per as well as a \Sor origin is possible. Given the great impact of \Per on this dialect, the former option seems preferable. Alongside the \transc{ke} \rel* we find also a \transc{ya} \rel*, which may be related to the \Per \ez*, the latter sometimes appearing with an initial glide. We note indeed that while \transc{ke} is compatible with the \prim-marking \ez*, this is not the case with \transc{ya}. Various usage examples of both are given in \sref{ss:JSan_rel}.


\section{Case study: The marking of ordinal numbers}\label{ss:case-study}

To show the diversity of \ili{NENA} dialects in a nutshell, it is illuminating to study a system on the fringes of the AC system, namely the qualification of nouns by ordinal numbers. Ordinal numbers (\transl{first}, \transl{second}, etc.) are akin to adjectives in many languages, in contrast to the cardinal numbers (\transl{one}, \transl{two}, etc.) which relate directly to quantification. In \ili{Semitic} languages, and in \ili{NENA} in particular, \isi{ordinals} show a special behaviour, mixing characteristics of adjectives with those of nominal attributes.\footnote{For example, in the Biblical \ili{Hebrew} phrase \foreign{\texthebrew{יום השישי} yōm haš-šišši}{the sixth day} (Genesis 1:31), the ordinal \transc{šišši} agrees in gender and number with the head \transc{yōm} as an adjective, but the placement of the \isi{definite article} \transc{ha-} is typical of the \ili{Hebrew} CSC. Incidently, a similar construction is found in \Iraq; see \example{2016}.} 
The \ili{NENA} ordinal system is interesting in that it conserves some characteristics of ancient strata (as shown by its affinity to Syriac), while at the same time it shows similarities to contact languages.

As the ordinal \transl{first} behaves specially in some respects, it deserves a separate discussion in the next section, followed by a treatment of the higher \isi{ordinals}. 

\subsection{The ordinal \transl{first}}

A particularity of the ordinal \transl{first} in \ili{Semitic} languages is that every \ili{Semitic} language-branch has a unique form to express it, and it is thus not part of a shared \ili{Semitic} heritage, but rather an independent innovation of each branch \citep{LoewenstammFirst}. The ordinal \transl{first} shows often, moreover, a special morpho-syntactic behaviour as compared to the other numerals, and \ili{NENA} is no exception. 

In many \ili{NENA} dialects we find an adjectival ordinal \foreign{qamāya}{first}, related to
 \Syr \textsyriac{ܩܲܕ݂ܡܵܝܵܐ} \transc{qaḏmāyā}. Like other adjectives, it usually appears after the qualified noun and agrees with it in number and gender, as in the following example (=\example{1845}).

\acex[\Alq]
{Noun}{Ordinal (first)}{1845bis}
{yóma qamā́ya}
{day(\masc) first.\masc}
{the first day}
{CoghillAlqosh}{293 {[A:137]}}

This ordinal can be nominalized like other adjectives, namely by putting a determiner before it:

\acex[\CArd]
{\zero}{Ordinal (first)}{1794}
{aw qamāya}
{\defi.\masc{} first.\masc{}}
{the first}
{KrotkoffAradhin}{22 {[112]}}

Some dialects have borrowed the \Arab ordinal \foreign{\textarabic{أَوَّل} ʾawwal}{first}. 
In \ili{Arabic}, both the written standard and the \Iraq vernacular, this ordinal (like others) can appear after the qualified noun as an inflecting adjective, or preceding the qualified noun while forming with it a CSC, invariably in the \masc\ form \parencites[224]{SchulzArabic}[366f.]{ErwinIraqi}.\footnote{To be sure, in the written standard language the ordinal \transl{first} can inflect in the pre-nominal position as well, but the \masc\ form is frequently used with disregard for gender agreement \citep[271]{BadawiCarter}.}
The adjectival usage of \transc{ʾawwal} has not been adopted in \ili{NENA}, but in several \ili{NENA} dialects we find the uninflecting ordinal \transc{ʾawwal} (or a similar form) preceding the qualified noun, mimicking the structure of the \ili{Arabic} CSC. 

\acex[\JArb]
{Noun}{Ordinal (first)}{1263}
{ʾáwwal baxtá}
{first woman}
{the first woman}
{KhanArbel}{181}

\citet[92, fn.\ 102]{FassbergChalla} notes that the form \transc{ʾawwal} is attested in {Palestinian Aramaic}\il{Aramaic!Palestinian} since the Middle Ages, and also in \ili{NENA} it is found in the earliest strata, namely \NrT, as shown in the examples below. Note, however, that these are very particular occurrences, having in common the \isi{adverbial} meaning of \transl{first time}.\footnote{\citet[92, fn.\ 102]{FassbergChalla} apparently did not consult the actual examples but only the occurrence of \transc{ʾawwal} in the glossary of \citet[248]{SabarMidrashim}. For this reason he relates it to the somewhat different construction in example \ref{ex:1710}. Note that a similar restricted use of \transc{ʾawwal} is found also in contemporary \ili{NENA}, such as in \Alq, where it is found only in the expressions \foreign{ʾáwwal-ga}{the first time} and \foreign{ʾáwwal-məndi}{firstly} \citep[284]{CoghillAlqosh}.}

\largerpage
\hebacex[\NrT]
{Noun}{Ordinal (first)}{1913}
{אאהין וילא אול ג̇אר דפלטלא מנד קוותי}
{ʾāhin we-la ʾawwal jār d\cb{} pləṭ-la mənn-əd  quwətt-i}
{3\fem{} \cop.\pst-3\fem{} first time \lnk\cb{} left-3\fem{} from-\cst{} strength-\poss.1\sg}
{It was the first time that it came out of my potency.}
{(\textit{Midraš Parašat Wayḥi} 29:3 ed.\ by \cite[85, line 27]{SabarMidrashim})}\antipar 

\hebacex[\NrT]
{Adverbial Clause}{Ordinal (first)}{1914}
{אוול מאד מירתי}
{ʾawwal ma-d mīrr-ət-ti}
{first what-\cst{} said-\patient2\masc-\agent1\sg}
{at first when I told you}
{(\textit{Midraš Parašat Wayḥi} 27:29 ed.\ by \cite[51, line 2]{SabarMidrashim})}\antipar
\newpage


In vernacular \Iraq we find additionally a construction in which the ordinal appears as a \secn of the  CSC. 

\acex[\Iraq]
{Noun}{Ordinal}{2016}
{yōm il-ʾawwal}
{day(\masc) \defi-first.\masc}
{the first day}
{ErwinIraqi}{367}

\acex[\Iraq]{Noun}{Ordinal}{2017}
{san-t il-ʾūla}
{year-\fem.\cst{} \defi-first.\fem}
{the first year}
{ErwinIraqi}{367}

A similar construction is found in some \ili{NENA} dialects, but the ordinal \transc{ʾawwal} never inflects.  Both in \Iraq and in \ili{NENA} this construction is in fact not limited to the ordinal \transl{first}.

\acex[\Cal]
{Noun}{Ordinal (first)}{1710}
{yom ʾawwal}
{day.\cst{} first}
{the first day}
{FassbergChalla}{92}



\acex[\JArb]
{Noun}{Ordinal (first)}{1260}
{tré\cb{} yom-it ʾàwwal}
{two\cb{} day-\cst{} first}
{two first days}
{KhanArbel}{181 {[B:72]}}


While this construction seems thus to be a pattern-cum-\isi{matter replication} from vernacular \ili{Arabic}, we find in \JArb the Kurdish \Kur form  \transc{ʾawwalí} alongside \transc{ʾáwwal} \citep[cf.][181]{KhanArbel}.\footnote{In Kurmanji orthography it is written \transc{ewel(î)}. As discussed in \sref{ss:kurd_ez_ord} the final \transc{-î} ending probably reflects a lexicalised \Kur \obl* ending.}  

\acex[\JArb]
{Noun}{Ordinal (first)}{1259}
{gor-it ʾawwalí}
{man-\cst{} first}
{the first man}
{KhanArbel}{181}

The above example may further hint at a \Kur influence, if we consider the \ed suffix to be the functional  equivalent of the Kurdish Ezafe, since in Kurdish (both \Kur and \Sor) \isi{ordinals} follow the \ez* (see \ref{ss:kurd_ez_ord}). Instead of arguing decisively in favour of one option or another, it seems that similarly to  other domains of the AC system, this pattern (head-marked \prim followed by an ordinal) represents an areal phenomenon (see further discussion of this in the next section).
 
 Finally, in the dialects of \JSul and \JSan we find the \ili{Arabic} ordinal \transl{first} (borrowed through \Sor \transc{(h)eweḷ}\footnote{In \Sor we find the forms \transc{\textarabic{ئەوەڵ} ʾeweḷ} and \transc{\textarabic{هەوەڵ} heweḷ} \citep[48]{HakemSorani} alongside the regularly formed \transc{\textarabic {يەكەم} yêkem} \citep[18]{ThackstonSorani}.}) following the noun without any further marking.\footnote{In \JSan an \ez* may intervene, or alternatively  the ordinal may precede the head. Compare examples \vref{ex:35}, \vref{ex:36} and \vref{ex:276}.} This is expected in these dialects, given their widespread usage of the \isi{juxtaposition} construction (see \sref{ss:loss_marking}).
 
 \acex[\JSul]
	 {Noun}{Ordinal (first)}{1213}
 {gorá hawwál}
 {first man}
 {the first man}
 {KhanSulemaniyya}{277}
 
 In \JSul, but not in \JSan, a native inflecting ordinal \foreign{qamayna}{first.\masc} may be used instead \citep[206]{KhanSulemaniyya}.
 
 \subsection{Higher ordinals} \label{ss:NENA_high_ordinals}
 
In most \ili{NENA} dialects, the remaining \isi{ordinals} are formed  by putting a cardinal \isi{numeral} after a \cst* noun or a \lnk*. In some dialects the \isi{numeral} agrees, moreover, with the \isi{head noun} in number and gender (at least for the numerals 2--10).\footnote{In general, this happens in the dialects which have conserved a gender distinction in the cardinal system.} This system is clearly inherited from prior strata of Aramaic, as it appears in Syriac as well (see \sref{ss:syr_lnk_ord}).

\largerpage
Recall that in Syriac the \isi{numeral} must appear after the \lnk* \d, as described in \sref{ss:syr_lnk_ord}. In \ili{NENA} dialects, however, we find the numerals after different kinds of \lnk*s, as well as the Neo-CSC (\ed suffix) and apocopate CSC. Thus, in \ili{NENA} we see a generalisation of the Syriac pattern in that all types of \cst* heads, irrespective of their forms, can govern a \isi{numeral} acting as an ordinal. This is shown in the following examples (\ref{ex:307bis}=\example{307}).

\acex[\Cal]
{Noun}{Ordinal}{1711}
{yom tre}
{day.\cst{} two}
{the second day}
{FassbergChalla}{92} 

\acex[\Cal]
{Noun}{Ordinal}{1686}
{yarx-əd ʾarba}
	{month-\cst{} four}
{the fourth month}
{FassbergChalla}{45}

\acex[\JZax]
{Noun}{Ordinal}{307bis}
{ē baxta dīd ṭḷāha}
{\defi.\fem{} woman \lnk{} three}
{the third wife}
{CohenZakho}{95 (8)}

The closest resemblance to \Syr is shown by the \Qar dialects, where an agreeing \isi{numeral} is preceded by the \lnk* \d, as in the following example (=\example{511}). Note that \Qar has also borrowed an \ili{Arabic} pattern; see \example{696bis}.

\acex[\Qar]{Noun}{Ordinal}{511biss}
{báxta d\cb{}  tə́ttə}
{woman \lnk\cb{}  second.\fem}
{the second woman}
{KhanQaraqosh}{225}

As a \lnk* is pronominal, it suffices to nominalize the ordinal, yet it is often preceded by a determiner. Consider the following two \Alq examples:

\acex[\Alq]
{Noun}{Ordinal}{1845}
{yóma qamā́ya lɛ̀θ \cb{}uˈ \textbf{de\cb{}} \textbf{trḗ} lɛ̀θ \cb{}uˈ}
{day(\masc) first.\masc{} \neg.\exist{} \cb{}and \lnk\cb{} two.\masc{} \neg.\exist{} \cb{}and}
{The first day there was nothing, and the second (day) there was nothing.}
{CoghillAlqosh}{293 {[A:137]}}

\newpage 
\acex[\Alq]
{\zero}{Ordinal}{1844}
{ʾɛ\cb{} t\cb{} xámmeš}
{\defi.\fem{} \lnk\cb{} five.\fem}
{the fifth one}
{CoghillAlqosh}{293 {[A]}}


See also the similar \JZax \examples{306}{348}.

While these constructions clearly show continuity with Syriac, they are also   structurally similar to constructions in neighbouring languages. As we saw in \examples{2016}{2017}, the \Iraq CSC is also used with inflecting \isi{ordinals} as \secns. 

\acex[\Iraq]
{Noun}{Ordinal}{2018}
{marr-t iθ-θānya}
{time-\fem.\cst{} \defi-second.\fem}
{the second time, the next time}
{ErwinIraqi}{367}

Note that in contrast to the \ili{NENA} and Syriac constructions, the \secn is a true adjectival ordinal, distinct from the corresponding cardinal form, e.g.\ \foreign{θnēn}{two}  \citep[see][268]{ErwinIraqi}.

The \ili{NENA} construction is, moreover, similar to the construction found in the neighbouring \ili{Iranic} languages. Thus, both in \Kur and \Sor, as well as in \Per, we find the ordinal numbers follow an \ez* marked \isi{head noun}, which is functionally equivalent to a \cst* noun in Aramaic. Similarly to example \ref{ex:1845}, the \isi{ordinals} can also be nominalized by appearing after an independent \ez* (\reex{750}; \reex{924}; \reex{808}): 

\acex[\Kur]{Noun}{Ordinal}{750bis}
{roj-a sisi-yan}
{day-\ez.\defi.\fem{} three-\obl.\pl}
{(on the) third day}
{ThackstonKurmanji}{25}

\acex[\KSul]{Noun}{Ordinal}{924bis}
{řêga-y sê-hem}
{third-\ez{} three-\ord}
{the third road}
{MacKenzie}{72 {[47]}}

\acex[\Ak]{\zero}{Ordinal}{808bis}
{yê dw-ê ... yê sê-yê}
{\ez.\masc{} two-\ord{} ... \ez.\masc{} three-\ord}
{the second... third one}
{MacKenzie}{163 {[562]}}

Morphologically, the Kurdish numerals show a number of differences in comparison to native Aramaic numerals. First, in accordance with some \ili{NENA} dialects but in contrast with the Syriac construction and the more conservative \ili{NENA} dialects, the numerals do not show gender or number features. In contrast to all \ili{NENA} dialects and the \isi{ordinals} appearing in the Syriac \lnk* construction, the \isi{ordinals} are clearly marked as distinct from the corresponding cardinals by a dedicated suffix.\footnote{As discussed in \sref{ss:kurd_ez_ord}, in standard \Kur the suffix \transc{-yan} represents the \pl* \obl* case suffix, which is grammatically expected due to the syntactic position of \isi{ordinals} following the \ez*. In the dialectal data (as in \Ak above), \citet[170]{MacKenzie} reports rather on the usage of the suffix \transc{-ê}, which is possibly related to the superlative suffix (see \vref{ft:superlative_e}). In \Sor, on the other hand, the \obl* case is no longer productive, and the suffix \transc{-hem}, while historically possibly related to the \pl* \obl* \isi{case marker}, must be seen as an ordinal derivational suffix. }  In this respect they show a certain affinity with the Syriac adjectival \isi{ordinals} which are marked by the  derivational suffix \transc{-āyā}, but recall that these \isi{ordinals} are used in the adjectival \isi{juxtaposition-cum-agreement} construction (see \example{1047}).

Given the partial similarities with Syriac, vernacular \ili{Arabic}, as well as with Kurdish dialects, should the \ili{NENA} construction be related to internal developments or to \isi{language contact}? Some authors prefer the latter option. \citet[172]{Garbell1965impact} thus asserts that the \ili{NENA} construction is an \enquote{exact parallel to the K[urdish] construction.}  \citet[214]{Noorlander} claims that the Syriac \lnk* construction was used only in a \enquote{chronological sense} (i.e.\ for enumerating days, months etc.) and that in \ili{NENA} its scope was \enquote{extended and [it] ultimately replaced the originally [Syriac-like] productive ordinal adjectives [...] most likely due to contact with Kurdish.} Yet,  just as in the case of the adoption of the Neo-CSC suffix \ed (see \sref{ss:role_contact}), we cannot be sure what  the precise role of \isi{language contact} was, or even what the direction of borrowing was: from Kurdish to Aramaic or vice versa (or both, in different periods). Even the disappearance of agreement features of the numerals, which may be attributed to a Kurdish origin, reflect an areal (and possibly universal) tendency to erode inflectional features with time. 

While \isi{pattern replication} is thus difficult to ascertain, we find clear cases of matter borrowing, namely the borrowing of the \Sor derivational affix \transc{-emîn}. In \Sor, an ordinal thus marked must precede the \isi{head noun}, as in the following example (=\example{1915}):\footnote{As discussed in \sref{ss:Kurd_inv_juxt}, this  suffix is composed of two parts: \transc{-em}, the ordinal suffix proper, and \transc{-în},  which is normally used to form superlative adjectives. Superlative adjectives thus marked precede their \isi{head noun}. \textit{In passim}, we note here two further areal phenomena: 1) The relationship between superlative form and \isi{ordinals} exists also in \ili{Arabic}, though only for the ordinal \transl{first}. 2) The positioning of superlative adjectives before the qualified noun is found also in \ili{Arabic}, as well as in \ili{NENA}; see \JZax \examples{335}{465} and \ref{ft:JZax_superlative} there. \label{ft:superlative_areal}} 


\acex[\KSul]
{Noun}{Ordinal}{1915bis}
{yek-em-în jar}
{one-\ord-\supr{} time}
{for the first time}
{MacKenzie}{73}

The suffix \transc{-em(in)} is also found in \Per, where it behaves similarly to \Sor \citep[262]{BalayEsmaili}, and is in fact also borrowed into \Kur \citep[25]{ThackstonKurmanji}, but without changing the post-nominal position of the ordinal.
  

Some \ili{NENA} dialects (\JArb, \JKoy, \JSul, \JUrm and \JSan) have  borrowed the \Sor ordinal suffix as \transc{-min}, either directly from the latter or through \Kur \citep[see also][166, \S 1.22.3]{Garbell1965impact}. In  \JUrm and \JSan the \isi{ordinals} thus formed can appear before the qualified noun (see examples \vref{ex:282bis} and \vref{ex:133}),\footnote{In \JUrm we find the suffix \transc{-mənji}. \citet[166]{Garbell1965impact} explains the ending \transc{-ji} as a further ordinal suffix borrowed from \ili{Azeri}.}   while in \JArb and \JKoy the \isi{ordinals} must follow a \cst* noun or a \lnk*, thus showing greater affinity with the \Kur construction, as well as with the native Aramaic construction.\footnote{In \JUrm the ordinal suffix may be absent when numerals higher than ten are used, and then the \isi{numeral} must appear in the CSC or ALC \citep[187]{KhanUrmi}.}

\acex[\JArb]
{Noun}{Ordinal}{1261}
{báxt-ət tre-mín}
{woman-\cst{} two-\ord}
{the second woman}
{KhanArbel}{181}

\acex[\Koy]
{Noun}{Ordinal}{1516}
{yā́ləd ʾod tre-min}
{child \lnk{} two-\ord}
{the second child}
{MutzafiKoySanjaq}{168}


In \JSan we find in fact both the pre-nominal and post-nominal positions for derived \isi{ordinals}, the latter optionally following an \ez*. Although arguably \JSan is the \ili{NENA} dialect most influenced by \ili{Iranic} \isi{language contact}, it permits the optional suffixation of an Aramaic inflectional suffix on top of the \ili{Iranic} derivational suffix (\reex{282}; \reex{281} and \example{279}). 

\acex[\JSan]
{Noun}{Ordinal}{282bis}
{tre-min-\opt{ta} baxta}
{two-\ord-\opt{\fem} woman}
{the second women}
{KhanSanandaj}{213}

\acex[\JSan]
{Noun}{Ordinal}{281bis}
{baxtá-\opt{e} tre-min-\opt{ta}}
{woman-\opt{\ez} two-\ord-\opt{\fem}}
{the second woman}
{KhanSanandaj}{213}


The \JSan ordinal is thus adjectival in nature and resembles the Syriac adjectival \isi{ordinals} structurally, notwithstanding the use of a loan-morpheme. When following the \ez*, it approaches moreover the Syriac ordinal \lnk* construction; recall, however, that in Syriac the \secns of the latter construction are inflecting cardinals (which have a special inflection pattern), rather than the adjectival \isi{ordinals}. 

Finally, in  \Qar, being in close contact with the \ili{Arabic}-speaking city of Mosul, we find a pattern-cum-\isi{matter replication} of the \ili{Arabic} CSC headed (syntactically) by ordinal adjectives. Compare the following two examples (\reex{696}):

 
 \acex[\Qar]{Noun}{Ordinal}{696bis}
 {ʾu\cb{} θáləθ yóma}
 {and\cb{} third day}
 {the third day}
 {KhanQaraqosh}{640 {[F:72]}}
 

 \acex[\Iraq]
 {Noun}{Ordinal}{2021}
 {θāliθ šahar}
 {third.\masc{} month}
 {the third month}
 {ErwinIraqi}{368}
 
\newpage 
Note  that both in the \ili{Arabic} vernacular and in the \Qar example the ordinal is invariably in the \masc\ form, and also that  this construction yields a definite meaning, although no definite marker is present.





\subsection{Case study conclusions} \label{ss:case_study_conclusions}

To conclude, in the sub-system of ordinal numbers the \ili{NENA} dialects exhibit constructions that  continue classical Aramaic strategies while  resembling patterns from contact languages. The interaction between these two sources lead to high dialectal variation, each dialect showing a unique combination of constructions and features, as shown in \vref{tb:ordinals}.




























\begin{table}[h!]
\centering
\begin{tabular}{llcccc}
\toprule
\multicolumn{2}{c}{} & N Ord			& Ord N	& N.\cst\ Ord	& \lnk\ Ord \\
\midrule
\multicolumn{2}{c}{\Syr}			& \ord.\agr	& 		&				& \agr		\\
\multicolumn{2}{c}{\Iraq}			& \ord.\agr	& \ord	&	\ord.\agr	& 			\\
\multicolumn{2}{c}{\Kur}			& 				& 		&	\ord		& \ord 	\\
\multicolumn{2}{c}{\Sor}			& 				& \ord	&	\ord		& 			\\
\midrule
\multicolumn{2}{c}{\ili{NENA} dialects} \\

\midrule
SE Turkey &	\Cal		&				& 		&	+			&  		\\
NW Iraq	& \JZax		& \first.\agr	& 		&	+			& +			\\
NW Iraq	&	\Qar		& \first.\agr	& =\Arab&				& \agr		\\
NW Iran	&\JUrm		& 				& \ord	&	\opt{\ord}		& \opt{\ord}		\\
NE Iraq	& \Arb		& 				& \first&	\opt{\ord} & 			\\
NE Iraq &\JSul	& \opt{\first.\textsc{agr}} & \first& \first		& 			\\
W	Iran& \JSan	& \ord.\opt{\textsc{agr}} & \ord.\opt{\textsc{agr}} & \ord.\opt{\textsc{agr}} &  \\ 
\bottomrule 
\end{tabular}

 \caption[Properties of ordinal constructions]{Ordinal constructions in select \ili{NENA} dialects and model languages. A (+) or non-blank entry denotes a construction is available, with following \opt{optional} qualifications: \ord=numerals marked as \isi{ordinals} (distinct from cardinals); \agr=\isi{numeral} agrees with \prim; \first = restricted to ordinal \transl{first}, \Arab = \ili{Arabic} loan-\isi{ordinals}.} \label{tb:ordinals}
\end{table}

At the same time, a comparison of  the various dialects and contact languages reveals also recurring patterns, notably the \cst-marked construction with \isi{numeral} \secns. This situation is symptomatic of the entire AC system of \ili{NENA} dialects: While each dialect permits idiosyncratic constructions and variations, the usage of a head-marking construction reoccurs again and again, both in the \ili{NENA} dialects and in the contact languages. 






