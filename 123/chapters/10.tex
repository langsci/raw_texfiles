













\chapter{The development of D-markers in NENA dialects} \label{ch:diachrony1}

\section{Introduction}

In the previous chapters I have surveyed some AC systems of certain \ili{NENA} dialects from a synchronic perspective. We have seen that these dialects permit a wealth of constructions to mark the \isi{attributive relation}, while at the same time some key strategies re-appear. In this and the following chapter I take a broad cross-dialectal view and compare the occurrence of the main constructions across all \ili{NENA} dialects in the survey. This comparison will permit us to formulate some plausible hypotheses regarding the origin of these constructions. A key question in this regard is to evaluate the role \isi{language contact} played in the \ili{NENA} developments, as opposed to \isi{internal development}s. Of course, both these factors play a role in the development of every language, but sometimes they can be shown to go hand in hand, while on other occasions they seem to block each other. As contact languages, I consider especially Kurdish dialects, whose AC system was presented in some detail in \ref{ch:Kurdish}. To assess the internal development scenarios, I treat \Syr, whose AC system was presented in \ref{ch:Syriac},  as an approximative Proto-\ili{NENA} stage, without entering into the methodological debate whether a Proto-\ili{NENA} existed at all.  Some allusions, moreover, will be made to other classical forms of Aramaic (notably \JBA), Early Jewish Neo-Aramaic (\NrT), as well as other Neo-Aramaic varieties (\WNA and \Midn).

In this chapter I concentrate on the development of the \concept{D-markers} in \ili{NENA} dialects, i.e.\ AC markers containing a \phonemic{d} segment which is a reflex of the \il{Aramaic!Classical}Classical Aramaic \d \lnk* or a cognate thereof. As such, the chapter is closely tied to \ref{ch:synchrony}, which presents the synchronic analysis of these markers in \ili{NENA} dialects. While this chapter aims to track the development of these markers, it has a comparative part as well, as it presents the distribution of the various constructions cross-dialectally. 

\largerpage
To situate the development of the D-markers, \sref{ss:d_lnk_dist} discusses first the retention of the \il{Aramaic!Classical}Classical Aramaic Analytic Linker Construction (=ALC, i.e.\ \Syr \foreign{bayta d\cb{}malkā}{the house of the king}, see \sref{ss:syr_ALC}) in \ili{NENA} dialects. \Sref{ss:dist_DAC}, on the other hand, discusses the non-retention of the Syriac Double Annexation Construction (=DAC, i.e.\ \Syr \transc{bayt-ēh d\cb{}malkā}, see \sref{ss:syr_DAC})

\Sref{ss:neo-CSC} discusses the arguably most prominent AC in \ili{NENA}, namely the \concept{neo-construct-state construction} (=Neo-CSC, i.e.\ \Bes \transc{bayt-əd malka}).\footnote{I use the term \concept{neo-construct} differently from \citet[3, fn. 15]{MutzafiBarzani}, who uses it to refer to the innovated apocopated \cst* formation,  not being a reflex of the historical \cst* formation. Since the distinction between the historical and the innovated apocopated \isi{construct state} formations is not always obvious, I subsume both under the heading Apocopate-CSC, reserving the term Neo-CSC for the forms marked by the suffix \ed, stressing the fact that this is the main structural (head-marking)  equivalent of the classical \ili{Semitic} \cst* in \ili{NENA} dialects. A discussion of the development of the innovated  Apocopate-CSC and the retention of the historical CSC is found in \sref{ss:apcopate}.} In particular, \Sref{ss:NeoCSC_Origin} discusses the origin of this construction, whether it is the \il{Aramaic!Classical}Classical Aramaic ALC or rather the DAC. \Sref{ss:role_contact} discusses the role \isi{language contact} might have played in its development.

\Sref{ss:genitive_development} discusses the development of the \gen* prefix \d, with a special emphasis on the role of \isi{language contact}.

\Sref{ss:alt_lnk} discusses the distribution and development of alternative \lnk*s in  NE\-NA. The usage of the \lnk* \transc{did} is discussed both as a basis for the attributive pronouns (\sref{ss:pron_base}) and as an independent \lnk* (\sref{ss:did_lnk}). Other \lnk*s are discussed as well, notably \transc{ad} or \transc{od} (\sref{ss:od_lnk}), and the \JUrm \transc{ay} \lnk*, in which the \phonemic{d} segment is arguably no more apparent (\sref{ss:JUrm_ay}). \Sref{ss:mara} discusses the possible \isi{grammaticalisation} of \foreign{mar-}{owner} as a \lnk*; while it is not related to the \d \lnk*, it it treated here due to its possible functional equivalence.



\section{The distribution of the inherited ALC: X \d{}Y} \label{ss:d_lnk_dist}

As seen in \ref{ch:Syriac}, the main AC in Syriac is the \isi{analytic linker construction} (=ALC), a linker construction in which a linker \d mediates between the \prim and the \secn, without any further marking on the \prim. This construction, with the very same linker \d (sometimes realized \phonemic{də-} or even \phonemic{ʾəd-}\footnote{While the form \phonemic{ʾəd-} can be seen as a phonetic variant of \d, with the \isi{schwa} added as an \isi{epenthetic} vowel, it may also represent an alternative \lnk* form, similarly to \transc{ad} or \transc{od} discussed in \sref{ss:od_lnk}. Since in all dialects, in which the form \phonemic{ʾəd-} is found, one finds also the basic form \phonemic{d-}, this question does not affect the current discussion. \label{ft:əd_lnk}}) is retained in many \ili{NENA} dialects, but often with various restrictions. I distinguish between cases where the \d \lnk* mediates between two nouns, and cases where it mediates between a (pro)noun and clause. The dialectal distribution of these two possibilities is given in \vref{tb:X_d-Y}.
 Cases where the \secn is a pronoun, on the other hand, must be treated separately, as well as cases where the \prim is an \isi{adverbial}. In the present discussion I exclude cases where I consider the \d segment to be re-analysed as a morphological \gen* marker, i.e.\ preceding a vowel-initial determiner/pronoun (these cases are discussed in \sref{ss:genitive_development}).

\begin{table}[t]
\centering
\begin{tabularx}{\textwidth}{XX c c}
\toprule
Region & Dialect &  N \transc{d}-Noun  &  N \transc{d}-Clause \\
\midrule
{SE Turkey} & \Her & (+)  & + \\
					& \Boh &   & \\
					& \Bes &   & \\
					& \Gaz &   & \\
					& \Baz  & + & + \\
					& \Cal  & + & (+) \\
					& \Jil  & + & + \\
\midrule
NW Iraq		& \JZax &  & + \\
					& \JArd &  &   \\
					& \CArd & + & + \\
					& \Barw & + & + \\
					& \Betn &   & + \\
					& \Amd & + & +\\
					& \Barz \\
					& \Alq & + & +\\
					& \Qar & + & + \\ 
\midrule
NW Iran		& \JUrm &  & \\
					& \Sar \\
\midrule
NE Iraq 
					& \Diy & + & ? \\
					& \Arb &   & + \\
					& \JKoy &   &   \\

					& \JSul & + & (+) \\

\midrule
W. Iran			& \JSan &  & \\
					& \CSan \\
\bottomrule
\end{tabularx}
\caption[Dialectal distribution of \d \lnk*]{Dialectal distribution of Noun + \d + Noun/Clause constructions. (+) indicates cases where the \prim is \zero\ or pronominal.}
 \label{tb:X_d-Y}
\end{table}


Out of the 24 dialects surveyed in \vref{tb:X_d-Y} the \d appears as mediating between two nouns only in  10 dialects.  Moreover, the usage of this construction is often qualified. Thus, \citet[192]{KhanSulemaniyya} reports that in \JSul this construction is used in \enquote{isolated instances}. In \Barw  it is reported to be in occasional use only \parencites[398]{KhanBarwar}. In \CArd an N \D-N construction is not found (in my survey), but the construction [N+Adj] \D-N  is found once.\footnote{In general there is a tendency to use the ALC with phrasal \prims, possibly due to the prosodic independence of the \d phrase. Nevertheless, phrasal \prims can appear also in other ACs, notably the Neo-CSC.} The usage of the construction is often motivated by morpho-phonological factors. In \Cal it is used when the \prim is a loanword, typically not adapted to Aramaic word-structure \citep[46]{FassbergChalla}. In \Jil \citet[60]{FoxJilu} asserts that the \lnk* appears after \prims ending in consonants, those being in fact also unadapted \isi{loanwords}. 




Taking into account also clausal \secns, 
one finds this construction in three more  dialects: \JZax, \Betn, and \Arb.\footnote{It should be noted that these dialects make use of the ALC for nominal \secns, but with other \lnk*s. Thus, \JZax and \Betn use the \lnk* \transc{did} (see \sref{ss:did_lnk}), while \Arb uses the \lnk* \transc{od} (\sref{ss:od_lnk}).} 
 The most common type of clausal \secns are those which start with a \isi{copula}, which is typically vowel-initial. This phonological environment may have been favourable for the retention of the \d \lnk* as a \rel*, as the \lnk* could easily syllabify with the vowel-initial \isi{copula}, creating an \textit{optimal} CV syllable.\footnote{The retention of morphemic segments before vowels (including glottal stops, these being weak consonantal onsets) is a well-known phenomenon in \ili{NENA}, especially in the verbal domain: In some dialects, the indicative marker \transc{k-} is only conserved before vowel-initial (or \phonemic{ʾ}-initial) verbal stems \citep[e.g.\ \JArb: ][248]{KhanArbel}. \label{ft:pre_vocalic_retention}} Support for this idea comes from  \JZax, which has gone beyond mere retention of the \d \lnk*, and has re-analysed the combination \transc{d}+\cop\  as an attributive form of the \isi{copula} (\cite{CohenNucleus}; see discussion in \sref{ss:JZax_genitive_clauses}). 


The construction is entirely  lacking in Iranian-located dialects\footnote{It is interesting to note that in \Sar the \d marker survives only with the \isi{copula}, as a verbal aspectual marker (see \example{1768}). This shows that the one of the last uses of the \d marker before its disappearance from the \ili{NENA} AC system is with clausal \secns, and more precisely with copular \secns. In \JSan and \JUrm, on the other hand, the \d survives as a \gen* marker on certain pronouns.} as well as the peripheral  dialects of Turkey (with the exception of \Her). In other words, the \d \lnk* is better conserved in the central dialects, while it is  lost in the periphery. 

  
From the above, two conclusions arise. First, the use of the ALC, which was a major AC construction in Eastern \il{Aramaic!Classical}Classical Aramaic has been greatly reduced in modern dialects. Second, the \d linker in its role as a \isi{relativizer} proved to be more durable, possibly due to the favourable role played by copular \secns, as explained above. 

The reason for the decline of the ALC in \ili{NENA} can be attributed to two main reasons: 1) The replacement of the \d linker by other linkers (see \sref{ss:alt_lnk}); 2) The replacement of the linker construction by a head-marking construction, namely the Neo-CSC (\sref{ss:neo-CSC}). 

\section{The Syriac double annexation construction: X-y.\poss\ \d{}Y} \label{ss:dist_DAC}

While the use of the ALC has been reduced in \ili{NENA}, its fate has been better than the Double Annexation Construction (=DAC). Recall that the DAC is a construction in which the \secn is indexed by a \isi{possessive pronoun} on the \prim, followed by the \d \lnk* and the \secn itself (for example, \foreign{bayt-ēh d\cb{}malkā}{the king's house}). This construction has completely disappeared from \ili{NENA} dialects.\footnote{This is not to imply that it died out. Rather, as shown in \sref{ss:NeoCSC_Origin}, it seems to have evolved into the Neo-CSC. Yet,  in its Classical form, the DAC does not occur in \ili{NENA} dialects (\textit{Pace} \citet[383]{MengozziExtended} who claims that \enquote{it is still used in certain varieties of NENA}). Only with pronominal \secns does one find a similar construction in some dialects, used chiefly to disambiguate the usage of \third person possessors (see \example{1389}).} The only attested cases I could find of this construction in a modern \ili{NENA} corpus are the Gospel translations in \Qar, which clearly preserve the original Syriac wording (see \sref{ss:Qar_DAC}).

\section[Development of the Neo-CSC in NENA: X\ed Y]{Development of the Neo-Construct-State construction in NENA: X\ed Y} \label{ss:neo-CSC}
\largerpage[-2]


\begin{table}[p] 
\centering
\begin{tabular}{l l | c c c | c c c}
\toprule
		&					& \multicolumn{3}{c}{\Prims}& \multicolumn{3}{|c}{\Secns} \\
Region 	& Dialect			& Noun	& Adj. & Inf.		& NP & Ordinal & Clause \\	
\midrule
{SE Turkey} & \Her 	& +		&		&	(+)		&	+		&	+	&	(+)		\\
					& \Boh 	& +		&		&			&	+		&		&	(+)		\\
					& \Bes 	& +		&		&			&	+		&		&			\\
					& \Gaz 	& +		&		&			&	+		&		&			\\
					& \Baz  & +		&		&			& +			&		&			\\
					& \Cal  & +		&		&	+		& +			&	+	&	+		\\
					& \Jil  & +		&		&			&	+		&		&			\\
\midrule
NW Iraq		& \JZax & +		&	+	&	+		& +			& +		&	+	\\
					& \JArd & +		&		&			& +			& 		&	(+) \\
					& \CArd & +		&		& +			& +			& +		&	+	\\
					& \Barw & +		& +		&			& +			& +		&	+	\\
					& \Betn & +		&		&	+		& +			& +		&	+	\\
					& \Amd 	& +		&		&			& +			& +		&	+\\
					& \Barz & +		&		&			&			&		& 	\\
					& \Alq 	& +		&		&			& +			& +		&	+\\
					& \Qar  & +		&	+	&	+		&	+		&		&	+	\\
\midrule
NW Iran		& \JUrm & +		&	+	&			&	+		&	+	&	+	\\
					& \Sar 	&	+	&		&			&	+		&	+	&	+	\\ 
\midrule
NE Iraq 	& \Rus  & 		&		&			&			&		&	(+)	\\
					& \DiyZ  & +		& +		&			& +			& +		&	?	\\
					& \Arb 	& +		&		&			& +			&		& 	+ \\
					& \JKoy & +		&		&			& +			&	+	&	+ \\
					& \JSul & +		&		&			&	+		&		&	(+) \\
					

\midrule
W. Iran			& \JSan & 		&		&			&			&		&	\\
					& \CSan &	+	&		&			&	+		&	+	&	\\
\bottomrule
\end{tabular}
\caption[Distribution of the suffixed \isi{construct state}]{Distribution of the suffixed \cst*. The entry (+) indicates clausal \secns following only pronominal \prims or a \cst* tautological infinitive (\Her only).} \label{tb:suff_cst}
\end{table}

As stated above, the ALC and DAC, extant in Syriac, have been to a large extent replaced by  the Neo-CSC of \ili{NENA}, in which the \prim is marked by a suffixed morpheme \ed. As \vref{tb:suff_cst} shows, this construction is extant in all surveyed \ili{NENA} dialects, with the notable exception of \JSan. The extent to which the construction is used with \prims and \secns other than nouns, however, varies quite a lot.\footnote{Some of the variation, however, is probably attributable to variable corpus sizes available for each dialect.} Some of the major categories are given in \ref{tb:suff_cst}, with the notable exclusion of \isi{adverbial} \prims (i.e.\ prepositions and conjunctions), as these are lexically determined in each dialect.\footnote{As for adjectival \secns, these appear regularly in this construction only in \Arb (see \example{1230}) and to a limited extent, which is probably non-productive, in \Barw (e.g., \foreign{xəṭṭət romaye}{roman wheat}; \cite[523]{KhanBarwar}) and \Barz (\foreign{kalekūvid ʾuṛwa}{the great wild ram}; \cite[4, fn.\ 33]{MutzafiBarzani}).}

  
Two questions arise regarding this diachronic development of this marking:

\begin{enumerate}

\item What is the origin of the Neo-CSC\is{neo-construct-state construction}? Is it the ALC, the DAC or both? 
\item How did the Neo-CSC develop\is{neo-construct-state construction}? Specifically, what is the role of \isi{language contact}? 
\end{enumerate}  

In the following sections, I shall attempt to answer these questions.

\subsection{Origin of the Neo-CSC} \label{ss:NeoCSC_Origin}



\citet[378--380]{MengozziExtended}, following \citet[169]{KhanArbel}, gives three possible hypotheses regarding the emergence of the Neo-CSC. In all accounts, it is clear that the suffixed segment \phonemic{-d} results from the \isi{encliticization} of the Syriac \isi{proclitic} \d. What is less clear is the source of the \isi{schwa} vowel which precedes it, forming the suffixed morpheme \phonemic{-əd}\~\phonemic{-ət}. Recall that the \isi{schwa} replaces as a vocalic nucleus the \isi{free state} endings \phonemic{-a}\~\phonemic{-e} of words of Aramaic origin. Indeed, this replacement of the  \isi{free state} endings is one of the main reasons I have alluded to in considering the \ed ending as a morphologically integrated suffix of the noun stem (see \sref{ss:morph_paradigm}).\footnote{Nouns of foreign origin ending in consonants can also get the \ed suffix in some dialects, such as the Kurdish loan \foreign{xadām}{servant} in the following example:

\acexfn[\Arb]
{Noun}{Noun Phrase}{1252}
{xadā́m-it [bā́b-it ʾiyyà faqī́r]}
{servant-\cst{} father-\cst{} \dem.\near{} poor.\sg}
{the servant of the father of this poor man}
{KhanArbel}{424 {[S:31]}}

 Foreign nouns whose final vowel is not seen as the \isi{free state} ending may get a simple \phonemic{-d} suffix in the \cst* (see \sref{ss:morpo_phon_idio}). Recall also that in \JUrm the suffix may be \phonemic{-ad} under the influence of vowel harmony (see \sref{ss:JUrmi_CST}).} 

\citet[379f.]{MengozziExtended}, citing \citet[169]{KhanArbel}, mentions three hypotheses regarding the origin of the \isi{schwa}:

\begin{enumerate}

\item It results from a phonetic reduction of the \phonemic{-a}\~\phonemic{-e} \isi{free state} suffixes appearing on the \prim of the ALC.

\item It is a reflex of a fossilized 3\masc\ \isi{possessive pronoun} \transc{-ēh} originating in the DAC, which was phonetically attenuated to \transc{-ə} (often realised as \phonetic[ɪ] or \phonetic[ɘ]).

\item It is a reflex of a demonstrative element \textit{ə}<\textit{ay}, originating in the ALC with an inserted \isi{demonstrative pronoun} acting as determiner (i.e. \prim + \dem\ + \lnk\ + \secn; see \sref{ss:syr_corr} for \Syr examples).\footnote{This construction is probably the source of the \ili{NWNA} \Midn \concept{heavy possessive suffixes}, which originated in the \isi{encliticization} of the sequence \transc{ay-ḏ}+\poss\ to a \prim noun, yielding for instance \foreign{ʾu\cb{}bayt-ayḏe}{his house}. See \textcites[52, \S 47]{JastrowMidin}[58]{JastrowLehrbuch}, who offers, however, a different development path. \label{ft:Midn_ayd} }

To this one may add two supplementary hypotheses:

\item It is an \isi{epenthetic} vowel added before a \transc{-d} suffix (following the removal of the \isi{free state} suffixes where present).

\item It is a reflex of the \Sor \ez* suffix \transc{-î} (=\phonetic[i]\~\phonetic[ɪ]), or an   fossilized and attenuated \Kur 3\masc\ \ez* suffix \transc{-ê}\~\transc{-î} (=\phonetic[e]\~\phonetic[i]).

\end{enumerate}

\citet[380]{MengozziExtended} prefers the second hypothesis, since it explains the occurrence of prepositional \prims with the \ed ending. Prepositions in \il{Aramaic!Classical}Classical Aramaic cannot appear in the ALC, but rather must appear in the DAC (if a \lnk* is present at all). Thus, only a DAC-origin hypothesis can explain their distribution with the \ed ending in \ili{NENA}.\footnote{As some prepositions, notably \transc{l-} and \transc{b-}, do not occur in the DAC in \Syr, one is obliged, moreover, to assume analogy across prepositions in order to explain their \cst* marked forms, namely \transc{ʾəlləd} and \transc{ʾəbbəd}. According to \citet[330, \S 231]{NoldekeMandaic}, in \CMand the preposition \transc{b-} does occur {very occasionally} (\textit{ganz vereinzelt}) in the DAC, but not the preposition \transc{l-} \citep[see also][112]{PatElStudies}.} \citeauthor{MengozziExtended}'s examples are reproduced in \vref{tb:DAC-origin}.

\begin{table}[h!]
\centering
\begin{tabular}{l ll l ll}
\toprule
\multicolumn{3}{c}{Classical Aramaic}  &  \multicolumn{3}{c}{\ili{NENA} dialects} \\
\midrule
\multirow{4}{*}{ALC} 	& \transc{baytā} 		& \transc{d\cb malkā} \\
						& house.\free	& \lnk\cb king \\
						& *\transc{ʿammā}		& \transc{d\cb malkā} \\
						& *with.\free	& \lnk\cb king \\
\midrule
\multirow{4}{*}{DAC} 	& \transc{bayt-ēh} 		& \transc{d\cb malkā} &  \multirow{4}{*}{> Neo-CSC} & \transc{bayt-əd} & \transc{malka} \\
						& house-\poss.3\masc	& \lnk\cb king 			& & house-\cst & king \\	
						& \transc{ʿamm-ēh}		& \transc{d\cb malkā} 	&  & \transc{ʾəmm-əd} & \transc{malka} \\ 
						& with-\poss.3\masc	& \lnk\cb king 			&  & with-\cst & king \\
\bottomrule
\end{tabular}
\caption{Mengozzi's argumentation regarding the origin of the Neo-CSC} \label{tb:DAC-origin}
\end{table}

A further fact substantiating this hypothesis is the fact the DAC is virtually absent in \ili{NENA} dialects, as discussed in \sref{ss:dist_DAC}. This is easily explained if the DAC evolved into  the Neo-CSC. The fact that the ALC remains to a certain extent in \ili{NENA}, as shown in \sref{ss:dist_DAC}, indicates conversely that it was probably not the source of the Neo-CSC. 

To round off the picture in favour of this hypothesis, note that \citet[122]{Socin} (cited by \cite[230]{Tsetsereli}) brings the \Jil example \foreign{šímm-o-d báxta}{the name of the woman}, in which the \transc{-o-} element corresponds to the 3\fem\ \isi{possessive suffix}. This construction looks very much like the Neo-CSC, as the \d \lnk* is encliticized to the \prim, yet the \fem\ \isi{possessive suffix} is a clear indication of a DAC-origin. This example seems to reflect an earlier stage of \ili{NENA} in which the \isi{possessive suffix} was not yet fossilized and attenuated as an \phonemic{ə} segment.  No such example, however, is attested in modern descriptions of \ili{NENA} dialects (including \cite{FoxJilu} describing \Jil), so this example may rather reflect a certain purist or prescriptive approach to language (imitating the Syriac construction) rather than normal usage. Indeed, judging from the examples of \citet[374f.]{MengozziExtended}, already Early Christian \ili{NENA} (manuscripts of the 17\th century) had a fossilized, and possible phonetically attenuated, 3\masc\ \isi{possessive suffix} in the DAC.\footnote{Confusingly, it was spelled sometimes as a final \Syr Aleph \textsyriac{ܐ}, rendering it orthographically similar to the \isi{free state} suffix. Note that an Aleph has the consonantal value of a \isi{glottal stop} \phonemic{ʾ}, but it was likely not  pronounced word-finally.} Also  some Neo-Aramaic writers using the 19\th century Syriac script developed by missionaries in Urmi, notably Paul Bedjan, wrote a fossilized 3\masc\ \isi{possessive suffix} on the \prim preceding a \isi{proclitic} \d \citep[192, \S 6.2.6; 198, \S 6.3.6, fn.\ 33]{MurreUrmi}.\footnote{Note, however, that \citet[175]{MurreUrmi}, who adopts an ALC-origin view of the \ed suffix, sees this fossilized 3\masc\ \isi{possessive suffix} as a \textit{post-hoc} adaptation of the \ed suffix to grammar of Syriac.}


While the DAC-origin hypothesis seems thus highly plausible, it does not exclude the alternative  explanations completely. First, as \citet[382]{MengozziExtended} himself notes, this origin is problematic in explaining the use of the \ed suffix before clausal \secns, since in Syriac these could only appear in the ALC. In order to explain the availability of the Neo-CSC construction in such cases, Mengozzi brings forth the first hypothesis, namely the ALC-origin hypothesis, and concludes that \enquote{the phonetic reduction that gave rise to the endings \textit{-ed}, \textit{-it}, etc.\ neutralized the morpho-phonetic oppositions between two earlier constructions [the ALC and the DAC]}.\footnote{He relates, moreover, the extended usage of the \ed suffix to the Kurdish \ez*, a question which I shall examine in more detail below.}

In order to reconcile the two origins, one can posit a double-origin hypothesis. In such a scenario, following the transformation of the DAC to the Neo-CSC, cases where the \d \lnk* of the ALC is cliticized to the \prim (as may happen due to prosodic reasons), are levelled by analogy to the Neo-CSC: e.g.\ ALC \transc{bayt-a\cb{}d malka} > Neo-CSC \transc{bayt-əd malka}. This would naturally also include cases with clausal \secns.




















One may wonder why  the DAC (\transc{bayt-ēh d\cb{}malkā}) was completely transformed into the Neo-CSC, while the ALC remains in \isi{complementary distribution} with the latter. This is partially answered by the hypothesis that the transformation ALC>Neo-CSC is a later development, that may not yet have reached its culmination.\footnote{In this respect, it would be interesting to follow the recent  evolution of this construction in contemporary \ili{NENA} dialects, now spoken for a large part in the diaspora.} Yet  also structural reasons may be called upon:

 First, since the 3\masc\ \isi{possessive suffix} is normally realized in \ili{NENA} as a vowel \transc{-e} or \transc{-u},  the \isi{encliticization} of the \d to it is highly facilitated, being in fact a phonetic re-syllabification.\footnote{In \il{Aramaic!Classical}Classical Aramaic a weak consonantal segment \phonemic{h} follows the vocalic nucleus yielding \mbox{transc{-ēh}}. Yet  in most \ili{NENA} dialects this segment has been elided, or conserved only in restricted morpho-phonological contexts (e.g.\ in in \JZax before a \third person \isi{copula}; see \cite[450]{CohenZakho}). In some dialects it has been conserved or even strengthened to \phonemic{ḥ} segment  \citep[96]{CoghillNotable}. The latter is the case for instance in \Alq, yet this had no effect on the emergence of the Neo-CSC in the dialect. This hints that the elision of the \phonemic{h} segment in the \isi{possessive suffix} of the DAC was independent of its development in other places, in line with the idea that the \isi{possessive suffix} of the DAC was fossilized.} 
   In the ALC, however, the \prim may in principle end in a consonant (especially if it is an unadapted loanword\footnote{In Syriac texts there are numerous Greek \isi{loanwords}, for example.}) thus preventing such a \isi{resyllabification}, and conserving the availability of the ALC. 

\largerpage[-2]
Second, from a more general point of view, the principle of \concept{economy} seems to have played a role.\footnote{Recently, \citet{Cristofaro} argued that \concept{economy} should not be advanced as responsible for \isi{language change}, but rather specific morpho-phonological processes of \isi{language change} should be specified. Clearly,  \isi{language change} is driven by specific processes (as is detailed in this chapter), yet I believe that the principle of \isi{economy} can give further insight about linguistic change as it relates to the general cognitive organisation of the linguistic system.}  In \il{Aramaic!Classical}Classical Aramaic, the marking of the \prim by a \isi{possessive pronoun} was part of a more general strategy of using \concept{proleptic pronominal suffixes} to mark \isi{definiteness}.\footnote{This is still conserved in \ili{NENA} dialects in the verbal domain, where definite objects are often indexed on the verb with proleptic pronouns. See \citet{CoghillDOM} for a discussion.} Yet  over time the role of the \isi{proleptic pronoun} as marking \isi{definiteness} of the DAC must have eroded (probably hand-in-hand with the fossilization), as one finds in \ili{NENA} the Neo-CSC\is{neo-construct-state construction} used with indefinite nouns:\footnote{An alternative explanation would be to assume that the indefinite usage of the Neo-CSC originated in the ALC, the latter not being tied to \isi{definiteness}.} 

  
\acex[\Amd]{Noun}{Noun}
{1581}
{xa šaqqiθ-əd ṃaye}
{\indef{} channel-\cst{} water}
{a channel of water}
{GreenblattAmidya}{72}

The erosion of the \isi{definiteness} value arose possibly due to the development of other means to mark \isi{definiteness} (see \sref{ss:intro_NPstructure}), or since ACs are in general definite anyhow \citep[cf.][231]{HaspelmathArticle}. Be that as it may, this led necessarily to the reanalysis of the \isi{proleptic pronoun} as a pure \prim-marker of the AC, on top of the \lnk*, rendering the DAC a double-marked AC. But, by the principle of \concept{economy}, it is preferable to transform the  double-marked DAC to a single-marked Neo-CSC, thus reducing the cognitive burden of marking the construction on two separate {loci}\is{locus}.\footnote{This can be contrasted to the situation in \Turk, in which the double-marked construction is productive since it marks the \isi{definiteness} of the AC (see \example{Turk1}).} The ALC, on the other hand, is single-marked (dependent-marked), thus showing equal structural complexity as the head-marked Neo-CSC.  

What about the other hypotheses mentioned above? Regarding the third hypothesis, Mengozzi asserts that no evidence for the origin construction (X \dem\ \lnk\ Y) is found in the Early \ili{NENA} manuscripts he investigated. In Syriac, one finds instances of this construction (see \ref{ss:syr_corr} and most notably \example{1034}: \foreign{rawmā haw da\cb{} šmayyā}{the height of heaven}), but not with prepositions as \prims . In any case, assuming this would be the origin of the Neo-CSC would require further explanation of the disappearance of the \emp* suffixes (\transc{-ā} in the cited example), unless it is assumed that they coalesced with the \isi{demonstrative pronoun}. It is rather more probable that such a construction developed into an alternative linker such as \transc{ad} or \transc{od}, discussed in \sref{ss:od_lnk}.
 
As for the fourth hypothesis, while the \isi{schwa} segment in the \ed suffix arose from an attenuation of the 3\masc\ \isi{possessive suffix}, synchronically one could indeed argue that in most \ili{NENA} dialects it has been re-analysed as merely an \isi{epenthetic} vowel, enabling the syllabic addition of the \transc{-d} suffix to the nominal stem.\footnote{A similar claim is made by \citet[112, \S 107.f]{SpitalerWNA} regarding the \isi{schwa} segment in the \WNA Neo-\cst\  suffix \transc{-əl}. I am grateful to \name{Ivri}{Bunis} for drawing my attention to this reference.} Thus, in the \Barw dialect, whenever a \prim noun ends in a vowel other than \transc{-a} or \mbox{\transc{-e}} serving as the Aramaic inflectional ending, only a \phonemic{-d}\~\phonemic{-t} suffix is added \citep[397]{KhanBarwar}.\footnote{\citet[397]{KhanBarwar} reports one possible exception to this rule, occurring supposedly when \ed is suffixed to \enquote{[a]n unadapted loanword that has a final vowel that it has retained from the source language}. In such cases the \isi{schwa} is retained by an insertion of the glide \phonemic{y}. He brings one example of this phenomenon:

\acexfn[\Barw]{Noun}{Noun}
{1863}
{ḥabba-y-ət xəṭṭəθa}
{seed-\transc{glide}-\cst{} wheat\_grain}
{a seed of grain}
{KhanBarwar}{397}

Yet  the validity of this analysis can be questioned, as in the \Iraq dictionary of \citet[89]{WoodheadBeene} \transc{ḥabbāya} is listed as a variant  of \transc{ḥabba}. Thus, the \phonemic{y} segment is simply part of the lexical stem, and the \isi{schwa} replaces the final \transc{-a}.}  The same phenomenon happens sometimes when the \prim ends in a liquid, as in the following \Gaz example:

\lex{\Gaz}{1833}
{ahl-d Gaznax}
{people-\cst{} G.}
{the people of Gaznax}
{GutmanGaznax}{318 (7)}


Thus, synchronically one could argue that the \isi{schwa} is not a phonemic part of the \ed suffix, but rather an \isi{epenthetic} vocalic nucleus needed due to the removal of the vocalic \isi{free state} suffixes (but see the \Qar \example{486}, where the \isi{schwa} is the sole exponent of the \cst*).



















What about the idea that the \isi{schwa} is related to the \ez* particle? Assuming that it results from the \Kur \ez* raises analytical difficulties, since the latter shows gender and number inflection, so one would have to stipulate an extra step of fossilization of the \ez* suffix, which is not observed in \Kur. The \Sor \ez*, on the other hand, may be a better candidate, as it is an uninflecting particle. This idea gains further support from the fact that in some dialects, especially in NE Iraq, an \ez* suffix \transc{-i} stands in \isi{complementary distribution} with an \ed suffix (see \sref{ss:i_ezafe}). Given the phonetic similarity of the \isi{schwa} and this \ez* (both roughly realized as \phonetic[ɪ]), it may indeed be the case that bilingual speakers conflated the two. Yet, since the \ed suffix is found also in dialects which have not integrated any \ez* marking, and also in the Kurmanji speaking area, it seems rather implausible to place the origin of the \isi{schwa} in a \isi{matter replication} of the \ez*. As for the different question of whether the Neo-\cst\ suffix \ed developed due to \isi{pattern replication} of the Kurdish \ez*, this is dealt with in the next section (see in particular \sref{ss:infl_lnk}). 

To conclude this section, I present the development of the Neo-CSC, as outlined above, in six distinct stages shown in \vref{tb:Neo-CSC_dev}. To better apprehend the fossilization of the \isi{possessive suffix}, the model \Syr expression is \foreign{bayt-āh d\cb{}malktā}{house of the queen}, as I assume that the \pl* possessive suffixes shifted to 3\masc\ \transc{-ēh}.


\begin{table}[p!]
\centering
\begin{tabularx}{\textwidth}{l p{5cm} l X}
\toprule
	&	Classical Aramaic\il{Aramaic!Classical}		& DAC							& 		ALC \\
\midrule
0	& Initial state		& \transc{bayt-āh d\cb malktā}	& \transc{bayt-ā d\cb malktā} \\
1	& The DAC \isi{possessive suffix}  is fossilized to the 3\masc\ form \transc{-ēh}, possibly losing its \isi{definiteness} marking function.
								& \transc{bayt-ēh d\cb malktā}	& \\
2	& The DAC \isi{possessive suffix} loses its consonantal coda and is centralized to \transc{-ə}.
								& \transc{bayt-ə d\cb malktā}	& \\
3	& The \d \lnk* of the DAC re-syllabifies with the \prim.	This happens occasionally also in the ALC. 	
								& \transc{bayt-ə\cb d malktā}	& \transc{bayt-a\cb d malktā}\\
4 	& The resulting \ed segment in the DAC is reanalysed as a unitary \cst* suffix.
								& \transc{bayt-əd malktā}		& \\
5	& By analogy, the \transc{-ad} sequence in the ALC (\transc{-ed} in \pl*) is levelled  to the \cst* suffix \ed.{*} 
								&						 		&\transc{bayt-əd malktā}\\
6	& The \phonemic{ə} segment is re-interpreted as an \isi{epenthetic} vowel, added only when the syllabic structure requires it.
								& \multicolumn{2}{c}{\transc{bayt-\opt{ə}d malktā}} 	\\
\midrule
* & The Neo-CSC construction co-exists in \isi{complementary distribution} with remnants of the ALC.							& \transc{bayt-əd malktā} \
								& \transc{bayt-ā d\cb malktā}							\\
\midrule
	& NENA\il{NENA}						& Neo-CSC		& ALC \\
\bottomrule
\end{tabularx}
\caption[Development of the DAC and the ALC into the Neo-CSC]{Possible development path of the DAC and the ALC into the Neo-CSC, tracing the development of the model expression \transl{house of the queen}.} \label{tb:Neo-CSC_dev}
\end{table}











\subsection{The role of language contact} \label{ss:role_contact}

According to the scenario outlined in \ref{tb:Neo-CSC_dev}, the key stages in the emergence of the neo-\cst* suffix \ed were the \isi{encliticization} of the \lnk* \d to the primary (stage 3) and its subsequent reanalysis as a head-marking suffix (stages 4--5). The \isi{encliticization} itself may be quite natural due to the syllabic structure (the \prim ending in a vowel, either due to the \emp* suffix, or the \isi{possessive suffix}) as well as to the frequent prosodic boundness of the \prim and \secn. Furthermore, \citet{LahiriPlank} have suggested (from a Germanic perspective) that cross-linguistically there may be a tendency of \isi{encliticization} of functional elements to preceding hosts. Yet  \isi{encliticization} does not necessarily mean reanalysis as a head-marking construction.\footnote{For example, while \citet[376]{LahiriPlank} claim that the expression \enquote{drink a pint of milk a day} is prosodically organised as [drink a][pint of][milk a][day]. Yet  the preposition \enquote{of} cannot be said to have been reanalysed as a head-marker.} Thus, a natural question is: what led to the re-analysis?


 A possible answer is to suppose that some external factor, such as \isi{language contact}, may have played a role in this reanalysis. Indeed, such a proposal has been made by \citet[121ff.]{CohenEzafe}. Cohen, examining data from \JZax, argues that  its Neo-CSC emerged as a \isi{pattern replication} \citep[in the sense of][]{MatrasSakel} from co-territorial Kurmanji Kurdish. A similar proposal was made by \citet[171, \S 2.21.2]{Garbell1965impact} regarding \JUrm, attributing its Neo-CSC to Sorani influence (\enquote{Central Kurdish} in her terminology). Note, however, that also \JUrm is co-territorial with Kurmanji \citep[see map of][171]{IzadyKurds}. 

In the following sections I shall present Cohen's proposal,  and then evaluate it, taking into account data from different \ili{NENA} dialects as well as Kurmanji and Sorani Kurdish and Syriac.\footnote{The argumentation in this section is similar to the presentation in \citet{GutmanContact}, but with some added details and arguments.}

\subsubsection{Parallels between Kurmanji and NENA Attributive Constructions} \label{ss:parallel_ez}

Recall that in Kurmanji Kurdish the \ez* morpheme marking attribution can be suffixed to the \isi{head noun} (see \ref{ex:711bis} below=\example{711}) or, when it does not directly follow the \isi{head noun}, appear as an independent morpheme (see the  morpheme in bold in \ref{ex:727bis} below=\example{727}; see further \sref{ss:three_Ez} and the following sections).

\acex[\Kur]{Noun}{Noun Phrase}{711bis}
{kitêb-ên [keç-{a} mirov]}
{book-\ez.\pl{} girl-\ez.\fem{} man}
{the man's daughter's books}
{ThackstonKurmanji}{13}


\acex[\Kur]{Noun Phrase}{Noun}{727bis}
{[hejmar-ek-e nû] \textbf{ya} kovar-ê} 
{issue-\indef-\ez.\fem{} new \lnk.\ez.\fem{} journal-\obl.\masc}
{a new issue of the journal}
{ThackstonKurmanji}{15}

\citet{CohenEzafe} argues that the independent \ez* morpheme acted as a \concept{pivot} in the \isi{pattern replication} of the Neo-CSC. The \isi{proclitic} pronominal \lnk* \d was matched to the independent \ez*, and consequently was {encliticized} to the construction's head and reanalysed as a head-marking suffix by analogy with the suffixed \ez*.\footnote{From a diachronic perspective, also within the \ili{Iranic} language family the suffixed \ez* arose from the \isi{encliticization} of an independent element \citep{HaigLinker}. \citet{HaiderZwanziger} claim more specifically that it originates in a relative pronoun, which lost its case inflection and subsequently became the \ez*.} Note that this proposal supposes that the \d was encliticized to a noun in the ALC, not the DAC. 


As a further piece of evidence for the affinity between the two languages Cohen notes  that both in Kurmanji and in \ili{NENA} a head-marked noun can precede a clausal attribute, as in the following examples (=\example{759} and \example{319}):

\acex[\Kur]{Noun}{Clause}
{759bis}
{ṭişṭ-ên [min nivisîbûn]}
{thing-\ez.\pl{} 1\sg.\obl{} written}
{the things I had written}{ThackstonKurmanji}{77}

\acex[\JZax]{Noun}{Clause}
{319bis}
{xabr-ıt mír-rē-la}
{word-\cst{} said-\agent3\masc-\dat3\fem}
{the word(s) he told her}
{CohenZakho}{97 (24)}\antipar

\subsubsection{Mismatches between the Kurmanji and NENA constructions}

Notwithstanding the appeal of the above explanation of the source of the \ili{NENA} Neo-CSC, it presents some difficulties. First, it is worth noting that this is a somewhat unusual kind of \isi{pattern replication}, as outlined in \citet[836]{MatrasSakel}. According to their model, it is the \enquote{functional scope} of the source construction which is replicated to the recipient language. Yet  in this case, it is not the functional scope which is replicated (since the \ez* and the \d \lnk* have the same functions to begin with) but rather the distributional-prosodic properties of the \ez*, namely its ability to occur as a head-marking suffix, rather than an independent morpheme, which is replicated.  

Second, looking closely at the linguistic data from a cross-\ili{NENA} perspective, one sees that there is no perfect match between the Kurmanji construction and the parallel \ili{NENA} construction. It should be immediately emphasized that the observed mismatches, surveyed below, cannot preclude an imperfect \isi{pattern replication} scenario. Indeed, \citet[836]{MatrasSakel} clearly state that any \isi{pattern replication} must be accommodated to constraints of the recipient language. Yet, given that in some respects the \ili{NENA} construction is in fact more similar to the Sorani \ez* construction (and in some respects to neither to Sorani nor Kurmanji), these mismatches may indicate that the Kurmanji \ez* construction is not necessarily  the sole or even the main source of this linguistic change. 

Indeed, since the Neo-CSC is encountered both in the \Kur speaking-area and the \Sor speaking-area,\footnote{Arguably, in the Sorani speaking area, the Neo-CSC is somewhat less wide-spread, as some dialects, in particular \JSan and \JSul prefer the \isi{juxtaposition} construction; see \cref{ch:Sanandaj}. However, as my sample of this area (NE Iraq and W Iran) is less comprehensive, I cannot draw firm conclusions out of this observation.}
 if one assumes that it results from \isi{language contact} with one source language, one must further explain its propagation throughout the \ili{NENA} speaking-zone (either by a wave model, or by assuming a common ancestor). Yet, given the partial similarity with each of the proposed source languages, such an assumption is not necessary. Accordingly, the aim of the following arguments is to show that the different facets of Neo-CSC cannot be attributed to contact with a single language, but they are better explained as an areal phenomenon related to the very long history of language convergence in this linguistic area.

\paragraph{Non-inflection of the \ili{NENA} \cst* marker} \label{ss:infl_lnk}
 
In contrast to the Kurmanji \lnk* \ez*, the \il{Aramaic!Classical}Classical Aramaic \lnk* \d does not inflect.  Moreover, the innovated \cst* \ed suffix does not inflect as well, again in contrast  to the \Kur construct \ez*. Thus, any pivot matching  between the two is partial at most.\footnote{One may argue, as a reviewer of \citet{GutmanContact} did, that  there is a general tendency of the languages \enquote{of the area} to evolve towards morphological simplification and loss of nominal inflection, and thus it would be remarkable if the Aramaic \lnk* were to gain inflection. Yet  Kurmanji is one of the exceptional languages that have conserved a relatively rich nominal morphology, as attested also by the conservation of its case system. Thus, the mismatch in inflection is relevant when evaluating the specific hypothesis that Kurmanji served as the model for the development of the Neo-CSC, though it cannot by itself invalidate it.} Of relevance is the fact that Early Jewish Cis-Zab\il{Cis-Zab NENA dialects} NENA (see \sref{ss:intro_dialects}) made use of inflecting demonstrative \isi{determiners} joined to an \isi{enclitic} \d linker, presenting a better parallel to the inflecting \ez*. This can be observed in the  Nerwa Texts, Jewish homilies from the 16\th century written in Nerwa in NW Iraq, whose language is close to the ancestor stratum of  \JZax:\footnote{Arguably, the \phonemic{-d} segment in these examples is already re-analysed as the \cst* suffix, as it regularly occurs also with nominal heads in \Nrt. Be this as it may, in some earlier state at least the \dem* and the \lnk* must have been conceived as independent morphemes.}

\hebacex[\Nrt]{\zero}{Noun}{2003}
{אוד אנואר [...] ואוד ג'מאם [...]}
{ʾaw \cb{}d anwār … u\cb{} ʾaw \cb{}d ġamām}
{\dem.\masc{} \cb{}\lnk{} lights {} and\cb{} \dem.\masc{} \cb{}\lnk{} clouds}
{that (the pillar) of fire ... and that of clouds} 
{(\textit{Pəšaṭ Wayəhî Bəšallaḥ} 22:5 ed.\ by \cite[68]{SabarNerwa})}

\hebacex[\Nrt]{Noun}{Clause}{2001}
{שבועה איד מומכלוך}
{šəḇûʿa,\footnotemark{} ʾay \cb{}d mōm-ax-lux}
{oath(\fem) \dem.\fem{} \cb{}\lnk{} put.\pst-\patient1\sg-\agent2\masc}
{the oath which you put us under}
{(\textit{Pəšaṭ Wayəhî Bəšallaḥ} 4:3 ed.\ by \cite[43]{SabarNerwa})}

\footnotetext{The comma, indicating a possible prosodic break, is added to the apparatus by \citeauthor{SabarNerwa} and is not part of the original manuscript \citep[see][XLVII]{SabarNerwa}.}

Thus, if \Kur was indeed the model language, one could expect a pivot match with these inflecting \enquote{linkers}. However, although such inflecting elements are conserved in some \ili{NENA} dialects such as C. \Barw or J. \Arb  (see \sref{ss:Arb_Barw_lnks}), they are never encliticized as such to the head-noun (see further the discussion in \sref{ss:Arb_Barw_lnks}).\footnote{The fact that such an \isi{encliticization} is in principle possible may be confirmed by the \ili{NWNA} \Midn dialect. See in this respect \vref{ft:Midn_ayd}.}

From the point of view of inflection, The \ili{NENA} Neo-\cst\ suffix is in fact more similar to the Sorani Kurdish uninflecting \ez*, which is always a fixed \transc{-ī}\~\transc{y}, as in the following example (=\example{897}):\footnote{In fact, there is no grammatical gender in \Sor.}

\lex{\KSul}{897bis}
{ser-î binîadem}
{head-\ez{} man}
{men's heads}{MacKenzie}{63}

As mentioned in \sref{ss:NeoCSC_Origin}, the phonetic similarity between the \Sor \ez* and the \phonemic{ə} segment of the \ili{NENA} \cst* suffix \ed  and their similar distribution, may have led bilingual speakers to conflate the two, but it is unlikely to have been the source of the  \isi{schwa} segment. It is equally unlikely that the \Sor \ez* could have served as a pivot morpheme comparable to the Aramaic \d linker, given their different distribution: in contrast to the \d \lnk*, the \Sor \ez*  cannot appear as an independent morpheme, except in those few cases in which it is not preceded by any nominal head at all (see \examples{900}{923}).\footnote{Due to its pronominal nature, the \d linker itself can also appear without a nominal antecedent preceding it (see \examples{987}{986}). Yet, judging by \ili{NENA} examples, outside the predicative position it typically appears with a nominal antecedent or with a demonstrative/determiner preceding it, as in \examples{2003}{2001}. Thus, it seems that the case of phrase-initial \d \lnk*s are not frequent enough to drive this kind of \isi{language change} scenario.}









 
 



\paragraph{Clausal \secns and the usage of a subordinating particle} \label{par:clausal_secn}
\largerpage[2]
In Kurmanji (as well as Sorani), clausal \secns may to follow the subordinating particle \transc{ḳu} (Sorani \transc{ke}), as in the following example (=\example{753} and see further there): 

\acex[\Kur]{Noun}{Clause}{753bis}
{[wî ziman]-ê [ḳu li\cb{}ber mir-in-ê ye]}
{\dem.\far.\obl{} language-\ez.\masc{} \rel{} before die-\inf-\obl.\fem{} \cop.3\sg}
{\transl{this language, which is on the verge of dying.}}
{ThackstonKurmanji}{75}\antipar

\newpage 


Most \ili{NENA} dialects, on the other hand, do not have a dedicated \isi{relativizer} in this position, but rely either on the \cst* ending or on the \lnk* \d (or derivative forms of it), \example{319} being typical. One dialect which does mimic completely the Kurdish pattern and particle is \JUrm, situated at the eastern periphery of the Kurmanji speaking area. This is shown in the following example (=\example{113}):\footnote{Yet as \citet[88]{GarbellUrmi} notes, clauses without an explicit subject NP can optionally appear directly after the \cst* suffix.}

\acex[\JUrm]{Noun}{Clause}{113bis}
{naš-it [ki lóka wélu]}
{people-\cst{} \rel{} there \cop.\pst-3\pl}
{the people who were there}
{GarbellUrmi}{55}

Another dialect which borrowed the particle, but without any \isi{construct state} marking, is  \JSan, located in the southern limit of the Sorani speaking area.\footnote{In general attribution is marked by mere \isi{juxtaposition} in \JSan (see \sref{ss:JSan_juxt}), so it should come as no surprise that no \cst* marking is present. \JSan has also borrowed the actual \ili{Persian} \ez* morpheme which can co-occur with the \isi{relativizer} following some conjunctions (see \example{17}).} This construction is shown in the following example (=\example{275} and see further \sref{ss:JSan_rel}):

\acex[\JSan]{Noun}{Clause}{275bis}
{xá\cb{} ʿəda našé [ke\cb{} ga\cb{} xá meydā́n smix \cb{}èn]}
{\indef\cb{} few people \rel\cb{} in\cb{} \indef{} square stood.\resl{} \cb{}\cop.3\pl}
{a group of people who were standing in a square}{KhanSanandaj}{380 (1)}

With the exception of these dialects, most \ili{NENA} dialects do not replicate the relativizer-marked clausal attribution construction found in Kurmanji.\footnote{A reviewer claimed that the usage of the \isi{relativizer} is typical of \enquote{local Turkic, \ili{Persian}, and a general areal feature of the languages of Urmi, western Iran and NE Anatolia} while Kurmanji dialects in SE Anatolia and northern Iraq tend to omit the \rel* and thus are more similar to the \ili{NENA} spoken in these regions like \JZax. The evaluation of this claim would require a thorough corpus study of the relevant Kurmanji dialects; in the meanwhile, one can note that \citet[203]{MacKenzie} gives numerous examples of the usage of the \isi{relativizer} in Kurmanji dialects of northern Iraq, while the co-territorial \ili{NENA} dialects lack such a construction, as stated above.}


\paragraph{Marking of prepositions with the \cst* suffix}

In \ili{NENA}, many prepositions can be optionally marked by the \cst* suffix. This could be readily explained for prepositions of nominal origin, but it also holds true for {pure prepositions} which cannot be related to any noun, yielding variant forms such as \transc{ʾəbb-əd}\~\foreign{b-}{in}, \transc{ʾəll-əd}\~\transc{ʾəll-}\~\foreign{l-}{to}, \transc{mənn-əd}\~\foreign{m-}{from} \citep[79]{GoldenbergEarly}. Recall that this fact was one of the main reasons for positing a DAC-origin for the Neo-CSC construction, following \citet{MengozziExtended}, whose argumentation is summarized in \vref{tb:DAC-origin}. 









In contrast to the situation in \ili{NENA}, in \Kur only prepositions of nominal origin can be marked by the \ez*. 
Cohen mentions in this respect the Kurmanji temporal conjunctions, namely \textit{dema}, \textit{gava}, \textit{çaxê} and \textit{wexta}.  To this short list I could  add some more prepositions of nominal character, which take invariably a un-inflecting \ez* \transc{-î}. The relation of this suffix to the inflecting \ez* is somewhat obscure, since this form normally follows the \isi{indefinite suffix} \transc{-ek}. This is shown in the following examples (\ref{ex:784bis}=\example{784} and see further there):

\acex[\Ak]{Adverbial Noun}{Noun}{784bis}
{nêzîk-î ḥakim-i}
{near-\ez{} judge-\obl.\masc}
{near the judge}
{MacKenzie}{161 {[602]}} 

\acex[\Am]{Adverbial Noun}{Adverb}{785}
{pişṭ-î hingî}
{back(\textsc{f})-\ez{} then}
{after that}{MacKenzie}{161}


In other words, in contrast to \ili{NENA}, basic Kurdish prepositions such as \foreign{di}{in} (taking part in circum-positional expressions) never take an \ez* ending.


\acex[\Kur]{Preposition}{Noun}{712}
{di gund-an de}
{in village-\pl.\obl{} in}
{in the villages}{ThackstonKurmanji}{13}

To conclude,  in \ili{NENA}, \cst* marking on prepositions is more readily available than in Kurmanji, and, moreover, this marking is morphologically more transparent. In this, \ili{NENA} resembles in fact \Per, where one finds the \ez* marking also on some prepositions which cannot be considered to be of nominal origin \citep[345, example (16)]{SamvelianHead}.

\paragraph{Adjectival \prims}

In \ili{NENA} adjectives can stand as the heads of an \isi{attributive construction}, and consequently be marked by the \cst* suffix. Such constructions can have several functions, such as marking the adjective as \supr* or as {emotive}\is{emotive genitive} (see for example \sref{ss:JZax_Adj_Head} regarding \JZax).  Another usage, not necessarily the most frequent, is the specification of the adjectival lexeme itself, as in the following example (=\example{557}):

\acex[\Qar]{Adjective}{Noun}{557bis}
{góra xwár-əd kósa}
{man white-\cst{} hair}
{a white-haired man}{KhanQaraqosh}{281}

This last usage is typical of \ili{Semitic} languages, and has been labelled in \ili{Semitic} grammatical tradition \concept{impure annexation}.\footnote{See \citet{GoldenbergAdjectivization} for an analysis of the phenomenon in \ili{Arabic}, and \citet{DoronAdjectival} for a analysis of the phenomenon in \MHeb, cast in formal semantics terminology.} It appears also in \Syr, in which one finds the adjective in the original \isi{construct state} forms, as in the following example (=\example{1039}).

\syacex{Adjective}{Noun}{1039bis}
{ܕܐܢܬ ܗܘ ܡܪܝܐ ܢܓܝܪ ܪܘܚܐ܁ ܘܡܪܚܡܢܐ ܘܣ̇ܓܝ ܚܢܢܐ}
{ʾat \cb{}\textsuperscript{h}u māryā ngir ruḥā wa\cb{} mraḥmānā w\cb{} saggi ḥnānā} 
{2\masc{} \cb{}3\masc{} Lord long.\cst{} spirit and\cb{} merciful and\cb{} great.\cst{} compassion}
{You are the Lord, long-suffering and merciful and of great compassion.}
{\Pesh, Prayer of Manasseh, ed. \cite[A7]{BaarsSchneider};  \cite[217 (7a)]{GutmanVanPeursen}}


In Kurmanji, however, such a construction is rarely found, as adjectives do not inflect in Kurmanji, and cannot receive an \ez* suffix (but see \example{1946} for a possible counter-example). It is rather in Sorani that one finds a similar construction, in which adjectives are head-marked by the \ez*, as in the following example (=\example{915}):

\acex[\KSul]{Adjective}{Noun}{915bis}
{tûş-î em derd-e}
{afflicted-\ez{} \dem.\near{} trouble-\defi}
{afflicted by this trouble}
{MacKenzie}{65 {[67]}}

Note, however, that the corresponding \ili{NENA} construction (\example{557}) occurs also in dialects which are in contact with \Kur dialects. Thus, two possibilities arise: either the construction was borrowed from \Sor and spread beyond the original contact zone; or, more likely, it is a retention of a construction that already existed in the language, but with new morphological marking. 

\paragraph{Adjectival \secns} \label{par:adj_secn}

Another challenge for the pattern borrowing theory is the fact that, while adjectives may follow the \ez* in Kurmanji (see \example{727}), this is not the case in most \ili{NENA} dialects. Adjectives in these dialects never follow  a \isi{construct state} noun. Rather, they stand in \isi{apposition} with a free (non-construct)  \isi{head noun}, while agreeing  in number and gender features. This is demonstrated in the following example (=\example{441}):

\acex[\JZax]{Noun}{Adjective}{441bis}
{xa xamsa sqəl-ta}
{\indef{} maiden(\fem) beautiful-\fem}
{a beautiful maiden}{CohenZakho}{214} 

Yet in  Syriac, one finds an alternative structure, in which adjectives in \isi{absolute state} (glossed \abs) can follow the \d linker:

\syacex{Noun Phrase}{Noun}{1027}
{ܪܘܚܗ ܕܐܢܫܐ ܕܬܒܝܪܐ}
{[ruḥ-ēh d\cb{} nāšā] da\cb{} tbirā}
{spirit(\fem)-\poss.3\masc{} \lnk\cb{} man \lnk\cb{} broken.\abs.\fem}
{the broken spirit of the person} 
{\Pesh, Sirach 4:2 \apud \cite[232]{PeursenBenSira}}

As discussed in \sref{ss:syr_adjSecn} the \abs* of adjectives in Syriac is typical of their predicative usage, and consequently the adjectival \secn in this construction is normally considered to be a nominal clause without an explicit subject argument, or alternatively a quasi-verbal predicate with a \zero\ exponent of the subject.\footnote{Recall that other nominal predicates, including \emp* adjectives, require generally in Syriac a mention of the subject in the form of the \isi{enclitic} personal pronoun. For a discussion of the use of the different states of the adjective, see \citet{GoldenbergPredicative}.}
 Be that as it may, from the perspective of the overt constituents such examples are parallel to the following Kurmanji pattern:

\acex[\Kur]{Noun Phrase}{Adjective}{728}
{[nav-ê wî mirov-î] yê rastîn}
{\hspace{0.7ex}name-\ez.\masc{} \dem.\far.\obl{}  man-\obl.\masc{} \ez.\masc{} real}
{that man's real name}{ThackstonKurmanji}{15}

Examples such as the above could trigger in \ili{NENA} the same  pivot matching process Cohen describes in \JZax for adoption of the Neo-CSC with nominal and clausal attributes; however in most \ili{NENA} dialects it does not occur with adjectives. An exceptional dialect in this respect is \Arb which has cases like the following (and see also \example{1887}):

\lex{\Arb}{1230}
{brā́t-it rubtá}
{daughter-\cst{} big.\fem}
{the eldest daughter} 
{KhanArbel}{229 {[Y:109]}}
 
 Note that, similarly to the Syriac construction, but unlike the Kurmanji one, the adjective agrees with the \isi{head noun}. The discrepancy is not surprising, given that adjectives in Kurmanji cannot inflect
 
 Acknowledging the exceptional case of \JArb, how can the lack of this construction in the majority of dialects be explained?  One possible reason may lie in the above mentioned claim that  the adjectival attribute in Syriac is a minimal nominal clause, marked as predicate by the \isi{absolute state}. In \ili{NENA}, however, the \isi{absolute state} is no longer productively used, and reduced clauses are in general not possible any more, due to the innovation of a quasi-mandatory \isi{copula} paradigm \citep{GoldenbergPronouns, GoldenbergEarly}.\footnote{Occasionally, nominal sentences without a \isi{copula} are found, typically in introductory clauses (see the \Diy examples in \cite[315, \S 13.3]{NapiorkowskaDiyana}). Such clauses are also reported in the dialect of Tel-Kepe (\name{Eleanor}{Coghill}, p.c.). See also the apparently asyndetic relative clauses lacking a \isi{copula} in \JZax \example{451} or \Qar \example{594}, though their clausal status is debated. As for the disappearance of the \abs*, the situation in \ili{NENA} can be contrasted with that in \WNA, where adjectives  can still appear in \abs* \citep[363]{WernerWNA}, and adjectival \secns following a \lnk* can be found: 
  
  \acexfn[\Mal]{Noun}{Adjective}{1884}
  {hanna, ti ipxel}
  {\dem.\masc{} \lnk{} stingy.\abs{}}
  {he who is stingy}
  {WernerWNALehrbuch}{16}
 
   }
  
   
  
  A second reason may lie in the inflecting nature of the Aramaic (indeed \ili{Semitic}) adjective. In contrast to Kurdish, the Aramaic inflecting adjective can be referential and can stand by its own without an explicit nominal antecedent or \lnk* (see \Syr \example{1122}). Thus, cases like \Syr example \ref{ex:1027} or \Arb example \ref{ex:1230} are superfluous, both with respect to the multiple-marking (AC marker + agreement), and with respect to the existence of a simpler \isi{juxtaposition} + agreement pattern. Given that already in \Syr the usage-conditions of the ALC with adjectives are difficult to pinpoint (see discussion in \sref{ss:syr_adj_vs}), it is indeed natural that the \ili{NENA} dialect ousted this construction rather than further grammaticalising it.\footnote{As noted in \vref{ft:ex462}, \name{Eran}{Cohen} (p.c.) suggested to me that it is rather constructions like \foreign{axōna aw rūwa}{the older brother} (=\example{462}), in which the determiner moves to the pre-adjectival position, that replicate the \Kur structure. Yet  a similar construction existed already in Syriac (see \example{2029}). \label{ft:ez_adj}}
  
  Note that in both accounts, internally-motivated developments are more prominent than a possible contact-induced \isi{pattern replication} scenario, thus blocking the occurrence of this construction in most \ili{NENA} dialects. 
 
 \subsubsection{Interim Conclusions} \label{ss:language_contact_conclusions}
 \largerpage
 While the pattern borrowing hypothesis has  merit in its simplicity and apparent elegance, it raises some difficulties in that the Kurmanji pattern is not exactly replicated in most \ili{NENA} dialects. Indeed, taking a broad cross-dialectal perspective, one can establish parallels with various aspects of the Kurmanji pattern (such as the use of adjectives in \JArb, or the \isi{relativizer} in \JUrm), but no single dialect seems to replicate entirely the Kurmanji pattern. While \isi{pattern replication} is never expected to be perfect, it raises the question of whether Kurmanji is indeed the sole source language. In some respects, as stated above, the NENA pattern is in fact more similar to the Sorani pattern. \ref{tb:cst_comp} presents the features discussed above, contrasting 3 NENA dialects, Early J. Cis-Zab\il{Cis-Zab NENA dialects} NENA (\Nrt), and the two main Kurdish varieties. 
 

 \begin{table}[h!t]
 \centering
 \begin{tabularx}{.8\textwidth}{X c c c c c c c c}
 \toprule
					& Ex.			& (a) & (b) & (c) & (d) & (e) & (f) & (g) 	\\
\midrule
 N(P) \lnk\ N(P)	& \ref{ex:727}	& +		& +		& +		& +		& +		& +		& 		\\
 N.\cst\ N(P)	 	& \ref{ex:711}	& +		& +		& +		& +		& +		& +		& +		\\
 \lnk\ inflects 	& \ref{ex:727}	& 		& 		& 		& (+)	&  	& +		& 	\\
 \cst\ inflects 	& \ref{ex:711}	& 		& 		& 		& 		& 		& +		& 		\\
 N(P) \lnk\ Cl.		& \ref{ex:759}	& +		& +		& 		& +		& +		& (+)	& 		\\
 N.\cst\ \rel\ Cl.	& \ref{ex:753}	& 		& 		& +		& 		& 		& +		& +		\\
 Prep.\cst\ N		& \ref{ex:784bis}	& +		& +		& +		& +		& +		& (-)	& 		\\
 Adj.\cst\ N  		& \ref{ex:915bis}	& +		& 		& +		& ?		& +		& 		& +		\\
 Adj. \secn 		& \ref{ex:728}	& 		& +		& 		& ?		& +		& +		& +		\\
 \bottomrule
 \end{tabularx}
 \caption[Comparison of the CSC and ALC across \ili{NENA}, Syriac and Kurdish]{Comparison of the CSC and ALC across \ili{NENA} dialects, Syriac and Kurdish. The example numbers refer to a Kurdish example of respective feature. Entries (+) or (-) indicate some reservations discussed in the appropriate section. (a) = \JZax, (b) = \JArb, (c) = \JUrm, (d) = \Nrt, (e) = \Syr, (f) = \Kur, (g) = \Sor} \label{tb:cst_comp}
 
 \end{table}
 
 
 Clearly, there is a functional similarity between the \ez* marking and the \cst* marking, in that both are head-markers of attribution, and a diachronic similarity in that both originated in \isi{encliticization}.\footnote{The functional similarity has been noted before, for instance by  \citet[381]{MengozziExtended}, and by \citet[4, fn.\ 33]{MutzafiBarzani}.}  \citet[121ff.]{CohenEzafe} attributes the functional similarity    to a specific Kurmanji pivot matching and \isi{pattern replication}, but a viable alternative is to relate it to a more general phenomenon of areal linguistic convergence favouring head-marking of attributive constructions. 
 
 
 
 As  stated above, the \isi{encliticization} process itself, while being clearly an innovation in \ili{NENA}, may be internally motivated, in line with a universal tendency of \isi{encliticization} of functional elements to preceding hosts, as proposed by \citet[395]{LahiriPlank}. It is rather the re-analysis of the resulting \phonemic{-əd} segment as a \cst* suffix which may need an external impetus. Yet in contact situations like the one discussed here, one cannot in fact reliably rule out one explanation in favour of the other. I concur with Cohen that the Kurmanji pattern probably played a role in the formation of the \ili{NENA} Neo-CSC\is{neo-construct-state construction}. Yet, following the \isi{encliticization} process, it could also have risen out of internal analogy with the existing historical \cst* marking, or due to contact with other languages of the area exhibiting \cst* or head-marking morphology.\footnote{These may include \ili{Arabic}, \ili{Syriac} or \ili{Hebrew} in liturgical use, other \ili{Iranic} languages or Kurdish dialects using the \ez* construction and even \ili{Turkish}.} It seems that a reasonable  position would be to relate the Neo-CSC to a linguistic feature present in the \ili{NENA}/Kurdish \isi{Sprachbund}, namely a preference to head-mark attributive constructions, without relating its source to any specific language.\footnote{One may argue that the preference of head-marking is going beyond the nominal domain, since  in the verbal domain there is also a preference for indexing arguments on the verbs rather than marking the arguments by means of case or adpositions.}  Such a position can explain the partial similarities with Kurmanji and Sorani as well as ancient Aramaic strata. One can also go further and propose that the head-marked \ili{Iranic} construction might have its origin in the original \isi{construct state} construction of Aramaic (\il{Aramaic!Classical}Classical Aramaic or possibly anterior strata), which was a language of high prestige in the region in antiquity.\footnote{While at the current state of knowledge this suggestion may sound speculative in nature, it is worthwhile to note in this respect that Middle \ili{Persian}, in which the \ez* construction started to stabilize, is contemporary with \il{Aramaic!Classical}Classical Aramaic, and was clearly influenced by Aramaic by means of the \ili{Pahlavi} (Aramaic-based) script. A somewhat similar suggestion was made by \citet[70f.]{Utas}, who notes that the \transc{ZY} logogram used  to write the \ez* in \ili{Pahlavi} stems from the Aramaic linker \transc{ðı̄}. His account, however, suggests that the \ez* construction is related to the \isi{analytic linker construction}, rather than to the \isi{construct state} construction. Further investigation is needed to elucidate this question.}
  
  
  An argument in favour of this more general explanation is the rise of a Neo-CSC in \WNA. In these dialects a similar suffixed \cst* marker \transc{-il} arose out of the Syriac dative linker construction (see \sref{ss:dat_lnk}). 

\lex{\Mal}{1867}
{berč-il ǧabrōna}
{daughter-\cst{} man}
{the daughter of the man}
{WernerWNA}{301}
 

 
 Since \WNA was not in known contact with any \ili{Iranic} language one must conclude that, in this case, the \isi{encliticization} and reanalysis of the DLC yielding the \transc{-il} suffix were mostly internal processes, possible influenced by the vernacular \ili{Arabic} dialects, which, however, show \cst* marking by stem reduction (and not by suffix). If such influence took place, it was purely a functional one, favouring a head-marked AC, in line with the hypothesis outlined above regarding a general areal preference for head-marking.
 
 
\section{Development of the genitive prefix} \label{ss:genitive_development}

Following the argumentation in \sref{ss:d_gen}, I treat a \phonemic{d} segment preceding certain demonstratives which begin with a \isi{glottal stop} or a (semi)-vowel as a \gen* marker.\footnote{The \JZax \isi{interrogative pronoun} \foreign{ēma}{which}, which exhibits  the genitive form \transc{dēma} can tentatively be analysed as being composed of a frozen \dem* \transc{ē} (identical to the \fem* \dem*) + interrogative \foreign{ma}{what}. An alternative analysis in which the \phonemic{ē} segment is  a reflex of the \ili{Semitic} interrogative *\transc{ay} (cf.\ \BHeb \texthebrew{אֵי}) is less viable due to the presence of the interrogative element \phonemic{ma}, while \BHeb \texthebrew{אֵי} normally combines with deictic elements (\texthebrew{איזה, איפה} etc.). Be it as it may, I assimilate \transc{ēma} to the category of demonstratives in the current discussion.\label{ft:d_ema}} 

In some dialects the justification for such an analysis is clearer, while in others still more research is needed, but as it occurs in quite distinct corners of the \ili{NENA} speaking-area, it seems reasonable to conclude that it is a cross-\ili{NENA} phenomenon, representing a possible shared feature of the \ili{NENA} precursors.\footnote{For the possibility that this development represents a wider areal phenomenon, encompassing also \NMand, see \vref{ft:NMand_Gen}.} 


In \vref{tb:d_gen_dem} I contrast four environments where one finds D-marked demonstratives (acting as \isi{determiners} unless stated otherwise).\footnote{Recall that \citet[90]{CohenNucleus} identifies the same \d prefix as an \concept{attributive} marker of the subordinated \isi{copula} in \JZax (also \sref{ss:JZax_genitive_clauses}). However,  as similar copular forms in other dialects are yet to be investigated, I have not included them in the current comparative study. \label{ft:d_copula}}
The D-markers in the first environment may be analysed either as \gen* markers or as \lnk*s, while in the other 3 environments their analysis as  \gen* markers is more straightforward, as a \lnk* analysis is hardly tenable (see again \ref{ss:d_gen}):\footnote{An empty cell in the table marks the absence of the corresponding construction in the database, but it does not completely exclude its existence in a given dialect, especially for the less-described dialects.}

\begin{enumerate}
\item  Following a noun in \free* (e.g.\ \Amd \foreign{šula d-eyya ṣawaʾa}{the job of this dyer}\footcite[72]{GreenblattAmidya}): Such cases can simply be analysed as instances of the ALC, in which \d serves as a \lnk*.\footnote{The dialects marked by (-) are those in which the available examples include only loan-nouns as \prims,  lacking the distinct \transc{-a} \free* suffix, e.g.\ \JZax \foreign{sabab d-o ʾīzāla dīd-ax}{the reason of your going} (=\example{366}).} Of special interest are dialects which exhibit the N \D-N construction (see \vref{tb:X_d-Y}), but not the \mbox{N \D-[\dem+N]} construction (\Cal, \Jil, \Alq and \Qar),\footnote{In \Diy, the latter construction is found, but \citet[95]{NapiorkowskaDiyana} mentions that  \enquote{the independent relative particle [in fact, the \isi{genitive prefix}] on its own is not always sufficient to express the genitive [=attributive] relation/possession, e.g.: ?\transc{čtawa d-ʾawən}}. } and, conversely, dialects which do not have \d before nouns but preserve it before \dem*s (only \JSan).

\item  Following a \lnk* (e.g.\ \JZax \foreign{ʾōda dīd d-aw gōra}{the room of the man}\footnote{\Example{355}.}): only a handful of dialects (with various linker shapes) show this pattern. Since this pattern exhibits a morpheme serving as a \lnk*, the subsequent \D-marker is naturally analysed as a \gen* marker.\footnote{In \Gaz, the \gen* marking is not certain in this position; see \citet[316 (22)]{GutmanGaznax}. In \Diy I have found this pattern only with an independent demonstrative: \foreign{čtawa ʾəd\cb\ d-ʾawen}{his book} \citep[95]{NapiorkowskaDiyana}.}

\item Following a noun in \cst*: the \cst* may be marked by an \ed  suffix (\Jil \foreign{xabr-əd d-a sawa}{the word of this old man}\footnote{\Example{1477}.}), by \ez* (\JSan \foreign{fešár-e d-o màe}{the pressure of the water}\footnote{\Example{4}.}) or by \isi{apocope} (\JZax \foreign{bēs d-aw gōra}{the house of the man}\footnote{\Example{358}.}).

\item Following a preposition: the preposition itself may be marked as \cst* (\Betn \foreign{bəd d-ayya nūra}{by this fire}\footcite[121 {[500]}]{MutzafiBetanure}), or not (\Betn \foreign{gu d-é\cb{} dāna}{at that time}\footcite[120 {[354]}]{MutzafiBetanure}). 

\end{enumerate}

\begin{table}[p!]
\centering
\begin{tabularx}{\textwidth}{X l c c c c}
\toprule
Region & Dialect			& \free & \lnk	& \cst	&  Prep.  \\
\midrule
{SE Turkey} & \Her 	& (-)  			& 				& + 			& + \\
					& \Boh 	& 				& 				& +				&  \\
					& \Bes 	& 				& 				& 				& + \\
					& \Gaz 	& 				& (+)			& +				& + \\
					& \Baz  & 				& 				& 				& + \\ 
					& \Cal  &  			& 				& 				& + \\
					& \Jil  &  			& 				& + 			& + \\
\midrule
NW Iraq		& \JZax & (-) 			& +				& + 			& + \\ 
					& \JArd &  			& 				& 				&  \\
					& \CArd & + 			& 				& 				& + \\
					& \Barw & + 			& +				& + 			& + \\
					& \Betn & 				& +				& 				& + \\
					& \Amd 	& + 			& 				& +				& + \\
					& \Barz & 				& 				& 				& + \\
					& \Alq 	& 				& 				& 				&  \\
					& \Qar  &  			& 				& (-) 			& + \\ 
\midrule
NW Iran		& \JUrm & (-) 			& +				& +				& + \\ 
					& \Sar 	& 				& 				& +				& + \\
\midrule
NE Iraq 	& \Rus  & 				& 				& 				&  \\ 
					& \DiyZ	& +				& +				& +				& + \\
					& \Arb 	&    			& 				& + 			&  \\ 
					& \JKoy & 				& 				& 				&  \\
					& \JSul & + 			& 				&  			& + \\

\midrule
W. Iran			& \JSan & + 			& 				& +				& + \\ 
					& \CSan & (-)			& 				& +				& + \\
					& \Ker	& 				& 				& 				&  \\
\bottomrule
\end{tabularx}
\caption[Distribution of D-marked demonstrative]{Distribution of D-marked demonstrative following 1) Free state nouns 2) Linkers 3) Construct state nouns 4) Prepositions} \label{tb:d_gen_dem}
\end{table}


As in the development of the Neo-\cst\ \ed suffix, in tracing the development of the \isi{genitive prefix} \d one must clearly distinguish between 1) the phonological process leading to a retention of the \d prefix before the above-mentioned \isi{determiners}, and 2) the morphological reanalysis of this segment as a \gen* marker. Indeed, it is the second stage that may explain the differences in distribution of the \gen* prefix in various \ili{NENA} dialects.

Considering the first stage, recall that the \d prefix is retained before those \isi{determiners} and pronouns that begin either with a weak consonantal onset (typically a \isi{glottal stop} \phonemic{ʾ} but also the semi-vowels \phonemic{w} and \phonemic{y}) or with a vowel. From an articulatory perspective, all these cases can be considered to be vowel-initial.\footnote{An initial \isi{glottal stop} may in these cases be considered as a phonetic support for the initial vowel, rather than a phonemic segment.} 

Thus, a natural hypothesis is to assume that the \gen* prefix originated in the \il{Aramaic!Classical}Classical Aramaic \lnk* \d. As explained above, the \lnk* \d of the ALC or DAC could re-syllabify with vowel-final \prims for syllabic reasons; yet this \isi{resyllabification} was blocked whenever the \secn started with a vowel.\footnote{See \vref{ft:pre_vocalic_retention}.}  This tendency may be still operative in some \ili{NENA} dialects, although exact statistics are hard to gather. As an illustration, in the grammar of \Amd, out of the 4 examples given by \citet[72]{GreenblattAmidya} representing the ALC, 3 have vowel-initial \secns.\footnote{The fourth example has a \phonemic{t}-initial \secn, leading to assimilation of the \lnk* to \phonemic{t}.}

This explanation readily explains the retention of the \d segment before the vowel-initial demonstratives, but it does not provide any reasons for its reanalysis as a \gen* marker. Indeed, since the \d \lnk* is retained as a \isi{proclitic} in such a scenario, there is no change whatsoever in the construction: the ALC and the DAC remain the same. This corresponds to the first column of \ref{tb:d_gen_dem} (labelled \free), but does not explain the occurrence of the \d prefix in the other columns.



A second hypothesis may solve this difficulty.\footnote{I'm indebted to my doctoral supervisor, Eleanor Coghill, for providing me with this idea.} According to this hypothesis, the origin of the \gen* \d is not in the \d \lnk* but rather in the Neo-\cst\ suffix \ed. Given a vowel-initial \secn, the final \phonemic{-d} of the \ed suffix would have a tendency to syllabify with the \secn. This would leave, however, a stranding \isi{schwa} at the end of the \prim. Since a \isi{schwa} in an open syllable is phonologically undesirable \citep[cf.][89]{CoghillAlqosh}, the speakers may rearrange the phonological material in two  distinct ways:

\begin{enumerate}
\item Dropping the \isi{schwa} altogether.\footnote{This is in line with a general phenomenon  of eliding final schwas in some \ili{NENA} dialects, especially when this does not result in a consonant cluster. See \textcites[49f.]{KhanQaraqosh}[88f.]{CoghillAlqosh}. \label{ft:schwa_drop}}



\item Geminating the final \phonemic{-d} segment, leaving the \isi{schwa} in a closed syllable.

\end{enumerate}

In either case, the result is the same: the \prim can be interpreted as being marked for \cst*, either by \isi{apocope} (first case) or by \ed suffix (second case), followed by the \secn marked by a prefix \d. 


In the \Qar dialect such a \isi{resyllabification} is still operative. It is not restricted to \isi{determiners}, and it happens not only before vowel-initial \secns, but also before consonant clusters, as the following examples show (= \example{483} -- \example{480}):\footnote{It may very well be the case that similar \isi{resyllabification} processes are operative in other dialects as well, but due to transcript normalisation practised generally by linguists, this is not always evident in the corpus data. The grammars of Khan are exceptional in this respect, in that the transcription strives to reflect the prosodic structure of the language as accurately as possible.}

\lex{\Qar}{ }
{yál d\cb{} axòna}
{child(ren) -\cst\cb{} brother}
{children of the brother}
{KhanQaraqosh}{208 {[F:3]}}

\lex{\Qar}{480bis}
{ʾít-ə -də\cb{} Šmòni}
{church-\\isi{schwa}{} -\cst\cb{} S.}
{the church of Shmoni}
{KhanQaraqosh}{208 {[K:21]}}

In one example of \Qar, one finds the gemination of the \phonemic{d} segment before an attributive demonstrative, providing the exact environment where it could be reanalysed as a \gen* prefix, though this did not seem to happen in \Qar\ (=\example{494} with a different gloss):

\acex[\Qar]{Noun}{Noun}{494bis}
{b\cb{} paqárt-əd d\cb{} áne ḥawāwī̀n}
{in\cb{} neck-\cst{} \gen(?)\cb{} \dem.\far.\pl{} animals}
{on the neck of those animals}
{KhanQaraqosh}{208 {[B:72]}}

















In other dialects, where the  \gen* prefix is better established, it is also possible to observe examples of the intermediate stage, where the \prim is marked only be a \isi{schwa} suffix, instead of a full \ed suffix or \isi{apocope}. This is the case in \Barw:\footnote{In \Barw one also finds  examples of \isi{resyllabification} of the \cst* suffix before the \isi{copula}, which may be the first stage for the creation of a special subordinated \isi{copula}, as in \JZax (see \sref{ss:JZax_genitive_clauses}): 

\acexfn[\Barw]{Noun}{Clause}{1454}
{kú= duk-ə -ṱ\cb{} íla mɛθ-ə̀t-la-li.ˈ}
{every= place-\\isi{schwa}{} -\cst\cb{} \cop.3\fem{} bring-\agent2\masc-\patient3\fem-\dat1\sg}
{Bring her to me wherever she is.}
{KhanGenitive}{82} 

}

\lex{\Barw}{1361}
{gu\cb{} ṣádr-ə d-áwwa sùsa}
{in\cb{} chest-\cst{} \gen-\dem.\masc{} horse}
{in the chest of this horse}
{KhanBarwar}{397 {[A14:67]}}

In \JUrm this phenomenon is found with \isi{adverbial} \prims (=\example{201}): 

\acex[\JUrm]{Adverbial Noun}{Noun}{201bis}
{⁺m-qulb-ə d-o gora}
{from-stead-\cst{} \gen-\dem.\far.\sg{} man}
{instead of that man}
{KhanUrmi}{196}





As the above \Qar examples show, the prosodic \isi{resyllabification} of the \phonemic{d} segment with \secns is not restricted to demonstrative \secns. Yet  only with demonstratives was this segment re-analysed as a \gen* prefix, leading in turn to its  occurrence in contexts where there was no \d segment initially, such as  in the following example (=\example{1409}):

\acex[\Barw]{Noun}{Noun Phrase}{1409bis}
{gnáy-ət [táwra d-o\cb{} gòṛa]}
{fault-\cst{} \hspace{0.7ex}ox \gen-\defi.\masc\cb{} big.\masc}
{the fault of the big ox}
{KhanBarwar}{517 {[D2:19]}}




 This leads to the following question: why was the \d segment reanalysed as a \gen* marker only before demonstratives?\footnote{As noted in \vref{ft:d_ema}, the \isi{interrogative pronoun} \foreign{ēma}{which} is analysed here as containing a \dem* element \transc{ē}. On the other hand, I do not treat here the question of the \d marked subordinated \isi{copula}, which has only been clearly analysed as such in \JZax (but see previous footnote).} This question is even more pertinent, as the reintroduction of a \gen* \isi{case marker} goes against the above-discussed areal preference of head-marking and the lack of a case system in Aramaic since antiquity. Moreover, the fact that this marker is a prefix goes against the cross-linguistic dis-preference of prefixes.\footnote{\citet{HaspelmathWord} claims that cross-linguistic generalisations about affixes are problematic due to the difficulty of defining the notion of affix as a \concept{comparative concept}. \citet{DryerAffix}, however, shows how to establish it as a comparative concept and affirms the above mentioned tendency.}

One possibility is that the  vowel-initial demonstratives occur in high frequency amongst the vowel-initial \secns. Since the number of these items is quite limited, their appearance with the \d marker  is high enough to permit a reanalysis. 

Setting aside pure frequency effects, one may seek a structural motivation for re-analysis. Thus, \citet[71]{KhanGenitive} suggests that the introduction of the \d segment in these contexts arises from an analogy to the genitive independent pronouns, some of which start with \d. As an example, he shows the analogy between \Barw \transc{bɛθa diy-a} and \transc{bɛθa d-ay}, both meaning \transl{her house}. This, however, would seem to explain only the occurrence of \d with independent demonstrative pronouns; its co-occurrence with attributive demonstratives would need a further step of analogy.

Another possible source of analogy is \isi{language contact}. As is shown in \vref{tb:kur_dem}, Kurmanji Kurdish possesses a series of \obl* demonstratives, which, in contrast to the nominative demonstratives, are consonant initial \citep[10]{ThackstonKurmanji}.\footnote{The series of the far-deixis demonstratives is the same, with the \phonemic{v} replaced by \phonemic{w}.} Note that, similarly to the \ili{NENA} demonstratives, the \Kur ones function both attributively and independently.

\begin{table}[h!]
\centering
\begin{tabular}{l l l}
\toprule
		& \nom 							& \obl \\
\midrule
\masc	& \multirow{3}{*}{\transc{ev}}	& \transc{vî} \\
\fem	&								& \transc{vê} \\
\pl		&								& \transc{van} \\
\bottomrule
\end{tabular}
\caption{Kurmanji near-deixis demonstratives}  \label{tb:kur_dem}
\end{table}

 Kurmanji \obl* case is used for 3 main functions: 
 
 \begin{enumerate}
 
 \item Marking of complements of verbs, corresponding to an \acc* use.\footnote{In the past tense, which exhibits an ergative alignment, the ergative argument is marked in the \obl* case.}
 
 \item Marking nominal complements (typically possessors) of \ez*-marked nominal heads, corresponding to a \gen* use. 
 
 \item Marking complements of adpositions, corresponding also to a \gen* case (at least in the \ili{Semitic} case-marking languages).
 \end{enumerate}
 
 Note that the \ili{NENA} innovated \gen* prefix occurs in functions 2 and 3. In these environments the Kurmanji oblique demonstratives may have served as pivots for the reanalysis of the \ddet\ complex as a case-marked \dem*. The similar syllabic structure of the two elements (CV) may have been a further facilitating factor. 
 
 As \citet[124]{CohenEzafe} notes, such an hypothesis poses a difficulty, since the \ili{NENA} \d marked demonstratives are not used for complements of verbs, i.e.\ in an \acc* context.\footnote{An exceptional dialect in this respect is \JSan, which does use the \d marker to mark complements of verbs; see \ref{ss:JSan_gen_verbal} and in particular \example{JSan1bis}. In other \ili{NENA} dialects, the \gen* case is sometimes used to mark complements of verbal nouns, i.e.\ infinitives \citep[cf.][37, fn.\ 8]{SabarNerwa} as well as complex-predicate nouns, but this is in all probability related to the nominal character of these heads.} One may solve this difficulty, however, by assuming that the \ili{NENA} speakers did not generalize the occurrence of the \d segment outside its initial domain of appearance, but restricted its reanalysis to the AC domain. 
 
A partial corroboration of the above hypothesis lies in the fact the usage of \gen* demonstratives seems to be more restricted in the non-Kurmanji speaking areas (roughly North-East Iraq south of Arbel and West-Iran). Thus, while about 80\% of the dialects surveyed in the Kurmanji-speaking area show \gen* case after prepositions, this is true only in  about half of the dialects in the non-Kurmanji speaking area.\footnote{The last figure should be taken with some caution due to the small sample of dialects in the non-Kurmanji area.} Similarly, marking of \gen* case after the \lnk* is present only in the Kurmanji-speaking area. Indeed, from \vref{tb:d_gen_dem} one may tentatively conclude that the innovation of the \gen* marker as such occurred at first in North-West Iraq, in the heart of the Kurmanji speaking zone, and spread out from there. 

\section{Development of alternative linkers} \label{ss:alt_lnk}

In \sref{ss:d_lnk_dist} I have surveyed the distribution of the inherited Syriac \lnk* \d. Many dialects, however, exhibit alternative linker forms, which may co-exist or supersede the \d \lnk*. \Vref{tb:alt_lnk} summarizes the various alternative \lnk*s which are found in each dialect. The first column, essentially identical to the first column of \vref{tb:X_d-Y}, states whether the \d \lnk* (\~\phonemic{əd}\~\phonemic{də}\~\phonemic{t}) is found in each dialect before nominal \secns.\footnote{Note that in \Her one finds the \d \lnk* only without immediate \prims.} The second column gives alternative forms which function as \lnk*s in each given dialect.\footnote{Some of these \lnk*s can appear both before clausal and nominal \secns, while others select only nominal \secns.} The third column gives the bases of the independent attributive pronouns (formed as base + \isi{possessive suffix}, often termed \concept{independent genitive pronouns}), which in some respects can be analysed as \lnk*s (see discussion below).

\begin{table}[p!]
\centering
\begin{tabular}{l l c c c}
\toprule
Region & Dialect			& \d \lnk* & Alt. \lnk* &  Pronominal base \\
\midrule
{SE Turkey} & \Her 	& (+) & \transc{did} & \transc{did-}, \transc{d-} (alt. 2\&3\pl) \\ 
					& \Boh 	&  & &  \transc{did-}, \transc{d-} (2\&3\pl)\\
					& \Bes 	&  & \transc{ad} & \transc{diy-} \\
					& \Gaz 	&  & \transc{ad} & \transc{diy-}\\
					& \Baz  & + & \transc{ʾəd} & \transc{diyy-}\\ 
					& \Cal  & + & \transc{ʾəd} & \transc{did-}, \transc{d-} (1\&2\pl)\\
					& \Jil  & + & & \transc{diy-} \\
\midrule
NW Iraq		& \JZax &  & \transc{dīd}, (\transc{ʾōd}) & \transc{did-} (\sg), \transc{d-} (\pl) \\ 
					& \JArd &  & & \transc{did-} (\sg), \transc{d-} (\pl)\\
					& \CArd & + & & \transc{dīy-}\\
					& \Barw & + & & \transc{diy-}\\
					& \Betn &  & \transc{dəd} & \transc{did-} (\sg), \transc{d-} (\pl) \\
					& \Amd 	& + &\transc{dəd} & \transc{did-} (\sg), \transc{d-} (\pl)\\
					& \Barz &  & \transc{ʾod} & \transc{did-}\\
					& \Alq 	& + &  & \transc{diy-} \\
					& \Qar  & + & \transc{ʾəd}  & \transc{did-} \\
\midrule
NW Iran		& \JUrm &  & \transc{ay} & \transc{did-} \\ 
					& \Sar 	&  & & \transc{əd-}\\
\midrule
NE Iraq 	& \Rus  &  & \transc{i} & \transc{did-} \\
					& \DiyZ  & + & \transc{ʾəd}	& \transc{did-}, \transc{diy-}, \d (alt.\ 3pers.) \\
					& \Arb 	&  & \transc{ot} & \transc{did-} \\

					& \JKoy &  & \transc{od} & \transc{did-} \\
					& \JSul & + &   & \transc{did-}, \transc{d-} (alt. 3pers.)\\

\midrule
W. Iran			& \JSan &  & & \transc{did-} (1\&2), \transc{d-} (3)\\ 
					& \CSan &  & & \transc{did-}, \transc{diy-} \\
					& \Ker	&  & & ? \\
\bottomrule& 
\end{tabular}
\caption{Alternative linkers} \label{tb:alt_lnk}
\end{table}

\subsection{Bases of independent attributive (genitive) pronouns} \label{ss:pron_base}

Before discussing the forms of the independent attributive pronouns' bases, we must clarify their relation to the (independent) pronominal linkers. Given a form like \transc{did-i}, there are two distinct synchronic analyses available:

\begin{enumerate}

\item \transc{did-} is a pronominal \lnk* representing a \prim; the \isi{possessive suffix} \mbox{\transc{-i}} represents a pronominal \secn. This is clearly the case in \JZax, where one finds examples such as the following (part of \example{304}). To clarify the analysis, the two pronominal elements are represented in the literal translation by subscripts.

\acex[\JZax]
{Preposition}{Pronoun}{2015}
{mən dīd-i}
{from \lnk-\poss.1\sg}
{from mine, lit. from that\textsubscript{1} which belongs to me\textsubscript{2}}
{CohenZakho}{\newline 95 (5)}


\item \transc{did-} is a semantically empty base, thus not representing pronominally a \prim and having no semantic contribution. It is present only to enable the \isi{possessive suffix} \transc{-i} to stand as part of an independent word. For this function I use the term \concept{genitive base}, as the resulting word is the genitive counter-part of the independent (nominative) pronoun \foreign{ana}{I}. This is the case in \JUrm, where one finds the following example (=\example{187}):

\acex[\JUrm]
{Preposition}{Pronoun}{187bis}
{bo-d\cb{} did-ew}
{because-\cst\cb{} \gen-\poss.3\masc}
{because of him\textsubscript{2}}
{KhanUrmi}{192}

\end{enumerate}

\textit{A priori}, in dialects where the attributive pronominal base and the \lnk* have the same form (ignoring allophony related to stress placement\footnote{Thus, \transc{did-} is equivalent to \transc{dəd}, and arguably also \transc{diy-} is equivalent to \transc{d-} with a glide inserted for phonological reasons (see \vref{ft:diy-}).}), there is no reason to analyse them as two different morphemes, and the first analysis should be favoured. Thus, in these dialects, the pronominal \lnk*, similarly to other nouns which it replaces, can be followed either by a noun or by a pronominal \isi{possessive suffix} (see for instance the discussion of the \JZax \lnk* in \sref{ss:JZax_Lnk}). 

In other dialects, where the form of the \lnk* differs from the attributive bases (such as \JUrm, where the \lnk* \transc{ay} is clearly distinct from the pronominal base \transc{did-}), the motivation for such an analytical move is weaker, and the second analysis is probably more appropriate. Indeed, in some of these dialects the independent attributive pronouns  occur after \cst* marked prepositions or nouns; see \JUrm \example{187bis} or \Barz \example{820}. As this is atypical for a pronominal \lnk* (see \sref{ss:Analytic_AC}), it is a further indication that they should be analysed as separate morphemes semantically bleached of a pronominal \prim. The    dialectal distribution of this construction is given in \vref{tb:double}, and see also discussion in \ref{ss:double} there.

It should be noted, however, that in predicative position the difference between the two functions (\lnk* vs.\ \gen*) is neutralised, since predicates are in general non-referential and thus lack pronominal force.\footnote{This is also true of adjectives, which in general lose their referential function, or covert \isi{pronominal head}, in predicative position. Recall that in Syriac this is manifested by the use of the \abs*, as discussed in \vref{ft:syr_abs_adj}.} Thus,  an independent genitive pronoun, just as a pronominal \lnk*, can appear in predicative position without an explicit \prim. This is illustrated in \JUrm \example{231} and the following similar \Arb example: 


\acex[\Arb]
{\zero}{Pronoun}{1277}
{kullà \zero{} did-ŏ́x \cb{}ilu.ˈ}
{all \zero{} \gen-\poss.2\masc{} \cb{}\cop.3\pl}
{They all belong to you.}
{KhanArbel}{220 {[S:84]}}




Considering now the form of the these bases, it is clear that across the \ili{NENA} dialects they present a coherent form being in general \transc{did-} or \transc{diy-} (with the rare exception of \Sar \transc{əd-}).\footnote{In \ili{NWNA} \Midn one finds \transc{diḏ-} as well \citep[43]{JastrowMidin}.} It is thus safe to assume that these forms stem back to the \ili{NENA} precursors. The form \transc{did-} as a pronominal base is present since antiquity in \JBA, where it is usually assumed to be a product of assimilation of the Official Aramaic pronominal base \transc{dil-}, which is in turn retained in Syriac \citep[108]{BarAsherJBA}.\footnote{For Nöldeke's explanation, see \vref{ft:Noldeke_did}.} 
Since no \ili{NENA} dialect shows the base \transc{dil-}, one can assume that in this respect the \ili{NENA} precursor (or precursors) diverged from Syriac and was closer to \JBA. As for the form \transc{diy-}, present especially in Turkey and some Iraqi dialects (as far south as Tel-Kepe\footnote{\name{Eleanor}{Coghill}, p.c.}), this may result from a further phonetic mutation of \transc{did-}, or as a phonetic extension of the simple \lnk* \transc{d(i)-} before possessive suffixes.\footnote{The latter view is endorsed by \citet[72]{SinhaBespen}. Evidence for this can be found in the fact that the 1\sg\ form in \Bes is \transc{diʾi}\~\transc{di}. Furthermore, other prepositions in the  \ili{Judi-dialects} of Turkey are extended in this way. For instance, the preposition \transc{b-} in \Gaz is rendered \transc{biy-} before pronominal suffixes \citep[315]{GutmanGaznax}. A similar analysis for \Diy in NE Iraq is given as a possibility by \citet[93]{NapiorkowskaDiyana}. She suggests, moreover, that \transc{diy-} may originate in influence from the literary Christian Urmi \ili{NENA} dialect \citep[cf.][198]{MurreUrmi}. The latter idea may also explain the occurrence of \transc{diy-} in \CSan.\label{ft:diy-}} If the latter view is true, this may mean that the ancestors of these dialects never made use of a \transc{did-} base, but rather contended with a \d base, identical to the erstwhile \lnk*. Indeed, in many \ili{NENA} dialects (mostly Jewish) the base \d is also used alongside a \transc{did-} base, but it is normally restricted to some or all of the \pl* persons.\footnote{In some dialects, namely \JSan, \JSul and \Diy, the \d prefix is used  with \third person genitive pronouns, singular and plural.  Note though that in these cases these forms are identical or similar to the genitive demonstratives and do not incorporate the pronominal suffixes.} The reason for this seems to be syllabic: the \pl* pronominal suffixes are bi-syllabic, while the \sg* ones are mono-syllabic. Using alternatively the \d and \transc{did-} bases guarantees bi-syllabicity across the paradigm. This is demonstrated in \vref{tb:Amd_gen_pron} with the paradigm from \Amd \citep[81]{GreenblattAmidya}, which is identical to the paradigm of \JZax \citep[453]{CohenZakho} except for the marking of length. A similar system is attested in \Nrt \citep[135]{SabarNerwa}, representing early J. Cis-Zab\il{Cis-Zab NENA dialects} dialects. This corroborates the idea that the \d \lnk* was  available in Proto-Cis-Zab to act as a pronominal basis, and possibly also in other NENA precursors.

\begin{table}[h!]
\centering
\begin{tabular}{l l l}
\toprule
			& \sg				& \pl \\
\midrule
1			& \transc{did-i} 	& \transc{d-eni} \\
2\textsc{m}	& \transc{did-ux}	& \multirow{2}{*}{\transc{d-oxun}} \\
2\textsc{f} & \transc{did-ax}	& \\
3\textsc{m} & \transc{did-e	}	& \multirow{2}{*}{\transc{d-ohun}} \\
3\textsc{f}	& \transc{did-a} 	& \\
\bottomrule
\end{tabular}
\caption{Independent genitive pronouns in \Amd or \JZax} \label{tb:Amd_gen_pron}
\end{table}


\subsection{The \transc{did} linker} \label{ss:did_lnk}

In a group of dialects in North-West Iraq (\JZax, \Betn, \Amd), as well as \Her, the Syriac \d \lnk* is replaced for the most part by the form \transc{did} (\~\transc{dīd}\~\transc{dəd}) pre-nominally. In \JZax \transc{did} also appears alongside \d in its role as a \isi{relativizer} (as in \Amd), and as a \isi{complementizer}.\footnote{This may be true also of \Betn and \Her, but I haven't found such occurrences in the available sources.}

Assuming that the role of \transc{did-} as a genitive pronominal base is prior to its use as a \lnk*, a natural hypothesis would be that these dialects generalized its use from a pre-suffixal linker to a general linker, capable of appearing before any \secn (nominal as well as clausal). Indeed, in these dialects, there is no reason to analyse the pre-suffixal base and the \lnk* \transc{did} as two separate morphemes.

Another possibility is to relate the appearance of the \transc{did} \lnk* to the emergence of the Neo-\cst\ suffix \ed discussed above. As we have noted in the previous chapters, the \lnk* is functionally equivalent to a noun in \cst*. With the emergence of the suffix \ed the speakers had the possibility to mark this explicitly by suffixing \ed to the \lnk* itself, yielding \phonemic{d(ə)} + \phonemic{-əd} = \phonemic{dəd}\~\phonemic{dīd}.\footnote{The form \phonemic{dīd} is found frequently in \JZax. Note that \phonemic{ī} is simply the long counterpart of \phonemic{ə}. In some dialects closed mono-syllabic words are always realised with a long vocalic nucleus \citep[cf.][307, fn.\ 6]{GutmanGaznax}. Thus the lengthening of the vowel in \transc{dīd} is an automatic phonological process related to the fact that it becomes an independent stress-bearing word.}
 Note that in all dialects which have the \transc{did} \lnk* the suffix \ed is highly productive. Further evidence may be adduced by the fact that in Early J. Neo-Aramaic, namely \Nrt\, this being the closest predecessor of \JZax, \Amd\ and \Betn, the \ed suffix is productive but no \transc{did} \lnk* is apparent, except as a pre-suffixal basis. Thus, it seems indeed that the development of the \ed suffix pre-dated the appearance of \transc{did} as an independent \lnk*, at least in the case of the Jewish dialects.\footnote{The form \transc{did} is also lacking from the Early Christian Neo-Aramaic poetry published by \citet{MengozziPoetry2002, MengozziPoetry2011}, where only \transc{di-}\~\transc{diy-} and \transc{dil-} are present (see glossary of \cite[205f.]{MengozziPoetry2002}). These poems, however, originate in the region of Alqosh, where the \transc{did} \lnk* did not develop at all.}

The two above explanations are in fact not mutually exclusive but rather complementary. The development of the \isi{neo-construct} suffix \ed may have eased the integration of the pre-suffixal basis \transc{did-} as an independent \lnk*, due to its reanalysis as \phonemic{d+əd}. This may also explain why in these dialects the usage of the original \lnk* form \d has diminished, occurring in some dialects (such as \JZax) predominantly with clausal \secns.

\subsection{The \transc{ad}, \transc{od} and \transc{ʾəd} linkers} \label{ss:od_lnk}

Both in South-East Turkey and in Iraq one finds linkers consisting of a vowel followed by \phonemic{-d}. The forms \transc{ad} and \transc{od} stem in all probability from the demonstrative + linker construction present in Syriac (see \sref{ss:syr_corr}). Note that, in Syriac, the pre-linker demonstrative, traditionally termed \concept{correlative}, appears especially (but not exclusively) before clausal \secns, but in the \ili{NENA} dialects where such linkers appear they are regularly followed by nominal \secns. This may hint that the situation in Syriac was rather exceptional compared to the precursors of these \ili{NENA} dialects. 


 As for the \lnk* \transc{ʾəd}, it may  result from a phonetic reduction of the former linker forms, or it may be a phonetic variant of the simple \lnk* \phonemic{d(ə)-}.\footnote{See also \vref{ft:əd_lnk}.} For the present discussion I leave this question open, and I shall concentrate on the clear forms \transc{ad} and \transc{od}.

\subsubsection{J. Arbel contrasted with C. Barwar} \label{ss:Arb_Barw_lnks}

\citet[224]{KhanArbel} analyses the \Arb \lnk* \transc{ʾot} as stemming from \foreign{ʾo + t}{the one of}, and \citet[67]{PatElCorrelative} relates it to the Syriac construction, explaining it as a \enquote{conflation of a Syriac-like \transc{*haw d-}}. Indeed, the \sg* (far-deixis) \dem* in \Arb is \transc{ʾo}, with no number distinction \citep[85]{KhanArbel}. The emergence of the \lnk* \transc{ot} may be seen as product of the same process leading to the emergence of the Neo-\cst\ \ed suffix, namely \isi{resyllabification} of \d with the preceding element and their subsequent reanalysis as one morpho-syntactic unit (see \sref{ss:NeoCSC_Origin}). In this case, however, a further step of \isi{grammaticalisation} took place, since the \transc{ʾot} \lnk* lost the \sg\ number feature associated with the original demonstrative:

\lex{\Arb}{1285} 
{bšilmāneˈ ʾot\cb{} Àrbelˈ}
{Muslims \lnk\cb{} A.}
{the Muslims of Arbel}
{KhanArbel}{224 {[L:42]}}

Moreover, \textit{pace} Pat-El, in contrast to the Syriac source construction, the \transc{ʾot} \lnk* does not induce a definite reading on the entire NP, and it can have an indefinite antecedent (possibly with a generic reading):\footnote{Such examples, however, are relatively rare, possibly due to the general tendency of attribute constructions to be definite \citep{HaspelmathArticle}.}

\acex[\Arb]
{Noun Phrase}{Noun}{Eleanor_Arb}
{[xanči masale] ʾot\cb{} ʾarbel}
{some stories \lnk\cb{} A.}
{some stories of Arbel}
{KhanArbel}{224}

\lex{\Arb}{1301}
{(xá\cb{} sinn-it dī̀b k-imr-í-wā-le,ˈ) káka ʾod\cb{} dī̀bˈ}
{\indef\cb{} tooth-\cst{} wolf, \ind-say-\agent3\pl-\pst-\patient3\masc{} tooth \lnk\cb{} wolf}
{(It was called a wolf's tooth,) a tooth of a wolf.}
{KhanArbel}{228 {[L:209]}}

Furthermore, in contrast to demonstratives, the \lnk* itself can serve as a generic indefinite head:

\acex[\Arb]{\zero}{Clause}{1255}
{ʾot\cb{} k-e-wa xa\cb{} čày k-míx-wā-leˈ}
{\lnk\cb{} \ind-come-\pst{} \indef\cb{} tea \ind-bring-\pst-\dat3\masc}
{whoever came, we would bring tea for him}
{KhanArbel}{170, 454 {[L:229]}}

Thus, we can conclude that \transc{ot} was grammaticalised as a general pronominal \lnk*, losing the grammatical features and semantic weight associated with the original \dem* element \transc{ʾo}.
This can be contrasted with \Nrt, representing Early J. Cis-Zab\il{Cis-Zab NENA dialects} Neo-Aramaic, where the forms \transc{ʾaw-d} and \transc{ʾay-d} can still be analysed as inflecting \dem*s with a definite semantic value followed by an \isi{enclitic} \transc{-d} \lnk* or \cst* suffix (see \examples{2003}{2001} and discussion there).  A similar situation exists in \Barw, where an \transc{ʾo-t} element is clearly segmentable into two distinct elements, an inflecting attributive demonstrative (which I analyse as a definite determiner in \Barw\footnote{See \vref{ft:barw_defi}. Preceding a \lnk* it marks the \lnk* phrase as definite.}) and a \isi{clitic} \lnk*:\footnote{The \phonemic{t} segment is bound phonologically both forward and backward in Khan's transcription. I assume that both the determiner \transc{ʾo} and the \lnk* \transc{t} are proclitics.}



\acex[\Barw]{Noun Phrase}{Noun}{1389}
{xón-e diy-e ʾo\cb{} t\cb{} Nìnweˈ}
{brother-\poss.3\masc{} \gen-3\masc{} \defi.\masc\cb{} \lnk\cb{} N.}
{His brother from Nineveh}
{KhanBarwar}{493 {[A13:3]}}

\lex{\Barw}{1390}
{yale ʾan\cb{} t\cb{} xal-i}
{children \defi.\pl\cb{} \lnk\cb{} uncle-\poss.1\sg}
{The children of my maternal uncle}
{KhanBarwar}{493 {(32)}}

\acex[\Barw]{Noun Phrase}{Clause}{1453}
{(ʾə́θyɛ-le) [ʾo\cb{} gàwṛa díy-a],ˈ ʾo\cb{} t\cb{} [wéwa mùθyə-lla].ˈ}
{came-3\masc{} \defi.\masc{} man \gen-3\fem{} \defi.\masc\cb{} \lnk\cb{} \cop.\pst.3\masc{} bought-\patient3\fem}
{(He came back,) her husband, the one who had brought her.}
{KhanBarwar}{957 {[A12:53]}}

\acex[\Barw]{Noun Phrase}{Clause}{1421}
{m\cb{} [bnōn\cb{} mám-i] ʾan\cb{} t\cb{} [wɛ́wa gòṛe].ˈ}
{from\cb{} sons.\cst\cb{} uncle-\poss.1\sg{} \defi.\pl\cb{} \lnk\cb{} \cop.\pst.3\pl{} old.\pl{}}
{from my cousins, who were older}
{KhanBarwar}{518 {[B8:5]}}

The \Barw determiner can even be marked as \gen*, if it occurs within a genitive NP:

\acex[\Barw]{Noun}{Noun--Clause}{1452}
{bráta [d-o\cb{} Xáno Lapzèrin,ˈ d-o\cb{} t\cb{} wewa bə́nya Dəmdə̀ma]ˈ}
{daughter \gen-\defi.\masc{} X. L. \gen-\defi.\masc\cb{} \lnk\cb{} \cop.\pst.3\masc{} built D.}
{the daughter of that Xano the Golden Hand, who had built Dəmdəma}
{KhanBarwar}{957 {[A11:17]}}

\citet[83f.]{KhanGenitive} notes that in \Barw this construction is available only for definite antecedents with non-restrictive relative clauses, while in \Arb the usage of \transc{ʾot} is generalized to restrictive relative clauses as well. He considers, moreover, the \Barw situation to be \enquote{typologically more archaic}, representing, in other words, an earlier stage of the development of these constructions.\footnote{There is yet another distribution in \JSul, where \transc{ʾot} heads only free (antecedent-less) relative clauses, whether they are definite or not \citep[418]{KhanSulemaniyya}. Note that the \dem* \transc{ʾo} in \JSul does not carry  number or gender features \citep[77]{KhanSulemaniyya}. For the sporadic use of \transc{od} in \JUrm, see \sref{ss:JUrm_od}.

On the other hand, also in \Arb there are, alongside the \lnk* \transc{ʾot}, attributive demonstratives which can be marked by \cst* suffix: the \pl* \dem* \transc{ʾinná-t} and the \sg* (near-deixis) \dem* \transc{ʾiyyá-t}. In contrast to the \lnk* they conserve their definite reading:

\acex[\Arb]{Pronoun}{Clause}{1324}
{mindí ʾiyyá-t zwìn-ni bāq-áwˈ nbìl-lu-lleu.ˈ}
{thing \dem.\sg.\near-\cst{} bought-1\sg{} for-3\fem{} took-\agent3\pl-\patient3\masc}
{The thing that I bought for her - they took it}
{KhanArbel}{388 {[L:408]}}

\acex[\Arb]{Noun Phrase}{Noun Phrase + Clause}{1887}
{sìmun,ˈ nāš-ít saràyˈ kul-lù qṭolún-nuˈ ʾínna-t rúww-ake ʾód did-xùnˈ ʾínna-t itiwé-lu b\cb{} sarày.ˈ}
{go.\imp{} people-\cst{} S. all-3\pl{} kill.\imp-3\pl{} \dem.\pl-\cst{} great-\defi{} \lnk{} \gen-2\pl{} \dem.\pl-\cst{} sit.\pst-3\pl{} in\cb{} S.}
{Go and kill all the people of the \textit{saray} (government office), your great men, who reside in the \textit{saray}}
{KhanArbel}{170, 510 {[Y:174]}}

In some \ili{NENA} dialects, \Arb included, adjectives can be nominalized by means of a \dem*:\footnote{According to my survey, similar patterns exist in   \Amd,  \Barw, \CArd, \Diy, \JZax and \Qar, and possible more dialects.}

\acex[\Arb]{\zero}{Adjective}{1309}
{ʾó zurtá}
{\dem{} small.\fem}
{the small one}
{KhanArbel}{229 {[L:214]}}


Quite exceptionally in \Arb, however, the \dem* in this environment is interchangeable with the \lnk* \transc{ʾot}:


\acex[\Arb]{\zero}{Adjective}{1310}
{ʾot\cb{} rabtà}
{\lnk\cb{} big.\fem}
{the big one}
{KhanArbel}{230 {[B:10]}}

Indeed, the latter possibility is available in \Arb as part of the general availability in this dialect of having adjectival \secns following a \cst* head (see \example{1230}). Yet  the similarity with \Sor \lnk* \ez* construction used to nominalize adjectives (see \example{922}) should not be overlooked. In \Barw a similar construction is also available, but only when the \secn \enquote{adjective is extended by an intensifier or by the comparative particle \transc{biš}} \citep[509]{KhanBarwar}. 

\acex[\Barw]
{\zero}{Adjective Phrase}{1400}
{ʾo \cb{}t biš\cb{} daqìqaˈ}
{\defi.\masc{} \cb{}\lnk{} more\cb{} thin.\fem}
{the one that is thinner}
{KhanBarwar}{509 {[B10:49]}}


This is reminiscent of the situation in \Syr, in which the \lnk* \d is used especially when preceding multi-word adjective phrases (see discussion in \sref{ss:syr_adj_vs}). 



In \Arb, there are rare instances of \cst* marked \prims preceding the \lnk*. While these are marginal in \Arb, they may represent the first step of a process of \isi{grammaticalisation} in which the \cst* \ed suffix becomes obligatory in ACs, irrespectively of the appearance of the appearance of the \d \lnk* or a derivative thereof. Synchronically, this may be analysed as an (optional) agreement-in-state pattern between the \prim and the \lnk*  (see also \sref{ss:double}):

\lex{\Arb}{1289}
{jirān-ít ʾót hulaʾèˈ ga\cb{} Šaqlàwaˈ}
{neighbour-\cst{} \lnk{} Jews in\cb{} Š.}
{the neighbour of the Jews in Shaqlawa}
{KhanArbel}{224 {[L:411]}}

In \Koy, discussed below, this process has gone further.

\subsubsection{J. Koy Sanjaq} The \lnk* \transc{ʾod} of \Koy can similarly be analysed as a grammaticalised combination of the far-deixis demonstrative \transc{ʾo} + \cst* suffix \transc{-d}. In this dialect the original \lnk* \d is not used any more. A peculiarity of \Koy is that the \lnk* often follows a \cst* \prim, becoming effectively a construct state agreement marker. It can co-occur with (pro)nominal and clausal as well as ordinal \secns:  

\acex[\Koy]{Noun}{Pronoun}{1501} 
{bel-a ʾod did-i}
{house-\free{} \lnk{} \gen-1\sg}
{my house}
{MutzafiKoySanjaq}{62}

\lex{\Koy}{1509}
{b\cb{} wáxt-əd ʾod bāb-ew}
{in\cb{} time-\cst{} \lnk{} father-\poss.3\masc}
{in the time of his father}
{MutzafiKoySanjaq}{63 {[\N8]}}

\acex[\Koy]{Noun}{Clause}{1510}
{məndixā́n-əd ʾod ʾabe mən šār}
{items-\cst{} \lnk{} want.3\masc{} from city}
{the items that he wants from town}
{MutzafiKoySanjaq}{63 {[\N6]}}







\acex[\Koy]{Noun}{Ordinal}{1518}
{yarx-əd ʾod xamša}
{month-\cst{} \lnk{} five}
{the fifth month}
{MutzafiKoySanjaq}{168}

As in \Arb, the linker has lost the definite interpretation associated with the demonstrative, and can head generic-indefinite relative clauses:

\acex[\Koy]{\zero}{Clause}{1512}
{ʾod šté-le mən yá\cb{} māʿe damudast míl-le}
{\lnk{} drank-3\masc{} from \dem.\near.\sg\cb{} water immediately died-3\masc}
{whoever drank from this water died immediately}
{MutzafiKoySanjaq}{63}

In prepositional phrases, furthermore, the \lnk* has entirely lost its pronominal status and has become a pure linker. It can only (optionally) follow \cst* marked prepositions, establishing, as noted above, an agreement-in-state pattern between the prepositional \prim and the \lnk*:

\acex[\Koy]{Preposition}{Noun/Pronoun}{1521-7}
{gā́w \opt{-əd \opt{ʾod}} bela / did-ew}
{in {-\cst{} \lnk{}} house {} \gen-3\masc}
{in the house/in him}
{MutzafiKoySanjaq}{174}

\subsubsection{J. Barzani} 

\citet[3, fn.\ 15]{MutzafiBarzani} mentions the existence of the \lnk* (\enquote{independent particle of annexation}) \transc{ʾod} in this dialect. In the limited corpus available, there is only one clear example of its usage with a clausal \secn:\footnote{There is another example with a \zero\ \prim in the corpus, yet the translation is a bit strange: \foreign{ʾod gāwər ʾod hāwe}{whoever [wishes] to marry or to be [something]} \citep[6 (31)]{MutzafiBarzani}. An alternative is to understand the word \transc{ʾod}  as the imperative form of the verb \foreign{wāda}{to do} (attested in the corpus), in which case the translation would be \transl{make him marry, make him be}. Another theoretical possibility is that \transc{ʾod} is used in this example as a \comp*, but such a usage is not attested elsewhere in the corpus.}

\acex[\Barz]{Noun}{Clause}{1890}
{Xajoke, ʾod zəl-lan šūwá naqle ṭəlb-ā-lan}
{X. \lnk{} went-1\pl{} seven times asked-\patient3\fem-\agent1\pl}
{Khajoke, to whom we went seven times and asked for her hand}
{MutzafiBarzani}{4 (7)}

Additionally, there isan example with an adjectival \secn. In this case, the usage of the \lnk* seems to be motivated by a contrastive focus on the adjective.

\acex[\Barz]
{Noun}{Adjective}{1906}
{(ʾə́t-wa-li xa xona), xon-i ʾod zora}
{\exist-\pst-1\sg{} \indef{} brother brother-\poss.1\sg{} \lnk{} small.\masc{}}
{I had a brother, my youngest brother}
{MutzafiBarzani}{9 (13)}


\subsubsection{Judi-dialects}\label{ss:judi-dialects_lnk}

In the \ili{Judi-dialects} (represented here by \Bes and \Gaz) there is a similar \lnk* \transc{ad} which has almost entirely replaced the simple \lnk* \d. 

\lex{\Bes}{1660}
{suraye ad\cb{} kaldā́n}
{Christians \lnk\cb{} Chaldean}
{Chaldean Christians}
{SinhaBespen}{212 (179)}

\acex[\Bes]{Noun}{Ordinal}{1657}
{bayta ad\cb{} tre}
{house \lnk\cb{} two}
{the second house}
{SinhaBespen}{169}

\acex[\Gaz]{Noun Phrase}{Noun}{1828}
{šula zaḥme ʾad d-awa zalame}
{work hard \lnk{} \gen-\dem.\masc{} man}
{the hard work of this man}
{GutmanGaznax}{316 (22)}

\largerpage
In these dialects one finds a series of \isi{determiners} (i.e.\ exclusively attributive demonstratives), all starting with \transc{a-}, presented in \vref{tb:det_Judi} (adapted from \cite[73]{SinhaBespen}). The \lnk* form \transc{ad} has thus effectively erased all gender/number information, in line with its \isi{grammaticalisation} as a generalized \lnk*.
 
\begin{table}[h!]
\centering
\begin{tabular}{l l}
\toprule
\masc & \transc{áw} \\
\fem & \transc{áy} \\
\pl & \transc{án} \\
\bottomrule
\end{tabular}
\caption{Determiners of Judi-Dialects} \label{tb:det_Judi}
\end{table}

\subsubsection{J. Zakho} \label{ss:JZax_od}


In \JZax there is generally no \transc{ʾod} \lnk*, but rather a \transc{dīd} \lnk* (see \sref{ss:did_lnk}). In the context of Bible translations, however, the uninflecting form \transc{ʾōd} is regularly used as a translational equivalent of the \ili{Hebrew} \rel* \transc{\texthebrew{אֲשֶׁר} ăšɛr} \parencites[\texthebrew{\hebrewnumeral{30}}]{SabarGenesis}[152]{GoldenbergZaken}

\acex[\JZax]{Noun}{Clause}{660}
{(mpəq-la mən) d-ay dūka ʾōd [wēla tāma kutru kalāsa dīd-a ʾəmm-a].} 
{left-3\fem{} from \gen-\dem.\fem{} place \lnk{} \cop.\pst.3\fem{} there both daughter\_in\_law.\pl{} \gen-3\fem{} with-3\fem} 
{(She left) the place where she was with both her daughters-in-law.} 
{}{(Ruth 1:7; \cite[153]{GoldenbergZaken})}


The usage of the  form \transc{ʾōd} may reflect some of kind of archaism, 
as similar (though inflecting) forms are present in Early J. Cis-Zab\il{Cis-Zab NENA dialects} Neo-Aramaic (see \examples{2003}{2001}). Its composite form, moreover, may relate to a meta-linguistic reflection on \transc{ăšɛr} as a complex form.\footnote{In fact, the common view today is that \transc{ăšɛr} is ultimately derived from a \cst* form of an \ili{Akkadian} noun  \foreign{ašru}{place} \citep[59]{KleinHebrew}. Yet  its phonetic similarity to the \rel* \transc{\texthebrew{שֶׁ} šɛ-} easily leads to the idea that it is a complex form containing the relative \citep[cf.][465, \S 138, fn.\ 1]{Gesenius}.} In either case, the non-inflection of \transc{ʾōd} parallels the fixed nature of \transc{ăšɛr}, rather than originating in a process of \isi{grammaticalisation}.

\subsection{The J. Urmi \transc{ay} linker} \label{ss:JUrm_ay}

The \JUrm \lnk* \transc{ay} (see \sref{ss:JUrm_Lnk}) poses a special problem regarding its origin, as it does not contain any \phonemic{d} element. The \lnk* is identical in form with the \dem* \transc{ay}, which \citet[58]{GarbellUrmi} lists as an \enquote{archaic} variant of the \sg* proximal \dem* \transc{ya}. Like the \Koy \lnk* discussed above, most often than not it occurs after \cst* marked \prims, in which case the \phonemic{d} segment of the \cst* suffix re-syllabifies frequently with the \lnk*, giving rise to the form \transc{d-ay}, reminiscent of the \isi{genitive marking} of \dem*s. When this does not happen, it is an indication that \transc{ay} is indeed the \lnk*, as in the following example (see further examples in \sref{ss:JUrm_cst_lnk}):

\lex{\JUrm}{152bis}
{lél-ət ay\cb{} xlulà}
{night-\cst{} \lnk\cb{} wedding}
{the night of the wedding}
{KhanUrmi}{175 {[93]}}


Unlike the \dem*, the \lnk* \transc{ay} does not inflect for number, making it easily identifiable as such after \pl* \prims, such as \transl{children} in the following example:\footnote{Note that, due to the \cst* suffix \ed, the morphological plural marking is erased, and the plural meaning is deduced from the textual context.}


\acex[\JUrm]{Noun}{Noun Phrase}{223}
{dəmm-ə́t ay\cb{} [⁺yál-ət ay gomè]}
{blood-\cst{} \lnk\cb{} child(ren)-\cst{} \lnk{} Muslims}
{the blood of the children of the Muslims}
{KhanUrmi}{230 {[101]}}



The \dem* origin of the \lnk* leads \citet[8]{KhanUrmi} to suggest that it is an \textquote{imitation of the Kurdish relational morpheme (\textit{izafe}), which is demonstrative in origin} closely following the suggestion of \citet[171, \S 2.32.12]{Garbell1965impact}. Later on, \citet[176]{KhanUrmi} elaborates on this idea, presenting effectively the adoption of the \lnk* as a kind of \isi{pattern replication}: \blockquote{It is likely to have developed under the influence of the \textit{izafe} construction in Iranian languages. It appears not to be a direct loan from Iranian, in which the \textit{izafe} is in principle monosyllabic (\textit{e, i, a}), but rather an imitation of the \textit{izafe} using Aramaic morphological material.} \il{Iranic}










A difficulty, however, with the above proposal is found in the fact that the \ez* arose out of a relative pronoun, not a simple demonstrative \citep{HaiderZwanziger}. Moreover, the pronominal origin of the \ez* is quite old, going back at least to Middle \ili{Persian} (spoken up to the 7\th century) pre-dating the earliest attestation of \ili{NENA} dialects by a millennium at least. 

\newpage 
From a structural viewpoint, as discussed above in (\sref{ss:parallel_ez}), the Kurdish \ez* is typically encliticized to the \prim, which is rarely the case with the \lnk* \transc{ay} (see \example{219}). Furthermore, in contrast to the \ez*, the \transc{ay} \lnk* does not introduce a clausal \secn, but instead the \rel* \transc{ki} is used (see also discussion in \vref{par:clausal_secn}). Similarly, adjectival \secns are not introduced by the \lnk* \transc{ay} (although there are rare examples in which adjectives follow the homophonous determiner \transc{ay}; see \sref{ss:JUrm_apparant_adj_secns} and compare with \vref{par:adj_secn}). One may wonder also regarding the form of the \JUrm \lnk*: if the \JUrm speakers were indeed to borrow the \ez* as a \dem* element, wouldn't it be more natural to use the \JUrm \dem* \transc{ya} as a pivot, being identical in form to the \Kur \fem\ \ez*?  Considering all these facts, it seems that the association of the \transc{ay} \lnk* to the \ili{Iranic} \ez* is hardly justified.\footnote{\citet[171, \S 2.32.12]{Garbell1965impact} notes that in \Sol the \ez* particle \transc{i} (apparently of Sorani origin) is borrowed, but this does not entail that \transc{ay} is also borrowed, notwithstanding its functional similarity. Note that these localities, in contrast to Urmi, are on the border of the Sorani speaking area. Regarding borrowing of the \transc{i} \ez* see \sref{ss:i_ezafe}. \label{fn:Solduz_lnk}}

A possible alternative explanation is to assume that the \lnk* \transc{ay} originated  in a compound \lnk* \transc{ay-d}, analogical to the \lnk*s discussed above. Subsequently, due to a phonological reduction, the \lnk* became \transc{ay}. The environment which promoted such a reduction may be exactly the same environment which led to the development of \gen* marked demonstratives following the \ed suffix (see \sref{ss:genitive_development}), as in the following example (=\example{182}):

\lex{\JUrm}{182bis}
{⁺qayd-ət áy d-ò\cb{} tka}
{custom-\cst{} \lnk{} \gen-\dem.\far.\sg\cb{} place}
{the custom of that place}
{KhanUrmi}{176 {[144]}}

Another minor factor may have been that the conflation of \transc{ay-d} could lead to the form \transc{ād}\~\transc{āt} which is identical to the independent 2\sg\ pronoun. The desire to avoid ambiguity  may have played a role in the dropping of the \phonemic{d}.

Some support for this idea comes from comparing nominalized ordinal numbers. Comparing the following two examples (=\example{306} and \example{161}), we see that \JZax \transc{ay-d} corresponds closely to \JUrm \transc{ay}:

\acex[\JZax]
{\zero}{Ordinal}{306bis}
{ʾay d\cb{} treʾ}
{\defi.\fem{} \lnk\cb{} two}
{the second one (\fem)}
{CohenZakho}{95 (7)}
		
\acex[\JUrm]
{\zero}{Ordinal}{161bis}
{ay arbi}
{\lnk{} forty}
{the fortieth}
{GarbellUrmi}{88}



A similar piece of evidence in favour of this account comes from the Bible translation written in the \Ruw dialect, described by \citet{ReesTargum}. In this translation, the particle \transc{\texthebrew{אַי} ay} is used as a \rel* serving as the translation equivalent of \BHeb \rel* \transc{\texthebrew{אֲשֶׁר} ăšɛr} \citep[27]{ReesTargum}.\footnote{In contrast to \JUrm, it is not used before nominal \secns. In such context, the \Ruw translation uses consistently the Neo-\cst\ suffix \transc{-ət}, being the translation equivalent of the \BHeb \cst* \citep[82]{ReesTargum}. Yet  \citet[27]{ReesTargum} tentatively relates this use of \transc{ay} to \JUrm influence.} Recall that in \JZax Bible translations the equivalent of \transc{ăšɛr} is consistently \transc{ʾod}, as described in \sref{ss:JZax_od}. Thus, once again, we see that \transc{ay} corresponds to a \dem+\cst\ combination in \JZax. While these similarities are first and foremost functional and distributional, they corroborate the idea that \transc{ay} as a \lnk* (in \JUrm) or a \rel* (in \Ruw) originated in \transc{ay-d}.

\subsection{Emerging grammaticalisation of \transc{mārā}} \label{ss:mara}

\largerpage
Quite distinct from the \lnk*s that developed from the \d \lnk*, in some dialects one finds reflexes of the \cst* of the Aramaic word \foreign{mārā}{owner, master}, marking possession of qualities and goods. 

\acex[\Alq]{\textit{mara}}{Noun}{1840}
{mar\cb{} ʾérwe}
{owner.\cst\cb{} sheep}
{sheep owner}
{CoghillAlqosh}{250}\antipar 


\acex[\Qar]{\textit{mara}}{Noun}{505}
{gora mari šafqa}
{man owner.\cst{} hat}
{a man with a hat}
{KhanQaraqosh}{211}\antipar 
\newpage 

\acex[\JSul]{Noun}{Noun}{1088-9}
{baxta mara/mare pare}
{woman owner/owner.\cst{} money}
{a woman possessing money}
{KhanSulemaniyya}{193}

\citet[225]{CohenZakho} notes that when the \secn denotes a quality, the entire construction is similar to an adjective (similarly indeed to phrases headed by a \lnk*):

\acex[\JZax]
{\textit{mara}}{Noun}{2009}
{mare qūwəta}
{owner strength}
{possessor of strength, strong}
{CohenZakho}{225 (60)}

In \JZax one finds also an example where \transc{mare} is followed by an interrogative pronouns:

\acex[\JZax]
{\textit{mara}}{Interrogative Pronoun}{2012}
{mare mā wē-tən?}
{owner what \cop-2\masc}
{What is on your mind}
{SabarDictionary}{210}

In all these dialects \transc{mar-} has lost its gender inflection, though in \Alq and \JSul it inflects for number. In \Qar and \JZax it is completely invariable. In these expressions \transc{mar-} keeps to a large extent its original lexical semantics of ownership (clearly so in examples such as \JZax \foreign{mare bēsa}{landlord}\footcite[224]{CohenZakho}), so we cannot properly speak of a   \isi{grammaticalisation} of \transc{mar-}. Yet  in some respects it shows the first signs of \isi{grammaticalisation}, such as the loss of its inflectional features, and its usage as a grammatical head of adjectival-like phrases . If its usage became wider and more abstract it might emerge as a new \lnk*.



\section{Conclusions}

In this chapter I have traced the development of the various D-markers in \ili{NENA} dialects. I have showed that virtually all the developments involved can be explained by prosodic mechanisms of re-syllabification and cliticization. Yet  the \isi{grammaticalisation} of the resulting segmental material (be it the suffix \ed, the \gen* prefix \d, or indeed \lnk*s such as \transc{od}) needs a further stage of morpho-syntactic re-analysis. It is in this last stage where \isi{language contact} may play a crucial role. Nonetheless, as I have showed in the discussion of the emergence of the Neo-\cst\ suffix \ed (\sref{ss:role_contact}), it is quite difficult to pinpoint the influence to a specific model language, due to the profusion of the features of these AC systems throughout the languages of the regions, forming effectively a \isi{Sprachbund}. 

To summarize my claims, I present the following model of development of the different D-markers. The division into stages helps conceptualize the process, but it should not be taken as a strict chronological ordering of the steps involved. As the beginnings of stage one are apparent in manuscripts of the 17\th century (see \sref{ss:neo-CSC}) one may cautiously date the start of the process to that time, though it may in fact have started even earlier.\footnote{Such an estimate must be taken with a grain of salt, as manuscripts often lag behind the synchronic developments of language, due to a tendency of scribes to use archaic spelling conventions.}

\begin{description}

\item[Stage 1] The \d \lnk* of the \il{Aramaic!Classical}Classical Aramaic DAC transforms, by means of \isi{encliticization}, into the Neo-\cst\ suffix \ed (\sref{ss:neo-CSC}).

\item[Stage 2] Due to analogy, the same process occurs to a limited extent with the \d \lnk* of the ALC. In contrast to the DAC, the ALC remains in \isi{complementary distribution} with the Neo-CSC. 

\item[Stage 3] The \ed suffix procliticizes to vowel-initial \secns, being reanalysed as a \gen* marker before vowel-initial demonstratives, possibly under the influence of \Kur (\sref{ss:genitive_development}).

\item[Stage 4] In some dialects, the emergence of the \ed suffix facilitates the generalisation of the pronominal base \transc{did-} as an independent \lnk* (\sref{ss:did_lnk}).

\item[Stage 5] In other dialects, combinations of  \dem+\lnk\ \d are grammaticalised to become new \lnk*s such as \JArb \transc{ot} (\sref{ss:Arb_Barw_lnks}) or \Gaz \transc{ad} (\sref{ss:judi-dialects_lnk}). In \JUrm, the \phonemic{d} segment is lost, giving rise to the \lnk* \transc{ay} (\sref{ss:JUrm_ay}).

\end{description}

As we shall in the next chapter, these developments triggered further changes in the attributive systems of the \ili{NENA} dialects.



