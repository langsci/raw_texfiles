\chapter{Attributive constructions in the Jewish dialect of Sanandaj} \label{ch:Sanandaj}

\renewcommand{\defaultDialect}{\JSan}

\section{Introduction}

The Iranian city of Sanandaj is located at the eastern extremity of the \ili{NENA} speaking zone. Compared to the three dialects surveyed so far, the grammar of the Jewish dialect of Sanandaj is the most divergent. This is certainly true for the AC system, which will be surveyed below, but can also be said about other domains of grammar, such as the verbal system. While the latter is outside the scope of this work, it is worthwhile noting two innovative features of the verbal system, which are of relevance to the current survey: First, the language exhibits an OV order (in contrast to the typical VO order found in most \ili{NENA} dialects); and second, the language makes extensive use of complex predication, i.e.\ predicates consisting of a combination of a light verb and a noun (termed here CP noun).\footnote{For an elaborate syntactic and semantic analysis of complex predication in \Per, see \citet{SamvelianComplex}.} 
 These and other features are in all probability related to an extensive \isi{language contact} with \Sor and \Per \citep[11f.]{KhanSanandaj}.\footnote{As \citet[11]{KhanSanandaj} notes, the Kurdish dialect of Sanandaj is not systematically described. Instead, I refer to standard \Sor for the sake of comparison. It should also be noted that \Hawr, a \ili{Gorani} language closely related to Kurdish, is spoken in the vicinity of Sanandaj.}
  While one may speculate that the divergence of \JSan is related to its peripheral location, it is worthwhile noting that the Christian dialect spoken in the same city presents a much more conservative grammar, but unfortunately it has not yet received  a detailed grammatical description.\footnote{See however \citet{PanoussiSenaya,HeinrichsSenaya} and the list of publications given in \citet{McPhersonCaldani}.}

The data for \JSan is based mainly on the grammatical description of \citet{KhanSanandaj}.\footnote{Khan's examples are cited according to the page in the grammar in which they are treated. Additionally, a reference to the textual corpus, if available, is given in square brackets according to Khan's system: a letter indicating the informant (A--E) and a sentence number. I have also consulted the grammatical description of \citet{SchallerSanandaj}, but as this description is mostly devoted to the verbal system, no examples are drawn from there.} Additional examples are drawn from an elicitation session I have conducted in Jerusalem with an elderly native speaker of the dialect, Ḥabib Nurani.\footnote{In Khan's description, he is marked as informant A.} \JSan is in some respects similar to \JSul, of which I give some comparative examples drawn from \citet{KhanSulemaniyya}. I present also some sporadic comparisons with \NMand, another Neo-Aramaic language spoken in Iran. 

The structure of the chapter is as follows:

First, I treat the usage of the possessive pronominal suffixes. In contrast to most other \ili{NENA} dialects, these are phrasal suffixes, as discussed in \sref{ss:JSan_X-y.poss}. 

A major difference in \JSan in comparison  to the dialects discussed so far is that the main AC in \JSan is not the CSC, but rather the zero-marked \isi{juxtaposition} construction, which is discussed in \sref{ss:JSan_juxt}. This construction has two further variants: \isi{juxtaposition} with agreement of the \secn with the \prim (see \sref{ss:JSan_juxt_agr}), and {inverse juxtaposition}\isi{inverse juxtaposition construction} with the \secn preceding the \prim (see \sref{ss:JSan_inverse}). 

The use of borrowed \ili{Iranic} \rel*s with clausal \secns is discussed in \sref{ss:JSan_rel}.

While \JSan does not make use of the Neo-CSC found in other dialects, it has a structural parallel formed by marking the \prim with the \ili{Iranic} \ez* suffix. This construction, as well as the idiomatic retention of the historical CSC and the possible emergence of a new CSC related to stress retraction, is discussed in \sref{ss:JSan_cst}.

From the above it is clear that \JSan has hardly retained any reflex of the \il{Aramaic!Classical}Classical Aramaic \d \lnk*. Indeed, \JSan has  only one reflex of this \lnk*, namely the \gen* marking of vowel-initial demonstratives. This is discussed in \sref{ss:JSan_gen}. On the other hand, \JSan has retained to a small extent the usage of the \isi{dative preposition} \transc{əl-} for marking \secns, as discussed in \sref{ss:JSan_dat}.

Conclusions and a general discussion of the various constructions are presented in \sref{ss:JSan_conclusions}.

\section{Possessive pronominal suffixes (X-y.\poss)} \label{ss:JSan_X-y.poss}



As in other \ili{NENA} dialects, a pronominal \secn may be expressed by a \isi{possessive suffix}. The \isi{possessive suffix} replaces the inflectional suffix (\transc{-a} or \transc{-e}) of the nominal \prim it attaches to:

\acex{Noun}{Pronoun}{14}
{bel-ef}
{house-\poss.3\masc}
{his house}
{KhanSanandaj}{61}

A particularity of \JSan in comparison with most other \ili{NENA} dialects is that the \isi{possessive pronoun} is suffixed NP-finally, rather than directly on the \prim noun, whenever the NP consists of a Noun+Adj. combination \citep[251]{KhanSanandaj}.
 
 \acex{Noun Phrase}{Pronoun}{1}
 {[xa ʾăxóna xet]-àfˈ}
 {\indef{} brother other-\poss.3\fem}
 {another brother of hers}
 {KhanSanandaj}{53 {[A:6]}}
 
 A similar pattern is found in \JSul:
 
 \acex[\JSul]
 {Noun Phrase}{Pronoun}{1104}
 {ʾaxón-a ruww-í}
 {brother-\free{} big-\poss.1\sg}
 {my elder brother}
 {KhanSulemaniyya}{262 {[R:94]}}
 
 
 In the current framework, this distribution makes the possessive suffixes of \JSan and \JSul phrasal suffixes rather then word-level suffixes (see \sref{ss:clitics_affixes}). The usage of the possessive suffixes NP-finally may very well be due to \isi{pattern replication} from \Sor (see \example{836}).

When attached to a verbal noun (such as an infinitive or a CP noun of a transitive verb), the pronoun denotes the object: 

\acex{Infinitive}{Pronoun}{92}
{(ʾila di-li ba\cb{}) găroš-ef\,\footnotemark}
{hand placed-1\sg{} in\cb{} pull.\inf-\poss.3\masc}
{(I began) to pull him.}
{KhanSanandaj}{331}

\footnotetext{One may want to analyse the combination \transc{ba}+infinitive as forming a gerund, as in \JZax (see \vref{ft:JZax_gerund}). As I am unaware of a gerund category in \JSan, I prefer to analyse the preposition \transc{ba} here, as well as in \example{91}, as forming part of the verbal complex.}

\acex{CP Noun}{Pronoun}{51}
{daʿwăt-ì (k-ol-í)ˈ}
{invitation-\poss.1\sg{} \ind-do-3\pl}
{They will invite me.}
{KhanSanandaj}{482 {[D:8]}}

  
When attached to a preposition, it denotes its complement. Note, however, that not all prepositions allow for a suffixed pronoun.

\acex{Preposition}{Pronoun}{62}
{reš-ef}
{on-\poss.3\masc}
{on it}
{KhanSanandaj}{224}

Interestingly, a pronoun attached to a true adverb can convert it to a noun:
\acex{Adverb}{Pronoun}{15}
{(gbé-wa xa\cb{}párča zayrá dă-en  ba\cb{}) lăxa-u}
{\ind.want.3\masc-\pst{} \indef\cb{}fabric yellow place-3\pl{} on\cb{} here-\poss.3\pl}
{(They had to put a yellow patch on) their (body place) here.\footnotemark}
{KhanSanandaj}{579 {[A:78]}}\vspace*{-4mm}
\footnotetext{From the context it seems that the informant pointed on a spot on his body (\transl{here}) referring to the same spot on the body of the referents.}

 \section{Simple juxtaposition (X Y)} \label{ss:JSan_juxt}
 \largerpage[2]
 The paradigmatically richest and most common construction in \JSan is the \isi{juxtaposition} construction, devoid of any special marking. In cases where a noun is modified by another noun, the \isi{juxtaposition} construction is the  functional parallel in \JSan of the CSC or the ALC in the previously surveyed  \ili{NENA} dialects. 
 
\acex{Noun}{Noun}{10}
{lišana bšəlmane}
{language Muslims}
{the language of the Muslims}
{KhanSanandaj}{199}

  
In this \JSan is very similar to \JSul, which also makes extensive use of the \isi{juxtaposition} construction:\footnote{Yet, in contrast to \JSan, \JSul also makes use of the Neo-\cst\ suffix \ed, as well as the \lnk* \d:

\acexfn[\JSul]
{Noun}{Noun}{1068}
{xə́zm-əd kaldà}
{family-\cst{} bride}
{the family of the bride}
{KhanSulemaniyya}{192 {[A:8]}}
\vspace*{-2mm}

\acexfnii[\JSul]
{Noun}{Noun}{1063}
{məšxa d\cb{} zetùne}
{oil \lnk\cb{} olives}
{olive oil}
{KhanSulemaniyya}{192 {[R:98]}}
\vspace*{-1mm}
}

\acex[\JSul]
{Noun}{Noun}{1059bis}
{šəmma brona}
{name son}
{the name of the boy}
{KhanSulemaniyya}{192}
   

While the above usage of the \isi{juxtaposition} construction in \JSan and \JSul for nominal modification
marks these dialects as special in comparison to the majority of \ili{NENA} dialects, there are also more {trivial} cases of \isi{juxtaposition},  such as its usage with \isi{adverbial} \secns:


\acex{Noun}{\PP}{13}
{ʾo gorá ga\cb{} lăxa (bărux-i \cb{}ye)}
{\textsc{dem.m} man in\cb{} here friend-\poss.1\sg{} \cb{}\cop.3\masc}
{the man here (is my friend)}
{KhanSanandaj}{252}

Another use of the \isi{juxtaposition} construction, which is  cross-dialectally common and present also in \JSan, is in quantification expressions (see also \example{1939}):

\acex
{Q. Noun Phrase}{Noun}{1933}
{[xa lewan] rəzza}
{one cup rice}
{one cup of rice}
{}{(own fieldwork)}

Adjectival and ordinal \secns normally appear in the \isi{juxtaposition-cum-agreement} construction, discussed in the next section. Yet, when the lexical items in question are invariable, such as loan-adjectives or the loan-ordinal \foreign{ʾăwal}{first}, they necessarily appear in the simple \isi{juxtaposition} construction:

\acex{Noun}{Adjective}{278}
{mal-ăwae qarwa}
{village-\pl{} near(\invar)}
{nearby villages}
{KhanSanandaj}{207}

\acex{Noun}{Ordinal}{36}
{gora ʾăwal}
{man first(\invar)} 
{the first man}
{KhanSanandaj}{213}

 

Clausal \secns are also found in this construction, yielding \concept{asyndetic relative clauses}.

\acex{Noun}{Clause}{43}
{(măt-í-wa-le ga\cb{}) xá \cb{}tʷka [qărirà hăwé].ˈ}
{put-\agent.3\pl-\pst-\patient.3\masc{} in\cb{} \indef{} \cb{}place cool \subj.be}
{They put it in a place that was cool.}
{KhanSanandaj}{381 {[A:83]}}


Asyndetic clausal \secns can also follow pronominal \prims, such as the indefinite pronoun \foreign{xa}{one}. In the following example there are two asyndetic relative clauses, one embedded in the other.\footnote{Note that the embedded \isi{relative clause} is separated from its \prim \transc{sawzì} by the \isi{copula} and a prosodic break. Alternatively, it could be analysed as an asyndetically conjoined \isi{relative clause} governed as well  by the \prim \transc{xa}.}


\acex{Pronoun}{Clause}{44}
{(bár kŭ̀leˈ kyà-waˈ) xa\cb{} [sawzì \cb{}ye,ˈ šaplultà kəmr-í-wa baq-éf].ˈ}
{after all \ind.come.3\fem-\pst{} one\cb{} vegetable \cb{}\cop.3\masc{} š. call-\agent.3\pl-\pst{} for-\poss.3\masc{}}
{(After everything else there came) something that is a vegetable, which is called \textit{šaplulta}.}
{KhanSanandaj}{382 {[B:68]}}

Examples of clausal \secns following demonstrative pronouns acting as \prims are given in \sref{ss:gen_pron_clause}. 

\largerpage
Occasionally, the \isi{juxtaposition} construction is used with an infinitival \prim followed by a nominal \secn, corresponding to the direct object of a transitive verbal lexeme.\footnote{Such cases can also be analysed as exponents of the \isi{completive relation} rather that the attributive one (see \sref{ss:threeRel}). Yet I prefer to to analyse them as attributive constructions, as discussed in \sref{ss:JSan_inverse}.}

\acex{Infinitive}{Conjoined Nouns (objects)}{91}
{(šerúʾ wí-lu ba\cb{}) yălopé hulaulà \cb{}uˈ yălopé făransà \cb{}uˈ ʿəbrì,ˈ fàrsi.ˈ} 
{start do-3\pl{} in learn.\inf{} Judaism \cb{}and learn.inf{} \ili{French} \cb{}and \ili{Hebrew} Persian}
{(They started) to learn Judaism, and to learn \ili{French}, \ili{Hebrew} and \ili{Persian}.}
{KhanSanandaj}{330 {[B:12]}}

 
Similarly, prepositions or conjunctions are complemented  by nouns or clauses without any special marking:

\acex{Preposition}{Noun}{54}
{reša mez}
{on table}
{on the table}
{KhanSanandaj}{224}

\acex{Conjunction}{Clause}{77}
{mangól [ga\cb{} lăxa k-olí]ˈ}
{as in\cb{} here \ind-do.3\pl}
{as they do here\footnotemark}
{KhanSanandaj}{393 {[B:67]}}

\footnotetext{An alternative analysis of this example is to see only the \isi{prepositional phrase} \transc{ga lăxa} as the complement of \transc{mangól}, the  verb \transc{kolí} being the main verb. This would correspond to the translation \transl{They do as (it is) here}.}

An adjective can also serve as the \prim of the \isi{juxtaposition} construction, whenever it is further specified by a nominal \secn. While the \secn in such cases is an \isi{adverbial} specification of the adjectival \prim, formally it uses the same \isi{juxtaposition} construction as the above examples (compare to \example{98}, where the AC is explicitly marked by the \ez* suffix): 


\acex{Adjective}{Noun}{99}
{(tamā́m-e yomá) ḥărík ḥaštà (xirá \cb{}y)ˈ} 
{entire-\ez{} day busy work become.\resl{} \cb{}\cop.3\masc} 
{(All day he has been) busy with work}
{KhanSanandaj}{570} 

\section{Juxtaposition-cum-agreement (X Y.\agr)} \label{ss:JSan_juxt_agr}

Similarly to other \ili{NENA} dialects, inflecting adjectives (which are typically but not exclusively of Aramaic origin) formally use  the \isi{juxtaposition} construction, and at the same time show agreement features with the \prim noun. 

\acex{Noun}{Adjective}{8}
{bela rŭwa}
{house(\masc) big.\masc}
{a big house}
{KhanSanandaj}{251}

Similarly, \isi{ordinals} above one, juxtaposed to their \prim, can optionally agree with it, similarly to adjectives:\footnote{Note that \isi{ordinals} are derived from the corresponding cardinals by means of the suffix \transc{-min}, being of \Per or \Sor origin (see \sref{ss:kurd_ez_ord}).}

\acex{Noun}{Ordinal}{279}
{baxta tre-min-\opt{ta}}
{woman(\fem) two-\ord-\opt\fem}
{the second woman}
{KhanSanandaj}{213}


\section{Inverse juxtaposition (Y X)} \label{ss:JSan_inverse}

\isi{inverse juxtaposition construction}Of special interest are constructions in which the order of the \secn and the \prim is reversed, so that the \secn precedes the \prim. There are two distinct kinds of these constructions, one which involves a verbal noun acting as a \prim, and the second which involves an adjective or an ordinal as the \secn.\footnote{Recall that the titles of the examples always reflect  the order \textbf{Primary--Secondary}, irrespective of the order of these constituents in the example.}

\subsection {Verbal nouns as \prims} \label{ss:JSan_inv_juxt_verbal}

The category of verbal nouns includes the infinitive and the active participle.\footnote{The resultative participle, on the other hand, does not participate in ACs, as its distribution is restricted to some compound tenses \citep[90--96]{KhanSanandaj}. Some resultative \isi{participles} have acquired an adjectival meaning, but in this case they do not function differently from other inflecting adjectives \citep[204]{KhanSanandaj}.} These nouns have the particularity that they can be complemented by a \secn which acts semantically as the direct object of the verbal lexeme. Moreover, I include in the category of verbal nouns also nouns participating in complex predicate formation (CP nouns), as their \secns are semantically the direct object of the entire complex predicate. Note that the extensive usage of complex predication in \JSan originates in the replication of an \ili{Iranic}, probably \Per, pattern \citep[of which see][]{SamvelianComplex}.

One may doubt whether constructions involving verbal nouns together with their complements should be regarded as ACs, rather than expressing simply a completive  relation (see \sref{ss:threeRel}). However, since verbal nouns behave categorically as nouns  (they share the privilege of occurrence of nouns), and complementation of  nouns yields by definition an AC, it seems justifiable to regard these constructions as ACs, albeit of a special kind. Two observations strengthen this claim: First, nouns and their complements participate sporadically in explicitly marked ACs (see \examples{49}{47} for verbal nouns modified by a \isi{possessive suffix}). Secondly, whenever their complement is an independent pronoun it is explicitly marked as genitive (see \sref{ss:JSan_gen_verbal}).

Notwithstanding the above analysis, verbal nouns expanded by a complement exhibit a key property of the verbal phrase of \JSan, namely the pre-verbal position of the complement. In fact, the OV order of \JSan is very probably a contact feature originating in \ili{Iranic} languages, as most \ili{NENA} dialects have a VO order. Thus, these ACs are of the {inverse juxtaposition}\isi{inverse juxtaposition construction} type, in which the \secn precedes the \prim (and see also \example{97}):

\acex{Participle}{Noun}{87}
{xola garš-ana}
{rope pull-\prtc}
{rope puller}
{KhanSanandaj}{252}

\acex{Infinitive}{Noun}{90}
{(ʾila hiw-li ba\cb{}) xola garoše}
{hand gave-1\sg{} in\cb{} rope pull.\inf}
{(I began) to pull the rope.}
{KhanSanandaj}{330}




\subsection{Adjectival and ordinal \secns} \label{ss:JSan_juxt_inverse_adj}

Normally an adjectival \secn follows the \prim noun (see \example{8}). However, according to \citet[251]{KhanSanandaj}, \textquote{[i]n some isolated cases the adjective is placed before the head [=the primary]. This is found where the adjective is evaluative, i.e.\ expressing the subjective evaluation by the speaker rather an objective description of the referent.} The following example is given:

\acex{Noun}{Adjective}{84}
{ʿáyza kā́sbi (hùl ta\cb{} nóš-ox).ˈ}
{good.\masc\footnotemark{} gain(\fem) take.\imp{} for\cb{} \refl-\poss.2\masc}
{Take the good earnings for yourself.}
{KhanSanandaj}{251 {[A:103]}}
\vspace*{-2mm}

\footnotetext{This example is peculiar in that the adjective disagrees in gender with the \isi{head noun}. It may be that some speakers treat \transc{ʿáyza} as an invariable adjective, being probably of foreign origin.}

Ordinal \secns can similarly appear before the \prim, in this case without any {evaluative} semantics. In the case of the ordinal \foreign{ʾăwaḷ}{one}, borrowed ultimately from \Arab, this yields the typical \ili{Arabic} order, but in \JSan this is only one possibility (contrast with  examples \vref{ex:36} and \vref{ex:35}):

\acex{Noun}{Ordinal}{276}
{ʾăwaḷ gora}
{first man}
{the first man}
{KhanSanandaj}{213}

Ordinals above one can optionally agree with the \prim noun, also when they precede it (compare \example{279}):

\acex{Noun}{Ordinal}{282}
{tre-min-\opt{ta} baxta}
{two-\ord-\opt\fem{} woman(\fem)}
{the second woman}
{KhanSanandaj}{213}

\section{Usage of relativizer (X \textsc{rel} Y)} \label{ss:JSan_rel}

Clausal \secns can be marked as such by the use of a \rel*.  Two distinct relativizers are available in \JSan: \transc{ya} and \transc{ke}, both borrowed from \ili{Iranic} languages. In particular, one finds \transc{ke} as a \rel* in \ili{Persian} \citep[136]{BalayEsmaili}. 

The \isi{relativizer} \transc{ya} is used mostly with definite \prims, while the \isi{relativizer} \transc{ke} has no such restriction. The exact distribution of these relativizers is outside the scope of this work. Prosodically, both relativizers are part of the clausal \secn, as they often cliticize to its first word.

\acex{Noun}{Clause}{39}
{ʾo\cb{} našé ya\cb{} [daʿwàt k-ol-í-wa-lu]ˈ}
{\definite\cb{} people \rel\cb{} invitation \ind-do-\agent\pl-\pst-\patient3\pl}
{the people whom they invited}
{KhanSanandaj}{378 {[A:42]}}

\acex{Pronoun}{Clause}{41}
{ʾonyé yá [ṭăbăqá ʾăwaḷ \cb{}ye-lù]ˈ}
{3\pl{} \rel{} class first \cb{}\cop.\pst-3\pl{}}
{Those who were the first class}
{KhanSanandaj}{379 {[B:5]}}

\acex{Noun}{Clause}{275}
{xá\cb{}ʿəda našé ke\cb{} [ga\cb{}xá meydā́n smix \cb{}èn]ˈ}
{\indef\cb{}few people \rel\cb{} in\cb{}\indef{} square stood.\resl{} \cb{}\cop.3\pl}
{a group of people who were standing in a square}
{KhanSanandaj}{380 {[A:109]}}

In \JSul, one finds conversely the cognate \rel* \transc{ga}\~\transc{ka} mostly with definite \prims, in restrictive relative clauses \citep[414]{KhanSulemaniyya}:\footnote{A similar restriction is found with the \isi{relativizer} \transc{ke} borrowed in \NMand \citep[165]{HaberlMandaic}.}

\acex[\JSul]
{Noun}{Noun}{1189}
{yóma ga\cb{} gezí ta\cb{} Merònˈ mìl-a.ˈ}
{day \rel\cb{} go-3\pl{} to\cb{} M. died-3\fem}
{She died on the day that they went to Mount Meron.}
{KhanSulemaniyya}{415 {[R:185]}}

In \JSan, the \rel* \transc{ke} is found following certain adverbials, notably \foreign{qắme}{before}: 

\acex{Conjunction}{Clause}{76}
{qắme ké hètˈ}
{before \rel{} \subj.come.2\masc}
{before you came}
{KhanSanandaj}{391}

In this usage, it can also combine with the \ez*  marking; see  \example{17}.

The \rel* \transc{ya} occurs once in the corpus of \citet{KhanSanandaj} complementing a temporal adverb. In this case, the entire construction gets a temporal meaning:

\acex{Adverb}{Clause}{42}
{ʾăta ya\cb{} [daʿwăt-í wilà \cb{}y]ˈ}
{now \rel\cb{} invitation-\poss.1\sg{} done.\resl{} =\cop.\masc}
{now that  they have invited me}
{KhanSanandaj}{379 {[D:15]}}
\antipar

  
\section{The construct state construction (X.\textsc{cst} Y)} \label{ss:JSan_cst}

\JSan has 3 different morphological means which can be classified under the broad category of \isi{construct state} as defined in \sref{ss:state}:

\subsection{The historical construct state marking}

The historical \il{Aramaic!Classical}Classical Aramaic \isi{construct state} marking, formed by \isi{apocope} of the \prim noun, is not productive any more in \JSan, yet a reflex of it is retained  in some collocations and idioms. For example, in the following example, the noun \foreign{belá}{house} appears as a reduced form \transc{be} with the meaning \transl{family of} (compare \Qar  \example{504}):

\acex{Noun}{Noun}{7}
{be\cb{} kalda}
{house.\cst\cb{} bride}
{family of the bride}
{KhanSanandaj}{201}

Similarly, two prepositions of nominal origin have retained an \isi{apocopated form} alongside their full form. These are the prepositions \foreign{reša}{on} (derived from the noun \foreign{reša}{head} by \isi{pattern replication} of Kurdish; see \vref{ft:reš}), which also has the \isi{apocopated form} \transc{reš}, and the preposition \foreign{txela}{under}, which also has the \isi{apocopated form} \transc{txel}. While both forms require a complement, I consider only the apocopated one to be positively marked as \cst*.

\largerpage
\acex{Preposition}{Noun}{59}
{reša/reš mez}
{on/on.\cst{} table}
{on the table}
{KhanSanandaj}{224}\antipar

\acex
{Preposition}{Noun}{59other}
{txela/txel mez}
{under/under.\cst{} table}
{under the table}
{KhanSanandaj}{225}\antipar
\newpage

\subsection{The Ezafe construction} \label{ss:JSan_Ez}

The closest structural parallel of \JSan to the Neo-CSC present in other dialects is the borrowed \ez* construction, in which an \ez* suffix \transc{-e}\~\transc{-y} marks the \prim as such.\footnote{The  variant form \transc{-y} is found in my  fieldwork data.} The form of the \ez* in \JSan seems to indicate a \Per origin, an assumption which is corroborated by its frequent usage with \Per words (see \example{280}).\footnote{The \Per \ez* is \transc{-e}, while in \Sor, it is normally \transc{-i} (but see \sref{ss:Kurdish_cst} for possible variation). Note that in the nearby \Hawr dialect the plural \ez* is realized as \transc{-e} \citep[133]{HolmbergOdden}.}

Indeed, the usage of the \ez* is most frequent \enquote{when the noun is an unadapted  loanword that ends in a consonant rather than in a nominal inflectional vowel} \citep[199]{KhanSanandaj}. These \isi{loanwords} are not necessarily of \ili{Iranic} origin. For instance, in the following example the \prim is the \MishHeb loan-noun \texthebrew{שַׁמָּשׁ} \transc{šămmaš}.  

\acex{Noun}{Noun}{2}
{šămáš-e kništà}
{beadle-\ez{} synagogue}
{the beadle of the synagogue}
{KhanSanandaj}{199 {[A:43]}}

Note that a similar restriction appears in \JSul, where the \Sor borrowed \ez* suffix \transc{-i} is most frequently used with \isi{loanwords} \citep[192f.]{KhanSulemaniyya}:

\acex[\JSul]
{Noun}{Noun}{1220}
{maktáb-i hulayè}
{school-\ez{} Jews}
{school of Jews}
{KhanSulemaniyya}{514 {[R:141]}}

\citet[199]{KhanSanandaj} also gives  examples (possibly elicited) of native Aramaic \prims marked by the \ez*. In these cases, the final number suffix (\sg: \transc{-a} or \pl:   \transc{-e}) is normally retained, but can also be elided in \enquote{fast speech} (in which case the stress falls on the \ez* suffix).

\acex
{Noun}{Noun}{1930}
{bel-\opt{á}-e bărux-i}
{house-\opt\sg-\ez{} friend-\poss.1\sg}
{the house of my friend}
{KhanSanandaj}{199f.}\antipar
\newpage

\acex
{Noun}{Noun}{1930plural}
{bat-\opt{é}-e bărux-i}
{houses-\opt\pl-\ez{} friend-\poss.1\sg}
{the houses of my friend}
{KhanSanandaj}{199f.}\antipar


The \pl* of \foreign{bela}{house} can be the irregular form \transc{baté} or the regular \transc{belé}. Therefore, one could argue that in \exampleabove{1930} the \prim's number distinction is lost. Yet, in my own elicitation I observed a slight phonetic difference between \foreign{belé}{houses} and \foreign{bel-é}{house.\sg-\ez}: in the latter form the \ez* is produced as \phonetic[æ] (and not as the expected \phonetic[e]), which is understandable if this vowel is analysed as a coalescence of a \sg* suffix \phonemic{-a} and an \ez* suffix \phonemic{-e}. On the other hand, the coalescence of the \ez* suffix with the \pl* suffix yields a construction identical to the \isi{juxtaposition} construction, as \citet[200]{KhanSanandaj} notes.


Similarly to the Neo-CSC of other \ili{NENA} dialects, as well as the  CSC of classical \ili{Semitic} languages (such as \BHeb or \Akk), the \ez* construction can be used not only with nominal \secns, but also with   infinitival or clausal \secns:

\acex{Noun}{Infinitival phrase}{97}
{(ʾaná) ḥawṣălá-e [ʾắra tărošè] (lít-i \cb{}u)ˈ}
{1\sg{} patience-\ez{} land build.\inf{} \neg.\exist-1\sg{} \cb{}and}
{(I don't have) the patience to build on the land.}
{KhanSanandaj}{571 {[C:6]}}

\acex{Noun}{Clause}{16}
{ʾo\cb{} baxtá-e [ləxm-ăkè k-ol-a-wa-le \cb{}ó]ˈ}
{\definite\cb{} woman-\ez{} bread-\definite{} \ind-do-\agent3\fem-\pst-\patient3\masc{} \cb{}open}
{the woman who made (lit.\ opened) the bread}
{KhanSanandaj}{381 {[B:22]}}

In contrast to the classical \ili{Semitic} CSC, however, the \ez* construction is also used with adjectival or ordinal \secns. In some cases, where both the \prim and the \secn are \Per words, the entire expression can be seen as a code-switch to \Per, as in the following example, where the AC corresponds to \Per \foreign{\textarabic{لباسِ خراب} ləbā́s-e xărā́b}{ragged clothes}: 

\acex{Noun}{Adjective}{280}
{ləbā́s-e xărā́b (lòš-wa)ˈ}
{clothing(\masc)-\ez{} bad(\invar) wear.3\masc-\pst}
{(He wore) ragged clothes.}
{KhanSanandaj}{251 {[A:108]}}\antipar
\newpage 

 
When used with native Aramaic adjectives as \secns, these inflect as expected:\footnote{Optional inflection of adjectives following the \ez* is attested sporadically also in \Hawr: 

\acexfn[\Hawr]
{Noun}{Adjective}{Hawr1}
{žæn-i zɪl-\opt{æ}}
{woman-\ez{} big-\opt\fem}
{big woman}
{HolmbergOdden}{130, fn.\ 2}\antipar
}

\acex{Noun}{Adjective}{11}
{bel-\opt{á}-e rŭwa}
{house(\textsc{m})-\opt{\sg}-\ez{} big.\masc}
{a big house}
{KhanSanandaj}{251}

\acex
{Noun}{Adjective}{1934}
{pəstan-e ʿista}
{gown(\textsc{f})-\ez{} beautiful.\fem}
{a beautiful gown}
{}{(own fieldwork)}

Ordinals behave similarly to adjectives. The loan-ordinal \foreign{ʾăwaḷ}{first} is invariable, while higher \isi{ordinals} show optional agreement:

\acex{Noun}{Ordinal}{35}
{gorá-e ʾăwaḷ}
{man-\ez{} first(\invar)}
{the first man}
{KhanSanandaj}{213}



\acex{Noun}{Ordinal}{281}
{baxtá-e tre-min-\opt{ta}}
{woman(\fem)-\ez{} two-\ord-\opt\fem}
{the second woman}
{KhanSanandaj}{213}


Note that  adjectives can also serve as the \prim of the \ez* construction (compare \example{99}):

\acex{Adjective}{Infinitive}{98}
{(ʾo\cb{} tré) ḥarik-é šyakà (\cb{}ye-lu).ˈ}
{\definite\cb{} two busy-\ez{} wrestle.\inf{} \cb{}\cop-3\pl}
{The two of them were busy wrestling.}
{KhanSanandaj}{331 (2)}
\antipar
\newpage 

The \ez* construction can easily be embedded. In the following examples, the \prims are NPs consisting themselves of the \ez* construction:

\acex
{Noun Phrase}{Adjective}{1935}
{[pəstan-e kald]-e zărif}
{gown-\ez{} bride-\ez{} beautiful(\invar)}
{a beautiful bridal gown}
{}{(own fieldwork)}

\acex
{Noun Phrase}{Noun}{1936}
{[bel-e smoq]-e tat-i}
{house-\ez{} red(\textsc{m})-\ez{} father-\poss.1\sg}
{the red house of my father}
{}{(own fieldwork)}

In such cases the \ez* behaves very similarly to its \Per model, and can similarly be analysed as a \isi{phrasal suffix} (see \sref{ss:ezafe_dispute}). There are also cases where the \secn consists of the \ez* construction:

\acex
{Noun}{Noun Phrase}{1937}
{bel-e [brat-e ʾamm-i]}
{house-\ez{} daughter-\ez{} aunt-\poss.1\sg}
{the house of my cousin (daughter of my aunt)\footnotemark}
{}{(own fieldwork)}

\footnotetext{Surprisingly, the speaker translated \transl{aunt} as \transc{ʾamma} and not as the expected \transc{ʾamta} \citep[cf.][538]{KhanSanandaj}.}

\acex
{Quantifier}{Noun Phrase}{1938}
{tamam-e [bat-e tat-i]}
{all-\ez{} houses-\ez{} father-\poss.1\sg}
{all the houses of my father}
{}{(own fieldwork)}

Conspicuously missing, in contrast to the \Per model, are cases with \isi{adverbial} \secns (see \example{1917}).  In Khan's description there is only one such case, consisting of the fixed \isi{prepositional phrase} \foreign{ʿăla ḥăda}{aside}, borrowed through \Per from \Arab \textarabic{على حد} \transc{ʿalā ḥadd} \citep[569]{KhanSanandaj}.

\acex
{Noun}{\PP}{12}
{tănurá-e ʿăla-ḥădá}
{oven-\ez{} on-edge}
{a separate oven}
{KhanSanandaj}{252 {[B:18]}} 

 
Productive prepositional phrases are lacking from Khan's description, and did not show up in my elicitation. On the other hand, prepositions and nouns serving as adverbials  can appear as \prims of the \ez* construction. When complemented by clauses  the \isi{relativizer} \transc{ke} is sometimes used as well:

\acex{Preposition}{Noun}{69}
{dawr-e mez}
{around-\ez{} table}
{around the table}
{KhanSanandaj}{220}

\acex{Adverbial noun}{Clause}{78}
{wáxt-e [híye bel-àn]ˈ}
{time-\ez{} came.3\masc{} house-1\pl}
{when he came to our house}
{KhanSanandaj}{394 (4)}

\acex{Adverbial noun}{Clause}{17}
{ta\cb{} zămān-e ke\cb{} [ʾanà xlulá wilí]}
{until\cb{} time-\ez{} \rel\cb{} 1\sg{} wedding did.1\sg}
{Until the time that I married}
{KhanSanandaj}{381 {[A:4]}}

Another case where the \ez* construction is not found is whenever the \prim is a noun serving to quantify the \secn. In such cases the \isi{juxtaposition} construction is used. Consider the following example, where the \secn itself is an \ez* construction (compare \example{1933}).

\acex
{Q. Noun Phrase}{Noun Phrase}{1939}
{[xa lewan] [reza-y yarixa]}
{one cup rice(\textsc{m})-\ez{} long(\masc)}
{one cup of long rice}
{}{(own fieldwork)}\antipar



\subsection {Stress retraction as emerging \isi{construct state} marking} \label{ss:JSan_cst_stress}

In \JSan, stress is commonly word-final. However, in non-pausal contexts, the stress of nouns and pronouns may be retracted \citep[53]{KhanSanandaj}. While this phenomenon occurs more widely than just in ACs, it may be seen as an \textit{emerging} construct-state marking.\footnote{Recall that the historical \ili{Semitic} \cst* began also as a prosodic phenomenon of stress-shift; see \sref{ss:cst_Semitic}.} Consider the following example, with attention to the stress position on the head:

\acex{Noun}{Noun}{9}
{bróna Jăhā̀n}
{son J.}
{the son of Jahan}
{KhanSanandaj}{53 {[A:17]}}

The same phenomenon of stress retraction occurs on the noun \foreign{ʾăxóna}{brother} appearing before an adjective in \example{1}.



\section{Genitive marking of \secns} \label{ss:JSan_gen}

A reflex of the \il{Aramaic!Classical}Classical Aramaic \lnk* \d is retained in \JSan only in one  environment, namely optionally preceding  vowel-initial demonstrative pronouns. As such, it has the same distribution as the \gen* prefix \d found in other dialects, and indeed, in \JSan too it can be analysed as a \gen* prefix, as it has no pronominal force typical of the \lnk* \d.\footnote{Cf.\ however \citet[200]{KhanSanandaj}, who assimilates it to the \lnk*, or \enquote{genitive particle} in his terminology: \textquote{The Aramaic genitive particle \textit{d} is used only when the dependent component of an \isi{annexation} construction contains a \isi{demonstrative pronoun}.} Khan uses the notion \enquote{particle} but in fact it is a bound morpheme.} The detailed argumentation for this analysis is given in \sref{ss:d_gen}, and see in particular \sref{ss:gen_dialectal} regarding \JSan.

The situation in \JSan can be contrasted with the situation in the closely related dialects of Kerend and Qarah Hasan, which have lost all trace of the \d \lnk* and always use the unmarked independent pronouns in the \secn position (\cite[11]{KhanSanandaj}, and see \example{Ker1}). 
 
 In the following, I discuss separately the occurrence of \gen* marked demonstratives as \isi{determiners} and conversely as independent genitive pronouns. A third subsection is devoted to \gen* marked demonstratives preceding clausal \secns.  

\subsection {Genitive determiners} 

Demonstrative pronouns used as \isi{determiners} of \secns  take an optional \gen* marking both after nominal and \isi{adverbial} \prims. Since the marking is optional, in contrast to \JZax, the unmarked forms cannot be analysed as non-genitive, but must rather be seen as unspecified forms (±\gen):

\acex{Noun}{Pronoun}{3}
{bela \opt{d}-o naša}
{house \opt{\gen}-\dem.\far{} man}
{the house of that man}
{KhanSanandaj}{200}

\acex{Preposition}{Pronoun}{58}
{reša/reš \opt{d}-o mez}
{on/on.\cst{} \opt{\gen}-\dem.\far{} table}
{on that table}
{KhanSanandaj}{224}

Note that in the last example, the preposition \transc{reša} can appear either in its full form or in its apocopate \cst* form \transc{reš}. Similarly, the \gen* prefix also follows  \prim nouns marked as \cst* by means of the \ez* (=\example{4bis}):

\acex{Noun}{Noun}{4}
{fešár-e d-o màe}
{pressure-\ez{} \gen-\dem.\far{} water}
{the pressure of the water}
{KhanSanandaj}{200 {[A:59]}}

\subsection {Independent genitive pronouns}  
The demonstrative pronouns may also appear as independent pronouns in the \secn position. In this case  they are obligatorily marked by the \gen* prefix. 


\acex{Noun}{Pronoun}{5}
{bela d-o}
{house \gen-3\sg}
{his house}
{KhanSanandaj}{200}

This situation can be contrasted with the closely related dialects of Kerend and Qarah Ḥasan, where such marking is always absent:

\acex[\Ker]
{Noun}{Pronoun}{Ker1}
{bela o}
{house 3\sg}
{his house}
{KhanSanandaj}{11}

As with the \gen* \isi{determiners}, the \gen* marking also appears  after prepositions, including those marked by apocopate \cst* (compare to \example{58} and contrast with \example{62}). Note that the prepositions often pro-cliticize to their complement, obscuring the fact that the \d is part of the \secn:

\acex{Preposition}{Pronoun}{63}
{ba\cb{} d-o}
{in\cb{} \gen-3\sg}
{in it}
{KhanSanandaj}{218}

\acex{Preposition}{Pronoun}{64}
{reša/reš d-o}
{on/on.\cst{} \gen-3\sg}
{on it}
{KhanSanandaj}{224}





Here too, the \gen* marking also occurs after the \ez*, irrespective of the category of the \prim (and see also \example{49}):


\acex{Noun}{Pronoun}{6}
{belá-e d-o}
{house-\ez{} \gen-3\sg}
{his house}
{KhanSanandaj}{200}



\acex{Preposition}{Pronoun}{73}
{ba-dawr-e d-o}
{in-around-\ez{} \gen-3\sg}
{around it}
{KhanSanandaj}{220}

\acex{Adjective}{Pronoun}{100}
{(ʾā́t hămešá) ḥărík-e d-èaˈ}
{2\sg{} always busy-\ez{} \gen-\dem.\sg}
{You are always busy with this.}
{KhanSanandaj}{570 {[A:102]}}


\subsection {Genitive pronouns preceding clausal \secns} \label{ss:gen_pron_clause}

Certain prepositions can be complemented by a clausal \secn, with the help of an intervening \isi{demonstrative pronoun}, itself marked by the \gen* prefix. 

\acex{Preposition}{Clause}{82}
{bar\cb{} d-èa [ʾay ḥášta wil-à-lu]}
{after\cb{} \gen-\dem.\near.\sg{} \dem.\near{} work(\fem) did-\patient.3\fem-\agent.3\pl}
{after they had done this work}
{KhanSanandaj}{392 {[B:17]}}

\acex
{Preposition}{Clause}{82other}
{qắme d-óa [ʾána b\cb{} ʿolā́m henàˈ]}
{before \gen-\dem.\far.\sg{} 1\sg{} in\cb{} world come.\subj-1\masc}
{before I was born}
{KhanSanandaj}{392 {[A:50]}}

The \gen* marking shows that the \isi{demonstrative pronoun} in each example acts as the \secn, i.e.\ the direct complement of the preposition. As it is followed by a clause, one cannot analyse the \dem* in this position as an NP determiner.\footnote{In \JZax, there are rare cases where a determiner is followed directly by a clause, as in \example{434}, yet in such cases the determiner/\dem* has referential power, quite distinct from the cases discussed here.} Indeed, \citet[392]{KhanSanandaj} writes that \textquote{[the] \isi{demonstrative pronoun} [...] is bound anaphorically to the following content clause}. Yet  what is the exact syntactic relation between the demonstrative and the clause? The role of the \dem* is to provide a nominal head acting as the complement of the preposition. As a nominal head, it governs the   clause and embeds it within an NP. It
 follows that the \isi{demonstrative pronoun} and  the clause stand in an \isi{attributive relation} with each other. Yet  this relationship is not marked, as it is instantiated by the \isi{juxtaposition} construction, discussed in \sref{ss:JSan_juxt}. Only the \isi{attributive relation} between the preposition and the \dem* is positively  marked  by means of a \gen* case prefix. The two attributive relations are schematized in \ref{tb:JSan_prep_clause}: 

\begin{table}[h!]
\centering
\begin{tabular}{ccccc}
\toprule
Prep. & $\mapsto_1$ 		& [\dem\ & $\mapsto_2$ & Clause]\textsubscript{NP.\gen} \\
	  & Genitive-marked		&		 & Zero-marked &								\\
\bottomrule
\end{tabular}
\caption{Clausal complement of a preposition mediated by a demonstrative} \label{tb:JSan_prep_clause}
\end{table}

In other dialects, also the second \isi{attributive relationship} is  marked by means of a \cst* marking of the pronominal \prim. Such is the case in \JUrm, as is shown in \examples{264}{265}. In fact, also in \JSan there are examples in which this relation is marked by a \rel* which follows the \isi{demonstrative pronoun}. 

\acex{Preposition}{Clause}{83}
{bár\cb{} d-ea ke\cb{} [xostá xlulá wil-wa-lù]}
{after\cb{} \gen-\dem.\near.\sg{} \rel\cb{} request marriage did-\pst-3\pl}
{after they made the request of the wedding}
{KhanSanandaj}{392 {[A:34]}}

\subsection{Genitivally marked complements of verbal nouns and verbs} \label{ss:JSan_gen_verbal}

The \isi{genitive case} is also used to mark complements of verbal nouns, be they infinitives, \isi{participles}, or complex-predicate nouns. When this marking appears alongside another AC marking, such as the \ez* in the following example, it is quite clear that it too marks the \isi{attributive relation}.

\acex{CP Noun}{Pronoun}{49}
{daʿwắt-e did-ăxun (wilì)ˈ}
{invitation-\ez{} \gen-2\pl{} (did.1\sg)}
{I invited you.}
{KhanSanandaj}{482 {[D:8]}}

Yet, in other cases, one finds the \gen* marking as the sole exponent of the \isi{attributive relation}, as in the following example which instantiates the \isi{inverse \isi{juxtaposition} construction} (see \sref{ss:JSan_inv_juxt_verbal}):

\acex{CP Noun}{Pronoun}{47}
{(kŭ́le ʾaṣər) did-án daʿwàt (k-ol-í)}
{every evening \gen-1\pl{} invitation \ind-do-3\pl}
{They will invite us every evening.}
{KhanSanandaj}{480 {[D:6]}}

Such cases pose an analytic difficulty as \JSan makes use of the \gen* pronouns  also to mark complements of finite verbs. 

\acex
{Verb}{Pronoun}{1944noPrep}
{did-ox grəš-li}
{\obl-2\masc{} pulled-1\sg}
{I pulled you.}
{KhanSanandaj}{159}

\acex
{Verb}{Pronoun}{JSan1bis}
{d-o grəš-le}
{\obl-3\masc{} pulled-\agent.3\masc}
{He pulled him.}
{KhanSanandaj}{159}


In light of such examples, one may re-interpret the \D\~\transc{did-} morphemes not as \gen* case markers but rather as \obl* case markers, fusing together \acc* and \gen* marking. This may be regarded as a development due to \isi{language contact}, as \obl* case is known in \ili{Iranic} languages, notably in \Kur (see \ref{ss:Kurd_obl}), which is however not in direct contact with \JSan, but also in \Hawr \citep[13]{MacKenzieHawrami}, spoken in closer proximity to Sanandaj. \citet[158]{KhanSanandaj} proposes an alternative cause, explaining \example{1944noPrep} as being a derivation of example \ref{ex:1944mod} below, in which the \acc* preposition \transc{həl} was dropped. The \gen* marking is thus justified, as the pronoun is a complement of a preposition:

\acex
{Verb}{Pronoun}{1944mod}
{həl\cb{} did-ox grəš-li\footnotemark}
{\acc\_Prep\cb{} \gen-2\masc{} pulled-1\sg}
{I pulled you.}
{KhanSanandaj}{158 {[modified]}}

\footnotetext{I took the liberty of changing the agent from \third to \first person, in order to provide a clear parallel to the previous example.}

Be that as it may,  the development of \D\~\transc{did-} into an oblique \isi{case marker} permits us to analyse its occurrence in example \ref{ex:47} as marking the object of the entire verbal complex \foreign{daʿwàt k-ol-í}{they invited} rather than marking an attributive \secn of \transc{daʿwàt} alone. Yet, taking into consideration clear cases such as example \ref{ex:49}, I prefer to analyse \transc{did-} first and foremost as a \gen* \isi{case marker}, being an exponent of the \isi{attributive relation}, whenever this is possible, and see any other grammatical functions as being secondary.

In this vein, I consider the following example to be showing a conjoined NP in the \secn position of the \isi{inverse \isi{juxtaposition} construction}, although only the pronominal complement is marked by \gen* case. 


\acex{CP Noun}{Pronoun+Noun}{48}
{(ʾaxtú tămà) [did-í \cb{}u daăk-í] daʿwát (lá kol-étun)ˈ}
{2\pl{} why \gen-1\sg{} \cb{}and mother-\poss.\sg{} invitation \neg{} \ind.make-2\pl}
{Why do you not invite me and my mother?}
{KhanSanandaj}{482 {[D:8]}}

Similarly, I consider pronominal complements of infinitives as being in \gen* case, as in the following example, exhibiting again the inverse order of constituents typical of verbal constructions in \JSan:

\acex{Infinitive}{Pronoun}{93}
{(ʾila di-le ba\cb{}) did-i găroše}
{hand placed-3\masc{} in\cb{} \gen-1\sg{} pull.\inf}
{He began to pull me.}
{KhanSanandaj}{331}


\section{Dative marking of \secns} \label{ss:JSan_dat}

The elicitation session revealed two examples of an elaborate construction in which the \prim is marked by an \ez* suffix and the \secn is marked by the \isi{dative preposition} \transc{əl-}.\footnote{I did not find any mention of this usage in \citet{KhanSanandaj}.} In both cases, a short pause or hesitation is marked after the \prim, which may explain the speaker's need to re-mark the \secn as such by means of the preposition. Note that the usage of the \isi{dative preposition} to mark \secns is not an innovation but rather a retention, as it is attested also in Syriac (see \sref{ss:dat_lnk}). A similar usage of this preposition is  attested in \NMand \citep[152]{HaberlMandaic}.

\acex
{Noun Phrase}{Noun}{1943}
{[xa\cb{} dana bela]-e ... əl\cb{} [brata amm-i]	}
{one\cb{} unit house-\ez{} {} \dat\cb{} daughter aunt-\poss.1\sg}
{one house of my cousin}
{}{(own fieldwork)}


\acex
{Noun Phrase}{Noun}{1942}
{[bel-e raba ʿayza]-y ... əl\cb{} tat-i}
{house-\ez{} much beautiful(\masc) {} \dat\cb{} father-\poss.1\sg}
{the very beautiful house of my father }
{}{(own fieldwork)}

Note that a similar usage of the \isi{dative preposition} is found in \JSul, but without the \ez* marking:

\acex[\JSul]
{Noun Phrase}{Pronoun}{1106}
{ʾaxonàˈ biš\cb{} zor-ăke ʾəl did-ànˈ}
{brother more\cb{} small-\definite{} \dat{} \gen-1\pl}
{the younger brother of ours}
{KhanSulemaniyya}{262 {[R:104]}}\antipar
\newpage 
 

Similarly, in predicative position one finds \secns marked by the \isi{dative preposition}.  The semantic \prim in such cases is the subject of the clause, but it does not form a syntactic constituent with the \secn. Therefore, I treat this construction as lacking a \prim.

\acex
{\zero}{Noun}{1940}
{ay bela [\zero{} əl\cb{} tat-i] \cb{}y}
{\dem.\near{} house \hphantom{[}\zero{} \dat\cb{} father-\poss.1\sg{} \cb{}\cop}
{This house is my father's.}
{}{{(own fieldwork)}}

Again, there is a very similar construction in \JSul:

\acex[\JSul]
{\zero}{Noun}{1098}
{ˈay\cb{} belá [\zero{} ʾəl\cb{} barux-ì] \cb{}ye.ˈ}
{\dem\cb{} house \hphantom{[}\zero{} \dat\cb{} friend-\poss.1\sg{} \cb{}\cop}
{This house belongs to my friend.}
{KhanSulemaniyya}{262 (9)}

Other dialects use the \lnk* \d in the predicative position (see for instance \examples{548}{549} on \Qar). As \JSan has lost the \lnk* \d, it uses instead the \isi{dative preposition} \transc{əl-}. 



\section{Conclusions} \label{ss:JSan_conclusions}

The AC system of \JSan is highly divergent in comparison to most other \ili{NENA} dialects, and in particular the dialects surveyed in the previous chapters. This divergence is at least partly related to extensive \isi{language contact} with \Sor and \Per. 

The most important innovation of \JSan (and related dialects such as the dialect of \Ker) is the loss of the \il{Aramaic!Classical}Classical Aramaic \lnk* \d. Not only is the \lnk* as such lost, but also its head-marking reflex, the \ed \cst* suffix found in other dialects, is absent in \JSan. The loss of these markers is clearly correlated with the rise of the usage of the zero-marked \isi{juxtaposition} construction in the dialect, discussed in \sref{ss:JSan_juxt}. Yet   the usage of the \isi{juxtaposition} construction is not necessarily a direct consequence of the loss of the D-markers: The D-marked constructions can coexist with the \isi{juxtaposition} construction, as is the case in \JSul (see \examples{1068}{1063}). Rather, it is probable that the \isi{juxtaposition} construction itself is contact induced, as will be discussed in \sref{ss:Juxt_general_usage}.

\newpage  
The only remnant of the \d \lnk* in \JSan is  the \gen* prefix \d used before vowel-initial demonstrative pronouns (see \sref{ss:JSan_gen}). Indeed, this very retention is one of my arguments in favour analysing the \phonetic[d] segment in this position as a \gen* prefix, since it follows an independent development path as compared to the \lnk* \d (see \sref{ss:gen_dialectal}). Possibly through contact with Kurdish or \Hawr, the \d prefix in \JSan shows moreover some progress towards becoming an \obl* \isi{case marker} (see \sref{ss:JSan_gen_verbal}). In the closely related dialects of Kerend and Qarah Ḥasan, on the other hand, even the \gen* prefix  is lost. 

Another interesting retention shared by \JSan and \JSul is the sporadic usage of the \isi{dative preposition} \transc{əl-} to mark \secns, found also in Syriac (see \sref{ss:JSan_dat} and compare to \sref{ss:dat_lnk}). Again, it seems that this retention is correlated with the demise of the usage of the \lnk* \d.


Alongside the extensive usage of the \isi{juxtaposition} construction, there is another construction replacing structurally the \ili{Semitic} CSC, namely the \ili{Iranic} \ez* construction, present both in  \JSul and \JSan. The fact that this construction is still largely confined to \isi{loanwords} may indicate that its introduction to these dialects is a relatively late process, not directly related to the loss of the D-markers or the usage of the \isi{juxtaposition} construction. On the other hand, it may be an indication of the cyclic nature of \isi{language change}: The loss of old grammatical markers (the D-markers) is subsequently compensated by adoption of new grammatical markers (the \ez*). This reasoning has also led us to postulate the possible emergence of a new \cst* marking due to stress shift (see \sref{ss:JSan_cst_stress}).

The grammatical developments discussed above have caused an important structural change in \JSan: In contrast to the situation in \il{Aramaic!Classical}Classical Aramaic, conserved in most \ili{NENA} dialects, the distributional distinction between nominal \secns (occurring typically after \cst* nouns or the \lnk*) and adjectival \secns (occurring typically after \free* nouns) has been levelled, as both the \ez* construction and the \isi{juxtaposition} construction treat these two types of \secns alike, the only difference being that native adjectival \secns agree in gender and number with the \prim. On the other hand, clausal \secns are sometimes signalled as such, as they are optionally preceded by borrowed \rel*s in both these constructions (see discussion in \sref{ss:JSan_rel}).






Finally, another important effect of \isi{language contact} is the emergence of the \isi{inverse \isi{juxtaposition} construction}, in which the \secn precedes the \prim. When this construction occurs with verbal nouns as \prims (see \sref{ss:JSan_inv_juxt_verbal}), this can be explained as a consequence of the general shift of the language to an OV order in the verbal domain. The usage of the inverse construction with ordinal \secns (see \sref{ss:JSan_juxt_inverse_adj}) is most probably a converging borrowing from \ili{Arabic} and Sorani (see \vref{tb:ordinals}). The rare usage of the inverse construction with adjectival \secns may also be related to contact (possibly with \Azr), but this requires further investigation.




