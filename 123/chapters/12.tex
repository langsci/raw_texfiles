





\chapter{General conclusions}

The current study aims at a morpho-syntactic comparison of a particular grammatical domain, the attributive system, across \ili{NENA} dialects. It is worthwhile noting that this is the first monograph-sized comparison of Neo-Aramaic dialects ever produced, to the best of my knowledge. As such, it is my hope that it will open the way to further broad comparative studies of various grammatical phenomena in this fascinating language group. In order to appeal to a broad linguistic audience, the approach taken in this study combined methodology and insights from various fields of linguistics, namely typology, contact linguistics and historical linguistics. 


This study had as its starting point the \ili{Semitic}  \concept{annexation} construction, used primarily to modify nominals by other nominals, but also by prepositional phrases and clauses. Following \citet{GoldenbergAttribution}, I identified this construction as the exponent of the \isi{attributive relationship}, and consequently I defined the notion of \concept{attributive construction} as any construction marking this relationship. This permits us to move beyond the specific morphological marking associated with the \isi{annexation} construction (namely the historical \cst*) and examine a wider range of constructions sharing a common functional denominator. 

In the following sections, I shall summarize the major ideas and contributions advanced by this study. The first three sections discuss contributions to general linguistic theory or \isi{linguistic typology}: \ref{ss:cst_universal} discusses the \cst* as a cross-linguistic category.  \Ref{ss:complex_typolgoy} emphasizes the importance of a complex typology of attributive constructions, while \ref{ss:universal_language_change} discusses \isi{encliticization} and cyclicity as possible universals of \isi{language change}.

The following sections, conversely, are specific to the study of \ili{NENA} dialects: \ref{ss:morphemic} re-discusses the various morphemic markers developed from \il{Aramaic!Classical}Classical Aramaic \d, while \ref{ss:variation} gives a short summary of the variation found within the \ili{NENA} AC systems. \Ref{ss:historical} re-traces in broad lines the developments of these systems.

\Ref{ss:lang_contact_challange} concludes again on a more general tone, addressing the difficulty of establishing the direction of \isi{language contact} within \isi{linguistic convergence zones} (\isi{Sprachbund}s). \Ref{ss:futher} ends the chapter with some suggestions for further research questions and directions.



\section{The construct state as a cross-linguistic category} \label{ss:cst_universal}

In the domain of linguistic theory, this study emphasizes  the importance of recognizing  \concept{state morphology}, and in particular the \isi{construct state}, as a cross-linguistic category. As discussed in \ref{ss:state}, the recognition of the notion of \cst* as a valid cross-linguistic category has been suggested before (notably by \cite{CreisselsConstruct}). In this work, I proposed to define the category of  \concept{state} as a non-projecting morphological category which marks the \concept{syntactic valency} of nominals: Construct state nominals require a complement, while \isi{free state} nominals can do without. The {syntactic valency} should  be kept apart from the \concept{semantic valency} of nouns, which corresponds to their inherent argument structure. The latter is encoded morphologically in some languages by an \concept{alienability} split present in the  nominal system \citep[]{BickelNicholsWals58}, but this is quite different from \isi{state morphology}, as the latter encodes the \emph{ad-hoc} \isi{valency} of a nominal. By way of analogy, I proposed to see the \isi{construct state}  as the nominal counterpart of causative morphology in verbs: both add one argument to the argument structure of their host in a given syntactic context.

The \isi{construct state} thus defined is the mirror-image of \isi{genitive case}. The latter marks (\textit{inter alia}) a nominal as a dependent of another nominal, while the former marks a nominal as governing another nominal. Intrinsically \isi{state morphology} is a head-marking device \citep[cf.][]{NicholsBickelWals}. For this reason, \isi{state morphology} is invisible outside the domain of its NP, i.e.\ it is non-projecting.  Notwithstanding its non-projecting property, we have seen that some \ili{NENA} dialects, notably \JUrm, have possibly developed an optional \concept{agreement in state} rule, by which a pronominal \lnk*, being syntactically in \cst*, induces \cst* marking on its antecedent, being the \prim of the construction (see \sref{ss:JUrm_cst_lnk}). As such constructions form cohesive NP units, under this analysis too the \isi{construct state} feature does not project beyond the domain of the NP. 

I have claimed, moreover, that the three-way state system present in \il{Aramaic!Early}{Early Aramaic} should be seen as an idiosyncrasy of these languages. These Aramaic strata distinguish \cst*, \abs* and \emp*. The latter two are instances of the \free*, and the opposition between them relates to the domain of determination: the \isi{emphatic state} was used in the \il{Aramaic!Early}{Early Aramaic} strata for definite nouns  while the \isi{absolute state} was in general used for indefinite nouns. The three-way state distinction is justified in that a \isi{construct state} noun is by itself not determined, but rather the entire NP takes its determination from the complement. As such, the three-state system can be seen as a particular \ili{Semitic} case of the confounding of possessors and determination \citep[cf.][]{HaspelmathArticle}. At the semantic level, however, nominal \isi{valency} and determination are logically independent. 

The usefulness of the category of state becomes clear when examining the debate regarding the \Per \ez*, presented in \sref{ss:ezafe_dispute}. The notion of \isi{construct state} provides a clear notional framework to analyse this particle, thus avoiding much of the controversy surrounding this construction. Moreover, the recognition of a functional category of \isi{construct state} has permitted us to make comparisons between languages which realize this category differently, be it by the \ez* suffix, the \ed suffix or \isi{apocope}. At the same time, the \cst* is differentiated from pronominal cross-reference head-marking, thus rendering it different to a mere synonym of the notion of head-marking. 

\section{Complex typology of attributive constructions} \label{ss:complex_typolgoy}

Building on the works of \citet{PlankIntro} in typology and \citet[Ch.\ 14]{GoldenbergSemitic} in \ili{Semitic} linguistics, I have shown that establishing a simple head- vs.\ depen\-dent-marking typology of adnominal modification is often too simplistic. Rather, as discussed in \ref{ch:attributive}, it is profitable to distinguish additionally between two types of markers, namely simple AC markers (\cst* and \gen* case) and pronominal markers, the latter indexing one member of the construction on the other. The recognition of these different types of markers provides a better account of the diversity of constructions found in \ili{NENA} dialects (and in all probability across the languages of the world). It permits one, moreover, to better trace \isi{language change}, as a pronominal marker can easily fossilize, thereby losing its pronominal value and becoming a simple marker. As discussed in \ref{ss:neo-CSC}, this is very probably what happened in \ili{NENA}, whereby the double-marked DAC, containing  pronominal \prim and \secn markers was simplified to become the head-marked Neo-CSC, exhibiting only a simple \prim marker. 

\section{Universal tendencies of language change} \label{ss:universal_language_change}

This study did not take an \emph{a priori} approach regarding linguistic universals. Yet the observations regarding the changes in the \ili{NENA} attributive systems corroborate certain claims regarding universal tendencies of \isi{language change}. The emergence of the Neo-CSC  (see \sref{ss:neo-CSC}) supports the idea that \isi{encliticization} of functional elements to a preceding host (disregarding their syntactic scope) is  a universal tendency, which  could also explain the general preference of suffixation over prefixation \citep[cf.][]{DryerAffix}. In this respect, the expectations put forward by \citeauthor{LahiriPlank} are indeed borne out: \blockquote[{\cite[395]{LahiriPlank}}]{There is probably something to be said for the wider validity of the trochaic/ dactylic
phrasing preference beyond Germanic and these other families; but we won’t say it here. It
would be of considerable interest, though, because it would help explain the near-universal
preference for suffixing over prefixing. If it does not matter where grammatical words are
positioned relative to the lexical words which they belong with syntactically, before or after,
since phonologically they will always prefer to associate leftwards; and if cliticisation is what
eventually leads to affixation -- then the result will be suffixation rather than prefixation
whatever the syntactic point of departure.}

An important point I have stressed in this regard, however, is that \isi{encliticization} by itself is not enough for \isi{language change}, but it must be accompanied by a subsequent step of re-analysis in order to have a lasting change on the linguistic system. Such a re-analysis may be motivated by external reasons (\isi{language contact}) or internal factors, such as the force of \concept{economy}.

Another observation, stemming from the study of re-emerging double-marked constructions (see \sref{ss:double}), is the cyclic nature of \isi{language change}. While the emergence of the Neo-CSC may have been partially motivated by the economic reduction of a double-marked construction to a single-marked construction, later developments re-introduce double-marked constructions, though in a different guise. I attributed these changes to the meta-linguistic force of \concept{clarity}, favouring more elaborate structures, but one can equally well relate this to the dynamic nature of linguistic systems, always being in a state of transition. This is due to the \concept{creativity} of speakers, who constantly create new linguistic constructions, whether consciously or unconsciously. 



\section[Morphemic differentiation of NENA AC markers]{Morphemic differentiation of NENA attributive construction markers} \label{ss:morphemic}

As we have seen in \ref{ch:synchrony}, many scholars  bundle together the various attributive constructions present in the \ili{NENA} dialects, especially these that contain a reflex of the \il{Aramaic!Classical}Classical Aramaic \lnk* \d, as one construction exploiting the \enquote{\isi{annexation} particle \d}. Building on the work of \citet{CohenNucleus} who discusses \JZax, I have claimed that at least two, if not three, distinct D-markers should be clearly differentiated across \ili{NENA} dialects. Most importantly, the suffix \ed is analysed as a \cst* suffix, instantiating a construction different to that of the \lnk* \d (or various alternative \lnk*s). Similarly,  the \phonemic{d} segment itself should be analysed as two distinct morphemic markers: the pronominal \lnk* \d, and the \isi{genitive prefix} \d, present before certain demonstratives and \isi{determiners}. While the \isi{genitive prefix} is more difficult to ascertain across all \ili{NENA} dialects (for instance, its presence is debatable in the \Qar dialect; see \ref{ch:Qaraqosh}), it is nonetheless useful to recognize its potential occurrence in various \ili{NENA} dialects, in order to trace the development of these constructions.


\section{Variation and uniformity in NENA dialects} \label{ss:variation}

In the introduction I set out two research questions directly related to the study of \ili{NENA} dialects:

\begin{enumerate}

\item What is the extent of the variation among attributive constructions in the documented \ili{NENA} dialects? Which different constructions exist in the various dialects to express the \isi{attributive relationship}?

\item How do these constructions relate to the contact languages of \ili{NENA} \textit{vis à vis} the historical background of \ili{NENA}? In other words, what was the role of \isi{language contact} in shaping the synchronic manifestations of the \isi{attributive construction}s in \ili{NENA} dialects?

\end{enumerate}





The research clearly demonstrates that the various \ili{NENA} dialects present a wealth of different constructions and sub-constructions within the attributive domain. The richness of these systems (within each dialect and across the group as a whole) is due to the wide geographic spread of these dialects, allowing for different source constructions for each dialect's system: On the one hand, one finds numerous diverging contact languages potentially affecting  each dialect. On the other hand, the various dialects may themselves be descendants of diverging anterior dialects, representing an ancient undocumented dialectal continuum not necessarily originating in a unique Proto-\ili{NENA} dialect. Indeed, the high diversity found in the study corroborates rather the view that no unique Proto-\ili{NENA} dialect of the \il{Aramaic!Classical}Classical Aramaic period existed.\footnote{Such a view is also advocated by \citet{KimStammbaum}. It is interesting to note that \citet[558f.]{Hoberman1988history}, while acknowledging the existence of such an ancient dialectal continuum, nonetheless posits the existence of a Proto-\ili{NENA} dialect, possibly for methodological reasons.}

Yet, glossing over some of the finer details, there are two constructions (in a broad sense) that re-occur again and again in the various dialects:

\begin{description}

\item[The \isi{construct state} construction:] In this construction the \prim is marked morphologically by the \cst*. In the vast majority of dialects this is achieved by the Neo-\cst\ suffix \ed, but some dialects have revived the use of apocopate \cst\ while others have borrowed the \ili{Iranic} \ez* as a \cst* marker. 

\item[The \isi{analytic linker construction}:] In this construction type a \lnk* is joined syntactically  with the \secn, while representing pronominally the \prim. In some dialects the \lnk* is a direct reflex of the \il{Aramaic!Classical}Classical Aramaic \lnk* \d, while in other dialects alternative \lnk* forms are used, typically reflexes of a \dem+\lnk\ combination or of the pronominal base \transc{did-} present in \JBA. 

\end{description}

It is fascinating to see that while the morphemic material of these two constructions is very often innovated in \ili{NENA} (such as the ubiquitous Neo-\cst\ suffix \ed), they actually represent continuity with older strata of Aramaic, and indeed with the \ili{Semitic} language family as a whole, since these two construction types are documented in \ili{Semitic} languages since their  earliest attestations (see \sref{ss:AC_sem}). Indeed, as \citet{CohenNucleus} notes, the \ili{NENA} dialects have re-introduced structural features present in ancient \ili{Semitic} languages but lost in the \il{Aramaic!Classical}Classical Aramaic stratum, notably the possibility of having clausal \secns in the CSC. Evaluating the role of \isi{language contact} in these developments, I gave a nuanced picture, pointing out that they can be conceived both as products of \isi{language contact} with diverse languages and as internal developments. The general difficulty with ascertaining \isi{language contact} is re-iterated in \sref{ss:lang_contact_challange}. 

Alongside these two construction types, one finds two minor construction types occurring in some \ili{NENA} dialects:

\begin{description}

\item[The \isi{juxtaposition} construction:] In a typical grammar of a \ili{Semitic} language, \isi{apposition} is expressed by the  \isi{juxtaposition-cum-agreement} construction, which typically occurs with adjectival \secns. Some \ili{NENA} dialects, however, have extended the use of \isi{juxtaposition} (without agreement) to be a genuine marker of the \isi{attributive relation}, both with nominal and clausal \secns (see \sref{ss:loss_marking}). 

\item[The \isi{relativizer} construction:] In dialects which are in intensive contact with \ili{Iranic} languages, a matter-cum-\isi{pattern replication} of the Relativizer Construction can be found, yielding a construction in which a subordinating particle introduces clausal \secns (\sref{ss:borrow_ke}).

\end{description}


In contrast to the two former constructions, the two latter constructions can be qualified as true innovations of \ili{NENA} dialects with respect to anterior strata of Aramaic. The \isi{relativizer} is clearly an instance of \isi{matter replication}, but the research also corroborates  the hypothesis that the \isi{juxtaposition} construction is a case of \isi{pattern replication}, as it occurs especially in areas where intensive \isi{language contact} with \ili{Iranic} languages took place. Indeed, the examination of the co-territorial 	dialectal Kurdish and \ili{Gorani} data establishes a direct connection with the \isi{juxtaposition} constructions extant in these languages.\footnote{As noted in \sref{ss:AC_sem}, the usage of the \isi{juxtaposition} construction with clausal \secns and indefinite \prims is very probably related to contact with \Arab vernaculars, and indeed this kind of usage (in contrast to the generalized use of \isi{juxtaposition} to mark the \isi{attributive relation}) is more widespread across \ili{NENA} dialects.} 

Another innovation found across \ili{NENA} dialects is the introduction of the \d prefix as a genitive \isi{case marker} of certain \isi{determiners} and pronouns (mostly demonstratives). I noted that this major innovation in the grammar of Aramaic goes against the supposed universal dispreference of prefixation. Consequently, I attributed it to the effect of \isi{language contact}, and more specifically to \isi{pattern replication} of the Kurmanji oblique demonstratives (see \sref{ss:genitive_development}). 

\section{Historical development of the NENA attributive constructions} \label{ss:historical}

In Chapters 10--11 I have advanced several hypotheses regarding the development of the \ili{NENA} attributive constructions. These amount to the conception of a \enquote{Domino model} or a \concept{chain reaction}: The initial point for the re-shaping of the \ili{NENA} AC system was the re-analysis of the DAC (\foreign{bayt-ēh d\cb{}malkā}{house of the king}) as the Neo-CSC (\transc{bet\ed\ malka}). This pivotal change in the system led to further phonological reshuffles, which eventually brought about the \isi{genitive case} prefix (\sref{ss:genitive_development}) as well as the innovated apocopate CSC (\sref{ss:apcopate}). The latter point is especially worth emphasizing, as I reject the view that the apocopate CSC is a re-generalisation of the historical apocopate CSC. 

Also the emergence of new \lnk*s can be related to this \enquote*{chain reaction}. I claimed, for instance, that the \lnk* \transc{did}, originally serving only as a pronominal base (e.g.\ in \JBA), could be generalized due to its apparent \cst* suffix \transc{-d} (\sref{ss:did_lnk}). As for other \lnk*s, such as \transc{od} or \transc{ad}, these too are the result of the \isi{encliticization} of the original \lnk* \d to its \prim, a development in line with the emergence of the Neo-\cst\ suffix \ed (\sref{ss:od_lnk}). Finally, the emergence of the \lnk* \transc{ay} in \JUrm (and related dialects) originates in a further deletion of the  \transc{-d} segment (\sref{ss:JUrm_ay}). 

Certainly, these claims can be challenged, and to a certain extent they should be seen as hypotheses rather than firm facts. Yet the study's model has the advantage of giving a unified account to the majority of changes in the \ili{NENA} attributive systems, thus explaining  the striking similarity between many \ili{NENA} dialects, without necessarily postulating a putative Proto-\ili{NENA}. The fact that the initial change was motivated by the \isi{encliticization} of the \lnk* \d to the \prim, a change which is in line with the claimed universal tendency of \isi{encliticization} of functional elements (see \sref{ss:universal_language_change}), means moreover that this change could have happened independently in several pre-\ili{NENA} dialects, without necessarily sharing a common ancestor of the \il{Aramaic!Classical}Classical Aramaic period. This is to some extent corroborated by the fact that also \WNA shows a similar re-structuring. Yet the fact that in \WNA the subsequent changes did not take place  may hint that at least some of the subsequent re-analysis occurred under the influence of \isi{language contact}. 

\section{Language contact and linguistic convergence} \label{ss:lang_contact_challange}

A major research question posed at the outset was to identify which \ili{NENA} constructions are due to \isi{language contact}, and which are due to internal developments. It was hoped that clear differences in the geographic distribution of certain patterns would allow the discovery of clear \isi{language contact} effects. This expectation was borne out only partially. Indeed, according to the data gathered, it seems that the dialects in the south-eastern periphery of the \ili{NENA} speaking zone, i.e.\ the dialects in the \Sor Kurdish speaking area (roughly from Arbel southwards), show greater susceptibility to contact effects, as is apparent from the numerous cases of \isi{matter replication} in these dialects, be it the \ez* suffix or the \ili{Iranic} \rel* (see \vref{tb:borrowed}). In this vein, it seems reasonable to conclude that the generalized \isi{juxtaposition} construction found in \JSan and \JSul is a product of \isi{language contact} (see \sref{ss:Juxt_general_usage}). 

Yet, as discussed in \ref{ss:role_contact}, such a clear assertion becomes more difficult when dealing with the most important construction of \ili{NENA}, namely the Neo-CSC. Part of the problem lies in the fact that this construction is widespread and occurs  in virtually all corners of the \ili{NENA} speaking zone (see \vref{tb:suff_cst}). Another difficulty lies in the fact that it shows affinity not only  with various Kurdish dialects (both \Sor and \Kur) but also with \il{Aramaic!Classical}Classical Aramaic languages such as Syriac (see \vref{tb:cst_comp}). Moreover, the development of the Neo-\cst\ suffix seems to follow from the universal tendency of \isi{encliticization} of functional elements. The last two factors go against \isi{language contact} as the source of this construction, yet the great functional and structural similarity with contact languages seems to indicate that some sort of contact must be involved. I concluded that the Neo-CSC is an instance of a linguistic convergence, in line with the areal preference to head-mark attributive constructions, but without positing a specific source language.

This conclusion places the inquiry into the development of the Neo-CSC in a wider context, namely the dynamics of linguistic convergence zones, known also as \isi{Sprachbund}s.\footnote{The recognition of the impact of linguistic convergence areas on the \ili{NENA} grammar has been highlighted in numerous recent publications. \citet{NoorlanderStilo} discuss the verbal system of \ili{NENA} dialects as part of the Araxes-Iran Linguistic Area, while \citet{Gandon} discusses the relativization strategy of Iranian \ili{NENA} dialects as part of the Caucasus-Western Iran area. See also the preface of \citet[VII]{KhanNapiorkowska}: \enquote{[T]he historical development of Neo-Aramaic cannot be fully understood without taking into account the structures of the languages with which the dialects have been in contact. [...] the parallels have developed in the Neo-Aramaic dialects by varying degrees of convergence with other languages}.}
 Since the languages in such areas show high structural similarity due to a  long history of contact, it is very difficult for any specific grammatical construction to ascertain the direction of contact and consequently whether a given construction is native to a certain language or not.  A somewhat similar difficulty is addressed by \citet{PatElContact}, who wishes to distinguish between internal developments and \isi{language contact} between \enquote{genetically related languages}. Methodologically, this is apparently a different question, since a convergence area brings together non-related languages, yet in both cases one is faced with the difficulty of reconstructing the historical development of languages with abundant structural similarities. Pat-El suggests to remedy the difficulty by scrutinizing 1) intermediate stages of \isi{language change} processes and 2) the generalization of a construction across categories. As we have seen in the study of the Neo-CSC, these measures are not helpful in this case, since both in \ili{NENA} and in \ili{Iranic} languages one can observe intermediate stages and generalisation across categories. 

To conclude this point, in convergence zones \isi{language contact} clearly plays a role, yet it is difficult, and in some cases maybe even impossible, to relate a specific structural feature to one source language. In \sref{ss:language_contact_conclusions} I formulated the (still speculative) hypothesis, that the preference for head-marking of ACs (i.e., \cst* morphology) is originally an Aramaic  feature transferred into \ili{Iranic} languages, and then re-transferred into Neo-Aramaic. If this is true, the Neo-\cst\ is effectively both a product of contact and of native Aramaic grammar.

\section{Further research questions} \label{ss:futher}

The aim of this study was to describe and compare the attributive systems of various \ili{NENA} dialects, attempting as well to reconstruct their development. Special emphasis was given to the question of the effect of \isi{language contact} in the emergence of these systems. Yet, from a more general typological point of view two further questions can be asked:

\begin{enumerate}
	\item Are the \ili{NENA} AC systems typical or exceptional amongst the languages of the world? Do they show typical patterns of head-marking languages? What do they teach us about head-marking languages?
	
	\item In the study I have traced several changes in the AC systems over time. For instance, I claimed that the emergence of the \isi{genitive case} was subsequent to the emergence of the Neo-\cst\ suffix. Are these claims in line with known universals of \isi{language change}? Do they corroborate exiting implicational universals? Do they allow deducing new implicational scales from them?
	
\end{enumerate}

Another further direction of study regards the methodology of the research.
Methodologically, the analysis used in this study is purely a qualitative one. The data gathered could be exploited instead  in a quantitative approach, relying on the recent advances in the production of phylogenetic trees in \isi{linguistic typology}. The availability of constructions in the various dialects, presented in tabular format in Chapters 10--11, could be seen as \concept{features} fed into this type of analysis. These results may elucidate the question of classification of the \ili{NENA} dialects, and could be compared with classification done using the traditional comparative method \citep[e.g.][]{Hoberman1988history,MutzafiTransZab}. Moreover, as discussed in the last section, the question of the importance of \isi{language contact} still remains somewhat ambivalent. One could try to disentangle this question using a phylogenetic tree, by taking into account also data from the contact languages surveyed, and observe whether they cluster with specific \ili{NENA} dialects. In this respect, while the study has not resolved all the questions raised, it provides a wealth of ready-made data useful for further investigation. 

Finally, the AC system is just one part of the NP domain. In \sref{ss:intro_NPstructure} I have touched briefly upon the question of determination and quantification of the \ili{NENA} NP,  but these questions would in fact profit from a dedicated comparative research. As we have seen, this topic is not disconnected from the AC system: in \ili{Semitic} languages in general, determination interacts with \isi{state morphology} (\sref{ss:CSCdet}), and in \ili{NENA} in particular some of the \lnk*s are clearly related to \isi{determiners} (notably \JUrm \transc{ay}). Thus, the study of the determination system of \ili{NENA} would be the natural continuation of the current study. Indeed, as I stated above, it is my hope that this study would provide an example for further in-depth and broad comparative studies of Neo-Aramaic dialects. 






