\chapter{Physical Interaction}\label{ch:interaction}
\index{interaction!physical}

As pointed out before, grounding an ontology and lexicon is supposed to be influenced for a great deal by agents' physical interaction with their environment. In this chapter several influences of these physical interaction are investigated. The robot's interaction schemes are varied in two cases, the physical body has been changed in one experiment and in still another experiment both the environment and the physical body has been changed. Naturally, all experiments are compared with the basic experiment.

\p
In the basic experiment, the robots decided to stop after two rotations based on finding a maximum intensity of IR on one back IR sensor. In the description of the model the robots stopped after they aligned their backs towards each other using IR taxis (section \ref{s:robots:PDL}). In section \ref{s:int:taxis} two experiments are shown where taxis is used to align the robots. In the first experiment it was observed that the gearings of the robots were worn off. In the second experiment the gearings were replaced by new ones. These experiments show how coordination abilities and physical fitness may influence the quality of interactions.

In \cite{steelsvogt:1997} the robots did not rotate twice aligning back-to-back while doing the perception, but only once aligning face-to-face. This experiment has been repeated in section \ref{s:int:original} to see what the differences are.

The adaptation of an agent to its environment and the agent's ability to perceive the environment with enough precision is likely to be very important. In section \ref{s:int:close} the environment and robots are changed such that the resolution of the robots decreases.

In all the above experiments there were constantly 4 light sources present in the robots' environment. What happens when in each situation there are only 3 light sources present, while the robots' niche has 4 light sources? This question shall be investigated in section \ref{s:int:3refs}.

\section{Alignment Using IR Taxis}\label{s:int:taxis}
\index{infrared taxis|see{phototaxis}}
\index{phototaxis}

Applying IR taxis (or taxis for short) for alignment has been explained in the original description of the model in chapter \ref{ch:lg}, so it will not be repeated here. When the robots use taxis their alignment is better than when they use one IR sensor and finding the maximum intensity. However, the physical behavior is more complex and less games succeed in performing the complete physical behavior. Thus the acquisition of sensory data takes much more time. A reason why for other data sets the maximum intensity condition was used. 

Although the robots are better at aligning each other, it is not likely that the performance increases because the perception is done in the middle of the two rotations. Perhaps the influence of worn off gearings can be observed.

\index{gearings|(}
\subsection{Worn off gearings}

The data in this experiment has been recorded using taxis. During the recording it was found that the gearings were worn off, causing the robots to move less smoothly than they would when the gearings are brand new. A data set of 606 situations has been recorded\footnote{The fact that the robots had problems in rotating was the reason why the data recording has been abandoned at this early stage.}. The average context size per situation was 3.64 for robot r0 and 3.83 for r1, so the a priori communicative success would be 0.268. The potential understandability is 0.897 for r0 and 0.824 for robot r1. It seems as if the slower rotation of the robots allowed them to perceive more coherent contexts. Off-board processing of the data is done exactly as in the basic experiment.

\p
Table \ref{t:int:taxis} shows the averaged measures of 10 runs of 5,000 language games. The experiment is in most ways similar to the basic experiment. Only the discrimination game is more successful (approximately 2 \%, $p=0.0004$). Specificity is higher and consistency is lower, but their significance is low ($p=0.1704$ and $p=0.2798$ resp.).

\begin{table}
\centering
\begin{tabular}{||l|c|c||}
\hline\hline
Score & Avg & Std\\\hline
CS & 0.350 & 0.005\\\hline
DS0 & 0.945 & 0.006\\\hline
DS1 & 0.944 & 0.001\\\hline
D0 & 0.956 & 0.000\\\hline
D1 & 0.960 & 0.000\\\hline
P0 & 0.852 & 0.006\\\hline
P1 & 0.880 & 0.007\\\hline
S0 & 0.849 & 0.006\\\hline
S1 & 0.869 & 0.011\\\hline
C0 & 0.802 & 0.001\\\hline
C1 & 0.828 & 0.006\\\hline
\hline
\end{tabular}
\caption{The average results of the experiment involving taxis and old gearings.}
\label{t:int:taxis}
\end{table}

So, although the gearings of the robots were really at their ends, the communication system that emerges is not worse than the basic experiment. Question is if this result is biased by the fact that this data set only consists of 606 situations rather than 1,000. Table \ref{t:int:basis606} presents the results of the basic experiment using 606 situations taken from the basic data set used. The table shows that using only 606 situations does not alter the results of the basic experiment very much, so the smaller data set does not really bias the experiment. 

\begin{table}
\centering
\begin{tabular}{||l|c|c||}
\hline\hline
Score & Avg & Std\\\hline
CS & 0.354 & 0.016\\\hline
DS0 & 0.794 & 0.009\\\hline
DS1 & 0.816 & 0.008\\\hline
D0 & 0.959 & 0.000\\\hline
D1 & 0.960 & 0.001\\\hline
P0 & 0.869 & 0.004\\\hline
P1 & 0.877 & 0.002\\\hline
S0 & 0.849 & 0.019\\\hline
S1 & 0.853 & 0.007\\\hline
C0 & 0.820 & 0.014\\\hline
C1 & 0.831 & 0.014\\\hline
\hline
\end{tabular}
\caption{The results of the basic experiments using only 606 situations from the basic data set.}
\label{t:int:basis606}
\end{table}

\subsection{The Repaired Robots: New Gearings}
\index{data set!taxis \& gearing}
\index{a priori success}
\index{understandability}

The results of the experiment where the robots controlled their physical behavior is shown in table \ref{t:int:gearing}. The rest of the experimental setup is like the previous experiment. 934 situations have been recorded in the so-called {\em taxis \& gearing data set}. From this data set it is found that the average context sizes of the two robots are 3.283 and 3.485, thus the a priori communicative success is: 0.296. The potential understandability is 0.808 for robot r0 and 0.779 for r1. This is lower than in the previous experiment and approximately equal to the basic data set. It is important to note that the robot now moves at a higher speed than with worn off gearings. Rotating faster reduces the resolution of the perception. Hence the robots may miss a small perturbation of a distant light source. The basic data set has been recorded immediately after this experiment, so there the gearings were still okay.

The communicative success is 2.8 \%  better than the basic experiment ($p=0.1230$). It is also better than the taxis experiment with old gearings with a significance of $p=0.0752$. The discriminative success is more or less equal compared with the basic experiment and is $\pm 2.5$ \% lower for the old gearings ($p=0.0008$). There are no significant differences when comparing the distinctiveness, parsimony, specificity and consistency with the two other experiments. So, also using new gearings does not influence the ability for the robots to construct ground a language very much. Slight improvement is found when comparing to the basic experiment, although this is not significant.

\begin{table}
\centering
\begin{tabular}{||l|c|c||}
\hline\hline
Score & Avg & Std\\\hline
CS & 0.379 & 0.013\\\hline
DS0 & 0.918 & 0.003\\\hline
DS1 & 0.917 & 0.003\\\hline
D0 & 0.959 & 0.000\\\hline
D1 & 0.960 & 0.001\\\hline
P0 & 0.864 & 0.001\\\hline
P1 & 0.858 & 0.002\\\hline
S0 & 0.837 & 0.014\\\hline
S1 & 0.824 & 0.018\\\hline
C0 & 0.803 & 0.006\\\hline
C1 & 0.794 & 0.004\\\hline
\hline
\end{tabular}
\caption{Results of 10 runs of 5,000 language games in which the robots got new gearings.}
\label{t:int:gearing}
\end{table}

\p
So, it seems that when the robots can better control their movements by using new gearings, their ability is slightly better than when they use old gearings. It is striking, however, that in comparison to the basic experiment the worn off gearings experiment outperforms the basic experiment in some ways. The first important difference is the discrimination success. The second difference is that the potential understandability is lower in the `new gearings' data set than in the `worn off gearings' data set. 
\index{gearings|)}

\section{Face-to-face alignment}\label{s:int:original}

In the original implementation the robots rotate only once starting face-to-face \cite{steelsvogt:1997} rather than rotating twice and starting back-to-back. When the robots rotate once they immediately start the perception, while when rotating twice they start perception when the rotating robot faces its opponent. This way the robot is already moving at a constant speed, whereas in the original implementation the robots first have to accelerate. When the robots first have to accelerate, the landscape view initially is somewhat warped. 

Statistics of the recorded data set yielded the following: The data set has 1360 recorded situations. The average context sizes are 3.546 for r0 and 3.354 for r1, yielding an a priori communicative success of 0.290. Robot r0 has a potential understandability of 0.722 and r1's potential understandability is 0.764. The context size is almost the same as in the basic experiment, but the potential understandability is pretty much lower. This latter finding is likely to be caused by the fact that it is not assured that the robots really rotate $360^o$. However, it may also be caused by the initial acceleration phase.

This lower understandability seems to have little effect on the results (table \ref{t:int:original}). The distinctive success is about 3 \% lower, which is significant ($p=0.0000$). Also the communicative success is lower: 2 \%, but with $p=0.0770$. All other differences are insignificant. So, the onset of acceleration cannot be observed as an important difference. However, when pointing is involved using the physical method this has much influence (see next chapter).

\begin{table}
\centering
\begin{tabular}{||l|c|c||}
\hline\hline
Score & Avg & Std\\\hline
CS & 0.331 & 0.008\\\hline
DS0 & 0.883 & 0.002\\\hline
DS1 & 0.891 & 0.001\\\hline
D0 & 0.957 & 0.011\\\hline
D1 & 0.956 & 0.011\\\hline
P0 & 0.861 & 0.012\\\hline
P1 & 0.855 & 0.009\\\hline
S0 & 0.826 & 0.009\\\hline
S1 & 0.823 & 0.009\\\hline
C0 & 0.818 & 0.007\\\hline
C1 & 0.809 & 0.009\\\hline
\hline
\end{tabular}
\caption{The average results of 10 runs of 5,000 language games where the robots rotated only once during the perception.}
\label{t:int:original}
\end{table}

\section{Reducing environmental distinctiveness}\label{s:int:close}

In the environment used so far, the light sources and sensors were placed at different heights with a difference of 3.9 cm. This way the environment was made rather distinctive as can be seen in figure \ref{f:robots:calibration} on page \pageref{f:robots:calibration}. What happens if the difference in heights are made smaller? Naturally it is expected that the robots have more difficulty in discriminating and identifying the light sources.

\index{sensors!white light|(}
In this experiment the difference in heights were reduced to 1.9 cm. Figure \ref{f:int:calibration} shows the characteristics of the sensors as measured for different distances when facing a light source. It is obvious that the further a robot gets away from the light source, the closer the different sensor readings are. Furthermore, it should be clear that when the distance between robot and light source is larger, correspondence between sensor and light source is unreliable. Hence, the feedback mechanism is unreliable. Interesting to see is that when the robot is close to the light source the non-corresponding sensors hardly sense light, but up to 40 cm the intensities increase. This is because at close distance the light source is invisible for these sensors and at larger distance the divergent light emission falls on the sensors.

\begin{figure}
\subfigure[L0]{\psfig{figure=physical//sensors0.eps,width=5.6cm}}
\subfigure[L1]{\psfig{figure=physical//sensors1.eps,width=5.6cm}}\\
\subfigure[L2]{\psfig{figure=physical//sensors2.eps,width=5.6cm}}
\subfigure[L3]{\psfig{figure=physical//sensors3.eps,width=5.6cm}}
\caption{The characteristics of sensors s0, s1, s2 and s3 of robot r0 while looking at light sources (a) L0, (b) L1, (c) L2 and (d) L3. The light sources are placed at heights with a difference of 1.9 cm in between. Note that the characteristics of L3 may be inaccurate since the characteristics is quite different from all other characteristics.}
\label{f:int:calibration}
\end{figure}

\p
The statistics of the data set reveal the following: The context sizes are 3.530 (r0) and 3.483 (r1), thus the a priori communicative success is 0.285. Potential understandability is $0.639\pm 0.292$ (r0) and $0.679 \pm 0.321$. Again this is much smaller than in the basic experiment. Note that these are unreliable statistics since the method for relating an observation to a referent is not correct when the robots are at larger distances (see figure \ref{f:int:calibration}). The recorded data set has 953 situations.

Again 10 runs of 5,000 language games are done. The results are presented in table \ref{t:int:close}. The communicative success is around the a priori value; its significance in comparison to the basic experiment is $p=0.0000$. The discriminative success is similar to the basic experiment. 

The distinctiveness seems approximately the same as in the basic experiment, but its $p$-value is $p=0.0114$, which is not very high. So, it seems likely that the two experiments yield different distinctiveness, but its difference is not large ($\leq 0.002$). Since the difference is so small, no further implications will be made.

Besides the specificity which does not show a significant difference, the parsimony and consistency ($p=0.0068$ and $p=0.0028$ resp.) are significantly different and lower than in the basic experiment. Obviously this has to with the large overlap in the sensory characteristics.


\begin{table}
\centering
\begin{tabular}{||l|c|c||}
\hline\hline
Score & Avg & Std\\\hline
CS & 0.281 & 0.007\\\hline
DS0 & 0.913 & 0.004\\\hline
DS1 & 0.917 & 0.004\\\hline
D0 & 0.954 & 0.002\\\hline
D1 & 0.955 & 0.002\\\hline
P0 & 0.823 & 0.001\\\hline
P1 & 0.822 & 0.001\\\hline
S0 & 0.829 & 0.015\\\hline
S1 & 0.840 & 0.014\\\hline
C0 & 0.778 & 0.008\\\hline
C1 & 0.778 & 0.007\\\hline
\hline
\end{tabular}
\caption{The results of an experiment where the distance between light source heights have been made closer.}
\label{t:int:close}
\end{table}
\index{sensors!white light|)}

\section{A dynamic environment}\label{s:int:3refs}

This section presents an experiment where in every situation the robots recorded there were only three light sources present. The height of the light sources were the same as in the basic experiment. After every few games, one of the light sources was removed and the one that was already out of the environment has been placed back. Whereas in the other experiments all light sources stayed roughly at the same place, the position of the light sources changed in this experiment as well. This way a dynamic environment was created.

A data set of 980 situations has been recorded. The average context sizes were measured to be 2.857 for r0 and 2.899 for r1, yielding an a priori communicative success of 0.347. The potential understandability was $0.757 \pm 0.314$ for $r0$ and $0.738 \pm 0.308$ for r1. 

Table \ref{t:int:3refs} shows the results of an experiment of 10 runs of 5,000 language games. The communicative success is about 2.5 \% higher than the a priori value, and about 2 \% higher than the basic experiment, but this latter result is not very significant ($p=0.0892$). Distinctiveness, specificity, parsimony and consistency show no significant difference with the basic experiment. Discriminative success looks higher than in the basic experiment, but its significance is low: $p=0.0630$.

\p
The fact that the CS is only slightly higher than the a priori success makes it hard to draw a meaningful conclusion. Nevertheless, it seems that the robots perform as if there are 4 referents. This is of course correct, there are 4 referents in the world, but in each situation there are only 3. This may be why the robots have some difficulty in performing with the same specificity and consistency as when all referents are continuously in their proximity.

\begin{table}
\centering
\begin{tabular}{||l|c|c||}
\hline\hline
Score & Avg & Std\\\hline
CS & 0.372 & 0.018\\\hline
DS0 & 0.927 & 0.005\\\hline
DS1 & 0.932 & 0.003\\\hline
D0 & 0.959 & 0.000\\\hline
D1 & 0.958 & 0.000\\\hline
P0 & 0.858 & 0.003\\\hline
P1 & 0.847 & 0.001\\\hline
S0 & 0.807 & 0.009\\\hline
S1 & 0.812 & 0.007\\\hline
C0 & 0.814 & 0.008\\\hline
C1 & 0.812 & 0.008\\\hline
\hline
\end{tabular}
\caption{Results of 10 runs of 5,000 language games in which the environment consists of only 3 referents.}
\label{t:int:3refs}
\end{table}


\section{Summary}

In this chapter the influence of different types physical interactions have been explored. In the first experiment IR taxis was used to let the robots align to each other after perception. In this experiment it has been observed while recording the data that the gearings were worn out. In the second experiment these have been replaced by new ones, still using taxis. The third experiment was setup like the original implementation \cite{steelsvogt:1997} rotating only once to do the perception. The environment and the robots were changed in the fourth experiment. The light sources were placed at heights that are closer to one another; the sensors were adjusted accordingly. In the last experiment the robots played language games where for each situation there were only three referents. Figure \ref{f:int:results} shows an overview of the results.

\begin{figure}
\subfigure[CS]{\psfig{figure=physical//cs.eps,width=5.6cm}}
\subfigure[DS]{\psfig{figure=physical//ds.eps,width=5.6cm}}\\
\subfigure[S]{\psfig{figure=physical//spec.eps,width=5.6cm}}
\subfigure[D]{\psfig{figure=physical//dist.eps,width=5.6cm}}\\
\subfigure[C]{\psfig{figure=physical//cons.eps,width=5.6cm}}
\subfigure[P]{\psfig{figure=physical//pars.eps,width=5.6cm}}
\caption{An overview of the results of the experiments presented in this chapter. The experiments investigating the influence of taxis with old gearings (T) and new gearings (G), one rotation (O), different heights (H) and 3 referents (3) are compared with the basic experiment (B).}
\label{f:int:results}
\end{figure}

\p
A striking result has been observed when the robots use taxis with or without new gearings. These experiments are qualitatively more or less similar to the basic experiment. Differences in discrimination success in the taxis experiment with old gearings may lie in the fact that this was the first experiment after the sensors have been calibrated. It is not unlikely that the accuracy of the sensors become less reliable through time. That taxis as such has no influence on the language games is because the perception is completed before the robots start taxis. Beginning and end point of the perception are the same, so the robots are well capable of doing perception for $360^o$ in experiments with or without taxis.

The experiment where the robots rotate only $360^o$ and where there are only three referents present are also qualitatively similar as the basic experiment. So, the slow onset of movement has little impact on the robots performance in these experiments. Furthermore, the robots seem to be well capable of dealing with a dynamic environment. Although the a prior success is higher, the robots appear to perform as if there are four referents. All these experiments show that the data recording can be repeated without influencing the experiments very much.

When the environment is changed such that it is less distinctive the performance is significantly worse than the basic experiment. Surprising is that this does not hold for the discrimination success. It seems to have more impact on the ability to provide reliable feedback. However, the results might indicate the importance of agents' physical adaptation to their environment as a basis for language origins.

\p
Physical interactions are also a part of how joint attention and feedback can be provided to the agents. However, these processes additionally require cognitive capabilities. Experiments investigating the influence of these interaction strategies are presented in the next chapter.
