\chapter{Joint Attention and Feedback}\label{ch:feed}

In chapter \ref{ch:theory} the possible roles of joint attention and feedback have been discussed. While presenting the cognitive model (chapter \ref{ch:cm}) different methods have been introduced that control the way an agent may or may not be provided with joint attention. Furthermore, methods have been introduced to model the way feedback is provided. This chapter presents the experiments where the different joint attention and feedback methods and strategies have been applied.

\p
In the original experimental setup up the robots implemented physical pointing as a way to provide prior topic knowledge \cite{steelsvogt:1997}. Later the physical pointing has been simulated to provide both prior topic knowledge and feedback in the language games \cite{vogt:1998b}. In \cite{vogt:1998c} different strategies for providing joint attention and feedback have already been investigated. Here it was observed that the scheme that has later been called the {\em guessing game} worked just as well as when joint attention was provided stochastically by the speaker beforehand. However, it was found later that this experiment was biased for naming the robots themselves, which had been implemented as an artifact in the feedback. In the current experimental setup the robots do not try to name themselves, and thus the implemented artifact is omitted. 

Since the implementation of pointing revealed a poor performance, the way joint attention was modeled has been changed to improve the performance. One strategy that has already been presented is the guessing game. Section \ref{s:feed:comb} presents the results of a set of experiments where different schemes of combinations of joint attention and feedback have been investigated. 

Section \ref{s:feed:oli} introduces other experiments where no directed feedback is used to control a language game. Feedback in these experiments is only based on the co-occurrence of word-meaning associations, not on their success.

\p
So, this chapter will stress the importance of non-linguistic information an agent might need to, at least, bootstrap a language. Section \ref{s:feed:summary} will finally summarize the results of the experiments presented in this chapter.


\section{Exploring non-linguistic information}\label{s:feed:comb}
\index{feedback|(}
\index{joint attention|(}

Six different relating schemes of combining joint attention and feedback are discussed in this section. The schemes are a combination of methods presented in chapter \ref{ch:cm}. The methods that have been combined are presented in table \ref{t:feed:comb}.

\begin{table}
\centering 
\begin{tabular}{||c|l|l||}
\hline\hline
Exp. & Joint Attention & Feedback\\\hline
xi & None & Correspondence\\\hline
ci & Cross-correlation & Correspondence\\\hline
cc & Cross-correlation & Cross-correlation\\\hline
ic & Correspondence & Cross-correlation\\\hline
xc & None & Cross-correlation\\\hline
ii & Correspondence & Correspondence\\\hline
\end{tabular}
\caption{The combinations of joint attention and feedback explored. The different methods have been explained in chapter \ref{ch:cm}. Note that {\bf xi} is the basic experiment.}
\label{t:feed:comb}
\end{table}

\index{guessing game}
\index{cross-correlation|(}
\index{language!game!ostensive|(}
\index{correspondence}
Experiment {\bf xi} is the guessing game that is presented as the basic experiment in chapter \ref{ch:basic}. Another guessing game, but that takes stochastic feedback provided by the cross-correlation method is experiment {\bf xc}. Experiments {\bf ci}, {\bf cc}, {\bf ic} and {\bf ii} are ostensive language games. Joint attention provides either absolute knowledge by the correspondence criterion or indicative topic knowledge by the cross-correlation method. When the correspondence criterion is used ({\bf ic} and {\bf ii}), both robots know exactly what the topic is. When the cross-correlation is used ({\bf ci} and {\bf cc}), the hearer considers all segments that have non-zero likelihood of being the topic (i.e. the cross-correlation of the feature values with the speaker's topic is unequal to zero). The likelihood of the topic is forwarded in the matrix from which the hearer selects the word-meaning association and consequently its topic. In the ostensive language games feedback is also either provided by the correspondence criterion or the cross-correlation.


\begin{table}
\centering
\begin{tabular}{||r|c|c|c||}
\hline\hline
	&	xi	&	       ci                      	&	       ii	\\\hline
CS      	&	0.351 $\pm$  0.010 	&	          0.595$\pm$      0.002	&	       0.671$\pm$   0.004	\\\hline
AS      	&	0.351 $\pm$  0.010 	&	          0.595$\pm$      0.002	&	       0.671$\pm$   0.004	\\\hline
DS0     	&	0.916 $\pm$ 0.004 	&	          0.915$\pm$      0.004	&	       0.935$\pm$   0.002	\\\hline
DS1     	&	0.920 $\pm$ 0.004 	&	          0.919$\pm$      0.003	&	       0.936$\pm$   0.002	\\\hline
D0      	&	0.956 $\pm$ 0.002 	&	          0.955$\pm$      0.002	&	       0.959$\pm$   0.000	\\\hline
D1      	&	0.955 $\pm$ 0.002 	&	          0.957$\pm$      0.000	&	       0.960$\pm$   0.000	\\\hline
P0      	&	0.852 $\pm$ 0.004 	&	          0.842$\pm$      0.005	&	       0.856$\pm$   0.001	\\\hline
P1      	&	0.851 $\pm$ 0.002 	&	          0.846$\pm$      0.001	&	       0.851$\pm$   0.002	\\\hline
S0      	&	0.822 $\pm$ 0.017 	&	          0.677$\pm$      0.035	&	       0.647$\pm$   0.046	\\\hline
S1      	&	0.817 $\pm$ 0.011 	&	          0.669$\pm$      0.031	&	       0.647$\pm$   0.045	\\\hline
C0      	&	0.816 $\pm$ 0.008 	&	          0.815$\pm$      0.026	&	       0.823$\pm$   0.037	\\\hline
C1      	&	0.811 $\pm$ 0.007 	&	          0.820$\pm$      0.023	&	       0.821$\pm$   0.026	\\\hline
\hline\hline\hline							
	&	       xc              	&	       cc              	&	       ic              	\\\hline
CS      	&	       0.368$\pm$   0.008	&	       0.597$\pm$   0.007	&	       0.634$\pm$ 0.000	\\\hline
AS      	&	       0.349$\pm$  0.009	&	       0.564$\pm$ 0.007		&	       0.627$\pm$ 0.001		\\\hline
DS0     	&	       0.920$\pm$   0.004	&	       0.912$\pm$   0.002	&	       0.930$\pm$   0.004	\\\hline
DS1     	&	       0.921$\pm$   0.003	&	       0.917$\pm$   0.002	&	       0.935$\pm$   0.002	\\\hline
D0      	&	       0.955$\pm$   0.002	&	       0.955$\pm$   0.002	&	       0.956$\pm$   0.002	\\\hline
D1      	&	       0.955$\pm$   0.001	&	       0.959$\pm$  0.000	&	       0.956$\pm$   0.002	\\\hline
P0      	&	       0.856$\pm$   0.002	&	       0.854$\pm$   0.001	&	       0.852$\pm$   0.002	\\\hline
P1      	&	       0.850$\pm$   0.002	&	       0.852$\pm$   0.002	&	       0.848$\pm$   0.003	\\\hline
S0      	&	       0.819$\pm$   0.009	&	       0.831$\pm$   0.024	&	       0.549$\pm$   0.061	\\\hline
S1      	&	       0.818$\pm$   0.008	&	       0.836$\pm$   0.023	&	       0.554$\pm$   0.061	\\\hline
C0      	&	       0.803$\pm$   0.017	&	       0.855$\pm$   0.005	&	       0.718$\pm$   0.043	\\\hline
C1      	&	       0.807$\pm$    0.01	&	       0.856$\pm$   0.006	&	       0.728$\pm$   0.038	\\\hline
\hline
\end{tabular}
\caption{The results of the experiments with different schemes of using prior topic knowledge and feedback.}
\label{t:feed:combination}
\end{table}

\begin{figure}
\subfigure[CS]{\psfig{figure=feedback//cs0.eps,width=5.6cm}}
\subfigure[DS]{\psfig{figure=feedback//ds0.eps,width=5.6cm}}\\
\subfigure[S]{\psfig{figure=feedback//s0.eps,width=5.6cm}}
\subfigure[D]{\psfig{figure=feedback//d0.eps,width=5.6cm}}\\
\subfigure[C]{\psfig{figure=feedback//c0.eps,width=5.6cm}}
\subfigure[P]{\psfig{figure=feedback//p0.eps,width=5.6cm}}
\end{figure}
\begin{figure}
\centering
\subfigure[AS]{\psfig{figure=feedback//as0.eps,width=5.6cm}}
\caption{An overview of the averaged results of varying the strategies for establishing joint attention and providing feedback.}
\label{f:feed:comb}
\end{figure}

In these experiments the actual success (see chapter \ref{ch:basic}) become important. The experiments where feedback is provided by a correspondence of sensory channels has an actual success that is equal to the communicative success because the feedback criterion is the same by which the AS is calculated. However, when the feedback is provided by the cross-correlation then the CS need not be the same as the AS.

Table \ref{t:feed:combination} and figure \ref{f:feed:comb} show the average results of 10 runs of 5,000 language games. This table and figure show that all combinations that do explore prior topic knowledge have higher communicative success with $p=0.0000$ (i.e. those experiments that do not have an `x' in front). So, the guessing game variants are less successful. 

The communicative success is highest when joint attention is established by the correspondence criterion ({\bf ii} and {\bf ic}). The differences between {\bf ii} and {\bf ic} become clear when the actual success is compared. Then experiment {\bf ii} appears to outperform {\bf ic} by 5 \% ($p=0.0000$). Next highest CS is observed when joint attention is established by cross-correlations ({\bf ci} and {\bf cc}). Again these results are similar until the AS is compared. The AS of {\bf ci} is about 3 \% higher ($p=0.0038$). So, the feedback provided by the correspondence criterion shows better performance than when it is provided by cross-correlations. Nevertheless, when comparing the AS of experiment {\bf ic} with experiment {\bf ci}, the first one is still more than 3 \% higher with a significance in difference of $p=0.0020$. Hence it seems that the quality of joint attention has more influence than the quality of the feedback. 

\p
When looking at the specificity, other interesting results are observed. Figure \ref{f:feed:comb} (c) clearly shows that when feedback is given by the correspondence criterion, the specificity decreases when the joint attention is focussed more narrow. So it appears that more referential polysemy enters the lexiconsynonymy. Although not shown clearly, it appears as if this also holds when feedback is provided by cross-correlations. A closer look at the different runs has been taken, and it appears that the specificity of the different runs when joint attention was established varies very much. This is also made clear with the standard deviations of these experiments. Whereas the standard deviation of the specificities of the guessing games is approximately 0.01, the standard deviations for the ostensive games varies between 0.023 and 0.061. In some runs a specificity has been observed of approximately 0.25. Since these anomalies were observed more often, such values could not be thrown away.

Although to a lesser extend, similar observations can be made for the consistency. The consistency of experiments {\bf ci}, {\bf ii} and {\bf ic} all have standard deviations between 0.023 and 0.043. Hence the distributions of the results is very much spread. Experiment {\bf ic} for example shows for one run a consistency of 0.40. Although for experiments {\bf ci}, {\bf ii} and {\bf cc} the consistency is higher than in the basic experiment ($p=0.6842$, $p=0.0186$ and $p=0.0020$ resp.), this is not the case for experiment {\bf ic} ($p=0.0354$). Although the $p$-values indicate little significance, it is difficult to use say that nothing is wrong. The distributions of these experiments are apparently completely different than for the basic experiment.

It seems as if joint attention helps to increase communicative success, but there is an increased level of referential polysemy and sometimes an increase in referential . The robots have less difficulty understanding each other when they both know what segment is the topic. However, this has the side-effect that there is little pressure to disambiguate the system. This will become clearer in the next section and in chapter \ref{ch:opt}.

\p
The differences observed in the measures monitoring the conceptualization processes are not very large nor significant. So, there is no further need in discussing the DS, D and P. Clearly the influence of joint attention and feedback is working on the lexicon formation.
\index{language!game!ostensive|)}

\section{No Directive Feedback}\label{s:feed:oli}
\index{language!game!observational|(}

Since the no negative feedback evidence states that there appears to be no feedback on the outcome of a language game, it is interesting to investigate what happens when there is no such {\em directive} feedback. The experiments reported in this section are inspired by the experiments done by Mike Oliphant \cite{oliphant:1997,oliphant:1998,oliphant:2000}, although they are necessarily not the same. A big first difference is inherent on the use of robots. \index{Oliphant, Mike}

Oliphant's agents have access to both the signal and the meaning of it during a language game. The learning mechanism tries to converge an ideal communication system based on the word-meaning associations. This is also done in our experiments. However, the robots have, in principle, no access to the meaning of a signal other than to its referent. Another difference is in the learning algorithm (or the update of the scores). Where Oliphant uses a.o. Hebbian- and Bayesian learning, a different update rule is used here.

The experiments reported in this section builds up an observational language game in the sense of Oliphant's communication strategy. It will be shown how a robot can develop a lexicon based on an interaction scheme and learning mechanism that does not make use of directive feedback. The feedback that is available is extracted from the language game by each robot itself. It is based on the use of a word-meaning pair. The association score of a {\em used} WM association $\sigma_{\mbox{WM}}$ is updated when it is used in the communication, whether or not the language game is actually successful:

\begin{eqnarray}
\sigma_{\mbox{WM}} = \eta \cdot \sigma_{\mbox{WM}}' + (1-\eta)
\label{e:feed:sigma}
\end{eqnarray}

\n
The equation is the same as used before, it is only updated under different circumstances.

Table \ref{t:feed:comb1} shows the setup of the 4 experiments reported. Experiment {\bf xx} uses no directive feedback, nor joint attention. Experiment {\bf ix$_0$} uses the correspondence criterion to establish joint attention. In both experiments no lateral inhibition is applied on the update of the association scores. This is applied in experiment {\bf ix$_1$}. Incorporating lateral inhibition makes equation \ref{e:feed:sigma} as follows:\index{lateral inhibition}

\begin{eqnarray}
\sigma_{\mbox{WM}} = \eta \cdot \sigma_{\mbox{WM}}' + (1-\eta) \cdot S
\label{e:feed:sigma1}
\end{eqnarray}

\n
where

\begin{eqnarray}
S = \left\{ \begin{array}{rl}
1 & \mbox{if WM}' = \mbox{WM}\\
0 & \mbox{if } (\mbox{WM}' \neq \mbox{WM}) \wedge ((F'=F) \vee (\Gamma' = \Gamma))
\end{array}\right.
\end{eqnarray}

\n
Recall that a WM association has been defined as $\mbox{WM}=(F,\Gamma,\sigma)$ where $F$ is a word-form and $\Gamma$ a meaning. Naturally $\mbox{WM}$ is the association that is used and $\mbox{WM}'$ is any other association of the agent. $\sigma$ is updated whether or not the game is successful.

The last column of table \ref{t:feed:comb1} gives the value of the word-form creation probability $P_s$. In the basic experiment and in all other previous experiment this parameter was set to $P_s=0.02$. In experiment {\bf ix$_2$} $P_s=0.40$, the rest of the settings of this experiment is as {\bf ix$_1$}.

Although the robots do not evaluate the success of the language games, the communicative success can still be calculated. The robots could evaluate themselves to be successful when both robots could identify a matching WM association whether or not they referred to the same referent. From these cases the CS is calculated. The actual success calculates whether the robots are actually successful in identifying the same referent.

\begin{table}
\centering
\begin{tabular}{||l|c|c|c||}
\hline\hline
Exp. & J.A. & L.I. & $P_s$\\\hline
xx & none & no & 0.02\\\hline
ix$_0$ & corr. & no & 0.02\\\hline
ix$_1$ & corr. & yes & 0.02\\\hline
ix$_2$ & corr. & yes & 0.40\\\hline
\hline
\end{tabular}
\caption{An overview of the experiments investigating no feedback.}
\label{t:feed:comb1}
\end{table}

\begin{table}
\centering
\begin{tabular}{||r|c|c|c||}
\hline\hline
&	{\bf xi}	&	{\bf xx}	&	{\bf ix$_0$}\\\hline
CS&	0.351$\pm$	0.010&	0.818$\pm$	0.006&	0.830$\pm$	0.001\\\hline
AS&	0.351$\pm$	0.010&	0.241$\pm$	0.008&	0.651$\pm$	0.002\\\hline
DS0&	0.916$\pm$	0.004&	0.912$\pm$	0.004&	0.937$\pm$	0.003\\\hline
DS1&	0.920$\pm$	0.004&	0.915$\pm$	0.005&	0.936$\pm$	0.003\\\hline
D0&	0.956$\pm$	0.002&	0.959$\pm$	0.000&	0.960$\pm$	0.001\\\hline
D1&	0.955$\pm$	0.002&	0.958$\pm$	0.000&	0.961$\pm$	0.000\\\hline
P0&	0.852$\pm$	0.004&	0.866$\pm$	0.002&	0.830$\pm$	0.001\\\hline
P1&	0.851$\pm$	0.002&	0.860$\pm$	0.003&	0.827$\pm$	0.000\\\hline
S0&	0.822$\pm$	0.017&	0.808$\pm$	0.031&	0.617$\pm$	0.094\\\hline
S1&	0.817$\pm$	0.011&	0.810$\pm$	0.031&	0.622$\pm$	0.095\\\hline
C0&	0.816$\pm$	0.008&	0.814$\pm$	0.005&	0.709$\pm$	0.114\\\hline
C1&	0.811$\pm$	0.007&	0.812$\pm$	0.005&	0.707$\pm$	0.112\\\hline
\hline\hline	\hline					
&	{\bf ix$_1$}	&	{\bf ix$_2$}&	\\\hline		
CS&	0.847$\pm$	0.003&	0.827$\pm$	0.005&	\\\hline
AS&	0.667$\pm$	0.003&	0.657$\pm$	0.005&	\\\hline
DS0&	0.937$\pm$	0.001&	0.960$\pm$	0.001&	\\\hline
DS1&	0.941$\pm$	0.004&	0.963$\pm$	0.001&	\\\hline
D0&	0.960$\pm$	0.000&	0.960$\pm$	0.000&	\\\hline
D1&	0.960$\pm$	0.001&	0.961$\pm$	0.000&	\\\hline
P0&	0.858$\pm$	0.002&	0.853$\pm$	0.001&	\\\hline
P1&	0.851$\pm$	0.000&	0.848$\pm$	0.001&	\\\hline
S0&	0.684$\pm$	0.144&	0.927$\pm$	0.002&	\\\hline
S1&	0.688$\pm$	0.132&	0.927$\pm$	0.002&	\\\hline
C0&	0.772$\pm$	0.117&	0.838$\pm$	0.003&	\\\hline
C1&	0.787$\pm$	0.099&	0.839$\pm$	0.003&	\\\hline
\hline
\end{tabular}
\caption{An overview of the averaged results of the experiments without directive feedback.}
\label{t:feed:oli}
\end{table}

\begin{figure}
\centering
\subfigure[CS]{\psfig{figure=feedback//cs1.eps,width=5.6cm}}
\subfigure[DS]{\psfig{figure=feedback//ds1.eps,width=5.6cm}}\\
\subfigure[S]{\psfig{figure=feedback//s1.eps,width=5.6cm}}
\subfigure[D]{\psfig{figure=feedback//d1.eps,width=5.6cm}}
\subfigure[C]{\psfig{figure=feedback//c1.eps,width=5.6cm}}
\subfigure[P]{\psfig{figure=feedback//p1.eps,width=5.6cm}}\\
\end{figure}
\begin{figure}[t]
\centering
\subfigure[AS]{\psfig{figure=feedback//as1.eps,width=5.6cm}}
\caption{The results of the experiments without directive feedback compared with the basic experiment ({\bf ix}).}
\label{f:feed:oli}
\end{figure}

Table \ref{t:feed:oli} and figure \ref{f:feed:oli} show the results of the four experiments compared with the basic experiment. When looking at the CS it seems as if all experiments work evenly well. All experiments yield a communicative success above 80 \%. More than doubling the success of the basic experiment.

However, when looking at the AS (figure \ref{f:feed:oli} (g)) only the {\bf ix} experiments yield a high success of approximately 65 \%. When no joint attention is established ({\bf xx}) the robots are communicating very unsuccessful; even below the a priori success of 29 \%. All differences of the AS compared to the basic experiment have $p$-values of $p=0.0000$. So, joint attention is very necessary when no directed feedback is available. This is confirmed by the findings of \citeN{oliphant:1997}, who calls this phenomenon {\em observation}.\index{Oliphant, Mike}

Although all {\bf ix} experiments have similar results in the AS, there are still main differences in these three experiments. These differences become overt in the specificity and consistency. Experiments {\bf ix$_0$} and {\bf ix$_1$} have much lower values than experiment {\bf ix$_2$}. These difference have $p=0.0000$ for the specificity and {\bf ix$_0$}'s consistency; the difference in consistency between {\bf ix$_1$} and {\bf ix$_2$} has $p=0.1432$.  Note that the standard deviations of experiments {\bf ix$_0$} and {\bf ix$_1$} is about 0.1 for both the specificity as the consistency. The results of the experiments vary a lot from run to run. Comparing {\bf ix$_0$} with {\bf ix$_1$} the second experiment yield higher values with a significance in difference of $p=0.0186$ for the consistency and $p=0.0892$ for the specificity. Nevertheless, it seems that lateral inhibition\index{lateral inhibition} does have a positive effect on the language formation. This is made clear most obvious when comparing the RF competition diagrams of the two experiments (figure \ref{f:feed:comp}).

\begin{figure}
\centering
\subfigure[{\bf ix$_0$}]{\psfig{figure=feedback//rf0-0oli1.eps,width=5.6cm}}
\subfigure[{\bf ix$_1$}]{\psfig{figure=feedback//rf0-0oli.eps,width=5.6cm}}
\caption{The referent-form competition diagrams of robot r0 for referent L0.}
\label{f:feed:comp}
\end{figure}

\begin{figure}
\centering
\subfigure[{\bf ix$_1$}]{\psfig{figure=feedback//fr0-0.eps,width=5.6cm}}
\subfigure[{\bf ix$_2$}]{\psfig{figure=feedback//fr0-4olib.eps,width=5.6cm}}
\caption{The form-referent competitions of experiments {\bf ix$_1$} and {\bf ix$_2$}.}
\label{f:feed:comp1}
\end{figure}

It is clear from this figure that in experiment {\bf ix$_0$} more referential synonymy exists than in {\bf ix$_1$}. Lateral inhibition causes the associations to be disambiguated. More clearcut differences between lateral connections of a WM pair arise and competition for the effective pair is more difficult to win. As will become clear in the next chapter, the creation probability of $P_s-0.02$ is too low. When it is higher as in the experiment {\bf ix$_2$}, formed associations become more specific, which is best illustrated in the specificity of 0.93, but can also be illustrated with figure \ref{f:feed:comp1} where the FR competition is shown for experiments {\bf ix$_1$} and {\bf ix$_2$}. When the creation probability is too low, there is much referential polysemy, which is disambiguated by allowing more WM associations in the last experiment.

The other measures of the experiments do not reveal interesting values in that they are more or less the same for all experiments. 

\p
It is clear that the robots can ground a reasonable communication system with a reasonable success without the availability of directive feedback. This, however only works when joint attention is established, lateral inhibition is incorporated and the word-form creation rate is sufficiently high.

It is interesting to note that the high average AS is likely to be mainly caused by the joint attention mechanism and the easiness of establishing communicative success. Recall that the potential understandability is approximately 79 \%. When multiplying this value with the average CS of approximately 83 \% for the three {\bf ix} experiments, this yields 65.6 \% which corresponds nicely to the average actual successes. Note again that like in the ostensive language game, joint attention makes the linguistic communication redundant. Hence the pressure on disambiguating the lexicon is low. \index{disambiguation}
\index{language!game!observational|)}
\index{feedback|)}
\index{joint attention|)}


\section{Summary}\label{s:feed:summary}

In this chapter experiments were presented in which the role of joint attention and feedback has been investigated. Two new language games have been introduced: the ostensive language game and the observational language game. As the results show, the impact of different schemes is huge. The different methods of establishing joint attention and providing feedback have been introduced in chapter \ref{ch:cm}. Figure \ref{f:feed:results} summarizes the results of the experiments compared to the basic experiment.

\begin{figure}
\centering
\subfigure[CS]{\psfig{figure=feedback//cs.eps,width=5.6cm}}
\subfigure[DS]{\psfig{figure=feedback//ds.eps,width=5.6cm}}\\
\subfigure[S]{\psfig{figure=feedback//spec.eps,width=5.6cm}}
\subfigure[D]{\psfig{figure=feedback//dist.eps,width=5.6cm}}\\
\subfigure[C]{\psfig{figure=feedback//cons.eps,width=5.6cm}}
\subfigure[P]{\psfig{figure=feedback//pars.eps,width=5.6cm}}
\end{figure}
\begin{figure}[t]
\centering
\subfigure[AS]{\psfig{figure=feedback//as.eps,width=5.6cm}}\\
\caption{An overview of the results presented in this chapter compared with the basic experiment {\bf xi}.}
\label{f:feed:results}
\end{figure}

The most important differences are found in the measures concerning communication, the CS, AS, S and to a lesser extend C. Apparently, joint attention and feedback have little influence on the conceptualization, which should not be a surprise. 

According to the CS, the three observational language games ({\bf ix}) are the best experiments, followed by the ostensive games {\bf ii}, {\bf ic}, {\bf ci} and {\bf cc}. The ostensive language games are about 25 to 30 \% better than the basic experiment. The observational language games are about 45 \% better. 

When looking at the AS (figure \ref{f:feed:results}) the best experiment is the ostensive game {\bf ii}, followed by the three observational games and then by the three other ostensive games. The AS of the ostensive language games are slightly below their CS, whereas it is about 15 \% lower for the observational language games. Apparently, directive feedback puts more restrictions on the CS than when it is not applied, while the actual success remains approximately the same. The actual success is completely conform the multiplication of the CS with the potential understandability. Hence the availability of joint attention is very much involved in the success of the language games. 

The next striking results are found in the specificity of the different experiments. The experiments {\bf xc}, {\bf cc} and {\bf xx} show similar results as the basic experiment. All other experiments have lower specificity, except experiment {\bf ix$_2$}, which is approximately 0.10 higher. As will become clear in the next chapter, this latter difference has much to do with the higher creation probability. In the experiments that have lower specificity joint attention is applied. Furthermore, each of these experiments have the correspondence method for either joint attention or feedback. So, it seems likely that the correspondence criterion combined with joint attention has this negative influence, although experiment {\bf ix$_2$} shows that this need not be so.

The differences in consistency reveal lower values for experiments {\bf ic}, {\bf ix$_0$} and {\bf ix$_1$}. So, it appears that establishing joint attention reliably and the availability of unreliable feedback has this negative influence. However, the differences have revealed low significance, so it is difficult to make hard claims about this.

\p
All with all, this chapter revealed that the observational language game is the most successful scenario observed so far. Provided the game incorporates lateral inhibition, which is also active in both the guessing game and the ostensive game, and the creation probability is sufficiently high. This latter result might be a little bit biased, since the creation probability is not varied in the other experiments up to now.

Up to now various interesting variants of the basic experiment have been introduced. Besides different categorization experiments, different interaction schemes have been compared. The next chapter investigates different learning mechanisms and parameters for the lexicon formation, amongst which the creation probability.
