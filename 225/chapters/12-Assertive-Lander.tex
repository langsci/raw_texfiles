\documentclass[output=paper]{langsci/langscibook} 
\ChapterDOI{10.5281/zenodo.3402076}

% Chapter 12

\title{The Mehweb ``assertive'' copula {gʷa}: a sketch of a portrait}

\author{Yury Lander\affiliation{National Research University Higher School of Economics}}

\abstract{Mehweb Dargwa features a particle \emph{gʷa}, a peculiar element   which is basically used for emphasizing the assertion. The paper  explores some grammatical characteristics of this particle. It is   shown that, in both verbal and non-verbal clauses, \emph{gʷa} serves as a  predicative marker forming a complete predication and is an equivalent of a copula (even though, unlike the neutral copula in  Mehweb, it lacks inflection). Similarly to typical East Caucasian   predicative markers, \emph{gʷa} may occur in different positions, though  its place is syntactically constrained (e.g., it cannot be embedded   within syntactic islands). Still, Mehweb speakers allow \emph{gʷa} not to  be adjoined to either the predicate or the focus. This makes the   distribution of the particle surprising as compared with similar  predicative markers in well-described East Caucasian languages,   where they may either occur on the predicate or immediately follow   the focused element.}

\begin{document}
\maketitle

% 1.
\section{Introduction}

This paper presents a preliminary description of the particle \emph{gʷa}
in Mehweb, a language of the Dargwa branch of the East Caucasian family.
The following examples illustrate the use of this marker in a verbal
clause (\ref{ex:12:1}) and in an equative clause (\ref{ex:12:2}):

\ea \label{ex:12:1} % (1)
\gll ʔudidi-li ħark'ʷ-li ar-χ-uwe \textbf{gʷa}!\\
under.\textsc{el}-\textsc{atr} river-\textsc{erg} away-bring:\textsc{ipfv}-\textsc{cvb.ipfv} \textsc{asrt}\\
\glt `The river carries away the one who is downstream!' (Molla Rasbaddin
goes to the market place: 1.11)

\pagebreak[3]

\ex \label{ex:12:2} % (2)
\gll hel čudu \textbf{gʷa} di-la.\\
this chudu \textsc{asrt} I.\textsc{obl}-\textsc{gen}\\
\glt `This chudu (a kind of pie) is mine.'
\z

The function of \emph{gʷa} is not at all obvious. Etymologically, this
particle is likely to originate from the imperative of the verb `see'
(which, as an imperative, is not fully felicitous – see \citealt{dobrushina2019}
[this volume]). \citet[128]{magometov1982} translated \emph{gʷa} by the Russian
particles \emph{ved'} and \emph{že}, whose semantics are by no means
clear. The speakers often suggest that \emph{gʷa} is frequent in
disputes and emphasizes a claim. Given this, I will label it an
\emph{assertive} marker. Further research is needed for an exhaustive
description of the rules that govern its use. What I will argue are the
following two specific points:
\begin{enumerate}[topsep=\medskipamount,itemsep=0pt,partopsep=0pt,parsep=0pt,label={(\roman*})]
\item % (i)
  \emph{gʷa} is a copula,

\item % (ii)
  the position of \emph{gʷa} does not necessarily depend on the
position of the predicate or of the focus.
\end{enumerate}

The latter makes \emph{gʷa} look quite peculiar against the background
of what we know about copulas in many East Caucasian languages and in
Dargwa languages in particular.

The issue of copula-ness is addressed in \sectref{the-assertive-marker-as-a-copula}. In \sectref{verbal-predications}, I
discuss the use of the marker in verbal predications and describe
syntactic restrictions on its position. \sectref{non-verbal-predication} describes the use of
\emph{gʷa} in non-verbal predications. The last section presents
conclusions.

% 2.
\section{The assertive marker as a copula}\label{the-assertive-marker-as-a-copula}

Many East Caucasian languages have elements which are described as
copulas or \isi{predicative marker}s, i.e.\ as markers which are normally added
to some lexical material in order to form complete predications (finite,
unless these copulas themselves take a subordinate form).\footnote{Some
  important studies addressing the behaviour of predicative markers in
  East Caucasian (especially with respect to their interaction with
  focus) include
  % Harris
  \citet{harris2000,harris2002} on \ili{Udi}, \citet{kazenin2002} on \ili{Lak},
  \citet{sumbatova2011} and \citet{sumbatova-lander2014} on \ili{Tanti Dargwa}.
  \citet{forker2013} discusses question particles which typically represent a
  kind of predicative markers in these languages. \citet{testelets1998},
  \citet{kalinina-sumbatova2007} and \citet{forker-belyaev2016} describe
  the influence of the position of some predicative markers on the
  overall clause structure.} Although their individual morphological and
syntactic properties vary, these elements are clearly distinguishable
from verbs. There are typically several predicative markers in a single
language: for example, many languages have dedicated predicative markers
used in questions in addition to those used in simple declaratives.

Predicative markers appear both in verbal and non-verbal predications.
Below I will illustrate their use with a few examples from Udi, a
language belonging to the Lezgic branch of the East Caucasian family,
thus only distantly related to Mehweb.\footnote{Here I omit some
  important details of the Udi system, including the existence of a
  series of dative clitics and a more verb-like copula-like element used
  in existential\is{existential predication}, \isi{possessive predication}, and identificational\is{identificational predication} clauses, which also
  takes a predicative marker.}

Predicative\is{predicative marker} markers in \ili{Udi} are highly grammaticalized and now commonly
described as clitics (\citealt{harris2000}, \citeyear{harris2002}). They include personal markers
which usually show agreement with the subject (S or A) and the question
marker, which only appears in interrogative contexts and is not
discussed here (but see \citealt{harris1992}). The following examples illustrate
the use of the 1\textsuperscript{st} person plural personal marker
\emph{꞊jan} in a non-verbal predication (\ref{ex:12:3}) and in verbal predications
(\ref{ex:12:4}–\ref{ex:12:5}):\footnote{The Udi examples are from the corpus of text in the
  Nizh dialect of Udi collected by Dmitry Ganenkov, Timur Maisak and the
  author.}

\ea \label{ex:12:3} % (3)
\gll jan꞊al tːe χalg-aun mand-i χalg꞊\textbf{jan}.\\
we꞊\textsc{add} that nation-\textsc{abl} remain-\textsc{aor}(\textsc{ptcp}) nation꞊1\textsc{pl}\\
\glt `We are the nation that continues (lit.\ remains from) that nation.'

\ex \label{ex:12:4} % (4)
\gll me äš-urχo lap mat mand-e꞊\textbf{jan}.\\
this affair-\textsc{pl}(\textsc{dat}) very surprised remain-\textsc{prf}꞊\textsc{1pl}\\
\glt `We really remained surprised at these facts.'

\ex \label{ex:12:5} % (5)
\gll pajiz-e dirij-a꞊\textbf{jan} kašˤ-e.\\
autumn-\textsc{dat} vegetable.garden-\textsc{dat}꞊1\textsc{pl} dig-\textsc{lv}:\textsc{prs}\\
\glt `In autumn, we dig in the vegetable garden.'
\z

Note that predicative markers attach not only to the lexical predicate (\ref{ex:12:4})
but also to the focused\is{focus} element (\ref{ex:12:5}). This can be viewed as a kind of
competition for acquiring head properties between the semantic head (the
predicate) and the most relevant element of the clause (i.e.\
focus).\footnote{See \citet{lander2009} for some discussion of competition
  between semantically obligatory elements and the most relevant
  elements for the head properties.}

In Dargwa languages, predicative markers are less grammaticalized than
in Udi. In particular, they show some properties of autonomous words.
Many such markers readily constitute autonomous expressions (such as
`yes' or `no'). Some of them take attributive and adverbial morphology
and hence are akin to content words.

The primary Mehweb predicative marker is the \isi{copula} \emph{le-}\textsc{cl} (for
morphology, see \citealt{daniel2019} [this volume]), with a gender agreement slot 
controlled by the absolutive argument. Its use in non-verbal
predications is shown in (\ref{ex:12:6}–\ref{ex:12:7}), while its use in verbal predications
is illustrated in (\ref{ex:12:8}–\ref{ex:12:9}).

\ea \label{ex:12:6} % (6)
\gll ʁača ħa-la aħin, di-la \textbf{le-b}.\\
calf you.sg.\textsc{obl}-\textsc{gen} {be}:\textsc{neg} I.\textsc{obl}-\textsc{gen} {be}-\textsc{n}\\
\glt `The calf is not yours, (it) is mine.' (A blind judge: 1.11)

\ex \label{ex:12:7} % (7)
\gll arci-ze-b \textbf{le-b-re} ħa-la daˤʜ-la surat.\\
money-\textsc{inter}-\textsc{n}(\textsc{ess}) {be}-\textsc{n}-\textsc{pst} you.sg.\textsc{obl}-\textsc{gen} face-\textsc{gen} picture\\
\glt `On the coin (lit., money), there was a picture of your face.' (The
Story of Akula Ali, 1.21)

\ex \label{ex:12:8} % (8)
\gll xunuj-s ruzi ħa-d-ig-es d-aʔ-i-le \textbf{le-r}.\\
wife.\textsc{obl}-\textsc{dat} sister \textsc{neg}-\textsc{f1}-love:\textsc{ipfv}-\textsc{inf} \textsc{f1}-start:\textsc{pfv}-\textsc{aor}-\textsc{cvb} \rlap{\textsc{aux}-\textsc{f}}\\
\glt `The wife disliked (her husband's) sister.' (A brother and sister: 1.6)

\ex \label{ex:12:9} % (9)
\gll wallahi, k'as \textbf{le-b} q'-oˤwe\\
Allah big.fish \textsc{aux}-\textsc{n} go:\textsc{ipfv}-\textsc{cvb.ipfv}\\
\glt `My God, a whale is going (here).' (Two sons: 1.65)
\z


Like in \ili{Udi}, the Mehweb \isi{predicative marker} in verbal clauses can follow
either the verb or the focused\is{focus} constituent. However, unlike in Udi, the
Mehweb \isi{copula} requires that a verb \emph{be} in a non-finite
(participial or converbal) form, while finite verb forms do not combine
with the predicative marker. In fact, combinations of a copula and a
lexical verb look like periphrastic forms, although the issue of
monoclausality of these constructions is tricky.\footnote{See \citet{sumbatova-lander2014}
  for a detailed discussion of this issue in \ili{Tanti Dargwa}, another Dargwa variety.}

Turning to the assertive marker \emph{gʷa}, it can be shown that it has
the distribution of a copula. There are two pieces of evidence for this.
First, similarly to \emph{le}-\textsc{cl}, the assertive marker cannot appear in
clauses that contain finite verb forms (\ref{ex:12:10}):

\ea \label{ex:12:10} % (10)
\ea % a.
\gll doˤʜi ar-b-ik-ib \(*\textbf{gʷa}\).\\
snow \textsc{pv}-\textsc{n}-fall:\textsc{pfv}-\textsc{aor} \textsc{asrt}\\
\glt `The snow fell.'

\ex % b.
\gll mator b-uz-an \(*\textbf{gʷa}\).\\
engine \textsc{n}-work:\textsc{ipfv}-\textsc{hab} \textsc{asrt}\\
\glt `The engine works.'
\z
\z

Second, the assertive marker cannot combine with a copula (\ref{ex:12:11}a–b),
unless the latter does not appear in a non-finite form, as in (\ref{ex:12:11}c). If
we assume that \emph{gʷa} is a copula, this is explained: a clause
cannot contain two copulas.

\ea \label{ex:12:11} % (11)
\ea % a.
\gll dag it derbenti-ze-la w-ak'-i-le \textbf{le-w} \(*\textbf{gʷa}\).\\
yesterday that Derbent-\textsc{inter}-\textsc{el} m-come:\textsc{pfv}-\textsc{aor}-\textsc{cvb} \textsc{aux}-\textsc{m} \textsc{asrt}\\

\ex % b.
\gll dag it derbenti-ze-la w-ak'-i-le \textbf{gʷa} \(*\textbf{le-w}\).\\
yesterday that Derbent-\textsc{inter}-\textsc{el} \textsc{m}-come:\textsc{pfv}-\textsc{aor}-\textsc{cvb} \textsc{asrt} \textsc{aux}-\textsc{m}\\

\ex % c.
\gll dag it derbenti-ze-la w-ak'-i-le \textbf{le-w-le} \textbf{gʷa}.\\
yesterday that Derbent-\textsc{inter}-\textsc{el} \textsc{m}-come:\textsc{pfv}-\textsc{aor}-\textsc{cvb} \textsc{aux}-\textsc{m}-\textsc{cvb} \textsc{asrt}\\
\glt `Yesterday he came from Derbent.'
\z
\z

It is worth mentioning, however, that \emph{gʷa} differs from
\emph{le}-\textsc{cl} in that it does not take any morphology.

% 3.
\section{Verbal predications}\label{verbal-predications}

Just like the \isi{copula} \emph{le}-\textsc{cl}, the assertive marker need not follow
the verb but can appear after focused\is{focus} elements:

\ea \label{ex:12:12} % (12)
\ea % a.
\gll nuša-jni \textbf{gʷa} kulubi-s remont b-aq'-i-le\\
we-\textsc{erg} \textsc{asrt} club-\textsc{dat} renovation \textsc{n}-do:\textsc{pfv}-\textsc{aor}-\textsc{cvb}\\
\glt `It was us who made the renovation for the club.'

\ex % b.
\gll nuša-jni kulubi-s \textbf{gʷa} remont b-aq'-i-le\\
we-\textsc{erg} club-\textsc{dat} \textsc{asrt} renovation \textsc{n}-do:\textsc{pfv}-\textsc{aor}-\textsc{cvb}\\
\glt `It was the club for which we made the renovation.'
\z
\z

I will distinguish between the wide scope use of \emph{gʷa}, where it
has a scope over the whole sentence or over the predicate and follows
this predicate, and the narrow scope use of \emph{gʷa}, where it should
follow exactly the focused\is{focus} phrase. In verbal clauses, the wide scope
\emph{gʷa} is found with the {neutral converb}\is{converb, neutral} (\ref{ex:12:13}) and with the
\isi{infinitive} (\ref{ex:12:14}–\ref{ex:12:15}) but not with the \isi{participle} (cf.\ the
infelicitous (\ref{ex:12:16}) with (\ref{ex:12:19}) below):\footnote{Presumably, the assertive marker should
  combine with the participle where it functions as the head of the
  nominal predicate in a nominal clause. However, I lack relevant
  examples.}

\ea \label{ex:12:13} % (13)
\gll qʷe b-iq'-uwe \textbf{gʷa}, ħu ħa-k-i-le ħa-wʔ-iša.\\
vow \textsc{n}-do:\textsc{ipfv}-\textsc{cvb.ipfv}:\textsc{ipfv} \textsc{asrt} you.sg \textsc{neg}-bring:\textsc{pfv}-\textsc{aor}-\textsc{cvb} \textsc{neg}-\textsc{m}.be-\textsc{fut}.\textsc{ego}\\
\glt `I swear I will take you (as a wife).' (Widow)

\ex \label{ex:12:14} % (14)
\gll durʡa uh-ub-i-li derqʷ uh-ub-i-s ca dus-li quli-w w-at-ul-le uz-es \textbf{gʷa}.\\ 
lose \textsc{m}.\textsc{lv}:\textsc{pfv}-\textsc{aor}-\textsc{atr}-\textsc{erg} winning \textsc{m}.become:\textsc{pfv}-\textsc{aor}-\textsc{atr}-\textsc{dat}
one year-\textsc{erg} house.\textsc{ess}-\textsc{m}(\textsc{ess}) \textsc{m}-put:\textsc{ipfv}-\textsc{ptcp}-\textsc{advz} \textsc{m}.work:\textsc{ipfv}-\textsc{inf} \textsc{asrt}\\
\glt `The one who will lose will work as a servant for the one who will win,
for one year.' (Widow)

\ex \label{ex:12:15} % (15)
\gll ħad hete ħunt'a-l qul-le-šu uˤq'-es \textbf{gʷa}.\\
you.sg.\textsc{dat} there(\textsc{lat}) red-\textsc{atr} house-\textsc{pl}-\textsc{ad}(\textsc{lat}) \textsc{m}.go:\textsc{pfv}-\textsc{inf} \textsc{asrt}\\
\glt `You should go there, to the red houses.'

\ex \label{ex:12:16} % (16)
\gll *musa-ni poˤroˤm b-oˤrʡ-aq-ib-i \textbf{gʷa}.\\
Musa-\textsc{erg} glass \textsc{n}-break:\textsc{pfv}-\textsc{caus}-\textsc{aor}-\textsc{atr} \textsc{asrt}\\
\glt (`Musa broke the glass.')
\z


If the assertive marker follows a constituent other than the predicate,
the choice of the verb form is less restricted. In this construction not
only a converbal\is{converb, neutral} form (\ref{ex:12:17}) and an infinitive (\ref{ex:12:18}) but also a participial\is{participle}
form (\ref{ex:12:19}) is allowed:\footnote{These combinations stand in parallel with
  similar combinations of converbs, infinitives and participles with the
  standard \isi{copula} (cf.\ \citealt{daniel2019} [this volume]).}

\ea \label{ex:12:17} % (17)
\gll maħmud-ini \textbf{gʷa} b-ilt'-uwe heš surat.\\
Mahmud-\textsc{erg} \textsc{asrt} \textsc{n}-take.out:\textsc{ipfv}-\textsc{cvb.ipfv} that picture\\
\glt `It was Mahmud who is drawing that picture.'

\ex \label{ex:12:18} % (18)
\gll rasuj-ni \textbf{gʷa} nu k-es.\\
Rasul.\textsc{obl}-\textsc{erg} \textsc{asrt} I bring:\textsc{pfv}-\textsc{inf}\\
\glt `It is Rasul who will bring me here.'

\ex \label{ex:12:19} % (19)
\gll musa-ni \textbf{gʷa} poˤroˤm b-oˤrʡ-aq-ib-i.\\
Musa-\textsc{erg} \textsc{asrt} glass \textsc{n}-break:\textsc{pfv}-\textsc{caus}-\textsc{aor}-\textsc{atr}\\
\glt `It was Musa who broke the glass.'
\z

In examples (\ref{ex:12:17}–\ref{ex:12:19}) we observe the assertive copula following focused\is{focus}
NPs. (\ref{ex:12:20}–\ref{ex:12:22}) demonstrate that \emph{gʷa} can follow other kinds of
constituents, such as adverbs and embedded clauses:

\ea \label{ex:12:20} % (20)
\gll išbari \textbf{gʷa} nuni praznik b-aq'-ib-i \textup/ b-aq'-i-le.\\
today \textsc{asrt} I.\textsc{erg} feast \textsc{n}-do:\textsc{pfv}-\textsc{aor}-\textsc{atr} / \textsc{n}-do:\textsc{pfv}-\textsc{aor}-\textsc{cvb}\\
\glt `It was today that I organized the feast.'

\ex \label{ex:12:21} % (21)
\gll it q'aˤju \textbf{gʷa} w-aš-uwe.\\
that slowly \textsc{asrt} \textsc{m}-go:\textsc{ipfv}-\textsc{cvb.ipfv}\\
\glt `He is moving SLOWLY.'

\ex \label{ex:12:22} % (22)
\gll musa rasuj-šu quli w-ak'-ib-i-jaʁe \textbf{gʷa} χamis g-ub-le.\\
Musa Rasul.\textsc{obl}-\textsc{ad}(\textsc{lat}) house.\textsc{ess}(\textsc{lat}) \textsc{m}-come:\textsc{pfv}-\textsc{aor}-\textsc{atr}-\textsc{ante} \textsc{asrt} Khamis see:\textsc{pfv}-\textsc{aor}-\textsc{cvb}\\
\glt `After MUSA'S COMING TO RASUL, he saw Khamis.'
\z


Still, we do find restrictions on what can be focused\is{focus} by means of
\emph{gʷa}.\footnote{I hypothesize that these restrictions hold for the
  neutral \isi{copula} as well, but I lack the necessary data.} For example,
the assertive marker cannot immediately follow postpositional objects, but
rather occurs after the whole postpositional phrase:

\ea \label{ex:12:23} % (23)
\ea % a.
\gll *heč' dubur-li-če \textbf{gʷa} aqu-r dirigʷ хaʔ d-uh-ub-le.\\
that mountain-\textsc{obl}-\textsc{super}(\textsc{lat}) \textsc{asrt} up-\textsc{npl}(\textsc{ess})
cloud appear \textsc{npl}-become:\textsc{pfv}-\textsc{aor}-\textsc{cvb}\\

\ex % b.
\gll heč' dubur-li-če aqu-r \textbf{gʷa} dirigʷ хaʔ d-uh-ub-le.\\
that mountain-\textsc{obl}-\textsc{super} upper-\textsc{npl}(\textsc{ess}) \textsc{asrt} cloud appear \textsc{npl}-become:\textsc{pfv}-\textsc{aor}-\textsc{cvb}\\
\glt `It is over that mountain that the cloud appeared.'
\z
\z

Further, the assertive marker cannot be embedded in an NP. In
particular, it cannot occur immediately after an adjective attribute (\ref{ex:12:24}),
an attributive demonstrative (\ref{ex:12:25}) or a quantifier (\ref{ex:12:26}) when they
precede the head noun:

\ea \label{ex:12:24} % (24)
\ea % a.
\gll *ħunt'a-l \textbf{gʷa} burχa-li-če-r ʁarʁ-ube.\\
red-\textsc{atr} \textsc{asrt} roof-\textsc{obl}-\textsc{super}-\textsc{npl}(\textsc{ess}) stone-\textsc{pl}\\

\ex % b.
\gll ħunt'a-l burχa-li-če-r \textbf{gʷa} ʁarʁ-ube.\\
red-\textsc{atr} roof-\textsc{obl}-\textsc{super}-\textsc{npl}(\textsc{ess}) \textsc{asrt} stone-\textsc{pl}\\
\glt `There are stones on the RED roof.'
\z

\ex \label{ex:12:25} % (25)
\ea % a.
\gll *heš \textbf{gʷa} ʁʷet'i-če-r d-aq-il inc-be d-urh-uwe.\\
that \textsc{asrt} tree-\textsc{super}-\textsc{npl}(\textsc{ess}) \textsc{npl}-much-\textsc{atr} apple-\textsc{pl} \textsc{npl}-become:\textsc{ipfv}-\textsc{cvb.ipfv}\\

\ex % b.
\gll heš ʁʷet'i-če-r \textbf{gʷa} d-aq-il inc-be d-urh-uwe.\\
that tree-\textsc{super}-\textsc{npl}(\textsc{ess}) \textsc{asrt} \textsc{npl}-much-\textsc{atr} apple-\textsc{pl} \textsc{npl}-become:\textsc{ipfv}-\textsc{cvb.ipfv}\\
\glt `There are many apples growing on THAT tree.'
\z

\ex \label{ex:12:26} % (26)
\ea % a.
\gll *har-il \textbf{gʷa} urši-li-s midal g-i-le.\\
each-\textsc{atr} \textsc{asrt} boy-\textsc{obl}-\textsc{dat} medal give:\textsc{pfv}-\textsc{aor}-\textsc{cvb}\\

\ex % b.
\gll har-il urši-li-s \textbf{gʷa} midal g-i-le.\\
each-\textsc{atr} boy-\textsc{obl}-\textsc{dat} \textsc{asrt} medal give:\textsc{pfv}-\textsc{aor}-\textsc{cvb}\\
\glt `He gave a medal to EACH boy.'
\z
\z


One natural way to \isi{focus} an attribute is to place the assertive copula
after the whole NP. Alternatively, one can split the description of a
participant into two NPs with a semantic attribute being nominalized and
taking its own case marker. Since the semantic attribute itself
constitutes a complete NP in this construction, it becomes possible to
place \emph{gʷa} immediately after it (\ref{ex:12:27}). Notably, for absolutive NPs
this results in the illusion of the embedding of the assertive marker in
an NP (\ref{ex:12:28}), but this is likely to be a consequence of the fact that
absolutive NPs do not receive overt case marking, so the two adjoined
absolutive NPs look as a single phrase.

\ea \label{ex:12:27} % (27)
\gll ħunt'aj-če-r \textbf{gʷa} burχa-li-če-r ʁarʁ-ube.\\
red.\textsc{obl}-\textsc{super}-\textsc{npl}(\textsc{ess}) \textsc{asrt} roof-\textsc{obl}-\textsc{super}-\textsc{npl}(\textsc{ess}) stone-\textsc{pl}\\
\glt `There are stones on the RED roof.'

(Lit., `There are stones on the red one, on the roof.')

\ex \label{ex:12:28} % (28)
\gll b-urq'-il \textbf{gʷa} bartbisu iχ-ini ħa-s-i-le.\\
\textsc{n}-old-\textsc{atr} \textsc{asrt} carpet that-\textsc{erg} \textsc{neg}-take:\textsc{pfv}-\textsc{aor}-\textsc{cvb}\\
\glt `He did not buy the OLD carpet.'
\z

Further, \emph{gʷa} cannot occur within \is{syntactic island|(}syntactic islands.
For example,
it cannot be embedded in a coordination construction (\ref{ex:12:29}) or in a
converbal clause (\ref{ex:12:30}).

\ea \label{ex:12:29} % (29)
\gll *rasuj-ni꞊ra \textbf{gʷa} nu-ni꞊ra past'an b-erʁ-u-le.\\
Rasul.\textsc{obl}-\textsc{erg}꞊\textsc{add} \textsc{asrt} I-\textsc{erg}꞊\textsc{add} vegetable.garden \textsc{n}-dig:\textsc{pfv}-\textsc{aor}-\textsc{cvb}\\
\glt (`RASUL and I digged the vegetable garden.')

\ex \label{ex:12:30} % (30)
\ea % a.
\gll *b-urq'-il bartbisu \textbf{gʷa} b-ic-i-le, d-aq-il arc d-aq'-i-le.\\
\textsc{n}-old-\textsc{atr} carpet \textsc{asrt} \textsc{n}-sell:\textsc{pfv}-\textsc{aor}-\textsc{cvb} \textsc{npl}-much-\textsc{atr} money \textsc{npl}-do:\textsc{pfv}-\textsc{aor}-\textsc{cvb} \\

\ex % b.
\gll b-urq'-il bartbisu b-ic-i-le \textbf{gʷa}, d-aq-il arc d-aq'-i-le.\\
\textsc{n}-old-\textsc{atr} carpet \textsc{n}-sell:\textsc{pfv}-\textsc{aor}-\textsc{cvb} \textsc{asrt} \textsc{npl}-much-\textsc{atr} money \textsc{npl}-do:\textsc{pfv}-\textsc{aor}-\textsc{cvb} \\
\glt `After selling THE OLD CARPET, he got much money.'
\z
\z

Unlike most Dargwa varieties, Mehweb has developed a \isi{biabsolutive
construction}\footnote{Biabsolutive (binominative) constructions are
  quite widespread in the East Caucasian family, but are not typical for
  the Dargwa branch, where they have been previously only reported for
  Itsari Dargwa \citep{mutalov-sumbatova2003}. See \citet{forker2012} and
  \citet{gagliardi-etal2014} for surveys of some properties of this kind of
  constructions as well as for a discussion of their diversity and
  possible analyzes.} (see also \citealt{daniel2019} [this volume] and \citealt{ganenkov2019} [this volume]). In this construction, a transitive verb appears as a converb
and requires a copula but the actor appears in the absolutive, as does
the undergoer. This construction is possible with \emph{gʷa} (\ref{ex:12:31}a–b),
yet the assertive copula cannot occur between the P-argument and the
converb (\ref{ex:12:31}c).\footnote{The same set of facts is observed for the simple
  \isi{copula} \emph{le}-\textsc{cl}.} This contrasts the biabsolutive construction
with a simple combination of the converb with a copula and suggests that
this pattern contains an embedded converbal clause which is an island,
at least with respect to \emph{gʷa}:

\ea \label{ex:12:31} % (31)
\ea % a.
\gll musa kaš d-uk-uwe \textbf{gʷa}.\\
Musa kasha \textsc{npl}-eat:\textsc{ipfv}-\textsc{cvb.ipfv} \textsc{asrt}\\
\glt `Musa is eating kasha.'

\ex % b.
\gll musa \textbf{gʷa} kaš d-uk-uwe.\\
Musa \textsc{asrt} kasha \textsc{npl}-eat:\textsc{ipfv}-\textsc{cvb.ipfv}\\
\glt `It is Musa who is eating kasha.'

\ex % c.
\gll *musa kaš \textbf{gʷa} d-uk-uwe.\\
Musa kasha \textsc{asrt} \textsc{npl}-eat:\textsc{ipfv}-\textsc{cvb.ipfv}\\
\glt Intended `It is kasha that Musa is eating.'
\z
\z

With clausal complements, the situation is less obvious: some (but by no
means all) speakers allow positioning \emph{gʷa} within a clausal
complement (\ref{ex:12:32}–\ref{ex:12:33}).\footnote{The superscripted \emph{\%} in these
  examples refers to the fact that there is considerable variation among
  speakers in the acceptance of such examples.}

\ea \label{ex:12:32} % (32)
\gll \textsuperscript{\%}it kaltuška \textbf{gʷa} d-elʡʷ-eˤs d-aʔ-i-le.\\
that potato \textsc{asrt} \textsc{npl}-seed:\textsc{ipfv}-\textsc{inf} \textsc{npl}-start:\textsc{pfv}-\textsc{aor}-\textsc{cvb}\\
\glt `She started to plant potatoes.'

\ex \label{ex:12:33} % (33)
\gll \textsuperscript{\%}heš kʷiha \textbf{gʷa} b-eqʷ-es aħmad-ini di-ze hari b-aq'-i-le.\\
that ram \textsc{asrt} \textsc{n}-cut:\textsc{pfv}-\textsc{inf} Ahmad-\textsc{erg} I.\textsc{obl}-\textsc{inter}(\textsc{lat}) request \textsc{n}-do:\textsc{pfv}-\textsc{aor}-\textsc{cvb}\\
\glt `Ahmed asked me to cut this ram.'
\z

While the placement of \emph{gʷa} after a constituent other than the
predicate usually indicates \isi{focus} shift, even in this case it does not
need to follow the constituents that are (likely to be) focused\is{focus}.
Consider the following examples:

\ea \label{ex:12:34} % (34)
\ea %  a.
\gll χadižat-ini꞊ra heš kung \textbf{gʷa} b-elč-u-we.\\
Khadizhat-\textsc{erg}꞊\textsc{add} that book \textsc{asrt} \textsc{n}-read:\textsc{pfv}-\textsc{aor}-\textsc{cvb}\\

\ex % b.
\gll χadižat-ini꞊ra \textbf{gʷa} heš kung b-elč-u-we.\\
Khadizhat-\textsc{erg}꞊\textsc{add} \textsc{asrt} that book \textsc{n}-read:\textsc{pfv}-\textsc{aor}-\textsc{cvb}\\
\glt `Even Khadizhat has read that book.'
\z
\z

In (\ref{ex:12:34}) one can hypothesize that the focused\is{focus} constituent is the ergative
NP, since it is marked with the additive clitic meaning `even'. Yet as
shown by these examples, the assertive copula may but need not be
adjacent to the focused\is{focus} phrase: indeed, in (\ref{ex:12:34}a) it follows the
absolutive argument. These examples suggest that focus is possibly not
the only factor which determines the position of \emph{gʷa}. More
generally, we conclude that in verbal clauses the grammatical position
of \emph{gʷa} should be determined neither by the predicate nor by focus.

\is{syntactic island|)}

% 4.
\section{Non-verbal predication}\label{non-verbal-predication}

Non-verbal predication is represented by two types, namely existential
clauses and non-existential clauses with non-verbal predicates (nouns,
adjectives, numerals, demonstratives, etc.). In Mehweb, the latter type
allows the absence of a copula while the former normally does
not.\footnote{An important exception is the use of NPs denoting events,
  which allow the absence of copula, as in (i):

  \begin{exe}\exi{(i)}
    \gll išbari meħʷe-b beʁ.\\
  today in.Mehweb-\textsc{n}(\textsc{ess}) wedding\\
  \glt   `There is a wedding in Mehweb today.'
\end{exe}


  \removelastskip
  \vspace{-\baselineskip}
} The assertive copula can appear
in both types.

  (\ref{ex:12:35}–\ref{ex:12:36}) show examples of the use of \emph{gʷa} in \isi{existential predication} that assert the existence of entities or events described
by an NP. Note that, in Mehweb, this type includes \isi{possessive
predication} (\ref{ex:12:37}).

\ea \label{ex:12:35} % (35)
\gll ʁuni-b \textbf{gʷa} muzej!\\
in.Gunib-\textsc{n}(\textsc{ess}) \textsc{asrt} museum\\
\glt `There is a museum in Gunib!'

\ex \label{ex:12:36} % (36)
\gll išbari meħʷe-b beʁ \textbf{gʷa}!\\
today in.Mehweb-\textsc{n}(\textsc{ess}) wedding \textsc{asrt}\\
\glt `There is wedding in Mehweb today!'

\ex \label{ex:12:37} % (37)
\gll pat'imat-la q'ʷaˤl \textbf{gʷa}!\\
Patimat-\textsc{gen} cow \textsc{asrt}\\
\glt `Patimat has a cow!'
\z

The assertive copula is also found in clauses emphasizing the existence
of the already known entities (sometimes in combination with the
converbal form of the copula; cf.\ (\ref{ex:12:38})) or describing the location of
the already known entities (\ref{ex:12:39}):

\ea \label{ex:12:38} % (38)
\gll meħʷe (le-b-le) \textbf{gʷa}!\\
in.Mehweb {be}-\textsc{n}-\textsc{cvb} \textsc{asrt}\\
\glt `Mehweb does exist!'

\ex \label{ex:12:39} % (39)
\gll musa ʁuni-w \textbf{gʷa}.\\
Musa in.Gunib-\textsc{m}(\textsc{ess}) \textsc{asrt}\\
\glt `Musa is in Gunib.'
\z

  (\ref{ex:12:40}–\ref{ex:12:41}) show examples of the use of \emph{gʷa} in clearly
non-existential predications.

\ea \label{ex:12:40} % (40)
\gll heš-di hum-be \textbf{gʷa} ʜaˤb dek'ar-i.\\
that-\textsc{pl} road-\textsc{pl} \textsc{asrt} three different-\textsc{atr}\\
\glt `These roads are three different (roads).' (Two sons)

\ex \label{ex:12:41} % (41)
\gll ħa-la k'unk'ul-li-ʔini b-aq'-ib-il k'unk'ur \textbf{gʷa} iš.\\
you.sg.\textsc{obl}-\textsc{gen} cauldron-\textsc{erg} \textsc{n}-do:\textsc{pfv}-\textsc{aor}-\textsc{atr} cauldron \textsc{asrt} that\\
\glt `This (cauldron) is the cauldron originating from (lit., made by) your
cauldron.' (Molla Rasbaddin and the neighbour's cauldron, 1.5)
\z

At least if the assertive marker follows the demonstrative, their
combination can be embedded within the alleged subject phrase. In (\ref{ex:12:42})
the phrase \emph{heš} \emph{gʷa} `that is' is embedded within the
relative clause construction `the house which Rasul built'.

\ea \label{ex:12:42} % (42)
\gll rasuj-ni [heš \textbf{gʷa}] b-aq'-ib-i qali.\\
Rasul.\textsc{obl}-\textsc{erg} that \textsc{asrt} \textsc{n}-do:\textsc{pfv}-\textsc{aor}-\textsc{atr} house\\
\glt `The house that Rasul built is that one.'
\z

Negative non-verbal predication in Mehweb contains a dedicated negative
copular verb. If \emph{gʷa} is needed, this copula appears in a converbal
form:

\ea \label{ex:12:43} % (43)
\gll it učitel aħi-je \textbf{gʷa}.\\
that teacher {be}:\textsc{neg}-\textsc{cvb} \textsc{asrt}\\
\glt `He is not a teacher!'
\z

In equative clauses, determining what  the predicate is presents a
complex issue because of the formal similarity between the subject and
the nominal predicate. Still, one can find indirect evidence for the
predicate status of one of the noun phrases based on various semantic
and syntactic tests. By using these tests, it is possible to show that
the assertive marker does not have to immediately follow the 
predicate.

First, if a nominal phrase in an equative clause includes a \isi{reflexive pronoun}
bound by the other part of the clause, it is likely that it is a
predicate and the reflexive is bound by the subject. However, \emph{gʷa}
need not follow such a nominal predicate:

\ea \label{ex:12:44} % (44)
\gll šamil \textbf{gʷa} sune-s-al weˤʡ.\\
Shamil \textsc{asrt} self.\textsc{obl}-\textsc{dat}-\textsc{emph} master\\
\glt `Shamil is a boss of himself.'
\z

Second, in an equative clause, an expression with a true \isi{distributive
quantifier} arguably should not function as a predicate (\citealt{partee1987}; but
see \citealt{arkadiev-lander2013} for counterevidence). Yet, \emph{gʷa} is
possible with the quantified NP:

\ea \label{ex:12:45} % (45)
\gll har insan \textbf{gʷa} sune-s-al uħna-w rasul ħamzatow\\
every person \textsc{asrt} self.\textsc{obl}-\textsc{dat}-\textsc{emph} \textsc{m}.inside-\textsc{m}(\textsc{ess}) Rasul Gamzatov.\\
\glt `Everyone is Rasul Gamzatov (a famous Daghestanian writer) deep inside.'
\z

Finally, if an equative clause contains an adjunct, the assertive copula
may follow this adjunct:

\ea \label{ex:12:46} % (46)
\gll anwar meħʷe-ja uškuj-ħe-w \textbf{gʷa} učitel.\\
Anwar in.Mehweb-\textsc{gen} school.\textsc{obl}-\textsc{in}-\textsc{m}(\textsc{ess}) \textsc{asrt} teacher\\
\glt `Anwar is a teacher at the Mehweb school.'
\z

Thus, the assertive marker need not follow the predicate. At the same
time, it is not obvious that \emph{gʷa} always follows the \isi{focus}. For
instance, in the elicited dialog (\ref{ex:12:47}), \emph{gʷa} is attached to the
first part of the clause `Shamil is a singer', while its focus is
constituted by its second part. In answers to content questions,
\emph{gʷa} is by default attached to the part of the utterance which
does not contain new information, as in (\ref{ex:12:48}) and (\ref{ex:12:49}).

\ea \label{ex:12:47} % (47)
\gll šamil učitel. – aħin! šamil \textbf{gʷa} dalaj uk'-an-či!\\
Shamil teacher {} {be}:\textsc{neg} Shamil \textsc{asrt} song \textsc{m}.say:\textsc{ipfv}-\textsc{hab}-\textsc{ag}\\
\glt `Shamil is a teacher. – No! Shamil is a singer!'

\ex \label{ex:12:48} % (48)
\gll meħʷe-la χʷalajli či-ja? – meħʷe-la χʷalajli \textbf{gʷa} israpil.\\
in.Mehweb-\textsc{gen} chief who-\textsc{q} {} in.Mehweb-\textsc{gen} chief \textsc{asrt} Israpil\\
\glt `Who is the head of Mehweb? – The head of Mehweb is Israpil.'

\ex \label{ex:12:49} % (49)
\gll israpil či-ja? – israpil \textbf{gʷa} meħʷe-la χʷalajli.\\ 
Israpil who-\textsc{q} {} Israpil \textsc{asrt} in.Mehweb-\textsc{gen} chief\\
\glt `Who is Israpil? – Israpil is the head of Mehweb.'
\z

Thus, we find that, in non-verbal predications as well as in verbal
predications, the assertive copula does not necessarily follow the
predicate and the focused\is{focus} element.

% 5.
\section{Conclusion}

To sum up, the assertive marker \emph{gʷa} has the distribution of a
copula (though lacking non-finite forms which are available for the
copula), but its position does not fit into the picture that is usually
documented in East Caucasian languages in that it does not need to be
adjacent to the predicate or focus. At the same time, we observe some
constraints on its distribution in complex constructions (in particular,
its reluctance to syntactic islands). I conclude that more research is
needed both to approach the functions of \emph{gʷa} and to understand
the principles that govern its syntactic position.

Further, it seems that our assumed knowledge of the principles regarding
other kinds of predicative markers is overestimated. Indeed, while the
idea of focus-determined positions of copulas is important for East
Caucasian, I am aware of no detailed corpus-based study of the position
of predicative markers for any language of the family. Given the fact
that during the last years the amount of corpora of East Caucasian
languages has been increasing, one may hope that such studies will soon
appear.

Moreover, as I emphasized in \sectref{the-assertive-marker-as-a-copula}, predicative markers differ in
their behavior, both within a single language and cross-linguistically.
For East Caucasian, we need a more elaborated intragenetic typology of
predicative markers. The present paper is to be considered a
contribution to this line of investigation.

\section*{Acknowledgements}

I am grateful to all my consultants in Mehweb for their patience and
to Michael Daniel, Nina Dobrushina and two anonymous reviewers for
their useful comments on earlier versions of the paper.

% \clearpage

\section*{List of abbreviations}

\begin{longtable}[l]{@{}ll@{}}
\textsc{1pl}	& first person plural \\
\textsc{abl}	& ablative \\
\textsc{ad}	& spatial domain near the landmark \\
\textsc{add}	& additive particle \\
\textsc{advz}	& adverbializer \\
\textsc{ag}	& nomen agentis \\
\textsc{ante}	& anteriority converb \\
\textsc{aor}	& aorist \\
\textsc{asrt}	& assertive particle \\
\textsc{atr}	& attributivizer \\
\textsc{aux}	& auxiliary \\
\textsc{caus}	& causative \\
\textsc{cl}	& gender (class) agreement slot \\
\textsc{cvb}	& converb \\
\textsc{dat}	& dative \\
\textsc{ego}	& egophoric \\
\textsc{el}	& motion from a spatial domain \\
\textsc{emph}	& emphasis (particle) \\
\textsc{erg}	& ergative \\
\textsc{ess}	& static location in a spatial domain \\
\textsc{f}	& feminine (gender agreement) \\
\textsc{f1}	& feminine (unmarried and young women gender prefix) \\
\textsc{fut}	& future \\
\textsc{gen}	& genitive \\
\textsc{hab}	& habitual (durative for verbs denoting states) \\
\textsc{in}	& spatial domain inside a (hollow) landmark \\
\textsc{inf}	& infinitive \\
\textsc{inter}	& spatial domain between multiple landmarks \\
\textsc{ipfv}	& imperfective (derivational base) \\
\textsc{lat}	& motion into a spatial domain \\
\textsc{lv}	& light verb \\
\textsc{m}	& masculine (gender agreement) \\
\textsc{n}	& neuter (gender agreement) \\
\textsc{neg}	& negation (verbal prefix) \\
\textsc{npl}	& non-human plural (gender agreement) \\
\textsc{obl}	& oblique (nominal stem suffix) \\
\textsc{pfv}	& perfective (derivational base) \\
\textsc{pl}	& plural \\
\textsc{prf}	& perfect \\
\textsc{prs}	& present \\
\textsc{pst}	& past \\
\textsc{ptcp}	& participle \\
\textsc{pv}	& preverb (verbal prefix) \\
\textsc{q}	& question (interrogative particle) \\
\textsc{super}	& spatial domain on the horizontal surface of the landmark \\
\end{longtable}

\printbibliography[heading=subbibliography,notkeyword=this]


\end{document}

%%% Local Variables:
%%% mode: latex
%%% TeX-master: "../main"
%%% End:
