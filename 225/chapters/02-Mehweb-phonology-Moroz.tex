\documentclass[output=paper]{langsci/langscibook} 
\ChapterDOI{10.5281/zenodo.3402056}

% Chapter 2

\title{Phonology of Mehweb}

\author{George Moroz\affiliation{Linguistic Convergence Laboratory, National Research University Higher School of Economics}} % A. 

\abstract{In this paper, I describe the phonetic inventory of Mehweb, consonants and vowels, as well as the main productive alternations. Two separate sections treat the rules of syllable\is{syllable structure} structure and give a preliminary treatment of \isi{pharyngealization}. In Mehweb, \isi{pharyngealization} is a feature which extends the basic vowel inventory (\emph{i}, \emph{e}, \emph{a}, \emph{u}) to include \emph{oˤ} (the pharyngealized\is{pharyngealization} variant of \emph{u}, along with pharyngealized\is{pharyngealization} \emph{iˤ}, \emph{eˤ}, \emph{aˤ}, \emph{uˤ}) and the inventory of radical and laryngeal consonants by the process of epiglottalization (where \emph{ʡ} is a pharyngealized\is{pharyngealization} variant of \emph{ʔ} and \emph{ʜ} is a pharyngealized\is{pharyngealization} variant of \emph{ħ}).

\emph{Keywords:} syllabification, stress, vowels, consonants,
\isi{pharyngealization}, alternations.}

\begin{document}
\maketitle

% 1. 
\section{Introduction}



This paper is an overview of the phonology of
Mehweb. It is primarily descriptive and is intended to make phonological
aspects of Mehweb clear to the reader. The paper is organized as
follows. In \sectref{consonants} and \sectref{vowels} I describe ​the​​ language's consonant and vowel systems.
\sectref{syllable-and-word-structure} is dedicated to syllable\is{syllable structure} and word structure of Mehweb.
\sectref{stress} deals with \isi{stress}. In \sectref{some-phonological-and-morphophonological-alternations} I introduce basic phonological and
morphophonological alternations. In the last section I describe
\isi{pharyngealization} and how it affects segments.



% 2.
\section{Consonants}\label{consonants}

The inventory of consonants is given in \tabref{t3-1}. Sounds provided in
parentheses are allophones, distributed either contextually or socially,
as described below.

\pagebreak

\begin{table}
% Table 1. 
\caption{Mehweb consonant phonemes\protect\footnotemark}\label{t3-1}

\footnotesize\tabcolsep3.3pt
\begin{tabular}{@{}lrccc@{\hskip9pt}c@{\hskip12pt}*{5}{cc}@{}} % {@{}lr|c|c|c|c*{5}{|cc}@{}}
\toprule % \hlx{v}
  && \makebox[12pt][l]{\raisebox{-15pt}[0pt][0pt]{\rotatebox{50}{labial}}}
  & \makebox[12pt][l]{\raisebox{-16pt}[0pt][0pt]{\rotatebox{50}{dental}}}
  & \makebox[12pt][l]{\raisebox{-18pt}[0pt][0pt]{\rotatebox{50}{\addfontfeature{LetterSpace=-.5}alveolar}}}
  & \makebox[12pt][l]{\raisebox{-16pt}[0pt][0pt]{\rotatebox{50}{palatal}}} &
 \multicolumn{2}{c}{velar} &
 \multicolumn{2}{c}{uvular} &
 \multicolumn{2}{c}{\addfontfeature{LetterSpace=-.5}pharyngeal} & 
 \multicolumn{2}{c}{\addfontfeature{LetterSpace=-.5}epiglottal} &
 \multicolumn{2}{c@{}}{glottal}\tabularnewline  \cmidrule(l{\tabcolsep}r{\tabcolsep}){7-8}  \cmidrule(l{\tabcolsep}r{\tabcolsep}){9-10}  \cmidrule(l{\tabcolsep}r{\tabcolsep}){11-12}  \cmidrule(l{\tabcolsep}r{\tabcolsep}){13-14} \cmidrule(l{\tabcolsep}){15-16}
&&&&&& -lab & +lab & -lab & +lab & \makebox[2em]{-lab} & \makebox[2em]{+lab} &-lab & +lab &-lab & +lab \tabularnewline \midrule % \hlx{vhv}
& +v & b & d & & & g & gʷ & & & & & & &&\tabularnewline
plosive & -v & p & t & & & k & kʷ & q & qʷ &&& (ʡ)& (ʡʷ)&ʔ&ʔʷ \tabularnewline
& ej & p' & t' & & & k' & k'ʷ & q' & q'ʷ &&&&&\tabularnewline \midrule % \hlx{vhv}
\raisebox{-5pt}[0pt][0pt]{fricative} & +v & & z & ž & & (ɣ) & & ʁ & ʁʷ &&&&&(ɦ)\tabularnewline
& -v & & s & š & & x & xʷ & χ & χʷ &ħ &ħʷ &(ʜ)&(ʜʷ)&h&hʷ \tabularnewline \midrule % \hlx{vhv}
& +v & & (ʒ) & (ǯ) & & & & &&&&&&\tabularnewline
affricate & -v & & c & č & & & & &&&&&&\tabularnewline
& ej & & c' & č' & & & & &&&&&&\tabularnewline \midrule % \hlx{vhv[4]}
\multicolumn{2}{@{}l}{sonorant} & m, w & \multicolumn{2}{c@{\hskip11pt}}{n, l, r} & j & & & & & &&&&\tabularnewline
\bottomrule % \hlx{v[4]h}
\end{tabular}
\end{table}


\footnotetext{In the table, +v stands for
 voiced, -v stands for voiceless, ej stands for ejective, lab stands
 for \isi{labialization}. Some allophones are presented in brackets. To be
 consistent with the transcription system used in the other
 contributions to this collection, I use the following transcriptions:
\centerline{\begin{tabular}{lccccccc}
CT& g& ž& š& č& c& ʒ& ǯ\\
IPA& ɡ& ʒ& ʃ& tʃ& ts& dz& dʒ
 \end{tabular}}}

There are 41 consonant phonemes in Mehweb, which are listed in \tabref{t3-1}.
Most plosives and affricates form three-way oppositions (voiced vs.
voiceless vs. ejective), but there are no radical\footnote{I use
  \emph{radical} after \citet{ladefoged1996}  % (Ladefoged, Maddieson 1996)
  to denote pharyngeal\is{pharyngealization} and epiglottal sounds.} voiced segments except some rare realizations
of \emph{h} as \emph{ɦ}. I don't mark concrete place of articulation for the
sonorants \emph{n}, \emph{l} and \emph{r}, since they ​ can be realized as either dental or alveolar. All postvelar consonants and velar plosives have
labialized counterparts, which occur in word-initial, medial
intervocalic, medial preconsonantal and final position. Some Dargwa
languages have voiceless geminate consonants. They correspond to voiced
consonants in Dargwa languages lacking geminates. There are no geminates
in Mehweb (contra \citealt[8]{magometov1982}).   Sequences of homorganic
consonants, however, are realised as geminates phonetically (cf.\ example (\ref{ex:1})):

\ea \label{ex:1}
\gll it-di-ni~{\upshape\textgreater}~itːini\\
this-\textsc{pl}-\textsc{erg}\\
\z

The voiced velar fricative \emph{ɣ} is attested only word initially in a
few roots and only in the speech of older consultants (cf.\ examples
(\ref{ex:2}–\ref{ex:4})). Younger consultants use the velar stop \emph{g} instead.

\ea \label{ex:2} % (2) 
\textbf{ɣ}an\\ 
\glt `snake' 

\ex \label{ex:3} % (3) 
\textbf{ɣ}uli\\
\glt `hide'

\ex \label{ex:4} % (4)
 \textbf{ɣ}ala\\
\glt `pitchfork'
\z

The voiced affricates \emph{ʒ} and \emph{ǯ} are allophones of the voiced
fricatives \emph{z} and \emph{ž}. They are attested only in the speech
of older consultants (cf.\ example (\ref{ex:5}a–b)):

\ea \label{ex:5} % {(5a)}
\ea \xbox{.3\textwidth}{\textbf{ʒ}e \textup{(older speakers)}
\glt `salt'}
%
\exsameline % {(5b)}
\xbox{.3\textwidth}{\textbf{z}e \textup{(younger speakers)}
\glt `salt'}
\z
\z

Some realizations of \emph{s} in intervocalic position seem geminate and are 
perceived as such by some of our consultants, including 
\emph{isːes} `take (\textsc{ipfv})', \textsc{cl}-\emph{isːes} `weep (\textsc{ipfv})', 
\emph{usːes} `grind'. These are the only  three verbal roots with intervocalic \emph{s} known to us, 
and we have no comparable evidence for nouns. There is thus no clear evidence that geminate 
\emph{sː} is phonologically distinct from simple
\emph{s}. The issue requires further investigation.

The glottal stop \emph{ʔ} is usually deleted in initial and intervocalic
position. Some older speakers occasionally produce the voiced glottal
fricative \emph{ɦ} instead of voiceless \emph{h} in intervocalic
position.

In non-final position epiglottal \emph{ʡ} and \emph{ʜ} are in most cases
followed and/or preceded by a pharyngealized\is{pharyngealization} vowel. The
segments \emph{ʔ} and \emph{ħ} are never followed or preceded by a
pharyngealized\is{pharyngealization} vowel\footnote{The situation is however more complex. First of all, the difference between \emph{ʜ} and \emph{ħ} is not perceived by all speakers; the others blame it on the quality of the preceding or the following vowel. Second, on \emph{a} and \emph{u}, the presence of the pharyngeal feature is very hard to perceive, even if the speakers were able to recognize the few minimal pairs we were able to find. One could then simply assume that \emph{ʔ} and \emph{ħ} only appear in non-pharyngealized contexts and \emph{ʡ} and \emph{ʜ} only appear in pharyngealized contexts. However, in the perceptually clearest cases, which are a combination of a pharyngeal stop \emph{ʡ} with the vowel \emph{a}, in some words, epiglottal \emph{ʡ} can be followed by non-pharyngealized\is{pharyngealization} \emph{a}. Some of these are Avar loanwords, including \emph{ʡat'} `dough, flour' (cf.~Avar \emph{ʕat'} `flour'; providing a pseudo minimal pair \emph{ʡat'} `dough' vs. \emph{ʡaˤt'a} `frog'), \emph{ʡaraq} `haystack' (cf.~Avar \emph{ʕaraqχ} `haystack'), \emph{ʡamal} `temper' (Avar \emph{ʕamal} `temper'), \emph{maʡna} `meaning' (~Avar \emph{maʕna} `meaning'). But other seem to be native, including \emph{ʡarʁal} `long', \emph{ʡa‹b›ad} `behind', \emph{ʡaraʁa} `last', \emph{beʡʷes} `seed', and particle \emph{ʡaj}. In combination, all this lead to inconsistencies in our transcription of pharyngeals throughout the book. Pharyngealized vowels other than \emph{a} and \emph{u} may also have been lost in transcription.}. In \sectref{pharyngealization} I will discuss some examples of
\emph{ʔ}/\emph{ʡ-} and \emph{ħ/ʜ}-alternations triggered by \isi{pharyngealization},
where I will also consider evidence for the independent and
suprasegmental nature of the pharyngeal\is{pharyngealization} feature.


\begin{wrapfigure}[8]{r}{4cm}
\vspace{-1.2\baselineskip}
  \begin{vowel}[plain]
\putcvowel{i, (iˤ)}{1}
\putcvowel{e, (eˤ)}{2}
\putcvowel{u, (uˤ)}{8}
\putcvowel{oˤ}{7}
\putcvowel{aˤ}{16}
\putcvowel{a}{4}
\end{vowel}
\caption{Vowel system}
\end{wrapfigure}



% 3.
\section{Vowels}\label{vowels}

There are four plain vowels and five pharyngealized\is{pharyngealization} vowels. Length is
not distinctive. Some pharyngealized\is{pharyngealization} vowels such as \emph{iˤ},
\emph{eˤ}, \emph{uˤ} are very rare, the phonological
%
%    
status of these
sounds thus is not clear, so they are written in brackets.
Pharyngealized\is{pharyngealization} vowels occur most often adjacent to, or in forms
containing, epiglottals (\emph{ʡ}, \emph{ʜ}) or uvulars (\emph{q}, \emph{χ},
\emph{ʁ}). However, \emph{aˤ} is also attested in some stems without
those segments:

\ea \label{ex:6} % (6)
l\textbf{aˤ}ži
\glt `cheek'

\ex \label{ex:7} % (7) 
kʷ\textbf{aˤ}š
\glt `handful'

\ex \label{ex:8} % (8)
t\textbf{aˤ}j
\glt `foal'
\z

Pharyngealized\is{pharyngealization} vowels are not common in Mehweb, and some are rarer than
others. For example, pharyngealized\is{pharyngealization} \emph{iˤ} and \emph{eˤ} are only
attested in very few words. Pharyngealized\is{pharyngealization} \emph{oˤ} seems to be a
realization of \emph{u} in pharyngealized\is{pharyngealization} syllables\is{syllable structure}; however, while in
some roots only \emph{oˤ} is attested (\ref{ex:9}a), in other forms \emph{uˤ}
occurs as a variant (\ref{ex:9}b). This distribution may also depend on individual speakers.

\ea \label{ex:9} % {(9a)}
\ea \xbox{.11\textwidth}{d\textbf{oˤ}rʜ\textbf{oˤ}
\glt `cub'}%
\exsameline % {(9b)} 
\xbox{.23\textwidth}{malʡ\textbf{uˤ}n, malʡ\textbf{oˤ}n
\glt `wolf'}%
\z
\z


Vowels, as well as pharyngeal\is{pharyngealization} and epiglottal consonants, rarely show
clear evidence of independent behavior of the pharyngeal\is{pharyngealization} feature. Pharyngealized\is{pharyngealization} vowels show alternations in
e.g.\ plural stem formation, as shown in examples (\ref{ex:10}–\ref{ex:15}); see also
\citet{chechuro2019}.

\ea \label{ex:10} % {(10a)}
\ea \xbox{.11\textwidth}{\gll j\textbf{aˤ}bu\\
horse\\
\glt `horse'}%
%
\exsameline % {(10b)} 
\xbox{.3\textwidth}{\gll j\textbf{aˤ}b-ne\\
horse-\textsc{pl}\\
\glt `horses'}
\z

\ex \label{ex:11} % {(11a)} 
\ea \xbox{.11\textwidth}{\gll t\textbf{aˤ}j\\
foal\\
\glt `foal'}%
%
\exsameline % {(11b)} 
\xbox{.3\textwidth}{\gll t\textbf{uˤ}j-re\\
foal-\textsc{pl}\\
\glt `foals'}
\z

\ex \label{ex:12} % {(12a)}
\ea \xbox{.11\textwidth}{\gll č'aˤʡ\textbf{aˤ}\\
cane\\
\glt `cane'}%
%
\exsameline % {(12b)}
\xbox{.3\textwidth}{\gll č'aˤʡ\textbf{uˤ-}be,~č'aˤʡ\textbf{oˤ-}be\\
cane-\textsc{pl}\\
\glt `canes'}
\z

\ex \label{ex:13} % {(13a)}
\ea \xbox{.11\textwidth}{\gll č'\textbf{u}ʡaˤ\\
straw\\
\glt `straw'}%
%
\exsameline % {(13b)}
\xbox{.3\textwidth}{\gll č'\textbf{uˤ}ʡ\textbf{-}ne,~č'\textbf{oˤ}ʡ\textbf{-n}e\\
straw-\textsc{pl}\\
\glt `straws'}
\z

\ex \label{ex:14} % {(14a)}
\ea \xbox{.11\textwidth}{ \gll \textbf{u}ʡaˤ\\
 cheese\\
\glt `cheese'}%
%
\exsameline % {(14b)}
\xbox{.4\textwidth}{\gll ʡ\textbf{aˤ}ʡ\textbf{-}ne,~ʡ\textbf{uˤ}ʡ\textbf{-}ne,~ʡ\textbf{oˤ}ʡ\textbf{-}ne\\
 cheese-\textsc{pl}\\
\glt `cheeses'}
\z

\ex \label{ex:15} % {(15a)}
\ea \xbox{.11\textwidth}{\gll {ʜ\textbf{uˤ}li}\\
 fat\\
 \glt `fat'}%
%
\exsameline % {(15b)}
\xbox{.3\textwidth}{\gll {ʜ\textbf{aˤ}l-me}\\
 fat-\textsc{pl}\\
 \glt `fats'}
\z
\z


\tabref{t3-2} sums up the vowel alternation patterns shown in (\ref{ex:10}) to (\ref{ex:15}).
Pharyngealization\is{pharyngealization}\hskip0pt-related processes are explained at the end of \sectref{pharyngealization}.

\begin{table}
% Table 2. 
\caption{Examples of alternation patterns}\label{t3-2}

\begin{tabular}{@{}lllllll@{}}
\toprule
\textsc{sg} & \emph{aˤ} (\ref{ex:10}a) & \emph{aˤ} (\ref{ex:11}a) & \emph{aˤ} (\ref{ex:12}a) &
\emph{u} (\ref{ex:13}a) & \emph{u} (\ref{ex:14}a) & \emph{uˤ} (\ref{ex:15}a)\tabularnewline \midrule
\textsc{pl} & \emph{aˤ} (\ref{ex:10}b) & \emph{uˤ} (\ref{ex:11}b) & \emph{uˤ, oˤ} (\ref{ex:12}b) &
\emph{uˤ, oˤ} (\ref{ex:13}b) & \emph{aˤ, uˤ, oˤ} (\ref{ex:14}b) & \emph{aˤ}
(\ref{ex:15}b)\tabularnewline
\bottomrule
\end{tabular}
\end{table}

Vowel frequencies in a list of 596 noun roots are as follows: \emph{a} –
38\%, \emph{i} – 27\%, \emph{u}~– 23\%, \emph{e} – 6\%, \emph{aˤ} –
6\%, other vowels less than 2\%. The most frequent vowel structure in
bisyllabic words is \emph{a}–\emph{a}.

The most complex phenomenon in Mehweb phonology is \isi{pharyngealization}.
Pharyngealization\is{pharyngealization} seems to be associated with uvular, pharyngeal\is{pharyngealization} and
epiglottal consonants, but there are some cases where it is not; cf.\
(\ref{ex:6}–\ref{ex:8}). Pharyngealized\is{pharyngealization} vowels typically appear after radical or uvular
consonants, e.g.\ (\ref{ex:13}a–\ref{ex:15}a); but sometimes they may precede them, e.g.\
(\ref{ex:13}b–\ref{ex:15}b); or occur both preceding and following them; e.g.\ (\ref{ex:12}a) and
(\ref{ex:12}b). For a discussion of an approach to \isi{pharyngealization} as a
suprasegmental feature, see \sectref{pharyngealization}.


% 4. 
\section{Syllable and word structure}\label{syllable-and-word-structure}

Except in some borrowings, the \isi{syllable structure} of most words can be
described as (C)V(C)(C). In other words, possible syllables\is{syllable structure} are: CV,
CVC, CVCC, VC, VCC, and V (cf.\ (\ref{ex:16}–\ref{ex:21})). If the coda is complex, the
first consonant is most frequently either a liquid or a nasal, as in
examples (\ref{ex:16}) and (\ref{ex:18}). Clusters of sonorants in the same syllable\is{syllable structure} are
not attested. Consonant sequences cannot be longer than three segments,
as in (\ref{ex:21}), and appear only at morphological boundaries. I treat such
sequences as divided between two syllables\is{syllable structure}. All native words can be
divided into syllables\is{syllable structure} according to the above schemata, but no
experiments with speakers' judgments on the location of syllable\is{syllable structure}
boundaries have been conducted.

\ea \label{ex:16} % (16)
nerʔ\\
\glt `louse' 

\ex \label{ex:17} % (17)
bec'\\
\glt `wolf'  

\ex \label{ex:18} % (18)
ims\\
\glt `moth'  

\ex \label{ex:19} % (19)
u\\
\glt `bottom'

\ex \label{ex:20} % (20)
qi\\
\glt `horn'            

\ex \label{ex:21} % (21)
ims-la\\
\glt moth-\textsc{gen} 
\z


The two action nominals \emph{w-ilsk'-ri}
(\textsc{m}-look:\textsc{ipfv}-\textsc{nmlz}) and \emph{w-ebk'-ri}
(\textsc{m}-die:\linebreak[0]\textsc{pfv}-\textsc{nmlz}) are the only examples known
so far to show a deviant syllable\is{syllable structure} structure. Note that there is some
evidence from nominal inflection \citep{chechuro2019} that \emph{b} may be
treated as a sonorant.

In Mehweb, the sonority sequencing principle\footnote{This principle can
 be formulated as follows: the overall acoustic energy of segments
 should increase from the beginning of the syllable\is{syllable structure} towards the
 syllable\is{syllable structure} nucleus, and decrease from the nucleus toward the end of the
 syllable\is{syllable structure}. I use a shortened version of the Sonority Hierarchy:
 obstruents \textless{} sonorants \textless{} vowels.} is thus rarely
violated: codas are predominantly sequences of a sonorant and an
obstruent. Sequences of sonorants or vowels are not allowed.

Noun stems can have from one to five syllables\is{syllable structure} (cf.\ (\ref{ex:22}–\ref{ex:26})). Most
common are one- and two-syllable\is{syllable structure} roots. \tabref{t3-3} shows the proportion of
one-, two-, \mbox{three-}, four- and five-syllable\is{syllable structure} noun stems, based on a list
of over 500 noun entries.

\begin{table}[h]
  % Table 3.
  \caption{Distribution of one-, two-, three-, four- and five-syllable\is{syllable structure}
noun stems}\label{t3-3}

\begin{tabular}{@{}cccccc@{}}
\toprule
\small one-syllable\is{syllable structure} & \small two-syllable\is{syllable structure} & \small three-syllable\is{syllable structure} & \small four-syllable\is{syllable structure} &
\small five-syllable\is{syllable structure} &\small Total\tabularnewline \midrule
132 & 284 & 65 & 22 & 1 & 504\tabularnewline
26\% & 56\% & 13\% & 4\% & \textless{}1\% & 100\%\tabularnewline
\bottomrule
\end{tabular}
\end{table}

\ea \label{ex:22} % (22)
{bec'}
\glt `wolf'

\ex \label{ex:23} % (23)
{darša}
\glt `thread'

\ex \label{ex:24} % (24)
{urculi}
\glt `wood'

\ex \label{ex:25} % (25)
{pušduk'ani}
\glt`sledgehammer'

\ex \label{ex:26} % (26)
{urʁaˤdiq'aˤni}
\glt `fat tail'
\z



Most verbal stems are monosyllabic. Out of 150 verbs collected so far,
only five are disyllabic (cf.\ (\ref{ex:27})).

\ea \label{ex:27} % (32) -> (27)
\gll {usaʔʷ-as}\\
\textsc{m}.sleep:\textsc{pfv}-\textsc{inf}\\
\glt `sleep'        
\z

There are also five
\is{irregular verbs}
irregular verbal stems (cf.\ (\ref{ex:28}–\ref{ex:32})) which, in some word forms, only consist of
one consonant or, in the case of `say' (cf.\ (\ref{ex:29})), may be considered to be
realized as zero morphs. The vast majority of Mehweb verbs have two
stems, a perfective stem and an imperfective stem. It is worth pointing
out that all irregular mono-consonantal stems are perfective.

\ea \label{ex:28} % (27) -> (28) etc
\gll {k-ib}\\
bring.to:\textsc{pfv}-\textsc{aor}\\
\glt `s/he brought something to somebody'

\ex \label{ex:29} % (28)
\gll {ib}\\
say:\textsc{pfv}.\textsc{aor}\\
\glt `s/he said'

\ex \label{ex:30} % (29)
\gll {g-ub}\\
see:\textsc{pfv}-\textsc{aor}\\
\glt `s/he saw'

\ex \label{ex:31} % (30)
\gll {g-ib}\\
give:\textsc{pfv}-\textsc{aor}\\
\glt `s/he gave'    

\ex \label{ex:32} % (31)
\gll {χ-ib}\\
bring:\textsc{pfv}-\textsc{aor}\\
\glt `s/he brought'

\z


These examples show a difference in number of syllables\is{syllable structure} in nominal and
verbal stems: nominal stems tend to be disyllabic, while verbal stems
are mostly monosyllabic. This type of asymmetry is typical for the other
Dargwa varieties as well.


% 5. 
\section{Stress}\label{stress}

\is{stress}


As compared with different Dargwa varieties, Mehweb has more or less fixed \isi{stress} (cf.\ \citealt{moroz2014}). In nearly all polysyllabic forms the \isi{stress} is on the second syllable\is{syllable structure}.

\ea % {(33a)}
\ea \xbox{.13\textwidth}{\gll {uq'l\textbf{á}ha}{\upshape\footnotemark}\\
  window\\
  \glt `window'}%
\exsameline % {(33b)}
  \xbox{.18\textwidth}{\gll {uq'l\textbf{á}ha-jni}\\
  window-\textsc{erg}\\}%
\exsameline % {(33c)}
  \xbox{.39\textwidth}{\gll {uq'l\textbf{á}ha-li-če-r}\\
  window-\textsc{obl}-\textsc{super}-\textsc{f}[\textsc{ess}]\\
  \glt `on the window'}
\z

\footnotetext{The nucleus of the stressed syllable\is{syllable structure}
  is marked by an acute accent mark.}

\ex % {(34a)}
\ea  \xbox{.23\textwidth}{\gll {w-ak'-\textbf{í}b}\\
  \textsc{m}-come:\textsc{pfv}-\textsc{aor}\\
  \glt `he came'}\nobreak\hskip-5pt\nobreak
\exsameline % {(34b)}
\xbox{.285\textwidth}{\gll {w-ak'-\textbf{í}ša}\\
  \textsc{m}-come:\textsc{pfv}-\textsc{fut.ego}\\
  \glt `I (male) will come'}\nobreak\hskip-5pt\nobreak
\exsameline % {(34c)}
 \xbox{.225\textwidth}{ \gll {w-ak'-\textbf{á}s}\\
  \textsc{m}-come:\textsc{pfv}-\textsc{fut}\\
  \glt `he will come'}
\z
\z


There are, however, some exceptions and even some minimal pairs
distinguished by the position of the \isi{stress} (cf.\ (\ref{ex:35}–\ref{ex:38})).


\ea \label{ex:35} % (35)
\gll {bek'\textbf{á}}\\
hill\\
\glt `hill'


\ex \label{ex:36} % (36)
\gll {b-\textbf{é}k'-a}\\
\textsc{hpl}-choose:\textsc{pfv}-\textsc{imp}.\textsc{tr}\\
\glt `choose (them)!'

\ex \label{ex:37} % (37)
\gll {duž\textbf{é}}\\
night\\
\glt `night'

\ex \label{ex:38} % (38)
\gll {d-\textbf{ú}ž-e}\\
\textsc{npl}-drink:\textsc{ipfv}-\textsc{imp}\\
\glt `drink (it)!'   
\z

When a suffix is added to a monosyllabic root, the \isi{stress} is placed on
the second syllable\is{syllable structure}, as shown in (\ref{ex:39}–\ref{ex:40}).

\ea \label{ex:39}
\ea % {(39a)}
  \xbox{.17\textwidth}{\gll {b\textbf{é}č'}\\
  head\\
  \glt`head'}%
\exsameline % {(39b)}
  \xbox{.28\textwidth}{  \gll {beč'-l\textbf{á}}\\
  head-\textsc{gen}\\
  \glt `(e.g.\ part) of a head'}%
\exsameline % {(39c)}
  \xbox{.22\textwidth}{ \gll {buč'-r\textbf{é}}\\
  head-\textsc{pl}\\
  \glt `heads'}          
\z

\ex \label{ex:40} % (40a)}
\ea \xbox{.17\textwidth}{\gll {g-\textbf{ú}b}\\
  see:\textsc{pfv}-\textsc{aor}\\
  \glt `(s)he saw'}%
\exsameline % {(40b)}
  \xbox{.28\textwidth}{\gll {gʷ-iš\textbf{á}}\\
  see:\textsc{pfv}-\textsc{fut.ego}\\
  \glt `I will see'}%
\exsameline % {(40c)}
\xbox{.22\textwidth}{\gll {gʷ-\textbf{é}s}\\
  see:\textsc{pfv}-\textsc{fut}\\
  \glt `he will see'}
\z
\z

Some verbal forms are more complex. In (\ref{ex:41}b) and (\ref{ex:41}c), as compared to
(\ref{ex:41}a), the \isi{stress} is on the second syllable\is{syllable structure}, as expected. Example (\ref{ex:41}d),
the only type of structure where two initial syllables\is{syllable structure} are added in
inflection, shows that the \isi{stress} may not leave the verbal stem:

\ea \label{ex:41} % {(41a)}
\ea \xbox{.3\textwidth}{\gll {b-ik-\textbf{í}b}\\
  \textsc{n}-become:\textsc{pfv}-\textsc{aor}\\
  \glt `he became'}%
\exsameline % {(41b)}
\xbox{.4\textwidth}{\gll {ħa-b-\textbf{í}k-ib}\\
  \textsc{neg}-\textsc{n}-become:\textsc{pfv}-\textsc{aor}\\
  \glt `he didn't become'}
  
\ex % {(41c)}
\xbox{.3\textwidth}{ \gll {ar-b-\textbf{í}k-ib}\\
  \textsc{pv}-\textsc{n}-fall:\textsc{pfv}-\textsc{aor}\\
  \glt `he fell'}% 
\exsameline % {(41d)}
\xbox{.4\textwidth}{ \gll {ar-ħa-b-\textbf{í}k-ib}\\
  \textsc{pv}-\textsc{neg}-\textsc{n}-fall:\textsc{pfv}-\textsc{aor}\\
  \glt `he didn't fall'}
\z
\z

A form that goes against the second-syllable\is{syllable structure} \isi{stress} generalization is
the vocative. A special vocative form only exists for two-syllable\is{syllable structure} stems
which denote humans. These forms are treated as a special \isi{stress}
pattern, with the \isi{stress} on the first syllable\is{syllable structure}. However, an acoustic study is necessary to find out whether this salience should be treated as
\isi{stress} or, alternatively, as a special vocative intonation. In these
forms \isi{stress} always on the first syllable\is{syllable structure} (cf.\ (\ref{ex:42}–\ref{ex:43})).

\ea \label{ex:42} % {(42a)}
\ea \xbox{.13\textwidth}{ \gll {ad\textbf{á}j}\\
  father\\
  \glt `father'}%
\exsameline % {(42b)}
\xbox{.3\textwidth}{  \gll {\textbf{á}daj}\\
  father[\textsc{voc}]\\
  \glt `fatherǃ'}
\z

\ex \label{ex:43} % {(43a)}
\ea \xbox{.13\textwidth}{ \gll {urš\textbf{í}}\\
  {brother}\\
  \glt `brother'}%
\exsameline % {(43b)}
\xbox{.3\textwidth}{ \gll {\textbf{ú}rši}\\
  {brother[\textsc{voc}]}\\
  \glt `brotherǃ'}
\z
\z

Another exception is the optative form: the optative marker is never
stressed (cf.\ (\ref{ex:44}–\ref{ex:45})):

\ea \label{ex:44} % (44)
\gll {l\textbf{ú}č'-ab}\\
read:\textsc{ipfv}-\textsc{opt}\\
\glt `if only he would read'

\ex \label{ex:45} % (45)
\gll {\textbf{ú}rc-ab}\\
fly:\textsc{ipfv}-\textsc{opt}\\
\glt  `if only he would fly'
\z

Imperative forms never have the \isi{stress} in the final position – as in
the optative, in the imperative the stem is stressed. Plural forms,
however, where the imperative is suffixed with the plural-of-addressee
marker \emph{-na}, have the pattern with stress on the second syllable.

\ea \label{ex:46} % (46)
\gll {\textbf{á}rc-e}\\
fly:\textsc{pfv}-\textsc{imp}\\
\glt `fly!'

\ex \label{ex:47} % (47)
\gll {arc-\textbf{é}-na}\\
fly:\textsc{pfv}-\textsc{imp}-\textsc{imp}.\textsc{pl}\\
\glt `fly! (to a group of people)'
\z 

There are numerous Arabic borrowings and proper names which are stressed
mostly as in Arabic (cf.\ (\ref{ex:48}–\ref{ex:51})):

\ea \label{ex:48} % (48)
{aman\textbf{á}t}\\
\glt `assignment'

\ex % (49)
{paraq'\textbf{á}t}\\
\glt `calm'

\ex % (50)
{ʡ\textbf{áˤ}q'lu}\\
\glt `wit'

\ex \label{ex:51} % (51)
{m\textbf{á}sala}\\
\glt `for example'
\z



% 6. 
\section{Some phonological and morphophonological alternations}\label{some-phonological-and-morphophonological-alternations}

\is{sound alternation|(}

In Mehweb hiatus is not allowed, and the underlying forms are changed in
various ways whenever such configurations arise. If the verb stem is
\emph{i}C or \emph{u}C, \emph{i} becomes \emph{j} (as in (\ref{ex:52}) and (\ref{ex:53}))
and the vowel \emph{u} (\emph{uˤ}, \emph{oˤ}) becomes \emph{w} (as in (\ref{ex:54})
and (\ref{ex:55})).

\ea \label{ex:52} % (52)
\gll {ħa\textbf{j}hub}~/{ħa-\textbf{i}h-ub}/\\
\textsc{neg}-throw:\textsc{pfv}-\textsc{aor}\\
\glt `(he) didn't throw'

\ex \label{ex:53} % (53)
\gll {ħa\textbf{j}gʷan}~/{ħa-\textbf{i}gʷ-an}/\\
\textsc{neg}-burn:\textsc{ipfv}-\textsc{hab}\\
\glt `(it) doesn't burn'

\ex \label{ex:54} % (54)
\gll {ħa\textbf{w}cib}~/{ħa-\textbf{u}c-ib/}\\
\textsc{neg}-\textsc{m}.catch:\textsc{pfv}-\textsc{aor}\\
\glt `(he) didn't catch him'

\ex \label{ex:55} % (55)
\emph{ħawrib~/ħa-\textbf{u}r-ib/}\\
\textsc{neg}-rain:\textsc{ipfv}-\textsc{ipft}\\
\glt  `it didn't rain'

\z

Whenever the verbal stem consists of two consonants, the root-initial
vowel deletes after the negation marker (as in (\ref{ex:56})).

\ea \label{ex:56} % (56)
\gll {ħalʔun}~/{ħa-\textbf{e}lʔ-un}/\\
\textsc{neg}-count:\textsc{pfv}-\textsc{aor}\\
\glt `he didn't count'
\z

The vowel \emph{u}, when followed by a consonant cluster, is deleted and
triggers the \isi{labialization} of the final consonant (compare (\ref{ex:54}–\ref{ex:55}) and
(\ref{ex:57}–\ref{ex:59})). Most labialized consonants that appear as a result of this
rule also occur as independent phonemes (see \tabref{t3-1}), but some
labialized consonants, e.g.\ \emph{zʷ}, only appear as a result of this
process.

\ea \label{ex:57} % (57)
\gll {ʜaˤrχ\textbf{ʷ}ib}~/{ħa-\textbf{uˤ}rχ-ib}/\\
\textsc{neg}-\textsc{m}.touch:\textsc{pfv}-\textsc{aor}\\
\glt `didn't touch him'

\ex \label{ex:58} % (58)
\gll{ħabk'\textbf{ʷ}an}~/{ħa-\textbf{u}bk'-an}/\\
\textsc{neg}-\textsc{m}.die:\textsc{ipfv}-\textsc{hab}\\
\glt `he doesn't die'

\ex \label{ex:59} % (59)
\gll {ħarzʷan}~/{ħa-\textbf{u}rz-an}/\\
\textsc{neg}-\textsc{m}.praise:\textsc{ipfv}-\textsc{hab}\\
\glt `didn't praise him'

\z

The alternation from the examples above can be generalized as follows:

\ea \label{ex:60} % (60)
\ea % a.
\upshape /\emph{a-u}C/ → [\emph{aw}C]

\ex % b.
\upshape /\emph{a-u}CC/ → [\emph{a}CC\emph{ʷ}]
\z
\z

The behavior of the \isi{labialization} feature can be explained by
phonotactic constraints. As stated in \sectref{syllable-and-word-structure}, if the coda is complex,
the first consonant is most frequently a sonorant, no complex onsets
are allowed, and clusters of sonorants in the same syllable\is{syllable structure} are not
attested. The rule in (\ref{ex:60}b) provides a resolution of unacceptable
consonant cluster (\emph{w}-sonorant-plosive).

The marker of the prohibitive and the negative optative
(\textsc{negvol}) \emph{m}(V)- has an unspecified vowel that, when
appearing before CVC or \textsc{cl}-VC roots, copies the vowel of the root (cf.\
(\ref{ex:61}–\ref{ex:63})):

\ea \label{ex:61} % (61)
\gll {m\textbf{u}-luč-adi}\\
\textsc{negvol}-read:\textsc{ipfv}-\textsc{proh}\\
\glt `don't read'

\ex \label{ex:62} % (62)
\gll {m\textbf{i}-d-ic'-adi}\\
\textsc{negvol}-\textsc{npl}-thaw:\textsc{ipfv}-\textsc{proh}\\
\glt  `don't thaw it'

\ex \label{ex:63} % (63)
\gll {m\textbf{a}-m-aš-adi-na~/m\textup{V}-b-aš-adi-na/}\\
\textsc{negvol}-\textsc{m}-walk:\textsc{ipfv}-\textsc{proh}-\textsc{pl}\\
\glt `don't go (to several people)'
\z

The gender marker \emph{b}- assimilates to the nasality of the
preceding \textsc{negvol} marker \emph{m}V-; cf.\ (\ref{ex:64}).

\ea \label{ex:64} % {(64a)}
\ea
\xbox{.4\textwidth}{\gll {mi-\textbf{d}-ilc-adi}\\
  \textsc{negvol}-\textsc{npl}-sell:\textsc{ipfv}-\textsc{proh}\\
  \glt `don't sell them (non-humans)'}%
\exsameline % {(64b)}
\xbox{.37\textwidth}{\gll {mi-\textbf{m}-ilc-adi}~/{m\textup{V}-\textbf{b}-ilc-adi}/\\
  \textsc{negvol}-\textsc{hpl}-sell:\textsc{ipfv}-\textsc{proh}\\
  \glt  `don't sell them (humans)'}\pagebreak[3]
\z
\z



The same segment in the verb root does not undergo assimilation:

\ea \label{ex:65} % (65)
\gll {m-i\textbf{b-}adi}~\textup{(*}{m-i\textbf{m-}adi}\textup{)}\\
\textsc{negvol}-sew:\textsc{ipfv}-\textsc{opt}\\
\glt `don't sew'
\z

There are some assimilations triggered by \emph{l} and involving
\emph{n} and \emph{l}. The sequences \emph{nl}V or \emph{ll}V in final
position can become \emph{w} or \emph{jj} after \emph{u} (\ref{ex:66}, \ref{ex:67}) and
\emph{jj} elsewhere (\ref{ex:68}, \ref{ex:69}).

\ea \label{ex:66} % (66)
\gll {xunu\textbf{w}a,~xunu\textbf{jj}a}~/{xunu\textbf{l-l}a}/\\
female-\textsc{gen}\\

\ex \label{ex:67} % (67)
\gll {buk'u\textbf{w}a,~buk'u\textbf{jj}a}~/{buk'u\textbf{n-l}a}/\\
shepherd-\textsc{gen}\\

\ex \label{ex:68} % (68)
\gll {t'a\textbf{jj}a}~/{t'a\textbf{l-l}a}/\\
pillar-\textsc{gen}\\

\ex \label{ex:69} % (69)
\gll {šaˤʜba\textbf{jj}a}~/{šaˤʜba\textbf{n-l}a}/\\
filbert-\textsc{gen}\\
\z

There is a correlation between the age of the speaker and the preferred
type of the alternation in nouns: older speakers tend to use the
\emph{w}-variant of the genitive, middle-aged speakers consider both
\emph{w}-variants and \emph{jj}-variants as well-formed, and younger
speakers tend to use the \emph{jj}-variant only. In the imperfective
converb, only \emph{w} is available for all speakers.

\ea \label{ex:70} % (70)
\gll {wik'u\textbf{w}e}~/{w-ik'-u\textbf{l-l}e}/\\
\textsc{m}-come:\textsc{ipfv}-\textsc{ptcp}.\textsc{cvb}\\
\glt `coming (\textsc{m})'

\ex \label{ex:71} % (71)
\gll {luč'u\textbf{w}e}~/{luč'-u\textbf{l-l}e}/\\
read:\textsc{ipfv}-\textsc{ptcp}.\textsc{cvb}\\
\glt `reading'
\z

In medial position, the sequences \emph{nli} or \emph{lli} become
\emph{j} and cause \isi{vowel deletion} (cf.\ (\ref{ex:72}–\ref{ex:75})):

\ea \label{ex:72} % (72)
\gll {xunu\textbf{j}ze}~/{xunu\textbf{l-l}i-ze}/\\
female-\textsc{obl}-\textsc{inter}[\textsc{lat}]\\

\ex \label{ex:73} % (73)
\gll {buk'u\textbf{j}ze}~/{buk'u\textbf{n-li-}ze}/\\
shepherd-\textsc{obl}-\textsc{inter}[\textsc{lat}]\\

\ex \label{ex:74} % (74)
\gll {t'a\textbf{j}ze}~/{t'a\textbf{l-l}i-ze}/\\
pillar-\textsc{obl}-\textsc{inter}[\textsc{lat}]\\

\ex \label{ex:75} % (75)
\gll {šaˤʜba\textbf{j}ze}~/{šaˤʜba\textbf{n-l}i-ze}/\\
filbert-\textsc{obl}-\textsc{inter}[\textsc{lat}]\\
\z

The sequences \emph{n}\textbf{V}\emph{l} or \emph{l}\textbf{V}\emph{l} after
\emph{u} show deletion of a medial vowel, which feeds the
\emph{nl}/\emph{ll} alternations above; cf.\ (\ref{ex:76}):

\ea \label{ex:76} % {(76a)}
\ea
\gll {hu\textbf{n}i}\\
  road\\
  
% {(76b)}
\ex  \gll {hu\textbf{j}zé}~/{hu\textbf{n}-\textbf{l}i-ze~←~hu\textbf{ní}-\textbf{l}i-ze}/\\
  road-\textsc{obl}-\textsc{inter}[\textsc{lat}]\\

% {(76c)}
\ex  \gll {hu\textbf{w}á}~/{hu\textbf{n}-\textbf{l}a~←~hu\textbf{ní}-\textbf{l}a}/\\
  road-\textsc{gen}\\
  \z
  \z

% \refstepcounter{equation}

When the clusters \emph{n\textup{V}l} or \emph{l\textup{V}l} follow any other
vowel, only an unstressed vowel can be deleted, and this deletion also
feeds the \emph{nl}/\emph{ll}/\emph{jj} alternation described above (cf.\
(\ref{ex:77}–\ref{ex:80})):

\ea \label{ex:77} % (77)
\gll {qarč'á\textbf{jj}a}~/{qarč'á\textbf{l}-\textbf{l}a~←~qarč'á\textbf{la}-\textbf{l}a}/\\
shoulder-\textsc{gen}\\

\ex \label{ex:78} % (78)
\gll {qarč'á\textbf{j}ze}~/{qarč'a\textbf{l}-\textbf{l}i-ze~←~qarč'á\textbf{la}-\textbf{l}i-ze}/\\
shoulder-\textsc{obl}-\textsc{inter}[\textsc{lat}]\\

\ex \label{ex:79} % (79)
\gll{ba\textbf{lá-l}a}~\textup{(*}{ba\textbf{jj}a}\textup{)}\\
wool-\textsc{gen}\\

\ex \label{ex:80} % (80)
\gll {ča\textbf{ná-l}a}~\textup{(*}{ča\textbf{jj}a}\textup{)}\\
sledge-\textsc{gen}\\
\z


There are some exceptions to the \isi{vowel deletion} rule, illustrated in
(\ref{ex:76}). While (\ref{ex:81}) shows non-deletion of a stressed vowel, in (\ref{ex:82}–\ref{ex:83})
the stressed vowel is deleted:

\ea \label{ex:81} % (81)
\gll {culála}\\
tooth-\textsc{gen}\\

\ex \label{ex:82} % {(82a)}
\ea \xbox{.31\textwidth}{\gll {šajjá}~/{ša\textbf{l}-\textbf{l}a~←~ša\textbf{lí}-\textbf{l}a}/\\
side-\textsc{gen}\\}%
\exsameline % {(82b)}
\xbox{.4\textwidth}{\gll {ša\textbf{j}zé}~/{ša\textbf{l}-\textbf{l}i-ze~←~ša\textbf{lí}-\textbf{l}i-ze}/\\
side-\textsc{obl}-\textsc{inter}[\textsc{lat}]\\}
\z

\ex \label{ex:83} % {(83a)}
\ea \xbox{.31\textwidth}{\gll {ejjá}~/{e\textbf{l}-\textbf{l}a~←~e\textbf{lí}-\textbf{l}a}/\\
child-\textsc{gen}\\}%
\exsameline % {(83b)}
\xbox{.4\textwidth}{\gll {e\textbf{j}zé}~/{e\textbf{l}-\textbf{l}i-ze~←~e\textbf{lí}-\textbf{l}i-ze}/\\
child-\textsc{obl}-\textsc{inter}[\textsc{lat}]\\}
\z
\z


Finally, \emph{r} can assimilate to \emph{n} and \emph{l} (cf.\
(\ref{ex:84}–\ref{ex:88})), including after applying \isi{vowel deletion} (cf.\ (\ref{ex:97}) and
(\ref{ex:98})), which then feeds the \emph{r}-assimilation.

\ea \label{ex:84} % (84)
\gll {qa\textbf{ll}ize}~/{qa\textbf{r-l}i-ze}/\\
sheepskin.coat-\textsc{obl}-\textsc{inter}[\textsc{lat}]\\

\ex \label{ex:85} % (85)
\gll {belč'u\textbf{nn}a}~/{b-elč'-u\textbf{n-r}a}/\\
\textsc{m}-read:\textsc{pfv}-\textsc{aor}-\textsc{ego}\\
\glt `I've read'

\ex \label{ex:86} % (86)
\gll {aħi\textbf{nn}a}~/{aħi\textbf{n-r}a}/\\
{be}:\textsc{neg}-\textsc{ego}\\

\ex \label{ex:87} % (87)
\gll {bata\textbf{ll}a}~/{batari-\textbf{l}a}/\\
wing-\textsc{gen}\\

\ex \label{ex:88} % (88)
\gll {bata\textbf{ll}ize}~/{batari-\textbf{l}i-ze}/\\
wing-\textsc{obl}-\textsc{inter}[\textsc{lat}]\\
\z

In some cases, this assimilation is optional (cf.\ (\ref{ex:89}–\ref{ex:91})):

\ea \label{ex:89} % (89)
\gll {qa\textbf{rl}á,~qa\textbf{ll}a}~/{qar-la}/\\
sheepskin.coat-\textsc{gen}\\

\ex \label{ex:90} % (90)
\gll {ši\textbf{nn}á,~ši\textbf{nr}á}~/{ši\textbf{n꞊r}a}/\\
water꞊\textsc{add}\\

\ex \label{ex:91} % (91)
\gll {t'u\textbf{ll}a,~t'u\textbf{lr}a}~/{t'u\textbf{l꞊r}a}/\\
finger꞊\textsc{add}\\
\z

The \emph{r}-assimilation would increase the number of forms to which
\emph{nl-} and \emph{ll-}mutations would apply. This does not happen,
however, so I postulate that \emph{r}-assimilation applies after
\emph{nl-}/\emph{ll-}mutations (counterfeeding order, see \citealt{kiparsky1968}): % (Kiparsky 1968))


\begin{table}[h]
  % Table 4.
  \caption{Interaction of the \emph{nl-}/\emph{ll-} mutation rule and the
\emph{r}- assimilation rule}\label{t3-4}

\begin{tabular}{@{}ll@{ }ll@{ }l@{}}
\toprule
& (\ref{ex:85}) & /\emph{belč'u\textbf{n-r}a/} & (\ref{ex:67}) &
\emph{/buk'u\textbf{n-l}a/}\tabularnewline \midrule
\emph{nl}- and \emph{ll}-mutation & & d. n. a. & &
\emph{buk'u\textbf{w}a, buk'u\textbf{jj}a}\tabularnewline
\emph{r}-assimilation & & \emph{belč'u\textbf{nn}a} & & not
applied\tabularnewline
\bottomrule
\end{tabular}
\end{table}

The rule for \isi{vowel deletion} between the consonants \emph{r}, \emph{l} or
\emph{n} can be generalized as follows:

%%% СЮДА НУЖНО ВСТАВИТЬ НОМЕР

\medskip
\centerline{\textbf{\isi{vowel deletion} rule}: V → ∅ / [+cons;+son;
DORSAL]\_\_[+cons;+son; \mbox{DORSAL}]}


\pagebreak
\medskip
\tabref{t3-5} summarises the rules discussed in this section.

\begin{table}
  % Table 5.
  \caption{Interaction of the \emph{nl-}/\emph{ll-} mutation rule, the \emph{r}-
assimilation rule and the \isi{vowel deletion} rule}\label{t3-5}

\small
\advance\tabcolsep-2pt
\begin{tabular}{@{}m{6em}<{\raggedright}l@{ }ll@{ }m{5em}<{\raggedright}l@{ }ll@{ }l@{}}
\toprule
& (\ref{ex:85}) & /\emph{belč'u\textbf{n-r}a}/ & (\ref{ex:67}) &
/\emph{buk'u\textbf{n-l}a}/ & (\ref{ex:87}) & /\emph{batari-\textbf{l}a}/ & (\ref{ex:76}c)
& /\emph{hu\textbf{ni-l}a}/\tabularnewline  \midrule
vowel deletion & & not applied & & not applied & &
\emph{batar\textbf{l}a} & & \emph{hu\textbf{nl}a}\tabularnewline  
\emph{nl}- and \emph{ll}- mutation &  & not applied  &  & \emph{buk'u\textbf{w}a, buk'u\textbf{jj}a} &  & not applied &  & 
\emph{hu\textbf{w}a}\tabularnewline 
\emph{r}-assimilation & & \emph{belč'u\textbf{nn}a} & & not applied & &
\emph{bata\textbf{ll}a} & & not applied\tabularnewline
\bottomrule
\end{tabular}
\end{table}

\is{sound alternation|)}


% 7. 
\section{Pharyngealization}\label{pharyngealization}

I suggest that \isi{pharyngealization} is a suprasegmental feature. By this I
mean that the \isi{pharyngealization} is not associated with a specific
consonant or vowel but with a whole syllable\is{syllable structure}; under certain conditions,
it may spread to other syllables\is{syllable structure}. I will mark the presence of the
pharyngeal\is{pharyngealization} feature on the nucleus of the syllable\is{syllable structure} by \emph{ˤ}.
Phonetically, \isi{pharyngealization} causes centering of vowels and
epiglottalization of the consonants \emph{ʔ} and \emph{ħ}:

\begin{table}[h]
  % Table 6
  \caption{Effect of pharyngeal\is{pharyngealization} feature on vowels and consonants}\label{t3-6}

  \advance\tabcolsep3pt
\begin{tabular}{@{}lllllll@{}}
\toprule
underlying segments & /iˤ/ & /eˤ/ & /aˤ/ & /uˤ/ & /ʔˤ/ &
/ħˤ/\tabularnewline
surface segments & [eˤ] & [ɛˤ] & [æˤ] & [uˤ], [oˤ] &
[ʡ] & [ʜ]\tabularnewline
\bottomrule
\end{tabular}
\end{table}

The evidence that the surface segment \emph{ʡ} and \emph{ʜ} are
underlyingly \emph{ʔ} and \emph{ħ} comes not only from the fact that the
latter segments do not co-occur with \isi{pharyngealization} (see note 2 above) but also from different realizations of
the same morphological segments in inflection and derivation. Consider
the following examples:


\ea \label{ex:92} % {(92a)}
\ea \xbox{.2\textwidth}{\gll {\textbf{u}ʡaˤ~\textup{\textless}}~/{ʔ\textbf{u}ʔaˤ/}\\
  cheese\\
  \glt `cheese'}%
\exsameline % {(92b)}
\xbox{.4\textwidth}{\gll {\textbf{ʡ}uˤʡ-ne~\textup{\textless}~/ʔ\textbf{u}ʔaˤ-ne/}\\
  cheese-\textsc{pl}\\
  \glt  `cheese (plural)'}
\z


\ex \label{ex:93} % (93)
\gll {\textbf{a}r-b-uχ-ib}\\
away-\textsc{n}-bring:\textsc{pfv}-\textsc{aor}\\
\glt `took it away'

\pagebreak[3]

\ex \label{ex:94} % (94)
\gll {\textbf{ʡaˤr}-d-aˤq'-un~\textup{\textless}~/ʔ\textbf{a}r-d-aˤq'-un/}\\
away-\textsc{f1}-go:\textsc{pfv}-\textsc{aor}\\
\glt
`she is gone'
\end{exe}

As stated in \sectref{consonants}, the glottal stop \emph{ʔ} in intervocalic and
initial position is often deleted. Glottal stops in initial and
intervocalic position can be deleted and appear only in the formal speech styles.
I stipulate that at the underlying level vowel initial morphemes
have the initial glottal stop. Examples (\ref{ex:92}–\ref{ex:93}) show that the
pharyngeal\is{pharyngealization} feature can spread backward, under which condition an
underlying \emph{ʔ} and \emph{ħ} become epiglottal and cease to be
affected by the deletion rule. This provides a uniform underlying
representation of the prefix, as shown in \tabref{t3-7}.

\begin{table}[h]
  % Table 7.
  \caption{Pharyngealization\is{pharyngealization} of underlying initial glottal stop}\label{t3-7}

\begin{tabular}{@{}lll@{}}
\toprule
& \emph{\textbf{/ʔa}r-b-uχ-ib/} &
\emph{\textbf{/ʔa}r-d-aˤq'-un/}\tabularnewline \midrule
{\isi{pharyngealization}} \textsc{spread} & not applied &
\emph{\textbf{ʡaˤ}r-d-aˤq'-un}\tabularnewline % \midrule
{deletion of initial ʔ} & \emph{\textbf{a}r-b-uχ-un} & not
applied\tabularnewline
& \emph{\textbf{a}r-b-uχ-un} &
\emph{\textbf{ʡaˤ}r-d-aˤq'-un}\tabularnewline
\bottomrule
\end{tabular}
\end{table}

In the first wordform, there is no lexical pharyngeal\is{pharyngealization} feature on the
root. Pharyngealization\is{pharyngealization} does not spread leftward and does not change the
underlying glottal stop to \emph{ʡ}; it can then be deleted. On the
contrary, in the second wordform, the lexical pharyngeal\is{pharyngealization} feature of the
root spreads leftwards and changes the glottal stop to epiglottal, which
cannot be dropped.

There is another argument for the \emph{ʔ}-to-\emph{ʡ} \isi{pharyngealization}
hypothesis. Examples of the sequences of the epiglottal \emph{ʡ} and
plain vowels are rare and seem to be detectable as Avar borrowings. This
interpretation creates some minimal pairs distinguished by the
pharyngeal\is{pharyngealization} feature alone (cf.\ (\ref{ex:95}–\ref{ex:98})):

\ea \label{ex:95} % (95)
{ʔe}\\
\glt `winter'\\

\ex \label{ex:96} % (96)
{ʡeˤ~\textup{\textless}~/ʔeˤ/}\\
\glt `summer'\\

\ex \label{ex:97} % (97)
\gll {d-irʔ-an}\\
\textsc{npl}-gather:\textsc{ipfv}-\textsc{hab}\\
\glt `gathers them'

\pagebreak[4]

\ex \label{ex:98} % (98)
\gll {d-irʡ-aˤn~\textup{\textless}~/d-irʔˤ-an/}\\
\textsc{npl}-freeze:\textsc{ipfv}-\textsc{hab}\\
\glt `they are freezing'
\z

Pharyngealization\is{pharyngealization} in (\ref{ex:98}) may be explained as a floating feature
(similarly to floating tone in \citealt{goldsmith1976}) %  Goldsmith 1976
that attaches to
the post-root syllable\is{syllable structure} of \emph{-irʔ}; the ending \emph{-an} becomes
pharyngealized\is{pharyngealization}.

Evidence for \emph{ħ} becoming \emph{ʜ} in a syllable\is{syllable structure} with the
pharyngeal\is{pharyngealization} feature is provided by the negation prefix \emph{ħa-} in
contexts of the pharyngeal\is{pharyngealization} feature spreading backward (cf.\ (\ref{ex:99}–\ref{ex:100})):

\ea \label{ex:99} % (99)
\gll {\textbf{ħa}}-{d-irʔ-an}\\
\textsc{neg}-\textsc{npl}-gather:\textsc{ipfv}-\textsc{hab}\\
\glt `does not gather them'

\ex \label{ex:100} % (100)
\gll {\textbf{ʜaˤ}-d}-{irʡ-aˤn}\\
\textsc{neg}-\textsc{npl}-freeze:\textsc{ipfv}-\textsc{hab}\\
\glt  `they are not freezing'
\z

In nouns, some of the plural CV-morphemes may delete the stem-final
vowel. If the deleted vowel is pharyngealized\is{pharyngealization}, the pharyngeal\is{pharyngealization} feature
moves to the previous syllable\is{syllable structure} (\ref{ex:101}–\ref{ex:103}):


\ea \label{ex:101}
  \ea  % {(101a)}
  \xbox{.12\textwidth}{\gll{č'\textbf{u}ʡaˤ}\\
  straw\\
  \glt `straw'}
%
  \exsameline %  {(101b)}
  \xbox{.3\textwidth}{\gll {č'\textbf{uˤ}ʡ-ne}\\
  straw:\textsc{pl}-\textsc{pl}\\
  \glt `straws'}
\z

\ex \label{ex:102} % {(102a)}
  \ea \xbox{.12\textwidth}{\gll {\textbf{u}ʡaˤ}\\
  cheese\\
  \glt `cheese'}
%  
\exsameline % {(102b)}
  \xbox{.3\textwidth}{\gll {ʡ\textbf{uˤ}ʡ-ne~/ʔ\textbf{u}ʔaˤ-ne/}\\
  cheese:\textsc{pl}-\textsc{pl}\\
  \glt `cheese (plural)'}
\z

\ex \label{ex:103} % {(103a)}
\ea  \xbox{.12\textwidth}{\gll {čiqʷ\textbf{aˤ}}\\
  bird\\
  \glt `bird'}
%  
\exsameline % {(103b)}
  \xbox{.3\textwidth}{\gll {č\textbf{iˤ}qʷ-ne}\\
  bird:\textsc{pl}-\textsc{pl}\\
  \glt `birds'}
\z
\z

I suggest that, in examples (\ref{ex:101}a), (\ref{ex:102}a) and (\ref{ex:103}a), only the second
syllable\is{syllable structure} of the underlying form is pharyngealized\is{pharyngealization}. In examples (\ref{ex:101}b),
(\ref{ex:102}b) and (\ref{ex:103}b), the plural morpheme deletes the nucleus of the
pharyngealized\is{pharyngealization} syllable\is{syllable structure}, and the feature spreads to the previous
syllable\is{syllable structure}. We thus observe \emph{ʡ} in examples (\ref{ex:101}b) and (\ref{ex:102}b).

Pharyngealization\is{pharyngealization} rules in Mehweb represent a complex phonological
phenomenon that requires further study. I will summarize its most
prominent properties:
\begin{enumerate}[topsep=\medskipamount,itemsep=0pt,partopsep=0pt,parsep=0pt]
\def\labelenumi{\arabic{enumi})}
\item
 the pharyngeal\is{pharyngealization} feature shows a strong association with uvular or
 epiglottal consonants, but also appears in some stems lacking these
 segments
\item
 acoustically, it is most visible on vowels adjacent to these
 consonants, but may spread backward as far as to the verbal prefixes
 (as in (\ref{ex:92}), (\ref{ex:94}) and (\ref{ex:100}))
\item
 all vowels can be pharyngealized\is{pharyngealization}, but \emph{iˤ} and \emph{eˤ} are
 extremely rare, and \emph{aˤ} is the most frequent
\item
 I treat \emph{ʡ} and \emph{ʜ} as realizations of \emph{ʔ} and \emph{ħ}
 in syllables\is{syllable structure} with the pharyngeal\is{pharyngealization} feature
\end{enumerate}

% 8. 
\section{Conclusion}\label{conclusion}

This paper explored phononological characteristics of Mehweb. The main
generalizations are as follows. Most plosives and affricates form
three-way oppositions (voiced vs. voiceless vs. ejective). There are
epiglottal consonants and pharyngealized\is{pharyngealization} vowels that can be described as
a result of the realization of suprasegmental pharyngeal\is{pharyngealization} feature.
The\is{syllable structure} majority of native Mehweb words can be
described as (C)V(C)(C). Nearly all polysyllabic forms have the \isi{stress}
on the second syllable\is{syllable structure}. To describe alternations (including vowel
deletion, \emph{nl}- and \emph{ll}- mutation rules and
\emph{r}-assimilation), I stipulate that \isi{vowel deletion} feeds all other
rules, and \emph{r}-assimilation counterfeeds \emph{nl-}/\emph{ll}-mutations.

\section*{Acknowledgements}

I would like to thank a number of people for reading this paper and
discussing its contents with me; in particular, Michael Daniel and
Alexandre Arkhipov, as well as to the anonymous reviewers of Language
Science Press.



\section*{List of abbreviations}


\begin{longtable}[l]{@{}ll@{}}
\textsc{add}	& additive particle \\
\textsc{aor}	& aorist \\
\textsc{cl}	& gender (class) agreement slot \\
\textsc{cvb}	& converb \\
\textsc{ego}	& egophoric \\
\textsc{erg}	& ergative \\
\textsc{ess}	& static location in a spatial domain \\
\textsc{f}	& feminine (gender agreement) \\
\textsc{f1}	& feminine (unmarried and young women gender prefix) \\
\textsc{fut}	& future \\
\textsc{gen}	& genitive \\
\textsc{hab}	& habitual (durative for verbs denoting states) \\
\textsc{hpl}	& human plural (gender agreement) \\
\textsc{imp}	& imperative \\
\textsc{inf}	& infinitive \\
\textsc{inter}	& spatial domain between multiple landmarks \\
\textsc{ipft}	& imperfect \\
\textsc{ipfv}	& imperfective (derivational base) \\
\textsc{lat}	& motion into a spatial domain \\
\textsc{m}	& masculine (gender agreement) \\
\textsc{n}	& neuter (gender agreement) \\
\textsc{neg}	& negation (verbal prefix) \\
\textsc{negvol}	& negation in volitional forms (negative imperative, negative optative) \\
\textsc{nmlz}	& nominalizer \\
\textsc{npl}	& non-human plural (gender agreement) \\
\textsc{obl}	& oblique (nominal stem suffix) \\
\textsc{opt}	& optative \\
\textsc{pfv}	& perfective (derivational base) \\
\textsc{pl}	& plural \\
\textsc{proh}	& prohibitive \\
\textsc{ptcp}	& participle \\
\textsc{pv}	& preverb (verbal prefix) \\
\textsc{sg}	& singular \\
\textsc{super}	& spatial domain on the horizontal surface of the landmark \\
\textsc{tr}	& transitive \\
\textsc{voc}	& form of address \\
\end{longtable}


\printbibliography[heading=subbibliography,notkeyword=this]

\clearpage

% \section*{References}

% Kiparsky, P. (1968). Linguistic universals and linguistic change. In E.
% Bach and R. Harms (eds), Universals in linguistic theory. New York:
% Holt, 170–202.

% Goldsmith, John. 1976. Autosegmental Phonology. Cambridge: MIT, PhD.
% dissertation. Distributed by IULe.

% Ladefoged, P., Maddieson, I. (1996). The sounds of the world's
% languages. \emph{Malden, MA (USA): Blackwell Publishing}.

% Magomedov, A. A. (1982) Megebskiy dialekt darginskogo jazika. Tbilisi:
% Mecniereba.

\end{document}

%%% Local Variables:
%%% mode: latex
%%% TeX-master: "../main"
%%% End:
