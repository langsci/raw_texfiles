\addchap{Preface}

\markboth{Preface}{Preface}

\begin{refsection}
This volume presents several papers on Mehweb, a one-village language
spoken in the central part of Daghestan, a republic of the Russian
Federation. The language has a relatively low number of speakers (about
800) but is not immediately endangered, as shown in the first
contribution by Nina Dobrushina, which is an introduction to the
sociolinguistic situation of Mehweb. The contribution covers the
geographical position of Mehweb and its economic situation, the official
status of the language, the ethnic affiliation of the villagers, the recent
history of Mehweb, its neighbours and the patterns of multilingualism observed.
While there are no visible signs of first language loss, the paper shows
that there is a strong tendency towards the loss of traditional patterns
of multilingualism, with Russian replacing all other languages for
interethnic communication.

Mehweb belongs to the Dargwa branch of the East Caucasian
(Nakh-Daghe\-stanian) language family. It is often considered as a dialect
of Dargwa \citep{magometov1982}, along with many other lects within the
Dargwa branch. A different tradition treats Mehweb as a separate
language \citep{khajdakov1985, koryakov-sumbatova2007}. The survey of
Dargwa idioms in \citet{sumbatova-lander2014} indicates that Mehweb is most
often classified as belonging to the northern group of Dargwa dialects.
Although the residents of Mehweb presently consider themselves to be the
descendants of re-settlers from the village of Mugi, where the Akusha
dialect of Dargwa is spoken (\citealt{uslar1892}; see also \citealt{dobrushina2019} [this volume]), there is no linguistic analysis that shows any special affinity
between Mehweb and Mugi. According to lexicostatistical analysis, Mehweb
is a member of the north-central group of Dargwa and shows more
similarities to Murego-Gubden than to Mugi \citep{koryakov2013}.

The first linguistic source on Mehweb is a reference by \citet{uslar1892}.
This short grammar describes another dialect of Dargwa, but
starts with a brief survey of different Dargwa languages and dialects.
Among these dialects Uslar also mentions Mehweb, qualifying it as a
dialect spoken in Mugi, but ``notably degraded''. Two descriptions of
Mehweb appeared in the 1980s, both in Russian. The first is a
grammar of Mehweb which describes its phonology and morphology but not
its syntax \citep{magometov1982}. This description, extremely clear and explicit,
considers only the main morphological forms while excluding some less frequent ones,
and does not provide a detailed analysis of their semantics. The second, 
a book by Khaidakov, was written at almost the same time as
Magometov's grammar. It compares the formal morphology of several Dargwa
languages and dialects, including Mehweb.

In 1990, a field team from Moscow State University came to work
on Mehweb, but no publications followed. In the aftermath of this trip,
in the 2000s, Nina Sumbatova started to work on Mehweb and, among other
things, compiled a list of glosses and suggested an analysis of Mehweb verbal
inflection, some elements of which are integrated into this volume
(Sumbatova manuscript).

The only dictionary of Mehweb which exists to date is a small vocabulary
supplement in \citet{magometov1982}. One of the aims of our study was to
compile a dictionary and document the main inflectional forms. The
dictionary is being developed by Michael Daniel with the participation of
many members of the field team, especially George Moroz, and implemented
as a web page by Aleksandra Kozhukhar. The current version of the
dictionary is available online – \url{https://linghub.ru/mehwebdict/}.

Mehweb texts were first published by Magometov (ibid.) with translation,
but without morphological glossing. New texts
were recorded and glossed during this project by Michael Daniel,
including a sample of Pear Stories (currently transcribed but not yet
glossed). The corpus includes 35 texts (including 13 from \citealt{magometov1982})
comprising about 1,000 sentences and 10,000 tokens and is also being
prepared for open access.

The following brief overview is intended for the reader who is not
familiar with East Caucasian languages. It provides background on the most
important features of the language.

The consonant inventory includes voiced and voiceless consonants. Stops
(but not other consonants) also have an ejective series. Unlike some
other Dargwa dialects, Mehweb lacks phonologically distinctive geminate
stops. The vocalic system has four members with a gap in the mid back
position [i, e, a, u]; [oˤ] only appears as a realization of
[u] with the pharyngeal feature. Velar, uvular and radical consonants
may be labialized. In addition to [ʔ, ħ, h], Mehweb also has the less
common [ʜ, ʡ] which seem to be phonologically secondary, appearing
only as pharyngealized counterparts of [ħ, ʔ], respectively.
Pharyngealization is strongly~– but not exclusively – associated with
uvulars, pharyngeal and laryngeal consonants. For further details on
phonetic inventories and pharyngealization see the contribution by
George Moroz, who discusses details of the inventory, syllable
structure, stress placement, morphophonological alternations and
pharyngealization.

\largerpage

Mehweb morphology is agglutinative. Mehweb is ergative in terms of both gender
agreement and case marking. To start with the latter, the case inventory
includes the nominative (absolutive), the ergative, the genitive, the
dative, the comitative and some peripheral case forms. Note that some
authors of the volume follow \citet{kibrik1997} in the use of the term
\emph{nominative} for ergative alignment. Spatial forms are bimorphemic, as
is typical of East Caucasian. The first category is that of localization,
defining a spatial domain with respect to the ground (in Mehweb: `on',
`near', `at', `in(side)', `among'). The second category is that of
orientation, defining the figure's motion with respect to this domain
(Goal, Source, Path) or absence thereof (Static location). Unlike other
branches of East Caucasian – but as in the other lects of the
Dargwa branch – the lative form (Goal) is zero marked and the essive
form (location) is marked by the presence of a gender agreement slot
controlled by the nominative argument of the clause. The plural is expressed
by a number of suffixes, sometimes accompanied by alternations. For more
on nominal morphology, see the contribution by Ilya Chechuro, dealing
with plural formation, the oblique stem, case formation and formation of
irregular locatives. There is also a brief discussion of the use of the
case forms.

Mehweb verb inflection is by and large similar to that of other Dargwa
languages. It resides upon a fundamental distinction between two stems,
perfective and imperfective, from which all other forms are derived. The
formal relation between the stems is irregular and can involve
alternations, infixation and loss of agreement slots. Most
forms are derived from both perfective and imperfective stems, except
the prohibitive and the present/habitual, which are only available in the
imperfective. The combination of the irregular relation between perfective
and imperfective stems and the almost perfectly parallel inflection
based on the two stems partly assimilates the Mehweb (and generally
Dargwa) aspectual system to that of derivational aspect. Irregular verbs
include verbs of motion, the verb `give', the verb `say' and some others.
For more on verbal morphology, see the contribution by Michael Daniel.

\largerpage

Zooming in on one fragment of the verb morphology, Nina Dobrushina
provides a detailed analysis of both form and meaning in the irrealis
domain. Several features are typologically infrequent, although common
for East Caucasian languages: the formal split between transitive and
intransitive imperatives, the expression of the negative imperative by a
dedicated inflectional form (the prohibitive), and the presence of a dedicated
inflectional optative used in blessings and curses. The presence of a
dedicated apprehensive is rare even within East Caucasian. The jussive
and the hortative are expressed periphrastically. A detailed analysis of
another fragment of verbal morphosyntax is provided in the contribution by Daria
Barylnikova. She provides a survey of periphrastic constructions based
on `drive' and `let' and explains the ways in which these constructions
show incipient signs of grammaticalization into expressions of factitive and permissive
causation, respectively.

Gender agreement in Mehweb follows strict semantic assignment:
in the vast majority of cases, it is enough to know the semantics of the
noun to determine its agreement pattern. Mehweb gender (class) agreement
distinguishes masculine, feminine and neuter in the singular and human
and non-human in the plural. One complication is connected to mass
nouns; although morphologically singular (and capable of forming
morphological plurals), these nouns control non-human plural agreement.
While this behavior of mass nouns is typical of Dargwa languages, the
next twist is an innovation and probably results from contact with Lak.
The majority of feminine nouns have moved from the original Dargwa
feminine (\emph{r-}, glossed as \textsc{f} in the book) gender to a gender
identical to non-human plural (\emph{d-}, glossed as \textsc{f1}). The
distribution is roughly between married/old~(\textsc{f}) and unmarried/young (\textsc{f1})
women. The choice between the two agreement patterns is still partly
flexible and may become a tool of language game or insults. One could
speculate that the source of this development is some kind of indirect
reference motivated by politeness. Another development in agreement is
that personal agreement on the verb, well attested in Dargwa languages,
developed into the typologically rare phenomenon of egophoric agreement;
the suffix \emph{-ra} (glossed \textsc{ego}) appears with first
person subject in the affirmative and with second person subject in
the interrogative. Unlike gender, personal agreement works on
an accusative rather than an  ergative basis.

Clause subordination is based on dependent verb forms, including action
nominals, infinitives, participles and converbs, rather than on finite
predication introduced by conjunctions. Converbs include two general (contextual)
converbs (perfective and imperfective) whose relation to the main clause
is context-determined and several special converbs that specify this relation
(in Mehweb, immediate anteriority, gradual accumulation, cause,
concession etc.~– see the contribution by Maria Sheyanova). Some
aspects of the syntax of general converbs are presented in the
contribution by Marina Kustova, who covers periphrastic converbs,
independent uses of converbs and their use in imperative contexts, and
different strategies for how the converb clause may share its arguments
with the main clause. In the absence of true clause co-ordination, the
respective discourse/narrative function is performed by chains of
general converbs. Kustova's contribution attempts to address this issue by
considering several tests targeting the subordination – co-ordination distinction.

One apparent exception to the non-use of finite predication in
subordination is constituted by reported speech constructions. Reported speech in
Mehweb, as generally in East Caucasian, is structurally similar to
direct reporting and typologically distant from true subordination.
Mehweb has a pronominal stem \emph{sa}‹\textsc{cl}›\emph{i}, used with a wide
range of functions, from logophoric function in reported speech to resumptive to
reflexive, considered in the contribution by Aleksandra Kozhukhar. The
author suggests that, in Mehweb, there is neither a morphological nor
a (sharp) syntactic distinction between logophoric and long-distance uses
of the pronoun.

The three other contributions on syntax are the chapters by Dmitry
Ganenkov (syntax – case assignment and personal agreement – of the
simple clause), Yuri Lander and Aleksandra Kozhukhar (the relative clause)
and Yuri Lander (a survey of the uses of the focus particle). Ganenkov
shows how the distribution of personal and gender agreement control
classifies Mehweb verbs into several morphosyntactic classes,
non-trivially connected to their transitivity, and demonstrates how this
distribution is linked to conventional subject properties such as
control of reflexivization. Lander and Kozhukhar argue that the use of
the reflexive pronoun has been specialized as resumptive in relative
clauses, taking as evidence the restrictions on its use as compared to
the use of simple reflexives. Finally, Lander argues that the focus
particle \emph{gʷa}, formally identical to the imperative of
`see', surprisingly does not have to be adjacent to the constituent in
the scope of the focus.

\medskip
\centerline{*\ *\ *}

\medskip
This volume is the result of a collective field research project run by the
linguists from the School of Linguistics of HSE University, Moscow. Part of the team consisted of bachelor's
students who conducted their research under the supervision of the more
experienced members of the team. Collective field research is a practice
developed by Aleksandr Kibrik, an eminent Russian typologist who
organized more than 40 field trips attracting hundreds of young people
to the description of minority languages. Kibrik edited numerous grammars
where chapters were contributed by all team participants, including
students in their early years at university.

In 1990, Aleksandr Kibrik brought to Mehweb a large group of students
which included, among others, Michael Daniel and Nina Dobrushina. This specific field trip produced relatively little in terms of scholarly output, the most important result being a three-page sketch of Mehweb morphology (a list of the major forms
and morphemes) by Nina Sumbatova.

The more important legacy of the 1990 expedition was a personal/human one. Anvar Musaev and Maisarat
Muslimova (now Musaeva), two teachers at the local school, took an active part in the
organization of the life of the expedition. A long-lasting
human bond was established with them. In 2010, Michael Daniel and Nina
Dobrushina decided to pass by Mehweb on their way from Archib to
Makhachkala. Anvar and Maisarat, this time a married couple with
grown-up children, were so open and hospitable, and so full of memories
of the 1990 visit, that the idea of working on Mehweb came very
naturally. In 2013, five students from the Higher School of Economics
accompanied by Michael Daniel, Nina Dobrushina, Dmitry Ganenkov, Yuri
Lander and George Moroz came to Mehweb to start working on a description
of the language. In the course of four field trips in 2013, 2014, 2015 and 2016, each
lasting about two weeks, we recorded texts, compiled a small
dictionary, and wrote several papers. The student team was not always
the same. Some of the students involved did not participate directly in this volume, 
but they all made a contribution to the analysis of the data. It is thus our
pleasure to list the participants of all field trips over these four
years: Ekaterina Ageeva, Darya Barylnikova, Ilya Chechuro, Violetta
Ivanova, Aleksandra Khadzhijskaya, Aleksandra Kozhukhar, Marina Kustova,
Yevgeniy Mozhaev, Olga Shapovalova, Semen Sheshenin, Aleksandra
Sheshenina, Mariya Sheyanova.

Anvar and Maisarat and their family invariably provided us with housing
and logistical support and never grew tired of being our primary native
consultants, including over email, Skype and now, in the final days of
our work on the book, also over WhatsApp, a very useful tool for instant
proofreading of examples. We are also infinitely grateful to our friends
and consultants Abakar and Zalmu Sharbuzovy, to their daughters Patimat
and Kamila, so intelligent and helpful, to the indefatigable Kazim, foe
of all tea parties, his wife Munira and his sister Bulbul; to Patimat
Tagirovna, who deserves to become the first announcer on Mehweb radio, if it
is ever established; to Khavsarat, Magomedzagid, Mariam and many
other Mehweb people the limits of whose patience we have been stretching for
too many years. We remember the touch of the hand of Aminat, Maisarat's
mother.

The authors are very grateful to Samira (Helena) Verhees who proofread
most of the papers presented here, to our very patient type-setter, Vadim Radionov, and to the reviewers of drafts of individual chapters of the volume: Aleksandr Arkhipov, Gilles Authier, Oleg Belyaev, Denis
Creissels, Francesca Di Garbo, Diana Forker, Martin Haspelmath, Olesya
Khanina, Timur Maisak, Nina Sumbatova, Yakov Testelets, as well as to
the anonymous reviewers of Language Science Press.

This volume was prepared within the framework of~the Academic Fund
Program at~the National Research University Higher School of~Economics
(HSE) in~2015–2016 (grant \#15-05-0021) and by~the Russian Academic
Excellence Project «5-100».

% \medskip
\rightline{\em Michael Daniel and Nina Dobrushina}


\printbibliography[heading=subbibliography]

\end{refsection}

\clearpage

\begin{russian}

\addchap{Предисловие}

\begin{refsection}

\emergencystretch1em

Настоящий сборник – результат многолетней работы исследовательской
группы Школы лингвистики Национального исследовательского университета
Высшая школа экономики. В проекте приняли участие студенты бакалавриата,
которыми руководили более опытные исследователи. Коллективная полевая
работа – практика, введенная Александром Евгеньевичем Кибриком,
выдающимся советским российским типологом, организовавшим более сорока
лингвистических экспедиций, в ходе которых в полевых исследованиях малых
языков приняли участие сотни студентов. А.Е.~Кибрик выпустил большое
число грамматик, главы которых писались в том числе студентами, лишь
недавно начавшими учебу в университете.


В 1990~г. А.Е.~Кибрик привез в селение Мегеб (Гунибский район Республики
Дагестан) большую группу студентов, участниками которой были, в том
числе, М.~Даниэль и Н.~Добрушина. От этой поездки сохранилось не так
много материалов. Важным результатом стал краткий обзор мегебской
морфологии (список основных форм и морфем), составленный Н.~Сумбатовой.

С точки зрения человеческих отношений самым главным приобретением
экспедиции 1990~г. стало знакомство с Анваром Мусаевым и Майсарат
Муслимовой (ныне Мусаевой), молодыми учителями мегебской школы, которые
приняли активное участие в жизни экспедиции. В 2010~г. мы (Н.~Добрушина
и М.~Даниэль) решили заехать в Мегеб на обратной дороге из Чародинского
района в Махачкалу. Майсарат и Анвар, к этому времени – семейная пара с
двумя взрослыми детьми, приняли нас настолько радостно и тепло, были так
полны воспоминаниями о той давней поездке, что идея возобновить работу
над мегебским языком показалась совершенно естественной и даже
неизбежной. В 2013~г. пять студентов ВШЭ под руководством М.~Даниэля, 
Н.~Добрушиной, Д.~Ганенкова, Ю.~Ландера и Г.~Мороза приехали в Мегеб для
работы над грамматикой этого языка. В результате четырех поездок
(2013–2016~гг.), каждая продолжительностью около двух недель, мы
записали некоторое количество текстов, собрали небольшой словарь и
написали несколько черновых статей. Студенческий состав не оставался
постоянным. Некоторые из участников этих экспедиций не приняли участие в
написании настоящего очерка, но каждый из них внес тот или иной вклад в
сбор и анализ данных. Мы приводим полный список участников всех
экспедиций: Екатерина Агеева, Дарья Барыльникова, Виолетта Иванова,
Александра Кожухарь, Марина Кустова, Евгений Можаев, Александра
Хаджийская, Илья Чечуро, Ольга Шаповалова, Семен Шешенин, Александра
Шешенина, Мария Шеянова.

Нашими неизменными хозяевами и главными переводчиками были Майсарат и
Анвар. Они и их сыновья обустраивали нашу жизнь и неутомимо отвечали на
наши вопросы о мегебском языке, в том числе по электронной почте,
скайпу, а в последнее время – по WhatsApp'у, совершенно незаменимому
инструменту для того, чтобы в последний момент вносить правку в
корректуру статей по малым языкам. Кроме того, мы бесконечно благодарны
нашим друзьям и переводчикам – Абакару и Залму Шарбузовым, их дочерям
Патимат и Камиле, таким умным и всегда готовым поделиться своим
временем, неутомимому чаененавистнику Казиму, его жене Мунире и его
сестре Булбул; Патимат Тагировне, которая несомненно заслуживала бы роли
первого диктора мегебского радио, если таковое когда-нибудь начнет
вещание; Исрапилу, Кавсарат, Магомедзагиду, Марьям, Саиде и многим
другим мегебцам, границы терпения которых мы испытывали в течение
стольких лет. Всем им мы желаем долгих лет жизни и здоровья. 

Мы помним рукопожатие Аминат, мамы Майсарат, Муниры и Марьям. 

Авторы сборника очень признательны Самире (Хелене) Ферхеес, которая
вычитала многие из статей, Вадиму Радионову, который взял на себя сложную верстку тома, рецензентам первых версий статей –
Александру Архипову, Жилю Отье, Олегу Беляеву, Дени Кресселю, Франческе
Ди Гарбо, Диане Форкер, Мартину Хаспельмату, Олесе Ханиной, Тимуру
Майсаку, Нине Сумбатовой, Якову Тестельцу, а также анонимным рецензентам
издательства Language Science Press.

Сборник был подготовлен в~ходе проведения исследования №~15-05-0021
в~рамках Программы «Научный фонд Национального исследовательского
университета „Высшая школа экономики“ (НИУ ВШЭ)» в~2015–2016~гг.
и~в~рамках государственной поддержки ведущих университетов Российской
Федерации «5-100».

\medskip
\rightline{\em Михаил Даниэль и Нина Добрушина}
\end{refsection}
\end{russian}

%%% Local Variables:
%%% mode: latex
%%% TeX-master: "../main"
%%% End:
