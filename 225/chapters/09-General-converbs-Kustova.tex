\documentclass[output=paper]{langsci/langscibook} 
\ChapterDOI{10.5281/zenodo.3402070}

% Chapter 9

\author{Marina Kustova\affiliation{National Research University Higher School of Economics}}

\title{General converbs in Mehweb}

\abstract{This paper deals with the morphological and syntactic properties of general converbs in Mehweb. I discuss the markers used to form general converbs, periphrastic converbs, independent uses of converbs, their behaviour in combination with verbs in the imperative, different strategies of argument sharing between the converb clause and the main clause, and coordination/subordination properties of the general converb. The description of the syntactic properties of the converbs is based on both elicited examples and corpus evidence.}

\begin{document}
\maketitle

% 1.
\section{Introduction}

Converbs are determined as nonfinite verb forms whose main function 
is to mark adverbial subordination (\citealt[3]{haspelmath1995a}). 
Mehweb specialized converbs, i.e. converbs which specify
the semantic relation between the main and the converb clause (e.g.\
causal, immediate precedence in time, other temporal relations and so on), are discussed in
 \citet{sheyanova2019} [this volume]. This paper is devoted to general converbs
which do not specify this relation – or, at least, do it in a more
subtle way, leaving some room for contextual interpretation (hence an alternative label for this category is \emph{contextual} converbs).

In \sectref{perfective-and-imperfective-converbs}, the basic uses and morphology of perfective and
imperfective converbs will be discussed, \sectref{periphrastic-converbs} describes
periphrastic converbs, and \sectref{independent-use} deals with independent uses of
general converbs in Mehweb. \sectref{argument-sharing} discusses different patterns 
of argument sharing between converb clauses and main clauses. 
Finally, in \sectref{coordination-and-subordination-properties} I discuss the coordination and
subordination properties of the Mehweb general converb.

% 2.
\section{Perfective and imperfective converbs: background information}\label{perfective-and-imperfective-converbs}

General converbs in Mehweb are derived from perfective and
imperfective stems. Below I will refer to them as perfective and
imperfective converbs, respectively. The perfective converb is formed by
adding the converb marker \emph{-le} to the verb in the aorist
\citep[110]{magometov1982}; the affix undergoes a number of morphophonological alternations
(see \citealt{moroz2019}; \citealt{daniel2019}). The formation of
perfective converbs is presented in \tabref{tab:9:1}.


\begin{table}[h]
\vskip-\jot
  % Tab. 1.
\caption{The formation of the perfective converb}\label{tab:9:1}
\small 
\begin{tabular}{@{}m{.11\textwidth}<{\raggedright}m{.25\textwidth}<{\raggedright}m{.25\textwidth}<{\raggedright}m{.28\textwidth}<{\raggedright}@{}}
\toprule
~ & 1\textsuperscript{st} conjugation class & 2\textsuperscript{nd}
conjugation class & 3\textsuperscript{rd} conjugation
class\tabularnewline \midrule 
Aorist & \emph{b-at-ur}

\textsc{n}-leave:\textsc{pfv}-\textsc{aor}

`left' & \emph{b-ic-ib}

\textsc{n}-sell:\textsc{pfv}-\textsc{aor}

`sold' & \emph{b-elč'-un}

\textsc{n}-read:\textsc{pfv}-\textsc{aor}

`read'\tabularnewline \midrule
Perfective converb & \emph{b-at-ul-le} (\textless{}\emph{b-at-ur-le})

\textsc{n}-leave:\textsc{pfv}-\textsc{aor}-\textsc{cvb}

`having left' & \emph{b-ic-i-le} (\textless{}\emph{b-ic-ib-le})

\textsc{n}-sell:\textsc{pfv}-\textsc{aor}-\textsc{cvb}

`having sold' & \emph{b-elč'-uwe} (\textless{}\emph{b-elč'-ul-le})

\textsc{n}-read:\textsc{pfv}-\textsc{aor}-\textsc{cvb}

`having read'\tabularnewline
\bottomrule
\end{tabular}


\vskip-\jot
\end{table}


The imperfective converb is formed by adding \emph{-uwe} to the
imperfective stem. Here, the process is the same for all conjugation classes and
could be interpreted as a combination of the participle suffix
\emph{-ul} and the converb suffix \emph{-le} \citep[112]{magometov1982}. The
formation of imperfective converbs is presented in \tabref{tab:9:2}.

\begin{table}[h]
\vskip-\jot

  % Tab. 2.
  \caption{The formation of the imperfective converb}\label{tab:9:2}

  \small
  \begin{tabular}{@{}m{.14\textwidth}<{\raggedright}m{.28\textwidth}<{\raggedright}m{.24\textwidth}<{\raggedright}m{.24\textwidth}<{\raggedright}@{}}
\toprule
  ~ & 1\textsuperscript{st} conjugation class & 2\textsuperscript{nd} conjugation class & 3\textsuperscript{rd} conjugation class\tabularnewline \midrule 
Present participle & \emph{b-alt-es}

\textsc{n}-leave:\textsc{ipfv}-\textsc{inf}

`leaving' & \emph{b-ilc-es}

\textsc{n}-sell:\textsc{ipfv}-\textsc{inf}

`selling' & \emph{luč'-es}

\textsc{n}-read:\textsc{ipfv}-\textsc{inf}

`reading'\tabularnewline \midrule
Imperfective converb & \emph{b-alt-uwe}

\textsc{n}-leave:\textsc{ipfv}-\textsc{cvb.ipfv}

`(while) leaving' & \emph{b-ilc-uwe}

\textsc{n}-sell:\textsc{ipfv}-\textsc{cvb.ipfv}

`(while) selling' & \emph{luč'-uwe}

read:\textsc{ipfv}-\textsc{cvb.ipfv}

`(while) reading'\tabularnewline
\bottomrule
  \end{tabular}

\vskip-\jot
\end{table}

The perfective converb is used to describe an event preceding the
situation denoted in the main clause. Situations that take place
simultaneously with the main event are described by the imperfective
converb. Both imperfective and perfective converbs can be combined with
finite verbs with present or past time reference, cf.:

\pagebreak[3]

\ea \label{ex:9:1} % (1)
\gll {deč'꞊ra b-aq'-i-le musa w-aˤq'-un quli.}\\
song꞊\textsc{add} \textsc{n}-do:\textsc{pfv}-\textsc{aor}-\textsc{cvb} Musa \textsc{m}-go:\textsc{pfv}-\textsc{aor} house(\textsc{lat})\\
\glt `Having sung a song, Musa went home.'

\ex \label{ex:9:2} % (2)
\gll {deč'꞊ra b-iq'-uwe musa w-aˤq'-un quli.}\\
song꞊\textsc{add} \textsc{n}-do:\textsc{ipfv}-\textsc{cvb.ipfv} Musa \textsc{m}-go:\textsc{pfv}-\textsc{aor} house(\textsc{lat})\\
\glt `Singing a song, Musa went home.'

\ex \label{ex:9:3} % (3)
\gll {deč'꞊ra b-aq'-i-le musa ʡaˤr-q'-uwe le-w quli.}\\
song꞊\textsc{add} \textsc{n}-do:\textsc{pfv}-\textsc{aor}-\textsc{cvb} Musa away-go:\textsc{ipfv}-\textsc{cvb.ipfv} \textsc{aux}-\textsc{m} house(\textsc{lat})\\
\glt `Having sung a song, Musa is going home.'

\ex \label{ex:9:4} % (4)
\gll {deč'꞊ra b-iq'-uwe musa ʡaˤr-q'-uwe le-w quli.}\\
song꞊\textsc{add} \textsc{n}-do:\textsc{ipfv}-\textsc{cvb.ipfv} Musa away-go:\textsc{ipfv}-\textsc{cvb.ipfv} \textsc{aux}-\textsc{m} house(\textsc{lat})\\
\glt `Singing a song, Musa is going home.'
\z

In sentence (\ref{ex:9:1}), a perfective converb is combined with a finite verb in
the aorist, in (\ref{ex:9:2}) an imperfective converb is combined with a verb in
the aorist, in (\ref{ex:9:3}) a perfective converb is combined with a verb in the present
tense, and in (\ref{ex:9:4}) an imperfective converb is combined with a verb in the
present tense.

% 3.
\section{Periphrastic converbs}\label{periphrastic-converbs}

\is{converb, periphrastic|(}

Apart from the perfective and imperfective converbs described above,
most speakers of Mehweb allow forms consisting of a converb and a copula
in the converb form. Essentially, these are converbs formed from
periphrastic verb forms. Below I refer to such forms as periphrastic
converbs.

A periphrastic converb consisting of a perfective converb and a
copula in the converb form corresponds to the resultative, a finite
periphrastic form consisting of a perfective converb and a tensed copula.

% \pagebreak

\ea \label{ex:9:5} % (5)
\gll {jaˤbu b-ic-i-le le-b-le maˤʜmud-ini χʷe as-ib.}\\
horse \textsc{n}-sell:\textsc{pfv}-\textsc{aor}-\textsc{cvb} \textsc{aux}-\textsc{n}-\textsc{cvb} Mahmud-\textsc{erg} dog take:\textsc{pfv}-\textsc{aor}\\
\glt `Having sold a horse, Mahmud bought a dog.'
\z

The same construction with an imperfective converb corresponds to the
present progressive, which \citet[87]{magometov1982} terms definite imperfect.

\ea \label{ex:9:6} % (6)
\gll {jaˤbu b-ilc-uwe le-b-le maˤʜmud le-w w-is-uwe.}\\
horse \textsc{n}-sell:\textsc{ipfv}-\textsc{cvb.ipfv} \textsc{aux}-\textsc{n}-\textsc{cvb} Mahmud \textsc{aux}-\textsc{m} \textsc{m}-weep:\textsc{ipfv}-\textsc{cvb.ipfv}\\
\glt `While selling a horse, Mahmud is crying.'
\z

Speakers also allow sentences like (\ref{ex:9:7}) and (\ref{ex:9:8}), where the copula in
the converb form is preceded by a perfective or an imperfective
infinitive. Morphologically, these forms correspond to the future
\isi{resultative} (composed of a perfective converb and a copula in the
converb form) and the future \isi{progressive} (an imperfective converb and a
copula in the converb form). The semantic difference between the two
periphrastic converbs remains unclear.

\ea \label{ex:9:7} % (7)
\gll {jaˤbu b-ic-es le-b-le maˤʜmud-ini χʷe as-ib.}\\
horse \textsc{n}-sell:\textsc{pfv}-\textsc{inf} \textsc{aux}-\textsc{n}-\textsc{cvb} mahmud-\textsc{erg} dog take:\textsc{pfv}-\textsc{aor}\\
\glt `Going to sell a horse, Mahmud bought a dog.'

\ex \label{ex:9:8} % (8)
\gll {jaˤbu b-ilc-es le-b-le maˤʜmud le-w w-is-uwe.}\\
horse \textsc{n}-sell:\textsc{ipfv}-\textsc{inf} \textsc{aux}-\textsc{n}-\textsc{cvb} mahmud \textsc{aux}-\textsc{m} \textsc{m}-weep-\textsc{cvb}\\
\glt `Going to sell a horse, Mahmud is crying.'
\z

\removelastskip
\is{converb, periphrastic|)}

% 4.
\section{Independent use}\label{independent-use}

In most cases, converbs are used in complex clauses that also contain
main finite clauses. However, some speakers allow sentences that contain
only converbal predication.

When used independently, the perfective converb can have \isi{resultative}
semantics, as in (\ref{ex:9:9}).

\ea \label{ex:9:9} % (9)
\gll {urši-ni diʔ b-erk-uwe.}\\
boy-\textsc{erg} meat \textsc{n}-eat:\textsc{pfv}-\textsc{aor.cvb}\\
\glt `A boy has eaten the meat (he finished it, so there is none left for
me).'
\z

Imperfective converbs can have the same semantics as \isi{habitual} forms,
i.e.\ sentences (\ref{ex:9:10}) and (\ref{ex:9:11}) have the same meaning.

\pagebreak[4]

\ea \label{ex:9:10} % (10)
\gll {urši-ni diʔ b-uk-uwe.}\\
boy-\textsc{erg} meat \textsc{n}-eat:\textsc{ipfv}-\textsc{cvb.ipfv}\\
\glt `A boy eats meat.'

\ex \label{ex:9:11} % (11)
\gll {urši-ni diʔ b-uk-an.}\\
boy-\textsc{erg} meat \textsc{n}-eat:\textsc{ipfv}-\textsc{hab}\\
\glt `A boy eats meat.'
\z

Although examples with a converb as a sole predicate are allowed by some
speakers, the corpus (about 900 sentences) does not contain any instances of such sentences.

% 5.
\section{Argument sharing}\label{argument-sharing}

\is{argument sharing|(}

In Mehweb, the S, A, P or other argument of the converb clause may –
but need not – be referentially identical to an argument of the main clause. 
This shared argument can be expressed
in either of the two clauses. Below I will refer to such situations as
\emph{argument sharing.} In this part I discuss sharing of core
arguments, including S, A and P. Logically, a large list of different
argument sharing configurations could be derived by alternating
syntactic parameters, including the role of the shared argument in the main
clause, the role of the shared argument in the converb clause and the locus of
its expression (main or converb clause). However, not all of them are grammatical. Below I classify
different argument sharing strategies in accordance with the
consultants' ability to interpret them. Note that some of the sentences
may be grammatical when interpreted in a different way, so I checked not
just grammaticality but also the availability of the intended
interpretation with shared arguments.

Generally, all configurations which include sharing of two S-arguments
or an S-argument and an A-argument, regardless of the clause where it is
expressed (the main or the converb clause), are interpretable, cf.\ (\ref{ex:9:12}) and~(\ref{ex:9:13}).

\ea \label{ex:9:12} % (12)
\gll {musa w-ak'-i-le rasuj-če b-aˤq-ib.}\\
Musa \textsc{m}-come:\textsc{pfv}-\textsc{aor}-\textsc{cvb} Rasul-\textsc{super}(\textsc{lat}) \textsc{n}-hit:\textsc{pfv}-\textsc{aor}\\
\glt `When Musa came, (he) hit Rasul.'~\footnote{The verb \emph{꞊aqˤas} 
`hit' takes the instrument as S, though it does not have to be expressed in the
  sentence. This is why the noun \emph{Rasul} is not marked as S and the
  verb has a neutral gender agreement marker.} (The S-argument of the
intransitive converb clause is coreferential with the A-argument of the
transitive main clause and is expressed in the converb clause.)

\ex \label{ex:9:13} % (13)
\gll {dag χʷe har-b-uq-uwe išbari ʡaˤš-b-aˤq-ib.}\\
yesterday dog \textsc{pv}-\textsc{n}-flee:\textsc{pfv}-\textsc{aor.cvb} today back-\textsc{n}-come:\textsc{pfv}-\textsc{aor}\\
\glt `Yesterday the dog ran away, today (it) returned.' (Two intransitive
clauses sharing their S-argument, which is expressed in the converb
clause)
\z

In example (\ref{ex:9:12}), the fact that the shared argument is expressed in the
converb clause is obvious from case marking. The verb \emph{꞊ak'es} `come' 
is intransitive and takes an S-argument, while \emph{꞊aˤqas} `hit' 
is transitive, with its A-argument in the ergative. Since the
shared argument takes S-marking (nominative), it is dependent of the
converb, not of the main verb. Therefore, it belongs to the converb
clause.

As for (\ref{ex:9:13}), the same fact can be established on the basis of word order.
The word \emph{dag} `yesterday' belongs to the converb clause, and the
shared argument stands between this adverb and the converb. Therefore,
I conclude that the shared argument belongs to the converb clause.

Sentences that include no argument sharing at all, like (\ref{ex:9:14}) and (\ref{ex:9:15}),
are perfectly grammatical as well.

\ea \label{ex:9:14} % (14)
\gll {maˤʜmud-ini diʔ as-i-le pat'imat-ini χʷe dub aˤʡ-aq-ib.}\\
Mahmud-\textsc{erg} meat take:\textsc{pfv}-\textsc{aor}-\textsc{cvb} Patimat-\textsc{erg} dog eat \textsc{lv}:\textsc{pfv}-\textsc{caus}-\textsc{aor}\\
\glt `Mahmud bought some meat, Patimat fed the dog.'

\ex \label{ex:9:15} % (15)
\gll {adami-li-ni q'ar b-iˤšq-i-le xunuj-ni buruš b-aq'-ib.}\\
husband-\textsc{obl}-\textsc{erg} hay \textsc{n}-mow:\textsc{pfv}-\textsc{aor}-\textsc{cvb} wife.\textsc{obl}-\textsc{erg} bed \textsc{n}-make:\textsc{pfv}-\textsc{aor}\\
\glt `The husband mowed the hay, the wife made the bed.'
\z

% \pagebreak


Sharing that involves P-argument, like (\ref{ex:9:16}) and (\ref{ex:9:17}), is less straightforward.

In (\ref{ex:9:16}), both clauses are transitive, the P-argument of the converb
clause is coreferential with the A-argument of the main clause and the shared
argument is expressed in the main clause (which can again be seen from
the case marking of the shared argument):

\ea \label{ex:9:16} % (16)
\gll {maˤʜmud-ini as-i-le gatu-ini waca b-uc-ib.}\\
Mahmud-\textsc{erg} take:\textsc{pfv}-\textsc{aor}-\textsc{cvb} cat-\textsc{erg} mouse \textsc{n}-catch:\textsc{pfv}-\textsc{aor}\\
\glt `Mahmud bought a cat and it caught a mouse.'
\z

Note that in (\ref{ex:9:16}) the P-argument of the main clause cannot be coreferential with the P-argument 
of the converb clause, i.e.\ the example cannot mean that Mahmud bought a
mouse who was then caught by a cat.

In (\ref{ex:9:17}), both clauses are transitive\is{transitivity}
and share both their A- and P-arguments. The shared A-argument is 
expressed in the converb clause, and the shared
P-argument belongs to the main clause (evidence based on word order, as
in (\ref{ex:9:13})):

\ea \label{ex:9:17} % (17)
\gll {dag ħamzat-ini as-i-le išbari kʷiha b-erh-un.}\\
yesterday Hamzat-\textsc{erg} take:\textsc{pfv}-\textsc{aor}-\textsc{cvb} today lamb \textsc{n}-slaughter:\textsc{pfv}-\textsc{aor}\\
\glt `Yesterday Hamzat bought a lamb, today he slaughtered it.'
\z

Sentences where A- and P-arguments of one transitive clause were intended to be 
criss-cross coreferential with the P- and A-arguments of the other transitive clause were
not interpreted in this way by any of the speakers. Cf.\
(\ref{ex:9:18}):

\ea \label{ex:9:18} % (18)
\gll {rasul uc-i-le musa w-aˤbʡ-ib.}\\
Rasul \textsc{m}.catch:\textsc{pfv}-\textsc{aor}-\textsc{cvb} Musa \textsc{m}-kill:\textsc{pfv}-\textsc{aor}\\
\glt Intended *`Musa caught Rasul, Rasul killed Musa.'

Possible interpretation: `Rasul was caught, Musa was killed.'
\z

\tabref{tab:9:3} below shows the distribution of different argument sharing strategies
according to the native speakers' ability to interpret them in the
intended way.

\begin{table}
  % Tab. 3.
  \caption{The acceptability of different core argument sharing
    strategies}\label{tab:9:3}


\begin{tabular}{@{}*3{p{.31\textwidth}<{\centering}}@{}}
\toprule
\small configurations that were always interpreted as expected & \small configurations
that were ambiguous or difficult for some speakers & \small configurations that were never
understood in the intended way \tabularnewline \midrule 
S=S & S=P & A=P \& P=A \tabularnewline
S=A & A=A & \tabularnewline
no sharing & P=P & \tabularnewline
& A=P & \tabularnewline
& A=A \& P=P & \tabularnewline
\bottomrule
\end{tabular}
\end{table}

Note that not all theoretically possible configurations are included in the
resulting table. It appears that configurations where the X-argument of
the converb clause is coreferential with the Y-argument of the main clause
behave in exactly the same way as those where the X-argument of the main
clause is coreferential with the Y-argument of the converb clause. The locus of
expression did not seem to matter, either. The configurations in the
table are thus only represented by the arguments which are shared.
%
\is{argument sharing|)}

% 6.
\section{Coordination and subordination properties}\label{coordination-and-subordination-properties}


An\is{coordination|(}\is{subordination|(}
English translation equivalent for a converb construction is often
coordination \citep[8]{haspelmath1995a}. The syntactic status of this parallel
is treated in \citet{kibrik2007}. Below I will explore the syntactic
properties of the Mehweb converb construction in terms of coordination
vs. subordination.

% 6.1.
\subsection{Three syntactic tests}

To find out whether the converbal construction in Mehweb is subordinate
to the main verb or not, three syntactic tests were applied, 
including changing the linear order (\sectref{linear-order-of-the-clauses}),
embedding
the converb clause in the main clause (\sectref{embedding}), and relativization
(\sectref{relativization}) (the tests are described in \citealt[143–145]{creissels2012}).

To run the tests, I will use sentences (\ref{ex:9:19}) and (\ref{ex:9:20}). In sentence (\ref{ex:9:19}), the
converb clause shares its A argument with the main clause, while
sentence (\ref{ex:9:20}) has no argument sharing.

\ea \label{ex:9:19} % (19)
\gll {musa-ini qali b-ic-i-le iz-es w-aʔ-ib.}\\
Musa-\textsc{erg} house \textsc{n}-sell:\textsc{pfv}-\textsc{aor}-\textsc{cvb} be.ill:\textsc{ipfv}-\textsc{inf} \textsc{m}-begin:\textsc{pfv}-\textsc{aor}\\\unskip
\glt `Musa, having sold the house, became ill.'

\ex \label{ex:9:20} % (20)
\gll {adami-li-ni q'ar b-iˤšq-i-le xunuj-ni buruš b-aq'-ib.}\\
husband-\textsc{obl}-\textsc{erg} hay \textsc{n}-mow:\textsc{pfv}-\textsc{aor}-\textsc{cvb} wife.\textsc{obl}-\textsc{erg} bed \textsc{n}-do:\textsc{pfv}-\textsc{aor}\\
\glt `The husband mowed the hay, the wife made the bed.'
\z

% 6.1.1.
\subsubsection{Linear order of the clauses}\label{linear-order-of-the-clauses}

When two or more coordinate clauses describe a sequence of events, their
order is iconic and cannot be changed without changing the sense of the
entire sentence. In contrast, if one of the clauses is subordinate, the
order can be changed with no influence on the general meaning. For
instance, \emph{I came, I saw, I conquered} is not semantically
identical to \emph{I came, I conquered, I saw}. However, the sentences
\emph{Having seen it, I conquered it} and \emph{I conquered it, having seen it} are
both possible and described the same sequence of events. In this respect, Mehweb
general converbs seem to behave more like English subordinate clauses:

\ea \label{ex:9:21} % (21)
\gll {iz-es w-aʔ-ib musa-ini qali b-ic-i-le.}\\
be.ill:\textsc{ipfv}-\textsc{inf} \textsc{m}-begin:\textsc{pfv}-\textsc{aor} Musa-\textsc{erg} house \textsc{n}-sell:\textsc{pfv}-\textsc{aor}-\textsc{cvb}\\\unskip
\glt `Musa became ill, because he had sold the house.'

\ex \label{ex:9:22} % (22)
\gll {xunuj-ni buruš b-aq'-ib, adami-li-ni q'ar b-iˤšq-i-le.}\\
wife.\textsc{obl}-\textsc{erg} bed \textsc{n}-make:\textsc{pfv}-\textsc{aor} husband-\textsc{obl}-\textsc{erg} hay \textsc{n}-mow:\textsc{pfv}-\textsc{aor}-\textsc{cvb}\\
\glt `The wife made bed, because the husband had mowed the hay.'

\z

As can be seen from comparison of these examples with (\ref{ex:9:19}) and (\ref{ex:9:20}), in both cases the main and
the converb clause can change places. It does not affect the interpretation of the order of
the events. However, note that the translations provided by native speakers
for both modified sentences changed so that their English translations now include the word `because'. 
This fact will be discussed further in the paper.

% 6.1.2.
\subsubsection{Embedding}\label{embedding}

\is{embedding|(}

Further evidence for the subordination analysis is the possibility of
embedding the converb clause in the main one.

In Mehweb, it is perfectly fine to place a converb clause that shares
its A-argument with the main clause between the main verb and its
dependents, cf.\ (\ref{ex:9:23}):

\ea \label{ex:9:23} % (23)
\gll {musa qali b-ic-i-le iz-es w-aʔ-ib.}\\
Musa house \textsc{n}-sell:\textsc{pfv}-\textsc{aor}-\textsc{cvb} be.ill:\textsc{ipfv}-\textsc{inf} \textsc{m}-begin:\textsc{pfv}-\textsc{aor}\\
\glt `Musa, as he sold the house, became ill.'
\z

In this sentence, it is clear that the shared argument belongs to the
main clause because of its case marking. The verb \emph{izes ꞊aʔes} 
`become ill' is intransitive, which is why its only argument stands in
the nominative. If the noun belonged to the converb clause, it would appear 
the ergative, cf.\ (\ref{ex:9:24}):

\ea \label{ex:9:24} % (24)
\gll {musa-ini qali b-ic-ib.}\\
Musa-\textsc{erg} house \textsc{n}-sell:\textsc{pfv}-\textsc{aor}\\
\glt `Musa sold the house.'
\z

In the absence of argument sharing, however, embedding is severely
degraded: speakers tend to either assign another interpretation or judge
the sentence as unacceptable:

\ea \label{ex:9:25} % (25)
\gll {xunuj-ni, adami-li-ni q'ar b-iˤšq-i-le, buruš b-aq'-ib.}\\
wife.\textsc{obl}-\textsc{erg} husband-\textsc{obl}-\textsc{erg} hay \textsc{n}-mow:\textsc{pfv}-\textsc{aor}-\textsc{cvb} bed \textsc{n}-make:\textsc{pfv}-\textsc{aor}\\
\glt `The wife and the husband, having mowed the hay, made the bed.'
\z

In (\ref{ex:9:25}), the converb clause with no argument sharing is embedded to
the main clause. When the ergative arguments of the different clauses
are placed next to each other as in (\ref{ex:9:25}), they are interpreted as
belonging to one and the same clause (which can be either the converb
clause or the main clause). As a result, interpretation of the sentence becomes problematic.
%
\is{embedding|)}

% 6.1.3.
\subsubsection{Relativization}\label{relativization}

\is{relativization|(}

Generally, clause coordination tends to place more severe
restrictions on the use of relativization strategies than clause
subordination. For instance, the English sentence \emph{The girl ran away when the boy punched her} 
can be relativized as \emph{The girl who ran away when the boy 
punched her came back}, whereas no such construction is
possible with a sentence like \emph{The boy punched the girl, and she ran away} 
(\emph{*The boy, who punched the girl, and she ran away, felt sorry}). 
Thus, where the relative construction is allowed, I
will consider this an argument for the subordinate status of the
converb. Unavailability of relativization will be considered as evidence
in favor of coordination.

In Mehweb, relativization is allowed if the converb clause shares its S-
or A-argument with the main clause:

\ea \label{ex:9:26} % (26)
\gll {qali b-ic-i-le iz-es w-aʔ-ib-i musa w-ebk'-ib.}\\
house \textsc{n}-sell:\textsc{pfv}-\textsc{aor}-\textsc{cvb} be.ill:\textsc{ipfv}-\textsc{inf} \textsc{m}-begin:\textsc{pfv}-\textsc{aor}-\textsc{atr} Musa \textsc{m}-die:\textsc{pfv}-\textsc{aor}\\
\glt
`Musa, who became ill because of selling the house, died.'
\z

In (\ref{ex:9:27}), where no argument is shared, none of the speakers
suggested the expected interpretation (`The wife, who made the bed after her
husband had mowed the hay, came here'). They all suggested the paratactic reading, with  
the participle interpreted as the predicate of an independent main clause:

\ea \label{ex:9:27} % (27)
\gll {adami-li-ni q'ar b-iˤšq-i-le buruš b-aq'-ib-i, xunul iše r-ak'-ib.}\\
husband-\textsc{obl}-\textsc{erg} hay \textsc{n}-mow:\textsc{pfv}-\textsc{aor}-\textsc{cvb} bed \textsc{n}-make:\textsc{pfv}-\textsc{aor}-\textsc{atr} wife here\textsc{lat} \textsc{f}-come:\textsc{pfv}-\textsc{aor}\\
\glt `The husband mowed the hay and made the bed (for his wife), the wife came
here.'

* `The wife, who made the bed after her husband mowed the hay, came here'
\z

I conclude that, with respect to relativization, sentences with no
argument sharing display more coordinate properties, while those with
argument sharing tend to behave more like subordinate clauses. With respect to clause
order, the constructions behave similarly, irrespective of the presence or
absence of a shared argument: they both allow main clause -- converb
clause order, but the speakers then specify the causal relation between
the two events.
%
\is{relativization|)}

% 6.2
\subsection{Semantic properties of the converb clause}

If two or more clauses are coordinated, each of them has a range of
properties of their own, which means that features like tense, aspect
and mood (and some others) are assigned to each predicate independently.
A subordinate clause can, however, inherit some features from a main
clause – or, in other words, fall under their scope. In this section,
I will explore some of the converb clause properties which can
potentially be inherited from the main clause. For each of the
(non-)shared features, I will suppose that inheriting a 
feature implies that the construction behaves more like a subrodinate clause,
and the absence of such inheritance will make an argument for the
coordination analysis.

% 6.2.1
\subsubsection{Tense and taxis}

\is{tense|(}
\is{taxis|(}

As was mentioned in \sectref{perfective-and-imperfective-converbs}, the perfective converb describes an
event preceding the situation denoted in the main clause, whereas
the imperfective converb describes an event which takes place
simultaneously with the main event. In other words, the
converb clause usually does not have a tense of its own, and its time
reference fully depends on that of the main clause.

Sentences which imply the presence of independent time reference within
the converb clause may nevertheless be accepted as fully grammatical,
cf.\ (\ref{ex:9:28}):

\pagebreak

\ea \label{ex:9:28} % (28)
\gll {išbari duči-rk'-uwe dag pat'imat pašmaje le-l-le.}\\
today laugh-\textsc{lv}:\textsc{ipfv}-\textsc{cvb.ipfv} yesterday Patimat sad.\textsc{advz} {be}-\textsc{f}-\textsc{cvb}\\\unskip
\glt `Today Patimat is smiling, yesterday she was sad.' (`Today smiling,
yesterday Patimat was sad.')
\z

Note that, however, such sentences are judged as ungrammatical if the
converb clause is embedded to the main one, cf.\ (\ref{ex:9:29}):

\ea \label{ex:9:29} % (29)
\gll {*dag pat'imat išbari duči-rk'-uwe pašmaje le-l-le.}\\
yesterday Patimat today laugh-\textsc{lv}:\textsc{ipfv}-\textsc{cvb.ipfv} sad.\textsc{advz} {be}-\textsc{f}-\textsc{cvb}\\\unskip
\glt `Today Patimat is smiling, yesterday she was sad.'
\z

The same happens if the converb clause is placed after the main one:
sentence (\ref{ex:9:30}) is ungrammatical as well.

\ea \label{ex:9:30} % (30)
\gll {*dag pat'imat pašmaje le-l-le išbari duči-rk'-uwe.}\\
yesterday Patimat sad.\textsc{advz} {be}-\textsc{f}-\textsc{cvb} today laugh-\textsc{lv}:\textsc{ipfv}-\textsc{cvb.ipfv}\\\unskip
\glt `Today Patimat is smiling, yesterday she was sad.'
\z

Overall, it seems that the Mehweb converb is capable of having a tense
of its own, i.e.\ be tensed independently of the main clause. However,
converbs inflected for a different tense than the main verb cannot be
embedded to the main clause or placed after it. In other words, they
fail the test on subordination. In this case, the converb clause is less
clearly subordinate to the main clause.
%
\is{tense|)}
\is{taxis|)}


% 6.2.2
\subsubsection{Illocutionary force}

\is{imperative|(}

When a subordinate predication depends on an imperative, it may or may not
inherit the illocutionary force of the main clause. This means that the
situation described in the subordinate predication can either be a part of the
situation that the speaker wants to happen, or not. For instance,
the English sentence \emph{Having drunk wine, don't drive} does not
mean that the speaker wants the addressee to drink the wine and then not
to drive. This means that \emph{Having drunk wine} does not
inherit the illocutionary force of the main predication. On the
contrary, the sentence \emph{Having cut the tomatoes, add them to the
salad}, which can easily be a part of a bigger instruction, does imply
that the speaker wants the addressee both to cut the tomatoes and to add
them to the salad. In this case, the subordinate clause inherits the
main clause's illocutionary force.

In Mehweb, a converb depending on an imperative form may or may not
inherit the illocutionary force of the main clause.

\ea % (\ref{ex:9:31})
\gll {aquli huji-s nuša-la šaˤ-baˤʜ w-ak'-i-le, nuša-šu quli w-ak'-e.}\\
next time.\textsc{obl}-\textsc{dat} we-\textsc{gen} village-\textsc{dir} \textsc{m}-come:\textsc{pfv}-\textsc{aor}-\textsc{cvb} we-\textsc{ad(lat)} house(\textsc{lat}) \textsc{m}-come-\textsc{imp}\\
\glt `When you arrive at our village next time, come at our place.'

\ex \label{ex:9:32} % (32)
\gll {kaltuška d-iˤšq-i-le ħarši d-aq'-a.}\\
potato \textsc{npl}-peel:\textsc{pfv}-\textsc{aor}-\textsc{cvb} soup \textsc{npl}-do:\textsc{pfv}-\textsc{imp}\\
\glt `Having peeled the potatoes, cook the soup.'
\z

In the contexts where the converb falls under the scope of the main
verb's illocutionary force, using another imperative instead of the
converb is possible. Thus, sentence (\ref{ex:9:33}) has almost the same reading as
sentence (\ref{ex:9:32}).

\ea \label{ex:9:33} % (33)
\gll {kaltuška d-išq-aˤ ħarši d-aq'-a.}\\
potato \textsc{npl}-peel:\textsc{pfv}-\textsc{imp} soup \textsc{npl}-cook:\textsc{pfv}-\textsc{imp}\\
\glt `Peel the potatoes and cook the soup.'
\z

The meaning of the two, however, is slightly different. Some speakers
claim that (\ref{ex:9:32}) implies that potatoes should be peeled and then added to
the soup, whereas (\ref{ex:9:33}) does not have this implication. Probably, using
converbs with imperatives implies that there is a closer semantic link
between the two events than there would be in a sentence with two
imperatives. A similar phenomenon is described in \citet{dobrushina2008}
for \ili{Archi}.
%
\is{imperative|)}


% 6.3
\subsection{Coordination vs.\ subordination}

According to \citet{creissels2010}, if it is difficult to determine 
whether a construction is a a case of coordination or 
subordination, there are a number of analytical possibilities. In
particular, if one and the same construction within the same sentence
can show both coordinate and subordinate properties, this would
represent an instance of what he calls co-subordination. If a
construction shows either coordinate or subordinate properties
depending on the context, this is analysed as coordination in some of
its uses and subordination in others.

After applying the tests to different sentences containing converbal
predication, it seems that Mehweb converbal construction displays
different coordination/subordination properties under different
circumstances. I will take a closer look at the conditions that
influence the syntactic properties of the constructions.

First, as can be seen from examples (\ref{ex:9:21}–\ref{ex:9:23}) and (\ref{ex:9:26}), in all the
cases where the subordination tests worked, some sort of causal relation
between the main clause and the converb clause is implied. I suggest that the
coordinate or subordinate characteristics of the construction mostly
depend on the semantic relationship between the main clause and the converb
clause. In other words, when a semantic link between the two appears,
the converb construction is very likely to become subordinate.

Another important factor seems to be the presence or absence of argument sharing
between the main and the converb clause. Examples (\ref{ex:9:25}) and (\ref{ex:9:27}) show
that if the embedding test and the relativization test are applied to
sentences with no argument sharing, the results may include the
re-interpretation of the intended syntactic structure and lead to a
different semantic interpretation. Relativisation and embedding of
converb clauses without argument sharing is ungrammatical.

All in all, it seems that the behavior of the converb construction
depends on (a) the semantic relation between the main and the converb
clause and (b) the presence or absence of argument sharing between the clauses.

This seems very similar to the situation in Tsakhur as described by
% Kazenin and Testelets
\citet{kazenin-testelets2004}. In this paper, the
authors applied several tests for coordination vs. subordination to
sentences containing general converbs. The tests turned out to give
different results for one and the same sentence, depending on whether
there was a causal relation between the converb clause and the main clause. If
a Tsakhur sentence contains a converb construction and its semantics may
imply some causal relation between the main clause and the converb clause, then
embedding the converb clause into the main one is only possible with
a causal interpretation. To put it differently, subordination tests
produce positive results only if there exists a causal relation between
the main clause and the converb clause. However, center embedding can also
work without a causal relation between the clauses, if they both have the
same subject.
%
\is{coordination|)}
\is{subordination|)}

% 7.
\section{Conclusion}

In this paper I have considered the properties of general converbs in Mehweb
Dargwa. I have described the converb marker and its morphophonological
features, the distribution of perfective and imperfective converbs, the
use of periphrastic converbs, the independent use of converbs, the way they
can combine with imperatives, and how they may share their S-, A- or P-arguments with
the main clause. Coordination and subordination properties of the Mehweb
general converb were discussed. The syntactic status of converb clauses
is either coordinate or subordinate, depending on (a) whether there is a
causal relation between the main clause and the converb clause, and (b) whether
the converb clause shares its main argument with the main clause or not.
Which of the principles (a) and (b) is prior, however, is still a
question to be discussed.


\section*{List of abbreviations}

\begin{longtable}[l]{@{}ll@{}}
\textsc{add}	& additive particle \\
\textsc{advz}	& adverbializer \\
\textsc{dir}	& motion directed towards a spatial domain \\
\textsc{aor}	& aorist \\
\textsc{atr}	& attributivizer \\
\textsc{aux}	& auxiliary \\
\textsc{caus}	& causative \\
\textsc{cvb}	& converb \\
\textsc{dat}	& dative \\
\textsc{erg}	& ergative \\
\textsc{f}	& feminine (gender agreement) \\
\textsc{gen}	& genitive \\
\textsc{hab}	& habitual (durative for verbs denoting states) \\
\textsc{imp}	& imperative \\
\textsc{inf}	& infinitive \\
\textsc{ipfv}	& imperfective (derivational base) \\
\textsc{lat}	& motion into a spatial domain \\
\textsc{lv}	& light verb \\
\textsc{m}	& masculine (gender agreement) \\
\textsc{n}	& neuter (gender agreement) \\
\textsc{neg}	& negation (verbal prefix) \\
\textsc{npl}	& non-human plural (gender agreement) \\
\textsc{obl}	& oblique (nominal stem suffix) \\
\textsc{pfv}	& perfective (derivational base) \\
\textsc{pv}	& preverb (verbal prefix) \\
\textsc{super}	& spatial domain on the horizontal surface of the landmark \\
\end{longtable}

% \nocite{daniel2015:hse,sheyanova2015,}
\printbibliography[heading=subbibliography,notkeyword=this]


% \section*{References}

% Haspelmath, M. (1995). The converb as a cross-linguistically valid
% category. \emph{Converbs in cross-linguistic perspective}, 1–55.

% Sheyanova, M. V. (2015). Specialized converbs in Mehweb. \emph{Higher
% School of Economics Research Paper No. WP BRP}, 31.

% Magometov, A. A. (1982). \emph{Megebskij dialekt darginskogo yazyka:
% issledovanie i teksty}. Metsniereba.

% Daniel, M. (2015). Mehweb Verb Morphology. \emph{Higher School of
% Economics Research Paper No. WP BRP}, 28.

% Dobrushina, N. R. (2008). Imperativnyj konverb v archinskom yazyke.
% \emph{Fonetika i nefonetika. K 70-letiju Sandro V. Kodzasova}, 195–206.

% Creissels, D. (2012). External agreement in the converbal construction
% of Northern Akhvakh. \emph{Clause Linkage in Cross-Linguistic
% Perspective: Data-Driven Approaches to Cross-Clausal Syntax, 249}, 127.

% Creissels, D. (2010). Specialized converbs and adverbial subordination
% in Axaxdərə Akhvakh. \emph{Clause Linking and Clause Hierarchy: Syntax
% and pragmatics}, 121, 105.

% Kazenin, K. I., \& Testelets, Y. G. (2004). Where coordination meets
% subordination: Converb constructions in Tsakhur (Daghestanian).
% \emph{TYPOLOGICAL STUDIES IN LANGUAGE, 58}, 227–240.

% Kibrik, А. Е. (2007). Printsipy i strategii klauzal'nogo sochineniya v
% dagestanskikh yazykakh. \emph{Voprosy yazykoznaniya, 3}, 78–120.


\end{document}


%%% Local Variables:
%%% mode: latex
%%% TeX-master: "../main"
%%% End:
