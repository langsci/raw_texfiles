\author{Michael Daniel\and Nina Dobrushina\lastand Dmitry Ganenkov} 
\title{The Mehweb language}  
\subtitle{Essays on phonology, morphology and syntax}
\BackTitle{The Mehweb language} % Change if BackTitle is different from Title

\renewcommand{\lsSeries}{loc} % use lowercase acronym, e.g. sidl, eotms, tgdi
\renewcommand{\lsSeriesNumber}{1} %will be assigned when the book enters the proofreading stage

\BackBody{This book is an investigation into the grammar of Mehweb (Dargwa, East Caucasian also known as Nakh-Daghestanian) based on several years of team fieldwork. Mehweb is spoken in one village community in Daghestan, Russia, with a population of some 800 people, In many ways, Mehweb is a typical East Caucasian language: it has a rich inventory of consonants; an extensive system of spatial forms in nouns and converbs and volitional forms in verbs; pervasive gender-number agreement; and ergative alignment in case marking and in gender agreement. It is also a typical language of the Dargwa branch, with symmetrical verb inflection in the imperfective and perfective paradigm and extensive use of spatial encoding for experiencers. Although Mehweb is clearly close to the northern varieties of Dargwa, it has been long isolated from the main body of Dargwa varieties by speakers of Avar and Lak. As a result of both independent internal evolution and contact with its neighbours, Mehweb developed some deviant properties, including accusatively aligned egophoric agreement, a split in the feminine gender, and the typologically rare grammatical categories of verificative and apprehensive. But most importantly, Mehweb is where our friends live.}

%\dedication{Change dedication in localmetadata.tex}
\typesetter{Vadim Radionov}
\proofreader{Ahmet Bilal Özdemir,
Andreas Hölzl,
Bev Erasmus,
Eva Schultze-Berndt,
Havenol Schrenk,
Ivica Jeđud,
Jeroen van de Weijer,
Jean Nitzke,
Sebastian Nordhoff,
Steven Kaye,
Sune Ryg{\aa}rd
Tatiana Philippova 
}

\renewcommand{\lsID}{225} 
\BookDOI{10.5281/zenodo.3374730} 
\renewcommand{\lsISBNdigital}{978-3-96110-208-2}
\renewcommand{\lsISBNhardcover}{978-3-96110-209-9} 
