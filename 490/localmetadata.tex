\author{Lisa Schäfer} %use this field for editors as well
\title{The licensing and usage of topic drop in German}
% \subtitle{Add subtitle here if it exists}
\ISBNdigital{978-3-96110-516-8}
\ISBNhardcover{978-3-98554-145-4}
\BookDOI{10.5281/zenodo.15584708}
\typesetter{Lisa Schäfer}
\proofreader{Amir Ghorbanpour,
David Carrasco Coquillat,
Elliott Pearl,
George Walkden,
Jean Nitzke,
Lachlan Mackenzie,
Maria Onoeva,
Mary Ann Walter,
Nicoletta Romeo,
Rainer Schulze,
Tabea Reiner
}
% \lsCoverTitleSizes{51.5pt}{17pt}
\dedication{\raggedleft\normalsize Es gibt gewisse Möglichkeiten die Informationstheorie in der Linguistik anzuwenden.
Shannon selbst war der erste, der auf solche Anwendungen hingewiesen hat.
Allen gemeinsam ist, daß sie zwar vom linguistischen Standpunkt aus
nicht übermäßig vielsagend sind, aber doch ganz unterhaltsam. \medskip

{\raggedleft\normalsize{\textsc{Flemming Topsoe}}\par\medskip}%
  
{\raggedleft\normalsize{\textit{in: Informationstheorie. Eine Einführung. (1974) \linebreak Stuttgart: B. G. Teubner: p. 62.}}%

}}

\renewcommand{\lsSeries}{ogl}  
\renewcommand{\lsSeriesNumber}{12}

\renewcommand{\lsID}{490}

\BackBody{This book is concerned with the licensing and usage of the elliptical construction topic drop in German. The term topic drop refers to the omission of the preverbal constituent in declarative verb\hyp second sentences, for example, the omission of the subject \textit{ich} (‘I’) in the sentence \textit{Bin gleich zurück} (‘Am right back’). Topic drop exists in most of the Germanic verb\hyp second languages and typically occurs in spoken language and text types such as SMS, chats, notes, etc.

While much of the previous research has focused on individual specific properties of topic drop, often adopting a purely theoretical perspective, this book presents a systematic investigation of both the syntactic properties and usage conditions of topic drop based on empirical evidence from a corpus study and 12 acceptability rating studies.

The first part of the book investigates the licensing of topic drop, in particular its restriction to the preverbal “prefield” position. The results of four rating studies on topic drop in different prefield configurations lead to a refined prefield condition based on proposals by Rizzi (1994) and Freywald (2020) that is independent of topicality. Moreover, they inform the discussion on the most suitable syntactic analysis of topic drop, supporting a PF-deletion approach.

The second part of the book presents and tests an information\hyp theoretic account of topic drop usage that builds on the Uniform Information Density hypothesis (Levy \& Jaeger 2007). In a corpus study and seven rating studies, several potential usage factors are investigated, including grammatical person and verb predictability. The results provide initial evidence suggesting that topic drop usage can be explained by general processing principles: The prefield constituent is omitted when it is redundant, and realized overtly when it facilitates the processing of the following verb. This information-theoretic explanation is based on independently evidenced processing mechanisms, bundles isolated claims from the theoretical literature, and allows for a unified analysis of topic drop with other types of ellipsis and reduction.}

