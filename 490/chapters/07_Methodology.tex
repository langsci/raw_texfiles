\chapter{Methodology of the empirical studies} \label{ch:methodology}
The following four chapters each address one or more closely related factors that potentially influence the usage of topic drop:
Chapter \ref{ch:usage.function} syntactic function, Chapter \ref{ch:factor.topicality} topicality, Chapter \ref{ch:usage.person} grammatical person, verbal inflection, and ambiguity avoidance, and Chapter \ref{ch:usage.verb} verb type and verb surprisal.
These chapters all have the same basic structure.
First, I outline the central claims from the theoretical literature and the most important results from previous empirical studies.
From these results, several hypotheses for the usage of topic drop emerge.
Second, I discuss the information-theoretic predictions concerning the factor(s) and their relation to the hypotheses from the literature.
Third and finally, I present my empirical studies in the form of a corpus study, acceptability rating studies, or both.
The current chapter provides the methodological background for these empirical studies.

It is important to note that most of these studies examined multiple factors at once to determine if they collectively / in combination have an impact on the usage of topic drop.
In my corpus study, for example, I considered grammatical person, verbal inflection, verb type, and verb surprisal, while in experiments \ref*{exp:top.q1}, \ref*{exp:top.s.fv}, and \ref*{exp:top.s.mv} I looked at grammatical person and topicality in combination.
Since the following chapters are built around the factors rather than the individual studies, the results of the studies are discussed in several places, e.g., experiment \ref*{exp:top.q1} is addressed in both Chapter \ref{ch:factor.topicality} on topicality and Chapter \ref{ch:usage.person} on grammatical person.
I give the details on the corpus study in a bundled way in \sectref{sec:corpus} in this chapter, while for each experiment, I present the basic information concerning design, stimuli, and analysis at the first mention.

This chapter is structured as follows:
First, I give an overview of the connection between the corpus study and the acceptability rating experiments and, more generally, the connection between production and perception.
Then, I present the details of the fragment corpus FraC and three data sets derived from that corpus, which I used for my corpus investigations.
I conclude this chapter with some brief remarks on the methodology of the experiments.

\is{Corpus|(}\is{Production|(}
\section{Production, perception, and processing}\label{sec:production.perception} \is{Acceptability rating study|(}
For my empirical studies, I employed a combination of methods to look both at the production, i.e., the frequency, and the perception, i.e., the acceptability, of topic drop.
This combination was both practically and conceptually motivated.
The practical motivation was that I could not investigate all factors of interest in the fragment corpus FraC.
First, the FraC is not suitable for studying topicality because it is not annotated for information structure and for many utterances, the precontext required for such an annotation is not available.
Second, it is a rather small corpus and certain factor combinations occur too rarely in there to investigate their impact on topic drop systematically.
Such a systematic investigation was possible with my rating studies, which could at least partly offset the data sparsity problem.
The conceptual reason for using both corpus and experiments was to investigate whether the effects found in the corpus study also show up in the experiments and vice versa.
The intention was to provide a first indication of whether or not audience design \is{Audience design|(} \citep{bell1984} is a reasonable assumption as a linking hypothesis for the usage of topic drop, as discussed in \sectref{sec:info.theory.summary}.

The corpus data allowed me to focus on the production of topic drop and to determine its frequency relative to certain influencing factors.
Looking at the production is a natural first step to test the information-theoretic predictions for topic drop usage since \textit{UID}, \is{Uniform information density} a core component of my approach, is in the first place a hypothesis about language production, i.e., how speakers optimize their utterances (see \sectref{sec:info.theory.uid}).
With the authentic corpus data, I could directly investigate whether produced utterances with and without topic drop differ significantly with respect to relevant parameters, such as the grammatical person of the overt or covert prefield constituent or the surprisal of the following verb.

I examined the perception of topic drop through experiments.
The linking hypothesis that connects production and perception is audience design \citep{bell1984}, i.e., the idea that a speaker shapes their linguistic production in such a way that the processing of the hearer is facilitated. \is{Processing effort|(}
If an utterance with topic drop is easier to understand for a hearer, a speaker should be more inclined to use it.%
%% Footnote
\footnote{It could be argued that processing difficulties on the part of the hearer could always be avoided by the speaker being as explicit as possible and omitting nothing.
However, according to \textit{UID}, \is{Uniform information density} such overly explicit and thus partly redundant utterances waste resources from the speaker's perspective and may underchallenge the hearer (see \sectref{sec:avoid.troughs}).
Furthermore, they may lead to undesired implicatures for violating the maxim of relevance or quantity (see the research on atypicality inferences, e.g., \citet{kravtchenko.demberg2022}).}
%
In this way, I could indirectly investigate the usage of topic drop via perception experiments.
Since the acceptability rating experiments are an offline method, they do not allow for directly observing the processing of an utterance.
However, it is possible to indirectly draw conclusions about the processing via the perceived acceptability (see, e.g., \cite{fanselow.frisch2006}, \cite{hofmeister.etal2014}).
The general idea is that the more difficult it is to process an utterance, the less acceptable or natural should it be perceived, and the worse should it be rated.%
%% Footnote
\footnote{Support for this idea of a correlation between acceptability and processing comes, for example, from  \citet{hofmeister.etal2013} for superiority effects in multiple \textit{wh}-questions and from \citet{hofmeister.etal2015} for ``frozen'' extraposed constituents.}
%

Recall that in the information-theoretic  logic, processing difficulties are caused by a suboptimal distribution of information, i.e., by peaks and troughs in the information density profile resulting from too predictable \is{Predictability} or too unpredictable linguistic expressions (see \sectref{sec:info.theory.uid}).
This reasoning allows for connecting frequency as the dependent variable in the corpus study and acceptability as the dependent variable in the experiments.
An expression that generally occurs very frequently (in a certain position) has a high predictability \is{Predictability} (in that position), i.e., a low surprisal, and is likely to create an information trough.
According to the \textit{avoid troughs} principle (\sectref{sec:avoid.troughs}), hearers should prefer an utterance where this expression is omitted (topic drop) over an utterance where it is realized (full form), provided that grammar permits the former.
In contrast, a very rare expression should have a low predictability \is{Predictability} and a high surprisal, which might cause a processing overload for the hearer.
In this situation, it is to be expected from the \textit{avoid peaks} principle (\sectref{sec:avoid.peaks}) that hearers prefer an utterance where an additional prefield constituent is inserted before the expression with the high surprisal to reduce the peak of information and the processing effort (full form) over one where nothing is inserted (topic drop).
 
In summary, I used corpus data to determine frequencies of authentic utterances with and without topic drop.
My experiments provided acceptability judgments of systematically constructed sentences with and without topic drop and, thus, also indirect evidence for possible processing difficulties. \is{Processing effort|)}
When the same factors are investigated with both methods, I expect comparable results according to audience design \is{Audience design|)} as a linking hypothesis, i.e., that rarer topic drop constructions are also less acceptable and vice versa. \is{Acceptability rating study|)}\is{Production|)}

\section{Methodology: corpus study}\label{sec:corpus}
In this section, I introduce the fragment corpus FraC, which I used for my corpus study.
I outline how I annotated the instances of topic drop and how I extracted the corresponding full forms.
Finally, I briefly present the three data sets that are based on the FraC and in which I investigated topic drop in this book.

\subsection{The fragment corpus FraC}\label{sec:corpus.frac}
%\vspace{-1.5em}
\begin{figure}
\centering
\includegraphics[scale=0.6]{FraC_Torte.png}
\caption[The 17 text types  in the fragment corpus FraC, divided into spoken, written, and social media text types]{The 17 text types in the fragment corpus FraC, divided into spoken, written, and social media text types}
\is{Text type}
\label{fig:frac.overview}
\end{figure}

My corpus study, the results of which I present in the following chapters, was conducted on the fragment corpus FraC \citep{horch.reich2017}.
The FraC is a Standard German corpus dedicated to the investigation of fragments, \is{Fragment} i.e., non-sentential utterances \citep[e.g.,][]{morgan1973,merchant2004,reich2007,lemke2021}.
The corpus consists of 17 different text types,%
%% Footnote
\footnote{See \sectref{sec:def.texttype} for a discussion of the term \textit{text type} and how I use it in this book.}
%
\is{Text type} which range from prototypically written (e.g., news articles, legal texts, etc.) to prototypically spoken (e.g., dialogues, interviews, etc.) to so-called ``social media texts'' (e.g., online chats, text messages, etc.).
Figure \ref{fig:frac.overview} shows an overview of the different text types.\is{Text type}%
%% Footnote
\footnote{I created the figure myself in Microsoft Excel.}
%

There are about 2\,000 utterances of each text type \is{Text type} in the corpus, resulting in a total of about 34\,000 utterances and 380\,000 tokens \citep{horch.reich2017}.
The corpus was automatically POS-tagged and lemmatized using the TreeTagger \citep{schmid1994, schmid1995} and is manually annotated for a variety of categories.
Relevant to my work is the annotation of omission types.
919 utterances were classified as containing topic drop.
However, the annotation of the FraC is based on a different definition of topic drop than the one that I assume and also includes V1 declaratives. \is{V1 declarative}
Therefore, I manually reviewed all 919 utterances and excluded those that did not match the definition of topic drop developed in \sectref{sec:definition} and that were neither covert subjects nor objects.%
% Footnote
\footnote{Furthermore, I did not consider variants of the abbreviation \textit{HD(GD)L} (\textit{hab dich} (\textit{ganz doll}) \textit{lieb} (`love you (very much)'), which occur around  30 times in the text message subcorpus.
Since the abbreviation is conventionalized in German chatspeak without the subject, i.e., usually there is no \textit{IHDL} (\textit{ich hab dich lieb} (`I love you')), i.e., there is actually no real full form that the speaker could use alternatively.
}
%
This resulted in a total of 873 instances of topic drop, which provide the basis for my corpus investigations.
Note that the number of instances is not equal to the number of utterances with topic drop, since some utterances contained two or more cases of topic drop (separated by commas and not split into individual sentences in the corpus).

\subsection{Annotation procedure}\label{sec:corpus.annotation}
In the following, I describe first how I annotated the instances of topic drop in the FraC and then how I obtained and annotated the corresponding full forms that serve as reference data.

\subsubsection{Annotating the instances of topic drop}
I annotated each of the 873 instances of topic drop in the FraC manually for several categories.
In what follows, I list these categories and illustrate the annotation process with example \ref{ex:frac.anno}.

\ex.\label{ex:frac.anno}
\ag. Sein fast 70jähriges Herz?\\
his almost 70-year-old heart\\
`His almost 70-year-old heart?'
\bg.\label{ex:frac.anno.td}$\Delta$ Schlägt gut und regelmäßig.\\
it beats well and regularly\\
`(It) beats well and regularly.' [FraC B47--B48]

For each instance, I provided a subjective intuitive reconstruction of the omitted element based on the context and the morphological information in the form of a personal pronoun, a demonstrative pronoun, or a proper name (if this name cannot be naturally replaced by a pronoun) and also annotated the category of this reconstruction (personal pronoun, demonstrative pronoun, proper name, etc.).
In the case of \ref{ex:frac.anno}, I reconstructed the personal pronoun \textit{es} (`it').

I annotated the grammatical function, person, number, and (if applicable) gender of the omitted constituent.
In \ref{ex:frac.anno}, it is a 3rd person singular neuter subject.

Next, I determined whether there is an antecedent or a postcedent present in the linguistic context and annotated its category. \is{Antecedent}
By default, I only considered postcedents if the precontext provided no suitable antecedent at all. \is{Antecedent}
I also determined the distance of the antecedent \is{Antecedent} or postcedent in the number of utterances (0 means that the antecedent occurs in the same utterance as topic drop).%
%% Footnote
\footnote{\label{note:precontext}Note that for some utterances, in particular for a part of the text messages, the precontext is not available.
In these cases I could often infer the antecedent \is{Antecedent} but not determine whether it was linguistically present and in what distance.}
%
In example \ref{ex:frac.anno}, there is a linguistic antecedent \is{Antecedent} in the last utterance (distance $1$), namely the determiner phrase \textit{sein fast 70jähriges Herz} (`his almost 70-year-old heart').
See \sectref{sec:recover.ling} for a discussion of the distance annotation.

For the verb following topic drop, I annotated the verb type distinguishing copular verbs \is{Copula} (\textit{sein} (`to be') and \textit{werden} (`will') when used without another verb form), auxiliaries \is{Auxiliary} (\textit{sein} (`to be'), \textit{haben} (`to have'), and \textit{werden} (`will') when used to form periphrastic verb forms), modal verbs \is{Modal verb} (\textit{dürfen} (`may'), \textit{können} (`can'), \textit{mögen} (`may'), \textit{müssen} (`must'), \textit{sollen} (`shall'), and \textit{wollen} (`to want')), reflexive verbs, \is{Reflexive verb} and lexical verbs. \is{Lexical verb}
Additionally, I annotated whether the verb form was syncretic \is{Syncretism} or distinct.
In example \ref{ex:frac.anno}, topic drop is followed by a lexical verb in present tense indicative, which constitutes a distinct verb form.

\subsubsection{Obtaining and annotating the reference data}\label{sec:corpus.reference}

\largerpage[-1]
To obtain relative numbers and omission rates based on usage factors, I extracted full forms as reference data.
This means that I considered those utterances in the FraC that could potentially be targeted by topic drop.
For the \textsc{FraC-TD-Comp} data set (\sectref{sec:corpus.comp}), which is based on the complete FraC, I used a semi-automatic approach.
For the smaller  \textsc{FraC-TD-SMS}  (\sectref{sec:corpus.mess}) and \textsc{FraC-TD-SMS-Part} (\sectref{sec:corpus.sms.part}) data sets, which are based on only the text message subcorpus, I manually reviewed all 1\,961 utterances in the text message subcorpus of the FraC.

To decide which utterances could be targeted by topic drop, it is necessary to verify that the licensing and the felicity condition of topic drop discussed in Chapters \ref{ch:topicality} and \ref{ch:recover} are met: prefield position \is{Prefield} and recoverability \is{Recoverability} of the corresponding constituent.%
%% Footnote
\footnote{Recall that I exempted non-referential expletives \is{Expletive} from the recoverability \is{Recoverability} condition (see \sectref{sec:recover.non.referential}).
\is{Recoverability|(} However,  I ensured that an overt constituent would be recoverable, if targeted by topic drop, by searching for all utterances with a substitutive pronoun in the prefield, as explained below.
This way, I also captured the cases with overt expletive \textit{es} in the preverbal position, as desired.
}
%
Since no linguistic context is available for several utterances in the FraC and since the extralinguistic context is not accessible, it is partially impossible to determine which realized referential prefield constituent would be recoverable and which would not if omitted.
To circumvent this problem, I restrict my reference data to those utterances where the prefield constituent is a substitutive pronoun.
I assume that the use of such pronouns is restricted to cases where their reference can be recovered from the linguistic or extralinguistic context.
That means that they have the same felicity condition of recoverability as referential topic drop.
In practice, I included every utterance in which one of the object (accusative \is{Accusative case|(} or dative)%
% Footnote
\footnote{There were no preverbal genitive \is{Genitive case} object pronouns in the corpus. 
I did find 20 instances of PPs with pronouns in the prefield, of which some were adverbial adjuncts such as \ref{ex:frac.adv} and some prepositional objects \is{Prepositional object} such as \ref{ex:frac.pobject}.
However, I argue that none of them can be targeted by topic drop (as indicated by the asterisks) because the prepositions \textit{mit} (`with') and \textit{von} (`from') cannot be recovered from context.
%\vspace{-0.5\baselineskip}
\ex.\label{ex:frac.adv}
\a.\textit{Schon beim zweiten Tagesordnungspunkt, der Wahl eines Vorsitzenden für das neue Gremium, wurde der Gewerkschafter Rappe ungeduldig -- und beantragte den Rückzug in Sitzungssaal Nummer zwei.}\\
`Already at the second item on the agenda, the election of a chairman for the new committee, the unionist Rappe became impatient -- and requested to retire to meeting room number two.'
\bg.*(Mit mir) geht sowas nicht, entrüstete er sich über die Gegenrede einer Studentin, ich mach so'n Zirkus nicht mit.\\
with me goes something.like.this not outraged he himself about the contradiction a.\textsc{gen} student.\textsc{fem} I make such.a circus not with\\
`That's not possible with me, he said indignantly in response to a counter-speech from a student, I don't take part in that kind of circus.' [FraC A2010--A2011]

%\vspace{-1\baselineskip}
\ex.\label{ex:frac.pobject}
\a.\textit{Ich bin Dozent für Computerlinguistik / Computerlexikographie in T-Stadt am Seminar für Sprachwissenschaft}\\
`I am a lecturer in computational linguistics / computational lexicography in T-city at the Seminar for Linguistics'
\bg.*(von mir) stammen die meisten Skripte (oder Skripts?)...\\
from me stem the most scripts or scripts\\
`Most of the scripts (or scripts?) are from me...' [FraC C64--C65]

%\vspace{-2em}
}%
\is{Dative case|(} or subject pronouns, listed in Table \ref{tab:weakly.pronouns}, occurs in the prefield.

\begin{table}
\caption{Subject and object pronouns considered in the corpus search for the full forms}
\is{Accusative case}\is{Dative case|)}\is{Nominative case}
\centering
\begin{tabular}{llll}
\lsptoprule
\Centerstack[l]{Grammati- \\ cal person} & Nominative & Accusative & Dative \\
\midrule
1SG & \textit{i}(\textit{ch}) & \textit{mich} & \textit{mir} \\
2SG & \textit{du} & \textit{dich} & \textit{dir} \\
3SG & \textit{er}, \textit{sie}, \textit{es}, \textit{man}, & \textit{ihn}, \textit{sie}, \textit{es}, & \textit{ihm}, \textit{ihr}, \\
& \textit{der}, \textit{die}, \textit{das} & \textit{den}, \textit{die}, \textit{das} & \textit{dem}, \textit{der} \\
& \textit{dieser}, \textit{diese}, \textit{dieses} & \textit{diesen}, \textit{diese}, \textit{dieses} & \textit{diesem}, \textit{dieser} \\
1PL & \textit{wir} & \textit{uns} & \textit{uns} \\
2PL & \textit{ihr} & \textit{euch} & \textit{euch} \\
3PL & \textit{sie}, \textit{die} & \textit{sie}, \textit{die} & \textit{ihnen}, \textit{denen} \\
\lspbottomrule
\end{tabular}
\label{tab:weakly.pronouns}
\end{table}

\largerpage[-2]
\noindent
I searched not only for the demonstratives with the stem \textit{dies*} but also for those with \textit{jen*}, but there were no occurrences in the preverbal position.
I also included the dialectal variant \textit{i} of \textit{ich} (`I'), which is common in southern Germany and occurs several times in the text message subcorpus.
Furthermore, I considered full forms with \textit{man} in the prefield because I found instances where it is omitted in the FraC.%
%% Footnote
\footnote{\label{note:man}
One such example of a covert \textit{man} is \ref{ex:man}.
%\vspace{-0.5\baselineskip}
\exg.\label{ex:man}Man kämpft mit den erstaunlich kräftigen Kindern, $\Delta$ presst sie in Klamotten. $\Delta$ Stopft die Füße in die Schuhe.\\
one fights with the astonishingly strong children one presses them in clothes one stuffs the feet in the shoes\\
`You fight with the astonishingly strong children, (you) press them into clothes. (You) stuff their feet into shoes.' [FraC B1640--B1641]

%\vspace{-0.5\baselineskip}
At first glance, the possibility of omitting \textit{man} seems to be problematic since \textit{man} is frequently classified as an indefinite pronoun (e.g., \cite[43]{zifonun.etal1997}, \cite[66--67]{gallmann.sitta2007}, \cite[87]{imo2016}, \cite[140--141]{thielmann2021}), and \citet{volodina.onea2012} explicitly deny that indefinite pronouns can be targeted by topic drop.
However, \citet[187]{eisenberg2020} characterizes \textit{man} not as an indefinite pronoun but as an ``impersonal personal pronoun'', while the Leibniz Institut für Deutsche Sprache (IDS) terms it a ``generalizing personal pronoun'' \citep{man}.
The IDS stresses that \textit{man} differs from indefinite pronouns in that consecutive occurrences can refer to the same person, as the example above demonstrates.
It seems that it is precisely this property that enables recoverability \is{Recoverability} and, thus, topic drop.}
\is{Recoverability|)}

\largerpage[-2]
\is{Accusative case|)}\is{Prefield|(}
Since the FraC is not annotated topologically, I operationalized the licensing condition of topic drop, i.e., the prefield restriction, in the semi-automatic approach that yielded \textsc{FraC-TD-Comp}, as follows:
I considered a pronoun to be in the prefield if it occurs in the first position of an utterance and is immediately followed by a finite verb.%
%% Footnote
\footnote{It has to be noted that the automatic extraction procedure that I used for \textsc{FraC-TD-Comp} might have yielded false positive and false negative instances, as it is dependent both on the POS tagging and the tokenization in sentence units.
The accuracy of both varies in particular with text type, e.g., for news articles it is higher than for dialogues.}
%%
In \sectref{sec:initial}, I discussed that certain elements such as conjunctions, \is{Conjunction} particles, and interjections can precede topic drop.
Consequently, I also considered corresponding full forms with both the semi-automatic and the manual approach, i.e., utterances where a word, separated by a comma, precedes the pronoun in the prefield.
For each full form, I annotated the prefield constituent, its category, its number, person, gender, and case, as well as the verb type of the following verb.\is{Prefield|)}

\subsection{FraC data sets}\label{sec:frac.data.sets}
I created three data sets:
(i) The \textsc{FraC-TD-Comp} data set contains all instances of topic drop in the FraC and (semi-automatically retrieved) corresponding reference data, i.e., full forms where the prefield constituent could potentially be omitted.
(ii) The \textsc{FraC-TD-SMS} data set consists of only the utterances with topic drop and the corresponding (manually extracted) full forms of the text message subcorpus of the FraC.
(iii) The \textsc{FraC-TD-SMS-Part} data set is a reduced variant of \textsc{FraC-TD-SMS} containing only the 1st and 3rd person singular subjects.%
% Footnote
\footnote{In \citet{schafer2021}, I also presented a corpus study of such a subset of the FraC, which yielded similar but partly different results.
For this book, I was able to increase the amount and quality of the data I analyzed by not only using automatically extracted instances of topic drop and full forms but by manually reviewing the entire text message subcorpus.
Additionally, I revised the criterion to choose the reference data from pronouns and full forms to only pronouns to ensure recoverability \is{Recoverability} (see \sectref{sec:corpus.reference}), refined the predictor that encodes whether the verb in the left bracket is syncretic \is{Syncretism} or not and also the annotation of the corresponding verbs, and additionally included a predictor for the verb type following topic drop.
}
As discussed in \sectref{sec:def.texttype}, topic drop is not (fully) licensed in certain text types \is{Text type|(} contained in the FraC, such as news articles or legal texts.
This means that these text types contribute full forms to \textsc{FraC-TD-Comp} but not instances of topic drop, making the full forms overrepresented in the data set.
This distorts relative measures such as the omission rate.
To circumvent this problem, I focused on the text message subcorpus in my further investigations because it contains the highest number of topic drop of all subcorpora and also has the highest omission rate of almost 64\%.
The corresponding \textsc{FraC-TD-SMS} data set allowed me to statistically compare relative values as a function of syntactic function, grammatical person, and verb type.
I used the third data set, \textsc{FraC-TD-SMS-Part}, to look at distinct inflectional marking and verb surprisal and to perform an inferential statistical analysis.
It simultaneously considered these two factors as well as grammatical person and verb type and evidenced an impact of them on the frequency of topic drop.

\subsubsection{\textsc{FraC-TD-Comp}}\label{sec:corpus.comp}
The data set \textsc{FraC-TD-Comp} (\textit{Fra}gment \textit{C}orpus \textit{T}opic \textit{D}rop \textit{Comp}lete) contains all 873 instances of topic drop in the FraC and reference data in the form of  3\,211 full forms that could potentially be targeted by topic drop (omission rate of 21.38\%), resulting in a total of 4\,084 instances, as shown in Table \ref{tab:complete.data.fraC}.

\begin{table}
\centering
\caption{Overview of the \textsc{FraC-TD-Comp} data set}
\begin{tabular}{rrrr}
\lsptoprule
Full form & Topic drop & Total & Omission rate \\
\midrule
$3\,211$ & $873$ & $4\,084$ & $21.38\%$\\
\lspbottomrule
\end{tabular}
\label{tab:complete.data.fraC}
\end{table}

\noindent
Each instance is annotated for the following categories:
\textsc{Completeness} (Does the instance contain topic drop or is it a full form?), 
\textsc{Syntactic Function} and \textsc{Case} of the omitted or realized prefield constituent, \textsc{Grammatical Person, Number}, and \textsc{Gender} of the omitted or realized prefield constituent, \textsc{Verb Type} of the verb following the omitted or realized prefield constituent and \textsc{Text Type} in which the instance occurs.

In the following chapters, I present the results of a descriptive look at the \textsc{FraC-TD-Comp} data set with respect to syntactic function, grammatical person, number, gender, and type of the following verb, as well as their potential role for topic drop.

\subsubsection{\textsc{FraC-TD-SMS}}\label{sec:corpus.mess}
As a result of the typical occurrence of topic drop in certain text types (see \sectref{sec:def.texttype}), I decided to reduce the corpus to a text type that allows for topic drop.
I created the data set \textsc{FraC-TD-SMS} (\textit{FraC} \textit{T}opic \textit{D}rop \textit{SMS}), which contains the 353 instances of topic drop and the 201 full forms of the text message subcorpus of the FraC (see Table \ref{tab:message.data.fraC}).

\begin{table}
\centering
\caption{Overview of the text message data set \textsc{FraC-TD-SMS}}
\begin{tabular}{rrrr}
\lsptoprule
Full form & Topic drop & Total & Omission rate\\
\midrule
$201$ & $353$ & $553$ & $63.83\%$ \\
\lspbottomrule
\end{tabular}
\label{tab:message.data.fraC}
\end{table}

\noindent
The restriction to the text message subcorpus seems appropriate for two reasons:
(i) By far the largest number of cases of topic drop in the FraC stem from text messages, 353 out of 873 cases, or 40.44\%, indicating that it is very natural for topic drop to occur in this text type. \is{Text type|)}
(ii) Analyses of this data set can be better compared to the existing corpus studies by \citet{androutsopoulos.schmidt2002} and \citet{frick2017} (see \sectref{sec:usage.person.studies} for details on their studies), which have also been conducted on text messages.
In the following chapters, I present analyses of the \textsc{FraC-TD-SMS} data set considering syntactic function, grammatical person, and following verb, which complement the analysis of \textsc{FraC-TD-Comp}.

\subsubsection{\textsc{FraC-TD-SMS-Part}}\label{sec:corpus.sms.part}
The final data set \textsc{FraC-TD-SMS-Part} is a further reduction that limits the data of \textsc{FraC-TD-SMS} to only the overt and covert subjects of the 1st and the 3rd person singular.
This is motivated by the observation that overall mainly subjects are omitted in the \textsc{FraC-TD-SMS} data set, above all subjects of the 1st person singular, followed with some distance by the 3rd person singular (see \sectref{sec:frac.td.sms.function}).
Since additionally for subject topic drop, the following congruent verb is predicted to play a role in the recovery \is{Recoverability} according to my information-theoretic account, at least if this verb is distinctly marked for inflection, \is{Verbal inflection} it seems reasonable to focus on the subjects, in particular subjects of the 1st and the 3rd person singular.
The contrast between these two grammatical persons furthermore allowed me to investigate a potential effect of syncretic inflectional endings, \is{Verbal inflection} which exist for the 1st and 3rd person singular in the present and past tense (see \sectref{sec:usage.person.theory}), and the overall influence of grammatical person on the likelihood of topic drop.
Table \ref{tab:frac.td.mess.part} shows the distribution of full forms and instances of topic drop as a function of the grammatical person in the resulting data set \textsc{FraC-TD-SMS-Part}.
In \sectref{sec:frac.td.part.regression.person}, I present the logistic regression analysis that I conducted on this data set and discuss its results there and in \sectref{sec:corpus.regression.rep}.

\begin{table}
\centering
\caption{Full forms, instances of topic drop, and omission rates as a function of grammatical person in the \textsc{FraC-TD-SMS-Part} data set}
\begin{tabular}{lrrrr}
\lsptoprule
Grammatical person & Full form & Topic drop & Total & Omission rate\\
\midrule
1SG & $131$ & $264$ & $395$ & $66.84\%$\\
3SG & $29$ & $41$ & $70$ & $58.57\%$\\
\tablevspace
Total & $160$ & $305$ & $465$ & $65.59\%$\\
\lspbottomrule
\end{tabular}
\label{tab:frac.td.mess.part}
\end{table}
\is{Corpus|)}

\section{Methodology: experiments }
\is{Acceptability rating study|(}\label{sec:experiments}
In the following chapters, I present a total of eight acceptability rating experiments that tested various factors potentially influencing the usage of topic drop.
I tested the impact of topicality, grammatical person, verb surprisal, and, more indirectly, verb type and ambiguity avoidance.
These factors were selected based on the discussions in the literature, as well as based on my information-theoretic account of topic drop usage.

There were two goals that I pursued with the experiments.
First, I tested several isolated hypotheses from the theoretical literature for the first time.
That is, I investigated whether grammatical person, topicality, and verb type impact the usage of topic drop, as hypothesized by authors such as \citet{auer1993}, \citet{zifonun.etal1997}, \citet{imo2013}, and \citet{helmer2016}.
Second, I related these results to the predictions of my information-theoretic account, more specifically to the \textit{avoid troughs} and the \textit{facilitate recovery} principles. \is{Recoverability}
The predictions of the \textit{avoid peaks} principle were then decidedly tested with the two experiments on verb surprisal.

The methodology of the experiments is basically identical to that of the experiments in the first part of this book.
I compared utterances with topic drop to a non-elliptical baseline to see whether ellipsis-specific effects of the influencing factors emerge in the form of an interaction.
The statistical analysis followed the procedure described in \sectref{sec:data.analysis}.
As discussed above, I interpret a degraded acceptability judgment as a display of processing difficulties \is{Processing effort} caused by a suboptimal distribution of information. \is{Acceptability rating study|)}
