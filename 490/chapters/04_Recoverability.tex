%*****************************************
\chapter{Recoverability as a prerequisite for topic drop}\label{ch:recover}
%*****************************************
\is{Recoverability|(}
Chapter \ref{ch:topicality} identified the prefield restriction, or more precisely the restriction to a prefield that is not c-commanded sentence-internally by a potential identifier or to the highest prefield of a root clause, as the main licensing condition of topic drop.
This chapter focuses on recoverability, which is often discussed in the literature as a further condition for topic drop \citep[e.g.,][]{fries1988,cardinaletti1990,helmer2016,freywald2020}.
For example, \citet[27]{fries1988} lists it as a discourse condition, while \citet[75]{cardinaletti1990} states that topic drop is ``only possible with contextually salient \is{Salience|(} elements, i.e. the reference of the null argument must be recoverable either from the linguistic or the extralinguistic context.''
In this chapter, I argue that recoverability should not be understood as a syntactic licensing condition, i.e., not analogous to the prefield restriction but as a felicity or usage condition.

Recoverability, roughly speaking, denotes the process of identifying the omitted constituent and, in the case of referential constituents, its reference.
I argue that a hearer is able to recover an omitted constituent (and its reference) if they can determine, based on several factors, which constituent (and reference) the speaker intended.
Possible referents are either directly available through the linguistic or extralinguistic context or can be inferred indirectly from the linguistic context.
It is important to note that recoverability is not an exclusive feature of topic drop but a prerequisite for any type of ellipsis or omission.
Presumably, this mechanism is even at work in other related phenomena, such as pragmatic enrichment, where something not explicitly verbalized must be co-understood, or the resolution of pronouns or anaphors in general, where the reference of a placeholder must be established.

I assume that the process of recovery requires cognitive effort on the part of the hearer and that this effort varies depending on how easy or difficult this process is. \is{Processing effort|(}
The ease of recoverability, in turn, may be influenced by several factors such as the type of context that is used for the recovery, the distance between the ellipsis site and the linguistic antecedent, and how direct the link between the ellipsis site and the linguistic antecedent is.
It may be possible to operationalize recoverability, at least of referential constituents, by recourse to the concept of givenness, as suggested in the literature \citep[e.g.,][]{fries1988,helmer2016,trutkowski2016}.
In this context, categorical givenness is often seen as a proxy for licensing the omission of a constituent or its recoverability, while gradual givenness is used to argue how well or poorly something can be omitted or recovered.
Hence, it seems useful to also discuss givenness as a concept and, in particular, to relate it to recoverability.

This chapter is structured as follows:
In the first two sections, I look at the type of context used to recover the reference of topic drop distinguishing between linguistic and extralinguistic context.
The third section briefly sketches the role of recoverability for expletives before I turn to givenness in the fourth section and discuss whether and to what extent (concepts related to) givenness can be used to describe recoverability.
Thereby, I also establish a link between recoverability and processing effort.
In the concluding section of this chapter, I summarize the results and come back to the above-given definition of recoverability.
There, I also already hint at a link between recoverability, processing effort, and the information-theoretic account of the usage of topic drop, which I outline in detail in Chapter \ref{ch:infotheory}. \is{Processing effort|)}

\is{Antecedent|(}
\section{Linguistic antecedents}\label{sec:recover.ling}
There is consensus in the literature that the reference of topic drop of referential constituents can be recovered via linguistic antecedents.
For example, \citet[100]{guenthner2006} and \citet[102]{schwitalla2012} state that the omitted constituents can be reconstructed based on a previous mention \citep[see also][]{fries1988, cardinaletti1990, zifonun.etal1997,volodina2011, volodina.onea2012}.
\citet[22]{trutkowski2016} argues that recovery is not only possible through an antecedent but also through a so-called \textit{postcedent}.
That means that the antecedent can also follow topic drop, as in the discourse-initial example \ref{ex:td.post}, where the subject of the first utterance is omitted \ref{ex:td.post.td} but can be recovered based on the co-referent overt subject of the second utterance \textit{Ortschronist Günter Bergner} \ref{ex:td.post.ff} (and based on an additional photograph).

\ex.\label{ex:td.post}
\ag.\label{ex:td.post.td}$\Delta$ Kennt die Straßen im Karlshorster Kiez wie seine Westentasche:\\
he knows the streets in.the Karlshorst's neighborhood like his vest.pocket\\
`(He) knows the streets in the Karlshorst neighborhood like the back of his hand:'
\bg.\label{ex:td.post.ff}Ortschronist Günter Bergner ist häufig auf Achse.\\
local.chronicler Günter Bergner is often on axis\\
`Local chronicler Günter Bergner is often on the road.' [DWDS: Corpus, Berliner Zeitung, 05/04/1995, p. 23] \citep[22]{trutkowski2016}

In example \ref{ex:td.post}, the antecedent and the omitted constituent are both a subject in the nominative case and, thus, formally identical.
\citet[27]{fries1988}  points out, however, that such a formal identity is not necessary, but that referential identity is sufficient to allow for recoverability. 
This means that mismatches in grammatical function and case between the antecedent and the omitted constituent are possible, as shown in example \ref{ex:recoverable.context}.
In the context sentence \ref{ex:recoverable.context.ling}, the linguistic antecedent \textit{Sofia} is a DP in the dative case that is part of a PP, but the non-realized prefield constituent in \ref{ex:recoverable.target} is required to be the subject in the nominative case.
However, recoverability works because \textit{Sofia} is a highly salient \is{Salience|)} potential antecedent (probably even the only available antecedent in this context), so it immediately suggests itself that both, the antecedent and topic drop, refer to the same person.

\ex.\label{ex:recoverable.context}
\ag.\label{ex:recoverable.context.ling}A: Was ist denn mit Sofia?\\
{} what is \textsc{part} with Sofia\\
A: `What's the matter with Sofia?'
\bg.\label{ex:recoverable.target}B: $\Delta$ Ist mir fremd gegangen!\\
{} she is me strange gone\\
B: `(She) cheated on me!' \citep[20]{fries1988}

\citet[76]{helmer2016} terms topic drop with referentially identical antecedents ``direct analepsis'' and distinguishes it from ``indirect analepsis'', similar to the distinction between direct and indirect anaphora \citep[e.g.,][284--286]{consten.schwarz-friesel2007}.%
%% Footnote
\footnote{\citet[284]{consten.schwarz-friesel2007} refer to similar concepts such as \citeg{clark1975} \textit{bridging} and \citeg{prince1981} \textit{inferrable} category (see below).
They likewise assume that a hearer establishes a relation between an expression and its ``anchor'' through an inference process.}
%
Indirect analepsis, which amounts to 20\% of the 541 instances of topic drop in Helmer's corpus \is{Corpus} (see \sectref{sec:prefield.detail}) \citep[74]{helmer2016}, is characterized by the fact that the referent of the antecedent and the referent of the omitted constituent are not identical but have a certain semantic relationship to each other, e.g., a metonymic one.
To this end, \citet{helmer2016} discusses example \ref{ex:indirect.analepsis}, taken from a conversation of a couple that wants to quit smoking. 
She argues that no immediate antecedent is available for the topic drop in \ref{ex:indirect.analepsis.td} and, based on the wider discourse context \citep[see][88--89, for details]{helmer2016}, that the referent of the covert expression would conceptually be something like \textit{the health (of all smokers) endangered by the many smoking}.%
%% Footnote
\footnote{Importantly, \citet[89, footnote 71]{helmer2016} clarifies that this is not to be understood as a lexical reconstruction but as a ``conceptual structure''.}
%
According to Helmer, it is linked to the discourse via a so-called \textit{linguistic anchor}, namely \textit{the many smoking} implicitly contained in A's ironic utterance \ref{ex:indirect.analepsis.anchor}.
Put simply, Helmer argues that the recovery of the omitted constituent works because there is a cause-effect relationship between the referent of the antecedent, i.e., the many smoking, and the referent of the omitted constituent, the endangered health.

\ex.\label{ex:indirect.analepsis}
\ag.A: Na, wollen wa noch eene roochen?\\
{} \textsc{inj} want we still one smoke\\
A: `Well, do we want to smoke another one?'
\bg.B: Joa\\
{} yep\\
B: `Yep.'
\cg.\label{ex:indirect.analepsis.anchor}A: Weil wa heut schon so wenig jeraucht ham?\\
{} because we today already so little smoked have\\
A: `Because we've already smoked so little today?'
\dg.\label{ex:indirect.analepsis.td}B: $\Delta$ is ja unsre jesundheit.\\
{}yes it is \textsc{part} our health\\
B: `(It) is our health.' [FOLK\_E\_00039\_SE\_01\_T\_03\_jesundheit] \citep[cited in][88, simplified]{helmer2016}

\largerpage
Example \ref{ex:indirect.analepsis.frac} from the fragment corpus \is{Corpus} FraC (see \sectref{sec:corpus.frac}) can serve as a less complex example of this type of indirect topic drop.
The topic drop in \ref{ex:indirect.analepsis.frac.td} cannot be recovered directly via a concrete linguistic antecedent but only indirectly by assuming what I informally term a process-result relationship.
The reference of the topic drop is the result of the mixing process described in \ref{ex:indirect.analepsis.frac.context}.

\ex.\label{ex:indirect.analepsis.frac}
\ag.\label{ex:indirect.analepsis.frac.context}Gleiche Mengen Giersch, Zitronenmelisse und Gundermannblättchen in Apfelsaft ziehen lassen, abseihen und mit Sprudelwasser (oder Prosecco?) verdünnen.\\
equal amounts goutweed lemon.balm and ground.ivy.leaves.\textsc{dim} in apple.juice steep let strain and with sparkling.water or prosecco dilute\\
`Steep equal quantities of goutweed, lemon balm, and ground ivy leaves in apple juice, strain and dilute with sparkling water (or Prosecco?).'
\bg.\label{ex:indirect.analepsis.frac.td}$\Delta$ Schmeckt prima an heißen Tagen...! \\
that tastes great on hot days\\
`(That) tastes great on hot days...!' [FraC B23--B24]

However, in both examples, the procedure of determining the reference via a cause-effect or process-result relationship would be analogous if there was an overt pronoun in the prefield.
This observation is supported by the fact that \citet[211]{helmer2016} classified 26.5\% of the \textit{das} (`that')-occurrences in her reference data (see \sectref{sec:usage.verb.type.studies} for details) as indirect anaphora, a figure that is even larger in absolute terms than the 20\% indirect cases of topic drop.

Despite the indirect relationship between the antecedent and the covert constituent in these examples, the respective utterances are nevertheless directly adjacent to each other.
\citet[217]{volodina.onea2012} argue that this is not coincidental but the result of a restriction.
They state that the referent of the omitted constituent needs to be named immediately before the omission without intervening utterances because only this would allow for what they call sufficient activation.%
%% Footnote
\footnote{They seem to make use of the concept of activation, discussed, for example, by \citet{chafe1994} (see below), although they do not explicitly refer to him.}
%
Consequently, they judge example \ref{ex:recover.volodina.onea.rain} as ungrammatical.%
%% Footnote
\footnote{\citet[299]{sandig2000} claims that topic drop is restricted to the utterance following the antecedent.}
%

\exg.\label{ex:recover.volodina.onea.rain}A: Hast du Marcus gesehen? Es regnet und es blitzt und es donnert so sehr. -- B: *(Er) ist im Haus.\\
{} have you.\textsc{2sg} Marcus seen it rains and it flashes and it thunders so very {} {} \phantom{*}he is in.the house\\
A: `Have you seen Marcus? It is raining and it is flashing, and it is thundering so much.' -- B: `(He) is in the house.' \citep[217, their judgment]{volodina.onea2012}

I do not share this judgment but consider the utterance with topic drop to be acceptable even with the intervening utterance.
This is supported by example \ref{ex:recover.frac.turns} from the dialogue subcorpus of the FraC,%
%% Footnote
\footnote{\is{Corpus} The data are transcripts of spoken dialogues about organizing a business trip collected as part of the Verbmobil project \citep{burger.etal2000} and stem from the Tübinger Baumbank des Deutschen/Spontansprache (TüBa-D/S, `Tübingen treebank of spoken German', \cite{hinrichs.etal2000}).
Unfortunately, the information about which speaker produced which utterance is not available.}
%
where several utterances lie between the antecedent \textit{das Maritim-Hotel} in \ref{ex:recover.frac.turns.ant} and the topic drop in \ref{ex:recover.frac.turns.td}.%
%% Footnote
\footnote{Theoretically, the antecedent of topic drop could also be the price per room from utterance \ref{ex:recover.frac.turns.price}.
However, the reading that the price is ``good middle class'' is very uncommon and, therefore, unlikely.
A more likely option is to assume that ``the price'' is needed for the statement that the hotel is middle class and that ``the hotel'' is reactivated by the immediately adjacent ``price'' from utterance \ref{ex:recover.frac.turns.price} in a kind of bridging (see above).
To determine the role of the distance independently of such alternative analyses, a controlled experiment is required.
}

\ex.\label{ex:recover.frac.turns}
\ag.\label{ex:recover.frac.turns.ant}tja, dann nehmen wir vielleicht gleich das Maritim-Hotel, das ist ziemlich zentral gelegen.\\
well then take we maybe right.away the Maritim-hotel that is pretty centrally located\\
`Well, then maybe we'll just take the Maritim Hotel, it is quite centrally located.'
\bg.Einzelzimmer hundert einundfünfzig Mark.\\
single.room hundred fifty.one German.mark\\
`Single room one hundred fifty-one German mark.'
\cg.\label{ex:recover.frac.turns.price}also, ich denke, das kann ich schon vertreten, ja, diesen Preis können wir schon noch nehmen.\\
well I think that can I \textsc{part} defend yes this price can we \textsc{part} still take\\
`Well, I think I can defend that, yes, we can still take this price.'
\dg.\label{ex:recover.frac.turns.backchannel}ja.\\
yes\\
`Yes.'
\eg.\label{ex:recover.frac.turns.td}gut, ja, $\Delta$ ist so gute Mittelklasse.\\
good yes it is so good middle.class\\
`Well, yes, (it) is good middle class.' [FraC D508--D512]

Examples like \ref{ex:recover.frac.turns} suggest that the statement by \citet{volodina.onea2012} according to which topic drop with intervening utterances is ungrammatical may be too strong.
Nevertheless, there seems to be a general tendency for the antecedent and the covert constituent to be close to each other. 
\citet[90]{helmer2016} reports that for the majority of topic drop cases in her data set, the utterance with the antecedent and the utterance with topic drop are immediately adjacent, or there are only two to three turns in between.
Furthermore, the intervening utterances are often incomplete through restarts or function merely as backchannelling, such as \ref{ex:recover.frac.turns.backchannel} above.
In these cases, they are argued not to disturb the linking between the antecedent and the covert constituent.

\largerpage[-2]
This observation of a short distance between the antecedent and topic drop coincides with the facts in the FraC, as shown in Figure \ref{fig:FraC.distance}.
The figure depicts the distance between the antecedent and the target for $322$ of the $873$ occurrences of topic drop in the FraC.
I measured this distance in the number of intervening utterances, i.e., $0$ means the antecedent and the target occur in the same utterance, $1$ means the utterances with the antecedent and the target are directly adjacent to each other, $2$ means  there is one utterance in between, etc.%
%% Footnote
\footnote{A distance could not be determined for the remaining $551$ cases because there was either no precontext available or no antecedent could be found in the precontext (or no postcedent could be found in the postcontext).
These were usually cases where the antecedent was known through the communicative situation, as in example \ref{ex:situation.antecedent} from the beginning of a blog post.
In this case, the text type blog post is such a strong hint toward the blogger writing about themselves that the pronoun referring to them can be omitted.
%\vspace{-0.5\baselineskip}
\exg.\label{ex:situation.antecedent}15.05.13 $\Delta$ Bin im Garten\\
15.05.13 I am in.the garden \\
`05/15/13 (I) am in the garden' [FraC B9--B10]

%\vspace{-1.5\baselineskip}
}

For the vast majority of the cases, the linguistic antecedent of topic drop occurs in the precontext.
This is in line with the central concept of my information-theoretic account of topic drop usage (see Chapter \ref{ch:infotheory}), namely predictability. \is{Predictability}
If an antecedent is present in the preceding discourse, the coreferential constituent becomes predictable \is{Predictability} and is more likely to be omitted (see \sectref{sec:avoid.troughs}).%
%% Footnote
\footnote{There are seven apparent counterexamples with a postcedent.
In three cases each, the postcedent is in the following or the same utterance, in one case in the utterance after the next.
They can be interpreted as the result of stylistic considerations rather than redundancy avoidance, e.g., in \ref{ex:postcedent} the intention is probably to first introduce the slogan and then the car.
%\vspace{-0.5\baselineskip}
\ex.\label{ex:postcedent}\ag.$\Delta$ MACHT DIE KURVE ZUR GERADEN \\
he makes the curve to.the straight \\
`(It) turns the curve into a straight line.'
\bg.DER NEUE NISSAN QASHQAI MIT CHASSIS CONTROL TECHNOLOGIE.\\
the new Nissan Qashqai with chassis control technology\\
`The new Nissan Qashqai with chassis control technology.' [FraC P143--P144]

%\vspace{-2em}
}
%

\begin{figure}
\centering
\includegraphics[scale=1]{Korpusplots/Bar_Distance_abb.pdf}
\caption[Distance between the antecedent (or postcedent) and topic drop in the FraC]{Distance between the antecedent or postcedent and topic drop in number of utterances in the FraC}
\label{fig:FraC.distance}
\end{figure}

In most of these cases ($211, 65.53\%$), the antecedent does not only occur somewhere in the preceding context but in the utterance that immediately precedes the utterance with topic drop.
A distance of two utterances is already less frequent with only $49$ instances ($15.22\%$) and includes many cases where topic drop occurs in two subsequent utterances after the antecedent.
Instances where the antecedent (or postcedent) occurs in the same utterance as topic drop ($13$, $4.04\%$) and instances where the distance is larger than two (in total $42$, $13.04\%$) are even rarer.%
%% Footnote
\footnote{See the file \textit{Examples\_Topic\_Drop\_Distances.pdf}, provided online at \url{https://osf.io/zh7tr}, for examples of different distances.
Most instances with a distance of more than five between the antecedent and the target stem from chat room conversations with numerous participants. 
In these conversations, multiple threads are often followed in parallel, and multiple utterances from different people can be understood as a response to an assertion.
These responses are then displayed one after the other, even if they have equal status as responses to the same antecedent utterance.
Therefore, these cases are probably also cases with distance one.}
%
This result supports the assumption that topic drop needs some proximity to its antecedent, which, in turn, is in line with the concept of predictability. \is{Predictability}
If a referent has been mentioned recently, that referent is most likely still (more) present in the hearer's mind, so any constituent that refers to that referent is also predictable \is{Predictability} and can be better omitted (see also the similar discussion on recoverability and givenness in \sectref{sec:recover.given}).

At the same time, the proximity between the antecedent and the target does not seem to be a unique property of topic drop.
Quite similar tendencies are observed for overt personal pronouns by \citet{portele.bader2016}.
They used the German DeWaC corpus \is{Corpus} \citep{baroni.etal2009} to analyze two random samples of about 500 overt p- and d-subject pronouns in sentence-initial position respectively, based on a total search result of more than 900\,000 instances.
They considered a precontext of five sentences and determined the last DP that was co-referent with the target pronoun as the antecedent \citep[11--13]{portele.bader2016}.
Table \ref{tab:portele.bader} shows the proportion of sentences with a certain distance between the antecedent and the target pronoun, subdivided for pronoun type, using the same measurement of the number of utterances as in Figure \ref{fig:FraC.distance} for topic drop.

\begin{table}
\centering
\caption[Proportions of distances between overt pronoun and antecedent in \citet{portele.bader2016}.]{Proportions of distances between an overt pronoun and its antecedent in \citet[16]{portele.bader2016}, original distance measure adapted to fit Figure \ref{fig:FraC.distance}}
\begin{tabular}{lccccc}
\lsptoprule
 & \multicolumn{5}{c}{Distance antecedent-target}\\
\multirow{-2}{*}{Pronoun type} & $1$ & $2$ & $3$ & $4$ & $5$\\
\midrule
P-pronoun & $91.1\%$ & \phantom{2}$6.7\%$ & \phantom{2}$2.6\%$ & \phantom{2}$0.0\%$& \phantom{2}$0.0\%$\\
D-pronoun & $95.8\%$ & \phantom{2}$3.5\%$ & \phantom{2}$0.5\%$ & \phantom{2}$0.2\%$ & \phantom{2}$0.0\%$\\
\lspbottomrule
\end{tabular}
\label{tab:portele.bader}
\end{table}

For their sample at least, the effect of recency seems to be even more pronounced for the overt pronouns than for topic drop.
For over 90\% of the sentence-initial subject pronouns, the antecedent occurs in the immediately preceding utterance.
\citet[19]{portele.bader2016} argue that unlike lexical NPs, which often allow for a longer distance to their antecedent according to \citet{arnold2010}, both pronoun types ``seem to require an antecedent that occurred recently and is therefore in an activated state in working memory.''
Consequently, not only the necessary proximity to the antecedent seems to be another parallel between topic drop and overt pronouns but also the relation to associated cognitive activity as the underlying explanation.

\section{Extralinguistic antecedents}\label{sec:recover.extra}
Besides the postulated restriction with respect to the distance between the antecedent and topic drop, \citet[217]{volodina.onea2012} additionally claim that for topic drop to be felicitous the referent needs to be explicitly (verbally) named.
By presenting  the contrast in examples \ref{ex:recover.volodina.onea} and \ref{ex:recover.volodina.onea.point}, they deny that topic drop can be recovered from the extralinguistic context.
However, the variant with the recovery from the extralinguistic context \ref{ex:recover.volodina.onea.point} is highly artificial because it is completely unclear why B would point to a portrait in this situation. 

\ex.\label{ex:recover.volodina.onea}
\ag.A: Hast du Marcus gesehen? \\
{} have you.\textsc{2sg} Marcus seen\\
A: `Have you seen Marucs?'
\bg.B: $\Delta$ Ist im Haus.\\
{} he is in.the house\\
B: `(He) is in the house.'

\exg.\label{ex:recover.volodina.onea.point}\emph{[}\emph{A} \emph{zeigt} \emph{auf} \emph{Marcus} \emph{Portrait.}\emph{]} B: *$\Delta$ Ist im Haus.\\
\phantom{(}A points to Marcus's portrait {} he is in.the house\\
[\emph{A points to Marcus's portrait.}] B: `(He) is in the house.' \citep[217, their judgments]{volodina.onea2012}

More natural examples like \ref{ex:recover.volodina.onea.point.better} indicate that topic drop can be acceptable although the referent is not explicitly named but only present in the current utterance situation.

\ex.\label{ex:recover.volodina.onea.point.better} [A, B, and C are in a restaurant. When C barks aggressively at the waiter, A asks B in a whisper:]
\ag.A: Was ist los?\\
{} what is loose\\
A: `What's wrong?'
\bg.B: $\Delta$ Hat schlecht geschlafen.\\
{} he has badly slept\\
B: `(He) slept badly.'

This view is shared by \citet{fries1988}, \citet{cardinaletti1990}, \citet{zifonun.etal1997},%
%% Footnote
\footnote{\citet{zifonun.etal1997} assume two different processes here depending on the nature of the context from which the ellipsis is recovered.
For linguistic contexts, they talk about \textit{Analepse} (`analepsis'), presumably a composition of \textit{Anapher} (`anaphora') and \textit{Ellipse} (`ellipsis') borrowed from \citet{blatz1896} \citep[569--571]{zifonun.etal1997}.
If the context is extralinguistic, they use the term \textit{Ellipse} (`ellipsis') \citep[413]{zifonun.etal1997}.
For topic drop, they explicitly discuss both possibilities, i.e., that its reference is recovered from the linguistic or the extralinguistic context \citep[636]{zifonun.etal1997}.}
%
and \citet{volodina2011}, who state that the referent of the omitted constituent can be recovered both from the linguistic and the extralinguistic context so that topic drop can also refer to objects or persons that are present in the current situation like the speaker, the hearer, or a movie in the form of a poster \ref{ex:recoverable.context.2}.

\ex.\label{ex:recoverable.context.2}
\a.\label{ex:recoverable.context.sit}[A and B standing in front of a poster for the new Downton Abbey movie.]\hfill(situational antecedent)
\bg.\label{ex:recoverable.target.2}B: $\Delta$ Muss ich unbedingt sehen!\\
{} that must I desperately  see\\
B: `I really have to see (it)!'

\citet[73]{frick2017} specifies that the recovery of topic drop from the extralinguistic context requires that the speaker and the hearer(s) share a mutual context in the concrete communicative situation, a context that is shaped by an orientation at the origo or deictic center (I, here, now) \citep[102]{buhler1965} (see also \sectref{sec:usage.person.theory}).
Accordingly, elements that can be recovered based on such a context orientation can be targeted by topic drop.

\section{Non-referential elements}\label{sec:recover.non.referential} \is{Expletive|(}
While resolving topic drop of referential constituents requires recovering both the omitted constituent and its referent, the situation is different for non-referen- tial constituents like expletive subjects.
Since expletives are not referential but semantically empty, there is no reference to recover.
The question, then, is how hearers handle utterances in which an expletive is omitted.
There are principally two options.
Hearers could reconstruct this expletive as a semantically empty element, i.e., insert it into the prefield during sentence processing.
Or they could process the utterance directly without a prefield constituent so that in this case no reconstruction takes place at all.
At this point, I cannot decide between the two variants, and it is also not clear to me which empirical method should succeed in such a differentiation.
In any case, a hearer will understand an elliptical utterance like \ref{ex:regnet} in the intended way, either because they can unambiguously recover the omitted \textit{es} (`it') as the prefield constituent, or because they do not need to recover anything, since all semantically relevant information is given.
Therefore, we can conclude with \citet[137]{frick2017} that expletives are uniquely recoverable and identifiable, which favors their omission.

\exg.\label{ex:regnet}$\Delta$ Regnet.\\
it rains\\
`(It) is raining.'

\is{Expletive|)}

\largerpage
\section{Recoverability and givenness}\label{sec:recover.given} \is{Givenness|(}
I now turn to the question of what kind of relationship exists between the omitted referential constituent and its linguistic antecedent \is{Antecedent|(} and/or extralinguistic referent that allows for recoverability.
In the literature, it is common to define this relationship as some form of categorical givenness relation.
For example, \citet[27]{fries1988} uses givenness as a proxy to recoverability when he demands that the constituent targeted by topic drop is coreferential to a ``known quantity''%
%% Footnote
\footnote{My translation, the original: ``bekannte[] Größe'' \citep[27]{fries1988}.}
%
in the discourse.
An extension of the givenness relation is proposed by \citet[19]{trutkowski2016}, who states in her ``[s]alience/givenness condition'' that ``[t]he referent of a topic dropped element must be inferrable/contextually given (via some antecedent \is{Antecedent|(} expression).''
Similarly, \citet[150]{freywald2020} postulates:
``The dropped element always refers to an entity that is highly salient%
%% Footnote
\footnote{Both authors use the concept of salience \is{Salience} differently from \citet{ariel1990}, who equates it with topicality, see below.
While \citet{freywald2020} simply states that salient means being inferrable, \citet{trutkowski2016} refers to \citet{chiarcos.etal2011} and adopts their definition of salience:
``Salience defines the degree of relative prominence of a unit of information, at a specific point in time, in comparison to the other units of information'' \citep[2]{chiarcos.etal2011}.
\citet[20]{trutkowski2016} admits that this definition is rather vague, and clarifies that for her a salient referent is (highly) accessible in a given discourse without having to be a topic. \is{Topic} 
See also the review article by \citet{boswijk.coler2020} on different concepts of salience in linguistics.
}
%
\is{Salience} in the discourse (i.e., that can be inferred from the discourse context).''%
%% Footnote
\footnote{My translation, the original: ``[D]as gedroppte Element referiert stets auf eine Entität, die im Diskurs hoch salient ist (d.h. aus dem Diskurskontext inferiert werden kann)'' \citep[150]{freywald2020}.}
%
From these two statements, it follows that the referent of topic drop need not necessarily be given (in the sense of, e.g., \citet{prince1981}); it is sufficient that it is inferrable, a term that neither \citet{trutkowski2016} nor \citet{freywald2020} explicitly defines.

\textit{Inferrable} is often used as a kind of middle point of givenness, located between given and new.
In this way, it was introduced as a linguistic term by \citet{prince1981}.
Prince created a taxonomy distinguishing a total of seven different states of givenness, which she terms ``assumed familiarity'' \citep[233; 237]{prince1981}, shown in Figure \ref{fig:prince}.
She distinguishes between new, given -- which she terms evoked -- and inferrable discourse entities and defines the latter in the following way:
``A discourse entity is Inferrable if the speaker assumes that the hearer can infer it, via logical --- or, more commonly, plausible --- reasoning, from discourse entities already Evoked or from other Inferrables'' \citep[236]{prince1981}.
She gives the example \ref{ex:prince.inferrable.basic}, where the discourse entity \textit{the driver} is a \textit{(noncontaining) inferrable} because it can be inferred from the textually evoked entity \textit{a bus} and from the world knowledge that buses usually have a driver.
In example \ref{ex:prince.inferrable.containing}, \textit{one of these eggs} is a so-called \textit{containing inferrable} since the hearer can infer it by a set-member inference from the situationally evoked discourse entity \textit{these eggs}.

\ex.\label{ex:prince.inferrable}
\a.\label{ex:prince.inferrable.basic} I got on a bus yesterday and the driver was drunk.
\b.\label{ex:prince.inferrable.containing} Hey, one of these eggs is broken! \citep[233]{prince1981}

So, if we combine the statements by \citet{trutkowski2016} and \citet{freywald2020} with \citeg{prince1981} taxonomy, recoverability in their sense means that the constituent targeted by topic drop must not be new, i.e., it must be either textually or situationally evoked or a noncontaining or containing inferrable.
This way, the concept of inferrable may also be suitable to describe the cases that I discussed as indirect topic drop above.
For example, in \ref{ex:indirect.analepsis.frac.td} above, the discourse entity to which the topic drop refers can be inferred from the ingredients mentioned in the previous utterance.

\begin{figure}
\centering
{\footnotesize
\begin{forest}, baseline, qtree, for tree={l sep=1.2cm,
    s sep=0.2cm},
[Assumed Familiarity, baseline
	[New
		[Brand-new
			[Brand-new\\(Unanchored)]
			[Brand-new\\Anchored]
		]
		[Unused]
	]
	[Inferrable
		[(Noncontaining)\\Inferrable]
		[Containing\\Inferrable]
	]
	[Evoked
		[(Textually)\\Evoked]
		[Situationally\\Evoked]
	]
]
\end{forest}}
\caption{\citeg{prince1981} levels of assumed familiarity, recreated from \citet[237]{prince1981}}
\label{fig:prince}
\end{figure}

While \citeg{prince1981} taxonomy is based on a categorical notion of givenness, there are also gradual approaches to givenness.
For example, \citet{chafe1994} defines it with recourse to activation, which is a cognitive concept used, roughly speaking, to describe the focus of a speaker's consciousness \citep[see also][]{chafe1974}.
According to \citet{chafe1994}, there are three activation states, i.e., \textit{active}, \textit{semiactive}, and \textit{inactive},%
%% Footnote
\footnote{For convenience, Chafe argues only with these three states, but basically, he assumes a continuum of activation states \citep[55]{chafe1994}.
Given the direct mapping between activation and givenness, he seems to also assume a gradual givenness concept.}
%
which he links to three levels of givenness -- given, accessible, and new -- using the concept of \textit{activation cost}:

\begin{quote}
Suppose that at a certain time, \emph{t\textsubscript{1}}, a particular idea is active, semiactive, or inactive. Suppose that at a later time, \emph{t\textsubscript{2}}, whatever its earlier state may have been, this idea is now active. If it was already active at \emph{t\textsubscript{1}}, we can say that at \emph{t\textsubscript{2}} it is given information. If it was semiactive at \emph{t\textsubscript{1}}, it is accessible at \emph{t\textsubscript{2}}. If it was inactive at \emph{t\textsubscript{1}}, it is new at \emph{t\textsubscript{2}}. It is helpful to think of these three processes in terms of cognitive cost: \is{Processing effort|(} given information is least costly in the transition from \emph{t\textsubscript{1}} to \emph{t\textsubscript{2}} because it was already active at \emph{t\textsubscript{1}}. Accessible information is somewhat more costly, and new information is the most costly of all, presumably because more mental effort is involved in converting an idea from the inactive to the active state. \citep[72--73]{chafe1994}
\end{quote}

\noindent
\citet[75]{chafe1994} claims that usually speakers use accented full noun phrases to express new and accessible information while given information is expressed using weakly accented pronouns or null pronouns.
The link between activation and referential form is picked up by \citet{volodina.onea2012} (without reference to Chafe) and by \citet{helmer2016} (with reference to Chafe).
They argue that a constituent can usually only be targeted by topic drop if the entity it refers to is given, i.e., if it was already active before the utterance with topic drop in the sense of \citet{chafe1994}.
According to \citet{volodina.onea2012} and \citet{helmer2016}, recoverability corresponds to having been active in the speaker's consciousness, so the cognitive cost \is{Processing effort|)} in accessing the referent of the omitted constituent is low.
This seems like a more concrete formulation of \citet[102]{schwitalla2012}, who claims that constituents can be omitted because their referent is still present in memory.

The link between activation cost and referential form has parallels to \citeg{ariel1990} continuous \textit{accessibility marking scale}.%
%% Footnote
\footnote{See also \citeg{ackema.neeleman2007} application of \citeg{ariel1990} accessibility theory to \textit{pro} drop \is{@\emph{pro}-drop} and topic drop in Early Modern Dutch, \il{Dutch} which I discussed briefly in Footnote \ref{note:ackema}.
}
Along this scale, \citet[73]{ariel1990}  arranges referential expressions as so-called ``accessibility markers'' from very explicit markers (e.g., full names plus modifiers) at the top to very implicit markers like gaps, i.e., null elements, at the bottom.
She argues that the closer a marker is toward the bottom, the more accessible its antecedent  \is{Antecedent} needs to be and vice versa.
A lowly accessible antecedent \is{Antecedent} is rather picked up with a full name plus a modifier (or a definite description, etc.), while a highly accessible antecedent can be referred to using a gap (or at least a cliticized or unstressed pronoun).
According to \citet{ariel1990}, accessibility%
% Footnote
\footnote{\citeg{ariel1990} term \textit{accessibility} is distinct from \citeg{chafe1994} \textit{accessible}.
For Ariel, \textit{accessibility} is comparable to \citeg{chafe1994} concept of activation, while Chafe uses \textit{accessible} to denote an intermediate givenness level.
}
%
is determined by at least the following four factors:

%\vspace{-0.5\baselineskip}
\begin{quote}
\begin{enumerate}
\item[a] Distance: The distance between the antecedent and the anaphor (relevant to subsequent mentions only).
\item[b] Competition: The number of competitors on the role of antecedent.
\item[c]	 Saliency:\is{Salience|(} The antecedent being a salient referent, mainly whether it is a topic or a non-topic. \is{Topic} 
\item[d] Unity: The antecedent being within vs. without the same frame\slash world\slash point of view\slash segment or paragraph as the anaphor. \citep[28--29]{ariel1990}
\end{enumerate}
\end{quote}
%\vspace{-0.5\baselineskip}

\noindent
\citet[29]{ariel1990} argues that an antecedent is ``more likely to be in a highly activated state in memory'' and, therefore, more likely to be picked up with an accessibility marker from the lower pole of the scale, e.g., with topic drop, if it is spatially close to this marker, if it has few competitors, if it is salient, which Ariel mainly defines as being a topic, and if it is more strongly connected in terms of coherence to the marker.
In sum, \citet{ariel1990} considers recoverability to be a form of activation or presence in memory or consciousness, similar to \citet{chafe1994}.
She argues that the degree of activation does not only depend on givenness but also on the four factors introduced: distance, competition, saliency, \is{Salience|)} and unity.
\is{Givenness|)}

\section{Recoverability as felicity or usage condition}\label{sec:recover.summ}
In this section, I come back to the definition of recoverability presented at the beginning of this chapter and expand it in light of the further insights gained.
I argued that recoverability means that the hearer must be able to uniquely identify the (reference of the) covert constituent.
For referential constituents, the recovery requires some kind of antecedent or ``anchor'' in the extralinguistic or linguistic context, as shown in Figure \ref{fig:recoverability}.%
%% Footnote
\footnote{I created this figure in Microsoft Word using pictures from pixabay. Source for the castle: \url{https://pixabay.com/de/vectors/highclere-castle-downton-abbey-4515425/} (visited on 05/09/2022). Source for the group of people: \url{https://pixabay.com/de/vectors/junge-kind-zusammenarbeit-vati-2026064/} (visited on 01/02/2024).}
%
In the case of linguistic context, topic drop and the antecedent tend to be close to each other, often in directly adjacent utterances.

\begin{figure}
\centering
\includegraphics[scale=0.2]{Recoverability2.PNG}
\caption{Relation between the omitted constituent, the antecedent, and the referent (loosely based on a figure by \cite[76]{helmer2016})}
\label{fig:recoverability}
\end{figure}

\noindent
Following \citet{helmer2016}, I distinguish between two ``modes of recovery'' for referential topic drop: direct and indirect topic drop.
For direct topic drop, the antecedent and the covert constituent do not have to be formally identical but  are typically coreferential.
This means that the hearer can directly determine the reference of the omitted constituent when they identify the coreferential antecedent since the referent of this antecedent is already known to them.
This is illustrated in the example of Figure \ref{fig:recoverability}, where \textit{the new Downton Abbey movie} is the referent of both the linguistic antecedent and the omitted constituent.
For indirect topic drop, however, the two referents are distinct but have a certain semantic or pragmatic relationship to each other that the hearer must determine.
The mechanisms sketched for recovering topic drop are parallel to the commonly assumed mechanisms that underlie the use of (personal) pronouns and, more generally, anaphora \citep[see, e.g.,][]{huang2000,bhat2004}.

In this book, I additionally draw on the idea that the recoverability of referential topic drop can be described by concepts of givenness.
While givenness can be understood as categorical, i.e., a constituent can be targeted by topic drop if its referent is not new, i.e., if it is given \is{Givenness} in or inferrable from the current discourse context in \citeg{prince1981} taxonomy, I pursue a gradual givenness concept.
Following \citet{chafe1994} and \citet{ariel1990}, I assume that the recoverability of a covert constituent depends on how accessible its referent is from memory or how strongly activated this referent is in consciousness.
Such a gradual givenness concept includes cognitive processes and is argumentatively closer to the information-theoretic  approach that I present and discuss in the second part of this book.
More specifically, in \sectref{sec:resolving}, I argue that the verb following topic drop may play a role in facilitating the recovery of the omitted constituent and reducing the overall processing effort \is{Processing effort} for the hearer.

A gradual notion of givenness almost inevitably leads to a gradual notion of recoverability, as I assume in this book.
The scale of recoverability ranges from (i) cases at the upper end, which can be recovered unambiguously, to (ii) cases at the lower end, where topic drop could be merely recovered by guessing.
Cases like (i) are the above-mentioned omission of the expletive \is{Expletive} \textit{es}, which can be uniquely recovered but also the omission of a 1st person singular subject before a verb that is distinctly marked for inflection. \is{Verbal inflection}
An example for cases like (ii) is \ref{ex:topic.unrecent.td} from Chapter \ref{ch:topicality}, repeated here as \ref{ex:topic.unrecent.td.rep}.
If it is uttered out of the blue, i.e., without a previous discourse in which John was mentioned or without John being present in the current situation, the hearer cannot determine the referent of topic drop but can only guess wildly.

\ex.\label{ex:topic.unrecent.td.rep}  Did you know? *(John) married last week.

I argue that in such cases, topic drop is not ungrammatical but simply infelicitous.
From this argumentation, it follows that recoverability is not a licensing condition for topic drop, as it is the prefield restriction, but rather a felicity or usage condition.
While in example \ref{ex:topic.unrecent.td.rep}, the speaker, unless completely lost in thought, should realize that their hearer has no chance of retrieving the referent of topic drop, there may be cases where this is not so obvious.
That is, it could be that the referent is sufficiently given \is{Givenness} to the speaker but not given \is{Givenness} enough to the hearer.
This idea is supported by \citet{helmer2016}.
She states that topic drop is generally well understood in her data despite instances of indirect topic drop and existing mismatches between the  antecedent and the target \citep[181]{helmer2016}.
However, she also found instances in her corpus \is{Corpus} where the hearer asked a comprehension question after an utterance with topic drop.
This indicates that they were unable to recover the reference of the ellipsis or were at least not completely sure if their reconstruction was correct \citep[190--191]{helmer2016}.
Such a case is shown in example \ref{ex:td.comp.question} from a conversation during a soccer manager game.
The player Simon produces a topic drop that another player, Jan, apparently cannot recover.
This leads first to Jan asking \textit{what?} and then Simon making the reference of the topic drop explicit by verbalizing the proposition the ellipsis referred to (\textit{dass du auch noch einsteigen willst}). 

\ex.\label{ex:td.comp.question}[\textit{Simon, Maik, and Jan are playing a soccer manager game with other friends and are currently bidding on a player.
Maik is in the lead with 14 million.}]
\ag.Simon: Vierzehn fürn Maik.\\
{} fourteen for.the Maik\\
Simon: `Fourteen for Maik.'
\bg.Simon: Jan.\\
{} Jan\\
Simon: `Jan.'
\cg. [...] Simon: $\Delta$ Hätt ja sein können.\\
{} {} it had.\textsc{conj} \textsc{part} be can\\
Simon: `(It) could have been.'
\dg. [...] Jan: Was?\\
{} {} what\\
[...] Jan: `What?'
\eg.Simon: $\Delta$ Hätt ja sein können, dass du auch noch einsteigen willst.\\
{} it had.\textsc{conj} \textsc{part} be can that you also still get.in want\\
Simon: `(It) could have been that you also wanted to get in.' \citep[190, adapted]{helmer2016}

\largerpage
In other cases, the speaker apparently preempts such a comprehension question by making the reference of the omitted constituent explicit and verbalizing it in the postfield or a subsequent utterance \citep[191--192]{helmer2016}.
This suggests that in many cases it depends not only on the referent's givenness but also on the hearers and their general and situational cognitive resources \is{Processing effort} whether and how well an omitted referent can be recovered and, consequently, omitted.

In sum, I consider recoverability to be a felicity or usage condition for topic drop.
Topic drop may be syntactically licensed, i.e., the omission occurs in the prefield as required, but it may still fail if the hearer cannot retrieve the omitted constituent or its reference, as in example \ref{ex:topic.unrecent.td.rep} above.
I assume that recoverability is gradual in nature, i.e., that topic drop can succeed more or less well depending on how easy it is to resolve the ellipsis, which in turn depends on how much cognitive effort \is{Processing effort} is required to do so.
Such a link between the usage of topic drop and assumed underlying cognitive processes anticipates the second part of this book.
In Chapter \ref{ch:infotheory}, I argue that the usage of topic drop is guided by the intention to distribute processing effort \is{Processing effort} efficiently across the utterance and that this is also associated with recovering the omitted constituent based on cues in the following context.
Therefore, I come back to recoverability in \sectref{sec:resolving}.

\is{Antecedent|)}\is{Recoverability|)}
