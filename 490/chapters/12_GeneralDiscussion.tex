\chapter{General discussion}\label{ch:discussion}
The goal of this work was to provide a comprehensive account of the ellipsis type topic drop in German that incorporates both aspects of licensing and aspects of usage.
To this end, I followed a strongly empirical approach, relying on corpus and experimental data.
After having defined topic drop, I focused on two research questions, each of which I will also summarize and discuss in a section below.
At the same time, this book also contributes to a systematic discussion and, most importantly, an empirical investigation of isolated claims from the theoretical literature, in some cases for the first time.
I will mention and highlight the corresponding results concerning these controversies at the appropriate places.

\section{Definition and typological perspective}
Before turning to the investigation of the two research questions, I first preliminarily defined topic drop in German as an ellipsis type that is restricted to the prefield of declarative V2 clauses. \is{Prefield}
I discussed that topic drop is typically used in certain text types. \is{Text type|(}
This property is not part of its definition, but it is nevertheless characteristic of topic drop.
By means of my corpus \is{Corpus} study on the text type-balanced \is{Text type} fragment corpus FraC \citep{horch.reich2017}, I could show that topic drop is not bound to mainly one specific text type, like null articles, \is{Article omission} null copulas (headlines), \is{Copula omission} or object omission \is{Object omission} (recipes and instructions) are. 
Instead, there are several text types in which topic drop occurs frequently, such as text messages, colloquial dialogues, blogs, and online chats, many of which share several features that \citet{koch.oesterreicher1985} describe as characteristic of conceptually spoken text types. \is{Text type}

In the second step, I distinguished topic drop from similar phenomena such as \textit{pro}-drop \is{@\emph{pro}-drop} and V1 declaratives.
In doing so, I was taking a stand on the first topic drop controversy because in the literature topic drop has occasionally been both analyzed as \textit{pro}-drop and equated with V1 declaratives. \is{V1 declarative}
Furthermore, I provided a typological overview of topic drop in other Germanic V2 languages and also discussed similarities of topic drop to omissions in English, \ili{French}, and Russian, \il{Russian} i.e., the left-peripheral positioning of topic drop and null subjects \is{Null subject} in \ili{English} and \ili{French} and the typical occurrence in certain registers or text types \is{Text type|)} of topic drop and the null arguments \is{Argument omission} in \ili{English}, \ili{French}, and Russian. \il{Russian}
I argued that a joint consideration of, in particular, left-peripheral register- or text-type-dependent ellipsis types in several languages could be a promising perspective for future research.

\section{When is topic drop licensed?}
The first research question of this work was the question of when topic drop is syntactically licensed in German.

\subsection{Prefield restriction as licensing condition}\is{Prefield|(}
Here, I looked in detail at the prefield restriction of topic drop.
I reviewed several hypotheses and observations from the literature and refined the restriction based on my results and on proposals by \citet{rizzi1994} and \citet{freywald2020}.
I argued on theoretical and empirical grounds that topic drop is not restricted to the deletion of topics in the prefield but may target any recoverable \is{Recoverability} constituent in a prefield that is not c-commanded \is{C-command|(} sentence-internally by a potential identifier or that is the highest prefield of a root clause.

In a first step, I rejected the widely held view in the literature that topic drop is restricted exclusively to topics, thereby touching upon another topic drop controversy.
I first argued with \citet{frey2000} against characterizing the prefield position as the topic position in German main clauses.
Second, I concluded, referring to the impossibility of omitting contrastive and non-recoverable topics, that topicality is not a (strictly) sufficient condition for topic drop.
Building on corpus \is{Corpus} data and a rating study, I was furthermore able to show that non-referential and, thus, non-topical expletive \is{Expletive} subjects, especially those of weather verbs, can be targeted by topic drop.
Consequently, I concluded that topicality is not a necessary condition for topic drop either.
It would presumably be more accurate to term topic drop not \textit{topic drop} but \textit{prefield ellipsis}.

I argued that the prefield restriction of topic drop is purely positional and narrowed down its nature more precisely with the help of three experiments.
In experiment \ref*{exp:pf.mf}, I compared topic drop in the prefield to corresponding omissions in the middle field \is{Middle field|(} and found that the omission from the prefield is significantly more acceptable.
This result is in line with the prefield restriction of topic drop and contributes to clarifying a further topic drop controversy  as it calls into question \citeg{helmer2016} claim that the omission of verb arguments \is{Argument omission}\is{Argument} in the middle field should also be considered as topic drop.
Experiment \ref*{exp:embedded} provided evidence that topic drop is not licensed in dependent V2 clauses that follow the matrix clause. \is{Embedding}
However, if the V2 complement precedes the matrix clause, topic drop is acceptable.
It is not fully clear, though, what the syntax of preceding V2 complements exactly looks like.
They can be analyzed as occupying the prefield (but see \cite{reis1997}), as main clauses with an integrated V1 parenthetical, or as forming a quasi-paratactic structure together with another clause.
Thus, since it cannot be conclusively clarified whether they are embedded \is{Embedding} clauses, it is also not possible to answer the question, controversially discussed in the literature, whether topic drop in embedded clauses is possible or not.

Following the proposals by \citet{rizzi1994} and \citet{freywald2020}, I took the results concerning the prefield restriction and the dependent V2 clauses as evidence that topic drop cannot occur in any prefield but only in a prefield that is not c-commanded sentence-internally by a potential identifier or that is the highest prefield of a root clause.
These proposals can also be reconciled with the results of experiment \ref*{exp:conjunctions}.
This study provided evidence that conjunctions \is{Conjunction} such as \textit{und} and \textit{aber} can precede topic drop, i.e., that it is not strictly restricted to the sentence-initial position.
Even if the conjunction \is{Conjunction} is taken to belong to the second conjunct, this conjunct would be a root clause and the prefield constituent would still not be c-commanded by a potential identifier but by a non-referential functional category. \is{Conjunction phrase} \is{C-command|)}
Referring to previous research, I discussed the three main generative approaches to modeling topic drop, the operator approach, \is{Operator analysis} the \textit{pro}-approach, \is{@\emph{pro}-analysis} and the PF-deletion approach. \is{PF-deletion analysis}
After weighing their respective advantages and disadvantages, I argued, following Occam's Razor,  for the PF-deletion approach \is{PF-deletion analysis} because it can account for expletive \is{Expletive} omission while requiring the fewest additional assumptions.

In summary, I was able to define the prefield restriction of topic drop as its central syntactic licensing condition more precisely and argued that the PF-deletion approach \is{PF-deletion analysis} currently offers the most promising analysis.
By simultaneously providing evidence against two claims from the literature, namely that topic drop can occur in the middle field \is{Middle field|)} and that it is restricted to topics, I have thus contributed to an empirically adequate account of topic drop licensing that can inform further theoretical, as well as empirical engagement with this ellipsis type.
\is{Prefield|)}

\subsection{Recoverability as felicity condition}\is{Recoverability|(}
Also related to my first research question, I examined the role of recoverability, which is not exclusive to topic drop but a prerequisite for any type of omission.
I concluded that it is a felicity or usage condition rather than a licensing condition. 
This means that an utterance with topic drop in which the omitted constituent cannot be recovered is not ungrammatical but infelicitous.
I reviewed the previous literature on the role of recoverability and argued that omitted referential constituents can have antecedents \is{Antecedent} in the linguistic as well as in the extralinguistic context. 
In the case of the former, the antecedent \is{Antecedent} and topic drop tend to occur close to each other.
Moreover, there can be a direct or indirect relationship between the antecedent and the covert constituent.
For expletive \is{Expletive} subjects, I argued that they are trivially recoverable because they convey no semantic content.

I proposed to model the recoverability of referential omitted constituents by a gradual givenness relation.
Following \citet{chafe1994} and \citet{ariel1990}, I consider a constituent to be more given \is{Givenness} when it is more strongly activated or accessible.
A higher degree of givenness also leads to better recoverability and a reduced processing effort \is{Processing effort} on the part of the hearer who has to recover the ellipsis.
Thus, while there are likely cases in which recovery always succeeds (e.g., with covert expletives) \is{Expletive} and cases in which it always fails (e.g., if the omitted constituent is neither mentioned in the preceding discourse nor present in the utterance situation), most cases will occupy an intermediate position.
That is, in these cases, how easy it is to recover topic drop depends on the current situation and the cognitive state of the hearer.
With this reasoning, I established a link between recoverability as a felicity condition and recoverability as a factor determining the usage of topic drop.
Here it became apparent that the boundary between licensing or felicity and usage cannot always be drawn sharply.
\is{Recoverability|)}

\section{When is topic drop used?}
The second part of my book was concerned with the usage of topic drop, i.e., with answering the question of when speakers decide to use topic drop instead of the corresponding full form, given that the ellipsis is licensed and recoverable.
To this end, I proposed an information-theoretic approach, according to which the usage of topic drop is guided by the intention to distribute processing effort \is{Processing effort|(} efficiently across utterances, following the three principles \textit{avoid troughs}, \textit{avoid peaks}, and \textit{facilitate recovery}. \is{Recoverability}

\is{Uniform information density|(}
\subsection{Information-theoretic account of topic drop usage}
The information-theoretic account that I advocated in this book is based on a probabilistic notion of information (surprisal) \is{Information} going back to the seminal work by \citet{shannon1948}.
It also draws on the \textit{uniform information density hypothesis} (\textit{UID}) proposed by \citet{levy.jaeger2007}.
\textit{UID} predicts that to communicate efficiently, speakers should prefer utterances with a more uniform distribution of information over utterances with a less uniform one, provided they are grammatical.
I argued, first, that conforming to \textit{UID} amounts to an efficient use of the processing resources available to the hearer, and, second, that the two principles derivable from \textit{UID}, \textit{avoid troughs} and \textit{avoid peaks}, guide the usage of topic drop.
Accordingly, speakers are assumed to use topic drop to omit predictable \is{Predictability} expressions and to prevent underutilization of the hearer's processing resources.
Speakers are predicted not to use topic drop if the overt prefield constituent is needed to reduce the processing effort of the following constituent.
In addition to these two principles derived from \textit{UID}, \is{Uniform information density} I proposed a third principle, \textit{facilitate recovery}. \is{Recoverability}
It predicts that, for example, a distinct inflectional ending \is{Verbal inflection} on the verb after topic drop can also reduce the processing effort on that verb that is caused by ellipsis resolution.

The proposed information-theoretic approach offers two advantages.
First, by analyzing the usage of topic drop in terms of information theory, I place myself in a tradition of similar approaches that have successfully accounted for the usage of various types of omissions by building on \textit{UID}.
Second, this approach explains how general processing principles can account for the usage of a very specific phenomenon such as topic drop, principles that are argued to shape language in general anyway. \is{Processing effort|)}
Certain properties of topic drop, which were previously explained by rules in the grammatical system, can be argued to largely follow from these principles, which unburdens and simplifies the grammatical system.

\subsection{Evidence for the \textit{avoid troughs} priciple}
Evidence for the \textit{avoid troughs} priciple was provided mainly by the attested influence of  grammatical person.
In my investigations of this factor, I focused on the observation made in the literature and confirmed by several previous corpus studies \is{Corpus} that the 1st person singular subject pronoun \textit{ich} (`I') is omitted particularly frequently.
As possible explanations, I discussed \citeg{auer1993} \textit{inflectional hypothesis}, as well as two types of \textit{extralinguistic hypotheses}.
The \textit{extralinguistic 1SG hypotheses} predict that the 1st person singular subject pronoun referring to the speaker can be easily recovered \is{Recoverability} because they are part of the origo of speaking and known through text type knowledge. \is{Text type}
The \textit{extralinguistic 1SG+2SG hypotheses} argue that both the 1st and 2nd person can be readily omitted because they refer to the speaker and the hearer as integral components of every communicative situation.
I argued that the \textit{inflectional} and both \textit{extralinguistic hypotheses} are covered by the information-theoretic approach, more specifically by the \textit{facilitate recovery} priciple \is{Recoverability} (for the \textit{inflectional hypothesis)} and the \textit{avoid troughs} priciple (for the \textit{extralinguistic hypotheses}).
\is{Verbal inflection}

In the corpus study, \is{Corpus} I found that 1st singular subjects indeed had the highest omission rate of all grammatical persons, both in the whole FraC and in the text message data sets.
1st person singular subjects were significantly more frequently omitted than 3rd person singular subjects before copular verbs, auxiliaries, and modal verbs and before verb forms perceived as unambiguous. \is{Copula} \is{Auxiliary} \is{Modal verb} \is{Ambiguity}
These production preferences were also reflected in acceptability judgments.
In experiments \ref*{exp:top.q1}, \ref*{exp:top.s.fv}, and \ref*{exp:top.s.mv}, topic drop of a 1st person singular subject was clearly preferred over topic drop of a 3rd person singular subject referring to a person, in the last experiment even in the absence of a distinct inflectional ending \is{Verbal inflection} at the following verb (contra the \textit{inflectional hypothesis}).
In experiments \ref*{exp:1sg.2sg} and \ref*{exp:1sg.2sg.spoken}, I furthermore found that topic drop of 1st and 2nd person singular subjects was rated comparably well.
This result provides support for those \textit{extralinguistic hypotheses} that predict that pronouns referring to the speaker \textit{and} the hearer can be omitted equally well.

In sum, I showed in line with the \textit{avoid troughs} priciple that those grammatical persons that are more predictable \is{Predictability} can also be more readily omitted because pronouns referring to them occur more frequently in general and/or because their referents as the speaker or hearer are predictable from the communicative situation. \is{Predictability}

Concerning topicality, I concluded in the first part of this book that it is neither a (strictly) sufficient nor a necessary condition for topic drop.
In this second part, I argued that its potential facilitating influence on topic drop could be attributable to the \textit{avoid troughs} priciple too.
The topic \is{Topic} as what is talked about and what is often held constant across multiple utterances should be more predictable \is{Predictability} than a non-topic.
There was no corresponding effect in experiments \ref*{exp:top.q1} and \ref*{exp:top.q2}, in which I manipulated the discourse topic. \is{Discourse topic}
However, in experiment \ref*{exp:top.s.fv}, where topic drop was followed by a lexical verb \is{Lexical verb} with a distinct inflectional ending, \is{Verbal inflection} and in the joint analysis of experiments \ref*{exp:top.s.fv} and \ref*{exp:top.s.mv}, in which the sentence topic \is{Topic} was set via the subject function, topic drop of a topical constituent was indeed rated better.
However, the effect did not show up in experiment \ref*{exp:top.s.mv} alone, where topic drop was followed by syncretic \is{Syncretism} verb forms.
This suggests that topicality affects the acceptability of topic drop in particular when not only the \textit{avoid troughs} priciple is at work but also the \textit{facilitate recovery} priciple \is{Recoverability} through the distinct ending of a subsequent full verb.

\subsection{Evidence for the \textit{avoid peaks} priciple}
For the second principle derived from \textit{UID}, the \textit{avoid peaks} priciple, the influence of verb surprisal provided particular evidence.
In my corpus study, \is{Corpus} I found a clear effect of unigram surprisal on the frequency of topic drop, i.e., that the probability/frequency of topic drop decreases when the surprisal of the following verb lemma increases.
However, I could not demonstrate an effect of verb surprisal in experiments \ref*{exp:surprisal} and \ref*{exp:surprisal.vt}, possibly in part for methodological reasons.
The experimental proof of a surprisal effect is, thus, still pending.

With respect to verb type, it also appeared that not all verb-specific effects can be explained by surprisal alone.
In addition to the surprisal effect at the lemma level, the corpus \is{Corpus} study also revealed an effect of verb type.
More specifically, 1st person singular topic drop was more common before copular verbs, auxiliaries, and modal verbs, as predicted by the IDS grammar  \citep{zifonun.etal1997}. \is{Copula} \is{Auxiliary} \is{Modal verb}
Further research is needed to investigate verb type and verb surprisal even more clearly.
However, it should be noted that the experimental studies do not support these observations.
In experiment \ref*{exp:surprisal.vt}, which contrasted lexical verbs \is{Lexical verb} with auxiliaries, \is{Auxiliary} and in experiments \ref*{exp:top.s.fv} and \ref*{exp:top.s.mv}, which compared lexical verbs \is{Lexical verb} and modal verbs, \is{Modal verb} there were no topic drop specific acceptability differences for verb type, which suggested that it was the copular verbs \is{Copula} that were the driving force for the effect in the corpus \is{Corpus} study.
\is{Uniform information density|)}

\subsection{Evidence for the \textit{facilitate recovery} priciple}
\is{Recoverability|(} \is{Verbal inflection|(}
An effect of the \textit{facilitate recovery} priciple was shown, at least in part, by the influence of a distinct inflectional ending on the verb after topic drop.
The role of verbal inflection is closely related to the influence of grammatical person and to ambiguity avoidance. \is{Ambiguity avoidance}
The relation between inflection and grammatical person becomes evident through \citeg{auer1993} \textit{inflectional hypothesis}, which, if understood literally, actually predicts that not only topic drop of the 1st person singular but topic drop of any person should be favored if followed by a verb with a distinct inflectional ending. 
Similarly, the \textit{facilitate recovery} priciple predicts that unambiguous forms help to resolve ellipsis on the verb following topic drop and, thus, facilitate processing this verb.  \is{Processing effort|(} \is{Recoverability|)}

\is{Ambiguity avoidance|(}
In my empirical studies, I found mixed evidence for the impact of verbal inflection and ambiguity avoidance.
In the corpus \is{Corpus} study, topic drop of the 1st person singular was more frequent than topic drop of the 3rd person singular before verbs that were perceived as unambiguous.
This is consistent with the \textit{facilitate recovery} priciple. \is{Recoverability}
Accordingly, a verb form perceived as unambiguous is more helpful for resolving topic drop of a 1st person singular subject because it allows direct reovery of \is{Recoverability} the speaker as the intended referent.
In contrast, the joint analysis of experiment \ref*{exp:top.s.fv} with distinct (unambiguous) verb forms and experiment \ref*{exp:top.s.mv} with syncretic \is{Syncretism} (ambiguous) verb forms showed no effect of verbal inflection or ambiguity on the acceptability of topic drop.
It might be that the difference between production \is{Production} (in the corpus \is{Corpus} study) and perception (in the experiments) plays a role here, i.e., speakers tend to avoid topic drop before ambiguous verb forms, but hearers do not really need this ``help'' and still find topic drop acceptable ceteris paribus.
However, an alternative explanation could be that there was actually no need to avoid the ambiguity because the items were only locally but not globally ambiguous.
That is, while hearers could not recover \is{Recoverability} the intended meaning through the verb form, they could when looking at the competing object pronoun. 
\is{Ambiguity avoidance|)} \is{Verbal inflection|)}

\subsection{Summary: information-theoretic account}
Overall, I was able to provide initial evidence for an information-theoretic account of the usage of topic drop.
According to this approach, the usage of topic drop can be described as being governed by three principles: the \textit{avoid troughs} principle, the \textit{avoid peaks} principle, and the \textit{facilitate recovery} principle. \is{Recoverability}
Adherence to these principles leads to an efficient distribution of information \is{Information} and thus facilitates the hearer's processing.
It was also shown that the individual factors do not only have an effect in isolation but partly interact and jointly impact the usage of topic drop.

\section{Open questions and outlook}
Although I have taken the first steps to a comprehensive account of topic drop, there are still some unanswered questions, some of which I briefly mention here.

What this book could not accomplish, and where I see a major research desideratum, is to use online methods to investigate the processing of topic drop in real time. 
Self-paced reading, eye-tracking, and ERPs could be used to explicitly test the implicitly assumed link between processing effort \is{Processing effort|)} on the one hand and production and perception preferences for topic drop on the other.
Specifically, it should be investigated whether the assumed processing difficulties for the troughs and peaks show up in corresponding correlates: reading times, eye movements, and ERP components. 
According to the \textit{avoid troughs} principle, the prefield constituent of a full form should be read faster in a predictive context, where it has a lower surprisal, than in a non-predictive context, where it has a higher surprisal.
In eye-tracking, participants should have faster first-pass and total reading times on that constituent, shorter fixation times, and fewer regressions to it in the predictive context (see, e.g., \cite{demberg.keller2008} and \cite{boston.etal2008} for eye tracking studies using surprisal as a predictor).
In an ERP study, I would expect an attenuated N400 and/or P600 in the predictive context, indicating lower surprisal of the prefield constituent (while, e.g., \cite{frank.etal2015, michaelov.etal2023} argue that surprisal explains N400 effects, according to \cite{brouwer.etal2021} it is the P600 that indexes at least surprisal effects related to plausibility).
According to the \textit{avoid peaks} priciple, it is expected that if a verb following topic drop has a high surprisal, this should be reflected in longer reading times on that verb, more regressions to it, and an increased amplitude of the ERP component related to surprisal, all compared to the same verb in a predictive context.

Furthermore, I consider it worthwhile to study topic drop more cross\hyp linguistically and to work out the similarities and differences between the Germanic V2 languages \is{V2 word order} more strongly, also empirically.
Here, it would make sense to use the same or similar experimental designs in several languages or to examine similar or even parallel corpora.
This applies to the syntactic licensing of topic drop as well as to its usage.
For example, it could be investigated, taking into account possible syntactic differences, whether a refined prefield restriction can also be found in the other Germanic V2 languages \is{V2 word order} and how the influence of topicality is to be evaluated there.
Since the information-theoretic approach is based on universal processing principles, the usage of topic drop in other languages should also be determined by the three principles described.
In a second step, the experiments and corpus analyses could be expanded to similar ellipsis types in non-V2 languages, such as subject omissions in \ili{English} and \ili{French}.

Thirdly, I consider it important to examine the typical occurrence of topic drop in certain text types \is{Text type} more closely.
While I tried to give first impulses in this direction, it should be investigated in more depth whether the text types in which topic drop occurs particularly frequently exhibit certain properties that predestine them for this ellipsis type.
Such properties could be, for example, a more fixed word order with the subject in the first position, a higher proportion of 1st and 2nd person (singular) pronouns, or a lower degree of formality.
Here, it seems to me indispensable to create and use a much larger corpus, \is{Corpus} preferably also containing several different text types. \is{Text type}

Such a larger corpus \is{Corpus} is also needed to investigate the role of object topic drop of different cases in more detail.
In particular, dative \is{Dative case} and genitive \is{Genitive case|(} objects were not omitted in the FraC.
However, since the corresponding full forms were also at least extremely rare, it could not be decided whether a preference for the overt realization of these objects is indeed evident here or whether it was only data sparsity that prevented finding corresponding instances of topic drop that are only rare but still exist.
An alternative or supplement to an extended corpus study is the experimental examination of topic drop of different objects, as a continuation of \citet{trutkowski2018}.
Here, however, formality (genitive \is{Genitive case|)} objects) and contrast (effect of objects in the prefield), among other things, must be considered as potential confounding factors.
