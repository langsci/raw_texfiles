\chapter{Syntactic function}\label{ch:usage.function}
As discussed in \sectref{sec:def.constituent}, the focus of this book lies on topic drop of verb arguments, \is{Argument} i.e., subjects and objects.
In this chapter, I discuss whether and to what extent the syntactic function of the preverbal constituent impacts its omission.
First, I provide a review of the literature on topic drop of subjects, objects, and predicatives and of previous empirical studies and their results.
In the second step, I turn to the information-theoretic predictions for syntactic function.
Finally, I present my empirical results in the form of a descriptive overview of the role of syntactic function in the fragment corpus FraC (see \sectref{sec:corpus.frac}).

\section{Theoretical overview}\label{sec:usage.function.theory}
\subsection{Subjects}
In the literature, it is uncontroversial that topic drop can target subjects,%
%% Footnote
\footnote{In an early paper that mentions topic drop as telegraphese, \citet[190]{reis1982} even suggested that only nominative \is{Nominative case} DPs, i.e., subjects, can be omitted in telegrams.
However, a broader notion of topic drop, not limited to the use in telegrams, also includes (at least) object drop (see \cite{ross1982,reis2000} and the next paragraph).
}
%
but there is dissent as to which kind of subjects can be omitted.
In \sectref{sec:topicality.ness}, I discussed in detail the controversy over whether only referential subjects \citep{fries1988, cardinaletti1990} or both referential and non-referential subjects \citep{reis2000, frick2017, ruppenhofer2018} can be targeted by topic drop.
My experiment \ref*{exp:ex} showed that non-referential topics, more precisely \textit{es} expletives, can very well be omitted (see \sectref{sec:exp.ex}).
This result is in line with previous results from corpus studies, which I discuss in \sectref{sec:usage.function.studies}.

\subsection{Objects}
The literature, starting with \citet{ross1982} and \citet{huang1984}, agrees that topic drop in German can also target objects.
However, there are three (possible) limitations of object topic drop that are frequently discussed.
First, it is debated whether object topic drop is as common and as acceptable as subject topic drop.
Second, the omission of direct objects in the accusative case \is{Accusative case|(} seems to be more felicitous than the omission of prepositional objects \is{Prepositional object|)} and indirect objects in the dative \is{Dative case} or the genitive case, \is{Genitive case} a difference explained by various approaches in the literature.
Third, there seems to be an asymmetry concerning grammatical person, according to which 1st and 2nd person objects cannot be omitted or are difficult to omit while omitting 3rd person objects is perfectly fine.%
%% Footnote
\footnote{In this respect, syntactic function overlaps with grammatical person, to which Chapter \ref{ch:usage.person} is devoted.
I discuss the object asymmetry in relation to grammatical person in the present chapter and focus on the grammatical person of omitted subjects rather than objects in the next chapter.
} %
In what follows, I present the relevant contributions to the discussion on each of the three issues.

\subsubsection{Subject vs. object topic drop}
Concerning the first issue, \citet[2]{trutkowski2016} states that subjects and objects can be equally well omitted, whereas \citet{volodina2011} claims that object topic drop is rarer than subject topic drop, and \citet{fries1988} suggests that objects are less likely to be omitted.
Fries refers to \citet[15]{klein1985}, who postulates that for context ellipsis,%
%% Footnote
\footnote{\citet[5]{klein1985} uses this term to subsume coordination ellipsis and what he terms ``Adjazenzellipse'' (`ellipsis under adjacency'), e.g., question-answer pairs, corrections, or continuations.
\citet{fries1988} extends the scope of this tendency by applying it to topic drop.}
%
how well an expression can be omitted tends to be determined by a hierarchy of syntactic functions in the form of subject > direct object > indirect object > prepositional \is{Prepositional object} complement.%
%% Footnote
\footnote{Similar hierarchies are proposed in centering theory \citep{grosz.etal1995} for \ili{English} \citep{walker.etal1998} and \ili{German} \citep{speyer2007} to determine the reference of anaphora (see \sectref{exp:top.s.fv.background}).
}
%%%
It seems that most authors discussing this issue take the view that object topic drop has a disadvantage over subject topic drop.

\subsubsection{Direct vs. indirect objects}
Regarding the second point, namely a possible difference between types of objects, i.e., direct vs. indirect objects or objects with a structural vs. objects with a lexical case,  \is{Lexical case|(} several authors argue for an asymmetry concerning how well they can be omitted.
I discuss the different arguments and approaches in detail in what follows.

First, there is the view that indirect objects can in principle be targeted by topic drop but less well or less often than direct objects.
For example, \citet[3]{schalowski2015} claims that topic drop of indirect objects is possible but less acceptable.
He notes that he did not find any such instances in a previous corpus \is{Corpus} study (\cite{schalowski2009}, cited in \cite{schalowski2015}).
According to \citet{ruppenhofer2018}, the omission of indirect objects is only possible to a very limited extent, in ``generic statements, especially ones about the efficacy of means and instruments'' \citep[224, footnote 14]{ruppenhofer2018} such as \ref{ex:rupp.ind.obj}.
There, however, the full form may not even obligatorily require an indirect object.%
%% Footnote
\footnote{While in any case, a subject referring to what promotes weight loss is omitted, the simplest full form is the generic statement \ref{ex:rupp.ind.obj.ff.simple} with \textit{es} (`it') in the prefield but without an indirect object. \vspace{-\baselineskip}
\exg.\label{ex:rupp.ind.obj.ff.simple}Es hilft beim Abnehmen.\\
it helps with.the losing.weight\\
`It helps to lose weight.'\par %\vspace{-0.5\baselineskip}
This is supported by the verb valence dictionary E-VALBU where for \textit{helfen} (`to help') in the sense of causing health improvement the dative argument is optional \citep{helfen1} and for the sense of enabling one to be successful at something, an example without a dative object is listed \citep{helfen2}.
To consider \ref{ex:rupp.ind.obj} as topic drop of an indirect object, \citet{ruppenhofer2018} has to assume a full form like \hyperref[ex:rupp.ind.obj.ff.2]{(iia)}, where a beneficiary like \textit{Ihnen} (`you.\textsc{pol}') is included in the prefield (or a full form like \hyperref[ex:rupp.ind.obj.ff.3]{(iib)} with \textit{Ihnen} in the middle field). %\vspace{-0.5\baselineskip}
\ex.\label{ex:rupp.ind.obj.ff.23}
\ag.\label{ex:rupp.ind.obj.ff.2}Ihnen hilft es beim Abnehmen.\\
you.\textsc{2sg.dat.pol} helps it with.the losing.weight\\
`It helps you to lose weight.'
\b.\label{ex:rupp.ind.obj.ff.3}Es hilft Ihnen beim Abnehmen.\par %\vspace{-0.5\baselineskip}
However, assuming the full forms in \ref{ex:rupp.ind.obj.ff.23} would result in \ref{ex:rupp.ind.obj} violating the restriction that topic drop can target only one constituent per clause.
In short, the dubiousness of Ruppenhofer's example does not allow me to validate the restriction to ``generic statements'' that he formulates.
}

\exg.\label{ex:rupp.ind.obj}$\Delta$ Hilft beim Abnehmen.\\
it helps with.the losing.weight\\
`(It) helps to lose weight.' \citep[224, footnote 14]{ruppenhofer2018}

In contrast to the assessment of \citet{schalowski2015} and \citet{ruppenhofer2018}, who assume that indirect objects can be omitted to a limited extent, other authors hold the view that topic drop cannot affect indirect objects at all.
\citet[418]{zifonun.etal1997} state that topic drop of dative objects \is{Dative case|(} and prepositional objects \is{Prepositional object} is not possible, regardless of the presence or absence of a linguistic antecedent.%
% Footnote
\footnote{\citet{zifonun.etal1997} seem to be one of the few researchers who mention topic drop of prepositional objects \is{Prepositional object} at all. 
But see also \citet[25]{trutkowski2016}.}
%
\is{Antecedent}
This view seems to be supported by \citet{jaensch2005} and the \citet{duden2016, duden2022} who do not list topic drop of indirect objects as an option (although they also do not explicitly deny it).

It is \citet{sternefeld1985} who is the first to propose that how well an object can be omitted depends on the distinction between \textit{structural case} \is{Structural case} and \textit{lexical case}.
While for verb arguments, \is{Argument} structural case \is{Structural case} depends on the sentence structure itself (e.g., the accusative as the case of the direct object), lexical case is determined by properties of the head that governs the corresponding DP (e.g., the German verb \textit{helfen} (`to help') requires an object in the dative case) \is{Dative case} \citep[see, e.g.,][]{haspelmath2009}.%
%% Fußnote
\footnote{In the literature, structural case \is{Structural case} is also frequently contrasted with the term \textit{inherent case}. \is{Inherent case}
Inherent case is used either for case assignment that is related to theta roles \citep[171]{chomsky1981}, later also called \textit{thematic case} \citep[e.g.,][]{reinhart.siloni2005}, or as an umbrella term for lexical case assignment and assignment by theta roles \citep{haspelmath2009}.
While the umbrella term would make lexical case a hyponym of inherent case, \citet{woolford2006} argues for a distinction between inherent and lexical case.
Since I restrict topic drop, as explained in \sectref{sec:def.constituent}, to the omission of arguments of the verb, it seems reasonable to focus on the concept of lexical case in this book.
}
\citet{sternefeld1985} claims that only arguments \is{Argument|(} with a structural case \is{Structural case} can be omitted, whereas arguments with a lexical case cannot.
However, he points out that this is not equivalent to distinguishing between direct objects in the accusative case \is{Accusative case} on the one hand and indirect objects in the dative \is{Dative case} or the genitive \is{Genitive case} case on the other.
He presents examples such as \ref{ex:no.acc}, where an accusative argument \is{Accusative case} cannot be omitted either, according to his judgment, and contrasts them with the examples in \ref{ex:sternefeld.acc}, where topic drop shall be possible.

\ex.\label{ex:no.acc}
\ag.*(Sie) Hat die Sache nicht interessiert.\\
she.\textsc{acc} has the thing not interested\\
`(She) was not interested in the matter.'
\bg. *(Ihn) Hat bedrückt, daß Ede krank ist.\\
him.\textsc{acc} has depressed that Ede ill is\\
`(He) was depressed that Ede is ill.' \citep[][407, adapted, his judgments]{sternefeld1985}

\ex.\label{ex:sternefeld.acc}
\ag. (Das) Kenn ich schon.\\
that.\textsc{acc} know I already\\
`I know (that) already.'
\bg. (Den) Haben wir gestern erst gesehen.\\
him.\textsc{acc} have we yesterday only seen\\
`We saw (it) only yesterday.' \citep[][407, adapted, his judgments]{sternefeld1985}

\citet{sternefeld1985} argues that in \ref{ex:no.acc}, the verbs \textit{interessieren} (`to interest') and \textit{bedrücken} (`to depress') are so-called ``flip verbs''.
In a classic generative analysis, these verbs are considered not to have a logical subject but two internal objects in the deep structure, of which one becomes the subject by receiving the nominative, \is{Nominative case} whereas the other becomes the object by receiving the accusative \is{Accusative case} \citep[400; 427] {sternefeld1985}.
\citet[427] {sternefeld1985} argues that this case assignment does not result in a structural but in a lexical accusative \is{Accusative case} required by the flip verb, and in the impossibility to omit this argument.
\citet{sternefeld1985} supports this claim only with the introspective judgments in \ref{ex:no.acc} and \ref{ex:sternefeld.acc}, but even if his predictions were confirmed in empirical studies yet to be conducted,%
% Footnote
\footnote{But see \citeg{trutkowski2018} rating study discussed below, in which a part of \citeg{sternefeld1985} predictions are tested.}
%
it would remain unclear why lexical case should block topic drop.%
%% Footnote
\footnote{As a starting point for an explanation, one could argue that lexical case is capable of blocking certain syntactic processes.
For instance, passivization only targets structural but not lexical case in German.
Topic drop could then be another such process.}
%
\il{German|(}
This also applies, first, to \citeauthor{fries1988}'s (\citeyear[31]{fries1988}) hypothesis that DPs with a lexical case and dative \is{Dative case} DPs seem to require structural and semantic constancy between antecedent \is{Antecedent} and target utterances to be able to be omitted, an idea that partly anticipates \citeg{trutkowski2016} argumentation that I discuss below.
Second, this open issue also concerns \citeauthor{haider2010}'s (\citeyear[269, footnote 22]{haider2010}) remark that topic drop is one of several phenomena that indicate a dichotomy between the nominative \is{Nominative case} and the accusative \is{Accusative case} on the one hand and the dative \is{Dative case} on the other.

\citet{trutkowski2016} presents a more elaborate account of the role of the case for (object) topic drop.
She differentiates between two types of topic drop: \textit{non-verbatim} (NVTD) and \textit{verbatim topic drop} (VTD).
NVTD can only target arguments \is{Argument|)} with a structural case, \is{Structural case} under which she subsumes the nominative \is{Nominative case} and the accusative \is{Accusative case} but allows for a change of the predicate.
This change can be accompanied by mismatches in case and theta role between the antecedent \is{Antecedent} and the omitted constituent \citep[3]{trutkowski2016}.
VTD does not allow for such mismatches, since it requires semantic identity, but it can target both arguments with a structural and a lexical case \citep[3]{trutkowski2016}.%
%% Fußnote
\footnote{\citet[31]{trutkowski2016} specifies that ``the crucial condition for the well-formedness of VTD is the identity of finely granulated (micro) theta roles assigned to antecedent \is{Antecedent} and gap, respectively, which is ensured when context and target predicates are `highly' synonymous (i.e. semantically equivalent).'' 
By ``micro theta roles'' she means very fine-grained semantically enriched theta roles \citep[31]{trutkowski2016}.
I do not go further into the details of her proposal but refer to her detailed account.
}
%%
Trutkowski illustrates the apparent contrast with topic drop of a genitive \is{Genitive case} object, i.e., an argument with a lexical case, which should only allow for VTD and, thus, block mismatches.
Therefore, the matching VTD in \ref{ex:td.trutkowski.vtd} should be grammatical, while the NVTD involving a change in predicate from \textit{gedenken} (`to commemorate') to \textit{schämen} (`to be ashamed') in \ref{ex:td.trutkowski.nnvtd} should not.%
%% Footnote
\footnote{What is omitted in \ref{ex:td.trutkowski.nnvtd} could also be the demonstrative pronoun \textit{dessen} referring back to the proposition that Hans commemorates Rosa Luxemburg.
Under this reading, the utterance would mean that many a politician is not ashamed of Luxemburg but of Hans commemorating her.}
%
According to \citet{trutkowski2016}, this is only possible if the antecedent \is{Antecedent} and the omitted constituent have a structural case, \is{Structural case} as example \ref{ex:td.trutkowski.nvtd} shall illustrate, where the omitted constituent is an accusative object. \is{Accusative case}
Therefore, the change from \textit{gedenken} to \textit{kennen} (`to know') should be fine.

\ex.\label{ex:td.trutkowski.vtd}
\ag.A: Der Hans gedenkt der Rosa Luxemberg [sic!].\\
{} the Hans commemorates the.\textsc{gen} Rosa Luxemberg \\
A: `Hans commemorates Rosa Luxemberg [sic!].'
\bg.B: $\Delta$ Gedenkt der Otto auch.\\
{} her.\textsc{gen} commemorates the Otto too\\
B: `Otto commemorates (her), too.' \citep[VTD,][4, her judgment]{trutkowski2016}

\ex.\label{ex:td.trutkowski.nnvtd}
\ag.A: Der Hans gedenkt der Rosa Luxemberg [sic!].\\
{} the Hans commemorates the.\textsc{gen} Rosa Luxemberg\\
A: `Hans commemorates Rosa Luxemberg [sic!].'
\bg.B: *$\Delta$ Schämt sich manch ein Politiker.\\
{} \phantom{*}her.\textsc{gen} is.ashamed \textsc{refl} some a politician\\
B: `Many a politician is ashamed (of her).' \citep[NVTD impossible,][4, her judgment]{trutkowski2016}

\ex.\label{ex:td.trutkowski.nvtd}
\ag.A: Der Hans gedenkt der Rosa Luxemberg [sic!].\\
{} the Hans commemorates the.\textsc{gen} Rosa Luxemberg\\
A: `Hans commemorates Rosa Luxemberg [sic!].'
\bg.B: $\Delta$ Kennt der Otto gar nicht.\\
{} her.\textsc{acc} knows the Otto at.all not\\
B: `Otto doesn't know (her) at all.' \citep[NVTD possible,][4, her judgment]{trutkowski2016}

As Trutkowski herself notes, utterances with covert genitive \is{Genitive case} objects generally seem odd because of a register clash between topic drop as a phenomenon of informal speech and genitive \is{Genitive case} objects, which are rarely used in contemporary German and when they are, it is mainly in formal registers \citep[26]{trutkowski2016}.
Nevertheless, she claims that there are acceptable instances of topic drop of genitive \is{Genitive case} objects and tested respective utterances in a survey. \is{Acceptability rating study|(}
Her stimuli with unrealized genitive \is{Genitive case} objects such as \ref{ex:gen.stimulus}, however, still seem highly marked in an informal setting%
%% Footnote
\footnote{This is supported by the fact that the Duden classifies the verb \textit{erwehren} (`to resist') from example \ref{ex:gen.stimulus} as ``elevated'' \citep{erwehren}.}
%%
 and received correspondingly low ratings.%
%% Footnote
\footnote{The mean rating for utterances with omitted genitive \is{Genitive case} objects was at -0.775 on a scale from -2 (worst) to 2 (best) and consists of ratings for 4 items with different case combinations, as shown in the following table taken from \citet[48]{trutkowski2016}: \is{Accusative case}\is{Dative case|)} \is{Nominative case} \is{Prepositional object}
\begin{center}
\begin{tabular}{lrr}
\lsptoprule
Case combination & Mean rating & Standard deviation \\
\midrule
Nominative--genitive & $-1.15$ & $1.21$ \\
Accusative--genitive & $-1.25$ & $1.07$ \\
Dative--genitive & $0.12$ & $1.39$ \\
Prepositional phrase--genitive & $-0.82$ & $1.37$\\
\lspbottomrule
\end{tabular}
\end{center}

\noindent In particular, the mean ratings for the items with nominative--genitive and with accusative--genitive are heavily degraded given that the best mean rating of the inquiry was 1.88 (0.32) for topic drop of a nominative DP with a PP antecedent. \is{Antecedent}
As \citet[39]{trutkowski2016} admits, her survey does not meet the requirements of a linguistic experiment, since she tested 30 different conditions with only one item each.
This makes a clean statistical analysis impossible and prevents the generalizability of the results since possible effects could also depend on the lexicalization and not on the case combinations.
}
%% 

\ex.\label{ex:gen.stimulus}
\ag. Mir gefällt Martin so gut. \\
me.\textsc{dat} pleases Martin so well \\
`I like Martin so well.'
\bg. $\Delta$ Konnte ich mich bislang auch nicht erwehren.\\
him.\textsc{gen} could I me yet also not resist\\
`So far, I have not been able to resist (him) either.' \citep[46, her judgment]{trutkowski2016}

I come back to Trutkowski's account in \sectref{sec:usage.function.studies}, where I discuss an experiment she conducted in \citet{trutkowski2018} to test her hypotheses.
\is{Acceptability rating study|)}

In sum, several authors argue for an asymmetry between topic drop of direct and topic drop of indirect.
\citet{sternefeld1985} and \citet{trutkowski2016} attribute this asymmetry to case differences (structural vs. lexical case \is{Lexical case|)} or verbatim and non-verbatim topic drop, respectively).
However, it remains unclear from \citeg{sternefeld1985} account why such case differences should block topic drop and \citeg{trutkowski2016} account is based on judgments, which are at least dubious.
\is{Accusative case|)} 

\subsubsection{1st and 2nd person objects vs. 3rd person objects}
Besides the difference between direct and indirect objects, there is a second asymmetry within object topic drop, which is the third issue concerning object topic drop.
Topic drop of objects is mainly restricted to the 3rd person, whereas the 1st and 2nd person object pronouns can hardly be omitted, if at all \citep{fries1988, auer1993, jaensch2005, erteschik-shir2007, volodina.onea2012, duden2016, duden2022, trutkowski2016}.
This is illustrated by the contrast between \ref{ex:td.obj.2sg} and \ref{ex:td.obj.3sg}.%
% Footnote
\footnote{Note that I glossed the omitted constituent in \ref{ex:td.obj.3sg} as a demonstrative rather than a personal pronoun.
Below, I come back to the difference between demonstratives and personal object pronouns.}

\ex.\label{ex:td.obj.2sg}
\ag. A: Wo warst du so lange?\\
{} where were you.\textsc{2sg} so long\\
A: `Where have you been for so long?'
\bg.B: *$\Delta$ Hat die Chefin gesucht.\\
{} \phantom{*}you.\textsc{acc.2sg} has the boss searched\\
B: `The boss has been looking for (you).' \citep[§1378, adapted]{duden2016}

\ex.\label{ex:td.obj.3sg}
\ag. A: Wo ist bloß wieder mein Schlüssel?\\
{} where is \textsc{part} again my key\\
A: `Where is my key again?'
\bg.B: $\Delta$ Hast du in die oberste Schublade getan.\\
{} that.\textsc{acc} has you.\textsc{2sg} in the top drawer put\\
B: `You put (it) in the top drawer.'

There are different explanations for this pattern.\is{Reflexivity|(}
\citet{fries1988} traces it back to the fact that in German, the reflexive and non-reflexive forms of the 1st and 2nd person object pronouns are identical, while with \textit{sich}, there is a reflexive form of its own for the 3rd person. \is{Ambiguity|(}
He argues that in contexts where the antecedent \is{Antecedent} is neither distinctly marked for reflexivity nor non-reflexivity, this syncretism \is{Syncretism|(} blocks topic drop \citep[33]{fries1988}, even for the 3rd person.
\ref{ex:td.refl.1} is such an example where according to \citet{fries1988}, the 3rd person reflexive \textit{sich} cannot be omitted, as the antecedent \textit{ihm} (`him') does not entail reflexivity.
In \ref{ex:td.refl.2}, in turn, speakers should even be able to omit the 1st person reflexive because the \textit{selbst} (`self') in A's question entails a reflexive usage.

\ex.\label{ex:td.refl.1}
\ag.A: Gisbert hat den Sascha rasiert, und was ist mit ihm?\\
{} Gisbert has the Sascha shaved and what is with him\\
A: `Gisbert shaved Sascha, and what about him?'
\bg.B: *$\Delta$ Rasiert er auch.\\
{} \phantom{*}himself shaves he too\\
B: `He shaves too.' (Lit. `(Himself,) he shaves too.') \citep[33, his judgment]{fries1988}

\ex.\label{ex:td.refl.2}
\ag.A: Mich hast du rasiert, und was ist mit dir selbst?\\
{} me has you shaved and what is with you.\textsc{dat} self\\
A: `You shaved me, and what about yourself?'
\bg.B: $\Delta$ Rasier' ich auch.\\
{} me shave I too\\
B: `I shave too.' (Lit. `(Myself,) I shave too') \citep[33, his judgment]{fries1988}

These subtle judgments require an empirical investigation, which cannot be accomplished in this book.%
% Footnote
\footnote{Note that \textit{ihm} (`him') in \ref{ex:td.refl.1} is also ambiguous \is{Ambiguity} as to whether it refers to Gisbert or to Sascha (with a preference for Gisbert in interpreting the discourse as a whole), further complicating matters.}
%
However, even if they were correct, it remains unclear why such a syncretism should block topic drop when other ambiguities \is{Ambiguity|)} are generally possible with topic drop, like the one caused by syncretic \is{Syncretism|)} 1st and 3rd person singular forms of modal verbs \is{Modal verb} tested in experiment \ref*{exp:top.s.mv} (see \sectref{sec:exp.top.s.mv.res.person}).\is{Reflexivity|)}

A further explanation for why object topic drop seems to be restricted to the 3rd person is given by \citet{schulz2006}.
She claims that only recoverable \is{Recoverability} constituents can be targeted by topic drop and that a necessary prerequisite for recoverability \is{Recoverability} is that the omitted constituent is a continued topic \is{Topic} \citep[8]{schulz2006}.
\citet[11]{schulz2006} argues that topics in subject function and topics in object function with the 3rd person are more easily recoverable \is{Recoverability} than 1st and 2nd person object topics because the latter are less typical, i.e., more marked continued topics.
She bases this claim on universal prominence hierarchies.
More specifically, she uses the concept of harmonic alignment from optimality theory \is{Optimality theory} \citep{prince.smolensky2004} and applies it to the two subhierarchies for syntactic function and grammatical person \citep[11--12]{schulz2006}.%
% Footnote
\footnote{For the hierarchy of syntactic function, she refers to \citet[66]{keenan.comrie1977} but omits the genitive:
Subject > direct object > indirect object > major oblique case NP (> genitive) > object of comparison (``>'' means ``more accessible'').
For grammatical person she follows \citet[674]{aissen1999}, who in turn derived the hierarchy from \citet{silverstein1976}: local person (1st \& 2nd) > 3rd.
}
As a result, 1st and 2nd person objects are ranked lowest and, therefore, should not be able to be omitted \citep[13; 17]{schulz2006}.%
% Footnote
\footnote{Based on this result, \citet{schulz2006} derives an optimality theoretic \is{Optimality theory} model with several constraints for the omission pattern in German, which can also be applied to \ili{Japanese} null topics.}
%
While Schulz's approach is more elaborate than \citeg{fries1988} proposal, two aspects seem problematic.
First, it presupposes that all omitted constituents must be continuous topics, i.e., that topicality is a necessary condition for topic drop.
In \sectref{sec:topicality.ness}, I argued against this view and showed that non-topical constituents can also be omitted.
Second, the relationship between salience, \is{Salience} topicality, and the possibility of omitting a constituent remains vague.
While it is plausible to assume, based on the two hierarchies discussed, that 1st or 2nd person objects are more marked, it is not clear why this makes them necessarily less good continuous topics, thus, blocking omission.

As a related explanation, \citet[290]{sigurdsson2011} introduces the so-called \textit{relative specificity constraint} for Germanic topic drop, according to which an omitted object cannot be more specific than the overt subject.
He argues that the 1st and 2nd person are more specific than the 3rd person and that [+human] is more specific than [-human].
Thus, a 1st or 2nd person object cannot be omitted if a 3rd person subject is simultaneously realized overtly, as shown in the Swedish \il{Swedish} example \ref{ex:sigurdsson.obj} with a clitic subject pronoun and the German equivalent \ref{ex:sigurdsson.obj.dt} (slightly adapted) with a full subject pronoun.

\ex. [Context: `That is Johnson over there, the new manager. We should say hello to him.']
\ag.*$\Delta$ Vill'an säkert inte prata med nu.\label{ex:sigurdsson.obj}\\
us/me wants'(h)e certainly not talk with now\\
`He certainly doesn't want to talk to (us/me) now.' \il{Swedish} (\cite[290]{sigurdsson2011}, his judgment, translation added)
\bg.*$\Delta$ will er jetzt bestimmt nicht sehen.\label{ex:sigurdsson.obj.dt}\\
us/me wants he now certainly not see\\
`He probably doesn't want to see (us/me) now.'


\largerpage[-2]
He states that this restriction is an intervention effect related to his proposed C/edge-linking generalization (see \sectref{sec:syntax}) \citep[291]{sigurdsson2011}.
\citet[292]{sigurdsson2011} argues that the subject must also be C/edge-linked and that ``the dropped object cannot be featurally 'bigger'\,'' than the subject, as it would otherwise violate Relativized Feature Minimality \citep{rizzi2001}.
Since he limits his argument to cases with clitic subject pronouns in Swedish, \il{Swedish} Norwegian, \il{Norwegian} and Icelandic \il{Icelandic} \citep[291, footnote 36]{sigurdsson2011},
it is unclear whether and how it could be applied to German and Dutch, \il{Dutch} where objects can also be omitted in the presence of full subject pronouns.%
%% Footnote
\footnote{In the FraC, there seems to be no instance that violates the relative specificity constraint.}

\citet{mornsjo2002}, as well as \citet{volodina.onea2012} and the \citet{duden2016, duden2022}, propose a fourth explanation.
\citet[70]{mornsjo2002} considers example \ref{ex:cardinaletti.1sg}, which \citet{cardinaletti1990} discusses to illustrate the apparent ban on omitting 1st and 2nd person objects, a ``pragmatically inappropriate'' answer to the question \textit{Did I disturb you?}, independently of the ellipsis.
She argues that, at least in Swedish, \il{Swedish} the full form would be acceptable if the direct object \textit{mich} (`me') is stressed to indicate contrastiveness (see the contrastive topics \is{Contrastive topic} discussed in \sectref{sec:topicality.suff}), but in this case, it cannot be omitted \citep[71]{mornsjo2002}.
Similarly, \citet[219]{volodina.onea2012} argue that when 1st and 2nd person object pronouns are placed in the prefield, this positioning is marked and involves stress, which, in turn, blocks topic drop.

\exg.*(Mich) hast du sehr gestört.\label{ex:cardinaletti.1sg}\\
me have you.\textsc{2sg} very disturbed\\
`You disturbed (me) a lot.' (\cite[79]{cardinaletti1990}, cited in \cite[70]{mornsjo2002})

A related argument comes from the \citet{duden2016, duden2022}.
There, it is stated that weakly stressed object pronouns such as \textit{mich} and \textit{dich} are rarely placed in the prefield and, therefore, rarely omitted from this position (\cite[§1378]{duden2016}; \cite[§35]{duden2022}).
The fact that 3rd person objects can be better omitted may be explained by the fact that in this case, the omitted constituent is not a weak personal pronoun as in \ref{ex:td.obj.pers.3SG} but a demonstrative pronoun as in \ref{ex:td.obj.dem.3SG}.

\exg.\label{ex:td.obj.demonstratives}Was ist mit Tino?\\
what is with Tino\\
`What's with Tino?'
\ag.\label{ex:td.obj.pers.3SG}(Ihn) Hab ich gesehen.\\
him.\textsc{acc} have I seen\\
`I have seen (him).' 
\bg.\label{ex:td.obj.dem.3SG}(Den) Hab ich gesehen.\\
him.\textsc{acc.dem} have I seen\\
`I have seen (him).' 


\largerpage[-2]

This is supported by the fact that demonstrative pronouns in object function of the sort \textit{den}, \textit{die}, and \textit{das} are a lot more frequent in the prefield \is{Prefield|(} than the corresponding personal pronouns, as evidenced by \citet{bosch.etal2007}.
In a corpus study of the NEGRA newspaper corpus,%
%% Footnote
\footnote{\citet[148, footnote 4]{bosch.etal2007} state that the NEGRA corpus, a POS-tagged and syntactically annotated subset of the Frankfurter Rundschau corpus, consists of 355\,000 words from news articles.}
%
\is{Corpus} they found that less than 0.5\% of the 3rd person singular personal pronouns in object function occurred in the prefield, whereas for 3rd person singular demonstrative object pronouns, it were around 20\% \citep[149--150]{bosch.etal2007}.%
%% Footnote
\footnote{\citet[116--117]{zifonun2001} points out that the object personal pronoun of the 3rd person singular neuter \textit{es} cannot be placed in the prefield at all \ref{ex:td.obj.demonstratives.es}. %\vspace{-0.5\baselineskip}
\exg.\label{ex:td.obj.demonstratives.es}Was ist mit dem Kind?\\
what is with the child\\
`What's with the child?'
\ag.\label{ex:td.obj.pers.3SG.es}(*Es) sehe ich nicht.\\
\phantom{(*}it.\textsc{acc} see I not\\
`I don't see (it).' 
\bg.\label{ex:td.obj.dem.3SG.es}(Das) sehe ich nicht.\\
it.\textsc{acc.dem} see I not\\
`I don't see (it).'\par \vspace{-1\baselineskip}
}
Since these demonstratives are regularly placed in the prefield without a special marking or stress, they can be omitted from there more easily.
No demonstrative counterparts are available for the 1st and 2nd person singular, since they are deictic by themselves.
Thus, their positioning in the prefield almost always involves stress, which blocks topic drop.
Overall, this last approach to the object asymmetry of topic drop in German seems to me to be the most promising one to explain the pattern.
It is based on the incompatibility of prosodic prominence \is{Prosody} and ellipsis and does not rely on only introspective data or additional assumptions concerning topic continuity or specificity, like the other three approaches.
\is{Prefield|)} \il{German|)}
    
\subsection{Predicatives}
For completeness, I also mention topic drop of predicatives, even though it is generally a rare phenomenon and, therefore, 
has not received as much attention in the literature as the omission of subjects and objects.
\citet[406]{sternefeld1985}  gives example \ref{ex:predicative.not} to prove that topic drop cannot target predicatives.
\citet[29]{fries1988} counters this with example \ref{ex:predicative}, from which he concludes that predicatives can be omitted after all, provided they are used referentially.
The \citet[§1378]{duden2016} (also \cite[§35]{duden2022}) also argues that predicatives can be omitted in the form of weakly stressed demonstratives.
I endorse this view.

\ex.\label{ex:predicative.not}
\ag.A: Weißt du was über Säugetiere?\\
{} know you something about mammals\\
A: `Do you know anything about mammals?'
\bg.B: *$\Delta$ Sind z.B. Wale.\\
{} \phantom{*}that are for.example whales\\
B: `(These) are, for example, whales.' \citep[406, his judgment]{sternefeld1985}

\ex.\label{ex:predicative}
\ag.A: Wale sind Säugetiere.\\
{} whales are mammals\\
A: `Whales are mammals.'
\bg.B: $\Delta$ Sind Hunde auch.\\
{} that are dogs too\\
B: `So are dogs.' \citep[29]{fries1988}

\section{Previous empirical evidence}\label{sec:usage.function.studies}
In this section, I discuss the results of previous empirical studies that investigated the syntactic function of the omitted prefield constituent.
The goal is to validate the claims from the theoretical literature regarding the above-mentioned restrictions on subject and object topic drop.
The results of three corpus studies \citep{poitou1993,frick2017,ruppenhofer2018} (i) suggest that topic drop of referential and non-referential subjects as well as of direct objects occurs in natural speech data, (ii) do not evidence that subjects are more often omitted than objects, and (iii) provide no evidence that indirect objects are targeted by topic drop.
However, an acceptability rating study \citep{trutkowski2018} suggests that topic drop of an indirect object can be as acceptable as topic drop of a direct object under certain circumstances.

\largerpage
\is{Corpus|(}
\subsection{\citet{poitou1993}}
\citet{poitou1993} presents the results of a corpus study that he conducted on 200 instances of topic drop.
He states that the omission of subjects is most frequent in his corpus with about half of the instances, followed by object omissions \citep[119]{poitou1993}.
The rarest is a so-called ``rest group'' with cases where an adverbial \is{Adverbial} could occupy the preverbal position in the full form \citep[117]{poitou1993}.%
% Footnote
\footnote{As discussed in \sectref{sec:def.constituent}, I disregard such cases in this book.}
%
Poitou presents a subdivision for the omitted subjects by animacy. \is{Animacy}
Mostly inanimate subjects are omitted, whereas animate subjects only make up a fifth of the instances he found \citep[116]{poitou1993}.%
%% Footnote
\footnote{\citet{poitou1993} does not discuss what the animate \is{Animacy} or inanimate subjects refer to.
However, in example \ref{ex:duerrenmatt}, which he provides for the inanimate case, taken from Dürrenmatt's 1956 play \textit{Der Besuch der alten Dame} \textit{[The visit]}, the reference of the omitted constituent would be a proposition.
A paraphrase could be something like \textit{$\langle$Why nobody pays taxes$\rangle$ needs to be investigated}.
Based on this example, it can be speculated that the high number of omitted inanimate subjects is the result of many instances of propositional reference.
Interestingly, in the English translation by Patrick Bowles, the utterance is translated with overt subject as ``It'll have to be investigated''.
%\vspace{-0.5\baselineskip}
\ex.\label{ex:duerrenmatt}
\ag.Der Bürgermeister: Unsere Kassen sind leer. Kein Mensch bezahlt Steuern.\\
the mayor our coffers are empty no human pays taxes\\
The mayor: `Our coffers are empty. No one pays taxes.'
\bg.Der {Pfarrer [sic!]:} $\Delta$ Muß untersucht werden.\\
the priest that must investigated be\\
The priest [sic!]: `(That) must be investigated.' (\cite[116]{poitou1993}; in fact the bailiff [orig. Pfändungsbeamte] not the priest utters the topic drop)
\vspace{-0.75\baselineskip}
}
He further notes that there are cases of non-anaphoric \textit{es} (`it'), i.e., non-referential subject topic drop.
For objects, he states that only direct but no indirect objects are omitted in his corpus \citep[116, footnote 1]{poitou1993}.

However, in interpreting Poitou's results, we must take into account that his evidence comes from very diverse sources, i.e., interviews, phone calls, plays, and comics \citep[111]{poitou1993}.
In particular, the unclear proportion of fictional and potentially dialectal speech data extracted from, among others, older plays by the Austrian author Nestroy (19th century) and the Swiss author Dürrenmatt (mid of 20th century) represents a problem for the significance of his results, as it is not clear how authentic and representative the data are.
Furthermore, the results are presented without relative or absolute numbers and, thus, can only provide a tentative impression of how topic drop of different syntactic functions is distributed.

In sum, \citet{poitou1993} confirms that referential and non-referential subjects can be targeted by topic drop and his results tentatively suggest that indirect objects are omitted at least less frequently, if at all.
Since he does not provide or did not determine relative numbers, it cannot be concluded that subject topic drop is more frequent than object topic drop.

\subsection{\citet{frick2017}}
\citet[38, 42]{frick2017} investigated topic drop and other ellipsis types in a corpus of 3\,999 Swiss German text messages.
This corpus is a part of the Swiss SMS corpus \citep{ueberwasser2015}, which was created in the project sms4science.ch between 2009 and 2010.
She presents relative numbers for omitted subjects and objects.

Of 4\,385 subject pronouns in the prefield, 2\,059 (46.96\%) were realized and 2\,326 (53.04\%) were omitted,%
%% Footnote
\footnote{Note that the number of 2\,326 omitted subjects is calculated based on \citeg{frick2017} figure 8 on page 88.
It slightly deviates from the sum of all omitted subject pronouns, 2\,294, which Frick provides in figure 7 on page 86, but it is identical to the sum of the numbers given in figure 10 on page 93.}
%
with large differences in terms of grammatical person.
For instance, the omission rate of the 1st person singular (59.32\%) is higher than that of the 2nd person singular (47.01\%), which in turn is considerably higher than that of the 2nd person plural (13.79\%) (\cite[88]{frick2017}, see \sectref{sec:usage.person.studies} for details).
Besides referential subjects, she also discusses the omission of non-referential expletives. \is{Expletive}
Of 262 expletives in her data set, 179 (68.32\%) were omitted \citep[140]{frick2017}.
This provides further support that non-referential subjects can be targeted by topic drop and, in fact, are in natural speech data.

For objects, \citet[125--126]{frick2017} lists 68 (53.13\%) realized and 60 (46.88\%) omitted instances.
First, the numbers indicate that subjects are a lot more frequent in the prefield of Swiss German text messages than objects.
Second, the omission rate for subjects is overall larger than for objects in absolute terms.
However, a Pearson's chi-squared test with Yates's continuity correction, which I calculated in R \citep{rcoreteam2021}, reveals that this difference is not significant ($\chi^2(1) = 1.66, p > 0.19$).
\citet{frick2017} does not distinguish between topic drop of direct and indirect objects.
Rather than defining object topic drop syntactically, she follows \citet{auer1993} and \citet{zifonun.etal1997} by stating that an object can be omitted from the prefield if it is present in the context, which functions as orientation for the interlocutors \citep[125]{frick2017}.
However, the examples that she discusses are only those that contain an overt \textit{das} or those with topic drop where a \textit{das} could naturally be inserted into the prefield \citep[126--127]{frick2017}, i.e., direct objects in the accusative case Accusative case.
This suggests that there are at least mainly direct objects in her data set.

In summary, \citeg{frick2017} corpus study suggests that topic drop can target referential and non-referential subjects and direct objects, whereas there is no evidence that indirect objects are omitted.
Even though subjects seem to be a lot more frequent in the prefield than objects, subject topic drop is not significantly more frequent than object topic drop in her data set, as shown above.

\subsection{\citet{ruppenhofer2018}}
\citet{ruppenhofer2018} investigated the frequency of V1 constructions, which in addition to topic drop include conditional inversion, imperatives, yes–no questions, etc.%
% Footnote
\footnote{The entire list of the 17 constructions that he investigated contains conditional/concessive inversion, exclamative, apodosis stranding, formulas, reporting inversion, \textit{da}-drop, presentational inversion, contrast inversion, yes-no questions, formal imperative, infinitive imperative, hortative/optative, informal imperative, subject topic drop, subject expletive \is{Expletive} drop, cataphoric subject drop, and object topic drop.
See \citet[218]{ruppenhofer2018} for details.}
%
He examined 5 subcorpora containing news articles, websites, tweets, parliamentary speeches, and telephone conversations, as listed in Table \ref{tab:ruppenhofer.data}.

\begin{table}
\caption{Composition of Rupppenhofer's corpus, taken from \citet[209]{ruppenhofer2018}, adapted}
\centering
\begin{tabular}{llr}
\lsptoprule
\multicolumn{1}{c}{Corpus name} & \multicolumn{1}{c}{Type} & \multicolumn{1}{c}{N(Tokens)} \\
\midrule
Huge German Corpus (HGC) & Newspaper articles &	$204\,813\,118 $	\\
German web corpus (deWaC) & Websites & 	$1\,627\,169\,557$ \\
Twitter corpus 	& Tweets &	$105\,074\,399$	\\
Bundestag corpus & Parliamentary speeches & $5\,756\,188$ \\
CALLHOME German speech 	& Telephone conversations &	$202\,964$ \\
\lspbottomrule
\end{tabular}
\label{tab:ruppenhofer.data}
\end{table}

\noindent
He randomly chose 100 V1 constructions%
% Footnote
\footnote{These were all sentences where the first token was a verbal form \citep[217]{ruppenhofer2018}.}
%
from each subcorpus and determined their type.
Of the 17 types that \citet{ruppenhofer2018} considers, four are of relevance here, which he terms subject topic drop, cataphoric subject drop, subject expletive \is{Expletive} drop, and object topic drop \citep[218]{ruppenhofer2018}.

Among the sample of 500 V1 constructions, he found 83 omitted referential subjects with an antecedent \is{Antecedent} and 4 with a postcedent (i.e., the cataphoric cases), 13 omitted expletives, \is{Expletive} and 22 instances of object topic drop.
He notes that the different V1 constructions vary clearly between the corpora he used.
For instance, the tweets contained the highest proportion of covert arguments.
This is in line with the result of my corpus study according to which tweets are the text type \is{Text type} in my corpus that exhibits the fourth-highest omission rate with about 35\% (see \sectref{sec:corpus.texttype} for details).

While \citet{ruppenhofer2018} provides relative numbers, these are only relative to other V1 utterances but not relative to the corresponding full forms.
Therefore, it is also impossible to determine from his corpus whether subject topic drop is more frequent than object topic drop.
\citet{ruppenhofer2018} found no instances where indirect objects were omitted, even when he explicitly searched for verbs that require an indirect object, such as \textit{helfen} (`to help'), \textit{schenken} (`to give as a gift'), or \textit{spenden} (`to donate') in Twitter data \citep[224--225]{ruppenhofer2018}.
\is{Corpus|)}

\subsection{\citet{trutkowski2018}}\is{Acceptability rating study|(}
\citet{trutkowski2018} presents an acceptability rating study to test her account of verbatim and non-verbatim topic drop.
An unknown number of participants rated the acceptability of 6 utterances%
% Footnote
\footnote{This seems to be a rather low number of items for a study with 6 experimental conditions and results in each participant seeing each condition only once.
When presenting the statistical results, \citet{trutkowski2016} also mentions herself that instead of testing 6 items, she should have actually tested 30 in order to be able to find significant results in the ANOVA item analysis (F2).} 
%
on a 7-point scale (7 = absolutely acceptable).
In a 2~$\times$~3 design, she crossed \textsc{Case Antecedent} (structural vs. lexical)%
% Footnote
\footnote{\citet{trutkowski2018} uses the term \textit{oblique case}, which traditionally meant any case besides the nominative \is{Nominative case|(}, but which is synonymous to inherent case \is{Inherent case} in generative grammar and in her usage.
For reasons of consistency, I use \textit{lexical case} \is{Lexical case} as the umbrella term for the dative \is{Dative case} and the genitive. \is{Genitive case|(}
However, it is unclear whether Trutkowski tested only objects in the dative \is{Dative case} or also objects in the genitive case.}
%
and \textsc{Case Gap + Predicate} (identical predicate, identical case vs. different predicate, identical case vs. different predicate, different case).% 
%% Footnote
\footnote{I renamed Trutkowski's variable \textsc{Case Gap} to \textsc{Case Gap + Predicate} to indicate that she considered not only the case of the omitted constituent but also whether the predicate was identical or different between the antecedent \is{Antecedent} and the target utterance.} 
%
In fact, it was a reduced 2 $\times$ 2 $\times$ 2 design with \textsc{Case Antecedent} (structural vs. lexical), \textsc{Case Gap} (structural vs. lexical), and \textsc{Predicate} (identical vs. different), as exemplified in \ref{ex:trutkowski.exp.1} and \ref{ex:trutkowski.exp.2}.
The combination of a mismatch between \textsc{Case Antecedent} \is{Antecedent} and \textsc{Case Gap} with an identical predicate is absent because a predicate usually takes arguments with the same case in German.%%
% Footnote
\footnote{The \textit{same predicate but different case} condition could theoretically be created by exploiting case alternations, e.g., by passivization, i.e., the change from the nominative \is{Nominative case|)} to the accusative  in the \textit{bekommen}-passive and from the dative to the nominative in the \textit{werden}-passive \citep[see, e.g.,][]{wegener1990}.
}
%
This rules out the \textit{lexical case antecedent, structural case gap, identical predicate} \is{Lexical case} \is{Structural case} condition and the \textit{structural case antecedent, lexical case gap, identical predicate} condition.
 
\largerpage
\exg.\label{ex:trutkowski.exp.1}A: Ich treffe den Peter.\\
{} I meet the.\textsc{acc} Peter\\
A: `I am meeting Peter.' \hfill (structural antecedent)
\ag.\label{ex:trutkowski.exp.1.match}B: $\Delta$ Treffe ich auch.\\
{} him.\textsc{acc} meet I too\\
B: `I am meeting (him), too.'\hfill(structural gap, identical predicate)
\bg.\label{ex:trutkowski.exp.1.smatch}B: $\Delta$ Mag ich wirklich gern.\\
{} him.\textsc{acc} like I really a.lot\\
B: `I really like (him).'\hfill(structural gap, different predicate)
\cg.\label{ex:trutkowski.exp.1.mmatch}B: $\Delta$ Vertraue ich nicht.\\
{} him.\textsc{dat} trust I not\\
B: `I do not trust (him).'\hfill(lexical gap, different predicate)

\exg.\label{ex:trutkowski.exp.2}A: Ich vertraue dem Peter.\\
{} I trust the.\textsc{dat} Peter\\
A: `I trust Peter.' \hfill (lexical antecedent)
\ag.\label{ex:trutkowski.exp.2.match}B: $\Delta$ Vertraue ich auch.\\
{} him.\textsc{dat} trust I too\\
B: `I trust (him) too.'\hfill(lexical gap, identical predicate)
\bg.\label{ex:trutkowski.exp.2.smatch}B: $\Delta$ Helfe ich auch gern.\\
{} him.\textsc{dat} help I too gladly\\
B: 'I am also happy to help (him).'\hfill(lexical gap, different predicate)
\cg.\label{ex:trutkowski.exp.2.mmatch}B: $\Delta$ Kenne ich kaum.\\
{} him.\textsc{acc} know I hardly\\
B: `I hardly know (him).'\hfill(structural gap, different predicate)\\
\phantom{.} \hfill \citep[3, adapted]{trutkowski2018}

Table \ref{tab:trutkowski.exp} shows the mean ratings per condition (Trutkowski provides no standard deviations).
Without knowing the corresponding fillers it is hard to judge the descriptive statistics, but the ratings seem relatively low on the 7-point scale.%
%% Footnote
\footnote{\citet{trutkowski2018} does not provide any information whether she included any fillers and if so, which kind of fillers.
If she had not used any fillers or marked or ungrammatical fillers, this would support the claim that the ratings for the materials with topic drop  were relatively low. 
However, if she had included grammatical and unmarked structures as fillers, it is likely that these structures would have received very high ratings and that, in turn, the more marked items would have been perceived as degraded relative to these ``good'' fillers.}
%
This could already indicate that the omission of (pronouns referring to) 3rd person singular proper names is somehow marked (see also experiments \ref*{exp:embedded}, \ref*{exp:top.q1}, \ref*{exp:top.q2}, \ref*{exp:top.s.fv}, and \ref*{exp:top.s.mv} in this book that tested topic drop of the 3rd person singular referring to persons).%
% Footnote
\footnote{A further factor could be that for the identical predicate conditions, a simpler and more natural form is available, which may compete with topic drop: the fragment \is{Fragment} \textit{ich auch} (`me too').
}

\begin{table}
\caption{Mean ratings per condition of \citeg{trutkowski2018} acceptability rating study, taken from \citet[3]{trutkowski2018}, adapted}
\centering
\begin{tabular}{lllrl}
\lsptoprule
Case Antecedent \is{Antecedent} & Case Gap & Predicate & Mean rating & Example\\
\midrule
Structural & Structural & Identical & $3.90$ & \ref{ex:trutkowski.exp.1.match} \\
Structural & Structural & Different & $3.67$ &\ref{ex:trutkowski.exp.1.smatch} \\
Structural & Lexical & Different & $2.60$ & \ref{ex:trutkowski.exp.1.mmatch} \\
Lexical & Lexical & Identical & $3.57$ & \ref{ex:trutkowski.exp.2.match}\\
Lexical & Lexical & Different & $2.53$ & \ref{ex:trutkowski.exp.2.smatch} \\
Lexical & Structural & Different & $3.43$ & \ref{ex:trutkowski.exp.2.mmatch}\\
\lspbottomrule
\end{tabular}
\label{tab:trutkowski.exp}
\end{table}

\noindent\citet{trutkowski2018} reports a significant main effect of \textsc{Case Gap + Predicate} and a significant interaction between \textsc{Case Gap + Predicate} and \textsc{Case Antecedent} in the two-way ANOVA subject analysis (F1) but not in the item analysis (F2).
Items in the \textit{identical case and identical predicate} condition were rated higher than items in the condition \textit{identical case but different predicate} condition, which, in turn, were rated higher than items with \textit{different case and different predicate}.

However, this gradation is only present in the conditions where the antecedent \is{Antecedent} has a structural case  \is{Structural case}, i.e., those in \ref{ex:trutkowski.exp.1}.
For antecedents with a lexical case, \is{Lexical case} the \textit{identical case but different predicate} condition \ref{ex:trutkowski.exp.2.smatch} received the worst mean rating.
The \textit{different case and different predicate} condition \ref{ex:trutkowski.exp.2.mmatch} was more acceptable, with a mean close to the \textit{identical case and identical predicate} condition \ref{ex:trutkowski.exp.2.match}.
I assume that this difference is caused by the fact that in the \textit{different case and different predicate} condition with an antecedent \is{Antecedent} with a lexical case, \is{Lexical case} a constituent in the accusative case is omitted \ref{ex:trutkowski.exp.2.mmatch}, which should generally work better than omitting a constituent in the dative \is{Dative case} or the genitive case. \is{Genitive case|)}

Table \ref{tab:trutkowski.exp} shows that topic drop of a constituent with a lexical case \is{Lexical case} is only as acceptable as topic drop of a constituent with a structural case \is{Structural case} if both the case and the predicate are identical in the context and the target utterance \ref{ex:trutkowski.exp.2.match}.
In contrast, the other conditions where a constituent with a lexical case \is{Lexical case} is omitted, i.e., when there is either a case mismatch between antecedent \is{Antecedent} and target \ref{ex:trutkowski.exp.1.mmatch} or a change in predicate between antecedent utterance and target utterance \ref{ex:trutkowski.exp.2.smatch}, are degraded.

From this, Trutkowski concludes that ``[t]opic drop of obliquely [i.e., lexically, LS] cased arguments \is{Argument} is fine when context and target predicates are the same'' and that ``case identity is not a sufficient condition for obliquely [i.e., lexically, LS] cased NPs to be dropped'' \citep[4]{trutkowski2018}.
However, it remains questionable, first, whether a rating of 3.57 on a 7-point scale for the \textit{identical case and identical predicate} condition with a lexical case \is{Lexical case} can really be called ``fine''.
Second, it is also dubious whether a difference in acceptability of one point to the other two conditions, in which the covert constituent is a constituent with a lexical case, \is{Lexical case} justifies the assumption that topic drop is grammatical in the first case and ungrammatical in the second.
This more gradual acceptability cline provides first support for treating the syntactic function not as a licensing condition but as a usage factor. \is{Acceptability rating study|)}

\section{Information-theoretic predictions for syntactic function}\label{sec:info.theory.function}  
As I indicated in \sectref{sec:avoid.troughs}, the \textit{avoid troughs} principle derived from \textit{UID} \is{Uniform information density} can potentially explain the influence of several factors on the speaker's choice between the full form and topic drop.
The syntactic function of the prefield constituent is one of these factors that have been discussed  in the previous literature, partly even as licensing factors for topic drop.
In \sectref{sec:usage.function.theory}, I reviewed three issues concerning syntactic function.
(i) Topic drop of subjects is argued to be more frequent than topic drop of direct objects, (ii) topic drop of direct objects or objects with a structural case \is{Structural case} shall be more frequent than topic drop of indirect objects or objects with a lexical case \is{Lexical case} or the latter should not be possible at all, and (iii) topic drop of 1st and 2nd person objects is said to be largely impossible in contrast to topic drop of 3rd person objects.
These effects could be explained by the relative frequency of the corresponding full forms.

\subsection{Subject vs. object topic drop}
\is{Accusative case|(}
While the relatively free word order in German\il{German|(} allows almost any constituent to occupy the prefield position, \is{Prefield|(} in practice, the sentence-initial constituent in V2 sentences is often the subject.
\citet[44]{engel1972} claims that it occupies the prefield in about 60\% of the cases, while arguments \is{Argument} in the dative \is{Dative case} or the accusative \is{Accusative case} seem to be very rare in this position.
This is supported by the results of several empirical studies.
\citet{hansen-schirra.gutermuth2017} analyzed the German-English CroCo corpus,  \is{Corpus|(} which contains original English and German texts, as well as translations from English into German and vice versa \citep{hansen-schirra.etal2012}.
What is of interest here are the 121 original German texts with about 288\,000 tokens \citep[31]{hansen-schirra.etal2012}.
There, the prefield position is occupied by the subject in 50.25\% of the declarative sentences, by an adverbial \is{Adverbial} in 31\%, by an object in 8.46\%, and by another element in 10.29\% of the sentences \citep[311]{hansen-schirra.gutermuth2017}.
\citet{zybatow2014} classified prefield constituents by syntactic category in 20 textbook texts each, for the subjects German, biology, and history for grades 5 to 9 \citep[97]{zybatow2014}.
She finds that in about 47\% of the cases, the prefield constituent is the subject, in about 45\% of the cases it is an adverbial, \is{Adverbial} and in about 5\% it is an object \citep[98]{zybatow2014}.%
% Footnote
\footnote{Zybatow presents a subdivision of her data by school subject and grade that reveals that objects in the prefield are most frequent in history books for grade 6, where about 10\% of the prefield constituents are objects.
See \citet[98]{zybatow2014} for more details.}
%
Further empirical evidence comes from \citet{bader.haeussler2010}, who compared, among other things, subject vs. object prefield positioning in a data set based on newspaper articles.
They restricted themselves to cases where the object was a DP with the masculine definite accusative singular or the dative \is{Dative case} plural article \textit{den} \citep[see for details][722--723]{bader.haeussler2010}.%
%% Footnote
\footnote{They provide the following examples for illustration:
%\vspace{-0.5\baselineskip}
\ex.
\ag.Wir haben den Studenten unterstützt.\\
we have the student.\textsc{acc.sg} supported\\
`We supported the student.'
\bg.Wir haben den Studenten geholfen.\\
we have the students.\textsc{dat.pl} helped\\
`We helped the students.' \citep[722]{bader.haeussler2010}\par
\vspace{-1.75\baselineskip}
%\vspace{-2em}
}
%
They found that in their subcorpus prefield SO/OS, where they looked at cases with a \textit{den} object at any position, the prefield was occupied by the subject in 58.5\% of the cases, by the \textit{den} object in 12.4\%, and by another phrase like an adverbial \is{Adverbial} phrase or a prepositional phrase in 29.1\% of the cases \citep[725]{bader.haeussler2010}.
\is{Corpus|)}

Despite the variation across empirical studies in terms of the specific proportion of objects in the prefield and without neglecting the diverse factors that determine the word order in German (see, e.g., \cite{hoberg1981}, \cite{kempen.harbusch2005},  \cite{rauth2020} for word/object order in the middle field \is{Middle field} and \cite{speyer2007}, \cite{filippova.strube2007}, \cite{bader.etal2017} for how the prefield is filled), it seems to be a priori, and if we neglect the precontext, more likely that the subject of the clause will appear in the prefield than another element, such as an object or even a predicative (which seems to be contained in the rest category of \cite{hansen-schirra.gutermuth2017} and which apparently did not appear in the texts investigated by \cite{zybatow2014}).
This increased frequency also raises the chance that the subject will be targeted by topic drop in suitable text types%
%% footnote
\footnote{Note that the studies that I just discussed investigated (conceptually) written text types, \is{Text type|(} where topic drop does not or only rarely occur.
This explains why the subjects in the prefield are realized despite their high frequency.} 
%
because according to the information-theoretic approach, the likelihood of a constituent being omitted increases with its predictability in context. \is{Predictability}
Therefore, the frequency of the full form may explain issue (i), the postulated higher relative frequency of subject topic drop compared to object topic drop.
As shown in \sectref{sec:usage.function.studies}, this frequency difference between subjects and objects has not yet been empirically proven.
Due to the small size of the FraC, I am not able to provide this evidence either (see \sectref{sec:corpus.function}).
This question must remain a research desideratum until a sufficiently large syntactically annotated corpus of a text type that allows for topic drop is available.\is{Prefield|)}

\subsection{Direct vs. indirect objects}
Issue (ii), the potential difference between direct and indirect objects or between objects with a structural vs. a lexical case, \is{Lexical case} could be explained similarly as the potential difference between subject and object topic drop.
The data from \citet{bader.haeussler2010} indicate that dative \is{Dative case|(} objects occur only half as often in the prefield than accusative objects \is{Accusative case}. \is{Prefield}
Consequently, the likelihood of their omission should also be reduced in relative terms.
However, not only did the corpus studies discussed in \sectref{sec:usage.function.studies} find fewer omissions of indirect objects, but they found none at all.
On the one hand, this could be because the data sets used were too small to find a sufficient number of the rare indirect objects.
On the other hand, the text types studied may also play a role.
For example, \citeg{bader.haeussler2010} examined news articles, whereas topic drop studies usually focus on text messages or spoken language.
In the latter two text types, \is{Text type|)} the proportion of indirect objects in the prefield could in turn be lower. \is{Prefield}
This would also result in a lower proportion of topic drop of indirect objects, possibly a proportion that is so small that it cannot be measured in the data sets used.
Here, corpus studies of more extensive data sets are pending.
As mentioned above, my corpus \is{Corpus} study was conducted on a rather small, mostly hand-annotated corpus and, thus, cannot provide insights into this issue either.
The results of \citeg{trutkowski2018} acceptability rating study suggest that topic drop of indirect objects can be as acceptable as topic drop of direct objects provided that the antecedent and the omitted constituent have the same case and the verb of the antecedent and the target utterance is the same. \is{Antecedent}
However, recall that the acceptability ratings were rather mediocre and that the study has some methodological drawbacks.
In information-theoretic terms, it could be that the previous mention of the prefield constituent in the local context with an identical verb could make it highly predictable, even if it is not generally predictable in this position. \is{Predictability}
Such contexts as the ones used by \citet{trutkowski2018}, two subsequent assertions about the same person with the same verb like \ref{ex:trutkowski.exp.2}, repeated as \ref{ex:trutkowski.exp.2.rep}, however, are special and probably not very frequent in natural language, not least because the fragment \is{Fragment} \textit{ich auch} (`me too') is the more natural alternative for \ref{ex:trutkowski.exp.2.rep.b}.

\ex.\label{ex:trutkowski.exp.2.rep}
\ag.A: Ich vertraue dem Peter.\\
{} I trust the.\textsc{dat} Peter\\
A: `I trust Peter.'
\bg.\label{ex:trutkowski.exp.2.rep.b}B: $\Delta$ Vertraue ich auch.\\
{} him.\textsc{dat} trust I too\\
B: `I trust (him) too.'  \citep[3, adapted]{trutkowski2018}

\subsection{1st and 2nd person objects vs. 3rd person objects}
This leaves us with issue (iii), which is also related to grammatical person: the nonexistence or at least extremely low frequency of object topic drop with the 1st and the 2nd person.
As discussed in \sectref{sec:usage.function.theory}, the \citet{duden2016, duden2022} argues that weakly stressed 1st and 2nd person object pronouns rarely occur in the prefield and, thus, are rarely omitted from there (\cite[§1378]{duden2016}, \cite[§35]{duden2022}).
Similarly, \citet[219]{volodina.onea2012} state that when 1st and 2nd person object pronouns occur in the prefield, this involves stress, which in turn blocks topic drop.
I conducted a corpus \is{Corpus|(} study to investigate the information-theoretic frequency hypothesis for issue (iii).
Specifically, I investigated whether the low number of topic drop of 1st and 2nd person object pronouns can be explained by a likewise low number of corresponding full forms with overt 1st and 2nd person object pronouns in the prefield.

I searched the German reference corpus DeReKo \citep{dereko2022} to assess the general frequency of subjects and objects, as well as their frequency in the prefield. \is{Prefield|(}
Specifically, I searched for subject and object pronouns with different cases and grammatical persons in all corpora of the archive TAGGED-T of  DeReKo.
This archive is large enough that one can expect to find a substantial amount of utterances with object pronouns in the prefield.
It contains almost 4.5 million texts (newspaper articles, Wikipedia articles, speeches, and interviews) with more than 1 billion words from 1997 to 2010, which were part-of-speech tagged using the TreeTagger \citep{schmid1994, schmid1995}.
I restricted myself to subject pronouns in the nominative \is{Nominative case|(} case and object pronouns in the accusative \is{Accusative case} and the dative \is{Dative case} case of the 1st and 2nd person singular and the 3rd person singular masculine to have as few syncretic \is{Syncretism} forms as possible%
% Footnote
\footnote{While the 3rd person masculine object pronoun in the dative \is{Dative case} case, \textit{ihm}, is identical to the form of the 3rd person neuter object pronoun in the dative, the paradigms of the plural persons and of the 3rd person singular of different gender exhibit even more syncretic \is{Syncretism} forms (marked in bold):
%\vspace{0.5em}

\begin{tabular}{llllll}
\lsptoprule
& 3SG \textsc{f} & 3SG \textsc{n} & 1PL & 2PL & 3PL \\
\midrule
Nominative & \textbf{\textit{sie}} & \textbf{\textit{es}} & \textit{wir} & \textbf{\textit{ihr}} & \textbf{\textit{sie}} \\
Accusative & \textbf{\textit{sie}} & \textbf{\textit{es}} & \textbf{\textit{uns}} & \textbf{\textit{euch}} & \textbf{\textit{sie}} \\
Dative & \textbf{\textit{ihr}} & \textbf{\textit{ihm}} & \textbf{\textit{uns}} & \textbf{\textit{euch}} & \textit{ihnen}\\
\lspbottomrule
\end{tabular}\bigskip

}
%
to ensure that the matches for one form correspond as good as possible to the matches for one function.
For each form, I assessed (a) how frequent it is generally in the data set and (b) how frequent it is in the prefield.
I approximated the positioning in the prefield by searching for sentence-initial occurrences that are followed by a finite verb.%
% Footnote
\footnote{I used the following query:\texttt{ mich /w0 <sa> /+w1 MORPH(VRB fin)}. 
For the demonstrative 3rd person pronouns, I additionally included an expression that ensured that the \textit{der}, \textit{den}, or \textit{dem} was used as a demonstrative and not as a determiner or relative pronoun:\texttt{ den /w0 MORPH(PRON dem sub) /w0 <sa> /+w1 MORPH(VRB fin)}.
}
Table \ref{tab:pronoun.freq.dereko} shows the results including the proportion of prefield occurrences in the total number of occurrences.

\begin{table}
\caption{Frequency of personal and demonstrative subject and object pronouns of the 1st person singular, the 2nd person singular, and the 3rd person singular masculine in the DeReKo TAGGED-T archive}
\centering
\begin{tabular}{lcccrrr}
\lsptoprule
Pronoun & \Centerstack{Person,\\number} & Gender & Case & \Centerstack[c]{Total oc-\\currences} & \Centerstack{Occur-\\rences in\\prefield} & \Centerstack{Propor-\\tion in\\prefield} \\
\midrule
\textit{ich} & 1SG & -- & \textsc{nom} & $2\,068\,332$ & $570\,809$ & $28.60\%$ \\
\textit{du} & 2SG & -- & \textsc{nom} & $130\,166$ & $10\,230$ & $7.86\%$ \\
\textit{er} & 3SG & \textsc{m} & \textsc{nom} & $3\,808\,561$ & $721\,514$ & $18.94\%$  \\
\textit{der}* & 3SG & \textsc{m} & \textsc{nom} & $7\,035$ & $3\,882$ & $55.18\%$  \\
\tablevspace
\textit{mich} & 1SG & -- & \textsc{acc} & $380\,986$ & $4\,534$ & $1.19\%$  \\
\textit{dich} & 2SG & -- & \textsc{acc} & $33\,224$ & $110$ & $0.33\%$  \\
\textit{ihn} & 3SG & \textsc{m} &\textsc{acc} & $410\,037$ & $3\,510$ & $0.86\%$ \\
\textit{den}* & 3SG & \textsc{m} & \textsc{acc} & $10\,172$ & $2\,795$ & $27.48\%$  \\
\tablevspace
\textit{mir} & 1SG & -- & \textsc{dat}\is{Dative case} & $346\,165$ & $18\,192$ & $5.26\%$  \\
\textit{dir} & 2SG & -- & \textsc{dat} & $28\,869$ & $121$ & $0.42\%$  \\
\textit{ihm} & 3SG & \textsc{m}/\textsc{n} & \textsc{dat} & $442\,615$ & $15\,465$ & $3.49\%$  \\
\textit{dem}* & 3SG & \textsc{m}/\textsc{n}  & \textsc{dat} & $11\,985$ & $9\,952$ & $83.04\%$ \\
\lspbottomrule
\rowcolor{white}
\multicolumn{7}{r}{*demonstrative pronoun} \\
\end{tabular}
\label{tab:pronoun.freq.dereko}
\end{table}

\noindent
I assume that two quantities are relevant to the argument here.
First, the total number of occurrences in the prefield is an estimate of how likely it is that any given utterance begins with the corresponding pronoun as the prefield constituent.%
% Footnote
\footnote{This number could be converted into a proportion by dividing it by the total number of (declarative) utterances in the corpus.
However, since this total number is the same for all pronouns, the ratios can also be read from the absolute numbers.}
%
Second, the proportion of occurrences in the prefield to the total number of occurrences serves as an estimate for the probability that an utterance begins with the corresponding pronoun as the prefield constituent, provided that the corresponding pronoun occurs in the utterance at all. 

The data confirm the observations just made about the difference in frequency between subjects and objects in the prefield.
The personal pronouns of all grammatical persons are more frequent as subjects than as objects, both generally and in the prefield (but not so the demonstrative pronouns, which we come to below).

In addition, we see that the 1st person singular object pronouns \textit{mich} and \textit{mir} are rare in the prefield according to both measures.
The number of occurrences in the prefield of \textit{mir} (about 18\,000) and \textit{mich} (about 4\,500) is remarkably lower than the number of occurrences of the 1st person singular subject pronoun \textit{ich} (about 570\,000).
This suggests that it is generally less likely for an utterance to start with either of the two object pronouns than with the corresponding subject pronoun.
The same result arises from the relative numbers.
While almost 30\% of all \textit{ich} occurrences are in the prefield position, for the dative \is{Dative case} object pronoun \textit{mir} it is just over 5\%, and for the accusative \is{Accusative case} object pronoun \textit{mich} even only around 1\%.
The 2nd person singular shows a similar tendency, with even lower absolute and relative values. 
At first glance, also the pattern for the 3rd person singular masculine seems to be the same.
The subject pronoun \textit{er} appears more than 720\,000 times in the prefield and in almost 20\% of the utterances.
In turn, the dative \is{Dative case} object pronoun \textit{ihm} occurs about 15\,000 times in the prefield and only in about 3.5\% of the utterances, and  the accusative \is{Accusative case} object pronoun \textit{ihn} only about 3\,500 times and in less than 1\% of the utterances.
The picture partly changes when we look not only at the 3rd person personal pronouns but also at the demonstratives.
The likelihood that an utterance starts with a 3rd person demonstrative pronoun is relatively low, about 2\,800 occurrences of \textit{den} occur in the prefield, about 10\,000 for \textit{dem}, and about 4\,000 for \textit{der}.
However, if they occur in an utterance, they do so particularly frequently in the prefield, in over 80\% of the cases for the dative, \is{Dative case} in more than 55\% for the nominative, \is{Nominative case|)} and still in around 27\% for the accusative. \is{Accusative case}
This confirms the results of \citet{bosch.etal2007} mentioned in \sectref{sec:usage.function.theory}, according to which demonstrative pronouns are more frequent in the prefield than personal pronouns.
Furthermore, it suggests that a more or less equivalent full form for a 3rd person singular topic drop should be formed with a demonstrative rather than a personal pronoun in the prefield.

The results provide evidence for the frequency-based information-theoretic explanation.
Full forms with 1st or 2nd person singular object pronouns in the prefield are rare in absolute and relative terms, the latter in particular in comparison to full forms with 3rd person demonstrative object pronouns in the prefield.
Consequently, \textit{mich}, \textit{dich}, \textit{mir}, and \textit{dir} are less predictable in this position and cannot be omitted equally well. \is{Predictability}

\largerpage
For the hypothesis that the positioning of 1st and 2nd person object pronouns in the prefield requires contrast or special stress, the corpus data from the DeReKo provide mixed evidence.
There are cases such as \ref{ex:dich.contr}, where \textit{dich} cannot be omitted according to my judgment and where the object pronoun in the prefield is indeed contrastive or would require stress.\is{Prefield|)}

\ex.\label{ex:dich.contr}
\a.\textit{Ihre Goldene Hochzeit feiern heute Renate und Helmut Urschel. Seit 50 Jahren geht das Bad Münsterer Ehepaar gemeinsam durchs Leben.}\\
`Renate and Helmut Urschel are celebrating their golden wedding anniversary today. The couple from Bad Münster has been going through life together for 50 years.'
\bg.,,*(Dich) heirate ich einmal``, sagte Helmut Urschel seiner späteren Frau gleich am ersten Abend beim Tanz im Hotel Kaiserhof.\\
you.\textsc{acc.2sg} marry I once said Helmut Urschel his eventual wife immediately on.the first evening at.the dance in.the hotel Kaiserhof\\
`\,`You're the one I'm going to marry one day,' Helmut Urschel told his future wife on the very first evening at the dance in the Hotel Kaiserhof.' [DeReKo, RHZ07/AUG.29363, Rhein-Zeitung, 08/30/2007], my judgment

For other examples such as \ref{ex:mir.ncontr}, I would likewise argue that \textit{mir} cannot be omitted, but the object pronoun in the prefield does not seem to be used contrastively nor does it necessarily require special stress.
The hypothesis that stress blocks omission cannot explain why topic drop is not possible in this case.

\ex.\label{ex:mir.ncontr}
\a.\textit{Ich spiele gerne Fußball, das ist mein liebstes Hobby. Ich spiele beim VfB Fallersleben in der E-Jugend und mache das schon seit fünf Jahren. Ich bin in meiner Mannschaft der rechte Stürmer.}\\
`I like playing soccer, it's my favorite hobby. I play for VfB Fallersleben in the under-11 team and have been doing so for five years. I'm the right striker on my team.'
\bg.*(Mir) macht das einfach Spaß, und Tore schießen ist natürlich am schönsten.\\
me.\textsc{dat} makes that simply fun and goals scoring is of.course at.the most.beautifully\\
`I just enjoy it, and scoring goals is the best thing, of course.' [DeReKo, BRZ08/JAN.12343, Braunschweiger Zeitung, 01/28/2008], my judgment

Also in information-theoretic terms, there seems to be no clear explanation for why topic drop is not possible or at least very marked.
The speaker, to whom the pronoun refers, is given \is{Givenness} and occurs linguistically as the subject in the previous utterances.
Therefore, it is predictable \is{Predictability} and its recovery in the critical utterance should be perfectly fine. \is{Recoverability|(}
Still, though, topic drop seems to be blocked.
So while the general pattern of  object topic drop with different grammatical persons is accounted for by the information-theoretic explanation based on frequencies, examples such as \ref{ex:mir.ncontr} suggest that it may not be sufficient to explain the whole picture.

At this point, it may be fruitful to briefly come back to the other proposed explanations for the potential ban on 1st and 2nd person object topic drop from the literature, discussed in \sectref{sec:usage.function.theory}.
\citet{schulz2006} proposes that 1st and 2nd person objects are harder to omit because they are more marked as continued topics.
Recall that she considers being a continued topic \is{Topic} a prerequisite for topic drop.
In example \ref{ex:mir.ncontr}, however, the 1st person singular object pronoun must be considered a rather optimal continued topic.
In the previous utterances, the speaker is clearly the topic denotation.
Thus, \textit{mir} can be considered an ideal continued topic expression, which should be easily recoverable. \is{Recoverability|)}
The fact that it still cannot be omitted, therefore, also poses a problem for \citeg{schulz2006} account.\is{Topic} 
\is{Reflexivity|(}
In contrast, \citeg{fries1988} approach according to which the syncretism \is{Syncretism} between 1st and 2nd person object and reflexive pronouns blocks their omission if the context is not explicitly marked for (non-)reflexivity can account for example \ref{ex:mir.ncontr}.
The form of the preverbal object pronoun \textit{mir} is syncretic \is{Syncretism} with the corresponding reflexive \textit{mir} (e.g., \textit{Ich merke mir das} (lit. `I memorize myself that')), the context does not explicitly provide a non-reflexive reading, and, therefore, topic drop is blocked, as predicted by \citet{fries1988}.\is{Reflexivity|)}
Furthermore, also \citeg{sigurdsson2011} relative specificity constraint according to which an omitted object cannot be more specific than the overt subject, correctly predicts topic drop to be impossible in \ref{ex:mir.ncontr}.
The omitted object \textit{mir} is a 1st person singular element and, thus, more specific than the overt subject \textit{das}, referring to the 3rd person referent \textit{Fußball spielen}, which is furthermore [-human].
Thus, also the higher specificity of the object compared to the subject may block its omission.

In sum, neither the information-theoretic account, nor the account based on stress and contrast, nor  \citeg{schulz2006} continued topic account can explain why topic drop seems to be impossible in example \ref{ex:mir.ncontr}.
By contrast, it is \citeg{fries1988} reflexivity \is{Reflexivity} account and \citeg{sigurdsson2011} relative specificity constraint that adequately describe the introspective judgment.
Here, then, the information-theore- tic approach motivated purely by frequencies seems to reach its limits for the first time.
It seems that an additional factor to explain the choice between full forms and topic drop is needed.
Determining what that factor is must be left to future research.

\section{Corpus study of syntactic function}\label{sec:corpus.function}
In the following, I present the results concerning syntactic function from two of the three data sets that I analyzed in my corpus study.
I looked at the frequencies and omission rates of referential and non-referential subjects, as well as of accusative \is{Accusative case} and dative \is{Dative case} objects.
Due to data sparsity, I cannot provide an inferential statistical analysis, but I present a descriptive overview that may inform in-depth research on this factor in the future.%
% Footnote
\footnote{The corpus data and the analysis scripts can be accessed online: \url{https://osf.io/zh7tr}.
For copyright reasons, I cannot provide the actual linguistic material but only the IDs and the annotations of each instance.}
%

\subsection{Syntactic function in \textsc{FraC-TD-Comp}}\label{sec:frac.td.comp.function}
As sketched in \sectref{sec:corpus.comp}, the \textsc{FraC-TD-Comp} data set contains 873 instances of topic drop and 3\,211 full forms that principally allow for topic drop (neglecting the typical occurrence of topic drop in certain text types, see \sectref{sec:def.texttype}).
Table \ref{tab:FraC.Comp.syn.function} shows the frequency per syntactic function and the omission rate in \textsc{FraC-TD-Comp}.

\begin{table}
\caption{Full forms, instances of topic drop, and omission rates as a function of syntactic function in the \textsc{FraC-TD-Comp} data set}
\centering
\begin{tabular}{lrrrr}
\lsptoprule
\multicolumn{1}{c}{Syntactic function} & \multicolumn{1}{c}{Full form} & \multicolumn{1}{c}{Topic drop} & \multicolumn{1}{c}{Total} &  \multicolumn{1}{c}{Omission rate}\\
\midrule
Subject & $3\,051$ & $807$ & $3\,858$ & $20.92\%$\\
-- referential & $2\,780$ & $778$ & $3\,558$  & $21.87\%$ \\
-- non-referential & $271$ & $29$ & $300$ & $9.67\%$\\ 
\tablevspace
Object & $160$ & $66$ & $226$ & $29.20\%$ \\
-- accusative \is{Accusative case} & $152$ & $66$ & $218$ & $32.14\%$ \\
-- dative \is{Dative case} & $8$ & $0$ & $8$ & $0.00\%$\\
\lspbottomrule
\end{tabular}
\label{tab:FraC.Comp.syn.function}
\end{table}

In the prefield position of the utterances in \textsc{FraC-TD-Comp}, there are either overt or covert subjects in the nominative case, \is{Nominative case} overt or covert objects in the accusative case, \is{Accusative case} or overt objects in the dative \is{Dative case}case.
There is no instance where topic drop targets an indirect object in the dative case. \is{Dative case}
Objects in the genitive case \is{Genitive case|(} or prepositional objects \is{Prepositional object} never occur in the prefield, \is{Prefield|(} neither overtly as pronouns nor covertly.

For subjects, it can be differentiated between referential and non-referential subjects (see the theoretical discussion in \sectref{sec:usage.function.theory}), namely the expletive \is{Expletive} \textit{es} in subject function.%
%% Footnote
\footnote{For the reasons discussed in Footnote \ref{note:man} and for consistency, I classified all occurrences of \textit{man} as referential.}
%
There are 271 instances with an overt expletive \is{Expletive} pronoun in the prefield and 29 cases where an expletive \is{Expletive} is omitted from this position.
This result is consistent with the corpus data discussed in \sectref{sec:topicality.ness}, as well as with the result of experiment \ref*{exp:ex}.
Expletives \is{Expletive} can be targeted by topic drop.

What stands out is the fact that objects in the prefield generally seem to be much rarer than subjects. \is{Prefield|)}
This is also true for accusative objects \is{Accusative case} but especially for dative \is{Dative case} objects, of which only 8 appear in pronominalized form in the prefield and none as topic drop.%
% Footnote
\footnote{The eight dative \is{Dative case} objects occur in seven different text types: \is{Text type} two in opinion pieces and one each in a news article, blog, online chat, email, radio transcript, and tweet.
The diversity of these text types does not hint at a systematic influence of the text type, as one might have assumed.}
%
Due to the small size of the corpus, no definitive conclusions can be drawn about the omission of indirect objects or objects with a lexical case, \is{Lexical case} but this observation is consistent with the assumption that topic drop of dative \is{Dative case} objects (and of genitive and prepositional objects) \is{Prepositional object|)} occurs rarely or not at all in corpus data because the corresponding full forms are already very infrequent.

\subsection{Syntactic function in \textsc{FraC-TD-SMS}}\label{sec:frac.td.sms.function}
Table \ref{tab:frac.syn.func} shows the frequency of topic drop and the omission rate by syntactic function in the \textsc{FraC-TD-SMS} data set, a subset of \textsc{FraC-TD-Comp} that is restricted to only the 353 occurrences of topic drop and the 201 full forms in the text message subcorpus of the FraC (see \sectref{sec:corpus.mess}).

\begin{table}
\centering
\caption{Full forms, instances of topic drop, and omission rates as a function of syntactic function in the \textsc{FraC-TD-SMS} data set}
\begin{tabular}{lrrrr}
\lsptoprule
\multicolumn{1}{c}{Syntactic function} & \multicolumn{1}{c}{Full form} & \multicolumn{1}{c}{Topic drop} & \multicolumn{1}{c}{Total} & \multicolumn{1}{c}{Omission rate} \\
\midrule
Subject & $199$ & $348$ & $547$ & $63.62\%$ \\
-- referential & $194$ & $337$ & $531$ & $63.47\%$\\
-- non-referential & $5$ & $11$ & $16$ & $68.75\%$\\
\tablevspace
Object & $2$ & $5$  & $7$  & $71.43\%$ \\
-- accusative \is{Accusative case} &  $2$ & $5$  & $7$ &  $71.43\%$\\
-- dative \is{Dative case}& $0$ & $0$ & $0$ & $0.00\%$\\
\lspbottomrule
\end{tabular}
\label{tab:frac.syn.func}
\end{table}

\noindent
Like in \textsc{FraC-TD-Comp}, most instances of topic drop in this data set are referential subjects.
They are significantly more often omitted than realized ($\chi^2(1) = 38.51$, $p~<~0.001$), according to a chi-squared test for given probabilities calculated in R \citep{rcoreteam2021} against a 50:50 baseline.
Their omission rate of 63.47\% is even 10\% higher than the 53\% that \citet{frick2017} attested in her larger data set of Swiss German text messages, a significant difference according to a Pearson's chi-squared test with Yates's continuity correction calculated in R ($\chi^2(1) = 20.3, p < 0.001$).

For both objects and non-referential subjects, there are fewer than 20 instances each in the \textsc{FraC-TD-SMS} data set.
Therefore, the resulting omission rates (non-referential subjects: 68.75\%, direct objects: 71.43\%) are hardly reliable.
A much larger corpus of text messages would be needed to find enough corresponding instances of object topic drop and non-referential subject topic drop to compare with \citeg{frick2017} Swiss German data.
This is even more necessary for indirect objects and predicates, for which I did not find any instances neither in the \textsc{FraC-TD-SMS} nor in the larger \textsc{FraC-TD-Comp} data set.

\section{Summary: syntactic function}\label{sec:usage.function.summary}
In this chapter, I bundled the relevant theoretical and empirical literature on the syntactic function of the prefield constituent as a factor for topic drop.
I presented my qualitative empirical results and discussed them in light of my information-theoretic predictions.
For subjects, I already showed in \sectref{sec:topicality.ness} in the first part of this book that topic drop is not restricted to referential subjects but can also target non-referential expletives. \is{Expletive}
This position was strengthened through the results of the corpus studies reported and through my own corpus results.
For object topic drop, I presented three issues in the theoretical overview.
(i) Object topic drop may be rarer than subject topic drop.
(ii) There may be a difference in how well direct vs. indirect objects or objects with a structural vs. a lexical case \is{Lexical case} can be omitted.
(iii) 3rd person objects may be better omitted than 1st and 2nd person objects (if the latter can be omitted at all).

The three corpus studies from the previous literature that I discussed and my corpus investigations provided no support for claim (i) that topic drop more frequently targets subjects than objects.
It must be noted, however, that it was only possible to determine and compare relative frequencies in \citeg{frick2017} study and my own since \citet{poitou1993} and \citet{ruppenhofer2018} did not determine full forms as baselines, i.e., complete utterances with subjects or objects in the prefield that could be omitted but were not.
Since Frick's study is based on Swiss German data and since my study suffered from data sparsity, the determination of possible frequency differences between subject and object topic drop in the Standard German of Germany remains a research desideratum.

The corpus studies also do not reveal how frequently indirect objects or objects with a lexical case \is{Lexical case} are targeted by topic drop.
Neither \citet{poitou1993}, \citet{frick2017}, \citet{ruppenhofer2018}, nor myself found instances of topic drop of dative \is{Dative case} or genitive \is{Genitive case|)} objects, not even with identical predicates.%
% Footnote
\footnote{Note that in neither of the corpus studies from the literature, nor in my own study there was a distinction between a lexical and a structural accusative case \is{Accusative case} as proposed by \citeg{sternefeld1985}.
Thus, there is no empirical basis to verify whether they behave differently.}
%
As a possible information-theoretic, frequency-based explanation, I discussed that they do not occur because the corresponding full forms with an overt indirect object in the prefield are already very infrequent.
Despite some methodological shortcomings, \citeg{trutkowski2018} acceptability rating study suggests at least that topic drop of an indirect object or an object with a lexical case \is{Lexical case} can be as acceptable as topic drop of a direct object or an object with a structural case \is{Structural case} if the antecedent and the omitted constituent have the same case and the verbs or predicates of the antecedent and the target utterance are the same. \is{Antecedent}

For (iii), the object asymmetry with respect to grammatical person, I argued initially that it is most reasonable to adopt \citeg{mornsjo2002} and \citeg{volodina.onea2012} explanation according to which 1st and 2nd person objects, unlike 3rd person objects, are usually stressed when they occur in the prefield and that this prosodic \is{Prosody} prominence blocks ellipsis.
The empirical studies from the previous literature did not report the grammatical person of the omitted objects, nor do the studies presented in Chapter \ref{ch:usage.person} on grammatical person (they focus on subjects).
Based on the results of my corpus study of the DeReKo, I argued that the presumed impossibility or very low frequency of topic drop of objects of the 1st and 2nd person at least partly hinges on the rarity of the corresponding full forms.
At the same time, however, it became evident that neither the frequency nor the incompatibility of stress and omission can explain the whole picture.
An authentic example with an unstressed 1st person dative \is{Dative case|)} object pronoun in the prefield that cannot be omitted suggests that the impossibility of 1st and 2nd person object topic drop is impacted by further factors such as the syncretism \is{Syncretism} with reflexive forms \citep{fries1988} or the relative specificity constraint \citep{sigurdsson2011}.

A final remark concerns predicatives.
They are not only rarely discussed in the theoretical literature but were also not considered in the empirical studies.
Thus, there is no corpus or rating data that allows me to judge how often or well they can be omitted.
Also, there were no cases of covert predicatives in the FraC data sets.
I leave it to future research to address this research desideratum. 

In the next chapter, I revisit the factor of topicality from the first part of this book, this time considering it as a usage factor.
\is{Corpus|)}\il{German|)} \is{Accusative case|)}
