\chapter{Grammatical person, verbal inflection, and ambiguity avoidance}\label{ch:usage.person}
As I already mentioned in Chapter \ref{ch:usage.function} on syntactic function, the grammatical person of the omitted constituent is discussed as a further important factor for the usage of topic drop.
Since object topic drop is mainly restricted to the 3rd person singular according to the theoretical literature, I focus on the impact of grammatical person and number on the omission of subjects in this chapter and only briefly mention its role for objects.
First, I provide a theoretical overview of the literature on grammatical person, and, since the three factors are argued to be closely related, also on the inflectional ending on the verb after topic drop and ambiguity avoidance.%
%% Footnote
\footnote{\sectref{sec:usage.person.theory} was published in a similar form in \citet{schafer2021}.}
%%
After presenting previous empirical results with respect to these three factors and the information-theoretic predictions, I turn to my investigations.
Grammatical person is the factor that I investigated in most detail, both corpus-linguistically and experimentally.
I first present the corpus results and then turn to the results of five acceptability rating studies.
In the corpus study and two of the experiments, I also looked at verbal inflection and ambiguity avoidance.

\section{Theoretical overview}\label{sec:usage.person.theory.general}
\subsection{Theoretical overview of grammatical person and verbal inflection}\label{sec:usage.person.theory}
\il{German|(}
\citet[272]{volodina2011} states that subject pronouns of all persons and numbers can be targeted by topic drop.
However, according to the previous literature, the 1st person singular subject pronoun \textit{ich} (`I') is especially often omitted \citep{tesak.dittmann1991, auer1993, volodina.onea2012, imo2013, imo2014}.
This prevalence of 1st person subject topic drop is explained by two different types of hypotheses, which I term \textit{inflectional hypothesis} and \textit{extralinguistic hypotheses}.

\is{Verbal inflection|(}
The \textit{inflectional hypothesis} was first proposed by \citet[198]{auer1993}%
%% Footnote
\footnote{It was later adapted by \citet{imo2013, imo2014}, who additionally suggests an \textit{extralingustic hypothesis}. See below.}
%
and connects the grammatical person of the omitted constituent and the inflection of the following verb.
His claim is that topic drop of the 1st person singular is easily possible because the verbal morphology in the present tense singular in German is still so differentiated that the grammatical person can be expressed only by inflection.%
%% Footnote
\footnote{Note that this argumentation has similarities to what is nowadays called \textit{Taraldsen's generalization}, i.e., the hypothesis that null subjects \is{Null subject} in \textit{pro}-drop \is{@\emph{pro}-drop}languages are licensed by a rich inflectional marking on the verb (\cite{taraldsen1980}, see, e.g., \cite{roberts2014} for an overview).
See \sectref{sec:prodrop} for a distinction of topic drop from \textit{pro}-drop and \sectref{sec:discourse.drop} for discourse (\textit{pro}-) drop languages, which allow for null subjects \is{Null subject} even though the verb is not marked for inflection at all.
}
%
So if a (1st person) singular subject is omitted, the inflectional marking on the verb should allow a hearer to reconstruct at least its grammatical person.
However, this reasoning applies only to the present tense and not to preterite present verbs.%
% Footnote
\footnote{According to \citet[1258]{zifonun.etal1997}, in German the preterite present verbs are \textit{wissen} (`to know') and the modal verbs except for \textit{wollen} (`to want'). \is{Modal verb}
They are characterized by the fact that their present forms were originally forms of the preterite, which has led to a syncretism \is{Syncretism} of the 1st and the 3rd person singular forms.
Such a syncretism \is{Syncretism} is also present in the remaining modal verb \textit{wollen} (`to want'), whose present forms originated from optative forms.
Therefore, when I speak of preterite present verbs in the following, I also always mean \textit{wollen}, which is a sensible simplification since we are concerned here exclusively with syncretism \is{Syncretism} as the central property of this verb group.
}
%
In the preterite and for preterite present verbs even in present tense, there is a syncretism \is{Syncretism} between the forms of the 1st and the 3rd person singular (e.g., \textit{ich war} -- `I was' vs. \textit{sie war} -- `she was' and \textit{ich kann} -- `I can' vs. \textit{er kann} -- `he can').
Consequently, not all 1st person singular verb forms are distinctly marked for inflection, so recoverability \is{Recoverability} through inflectional marking alone cannot explain the frequent omission of subject pronouns of this person.
Furthermore, \citet[279]{sigurdsson2011} notes that in Swedish, \il{Swedish} which does not mark verb forms for grammatical person at all, ambiguous \is{Ambiguity} utterances with topic drop and syncretic \is{Syncretism} verb forms are nevertheless interpreted as 1st person (singular) by default.
Therefore, \citeg{auer1993} \textit{inflectional hypothesis} falls short at least for this other Germanic language.
For German, I tested the \textit{inflectional hypothesis} in experiments \ref*{exp:top.s.fv} and \ref*{exp:top.s.mv}, where I compared topic drop before lexical verbs \is{Lexical verb} with a distinct inflectional marking to topic drop before syncretic \is{Syncretism} forms of modal verbs. \is{Modal verb}
I found that topic drop of the 1st person singular was preferred both before distinct and before syncretic \is{Syncretism} verb forms (see Sections \ref{sec:exp.top.s.fv.person} and \ref{sec:exp.top.s.mv.person}).

Besides the \textit{inflectional hypothesis}, the prevalence of 1st person (singular) topic drop is also explained by a group of hypotheses that I term \textit{extralinguistic hypotheses} because they postulate that the omitted constituent can be recovered \is{Recoverability} through some form of extralinguistic context or knowledge (see \sectref{sec:recover.extra} for the general possibility to have topic drop with extralinguistic referents).
Two groups can be further distinguished within the \textit{extralinguistic hypotheses}:
The first group argues that only the 1st person (singular) can be more easily omitted and recovered \is{Recoverability} (\textit{extralinguistic 1SG}).
The second group states that both the 1st and the 2nd person (singular) have this advantage (\textit{extralinguistic 1SG+2SG}).

\citet{imo2013, imo2014} is not only an advocate of Auer's \textit{inflectional hypothesis}, but he also proposes a further explanation for the prevalence of 1st person singular subject topic drop that belongs to the group of \textit{extralinguistic 1SG hypotheses}.
According to him, hearers can easily process utterances where a 1st person singular subject pronoun is omitted because ``the default `origo' of speaking, i.e. `I-here-now', can be activated in most cases so that the recipients can assume that the `missing' element is the unmarked `I'\,'' \citep[153--154]{imo2014}.
Thus, the recovery \is{Recoverability} of the omitted element also depends on a form of extralinguistic context, namely the hearer's knowledge about the deictic center of the utterance in question.
A further representative of the \textit{extralinguistic 1SG hypotheses} is \citet[272]{volodina2011}.
She points out that topic drop of the 1st person singular or plural can be used without anaphoric reference in certain text types. \is{Text type|(}
Similarly, \citet[4, footnote 1]{schalowski2015} states that 1st person pronouns can be omitted without a licensing context.
According to \citet[218]{volodina.onea2012}, this is because the reference of 1st person pronouns is usually evident through text type knowledge.
1st person subject topic drop is particularly frequent in certain conceptually spoken text types such as diaries and personal letters, which led some authors to treat it as a phenomenon in its own right, using terms such as \textit{diary drop} (e.g., \cite{haegeman1990} for \ili{English}, see \cite[272]{volodina2011}).

I discuss the second group of \textit{extralinguistic hypotheses} under the name \textit{extralinguistic} \textit{1SG+2SG}.
A central representative of these variants is the IDS grammar \citep[413]{zifonun.etal1997}.
They discuss 1st and 2nd person subject topic drop under their own name as \textit{Person-Ellipse} (`person ellipsis'), a subtype of situative ellipsis.
According to this concept, pronouns referring to the speaker and the hearer -- and in parallel to the author and the addressee in written discourse \citep[416]{zifonun.etal1997} -- can be omitted in a shared speech situation, where the roles of speaker and hearer are clearly determined \citep[414]{zifonun.etal1997}.
Reference to a group of speakers or hearers is also possible but rarely occurs \citep[415]{zifonun.etal1997}.
According to this \textit{extralinguistic hypothesis}, topic drop of the 1st and the 2nd person should be more frequent because both speaker and hearer are recoverable \is{Recoverability} from the speech situation and through knowledge about certain text types.

A similar argumentation can be found in \cite{ariel1990} (see \sectref{sec:recover.given} for her accessibility theory).
She claims that 1st and 2nd person pronouns ``correspond to assigned roles in conversations, while third-person pronouns refer to any person, excluding the above two'' \citep[47]{ariel1990}.%
%% Footnote
\footnote{\citet[128]{zdorenko2010} makes a similar statement, arguing that ``[f]irst- and second-person subjects [...] are always topics, i.e., they always have a discourse antecedent, \is{Antecedent} because the identities of the speaker and the hearer are taken as a background assumption in any conversation.''}
%
This difference has consequences not only for topic drop but for null pronouns in several languages where covert 1st and the 2nd person pronouns are more common than covert  3rd person pronouns.
\citet[48]{ariel1990} stresses that speaker and hearer are such salient \is{Salience|(} referents that the overt 1st and 2nd person pronouns are only used under special circumstances in languages with null pronouns, i.e., if they are less accessible in Ariel's terms.

A different variant of the \textit{extralinguistic 1SG+2SG hypotheses} is put forward by \citet[218]{volodina.onea2012}.
They state that 1st and 2nd person pronouns can be better omitted because they are not marked for grammatical gender,%
%% Footnote
\footnote{Note that there are languages that also or only mark the 1st and 2nd person pronouns for grammatical gender, such as Ngala (Ndu, Sepik; north-western Papua New Guinea), in which all three persons in the singular are marked for gender, and Korana (Central Khoisan; South Africa),  which distinguishes gender for all three persons and numbers \citep{wals-44}.}
%
which they believe reduces potential ambiguity. \is{Ambiguity}
While it may well be that ambiguity \is{Ambiguity} plays a role in the usage of topic drop (see also \sectref{sec:usage.ambiguity.theory}), it seems to me that the lack of gender marking is only a side effect of the uniqueness of speaker and hearer roles, which is the real factor that reduces ambiguity. \is{Ambiguity}
In a communication situation, there is usually one speaker at a time and one or several hearers to which the 1st and 2nd person pronouns unambiguously refer.
3rd person pronouns, in turn, can theoretically refer to any person, object, or proposition that is somehow present in the linguistic or extralinguistic context, provided that the grammatical gender matches in the singular cases.
Therefore, gender marking indeed reduces ambiguity \is{Ambiguity} in German but for the 3rd person singular.

In sum, two or rather three explanations for the prevalence of 1st person (singular) topic drop can be distinguished:
First, there is \citeg{auer1993} \textit{inflectional hypothesis} according to which the 1st person singular can be easily recovered  \is{Recoverability} because the following verb (often) has a distinct inflectional ending expressing the grammatical person of the omitted subject.
If one takes the hypothesis seriously, one would have to assume that, in principle, all subjects followed by a distinct verb form can be recovered \is{Recoverability} well and, thus, be omitted, i.e., the \textit{inflectional hypothesis} would then not only concern the 1st person singular.
Second, there are two variants of \textit{extralinguistic hypotheses}.
The first variant, which I termed \textit{extralinguistic 1SG}, argues that topic drop of the 1st person should be more felicitous because of the special status of the speaker in certain text types \is{Text type|)} and as the deictic center.
The second variant, \textit{extralinguistic 1SG+2SG}, explains an advantage of both 1st and 2nd person topic drop by the uniqueness of speaker and hearer roles.
The results of experiments \ref*{exp:1sg.2sg} and \ref*{exp:1sg.2sg.spoken} in Sections \ref{sec:exp.1sg.2sg} and \ref{sec:exp.1sg.2sg.spoken} suggest that there is no difference in acceptability between topic drop of the 1st and topic drop of the 2nd person singular, supporting the second type of \textit{extralinguistic hypotheses}.

\is{Null subject|(}
A more radical approach to explain the special status of 1st and 2nd person topic drop in German is taken by \citet{trutkowski2011,trutkowski2016}.
She assumes that in German, the omission of a prefield constituent can be caused by two distinct mechanisms and distinguishes topic drop, i.e., the omission of objects and 3rd person subjects, from ``out of the blue-drop'' (OBD), i.e., the omission of 1st and 2nd person subjects.
Consequently, she assumes two different ellipsis types depending on the grammatical person of the omitted constituent.
In the following, I reject this distinction for two reasons.
On the one hand, the justification behind it is problematic.
On the other hand, assuming two phenomena instead of one unnecessarily complicates the grammatical system.
According to Occam's razor, the simpler system is preferable if the more complicated one offers no advantages.

\largerpage
By OBD, Trutkowski refers to cases where ``[t]he gap is identified/licensed independently of a salient discourse antecedent'' \is{Antecedent} \citep[187]{trutkowski2016}.
According to her, OBD is limited to ``out of the blue-contexts'', i.e., real null contexts without salient \is{Salience|)} discourse antecedents, \is{Antecedent} which she assumes to exist in the form of book and song titles, headlines, and novel and conversation beginnings \citep[188]{trutkowski2016}.
She argues that OBD is distinct from \ili{English} diary drop (\cite{haegeman1990}, see \sectref{sec:english.french}) because the latter would require ``a (default) referent or an antecedent/addressee'' \is{Antecedent} \citep[192]{trutkowski2016}, while OBD does not.
In this context, she discusses, among others, the examples \ref{ex:obd.unclear.target}, to which I added the corresponding context in \ref{ex:obd.unclear.context} from the corpus,%
%
% Footnote
\footnote{Context added from corpus search results for ``Dann: Die lässt sich die Haare trotzdem so schneiden. Kennst sie doch.'', from the aggregated reference and newspaper corpus of the digital dictionary of the German language (Digitales Wörterbuch der deutschen Sprache),  \url{https://www.dwds.de/r/?q=Dann\%3A++Die+l\%C3\%A4sst+sich+die+Haare+trotzdem+so+schneiden.+Kennst+sie+doch.+&corpus=public&date-start=1465&date-end=2018&genre=Belletristik&genre=Wissenschaft&genre=Gebrauchsliteratur&genre=Zeitung&format=max&sort=date_desc&limit=10} (visited on 01/02/2025).}
%
\is{Corpus} and \ref{ex:obd.clear}.

\ex.\label{ex:obd.unclear}
\a.\label{ex:obd.unclear.context},,Du hättest ruhig diplomatischer sein können", sage ich zu meiner Schwester.
Sie schaut mich verwundert an und fragt, was ich denn Diplomatisches zu unserer Unterhaltung beigetragen hätte.
,,Hätte ich sagen sollen, tolle Frisur, steht dir bestimmt gut?'', fragt meine Schwester.\\
`\,`You could have been more diplomatic,' I say to my sister.
She looks at me in surprise and asks what I would have contributed diplomatically to our conversation.
`Should I have said, great hairstyle, looks good on you?' my sister asks.' [DWDS: Corpus, Berliner Zeitung, 01/22/2005]
\bg.\label{ex:obd.unclear.target}Dann: ,,Die lässt sich die Haare trotzdem so schneiden. $\Delta$ Kennst sie doch''\\ 
then she.\textsc{dem} gets \textsc{refl} the hair nevertheless in.this.way cut you.\textsc{2sg} know her \textsc{part}\\
`Then: `Anyway, she will get her hair cut like that. (You) know her.'\,' \citep[190]{trutkowski2016}

\exg.\label{ex:obd.clear}Bitte einmal abstimmen. $\Delta$ Würdet mir sehr mit eurer Meinung helfen!\\
please \textsc{part} vote you.\textsc{2pl} would me very with your.\textsc{2pl} opinion help \\
`Please vote. (You) would help me a lot with your opinion!' \citep[191]{trutkowski2016}

In Trutkowski's view, \ref{ex:obd.unclear} resembles the \ili{English} diary drop.
Therefore, she does not consider it to be OBD because ``the hearer (that is addressed by the speaker) is provided by the discourse situation'' \citep[191]{trutkowski2016}.
According to her, however, \ref{ex:obd.clear} is an undoubted case of OBD.
This distinction is questionable.
\ref{ex:obd.clear} is a post in a forum that is clearly addressed to the other users of this forum.
The addressee can just as well be traced back to the discourse situation as in example \ref{ex:obd.unclear}, even if this situation is digitally mediated.
It is also questionable whether the context in \ref{ex:obd.clear} is really an ``out of the blue-context'' or whether the imperative infinitive does in fact evoke the addressee as discourse antecedent. \is{Antecedent}

The concept of OBD seems to be disputable in general.
Even in the null contexts that \citet{trutkowski2016} proposes, such as the beginning of a song or a headline, it is clear to the hearers or readers that they are in some kind of communication situation involving the roles of speaker and hearer.
Therefore, they can easily recover \is{Recoverability} any omitted 1st or 2nd person subject pronouns.
Further evidence against OBD as a concept is given in the following.
I discuss a variant of the \textit{inflectional hypothesis} that \citet{trutkowski2016} proposes to explain the licensing of OBD:

\begin{quote}
(a) 1\textsuperscript{st}/2\textsuperscript{nd} person null subjects are well-formed \textit{out of the blue} because they can be identified/licensed by discrete verbal inflectional endings.\newline
(b) 3\textsuperscript{rd} person null subjects are identified and licensed in dependence of [sic!] an external antecedent \is{Antecedent} (cf. topic drop) -- as a consequence it does not matter whether 3\textsuperscript{rd} person verbal inflectional endings are syncretic with \is{Syncretism}1\textsuperscript{st}/2\textsuperscript{nd} person verbal inflectional endings. \citep[218]{trutkowski2016}
\end{quote}

\noindent
\citet{trutkowski2016} claims that syncretic \is{Syncretism} verb forms that are ambiguous \is{Ambiguity|(} between the 1st and the 3rd person do not pose a problem for interpreting null subjects.
If there is a suitable antecedent \is{Antecedent} that licenses 3rd person topic drop, the utterance will be interpreted as 3rd person topic drop.
If there is no such antecedent, \is{Antecedent} the null subject will be interpreted as OBD of the 1st person.
She backs up her claim with examples such as \ref{ex:cat.test1} with null subjects before syncretic \is{Syncretism} verb forms.

\exg.\label{ex:cat.test1}Hans und ich haben den Film schon gesehen.\\
Hans and I have the movie already seen\\
`Hans and I have already seen the movie.'
\ag.\label{ex:cat.test1.3SG}*$\Delta$ Will deshalb lieber zuhause bleiben.\\
\phantom{*}he want.\textsc{3sg}(/\textsc{1sg}) therefore rather at.home stay\\
`(He) therefore prefers to stay at home.'\hfill\mbox{(3rd person singular)}
\bg.\label{ex:cat.test1.1SG}$\Delta$ Will deshalb lieber zuhause bleiben.\\
I want.\textsc{1sg}(/\textsc{3sg}) therefore rather at.home stay\\
`(I) therefore prefer to stay at home.' \hfill (1st person singular) \citep[208, her judgments]{trutkowski2016}

According to her, the coordinated subject \textit{Hans und ich} in the first sentence cannot serve as the antecedent \is{Antecedent} for 3rd person topic drop in \ref{ex:cat.test1.3SG}, whereas OBD of the 1st person singular is possible in \ref{ex:cat.test1.1SG}.
The similar example \ref{ex:cat.test2} questions this intuition.%
%% Footnote
\footnote{I thank Robin Lemke for suggesting the example and for the helpful discussion.
}
%%

\exg.\label{ex:cat.test2}Hans und ich kommen heute später zum Training.\\
Hans and I come today later to.the training\\
`Hans and I will come later today for training.'
\ag.\label{ex:cat.test2.3SG}$\Delta$ Musste sich noch rasieren.\\
he must.\textsc{3sg}(/\textsc{1sg}).\textsc{pret} himself still shave\\
`(He) still had to shave.' (Lit. `(He) still had to shave himself.') \\\phantom{.} \hfill (3rd person singular)
\bg.*\label{ex:cat.test2.1SG}$\Delta$ Musste sich noch rasieren.\\
I must.\textsc{1sg}(/\textsc{3sg}).\textsc{pret} himself still shave\\
`(I) still had to shave.' (Lit. `(I) still had to shave himself.') \\\phantom{.}\hfill (1st person singular)

The difference is that despite the syncretic \is{Syncretism} verb form, the reflexive \textit{sich} \is{Reflexivity} makes it clear that the omitted subject has to be a 3rd person singular subject \ref{ex:cat.test2.3SG}.
An interpretation as 1st person singular is not possible \ref{ex:cat.test2.1SG}.
Therefore, the topic drop in \ref{ex:cat.test2.3SG} is possible and, importantly, \textit{Hans} in the coordinated subject of the first sentence can now serve as an antecedent, \is{Antecedent} whereas \ref{ex:cat.test2.1SG}, which \citet{trutkowski2016} refers to as OBD of the 1st person, is no longer an option.%
% Footnote
\footnote{Of course, if \textit{sich} is replaced by \textit{mich}, i.e., the 1st person singular reflexive, \is{Reflexivity} the pattern is inverse and only the reading with a covert 1st person singular subject is possible.}
%

\is{Verbal inflection|)}
This suggests that the contrast observed by Trutkowski in \ref{ex:cat.test1} is not categorical in nature.%% Footnote
\footnote{However, an experimental investigation of both \citeg{trutkowski2016} argument and my counterexample is still pending.
}
%%
Rather, there seems to be a preference for interpreting the covert constituent as a 1st person singular pronoun, which is based on probabilities and frequencies and which can change in an appropriate context, as example \ref{ex:cat.test2} shows.
This line of reasoning, therefore, raises again the question of whether it is legitimate to assume two different ellipsis types, i.e., to distinguish between OBD and topic drop.
I refrain from making such a distinction because it does not provide additional explanatory power but would complicate the grammatical system.
\is{Null subject|)}

The theoretical overview has shown that many authors agree in assuming that the 1st person can be omitted particularly well under topic drop but disagree on how to explain this assumption.
\citet{auer1993} proposes an \textit{inflectional hypothesis}, which was at least originally put forward to explain only a part of the data, i.e., only 1st person singular subject topic drop before distinctly marked verb forms.
\citet{zifonun.etal1997}, \citet{volodina2011}, \citet{volodina.onea2012}, \citet{imo2013, imo2014}, and \citet{schalowski2015} are advocates of what I termed \textit{extralinguistic hypotheses}, which come in two varieties.
Some authors assume that only the 1st person has an advantage for being the deictic center and the speaker/writer of certain text types, whereas others refer to the default roles of speaker and hearer in every communicative situation and assume that both the 1st and the 2nd person can be omitted particularly well.
The differentiation between 1st and 2nd person on the one hand and 3rd person on the other culminates in \citet{trutkowski2016} even assuming two different ellipsis types, topic drop and OBD, a view I rejected as unnecessary.\il{German|)}

\subsection{Theoretical overview of ambiguity avoidance }
\is{Ambiguity avoidance|(}
\label{sec:usage.ambiguity.theory}
Ambiguity avoidance could be a further factor that impacts the usage of topic drop.
It predicts that speakers avoid ambiguous linguistic structures to facilitate the processing for the hearer.\is{Processing effort}
Like \textit{UID}, ambiguity avoidance is, thus, based on the concept of audience design, \is{Audience design|(} i.e., the adaptation of linguistic structures by the speaker for the benefit of the hearer (see also \sectref{sec:info.theory.uid}).

A linguistic expression is ambiguous \is{Ambiguity} if it can be interpreted in more than one way, i.e., if it has more than one possible meaning.%
%% footnote
\footnote{\citet[238--239]{kennedy2019} delimits ambiguity \is{Ambiguity} from vagueness by stating that ambiguity is an uncertainty that results in variation with respect to truth conditions, whereas terms are vague if there remains ``uncertainty [...] about precisely what properties these terms ascribe to the objects to which they are applied''.
The expression \textit{funny} in \ref{ex:ambig} is ambiguous because it has two meanings.
Depending on which meaning is the intended one, the same utterance can be judged true or false in the exact same context.
For instance, \ref{ex:ambig} can be successfully uttered to say that Sterling's cousin is good at making people laugh but that she is not strange.
%\vspace{-0.5\baselineskip}
\ex.\label{ex:ambig} Sterling’s cousin is funny, but she is not funny. \citep[following][237]{kennedy2019}\par
%\vspace{-0.5\baselineskip}
This is different for \ref{ex:vague}, where the adjective \textit{tall} is vague. 
\textit{Tall} does not have two meanings but only one, something like ``having a height above the average'', therefore \ref{ex:vague} is not possible.
It is this fuzzy meaning that causes the vagueness because the average is not fixed.
%\vspace{-0.5\baselineskip}
\ex.\label{ex:vague}\#Sterling’s cousin is tall, but she is not tall.
\vspace{-0.75\baselineskip}
}
%
Sentences can be either globally (fully) ambiguous \is{Ambiguity} if it cannot be decided from the utterance alone which reading is the intended one, or locally (temporarily) ambiguous if the intended meaning becomes evident when the processing of the utterance is complete \citep{pritchett1988, ferreira2007}.
While there are several forms of ambiguity, \is{Ambiguity} e.g., lexical, structural, and scope ambiguity \citep{wasow2015},%
% Footnote
\footnote{For instance, the sentence \ref{ex:amb.lexical} is lexically ambiguous \is{Ambiguity} because the German word \textit{Bank} can refer to a financial institution (`bank') or a seat (`bench'), among other things.
The classic sentence \ref{ex:amb.struct} is structurally ambiguous because the PP \textit{with a telescope} can be attached either as an attribute to \textit{a man} or function as an instrumental adverbial specifying the action of seeing.
The sentence \ref{ex:amb.scope} exhibits scope ambiguity because it can either mean that every person in the room speaks any two languages or that two specific languages are spoken by all the people in the room. %\vspace{-0.5\baselineskip}
\exg.\label{ex:amb.lexical}Treffen wir uns bei der Bank!\\
meet we us at the bench/bank\\
`Let's meet at the bench/bank!'\par %\vspace{-1.5\baselineskip}
\ex.\label{ex:amb.struct} I saw a man with a telescope. \par %\vspace{-1.5\baselineskip}
\ex.\label{ex:amb.scope} Everyone in the room speaks two languages. \citep[35]{wasow2015} \par %\vspace{-1.5\baselineskip}
}
%
for topic drop, a special type of lexical ambiguity \is{Ambiguity} is important: the ambiguity of inflectional morphemes called syncretism \citep{wasow.etal2005}. \is{Syncretism|(}
It is common to distinguish systematic syncretisms from accidental syncretisms.
Systematic syncretisms can be captured by underspecification since they are simplifications of the feature space of a natural class.
This is not possible for the latter because the features involved are not part of a common natural class \citep{korth2017}.
\il{German|(}
I already discussed a type of systematic syncretism in \sectref{sec:usage.person.theory}.
In the present tense, the inflected forms of modal verbs \is{Modal verb} are identical for the 1st and the 3rd person singular (\textit{ich kann} (`I can') vs. \textit{sie kann} (`she can')), just like they are for the 1st and the 3rd person plural (\textit{wir können} (`we can') vs. \textit{sie können} (`they can')).
Following  \citet{korth2017}, the paradigm is underspecified as it lacks a feature expressing the opposition between the 1st and the 3rd person.
Consequently, an utterance with topic drop such as \ref{ex:amb.td.13} with a corresponding syncretic form is globally ambiguous between the 1st and the 3rd person singular (see also \cite[10]{trutkowski2016} for an overview of this syncretism listed in tabular form).

\exg.\label{ex:amb.td.13}$\Delta$ Kann heute arbeiten.\\
I/she/he/it can.\textsc{1sg}/\textsc{3sg} today work\\
`(I/She/He/It) can work today.'

\citet{poitou1993}, as well as \citet{zifonun.etal1997}, discuss another ambiguity \is{Ambiguity} arising from further syncretic forms.
They note that several forms of the 1st person singular present tense indicative are formally identical to the forms of the imperative \is{Imperative} singular \ref{ex:amb.td.imp}.
Since the indicative forms and the imperative \is{Imperative|(} do not form a natural class, this syncretism is best described as an accidental syncretism.

\exg.\label{ex:amb.td.imp}Arbeite heute nicht so viel.\\
work.\textsc{1sg.ind.prs}/\textsc{imp.sg} today not so much\\
`(I) will not work so much today.' / `Do not work so much today!'

\citet[415]{zifonun.etal1997} add that the two forms are also prosodically \is{Prosody} identical as both exhibit a falling boundary tone.
\citet{poitou1993} argues that this ambiguity \is{Ambiguity} is mostly negligible because topic drop would occur preferably before modal verbs, \is{Modal verb} which do not have an imperative, \is{Imperative} and auxiliaries, where the forms would not be syncretic. \is{Auxiliary} \is{Modal verb}
The latter point, however, is only correct for the auxiliary \is{Auxiliary} \textit{sein} (`to be') where the imperative \is{Imperative} \textit{sei} is distinct from the 1st person singular present tense indicative form \textit{bin}, but the forms of \textit{haben} (`to have') and \textit{werden} (`will') are identical: \textit{hab(e)} and \textit{werd(e)}.
More importantly, the corpus studies by \citet{androutsopoulos.schmidt2002} and \citet{frick2017}, discussed in \sectref{sec:usage.verb.type.studies}, show that although topic drop may be more frequent at least before copular \is{Copula} and modal verbs, \is{Modal verb} it is by no means rare before lexical verbs \is{Lexical verb} so that the ambiguity \is{Ambiguity} may well be relevant.

The syncretism between the imperative \is{Imperative} and the 1st person singular in present tense indicative occurs for verbs such as \textit{öffnen} (`to open'), for which both forms end in \textit{e} (schwa) \ref{ex:TDambiguous.full} \citep{imperativ}.
But also verbs such as \textit{anrufen} (`to call'), for which the imperative \is{Imperative} is usually formed without a final \textit{e}, in particular in spoken language, exhibit syncretic forms.
This is because the final \textit{e} of the 1st person singular is frequently omitted in colloquial speech, turning the actual standard form (\textit{ich}) \textit{rufe an} into the syncretic form (\textit{ich}) \textit{ruf an} \ref{ex:TDambiguous.e} \citep{imperativ}.
In Standard High German, the forms of verbs with an \textit{e}/\textit{i}-\textit{alternation} such as \textit{essen} (`to eat') are not syncretic \ref{ex:TDambiguous.not} \citep{imperativ}.
However, in certain dialects and occasionally also in colloquial speech, the present tense form is used as the imperative \is{Imperative} for these verbs as well, as the example \ref{ex:TDambiguos.FOLK} from the FOLK corpus  \is{Corpus} \citep{schmidt2014} shows.

\ex.\label{ex:TDambiguous}
\ag.\label{ex:TDambiguous.full}Öffne die Tür!\\
open.\textsc{1sg.ind.prs}/\textsc{imp.sg} the door!\\
`(I) open the door!' / `Open the door!'
\bg.\label{ex:TDambiguous.e}Ruf(e) gleich an!\\
call.\textsc{1sg.ind.prs}/\textsc{imp.sg} right.away \textsc{vpart}\\
`(I) will call right away!' / `Call right away!'

\ex.\label{ex:TDambiguous.not}
\ag.Iss den Kuchen!\\
eat.\textsc{imp.sg} the cake\\
`Eat the cake!' (Not: `(I) eat the cake!')
\bg.(Ich) esse den Kuchen.\\
I eat.\textsc{1sg.ind.prs} the cake\\
`(I) am eating the cake.' (Not: 'Eat the cake!')

\exg.\label{ex:TDambiguos.FOLK}ess mal dein zimt[hörnchen]\\
eat.\textsc{1sg.ind.prs}/eat.\textsc{imp.sg} \textsc{part} your cinnamon.croissant\\
`Eat your cinnamon croissant!' [FOLK\_E\_00309\_SE\_01\_T\_02]

\il{German|)}
In summary, in colloquial speech, there is potentially an ambiguity \is{Ambiguity} between topic drop of the 1st person singular in present tense and the singular imperative, \is{Imperative|)} which can arise in appropriate contexts, and if no other words in the utterance such as reflexive or object pronouns disambiguate.
\is{Syncretism|)}

The concept of ambiguity avoidance predicts that the existence of ambiguities \is{Ambiguity} such as those just discussed for topic drop constrains language usage.
Already Grice famously formulated ``Avoid ambiguity!'' as one of the submaxims of his maxim of manner \citep[46]{grice1975}.
Accordingly, one interlocutor should assume that if the other is cooperative, they will not use ambiguous linguistic structures, or if they do, it will be to trigger an implicature.
The concept of ambiguity avoidance is, thus, similar to \textit{UID} (see \sectref{sec:info.theory.uid}) in that it is based on the idea of audience design, i.e., that speakers adapt their utterances to the hearer.
Speakers should not use ambiguous structures in their language production for two reasons: \is{Production|(} \is{Audience design|)}
first, to avoid an increased processing effort \is{Processing effort|(} on the part of the hearers, who need to disambiguate the utterance, and, second, to avoid misunderstandings if the hearers disambiguate in an unintended way.
This line of reasoning has been extensively discussed for so-called \textit{garden path sentences} such as the famous \ref{ex:gardenpath} \citep[see, e.g.,][]{sanz.etal2013}.
Here, a temporarily ambiguous utterance can cause processing difficulties  but only when the hearer realizes at some point that the structure they initially considered to be the more likely one is not accurate.
That is, increased processing effort occurs not at the ambiguous but at the disambiguating point.

\ex.\label{ex:gardenpath}The horse raced past the barn fell. \citep[316]{bever1970}

Psycholinguistic research has found little evidence that speakers systematically avoid linguistic ambiguities, in particular, structural ones (see \cite{ferreira2008, wasow2015} for more detailed overviews).
For example, the experimental results of \citet{ferreira.dell2000} and the corpus \is{Corpus} results of \citet{rohdenburg2021} do not suggest that speakers use optional function words such as \textit{that} to disambiguate (temporarily) ambiguous clauses.
In experiments by \citet{arnold.etal2004}, speakers did not use constituent ordering to disambiguate local PP-attachment ambiguities.
\citet{snedeker.trueswell2003} found in their experiments that speakers avoided ambiguity by providing prosodic \is{Prosody} cues depending on the current situation.
However, the results by \citet{kraljic.brennan2005} and \citet{schafer.etal2005} show that while speakers also disambiguated prosodically \is{Prosody} in spontaneous speech, they did so regardless of whether the current situation demanded it or the hearer benefited from it, i.e., it could not be interpreted as a strategy of ambiguity avoidance.
In a study of lexical ambiguity by \citet{ferreira.etal2005}, speakers took linguistic ambiguity into account to some extent when describing objects presented in ambiguous displays, but they considered non-linguistic ambiguity more strongly and more consistently. 
This pattern was confirmed in an eye-tracking study by \citet{rabagliati.robertson2017}, who furthermore showed that speakers even proactively monitored for non-linguistic ambiguities, as well as self-monitored their produced utterances subsequently for how informative they were.%
% Footnote
\footnote{\citet{rabagliati.robertson2017} argue that for younger children, these two processes are still limited.}
%
Concerning the processing rather than the production of ambiguous structures, \citet{levy2008} points to the results of \citet{traxler.etal1998}, \citet{vangompel.etal2001}, and \citet{vangompel.etal2005}.
They found that several types of global attachment ambiguities did not result in slower but faster reading times, i.e., they were processed more easily.
This suggests that hearers do not necessarily experience difficulties processing \is{Processing effort|)} ambiguous structures (potentially even the opposite), thus, speakers may not be required to avoid them.
\citet[1152--1153]{levy2008} argues that this pattern follows naturally from the surprisal theory.
He assumes a ``fully parallel, incremental probabilistic parser \is{Parser} capable of online inference (that is, inference before input is complete)'' which assigns ``a probability distribution over the complete structures to which the already-seen input may possibly extend'' \citep[1132]{levy2008}.
In this framework, ambiguity is of relevance only for conditional word probabilities.
If at a given point in an incrementally parsed sentence, a local structural ambiguity at a word occurs so that the word would, up to this point, be consistent with several structural analyses of the complete sentence, the conditional probability of that word is fed by all these consistent structures.
This makes the word overall more probable and, therefore, easier to process.


While the research just discussed seems to call into question the role of ambiguity avoidance for speech production \is{Production|)} (and perception), it is debatable whether it can be generalized to syncretisms \is{Syncretism} relevant to topic drop given that primarily structural ambiguity was tested (see also the study by \cite{soares.etal2019} in \sectref{sec:usage.ambiguity.studies}).
For instance, \citet{volodina.onea2012} argue that ambiguity avoidance does play a role in how topic drop is used.
They state that the existing number of syncretisms \is{Syncretism} in the German verbal paradigm could lead to the situation that the hearer cannot reliably identify the referent of the omitted constituent.
Therefore, they postulate that topic drop is restricted to those cases where the omission does not lead to ambiguity \citep[214--215]{volodina.onea2012}.


Given the conflicting predictions of whether or not ambiguity avoidance affects speech production and, in particular, topic drop, it seems appropriate to test the impact of ambiguity avoidance empirically.
The first steps in this direction are considering verbal inflection as a factor in the corpus analysis in \sectref{sec:frac.td.part.regression.person} and comparing topic drop before unambiguously marked lexical verbs \is{Lexical verb} with topic drop before ambiguous verb forms in locally ambiguous utterances.%
%% Footnote
\footnote{\label{note:ambiguity}In another experiment, I also tested utterances that were globally ambiguous between a reading as an utterance with 1st person singular topic drop and a singular imperative \is{Imperative} reading in ambiguity-promoting and non-ambiguity-promoting contexts.
However, the results here were not conclusive.
Moreover, the experiment suffered from the fact that the baseline, which was actually intended to be non-ambiguous, was itself unintentionally a syncretic form.
For this reason, I do not report the results of this experiment here.
}
%
To this end, I conducted a post hoc analysis comparing data from experiment \ref*{exp:top.s.fv} (see \sectref{sec:exp.top.s.fv.person}) with distinctive verb forms and experiment \ref*{exp:top.s.mv} (see \sectref{sec:exp.top.s.mv.person}) with syncretic forms. \is{Ambiguity avoidance|)} \is{Ambiguity|)}

\section{Previous empirical evidence}
\subsection{Previous empirical evidence regarding grammatical person}\label{sec:usage.person.studies}
In the following, I discuss four corpus studies from the literature \citep{auer1993, androutsopoulos.schmidt2002,doring2002,frick2017} that evidence differences in the frequency of topic drop with respect to grammatical person.
According to them, the 1st person singular is indeed frequently omitted.
In particular, it is more often omitted than realized in text messages.
Topic drop of the 2nd person singular and the 1st person plural are also relatively frequent.
The 3rd person singular seems to be omitted particularly often when the prefield could be filled with the neuter demonstrative pronoun \textit{das} (`that'), which often refers to propositions.
The remaining plural persons are hardly ever targeted by topic drop but are also rare in the prefield of the corresponding full forms.

\subsubsection{\citet{auer1993}}
\is{Corpus|(}
\citet{auer1993} presents data from a corpus of spoken conversations that he does not specify further in terms of size, source, or date of creation.
He states that he extracted a bit more than 100 instances of topic drop and V1 declaratives \citep[195]{auer1993}. \is{V1 declarative}
Topic drop accounts for about 75\% of the cases \citep[198]{auer1993}.
The 3rd person subject and object pronoun \textit{das} (`that'), which refers to propositions, groups of propositions, verbs, or predicatives, is omitted most often \citep[200]{auer1993}.
Auer adds that the 1st person singular subject pronoun \textit{ich} (`I') is frequently omitted \citep[198]{auer1993}, as well as the 2nd person singular, whereas the 1st and 2nd person plural are rarely unrealized \citep[199]{auer1993}.
He does not provide information about the frequency of omitted 3rd person singular pronouns other than \textit{das}, nor about 3rd person plural pronouns but only presents an example of the latter \citep[199, example 10]{auer1993}.
Since he does not report absolute or relative numbers, frequency tables, or a statistical analysis, his results can only be interpreted as anecdotal.

\subsubsection{\citet{androutsopoulos.schmidt2002}}
In one of the earliest empirical studies of text messages in German, \citet{androutsopoulos.schmidt2002} analyzed a corpus of 934 text messages.
The messages had been produced during a period of 8 weeks in the year 2000 by a small group of mainly 5 people in their late twenties \citep[55--56]{androutsopoulos.schmidt2002}.
In their corpus, \citet[68--69]{androutsopoulos.schmidt2002} found 229 instances of subject topic drop and compared them to 197 complete utterances where the subject pronoun in the prefield could have been omitted.
They did not look at object topic drop.%
%% Footnote
\footnote{They also state that they excluded ``elliptische Formeln'' (`elliptical formulae', my translation) such as \textit{geht so} (`Not too bad', e.g., as a response to the question \textit{How are you?}, lit. `(it) goes so') without further discussing what they consider to be a formula and how many of them were excluded \citep[68, footnote 24]{androutsopoulos.schmidt2002}.}
%
Table \ref{tab:androutsopoulos.schmidt} shows the omission rates by pronoun that they determined.
For the ambiguous pronoun \textit{sie}, which can either be the 3rd person singular feminine (`she') or the 3rd person plural (`they'), they do not provide a subdivision.

\begin{table}
\caption[Omission rates by pronoun from \citet{androutsopoulos.schmidt2002}]{Omission rates by pronoun in the text message corpus of \citet{androutsopoulos.schmidt2002}, taken from \citet[69]{androutsopoulos.schmidt2002}, adapted}
\centering
\begin{tabular}{llrrrr}
\lsptoprule
 & \multicolumn{1}{c}{Person, num-} & &  &  & \multicolumn{1}{c}{Omission} \\
\multicolumn{1}{c}{\multirow{-2}{*}{Pronoun}} & \multicolumn{1}{c}{ber, gender} & \multicolumn{1}{c}{\multirow{-2}{*}{\Centerstack[c]{Full\\form}}} & \multicolumn{1}{c}{\multirow{-2}{*}{\Centerstack[c]{Topic\\drop}}} & \multicolumn{1}{c}{\multirow{-2}{*}{Total}} & \multicolumn{1}{c}{rate} \\
\midrule
\textit{ich} (`I') & 1SG & $124$ & $187$ & $311$ & $60.13\%$\\
\textit{du} (`you') & 2SG & $20$ & $7$ & $27$ & $25.93\%$\\
\textit{er} (`he') & 3SG \textsc{m} & $1$ & $3$ & $4$ & $75.00\%$\\
\textit{sie} (`she'/`they') & 3SG \textsc{f}/3PL & $4$ & $0$ & $4$ & $0.00\%$\\
\textit{es/das} (`it'/`that') & 3SG \textsc{n} & $19$ & $20$ & $39$ & $51.28\%$\\
\textit{wir} (`we') & 1PL & $28$ & $12$ & $40$ & $30.00\%$\\
\textit{ihr} (`you') & 2PL & $1$ & $0$ & $1$ & $0.00\%$\\
\lspbottomrule
\end{tabular}
\label{tab:androutsopoulos.schmidt}
\end{table}

\noindent
The 1st person singular subject pronoun is omitted most often in absolute terms,%
% Footnote
\footnote{Actually, \textit{er} (`he') has the highest omission rate of 75\%, but this rate is unreliable because it is based on only four occurrences of \textit{er} in the entire corpus.
The same is true for the omission rates of \textit{sie} (`she') and \textit{ihr} (`you.\textsc{pl}'), which are also based on only four or even only one data point, respectively.} %
%
followed by \textit{es} (`it') and \textit{das} (`that'), i.e., 3rd person singular neuter pronouns. 
The difference between \textit{ich} and \textit{es}/\textit{das} is not significant according to a Pearson's chi-squared test with Yates's continuity correction, which I calculated in R \citep{rcoreteam2021} ($\chi^2(1) = 0.79, p > 0.37$).%
% Footnote
\footnote{The other two tests reported in the following were calculated in the same way.}
%
The 1st person plural \textit{wir} (`we') is omitted only half as often as the 1st person singular (significant, $\chi^2(1) = 11.91, p < 0.001$) but only slightly more often than the 2nd person singular \textit{du} (`you') (not significant, $\chi^2(1) = 0.007, p > 0.9$).%
% Footnote
\footnote{Note, however, that there are relatively few instances of \textit{wir} in the corpus and even fewer of \textit{du}.
Thus, the result of their comparison may not be reliable, i.e., for a larger data set there might be a significant difference between their omission rates.}
These quantitative results are similar to the tentative results of \citet{auer1993}:
\textit{Das} (\textit{es}) and \textit{ich} are most often omitted. 

\subsubsection{\citet{doring2002}} A further text message corpus of comparable size was analyzed by \citet{doring2002}.
The corpus consists of 1\,000  text messages with about 13\,000 tokens and was created in April and September 2001 \citep[102]{doring2002}.
Döring investigated several types of what she terms syntactic reduction \citep[105]{doring2002}, among which is the omission of subject pronouns, i.e., topic drop of subjects.
She does not provide absolute or relative numbers, but her results are in line with the results by \citet{androutsopoulos.schmidt2002} according to which the subject pronoun \textit{ich} (`I') is most often omitted \citep[107]{doring2002}.
By way of qualification, she notes that \textit{ich} (`I') and \textit{du} (`you') were at the same time the most frequent words in the text message corpus.
However, this information is of limited value because she does not consider the syntactic position of the pronouns.
Consequently, it remains unclear how many of them occur in the prefield and could theoretically be targeted by topic drop.

\subsubsection{\citet{frick2017}}
In her study of Swiss German text messages (see \sectref{sec:usage.function.studies} for details), \citet{frick2017} also assessed the grammatical person of her 2\,326 instances of subject topic drop and the 2\,059 realized subject pronouns in the prefield.%
%% Footnote
\footnote{Given that the number of instances is about ten times higher in \citeg{frick2017} corpus than in the corpus of \citet{androutsopoulos.schmidt2002}, Frick's results should be more reliable.
While her results are based on Swiss German data, the results of the other corpus studies discussed here and of my corpus study presented in \sectref{sec:factor.person.corpus} on German Standard German show a similar basic tendency.
}
%%
The distribution is shown in Table \ref{tab:frick.person}.

\begin{table}
\caption[Omission rates by pronoun from \citet{frick2017}]{Omission rates by grammatical person in the text message corpus of \citet{frick2017}, taken from \citet[88]{frick2017}}
\centering
\begin{tabular}{lrrrr}
\lsptoprule
Grammatical person & Full form & Topic drop & Total & Omission rate \\
\midrule
1SG & $1\,314$ & $1\,916$ & $3\,230$ & $59.32\%$\\
2SG & $195$ & $173$ & $368$ & $47.01\%$\\
3SG & $229$ & $158$ & $387$ & $40.83\%$\\
1PL & $279$ & $70$ & $349$ & $20.06\%$\\
2PL & $25$ & $4$ & $29$ & $13.79\%$\\
3PL & $17$ & $5$ & $22$ & $22.73\%$\\
\lspbottomrule
\end{tabular}
\label{tab:frick.person}
\end{table}

\noindent
Again, the omission rate of the 1st person singular subject pronoun is the highest.
More specifically, it is significantly higher than the omission rates of the other grammatical persons, as \citet[90, footnote 102]{frick2017} reports.
\textit{Ich} is omitted more often than it is realized.
Consequently, \citet[89]{frick2017} argues that in (Swiss German) text messages, topic drop of the 1st person singular has become the unmarked default case.
The 2nd person singular exhibits the second highest omission rate with about 47.01\% (according to \cite[90, footnote 102]{frick2017}, it is significantly higher than those of the remaining persons).
It is slightly more often realized in the prefield than omitted.
3rd person singular pronouns are omitted in 40.83\% of the cases and, thus, still significantly more frequently than all the plural persons, which have omission rates between about 14\% and 23\% \citep[90, footnote 102]{frick2017}.

At first glance, it is surprising that the omission rate of the 3rd person singular is not much lower than that of the 2nd person singular.
\citeg{androutsopoulos.schmidt2002} results suggest that the omitted 3rd person singular pronouns are rather \textit{es} (`it') and \textit{das} (`that'), which often refer to propositions, instead of \textit{sie} (`she') or \textit{er} (`he'), which refer to persons and objects.
While there are many persons and objects in the world, to which such a pronoun may refer, there is often only one very salient \is{Salience} proposition in discourse, namely the last one mentioned.
This proposition is therefore highly predictable \is{Predictability} and the pronoun referring to it can be easily omitted.%
%% Footnote
\footnote{It is not clear from \citeg{frick2017} statements whether in her corpus data, the omitted 3rd person singular subjects also most frequently refer to propositions.
While in her examples the covert 3rd person singular constituents are coreferential with concrete NPs in the previous discourse \citep[106]{frick2017}, she adds that the reference can also be a more abstract entity \citep[107, footnote 120]{frick2017}.}
%
This hypothesis is supported by the results of \citeg{helmer2016} corpus study, where she compared the type of referent for utterances with topic drop and utterances with the anaphoric subject and object pronoun \textit{das} in the prefield (see \sectref{sec:usage.verb.type.studies} for details on this reference data).
\citet[210--211]{helmer2016} states that in her data of spoken dialogues, most instances of topic drop (41.6\%) refer to a proposition, followed by 20.2\% that refer to a noun phrase, and 20\% that constitute cases of indirect topic drop (see \sectref{sec:recover.ling}).
The reference of the full forms with \textit{das} is mostly a noun phrase (33.5\%), followed by cases with indirect reference (26.5\%), and the reference to propositions in the third place (23.5\%).
\citet{helmer2016} states that topic drop, thus, refers significantly more often to propositions than the anaphor \textit{das}.
She adds that it occurs particularly often in evaluations, consents, objections, and responses that refer to propositions (this is similar to the socio-pragmatic functions of topic drop discussed in \sectref{sec:pusage.effects}).
\is{Corpus|)}

\subsection{Previous empirical evidence regarding ambiguity avoidance}
\is{Ambiguity avoidance|(}\label{sec:usage.ambiguity.studies} \il{Portuguese|(}\is{Null subject|(}
While there is, to the best of my knowledge, no study of topic drop and ambiguity avoidance in German, \citet{soares.etal2019} looked at the interplay of 1st person null subjects and syncretic \is{Syncretism|(} verb forms in Brazilian Portuguese.
Considering ambiguity avoidance in a different language and with a different form of null subjects takes me further away from topic drop in German than the studies mentioned in the context of the other factors do.
It still seems reasonable to me because the principle of ambiguity avoidance should be at work across languages if it plays any role at all.
This means that the conclusions drawn here should be transferable.

\citet{soares.etal2019} conducted a corpus \is{Corpus|(} study and two acceptability rating experiments.
For the corpus study, they used a corpus that was previously analyzed by \citet{duarte1995} and that contains 18 oral interviews with a total of 8\,032 of what they term ``inflected clauses'' \citep[3585]{soares.etal2019}.
\citet[3586--3587]{soares.etal2019} found that for the 1st person singular, the subject is more often realized overtly (80\% overt subjects) before a syncretic verb form than before a verb form that is distinctly marked (70\% overt subjects). \is{Corpus|)}

\is{Verbal inflection|(}
\is{Acceptability rating study|(}To verify this result in a controlled setting, they conducted two acceptability rating experiments in the form of 2 $\times$ 2 designs crossing \textsc{Inflection} (syncretic vs. distinct)%
% Footnote
\footnote{They kept the verb and the grammatical person constant and achieved variation in inflection by using verb forms in different tenses that are either syncretic or distinctive.}
%
 and \textsc{Subject} (null vs. overt).
The items consisted of a context sentence, a question, and a target sentence with the 1st person singular as the overt subject of the main clause and as the overt vs. covert subject of a subordinate clause, as shown in example \ref{ex:soares.item}.
In experiment 1, the target sentence contained an animate \is{Animacy|(} 3rd person singular object,\ref{ex:soares.item.1} and \ref{ex:soares.item.2}, whereas in experiment 2 this object was inanimate, \ref{ex:soares.item.3} and \ref{ex:soares.item.4}.

\ex.\label{ex:soares.item}
A: \textit{Maria estava muito nervosa. Você sabe quando ela ficou mais calma?}\\
A: `Mary was very nervous. Do you know when she’s got calmer?'
\ag.\label{ex:soares.item.1}B: Eu tranquilizei a Maria quando (eu) divulguei os resultados do exame.\\
{} I calm.down.\textsc{1sg.pst}  the Maria when I publish.\textsc{1sg.pst} the results of.the exam\\
B: `I calmed Mary down when I published the results of the exam.' \citep[3588]{soares.etal2019}
\bg.\label{ex:soares.item.2}B: Eu tranquilizei a Maria quando (eu) ia divulgar os resultados do exame.\\
{} I calm.down.\textsc{1sg.pst} the Maria when I was.going.to publish.\textsc{inf} the results of.the exam\\
B: `I calmed Mary down when I was going to publish the results of the exam.' \citep[3588]{soares.etal2019}
\cg.\label{ex:soares.item.3}B: Eu resolvi o problema quando (eu) divulguei os resultados do exame.\\
{} I solve.\textsc{1sg.pst} the problem when I publish.\textsc{1sg.pst} the results of.the exam\\
B: `I solved the problem when I published the results of the exam.' \citep[3592]{soares.etal2019}
\dg.\label{ex:soares.item.4}B: Eu resolvi o problema quando (eu) ia divulgar os resultados do exame.\\
{} I solve.\textsc{1sg.pst} the problem when I was.going.to publish.\textsc{inf} the results of.the exam\\
B: `I solved the problem when I was going to publish the results of the exam.' \citep[3592]{soares.etal2019}

In experiment 1, they found that utterances with null subjects were rated significantly worse before syncretic verb forms than before distinct ones compared to the overt subjects \citep[3591]{soares.etal2019}.
In experiment 2, no such interaction was present, but null subjects were generally preferred over overt subjects \citep[3593]{soares.etal2019}.
\citet[3593]{soares.etal2019} interpret these results as evidence of ambiguity avoidance, which, however, is only relevant if there is a syncretic verb form \textit{and} a ``competition between potential antecedents \is{Antecedent} for the null subject in contexts where the verb is ambiguous between first and third persons'' \citep[3593]{soares.etal2019}.
Such a competition was present in experiment 1, where the animate \is{Animacy|)} object represented a competing referent for the null subject but not in experiment 2, where the object was inanimate.
In sum, these results suggest that ambiguity avoidance does not generally play a role in null subjects and syncretic verb forms but only if there is competition between multiple potential antecedents. \is{Antecedent}\is{Verbal inflection|)}

In fact, there was such competition in my experiments \ref*{exp:top.s.fv} and \ref*{exp:top.s.mv}, which I used indirectly to study ambiguity avoidance.
The items also contained two potential referents for the covert constituent, like in \citeg{soares.etal2019} experiment 1.
A major difference to \citeg{soares.etal2019} study was, however, that the target utterances in experiment \ref*{exp:top.s.mv} with syncretic \is{Syncretism|)} verb forms were not globally but only locally ambiguous.
It is unclear whether speakers also avoid this local ambiguity.
I discuss this point again in \sectref{sec:exp.top.s.mv.person.diss}. \is{Acceptability rating study|)} \is{Ambiguity avoidance|)} \il{Portuguese|)}\is{Null subject|)}

\largerpage
\section[head={Information-theoretic predictions}]{Information-theoretic predictions for grammatical person and verbal inflection}\label{sec:info.theory.person}
In the following, I outline the information-theoretic predictions for grammatical person and distinct inflectional ending and argue that their impact can be captured by the \textit{avoid troughs} and the \textit{facilitate recovery} principles. \is{Recoverability}
On the one hand, those grammatical persons and numbers that are highly predictable \is{Predictability} in context should be better omitted.
On the other hand, a distinct inflectional ending \is{Verbal inflection} on the verb following subject topic drop should facilitate recovering \is{Recoverability|(} the omitted subject as it provides information about the person and number of that subject.

The overview of previous corpus studies that investigated grammatical person in \sectref{sec:usage.person.studies} has shown that topic drop of the 1st person singular subject pronoun \textit{ich} (`I') is particularly frequent.
Also the 2nd person singular subject pronoun \textit{du} (`you') and the 3rd person singular neuter subject pronouns \textit{das} (`that') and \textit{es} (`it') still exhibit relatively high omission rates.
It is reasonable to also infer for certain grammatical persons how well they can be omitted from their general frequency in the prefield, similar to the argument for syntactic function in \sectref{sec:info.theory.function}.
However, such a strategy falls short.
For instance, the data from the DeReKo study,  \is{Corpus|(} discussed in \sectref{sec:info.theory.function}, partly repeated here in Table \ref{tab:pronoun.freq.dereko.rep}, show that \textit{du} occurs strikingly less frequently in the corpus than the 1st person and the 3rd person singular masculine subject pronouns, both generally and in the prefield.
However, the existing studies of topic drop suggest that omitting \textit{du} is more frequent than omitting \textit{er}.

\begin{table}
\caption{Frequency of the 1st and 2nd person singular, as well as of the 3rd person singular masculine in the DeReKo TAGGED-T archive (repeated from page \pageref{tab:pronoun.freq.dereko})}
\centering
\begin{tabular}{lcccrrr}
\lsptoprule
%& \multicolumn{1}{c|}{Person,} &   &  & \multicolumn{1}{c|}{Total} & \multicolumn{1}{c|}{occurrences} & \multicolumn{1}{c}{Proportion} \\
%\rowcolor{gray!50}
%\multirow{-2}{*}{Pronoun} & \multicolumn{1}{c|}{number} & \multirow{-2}{*}{Gender}  & \multirow{-2}{*}{Case} & \multicolumn{1}{c|}{occurrences} & \multicolumn{1}{c|}{in prefield} & \multicolumn{1}{c}{in prefield} \\
Pronoun & \Centerstack{Person,\\number} & Gender & Case & \Centerstack{Total oc-\\currences} & \Centerstack{Occur-\\rences in\\prefield} & \Centerstack{Propor-\\tion in\\prefield}\\
\midrule
\textit{ich} & 1SG & -- & \textsc{nom} & $2\,068\,332$ & $570\,809$ & $28.60\%$ \\
\textit{du} & 2SG & -- & \textsc{nom} & $130\,166$ & $10\,230$ & $7.86\%$ \\
\textit{er} & 3SG & \textsc{m} & \textsc{nom} & $3\,808\,561$ & $721\,514$ & $18.94\%$  \\
\lspbottomrule
\end{tabular}
\label{tab:pronoun.freq.dereko.rep}
\end{table}

\largerpage
\noindent
While one could attribute the difference between frequency and omission rate to the different text types \is{Text type|(} used in the studies, news articles in the DeReKo study and text messages in the studies by \citet{androutsopoulos.schmidt2002} and \citet{frick2017}, this does not explain the whole pattern.
Also in \citeg{frick2017} text message data, the subject pronoun of the 1st person plural occurs more often in the prefield than that of the 2nd person singular but is nevertheless omitted less than half as often.
Thus, it stands to reason that there are further factors that influence the predictability \is{Predictability} of the prefield constituent and how well it can be omitted than its frequency and the occurrence in certain text types. \is{Text type|)}

For the 1st and the 2nd person, such factors are considered in what I termed \textit{extralinguistic 1SG+2SG hypotheses}.
They assume that the recovery of an omitted prefield 1st or 2nd person subject pronoun is possible through some form of extralinguistic context, e.g., knowledge about the speaker and hearer being uniquely determinable and integral parts of any speech situation.
Given that the speaker and the hearer share this knowledge, the speaker knows that pronouns referring to themselves or the hearer are highly predictable \is{Predictability} to both of them.
So if the speaker performs audience design, \is{Audience design} as \textit{UID} \is{Uniform information density} assumes, they should be more likely to omit a 1st or 2nd person pronoun referring to the speaker(s) or hearer(s) when using topic drop than a 3rd person pronoun referring to a person.
The case of the 3rd person singular subject and object pronouns \textit{das} (`that') and \textit{es} (`it') may be special because they frequently refer to propositions that are highly salient \is{Salience|(} in the current discourse, more specifically, to the last proposition expressed by the previous utterance.
Consequently, this proposition and the pronouns referring to it are very predictable \is{Predictability} so the latter should be omitted by the speaker to avoid surprisal minima.
The results of \citeg{helmer2016} corpus study mentioned in \sectref{sec:usage.person.studies} and of my corpus study discussed in \sectref{sec:factor.person.corpus} show that covert 3rd person singular constituents do indeed frequently refer to propositions or VPs \is{Verb phrase} but also that not all cases can be explained in this way.
About 40\% of the instances of topic drop refer to propositions in \citeg{helmer2016} data.
In my data, between 40\% and 50\% of the subjects have a propositional reference, for the objects the figure is higher, at almost 70\%.
\is{Corpus|)}

\is{Verbal inflection|(}
An information-theoretic explanation of the potential impact of grammatical person does not only capture the rationale behind the \textit{extralinguistic hypotheses}, but it can also account for \citeg{auer1993} \textit{inflectional hypothesis} and a potential role of verbal inflection in general.
In \sectref{sec:resolving}, I discussed how processing an utterance with topic drop requires additional effort \is{Processing effort} on the verb because the hearer needs not only to process the verb but also to resolve the ellipsis.
If this effort \is{Processing effort} is too high, i.e., if there is an information peak exceeding the channel capacity, \is{Channel capacity} the processing capacities of the hearer are overburdened resulting in processing difficulties.
I argued that the effort \is{Processing effort} associated with recovering \is{Recoverability|)} an omitted subject could be reduced if the following congruent verb is distinctly marked for inflection.
If it is marked for the 1st or the 2nd person, the referent of topic drop becomes evident because the roles of the speaker(s) and hearer(s) are uniquely defined in the utterance situation.
In contrast, the set of possible referents is only narrowed for the 3rd person.
This difference can explain why topic drop of the 1st person (and of the 2nd person, see experiments \ref*{exp:1sg.2sg} and \ref*{exp:1sg.2sg.spoken} in Sections \ref{sec:exp.1sg.2sg} and \ref{sec:exp.1sg.2sg.spoken}) is more felicitous than topic drop of a 3rd person pronoun that does not refer to the most salient \is{Salience|)} proposition in the current discourse.
\is{Verbal inflection|)}

\largerpage
\section{Corpus study of grammatical person and verbal inflection }
\is{Corpus|(}\label{sec:factor.person.corpus}
As mentioned above, I investigated the influence of the grammatical person on topic drop with a corpus study and a series of experiments.
In this section, I present the corpus results, more specifically a descriptive overview of grammatical person in the \textsc{FraC-TD-Comp} and the \textsc{FraC-TD-SMS} data sets, which were derived from the fragment corpus FraC (see \sectref{sec:corpus.frac}) and the results of the logistic regression analysis that I performed on the \textsc{FraC-TD-SMS-Part} data set.%
% Footnote
\footnote{The corpus data and the analysis scripts can be found online: \url{https://osf.io/zh7tr}.
For copyright reasons, I can only provide the IDs and the annotations of each instance but not the actual linguistic material.}
%
In the regression analysis, I furthermore considered information about the verb following topic drop including its inflectional ending, and interpreted the results also with respect to ambiguity avoidance.
I investigated whether topic drop of the 1st person singular subject pronoun \textit{ich} is indeed particularly frequent and whether such a preference could then be explained by the \textit{inflectional hypothesis}, i.e., whether the rate of topic drop increases if the verb in the left bracket has a distinct inflectional marking.

\subsection{Grammatical person in \textsc{FraC-TD-Comp}}\label{sec:frac.td.comp.person}
Table \ref{tab:FraC.Comp.person.subj} shows the grammatical person, number, and gender of the subjects in the prefield in the \textsc{FraC-TD-Comp} data  set, which consists of all instances of topic drop and the corresponding full forms in the FraC (see \sectref{sec:corpus.comp}).%
%% Footnote
\footnote{I follow \citet[43]{zifonun1995} in considering the anaphor \textit{es} to be of neuter gender.
This means that all occurrences of the expletive \is{Expletive|(} subject \textit{es} in this table and the following tables in this chapter are contained within the total number of 3rd person singular instances.
With \citet[68]{eisenberg2002}, however, it should be noted that already in the case of the pronominal \textit{es} the neuter gender is unmarked compared to the masculine and the feminine.
It can even be used when the antecedent \is{Antecedent} is, for example, a sentence and, thus, cannot be a carrier of gender congruence \citep[68]{eisenberg2002}.
For expletive \is{Expletive} \textit{es} occurrences, this reasoning should apply even more strongly since there is no antecedent and no reference at all. \is{Antecedent}
The category indeterminate consists of overt and covert indefinite pronouns \textit{man} (`one') and \textit{alle} (`everyone') and of omitted proper names referring to products or brands whose gender is uncertain.}
%
The data suggest a correlation between the frequency of the full forms with a certain grammatical person and number combination and the corresponding instances of topic drop,%
%% Footnote
\footnote{I correlated the number of full forms and the number of instances of topic drop for each combination of grammatical person and number.
For the 3rd person singular, I only included the subgroups obtained by considering gender and not the summed total.
I obtained a Pearson's $r$ of $0.93$ ($p < 0.001$), which hints at a strong positive correlation.
This value needs to be handled with care, though, since there are not many data points and in particular the high frequency of the 1st person singular seems to be the driving force for the observed effect.}
%
which I argue hints at a causal relationship.
The information-theoretic account that I advocate in this book predicts that the likelihood of omitting a certain grammatical person increases with its frequency in the prefield.

\begin{table}
\caption{Full forms, instances of topic drop, and omission rates as a function of the grammatical person, number, and gender for the subjects in the \textsc{FraC-TD-Comp} data set}
\centering
\begin{tabular}{lrrrr}
\lsptoprule
\Centerstack[c]{Grammatical person,\\number} & \multicolumn{1}{c}{\multirow{-1}{*}{Full form}}  & \multicolumn{1}{c}{\multirow{-1}{*}{Topic drop}} & \multicolumn{1}{c}{\multirow{-1}{*}{Total}} & \multicolumn{1}{c}{\multirow{-1}{*}{Omission rate}} \\
\midrule
1SG & $1\,077$ & $465$ & $1\,542$ & $30.16\%$\\
2SG & $83$ & $11$ & $94$ & $11.70\%$\\
3SG & $1\,186$ & $278$ & $1\,464$ & $18.99\%$\\
-- 3SG masculine & $137$ & $21$ & $158$ & $13.29\%$ \\
-- 3SG feminine & $131$ & $23$ & $154$ & $14.94\%$\\
-- 3SG neuter & $825$ & $206$ & $1\,031$ & $19.98\%$\\
-- indeterminate & $93$ & $28$ & $121$ & $23.14\%$\\
1PL & $387$ & $44$ & $431$ & $10.21\%$ \\
2PL & $26$ & $0$ & $26$ &  $0.00\%$\\
3PL & $292$ & $9$ & $301$ & $2.99\%$\\
\lspbottomrule
\end{tabular}
\label{tab:FraC.Comp.person.subj}
\end{table}

\noindent
Across all text types, 1st person singular subjects occur very frequently in the prefield both overtly and covertly, as well as 3rd person singular subjects, in particular those with neuter gender.
The fact that most instances of subject topic drop are omissions of the 1st person singular is in line with the statements in the literature and with the results of the previous corpus studies, discussed in \sectref{sec:usage.person.studies}.
Also, the observation that the number of utterances with topic drop of subject pronouns of the 3rd person singular neuter is about ten times higher than the number of utterances where a 3rd person singular masculine or feminine subject is omitted is in accord with the previous results.
Of these 200 cases, 28 are non-referential expletives, \is{Expletive} about 100 are cases where the antecedent \is{Antecedent} of topic drop is a proposition or a VP \is{Verb phrase|(} that is salient \is{Salience} in the current discourse, and in the remaining cases, the antecedent \is{Antecedent} is mainly a DP denoting concrete or abstract things.
This suggests that the explanation discussed in \sectref{sec:info.theory.person} that topic drop of the 3rd person singular neuter is so frequent because it targets mainly pronouns referring to salient \is{Salience} propositions is only tentatively correct.
Pronouns referring to physical or abstract entities are also frequently omitted.%
%% Footnote
\footnote{Since in German neuter DPs are often, though not always, inanimate, perhaps the animacy \is{Animacy} could play a role here too (similar to object topic drop in Dutch, \il{Dutch} see \sectref{sec:td.germanic}).}
%
A look at the full forms suggests that there is a similar difference in frequency per grammatical gender for the utterances with overt 3rd person singular pronouns as well, suggesting that the high proportion of the neuter gender is not a peculiarity of topic drop.
I return to this issue below.  
  
\begin{table}[t]
\caption{Full forms, instances of topic drop, and omission rates as a function of the grammatical person, number, and gender for the objects in the \textsc{FraC-TD-Comp} data set}
\centering
\begin{tabular}{lrrrr}
\lsptoprule
\Centerstack[c]{Grammatical person,\\number} & \multicolumn{1}{c}{\multirow{-1}{*}{Full form}}  & \multicolumn{1}{c}{\multirow{-1}{*}{Topic drop}} & \multicolumn{1}{c}{\multirow{-1}{*}{Total}} & \multirow{-1}{*}{Omission rate} \\
\midrule
1SG & $0$ & $0$ & $0$ & $0.00\%$ \\
2SG & $0$ & $0$ & $0$ & $0.00\%$ \\
3SG & $150$ & $61$ & $211$  & $28.91\%$\\
-- 3SG masculine & $7$ & $6$ & $13$ & $46.15\%$\\
-- 3SG feminine & $5$ & $0$ & $5$ & $0.00\%$ \\
-- 3SG neuter & $138$ & $55$ & $193$ & $28.50\%$\\
1PL & $1$ & $0$ & $0$  &  $0.00\%$ \\
2PL & $0$ & $0$ & $0$  & $0.00\%$ \\
3PL & $9$ & $0$ & $0$  & $0.00\%$ \\
undefined & $0$ & $5$ & $5$ & $100.00\%$ \\
\lspbottomrule
\end{tabular}
\label{tab:FraC.Comp.person.obj}
\end{table}


Table \ref{tab:FraC.Comp.person.obj} shows the grammatical person, number, and gender of the object data.
Of the 160 overt objects in the prefield, 150 are 3rd person singular, nine are 3rd person plural and one is 1st person plural.
Of the 66 covert objects, 61 are 3rd person singular and for five, the person and number cannot be determined due to the missing precontext.
As for the 3rd person singular subjects, also the vast majority of the 3rd person singular cases are of neuter gender, both in the instances of topic drop and the full forms.
Most of the topic drop cases, around 50, have a proposition or a VP as their antecedent. \is{Antecedent}
Potentially, the explanation that topic drop of the 3rd person singular neuter is frequent because it often targets pronouns referring to salient \is{Salience} propositions is more accurate for objects than for subjects.
Further research is needed to investigate whether there is a systematic difference.
While the numbers for the objects are overall relatively low, the result is in line with the observation made in \sectref{sec:usage.function.theory} that topic drop of 1st and 2nd person objects is at least very infrequent.
Similarly, the corresponding full forms appear to be rare, which is compatible with a frequency-based approach to explain their omission.
They cannot be omitted (well) because they are not very predictable \is{Predictability} in the prefield (see \sectref{sec:info.theory.function}).
This argumentation has to be qualified by the fact that objects of the 3rd person singular masculine were omitted almost as often as they were realized, even though they were clearly rarer than the 3rd person singular neuter objects with only 13 occurrences in total.
Thus, it seems that the frequency hypothesis (simplified: if the full form is frequent, topic drop is frequent) cannot fully explain the occurrence of object topic drop or that at least larger amounts of data are needed to make conclusive statements.

\subsection{Grammatical person in \textsc{FraC-TD-SMS}}\label{sec:frac.td.sms.person}
The \textsc{FraC-TD-SMS} data set primarily contains overt or covert subjects in the prefield (see \sectref{sec:corpus.mess}).
Since there are only seven objects,%
%% Footnote
\footnote{There are two realized objects in the form of the 3rd person singular demonstrative pronoun \textit{das} (`that').
For the five instances of topic drop, the grammatical person and number cannot be unambiguously determined due to the missing precontext.}
%%
a number too low for meaningful conclusions, I focus only on the subjects in this section.
Table \ref{tab:frac.gr.pers} shows their distribution by grammatical person.%
%% Footnote
\footnote{The two indeterminate cases are omissions of the indefinite pronoun \textit{man} (`one') in the prefield.}

The majority of omitted or realized subjects in the prefield position are 1st person singular pronouns.
Their omission rate of about 67\% is the highest.
This is generally in line with the previous research presented in \sectref{sec:usage.person.studies}, according to which the 1st person singular is especially often targeted by topic drop \citep{auer1993}, in particular in text messages \citep{androutsopoulos.schmidt2002,doring2002,frick2017}, although the 1st person plural and the 3rd person singular neuter have compatible high omission rates.%
%% Footnote
\footnote{The omission rate of the 1st person singular even is about 7\% higher in absolute terms in the \textsc{FraC-TD-SMS} data set than in the corpora of \citet{androutsopoulos.schmidt2002} and \citet{frick2017}.
The difference to Androutsopoulos and Schmidt is marginally significant according to a Pearson's chi-squared test with Yates's continuity correction calculated in R \citep{rcoreteam2021} ($\chi^2(1) = 3.11, p < 0.1$); the difference to Frick is significant ($\chi^2(1) = 7.98, p < 0.01$). 
(Recall that the amount of data investigated in \cite{frick2017} is considerably higher than that in \cite{androutsopoulos.schmidt2002}, yielding higher statistical power.)
Also the omission rate of the 3rd person singular with about 59\% is slightly higher than the corresponding rates by Androutsopoulos and Schmidt (51\% for \textit{es} (`it') and \textit{das} (`that'); not significant ($\chi^2(1) = 0.28, p > 0.5$)), and Frick (41\%, significant ($\chi^2(1) = 6.69, p < 0.01$)).
The rate of about 63\% for the 1st person plural is significantly higher than the 30\% in \citet{androutsopoulos.schmidt2002} ($\chi^2(1) = 8.6, p < 0.01$) and the 20\% in \citet{frick2017} ($\chi^2(1) = 43.08, p < 0.001$).
With 32\% the omission rate of the 2nd person singular lies between the 26\% that Androutsopoulos and Schmidt report and Frick's 47\%, the differences are not significant however ($\chi^2(1) = 0.103, p > 0.8$; $\chi^2(1) = 1.56, p > 0.2$).
These discrepancies with the data from the previous studies cannot be readily explained.
Here, a more detailed examination of the text messages contained in each data set might be required.}
%
It seems that, on the one hand, pronouns referring to the speaker(s) are particularly often omitted in the \textsc{FraC-TD-SMS} data set.

\begin{table}
\caption{Full forms, instances of topic drop, and omission rates as a function of grammatical person, number, and gender for the subjects in the \textsc{FraC-TD-SMS} data set}
\centering
\begin{tabular}{lrrrr}
\lsptoprule
\Centerstack[c]{Grammatical person,\\number} & \multicolumn{1}{c}{\multirow{-1}{*}{Full form}}  & \multicolumn{1}{c}{\multirow{-1}{*}{Topic drop}} & \multicolumn{1}{c}{\multirow{-1}{*}{Total}} & \multicolumn{1}{c}{\multirow{-1}{*}{Omission rate}} \\
\midrule
1SG & $131$ & $264$ & $395$ & $66.84\%$ \\
2SG & $17$ & $8$ & $25$ & $32.00\%$ \\
3SG & $29$ & $41$ & $70$ & $58.57\%$ \\
-- 3SG masculine & $5$ & $1$ & $6$ & $16.67\%$\\
-- 3SG feminine & $5$ & $3$ & $8$ & $37.50\%$\\
-- 3SG neuter & $17$ & $37$ & $54$ & $68.52\%$\\
-- indeterminate & $2$ & $0$ & $2$ & $0.00\%$ \\
1PL & $21$ & $35$ & $56$ &  $62.50\%$ \\
2PL & $1$ & $0$ & $1$ & $0.00\%$ \\
3PL & $0$ & $0$ & $0$ & $0.00\%$ \\
\lspbottomrule
\end{tabular}
\label{tab:frac.gr.pers}
\end{table}

\noindent
On the other hand, pronouns of the 3rd person singular are also more often omitted than realized.
Of the 41 instances of 3rd person singular subject topic drop in the data set, 37 are neuter.
Their omission rate of 68.52\% is even higher than the rate of the 1st person singular.
Of these neuter pronouns, 11 are expletives \is{Expletive} and 17 refer to propositions or VPs in the previous discourse, while the remaining 9 refer to entities.
This indicates, similar to the results from the \textsc{FraC-TD-Comp} data set, that a 3rd person singular pronoun is indeed often omitted when it refers to a salient \is{Salience} proposition, as discussed in Sections \ref{sec:usage.person.studies} and \ref{sec:info.theory.person}, but it is by no means restricted to only these cases.
Additionally, for the 17 neuter full forms a similar picture emerges.
There are 5 expletives \is{Expletive|)} and only 2 neuter pronouns referring back to an entity, while the remaining neuter pronouns are either clearly (3 instances) or presumably (7 instances) referring back to a proposition or a VP.
To detect a potential difference between topic drop and full forms in terms of how frequently the overt or covert prefield element refers to propositions or VPs, a larger data set is needed. \is{Verb phrase|)}

\is{Verbal inflection|(}
\subsection{Inflectional ending in \textsc{FraC-TD-SMS-Part}}\label{sec:frac.td.sms.part.inflection} \is{Syncretism|(}
In the \textsc{FraC-TD-SMS-Part} data set, which contains all utterances from the text message subcorpus with an overt or covert 1st and 3rd person singular subject pronoun in preverbal position (see \sectref{sec:corpus.sms.part} and the repeated Table \ref{tab:frac.td.mess.part.rep}, for an overview), I investigated not only the role of the grammatical person of the subjects but also the role of the inflectional ending on the verb in the left bracket.

%\vspace{-1em}
\begin{table}
\centering
\caption{Full forms, instances of topic drop, and omission rates as a function of grammatical person in the \textsc{FraC-TD-SMS-Part} data set (repeated from page \pageref{tab:frac.td.mess.part})}
\begin{tabular}{lrrrr}
\lsptoprule
Grammatical person & Full form & Topic drop & Total & Omission rate\\
\midrule
1SG & $131$ & $264$ & $395$ & $66.84\%$\\
3SG & $29$ & $41$ & $70$ & $58.57\%$\\
\tablevspace
Total & $160$ & $305$ & $465$ & $65.59\%$\\
\lspbottomrule
\end{tabular}
\label{tab:frac.td.mess.part.rep}
\end{table}
%\vspace{-0.5\baselineskip}

\noindent
For each instance in \textsc{FraC-TD-SMS-Part}, I manually annotated whether the verb in the left bracket following a realized or omitted subject is distinctly marked for inflection or whether the verb form is syncretic.
I pursued a strict approach so that any verb form that could have more than one meaning/function was annotated as syncretic, regardless of how frequent/common the forms are respectively.
The results of this annotation process are shown in Table \ref{tab:frac.inflection}.

\begin{table}
\centering
\caption{Full forms, instances of topic drop, and omission rates as a function of inflectional ending in the \textsc{FraC-TD-SMS} data set according to a strict annotation}
\begin{tabular}{lrrrr}
\lsptoprule
\multicolumn{1}{c}{Verb form} & \multicolumn{1}{c}{Full form} & \multicolumn{1}{c}{Topic drop} & \multicolumn{1}{c}{Total} & \multicolumn{1}{c}{Omission rate} \\
\midrule
Syncretic & $135$ & $241$ & $376$ & $64.10\%$\\
-- 1SG and 3SG & $44$ & $71$ & $115$ & $61.74\%$\\
-- 1SG and \textsc{imp.sg} & $84$ & $157$ & $241$ & $65.15\%$\\
-- 3SG and 2PL & $7$ & $13$ & $20$ & $65.00\%$\\
Distinct & $25$ & $64$ & $89$ & $71.91\%$ \\
\lspbottomrule
\end{tabular}
\label{tab:frac.inflection}
\end{table}
%\vspace{-0.5\baselineskip}

\is{Ambiguity|(}
In absolute terms, there are 376 syncretic verb forms and 89 distinct ones.
The omission rate is higher before distinctly marked verb forms ($71.91\%$) than before syncretic forms ($64.1\%$).
Nevertheless, the subject is more often omitted than realized before syncretic verb forms.
This observation speaks against the literal, i.e., categorical version of \citeg{auer1993} \textit{inflectional hypothesis} because topic drop also frequently occurs before verb forms that do not indicate the grammatical person and number of the omitted subject through their inflection.
However, it is in line with a gradual variant and with the information-theoretic \textit{facilitate recovery} principle \is{Recoverability} because the preference for topic drop seems to be higher before distinctly marked verb forms.

In \textsc{FraC-TD-SMS-Part}, there are three relevant types of syncretisms, which I annotated manually:
(i) The first one is the form equivalence between the 1st and 3rd person singular in the preterite and for preterite-present verbs also in the present tense, which I discussed in \sectref{sec:usage.person.theory}.
This concerns 115 verb forms.
In the majority of the cases (35 or 79.55\% for the full forms, 57 or 80.28\% for the utterances with topic drop), the intended form is that of the 1st person singular.
(ii) The second type is the syncretism between the 1st person singular present tense and the imperative \is{Imperative} singular discussed in \sectref{sec:usage.ambiguity.theory}.
It contributes the largest proportion of syncretic forms with 241 instances (84 full forms, 157 utterances with topic drop).
Their omission rate of about 65\% is the highest (together with that for the syncretism between the 3rd person singular and the 2nd person plural), which suggests that this syncretism is especially unproblematic for topic drop.
This hypothesis is supported by the fact that besides the 241 reported instances of this type of syncretism that are indicative forms, there are additionally 80 imperative \is{Imperative} occurrences, of which 77 are syncretic with the indicative forms.%
%% Footnote
\footnote{I manually searched the complete text message subcorpus for imperative \is{Imperative} singular forms at the left edge of the utterances.
The search revealed 7 ambiguous forms and 3 clear imperative forms (\textit{vergiss} (`forget'), \textit{nimm} (`take'), and \textit{gib} (`give')).
}
This sheer amount of 234 sentence-initial verb forms (157 instances of topic drop and 77 syncretic imperative forms) that are ambiguous between the imperative singular and the 1st person singular indicative present tense suggests indeed that writers seem to consider this ambiguity relatively unproblematic.
However, this has to be qualified by the fact that many of the 1st person singular verb forms that are formally identical to imperative forms are not very common as an imperative, \is{Imperative} such as \textit{komme} (`come'), which is usually formed without the final schwa as \textit{komm}, or are not very plausible as a demand, such as \textit{brauch}(\textit{e}) (`need').
Therefore, it seems useful, in particular for this group of syncretic verb forms, to distinguish between the theoretical form equality and the ambiguity present in practice, as suggested below.
(iii) As a third type, I considered a previously unmentioned form equivalence for verbs without stem vowel alternation between the 3rd person singular present tense and the 2nd person plural present tense or the imperative \is{Imperative} plural, e.g., \textit{er geht} (`he goes') vs. \textit{ihr geht} (`you go') vs. \textit{Geht!} (`Go!').
With only 20 instances (7 full forms, 13 utterances of topic drop), this group is the rarest syncretism in \textsc{FraC-TD-SMS-Part} but together with type (ii) the one with the highest omission rate (however, this rate is less expressive for only 20 instances).
Given the fact that the 2nd person plural generally occurs very rarely in the text message subcorpus of the FraC, as does the imperative \is{Imperative} plural, this syncretism may not be problematic.
It may be that this type of potentially ambiguous verb form behaves more like a distinct form.
That is, the hearer simply interprets it as the a priori more likely 3rd person singular form, and only realizes that there is an ambiguity when they are forced to reanalyze the structure because the subsequent context does no longer fit with their initial interpretation.
The data of these three types of syncretisms as a result of the strict annotation are used in the regression analysis, which is presented in \sectref{sec:frac.td.part.regression.person}.

The inflection predictor also includes an ``informed'' annotation of syncretisms to indicate which of the theoretical existing syncretisms matter in practice, i.e., which verb forms are ambiguous for a hearer or reader.
To this effect, three student assistants,%
%% Footnote
\footnote{I thank Annika Schäfer, Ricarda Scherer,  and the third person who wished to remain anonymous.}%
%
undergraduate students of German studies from Saarland University, manually annotated all verb forms in the \textsc{FraC-TD-SMS-Part} data set.
They read the immediately preceding context utterance, if available, and then only the verb form of the target utterance, i.e., the prefield constituent was removed if originally present, just like the rest of the utterance.
This was intended to simulate incremental parsing at the verb, i.e., to mimic a hearer who processes the verb of an utterance with topic drop in context.
The student assistants' task was then to annotate this verb form for grammatical person, number, verb mode, and tense.
They were told that it is possible to indicate as a comment that there was more than one possible reading for the verb form, but it was not spelled out explicitly that they should pay special attention to potential ambiguities. 
I considered every verb form as syncretic in the sense of this informed annotation for which an alternative analysis was proposed by at least one of the three student assistants.
Table \ref{tab:frac.inflection.3level} shows the distribution of syncretic and distinct forms across full forms and utterances with topic drop for both the strict and the informed annotation in comparison.

\begin{table}
\caption{Full forms, instances of topic drop, and omission rates as a function of inflectional ending in the \textsc{FraC-TD-SMS} data set according to a strict vs. an informed annotation}
\centering
\begin{tabular}{lrrrr}
\lsptoprule
\multicolumn{1}{c}{Verb form} & \multicolumn{1}{c}{Full form} & \multicolumn{1}{c}{Topic drop} & \multicolumn{1}{c}{Total} & \multicolumn{1}{c}{Omission rate} \\
\midrule
Distinct (strict) & $25$ & $64$ & $89$ & $71.91\%$ \\
Syncretic (strict) & $135$ & $241$ & $376$ & $64.10\%$ \\
\tablevspace
Total & $160$ & $305$ & $465$ & $65.59\%$\\
\lspbottomrule
\multicolumn{5}{c}{} \\
\lsptoprule
\multicolumn{1}{c}{Verb form} & \multicolumn{1}{c}{Full form} & \multicolumn{1}{c}{Topic drop} & \multicolumn{1}{c}{Total} & \multicolumn{1}{c}{Omission rate} \\
\midrule
Distinct (informed) & $87$ & $187$ & $274$ & $68.25\%$ \\
Syncretic (informed) & $73$ & $118$ & $191$ & $61.78\%$ \\
\tablevspace
Total & $160$ & $305$ & $465$ & $65.59\%$\\
\lspbottomrule
\end{tabular}
\label{tab:frac.inflection.3level}
\end{table}

\noindent
The total number of informed syncretic forms (191) is significantly lower than the total number of strictly syncretic forms (376) ($\chi^2(1) = 152.98, p < 0.001$).
A closer inspection reveals that the former are a proper subset of the latter.
Additionally, the omission rates before the syncretic verb forms are lower than before the distinct verb forms for both the strict and the informed annotation.
This suggests that distinct inflection may indeed play a role in how topic drop is used.
In the informed annotation, the majority of the recognized syncretisms are cases of ambiguity between the 1st person singular and the imperative \is{Imperative} singular (115), which is also the largest group in the strict annotation, followed by the syncretism between the 1st and the 3rd person singular (73),%
% Footnote
\footnote{Interestingly, one annotator consistently provided as alternatives for these syncretisms a declarative utterance with topic drop of the 1st person singular and an interrogative utterance with an overt 3rd person singular subject in the middle field.
This can be interpreted as anecdotal evidence that topic drop is considered to be a more likely alternative for the 1st than for the 3rd person singular.}
%
whereas there are only 3 cases of the syncretism between 3rd person singular and 2nd person plural.
It seems that in most cases, a reader notices whether a verb form is syncretic, especially for the 1st person singular and the imperative. \is{Imperative}
In these cases, as Table \ref{tab:frac.inflection.3level} suggests, the syncretism and the resulting ambiguity lead writers to use topic drop less frequently, but with over 60\% omission rate, topic drop is still clearly preferred over the full form.
In the next section, I investigate whether these descriptively observed tendencies are confirmed in a statistical analysis. \is{Ambiguity|)}

\subsection{Grammatical person and inflectional ending in \textsc{FraC-TD-SMS-Part} -- logistic regression analysis}\label{sec:frac.td.part.regression.person}
As the centerpiece of my corpus study,  I conducted a logistic regression analysis on the \textsc{FraC-TD-SMS-Part} data set.
I investigated whether the likelihood of subject topic drop of the 1st and the 3rd person singular varies as a function of grammatical person, inflectional ending, verb surprisal, and verb type.
In this section, I outline the analysis but focus only on grammatical person and inflection in the interpretation.
The other factors are discussed in the corresponding Chapter \ref{ch:usage.verb}, namely in \sectref{sec:corpus.regression.rep}.

Based on 305 utterances with topic drop and 160 full forms, I predicted the likelihood of the binary dependent variable \textsc{Completeness} (full form vs. topic drop) from the independent variables grammatical \textsc{Person}, \textsc{Inflection}, verb \textsc{Surprisal}, and \textsc{Verb Type} in a logistic regression analysis.
\textsc{Person} was a binary predictor distinguishing between the 1st and the 3rd person singular.
\textsc{Inflection} had three levels: \textit{distinct}, \textit{strictly syncretic}, and \textit{informed syncretic}.
\textit{Strictly syncretic} accounted for every theoretically possible form of syncretism, while \textit{informed syncretic} considered only those cases where the ambiguity resulting from the syncretism mattered for at least one of three annotators, as described in \sectref{sec:frac.td.sms.part.inflection}.
I used forward coding to create two contrasts for this predictor: \textsc{Inflection Distinct} compared \textit{distinct} to both \textit{strictly} and \textit{informed syncretic} to see whether any form of syncretism is of relevance, while \textsc{Inflection Informed} compared \textit{informed syncretic} to \textit{distinct} and \textit{strictly syncretic} to look at a potential difference between theoretically and practically relevant ambiguity.
\textsc{Surprisal} was a numeric predictor for the unigram surprisal per verb lemma of the verb following topic drop.
The binary predictor \textsc{Verb Type} encoded whether this verb belonged either to the classes copular, \is{Copula} auxiliary, \is{Auxiliary} or modal verb \is{Modal verb} or to the classes lexical \is{Lexical verb} or reflexive verbs \is{Reflexive verb} (see \sectref{sec:corpus.regression.rep} for details).

\subsubsection{Predictions}
From the isolated hypotheses developed in the theoretical literature discussed in \sectref{sec:usage.person.theory} and from the predictions of my unifying information-theoretic account presented in \sectref{sec:info.theory.person}, I derive the following predictions concerning the likelihood of topic drop as a function of grammatical person and verbal inflection (see Table \ref{tab:frac.td.mess.partpredictions} for an overview and \sectref{sec:corpus.regression.predictions.rep} for the other two factors).

\begin{table}
\centering
\caption{Predictions for grammatical person and inflection to be tested on \textsc{FraC-TD-SMS-Part}: a checkmark indicates an expected effect on the likelihood of topic drop.}
\begin{tabular}{lcccc}
\lsptoprule
& Information- & \multicolumn{2}{c}{Theoretical literature} & \\
& theoretic &\textit{Inflectional}  & \textit{Extralinguistic}  &  \\
& account & \textit{hypothesis} &  \textit{hypotheses}  & \multirow{-3}{*}{\Centerstack{Ambiguity\\ avoidance}}\\
\midrule
(i--\textsc{Person}) & \ding{51} & (\ding{51}) & \ding{51} & --\\
(ii--\textsc{Inflection}) & \ding{51} & \ding{51} & -- & \ding{51} \\
\lspbottomrule
\end{tabular}
\label{tab:frac.td.mess.partpredictions}
\end{table}

\noindent
(i--\textsc{Person})
My information-theoretic account predicts a significant main effect of \textsc{Person}, according to which topic drop should more frequently target the 1st person singular than the 3rd person singular.
This effect is the result of the increased predictability \is{Predictability} of the 1st person singular that feeds on two sources.
First, the reference of an omitted 1st person singular pronoun can be uniquely determined in the text type text messages as the writer of the message, whereas for an omitted 3rd person singular constituent, there are usually several potential referents.
This means that the probability mass is usually split among several potential referents in the 3rd person singular case, whereas it is completely concentrated on the speaker/writer in the 1st person singular case.
Second, the predicted effect is a consequence of general frequency considerations.
The writer is not only uniquely determinable, but they are also very prominent in text messages and often text about themselves.
This, in turn, results in the 1st person singular pronoun \textit{ich} (`I') occurring more frequently in text messages than 3rd person singular pronouns, as discussed in Sections \ref{sec:usage.person.studies} and \ref{sec:factor.person.corpus}.
Therefore, \textit{ich} (`I') becomes highly predictable and is very likely to be omitted.

In this reasoning, the information-theoretic  account draws on what I termed \textit{extralinguistic hypotheses} as one source of the predictability \is{Predictability} of the prefield constituent.
These \textit{extralinguistic hypotheses} likewise predict a significant main effect of \textsc{Person}.
The idea is that the information from the extralinguistic context such as world knowledge, knowledge about the components of a communication situation, as well as text type knowledge facilitate the reconstruction and, thus, the omission of the 1st and, possibly, also the 2nd person.
In the case of text messages, it is, as mentioned above, the knowledge that in a text messaging setting there is usually exactly one writer of a text message and that this writer often texts about themselves that serves as an explanation for the higher frequency of 1st person singular subject topic drop.

Partly, also \citeg{auer1993} \textit{inflectional hypothesis} predicts an effect of \textsc{Person} but only in interaction with \textsc{Inflection}.
Topic drop of the 1st person singular should be more frequent before verbs that are explicitly marked for inflection but not (necessarily) before syncretic verb forms.

(ii--\textsc{Inflection})
However, if we take \citeg{auer1993} \textit{inflectional hypothesis} literally,%
% Footnote
\footnote{``Die Morphologie des Deutschen ist im Singular des Präsensparadigmas noch differenziert genug, um auch ohne pronominale Markierung die Person flektivisch ausdrücken zu können'' (My translation: `The morphology of \ili{German} is still differentiated enough in the singular of the present tense paradigm to be able to express the person inflectively even without pronominal marking') \citep[198]{auer1993}.}
%
it can be generalized to mean that topic drop should generally be possible only (or at least be more frequent) if the verb form in the left bracket is not syncretic but uniquely marked for inflection.
This predicts not an interaction between \textsc{Person} and \textsc{Inflection}, as stated in the last paragraph, but a main effect of \textsc{Inflection}.
If all that matters is being able to derive the grammatical person from the verb form, the relevant annotation should be the strict one, i.e., the main effect should be visible at the new contrast \textsc{Inflection Distinct} that compares distinct to strictly and informed syncretic verb forms.

While the \textit{extralinguistic hypotheses} do not make specific predictions for verbal inflection, my information-theoretic account does.  
It likewise predicts a main effect of \textsc{Inflection} according to which topic drop should be more frequent if the following verb is distinctly marked.
Following from the \textit{facilitate recovery} principle, \is{Recoverability} discussed in \sectref{sec:resolving}, such a distinct inflectional ending should reduce the overall processing effort \is{Processing effort|(} on the verb because it provides information about the omitted subject.
Thus, it should increase the likelihood of topic drop.
For the information-theoretic account,  it may matter how likely the two or more alternative meanings of a syncretic form are.
That is, a syncretic form that has one very likely and one (or more) very unlikely meaning(s) may cause less processing effort if the intended meaning is the likely one than a syncretic form where the competing meanings are approximately equally likely.
Therefore, it may not be the contrast \textsc{Inflection Distinct} for which the significant main effect is expected but rather \textsc{Inflection Informed}, where the distinct verb forms and the forms that are only theoretically syncretic are pooled and jointly compared to the informed syncretic verb forms.

A main effect of inflection is also predicted by the principle of ambiguity avoidance. \is{Ambiguity avoidance|(}
Topic drop should be less frequent if the following verb is not distinctly marked for inflection because syncretic verb forms lead to ambiguous utterances and potentially increased processing effort for the hearer (as described in \sectref{sec:usage.ambiguity.theory}).
As for the information-theoretic account, it might also matter from the perspective of ambiguity avoidance how likely the two or more different meanings of a syncretic verb form are.
That is, a speaker might be more likely to avoid ambiguity if the meanings are approximately likely than if the meaning they intend to communicate is the most likely one anyway.
Consequently, it is rather \textsc{Inflection Informed} than \textsc{Inflection Distinct} that is predicted to have a significant effect on the likelihood of topic drop according to ambiguity avoidance. \is{Ambiguity avoidance|)}

\subsubsection{Results}\label{sec:corpus.inference.results}
I conducted the logistic regression analysis using general linear models with family binomial in R \citep{rcoreteam2021}.
I predicted the likelihood of topic drop from the binary independent variables grammatical \textsc{Person}  (1SG vs. 3SG) and \textsc{Verb Type} (copular, auxiliary, or modal verbs vs. lexical or reflexive verbs), which were coded using deviation coding (1SG and lexical or reflexive verb were coded as $0.5$, 3SG and copular, auxiliary, or modal verb as $-0.5$ respectively), from the numeric unigram \textsc{Surprisal} of the verb following topic drop and from \textsc{Inflection}, as well as from all the two-way interactions between the predictors.
\textsc{Inflection} had three levels (distinct vs. strictly syncretic vs. informed syncretic) and was forward coded:
For \textsc{Inflection Distinct} \textit{distinct} was coded as $\sfrac{2}{3}$, \textit{strictly syncretic} and \textit{informed syncretic} as $-\sfrac{1}{3}$  respectively.
For \textsc{Inflection Informed} \textit{informed syncretic} was coded as $-\sfrac{2}{3}$, \textit{distinct} and \textit{strictly syncretic} as $\sfrac{1}{3}$ respectively.%
% Footnote
\footnote{The formula of the full model was as follows:
\texttt{\textsc{Completeness} \textasciitilde~  (\textsc{Person} + \textsc{Surprisal} + \textsc{Verb Type} + \textsc{Inflection Distinct})\textasciicircum2 + (\textsc{Person} + \textsc{Surprisal} + \textsc{Verb Type} + \textsc{Inflection Informed})\textasciicircum2.
}}
%
The fixed effects in the final model, which I obtained with a backward model selection, as described in \sectref{sec:data.analysis}, are presented in Table \ref{tab:frac.mess.part.model}.

\begin{table}
\centering
\caption{Fixed effects in the final GLM analyzing \textsc{FraC}‑TD‑SMS‑\textsc{Part}}
\begin{tabular}{lrrrll}
\lsptoprule
Fixed effect & Est. & SE & $\chi^2$ & \textit{p}-value &   \\
\midrule
\textsc{Intercept} & $2.81$ & $0.57$ & $27.20$ & $< 0.001$ & ***\\
\textsc{Person} & $0.35$ & $0.29$ & $1.46$ & $> 0.2$ & \\
\textsc{Surprisal} & $-0.27$ & $0.07$ & $17.85$ & $< 0.001$ & ***\\
\textsc{Verb Type} & $0.95$ & $0.39$ & $6.30$ & $< 0.05$ & *\\
\textsc{Inflection Informed} & $-0.33$ & $0.31$ & $1.15$ & $> 0.2$ & \\
\textsc{Person $\times$ Verb Type} & $-1.30$ & $0.61$ & $4.72$ & $< 0.05$ & *\\
\textsc{Person $\times$ Inflection Informed} & $1.47$ & $0.63$ & $5.85$ & $< 0.05$ & *\\
\lspbottomrule
\end{tabular}
\label{tab:frac.mess.part.model}
\end{table}

There was a significant interaction between grammatical \textsc{Person} and \textsc{Inflection Informed} ($\chi^2(1) = 5.85, p < 0.05$).
The likelihood of topic drop is increased for the 1st person singular before practically unambiguous verb forms (see Figure \ref{fig:frac.corpus.plot.freq.person}).

\begin{figure}
\centering
\includegraphics[scale=1]{Korpusplots/Corpus_FraCTDMessPart_PersonInflection.pdf}
\caption{Frequency of the full forms and the instances of topic drop as a function of grammatical \textsc{Person} and \textsc{Inflection Informed} in the \textsc{FraC-TD-SMS-Part} data set}
\label{fig:frac.corpus.plot.freq.person}
\end{figure}

The significant interaction between \textsc{Person} and \textsc{Verb Type} ($\chi^2(1) = 4.72, p < 0.05$) suggests furthermore that topic drop is more likely if the covert constituent is a 1st person singular pronoun and if, at the same time, the following verb is a copular, auxiliary, or modal verb.
There is a significant main effect of \textsc{Verb Type} ($\chi^2(1) = 6.3, p < 0.05$), which runs contrary to the interaction:
Accordingly, topic drop is less likely before copular, auxiliary, or modal verbs.
Finally, the significant main effect of verb \textsc{Surprisal} ($\chi^2(1) = 17.85, p < 0.001$) in the negative direction suggests that the likelihood of topic drop decreases with a higher verb surprisal.
All other main effects and interactions were not significant (all $p > 0.2$).

\subsubsection{Discussion}\label{sec:corpus.inference.diss}
The logistic regression analysis revealed two significant interactions in which grammatical person participated.
The results suggest that topic drop of the 1st person singular is more likely than topic drop of the 3rd person singular if (i) the following verb is not ambiguous \is{Ambiguity avoidance|(} to the hearer and (ii) if this verb is a copular, auxiliary, or modal verb.
While I turn to result (ii) in \sectref{sec:corpus.inference.rep.diss}, result (i) is in line with the narrower interpretation of \citeg{auer1993} \textit{inflectional hypothesis}:
A 1st person singular subject is more likely to be omitted if the following verb relatively unambiguously indicates the grammatical person of the congruent subject.
When discussing prediction (ii--\textsc{Inflection}) above, I argued that if we take \citeg{auer1993} \textit{inflectional hypothesis} literally, we would actually expect not an interaction but a main effect of \textsc{Inflection}.
Topic drop should be more likely whenever the following verb clearly indicates the grammatical person of the omitted subject.
Such a main effect was also predicted by my information-theoretic account and by ambiguity avoidance, but it was not significant in my analysis.

An explanation for why there was an interaction with grammatical person instead of the predicted main effect may be found if a semantic perspective is included on how the referent of the omitted prefield constituent is ultimately determined.
While the \textit{inflectional hypothesis} comes from a purely formal perspective, i.e., a distinct inflectional ending results in being able to recover \is{Recoverability|(} the omitted subject, it disregards an important difference between the 1st and 2nd person on the one hand and the 3rd person on the other.
For the 1st and the 2nd person (singular), the referent of the omitted constituent can be identified as the speaker or hearer, whereas there are usually several candidates for the referent of an omitted 3rd person singular subject, i.e., any gender-matching entity present in the current discourse or situation.
Consequently, using (relatively) unambiguous verb forms may be particularly beneficial for the 1st person singular. \is{Ambiguity avoidance|)}
In information-theoretic terms:  
It reduces the processing effort \is{Processing effort|)} associated with ellipsis resolution more strongly as it allows the hearer to directly identify the intended referent, as discussed in \sectref{sec:resolving}.

The results also suggest a preference for topic drop of the 1st person as predicted by (i--\textsc{Person}) and the \textit{extralinguistic hypotheses}, as well as by my informa- tion-theoretic account.
The 1st person singular as the pronoun referring to the speaker is generally more frequent in the text message subcorpus and, thus, more predictable and more easily recoverable. \is{Recoverability|)}
The fact that this preference for the 1st person is only reflected in the two interactions and not in an additional main effect is most likely due to the imbalanced data set, which contains generally a lot fewer instances of the 3rd person singular than the 1st person singular, in combination with the deviation coding, where ``the effect of each factor is coded to reflect how it influences the DV while collapsing across any other factors'' \citep[3]{brehm.alday2022}.
\is{Corpus|)}\is{Syncretism|)} \is{Verbal inflection|)}

\section{Experimental investigations of grammatical person and verbal inflection}\label{sec:usage.person.exp}
The results of the corpus study presented in \sectref{sec:factor.person.corpus} suggest that grammatical person and verbal inflection influence why and when speakers use topic drop.
In the following, I present the results of five studies that aimed to evidence the relevance of these two factors also experimentally.

I start this section with a further discussion of the three acceptability rating studies that I outlined already in Chapter \ref{ch:factor.topicality} on topicality (see \sectref{sec:topicality.experiments}).
Experiments \ref*{exp:top.q1}, \ref*{exp:top.s.fv}, and \ref*{exp:top.s.mv} investigated the impact of grammatical person and topicality on the acceptability of topic drop.
In this section, I focus on the results related to grammatical person and verbal inflection.
More specifically, by comparing the 1st and the 3rd person singular in these three experiments, I investigated whether the difference in frequency evidenced in the corpus study is reflected in acceptability.
Additionally, since experiments  \ref*{exp:top.s.fv} and \ref*{exp:top.s.mv} were identical except for the use of lexical verbs \is{Lexical verb} with a distinct inflectional marking (experiment \ref*{exp:top.s.fv}) vs. syncretic forms of modal verbs (experiment \ref*{exp:top.s.mv}), a comparison between both studies can also provide insights into the role of verbal inflection. \is{Verbal inflection}

In the second part of this section, I turn to experiments \ref*{exp:1sg.2sg} and \ref*{exp:1sg.2sg.spoken}, with which I extended the investigation of grammatical person.
I compared the 1st and 2nd person singular to distinguish between the two types of \textit{extralinguistic hypotheses} that aim to explain a preference for topic drop of only the 1st or of both the 1st and the 2nd person singular.
While experiment \ref*{exp:1sg.2sg} tested this with the established instant messages design, in experiment \ref*{exp:1sg.2sg.spoken} the items were presented as if they were spoken.
This also allows for a first comparison of topic drop between different communication forms.

\subsection{Experiment \ref*{exp:top.q1}: grammatical person (1SG vs. 3SG) }
\is{Acceptability rating study|(}\label{sec:exp.top.q1.person}
In this section, I return to experiment \ref*{exp:top.q1} to discuss the results for grammatical person.
In \sectref{sec:exp.top.q1.top}, I described in detail the design of this acceptability rating study, as well as the stimuli and the analysis.
Here, I only repeat the most central points.%%
% Footnote
\footnote{All items, fillers, and the analysis script can be found online: \url{https://osf.io/zh7tr}.}
%


Experiment \ref*{exp:top.q1} investigated not only the role of topicality using a question method to set the discourse topic but also the influence of grammatical person on topic drop.
It compared utterances with the overt or covert 1st person singular pronoun \textit{ich} (`I') in the prefield to corresponding utterances with an overt or covert 3rd person singular subject pronoun referring to a person.
As discussed in Sections \ref{sec:usage.person.theory} and \ref{sec:usage.person.studies}, the previous literature found a preference for omitting the 1st person singular subject pronoun, which was confirmed in several corpus studies, in particular of text messages. \is{Corpus}
I was able to add further evidence to this by showing that also in the text message subcorpus of the FraC, the 1st person singular was omitted particularly often (see Sections \ref{sec:frac.td.sms.person} and \ref{sec:frac.td.part.regression.person}).
In the existing literature, this preference was explained by an \textit{inflectional hypothesis} \is{Verbal inflection}\citep{auer1993} or two types of \textit{extralinguistic hypotheses} \citep[e.g.,][]{zifonun.etal1997,volodina.onea2012}, which trace back the fact that the 1st person or the 1st and the 2nd person can be better omitted to text type knowledge and fixed speaker and hearer roles (see \sectref{sec:usage.person.theory}).
In \sectref{sec:info.theory.person}, I discussed that the information-theoretic account can capture all three types of hypotheses and explains the potential effect of grammatical person in terms of predictability and reducing processing effort \is{Processing effort} on the verb following topic drop.

Experiment \ref*{exp:top.q1} had the form of a 2 $\times$ 2 $\times$ 2 within-subjects design with the three binary predictors \textsc{Topicality}, grammatical \textsc{Person}, and \textsc{Completeness}.
For \textsc{Person}, I expected to find an interaction with \textsc{Completeness} to the effect that topic drop of the 1st person singular subject pronoun should be more acceptable than topic drop of a 3rd person singular subject pronoun.
This result would show that the frequency differences found in the corpus, i.e., speaker-sided production preferences, are reflected in acceptability differences, i.e., hearer-sided preferences.

\subsubsection{Materials}
The structure of the 24 items and the fillers has been presented in \sectref{sec:exp.top.q1.mat}.

\subsubsection{Procedure}
The procedure of the experiment with 48 participants is described in \sectref{sec:exp.top.q1.procedure}.

\subsubsection{Results}
I repeat the descriptive statistics in Table \ref{tab:descriptives.top.q1.rep} and Figure \ref{fig:pl.top.q1.rep}. 

\begin{table}
\caption{Mean ratings and standard deviations per condition for experiment \ref*{exp:top.q1} (repeated from page \pageref{tab:descriptives.top.q1})}
\centering
\begin{tabular}{lllrr}
\lsptoprule
\multicolumn{1}{c}{\textsc{Completeness}} & \multicolumn{1}{c}{\textsc{Person}} & \multicolumn{1}{c}{\textsc{Topicality}} & \multicolumn{1}{c}{\Centerstack{Mean\\rating}} & \multicolumn{1}{c}{\Centerstack{Standard \\deviation}} \\
\midrule
Full form & 1SG & Topic & $5.48$ & $1.61$ \\
Topic drop & 1SG & Topic & $4.76$ & $1.87$ \\
Full form  & 3SG & Topic & $5.71$ & $1.45$ \\
Topic drop & 3SG & Topic & $4.13$ & $1.88$ \\
Full form & 1SG & No topic  & $5.45$ & $1.69$ \\
Topic drop  & 1SG & No topic  & $4.90$ & $1.69$ \\
Full form & 3SG & No topic  & $5.60$ & $1.58$ \\
Topic drop &  3SG & No topic & $3.71$ & $1.88$ \\
\lspbottomrule
\end{tabular}
\label{tab:descriptives.top.q1.rep}
\end{table}
%\vspace{-0.5\baselineskip}

\begin{figure}
\centering
\includegraphics[scale=1]{Experimenteplots/PL_TopFrage1.pdf}
 \caption{Mean ratings and 95\% confidence intervals per condition for experiment \ref*{exp:top.q1} (repeated from page \pageref{fig:pl.top.q1})}
\label{fig:pl.top.q1.rep} % pl for point line
\end{figure}

\noindent
They show the mean ratings based on the data from 42 participants.
From visual inspection, there seems to be a difference for grammatical person.
Topic drop of the 3rd person singular seems to be less acceptable than topic drop of the 1st person singular.

As described in \sectref{sec:exp.top.q1.results}, the final CLMM (repeated as Table \ref{tab:model.exp.top.q1.rep}) contained significant main effects of the two predictors \textsc{Completeness} ($\chi^2(1) = 26.67, p < 0.001$) and \textsc{Person} ($\chi^2(1) = 5.30, p < 0.05$) and a significant interaction between \textsc{Completeness} and \textsc{Person} ($\chi^2(1) = 15.07, p < 0.001$).
Full forms were significantly preferred over utterances with topic drop.
Utterances with an overt or covert 1st person singular pronoun in the prefield received better ratings than corresponding 3rd person utterances.
Utterances in which a 3rd person singular constituent was targeted by topic drop were especially degraded.

\begin{table}
\caption{Fixed effects in the final CLMM of experiment \ref*{exp:top.q1} (repeated from page \pageref{tab:model.exp.top.q1})}
\centering
\begin{tabular}{lrrrll}
\lsptoprule
Fixed effect & Est. & SE & $\chi^2$ & \textit{p}-value &   \\
\midrule
\textsc{Completeness} & $2.03$ & $0.34$ & $26.67$ & $< 0.001$ & ***\\
\textsc{Person} & $0.55$ & $0.23$ & $5.30$ & $< 0.05$ & *\\
\textsc{Completeness $\times$ Person} & $-1.57$ & $0.34$ & $15.07$ & $< 0.001$ & ***\\
\lspbottomrule
\end{tabular}
\label{tab:model.exp.top.q1.rep}
\end{table}

\subsubsection{Discussion}\label{sec:exp.top.q1.diss.person}
Experiment \ref*{exp:top.q1} was designed to examine the effects of topicality and grammatical person on the acceptability of topic drop.
For grammatical person, it provided evidence that, although participants generally rated utterances with a 1st person subject as more natural than utterances with a 3rd person subject, this preference was particularly strong for topic drop.
That is, topic drop of the 1st person singular subject pronoun \textit{ich} (`I') was perceived as more natural than topic drop of a 3rd person singular subject constituent referring to another human referent in the discourse.
This result is in line with previous claims from the theoretical literature, with the corpus results by \citet{auer1993, androutsopoulos.schmidt2002, frick2017}, and with my own corpus results (see \sectref{sec:factor.person.corpus}).
The production preference for topic drop with respect to grammatical person is also reflected in the addressees' preferences during perception.
This can be captured by the information-theoretic account.  
The 1st person singular subject pronoun is highly predictable \is{Predictability} in instant messages through text type knowledge and general frequency.
The writers of instant messages are just like the writers of text messages clearly identifiable and often text about themselves, which results in \textit{ich} (`I') being very frequent and very predictable.
According to the \textit{avoid troughs} principle, this enhanced predictability increases the preference to omit the \textit{ich}.

Since topic drop was always followed by a verb whose inflectional marking allowed readers to distinguish between the 1st and 3rd person singular, the result of this experiment is generally also in line with \citeg{auer1993} \textit{inflectional hypothesis}. \is{Verbal inflection}
Topic drop of the 1st person singular could be more acceptable because the verb's inflectional ending indicated the grammatical person. \is{Verbal inflection}
However, the 3rd person is just as clearly recognizable purely from the verb ending, but it can nevertheless not be omitted just as well.
It seems that the \textit{inflectional hypothesis} \is{Verbal inflection} must be supplemented at least by \textit{extralinguistic hypotheses} or a processing account to explain the data (as I did similarly in the corpus study, see \sectref{sec:corpus.inference.diss}).
There and in \sectref{sec:resolving}, I argued that due to the uniqueness of the hearer and speaker roles in a conversational situation and certain text types such as text messages, it is quite clear for the 1st person singular what the (covert or overt) pronoun refers to.
In contrast, the 3rd person pronoun can theoretically refer to a variety of things and people in the extralinguistic context and the wider linguistic context, even if there is only one matching linguistic antecedent \is{Antecedent} present in the immediate precontext, as was the case in experiment \ref*{exp:top.q1}.
This way, an inflectional ending \is{Verbal inflection} that distinctly encodes the 1st person singular can help to recover \is{Recoverability} the omitted subject to a larger extent than a 3rd person singular ending and, therefore, reduce the processing effort \is{Processing effort} associated with ellipsis resolution more strongly (see the \textit{facilitate recovery} principle \is{Recoverability} in \sectref{sec:resolving}). \is{Acceptability rating study|)}

\subsection{Experiment \ref*{exp:top.s.fv}: grammatical person (1SG vs. 3SG, lexical verbs)}
\is{Acceptability rating study|(}\label{sec:exp.top.s.fv.person}
In \sectref{sec:exp.top.s.fv}, I presented the details about experiment \ref*{exp:top.s.fv}, which was again a 2 $\times$ 2 $\times$ 2 acceptability rating study crossing \textsc{Completeness} (full form vs. topic drop), grammatical \textsc{Person} (1SG vs. 3SG), and \textsc{Topicality} (topic vs. no topic).%
%% Footnote
\footnote{The items and fillers, as well as the analysis script is accessible online: \url{https://osf.io/zh7tr}.}

There, I discussed its results for topicality, which was manipulated through the subject function.
In this section, I return to the study, but this time, I focus on grammatical person.
The goal was to replicate the result of experiment \ref*{exp:top.q1} that evidenced a preference for topic drop of the 1st person singular.

\subsubsection{Materials}
I described the 24 items and the fillers in detail in \sectref{sec:exp.top.s.fv.materials}.
Recall that the verb following the overt or covert subject in the prefield was a lexical verb \is{Lexical verb} in present tense and, therefore, distinctly marked for grammatical person.

\subsubsection{Procedure}
Information on the procedure and the 48 participants can be found in \sectref{sec:exp.top.s.fv.procedure}.

\subsubsection{Results}\label{sec:exp.top.fv.person.results.}
For convenience, I repeat Table \ref{tab:descriptives.top.full.gp} as Table \ref{tab:descriptives.top.full.gp.rep} and Figure \ref{fig:pl.top.full} as Figure \ref{fig:pl.top.full.rep} from \sectref{sec:exp.top.fv.top.results} to show again the descriptive statistics based on the rating data from 43 participants.
From visual inspection, it seems that topic drop of the 1st person singular was rated better than topic drop of the 3rd person singular.

\begin{table}
\caption{Mean ratings and standard deviations per condition for experiment \ref*{exp:top.s.fv} (repeated from page \pageref{tab:descriptives.top.full.gp})}
\centering
\begin{tabular}{lllrr}
\lsptoprule
\multicolumn{1}{c}{\textsc{Completeness}} & \multicolumn{1}{c}{\textsc{Person}} & \multicolumn{1}{c}{\textsc{Topicality}} & \multicolumn{1}{c}{\Centerstack{Mean \\rating}} & \multicolumn{1}{c}{\Centerstack{Standard\\ deviation}} \\
\hline
Full form & 1SG & Topic & $5.64$ & $1.44$ \\
Topic drop & 1SG & Topic & $5.38$ & $1.40$ \\
Full form & 3SG & Topic & $5.58$ & $1.43$ \\
Topic drop & 3SG & Topic & $4.65$ & $1.60$ \\
Full form & 1SG & No topic & $5.58$ & $1.37$ \\
Topic drop & 1SG & No topic & $4.91$ & $1.46$ \\
Full form & 3SG & No topic & $5.67$ & $1.45$ \\
Topic drop & 3SG & No topic & $4.21$ & $1.67$ \\
\lspbottomrule
\end{tabular}
\label{tab:descriptives.top.full.gp.rep}
\end{table}

\begin{figure}
\centering
\includegraphics[scale=1]{Experimenteplots/PL_TopFV.pdf}
\caption{Mean ratings and 95\% confidence intervals per condition for experiment \ref*{exp:top.s.fv} (repeated from page \pageref{fig:pl.top.full})}
\label{fig:pl.top.full.rep}
\end{figure}

\begin{table}
\caption{Fixed effects in the final CLMM of experiment \ref*{exp:top.s.fv} (repeated from page \pageref{tab:model.exp.top.fv})}
\centering
\begin{tabular}{lrrrll}
\lsptoprule
Fixed effect & Est. & SE & $\chi^2$ & \textit{p}-value &   \\
\midrule
\textsc{Completeness} & $1.72$ & $0.35$ & $20.75$ & $< 0.001$ & ***\\
\textsc{Person} & $0.52$ & $0.20$ & $6.10$ & $< 0.05$ & *\\
\textsc{Topicality} & $0.38$ & $0.15$ & $6.69$ & $< 0.01$ & ** \\
\textsc{Completeness $\times$ Person} & $-1.33$ & $0.35$ & $12.85$ & $< 0.001$ & ***\\
\textsc{Completeness $\times$ Topicality} & $-0.81$ & $0.33$ & $5.95$ & $< 0.05$ & *\\
\lspbottomrule
\end{tabular}
\label{tab:model.exp.top.fv.rep}
\end{table}

A detailed description of my analysis of the rating data with CLMMs can be found in \sectref{sec:exp.top.fv.top.results}.
The final model (repeated as Table \ref{tab:model.exp.top.fv.rep}) contained significant main effects of all three predictors \textsc{Completeness} ($\chi^2(1) = 20.75, p < 0.001$), \textsc{Person} ($\chi^2 = 6.1, p < 0.05$), and \textsc{Topicality} ($\chi^2(1) = 6.69, p < 0.01$), a significant interaction between \textsc{Completeness} and \textsc{Topicality} ($\chi^2(1) = 5.95, p < 0.05$), and a significant interaction between \textsc{Completeness} and \textsc{Person} ($\chi^2(1) = 12.85, p < 0.001$).
For grammatical person, it seems that utterances with the 3rd person singular in preverbal position were generally degraded but particularly strongly in the case of topic drop.

\subsubsection{Discussion}\label{sec:exp.top.fv.diss.person}
With experiment \ref*{exp:top.s.fv}, I investigated again whether grammatical person impacts the acceptability of topic drop, focusing on the contrast between 1st and 3rd person singular subjects. 
Its results replicate the results for grammatical person of experiment \ref*{exp:top.q1}:
The omission of a 1st person singular pronoun was preferred over the omission of a 3rd person singular constituent.
Since the verb following topic drop was likewise a lexical verb \is{Lexical verb} with distinct inflectional marking \is{Verbal inflection} for grammatical person, this result is still compatible with \citeg{auer1993} \textit{inflectional hypothesis}.
It can also be captured by the \textit{extralinguistic hypotheses} and by the information-theoretic account, as outlined in \sectref{sec:exp.top.q1.diss.person}.
I discuss the role of grammatical person in more detail in \sectref{sec:exp.top.s.mv.person.diss}, in the context of the next experiment. \is{Acceptability rating study|)}

\subsection{Experiment \ref*{exp:top.s.mv}: grammatical person (1SG vs. 3SG, modal verbs)}
\is{Acceptability rating study|(}
\is{Modal verb|(}\label{sec:exp.top.s.mv.person}
As mentioned already in \sectref{sec:exp.top.s.mv}, experiment \ref*{exp:top.s.mv} replicates experiment \ref*{exp:top.s.fv} but uses modal verbs instead of lexical verbs.
It was again a 2 $\times$ 2 $\times$ 2 design, which crossed \textsc{Completeness}, grammatical \textsc{Person}, and \textsc{Topicality}.%
%% Footnote
\footnote{All materials and the analysis scripts can be found online: \url{https://osf.io/zh7tr/}.}
%

In this experiment, I used the syncretic \is{Syncretism} verb forms of modal verbs (e.g., \textit{ich will} (`I want') and \textit{Sabrina will} (`Sabrina wants')).
Since they do not encode the grammatical person of the congruent subject, they do not help to recover \is{Recoverability} this subject if it is omitted.
Nevertheless, the hearers could still recover \is{Recoverability} the omitted subject in this experiment because the target utterance always contained an object pronoun in the middle field referring to one of the two possible referents.
This way, it was indicated that the covert subject refers to the other referent.

\is{Verbal inflection|(}
Considered together, experiment \ref*{exp:top.s.fv} with the distinctly marked forms of lexical verbs \is{Lexical verb} and experiment \ref*{exp:top.s.mv} with the syncretic \is{Syncretism} form of modal verbs investigated whether the inflectional ending of the verb in the left bracket impacts the acceptability of topic drop.
Such an impact is implicitly predicted by \citeg{auer1993} \textit{inflectional hypothesis}.
According to a narrow interpretation, there should be an interaction between \textsc{Completeness}, \textsc{Person}, and \textsc{Verb Type}.
Topic drop of the 1st person singular should be more acceptable only if it precedes a lexical verb \is{Lexical verb} that is distinctly marked for grammatical person. 
According to a wider interpretation, any distinct inflectional marking should improve topic drop, i.e., there could be an interaction between \textsc{Completeness} and \textsc{Verb Type}.
The same effect is predicted by the information-theoretic account because a distinct inflectional marking should generally reduce the processing effort \is{Processing effort} on the verb following topic drop.

An impact of distinct vs. syncretic \is{Syncretism} verb forms is also predicted by ambiguity avoidance. \is{Ambiguity avoidance}
Utterances with ambiguous \is{Ambiguity} syncretic verb forms should be avoided by speakers to facilitate the hearer's processing. \is{Processing effort}
However, the utterances used in this experiment are not globally ambiguous \is{Ambiguity} but only locally.
Since the disambiguation later in the sentence by the object pronoun is sufficient to avoid ambiguity, \is{Ambiguity avoidance} it is questionable whether the syncretic \is{Syncretism} verb form needs to be avoided at all.

\subsubsection{Materials}
The adjustment of the materials is described in \sectref{sec:exp.top.s.mv.materials}.

\subsubsection{Procedure}
Details on the procedure can be found in \sectref{sec:exp.top.s.mv.procedure}.

\subsubsection{Results}\label{sec:exp.top.s.mv.res.person}

\subsubsubsection{Analysis of experiment \ref*{exp:top.s.mv}}
The mean ratings of all 48 participants are repeated in Table \ref{tab:descriptives.top.modal.gp.rep} and Figure \ref{fig:pl.top.modal.rep}.
They indicate a  preference for the 1st over the 3rd person singular for topic drop.

%\vspace{-0.5\baselineskip}
\begin{table}
\caption{Mean ratings and standard deviations per condition for experiment \ref*{exp:top.s.mv} (repeated from page \pageref{tab:descriptives.top.modal.gp})}
\centering
\begin{tabular}{lllrr}
\lsptoprule
\multicolumn{1}{c}{\textsc{Completeness}} & \multicolumn{1}{c}{\textsc{Person}} & \multicolumn{1}{c}{\textsc{Topicality}} & \multicolumn{1}{c}{\Centerstack{Mean \\ rating}} & \multicolumn{1}{c}{\Centerstack{Standard\\ deviation}} \\
\midrule
Full form & 1SG & Topic & $5.41$ & $1.55$ \\
Topic drop & 1SG & Topic & $5.24$ &  $1.67$ \\
Full form & 3SG & Topic & $5.52$ & $1.43$ \\
Topic drop & 3SG & Topic & $4.43$ & $1.93$\\
Full form & 1SG & No topic & $5.35$ &  $1.67$\\
Topic drop & 1SG & No topic & $4.86$ & $1.77$\\
Full form & 3SG & No topic & $5.25$ &  $1.74$\\
Topic drop & 3SG & No topic &  $4.27$ & $1.77$\\
\lspbottomrule
\end{tabular}
\label{tab:descriptives.top.modal.gp.rep}
\end{table}

%\vspace{-1\baselineskip}
\begin{figure}
\centering
\includegraphics[scale=1]{Experimenteplots/PL_TopMV.pdf}
\caption{Mean ratings and 95\% confidence intervals per condition for experiment \ref*{exp:top.s.mv} (repeated from page \pageref{fig:pl.top.modal})}
\label{fig:pl.top.modal.rep} % pl for point line
\end{figure}
%\vspace{-1\baselineskip}

\noindent
In \sectref{sec:exp.top.s.mv.res}, I already outlined the statistical analysis of this experiment with CLMMs.
The final model (repeated as Table \ref{tab:model.exp.top.mv.rep}) contained significant main effects of \textsc{Completeness} ($\chi^2(1) = 18.07, p < 0.001$) and \textsc{Person} ($\chi^2(1) = 10.67, p < 0.01$) and a significant \textsc{Completeness} $\times$ \textsc{Person} interaction ($\chi^2(1) = 9.43, p < 0.01$).
Full forms were preferred over topic drop and utterances with the 1st person singular were preferred over utterances with the 3rd person singular.
Those instances of topic drop that targeted a 3rd person singular constituent were especially degraded.

\begin{table}
\caption{Fixed effects in the final CLMM of experiment \ref*{exp:top.s.mv} (repeated from page \pageref{tab:model.exp.top.mv})}
\centering
\begin{tabular}{lrrrll}
\lsptoprule
Fixed effect & Est. & SE & $\chi^2$ & \textit{p}-value &   \\
\midrule
\textsc{Completeness} & $1.08$ & $0.23$ & $18.07$ & $< 0.001$ & ***\\
\textsc{Person} & $0.61$ & $0.17$ & $10.67$ & $< 0.01$ & **\\
\textsc{Completeness $\times$ Person} & $-0.98$ & $0.29$ & $9.43$ & $< 0.01$ & **\\
\lspbottomrule
\end{tabular}
\label{tab:model.exp.top.mv.rep}
\end{table}

\subsubsubsection{Analysis of experiments \ref*{exp:top.s.fv} and \ref*{exp:top.s.mv}} \is{Lexical verb|(}
In \sectref{sec:exp.top.s.mv.res}, I presented a post hoc analysis of the combined data from experiments \ref*{exp:top.s.fv} and \ref*{exp:top.s.mv}.
The newly added predictor \textsc{Verb Type} was not involved in any significant effect in the final model (repeated as Table \ref{tab:model.exp.top.bv.rep}).
The two-way interaction between \textsc{Verb Type} and \textsc{Completeness} was not significant ($\chi^2(1) = 0.3, p > 0.5$), nor was the three-way interaction between \textsc{Verb Type}, \textsc{Completeness}, and \textsc{Person} ($\chi^2(1) = \chi^2(1) = 0.005, p > 0.9$).%
% Footnote
\footnote{It might be that even when collapsing the two experiments, the power was too low to find such a three-way interaction.
Recall that the studies were not originally designed to find this effect.}
%
The acceptability of topic drop before a syncretic \is{Syncretism} form of a modal verb did not differ significantly from that of topic drop before a distinctly marked form of a lexical verb, \is{Lexical verb} neither in general nor for only the 1st person singular.

As mentioned above, the fixed effects in the final model were similar to those of experiment \ref*{exp:top.s.fv}.
What is relevant to grammatical person is the significant interaction between \textsc{Completeness} and \textsc{Person} ($\chi^2(1) = 22.24, p < 0.001$) and the significant main effect of \textsc{Person} ($\chi^2(1) = 8.52, p < 0.01$).
There was a preference for utterances with a 1st person singular subject, in particular, in the case of topic drop.

\begin{table}
\caption{Fixed effects in the final CLMM of the joint analysis of experiments \ref*{exp:top.s.fv} and \ref*{exp:top.s.mv} (repeated from page \pageref{tab:model.exp.top.bv})}
\centering
\begin{tabular}{lrrrll}
\lsptoprule
Fixed effect & Est. & SE & $\chi^2$ & \textit{p}-value &   \\
\midrule
\textsc{Completeness} & $1.10$ & $0.18$ & $30.41$ & $< 0.001$ & ***\\
\textsc{Person} & $0.38$ & $0.12$ &  $8.52$ & $< 0.01$ & **\\
\textsc{Topicality} & $0.24$ & $0.10$ & $5.73$ & $< 0.05$ & *\\
\textsc{Completeness $\times$ Person} & $-1.02$ & $0.19$ & $22.24$ & $< 0.001$ & ***\\
\textsc{Completeness $\times$ Topicality} & $-0.53$ & $0.20$ & $6.98$ & $< 0.01$ & **\\
\lspbottomrule
\end{tabular}
\label{tab:model.exp.top.bv.rep}
\end{table}

\subsubsection{Discussion}\label{sec:exp.top.s.mv.person.diss}
With experiment \ref*{exp:top.s.mv}, I investigated whether the preference for topic drop of the 1st person singular persists if the verb is not distinctly marked for inflection.
The joint post hoc analysis of experiments \ref*{exp:top.s.fv} and \ref*{exp:top.s.mv} investigated whether a distinct verbal inflection on the verb in the left bracket impacts the acceptability of topic drop.

For grammatical person, experiment \ref*{exp:top.s.mv} replicated the result of experiments \ref*{exp:top.q1} and \ref*{exp:top.s.fv}.
Topic drop of the 1st person singular was preferred over topic drop of the 3rd person singular.
However, experiment \ref*{exp:top.s.mv} showed that this preference is independent of a distinct inflectional ending on the verb following topic drop.
The preference was also present for topic drop before syncretic \is{Syncretism|(} verb forms and, thus, does not hinge on the encoding of the grammatical person through a distinct inflectional verb ending.
This argues against \citeg{auer1993} \textit{inflectional hypothesis}, according to which the 1st person singular is particularly frequently targeted by topic drop because the following verb identifies the omitted subject through its inflectional ending.

Instead, the result supports those \textit{extralinguistic hypotheses} that trace back the preference for topic drop of the 1st person singular to the prominence of the speaker per se and in certain text types.
These hypotheses can be covered by the information-theoretic account according to which differences in likelihood affect how well the prefield constituent can be omitted, as predicted by the \textit{avoid troughs} principle.
In Sections \ref{sec:exp.1sg.2sg} and \ref{sec:exp.1sg.2sg.spoken}, I look more closely at the two different types of \textit{extralinguistic hypotheses} to refine the overall picture.

Concerning verbal inflection, the joint post hoc analysis of experiments \ref*{exp:top.s.fv} and \ref*{exp:top.s.mv} does not provide any evidence for a general influence of a distinct verbal inflection on topic drop.
This is unexpected under the information-theoretic \textit{facilitate recovery} principle, \is{Recoverability} discussed in \sectref{sec:resolving}.
This principle predicts that subject topic drop should be easier to process and, therefore, be more acceptable if it is followed by a distinctly marked verb form.
The inflectional ending is argued to facilitate recovering \is{Recoverability} the omitted subject and this way to reduce the overall processing effort \is{Processing effort} on the verb.
The fact that there was no difference in acceptability between topic drop before syncretic and before distinct verb forms suggests that the distinct inflectional marking does not provide a (substantial) advantage in processing that would be reflected in acceptability. \is{Processing effort}

\is{Ambiguity avoidance|(}
The absence of an effect of the verb type, i.e., the verbal inflection, is at first glance also unexpected from the perspective of ambiguity avoidance.
The study by \citet{soares.etal2019}, discussed in \sectref{sec:usage.ambiguity.studies}, suggested a tendency to avoid ambiguity by using an overt pronoun instead of a covert constituent before a syncretic \is{Syncretism|)} verb form but only if there is a competition between several potential antecedents. \is{Antecedent}
Although such a competition was present in the items of my experiment, the ambiguous utterances with a syncretic form of a modal verb were not degraded.
Unlike in \citeg{soares.etal2019} study, however, the utterances were only locally ambiguous.
As hypothesized already above, it is conceivable that such a local ambiguity does not need to be avoided, since the object pronoun later in the sentence allows the hearer to determine the intended meaning of the utterance with topic drop. \is{Acceptability rating study|)}
\is{Ambiguity avoidance|)} \is{Modal verb|)} \is{Lexical verb|)}
\is{Verbal inflection|)}

\refstepcounter{expcounter}\label{exp:1sg.2sg}
\subsection{Experiment \arabic{expcounter}: grammatical person (1SG vs. 2SG) }
\is{Acceptability rating study|(}\label{sec:exp.1sg.2sg}
The previously discussed experiments \ref*{exp:top.q1}, \ref*{exp:top.s.fv}, and \ref*{exp:top.s.mv} evidenced a preference for topic drop of the 1st person singular subject pronoun over topic drop of a 3rd person singular subject pronoun in line with the production preferences observed in the corpus study in \sectref{sec:frac.td.part.regression.person}.
In this experiment, which was part of the same study as experiment \ref*{exp:ex}, I likewise focused on grammatical person, but this time, I investigated whether there is a difference in acceptability between topic drop of 1st and 2nd person singular subjects.%
% Footnote
\footnote{All items, fillers, and the analysis script are available online: \url{https://osf.io/zh7tr}.}
%
The study had the form of a 2 $\times$ 2 design crossing grammatical \textsc{Person} of the preverbal subject (1st person singular vs. 2nd person singular) and \textsc{Completeness} (full form vs. topic drop).

\subsubsection{Background}
In \sectref{sec:usage.person.theory}, I summarized several approaches under the heading \textit{extralinguistic hypotheses} that attempt to explain why the 1st and partly also the 2nd person singular are particularly frequently omitted.
I distinguished between two types of these hypotheses: \textit{extralinguistic 1SG} and \textit{extralinguistic 1SG+2SG}.
The central representatives of the \textit{extralinguistic 1SG hypotheses} are \citet{imo2013,imo2014} and \citet{volodina.onea2012}.
\citet[153--154]{imo2014} claims that the 1st person singular is especially easy to recover \is{Recoverability} because the speaker it refers to is part of the default origo.
\citet[218]{volodina.onea2012} attribute the more frequent omission of the 1st person singular, in particular in certain text types, to knowledge about these text types and their typical speakers or writers.
While \citet{imo2013, imo2014} and \citet{volodina.onea2012} focus on the 1st person, the \textit{extralinguistic 1SG+2SG hypotheses} also take the 2nd person into account.
For instance, the IDS grammar \citep[414]{zifonun.etal1997} argues that both the 1st and the 2nd person can be omitted well because in a shared speech situation, the roles of speaker and hearer, to which the corresponding pronouns refer, are clearly established.
A similar claim is made by \citet[47]{ariel1990}, who points out that cross\hyp linguistically null pronouns for the 1st and the 2nd person are usually more common than null pronouns for the 3rd person.
Thus, it turns out that one half of the \textit{extralinguistic hypotheses}, \textit{extralinguistic 1SG}, (implicitly) predicts that the omission of the 1st person should have an advantage over the omission of the 2nd person because the 1st person refers to the origo or to the speaker/writer, who is particularly prominent in some text types.
The other half, \textit{extralinguistic 1SG+2SG}, (explicitly) predicts that both persons should behave comparably since they refer to the constitutive components of any communicative situation, speaker and hearer.

From the quantitative corpus \is{Corpus} studies by \citet{androutsopoulos.schmidt2002}, \citet{frick2017}, and myself (\sectref{sec:factor.person.corpus}) a mixed picture emerges.
While in all three studies, the 1st person singular was most often targeted by topic drop (from an omission rate of 60\% in \citet{androutsopoulos.schmidt2002} and \citet{frick2017} to a rate of 67\% in my corpus study of \textsc{FraC-TD-SMS}), there were greater differences for the 2nd person singular.
In \citeg{frick2017} corpus, \is{Corpus} the omission rate of the 2nd person singular was the second highest with about 47\%, but the 2nd person singular was omitted only in about 35\% of the cases in my data set \textsc{FraC-TD-SMS} (see \sectref{sec:frac.td.sms.person}) and in about 26\% of the cases in the corpus of \citet{androutsopoulos.schmidt2002}.
Consequently, the corpus \is{Corpus} frequencies suggest that topic drop of the 1st person singular and topic drop of the 2nd person singular do not occur with equal absolute and relative frequency, supporting the first half of the \textit{extralinguistic hypotheses}.

Experiment \arabic{expcounter} addressed the question of whether these production preferences are also reflected in acceptability.
If this were the case, there should be a significant interaction between the predictors \textsc{Completeness} and grammatical \textsc{Person}, according to which topic drop of the 1st person singular is more acceptable than topic drop of the 2nd person singular.
This would also be predicted by an information-theoretic approach that relies mainly on frequencies to predict differences in likelihood.
In the other case, and if the \textit{extralinguistic 1SG+2SG hypotheses} are correct, there should be no such interaction, i.e., topic drop of the 1st person singular and topic drop of the 2nd person singular should be comparably acceptable.
This could be captured by the information-theoretic account if not only plain corpus \is{Corpus} frequencies are considered but also the predictability through the extralinguistic context and its impact on recoverability. \is{Recoverability}
For instance, the knowledge about the roles of the speaker and hearer allows for easily identifying and recovering the reference of an omitted 2nd person subject pronoun although \textit{du} as a token may be less frequent in a corpus than \textit{ich}.

\subsubsection{Materials}
\subsubsubsection*{Items}
I constructed 24 items such as \ref{ex:1sg.2sg.item}, which consisted of a context utterance and a target utterance, both texted by the same person.
The target utterance always contained a modal verb in the left bracket, which was equally balanced between \textit{können} (`can'), \textit{sollen} (`shall'), and \textit{müssen} (`must') and which had an inflectional ending that indicated the grammatical person.
The utterance was either a statement about the speaker with a preverbal (realized or omitted) 1st person singular subject \ref{ex:1sg.2sg.item.1sg} or a request to the hearer with a preverbal (realized or omitted) 2nd person singular subject \ref{ex:1sg.2sg.item.2sg}.
The context utterance was designed to make both options, i.e., the statement about the speaker and the request or advice to the hearer, as equally natural as possible.%
% Footnote
\footnote{A possible difference in naturalness between the \textsc{Person} conditions would also show up in the full forms that function as a baseline and, thus, would be captured by a main effect of the predictor \textsc{Person}.}
%

\exg.\label{ex:1sg.2sg.item}Papa kriegt es scheinbar nicht gebacken, seinen neuen Laptop einzurichten\makebox[0pt][l]{\vspace*{-2cm}\raisebox{-0.35ex}{\includegraphics[scale=0.07]{Emojis/woman-facepalming_wa.png}}}\\
dad gets it apparently not baked his new laptop set.up\\
`Dad can't seem to get it together to set up his new laptop\makebox[0pt][l]{\vspace*{-2cm}\raisebox{-0.35ex}{\includegraphics[scale=0.07]{Emojis/woman-facepalming_wa.png}}}\phantom{m}'
\ag.\label{ex:1sg.2sg.item.1sg}(Ich) sollte ihn besser mal anrufen\\
I should him better \textsc{part} call\\
`(I) better give him a call' \hfill (1SG)
\bg.\label{ex:1sg.2sg.item.2sg}(Du) solltest ihn besser mal anrufen\\
you should him better \textsc{part} call\\
`(You) better give him a call' \hfill (2SG)

\subsubsubsection*{Fillers}
As described for experiment \ref*{exp:ex} in \sectref{sec:exp.ex.materials}, I included 80 fillers:
24 items with an over or covert expletive subject in the prefield, 24 gapping or right node raising structures and corresponding full forms, 24 sluicing or sprouting structures and corresponding full forms, and 8 ungrammatical catch trials.

\subsubsection{Procedure}
The procedure is described in \sectref{sec:exp.ex.procedure} for experiment \ref*{exp:ex}, which was part of the same study.

\subsubsection{Results}
As described in \sectref{sec:data.analysis}, the data from 42 participants could be used for the analysis.
Table \ref{tab:1sg.2sg.ratings} shows the mean ratings and standard deviations per condition, which are plotted with 95\% confidence intervals in Figure \ref{fig:pl.1sg.2sg}.
The visual inspection shows no differences between conditions.

\begin{table}
\caption{Mean ratings and standard deviations per condition for experiment \arabic{expcounter}}
\centering
\begin{tabular}{llrr}
\lsptoprule
\multicolumn{1}{c}{\textsc{Completeness}} & \multicolumn{1}{c}{\textsc{Person}} & \multicolumn{1}{c}{\Centerstack{Mean \\ rating}} & \multicolumn{1}{c}{\Centerstack{Standard\\ deviation}}\\
\midrule
Full form & 1SG & $5.90$ & $1.21$ \\
Topic drop & 1SG & $5.85$ & $1.29$ \\
Full form & 2SG & $5.95$ & $1.23$ \\
Topic drop & 2SG & $5.94$ & $1.26$ \\
\lspbottomrule
\end{tabular}
\label{tab:1sg.2sg.ratings}
\end{table}

\begin{figure}
\centering
\includegraphics[scale=1]{Experimenteplots/PL_1SG2SG_WA.pdf}
\caption{Mean ratings and 95\% confidence intervals per condition for experiment \arabic{expcounter}}
\label{fig:pl.1sg.2sg} % pl for point line
\end{figure}

I analyzed the responses with CLMMs from the package ordinal \citep{christensen2019}, following the procedure described in \sectref{sec:data.analysis}.
I modeled the ratings as a function of the binary predictors \textsc{Completeness} and \textsc{Person}, coded using deviation coding as $0.5$ (full form, 1SG) and $-0.5$ (topic drop, 2SG), as well as of the numeric scaled and centered \textsc{Position} of the trial in the experiment and included all two-way interactions between the independent variables.
The maximal random effects structure was included.%
% Footnote model call
\footnote{The formula of the full model was as follows: \texttt{Ratings \textasciitilde ~(\textsc{Completeness} + \textsc{Person} + \textsc{Position})\textasciicircum2 + (1 + (\textsc{Completeness} + \textsc{Person} + \textsc{Position})\textasciicircum2 | Subjects) + (1 + (\textsc{Completeness} + \textsc{Person} + \textsc{Position})\textasciicircum 2 | Items).}}
%

The final model obtained with a backward model selection did not contain any significant fixed effects.
Neither the interaction between \textsc{Completeness} and \textsc{Person} was significant ($\chi^2(1) = 0.57, p > 0.4$), nor the main effect of \textsc{Completeness} ($\chi^2 (1) = 0.01, p > 0.9$), nor any of the \textsc{Position} effects.
There was a marginally significant main effect of \textsc{Person} ($\chi^2(1) = 3.65, p > 0.05$) indicating that there was a general tendency for utterances with the 2nd person singular as the (realized or omitted) preverbal subject to be preferred over those utterances with the 1st person singular.

\subsubsection{Discussion}
Experiment \arabic{expcounter} had the purpose of comparing 1st and 2nd person singular topic drop in appropriate contexts.
Its results indicate that they are both equally acceptable in such contexts.
This is somewhat unexpected given the corpus results by \citet{androutsopoulos.schmidt2002}, \citet{frick2017}, and myself (\sectref{sec:factor.person.corpus}).
Apparently, the production preferences are not fully reflected in the perception preferences.
Instead, the results are in line with the \textit{extralinguistic 1SG+2SG hypotheses} as proposed by \citet{zifonun.etal1997} and \citet{ariel1990}, which rely on the fact that the speaker and hearer are constitutive parts of any communicative situation.
Therefore, the pronouns of the 1st and 2nd person singular referring to these roles are easily recoverable,\is{Recoverability|(} and to the same degree.
This result also suggests that the information-theoretic account should not rely on frequencies alone but also consider further influences on the likelihood such as the presence of the speaker and hearer in virtually any communicative situation.
In the next section, I present a replication of this experiment in a different text type. \is{Acceptability rating study|)}

\refstepcounter{expcounter}\label{exp:1sg.2sg.spoken}
\subsection{Experiment \arabic{expcounter}: grammatical person (1SG vs. 2SG, dialogues) }
\is{Acceptability rating study|(}\label{sec:exp.1sg.2sg.spoken}
Experiment \arabic{expcounter} replicated experiment \ref*{exp:1sg.2sg} by also investigating whether there is a difference in acceptability between topic drop of the 1st and the 2nd person singular but in another text type \is{Text type} or rather using a different form of presentation.%
% Footnote
\footnote{The materials and the analysis script can be accessed online: \url{https://osf.io/zh7tr}.}
%
Instead of showing the critical utterance as an instant message, it was presented in quotation marks as direct speech.
In this way, I wanted to validate the reliability of the result of the previous experiment.
At the same time, this experiment allows for an exploratory look at possible differences between text types \is{Text type} with regard to topic drop, as a continuation of the considerations made in \sectref{sec:corpus.texttype}.
This mainly concerns the question of whether topic drop is generally perceived as less natural in spoken language than in instant messages.
I do not expect any difference between text types \is{Text type} for grammatical person because, although the communicative situation is different in talking face-to-face vs. texting, the roles of speaker or writer and hearer or reader are equally fixed and unique.
The pronouns \textit{ich} (`I') and \textit{du} (`you') should therefore be equally predictable and recoverable \is{Recoverability|)} and topic drop should be equally acceptable.

Like experiment \ref*{exp:1sg.2sg}, experiment \arabic{expcounter} had the form of a 2 $\times$ 2 design crossing grammatical \textsc{Person} of the preverbal constituent (1st person singular vs. 2nd person singular) and \textsc{Completeness} (full form vs. topic drop).
I expected to replicate the result of experiment \ref*{exp:1sg.2sg} that topic drop of the 1st and the 2nd person singular is equally acceptable.

\subsubsection{Materials}
\subsubsubsection*{Items}
The 24 items used in this experiment were adaptions of the 24 items of experiment \ref*{exp:1sg.2sg}, as shown in \ref{ex:1sg.2sg.item.dialogue}.
Instead of two instant messages, they consisted of a context story and the target utterance, which was presented as direct speech.
This target utterance was slightly adapted compared to experiment \ref*{exp:1sg.2sg} to better fit with the context.
Each context story was three sentences long and introduced two characters.
One of these characters produced the target utterance, which was introduced by a reporting clause such as \textit{Jessica sagt} (`Jessica says'). 
The target utterance itself was enclosed in quotation marks to mark it as direct speech.
Like in experiment \ref*{exp:1sg.2sg}, this target utterance always contained a modal verb in the left bracket and was either a statement about the speaker (1SG conditions) or a request to the hearer (2SG conditions).

\sloppy
\ex.\label{ex:1sg.2sg.item.dialogue}
Marcel und Jessica kochen gemeinsam Ratatouille in Marcels Küche. Während sie das Gemüse klein schneiden, tauschen sie Neuigkeiten über ihren Freund Sven aus. Sie unterhalten sich darüber, dass Sven zögert, sich auf eine interessante Stelle zu bewerben.	Jessica meint:
\\
`Marcel and Jessica cook ratatouille together in Marcel's kitchen. While they chop the vegetables, they exchange news about their friend Sven. They talk about how Sven is hesitant to apply for an interesting job. Jessica says:'
\ag. ,,Ich muss ihn überreden.''\\
I must.\textsc{1sg} him persuade\\
`I must persuade him.' \hfill (full form, 1SG)
\bg. ,,Muss ihn überreden.''\\
must.\textsc{1sg} him persuade\\
`Must persuade him.' \hfill (topic drop, 1SG)
\cg. ,,Du musst ihn überreden.''\\
you.\textsc{2sg} must.\textsc{2sg} him persuade\\
`You must persuade him.' \hfill (full form, 2SG)
\dg. ,,Musst ihn überreden.''\\
must.\textsc{2sg} him persuade\\
`Must persuade him.' \hfill (topic drop, 2SG)

\fussy

\subsubsubsection*{Fillers}
I tested 72 fillers together with the 24 items.
Those fillers contained 24 items of an experiment on the verb \textit{brauchen} (`need'), 24 items of an experiment on preposition omission in short answer fragments, and 24 items on predictable vs. unpredictable fragment answers vs. full answers.
I included 8 ungrammatical fillers with two finite verbs in the left bracket as attention checks to be able to exclude inattentive participants.

\subsubsection{Procedure}
48 self-reported native speakers of German between the ages of 18 and 40 who had not taken part in any previous experiment on topic drop participated in the study.
They were recruited from \emph{Clickworker} and received a compensation of €4.00.
The experiment was presented online with LimeSurvey \citep{limesurveygmbh}.
The task of the participants was to rate the naturalness of the final utterance on a 7-point Likert scale (7 = completely natural), thereby considering the context story.
The 24 items were distributed across 4 lists according to a Latin square design so that each participant saw each token set only once and in one condition.
I mixed the items with the fillers and presented them to the participants in individual pseudo-randomized order.

\subsubsection{Results}
The data from 6 participants who had rated more than 4 of the 8 ungrammatical fillers with 6 or 7 points on the 7-point Likert scale, i.e., as completely natural or as almost completely natural, were excluded from the data set.
Table \ref{tab:1sg.2sg.spoken.ratings} shows the mean ratings and standard deviations per condition.
The mean ratings are also plotted with 95\% confidence intervals in Figure \ref{fig:pl.1sg.2sg.spoken}.
Unlike in experiment \ref*{exp:1sg.2sg}, there seems to be a difference between the two \textsc{Completeness} conditions.
Full forms received higher ratings than utterances with topic drop.
Additionally, there seems to be a tendency that the two \textsc{Completeness} conditions differed more strongly for the 1st than for the 2nd person singular.

\begin{table}
\caption{Mean ratings and standard deviations per condition for experiment \arabic{expcounter}}
\centering
\begin{tabular}{llrr}
\lsptoprule
\multicolumn{1}{c}{\textsc{Completeness}} & \multicolumn{1}{c}{\textsc{Person}} & \multicolumn{1}{c}{\Centerstack{Mean\\ rating}} & \multicolumn{1}{c}{\Centerstack{Standard\\ deviation}}\\
\midrule
Full form & 1SG & $5.92$ & $1.35$ \\
Topic drop & 1SG & $4.46$ & $1.87$ \\
Full form & 2SG & $5.85$ & $1.53$ \\
Topic drop & 2SG & $4.68$ & $1.85$ \\
\lspbottomrule
\end{tabular}
\label{tab:1sg.2sg.spoken.ratings}
\end{table}

\begin{figure}
\centering
\includegraphics[scale=1]{Experimenteplots/PL_1SG2SG.pdf}
\caption{Mean ratings and 95\% confidence intervals per condition for experiment \arabic{expcounter}}
\label{fig:pl.1sg.2sg.spoken} % pl for point line
\end{figure}

For the inferential statistics, I analyzed the responses of the remaining 42 participants with CLMMs from the package ordinal \citep{christensen2019} in R \citep{rcoreteam2021}, as described in \sectref{sec:data.analysis}.
I modeled the ratings as a function of the binary predictors \textsc{Completeness} and \textsc{Person}, as well as of the numeric scaled and centered \textsc{Position} of the trial in the experiment, and included all two-way interactions between the independent variables.
I coded \textsc{Completeness} and \textsc{Person} using deviation coding.
Full forms and the 1st person singular were coded as $0.5$, topic drop and the 2nd person singular as $-0.5$ respectively.
The random effects consisted of random intercepts for participants and items and of by-participant and by-item random slopes for all predictors and their two-way interactions.%
% Footnote model call
\footnote{The formula of the full model was as follows: \texttt{Ratings \textasciitilde~(\textsc{Completeness} + \textsc{Person} + \textsc{Position})\textasciicircum 2 + (1 + (\textsc{Completeness} + \textsc{Person} + \textsc{Position})\textasciicircum 2 | Subjects) + (1 + (\textsc{Completeness} + \textsc{Person} + \textsc{Position})\textasciicircum 2 | Items).}}
%
Table \ref{tab:1sg.2sg.model.spoken} summarizes the fixed effects in the final model with symmetric thresholds.

\begin{table}
\caption{Fixed effects in the final CLMM of experiment \arabic{expcounter}}
\centering
\begin{tabular}{lrrrll}
\lsptoprule
Fixed effect & Est. & SE & $\chi^2$ & \textit{p}-value &   \\
\midrule
\textsc{Completeness} & $2.09$ & $0.25$ & $43.71$ & $< 0.001$ & ***\\
\lspbottomrule
\end{tabular}
\label{tab:1sg.2sg.model.spoken}
\end{table}

\noindent
There was only a significant main effect of \textsc{Completeness} ($\chi^2(1) = 43.71, p < 0.001$), according to which participants preferred full forms over topic drop.
The interaction between \textsc{Completeness} and \textsc{Person} was not significant ($\chi^2(1) = 0.42, p > 0.5$), which indicates that topic drop of the 2nd person singular was not perceived as degraded compared to topic drop of the 1st person singular (or vice versa).
The main effect of \textsc{Person} was not significant either ($\chi^2(1) = 1.55, p > 0.2$).
Utterances with the 1st and 2nd person singular were rated equally acceptable, which indicates that the contexts made both readings of the target utterance equally possible.

\subsubsection{Discussion}
Experiment \arabic{expcounter} compared 1st and 2nd person singular topic drop in appropriate contexts and replicated the results of experiment \ref*{exp:1sg.2sg}.
Topic drop of the 1st and the 2nd person singular were equally acceptable.
This result is in line with the \textit{extralinguistic 1SG+2SG hypotheses}:
If the communicative situation makes it obvious how the roles of speaker and hearer are distributed, the pronouns referring to each of them can be omitted because the recovery \is{Recoverability} is easy.
This seems to hold for communication via instant messaging but also for spoken utterances in face-to-face conversations (at least if presented in written form).

Unlike in experiment \ref*{exp:1sg.2sg}, topic drop received generally worse ratings than the full forms in this experiment.
This may be due to the presentation as direct speech.
As I showed in \sectref{sec:corpus.texttype}, at least in the FraC, topic drop is less frequent in dialogues (omission rate of 16\%) than in text messages (omission rate of almost 64\%), the direct ancestors of instant messages.
Furthermore, while in text messages the 1st person singular subject pronoun \textit{ich} is most frequently targeted by topic drop (264 instances, 66.84\% of all omissions), this is less so in dialogues (11 instances, 12.94\% of all omissions).
In dialogues, it is mainly the subject and object demonstrative pronoun \textit{das} that is omitted (50 instances, 58.82\%), which in turn is less often covert in text messages (19 instances, 5.44\%).
Consequently, the degraded ratings for topic drop in this experiment may be caused by the fact that (i) topic drop is generally less frequent in dialogues and (ii) that topic drop of the 1st (and probably also the 2nd) person singular is even less common.
In summary, the two experiments \ref*{exp:1sg.2sg} and \arabic{expcounter} provide first experimental evidence that topic drop occurs preferably in certain text types, \is{Text type} which needs to be validated and extended in future research. \is{Acceptability rating study|)}

\section{Summary: grammatical person, verbal inflection, and ambiguity avoidance}
In \sectref{sec:usage.person.theory}, I discussed in detail that according to the previous literature \citep[e.g.,][]{tesak.dittmann1991, volodina.onea2012, imo2013, imo2014}, 1st person (singular) subjects are particularly often targeted by topic drop.
This observation was confirmed through corpus studies, anecdotally by \citet{auer1993, doring2002} and quantitatively by \citet{androutsopoulos.schmidt2002} and \citet{frick2017} (see \sectref{sec:usage.person.studies}), as well as by my own corpus study in Section  \ref{sec:factor.person.corpus}.
Both in the complete FraC and the text message subcorpus, the 1st person singular has the highest omission rate.

The acceptability rating studies, experiments \ref*{exp:top.q1}, \ref*{exp:top.s.fv}, and \ref*{exp:top.s.mv}, presented in this chapter consistently confirmed the pattern also experimentally.
The observed production preference was reflected in an acceptability difference, i.e., in perception.
In all three experiments, topic drop of the 1st person singular was significantly preferred over topic drop of the 3rd person singular.
It has to be noted, though, that in these studies, the 3rd person singular always referred to a human and was, thus, of either feminine or masculine gender, while my corpus study suggests that topic drop of the 3rd person singular is most frequent for the neuter gender.
It could be that for non-human 3rd person singular subjects of neuter gender, the effect would be less pronounced.
This should be tested in a future study.

Also in \sectref{sec:usage.person.theory}, I presented potential explanations for the preference to omit the 1st person (singular) from the theoretical literature.
On the one hand, there is \citeg{auer1993} \textit{inflectional hypothesis}, which traces back the preference to a distinct inflectional ending \is{Verbal inflection|(} on the verb following topic drop.
On the other hand, there are \textit{extralinguistic hypotheses} of two variants.
First, authors such as \citet{imo2014} and \citet{volodina.onea2012} argue that the 1st person singular can be omitted well because it refers to the speaker, who is easily recoverable \is{Recoverability} as being part of the origo of speaking and through text type knowledge (\textit{extralinguistic 1SG}).
Second, \citet{zifonun.etal1997} and \citet{ariel1990} propose that both speaker and hearer are easily recoverable \is{Recoverability} as they are essential parts of every communicative situation.
Therefore, the omission of both 1st and 2nd person (singular) should work well (\textit{extralinguistic 1SG+2SG}).
The results of experiment \ref*{exp:top.s.mv} speak against \citeg{auer1993} \textit{inflectional hypothesis} because they evidenced that the preference for topic drop of the 1st person singular persists even before syncretic \is{Syncretism} verb forms.
The two experiments \ref*{exp:1sg.2sg} and \ref*{exp:1sg.2sg.spoken} provided evidence for the \textit{extralinguistic 1SG+2SG hypotheses}.
Topic drop of the 1st person singular and the 2nd person singular are comparable in their acceptability, which suggests that both speaker and hearer can be equally well recovered. \is{Recoverability}

As discussed in \sectref{sec:info.theory.person}, my information-theoretic account of topic drop usage explains the effect of grammatical person by the \textit{avoid troughs} principle.
According to this principle (see \sectref{sec:avoid.troughs}), speakers should omit predictable expressions to distribute information more uniformly.
In the case of the 1st (and the 2nd) person singular, this predictability \is{Predictability} seems to be impacted by at least the following two factors:
(i) the frequency of the respective overt pronoun in certain text types and (ii) the likelihood of the referent in a communicative situation.
(i) Corpus studies \is{Corpus} have evidenced that, in particular in text messages, the 1st person singular subject pronoun \textit{ich} (`I') is very frequent in sentence-initial position, which makes it predictable there and likely to be omitted. \is{Predictability}
(ii) Speaker and hearer are necessary parts of every communicative situation.
This makes the 1st and 2nd person (singular) pronouns, which refer to the speaker and hearer, predictable \is{Predictability} and easier to omit.
Given the results of experiments \ref*{exp:1sg.2sg} and \ref*{exp:1sg.2sg.spoken}, which did not indicate an acceptability difference between 1st and 2nd person, it is reasonable to assume that factor (ii) has priority over factor (i).
Accordingly, the likelihood of a referring expressing may be more strongly impacted by the presence of its referent in a situation than by its pure frequency in a text type.
In summary, the results for grammatical person are consistent with the information-theoretic account and can be explained by the \textit{avoid troughs} principle.

As discussed above, \citet{auer1993} and, following him also \citet{imo2013, imo2014}, explain the high frequency of topic drop of the 1st person singular subject pronoun \textit{ich} with the easy recoverability \is{Recoverability} through the distinct inflectional marking on the verb following topic drop.
They argue that a subject can be easily omitted if the grammatical person is also expressed through the inflectional ending at the congruent verb.
While they only use this argument to explain the high frequency of the 1st person singular, this reasoning should apply in principle to every grammatical person for which the following verb is distinctly marked.
In \sectref{sec:resolving}, I proposed to explain the potential impact of verbal inflection as the consequence of trying to reduce the processing effort \is{Processing effort} on this verb, as predicted by the \textit{facilitate recovery} principle. \is{Recoverability}
Since a distinct inflectional ending indeed indicates the grammatical person of the omitted congruent subject, this ending can reduce the overall effort \is{Processing effort} associated with ellipsis resolution, i.e., determining the reference of the omitted constituent.
I furthermore argued that this reduction of processing effort \is{Processing effort} could be stronger for the 1st person singular than for the 3rd person singular.
If the hearer identifies the omitted subject as 1st person singular, they can  automatically conclude that the speaker is the intended referent, while for the 3rd person singular there are still, at least theoretically, always several potential referents present in the context. 

First support for this line of reasoning came from the corpus study, \is{Corpus} where I found that topic drop of the 1st person singular was indeed more frequent before distinctly marked verb forms.
Experiments \ref*{exp:top.s.fv} and \ref*{exp:top.s.mv}, however, did not provide evidence in this direction.
Topic drop of the 1st person singular was preferred to an equal degree over topic drop of the 3rd person singular, regardless of whether the following verb was distinctly marked for inflection or not.
There was no effect of a distinct vs. a syncretic \is{Syncretism} verb form at all in the joint analysis of both studies.
Thus, the result remains inconclusive concerning verbal inflection. \is{Verbal inflection|)}

\is{Ambiguity avoidance|(}
Also from the perspective of ambiguity avoidance, it is predicted that syncretisms \is{Syncretism} should be relevant to topic drop.
Utterances with subject topic drop followed by a syncretic verb form are ambiguous and should be avoided in order not to cause processing difficulties for the hearer. \is{Processing effort}
I only found partial evidence for this hypothesis in the corpus study but not in the experiments.
The corpus \is{Corpus} study suggests that topic drop of the 1st person singular but not of the 3rd person singular is rarer before verbs that are perceived as ambiguous.
In experiments \ref*{exp:top.s.fv} and \ref*{exp:top.s.mv}, there was no difference in acceptability between the utterances in which topic drop was followed by a distinct verb form and those in which the verb form was syncretic \is{Syncretism} and, thus, ambiguous. 
This cannot yet be taken as evidence against ambiguity avoidance because the object pronoun referring to the competing referent made the utterances with topic drop preceded by syncretic \is{Syncretism} verb forms only locally ambiguous, not globally.
The investigation of globally ambiguous topic drop structures needs to be left to future research (but see the attempt discussed in Footnote \ref{note:ambiguity} on page \pageref{note:ambiguity}).
In the next chapter, I discuss two further verb-related influencing factors: verb type and verb surprisal.
\is{Ambiguity avoidance|)}
