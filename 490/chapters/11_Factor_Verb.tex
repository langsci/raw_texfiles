\chapter{Verb type and verb surprisal}\label{ch:usage.verb}
After having reviewed, tested, and discarded \citeg{auer1993} \textit{inflectional hypothesis}, according to which a distinct inflectional marking on the verb following topic drop can facilitate the recovery of omitted 1st person subject pronouns, in Chapter \ref{ch:usage.person}, I turn to two further verb-related factors in this chapter: verb type and verb surprisal.
While verb type has been discussed at least briefly in the theoretical literature as a factor relevant to topic drop and has been examined in corpus studies, the same is not true for verb surprisal.
An influence of verb surprisal is not predicted by any hypotheses from the theoretical literature but exclusively by the information-theoretic approach presented in this book (see Sections \ref{sec:avoid.peaks} and \ref{sec:resolving}).

The structure of this chapter is as follows:
For verb type, I give a brief theoretical overview and discuss previous studies.
Then, I turn to the information-theoretic predictions for both factors.
Finally, I present, first, my corpus results and, second, my experimental results, which partially evidence an impact of verb type and verb surprisal on topic drop.

\section{Theoretical overview of verb type}\label{sec:usage.verb.type.theory} \is{Copula|(} \is{Auxiliary|(} \is{Modal verb|(} \is{Lexical verb|(}
In the theoretical literature on topic drop, some authors claim that it occurs preferably before certain verb types.
However, they do not provide any explanation or theoretical account of why there should be an effect of the type of verb following topic drop at all.
According to the IDS grammar \citep[415]{zifonun.etal1997}, subject topic drop of the 1st and the 2nd person, which Zifonun et al. term ``person ellipsis'', is particularly frequent before copular, auxiliary, and modal verbs.
They argue that topic drop of these grammatical persons before finite lexical verbs is more marked in terms of style \citep[415]{zifonun.etal1997}.
However, they do not justify these claims nor do they substantiate them empirically or even with introspective data.
Also without empirical validation, \citet[297]{imo2013}  claims that topic drop occurs particularly frequently before verba dicendi and verba sentiendi, such as \textit{wissen} (`to know'), \textit{finden} (`to find'), or \textit{glauben} (`to believe').

\section{Previous empirical evidence regarding verb type}\label{sec:usage.verb.type.studies}
Although a sophisticated account of verb type effects is pending, some corpus studies have investigated them. 
I summarize the results of four studies \citep{poitou1993,androutsopoulos.schmidt2002,helmer2016,frick2017}, which generally suggest that the omission rates for modal, auxiliary, and copular verbs are higher than for lexical verbs, while there is only sparse evidence for a higher frequency of topic drop before verba dicendi and verba sentiendi.
The most detailed study is again that by \citet{frick2017}, who not only has the most data but also distinguishes the largest number of verb categories and even briefly discusses the interplay between verb type and grammatical person.

\subsection{\citet{poitou1993}} 
\is{Corpus|(}
In his corpus study (for details see \sectref{sec:usage.function.studies}), \citet{poitou1993} observes that topic drop is almost always followed by modal verbs and auxiliaries, which form a verbal complex with the element(s) in the right bracket, but he does not support this claim with count data.

\subsection{\citet{androutsopoulos.schmidt2002}}
\citet{androutsopoulos.schmidt2002} report the frequency and the omission rates of verbs that follow topic drop in their text message corpus (see \sectref{sec:usage.person.studies} for details on their data set).
Table \ref{tab:verb.type.as} shows both figures for verb lemmas with more than 10 occurrences in the corpus.%
%% Footnote
\footnote{It is not clear from their paper why they drew the line at 10 occurrences and whether they considered only subjects or also objects for this table.}
%

\begin{table}
\caption[Verb frequency in \citeg{androutsopoulos.schmidt2002} corpus]{Full forms, instances of topic drop, and omission rates as a function of the following verb lemma for lemmas with more than 10 occurrences in the text message corpus of \citet{androutsopoulos.schmidt2002}, taken from \citet[69]{androutsopoulos.schmidt2002}}
\centering
\begin{tabular}{lrrrr}
\lsptoprule
\multicolumn{1}{c}{Verb} & \multicolumn{1}{c}{Full form} & \multicolumn{1}{c}{Topic drop} & \multicolumn{1}{c}{Total} & \multicolumn{1}{c}{Omission rate} \\
\midrule
\textit{sein} (`to be') & $51$ & $76$ & $127$ & $59.84\%$ \\
\textit{haben} (`to have') & $34$ & $44$ & $78$ & $56.41\%$ \\
\textit{wollen} (`to want') & $5$ & $12$ & $17$ & $70.59\%$ \\
\textit{gehen} (`to go') & $10$ & $7$ & $17$ &  $41.18\%$ \\
\textit{müssen} (`must') & $3$ & $11$ & $14$ & $78.57\%$ \\
\textit{können} (`can') & $3$ & $10$ & $13$ & $76.92\%$ \\
\textit{sitzen} (`to sit') & $5$ & $6$  & $11$ & $54.55\%$ \\
\textit{kommen} (`to come') & $6$ & $5$ & $11$ & $45.45\%$ \\
\lspbottomrule
\end{tabular}
\label{tab:verb.type.as}
\end{table}

The verbs with the highest omission rates in the corpus of \citet{androutsopoulos.schmidt2002} are the modal verbs \textit{müssen} (`must'), \textit{können} (`can'), and \textit{wollen} (`to want') with over 70\%.
The  verbs \textit{sein} (`to be') and \textit{haben} (`to have'), which are auxiliary verbs but can also function as a copular verb (\textit{sein}) or a lexical verb (\textit{haben}), have significantly lower omission rates of about 55\% to 60\% ($\chi^2(1) = 4.25, p < 0.05$).%
%% footnote
\footnote{I compared the summed frequencies of the modal verbs (\textit{müssen}, \textit{können}, and \textit{wollen}) and the summed frequencies of the two potential auxiliary verbs with a Pearson's chi-squared test with Yates's continuity correction in R \citep{rcoreteam2021}.}
%
The lexical verb \textit{sitzen} (`to sit') has a similar omission rate of just under 55\%, while the omission rates of the remaining two lexical verbs \textit{gehen} (`to go')  and \textit{kommen} (`to come') are even lower with about 40\% to 45\%.
The rates are on average significantly lower than those of the modal verbs ($\chi^2(1) = 6.89, p < 0.01$), but there is no significant difference from the potential auxiliaries ($\chi^2(1) = 1.57, p > 0.2$).%
%% Footnote
\footnote{Again, I used Pearson's chi-squared tests with Yates's continuity correction in R \citep{rcoreteam2021} to compare the summed frequencies between the verb types.
But consider that the numbers of occurrences for all verbs except \textit{sein} and \textit{haben} in \citeg{androutsopoulos.schmidt2002} data set are very small, resulting in potentially unreliable statistical results even after summing up.}
%
\citet{androutsopoulos.schmidt2002} do not provide a breakdown by grammatical person, but most of the instances in their corpus are 1st person singular pronouns (see \sectref{sec:usage.person.studies}).
Therefore, it is possible to interpret the data at least tentatively in terms of the claim by the IDS grammar \citep{zifonun.etal1997} that copular, auxiliary, and modal verbs most often follow topic drop of the 1st and the 2nd person.
For the modal verbs, it turns out that, at least based on the data in Table \ref{tab:verb.type.as},%
%% footnote
\footnote{It should be noted that \citet{androutsopoulos.schmidt2002} do not present the whole data.
It might be possible that there are verb lemmas with fewer than 10 occurrences in their corpus that behave differently.
For example, there could be lexical verbs with a higher omission rate than the lexical verbs presented in the table, or, in turn, it could be that the remaining modal verbs have lower omission rates.
Thus, the implications discussed in the following should be taken with caution.}
%
they do indeed follow topic drop (especially of the 1st person singular) particularly frequently, to be more precise, significantly more frequently than the listed lexical verbs.
This is consistent with the claim by the IDS grammar.
At the same time, however, their omission rates are also significantly higher than those of the potential auxiliaries, which, in turn, contrary to the prediction by the IDS grammar, are not significantly different from the lexical verbs in the table.
Thus, overall, we find at most partial evidence for the claim by the IDS grammar.

In the data by \citet{androutsopoulos.schmidt2002}, there is no evidence that topic drop occurs preferably before verba dicendi or verba sentiendi, as predicted by \citet{imo2013}.
There is no corresponding verb of saying or thinking in Table \ref{tab:verb.type.as}.

\subsection{\citet{helmer2016}} 
As sketched in \sectref{sec:prefield.detail.theory}, \citet{helmer2016} investigated 541 instances of topic drop in the corpora of spoken conversations FOLK and GIF.
She compared them to a random sample of 200 syntactically complete utterances from the FOLK, each of which contains the pronoun \textit{das} (`that') in subject or object function.%
%% Footnote
\footnote{Recall that \citet{helmer2016} considers topic drop to be possible in the middle field as well.
For this reason, her reference data also contain 29 full forms with \textit{das} in the middle field, i.e., after the finite verb in the left bracket, and 10 for which the topological position could either not be determined or that occurred in the postfield \citep[214]{helmer2016}.}
%
She argues that utterances with what she refers to as the ``unspecific'' anaphor \textit{das} are similar to utterances with topic drop in their semantic vagueness.
In her view, this allows for a comparison between overt and covert variants independently of the semantic content that a more specific anaphor would have \citep[197]{helmer2016}.

\citet{helmer2016} presents the frequency of topic drop and the full forms with \textit{das} with the four biggest%
%% Footnote
\footnote{She does not provide a complete list of all verb classes that she assumes.}
%
semantic verb classes in her data set, as shown in Table \ref{tab:verb.type.helmer}:
copular \textit{sein} (`to be'), modal verbs, what she terms mental verbs, and verba dicendi.
Note that what I term ``proportion'' in Table \ref{tab:verb.type.helmer} is different from the omission rate that I report in many tables in this book.
It indicates the proportion of instances of topic drop with a particular verb type out of the total number of 541 instances of topic drop, or the proportion of full forms with \textit{das} and a particular verb class out of the total number of all 200 full forms with \textit{das}.
Since \citet{helmer2016} does not provide numbers for the verb classes that are not listed in Table \ref{tab:verb.type.helmer}, these proportions do not sum to 100\%.

\begin{table}
\caption[Verb classes in \citet{helmer2016}]{Frequency and proportions of the full forms with \textit{das} and of topic drop for the four most frequent semantic verb classes in \citeg{helmer2016} data set, taken from \citet[212]{helmer2016}}
\centering
\begin{tabular}{lrrrr}
\lsptoprule
& \multicolumn{2}{c}{Full form} & \multicolumn{2}{c}{Topic drop}  \\
\multicolumn{1}{c}{Verb class} & Frequency & Proportion & Frequency & Proportion\\
\midrule
Copular \textit{sein}  & $109$ & $54.5\%$  & $169$ & $31.2\%$\\
Modal verbs & $19$ & $9.5\%$  & $78$ & $14.4\%$\\
Mental verbs  & $6$ & $3.0\%$ & $91$ & $16.8\%$ \\
Verba dicendi & $5$ & $2.5\%$ & $41$ & $7.6\%$ \\
\lspbottomrule
\end{tabular}
\label{tab:verb.type.helmer}
\end{table}

From the table, it can be seen that topic drop occurs most frequently before \textit{sein} as a copula in \citeg{helmer2016} corpus of spoken dialogues.
However, this verb class is also the most frequent for the variants with overt \textit{das}, even by a larger margin.
The second and third most frequent verb classes after topic drop are modal and mental verbs, but they differ by only about 2\%.
Verba dicendi follow in fourth place.
The finding that topic drop occurs most frequently before the copular verb \textit{sein} is partly consistent with the claim by \citet{zifonun.etal1997}.
However, the result should be qualified in that the full forms with \textit{das} occur even more frequently with this verb type and \citet{zifonun.etal1997} made their claim only for 1st and 2nd person subject topic drop.

\citet{helmer2016} interprets her result as support for \citeg{imo2013} claim that topic drop is particularly frequent before verba dicendi and sentiendi.
While \citeg{helmer2016} instances of topic drop do seem to occur more often with these verb classes than her  full forms with \textit{das}, they are at the same time even more frequent with \textit{sein} and modal verbs.

It must be noted, however, that her approach prevents true comparability and the computation of reasonable omission rates for two reasons.
First, the ellipses and the full forms come from different corpora, the ellipses from the FOLK and the GIF, the full forms only from the FOLK.
The Forschungs- und Lehrkorpus Gesprochenes Deutsch FOLK (`The research and teaching corpus of spoken German') \citep{schmidt2014} contains recordings of informal everyday conversations \citep[65]{helmer2016}.
The corpus Gespräche im Fernsehen GIF (`conversations on TV') consists of recordings from talk shows, discussions, and programs concerning the German federal election campaign of 2002 \citep{GIF}.
Second, the ellipses are not restricted in terms of their grammatical person, number, and gender while the full forms are.%
% Footnote
\footnote{With the exclusive fixation on full forms with \textit{das}, all of \citeg{helmer2016} full forms are 3rd person singular neuter.
In contrast, her topic drop data also contain omission of constituents of other grammatical persons, numbers, and genders.
She does not provide a corresponding breakdown and in most of her examples the omitted prefield constituent can be reconstructed with a \textit{das}, but she also discusses examples where the 1st person singular pronoun \textit{ich} is omitted \citep[e.g.,][98, example 19]{helmer2016}.}
%
Additionally, the data set seems to contain both subjects and objects.
\citet{helmer2016} does not provide information about their respective proportions.

In sum, there seems to be some tentative support for the two claims from the literature concerning verb type in \citeg{helmer2016} data set, but this support must be qualified  by the fact that her restricted full forms with \textit{das} do not allow for a real comparison between overt and covert variants.

\subsection{\citet{frick2017}}
\largerpage
For the subjects in her Swiss German text message corpus, \citet{frick2017} observes a similar tendency to \citet{androutsopoulos.schmidt2002} concerning verb type, as shown in Table \ref{tab:verb.type.frick}.

\begin{table}
\caption[Verb types after subject topic drop in \citet{frick2017}]{Full forms, instances of topic drop, and omission rates as a function of the type of the following verb for the subjects in \citeg{frick2017} text message corpus, taken from \citet[93]{frick2017}}
\is{Reflexive verb|(}
\centering
\begin{tabular}{lrrrr}
\lsptoprule
\multicolumn{1}{c}{Verb type} & \multicolumn{1}{c}{Full form} & \multicolumn{1}{c}{Topic drop} & \multicolumn{1}{c}{Total} & \multicolumn{1}{c}{Omission rate} \\
\midrule
Lexical verbs & $1\,381$ & $1\,353$ & $2\,734$ & $49.49\%$ \\
Copular verbs & $377$ & $482$ & $859$ &  $56.11\%$ \\
Modal verbs & $261$ & $340$ & $601$ & $56.57\%$ \\
Reflexive verbs & $34$ & $148$ & $182$ & $81.32\%$ \\
Auxiliary verbs & $7$ & $3$ & $10$ & $30.00\%$ \\
\lspbottomrule
\end{tabular}
\label{tab:verb.type.frick}
\end{table}
\noindent
Subject topic drop is about as common before copular verbs as it is before modal verbs ($\chi^2(1) = 0.01$, $p > 0.9$).
In comparison to lexical verbs, it is significantly more frequent before copular verbs ($\chi^2(1) = 11.21, p < 0.001$) and before modal verbs ($\chi^2(1) = 9.61$, $p < 0.01$).%
% Footnote
\footnote{I compared the numbers for the verb types with Pearson's chi-squared tests with Yates's continuity correction in R \citep{rcoreteam2021}.}
%
What is striking are the results for auxiliaries and reflexive verbs.
Auxiliaries seem to be surprisingly rare in Frick's data.
Their omission rate is lower than the rate determined by \citet{androutsopoulos.schmidt2002} for the potential auxiliaries \textit{sein} and \textit{haben},%
%% Footnote
\footnote{Recall that \citet{androutsopoulos.schmidt2002} did not distinguish between copular and auxiliary usage of \textit{sein} nor between auxiliary and lexical verb usage of \textit{haben}.
It is questionable whether this distinction is even relevant to topic drop.
As discussed in \sectref{sec:resolving}, I assume that topic drop is usually resolved on the verb in the left bracket, i.e., at a point in time when it is not yet clear to the reader whether the form of \textit{sein} is a copula or an auxiliary or whether \textit{haben} is used as an auxiliary or a full verb.
This would be different under parallel parsing theories, such as the one proposed by \citet{levy2008} and briefly discussed in \sectref{sec:usage.ambiguity.theory}.
Under his account, the ambiguity \is{Ambiguity} would not be resolved directly on the verb but only when the intended structure becomes apparent, in the right bracket or at the end of the clause.
In this framework, the difference between, e.g., a usage of \textit{haben} as auxiliary and a usage as lexical verb might be of relevance.}
%
but it is not very meaningful with only 10 data points.
The highest omission rate of all verb types is exhibited by reflexive verbs with over 80\%.%
% Footnote
\footnote{It is significantly higher than that of the modal verbs with the second highest rate ($\chi^2(1) = 35.39, p < 0.001$).}
%
However, without providing a proportion, Frick states that these are mainly instances of the verb \textit{sich freuen} (`to be glad') \citep[94]{frick2017}.
In other words, topic drop may not generally be frequent before reflexive verbs but only or mainly before \textit{sich freuen}.
\citet[95]{frick2017} speculates that the Swiss German utterance \textit{$\Delta$ freu mi} (`(I) am glad'), i.e., 1st person singular subject topic drop before \textit{sich freuen} in the present tense, is currently developing into a fixed formula.\is{Reflexive verb|)}

\citet[96]{frick2017} notes that the verb type following topic drop seems to differ depending on the grammatical person of the omitted subject.
This may be relevant to the claim by \citet{zifonun.etal1997} that subject topic drop of the 1st and the 2nd person is particularly frequent before copular, auxiliary, and modal verbs.%
% Footnote
\footnote{Note that the IDS grammar does not explicitly claim that topic drop of the 3rd person is less frequent before these verb types.
The 1st and 2nd person topic drop is discussed separately as ``person ellipsis'' and the verb type claim is only made there \citep[see][414--417]{zifonun.etal1997}.
}
%
Frick states that the 2nd person is preferably omitted before modal verbs \citep[96]{frick2017}.
The omission rate of the 2nd person singular is 64.41\% before modals vs. only 37.79\% before lexical verbs.
For the 2nd person plural, there are only 4 instances of topic drop, which admittedly all occur before modals, but which are too few to make a quantitative statement.
In contrast to the 2nd person, \citet[96]{frick2017} notes that the 3rd person singular is rarely omitted before modal verbs (13.33\% omissions vs. 30.77\% for lexical verbs).
She attributes this to the syncretism \is{Syncretism} with the 1st person singular discussed above and to ambiguity avoidance: \is{Ambiguity avoidance}
``Due to the danger of confusion with the first person singular, the omission before modal verbs is therefore blocked or only possible as an exception''%
%% Footnote
\footnote{My translation, the original: ``Durch die Verwechslungsgefahr mit der ersten Person Singular ist die Auslassung vor Modalverben deshalb geblockt bzw. nur ausnahmsweise möglich'' \citep[108]{frick2017}.}
%
\citep[108]{frick2017}.
However, in her corpus, only 13 overt 3rd person singular pronouns and 2 corresponding instances of topic drop occur before modal verbs.%
% Footnote
\footnote{This may be because in text messages, it is more likely to give the addressee advice of the type \textit{you should} / \textit{you can}, etc. than a third person.}
%
These figures are too small to base conclusions on them. 
The results of my experiments \ref*{exp:top.s.fv} and \ref*{exp:top.s.mv} (Sections \ref{sec:exp.top.s.fv.person} and \ref{sec:exp.top.s.mv.person}), which tested topic drop of the 1st and 3rd person singular before lexical verbs and before modal verbs, do not suggest that topic drop of the 3rd person singular is more degraded before modal than before lexical verbs.
For the 1st person, \citet{frick2017} does not discuss the effect of verb type, but her figure 11 \citep[96]{frick2017} shows that both the 1st person singular and the 1st person plural are omitted most often before lexical verbs.
Despite her remarks, \citet{frick2017} also does not provide an explanation for the different omission rates per verb type.

In short, the corpus studies are tentatively in line with the claim by the IDS grammar \citep{zifonun.etal1997} that topic drop of the 1st and the 2nd person occurs particularly often before copular and modal verbs.
For the auxiliaries, the results are less conclusive, which hinges partly on the extremely low number of auxiliaries that \citet{frick2017} claims to have found in her corpus.
Concerning \citeg{imo2013} claim that topic drop occurs preferably before verba sentiendi and dicendi, there is only tentative support from \citeg{helmer2016} study.
In her corpus, verbs of saying and thinking were in the top four verb classes that occurred most often with topic drop.
It is important to note that even the authors who empirically studied verb type do not explain why topic drop should appear particularly frequently with certain verb types.
In the following, I propose such an explanation from the perspective of information theory.
\is{Copula|)} \is{Auxiliary|)} \is{Modal verb|)} \is{Lexical verb|)}
\is{Corpus|)} 

\section{Information-theoretic predictions for verb type and verb surprisal}\label{sec:verb.info.theory.predictions}
\largerpage[2]

\begin{table}[b]
\small
\caption[Verb frequencies according to DeReWo]{Verbs in the frequency classes 3 to 7 in the DeReWo: the lower the frequency class, the more frequent the corresponding verb}
\centering
\hfill
\begin{tabular}{@{}l@{}r}
\lsptoprule
{Verb lemma} & Frequency class \\
\midrule
\textit{sein} (`to be') & $3$ \\
\textit{werden} (`will') & $3$ \\
\textit{haben} (`to have') & $3$ \\
\textit{können} (`can') & $5$ \\
\textit{sollen} (`shall') & $6$ \\
\textit{müssen} (`must') & $6$ \\
\textit{geben} (`to give') & $6$ \\
\textit{sagen} (`to say') & $6$ \\
\textit{kommen} (`to come') & $6$ \\
\textit{wollen} (`want') & $6$ \\
\textit{gehen} (`to go') & $6$ \\
\textit{machen} (`to make') & $6$\\
\lspbottomrule
\end{tabular}
\hfill
\begin{tabular}{l@{}r@{}}
\lsptoprule
{Verb lemma} & Frequency class \\
\midrule
\textit{stehen} (`to stand') & $6$ \\
\textit{sehen} (`to see') & $7$ \\
\textit{finden} (`to find') & $7$ \\
\textit{lassen} (`to let') & $7$ \\
\textit{bleiben} (`to remain') & $7$ \\
\textit{liegen} (`to lie') & $7$ \\
\textit{stellen} (`to put') & $7$ \\
\textit{nehmen} (`to take') & $7$ \\
\textit{zeigen} (`to show') & $7$ \\
\textit{dürfen} (`may') & $7$ \\
\textit{halten} (`to hold') & $7$ \\
\\
\lspbottomrule
\end{tabular}
\hfill
\label{tab:verb.freq}
\end{table}


In \sectref{sec:avoid.peaks}, I outlined how verb surprisal should impact the usage of topic drop according to my information-theoretic account.
I briefly repeat the corresponding reasoning here.
According to the \textit{avoid peaks} principle, the likelihood of topic drop should decrease with the surprisal of the following verb:
the higher the surprisal of the verb in the left bracket, the lower the likelihood of topic drop. 
This is because the overt prefield constituent may be needed to reduce the processing effort \is{Processing effort} on the verb for two reasons.
First, it can make the following verb more predictable, \is{Predictability} and, second, it avoids the processing effort \is{Processing effort} required to resolve the ellipsis.

\is{Copula|(} \is{Auxiliary|(} \is{Modal verb|(} \is{Lexical verb|(}
My information-theoretic account also predicts effects of verb type as an indirect consequence of the \textit{avoid peaks} principle.
Copular, auxiliary, and modal verbs are generally very frequent in German. \il{German|(}
This is illustrated in Table \ref{tab:verb.freq} based on the Korpusbasierte \is{Corpus} Wortgrundformenliste DeReWo \citep{derewo2013},%
% Footnote
\footnote{\label{note:wortgrundformliste}The Korpusbasierte Wortgrundformenliste (`corpus-based list of word base forms') DeReWo v-ww-bll-320000g-2012-12-31-1.0 \citep{derewo2013} is based on the Deutsches Referenzkorpus DeReKo (`German reference corpus') \citep{dereko2012}.
The frequency of words is indicated by their grouping into frequency classes.
Frequency class 0 contains the most frequent word, i.e., the definite determiner with the forms \textit{der}, \textit{die}, \textit{das}. \is{Article}
Each subsequent frequency class $n$ contains words that are approximately $2^n$ times less frequent than the word in frequency class 1 \citep[7]{derewo2013.doku}.
This means that \textit{sein} (`to be') in frequency class 3  is about $2^n = 2^3 = 8$ times less frequent than the definite determiner \textit{der}, \textit{die}, \textit{das}.
The higher the frequency class, the rarer the words contained in this class.}
%
which shows all verbs in the frequency classes 3 to 7 from the list.%
%% Footnote
\footnote{I limited myself to frequency classes 1 through 7 because that way all copular, auxiliary, and modal verbs are included in my selection.
Of these, \textit{dürfen} (`may') is the rarest in class 7.}



\textit{Sein} (`to be'), \textit{werden} (`will'), \textit{haben} (`to have'), and all modal verbs except for \textit{dürfen} (`may') are more frequent than or at least equally frequent as the most frequent lexical verbs, such as \textit{geben} (`to give') and \textit{sagen} (`to say').
Given that copular, auxiliary, and modal verbs are so frequent, it is generally unlikely for them to cause a peak in the information density profile.
Thus, the speaker does not need to insert the preverbal constituent to smooth the profile but can omit it whenever possible.

There are two qualifications to this conclusion.
First, the table does not consider the syntactic position of the verb, so the occurrences in the left and right brackets are combined, whereas only the left bracket is relevant to the informa- tion-theoretic frequency-based explanation of topic drop.
Second, the data are based on the DeReKo, i.e., the German reference corpus, \is{Corpus} which consists largely of news articles.
It could be that the distribution would be different in a correspondingly large text message corpus or another corpus of conceptually spoken text types. \is{Text type}
For the text message subcorpus of the fragment corpus FraC (see \sectref{sec:corpus.frac} for details), however, a similar distribution can be observed if we look at the 10 most frequent verb lemmas, again independent of the topological position in Table \ref{tab:verb.freq.frac}.

\begin{table}
\caption{The ten most frequent verb lemmas and their frequency in the text message subcorpus  of the FraC}
\is{Corpus}
\centering
\begin{tabular}{lr}
\lsptoprule
Verb lemma & Frequency \\
\midrule
\textit{sein} (`to be') &  $363$ \\
\textit{haben} (`to have') &  $221$\\
\textit{gehen} (`to go') & $76$ \\ 
\textit{kommen} (`to come') & $75$ \\
\textit{können} (`can') & $67$ \\
\textit{machen} (`to make') & $47$ \\ 
\textit{müssen} (`must') & $47$ \\
\textit{werden} (`will') & $44$ \\ 
\textit{wollen} (`to want') & $42$ \\
\textit{sagen} (`to say') & $35$ \\
\lspbottomrule
\end{tabular}
\label{tab:verb.freq.frac}
\end{table}

\is{Corpus|(}Again, the auxiliary and copular verb \textit{sein} (`to be') and the auxiliary and lexical verb \textit{haben} (`to have') are the most frequent verbs by a large margin.
Therefore, the predictions regarding auxiliary and copular verbs should translate to text message data. 
Although the modal verbs seem to be a bit less frequent relative to the most frequent lexical verbs, still three of them are in the top ten of verb lemmas in the text message corpus.
Therefore, at least for these modal verbs, it can also be expected that they tend to have a rather low surprisal.
Note that the information-theoretic prediction according to which topic drop before modal verbs should be more felicitous because they are more frequent clashes with the predictions of ambiguity avoidance \is{Ambiguity avoidance} discussed in \sectref{sec:usage.ambiguity.theory}.
According to ambiguity avoidance, \is{Ambiguity avoidance} speakers should not use topic drop before modal verbs, at least of the 1st and the 3rd person singular, because they exhibit syncretic \is{Syncretism} forms resulting in ambiguity.\is{Ambiguity}\il{German|)}

\section{Corpus study of verb type and verb surprisal}
In addition to grammatical person, verbal inflection and ambiguity avoidance, I also investigated the remaining verb-related factors both in my corpus study and experimentally.
In this section, I first present and discuss the corpus results.%
% Footnote
\footnote{The corpus data and the analysis scripts are accessible online: \url{https://osf.io/zh7tr}.
For copyright reasons, I only provide the IDs and the annotations of each instance but not the linguistic material.}
%
I start with the descriptive results concerning verb type in the \textsc{FraC-TD-Comp} data set.
With this data set, I furthermore refute \citeg{imo2013} claim concerning verb types.
As discussed in \sectref{sec:frac.data.sets}, given the typical occurrence of topic drop in certain text types, it is not reasonable to determine relative numbers and omission rates on the whole FraC.
Therefore, I focused on the text message data set \textsc{FraC-TD-SMS} to determine relative numbers for the type of the verb in the left bracket and on \textsc{FraC-TD-SMS-Part}  to calculate the surprisal of this verb.
For all three data sets, I discuss the claim of the IDS grammar \citep{zifonun.etal1997} that topic drop of the 1st and the 2nd person occur preferably before copular, auxiliary, and modal verbs.
However, it was tested systematically in the  logistic regression analysis of \textsc{FraC-TD-SMS-Part}, which I discussed already in \sectref{sec:frac.td.part.regression.person} and which I revisit in \sectref{sec:corpus.regression.rep}.

\subsection{Verb type in \textsc{FraC-TD-Comp}}
In the first step, I determined the verb type in the \textsc{FraC-TD-Comp} data set, which consists of all instances of topic drop and all full forms in the FraC.
Each instance is annotated for the type of the verb in the left bracket distinguishing between copular verbs, auxiliaries, modal verbs, reflexive verbs, \is{Reflexive verb} and lexical verbs \is{Lexical verb} (see \sectref{sec:corpus.annotation}).
As Table \ref{tab:FraC.Comp.verb.type.subj}%
%% Footnote
\footnote{The verb type is unascertainable for three full forms.
There, the sentence is interrupted right after the finite verb so that for verbs such as \textit{sein} (`to be') or \textit{haben} (`to have'), I could not decide whether it was used as a copular, auxiliary, or lexical verb.}
%
shows for the subjects and Table \ref{tab:FraC.Comp.verb.type.obj} for the objects, the instances of topic drop and the full forms principally behave in parallel concerning the verb type.
For subjects, lexical verbs are the most common, followed by copular verbs, auxiliaries, and modal verbs, while reflexive verbs \is{Reflexive verb} are the least common.
Overt and covert objects occur most frequently before lexical verbs, followed by modal verbs and auxiliaries but rarely before copular and reflexive verbs. \is{Reflexive verb|(}
The omission rates for both subjects and objects are the highest before copular verbs, followed by lexical verbs.

%\vspace{-0.5\baselineskip}
\begin{table}
\caption{Full forms, instances of topic drop, and omission rates as a function of the type of the following verb for the subjects in the \textsc{FraC-TD-Comp} data set}
\centering
\begin{tabular}{lrrrr}
\lsptoprule
\multicolumn{1}{c}{Verb type} & \multicolumn{1}{c}{Full form} & \multicolumn{1}{c}{Topic drop} & \multicolumn{1}{c}{Total} & \multicolumn{1}{c}{Omission rate}\\
\midrule
Lexical verb& $1\,450$ & $398$ & $1\,848$ & $21.54\%$ \\
Copular  verb& $602$ & $205$ & $807$ & $25.40\%$\\
Auxiliary  verb& $450$ & $94$ & $544$ & $17.28\%$ \\
Modal verb& $421$ & $73$ & $494$ & $14.78\%$ \\
Reflexive verb& $125$ & $31$ & $156$ & $19.87\%$ \\
Unascertainable & $3$ & $0$ & $3$ & $0.00\%$\\
\lspbottomrule
\end{tabular}
\label{tab:FraC.Comp.verb.type.subj}
\end{table}

%\vspace{-0.5\baselineskip}
\begin{table}
\caption{Full forms, instances of topic drop, and omission rates as a function of the type of the following verb for the objects in the \textsc{FraC-TD-Comp} data set}
\centering
\begin{tabular}{lrrrr}
\lsptoprule
\multicolumn{1}{c}{Verb type} & \multicolumn{1}{c}{Full form} & \multicolumn{1}{c}{Topic drop} & \multicolumn{1}{c}{Total} & \multicolumn{1}{c}{Omission rate}\\
\midrule
Lexical verb& $78$ & $35$ & $113$ &  $30.97\%$  \\
Copular verb &  $1$ & $1$ & $2$ &  $50.00\%$  \\
Auxiliary verb &  $29$ & $16$ & $45$ & $35.56\%$ \\
Modal verb&  $51$ & $20$ & $71$ & $28.17\%$ \\
Reflexive verb& $1$ & $0$ & $1$ & $0.00\%$  \\
\lspbottomrule
\end{tabular}
\label{tab:FraC.Comp.verb.type.obj}
\end{table}
%\vspace{-0.5\baselineskip}

This only partly matches the claim of the IDS grammar \citep[415]{zifonun.etal1997} according to which subject topic drop, at least of the 1st and the 2nd person, is more frequent before copular, auxiliary, and modal verbs than before lexical verbs.
Below in Sections \ref{sec:corpus.mess.verb} and \ref{sec:corpus.surprisal}, I come back to this issue in more detail and to verb type in general by looking at the \textsc{FraC-TD-SMS} and \textsc{FraC-TD-SMS-Part} data sets.

As discussed in \sectref{sec:frac.data.sets}, I only conducted statistical analyses of the text message data sets.
This is because we may not only expect differences between the text types in terms of the rate of utterances with topic drop but also in the frequency of the verb types.
For instance, the verb type is also influenced by which tenses are frequently used, e.g., using the perfect leads to a higher rate of auxiliary verbs, etc.
Theoretically, text types with little or no topic drop could distort the results if certain verb types are particularly frequent or rare there.
\is{Lexical verb|)} \is{Copula|)} \is{Auxiliary|)} \is{Modal verb|)} \is{Reflexive verb|)}

\largerpage
The claim by \citet[297]{imo2013}, according to which topic drop is said to occur most frequently before verba dicendi and verba sentiendi, such as \textit{wissen} (`to know'), \textit{finden} (`to find'), or \textit{glauben} (`to believe'),%
%% Footnote
\footnote{The original claim in German was: ``Es ist offensichtlich, dass uneigentliche Verbspitzenstellungen mit bestimmten, hoch frequenten verba dicendi et sentiendi wie \textit{wissen}, \textit{finden} oder \textit{glauben} besonders häufig auftreten'' \citep[297]{imo2013}.
My translation: `It is obvious that the improper verb top positionings occur particularly frequently with certain high-frequency verba dicendi and sentiendi such as to know, to find, or to believe.'}
%
is challenged by the data in Table \ref{tab:FraC.Comp.verb.lemma}.\clearpage

\begin{table}
\caption{Absolute frequencies and omission rates of selected verb lemmas in the left bracket of the full forms and the utterances with topic drop in the \textsc{FraC-TD-Comp} data set}
\centering
\begin{tabular}{rlrrrr}
\lsptoprule
Rank &  Verb lemma &  \Centerstack[c]{Full\\form} & \Centerstack[c]{Topic\\drop} & Total & \Centerstack[c]{Omission\\rate} \\
\midrule
1 & \textit{sein} (`to be') & $701$ & $216$ & $917$ & $23.56\%$\\
2 & \textit{haben} (`to have') & $419$ & $117$ & $536$ & $21.83\%$\\
3  & \textit{können} (`can') &  $195$ & $41$ & $236$ & $17.37\%$\\
4 & \textit{sich freuen} (`to be glad') & $34$ & $36$ & $70$ & $51.43\%$\\
5 & \textit{werden} (`to become') & $167$ & $34$ & $201$ & $16.92\%$\\
6 & \textit{müssen} (`must') & $117$ & $28$ & $145$ & $19.31\%$\\
7 & \textit{gehen} (`to go') & $56$ & $19$ & $75$ & $25.33\%$ \\
8 & \textit{suchen} (`to search') & $23$ & $19$ & $42$ & $45.24$\%\\
9 & \textit{machen} (`to make') & $38$ & $18$ & $56$ & $32.14$\%\\
10 & \textit{kommen} (`to come') & $30$ & $17$ & $47$ & $36.17$\%\\
\dots & & & & &\\
13 & \textit{wissen} (`to know') & $69$ & $14$ & $83$ & $16.87\%$\\
14 & \textit{finden} (`to find') & $35$ & $13$ & $48$ & $27.08\%$\\
\dots & & & & &\\
17 & \textit{denken} (`to think') & $53$ & $9$ & $62$ & $14.52\%$\\
\dots & & & & &\\
39 & \textit{glauben} (`to believe') & $40$ & $3$ & $43$ & $6.98\%$\\
\lspbottomrule
\end{tabular}
\label{tab:FraC.Comp.verb.lemma}
\end{table}


In absolute terms, the verbs that follow topic drop most often are forms of \textit{sein} (`to be') with more than 200 occurrences and \textit{haben} (`to have') with more than 100. 
The most frequent reflexive verb \is{Reflexive verb} is \textit{sich freuen} (`to be glad') with about 36 occurrences.%
%% Footnote
\footnote{As in \citeg{frick2017} corpus, this makes it by far the most frequent reflexive verb \is{Reflexive verb} following topic drop (see \sectref{sec:usage.verb.type.studies}).}
%
The lexical verbs \is{Lexical verb} \textit{wissen} (11 occurrences), \textit{finden} (13 occurrences), and \textit{glauben} (3 occurrences) are at most mid-table%%
% Footnote
\footnote{Recall that \textit{wissen} is also a preterite present verb with syncretic forms for the 1st and 3rd person singular in present tense (\textit{ich weiß} and \textit{sie weiß}).}
%
and their omission rates, except for \textit{finden}, are even below the mean of 22.08\% for the lexical verbs.
Taken together, there is nothing that suggests that these verbs are particularly frequent after topic drop or that topic drop occurs most often before them.
With this result, I consider \citeg{imo2013} strong claim that topic drop occurs very frequently with certain verba dicendi and sentiendi to be sufficiently invalidated and do not discuss it further.

\subsection{Verb type in \textsc{FraC-TD-SMS}}\label{sec:corpus.mess.verb}
Table \ref{tab:frac.verb.type} shows the omission rates in the \textsc{FraC-TD-SMS} data set as a function of the type of the following verb.
Because of the higher frequency of subject topic drop, I restrict myself to the verb type following the subjects, disregarding the 8 objects.%
% Footnote
\footnote{Again, the numbers for the objects are too small anyway to base conclusions on them.}

\is{Lexical verb|(} \is{Copula|(} \is{Auxiliary|(} \is{Modal verb|(} \is{Reflexive verb|(}
\begin{table}
\centering
\caption{Full forms, instances of topic drop, and omission rates as a function of the type of the following verb for the subjects in the \textsc{FraC-TD-SMS} data set}
\begin{tabular}{lrrrr}
\lsptoprule
\multicolumn{1}{c}{Verb type} & \multicolumn{1}{c}{Full form} & \multicolumn{1}{c}{Topic drop} & \multicolumn{1}{c}{Total} & \multicolumn{1}{c}{Omission rate} \\
\midrule
Auxiliary verb& $22$ & $57$ & $79$ & $72.15\%$ \\
Copular verb& $43$ & $85$ & $128$ & $66.41\%$ \\
Lexical verb& $95$ & $151$ & $246$ & $61.38\%$ \\
Modal verb& $29$ & $42$ & $71$ & $59.15\%$ \\
Reflexive verb& $10$ & $13$ & $23$ & $56.52\%$ \\
\lspbottomrule
\end{tabular}
\label{tab:frac.verb.type}
\end{table}

As discussed in \sectref{sec:usage.verb.type.theory}, \citet[415]{zifonun.etal1997} claim in the IDS grammar that subject topic drop of the 1st and the 2nd person is more frequent before copular, auxiliary, and modal verbs than before other verb types.
The observed frequency differences in the \textsc{FraC-TD-SMS} data set are only partially in line with this claim.%
%% Footnote
\footnote{While the overall tendency in \textsc{FraC-TD-SMS} and in \citeg{frick2017} study on Swiss German text messages using a comparable method (see \sectref{sec:usage.verb.type.studies}) is similar, the individual omission rates differ, in two cases considerably.
Topic drop is the rarest before auxiliaries in her data (30\%), but in \textsc{FraC-TD-SMS} data set it was the most frequent (72\%).
Since \citet{frick2017} reports to have found only 7 full forms and 3 instances of topic drop before auxiliaries, this discrepancy is most likely caused by different criteria for assigning a verb to a category.
Conversely, the verb type that most frequently follows topic drop in \citeg{frick2017} data, reflexives (80\%, based on 182 occurrences), has the lowest omission rate (57\%) in \textsc{FraC-TD-SMS}.
As in \citeg{frick2017} case, the data for reflexive verbs mostly stem from the verb \textit{sich freuen} (`to be glad').
In German Standard German, the topic drop variant, \textit{freu(e) mich} (`am glad'), does not yet seem to have become equally fixed as a formula as in Swiss German (at least at the time of the data collection), so the full form with \textit{ich} is still frequent.
The omission rates of copular verbs and lexical verbs are around 10\% higher in \textsc{FraC-TD-SMS} than in \citeg{frick2017} corpus (significant according to a Pearson's chi-squared tests with Yates's continuity correction calculated in R \citep{rcoreteam2021}, $\chi^2_{cop} (1) = 4.42, p < 0.05$; $\chi^2_{lex}(1) = 12.3, p < 0.001$), while the rate for modal verbs is only slightly higher (57\% in \citet{frick2017} and 59\% in \textsc{FraC-TD-SMS}, not significant, $\chi^2(1) = 0.08, p > 0.7$).
There is no obvious explanation for the differences in modal verbs and lexical verbs other than that of general differences between the corpora used.}
%%
While descriptively auxiliaries and copulas follow topic drop more often than lexical verbs, the differences are not significant according to a simple logistic regression predicting the probability of topic drop from the verb type and setting lexical verbs as the baseline ($\chi^2_{aux} (1) = 3.09, p > 0.07; \chi^2_{cop} (1) = 0.92, p > 0.3$), although there is a tendency in this direction for the auxiliaries.
What is unexpected is that the omission rate before the modal verbs in the data set is relatively low, even a bit lower than for the lexical verbs in absolute terms (the difference is not significant, $\chi^2 (1) = 0.11, p > 0.7$).
Both the IDS grammar and my information-theoretic approach would have predicted a higher omission rate for modal verbs.
While the IDS grammar does not justify this expectation, the information-theoretic approach argues that verbs in the closed class of modal verbs should have lower surprisal on average than members of the open class of lexical verbs.
Therefore, it should not be necessary to use the full form to facilitate their processing. \is{Processing effort}
I discuss the role of verb type more closely below.

\subsection{Verb type and surprisal in \textsc{FraC-TD-SMS-Part}}\label{sec:corpus.surprisal}
I also analyzed the verb type and additionally the verb surprisal of the instances in the \textsc{FraC-TD-SMS-Part} data set, which contains only the 1st and 3rd person subjects from \textsc{FraC-TD-SMS}  (see \sectref{sec:corpus.sms.part}).
Table \ref{tab:frac.verb.type.part.total} shows the omission rates per verb type.
The total figures are very similar to the slightly larger \textsc{FraC-TD-SMS} and also the order of verb types as a function of their omission rate is identical.
The omission rates seem to provide partial support for the claim from the IDS grammar \citep[415]{zifonun.etal1997} that subject topic drop of the 1st and the 2nd person is particularly frequent before copular, auxiliary, and modal verbs.

\begin{table}
\centering
\caption{Full forms, instances of topic drop, and omission rates as a function of the type of the following verb in the \textsc{FraC-TD-SMS-Part} data set}
\begin{tabular}{lrrrr}
\lsptoprule
Verb type & Full form & Topic drop & Total & Omission rate \\
\midrule
Auxiliary verb &  $14$ & $45$ & $59$ & $76.27\%$ \\
Copular verb & $30$ & $72$ & $102$ & $70.59\%$ \\
Lexical verb& $86$ & $142$ & $228$ & $62.28\%$\\
Modal verb & $20$ & $33$ & $53$ & $62.26\%$\\
Reflexive verb& $10$ & $13$ & $23$ & $56.52\%$ \\
\lspbottomrule
\end{tabular}
\label{tab:frac.verb.type.part.total}
\end{table}

Table \ref{tab:frac.verb.type.part} shows the omission rates per verb type subdivided by the two grammatical persons.

\begin{table}
\centering
\caption{Full forms, instances of topic drop, and omission rates as a function of the type of the following verb in the \textsc{FraC-TD-SMS-Part} data set divided by grammatical person}
\begin{tabular}{lrrrrrrrr}
\lsptoprule
& \multicolumn{4}{c}{1SG}  & \multicolumn{4}{c}{3SG}  \\
%& &  & & \multicolumn{1}{c|}{omis-}  &  & & & \multicolumn{1}{c}{omis-}  \\
Verb type & \Centerstack[c]{Full\\form} & \Centerstack[c]{Topic\\drop} & Total & \Centerstack[c]{Omis-\\sion\\rate} & \Centerstack[c]{Full\\form} & \Centerstack[c]{Topic\\drop} & Total & \Centerstack[c]{Omis-\\sion\\rate} \\
% & \multicolumn{1}{c|}{Full} & \multicolumn{1}{c|}{Topic} & & \multicolumn{1}{c|}{Omission} & \multicolumn{1}{c|}{Full} & \multicolumn{1}{c|}{Topic} &  & \multicolumn{1}{c}{Omission} \\
%\multirow{-3}{*}{Verb type} & \multicolumn{1}{c|}{form} & \multicolumn{1}{c|}{drop} & \multicolumn{1}{c|}{\multirow{-2}{*}{Total}} & \multicolumn{1}{c|}{rate} & \multicolumn{1}{c|}{form} & \multicolumn{1}{c|}{drop} & \multicolumn{1}{c|}{\multirow{-2}{*}{Total}} & \multicolumn{1}{c}{rate} \\
\midrule
Auxiliary & $13$ & $41$ & $54$ & $75.93\%$  & $1$ & $4$ & $5$ & $80.00\%$ \\
Copular & $14$ & $52$ & $66$ & $78.79\%$  & $16$ & $20$ & $36$ & $55.56\%$ \\
Lexical& $75$ & $126$ & $201$ & $62.69\%$ & $11$ & $16$ & $27$ & $59.26\%$ \\
Modal & $19$ & $32$ & $51$ & $62.75\%$ & $1$ & $1$ & $2$ & $50.00\%$ \\
Reflexive& $10$ & $13$ & $23$ & $56.52\%$ & $0$ & $0$ & $0$ & $0.00\%$ \\
\lspbottomrule
\end{tabular}
\label{tab:frac.verb.type.part}
\end{table}

It can be seen that both the 1st and the 3rd person singular are particularly frequently omitted before auxiliaries, whereas only the 1st person singular is also especially often omitted before copulas.
However, the omission rates of neither the 1st nor the 3rd person singular are higher before modal verbs than before lexical verbs.
It is worth noting that the rates of the 1st person singular are generally higher than those of the 3rd person singular for all verb types except auxiliaries and that overall the figures for the 3rd person singular are relatively low.
This suggests that there is a tendency that partly matches \citeg{zifonun.etal1997} claim.
To also test this claim statistically, I included a binary predictor \textsc{Verb Type} in my regression analysis, which contrasted copular, auxiliary, and modal verbs on the one hand with lexical and reflexive verbs on the other. 
I revisit this analysis in \sectref{sec:corpus.regression.rep}.
\is{Lexical verb|)} \is{Copula|)} \is{Auxiliary|)} \is{Modal verb|)} \is{Reflexive verb|)}

\is{Unigram surprisal|(}
With the data set \textsc{FraC-TD-SMS-Part}, I also looked at the surprisal of the verb following topic drop to test my information-theoretic approach.
Following from the \textit{uniform information density hypothesis} \is{Uniform information density} and the \textit{avoid peaks} principle (see \sectref{sec:avoid.peaks}), topic drop should be less frequent if the surprisal of the following verb is higher because the insertion of an overt prefield constituent may be necessary to lower the surprisal and, thus, the processing effort \is{Processing effort} on this verb.
This is supported by Figure \ref{fig:frac.surprisal.density},%
%% Footnote
\footnote{Figures \ref{fig:frac.surprisal.density}, \ref{fig:frac.surprisal.verb.type}, and \ref{fig:frac.surprisal.verb.type} were created in R with the package ggplot2 \citep{wickham2016}.}
%
which shows the proportion of topic drop as a function of the unigram surprisal of the verb in the left bracket and indicates that the rate of topic drop decreases when the verb surprisal increases.

\begin{figure}
\centering
\includegraphics[scale=1]{Korpusplots/Proportion_TD_Surprisal_FraC.pdf}
\caption[Distribution of the verb surprisal in \textsc{FraC-TD-SMS-Part}]{Proportion of topic drop as a function of the unigram surprisal of the verb in the left bracket in the \textsc{FraC-TD-SMS-Part} data set} 
\label{fig:frac.surprisal.density}
\end{figure}

\noindent
I relied on the lemma-based unigram surprisal of the verb in the left bracket as a measure of surprisal.
It was calculated on the lemmatized text message subcorpus of the FraC using the SRILM toolkit \citep{stolcke2002} and corresponds to the frequency of the lemma in this subcorpus regardless of its position in the clause.
That is, the more frequently a verb occurs in the text message corpus in any topological position, the lower its surprisal (measured in bits).
With this measure, I approximated the likelihood of the verbs in text messages, i.e., the context considered in the surprisal measure is the text type \is{Text type} text message:
$S(\textit{verb lemma}) \approx \mathbin{-}\log_{2}\ p(\textit{verb lemma}\mathbin{|}\textit{text message})$.
The lemma frequency is only a rough approximation to what would be a ``psychologically realistic'' surprisal, i.e., one that considers all relevant aspects of the linguistic and extralinguistic context.
Given, however, the data sparsity of \textsc{FraC-TD-SMS-Part}, the unigram surprisal conditioned on the text type \is{Text type} text message seems to be a reasonable approximation.
This holds also because, following from the V2 word order \is{V2 word order} in German, the verbs in my data set are predominantly either in the first or second position, i.e., higher n-gram models, which would require more data, may not be a lot more informative.

\is{Lexical verb|(} \is{Copula|(} \is{Auxiliary|(} \is{Modal verb|(} \is{Reflexive verb|(}
The violin plot in Figure \ref{fig:frac.surprisal.verb.type} shows the unigram surprisal per verb type.
The mean surprisal of all verbs in the corpus is at 8.18 (SD = 2.23).
The small and closed classes of copulas and auxiliaries are very frequent and, thus, have a low surprisal.
The mean surprisal of the likewise closed class of modal verbs lies between that of the auxiliaries and the reflexive and lexical verbs.
While the mean surprisal of the reflexive verbs is close to that of the lexical verbs, the latter is additionally distributed over almost the entire surprisal range from 5.5 to 13 with the mean at 9.6 (SD = 1.94).

%\vspace{-0.5\baselineskip}
\begin{figure}
\centering
\includegraphics[scale=1]{Korpusplots/VerbType_Surprisal_FraC.pdf}
\caption[Relation between verb surprisal and verb type in \textsc{FraC-TD-SMS-Part}]{Relation between verb surprisal and verb type in the \textsc{FraC-TD-SMS-Part} data set (diamonds are mean values)}
\label{fig:frac.surprisal.verb.type}
\end{figure}
%\vspace{-0.5\baselineskip}

The order of verb types based on their omission rates is very similar to that based on the mean unigram surprisal but not identical, as shown in Figure \ref{fig:frac.surprisal.rate.verb.type}.
The auxiliaries have a higher total omission rate than the copulas (but not after the 1st person singular), but the copulas have a lower mean surprisal.
This suggests that the verb surprisal is generally a good predictor for the omission rate before certain verb types, but it might not be the only relevant factor.

%\vspace{-0.5\baselineskip}
\begin{figure}
\centering
\includegraphics[scale=1]{Korpusplots/VerbType_Surprisal_OmissionRate.pdf}
\caption{Relation between verb surprisal, omission rate, and verb type in the \textsc{FraC-TD-SMS-Part} data set}
\label{fig:frac.surprisal.rate.verb.type}
\end{figure}
%\vspace{-0.5\baselineskip}

\subsection{Verb type and surprisal in \textsc{FraC-TD-SMS-Part} -- logistic regression analysis}\label{sec:corpus.regression.rep}
In \sectref{sec:frac.td.part.regression.person}, I presented the logistic regression analysis conducted on the \textsc{FraC-TD-SMS-Part} data set and discussed its results for grammatical person and verbal inflection.
In this section, I return to this analysis and focus on the two remaining factors in my discussion: verb type and verb surprisal.
Recall that I predicted the likelihood of topic drop vs. full form from the independent variables grammatical \textsc{Person}, verb \textsc{Surprisal}, \textsc{Verb Type}, and \textsc{Inflection}.
The binary \textsc{Person} predictor compared the 1st to the 3rd person singular and the three-level predictor \textsc{Inflection} compared \textit{distinct}, \textit{strictly syncretic}, and \textit{informed syncretic} verb forms with each other (for details see \sectref{sec:frac.td.part.regression.person}).
The independent variable \textsc{Verb Type} had two levels.
It encoded whether the verb in the left bracket was either part of the group of copular, auxiliary, and modal verbs or part of the group of lexical and reflexive verbs.
This way, I tested the claim from the IDS grammar that topic drop of the 1st and 2nd person is particularly frequent before copular, auxiliary, and modal verbs.
The numeric \textsc{Surprisal} predictor encoded the unigram surprisal per verb lemma of the verb in the left bracket.

In this section, I additionally present an analysis of only the instances in \textsc{FraC-TD-SMS-Part} with a lexical verb in the left bracket to evidence an impact of verb surprisal for only them as well.
Such an effect would provide stronger support for my information-theoretic account.
It would then be excluded that the surprisal effect, already described in \sectref{sec:frac.td.part.regression.person} for all verb types, is only an artifact of systematic surprisal differences between verb types.
Up to now, this possibility has not been ruled out because the predictor \textsc{Verb Type} only captures the difference between copular, auxiliary, and modal verbs on the one hand and lexical and reflexive verbs on the other hand in a binary way but not differences within the two groups.
This means that if I find a surprisal effect when focusing  only on the lexical verbs, this effect is independent of verb type and, thus, provides genuine evidence for the information-theoretic approach, or more precisely for the \textit{avoid peaks} principle.

\subsubsection{Predictions}\label{sec:corpus.regression.predictions.rep}
In the following, I complement the predictions (i--\textsc{Person}) and (ii--\textsc{Inflection}) presented in \sectref{sec:frac.td.part.regression.person}, according to which topic drop of the 1st person singular should be more frequent, in particular, before verbs with a distinct inflectional ending,  with two further ones regarding verb type and verb surprisal.

(iii--\textsc{Surprisal}) 
It is exclusively the information-theoretic account that predicts an influence of verb surprisal on topic drop, namely a main effect.
Following from the \textit{avoid peaks} principle (\sectref{sec:avoid.peaks}), topic drop should be less frequent if the following verb has a high surprisal.
In this case, the processing effort \is{Processing effort|(} of this verb is more likely to exceed the hearer's cognitive capacity. \is{Channel capacity}
Therefore, to prevent processing difficulties for the hearer, the speaker could  make the verb more predictable \is{Predictability} and avoid the effort  associated with ellipsis resolution by using the full form with an overt pronoun in the prefield.

(iv--\textsc{Verb Type})
From the theoretical claim in the IDS grammar \citep[415]{zifonun.etal1997}, we can derive the prediction of an interaction between the predictors \textsc{Verb Type} and grammatical \textsc{Person}.
Topic drop of the 1st person should be particularly frequent before copular, auxiliary, and modal verbs.
From the information-theoretic perspective, it may be that some verb types are generally less predictable \is{Predictability} in the left bracket and this may impact whether the prefield constituent is realized or omitted.
Thus, the information-theoretic account could explain an effect of \textsc{Verb Type} in addition to the more fine-grained surprisal effect.

\subsubsection{Results}
Details on the logistic regression analysis of all verb types can be found in \sectref{sec:corpus.inference.results}.
The final model (repeated here as Table \ref{tab:frac.mess.part.model.rep}) contained a significant interaction between grammatical \textsc{Person} and \textsc{Inflection Informed} ($\chi^2(1) = 5.85, p < 0.05$), a significant interaction between \textsc{Person} and \textsc{Verb Type} ($\chi^2(1) = 4.72, p < 0.05$), and significant main effects of \textsc{Verb Type} ($\chi^2(1) = 6.3, p < 0.05$) and \textsc{Surprisal} ($\chi^2(1) = 17.85, p < 0.001$).
Topic drop of a 1st person singular subject pronoun is more likely if the following verb is practically unambiguous and if it is a copular, auxiliary, or modal verb.
Topic drop is generally, i.e., independently of \textsc{Person}, less likely before copular, auxiliary, and modal verbs (see Figure \ref{fig:frac.corpus.plot.freq.person.verbtype}).
Additionally, the likelihood of topic drop decreases with the surprisal of the verb in the left bracket.

\begin{table}
\centering
\caption{Fixed effects in the final model analyzing \textsc{FraC-TD-SMS-Part} (repeated from page \pageref{tab:frac.mess.part.model})}
\begin{tabular}{lrrrll}
\lsptoprule
Fixed effect & Est. & SE & $\chi^2$ & \textit{p}-value &   \\
\midrule
\textsc{Intercept} & $2.81$ & $0.57$ & $27.20$ & $< 0.001$ & ***\\
\textsc{Person} & $0.35$ & $0.29$ & $1.46$ & $> 0.2$ & \\
\textsc{Surprisal} & $-0.27$ & $0.07$ & $17.85$ & $< 0.001$ & ***\\
\textsc{Verb Type} & $0.95$ & $0.39$ & $6.30$ & $< 0.05$ & *\\
\textsc{Inflection Informed} & $-0.33$ & $0.31$ & $1.15$ & $> 0.2$ & \\
\textsc{Person $\times$ Verb Type} & $-1.30$ & $0.61$ & $4.72$ & $< 0.05$ & *\\
\textsc{Person $\times$ Inflection Informed} & $1.47$ & $0.63$ & $5.85$ & $< 0.05$ & *\\
\lspbottomrule
\end{tabular}
\label{tab:frac.mess.part.model.rep}
\end{table}

\begin{figure}
\centering
\includegraphics[scale=1]{Korpusplots/Corpus_FraCTDMessPart_PersonVerbtype.pdf}
\caption{Frequency of the full forms and the instances of topic drop as a function of grammatical \textsc{Person} and \textsc{Verb Type} in the \textsc{FraC-TD-SMS-Part} data set}
\label{fig:frac.corpus.plot.freq.person.verbtype}
\end{figure}

\noindent
In the second step, I conducted a similar logistic regression analysis of only the lexical verbs.
The full model predicted the likelihood of topic drop from grammatical \textsc{Person}, \textsc{Surprisal}, and \textsc{Inflection Informed}, as well as from their two-way interactions.
For reasons of data sparsity, I was unable to include the contrast \textsc{Inflection Distinct}.
The final model contained a significant main effect of verb surprisal ($\chi^2(1) = 13.85, p < 0.001$), as shown in Table \ref{tab:frac.mess.part.model.2}.
Even when only the lexical verbs are considered, the likelihood of topic drop decreases if the verb surprisal increases.
All other effects were not significant.
That is, for only the instances with lexical verbs, neither grammatical \textsc{Person} nor \textsc{Inflection informed} impacted the likelihood of topic drop.

\begin{table}
\centering
\caption{Fixed effects in the final model analyzing only the instances with a lexical verb in \textsc{FraC-TD-SMS-Part}}
\begin{tabular}{lrrrll}
\lsptoprule
Fixed effect & Est. & SE & $\chi^2$ & \textit{p}-value &   \\
\midrule
\textsc{Intercept} & $3.21$ & $0.79$ & $19.06$ & $< 0.001$ & ***\\
\textsc{Surprisal} & $-0.28$ & $0.08$ & $13.85$ & $< 0.001$ & ***\\
\lspbottomrule
\end{tabular}
\label{tab:frac.mess.part.model.2}
\end{table}

\subsubsection{Discussion}\label{sec:corpus.inference.rep.diss}
In the main analysis, I found a significant interaction between verb type and grammatical person, which supports the prediction (iv--\textsc{Verb Type}), motivated by the claim of the IDS grammar \citep{zifonun.etal1997}.
Topic drop of the 1st person singular indeed seems to be more likely before copular, auxiliary, and modal verbs.
The frequency table presented in \sectref{sec:corpus.surprisal} suggests that this effect is mainly caused by the copular verbs.%
%% Footnote
\footnote{The numbers for the individual verb types are too small for statistical comparison, especially for the 3rd person singular.}
%%
Topic drop of the 1st person singular is only particularly frequent before this verb type compared to the 3rd person singular.
This interaction is qualified by the unexpected main effect of \textsc{Verb Type} in the opposite direction.
When averaging over the other predictors, topic drop before copular, auxiliary, and modal verbs is generally less likely than before lexical and reflexive verbs.
Moreover, since there are relatively few cases with the 3rd person singular in the data for both groups of verb types overall (see again Figure \ref{fig:frac.corpus.plot.freq.person.verbtype}), the interaction should be interpreted cautiously.
The claim from the IDS grammar should be investigated in future research on a larger data set.

In line with prediction (iii--\textsc{Surprisal}), both analyses evidenced a significant main effect of the surprisal of the following verb on the frequency of topic drop.
Topic drop is less likely before a verb with a high surprisal than before a verb with a low surprisal.
I found this effect when looking at all verb types and when looking at only the lexical verbs.
It provides support for the information-theoretic account of the usage of topic drop, in particular for the \textit{avoid peaks} principle.
The writers of text messages are indeed more likely to overtly realize the prefield constituent if the following verb has a high surprisal, i.e., if it causes higher processing effort for the receiver.
On the one hand, the overt prefield constituent saves the receiver the cognitive effort required to resolve the ellipsis on the verb.
On the other hand, it can also make the following verb more predictable \is{Predictability} and, thus, reduce its surprisal.
In this way, the surprisal peak on the verb can be lowered and the receiver's processing can be facilitated.
This result of a consistent surprisal effect in both corpus analyses provides the first genuine empirical support for my information-theoretic account of topic drop usage.

In \sectref{sec:corpus.inference.diss}, I already discussed the results of the regression analysis with respect to grammatical person and verbal inflection.
There, I argued that the likelihood of topic drop is higher for the 1st person singular than for the 3rd person singular if it precedes a verb form perceived as unambiguous.
In the analysis of only the lexical verbs, which I presented in this section, I did not replicate this result.
One reason could be that the observed effects are primarily due to the other verb types.
Another reason could be the small size of the data set.
More in-depth studies with larger data sets are needed here.

In summary, the results of the corpus analysis are only partially consistent with the predictions from the IDS grammar for verb type.
Even if larger data sets in future studies were to provide clearer evidence, it would nevertheless remain an open question why topic drop of the 1st and 2nd person should be more frequent before copular, auxiliary, and modal verbs.
\citet{zifonun.etal1997} do not provide a theoretical explanation for their prediction.
In contrast, my information-theoretic approach does explain the attested influence of verb surprisal on the frequency of topic drop.
Two analyses showed that the rate of topic drop decreases as the surprisal of the verb in the left sentence bracket increases and, thus, the effort required to process it.
This result is consistent with the predictions of the \textit{avoid peaks} principle and provides the first genuine evidence for the information-theoretic approach since the effect cannot be explained by any theoretical approach to the usage of topic drop discussed in the literature so far.
\is{Lexical verb|)} \is{Copula|)} \is{Auxiliary|)} \is{Modal verb|)} \is{Reflexive verb|)} \is{Unigram surprisal|)}

\newpage
\section{Experimental investigations of verb surprisal}\label{sec:exps.surprisal}
\refstepcounter{expcounter}\label{exp:surprisal}
The corpus study on the text message subcorpus of the FraC evidenced an effect of verb surprisal on the frequency of topic drop (see \sectref{sec:corpus.regression.rep}).
Topic drop was rarer if the following verb had a high surprisal.
In two experiments, I attempted to show the effect of verb surprisal that I found in the production of topic drop also in its perception.
To this end, I tried to explicitly manipulate the surprisal of the following verb in a controlled setting.
\is{Corpus|)}

\subsection{Experiment \arabic{expcounter}: verb surprisal (context)}
\is{Acceptability rating study|(}\label{sec:exp.surprisal}
In the first experiment, I used the likelihood of the verb given the preceding linguistic context to manipulate its surprisal.\footnote{The items and fillers, as well as the analysis scripts of both the pretest and the main experiment can be found online: \url{https://osf.io/zh7tr}.}

To this effect, I exploited the stereotypical association between certain professions and typical actions or tasks that I had beforehand confirmed with a pretest.
The assumption is that the knowledge about the profession of a person makes it more or less likely for this person to perform a certain action or task, impacting the surprisal of the verb denoting this action or task.
For example, \textit{to program} should be more likely for a computer scientist than for a bricklayer, and, thus, have a lower surprisal.
If we consider again the \textit{avoid peaks} principle discussed in \sectref{sec:avoid.peaks}, this means that topic drop should be more likely and, thus, more acceptable if the omitted subject before \textit{to program} is a pronoun referring to a computer scientist than if it refers to a bricklayer.
I argue that this is because, in the case of the bricklayer, the overt subject pronoun in the prefield may be needed to lower the processing effort on the verb by, first, indicating structurally that a congruent finite verb will follow and, second, by saving the effort \is{Processing effort|)} required to resolve the ellipsis.
The main experiment had the form of a 2 $\times$ 2 design crossing the factors \textsc{Completeness} (topic drop vs. full form) and \textsc{Profession} (predictive vs. non-predictive for the main verb).
With a pretest, I assessed whether the relations between professions and verbs do indeed hold.

\subsubsection{Pretest}\label{sec:exp.surprisal.pretest}
The pretest was intended to ensure that for each token set, the verb in the target utterance was indeed more likely after a predictive profession than after a non-predictive profession.
Using the example from above, \textit{programmieren} (`to program') should be more likely after \textit{Informatiker} (`computer scientist') than after \textit{Maurer} (`bricklayer').
I presented the participants the materials in either the predictive or the non-predictive condition (predictor \textsc{Profession}) and assessed the likelihood of (the beginning of) the answer, given the question and the introduction.

\subsubsubsection{Items}
I constructed 38 token sets such as \ref{ex:verb.surprisal.item.pt}, which were always built around a pair consisting of a profession such as \textit{Informatiker} (`computer scientist') and a verb denoting an action that is typically performed by a person carrying on this profession such as \textit{programmieren} (`to program').
This pair was then complemented by a second profession that typically does not perform this action as part of their job such as \textit{Maurer} (`bricklayer').

\ex.\label{ex:verb.surprisal.item.pt}
\a.\label{ex:verb.surprisal.item.context.pt}
\ag.\label{ex:verb.surprisal.item.context.p.pt}Anna schreibt mit Jan, der Informatiker ist:\\
Anna writes with Jan who computer.scientist is\\
`Anna is texting with Jan, who is a computer scientist:'\hfill(predictive)
\bg.\label{ex:verb.surprisal.item.context.np.pt}Anna schreibt mit Jan, der Maurer ist:\\
Anna writes with Jan who bricklayer is\\
`Anna is texting with Jan, who is a bricklayer:'\hfill(non-predictive)
\z.
\bg.\label{ex:verb.surprisal.item.question.pt}Anna: Was gibt's Neues bei dir?\\
{} what gives.it new at you.\textsc{dat.2sg}\\
Anna: `What's new with you?'
\cg.\label{ex:verb.surprisal.item.answer.pt}Jan: Ich programmiere \#\#\#\#\#.\\
{} I program \\
Jan: `I am programming \#\#\#\#\#'

The items consisted of an introductory sentence and a question-answer pair.
The first sentence \ref{ex:verb.surprisal.item.context.pt} introduced the two interlocutors, e.g., Anna and Jan, and specified the profession of the second person, e.g., Jan was either a computer scientist in the predictive condition \ref{ex:verb.surprisal.item.context.p.pt} or a bricklayer in the non-predictive condition \ref{ex:verb.surprisal.item.context.np.pt}.
This sentence was then followed by a general question asked by the first person, e.g., Anna \ref{ex:verb.surprisal.item.question.pt}.
The second person, e.g., Jan, answered this question with a declarative V2 clause with a finite lexical main verb in the left bracket, e.g., \textit{programmiere} \ref{ex:verb.surprisal.item.answer.pt}.
This verb was the same for the two \textsc{Profession} conditions, but in the predictive condition, it matched the profession, while it did not in the non-predictive condition.
In the pretest, I presented the answer as the full form, i.e., with the 1st person singular subject pronoun \textit{ich} in the prefield, and  presented the placeholder \#\#\#\#\# after the finite verb, as shown in Figure \ref{fig:pretest.verbsurprisal}.

\begin{figure}
\centering
\includegraphics[scale=0.4]{Pretest_Verbsurprisal.png}
\caption[Presentation of the pretest of experiment \arabic{expcounter}]{Presentation of the pretest of experiment \arabic{expcounter} using the item shown in \ref{ex:verb.surprisal.item.pt}}
\label{fig:pretest.verbsurprisal}
\end{figure}

\subsubsubsection{Procedure}
The experiment was conducted online using LimeSurvey \citep{limesurveygmbh}.
I recruited 48 participants, native speakers of German between the ages of 18 and 40, from Clickworker \citep{clickworker2022}, who were paid €1.30.
They read the introductory sentence with the profession, the question by person A, and the answer by person B cut off after the verb in the left bracket.
The participants' task was to rate how probable it is that the answer of person B would begin like this, given person A's question and the introductory sentence.
They rated this likelihood using a slider scale with labeled endpoints (from ``vollkommen unwahrscheinlich'' (`completely unlikely') to ``vollkommen wahrscheinlich'' (`completely likely')).
The slider ranged from 0 to 100, but the values were not visible to participants, i.e., they had to orient themselves only visually, as shown in Figure \ref{fig:pretest.verbsurprisal}.

For each participant, the experiment started with the same two filler trials which were not analyzed.
They mimicked the critical items and had the purpose of familiarizing the participants with the task.
The items were mixed with 9 catch trials, where the main verb in the answer was a verb that usually has no human subject, such as \textit{laichen} (`to spawn') or \textit{glimmen} (`to smolder').
Consequently, it should be unlikely for an utterance to start with \textit{ich} plus the congruent form of such a verb.

The items were distributed across two lists so that each participant rated 19 predictive and 19 non-predictive conditions.
Due to an error, there were actually 20 predictive and 18 non-predictive items in one list, because a token set appeared in both lists in the predictive condition.
This token set was excluded from further analysis.
The materials were presented in individually randomized order.

\subsubsubsection{Results}
I excluded the data from two participants whose mean score for the catch trials deviated by more than 2.5 standard deviations from the overall mean score for the catch trials.
Table \ref{tab:verb.surprisal.pretest} shows the mean ratings and standard deviations for the two conditions of the items and the catch trials.
The items received higher scores than the catch trials in both conditions and the items received higher scores in the predictive condition than in the non-predictive condition.

\begin{table}
\caption{Descriptive overview of the likelihood scores for the items and catch trials in the pretest of experiment \arabic{expcounter}}
\centering
\begin{tabular}{rrr}
\lsptoprule
& \multicolumn{1}{c}{Mean score} & \multicolumn{1}{c}{Standard deviation} \\
\midrule
Items, predictive & $82.82$ & $20.45$ \\ 
Items, non-predictive & $29.77$ & $25.99$ \\ 
Catch trials & $8.31$ &	 $13.45$ \\ 
\lspbottomrule
\end{tabular}
\label{tab:verb.surprisal.pretest}
\end{table}

This visual impression was supported by the statistical analysis.
I fitted a linear mixed effects regression model using the lme4 package \citep{bates.etal2015} in R with random intercepts for participants%
% Footnote
\footnote{More complex random effects structures resulted in singular fits.}
%
to compare the ratings for the critical items to the ratings for the catch trials.
I modeled the numeric likelihood scores as a function of the dummy coded predictor \textsc{Stimulus Type} with catch trial as the reference level, which was compared to the critical items.
The full and likewise final model (see Table \ref{tab:verb.surprisal.pretest.model.1}) revealed a significant main effect of \textsc{Stimuli Type} indicating that the items were generally rated higher than the fillers ($\chi^2(1) = 601.49, p < 0.001$).

\begin{table}
\caption{Fixed effect in the final LMER of the pretest of experiment \arabic{expcounter} considering \textsc{Stimulus Type}}
\centering
\begin{tabular}{lrrrll}
\lsptoprule
Fixed effect & Est. & SE & $\chi^2$ & \textit{p}-value &   \\
\midrule
\textsc{Stimulus Type.Item} & $47.99$ & $1.81$ & $601.49$ & $< 0.001$ & ***\\
\lspbottomrule
\end{tabular}
\label{tab:verb.surprisal.pretest.model.1}
\end{table}

With a second linear mixed effects regression model, I compared the ratings for the two \textsc{Profession} conditions with each other.
I focused on only the items and included the likelihood scores as the dependent variable and \textsc{Profession} as the independent variable using dummy coding with non-predictive as the reference level.
In the full and likewise final model (see Table \ref{tab:verb.surprisal.pretest.model.2}), the condition \textsc{Profession} had a significant main effect on the probability scores ($\chi^2(1) = 1602.8, p < 0.001$).
The beginning of the answer received higher probability scores in the predictive condition than in the non-predictive condition.
These results indicate that overall the manipulation worked in the intended way.

\begin{table}
\caption{Fixed effect in the final LMER of the pretest of experiment \arabic{expcounter} considering  \textsc{Profession}}
\centering
\begin{tabular}{lrrrll}
\lsptoprule
Fixed effect & Est. & SE & $\chi^2$ & \textit{p}-value &   \\
\midrule
\textsc{Profession.predictive} & $53.03$ & $1.02$ & $1602.8$ & $< 0.001$ & ***\\
\lspbottomrule
\end{tabular}
\label{tab:verb.surprisal.pretest.model.2}
\end{table}

\noindent
On the level of individual token sets, there were clear differences though, as can be seen from Figure \ref{fig:vt.pretest.results}.%
%% Footnote
\footnote{It should be noted that some of the German equivalents of the verbs have a clearer meaning and therefore match the occupations in the condition better than is suggested by the English translation.}
%
It shows the mean likelihood scores per token set and \textsc{Profession} condition.
For convenience, I added the verbs and the professions in the predictive and non-predictive conditions to the plot.

\begin{figure}
\centering
\includegraphics[scale=0.8]{Experimenteplots/VS_Pretest_per_TS.pdf}
\caption[Likelihood scores of the 32 token sets selected based on the pretest as a function of the \textsc{Profession} predictor, including verbs and professions]{Likelihood scores of the 32 token sets selected based on the pretest as a function of the \textsc{Profession} predictor, including verbs and professions}
\label{fig:vt.pretest.results}
\end{figure}

For example, token set 4 compared the likelihood of the verb \textit{komponieren} (`to compose') for the predictive profession \textit{Musiker} (`musician') to the non-predic- tive profession \textit{Surflehrer} (`surf instructor').
The mean rating for the predictive condition was 89, while it was 11 for the non-predictive one, resulting in a difference of 78 points.
In contrast, the predictive condition of token set 26 received a mean rating of 66 (\textit{Kosmetikerin} (`cosmetologist.\textsc{fem}') and \textit{zupfen} (`to pluck')), the non-predictive condition one of 47 (\textit{Winzerin} (`winemaker.\textsc{fem}') and \textit{zupfen} (`to pluck')), which made a difference of only 19 points.
To account for these differences between token sets and to make the contrast between the two item conditions as obvious as possible, I decided to reduce the token sets to those 32 where the difference in terms of z-scores between the predictive and the non-predictive condition was the highest.
For these 32 token sets, the difference in the absolute scores between the predictive and the non-predictive condition ranged between a minimum of 37.91 and a maximum of 78.04.
There is a clear dichotomy between predictive and non-predictive conditions, which justifies using a categorical predictor for \textsc{Profession} in the analysis of the rating study in \sectref{sec:exp.surprisal.results} below.


\subsubsection{Materials}
\subsubsubsection*{Items}
The 32 token sets selected by the pretest were completed by replacing the placeholder \#\#\#\#\# with a meaningful continuation, as shown in example \ref{ex:verb.surprisal.item}.
For the target utterance, I varied whether the prefield constituent, the 1st person singular pronoun \textit{ich} (`I'), was realized overtly or covertly, i.e., whether the utterance contained topic drop or not.

\ex.\label{ex:verb.surprisal.item}
\a.\label{ex:verb.surprisal.item.context}
\ag.\label{ex:verb.surprisal.item.context.p}Anna schreibt mit Jan, der Informatiker ist:\\
Anna writes with Jan who computer.scientist is\\
`Anna texts with Jan, who is a computer scientist:'\hfill(predictive)
\bg.\label{ex:verb.surprisal.item.context.np}Anna schreibt mit Jan, der Maurer ist:\\
Anna writes with Jan who bricklayer is\\
`Anna texts with Jan, who is a bricklayer:'\hfill(non-predictive)
\z.
\bg.\label{ex:verb.surprisal.item.question}Anna: Was gibt's Neues bei dir?\\
{} what gives.it new at you.\textsc{dat.2sg}\\
Anna: `What's new with you?'
\cg.\label{ex:verb.surprisal.item.answer}Jan: (Ich) programmiere gerade eine Software.\\
{} I program just a software\\
Jan: `(I) am currently programming a piece of software.'\\\phantom{.}\hfill (topic drop / full form)

\subsubsubsection*{Fillers}
Alongside the 32 critical items for this experiment, I tested a total of 64 fillers in the actual rating study, which had the same overall structure as the items:
an introductory sentence followed by a question-answer pair.
16 fillers were the items of another experiment on preposition omission.
The introductory sentence specified that one person is texting with a group of people.
The question-answer pair contained a fragmentary answer for which the definiteness and the presence or absence of a preposition were varied.
24 fillers were question-answer pairs with V2 or verb-final subordinate clauses with (potential) gapping structures in the answers, where the introductory sentence explained where the two interlocutors know each other from.
The final group of fillers were 24 further question-answer pairs.
The introductory sentence explained again where the two interlocutors know each other from, while the answer was a declarative sentence with an adverbial in the prefield.

\subsubsection{Procedure}
The main rating study was conducted as an online study implemented with LimeSurvey \citep{limesurveygmbh}.
48 participants, native German speakers (18 to 40 years old) who had not participated in any other of my topic drop experiments, were recruited from Clickworker \citep{clickworker2022} and received €4.00.
Their task was to rate the naturalness of the answer in the context of the question and the introductory utterance on a 7-point Likert scale (7 = completely natural).
The 32 items selected based on the pretest were distributed across four lists according to a Latin square design and mixed with 64 fillers.
Each participant saw the materials in an individually pseudo-randomized order, ensuring that no two items immediately followed each other.
The question-answer pairs of all materials were presented as instant messaging dialogues, as shown for the pretest in Figure \ref{fig:pretest.verbsurprisal}.

\subsubsection{Results}\label{sec:exp.surprisal.results}
Table \ref{tab:vs.ratings} shows the mean ratings and standard deviations per condition.
In Figure \ref{fig:pl.vs}, the mean ratings and 95\% confidence intervals are plotted.

\begin{table}
\caption{Mean ratings and standard deviations per condition for experiment \arabic{expcounter}}
\centering
\begin{tabular}{lrrrll}
\lsptoprule
\multicolumn{1}{c}{\textsc{Completeness}} & \multicolumn{1}{c}
    {\textsc{Profession}} & \multicolumn{1}{c}{\Centerstack{Mean\\
    rating}} & \multicolumn{1}{c}{\Centerstack{Standard\\ deviation}} \\
\midrule
Full form & Predictive & $5.41$ & $1.55$ \\
Topic drop & Predictive & $5.49$ & $1.53$ \\
Full form & Non-predictive & $3.40$ & $1.89$ \\
Topic drop & Non-predictive & $3.34$ & $1.85$ \\
\lspbottomrule
\end{tabular}
\label{tab:vs.ratings}
\end{table}

\begin{figure}
\centering
\includegraphics[scale=1]{Experimenteplots/PL_VS.pdf}
\caption{Mean ratings and 95\% confidence intervals per condition for experiment \arabic{expcounter}}
\label{fig:pl.vs} % pl for point line
\end{figure}

Their visual inspection indicates an enormous difference between the two \textsc{Profession} conditions, which seems to be independent of the \textsc{Completeness} condition.
Utterances in the non-predictive conditions were clearly degraded, regardless of whether they were syntactically complete or contained topic drop.

I analyzed the data with CLMMs in R \citep{christensen2019}, as described in \sectref{sec:data.analysis}.
The full model contained the ratings as the dependent variable and as the independent variables \textsc{Completeness} and \textsc{Profession},%
%% Footnote
\footnote{As discussed above in \sectref{sec:exp.surprisal.pretest}, I used a categorical predictor for \textsc{Profession} because there was a clear split between predictive and non-predictive conditions in the results of the pretest.}
%
coded with deviation coding (full form and predictive coded as $0.5$, topic drop and non-predictive coded as $-0.5$), as well as the numeric scaled and centered \textsc{Position} of the trial in the experiment and all two-way interactions between the predictors.
The random effects structure consisted of random intercepts for participants and items and of by-participant and by-item random slopes for \textsc{Completeness}, \textsc{Profession}, and their interaction.%
% Footnote
\footnote{The formula of the full model was as follows: \texttt{Response \textasciitilde\ (\textsc{Completeness} + \textsc{Profession} + \textsc{Position})\textasciicircum2 + (1 + \textsc{Completeness} * \textsc{Profession} | Subjects) + (1 + \textsc{Completeness} * \textsc{Profession}| Items)}.
Models including random slopes for \textsc{Position} did not converge.}

The final model obtained with a backward model selection had symmetric thresholds and contained the following significant fixed effects (see Table \ref{tab:model.exp.vs}):
a significant main effect of \textsc{Profession} ($\chi^2(1) = 44.51, p < 0.001$), indicating that the predictive conditions were highly preferred over the non-predictive conditions and an interaction between \textsc{Profession} and \textsc{Position} ($\chi^2(1) = 15.3, p < 0.001$), according to which the ratings for the predictive conditions improved in the course of the experiment, just as the ratings for the non-predictive conditions became worse.
The latter result suggests that the participants adapted their ratings more and more to the \textsc{Profession} manipulation.
The interaction between \textsc{Completeness} and \textsc{Profession} ($\chi^2(1) = 1.55, p > 0.2$) was not significant nor were any other effects.

\begin{table}
\caption{Fixed effects in the final CLMM of experiment \arabic{expcounter}}
\centering
\begin{tabular}{lrrrll}
\lsptoprule
Fixed effect & Est. & SE & $\chi^2$ & \textit{p}-value &   \\
\midrule
\textsc{Profession} & $3.08$ & $0.38$ & $44.51$ & $< 0.001$ & *** \\
\textsc{Position} & $0.02$ & $0.05$ & $0.12$ & $> 0.7$ & \\
\textsc{Profession $\times$ Position} & $0.40$ & $0.10$ & $15.30$ & $< 0.001$ & *** \\
\lspbottomrule
\end{tabular}
\label{tab:model.exp.vs}
\end{table}

\subsubsection{Discussion}
Experiment \arabic{expcounter} was intended to test the effect of verb surprisal on the acceptability of topic drop by exploiting the assumed strong association between a certain profession and a typical activity performed by this profession.
The idea was that a verb denoting the typical activity has a lower surprisal in the context of the predictive profession than in the context of another unrelated, non-predictive profession.
As a consequence, topic drop should be more acceptable before the verb in the context of the predictive profession than in the context of the non-predictive profession.
This is because, after the non-predictive profession, it may be necessary to reduce the processing effort \is{Processing effort} on the verb by inserting an overt preverbal constituent.
This constituent is argued to reduce the peak on the verb because it may increase the likelihood of that verb and it saves the effort \is{Processing effort} required to resolve the ellipsis. 

The results of the experiment do not support this hypothesis.
Topic drop and the full forms were rated comparably.
The ratings for both \textsc{Completeness} conditions were equally impacted by the manipulation of \textsc{Profession} but this in a massive way.
Participants more strongly preferred the target utterances in the context of a predictive profession than in the context of the non-predictive profession, i.e., when the action denoted by the verb was typical for this profession.
This preference even increased in the course of the experiment.
It could be the case that this massive effect of the \textsc{Profession} predictor masked any potential influence that the verb surprisal might have on the acceptability of topic drop, i.e., that participants focused on the different professions, neglecting the variation between full forms and topic drop.
More abstractly, the manipulation relying on world knowledge and pragmatics might have overridden the more subtle variation in syntactic form.
For this reason, I chose a different way to manipulate verb surprisal in the following experiment. \is{Acceptability rating study|)}

\refstepcounter{expcounter}\label{exp:surprisal.vt}
\subsection{Experiment \arabic{expcounter}: verb surprisal (verb type)}
\is{Acceptability rating study|(}\label{sec:exp.surprisal.vt}
Experiment \ref*{exp:surprisal} indicated no effect of verb surprisal on the usage of topic drop, but this might be caused by properties of the experimental design.
It might be that the context manipulation had such a massive effect on the acceptability of the target utterances that any possible difference between full forms and topic drop was masked by it.
Therefore, in experiment \arabic{expcounter}, I manipulated the verb surprisal not via the utterance context but via the verb type used, which I varied between a lexical and an auxiliary verb. \is{Lexical verb|(} \is{Auxiliary|(}
This resulted in a 2 $\times$ 2 acceptability rating study, crossing \textsc{Completeness} (full form vs. topic drop) and \textsc{Verb Type} (lexical verb vs. auxiliary).%
%% Footnote
\footnote{All items, fillers, and the analysis script can be accessed online: \url{https://osf.io/zh7tr}.}


In \sectref{sec:corpus.surprisal}, I showed that there is a correlation between verb type and verb surprisal.
Since auxiliary verbs such as \textit{haben} (`have'), which are used to build the perfect for most verbs, are very frequent, they generally have a low unigram surprisal.
Lexical verbs, in turn, vary more strongly in terms of their surprisal, but they are generally expected to be less frequent and, thus, have a higher surprisal.
It follows from the \textit{avoid peaks} principle (\sectref{sec:avoid.peaks}) that topic drop should on the whole be more acceptable before unsurprising auxiliary verbs than before more surprising lexical verbs.
Recall that also \citet[415]{zifonun.etal1997} claim that topic drop of the 1st and the 2nd person is more frequent or more acceptable before  copular, auxiliary, and modal verbs, however without substantiating this claim (see \sectref{sec:usage.verb.type.theory}).
Since in this experiment I only tested utterances with topic drop where the 1st person singular pronoun \textit{ich} (`I') was omitted, this claim would likewise predict that topic drop before auxiliaries should be rated as more acceptable than topic drop before lexical verbs.

\subsubsection{Materials}
\subsubsubsection*{Items}
I adapted the 32 items of experiment \ref*{exp:surprisal} by varying the tense in the target utterance between the original present tense \ref{ex:verb.surprisal.vt.item.p} and the present perfect \ref{ex:verb.surprisal.vt.item.pp}.
As a result, the verb in the left bracket was once a lexical verb with a higher surprisal and once an auxiliary verb with a lower surprisal, namely \textit{haben} (`have').%
% Footnote
\footnote{While in German some verbs form the perfect with \textit{sein} (`be'), all verbs in this experiment use \textit{haben}.}
%
Additionally, I removed the introductory sentence and changed the temporal adverbial in the prefield for some items to make the perfect conditions more natural.
For example, I changed \textit{gerade} (`just') to \textit{diese Woche} (`this week') in \ref{ex:verb.surprisal.vt.item}.

\ex.\label{ex:verb.surprisal.vt.item}
\ag.\label{ex:verb.surprisal.vt.item.question}Anna: Was gibt's Neues bei dir?\\
{} what gives.it new at you.\textsc{dat.2sg}\\
Anna: `What's new with you?'
\b.
\ag.\label{ex:verb.surprisal.vt.item.p}Jan: (Ich) programmiere diese Woche eine Software.\\
{} I program this week a software\\
Jan: `(I) am programming a piece of software this week.'\\\phantom{.}\hfill(lexical verb)
\bg.\label{ex:verb.surprisal.vt.item.pp}Jan: (Ich) habe diese Woche eine Software programmiert.\\
{} I have this week a software programmed\\
Jan: `(I) have programmed a piece of software this week.'\\\phantom{.}\hfill(auxiliary verb)

\subsubsubsection*{Fillers}
The same 64 fillers as in experiment \ref*{exp:surprisal} were used, complemented by 8 ungrammatical catch trials.

\subsubsection{Procedure}
Another 48 native speakers of German  were recruited from Clickworker  \citep{clickworker2022} and received €4.00 to participate in the study, which was again implemented with LimeSurvey \citep{limesurveygmbh}.
They were between the ages of 18 and 40 and had not taken part in any of my previous studies of topic drop.
Their task was again to rate the last utterance of the instant messaging dialogues in terms of its naturalness on a 7-point Likert scale (7 = completely natural).
The 32 items were distributed across 4 lists according to a Latin square design to ensure that each participant rated each token set only once and in one condition.
The items were mixed with the 64 fillers and presented in individual pseudo-randomized order as instant messaging dialogues.

\subsubsection{Results}
I excluded 5 participants who had rated 4 or more of the 8 ungrammatical catch trials as (almost) completely natural, i.e., with 6 or 7 on the 7-point Likert scale.
Table \ref{tab:vs.vt.ratings} and Figure \ref{fig:pl.vs.vt} provide an overview of the aggregated data for the remaining 43 participants, broken down by condition.
It looks as if there is no interaction, just a difference for \textsc{VerbType} and one for \textsc{Completeness}.
Full forms seem to be more acceptable than topic drop and, unexpectedly, utterances with lexical verbs seem to be preferred over utterances with auxiliaries.

\begin{table}
\caption{Mean ratings and standard deviations per condition for experiment \arabic{expcounter}}
\centering
\begin{tabular}{llrr}
\lsptoprule
\multicolumn{1}{c}{\textsc{Completeness}} & \multicolumn{1}{c}{\textsc{Verb Type}} & \multicolumn{1}{c}{\makecell{Mean\\ rating}} & \multicolumn{1}{c}{\makecell{Standard\\ deviation}}\\
\midrule
Full form & Lexical verb & $5.16$ &  $1.79$\\
Topic drop & Lexical verb & $5.36$ &  $1.65$\\
Full form & Auxiliary verb & $4.76$ & $1.79$\\
Topic drop & Auxiliary verb &  $5.02$ & $1.75$ \\
\lspbottomrule
\end{tabular}
\label{tab:vs.vt.ratings}
\end{table}

\begin{figure}
\centering
\includegraphics[scale=1]{Experimenteplots/PL_VS_VT.pdf}
\caption{Mean ratings and 95\% confidence intervals per condition for experiment \arabic{expcounter}}
\label{fig:pl.vs.vt} % pl for point line
\end{figure}

I used CLMMs (package ordinal \citep{christensen2019}) to analyze the data, as presented in \sectref{sec:data.analysis}.
I modeled the ordinal ratings as a function of the numeric \textsc{Position} of the trial in the experiment and of the two binary predictors \textsc{Completeness} and \textsc{VerbType}, as well as of all two-way interactions between them.
\textsc{Position} was scaled and centered.
\textsc{Completeness} and \textsc{VerbType} were coded using deviation coding (full form and lexical verb as $0.5$, topic drop and auxiliary as $-0.5$).
The random effects structure consisted of random intercepts for subjects and items and of by-subject and by-item random slopes for all three predictors and their two-way interactions.%
% Footnote
\footnote{The formula of the full model was as follows: \texttt{Ratings \textasciitilde (\textsc{Completeness} + \textsc{VerbType} + \textsc{Position})\textasciicircum2 + (1 + (\textsc{Completeness} + \textsc{VerbType} + \textsc{Position})\textasciicircum2 | Subjects) + (1 + (\textsc{Completeness} + \textsc{VerbType} + \textsc{Position})\textasciicircum2 | Items)}.}

The final model obtained with a backward model selection had flexible thresholds and contained the fixed effects shown in Table \ref{tab:model.exp.vs.vt}.
There were significant main effects of both binary predictors \textsc{Completeness} ($\chi^2(1) = 4.34, p < 0.05$) and \textsc{Verb Type} ($\chi^2(1) = 10.96, p < 0.001$), but their interaction was not significant ($\chi^2(1) = 0.38, p > 0.5 $).
Full forms were preferred over topic drop and utterances with lexical verbs were rated as more acceptable than utterances with auxiliaries.

\begin{table}
\caption{Fixed effects in the final CLMM of experiment \arabic{expcounter}}
\centering
\begin{tabular}{lrrrll}
\lsptoprule
Fixed effect & Est. & SE & $\chi^2$ & \textit{p}-value &   \\
\midrule
\textsc{Completeness} & $-0.40$ & $0.19$ & $4.34$ & $< 0.05$ & * \\
\textsc{VerbType} & $0.55$ & $0.16$ & $10.96$ & $< 0.001$ & ***\\
\lspbottomrule
\end{tabular}
\label{tab:model.exp.vs.vt}
\end{table}

\subsubsection{Discussion}
Experiment \arabic{expcounter} had the purpose of testing again for an effect of verb surprisal on the acceptability of topic drop.
It relied on the observation that auxiliary verbs are typically more frequent and, therefore, have a lower surprisal than lexical verbs.
Based on the \textit{avoid peaks} principle, this led to the prediction that topic drop should be more acceptable before auxiliary than before lexical verbs.

This prediction was not confirmed by the experimental results, since I found no significant interaction in my data.
Somewhat unexpectedly, participants overall preferred utterances with lexical verbs over utterances with auxiliaries, but there was no difference between full forms and topic drop in this respect.

An explanation for why topic drop was not, as predicted, more acceptable before auxiliary verbs, and for the main effect of \textsc{VerbType} is to assume that the participants considered the form \textit{habe} of the auxiliary, which was used in the experimental items, as less natural than the forms of the lexical verbs because they would have expected the more colloquial form \textit{hab} without final \textit{schwa}.
This may be relevant for two reasons.
First, if the inflected form of the auxiliary verb is unexpected in the given form of communication / text type, \is{Text type} it might not have a lower surprisal than the corresponding lexical verbs.
This means that the acceptability of topic drop might not vary between conditions because there is an equally strong need to insert an overt prefield constituent to facilitate the processing of the following verb. \is{Processing effort}
Second, it could be that the form \textit{habe} is marked in terms of style in the conceptually spoken text type instant messages.
If this were the case, this would account for the main effect, i.e., that the full forms and the instances of topic drop with \textit{habe} are degraded.

To check the assumption that \textit{habe} is unnatural in instant messages, I searched the Mobile Communication Database 2 (MoCoDa2) \citep{beisswenger.etal2020} for both verb forms \textit{habe} and \textit{hab}.
MoCoDa2 is a still growing corpus \is{Corpus|(} of WhatsApp instant messages, which at the time of my search in August 2023 consisted of about $980$ chats with about $38\,000$ messages and $309\,000$ tokens.

A search for \textit{habe} resulted in $683$ matches in $287$ chats while searching for \textit{hab} yielded $1\,300$ results in $436$ chats.
This shows that while \textit{hab} was indeed the preferred form with twice as many matches, \textit{habe} was by no means infrequent as it still occurred in more than a quarter of all chats in the MoCoDa2.
\textit{Habe} was also more frequent in this corpus than the forms of all lexical verbs used in the experiment.
Of these, the most frequent was \textit{schreibe} (`write.\textsc{1sg}') with 40 occurrences, while several others such as \textit{komponiere} (`compose.\textsc{1sg}'), \textit{entwerfe} (`design.\textsc{1sg}') and \textit{observiere} (`keep.\textsc{1sg} under surveillance') did not occur at all in the corpus.
This indicates that although \textit{habe} may be less frequent than the more colloquial \textit{hab} in WhatsApp messages, it was still a lot more frequent than all of the inflected forms of the lexical main verbs.
Consequently, it should also have a lower unigram surprisal.
Still, though, topic drop was not more acceptable before \textit{habe} than before the lexical verbs.
\is{Corpus|)}

In sum, both experiments \ref*{exp:surprisal} and \arabic{expcounter} failed to experimentally evidence an impact of verb surprisal.
Further research is required to investigate whether such an effect is indeed absent or whether only better methods or research designs are required to find it. \is{Acceptability rating study|)} \is{Auxiliary|)}

\section{Summary: verb type and verb surprisal}
\largerpage
In this chapter on the role of verb type and verb surprisal for topic drop, I first summarized the existing sparse literature on the matter.
While there is so far no explanatory approach in the theoretical literature for a possible effect of the following verb on topic drop, my information-theoretic approach provides one, as repeated below.
\is{Copula|(}\is{Auxiliary|(}\is{Modal verb|(}
For verb type, there are only two unsubstantiated claims in the theoretical literature, one by the IDS grammar \citep{zifonun.etal1997} and one by \citet{imo2013}.
According to \citet{zifonun.etal1997}, subject topic drop of the 1st and 2nd person is supposed to be more frequent before copular, auxiliary, and modal verbs.
\citet{imo2013} claims that topic drop is generally more frequent before verbs of saying and thinking.

Of the four previous corpus \is{Corpus} studies that I discussed, only one provides tentative support for \citeg{imo2013} claim.
Only \citet{helmer2016} found that verbs of saying and thinking are among the four most frequent verb classes to occur with topic drop.
In contrast, all corpus studies are at least consistent with the theoretical claim by \citet{zifonun.etal1997} that subject topic drop of the 1st and the 2nd person is particularly frequent before copular verbs and modal verbs.
The results for topic drop before auxiliaries were mixed (not least because of the surprisingly low number of auxiliary verbs in \citeg{frick2017} corpus) and still need additional investigation. \is{Corpus}

My empirical investigations aimed to examine this more systematically and to investigate the extent to which my information-theoretic account can explain the data.\is{Corpus|(}
In the corpus study, I found no evidence for \citeg{imo2013} hypothesis that topic drop is particularly frequent before verba dicendi and sentiendi.
Concerning the hypothesis by the IDS Grammar \citep{zifonun.etal1997} that in particular 1st and 2nd person subject topic drop is more frequent before copular verbs, auxiliaries, and modal verbs than before lexical verbs, the evidence from my corpus study was mixed.
While 1st person singular topic drop was indeed more frequent before these verb types in my text message data, the visual inspection has suggested that the effect was mainly caused by the copular verbs. \is{Corpus|)}
Since I did not test copular verbs in my experiments,%
%% Footnote
\footnote{In these experiments, I was less interested in the verb type than in the inflection (experiments \ref*{exp:top.s.fv} and \ref*{exp:top.s.mv}) or surprisal (experiment \ref*{exp:surprisal.vt}) of the verb.
Testing copular verbs could be done in a future study.}
%
this might also explain why I did not find comparable effects of verb type there.
In experiment \ref*{exp:surprisal.vt} presented in this chapter, there was no topic drop-specific difference between lexical verbs and auxiliaries.
In \sectref{sec:exp.top.s.mv.person}, I discussed that topic drop was found to be equally acceptable before lexical verbs and before modal verbs in experiments \ref*{exp:top.s.fv} and \ref*{exp:top.s.mv}.
However, those experiments did not focus on the effect of verb type but on whether a distinct inflectional ending on the verb impacts the acceptability of topic drop.
There was a one-to-one mapping between verb type and inflection so their effects could not be separated.
Every form of a modal verb tested  was syncretic \is{Syncretism} and vice versa, just as every form of a lexical verb was distinct and the other way around.
Thus, an explicit experimental investigation of different verb types has yet to be conducted.
\is{Copula|)}\is{Auxiliary|)}\is{Modal verb|)}

For verb surprisal, the corpus \is{Corpus|(} study revisited in \sectref{sec:corpus.regression.rep} provided robust evidence that topic drop is less likely before verbs with a high surprisal, as predicted by the \textit{avoid peaks} principle.
I found this effect in an analysis including all verb types and replicated it in an additional analysis of the lexical verbs only.
The latter analysis rules out potential confounds due to the correlation between surprisal and verb type.
The result provides the first genuine evidence for my \textit{UID}-based information-theoretic account. \is{Corpus|)}

In my experimental studies, experiments \ref*{exp:surprisal} and \ref*{exp:surprisal.vt}, I could not strengthen this evidence.
When I tried an exploratory approach of manipulating the surprisal of individual lexical verbs via the context in experiment \ref*{exp:surprisal}, I did not find a topic drop-specific difference in acceptability.
I argued that the pragmatic manipulation of the likelihood of verbs using world knowledge was probably too strong and thus obscured a possible difference concerning the variation between full forms and topic drop.
As mentioned above, also the manipulation of surprisal via the verb type, auxiliaries as less surprising than lexical verbs, did not result in an effect on the acceptability of topic drop. \is{Lexical verb|)} \is{Auxiliary}
Therefore, experimental evidence of a surprisal effect on topic drop is still pending.

All in all, my information-theoretic approach with the \textit{avoid peaks} principle provides an explanation for the influence of the following verb on the usage of topic drop.
Such an influence has been discussed sporadically in the theoretical literature for the type of this verb but has not yet been theoretically substantiated.
My approach remedies this situation by predicting, on the one hand, an influence of the inflectional ending of this verb on the processing effort \is{Processing effort} of the entire utterance, using the \textit{facilitate recovery} principle discussed in Chapter \ref{ch:usage.person}.
On the other hand, it also states that this processing effort \is{Processing effort} is influenced or, more precisely, can be lowered by the presence of an overt prefield constituent.
I was able to empirically demonstrate such an effect in my corpus study, independent of the influence of verb type.
Thus, I provided genuine evidence for my information-theoretic approach, which should be further extended using different experimental methods.
