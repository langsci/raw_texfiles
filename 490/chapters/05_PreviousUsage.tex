\chapter{Previous accounts of topic drop usage}\label{ch:pusage}
While the first part of this book was concerned with the licensing of topic drop and its syntactic properties, in the second part, I address the question of when speakers \textit{use} topic drop, assuming it is licensed.
Occasionally, I still cover aspects concerning licensing or syntax in this part since many authors do not sharply separate licensing and usage, and such a separation is not always easy either.

Before I propose an information-theoretic account of topic drop usage in Chapter \ref{ch:infotheory}, in this first chapter, I present existing approaches to topic drop usage, approaches that try to explain when and why topic drop is used, and what function it serves.
First, I discuss a classic explanatory approach to ellipsis in general, namely avoiding redundancy and achieving linguistic economy.
Then, I focus on the function of topic drop in creating cohesion, as well as the function of topic avoidance.
Afterward, I discuss what I termed socio-pragmatic functions, which include expressiveness, condensation, action orientation, and the usage in completions, elaborations, and responses, as well as rhetorical and social functions.
Finally, I deal with the potential role of prosody.

\section{Redundancy avoidance and economy}\label{sec:pusage.economy}
\largerpage[-1]
The first explanation of why speakers use topic drop is a classic one, namely ellipsis as a means of economy, which allows speakers to avoid redundancy.
\citet[1]{merchant2001} puts it as follows:
``Elliptical processes capitalize on the redundancy of certain kinds of information in certain contexts, and permit an economy of expression by omitting the linguistic structures that would otherwise be required to express this information.''
According to \citet[103]{schwitalla2012}, a speaker does not need to say more than the hearer needs to understand the intended message.

\citet[167]{helmer2016} applies this reasoning to topic drop and distinguishes two cases related to linguistic economy:
(i) Speakers can omit a constituent if it is redundant because all interlocutors know based on the previous discourse how to understand the elliptical utterance anyway.
(ii) Speakers can refrain from explicating a topic \is{Topic} if it would be uneconomic to boil down a complex, global discourse topic. \is{Discourse topic}
By (ii), \citet{helmer2016} primarily explains the usage of what she terms \textit{indirect analepsis}, and what I termed \textit{indirect topic drop} in Section \ref{sec:recover.ling}, that is, cases where determining the reference requires more complex inference processes.
In (i), one can argue that topic drop is economical both for the speaker and the hearer because the speaker does not need to produce redundancy and the hearer does not need to process it.
However, \citet[45]{helmer2016} argues that from the view of redundancy avoidance, topic drop is at first glance not as easy as possible to understand for the hearer (she quotes \citeg{sperber.wilson1986} relevance theory \is{Relevance theory} here) since the hearer must at least think about what the antecedent \is{Antecedent} of the ellipsis is.%
%% Footnote
\footnote{\citet[45]{helmer2016} points out that it is the cohesive function of topic drop that facilitates processing for the hearer. \is{Processing effort}
See Section \ref{sec:pusage.cohesion} for details.}
%
As there is often one very salient \is{Salience} antecedent \is{Antecedent} in the discourse, this will most of the time not result in increased processing effort. \is{Processing effort}
But this is different for (ii):
(ii) is mainly economical for the speaker as they do not need to explicate a potentially complex concept.
In contrast, the hearer may even be confronted with additional processing effort as they need to infer this intended concept \citep[169--170]{helmer2016}. \is{Processing effort}

\citet{poitou1993} argues that redundancy avoidance is in particular a useful explanation for the omission of the expletive \is{Expletive} \textit{es}, which is semantically empty and, as he puts it, in some sense superfluous since it simply serves the purpose of filling the prefield. \is{Prefield}
However, he also points out that there are limits to the explanation of redundancy avoidance.
Often even shorter expressions would be possible, for example, telegraphese \citep[123]{poitou1993}, but speakers still do not use them.
For instance, in example \ref{ex:td.telegraphese}, topic drop allows the speaker to omit the prefield constituent \textit{ich}, but in telegraphese and potentially also in spoken language the even shorter form \textit{Schlüssel gefunden}, i.e., verbalizing only the object as bare noun plus the verb participle, would be possible.
This shows that for many instances of topic drop, not all constituents that are redundant in some sense are omitted.

\largerpage[-1]

\exg.\label{ex:td.telegraphese}[(Ich) hab den] Schlüssel gefunden.\\
I have the key found\\
`[(I) have] found [the] key.'

However, the limitations of redundancy avoidance as the (only) explanation for the use of topic drop lie not only in the competition with shorter forms but also, conversely, in the presence of full forms.
As \citet[198]{helmer2016}  points out, relevance theory, \is{Relevance theory} for instance, predicts that when topic drop is licensed, it should always be used, with overt pronouns being avoided.
But still, we find cases where speakers or writers use the full form when topic drop would be possible.
This shows that redundancy avoidance as an explanation for the use of topic drop is plausible but falls short as the only explanation.%
%% Footnote
\footnote{As mentioned above, while one can imagine that the most reduced form of example \ref{ex:td.telegraphese} can not only occur in telegraphese but also in spoken language, it will most likely be rather marginal there.
Other reductions in telegrams are probably even impossible in spoken language.
\citet[1124]{tesak.dittmann1991} have therefore considered telegraphese as a register of its own.
\citet[404]{heidolph1992} argues that it has, like headlines, potentially its own grammar enabling special forms of omission.
From this point of view, the argument against redundancy avoidance that has just been put forward can be qualified to some extent.
If a shorter form is not available in a register/grammar, it cannot be used as an alternative to topic drop.
}

\section{Cohesion}\label{sec:pusage.cohesion}
\citet{auer1993}, \citet{sandig2000}, \citet{helmer2016}, and \citet{frick2017} consider topic drop to be a means of building cohesion.
In their seminal book \citetitle{halliday.hasan1976}, \citet{halliday.hasan1976} state that ``[c]ohesion occurs where the \textsc{interpretation} of some element in the discourse is dependent on that of another. The one \textsc{presupposes} the other, in the sense that it cannot be effectively decoded except by recourse to it'' \citep[4, original emphasis]{halliday.hasan1976}.
Halliday and Hasan consider ellipsis to be one of the grammatical resources that establish cohesion in texts.

\citet{auer1993} frames the relationship between cohesion and topic drop in two slightly different ways.
On the one hand, he states that positioning the verb in sentence-initial position can be an alternative to using an anaphoric pronoun with the same function of establishing cohesion to the previous discourse \citep[199]{auer1993}.
On the other hand, he argues that topic drop is used in particular when the corresponding utterance exhibits what he considers to be an extremely strong cohesion to the previous linguistic context or the context of the communicative situation \citep[203]{auer1993}.
The subtle difference between both statements is that in the first case, topic drop is a means to establish cohesion, whereas in the second case, topic drop results from the already existing strong cohesion relation.
Since \citet{auer1993} does not distinguish between the two variants, he might assume that both play a role.
That is, that topic drop is used when there is already cohesion between two utterances and then makes this relationship explicit and reinforces it if necessary.

The latter point is also made by \citet[128--129]{frick2017} in the context of object topic drop.
She states that it is precisely the omission of the prefield constituent that can reinforce the cohesion to the previous discourse because it signals that the referent of this constituent is known to the hearer.
Similarly, \citet[300--301]{sandig2000} argues that the omission of a verb argument \is{Argument} is meant to indicate that the current utterance is still about the referent of that omitted argument, and, thus, indicates local cohesion.
According to \citet[216]{helmer2016}, utterances with topic drop are strongly cohesive to a speaker's preturn, frequently to the immediately adjacent one.%
%% Footnote
\footnote{See also the discussion on the usual small distance between antecedent and target in Section \ref{sec:recover.ling}.}
%
She states that they add information to the precontext instead of establishing a new topic or a new ``communicative project'' and, thus, depend on this precontext.
\citet[45]{helmer2016} argues, again with recourse to \citeg{sperber.wilson1986} relevance theory, \is{Relevance theory} that the cohesive function of topic drop results in a ``positive cognitive effect'' for the hearer by indicating that the current discourse sequence continues. \is{Processing effort}

In sum, topic drop can be seen as a means to enforce an already existing cohesion relation between two utterances by indicating thematic continuity.
I already mentioned that \citet{auer1993} argues that topic drop is similar to the use of an overt anaphoric pronoun in this case.
However, he does not explain when speakers use one and when they use the other.
If topic drop strengthens cohesion and facilitates processing \is{Processing effort} for the hearer, it should always be preferred to the overt pronoun, but, as discussed in Section \ref{sec:pusage.economy}, this is not the case.
\citet{eckert1998.diss} explains the variation between utterances with overt (demonstrative) anaphoric pronouns and utterances with topic drop as a division of tasks.
She states that while ``demonstratives are used when reference is made to the previous utterance as a whole and the predicate supplies new information'', topic drop occurs ``when the semantic information of the verb is somehow given \is{Givenness} (explicitly or implicitly) and the participants expect a statement to be made about the previous utterance (eg [sic!] agreement or disagreement with it)'' \citep[217]{eckert1998.diss}.
I come back to the potential role of the verb for topic drop below in Section \ref{sec:pusage.effects}.
What emerges from Eckert's functional description, based on an analysis of authentic spoken data, is that both utterances with anaphoric pronouns and utterances with topic drop fulfill cohesive functions, but that utterances with topic drop often link more strongly to the previous discourse by commenting on it evaluatively.
I revisit this aspect in Section \ref{sec:pusage.effects}, too.

\section{Topic avoidance}\is{Topic|(} 
\citet{oppenrieder1987}, \citet{auer1993}, \citet{guenthner2006}, \citet{schwitalla2012}, and \citet{helmer2016} discuss that speakers use topic drop to avoid having to designate a constituent in the utterance as the topic and to focus on the comment or rheme.
\citet[179]{oppenrieder1987} speculates that topic drop might be a means to indicate that all remaining constituents are equally rhematic and/or equally important.
This claim is called into question by \citet[204]{auer1993}.
He argues that the remaining constituents in an utterance with topic drop also vary in terms of how rhematic they are, e.g., pronouns are less rhematic than full NPs, etc.
However, he agrees with Oppenrieder in assuming that in utterances with topic drop no constituent is designated as the topic because either the topic is known from the linguistic or extralinguistic context or none of the constituents is ``un-rhematic'' or thematic enough to function as a topic \citep[204]{auer1993}.
\citet[105]{guenthner2006} states that topic drop as the omission of thematically given \is{Givenness} information increases the ``rhemacity'' because the semantically more important information comes first \citep[see also][168--169]{helmer2016}.
Similarly, \citet[103]{schwitalla2012} argues that the omission of constant themes (or topics) allows the hearer to focus their attention on the rhemes.
While \citet{helmer2016} links this strategy to redundancy avoidance and thematic progression, \citet{oppenrieder1987} connects it to increased expressiveness (see Section \ref{sec:pusage.effects}).

However, as shown in Section \ref{sec:topicality.ness}, topic drop does not only affect topics but also non-topics, such as non-referential expletives. \is{Expletive}
For these elements, the explanation of topic avoidance cannot be valid, since the omitted prefield element is not a topic.
However, it could be argued that the omission of a semantically empty element also directs the focus more strongly to the rest of the utterance.
But here, too, the question remains of why this strategy is not pursued more often or, more precisely, whenever possible, i.e., why there are still utterances with topical elements or expletives \is{Expletive} in the prefield at all.
Topic avoidance or rheme focus can thus presumably only be a part of a broader explanation for the usage of topic drop.
\is{Topic|)} 

\section{Socio-pragmatic functions}\label{sec:pusage.effects}
In this section, I discuss jointly those functions of topic drop that it fulfills primarily in spoken or narrative texts according to the relevant literature and which may be termed socio-pragmatic functions.

\subsection{Expressiveness, condensation, and action orientation}\label{sec:pusage.expressiv}
\citet{auer1993} assumes that the function of topic drop varies according to the type of text it is used in and the linguistic acts as which the corresponding utterances occur.
He argues that topic drop can indicate a switch to narration and states that in the narrative genre, it stresses what he terms the action character of the proposition \citep[219]{auer1993}.
More specifically, he suggests that it highlights the semantics of the finite verb in the left bracket, marking action orientation in a general sense.
While topic drop does not necessarily draw the hearer's attention to a concrete action, since at least for analytic verb forms, the semantic content of an action is expressed through the infinite verb part in the right bracket \citep[218]{auer1993}, it often does so in the present tense, as  \citet[101]{guenthner2006} points out.
Accordingly, the frequent combination of the narrative present tense and topic drop allows the speaker to focus on the concrete action, which leads to the plot advancing.
According to \citet[104]{guenthner2006}, the combination of the present tense and topic drop, which she considers a technique of verb foregrounding, has the effect of ``liveliness, condensation and expressiveness of the utterance.''%
%% Footnote
\footnote{My translation, the original: ``Lebendigkeit, Dichte und Expressivität der Äußerung''  \citep[104]{guenthner2006}.}
%
Similarly, \citet[302]{sandig2000} interprets topic drop as a conventionalized form of expressing dramatics and pace.%
%% Footnote
\footnote{All three authors, \citet{auer1993}, \citet{sandig2000}, and \citet{guenthner2006}, furthermore emphasize the role of prosody \is{Prosody} and rhythm for topic drop (see Section \ref{sec:pusage.prosody}).}
%

In the last two statements, several claims about the effect of topic drop seem to be condensed and partly intermingled.
First, the focus on the verb and the action, including the concurrent omission of the agent or patient of the action, can indeed result in the effect of an increased pace of the narration -- one event follows the other.
In this respect, there may also be a similarity to the likewise action-oriented drama, which can justify the denotation as ``expressing dramatics''.
The term ``liveliness'', which \citet[104]{guenthner2006} uses, may be a more subjectively biased word for this, which already anticipates the aspect of expressiveness discussed below.

Second, \citet[104]{guenthner2006} argues that topic drop has the effect of condensation, which is not surprising given that omitting something always results in a shorter and denser structure.
\citet{auer1993} describes this process in more detail by stating that topic drop leads to a closer connection between two adjacent utterances that does not endanger their semantic or syntactic autonomy.
Furthermore, he links topic drop to the global strategies of condensation (analogous to syntactic subordination) and fragmentation (analogous to syntactic parataxis), which are at work in spoken language \citep[219]{auer1993}.

Third, \citeg{guenthner2006} statement about the expressiveness of topic drop is probably the vaguest one.
Alongside her, \citet[179--180]{oppenrieder1987} uses the term in a similarly unspecific way by stating that the strategy of ``rhematizing'' the utterances leads to ``a strongly expressive coloring'', which is enforced by the omission of a functionless prefield constituent.
What is meant by this expressiveness and why exactly it is supposed to be evoked or enhanced by topic drop remains unclear in both authors.
Here \citet[127--128]{poitou1993} is somewhat clearer, at least specifying what he means by expressiveness, namely that the speaker also communicates their own (emotional) attitude.
According to him, this function of topic drop would be expressed with modal particles or intonational patterns in the complete V2 sentence.

\citet{auer1993} lists five so-called conversational environments in which topic drop occurs.
The last one is the already discussed narration while the first and second ones are directly related to expressiveness.
First, topic drop is said to occur in modalizations.
According to \citet[207]{auer1993}, these are statements in which the speaker expresses their attitude to a proposition by judging the truth, the probability, the reliability, etc. of the information contained in these statements.
Second, \citet[208]{auer1993} lists evaluations as a further environment.
They also express the speaker's attitude but not in terms of truth values or probability but in terms of aesthetics and morality.
Here, we seem to run into a chicken-and-egg problem.
Is topic drop a means of expressiveness, as proposed by \citet{oppenrieder1987}, \citet{poitou1993}, and \citet{guenthner2006}, or does topic drop preferably occur in expressive environments, i.e., in statements with expressive function, as proposed by \citet{auer1993}?

\subsection{Elaborations/reformulations and responses}
The last two conversational environments discussed by \citet{auer1993} in which topic drop occurs preferentially also establish a relation to the previous utterance.
They are elaborations or reformulations and responses.
According to \citet[209]{auer1993}, the former are ``non-corrective (mostly self-)repairs.''%
%% Footnote
\footnote{My translation, the original: ``nicht-korrigierende (meist Selbst-)Reparaturen'' \citep[209]{auer1993}.}
%
Here, the speaker specifies, exemplifies, or reformulates their own previous utterance or (more rarely) the utterance of their interlocutor.
\citet{auer1993} illustrates this function with example \ref{ex:elaboration.auer}, an excerpt from an investment consultancy. 

\ex.\label{ex:elaboration.auer}
\ag.A: Was nich schlecht is s Siemens\\
{} what not bad is is Siemens\\
A: `What's not bad is Siemens'
\b.B: m`m
\cg.A: Also $\Delta$ isn, isn, isn starkes Papier\\
{} so that is.a is.a is.a strong paper\\
A: `So (this) is a strong security' \citep[209, simplified]{auer1993}

A's second utterance \textit{isn starkes Papier} is an addition or specification to their first utterance \textit{was nich schlecht is s Siemens}.
\citet[210]{auer1993}  emphasizes that elaborations or reformulations can be grouped with the modalizations and evaluations mentioned above.
In all three types of conversational environments, he argues, there is a particularly strong sequential link between the precontext and the actual utterance, without the actual utterance starting a new linguistic action or being more relevant.
Therefore, Auer characterizes them as backward-looking.

According to \citet[212]{auer1993}, this is somewhat different for the group of responses because the responses are at least as relevant as the question or, more generally, the preceding utterance to which they refer -- since they do not need to answer an explicit question.
This is shown in example \ref{ex:response.auer}, where the response follows a request.
\citet{auer1993} argues that the type of conversational responses with topic drop that he refers to are similar to the modalizations in that the same verba sentiendi are used.
He further speculates that the responses could be captured as modalizations or elaborations of an implicit or explicit direct yes-or-no response and thereby connects four of the five mentioned conversational environments \citep[212]{auer1993}.

\ex.\label{ex:response.auer}
\ag.A: Grüß alle!\\
{} greet all\\
A: `Say hello to all!'
\bg.B: $\Delta$ Mach ich, du auch!\\
{} that make I you too\\
B: `I will do (that), you too!' \citep[212, simplified]{auer1993}

\subsection{Completion or continuation}
\citet{poitou1993} discusses a further function of topic drop that can be termed completion.
He, too, makes the observation, discussed earlier in this book, that topic drop frequently occurs in utterances that are immediately adjacent to the utterance with the antecedent (Section \ref{sec:recover.ling}).
He adds, however, that this often takes place in dialogues and in such a way that the speaker who utters the utterance with topic drop would, in a sense, continue the utterance of the previous speaker \citep[125]{poitou1993}.
About half of his instances of subject topic drop, he says, are structured this way, and the vast majority of instances of object topic drop.
He states that in these cases, the utterance with topic drop continues and completes the preceding utterance or simply follows on from it.
Or in more flowery terms, he puts it:  ``So it's as if speaker B's sentence had actually begun before B starts speaking''%
%% Footnote
\footnote{My translation, the original: ``Es ist also so, als hätte der Satz des Sprechers B eigentlich schon begonnen, bevor B zu sprechen beginnt'' \citep[125]{poitou1993}.}
%
\citep[125]{poitou1993}.

\citet[202--203]{auer1993}, on the other hand, disputes that the cases of collaborative turn constructions, as he terms them, occur frequently.
He describes them as cases in which a speaker does directly continue or complete the utterance of the previous speaker with a topic drop so that both utterances together form one syntactic structure.
Auer even goes so far as to consider examples like \ref{ex:auer.fries}, from \citet{fries1988}, as irrelevant to an analysis of topic drop.
He argues that in such cases, both utterances F and A need to be considered in combination, as one syntactic structure, and that there is no topic drop because the verb would be in the second position in the complete structure, as usual.

\ex.\label{ex:auer.fries}
\ag.F: Mir las sie immer Simone de Beauvoir vor und dem Fritz?\\
{} me.\textsc{dat} read she always Simone de Beauvoir \textsc{vpart} and the.\textsc{dat} Fritz\\
F: `She used to read Simone de Beauvoir to me, and to Fritz?'
\bg.A: $\Delta$ Liest sie zur Zeit ``Wie kommt das Salz ins Meer?'' vor.\\
{} him.\textsc{dat} reads she to.the time how comes the salt in.the sea \textsc{vpart} \\
A: `She is currently reading `Why Is There Salt in the Sea?' (to him).' (\cite[31]{fries1988}, cited in \cite[202]{auer1993})

While I do not agree with this interpretation and also analyze examples like \ref{ex:auer.fries} as structures with topic drop,%
%% Footnote
\footnote{Although topic drop of a dative object is a rare case anyway.}
%
I would like to support Auer's view that such question-answer sequences with topic drop, as they are much discussed especially in (mainly) introspectively oriented works like those of \citet{fries1988} or also \citet{trutkowski2016}, often seem constructed and occur rather rarely in corpora of natural language.%
%% Footnote
\footnote{Remember that \citeg{poitou1993} corpus \is{Corpus} contains an undefined portion of literary texts, in which examples like \ref{ex:auer.fries} may be deliberately used as stylistic devices.}
%
Thus, it seems at least unlikely that the function of topic drop is to complete previous utterances.

\subsection{Rhetorical and social function}
Besides the functions redundancy avoidance and cohesion, \citet{helmer2016} also discusses a rhetorical and a social function of topic drop.
She states that topic drop can be used to allow the hearer to continue the individually preferred topic, i.e., a certain vagueness lets the hearer continue in their desired way \citep[174]{helmer2016}.
She exemplifies this with \ref{ex:td.rhetorical}, taken from an interview, where she argues that the question with topic drop is so vague that the interviewee has a large freedom in answering.

\ex.\label{ex:td.rhetorical}
\a. \textit{Nun sagt man ja oft Frauen nach, also man, Männer sagen das oft Frauen nach, dass Frauen sehr oft an sich zweifeln. Kann ich das überhaupt?} [...] \textit{Da wächst mir was aufn} [sic!] \textit{Kopf. Nein, lieber nicht. Lass ich die Finger davon.}\\
`Now it is often said of women, well, men often say of women that women very often doubt themselves. Can I do that at all? [...] Something is becoming too much to handle for me. No, better not. I'll leave it alone.\,'
\bg.$\Delta$ Hat bei Ihnen keine Rolle gespielt?\\
that has at you.\textsc{2sg.pol} no role played\\
`(That) didn't matter to you?' \citep[175, adapted]{helmer2016}

However, it seems to me that an utterance with an overt pronoun \textit{das} in the prefield would be likewise vague, i.e., offering several possible interpretations as it can refer back to different referents.
Therefore, it is dubious whether the rhetorical function of the utterance does in fact hinge on topic drop.

Furthermore, \citet[177]{helmer2016} assumes that certain utterances with topic drop function more or less as formulas or idioms, or more precisely, she treats them as social action formats following, e.g., \citet{thompson.couper-kuhlen2005} and \citet{fox2007}.
Social action formats are linguistic expressions frequently used to perform certain social actions such as requests, suggestions, or demands, with their propositional content receding into the background \citep[see also][]{deppermann2021}.
\citet[177]{helmer2016} argues that in her data set, social action formats with topic drop like those in \ref{ex:social.actions} are often a means to express empathy while at the same time terminating the current discourse topic. \is{Discourse topic}
Again, she argues that topic drop allows for semantic vagueness, which in these cases has the function of expressing understanding.

\ex.\label{ex:social.actions}
\ag.$\Delta$ Is ja auch nich so einfach immer.\\
it is \textsc{part} also not so easy always\\
`(It) is not always that easy.' \citep[178, shortened and simplified]{helmer2016}
\bg.$\Delta$ Wird schon\\
that will already\\
`(It) will be fine' \citep[179, shortened and simplified]{helmer2016}

While this analysis seems reasonable, again it is not clear to me why only or mainly topic drop and not the corresponding full forms with the vague anaphoric pronoun \textit{das} (`that') should have this function.

\section{Prosody}\label{sec:pusage.prosody}\is{Prosody|(}
Related to the effect of liveliness discussed in Section \ref{sec:pusage.expressiv}, \citet[218]{auer1993} argues that this effect is enhanced in spoken language through a special rhythmic structure, which is also discussed in \citet[104]{guenthner2006}.
They state that utterances with topic drop differ from the full forms in that they lack an anacrusis.
They do not begin with an unstressed or weakly stressed syllable, as the full forms with pronouns do, but directly with a more or less strongly stressed finite verb.
\citet[300]{sandig2000} agrees with this observation and states that the finite verb is more strongly stressed than it would be with a preceding pronoun.
She argues that the resulting structure is then more strongly intonationally marked and that a two-peaked tone pattern is created; however, she does not explain what the two peaks consist of. 

\citet{helmer2016} investigated prosodic differences between utterances with topic drop and utterances with an anaphoric pronoun using a data set based on the research and teaching corpus of spoken German – FOLK \is{Corpus} \citep{schmidt2014}.
It consists of 321 instances of topic drop and 200 instances that contain an overt anaphoric pronoun \citep[217]{helmer2016}.
In her rather concise presentation, which leaves some methodological and terminological questions unanswered, Helmer concentrates, among other aspects,%
%% Footnote
\footnote{See \citet[216--227]{helmer2016} for her complete comparison between utterances with topic drop and utterances with an anaphoric pronoun with respect to prosody, which also includes the rhythm and the length of the utterances, as well as the relation between prosody and social interaction.}
%
on the position of the focus accents \is{Focus} in both types of utterances.
She notes that in utterances with an overt anaphoric pronoun, the focus accent is often (in about 19.5\% of the instances) on the subject in predicative constructions of the form \textit{das Kopula NP} (`that copula NP') \is{Copula} but is also placed frequently (here, Helmer does not provide a proportion) on predicative adjectives in constructions of the form \textit{das Kopula ADJP} (`that copula ADJP').
From this, she concludes that in utterances with anaphoric pronouns often evaluative elements or elements in comments are stressed.
In turn, in utterances with topic drop,%
%% Footnote
\footnote{Recall that \citet{helmer2016} considers topic drop to be possible also in the middle field (see Section \ref{sec:prefield.detail} of this book for a critical discussion).
This means that her statements about the position of focus accents \is{Focus} do not only refer to utterances with a V1 word order.
Since in her complete data set, which contains not only 321 instances from the FOLK but also 220 instances from the corpus \is{Corpus} Gespräche im Fernsehen (`conversations on TV'), there are 55 instances of what she describes as topic drop in the middle field, there are at most 55 instances of topic drop in the middle field in the data set discussed here but presumably less.}
%
the focus \is{Focus} accent is placed most often on the lexical verb, in about 36.5\% of the cases, as \citet[217]{helmer2016} reports.
Furthermore, in many cases (again Helmer does not report a proportion), there is a verum focus \is{Focus} \citep[e.g.,][]{hoehle1992} on the full verb or adverbs and adjectives are focused \citep[217--218]{helmer2016}.
At this point, it remains unclear from \citeg{helmer2016} statements whether all cases where the lexical verb is accented are cases of verum focus \is{Focus} and if not, what the other cases are.
\citet[218]{helmer2016} points out that in the utterances with topic drop, mostly elements are stressed that mark the responsivity of the utterance, as in the verum focus \is{Focus} cases, where the prosodic structure emphasizes that the speaker considers the previously uttered proposition to be true.

In a second step, \citet[218]{helmer2016} investigated the position of the focus \is{Focus} accent and differentiated initial, medial, and final position, as shown in Table \ref{tab:helmer.focus}.

\begin{table}
\caption{Position of the focus accent in utterances with topic drop vs. utterances with an anaphoric pronoun in \citeg{helmer2016} data set, taken from \citet[218]{helmer2016}}
\is{Focus}
\centering
\begin{tabular}{lrrrr}
\lsptoprule
Position of & \multicolumn{2}{c}{Full form} & \multicolumn{2}{c}{Topic drop}  \\
focus accent & Frequency & Proportion & Frequency & Proportion  \\
\midrule
Initial  & $17$ & $8.5\%$ & $96$ & $29.9\%$ \\
Medial & $103$ & $51.5\%$  & $98$ & $30.5\%$ \\
Final & $80$ & $40\%$ & $127$ & $39.6\%$ \\
\lspbottomrule
\end{tabular}
\label{tab:helmer.focus}
\end{table}

\noindent
The initial cases, according to her, are those in which the first element of an utterance bears the focus accent. \is{Focus}
For final and medial, she is less clear, but it can be assumed that the final cases are accordingly those in which the very last element of an utterance is focused, and medial are all cases in which it is an element somewhere in between.
While both in utterances with topic drop and with an anaphoric pronoun the focus \is{Focus} accent is final in about 40\% of the cases, the values for initial and medial positions differ significantly between elliptical and non-elliptical utterances, as \citet[218]{helmer2016} points out.
In utterances with topic drop, the focus \is{Focus} accent is significantly more often on the utterance-initial element than in utterances with an anaphoric pronoun, where it is put significantly more often on elements in the utterance-medial position.
This partly supports the claims by \citet{auer1993}, \citet{sandig2000}, and \citet{guenthner2006}.
In comparison to their non-elliptical counterparts, utterances with topic drop more frequently start with a stressed element, which is usually the finite verb (except for Helmer's alleged cases of topic drop in the middle field).
However, the fact that still in about 70\% of the cases, the focus \is{Focus} accent is not on the initial element but later in the utterance does not indicate a very strong tendency to stress the initial verb.

It can be concluded that while topic drop allows the focus \is{Focus} accent to be on the initial verb, this possibility is not always used.
If it is, the initial focus \is{Focus} accent draws attention to the verb and the action, which brings us back to the effects of topic drop outlined in Section \ref{sec:pusage.effects}.
Thus, prosody seems less able to explain why topic drop is used than to be an additional factor that may, under certain circumstances, enhance the actual functions of topic drop, such as action emphasis.
In addition, it is at least not immediately evident to what extent prosody is relevant to topic drop in written (if also conceptually spoken) text types. \is{Text type}
\citet{fery2006}, for instance, argues that the human parser \is{Parser} not only takes syntax into account but also builds up an unmarked prosodic structure when processing silently read sentences, which defaults to wide focus. \is{Focus}\is{Prosody|)}

\section{Summary: previous accounts of topic drop usage} 
I started this chapter on previous accounts of topic drop usage with the most classic explanation that can probably be applied to any elliptical phenomenon:
linguistic economy through the avoidance of redundancy.
This explanation is also a main component of my information-theoretic  account of topic drop usage, sketched in Chapter \ref{ch:infotheory}.
Following \citeauthor{levy.jaeger2007}'s (\citeyear{levy.jaeger2007}) \textit{uniform information density hypothesis}, \is{Uniform information density} I argue that predictable \is{Predictability} expressions are more likely to be omitted by speakers to achieve communicative efficiency.
This probability-driven operationalization of economy may contribute to explaining the limits of the redundancy avoidance approach sketched above.

What my information-theoretic account does not explicitly consider are the potential socio-pragmatic functions of topic drop, as well as its function as a cohesive device.
As described above, they partly involve a kind of chicken-and-egg problem.
Is topic drop indeed deliberately used to create cohesion or to fulfill a certain socio-pragmatic function, or does, in turn, a higher cohesion or a certain socio-pragmatic function lead to more uses of topic drop?
In any case, it seems to be reasonable that topic drop is associated with these functions, functions that play a more important role in spoken and narrative texts.
Since such texts are not the focus of the present book, it does not contribute to this issue other than encourage future research.
The question of whether and how socio-pragmatic factors can be integrated into my proposed information-theoretic account must, thus, remain equally open.

The same applies to prosody, \is{Prosody} which is naturally restricted to spoken language to an even stronger degree.
For this factor, \citet{helmer2016} evidenced that topic drop is at least not consistently used only as a device to stress the sentence-initial verb.

In this chapter, I furthermore argued that topic avoidance and completion/continuation only play a marginal role, if any, in motivating the usage of topic drop since non-topical constituents are also omitted and it is rare in authentic speech data that one interlocutor completes the utterance of another one.

For the rhetorical and social function that \citet{helmer2016} discusses as well as for the use in certain conversational environments that \citet{auer1993} lists, it remains unclear whether topic drop is really needed to fulfill the respective functions or whether the corresponding full forms would work equally well.
Here, more research in the vein of \citet{eckert1998.diss} and \citet{helmer2016}, who focus on usage differences between utterances with topic drop and corresponding full forms, is needed.
In the following chapters, I turn to my information-theoretic approach, its implications, and my empirical investigations.

