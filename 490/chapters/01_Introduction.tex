\chapter{Introduction}
\largerpage[2]
One fundamental property of human language is that we frequently leave certain things implicit while communicating, as exemplified in \ref{ex:intro.ell}.% (see also Figure \ref{fig:td.headline}).

\ex.\label{ex:intro.ell}
\a.\label{ex:intro.sluicing}Someone was looking for you, but I don't know who $\Delta$.
\b.\label{ex:intro.rnr}I know that Robin loves $\Delta$ and Fabian hates coriander.
\cg.\label{ex:intro.td}Milliarden-Investor in Ensdorf: ``$\Delta$ Haben uns ins Saarland verliebt''\\ 
billions-investor in Ensdorf we have us in.the Saarland fell.in.love\\
`Billion-euro investor in Ensdorf: `(We) fell in love with the Saarland'\,' (Saarbrücker Zeitung, 02/02/2023, p. 1)

\noindent
These structures seem somehow ``incomplete'' because they lack elements that would usually be present in a ``normal'' sentence, such as a second verb phrase (VP) \is{Verb phrase} \ref{ex:intro.sluicing.ff}, a determiner phrase (DP) \is{Determiner phrase}in object function \ref{ex:intro.rnr.ff}, or the preverbal constituent \ref{ex:saarland.ff}.
The ``normal'', i.e., syntactically complete, sentences are referred to as \textit{full forms} in this book.

\ex.\label{ex:intro.ff}
\a.\label{ex:intro.sluicing.ff}Someone was looking for you, but I don't know who $\langle$was looking for you$\rangle$.
\b.\label{ex:intro.rnr.ff}I know that Robin loves $\langle$coriander$\rangle$ and Fabian hates coriander.
\c.\label{ex:saarland.ff}Milliarden-Investor in Ensdorf: ``$\langle$Wir$\rangle$ haben uns ins Saarland verliebt''

Despite their incompleteness, the utterances in \ref{ex:intro.ell} are nevertheless usually understood by their addressees -- including the ``omitted'' part in \ref{ex:intro.ff}.
This phenomenon has been already discussed since antiquity under the term \textit{ellipsis} (from Ancient Greek ἔλλειψις, \textit{élleipsis}, `omission', \cite{ellipsis}) and continues to be an intriguing and popular research topic in modern linguistics.%
%% Footnote
\footnote{This is evidenced, for example, by the publication of the \textit{The Oxford handbook of ellipsis} in \citeyear{vancraenenbroeck.temmerman2018.handbook} \citep{vancraenenbroeck.temmerman2018.handbook}, by the inclusion of articles on ellipses in handbooks of different disciplines, such as \citet{reich2011} in \textit{Semantics. An international handbook of natural language meaning} 
\citep{heusinger.etal2011} or \citet{vancraenenbroeck.merchant2013} in \textit{The Cambridge handbook of generative syntax} \citep{dikken2013}, and by the initiation of an annual workshop on experimental and corpus-based approaches to ellipsis in 2017 \citep[see][]{bilbiie.nykiel2023}.}
%
Still, many questions surrounding ellipsis have not been fully resolved, such as which types of ellipsis should be meaningfully distinguished and why, how ellipses differ across languages, how they can be adequately described grammatically, how hearers process them, and why hearers understand elliptical utterances anyway, to name just a few.

Cross-linguistically, a number of different ellipsis types have been discussed, such as \textit{sluicing} \is{Sluicing} \citep[e.g.,][]{ross1969,merchant2001} in \ref{ex:intro.sluicing} or \textit{right node raising} \citep[e.g.,][]{postal1974,hartmann2000} in \ref{ex:intro.rnr}.
This book is concerned with the type in \ref{ex:intro.td}, the omission of the preverbal constituent (to which I refer as the \textit{prefield constituent} later in this book) in verb-second (V2) declarative clauses in most Germanic languages:%
%% Footnote
\footnote{Although I only investigate topic drop in German in this book, it is to be expected that the results are at least partially transferable to the other Germanic V2 \is{V2 word order} languages, provided that language-specific peculiarities are taken into account.}%
%
 so-called \textit{topic drop} \citep[e.g.,][]{fries1988}.
A more detailed study of topic drop seems to be required because, although numerous claims about topic drop have been put forward in the theoretical literature, they have often been made in isolation, are partly contradictory, and have not or only rarely been empirically tested.
Therefore, this book has the goal of investigating the licensing and the usage of topic drop thoroughly using empirical methods, especially experiments.

Example \ref{ex:intro.td} illustrates a case of topic drop where the 1st person plural subject pronoun \textit{wir} (`we') is omitted from the clause.
The normally obligatorily filled preverbal position of the declarative clause is empty.
However, from the usage of the reflexive pronoun \textit{uns} (`ourselves'), it is evident that \textit{wir} must still be part of the sentence.
I usually term the unrealized constituent \textit{omitted} or \textit{covert}, making it clear that I consider topic drop to be a case of ellipsis.
The empty preverbal position is marked with the Greek capital letter Delta $\Delta$ and glossed and translated with an intuitive reconstruction of the omitted constituent.%
%% Footnote
\footnote{When presenting experimental items, I indicate the variation between topic drop and the corresponding full form by putting the prefield constituent into parentheses.}
%

\largerpage
While the omission of a referential subject as in \ref{ex:intro.td} is indeed a particularly typical case of topic drop in German, its occurrence in the text type newspaper headline is not.
This is because, as this book also evidences in line with previous research, topic drop typically occurs in certain text types. \is{Text type|(}
Figure \ref{fig:td.wa.mocoda2}, glossed and translated in example \ref{ex:td.wa.mocoda2}, therefore, shows topic drop in a more characteristic text type, namely, in a message of the instant messaging service WhatsApp, taken from the Mobile Communication Database 2 (MoCoDa2) \citep{beisswenger.etal2020}.%
%% Footnote
\footnote{\is{Corpus}The still growing Mobile Communication Database 2 consisted in January 2025 of about $1\,000$ instant messaging (WhatsApp) chats with roughly $39\,000$ messages and $318\,000$ tokens of about $3\,550$ writers, which were donated for scientific usage.}
%
In the present dialogue, there is a further omission of a referential subject (\textit{ich} -- `I'), as well as the omission of a direct object (\textit{das} -- `that') and of a non-referential subject (expletive \is{Expletive} \textit{es} -- `it'), showing a wider range of possible topic drop constructions.

\ex.\label{ex:td.wa.mocoda2}
\ag.MB: Möchstest [sic!] du in Mittelalter deine BA schrieben [sic!] oder warum willst du die MP in Neuzeit machen\\
{} would.like {} you in middle.ages your bachelor.thesis write {} or why want you the oral.exam in modern.era make\\
MB: `Do you want to write your bachelor thesis on medieval times or why do you want to do the oral exam on modern times?' 
\bg.TZ: Beides Neuzeit\\
{} both modern.times\\
TZ: `Both modern times'
\cg.MB: $\Delta$ Darf man nicht. Es sei denn das eine ist frühe Neuzeit\\
{} that may one not it is.\textsc{conj} \textsc{part} that one is early modern.times\\
MB: `You are not allowed to do (that). Unless the one is early modern times'
\dg.TZ: Wieso darf man das nicht?\makebox[0pt][l]{\vspace*{-2cm}\raisebox{-0.35ex}{\includegraphics[scale=0.07]{Emojis/thinking-face_wa.png}}}\phantom{mn} $\Delta$ Gibt doch 3 Körbe. $\Delta$ Nehme einmal 19. Jh und einmal 20\\
{} why may one that not it gives \textsc{part} 3 baskets I take once 19th century and once 20th\\
TZ: `Why is that not allowed?\makebox[0pt][l]{\vspace*{-2cm}\raisebox{-0.35ex}{\includegraphics[scale=0.07]{Emojis/thinking-face_wa.png}}}\phantom{mn} (There) are 3 baskets. (I) take once 19th century and once 20th.'

\begin{figure}
\centering
\includegraphics[scale=1.2]{WA_TD_MoCoDa2.PNG}
\caption{Excerpt from a WhatsApp dialogue between two male history students, taken from MoCoDa2, chat \#6WJlJ, \url{https://db.mocoda2.de/view/6WJlJ} (visited on 01/02/2025)}
\label{fig:td.wa.mocoda2}
\end{figure}

\noindent
In this book, I focus on topic drop of subjects and, to a lesser extent, objects in text messages and instant messaging chats.
I investigate both its licensing and its usage.

\section{Contextualization}
To the best of my knowledge, topic drop was first discussed in modern linguistics by \citefirstlastauthor{ross1982} and \citefirstlastauthor{reis1982} in 1982.
In early works, topic drop research was mainly concerned with issues of licensing and the syntactic description of the phenomenon (see \cite{fries1988}, for a first detailed investigation).
It has also been discussed repeatedly in the generative literature, for instance, in the context of related null subject \is{Null subject} phenomena such as \textit{pro}-drop \is{@\emph{pro}-drop} but especially in relation to the question of which null category could describe it most adequately \citep[e.g.,][]{huang1984, cardinaletti1990, haegeman1990, rizzi1994}.
In the German-language research, especially of the late 1980s and the 1990s, individual aspects of topic drop were repeatedly singled out \citep[e.g.,][]{sternefeld1985,oppenrieder1987, auer1993, poitou1993, zifonun.etal1997}.
As a result, authors often  put forward isolated hypotheses, concerning, e.g., the preferred occurrence of topic drop with certain grammatical functions or persons, as well as the role of topicality and the following verb.
Following \citet{auer1993}, who first discussed topic drop in corpus data, topic drop became increasingly a matter of interest in spoken language research after the turn of the millennium, where aspects of the usage and the effect of the ellipsis type came to the fore \citep[e.g.,][]{sandig2000, guenthner2000, guenthner2006, schwitalla2012, imo2013, imo2014, helmer2016}.
At the same time, topic drop was also increasingly investigated quantitatively as a typical phenomenon of the communication form text message and its successor instant message \citep{androutsopoulos.schmidt2002,doring2002,frick2017}.

The short sketch of topic drop research since the 1980s illustrates the different perspectives and foci on the phenomenon of topic drop.
However, what is still needed is, first, a systematic investigation of the several isolated claims and, second, an approach that simultaneously considers the licensing and the usage of topic drop, as well as their relation.
Such a double perspective is the goal of this book.
In other words, my aim is to determine the licensing conditions of topic drop and to answer the question of why speakers use topic drop, provided these conditions are met.
At the same time, I take a more empirical and quantitative approach than previous research, which often relied on unnatural introspective examples, some of which were judged inconsistently by different authors.
In addition to the corpus data already established in topic drop research, above all, I also take into account experimental results, in the form of acceptability judgments by naive participants. 

\section{Research questions}
This book focuses on two central aspects of topic drop: licensing and usage.
The first major research question I would like to answer is:
When is topic drop syntactically licensed in German?
This question can be divided into two sub-questions since, according to previous literature, there are two central conditions for topic drop, the restriction to the prefield and the recoverability of the omitted constituent.
Consequently, the first sub-question concerns the restriction of topic drop to the prefield, i.e., the preverbal position of declarative V2 main clauses, and the adequate description of this restriction.
While this prefield restriction is widely accepted in the literature, although it is not entirely uncontroversial, it has been formalized in different ways and captured both syntactically and in terms of information structure.
The central goals of the first part of this book are to confirm the prefield restriction empirically and to determine its nature more precisely through theoretical discussions and experiments.
To a lesser extent and exclusively from a theoretical perspective, I address the second sub-question:
What is the role of recoverability for topic drop and how can this concept be meaningfully grasped?

The second major research question is the following:
When is topic drop used, provided it is syntactically licensed?
This question is motivated by the observation that, in a context where topic drop is permitted, speakers%
%% Footnote
\footnote{I use the term \textit{speaker} to generally refer to the ``producer'' of language, may it be in spoken or in written form, i.e., a speaker could also be a writer.
Similarly, \textit{hearer} is used to denote the ``addressee'' of language.}
%
generally have a choice between using the ellipsis or a full form.
In fact, they sometimes use one and sometimes the other in similar contexts or situations.
To answer this question, I propose an information\hyp theoretic  account of topic drop usage and validate it by means of empirical studies.
The idea is that speakers (do not) use topic drop to optimally distribute the hearer's processing effort across an utterance. \is{Processing effort}
Since the information\hyp theoretic account is based on probabilities and general processing principles, it is transferable to different (elliptical) phenomena, as has already been done, and allows for explaining their usage in a uniform way.
In my empirical studies, I investigate, first, which of the factors proposed in the theoretical literature really impact the usage of topic drop.
Second, I examine whether the information\hyp theoretic  approach can provide a unified account of the previously isolated claims from the literature and whether it has additional explanatory power.

\section{Method}
To answer these two research questions, I begin with a discussion of the existing literature on the licensing and the usage of topic drop, respectively.
The discussion mainly concerns the theoretical literature, which, as already indicated above, often made isolated claims about topic drop in the past.
I derive testable hypotheses from these claims, which the theoretical literature itself often did not produce.
In the second part of this book on the usage of topic drop, I develop an information\hyp theoretic approach. 
It builds on existing approaches to information\hyp theoretic modeling of other phenomena involving optional omissions and, thus, contributes to a unified analysis of the usage of different types of ellipsis and related phenomena.
The information\hyp theoretic account predicts that the usage of topic drop is determined by an interplay of two tendencies:
(i) speakers use topic drop to omit predictable \is{Predictability} prefield constituents and (ii) speakers use the full form to reduce the processing effort on the following verb. \is{Processing effort}
With these principles, the approach can explain the effects of individual factors, which have been predicted by the theoretical literature, without additional assumptions, e.g., that topic drop of \textit{ich} (`I') is so frequent because it is usually highly predictable. \is{Predictability}
In addition, the approach predicts specific effects, such as the effect of the surprisal of the following verb, that cannot be explained by any other theory.
I empirically tested the hypotheses from the theoretical literature and the predictions of the information\hyp theoretic  account, partly individually, and partly in combination using two different methods:
a corpus study and acceptability rating experiments.
This allowed me to examine both the production and perception of topic drop.
According to the information\hyp theoretic account, they should behave in parallel since the speaker is argued to perform audience design \is{Audience design} \citep{bell1984} to shape their language production for the benefit of the hearer.

\is{Corpus|(}\is{Production|(}
With the corpus study, I focused on authentic speech data to investigate the \emph{production} of topic drop in different text types.
For this purpose, I extracted instances of topic drop from the entire text type-balanced fragment corpus FraC \citep{horch.reich2017} and compared them to corresponding full forms as reference data.
In the second step, I focused on the text message subcorpus and conducted an inferential statistical analysis of topic drop in this single text type.
Although there have been corpus studies of topic drop, they either lacked reference data in the form of full forms or statistical analysis or both \citep[e.g.,][]{auer1993,androutsopoulos.schmidt2002, doring2002, helmer2016}, or the analysis was not state-of-the-art \citep{frick2017}.
In my corpus studies, I collected reference data to be able to assess ellipsis-specific effects and used a more complex statistical method, i.e., logistic regressions, which allowed me to consider several factors simultaneously as well as to observe possible interactions. \is{Corpus|)}
\is{Production|)}

In this book, I present the results of a total of 12 acceptability rating studies \is{Acceptability rating study|(} that I employed to investigate the \emph{perception} of topic drop.
I used minimal pairs to systematically test possible factors influencing topic drop and collected the ratings of linguistic non-experts for this purpose.
This allowed me to also look at feature combinations that occur too rarely or not at all in the production \is{Production} data but for which my hypotheses derived from the literature or the information\hyp theoretic  approach nevertheless predict differences.
The focus on experimental methods, in this case on rating studies, is relatively new for topic drop research.
To the best of my knowledge, only \citet{trutkowski2018} has conducted an experimental study of topic drop so far, also a rating study.
My work, thus, also contributes to a greater methodological diversity in topic drop research.\is{Acceptability rating study|)}

My empirical investigations of topic drop are largely focused on two text types or types of communication (see \sectref{sec:def.texttype} for a terminological discussion), in which topic drop is particularly frequent.
As mentioned above, I zoomed in on the text message subcorpus in my corpus study \is{Corpus} for an inferential statistical analysis, while in all but one of my experiments, I presented the stimuli in the form of instant messaging chats, the successors of text messages.
This decision is motivated by the intention to investigate topic drop in text types where this ellipsis type is as natural as possible.
Thus, any observed differences in the usage of topic drop should depend on the varying factors and not on the text type.
The focus on text and instant messages has the consequence that the conclusions drawn in this book apply primarily to topic drop in these two text types. \is{Text type|)}
However, descriptively, there seem to be few differences between the results from the whole corpus and those from the text message subcorpus.
Similarly, experiment \ref*{exp:1sg.2sg.spoken}, which tested topic drop in spoken dialogues presented in written form, found similar effects of grammatical person as experiment \ref*{exp:1sg.2sg} with an instant messaging design.

\section{Contributions}
\subsection{Focus on licensing and usage}
This book examines the two central aspects of the ellipsis type topic drop:
on the one hand, its licensing, above all the question of the nature of the syntactic environment in which it is possible.
On the other hand, its usage, i.e., the question of when speakers of German decide against the full form and in favor of the ellipsis, provided the latter is licensed, and why they do so.
In doing so, it offers a descriptive and explanatory approach that brings together what can be described as two of the central strands of topic drop research:
the generative tradition, which is dedicated to adequately describing the syntactic properties of topic drop, and the tradition that approaches usage preferences with the help of authentic speech data, be it dialogues or text messages.

\subsection{Empirical approach}
To this effect, I take a strongly empirical approach.
First, I place myself in the tradition of investigating topic drop and its production \is{Production} through authentic language data in corpus studies. \is{Corpus}
Second, I focus even more on experimental work, first, to investigate the perception of topic drop and, second, to systematically test individual factors that in some cases could not be investigated in the corpus study due to data sparsity.
With these studies, I empirically test for the first time some claims that have so far only been postulated in the theoretical literature.

\subsection{Definition and typological perspective}
I present an explicit definition of topic drop in German that focuses on its elliptical nature and its prefield restriction.
Based on this definition, I distinguish topic drop from similar phenomena with which it has been equated in the past, most importantly \textit{pro}-drop \is{@\emph{pro}-drop} and verb-first (V1) declaratives. \is{V1 declarative}
From a typological perspective, I consider topic drop as a common phenomenon of the Germanic V2 languages \is{V2 word order} and discuss similarities and differences between these languages.
Furthermore, I point out the possibility of a common analysis of topic drop with register-dependent argument omissions in other languages.

\subsection{Prefield restriction as syntactic licensing condition}\is{Prefield|(}
Concerning the syntactic licensing of topic drop, I propose a refined prefield restriction based on the results of my experiments.
These experiments are, to my knowledge, the first to systematically study the licensing of topic drop.
With them, I first provide empirical evidence for the validity of the prefield restriction, which I and most of the literature assume, and then examine its properties in more detail.
I argue that topicality is not a licensing condition of topic drop, and thus the label \textit{topic drop} falls short of describing this ellipsis type in German.
Instead, my experiments suggest that the prefield restriction is a positional constraint that limits this ellipsis to the preverbal position (that is, to the specifier of the complementizer phrase \is{Complementizer phrase} [Spec, CP] in generative terms) of declarative V2 clauses (contra \cite{helmer2016}).
More specifically, I argue that topic drop is not possible in any prefield position but only in a prefield that is either not c-commanded \is{C-command} sentence-internally by a potential identifier \citep{rizzi1994} or that is the highest prefield of a root clause \citep{freywald2020}.
I suggest that, based on my data, a PF-deletion approach is best suited as an analysis of topic drop and should be preferred over the more complex operator or \textit{pro}-approaches, which do not provide greater explanatory power.
\is{Prefield|)}

\subsection{Recoverability as usage condition}
\is{Recoverability}
I argue that recoverability should not be treated as a further licensing condition of topic drop but rather as a felicity condition, which topic drop shares with other ellipsis types.
I operationalize recoverability by gradual givenness \citep{ariel1990,chafe1994}.
The recoverability of an expression is expected to increase with its givenness in context.\is{Recoverability}

\subsection{Information-theoretic account} 
Concerning the usage of topic drop, I propose an information\hyp theoretic approach as an explanation.
This approach stands in the tradition of a strand of work that explains optional omissions by the speaker's intention to distribute processing effort \is{Processing effort} for the hearer as optimal as possible across utterances (e.g., \cite{levy.jaeger2007, jaeger2010, asr.demberg2015, lemke2021}, to name just a few).
The approach is based on general production and processing mechanisms.
If certain properties of a phenomenon that were previously explained by specific grammatical rules can be traced back to these general mechanisms, this simplifies the grammar to be assumed.

For topic drop, I argue that its usage is determined by an interplay of the three information\hyp theoretic principles \textit{avoid troughs}, \textit{avoid peaks}, and \textit{facilitate recovery}. \is{Uniform information density}
According to the former two, which are derived from the \textit{uniform information density hypothesis} (\textit{UID}) \citep{levy.jaeger2007} and its precursors \citep[e.g.,][]{fenk.fenk1980}, speakers avoid regions of too low and too high processing effort for the hearer. \is{Processing effort}
The \textit{facilitate recovery} principle predicts that speakers lower the hearer's processing effort by providing cues that facilitate the resolution of ellipsis. \is{Processing effort}

\subsection{Empirical support for the information-theoretic account}
My empirical studies provide by and large support for these three principles.
For instance, the consistently observed preference for topic drop of the 1st person singular subject pronoun \textit{ich} (`I') can be accounted for by the \textit{avoid troughs} principle.
The high predictability \is{Predictability} of \textit{ich} in the investigated text types \is{Text type} text messages and instant messages results in very low processing effort \is{Processing effort} and in an increased chance of omitting the predictable constituent.
The corpus results for verb surprisal are in line with the \textit{avoid peaks} principle.
Topic drop is less frequent before verbs with high processing \is{Processing effort} costs because the preverbal constituent seems to be needed to make the verb easier to process.%
%% Footnote
\footnote{The experiments on verb surprisal provided no further evidence of an effect of verb surprisal.}
%
In the corpus study, there was some evidence of an impact of the \textit{facilitate recovery} principle.
Topic drop of the 1st person singular was shown to be more frequent when the following verb had a distinct inflectional ending, which facilitates ellipsis resolution as it unambiguously determines the reference of the omitted prefield constituent.


\section{Structure of this book}
In the following, I briefly outline the structure of this book.
It consists of an introductory definition chapter followed by two parts, each addressing one of the two major research questions.
Part one deals with the licensing of topic drop and part two with its usage.

In Chapter \ref{ch:definition}, I begin by proposing a working definition of topic drop.
In the second step, I delimit topic drop from similar phenomena such as \textit{pro}-drop and V1 declaratives, before I point out similarities to topic drop in other Germanic languages and to related ellipsis types in other languages.

The first part of this book is concerned with the question of topic drop licensing.
In Chapter \ref{ch:topicality}, I begin with the restriction of topic drop to the preverbal prefield position and work out the nature of this restriction in detail.
In \sectref{sec:topicality}, I conclude  that topicality is neither a (strictly) sufficient nor a necessary condition for topic drop.
In \sectref{sec:prefield.detail}, I provide experimental evidence for the prefield restriction, which most of the literature and I assume, and argue against \citeg{helmer2017} proposal of analyzing omissions in the middle field \is{Middle field} as topic drop as well.
Then, I determine the nature of the prefield restriction more precisely with two further experiments, each devoted to a special case of topic drop: topic drop in (potentially) embedded clauses in \sectref{sec:highest} and topic drop after conjunctions in \sectref{sec:initial}.
I conclude that topic drop either only occurs in a prefield that is not c-commanded \is{C-command} sentence-internally by a potential identifier \citep{rizzi1994} or in the highest prefield of a root clause \citep{freywald2020}.
In \sectref{sec:top.pf.summary}, I summarize my findings before I bring them together with previous generative analyses of topic drop in \sectref{sec:syntax}.
I conclude that the PF deletion approach can account for most of the properties of topic drop in German that I have identified while it requires the fewest additional assumptions.

In Chapter \ref{ch:recover}, I turn to recoverability, often considered another condition for topic drop, and look at its properties in Sections \ref{sec:recover.ling} to \ref{sec:recover.non.referential}.
In Sections \ref{sec:recover.given} and \ref{sec:recover.summ}, I argue that the relationship between the antecedent and the omitted referential constituent that enables recoverability is best described in terms of a gradual givenness concept.
Recoverability should be considered a felicity or usage condition rather than a licensing condition.

In the second part of this book, I focus on the usage of topic drop to answer the question of when topic drop is used, provided it is licensed.
In Chapter \ref{ch:pusage}, I first outline the previous approaches to topic drop usage, before I present my information\hyp theoretic account in Chapter \ref{ch:infotheory}.
After introducing basic information\hyp theoretic concepts in \sectref{sec:infotheory.basic}, I turn to the \textit{uniform information density hypothesis} (\textit{UID}) and its precursors in \sectref{sec:info.theory.uid}, according to which speakers strive to distribute information as evenly as possible across an utterance.
From \textit{UID}, the \textit{avoid troughs} principle and the \textit{avoid peaks} principle, discussed in Sections \ref{sec:avoid.troughs} and \ref{sec:avoid.peaks}, can be derived, which I argue guide the usage of topic drop.
I complement them with a further principle, the \textit{facilitate recovery} principle, discussed in \sectref{sec:resolving}, according to which  the ellipsis resolution also affects the processing of topic drop structures and how felicitous they are. \is{Processing effort}

In Chapter \ref{ch:methodology}, I outline the methodology of the empirical studies presented in the second part of this book.
I discuss the relation between production, perception, and processing in \sectref{sec:production.perception} and relate it to the information\hyp theoretic reasoning.
Sections \ref{sec:corpus} and \ref{sec:experiments} provide details about the corpus study and the experiments discussed in the subsequent chapters.

The following four chapters are then each dedicated to one or more factors that potentially influence the usage of topic drop.
In each chapter, I first discuss the state of research, as well as previous studies, outline the information\hyp theoretic predictions, and present my own corpus and/or experimental results.
Chapter \ref{ch:usage.function} is concerned with syntactic function and presents corpus results, while Chapter \ref{ch:factor.topicality}, based only on experimental data, investigates whether topicality is a usage factor.
In Chapter \ref{ch:usage.person}, I turn to grammatical person and number, verbal inflection, and ambiguity avoidance, and, finally, in Chapter \ref{ch:usage.verb}, to verb type and verb surprisal.
I investigated the factors in these two chapters with my corpus study and with experiments.

Finally, the conclusion of the book is provided by the general discussion in Chapter \ref{ch:discussion}.
There, I revisit the central research questions and answer each of them in light of the findings in this book.
In doing so, I also discuss the strengths and limitations of the proposed information\hyp theoretic approach and identify research desiderata.


