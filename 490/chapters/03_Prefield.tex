%*****************************************
\chapter{Syntactic licensing: prefield restriction of topic drop}\label{ch:topicality}
%*****************************************
\is{Prefield|(}\il{German|(}
The most cited licensing condition of topic drop in German is its structural restriction \citep[1850]{reich2011} to the preverbal position of declarative verb-second (V2) clauses, as shown by the contrast in \ref{ex:td.prefield} where \textit{ich} (`I') can only be omitted in \ref{ex:td.prefield.pf} but not in  \ref{ex:td.prefield.mf}.

\ex.\label{ex:td.prefield}
\ag.\label{ex:td.prefield.pf}$\Delta$ Habe das schon gemacht.\\
I have that already made\\
`(I) have already made that.'
\bg.*\label{ex:td.prefield.mf}Das habe $\Delta$ schon gemacht.\\
that have I already made\\
`(I) have already made that.'

In what follows, I address this prefield restriction in detail.
For this purpose, I discuss the positions from the theoretical literature and relate them to my empirical results from four acceptability rating experiments.
The resulting findings motivate a specification of the prefield restriction of topic drop and form the basis for a critical examination of the generative-syntactic analyses proposed so far in the literature.

While the absolute majority of researchers agree on the prefield restriction of topic drop, there is, to the best of my knowledge, only one dissenting opinion on this matter, namely \citet{helmer2016}, who argues that topic drop in German can also occur in the middle field. \is{Middle field}
I discuss her argumentation in \sectref{sec:prefield.detail} and assume for this overview, based on my definition from \sectref{sec:definition}, that the prefield restriction holds -- an assumption that is supported by the experimental evidence presented in \sectref{sec:exp.prefield}.

\is{Topological field model|(}
Pre-theoretically, the position to which topic drop is restricted can be defined as preverbal \citep{fries1988, sandig2000,frick2017}, i.e., as located immediately before the finite verb.
In the literature, this position is most frequently referred to as \textit{prefield} \citep[e.g.,][]{oppenrieder1987, auer1993, zifonun.etal1997, reis2000, reich2011, volodina2011, schalowski2015, trutkowski2016, frick2017}, following the terminology of the so-called \textit{topological field model} \citep{drach1937, hohle1986, wollstein2018}.
This model has its origins in the 19th century (e.g., \cite{herling1821,erdmann1886} and later \cite{drach1937}; see \cite{hohle1986}) and is used to describe the structure of German clauses by dividing them into several linearly ordered ``field'' and ``bracket'' positions \citep{wollstein2018}, in which specific elements occur. 
V2 declarative \is{V2 word order} clauses are usually described as consisting of five positions, but not all of them need to be filled: the prefield, which usually holds exactly one constituent, the left bracket, which only contains the finite verb, the middle field, \is{Middle field} which can hold any number of constituents, the right bracket, which contains verb particles or infinite verbs, and the postfield, which contains extraposed constituents \citep[see][]{wollstein2018}.
Table \ref{tab:top.model} illustrates the scheme for two example sentences, which contain \textit{ich} (`I') and \textit{der Mann} (`the man') as prefield constituents respectively.

\begin{table}
\caption{Analysis of two example sentences within the topological field model}
\centering
\begin{tabularx}{\textwidth}{ll l ll l lll}
\lsptoprule
\multicolumn{2}{c}{\small Prefield} & \multicolumn{1}{c}{\small Left bracket} & \multicolumn{2}{c}{\small Middle field \is{Middle field}} & \multicolumn{1}{c}{\small Right bracket} & \multicolumn{3}{c}{\small Postfield}  \\
\midrule
\multicolumn{2}{l}{\small Ich} & {\small habe} & {\small das} & {\small schon} & {\small gemacht.}& & & \\
\multicolumn{2}{l}{\emph{\small I}} & \emph{\small have} & \emph{\small that} & \emph{\small already} & \emph{\small made}& & &\\
\tablevspace
{\small Der} & {\small Mann} & {\small muss} & {\small ein} & {\small Buch} & {\small abholen} & {\small in} & {\small der} & {\small Stadt.}\\
\emph{\small the} & \emph{\small man} & \emph{\small must} & \emph{\small a} & \emph{\small book} & \emph{\small collect} & \emph{\small in} & \emph{\small the} & \emph{\small city} \\
\lspbottomrule
\end{tabularx}
\label{tab:top.model}
\end{table}

\noindent
\citet[159]{wollstein2018} illustrates how the topological field model can be brought into correspondence with a classic complementizer phrase \is{Complementizer phrase} / inflectional phrase \linebreak(CP/IP) model of generative grammar.
In this approach, the prefield position corresponds to [Spec, CP], as shown in Figure \ref{fig:top.cp}, so that, in generative terms, topic drop is restricted to [Spec, CP] \citep{ackema.neeleman2007, trutkowski2016}.

\begin{figure}
\centering
\begin{forest}
for tree={s sep*=1.75, parent anchor=south, child anchor=north}
[CP
	[SpecCP
		[DP
			[\phantom{j}Er\textsubscript{i}\phantom{j}\vspace{0.1em}, tier=word, roof], ]
		]
	[C'
		[C°
		[\phantom{j}hat\textsubscript{j}\phantom{j}, tier=word]]
		[IP, tikz={\draw[LimeGreen] (0.4,-1.8) rectangle (6.25,-10); \node[align=center]  at (3.5,-10.3) {\emph{\textcolor{LimeGreen}{Middle field}}};				
					\draw[cyan] (-2.5,-0.7) rectangle (-1.1,-10); \node[align=center]  at (-1.9,-10.3) {\emph{\textcolor{cyan}{Prefield}}};
				   \draw[red] (-0.9,-1.8) rectangle (0.2,-10); \node[align=center]  at (-0.25,-10.3) {\emph{\textcolor{red}{Left bracket}}};		
				   \draw[orange] (6.45,-1.8) rectangle (9.9,-10); \node[align=center]  at (8.35,-10.3) {\emph{\textcolor{orange}{Right bracket}}};}
			[\phantom{g}SpecIP\phantom{g}[t\textsubscript{i}\textquotesingle\textquotesingle, tier=word]]
			[I'
				[VP
					[SpecVP [t\textsubscript{i}\textquotesingle, tier=word]]
					[V'
						[VP
							[SpecVP [t\textsubscript{i}, tier=word]]
							[V'
								[DP [\phantom{j} ein Buch\phantom{j}, tier = word, roof]]
								[V°	[gekauft, tier=word]]
							]
						]
						[V°	[t\textsubscript{j}, tier=word]]]
					]
				[I°
					[t\textsubscript{j}\textquotesingle, tier=word]
				]
			]
		]
	]
]
\end{forest}
\caption{Correspondence between the classic CP/IP model of generative grammar and the positions of the topological field model, based on a figure by \citet[159]{wollstein2018}}
\is{Middle field}
\is{Complementizer phrase}
\label{fig:top.cp}
\end{figure}
\is{Topological field model|)}

Since [Spec, CP] is frequently considered one of several topic positions or even the only one in German (\cite[154]{pittner.berman2021}; \cite[158]{wollstein2018}), several authors assume that the restriction to [Spec, CP] is, in fact, a restriction to the topic position \citep{huang1984, auer1993, jaensch2005, volodina2011} or, more explicitly, a restriction to topics \citep{sternefeld1985, erteschik-shir2007, helmer2016}. 

\citet{rizzi1994} approaches the positional restriction of topic drop from a generative point of view by stating that subject topic drop -- similarly to early null subjects \is{Null subject} in child language -- is restricted to the ``specifier of the root'' \citep[155]{rizzi1994}, a position that c-commands \is{C-command|(} every other constituent in the sentence and is itself not c-commanded \is{C-command}sentence-internally by a potential identifier.%
%% Footnote
\footnote{The concept of c-command \is{C-command|)} goes back to \citet[32]{reinhart1976}, who defined it as follows:
``Node A c(onstituent)-commands node B if neither A nor B dominates the other and the first branching node which dominates A dominates B.''
}
%
\citet[167]{freywald2020} interprets the \textit{specifier of the root}-restriction as a restriction to the highest syntactic position of an autonomous or root clause, which results in a restriction of topic drop to the highest [Spec, CP] of a sentence since \citet{rizzi1994} assumes that the root of V2 clauses is usually a CP. \is{Complementizer phrase}

In the literature, it is a matter of debate whether topic drop is possible in (potentially) embedded \is{Embedding} contexts.%
%% Footnote
\footnote{There is also an ongoing discussion about the status and the syntactic position of V2 clauses in complement function.
I remain agnostic about this dispute, but I discuss it in \sectref{sec:highest}.}
%
While \citet[76]{cardinaletti1990} and \citet[272]{volodina2011} deny this option, \citet[101]{jaensch2005} states that topic drop can occur in embedded clauses in specific cases like question-answer pairs, such as \ref{ex:embedding.jaensch}, and that its acceptability seems to be highly dependent on the speakers being asked.

\ex.\label{ex:embedding.jaensch}
\ag.	Will Peter wirklich das neue Auto kaufen?\\
wants Peter really the new car buy \\
`Does Peter really want to buy the new car?'
\bg.Ich glaube [sic!] $\Delta$ hat er schon gekauft.\\
I believe {} it has he already bought\\
`I think he’s already bought (it).' \citep[98, her judgment]{jaensch2005}
 
\citet[224--225]{trutkowski2016} likewise assumes that topic drop is generally possible in embedded \is{Embedding} clauses but states that there are stronger identity conditions between antecedent \is{Antecedent} and gap for embedded than for unembedded cases.
The question of whether topic drop can or cannot occur in potentially embedded clauses is related to the characterization of the position of topic drop as clause- or sentence-initial (\cite[547]{huang1984}, \cite[1]{trutkowski2016}), left-peripheral \citep[150]{freywald2020}, or being at the left edge of an utterance \citep[99]{ackema.neeleman2007}.
It remains unclear whether this also entails that topic drop is restricted to the syntactically highest position and/or to the very first position of the clause-forming sequence of elements.
\il{German|)}

From the above, four questions arise regarding the positional restriction of topic drop, which I address through theoretical discussions and experimental studies: \\
1) Is the positional restriction of topic drop at the same time a restriction to only topical constituents? -- In \sectref{sec:topicality}, I argue that topic drop is not restricted to topics, that the prefield position is not a genuine topic position, and that topicality is neither a (strictly) sufficient nor a necessary condition for topic drop.\\
2) Is topic drop restricted to a preverbal, more specifically, prefield position or is it also possible postverbally in the middle field, as \citet{helmer2016} argues? -- The results of a rating study that is presented in \sectref{sec:prefield.detail} suggest that argument omissions in the middle field are degraded and that topic drop is restricted to the prefield position of V2 clauses. \\
3) Can topic drop occur in (potentially) embedded V2 clauses, i.e., is it possible in any prefield position? -- In \sectref{sec:highest}, I argue based on the results of a further rating study that subject topic drop is restricted at least to a prefield position where it is not c-commanded by a potential identifier from within the sentence or even to the prefield of autonomous or root clauses.\\
4) Is topic drop restricted to a strictly linearly sentence-initial position or can elements precede topic drop? -- The results of a rating study that is presented in \sectref{sec:initial} suggest that topic drop is not restricted to an absolute sentence-initial position, but that it can follow conjunctions \is{Conjunction} like \textit{und} (`and') and \textit{aber} (`but').

In \sectref{sec:top.pf.summary}, I summarize the answers to the four questions and thus the insights gained into the prefield restriction.
Finally, in \sectref{sec:syntax}, I examine the extent to which they can be reconciled with previous syntactic analyses of topic drop.

\section{Prefield restriction and topicality}\label{sec:topicality}
As I mentioned above, the prefield restriction of topic drop is often implicitly or explicitly interpreted as a restriction to topics, i.e., in a way that only topical constituents can be targeted by topic drop.
In this section, I investigate whether the equation of topic drop with the omission of topics is justified.
To do so, I first discuss the concept of \textit{topic}, which is essential to this book.
Based on the central literature, I propose a topic definition that serves as the basis for the theoretical discussion in this chapter and the experimental manipulations of topicality in \sectref{sec:topicality.experiments}.
Second, I show that the prefield position is not necessarily the topic position in German.
In the third step, I discuss the role of topicality for topic drop and conclude that it is neither a (strictly) sufficient nor a necessary condition.

\subsection{Defining \emph{topic}}
\is{Topic|(} \label{sec:topicality.lit}\is{Information structure|(}
The term \textit{topic} is one of the concepts that are used to describe the information structure \citep{halliday1967} of an utterance, also called its information packaging \citep{chafe1976}, i.e., ``the relation of what is being said to what has gone before in the discourse, and its internal organization into an act of communication'' \citep[199]{halliday1967}.%
%% Footnote
\footnote{According to \citet[28]{chafe1976} the concepts related to information packaging include but are not limited to givenness, focus \is{Focus} and contrast, definiteness, subjecthood, topicality, and point of view.
It is the multitude of these phenomena with their different, sometimes overlapping or contradictory definitions and their various relations to partly the same linguistic means of expression that have given information structure  \is{Information structure} the reputation of being an opaque field of research.
\citet[207]{musan2002} speaks here of a ``lush thicket'' of concepts, pairs of concepts, means of encoding, and functions that developed from the ``seed'' originally ``sown'' by the authors \citet{weil1844}, \citet{gabelentz1868}, and \citet{paul1919} and visualizes the complex relationships within the field of information structure with an intricately branching graphic.
Similarly, \citet[11]{molnar1991} characterizes the research around the concept of topic as ``chaotic'' and diagnoses a terminological confusion and a theoretical mess.}
%
As \citet[118]{lambrecht1994} and \citet[26]{musan2017} point out, topic as a concept has a long history, which goes back to Aristotle's distinction between (logical) subject and predicate, where the predicate says something about the subject.
This basic idea of a dichotomy between the thing about which something is said and what is said about it has persisted into modern research.%
% Footnote
\footnote{It was also discussed by \citet[378]{gabelentz1868} in the 19th century and by \citet[12]{paul1919} at the beginning of the 20th century.
See \citet[13--40]{molnar1991} for a detailed history of the topic concept from a functional and formal perspective.}
%
Nowadays it is mostly discussed under the term \textit{topic}, which goes back to \citet[201]{hockett1958}: 
``The most general characterization of predicative constructions is suggested by the terms `topic' and `comment': the speaker announces a topic and then says something about it.''%
%% Footnote
\footnote{\label{note:prague}The Prague School uses the similar conceptual pair \textit{theme} and \textit{rheme}. 
But while the theme is usually defined as given, \is{Givenness} old, or known information and the rheme as new or unknown information \citep[e.g.,][]{danes1970}, such an equation with old and new information is rejected for topic and comment by \citet{reinhart1981}, \citet{molnar1991}, and \citet{krifka2007}.
They argue that while topics are often given \is{Givenness} or old information, they do not necessarily have to be.
See \citet[204--205]{musan2002} and \citet[12--13; 60--62]{molnar1991} for a closer discussion and for more related terms.}
%
\citet[201]{hockett1958} illustrates this with the example sentence \emph{John ran away}, in which \textit{John} is announced as the topic and \textit{ran away} is the comment that is said about \textit{John}.%
%% Footnote
\footnote{This topic concept is, like its historical ancestors, a categorical one, according to which a constituent is either topical or not.
In contrast, some scholars of the Prague school assume a gradual topic-like notion.
For instance, \citet{firbas1971,firbas1992} defines the related concepts theme and rheme (see Footnote \ref{note:prague}) by means of so-called ``communicative dynamism'' as a part of his theory of functional sentence perspective.
According to this view, the theme is the element with the lowest degree of communicative dynamism, i.e., the element that contributes least to pushing communication forward (\cite[141]{firbas1971}, see also the discussions in \cite{molnar1991} and \cite{erteschik-shir2007}).
In this book, I do not employ such a gradual topic notion but stick to the now ``classic'' categorical topic concept.}
In this book, I assume the following topic definition:

\begin{theorem}
The topic of a sentence, represented through a topic expression, also called topic constituent, is the entity under which the information contained in the comment expression(s) should be stored in the common ground. \is{Common ground}
Thereby, the topic can be determined both as the entity the speaker talks about or as the entity the addressee gains additional knowledge about.
The topic and the corresponding proposition bear an aboutness relation relative to a particular discourse.
Topic expressions need to be referential, but they do not need to be subjects nor given \is{Givenness} information, although they frequently are both.
\end{theorem}

This definition is based on four influential topic definitions from the last 40 years by \citet{reinhart1981}, \citet{gundel1988}, \citet{lambrecht1994}, and \citet{krifka2007}, who all employ the aboutness criterion to define topic.
It is worth noting that they implicitly or explicitly mean sentence topics, i.e., the topics of a single utterance, as opposed to discourse topics, \is{Discourse topic} i.e., often more abstract ``topics of larger units'' \citep[54]{reinhart1981}.
In this book, I refer to sentence topics when I use the term \textit{topic}; when I mean discourse topics, I make this explicit.

\citeg{gundel1988} definition focuses on the interaction between speaker and hearer:
``An entity, E, is the topic of a sentence, S, iff in using S the speaker intends to increase the addressee's knowledge about, request information about, or otherwise get the addressee to act with respect to E'' \citep[210]{gundel1988}.
The topic is the anchor point of this interaction, the entity about which some form of information is exchanged from the speaker to the hearer.
According to Gundel, for an entity to be a topic, this entity has to be familiar or known to both speaker and hearer \citep[212]{gundel1988}.
Additionally, to store information about the topic, the hearer must be able to uniquely identify it \citep[214]{gundel1988}.
If we take \citeg{hockett1958} example sentence, \textit{John ran away}, \textit{John} is the anchor point, i.e., the entity about which the speaker wants to increase their addressee's knowledge.
It is assumed that the speaker and the addressee both know John and that the addressee can uniquely identify John.

\citet{lambrecht1994} provides a similar topic definition that also takes the hearer into account, but he focuses on the aspect of increasing the hearer's knowledge about a referent: 
``A referent is interpreted as the topic of a proposition if in a given situation the proposition is construed as being about this referent, i.e. as expressing information which is relevant to and which increases the addressee's knowledge of this referent'' \citep[131]{lambrecht1994}.
He explicitly distinguishes between topic referent and topic expression.
The former is the entity that serves as the topic, and the latter is the linguistic expression that is used to refer to that entity \citep[127--128]{lambrecht1994}.
Lambrecht makes two statements about this linguistic expression.
First, he argues, following \citet[67]{reinhart1981}, ``that only referential expressions can be topics'' \citep[156]{lambrecht1994}, ruling out semantically empty proforms such as expletive \is{Expletive} \textit{it} or \textit{there}, as well as indefinite pronouns.  
Second, he stresses the fact that there is a cross\hyp linguistically strong tendency for this expression to be the grammatical subject.
``[S]ubjects are \textsc{unmarked topics}'' \citep[132, original emphasis]{lambrecht1994}, but there is no one-to-one mapping between topic and subject.
Non-topics can be subjects and non-subjects can be topics \citep[see also][62]{reinhart1981}.
Finally, following \citet{reinhart1981}, \citet[127]{lambrecht1994} describes the \textit{topic-of}-relation as a pragmatic aboutness relation between a topic referent and a proposition relative to a particular discourse. 
In \citeg{hockett1958} example, \textit{John ran away}, the string \textit{John} is the topic expression, while the actual person John is the topic referent.
The proper name is a referential expression and functions as the subject of the sentence \textit{John ran away}, i.e., the subject is the topic.
Relative to the discourse in which  \textit{John ran away} is uttered, an aboutness relation between the referent John and the proposition \textit{John ran away} is established.

\citet{reinhart1981} provides a pragmatic topic definition using \citeg{stalnaker1978} notion of a \textit{context set}, in fact, she uses the term \textit{context set} but means \citeg{stalnaker1978} concept of \textit{common ground}. \is{Common ground}
\citet{stalnaker1974, stalnaker1978, stalnaker2002} defines the common ground%
% Footnote
\footnote{\citet{stalnaker1978, stalnaker2002} attributes the term to Paul Grice, who used it in the William James Lectures in the form of the common ground \is{Common ground} status of propositions \citep{grice1989}.}
%
as a shared background between interlocutors in the form of presupposed propositions, i.e., the set of propositions that the speaker and the hearer assume, believe, or accept to be true.
The context set, in turn, is the set of possible worlds that are compatible with the common ground. \is{Common ground}
\citet[78--79]{reinhart1981} states that she disregards the worlds aspect and considers ``the context set of a given discourse at a given point as the set of the propositions which we accept to be true at this point'',  which corresponds to the common ground. \is{Common ground}
To illustrate the organization of the common ground and its relation to topicality, she makes use of the metaphor of a library catalog.
The common ground is the library catalog, which is ordered in a useful way, and the propositions that are part of the common ground correspond to entries in this catalog \citep[79--80]{reinhart1981}.
When a new book is entered into the catalog, this book is stored under a defining entry, like the subject it deals with.
Similarly, when a new proposition is added to the common ground, this proposition needs to be stored under an entry.
This entry corresponds to the topic of the respective sentence.
We can illustrate this again with \citeg{hockett1958} example sentence \textit{John ran away}.
When a speaker utters this sentence and the hearer accepts it, the corresponding proposition is added to the common ground \is{Common ground} so that the comment \textit{ran away} is stored under the entry of the topic \textit{John}.

\largerpage
This metaphor is picked up by \citet{krifka2007}, who talks of the common ground \is{Common ground} as a file card system and of topics as the titles of these file cards.
He defines topic in the following way:
``The topic constituent identifies the entity or set of entities under which the information expressed in the comment constituent should be stored in the CG content'' \citep[41]{krifka2007}.
He states that utterances typically have exactly one topic, but, following \citet{lambrecht1994}, there may also be sentences with more than one topic constituent \ref{ex:multiple.topics} or so-called \textit{thetic} sentences with no topic constituent at all \ref{ex:thetic} \citep[42--43]{krifka2007}.
In \sectref{sec:topicality.ness}, I come back to this issue when discussing whether sentences with expletive subjects should be considered thetic.

\ex.
\a.As for John and Sue, they ran away.\label{ex:multiple.topics}
\b.It is raining.\label{ex:thetic}

\is{Information structure|)}

\subsection{Relation between topic drop and topicality}\label{sec:topicality.relation}
With this definition at hand, we can now proceed to the question of the relationship between topic drop and topic, the main point of this section.
In the research to date, two opposing views can be found.
First, \citet[19]{trutkowski2016} and \citet[67]{frick2017} state that topic drop and topicality are independent notions and that the omission is by no means restricted to topic expressions.
\citet{frick2017} clarifies:
``The notion of topic on which this [Frick's, LS] investigation is based is hence a purely structural one and refers to the preverbal positioning of the elements in the prefield'' \citep[67]{frick2017}.%
%% Footnote
\footnote{My translation, the original: ``Die der [Fricks, LS] Untersuchung zugrunde gelegte Auffassung von Topik ist also eine rein strukturelle und bezieht sich auf die präverbale Positionierung der Elemente im Vorfeld'' \citep[67]{frick2017}.}
%
The opposite opinion is held by \citet{sternefeld1985}, \citet{erteschik-shir2007}, and \citet{helmer2016, helmer2017}, who explicitly state that topic drop is the non-realization of a topic expression.
\citet{helmer2017} defines topic drop as follows:
``Analepses with topic drop are [...] sentence grammatically incomplete utterances [...] in which the topic of a previous utterance is not verbalized again'' \citep[2]{helmer2017}.%
%% Footnote
\footnote{My translation, the original: ``Analepsen mit Topik-Drop sind […] satzgrammatisch unvollständige Äußerungen, […] in denen das Topik einer vorherigen Äußerung nicht erneut verbalisiert wird'' \citep[2]{helmer2017}.}
%
In her dissertation, she specifies that both the sentence topic and the discourse topic \is{Discourse topic} can be omitted \citep[168]{helmer2016}.
Other researchers do not use the term \textit{topic} but the related \textit{theme} (see Footnote \ref{note:prague}).
They argue that only a thematic element can be omitted \citep{oppenrieder1987, poitou1993, zifonun.etal1997, sandig2000, guenthner2006}, or that no rhematic expressions can be targeted by topic drop \citep{fries1988,zifonun.etal1997}.
However, they do not explain which definition of theme they assume and do not relate it to topic. 
Even less clear is the position of authors such as \citet{huang1984}, \citet{fries1988}, \citet{auer1993}, \citet{jaensch2005}, and \citet{volodina2011}.
They do not define topic drop by using the information-structural topic concept but rather through the prefield restriction, which they term restriction to the topic position.
As mentioned above, the German \il{German|(} prefield position is frequently equated with a or even \textit{the} topic position in German.
When authors argue that topic drop is restricted to a syntactic position that they believe is reserved for topics, it is not unreasonable to assume that they at least indirectly consider topic drop to be restricted to topics.

From this literature review, two main issues arise:
1) Is the prefield position indeed the topic position in German so that an equation of prefield restriction and topic restriction is justified?
2) Is topic drop restricted to topics or, to put it more concretely, is being a topic both a sufficient and a necessary condition for a constituent to be targeted by topic drop?
I discuss these issues in the next three sections.

\subsection{Contra prefield position as topic position}\label{sec:topicality.prefield}
The positioning of a constituent in the prefield in German is frequently called topicalization \citep[e.g.,][]{cardinaletti1986}, which suggests a close relationship between topic and prefield \citep[e.g.,][]{pittner.berman2021}.
When authors such as \citet{huang1984} state that topic drop is restricted to ``sentence-initial -- namely, topic -- position'' (\cite[547]{huang1984}; see also \cite[198]{auer1993}, \cite[97]{jaensch2005}, \cite[272]{volodina2011}), this suggests that the prefield is exactly the position where the topic is placed in German.

However, the prefield position in German is neither restricted to topics nor can topics only occur in this position.
The first point is easily illustrated by elements in the prefield that are not referential and therefore cannot function as topics.
Examples are the expletive \is{Expletive} \textit{es} (`it') \ref{ex:prefield.no.topic.ex}, the placeholder \textit{es} \ref{ex:prefield.no.topic.ph}, the correlate \textit{es} \ref{ex:prefield.no.topic.corr}, or indefinite pronouns such as \textit{niemand} (`nobody') \ref{ex:prefield.no.topic.ind}.%
% Footnote
\footnote{In naming the various \textit{es} forms, I follow the terminology in \citet[190--193]{eisenberg2020}.}%
%

\ex.\label{ex:prefield.no.topic}
\ag.\label{ex:prefield.no.topic.ex}Es regnet heute schon den ganzen Tag.\\
it rains today already the whole day\\
`It's been raining all day today.'
\bg.\label{ex:prefield.no.topic.ph}Es spielen bekannte Rockbands und kleinere Indiegruppen.\\
it play known rock.bands and smaller indie.groups\\
`Well-known rock bands and smaller indie groups play.'
\cg.\label{ex:prefield.no.topic.corr}Es freut mich, dass es dir besser geht.\\
it pleases me that it you better goes\\
`I'm pleased that you are feeling better.'
\dg.\label{ex:prefield.no.topic.ind}Niemand mag Rosenkohl.\\
nobody likes Brussels.sprouts\\
`Nobody likes Brussels sprouts.'

Second, while there seems to be a general cross\hyp linguistic tendency for topics to occur at the left edge of an utterance \citep[e.g.,][471]{gundel2010.topic}, this tendency is by no means universal.
\citet[200]{lambrecht1994} shows this not only with the example of VOS or VSO languages but also with German and \ili{English}, ``in which topical non-subject constituents may appear in canonical argument \is{Argument} position after the verb''.
\citet[140; 145]{frey2000} argues for a designated topic position in the middle field \is{Middle field} in German, which immediately precedes the position of sentence adverbials \is{Adverbial} like \textit{wahrscheinlich} (`probably') or \textit{unglücklicherweise} (`unfortunately') and is independent of whether the prefield is filled with a topic expression or not.%
%% Footnote
\footnote{He illustrates the topic position in the middle field \is{Middle field} by means of the contrast in \ref{ex:frey}.
The subject must precede the sentence adverbial \is{Adverbial} and not follow it.

%\vspace{-0.5\baselineskip}
\ex.\label{ex:frey}
`I'll tell you something about Otto.'
\ag.Nächstes Jahr wird Otto wahrscheinlich seine Kollegin heiraten.\\
next year will Otto probably his colleague.\textsc{fem} marry\\
`Next year, Otto will probably marry his colleague.'
\b.\#Nächstes Jahr wird  wahrscheinlich Otto seine Kollegin heiraten. \citep[141, his judgments]{frey2000}
\vspace{-1em}
}
Finally, \citet{speyer2010} sketches an optimality-theoretic \is{Optimality theory} model of how the prefield is filled.
He argues that constituents that function as scene-setters or that indicate contrast are ranked higher than those that represent the topic and are therefore more likely to occur in the prefield.
In these cases, the topic must also occur in another topological position.

In summary, while the prefield position in German is a prototypical position for topics, it is not the only or exclusive topic position.
Topics may occur in other positions and also non-topical constituents can be placed in the prefield.
Consequently, the positional restriction of topic drop to the prefield position does not automatically entail a restriction of topic drop to topics.\il{German|)}

\subsection{Contra topicality as a sufficient condition}\label{sec:topicality.suff}
The question of whether topic drop is restricted to topic expressions has two aspects, namely whether topicality is a sufficient condition and whether it is a necessary condition for topic drop.
When people argue in favor of a topicality restriction in the literature, they usually refer exclusively to the aspect of necessity and not to sufficiency. 
They assume that topicality is the precondition for omitting a constituent but not that topicality follows from being able to omit the constituent.
Although necessity is thus the central point, I formalize and discuss both aspects in what follows.

For a condition $P$ to be sufficient, it has to be the case that if condition $P$ is met, the consequence $Q$ inevitably follows from $P$.
This means that whenever a prefield constituent%
%% Footnote
\footnote{If one abandons the equation of prefield and topic position and assumes that topics in German can also occur in the middle field, \is{Middle field} the prefield restriction does no longer follow from the restriction of topic drop to topics.
To account for the empirical facts (see \sectref{sec:exp.prefield}), proponents of a topic restriction would have to additionally assume an independently motivated prefield restriction.}
%
 is topical, it must be omittable, as indicated in equation \ref{eq:suff}.
\begin{equation}\label{eq:suff}
P \Rightarrow Q: \text{prefield constituent topical} \Rightarrow \text{constituent omittable}
\end{equation}

\noindent
This can be refuted by cases in which a prefield constituent is a topic expression but still cannot be targeted by topic drop.
A first counterexample are contrastive topics \is{Contrastive topic|(} \citep{buring1997}, i.e., ``topics with a rising accent'' that ``represent a combination of topic and focus'' \is{Focus|(} \citep[44]{krifka2007} \ref{ex:contrastive.topic}.

\ex.\label{ex:contrastive.topic} A: What do your siblings do?\\
B: [My [SISter]\textsubscript{Focus}]\textsubscript{Topic} [studies MEDicine]\textsubscript{Focus}, \\
and [my [BROther]\textsubscript{Focus}]\textsubscript{Topic} is [working on a FREIGHT ship]\textsubscript{Focus}. \\\citep[44]{krifka2007}

According to \citet[44]{krifka2007}, the function of the contrastive topic \is{Contrastive topic} \textit{my sister} is to indicate an alternative aboutness topic, namely \textit{my brother}, about whom the second conjunct is.
This way, B indicates that the first conjunct is an incomplete answer to A's question and that more information relevant to this question will come -- Krifka calls this ``a strategy of incremental answering in the CG [common ground, LS] management'' \citep[44]{krifka2007}.%
%% Footnote
\footnote{Not all usages of contrastive topics \is{Contrastive topic} have this function.
In example \ref{ex:contrastive.topic.2}, the contrastive topic indicates that B's assertion may not satisfy the expectations of speaker A \citep[45]{krifka2007}.
%\vspace{-0.5\baselineskip}
\ex.\label{ex:contrastive.topic.2}A: Does your sister speak Portuguese?\\
B: [My [BROther]\textsubscript{Focus}]\textsubscript{Topic} [[DOES]\textsubscript{Focus}]\textsubscript{Comment}.
\vspace{-0.75\baselineskip}
}

What is relevant to the question of sufficiency is the fact that such contrastive topics \is{Contrastive topic} cannot be targeted by topic drop, as \citet[28]{fries1988}, \citet[272]{volodina2011}, and \citet[217]{volodina.onea2012} point out.
Omitting \textit{my sister} in \ref{ex:contrastive.topic} is impossible because this would result in the loss of the prosodic \is{Prosody} marking through the focus and thus in the loss of the function of incremental answering.
Moreover, the omitted constituent \textit{my sister} would be difficult to recover \is{Recoverability|(} because the plural form \textit{siblings} implies that there are at least two persons, to whom the topic drop could refer.

This leads us to the second counterexample against sufficiency:
topics that cannot be omitted because the hearer is unable to link them to a previous mention in the discourse.%
%% Footnote
\footnote{Such cases pose no problem for approaches that assume that only continuous or continued topics can be omitted \citep[e.g.,][]{schulz2006,freywald2020}.}
%
\citet{krifka2007} discusses example \ref{ex:topic.unrecent} uttered by speaker A to speaker B in a context where both A and B know John but have not talked about him in their recent conversation, and where John is also not present in the situation.

\ex.\label{ex:topic.unrecent}  Did you know? [John]\textsubscript{Topic} [married last week]\textsubscript{Comment}. \citep[43, topic-comment structure added]{krifka2007}

A's second sentence clearly is about John.
The information about the recent marriage will be stored under the entry for John in the common ground of A and B. \is{Common ground}
This means that for John to be the topic of the utterance, he does not need to be part of the ``recent state of the CG content'', but it suffices that there is ``a long established and known interest of B in John'' \citep[43]{krifka2007}.
However, for the topic expression \textit{John} to be targeted by topic drop, the reference of the omitted constituent needs to be recoverable (see Chapter \ref{ch:recover}), like basically any type of omission.
An omission out of the blue, as in \ref{ex:topic.unrecent.td}, is not possible as B will not be able to identify and recover the omitted topic expression.

\ex.\label{ex:topic.unrecent.td}  Did you know? *$\Delta$ Has married last week.

The fact that contrastive topics \is{Contrastive topic} and unrecoverable topics cannot be targeted by topic drop constitutes counterevidence to topicality being a sufficient condition for topic drop.
These are topic expressions in the prefield that cannot be omitted so that we may conclude that not all topic expressions can be targeted by topic drop.
However, one could restrictively object that in the cases of contrastive topics \is{Contrastive topic|)} and non-recoverable \is{Recoverability|)} topics with focus/contrast \is{Focus|)} and lack of recoverability, there are, as it were, ``confounding factors'' $R$ that can override the possibility to omit the constituent conditioned by topicality.
Consequently, at most equation \ref{eq:suff.nonstrict} applies but not equation \ref{eq:suff}.%

\begin{equation}\label{eq:suff.nonstrict}
\begin{aligned}
P + \neg R \Rightarrow Q: \text{prefield constituent topical} \\ + \text{ no confounding factors} \Rightarrow \text{constituent omittable}
\end{aligned}
\end{equation}

\noindent
In this case, then, we can more conservatively conclude that topicality is at least not a \emph{strictly} sufficient condition for topic drop.

\subsection{Contra topicality as a necessary condition}\label{sec:topicality.ness}
The second aspect of the question of whether topic drop only targets topical constituents is whether topicality is a necessary condition for topic drop.
A condition $P$ is necessary if $P$ must be true whenever the consequence $Q$ is true, as shown in equation \ref{eq:ness}.
\begin{equation}\label{eq:ness}
Q \Rightarrow P : \text{constituent omittable} \Rightarrow  \text{prefield constituent topical}
\end{equation}

\noindent
This means that for any constituent that can be omitted from the prefield position, it must hold that this constituent is topical.
Only topic expressions should be targeted by topic drop, while non-topical constituents should not.
Since there is widespread agreement in the literature that non-referential constituents cannot be topics (see \sectref{sec:topicality.lit}), it would be an argument against necessity if non-referential prefield constituents can be omitted.
A first example of non-referential elements are quantifiers such as \textit{jeder} (`everybody') or \textit{keiner} (`nobody'), which apparently indeed cannot be targeted by topic drop.%
%% Footnote
\footnote{It is probably the impossibility of reconstructing the omitted constituent that blocks topic drop of quantifiers.
In example \ref{ex:quantifiers}, both \textit{jeder} and \textit{keiner} could theoretically have been omitted, but it could also be a referential 3rd person singular pronoun that is unrealized.
See also the related discussion on indefinite pronouns and the pronoun \textit{man} in Footnote \ref{note:man} on page \pageref{note:man}.

\exg.*$\Delta$ hat die Klausur bestanden.\label{ex:quantifiers}\\
everybody/nobody has the exam passed\\
`*(Everybody/nobody) passed the exam.'

}
%
The second example is the expletive \is{Expletive} \textit{es} (`it'),%
%% Footnote
\footnote{Recall that, unlike \citet{frick2017}, for example, I do not consider the ``omission'' of other \textit{es} types, such as the placeholder \textit{es} and the correlative \textit{es}, to be topic drop.
See Sections \ref{sec:def.constituent} and \ref{sec:def.v1} for details.}
%
which is used as the subject of weather verbs like \textit{regnen} (`to rain') \ref{ex:expletive.rain}, of copulas \is{Copula} in combination with adjectives like \textit{kalt} (`cold') \ref{ex:expletive.cold}, of verbs with further arguments in impersonal constructions like \textit{geben} (`to give') \ref{ex:expletive.pizza} and of verbs of perception like \textit{stinken} (`to stink') \ref{ex:expletive.stinks}, which also allow for referential subjects \ref{ex:expletive.garbage} \citep[see][191--192]{eisenberg2020}.

\ex.
\ag.Es regnet.\label{ex:expletive.rain}\\
it rains\\
`It is raining.'
\bg.Es ist kalt.\label{ex:expletive.cold}\\
it is cold\\
`It is cold.'
\bg.Es gibt Pizza.\label{ex:expletive.pizza}\\
it gives pizza\\
`There is pizza.'
\bg.Es stinkt.\label{ex:expletive.stinks}\\
it stinks\\
`It stinks.'
\bg.Der Müll stinkt.\label{ex:expletive.garbage}\\
the garbage stinks\\
`The garbage stinks.'

In the theoretical literature, there is strong dissent about whether topic drop can or cannot target expletives. \is{Expletive}
\citet[34]{fries1988}, \citet[81]{cardinaletti1990}, \citet[260]{kaiser2003}, \citet[116]{haegeman2007}, and \citet[213]{volodina.onea2012} judge examples like \ref{ex:expletives.contra} and \ref{ex:expletives.contra.2} as ungrammatical.

\exg.\label{ex:expletives.contra}A: Und wie ist das Wetter bei euch?\\
{} and how is the weather at you.\textsc{2pl}\\
A: `And how is the weather with you?'
\ag.B: *$\Delta$ Ist warm. / *$\Delta$ Ist kalt.\\
{} \phantom{*}it is warm {} \phantom{*}it is cold\\
B: `(It) is warm.' / `(It) is cold.'
\bg.B: *$\Delta$ Regnet. / *$\Delta$ Schneit. \\
{}  \phantom{*}it rains {} \phantom{*}it snows \\
B: `(It) is raining.'  / `(It) is snowing.'  \citep[34, his judgments]{fries1988}

\exg.*$\Delta$ Freut mich, dir zuzuhören.\label{ex:expletives.contra.2}\\
it pleases me you.\textsc{dat.2sg} listen.to\\
`I'm glad to listen to you.' \citep[34, his judgment]{fries1988}

In contrast, \citet[116]{poitou1993}, \citet[218]{reis2000}, \citet[271]{volodina2011}, \citet[120]{trutkowski2016}, \citet[67]{frick2017}, and \citet[220; 222--224]{ruppenhofer2018} discuss corpus \is{Corpus|(} examples demonstrating that speakers do use topic drop to omit expletives \is{Expletive} in the prefield.
\citet{frick2017} presents evidence from Swiss German text messages, such as \ref{ex:expletives.corpus.swiss}.
In her corpus data, she found 83 utterances with realized expletive \textit{es} in the prefield and 179 utterances where an expletive \textit{es} is omitted from this position.
The resulting omission rate of 68.32\% is significantly higher ($\chi^2(1) = 6.03$, $p < 0.05$)%%
%% Footnote
\footnote{I calculated a Pearson's chi-squared test with Yates's continuity correction in R \citep{rcoreteam2021}.}
%%
than the rate of the referential personal pronoun of the 3rd person singular neuter of 57.32\% (137 omitted instances vs. 102 realized instances) \citep[140]{frick2017}.

\ex.\label{ex:expletives.corpus.swiss}
\ag.hüt chani s tel dänn abnäh, $\Delta$ rägnet nüme und mir stecked nöd im morast\\
today can.I the phone then pick.up it rains no.more and we stick not in.the mud\\
`Then today I can pick up the phone, (it)'s no longer raining and we're not stuck in the mud.' \citep[150]{frick2017}
%\vspace{-1em}
\bg.$\Delta$ Wird ehner spat, $\Delta$ bliibe na chli.\\
it gets rather late I stay still a.little\\
`(It)'s getting rather late, (I)'ll stay a bit longer.' \citep[151]{frick2017}

\citet{ruppenhofer2018} supports this finding for the Standard German of Germany.
In a sample of German Twitter data, he found several cases of topic drop of expletives \is{Expletive} with weather verbs, although the possibility of omission seems to depend on the lexeme used \citep[222--223]{ruppenhofer2018}, see Table \ref{tab:rupp.ex}.

\begin{table}
\caption{Omission rates by weather verb in the twitter data by \citet{ruppenhofer2018}, taken from \citet[223]{ruppenhofer2018}}
\begin{tabular}{lrrrr}
\lsptoprule
Weather verb & Full form & Topic drop & Total & Omission rate \\
\midrule
\textit{regnen} (`to rain') & $26$ & $11$ & $37$ & $29.73\%$\\
\textit{pissen} (`to rain hard') & $63$ & $1$ & $64$ & $1.56\%$\\
\textit{nieseln} (`to drizzle') & $41$ & $1$ & $42$ & $2.38\%$\\
\textit{schiffen} (`to rain hard') & $35$ & $10$ & $45$ & $22.22\%$\\
\textit{schneien} (`to snow') & $97$ & $1$ & $98$ & $1.02\%$\\
\lspbottomrule
\end{tabular}
\label{tab:rupp.ex}
\end{table}

\largerpage
\noindent
Additionally, in the fragment corpus FraC (\cite{horch.reich2017}, see \sectref{sec:corpus.frac} for details), several cases of omitted expletives \is{Expletive} produced by different authors in different text types \is{Text type} can be found, such as \ref{ex:expletive.frac.gibt} from a conversation about a hotel (the omitted referential constituent refers to this hotel), the tweet \ref{ex:expletive.frac.warm}, or the text message \ref{ex:expletive.frac.besetzt}.%
%% Footnote
\footnote{Example \ref{ex:pf.ex} shows that the prefield restriction also applies to examples \ref{ex:expletive.frac.gibt}, \ref{ex:expletive.frac.warm}, and \ref{ex:expletive.frac.besetzt}.
Therefore, it seems legitimate to assume that the omission of non-referential constituents is just as much topic drop as that of referential ones.
%\vspace{-0.5\baselineskip}
\ex.\label{ex:pf.ex}
\a.*Hallenbad gibt $\Delta$ auch.
\b.*Zum Austoben gibt $\Delta$ einen Fitnessraum.
\c.*Draußen ist $\Delta$ ja fast schon warm.
%\vspace{-1.5\baselineskip}}
}

\ex.\ag.\label{ex:expletive.frac.gibt}ja, $\Delta$\textsubscript{expletive} gibt auch Hallenbad, $\Delta$\textsubscript{referential} ist zentral, $\Delta$\textsubscript{expletive} gibt einen Fitneßraum zum Austoben.\\
yes it gives also indoor.swimming.pool it is central it gives a fitness.room to.the work.out \\
`yes, (there) is also an indoor swimming pool, (it) is central, (there) is a fitness room to work out in.' [FraC D867]
\bg.\label{ex:expletive.frac.warm}Oh... $\Delta$ ist ja fast schon warm draußen.\\
oh it is \textsc{part} almost already warm outside\\
`Oh... (It) is almost warm outside already.'
[FraC T261]
\cg.\label{ex:expletive.frac.besetzt}$\Delta$ WIRD ZEIT FÜR DEINE RÜCKKEHR!\\
it becomes time for your.\textsc{2sg} return\\
`(It) is time for your return!'
 [FraC S748]

These examples indicate that expletives \is{Expletive} can be omitted, which challenges the assumption that topicality is a necessary condition for topic drop. \is{Corpus|)}

\citet{trutkowski2011, trutkowski2016} tries to reconcile, at least for weather verbs, the restriction of topic drop to topics and the possibility of omitting expletives.
She suggests that an expletive subject can function as a topic expression that refers to an event or situation, and in these cases can be targeted by topic drop:
\begin{aquote}{\cite[120--121]{trutkowski2016}; see also \cite[213]{trutkowski2011}}
[A]s situations are referential and representable by overt elements (also by expletive pronouns), nothing speaks against the fact that these overt elements can be dropped.
I.e., I assume that the (dropped) expletive \is{Expletive} pronoun represents the event that is the current situation.
\end{aquote}

\noindent
She illustrates this with the contrast in \ref{ex:trutkowski.weather}.
While she claims \ref{ex:trutkowski.weather.sit} to be fine because the situation is present through the speaker looking out of the window, she marks \ref{ex:trutkowski.weather.nosit} as ungrammatical because the situation the expletive refers to is not present in the discourse context:
``Whenever the event is actually present, the expletive \textit{es} can be omitted -- however, this is not possible when the event is not present in the given discourse'' \citep[120]{trutkowski2016}.%
% Footnote
\footnote{Note that there are also two types of usages of the expletive \is{Expletive} in example \ref{ex:trutkowski.weather}: an episodic \ref{ex:trutkowski.weather.sit} and a generic usage \ref{ex:trutkowski.weather.nosit}.
It may well be that this is the relevant difference that leads Trutkowski to consider \ref{ex:trutkowski.weather.nosit} as ungrammatical.
However,  Trutkowski does not mention this difference.
While in this book I only tested her proposal of the presence/absence of the situation, it may be beneficial in a future study to also contrast topic drop of different usages of expletives.
I thank Ingo Reich (p.c.) for pointing out the two usage types to me.
}

\ex.\label{ex:trutkowski.weather}
\ag.\label{ex:trutkowski.weather.sit}{[While looking out of the window:]} $\Delta$ Regnet grad'.\\
{} it rains right.now\\
[While looking out of the window:] `(It) is raining right now.'
\bg.*\label{ex:trutkowski.weather.nosit}$\Delta$ Regnet bestimmt, wenn wir in Urlaub fahren.\\
it rains definitely when we in vacation go\\
`(It) will definitely rain when we go on vacation.'\\\phantom{.} \hfill \citep[120, her judgments]{trutkowski2016}

Trutkowski bases her argumentation on \citet{falk1993}, who considers a topic drop interpretation for expletive \is{Expletive} drop in Swedish. \il{Swedish}
Falk states that in this case ``[t]he notion `topic' must be given a somewhat wider interpretation, including situations that are often expressed in impersonal expressions as a topic of discourse, like the weather'' \citep[172]{falk1993}.
It is worth noting that Falk, unlike \citet{trutkowski2011, trutkowski2016}, does not explicitly call the expletive \textit{es} the topic expression but states that impersonal expressions serve the purpose of expressing a situation in the form of a discourse topic. \is{Discourse topic}
Even clearer is \citet{firbas1982}, who terms sentences with weather verbs like \textit{regnen} (`to rain') \textit{thetic} sentences, following a philosophical tradition \citep[see, e.g.,][]{ulrich1986}.
He states that these sentences express an event but do not have a rheme so the expletive \is{Expletive} subject is ``a mere dummy element'' \citep[105]{firbas1982}.
\citet{krifka2007} argues that although thetic sentences do not have a topic expression, ``they do have a topic denotation, typically a situation that is given in the context'' \citep[43]{krifka2007}.
\citet{erteschik-shir1997, erteschik-shir2007} talks about \textit{stage topics} ``indicating the spatio-temporal parameters of the sentence (here-and-now of the discourse)'' \citep[16]{erteschik-shir2007}.
To sum up, sentences with expletive subjects and weather verbs are regularly classified as thetic sentences without a topic expression.
This means that the expletive subject \textit{es} is not a topic expression.
If \textit{es} can be omitted, this argues against topicality being a necessary condition for topic drop.

This is different with \citeauthor{trutkowski2016}'s (\citeyear{trutkowski2011}, \citeyear{trutkowski2016})  account, as she considers the possibility of identifying the topic with the expletive \is{Expletive} subject.
In what follows, I present an experiment with which I tested her proposal empirically to exclude the possibility that the omission of expletives before weather verbs is the omission of a topic in Trutkowski's sense specified above.
However, it is important to stress that Trutkowski's account was sketched for weather verbs and it remains unclear how other types of expletive \is{Expletive} subjects would be treated in cases where it may be not that easy to find a suitable situation.

\setcounter{expcounter}{0}
\refstepcounter{expcounter}\label{exp:ex}
\subsection{Experiment \arabic{expcounter}: topic drop of expletives}
\is{Acceptability rating study|(}\label{sec:exp.ex}  \is{Expletive|(}
Experiment \arabic{expcounter}%
% Footnote
\footnote{All materials and the analysis script can be found online: \url{https://osf.io/zh7tr}.}
%
was an acceptability rating study that tested the following claim by \citet{trutkowski2011, trutkowski2016}:
Topic drop of the expletive subject \textit{es} of weather verbs is said to be possible, provided that the subject represents a topic denotation in the form of the current weather situation, which must be present in the discourse.%
% Footnote
\footnote{\label{note:ex.pilot}At a poster presentation \citep{schafer2022}, I presented a pilot study of this experiment with eight token sets, which yielded a comparable result.}
%
Trutkowski's claim conflicts with the prevailing research opinion that non-referential expletive elements cannot serve as topics.
In this view, it would speak against topicality as a necessary condition for topic drop if their omission is acceptable.

As mentioned above, \citet{trutkowski2011,trutkowski2016} argues that the situation that serves as the topic must be present in the current discourse.
I operationalized this through a question that either asked for information about the current weather or not (see \sectref{sec:exp.ex.materials} below).
Following \citet{trutkowski2011,trutkowski2016}, the weather question should make the weather, i.e., the current situation, present in the discourse and set it as the topic.
With the other question proposing an activity, this was not the case.
The weather as the current situation did not occur until the target sentence, so the situation was not set as the topic.
Accordingly, the overt or covert expletive subject pronoun in the target sentence should once be a topical expression referring to the weather and once not. 

The experiment had the form of a 2 $\times$ 2 design crossing \textsc{Completeness} (full form vs. topic drop) and \textsc{Question Type} (weather vs. other).
If Trutkowski's (\citeyear{trutkowski2011,trutkowski2016}) approach is on the right track, there should be a significant interaction between both predictors so that topic drop is rated as more natural in the context of a weather question that sets \textit{es} (`it') as topic expression than in the context of an unrelated question, whereas the full forms should be natural after both types of questions.
If only referential expressions can be topics and if topicality is generally a necessary condition for topic drop, this predicts that topic drop should be degraded to the same extent compared to the full forms, regardless of the preceding question.
On the contrary, if topicality is not a necessary condition for topic drop, there should be no difference between the utterances with topic drop and the full forms in both question conditions.

\subsubsection{Materials}\label{sec:exp.ex.materials}
\subsubsubsection*{Items}
I tested 24 items in the form of question-answer pairs, such as \ref{ex:item.ex}.

\ex.\label{ex:item.ex}
\a.\label{ex:item.ex.w}
\ag.A: Was macht das Wetter bei dir?\\
{} what makes the weather at you.\textsc{dat.2sg}\\
A: `How's the weather with you?'
\bg.B: (Es) regnet leider schon wieder ziemlich heftig\\
{} it rains unfortunately already again pretty hard\\
B: `Unfortunately (it) is raining pretty hard again' \hfill (weather)
\z.
\b.\label{ex:item.ex.o}
\ag.A: Wolltest du nicht joggen gehen?\\
{} wanted you.\textsc{2sg} not jog go\\
A: `Didn't you want to go jogging?'
\bg.B: (Es) regnet leider schon wieder ziemlich heftig\\
{} it rains unfortunately already again pretty hard\\
B: `Unfortunately (it) is raining pretty hard again' \hfill (other)

The question either asked for the weather in the current location of the addressee(s) \ref{ex:item.ex.w} or did not mention the weather but suggested an activity or asked why an activity did not come about \ref{ex:item.ex.o}.
The answer was an utterance with a weather verb and the expletive subject \textit{es} realized in or omitted from the prefield.%
% Footnote
\footnote{I indicate the two levels of the predictor \textsc{Completeness}, full form and topic drop, by putting the prefield constituent into parentheses.
Note that the target sentences were shown without a final period, as such points are often omitted in instant messages or replaced by emojis.}
%

Since weather verbs tend to be a closed class with a limited number of members, of which again only a few are very frequent, I did not use 24 different verbs but tested six frequent verbs in four different lexical environments respectively:
\textit{regnen} (`to rain'), \textit{stürmen} (`to storm'), \textit{schneien} (`to snow'), \textit{donnern} (`to thunder'), \textit{nieseln} (`to drizzle'), and \textit{schütten} (`to pour (with rain)').%
%% Footnote
\footnote{For example, I did not consider (i) verbs like \textit{graupeln} (`to sleet'), \textit{winden} (`to be windy'), and \textit{gewittern} (`to thunder', lit. `to thunderstorm') because they are rarely used, (ii) synonyms for heavy raining like \textit{pissen} (`to piss'), \textit{gießen} (`to pour'), or \textit{schiffen} (`to pee', literally `to travel by ship') because they are very colloquial and partly vulgar, and (iii) verbs like \textit{pladdern} (`to rain with heavy drops') or \textit{drippeln} (`to drizzle') because they are regionally restricted.
Furthermore, I excluded \textit{blitzen} (`to flash', used for lightning) because it deviated remarkably from the other verbs already in the full forms in the pretest that I conducted (see Footnote \ref{note:ex.pilot}).
As a consequence, I also excluded \textit{hageln} (`to hail'), which was also tested in the pilot study, to have an even number of verbs (and token sets).
}
This means that there were four token sets respectively with each of the six verbs.
Another token set with \textit{regnen} (`to rain') is shown in \ref{ex:item.ex.regnen2}.

\ex.\label{ex:item.ex.regnen2}
\a.
\ag.A: Wie ist das Wetter bei euch?\\
{} how is the weather at you.\textsc{dat.2pl}\\
A: `How is the weather with you?' \hfill (weather)
\bg.A: Und, was hältst du von einem Spaziergang im Park?	\\
{} and what hold you.\textsc{2sg} of a walk in.the park\\
A: `So, what do you think about a walk in the park?' \hfill (other)
\z.
\bg.B: (Es) regnet wohl noch den ganzen Tag\makebox[0pt][l]{\vspace*{-2cm}\raisebox{-0.35ex}{\includegraphics[scale=0.07]{Emojis/unamused-face_wa.png}}}\\
{} it rains \textsc{part} still the whole day \\
B: `(It) will probably rain all day\makebox[0pt][l]{\vspace*{-2cm}\raisebox{-0.35ex}{\includegraphics[scale=0.07]{Emojis/unamused-face_wa.png}}}\phantom{mi}'

\subsubsubsection*{Fillers}
The 24 items were presented together with a total of 80 fillers.
24 of the fillers were also question-answer pairs that contained gapping or right node raising structures in the answers.
Another 24 fillers were the items of experiment \ref*{exp:1sg.2sg}, which tested topic drop of the 1st and 2nd person singular and consisted of two turns by one speaker (see \sectref{sec:exp.1sg.2sg} for details).%
%% Footnote
\footnote{Experiment \arabic{expcounter} and experiment \ref*{exp:1sg.2sg} were conducted as part of the same study.
Their items both contained topic drop but were otherwise unrelated.
Experiment \arabic{expcounter} tested topic drop of 3rd person singular expletives before weather verbs, while in experiment \ref*{exp:1sg.2sg} 1st and 2nd person singular pronouns referring to sender and addressee were omitted.
Having relatively many topic drop structures in the study should not be a problem since they are very common in instant messages (see the discussion in \sectref{sec:corpus.texttype}).
}
The same structure of two turns produced by one speaker was used for 24 fillers with sluicing and sprouting structures in the target utterances.
Additionally, I included eight ungrammatical catch trials with word order violations in the last utterance such as \ref{ex:filler.catch.ex}, which served as attention checks and exclusion criterion for the participants.

\ex.\label{ex:filler.catch.ex}
\ag.A: Hast du in letzter Zeit mal was von Ricardo gehört?\\
{} have you.\textsc{2sg} in last time \textsc{part} what about Ricardo heard\\
A: `Have you heard from Ricardo lately?'
\bg.B: Er in Köln wollte eine Ausbildung zum anfangen Mediengestalter\\
{} he in Cologne wanted a training to.the start media.designer\\
B: `He wanted to start training as a media designer in Cologne' \\
(Lit. `He in Cologne wanted a training as a start media designer') \\\phantom{.} \hfill (ungrammatical)%


\subsubsection{Procedure} \label{sec:exp.ex.procedure}
I used a web-based procedure to collect acceptability rating judgments.
I implemented a survey using the LimeSurvey online survey tool \citep{limesurveygmbh}.
As participants, I recruited 47 German native speakers between the ages of 18 and 40 from the crowdsourcing platform Clickworker \citep{clickworker2022}%
%% Footnote
\footnote{Due to technical problems, I could only collect data from 47 instead of the planned 48 participants.}
%
who were compensated with €3.30.%
%% Footnote
\footnote{For each study in this book, the compensation was based on the statutory minimum wage in Germany at the time of the study and a realistic estimate of the expected completion time.}
%
Their task was to read the instant messaging dialogues and to rate how natural the last utterance is in the given context on a 7-point Likert scale with labeled end points: 1 was labeled ``vollkommen unnatürlich'' (`completely unnatural'), 7 was labeled ``vollkommen natürlich'' (`completely natural').
Participants were asked to base their judgments on the question of whether they or people with whom they communicate via WhatsApp or similar services would write this message in the corresponding dialogue in this way.
They were told that there is no right or wrong but that all that matters is their intuition.

The 24 items were distributed across four lists according to a Latin square design so that each participant saw each token set exactly once and in only one condition.
Items and fillers were presented in an individual pseudo-randomized order, ensuring that no two stimuli of the same type immediately followed each other.
All materials were presented as instant messaging dialogues between two persons.
I added a matching emoji to some of the materials, such as the item in \ref{ex:item.ex.regnen2}, to increase their overall naturalness as instant messages.%
%% Footnote
\footnote{All emojis shown in this book are the WhatsApp versions obtained from \url{https://emojipedia.org/whatsapp/} (visited on 01/02/2025).
In the experiments, equivalent versions were shown to participants depending on the browser they used to participate in the experiment.}
%
The appearance of the chat resembled that of the world's largest instant messaging service WhatsApp \citep{statista2022}, as shown in Figure \ref{fig:wa.design}, and was created using CSS code by \citet{rocha2022}.

\begin{figure}
\centering
\includegraphics[scale=0.4]{Exp_Ex_Screenshot.png}
\caption{Screenshot of the instant messaging design of experiment \arabic{expcounter}}
\label{fig:wa.design}
\end{figure}

\subsubsection{Data analysis} \label{sec:data.analysis}
In what follows, I describe in detail how I analyzed the resulting acceptability rating data.
Since the procedure for all rating experiments in this book is more or less identical, I am not as detailed in describing the further experiments but only outline the most important steps and any deviations from the following general scheme.

Before the analyses, I excluded the data from presumably inattentive participants who gave too good ratings to the ungrammatical catch trials.
For experiment \arabic{expcounter}, I defined ``too good ratings'' beforehand as rating four or more of the eight catch trials with 6 or 7 points.
This is a liberal threshold, which allowed me to only exclude the data from subjects who repeatedly gave massively inappropriate ratings.
For experiment \arabic{expcounter}, this resulted in the loss of the data from nine participants.
I statistically analyzed the data from the remaining participants, 38 in experiment \arabic{expcounter}, using the programming language R (version 4.1.0, \cite{rcoreteam2021})%
%% Footnote
\footnote{All analyses in this book used the same R-version to maximize their comparability.}
%
and the integrated development environment RStudio (version 2022.02.3, \cite{rstudioteam2021}).

The dependent variable in all of the rating experiments were numeric values between 1 and 7 indicating the perceived naturalness of the experimental stimuli assessed with a 7-point Likert scale.
It is a matter of debate whether these data can be treated as interval data or need to be considered ordinal \citep{carifio.perla2008}.
While ordinal variables require a meaningful order, for interval variables furthermore the difference between two values needs to be equal \citep[26]{rasinger2013}.
Likert scale data are clearly ordered, e.g., a rating of 5 is higher than a rating of 4, but it is unclear whether the distance between 4 and 5 is equal to the distance between a rating of 1 and a rating of 2 \citep[33--34]{schutze.sprouse2014}.
Such an equal distance between scale points would be the prerequisite to analyzing the data with parametric tests like linear mixed effect models.
In this book, I pursue a rather conservative strategy by treating the rating data as ordinal (see, e.g., \cite{jamieson2004}; but cf. \cite{jaccard.wan1996}, \cite{carifio.perla2008}) and analyze them with cumulative link mixed models (CLMMs) from the package ordinal for ordinal data (version 2019.12-10, \cite{christensen2019}).
These CLMMs take into account that the distances between the points of a scale may not be equal, but they also allow for an adjustment of this assumption.
The threshold parameter of the CLMM function can be used to set constraints on the thresholds to simplify the corresponding model.
It can be changed from the default flexible thresholds (scale points are ordered, but there are no further restrictions) to symmetric (distance between scale points is symmetric around the central point), symmetric2 (the latent mean of the reference group is 0) or equidistant (equal distance between scale points) \citep{christensen2019}.
For the final model, I always started with the most liberal flexible thresholds and subsequently tried simpler models with more restricting symmetric, symmetric2, or equidistant thresholds.
I compared the complex and the simpler final model with likelihood ratio tests calculated with R's anova function \citep{rcoreteam2021} and kept the simpler model if the model fit was not significantly different between models.

CLMMs allow for the analysis of an ordinal response variable considering fixed effects of categorical or continuous independent variables, as well as random effects.
The independent variables in my experiments were usually binary predictors coded with deviation coding%
%% Footnote:
\footnote{Deviation coding is a coding scheme for categorical variables that ``compares the mean of the dependent variable for a given level to the overall mean of the dependent variable'' \citep{coding}, the so-called grand mean.
For a binary independent variable, the two levels are usually coded either as $1$ and $-1$ respectively (this is also called sum coding) or as $0.5$ or $-0.5$ (this is called deviation coding in the narrower sense).
For such a binary variable coded with deviation coding, the intercept corresponds to the mean between conditions, while the slope indicates the deviation between the conditions and the mean \citep{alday2022}.}
%
or, less often, numeric scores.
The models contained the main effects of these independent variables, as well as the scaled and centered numeric position at which the critical trial was presented in the experiment, and all the two-way interactions between the predictors.
To cover possible differences between the token sets and the participants, the models had random intercepts for items and subjects and by-item and by-subject random slopes.

In my analyses, I always started with a maximal model including all fixed and random effects \citep{barr.etal2013}.
In case of convergence issues, I simplified the random effects structure starting with eliminating random slopes for interactions until the model converged.
I systematically reduced the full model to the final model by using a backward model selection procedure for the fixed effects.
In each step, I excluded a non-significant fixed effect and compared the models with and without the effect through likelihood ratio tests performed with the anova function in R \citep{rcoreteam2021}.
If the more complex model did not have a significantly better model fit, I continued with the simpler model.
The random effects structure remained unaffected by this procedure, i.e., constant for all models.
The final model was always a model where all fixed effects were either significant or part of a significant higher-order interaction.
To obtain the test statistics in the form of $\chi^2$ values and the \textit{p}-values for the effects in the final model, I used the same procedure as for model selection.
Again, I compared a model with and a model without the effect in question with the anova function.
For each analysis in this book, I present the model call of the full model and the model table showing the fixed effects in the final model.

\subsubsection{Results}\label{sec:exp.ex.results}
Table \ref{tab:descriptives.ex} shows the mean ratings and standard deviations of the four conditions and Figure \ref{fig:pl.ex} the mean ratings and 95\% confidence intervals.%
%% Footnote
\footnote{All point-line plots in this book were created in R \citep{rcoreteam2021} using the package ggplot2 \citep{wickham2016}.}
%
They suggest that utterances following a weather question were preferred over utterances following another question, while there seems to be little difference between the full forms and the utterances with topic drop.

\begin{table}
\caption{Mean ratings and standard deviations per condition for experiment \arabic{expcounter}}
\centering
\begin{tabular}{llrr}
\lsptoprule
\textsc{Completeness} & \textsc{Question Type} & \Centerstack{Mean\\rating} & \Centerstack{Standard\\deviation} \\
\midrule
Full form & Weather question & $5.84$ & $1.35$ \\
Topic drop & Weather question & $5.88$ & $1.32$\\
Full form & Other question & $5.39$ & $1.55$ \\
Topic drop & Other question & $5.30$ & $1.60$ \\
\lspbottomrule
\end{tabular}
\label{tab:descriptives.ex}
\end{table}

%\vspace{-0.5\baselineskip}
\begin{figure}
\centering
\includegraphics[scale=1]{Experimenteplots/PL_Ex.pdf}
 \caption{Mean ratings and 95\% confidence intervals per condition for experiment \arabic{expcounter}}
\label{fig:pl.ex}
\end{figure}

I performed an inferential statistical analysis of the data, as described in the previous section.
The full CLMM contained the ordinal ratings as the response variable.
As fixed effects, I included the main effects of the independent variables \textsc{Completeness} (deviation coded, full form as $0.5$, topic drop as $-0.5$), \textsc{Question Type} (deviation coded, weather questions as $0.5$, other questions as $-0.5$), and of the numeric centered and scaled \textsc{Position} at which the trial was presented in the experiment, as well as all two-way interactions.
The random effects structure consisted of random intercepts for subjects and items and of by-subjects and by-items random slopes for all three predictors and the two-way interactions between them.%
%% Footnote
\footnote{The formula of the full model was as follows:
\texttt{Ratings \textasciitilde~(\textsc{Completeness} + \textsc{Question Type} + \textsc{Position})\textasciicircum2 + (1 + (\textsc{Completeness} + \textsc{Question Type} + \textsc{Position})\textasciicircum2 | Items) + (1 + (\textsc{Completeness} + \textsc{Question Type} + \textsc{Position})\textasciicircum2 | Subjects)}.}
%
By performing a backward model selection, I obtained the final model with symmetric thresholds and a significant main effect of \textsc{Question Type}, as shown in Table \ref{tab:ex.model}.

\begin{table}
\caption{Fixed effect in the final CLMM of experiment \arabic{expcounter}}
\centering
\begin{tabular}{lrrrll}
\lsptoprule
Fixed effect & Est. & SE & $\chi^2$ & \textit{p}-value &   \\
\midrule
\textsc{Question Type} & $0.91$ & $0.23$ &  $12.05$ & $< 0.001$ & ***\\
\lspbottomrule
\end{tabular}
\label{tab:ex.model}
\end{table}

\noindent
Regardless of topic drop, utterances in the context of a weather question were rated as more acceptable than utterances in the context of another question ($\chi^2(1) = 12.05$, $p < 0.001$).
The interaction between \textsc{Completeness} and \textsc{Question Type} was not significant ($\chi^2(1) = 0.05$, $p~>~0.8$), nor the main effect of \textsc{Completeness} ($\chi^2(1) = 0.11$, $p~>~0.7$).

\subsubsection{Discussion}
The results of experiment \arabic{expcounter} are inconsistent with an explanation according to which only referential expressions can be topical and according to which topicality is a necessary condition for topic drop.
The utterances with non-referential and, thus, non-topical expletive subjects of weather verbs received comparable acceptability ratings regardless of whether they were syntactically complete or contained topic drop.
There is also no evidence for the proposal by \citet{trutkowski2011, trutkowski2016}, according to which topic drop of such expletive subjects is only possible if the situation they supposedly refer to is present in the current discourse.
There was no interaction between the syntactic completeness of the utterance and the type of question used.
Topic drop was not particularly degraded if the question did not ask about the current weather.
The main effect of \textsc{Question Type} seems to be pragmatically motivated.
A discourse where (in)complete utterances with weather verbs follow questions asking about the weather is more coherent than a discourse where the utterance with a weather verb is used as an excuse (not) to do something.
In the latter case, the hearer needs to draw a relevance implicature%
%% Footnote
\footnote{I thank Robin Lemke (p.c.) for pointing this out.}
%
 as there is a more indirect relation between question and answer.

In line with the corpus results by \citet{frick2017} and \citet{ruppenhofer2018} and the examples from the FraC, we can conclude that topic drop of expletives is possible and that it is as acceptable as the corresponding full forms.
This indicates that a constituent does not need to be topical in order to be targeted by topic drop.
Therefore, the results do not support the idea that topicality is a necessary condition for topic drop. \is{Acceptability rating study|)}
\is{Expletive|)}

\subsection{Summary: topic drop and topicality}
In this section, I discussed in detail the potential role of topicality for topic drop.
After having developed a topic definition drawing on the most central literature of the last 40 years, I showed that the prefield position is not to be equated with \textit{the} topic position in German.
The prefield can be filled with non-topical constituents, and topical constituents regularly appear in the middle field \is{Middle field} as well.
In the second half of the section, I argued on both theoretical and empirical grounds that topicality is neither a (strictly) sufficient nor a necessary condition for topic drop.
I showed that contrastive topics \is{Contrastive topic} and non-recoverable topics cannot be omitted and that topic drop also targets elements that by definition cannot be topics, such as non-referential expletive \is{Expletive} subjects.
This suggests that the term \textit{topic drop} is at least misleading in German.
In Chapter \ref{ch:factor.topicality}, I return to topicality and investigate whether it is a factor that favors topic drop.
For the remainder of this chapter, I focus on the prefield restriction as a positional restriction.
\is{Topic|)} 

\section{Is topic drop restricted to the prefield?}\label{sec:prefield.detail}
\subsection{Theoretical background}\label{sec:prefield.detail.theory}
\is{Middle field|(}
In light of the results about the relationship between topic drop and topicality gained in the previous section, it seems reasonable to  interpret the prefield restriction of topic drop as structural \citep[1850]{reich2011}.
As mentioned above, the assumption that topic drop in German is restricted to the prefield is the prevailing research opinion.
It is, to the best of my knowledge, only contradicted by \citet{helmer2016}, who argues that topic drop can also occur postverbally in the middle field.%
%% Footnote
\footnote{But see \citet[10]{nygard2018} for a similar view on \ili{Norwegian} topic drop.\is{Middle field}}
%
She states that the widespread assumption of a prefield restriction is related to the basic tendency that given \is{Givenness} information precedes new information in sentences.
Topics, which are usually given \is{Givenness} information, occur more often in the prefield and are omitted more often from there \citep[26--27]{helmer2016}.
However, according to her, topic drop still regularly occurs in the middle field, \is{Middle field} which she proves with examples from her corpus study. \is{Corpus|(}
In this study, Helmer examined a total of about 35 hours of spoken conversations from
the Forschungs- und Lehrkorpus Gesprochenes Deutsch FOLK (`The research and teaching corpus of spoken German') \citep{schmidt2014} and the corpus Gespräche im Fernsehen GIF (`conversations on TV') \citep[65]{helmer2016}.
In these corpora, she found 541 instances of topic drop, which she defines as the non-realization of non-expletive%
%% Footnote
\footnote{In this respect, she deviates from my definition of topic drop which includes expletive \is{Expletive} arguments, see \sectref{sec:def.constituent}.}
%
subject and object complements of fully uttered verbs \citep[70]{helmer2016}.
Of the 541 instances, she classifies 55 (10.2\%) as topic drop in the middle field \citep[214]{helmer2016}, but she mentions only five of them and provides little or no context.
She claims that the majority of the 55 cases are responsive utterances with mental verbs like \textit{nicht glauben} (`not believe') \citep[214]{helmer2016} and cites \ref{ex:helmer.glaube}, an instance from the FOLK, \is{Corpus} which I complemented with the relevant precontext.
However, for this example and similar cases, an analysis as null complement anaphor (NCA), where ``the understood sentential or VP complement of the verb must be interpreted from context'' (\cite[411]{hankamer.sag1976}, see \cite[778]{klein1993} for German examples) seems to be more promising than the assumption of topic drop in the middle field.

\ex.\label{ex:helmer.glaube}
\ag. A: Welche Methoden des Literaturunterrichts gibt es denn?\\
{} which methods the.\textsc{gen} literature.classes.\textsc{gen} gives it \textsc{part}\\
A: `What are the methods of teaching literature?'
\bg. B: Also es gibt natürlich Unterrichtsgespräche. 
		Es gibt Textanalyse. 			
		Handlungs- und produktionsorientierte Methoden. 
		Nja. Ich glaub mir fehlt eine aber… \\
{} so it gives of.course class.discussions it gives text.analysis action and production-oriented methods well I believe me.\textsc{dat} lacks one but\\
B: `So there is class discussion, of course. There is text analysis. Ac- tion- and production-oriented methods. Well. I believe I'm missing one but...'
\cg.\label{ex:helmer.glaube.rel}A: Ich glaube nicht, jedenfalls nicht nach dem Modell von ((AUTOR A)).\\
{} I believe not at.least not following the model of author A\\
A: `I don't think so, at least not according to the model of ((AUTHOR A)).' [FOLK\_E\_00038], simplified

By considering \textit{Ich glaube nicht} in \ref{ex:helmer.glaube.rel} a case of topic drop in the middle field, Helmer has to assume that the corresponding full form is \ref{ex:helmer.glaube.rec.mf}, i.e., a pronoun \textit{das} is omitted in the middle field, which refers back to the last utterance of B.
However, an alternative and more natural full form is the structure in \ref{ex:helmer.glaube.rec.sc}, in which the reference to the preceding utterance is made explicit in the subordinate \textit{dass}-clause.
This structure can be transformed to \ref{ex:helmer.glaube.rel} via NCA, i.e., by omitting the subordinate clause.

\ex.
\ag.\label{ex:helmer.glaube.rec.mf}Ich glaube das nicht.\\
I believe that not\\
`I don't believe this.'
\bg.\label{ex:helmer.glaube.rec.sc}Ich glaube nicht, dass Ihnen eine Methode fehlt.\\
I believe not that you.\textsc{2sg.pol} a method lack\\
`I don't believe that you are missing a method.'

Since \textit{glauben} in the reading of `considering something likely' most often takes a clause as its complement \citep{glauben}, the NCA analysis is a promising strategy to cover this most frequent type of \citeg{helmer2016} apparent counterexamples and to maintain the prefield restriction.
Still, though, this explanation does not cover all the cases that Helmer discusses.%
% Footnote
\footnote{Other examples of \citet{helmer2016} such as \textit{Sagen Sie ruhig} (`Go ahead and say (it)') might be a case of argument omission \is{Argument omission} possible only in imperatives \is{Imperative} as described in \citet{kulpmann.symanczykjoppe2016} and \citet{kulpmann2021}.
In the case of the question \textit{kann ich da liegen lassen} (`Can I leave (it) there'), either the speaker might have produced an involuntary elision of a \textit{s}, turning a demonstrative \textit{das} (`that') referring to the omitted object into the adverb \textit{da} (`there'), or the question could be a V2 question, in which the missing direct object is omitted from the prefield not from the middle field.}
%
Therefore, I break away from \citeg{helmer2016} individual examples in the next section and address the question of whether topic drop is possible in the middle field more generally, using an acceptability rating study that compared pre- and postverbal argument omissions. \is{Argument omission} \is{Corpus|)}

\refstepcounter{expcounter}\label{exp:pf.mf}
\subsection{Experiment \arabic{expcounter}: prefield restriction }
\is{Acceptability rating study|(}\label{sec:exp.prefield}
Experiment \arabic{expcounter} tested the main licensing condition of topic drop, namely its restriction to the preverbal prefield position of declarative V2 clauses.%
% Footnote
\footnote{All materials and the analysis script can be found online: \url{https://osf.io/zh7tr}.}
%
While almost all previous literature on topic drop in German lists this restriction, it was, to the best of my knowledge, only discussed with introspective examples.%
% Footnote
\footnote{\citeg{helmer2016} corpus data, \is{Corpus} discussed in \sectref{sec:prefield.detail.theory}, are an exception that opposes not only the prevailing research opinion but also the introspective approach.
}
%
The present acceptability rating study aimed to complement the introspection with empirical data.
I systematically collected acceptability judgments for utterances with topic drop in the prefield and for utterances with corresponding argument omissions \is{Argument omission|(} in the middle field, as well as for the corresponding full form, which served as a baseline.
This resulted in a 2 $\times$ 2 design crossing \textsc{Completeness} (full form vs. topic drop) and \textsc{Topological Position} (prefield vs. middle field).
If the majority of the previous theoretical literature is correct and topic drop is restricted to the prefield position, there should be an interaction between the predictors \textsc{Completeness} and \textsc{Topological Position}.
Argument omissions in the middle field are expected to be degraded compared to omissions in the prefield.
In contrast, if \citeg{helmer2016} assessment is correct, there should be no interaction between both independent variables.
The omissions should be rated equally well in the prefield and the middle field.%
%% Footnote
\footnote{It should be noted, however, that the cases that \citet{helmer2016} discusses as topic drop in the middle field were rarer in absolute numbers in her corpus \is{Corpus} than the cases of topic drop in the prefield.
If this were confirmed by relative numbers with corresponding full forms as the baseline, this could indicate a preference for topic drop in the prefield and result in a weak interaction in the experiment.}
%

\subsubsection{Materials}\label{sec:exp.pf.mf.materials}
\subsubsubsection*{Items}
To test these predictions, I created 24 items with a similar structure, such as \ref{ex:item.pf.mf}.
Person A asked a question to which person B responded with three utterances, the last of which was critical, i.e., \ref{ex:item.pf.mf.pf} or \ref{ex:item.pf.mf.mf}.
In these utterances it is varied whether the subject, which is always the 1st person singular subject pronoun \textit{ich} (`I'), is realized or omitted and whether it is positioned in the prefield \ref{ex:item.pf.mf.pf} or in the middle field \ref{ex:item.pf.mf.mf}.
In the middle field conditions, the prefield position was occupied by another constituent of the clause, such as the temporal adverbial \is{Adverbial} \textit{jetzt} (`now') in \ref{ex:item.pf.mf}, modal adverbials such as \textit{leider} (`unfortunately'), conjunctive adverbs such as \textit{deshalb} (`therefore'), or (prepositional) objects such as \textit{auf das Essen dort} in \textit{Auf das Essen dort steh (ich) aber total} (`But (I) am totally into the food offered there.').%
%% Footnote
\footnote{It cannot be excluded that the prefield constituent influences the (im)possibility of an ellipsis in the middle field, at least if the ellipsis targets a topical constituent.
Recall that \citet{frey2000} argues that the topic denotation must precede the sentence adverbial, \is{Adverbial} at least in the middle field.
If this principle is expanded to include the prefield, it should not be possible to have a sentence adverbial \is{Adverbial} in the prefield followed by a topic denotation in the middle field.
If we still find such cases, it could be argued that the apparent topic is not a topic precisely because it is preceded by a sentence adverbial. \is{Topic}
Although it is unclear whether such an expansion of the ordering principle is reasonable, it suggests that the presence of a sentence adverbial could principally influence the acceptability of omitting a constituent in the middle field.
However, since I tested a large number of different prefield constituents in my experiment, of which only a few were sentence adverbials, and the effects reported below showed up across all items, it seems unlikely that there was a systematic influence of some sort in my experiment.
}

\ex.\label{ex:item.pf.mf}
\ag.A: Was habt ihr heute Abend geplant?\\
{} what have you.\textsc{2pl} today evening planned\\
A: `What do you have planned for tonight?'
\bg.B: Wir wollen uns den neuen Matrixfilm im Kino anschau- \newline en \makebox[0pt][l]{\vspace*{-2.5cm}\raisebox{-0.35ex}{\includegraphics[scale=0.07]{Emojis/smiling-face-with-smiling-eyes_wa.png}}}\\
{} we want us the new matrix.movie in.the theater watch\\
\vspace{-\baselineskip}
B: `We want to watch the new Matrix movie in the theater \makebox[0pt][l]{\vspace*{-2.5cm}\raisebox{-0.35ex}{\includegraphics[scale=0.07]{Emojis/smiling-face-with-smiling-eyes_wa.png}}}\phantom{mi}'
\cg.B: Die ersten drei Filme gefallen mir total gut\\
{} the first three movies please me.\textsc{dat} totally well\\
B: `I totally like the first three movies'
\d.
\ag.\label{ex:item.pf.mf.pf}B: (Ich) bin jetzt richtig gespannt auf den neuen Teil\\
{} I am now rightly keen on the new part\\
B: `(I) am now really excited about the new part' \\\phantom{.}\hfill (full form / topic drop, prefield)
\bg.\label{ex:item.pf.mf.mf}B: Jetzt bin (ich) richtig gespannt auf den neuen Teil\\
{} now am I rightly keen on the new part\\
B: `Now (I) am really excited about the new part' \\\phantom{.}\hfill (full form / topic drop, middle field)

%\vspace{-1\baselineskip}
\subsubsubsection*{Fillers}
The items were mixed with a total of 72 fillers.
24 fillers were embedded gapping structures and corresponding full forms with V2 or verb-final word order, which, like the items, consisted of four turns.
16 fillers functioned as items for another experiment.
They were question-answer pairs, more specifically, dialogues with two turns.
The second turn was a prepositional phrase (PP) fragment answer exhibiting or not exhibiting preposition omission and/or article omission.
Another 24 fillers were items from experiment \ref*{exp:embedded} with two turns, which tested potentially embedded vs. unembedded topic drop structures and full forms with 3rd person singular subject pronouns (see \sectref{sec:exp.embedded} for details).%
%% Footnote
\footnote{Experiment \arabic{expcounter} and experiment \ref*{exp:embedded} were conducted as part of the same study.
The items served as fillers for each other, which seems to be unproblematic because the manipulations were independent.
In this experiment, I tested only topic drop of the 1st person singular manipulating the topological position of the covert constituent, while in experiment \ref*{exp:embedded} the items only contained topic drop of the 3rd person singular and it was varied whether they occurred in a potentially embedded clause or not. 
}
%
I also included eight catch trials with severe word order violations, similar to those used in experiment \ref*{exp:ex} (see \sectref{sec:exp.ex.materials}), which served as attention checks for the participants.
Half of them consisted of two turns, the other half of four turns.

\subsubsection{Procedure} \label{sec:exp.pf.mf.procedure}
Like experiment \ref*{exp:ex}, experiment \arabic{expcounter} was conducted as a web-based acceptability rating study set up with LimeSurvey \citep{limesurveygmbh}.
I recruited another 48 participants from Clickworker \citep{clickworker2022} who had not taken part in any of my other experiments on topic drop.
They were German native speakers between the ages of 18 and 40 and received €3.20 for their participation.
They should read the instant messaging dialogues and rate the naturalness of the last utterance on a 7-point Likert scale (1 = completely unnatural, 7 = completely natural) based on their intuitions (see \sectref{sec:exp.ex.procedure} for details on the instructions).
The 24 items were distributed across four lists according to a Latin square design to ensure that each token set was seen exactly once and in only one condition by each participant.
To avoid two stimuli of the same type immediately following each other, items and fillers were mixed and presented in individual pseudo-randomized order.
Like in experiment \ref*{exp:ex}, the materials were presented as instant messaging dialogues between two persons and occasionally contained matching emojis, as shown in item \ref{ex:item.pf.mf}.

\subsubsection{Results}\label{sec:exp.pf.mf.results}
I excluded the data from three presumably inattentive participants who had rated four or more of the eight catch trials with 6 or 7.
Table \ref{tab:descriptives.pf.mf} shows the mean ratings and standard deviations for the four experimental conditions calculated on the data from the remaining 45 participants. 
In Figure \ref{fig:pl.pf.mf} the mean ratings and 95\% confidence intervals are plotted.
While the ratings for full forms and topic drop were comparable when the overt or covert constituent was in the prefield, there was a clear drop in acceptability of topic drop in the middle field compared to the corresponding full form.
The mean rating of 4.94 was still relatively good, although participants differed more strongly in their judgment of this condition, as indicated by the increased standard deviation of 1.74.

\begin{table}
\caption{Mean ratings and standard deviations per condition for experiment \arabic{expcounter}}
\centering
\begin{tabular}{llrr}
\lsptoprule
\textsc{Completeness} & \Centerstack{\textsc{Topological}\\\textsc{Position}} & \Centerstack{Mean\\rating} & \Centerstack{Standard\\deviation} \\
\midrule
Full form & Prefield & $6.13$ & $1.25$ \\
Topic drop & Prefield & $6.16$ & $1.34$ \\
Full form & Middle field & $6.03$ & $1.29$ \\
Topic drop & Middle field & $4.94$ & $1.74$\\
\lspbottomrule
\end{tabular}
\label{tab:descriptives.pf.mf}
\end{table}

\begin{figure}
\centering
\includegraphics[scale=1]{Experimenteplots/PL_PFMF.pdf}
 \caption{Mean ratings and 95\% confidence intervals per condition for experiment \arabic{expcounter}}
\label{fig:pl.pf.mf} % pl for point line
\end{figure}

\noindent
I analyzed the data with CLMMs \citep{christensen2019}, as described in \sectref{sec:data.analysis}.
The full model contained fixed effects of the binary predictors \textsc{Completeness} and \textsc{Topological Position}, for which I used deviation coding (full form and prefield were coded as $0.5$, topic drop and middle field as $-0.5$ respectively), and of the numeric scaled and centered \textsc{Position}, at which the trial appeared in the experiment, as well as of all two-way interactions between these three variables.
The random effects structure consisted of random intercepts for both items and subjects and of by-item and by-subject random slopes for \textsc{Completeness}, \textsc{Topological Position}, \textsc{Position}, and the interaction between \textsc{Completeness} and \textsc{Topological Position}.%
%% Footnote
\footnote{The formula of the full model was as follows: \texttt{Ratings \textasciitilde ~(\textsc{Completeness} + \textsc{Topological Position} + \textsc{Position})\textasciicircum2 + (1 + \textsc{Completeness} * \textsc{Topological Position} + \textsc{Position} | Items) + (1 + \textsc{Completeness} * \textsc{Topological Position} + \textsc{Position} | Subjects)}.}
%
The final model had symmetric thresholds and contained significant main effects of all three independent variables and a significant interaction between \textsc{Completeness} and \textsc{Topological Position}, as presented in Table \ref{tab:pf.mf.model}.

The ratings for the full forms were significantly higher than for the utterances with topic drop ($\chi^2(1) = 13.06$, $p < 0.001$).
Utterances with the overt or covert constituent in the prefield were rated significantly better than utterances where this constituent was in the middle field ($\chi^2(1) = 23.7$, $p < 0.001$).
Additionally, utterances with an overt constituent in the middle field were particularly degraded ($\chi^2(1) = 28.18$, $p < 0.001$).
Overall, the ratings improved in the course of the experiment ($\chi^2(1) = 7.62$, $p < 0.01$).
Since the significant main effects of the three predictors are not of theoretical relevance, I do not discuss them further.%
%% Footnote
\footnote{Looking at Figure \ref{fig:pl.pf.mf}, it appears anyway that the main effects of \textsc{Completeness} and \textsc{Topological Position} are inconsistent, i.e., qualified by the interaction between \textsc{Completeness} and \textsc{Topological Position} (see \cite{crump.etal2019} for more details), which is the result of the deviation coding.}
%

\begin{table}
\caption{Fixed effects in the final CLMM of experiment \arabic{expcounter}}
\centering
\begin{tabular}{lrrrll}
\lsptoprule
Fixed effect & Est. & SE & $\chi^2$ & \textit{p}-value &   \\
\midrule
\textsc{Completeness} & $0.92$ & $0.25$ & $13.06$ & $< 0.001$ & ***\\
\textsc{Topological Position} & $1.37$ & $0.24$ & $23.70$ & $< 0.001$ & ***\\
\textsc{Position} & $0.27$ & $0.10$ & $7.62$ & $< 0.01$ & **\\
\Centerstack[l]{\textsc{Completeness} $\times$ \\ \textsc{Topological Position}} & $-2.46$ & $0.42$ & $28.18$ & $< 0.001$ & ***\\
\lspbottomrule
\end{tabular}
\label{tab:pf.mf.model}
\end{table}

\subsubsection{Discussion}
Experiment \arabic{expcounter} tested argument omissions in the prefield and the middle field.
The analysis revealed a significant interaction between \textsc{Completeness} and \textsc{Topological Position} indicating that omissions in the middle field were particularly degraded.
This result supports the assumption of a prefield restriction of topic drop.
While topic drop was acceptable in the preverbal position with ratings comparable to the full forms, a comparable argument omission was significantly degraded in the middle field.
This degradation is arguably the result of violating the positional restriction of topic drop.
It is worth noting, however, that the mean rating of the omission in a postverbal position was still at 4.94 on a 7-point scale and, thus, relatively good.
At first glance, this seems to question the validity of the conclusions drawn here, but there are at least two possible explanations for the relatively good ratings.
First, the ratings for the omission in a postverbal position may have been driven up by the extremely poor catch trials, which, as described above, contained severe word order violations that are much more striking than the omission of a postverbal pronoun.
These catch trials received a mean rating of 2.03 (SD = 1.59,) while the mean ratings for all other filler types were also around 5 points.
Second, it could be that in some trials some participants simply overlooked the missing pronoun and did not realize that the sentence contained an omission.
This seems plausible since previous research has established that readers skip around a third of all words during initial reading \citep{rayner1998} and that the tendency to skip a word is stronger if the word is shorter and more predictable \is{Predictability} \citep[e.g.,][]{rayner.mcconkie1976,brysbaert.etal2005,rayner.etal2011}.
Consequently, readers should generally be likely to skip a predictable \is{Predictability} pronoun like \textit{ich} (`I') anyway so that they may not realize that this pronoun is not present in the sentence.
This line of reasoning seems to be supported by the ratings of up to a dozen participants who, based on visual inspection of the figure \mbox{\textit{Exp2\_Barplot\_Participants.pdf}} 
in the online repository,%
%% Footnote
\footnote{The figure can be found at: \url{https://osf.io/zh7tr}.}
%
made no or only a little difference between the topic drop and the full form in the middle field.%
%% Footnote
\footnote{This hypothesis could be validated in further studies.
Cases of topic drop in the middle field could be compared to cases where other short words are omitted from the middle field and the resulting structures cannot be interpreted as topic drop, such as the omission of prepositions, determiners, \is{Article} or even short nouns after determiners.
If these clearly ungrammatical structures receive ratings comparable to those of topic drop in the middle field by the same participants, this might support the skipping hypothesis.
This skipping hypothesis could also be addressed more directly using a lexical decision task where participants are asked whether or not they just read the pronoun.
}
%
It appears that the characteristics of the fillers and the way participants read the items may have caused the absolute ratings for the omission in the middle field to be higher than would have been expected.
Nevertheless, they are clearly worse than those for the omission in the prefield, which argues for the prefield restriction of topic drop. \is{Acceptability rating study|)} \is{Argument omission|)}
\is{Middle field|)}

\section{Is topic drop possible in (potentially) embedded clauses?}\label{sec:highest} \is{Embedding|(}
In what follows, I refine the prefield restriction of topic drop by considering two further aspects.
In this section, I investigate whether topic drop can occur in potentially embedded clauses (experiment \ref*{exp:embedded}), while in \sectref{sec:initial} I discuss whether the prefield restriction can be understood as a restriction to a sentence-initial position.

\subsection{Theoretical background}\label{sec:highest.theory}
There is dissent in the literature about whether topic drop is possible in embedded clauses, namely,
in embedded V2 clauses that exhibit a prefield like main clauses.%
%% Footnote
\footnote{There is an ongoing debate in the literature about the status and the syntactic position of these V2 clauses, to which I briefly turn below.}
%
\citet[76]{cardinaletti1990}, \citet[169]{rizzi1994}, and \citet[272]{volodina2011} claim that topic drop is not possible in clauses like \ref{ex:embedded.start}, which stems from the items of experiment \ref*{exp:embedded}.
In contrast, \citet[101]{jaensch2005} takes the view that topic drop is possible in embedded clauses, although its acceptability is said to vary stronger between speakers than for unembedded topic drop.
Similarly, \citet[224--225]{trutkowski2016} considers embedded topic drop to be possible but states that it is subject to stronger identity conditions.

\exg.?\label{ex:embedded.start}Am Freitag hat er mir gebeichtet, $\Delta$ hat seine neue Freundin betrogen.\\
on Friday has he me confessed he has his new girlfriend cheated\\
`On Friday he confessed to me, (he) cheated on his new girlfriend.'

If topic drop was not possible in a potentially embedded clause, this would require modifying the prefield restriction by stating that topic drop is not possible in every prefield position but only in a ``special'' one.
\citet{rizzi1994} and \citet{freywald2020} propose two different approaches on how to specify this special position.
Based on an alleged asymmetry between subject and object topic drop postulated by \citet{cardinaletti1990},
\citet{rizzi1994} analyzes cases of subject topic drop in \ili{German} as so-called ``root null subjects'', \is{Null subject} while treating object topic drop separately as an ``empty category bound by a discourse-identified null operator'' \is{Operator analysis} \citep[160]{rizzi1994} (see also \sectref{sec:syntax}).
For the subjects, on which I focus here, he assumes an analysis as a null constant without an operator that is restricted to ``the specifier of the root, the highest position of the structure, the position that c-commands \is{C-command} everything and is not c-commanded \is{C-command} by anything'' \citep[162]{rizzi1994}.% 
%% Footnote
\footnote{It seems that the c-command \is{C-command} relation plays a role in other types of null elements as well.
For instance, \citet{stowell1991, stowell1996} observes an asymmetry for null articles in headlines. \is{Article omission}
A DP \is{Determiner phrase} cannot have a null article if it is c-commanded \is{C-command|(} by another DP that has an overt article, as illustrated by the contrast in \ref{ex:stowell}.

%\vspace{-0.5\baselineskip}
\ex.\label{ex:stowell}
\a.Man bites a dog.
\b.*A man bites dog.

%\vspace{-0.5\baselineskip}
\citet{reich2017} explains the licensing of null articles \is{Article omission} (and also of null copulas) \is{Copula omission} with a strategy of discourse-orientation (in the sense of what \cite{huang1984} has proposed for languages that allow for null topics).
He argues that discourse-orientation as a strategy is available in headlines but not in the standard registers of \ili{German} or \ili{English}, which are said to follow a strategy of sentence-orientation. 
See also the discussion of \citeg{reich2017} proposal in \sectref{sec:syntax}.}
%
In this position, the null constant is exempted from a relaxed version of the empty category principle \citep[in the sense of][]{rizzi1990} -- according to which an empty category must be chain-connected to an antecedent, \is{Antecedent} if it can -- and is therefore available for ``discourse identification'', i.e., to ``receive its referential value in discourse'' \citep[162]{rizzi1994}.%
%% Footnote
\footnote{In \citeg{trutkowski2016} interpretation, this means ``that the prefield guarantees optimal access to the discourse, which provides the antecedent \is{Antecedent} for the gap'' \citep[19]{trutkowski2016}.
She adds that this is particularly important for topic drop because, unlike other ellipsis types, it allows for semantic and syntactic mismatches between antecedent \is{Antecedent} and target.
Therefore, in her view, a close connection to the antecedent \is{Antecedent} is of greater importance.
Note, however, that other types of ellipses also allow for mismatches but are not positionally constrained.
For instance, sluicing \is{Sluicing} \citep[e.g.,][]{ross1969,merchant2001} allows for mismatches in finiteness, tense, modality, and polarity, among others \citep[see, e.g.,][]{kroll.rudin2017,anand.etal2021}.
See also the discussions of the prefield restriction in several syntactic approaches in \sectref{sec:syntax}.}
\largerpage[2]
\citet{rizzi1994} assumes that in the unmarked case, the root category is the CP \is{Complementizer phrase|(} so that subject topic drop occurs in [Spec, CP].
However, he argues that the alleged ban on topic drop in embedded clauses shows that topic drop cannot occur in every [Spec, CP] position but only in the \textit{specifier of the root}:
``A null element can be discourse-identified only if it is not c-commanded sentence-internally by a potential identifier'' \citep[169]{rizzi1994}.
In sum, \citet{rizzi1994} predicts that topic drop should not be possible in every prefield but only in the prefield of root clauses, a position that is not c-commanded \is{C-command|)} clause-internally by a potential identifier.%
%% Footnote
\footnote{\label{note:ackema}Apparently independently of \citet{rizzi1994}, \citet{ackema.neeleman2007} develop a similar approach.
To account for the distribution of \textit{pro}-drop \is{@\emph{pro}-drop} in Early Modern Dutch, \il{Dutch} they add a structural condition to the factor of locality in \citeg{ariel1990} list of factors determining accessibility (see \sectref{sec:recover.given} for a description of her concept of an accessibility hierarchy).
Accordingly, an antecedent \is{Antecedent} is (more) highly accessible if it ``is part of the same finite CP \is{Complementizer phrase} as the anaphoric expression'' \citep[97]{ackema.neeleman2007}.
To avoid that this condition excludes the possibility of topic drop, where the positioning in [Spec, CP] rules out that it finds an antecedent \is{Antecedent} within the CP \is{Complementizer phrase|)} in which it is contained, they exempt null pronouns at the left edge from the structural condition, considering the left edge an ``\,`escape hatch' for anaphoric dependencies'' \citep[99]{ackema.neeleman2007}.}

\citet{freywald2020} proposes to extend the topic drop analysis to further phenomena that do not occur in the prefield, e.g., object omissions \is{Object omission} in directive infinitives.%
%% Footnote
\footnote{\citet{freywald2020} argues that object omissions \is{Object omission} in directive infinitives such as \ref{ex:object.omission} are not a syntactic phenomenon in their own right but can be subsumed under topic drop which she, unlike me, understands as the omission of topics.
According to her, treating object omission as topic drop is justified because the omitted objects are continuous topics, they can refer to an extralinguistically present entity, and there is evidence of a topic position at the left periphery in directive infinitives.

%\vspace{-0.5\baselineskip}
\exg.\label{ex:object.omission}Nach dem Öffnen $\Delta$ kühl lagern (6° bis 8°C) und $\Delta$ innerhalb weniger Tage verbrauchen.\\
after the opening it cool store.\textsc{inf} 6° to 8°C and it within few days consume.\textsc{inf}\\
`Once opened, store (it) in a cool place (6° to 8°C) and consume (it) within a few days.' \citep[148]{freywald2020}
\vspace{-0.75\baselineskip}
}
To do so, she refers to \citet{rizzi1994} but reinterprets the \textit{specifier of the root}-account by stating ``that topic drop can only occur in the top-most position of an autonomous sentence''%
%% Footnote
\footnote{My translation, the original: ``dass Topikdrop nur in der höchsten Position eines selbstständigen Satzes stattfinden kann'' \citep[167]{freywald2020}.}
%
\citep[167]{freywald2020}.
Thereby, she equates root clauses and autonomous%
%% Footnote
\footnote{I use the term \textit{autonomous} to refer to what is called \textit{selbständig} in German, while \textit{dependent} is used as equivalent to the German \textit{abhängig}.
As \citet[541]{reich.reis2013} point out, every dependent clause is non-autonomous, but independent clauses can also be non-autonomous, such as the conjuncts of a coordination.}
%
clauses \citep[150]{freywald2020}, a view that is questioned by \citet{reich.reis2013}.
They point out that not only autonomous clauses%
%% Footnote
\footnote{They define syntactic autonomy in the following way:
``S ist genau dann syntaktisch selbständig, wenn der maximale Satzknoten von S von keinem (anderen) Knoten dominiert wird'' (`S is syntactically autonomous if and only if the maximum clause node of S is not dominated by any (other) node') \citep[541, my translation]{reich.reis2013}.}
%
are root clauses%
%% Footnote
\footnote{According to \citet{reich.reis2013}, a root clause is defined as follows:
``Ein Satz S ist genau dann ein Wurzelsatz, wenn S in keinen Satz S’ im Sinne von (I) integriert ist.'' (`A clause S is a root clause if and only if S is not integrated into a clause S' in the sense of (I).') \citep[542, my translation]{reich.reis2013}.
For integration in the sense (I), they provide this definition:
``S2 ist genau dann in S1 \textit{integriert} (eingebettet), wenn der minimale Satzknoten von S1 den maximalen Satzknoten von S2 dominiert'' (`S2 is \textit{integrated} (embedded) in S1 if and only if the minimum clause node of S1 dominates the maximum clause node of S2.') \citep[537, my translation]{reich.reis2013}.
}
%
but also the conjuncts of an autonomous coordination or continuative relative clauses (see, e.g., \cite{blumel.etal2017} for this clause type in German), where the maximal sentence node is dominated by another node, but no integration relation exists between the two clauses.
When describing \citeg{freywald2020}  approach in the following, I preliminarily adopt her equation of root clause and autonomous clause.
In sum, \citet{freywald2020} predicts topic drop not to be possible in every prefield position but only in the highest syntactic position of an autonomous or root clause.
In principle, \citeg{rizzi1994} and \citeg{freywald2020} approaches are very similar and only make a different prediction for the special case of initial V2 clauses if they are in fact interpreted as embedded (see below).
In this book, it is neither possible nor my goal to distinguish between the two approaches.
I remain agnostic here and discuss my predictions and results in terms of both approaches.

Before I turn to these predictions, it is necessary to briefly sketch the ongoing debate about the status and syntactic position of presumably embedded V2 clauses, which can appear both in sentence-final \ref{ex:embedded.final.ill} and in sentence-initial position \ref{ex:embedded.initial.ill}.

\ex.
\ag.\label{ex:embedded.final.ill}Am Freitag hat er mir gebeichtet, $\Delta$ hat seine neue Freundin betrogen.\\
on Friday has he me confessed he has his new girlfriend cheated\\
`On Friday he confessed to me, (he) cheated on his new girlfriend.'
\bg.\label{ex:embedded.initial.ill}$\Delta$ hat seine neue Freundin betrogen, hat er mir am Freitag gebeichtet.\\
he has his new girlfriend cheated has he me on Friday confessed\\
`(He) cheated on his new girlfriend he confessed to me on Friday.'

Superficially, at least the full forms, i.e., the syntactically complete utterances, seem to be parallel to the corresponding embedded verb-last clauses with an overt complementizer \is{Complementizer} in \ref{ex:embedded.dass}.
Therefore, it is reasonable to assume that the V2 just like the verb-last structures appear in the prefield or postfield position of the matrix clause and are syntactically embedded.

\ex.\label{ex:embedded.dass}
\ag.\label{ex:embedded.final.dass}Am Freitag hat er mir gebeichtet, dass er seine neue Freundin betrogen hat.\\
on Friday has he me confessed that he his new girlfriend cheated has\\
`On Friday he confessed to me that he cheated on his new girlfriend.'
\bg.\label{ex:embedded.initial.dass}Dass er seine neue Freundin betrogen hat, hat er mir am Freitag gebeichtet.\\
that he his new girlfriend cheated has has he me on Friday confessed\\
`That he cheated on his new girlfriend he confessed to me on Friday.'

This seems to be the view of \citet{cardinaletti1990}, \citet{rizzi1994}, and \citet{volodina2011} because they, when arguing against the possibility of topic drop in embedded clauses, refer to them as such, i.e., as embedded, and do not discuss other options.
Similarly, \citet[253--254]{grewendorf1988} assumes the possibility of having V2 complement clauses \is{Complement clause} in the prefield.
\citet[138]{reis1997} notes that, at least at the time of her 1997 paper, the default assumption is that these kinds of V2 sentences are syntactic complements in the strict sense.
With this paper, however, \citet{reis1997} argued against this assumption and largely changed the prevailing opinion.

She terms final V2 clauses with argument \is{Argument} function \textit{syntactically relatively unintegrated dependent clauses}%
%% Footnote
\footnote{My translation, the original: ``syntaktisch `relativ unintegrierte Nebensätze'\,'' \citep[121]{reis1997}.}
%
\citep[121]{reis1997} and assumes, based on several diagnostics, that they are adjoined to the VP \is{Verb phrase} (\cite[138]{reis1997}, see also \cite{truckenbrodt2006}).%
%% Footnote
\footnote{\citet{meinunger2004, meinunger2006} proposes an adjunct \is{Adjunct} position next to the CP \is{Complementizer phrase} including so-called \textit{double access reading}, where the V2 clause receives theta-roles in the base position as the complement of VP \is{Verb phrase} and the licensing of binding \is{Binding theory} relations is permitted \citep[481]{meinunger2006}.}
%
\citet[139]{reis1997} furthermore claims that relatively unintegrated V2 clauses cannot be placed in the prefield \citep[see also][]{frank2000, truckenbrodt2006, freywald2009, freywald2013}.%
%% Footnote
\footnote{In more recent work, \citet[342, footnote 16]{freywald2016} argues that speaker judgments regarding what she calls the ``Vorfeldphobie'' (`prefield phobia') of V2 clauses are not very robust and that in particular, the ratings for topical V2 clauses vary widely.
However, the results of experiment \ref*{exp:embedded} provide no evidence for this statement.
While the mean ratings for utterances with initial embedding are generally slightly but significantly worse than for unembedded clauses, the standard deviations are comparable.
Participants appear to be as confident in their judgments of utterances with initial embedded V2 clauses as they are for utterances without embedding (whereas there appears to be more variation for utterances with final embedded clauses, where the standard deviations are larger).
See \sectref{sec:exp.embedded.results} for details.
}
%
For apparent counterexamples, she proposes two alternative analyses, one as parenthetical structures and one as quasi-paratactic structures.

In her view, example \ref{ex:v1.parenthetical} is an autonomous V2 clause with a final integrated V1-parenthetical.
\textit{Er kommt} is considered to be the matrix clause in which the V1 clause \textit{glaube ich} is syntactically integrated \citep[139, footnote 23]{reis1997}.

\ex.
\ag.\label{ex:v1.parenthetical}Er kommt, glaube ich.\\
he comes believe I\\
`He's coming, I think.' \citep[139, footnote 23]{reis1997}

\citet{reis1996} lists the following defining properties of V1 integrated parentheticals:
\begin{quote}
\begin{enumerate}
\item[(i)] Verb-first
\item[(ii)] Interpretational integration into the host clause
\item[(iii)] Prosodic integration into the host clause \is{Prosody}
\begin{enumerate}
\item[a)] no focus-background-structure of their own \is{Focus}
\item[b)] no stress/focus \is{Focus}
\item[c)] no intonational breaks (i.e. no `comma intonation') \citep[48]{reis1996}
\end{enumerate}
\end{enumerate}
\end{quote}

\noindent
For longer examples like the item of experiment \ref*{exp:embedded} in \ref{ex:embedded.initial.item} (see \sectref{sec:exp.emb.materials} for details on the materials), an analysis as an autonomous V2 clause with integrated V1 parenthetical seems unlikely (see also \cite[149--151]{pauly2013} for a critical discussion of Reis's arguments for a parenthetical reading based on binding data). \is{Binding theory}

\exg.\label{ex:embedded.initial.item}B: (Er) hat seine neue Freundin betrogen, hat er mir am Freitag gebeichtet\\
{} he has his new girlfriend cheated has he me on Friday confessed\\
B: `(He) cheated on his new girlfriend, he confessed to me on Friday'

The assumed V1 parenthetical \textit{hat er mir am Freitag gebeichtet} is longer than a typical V1 parenthetical  and \textit{beichten} (`to confess') is less typical as a verb compared to classic ``verbs of saying, thinking, and believing and verbs of asking''  like \textit{glauben} (`to believe'), \textit{fragen} (`to ask'), \textit{denken} (`to think'), \textit{meinen} (`to think'), \textit{sagen} (`to say'), and \textit{wissen wollen} (`want to know') \citep[56]{steinbach2007}.%
%% Footnote
\footnote{Note that \citet[194, footnote 20]{rapp2015} considers the very similar verb \textit{gestehen} (`to confess'/`to admit') to be a factive verb that only allows for V2 embeddings in conjunctive mood \ref{ex:gestehen.conj}.

%\vspace{-0.5\baselineskip}
\exg.\label{ex:gestehen.conj}Sie gesteht, sie könne/*kann diesen Mann nicht einschätzen.\\
she confesses she can.\textsc{conj}/can.\textsc{ind} this man not estimate\\
`She confesses she cannot evaluate this man.' \citep[194, footnote 20, her judgment]{rapp2015}

%\vspace{-0.5\baselineskip}
In the Deutsches Referenzkorpus DeReKo (`German reference corpus') \is{Corpus}(\cite{dereko2022}, see also \cite{kupietz.etal2010}) there are, however, isolated examples of \textit{gestehen} with V2 complement clauses \is{Complement clause} in indicative mood, such as \ref{ex:gestehen.dereko}.
Thus, at least this verb should generally allow for embedded V2 clauses, regardless of the mood.

%\vspace{-0.5\baselineskip}
\exg.\label{ex:gestehen.dereko}In der ``Bunten'' gesteht er, er mag den harten Rap des Amerikaners Eminem [...].\\
in the Bunte confesses he he likes.\textsc{ind} the hard rap the.\textsc{gen} American Eminem \\
`In the `Bunte' (tabloid, L.S.) he confesses that he likes the hard rap of the American Eminem [...].' [DeReKo, HMP11/APR.02340, Hamburger Morgenpost, 04/28/2011, p. 02]
%\vspace{-2em}}
}
Additionally, despite a close interpretational relation to the V2 clause, the V1 clause provides additional information in the form of the temporal adverbial \textit{am Freitag}.
In sum, there are good reasons not to treat examples like \ref{ex:v1.parenthetical} as autonomous V2 clauses with V1 parentheticals (though I come back to this option in the discussion in \sectref{sec:exp.embedded.discussion}).%
%% Footnote
\footnote{To exclude the possibility of a reading as V1 parenthetical it might be an option to test the respective structures with and without an intonational break between the two clauses.
Given \citeg{reis1996} defining properties, the variants with intonational break should not be interpretable as V1 parentheticals.
Nevertheless, they are not automatically ``true'' complement clauses \is{Complement clause} but could still be interpreted as unintegrated V2 parentheticals, as discussed in the following.
}

The second alternative to the analysis as a syntactically embedded V2 clause is assuming a quasi-paratactic structure.
The clause \textit{hat er mir am Freitag gebeichtet} in \ref{ex:embedded.initial.ill}, repeated here as \ref{ex:embedded.initial.ill.rep}, would then have to be interpreted as a V2 clause with topic drop \ref{ex:parenthetical.td}, as proposed in \citet{reis1995,reis1996} for unintegrated parentheticals, similar to \citeauthor{reis1995}'s (\citeyear[66]{reis1995}) example in \ref{ex:parathetical.reis}.

\ex.
\ag.\label{ex:embedded.initial.ill.rep}$\Delta$ hat seine neue Freundin betrogen, hat er mir am Freitag gebeichtet.\\
he has his new girlfriend cheated has he me on Friday confessed\\
`(He) cheated on his new girlfriend he confessed to me on Friday.'
\b.\label{ex:parenthetical.td}(Er) hat seine neue Freundin betrogen, (das) hat er mir am Freitag gebeichtet.

\exg.\label{ex:parathetical.reis}Hans -- (so/das) glaubt FRITZ -- wird KOMMen.\\
Hans {} so/that believes Fritz {} will come\\
`Hans -- (so) Fritz believes -- will come.' \citep[66]{reis1995}

The corresponding full form would have a demonstrative \textit{das} (`that') in the prefield referring back to the preceding clause \ref{ex:parenthetical.td}.
The other V2 clause \textit{Er hat seine neue Freundin betrogen} would not be in the prefield of the \textit{gebeichtet}-clause, but both clauses would be considered root clauses and form together a quasi-paratactic structure.

In sum, we end up with two proposals on how to analyze the final V2 clauses and with three for the initial V2 clauses.
In the following, I briefly sketch how \citeg{rizzi1994} and how \citeg{freywald2020} refinement of the prefield restriction account for these proposals.
For both analyses of the final V2 clauses, either as embedded or as an adjunct \is{Adjunct} to the VP, \is{Verb phrase} \citeg{rizzi1994} and \citeg{freywald2020} theoretical accounts predict topic drop to be impossible because the [Spec, CP] of the final V2 clause is not the highest [Spec, CP] of the autonomous or root clause and it is c-commanded \is{C-command|(} sentence-internally by a potential identifier, namely by the subject.
For the initial V2 clauses, there are the options to analyze them as parenthetical structures with a V1 and a V2 clause, or as quasi-paratactic structures with two independent V2 clauses.
Under both analyses, \citet{rizzi1994} and \citet{freywald2020} would predict that topic drop should be possible since the corresponding prefield would be the highest of an autonomous or root clause and it would not be c-commanded \is{C-command} sentence-internally by a potential identifier.
Only if the initial V2 sentences were indeed analyzed as embedded, and if we assumed a similar structure to \ref{ex:tree.embedded.initial} for \ref{ex:embedded.initial.r}, would the predictions of the two approaches differ.

\ex.
\ag.\label{ex:embedded.initial.r}(Er) kündigt, erzählte Tim.\\
he quits told Tim\\
`(He) quits, Tim told.'
\b.\label{ex:tree.embedded.initial}
%\hspace{-3.5em}
\begin{forest}
for tree={parent anchor=south, child anchor=north, s sep = 0.1cm}
[CP, baseline,  
	[CP
		[DP	, rectangle, draw, red, 
			[(Er), roof, tier=word]
		]
		[C'
			[C\textsuperscript{0}
				[kündigt, tier=word]
			]
			[IP
				[...\phantom{T}, roof, tier=word]
			]
		]
	]	
	[C'
		[C\textsuperscript{0}
			[erzählte, tier=word]		
		]
		[IP
			[DP
				[Tim, roof, tier = word]
			]
			[I'
				[..., roof, tier=word]
			]
		]
	]	
]
\end{forest}

\noindent
While, according to \citet{freywald2020}, topic drop should be impossible because the [Spec, CP] with \textit{er} is not the highest of the autonomous or root clause, it should be allowed according to \citet{rizzi1994} since the \textit{er} is not c-commanded \is{C-command|)} sentence-internally by a potential identifier.

While a distinction between the three analyses proposed for the initial V2 cases would be desirable in principle, it must be left to future research, just like a distinction between \citeg{freywald2020} and \citeg{rizzi1994} accounts.
In the following experiment, I investigated more generally whether topic drop is possible at all in the potentially embedded initial or final V2 sentences.
This is my main question here, as it allows for a refinement of the prefield restriction under investigation.

\refstepcounter{expcounter}\label{exp:embedded}
\subsection{Experiment \arabic{expcounter}: topic drop in (potentially) embedded clauses }
\is{Acceptability rating study|(}\label{sec:exp.embedded}
In experiment \arabic{expcounter}, I tested potentially embedded V2 clauses with topic drop and compared them to a simple baseline with the clause in isolation and to the corresponding full forms.%
% Footnote
\footnote{All experimental items, fillers, and the analysis script are available online at: \url{https://osf.io/zh7tr}.}
%
Moreover, I varied whether the potentially embedded V2 clause occurred sentence-initially or sentence-finally.
The acceptability rating study had the form of a 2 $\times$ 3 design with the two factors of \textsc{Completeness} (full form vs. topic drop) and \textsc{V2 Clause} (initial vs. final vs. baseline).
Table \ref{tab:predictions.embedding} summarizes the predictions for the analyses and approaches discussed in \sectref{sec:highest.theory}.

\begin{table}
\caption[Overview of the predictions for experiment \arabic{expcounter}]{Overview of the predictions for experiment \arabic{expcounter}: a checkmark indicates that topic drop is predicted to be possible.}
\centering
\begin{tabular}{lcccc}
\lsptoprule
& Final V2 & \Centerstack{Initial V2\\embedded} & \Centerstack{Initial V2\\parenthetical} & \Centerstack{Initial V2\\paratactic}\\
\midrule
\Centerstack[l]{General prefield\\restriction} & \ding{51} & \ding{51} & \ding{51} & \ding{51} \\
\citet{rizzi1994} & \ding{55} & \ding{51} & \ding{51} & \ding{51}\\
\citet{freywald2020} & \ding{55} & \ding{55} & \ding{51} & \ding{51} \\
\lspbottomrule
\end{tabular}
\label{tab:predictions.embedding}
\end{table}

\noindent
If there is a general prefield restriction, it should not matter whether the clause is potentially embedded or not, i.e., the acceptability of topic drop should be comparable in all three \textsc{V2 Clause} conditions because the ellipsis always occurs in a prefield position.
If \citeg{freywald2020} proposal of a restriction of topic drop to the highest [Spec, CP] of an autonomous or root clause is correct, topic drop in the final condition should be degraded compared to the baseline topic drop condition.
For the initial condition, topic drop should be possible for \citet{freywald2020} if the V2 clause is analyzed as autonomous, combined either with a V1 parenthetical or with a further V2 clause with topic drop but not if analyzed as an embedded clause.
According to \citeg{rizzi1994} approach, topic drop should be excluded in the final (possibly embedded) V2 clauses, while it should generally be possible in the initial V2 clauses, regardless of the analysis adopted.
Although I present the predictions of \citeg{rizzi1994} and \citeg{freywald2020} approaches here, my main goal is not to distinguish between them but to answer the question of whether topic drop is possible in any prefield.
Nevertheless, it seems reasonable to be able to discuss the implications of the results for these two proposals, as it may be the starting point for further research.

\largerpage[2]
Note that experiment \ref*{exp:pf.mf} and experiment \arabic{expcounter} were part of the same study.
As a consequence, the items served reciprocally as fillers (along with the additional fillers, see \sectref{sec:exp.emb.materials}).
Therefore, I only outline the main points of the experimental procedure, as the details were already sketched in \sectref{sec:exp.prefield}.\clearpage

\subsubsection{Materials}\label{sec:exp.emb.materials}
\subsubsubsection*{Items}
I constructed 24 items in the form of question-answer pairs, such as \ref{ex:item.embedded} .
The question asked for news about a third person X, either in the form \textit{Was gibt's Neues von X?} (`What's new from X?') or \textit{Wie ist die Lage bei X?} (`What is the situation with X?').
The answer provided the requested information in the form of a V2 clause with an omitted or realized subject in the prefield.
This clause was either presented on its own \ref{ex:item.embedded.no}, serving as a baseline, or in the initial \ref{ex:item.embedded.initial} or final \ref{ex:item.embedded.final} position relative to another clause, i.e., in a potentially embedded position.%
%% Footnote
\footnote{For a discussion of the question of whether the V2 clauses should be considered embedded or not, see \sectref{sec:highest.theory}.}
%
I only used the verbs \textit{erzählen} (`to tell'), \textit{beichten} (`to confess'), and \textit{gestehen} (`to admit') (as opposed to typical V1 parenthetical verbs like \textit{glauben} (`to believe'/`to think')), to make a parenthetical reading of the matrix clause less likely (see \sectref{sec:highest.theory}).
In addition, I expanded the clause with additional information in the form of temporal adverbs.
Both the verbs in the matrix clauses and the verbs in the embedded clauses were presented in perfect tense.

\exg.\label{ex:item.embedded}A: Was gibt's Neues von Tim?\\
{} what gives.it new from Tim\\
A: `What's new from Tim?'
\ag.\label{ex:item.embedded.no}B: (Er) hat seine neue Freundin betrogen\\
{} he has his new girlfriend cheated \\
B: `(He) cheated on his new girlfriend' \\\phantom{.}\hfill (topic drop / full form, baseline)
\bg.\label{ex:item.embedded.initial}B: (Er) hat seine neue Freundin betrogen, hat er mir am Freitag gebeichtet\\
{} he has his new girlfriend cheated has he me on Friday confessed\\
B: `(He) cheated on his new girlfriend, he confessed to me on Friday' \\\phantom{.}\hfill (topic drop / full form, initial)
\bg.\label{ex:item.embedded.final}B: Am Freitag hat er mir gebeichtet, (er) hat seine neue Freundin betrogen\\
{} on Friday has he me confessed he has his new girlfriend cheated\\
B: `On Friday he confessed to me, (he) cheated on his new girlfriend' \\\phantom{.}\hfill (topic drop / full form, final)

%\vspace{-0.5\baselineskip}

\subsubsubsection*{Fillers}
As described in \sectref{sec:exp.pf.mf.materials}, 72 fillers were included in the experiment:
24 (potential) gappings, 16 PP fragments, 24 clauses with overt or covert constituents in the prefield or the middle field, and eight catch trials with word order violations.

\subsubsection{Procedure}
The procedure was described in detail in \sectref{sec:exp.pf.mf.procedure}.
As stated above, the materials were presented as WhatsApp dialogues using the design shown in Figure \ref{fig:wa.design} on page \pageref{fig:wa.design} and partly contained emojis to increase the naturalness.

\subsubsection{Results}\label{sec:exp.embedded.results}
As described in \sectref{sec:exp.pf.mf.results}, I excluded three participants who did not pass the attention checks by rating four or more of the ungrammatical fillers with 6 or 7.
For the remaining 45 participants, Table \ref{tab:descriptives.embedded} shows the mean ratings and standard deviations per condition, and Figure \ref{fig:pl.embedded} the mean ratings and 95\% confidence intervals.
The visual inspection indicates that there was a three-part degradation independently of \textsc{Completeness}.
The baseline conditions were more acceptable than the initial conditions and the initial conditions, in turn, were more acceptable than the final conditions.
It is striking that topic drop in the final condition was significantly worse, with a mean score of only 3.93.

\begin{table}
\caption{Mean ratings and standard deviations per condition for experiment \arabic{expcounter}}
\centering
\begin{tabular}{llrr}
\lsptoprule
\textsc{Completeness} & \textsc{V2 Clause} & \Centerstack{Mean\\rating} & \Centerstack{Standard\\deviation} \\
\midrule
Full form & Baseline &  $5.62$ & $1.42$ \\
Topic drop & Baseline & $5.40$ & $1.53$ \\
Full form & Initial & $5.31$ & $1.44$ \\
Topic drop & Initial &  $5.13$ & $1.58$\\
Full form & Final &  $4.92$ & $1.62$ \\
Topic drop & Final & $3.93$ & $1.80$ \\
\lspbottomrule
\end{tabular}
\label{tab:descriptives.embedded}
\end{table}

\begin{figure}
\centering
\includegraphics[scale=1]{Experimenteplots/PL_Emb.pdf}
\caption{Mean ratings and 95\% confidence intervals per condition for experiment \arabic{expcounter}}
\label{fig:pl.embedded} % pl for point line
\end{figure}

The rating data were analyzed with CLMMs from the package ordinal \citep{christensen2019} in R, generally following the procedure described in \sectref{sec:data.analysis}.
For the three-level predictor \textsc{V2 Clause}, however, I used forward coding to compare every level to the subsequent one(s) using two contrasts:
\textsc{Baseline.Other} compares the level baseline, coded as $\sfrac{2}{3}$, to the other two levels, coded as $-\sfrac{1}{3}$.
\textsc{Other.Final} compares the levels baseline and initial, coded as $\sfrac{1}{3}$, to the level final, coded as $-\sfrac{2}{3}$.
Besides these two predictors that represent \textsc{V2 Clause}, I also included the deviation-coded predictor \textsc{Completeness} (full form coded as $0.5$, topic drop coded as $-0.5$), the centered and scaled \textsc{Position} of the trial in the experiment, as well as the two-way interactions between the three predictors \textsc{Completeness}, \textsc{Baseline.Other}, and \textsc{Position}, and the two-way interactions between \textsc{Completeness}, \textsc{Other.Final}, and \textsc{Position}.
The random effects structure consisted of random intercepts for subjects and items and of by-item and by-subject random slopes for the predictors \textsc{Completeness}, \textsc{Baseline.Other}, \textsc{Other.Final}, \textsc{Position}, and the two-way interactions of \textsc{Completeness} with \textsc{Baseline.Other} and \textsc{Other.Final}.%
%% Footnote
\footnote{The formula of the full model was as follows: \texttt{Ratings \textasciitilde ~\textsc{Completeness} + \textsc{Baseline.Other} + \textsc{Other.Final} + \textsc{Position} + \textsc{Completeness} : \textsc{Baseline.Other} + \textsc{Completeness} :  \textsc{Other.Final} + \textsc{Position} : \textsc{Baseline.Other} + \textsc{Position} : \textsc{Other.Final} + (1 + (\textsc{Baseline.Other} + \textsc{Other.Final}) * \textsc{Completeness} + \textsc{Position} | Items) + (1 + (\textsc{Baseline.Other} + \textsc{Other.Final}) * \textsc{Completeness} + \textsc{Position}  | Subjects)}.}
%
The fixed effects in the final model with symmetric thresholds are shown in Table \ref{tab:embedded.model}.

\begin{table}
\caption{Fixed effects in the final CLMM of experiment \arabic{expcounter}}
\centering
\begin{tabular}{lrrrll}
\lsptoprule
Fixed effect & Est. & SE & $\chi^2$ & \textit{p}-value &   \\
\midrule
\textsc{Completeness} & $0.76$ & $0.15$ & $20.57$ & $< 0.001$ & ***\\
\textsc{Baseline.Other} & $0.64$ & $0.26$ & $5.65$ & $< 0.05$ & *\\
\textsc{Other.Final} & $1.33$ & $0.22$ & $26.62$ & $< 0.001$ & ***\\
\textsc{Completeness $\times$ Other.Final} & $-1.25$ & $0.41$ & $8.96$ & $< 0.001$ & ***\\
\lspbottomrule
\end{tabular}
\label{tab:embedded.model}
\end{table}

\noindent
The final model contained a significant interaction between \textsc{Other.Final} and \textsc{Completeness} ($\chi^2(1) = 8.96$, $p < 0.001$) according to which topic drop in a (potentially embedded) final V2 clause is significantly degraded compared to topic drop in the initial V2 clause or the baseline condition.
The significant main effects of \textsc{Other.Final} ($\chi^2(1) = 26.62$, $p < 0.001$) and of \textsc{Baseline.Other} ($\chi^2(1) = 5.65$, $p < 0.05$) show a gradation for \textsc{V2 Clause} independently of \textsc{Completeness}.
Both full forms and utterances with topic drop were rated as more acceptable in the baseline condition than with an initial V2 clause.
Additionally, utterances with an initial V2 clause were in turn more acceptable than utterances with a final V2 clause (see Figure \ref{fig:pl.embedded}).
Finally, there was also a significant main effect of \textsc{Completeness} ($\chi^2(1) = 20.57$, $p < 0.001$), according to which utterances with topic drop received degraded ratings compared to full forms.

\subsubsection{Discussion}\label{sec:exp.embedded.discussion}
In this acceptability rating study, I investigated whether topic drop is possible in every prefield position by testing utterances with sentence-initial or sentence-final (potentially embedded) V2 clauses.
The results show that topic drop in a potentially embedded clause that is positioned sentence-finally was degraded, even though the constituent was omitted from a prefield position.
This suggests that the prefield positioning of the omitted constituent is a necessary but not a sufficient condition for topic drop.

Utterances with an initial V2 clause were degraded compared to the simple baseline, but unlike for the final V2 clauses, this was not restricted to topic drop but also held for the full forms.%
%
\footnote{This drop in acceptability observed for the initial condition independently of \textsc{Completeness} compared to the baseline condition could be caused by the more complex structures being less natural in instant messages than simple sentences.}
%
This means that topic drop in a (potentially embedded) V2 clause in sentence-initial position was not particularly degraded but can be considered to be possible.
This result can be explained in two ways:

a) If the initial conditions indeed contain embeddings, it could be concluded that topic drop in embedded clauses is as acceptable as the corresponding full form, as long as the embedding is initial.
This result would call into question \citeg{freywald2020} variant of the \textit{specifier of the root}-account since in this case topic drop would not occur in the highest [Spec, CP] of an autonomous or root clause but in a subordinate position.
However, it would not clash with \citeg{rizzi1994} variant because the [Spec, CP] at the left edge may not be the highest position, but it would be a position that is not c-commanded \is{C-command|(} sentence-internally by a potential identifier and should in principle allow for topic drop.

b) If we consider the possibility that the initial conditions do not contain embeddings but are either autonomous V2 clauses with integrated V1 parentheticals or V2 root clauses in a paratactic relation to a second V2 clause with topic drop, \citeg{freywald2020} variant would not be at stake either.
The [Spec, CP] of the autonomous or root V2 clause would be the highest [Spec, CP], from where topic drop should be possible.

These results do not allow for a decision between \citeg{rizzi1994} more liberal and \citeg{freywald2020} stricter variant of the \textit{specifier of the root}-account.
The important result is that topic drop is not just restricted to the prefield but only to a prefield that is at least not c-commanded \is{C-command} sentence-internally by a potential identifier or even  has to be the highest prefield of an autonomous or root clause.
Furthermore, both accounts explain why topic drop is not possible in the middle field \is{Middle field} because positions in the middle field of V2 clauses are necessarily c‑commanded sentence‑internally by a lexical category and are not the highest position of the autonomous or root clause in which they occur.%
%% Footnote
\footnote{The fact that topic drop does not seem to be possible in left dislocations \is{Left dislocation} (see \cite{volodina.onea2012} and the next section) suggests that it is not sufficient to restrict topic drop to a position that is not c-commanded sentence‑internally by a potential identifier.
The restriction to the prefield probably needs to be assumed independently.}  \is{Acceptability rating study|)}
\is{Embedding|)}

\section{Is topic drop restricted to the very first element?}\label{sec:initial}
The results of experiment \ref*{exp:embedded} on potentially embedded topic drop suggest that topic drop is not possible in every prefield position.
It is restricted to at least a prefield that is not c-commanded \is{C-command|)} sentence-internally by a potential identifier if not to the highest prefield of an autonomous or root clause.
In any case, the position is usually located at the left edge of an utterance.
\citet{trutkowski2016} argues that the restriction to this sentence-initial (\cite[547]{huang1984}, \cite[1]{trutkowski2016}), left-peripheral \citep[150]{freywald2020}, or left edge \citep[99]{ackema.neeleman2007} position is required to ensure that the antecedent \is{Antecedent} in the previous discourse is optimally accessible.
This brings up the question of whether the sentence-initial position is necessary, i.e., whether topic drop must also be the linearly first word in the string of words that form the utterance.

\subsection{Theoretical background}\label{sec:intial.theory}
Examples like \ref{ex:frac.preceding}, taken from the FraC \is{Corpus|(} (see \sectref{sec:corpus.frac}), suggest that topic drop does not generally need to be the very first element of an utterance.
The insulting address \textit{Vollidiot} (`complete idiot') \ref{ex:frac.preceding.vollidiot}, the particle \textit{ja} (`yes') \ref{ex:frac.preceding.ja}, the greeting \textit{hey-ho} \ref{ex:frac.preceding.heyho}, as well as the conjunctions \is{Conjunction} \textit{und} (`and') \ref{ex:frac.preceding.und}, \ref{ex:frac.preceding.und.asym} and \textit{aber} (`but') \ref{ex:frac.preceding.aber} all seem to be able to precede topic drop.%
%% Footnote
\footnote{This is also discussed by \citet[287]{sigurdsson2011}, who states that for topic drop ``high discourse particles and left-dislocated \is{Left dislocation} elements'' can precede topic drop.}
%

\ex.\label{ex:frac.preceding}
\ag.\label{ex:frac.preceding.vollidiot}Vollidiot $\Delta$ kannst mich mal!!! \\
complete.idiot you.\textsc{2sg} can me \textsc{part}\\
`Complete idiot, screw (you)!!!' [FraC S481]
\bg.\label{ex:frac.preceding.ja}ja, $\Delta$ wäre mir lieber.\\
yes that would.be me rather\\
`Yes, I would prefer (that).' [FraC D1416]
\cg.\label{ex:frac.preceding.heyho}HEY-HO, $\Delta$ WEIß NET, OB I NACH DER STRESSIGEN WOCHE SO LANGE DURCHHALTE,\\
hey-ho I know not if I after the stressful week so long last\\
`Hey ho, (I) don't know if I'll last that long after the stressful week' [FraC S1691]
\dg.\label{ex:frac.preceding.und}Wir hängen sehr an ihr. Und $\Delta$ mögen uns gar nicht vorstellen, wie es irgendwann ohne sie ist.\\
we hang very at her and we may us at.all not imagine how it sometime without her is\\
`We are very attached to her. And (we) don't even want to imagine what it will be like without her at some point.' [FraC B36--37]
\eg.\label{ex:frac.preceding.und.asym}Da habe ich nun ein schönes neues Tablet. Und $\Delta$ kann's nicht mit dem Wlan verbinden. \\
there have I now a pretty new tablet and I can.it not with the wifi connect \\
`Now I have a nice new tablet and can't connect it to the wifi.' [FraC B363]
\fg.\label{ex:frac.preceding.aber}ich habe ja jetzt schon {ferien [sic!]!} aber $\Delta$ fahre erst in {2Wochen [sic!]} weg!  \\
I have \textsc{part} now already vacations  but I go only in 2weeks away\\
`I am already on vacation now! But (I) am not leaving for another two weeks!' [FraC S417]

\is{Topological field model|(} \is{Corpus|)}
Apparently, topic drop can be preceded by elements that occupy a position to the left of the prefield, which is usually termed \textit{Vorvorfeld} (`pre-prefield') \is{Pre-prefield|(} or \textit{linkes Außenfeld} (`left outfield') (\cite[1577--1581]{zifonun.etal1997}, \cite{auer1997, gallmann2015, speyer.reich2020}) in the extended topological field model.%
%% Footnote
\footnote{\is{Topological field model}\citet{wollstein2014} distinguishes between a pre-prefield \is{Pre-prefield} and a left outfield while \citet{pasch.etal2003} propose the term \textit{Nullstelle} (`zero' or `root') because they assume that this position exists independently of a prefield position, for example, for V1 sentences as well.}
%
It is often divided into at least two subfields, a position for coordinating conjunctions \is{Conjunction} (COOR(D)) and one for so-called \textit{linksversetzte} (`left dislocated') \is{Left dislocation} constituents (e.g., \cite{speyer.reich2020, pittner.berman2021}; see also \citet[55]{pafel2011}, who calls the first \textit{Anschlussposition} (`connection position') and the second \textit{Topikfeld} (`topic field') and \citet[410]{eisenberg2020}, who equates left dislocation \is{Left dislocation} and pre-prefield). \is{Pre-prefield}\is{Topological field model|)}

\citet[1577]{zifonun.etal1997} further assume that left of the coordination position, relatively independent interactive units such as interjections, answer particles, and forms of address can be placed. 
In the Grammis system by the IDS, it is pointed out that these units do not contribute to the construction of sentences and tend to be the units with the loosest connection to the following sentence \citep{aussenfeld}.
Examples \ref{ex:frac.preceding.vollidiot}--\ref{ex:frac.preceding.heyho} fall into this category.
Given that they are frequently considered units on their own, the fact that they precede topic drop does not necessarily question a sentence-initial position of topic drop because these elements can be considered to be outside of the actual sentence.

This is less plausible as an explanation for left dislocations. \is{Left dislocation}
In \ref{ex:left.dislocation}, the dislocated DP in the pre-prefield  \is{Pre-prefield|)} \textit{den einen Schauspieler aus Breaking Bad} receives the accusative case from the verb \textit{mögen} in the clause, which suggests a syntactic integration of the preposed element.
However, \citet{volodina.onea2012} argue that topic drop is neither possible for left dislocated constituents because these constituents usually function as contrastive topics\is{Contrastive topic}  (see \sectref{sec:topicality.suff}), nor after left dislocations because the corresponding prefield constituent must be focal and focal constituents cannot be omitted \citep[230--231]{volodina.onea2012}.
The topic drop variant of \ref{ex:left.dislocation}, \ref{ex:left.dislocation.td}, indeed seems to be marked at least if not impossible.%
%% Footnote
\footnote{However, this is difficult to judge because, without a clear intonational pause, the dislocated constituent could also be understood as the prefield constituent of a full form without a pre-prefield.\is{Pre-prefield}}
%
Moreover, since there do not appear to be instances of topic drop after a left dislocation in either the FraC or in \citeg{frick2017} corpus, \is{Corpus} left dislocations also do not seem to challenge the sentence-initial positioning of topic drop either.%
%% Footnote
\footnote{The potential impossibility of topic drop after left-dislocated constituents could be explained in \citeg{rizzi1994} logic by assuming that the left-dislocated constituent is a possible identifier for the prefield constituent.
See also \citet[165--166]{frey2005}, who assumes, based on binding \is{Binding theory} and island effects, that the left-dislocated \is{Left dislocation} constituent and the resumptive pronoun, by which it is taken up again, together form one syntactic object and that the left-located constituent is in [Spec, CP], whereas the prefield is formed by [Spec, FinP].}

\ex.
\ag.\label{ex:left.dislocation}Den einen Schauspieler aus Breaking Bad, den mag ich total.\\
the.\textsc{acc} one.\textsc{acc} actor from Breaking Bad the.\textsc{acc} like I totally\\
`The one actor from Breaking Bad, I really like him.'
\b.?\label{ex:left.dislocation.td}Den einen Schauspieler aus Breaking Bad, $\Delta$ mag ich total.\is{Left dislocation}

What remains are the conjunctions \is{Conjunction} in the COORD position.
Examples \ref{ex:frac.preceding.und}--\ref{ex:frac.preceding.aber} show that they indeed occur before topic drop.
Following \citet{johannessen1998}, the phrase structure of coordinated structures can be modeled as in \ref{ex:tree.coord}, taken from \citet{reichtoappear}.
Johannessen assumes a conjunction \is{Conjunction} phrase \is{Conjunction phrase} \&P with the conjunction \& as the (functional) head.%
%% Footnote
\footnote{The fact that conjunctions \is{Conjunction} are functional rather than lexical heads becomes important below.
For a detailed justification of this assumption see \citet[96--105]{johannessen1998}.}
%
The conjunction \is{Conjunction} is not a direct part of either conjunct but syntactically closer to the second conjunct \citep{reichtoappear}.

\ex.\label{ex:tree.coord}
%\centering
\begin{forest}
for tree={s sep*=3, parent anchor=south, child anchor=north}
[\&P, baseline
	[S1
		[\phantom{SENT},roof]
	]
	[\&'
		[\&]
		[S2
			[\phantom{SENT},roof]	
		]
	]
]
\end{forest}\\
\phantom{.}\hfill\citep{reichtoappear}

\noindent
Nevertheless, it is not clear whether the conjunctions \is{Conjunction} in \ref{ex:frac.preceding.und}--\ref{ex:frac.preceding.aber} do function as conjunctions \is{Conjunction} linking two clauses or whether they should rather be considered discourse particles or discourse markers.
\citet[130]{proske2015} summarizes the properties of discourse markers in German as follows:
They are not independent, cannot form a clause or a turn on their own, and can but need not be realized as separate intonation units \citep[see also][]{auer1997}.
When discussing the research on discourse particles along various dichotomies, \citet{fischer2006} summarizes that a considerable number of researchers consider conjunctions \is{Conjunction} to be ``items that constitute parts of utterances'' \citep[8]{fischer2006}, i.e., they consider them to be syntactically integrated.
In contrast, \citet{gohl.gunthner1999} and \citet{imo2012.diskursmarker} argue that discourse markers are often not part of the following utterance and are only loosely syntactically connected to it.
In summary, there seems to be no consensus on whether conjunctions \is{Conjunction} are part of the subsequent utterance, regardless of whether they function as discourse markers or not.
However, they seem to be less independent than the interjections, forms of address, and greetings discussed above.
Therefore, they are the best candidate to test whether elements can precede topic drop.

Concerning the relationship between conjunctions \is{Conjunction} and topic drop, it is important to note that the structures in \ref{ex:frac.preceding.und}--\ref{ex:frac.preceding.aber} have clear parallels to subject gaps \is{Subject gap} in coordination, also called subject deletion, as has been pointed out by, for example, \citet{klein1993} and \citet{haegeman2007}.
Without the sentence boundary, \ref{ex:frac.preceding.und} and \ref{ex:frac.preceding.aber} could be analyzed as symmetric coordinations with a shared prefield (constituent), as in \ref{ex:frac.preceding.und.sg} and \ref{ex:frac.preceding.aber.sg} \citep[e.g.,][]{wilder1997}.%
%% Footnote
\footnote{See example \ref{ex:frac.preceding} above for the glossing.}
%
This is not possible for example \ref{ex:frac.preceding.und.asym}, though, which would have to be analyzed as an asymmetric coordination \ref{ex:frac.preceding.und.asym.sg}, more specifically, as a so-called \textit{SLF construction} for \textit{Subjektlücke in F-Sätzen} (`subject gaps \is{Subject gap} in F-clauses')%
%% Footnote
\footnote{F-clauses are clauses with a finite verb in the first or second position \citep[1]{hohle1983}.}
%
(\cite{hohle1983, hohle1990}; see also \cite{reich2007,reich2009,reich2009a,reich2013,bonitz2013,barnickel2017,oppermann2021}).
SLF coordination is characterized by the fact that in the non-initial conjuncts, the finite verb is fronted and no overt subject is present, while in the initial conjunct, the subject is located in the middle field \is{Middle field} so that an analysis in terms of a shared prefield is impossible \citep{reich2008,reich2009a}.

\ex.
\a.\label{ex:frac.preceding.und.sg}Wir hängen sehr an ihr und $\Delta$ mögen uns gar nicht vorstellen, wie es irgendwann ohne sie ist.
\b.\label{ex:frac.preceding.und.asym.sg} Da habe ich nun ein schönes neues Tablet und $\Delta$ kann's nicht mit dem Wlan verbinden.
\c.\label{ex:frac.preceding.aber.sg}ich habe ja jetzt schon ferien, aber $\Delta$ fahre erst in 2Wochen [sic!] weg!

It is important to note that not all conjunctions \is{Conjunction} allow (to an equal degree) for a subject gap \is{Subject gap} interpretation according to the literature.
Such an interpretation in both symmetric and asymmetric coordination is possible for the basic case \textit{und} (`and'), and also for \textit{oder} (`or'), \textit{sondern} (`but rather'), and \textit{aber} (`but') \citep{hohle1983, wunderlich1988}.
While \citet{hohle1983} argues that \textit{aber} is special because in asymmetric coordination it must not precede the second conjunct like \textit{und} does \ref{ex:aber.pos.ug} but follow the finite verb, as in \ref{ex:aber.pos.g}, \citet{wunderlich1988} does not mention a special status of \textit{aber} but presents example \ref{ex:aber.wunderlich}, where \textit{aber} precedes the second conjunct, as in \ref{ex:aber.pos.ug}.
As a mediating position, so to speak, \citet[504]{vandevelde1986} argues that the use of the subjunctive adverb \textit{aber} instead of the conjunction \is{Conjunction} in subject gap \is{Subject gap} constructions, while apparently preferred, is by no means obligatory.

\ex.
\ag.*\label{ex:aber.pos.ug}da standen ein paar Leute rum, aber $\Delta$ rührten keinen Finger\\
there stood a few people around but they moved no finger \\
`There were a few people standing around, but (they) didn't lift a finger'
\b.\label{ex:aber.pos.g}da standen ein paar Leute rum, $\Delta$ rührten aber keinen Finger \citep[13, his judgments]{hohle1983}

\exg.\label{ex:aber.wunderlich}In den Wald ging der Jäger, aber $\Delta$ fing nichts.\\
in the forest went the hunter but he caught nothing\\
`The hunter went into the forest, but (he) caught nothing.' \\(Lit. `Into the forest, the hunter went, but (he) caught nothing.') \citep[308, his judgment]{wunderlich1988}

A subject gap \is{Subject gap} interpretation is impossible for the empty conjunction \is{Conjunction} \citep[361]{reich2013}, the two-part conjunctions \textit{weder -- noch} (`(n)either -- (n)or') and \textit{sowohl -- als auch} (`both -- and') (\cite[14]{hohle1983}, \cite[310]{wunderlich1988}, \cite[361]{reich2013}), for the complex conjunctions \textit{entweder -- oder} (`either -- or') and \textit{einerseits -- andererseits} (`on the one hand -- on the other hand') (\cite[93]{reich2009}, \cite[361]{reich2013}), and for \textit{denn} (parordinating `because'),%
%% Footnote
\footnote{\label{note:denn}While, e.g., \citet[60]{zifonun.etal1997} and \citet[216]{eisenberg2020} classify \textit{denn} as coordinating conjunction, \is{Conjunction} the \textit{Handbuch der deutscher Konnektoren} (`Handbook of German connectives') \citep[584]{pasch.etal2003}, \citet[549]{reich.reis2013}, and \citet[16]{breindl2017} list it as ``syntaktischen Einzelgänger'' (`syntactic loner').
According to \citet[15]{breindl2017}, it deviates from the typical coordinating conjunctions \is{Conjunction} \textit{und} and \textit{oder} in not allowing for ellipsis in coordination \ref{ex:denn.ell}, nor for category variable internal conjuncts \ref{ex:denn.variable}, nor for embeddings \is{Embedding} \ref{ex:denn.emb} \citep[see also][549]{reich.reis2013}.
%\vspace{-0.5\baselineskip}
\ex.\label{ex:denn.ell}
\ag. Die Kinder sind heimgegangen \emph{und} die Eltern (sind heimgegangen).\\
the children are home.gone and the parents are home.gone\\
`The children have gone home and the parents (have gone home).'
\b.Die Kinder sind heimgegangen \emph{denn} die Eltern *(sind heimgegangen). \citep[15]{breindl2017}

%\vspace{-1\baselineskip}
\ex.\label{ex:denn.variable}
\ag.Wir laden Uwe \emph{und} Evi ein.\\
we invite Uwe and Evi \textsc{vpart}\\
`We invite Uwe and Evi.'
\b.*Wir laden Uwe \emph{denn} Evi ein. \citep[15]{breindl2017}

%\vspace{-1\baselineskip}
\ex.\label{ex:denn.emb}
\ag.\{Dass Otto krank ist \emph{und} heute nicht kam\}, hat alle überrascht.\\
that Otto sick is and today not came has all surprised\\
`That Otto is sick and did not come today surprised everyone.'
\b.*\{Dass Otto krank ist, \emph{denn} heute nicht kam\}, hat alle überrascht. \citep[15]{breindl2017}

%\vspace{-2em}
}
%
both in symmetric \ref{ex:denn.imp.sym} and asymmetric coordination \ref{ex:denn.imp.asym} (\cite[508]{vandevelde1986}, \cite[54]{bonitz2013}).

 \ex.
\ag.*\label{ex:denn.imp.sym}Ich kann heute nicht kommen, denn $\Delta$ bin krank.\\
 I can today not come because I am sick\\
 `I cannot come today because (I) am sick.'
\bg.*\label{ex:denn.imp.asym}Heute kann ich nicht kommen, denn $\Delta$ bin krank.\\
 today can I not come because I am sick\\
 `Today I cannot come because (I) am sick.' \citep[508, adapted]{vandevelde1986}

From these statements, it follows that topic drop with certain preceding conjunctions \is{Conjunction} might be acceptable because speakers use it analogously to or hearers reanalyze it as subject gap \is{Subject gap} in symmetric or asymmetric coordination.

Besides the fact that such an interpretation may not be possible for all preceding conjunctions, \is{Conjunction} two further facts speak against the analogy to subject gaps \is{Subject gap} as the only explanation.
First, example \ref{ex:no.sg.td} taken from an email by a student%
%% Footnote
\footnote{I thank Nele Hartung for providing me with this example.}
%
 suggests that topic drop after a conjunction \is{Conjunction} is also possible if the subject of the preceding clause is distinct from the omitted subject and a subject gap \is{Subject gap} analysis is not available.

\exg.\label{ex:no.sg.td}Es tut mir wirklich sehr Leid [sic!] und $\Delta$ hoffe Sie haben Verständnis.\\
it does me really very sorry {} and I hope you.\textsc{2sg.pol} have understanding\\
`I am really very sorry and (I) hope you understand.' (Lit. `It does me very sorry and (I) hope you understand.') (email by a student, 05/19/2022)

Second, a preceding conjunction \is{Conjunction} in the form of a discourse particle apparently can occur with object topic drop, see example \ref{ex:obj.td.conj} with \textit{aber} (`but'), a modification of the attested example \ref{ex:obj.td.noconj}, while object gaps in coordination are taken to be impossible (see \cite{horch2014} for experimental evidence against \cite{fortmann2005}, who considers object gaps to be possible under certain circumstances).%
%% Footnote
\footnote{An informal survey of colleagues suggests that the variant with the conjunction \is{Conjunction} is in fact even slightly preferred, although the general judgments for examples \ref{ex:obj.td.noconj} and \ref{ex:obj.td.conj} vary widely, which calls for a systematic empirical investigation in a future study.}
%

\ex.
\ag.\label{ex:obj.td.noconj}ach, Hotels habe ich. ja. zwei Möglichkeiten. [...] fünf Minuten vom Bahnhof mit Garage. $\Delta$ brauchen wir eh nicht, so ein Quatsch.\\
\textsc{inj} hotels have I yes two options {} five minutes from.the station with garage that need we anyway not such a nonsense\\
`Ah, hotels I have. Yes. Two options. [...] Five minutes from the station with garage. We don't need (that) anyway, such nonsense.' \newline [FraC D1974--1975]
\b.\label{ex:obj.td.conj} [...] fünf Minuten vom Bahnhof mit Garage. Aber $\Delta$ brauchen wir eh nicht, so ein Quatsch.

To complement the corpus and introspective data just presented with experimental evidence, I conducted an experiment on topic drop with and without preceding conjunctions, which is presented in the next section.

\refstepcounter{expcounter}\label{exp:conjunctions} \is{Conjunction|(} \is{Subject gap|(}
\subsection{Experiment \arabic{expcounter}: topic drop and preposed conjunctions }
\is{Acceptability rating study|(}\label{sec:exp.conjunctions}
In experiment \arabic{expcounter}, I investigated whether topic drop is restricted to the left-most element of a clause using the test case of topic drop after conjunctions.%
% Footnote
\footnote{All items, fillers, and the analysis scripts are available online: \url{https://osf.io/zh7tr}.}
%
I tested utterances with topic drop and corresponding full forms that were or were not preceded by a conjunction and collected acceptability judgments.
Since, as shown in \sectref{sec:intial.theory}, different conjunctions allow for a subject gap interpretation to varying degrees, I altered the type of conjunction between items for exploratory reasons.
A third of the items contained \textit{und} (`and') as a conjunction, a third \textit{aber} (`but'), and another third \textit{denn} (parordinating `because').%
% Footnote
\footnote{The term \textit{parordinating} was introduced by \cite[329]{hohle1986} for connectives that neither coordinate nor subordinate their conjuncts.
See below for a more exhaustive discussion of \textit{denn}.}
%
As a control predictor, I also varied whether or not the sentence preceding the critical utterance theoretically allows for interpreting the target sentence as an SLF construction, i.e., as containing a subject gap.
Based on the literature presented in the previous section, such an interpretation should be perfectly possible for \textit{und}, impossible for \textit{denn}, and possibly marked for \textit{aber}, given that in my materials \textit{aber} does not occur post- but pre-verbally in the second conjunct (see below).
However, since the claims about \textit{aber} and \textit{denn} have not, to my knowledge, been empirically confirmed, it does not seem unreasonable to vary the possibility of a subject gap reading for all three conjunctions.
In sum, the experiment had the form of a 2 $\times$ 2 $\times$ 2 design with \textsc{Completeness} (full form vs. topic drop), \textsc{Presence of Conjunction} (absent vs. present), and \textsc{Subject Gap} (possible vs. impossible).

\subsubsection{Materials}\label{sec:exp.conj.mat}
\subsubsubsection*{Items}
I constructed 24 items in the form of instant messaging dialogues between two persons, such as \ref{ex:item.conjunction}, adapting the materials of experiment \ref*{exp:pf.mf}.
Person A always asked a question, to which person B responded with three utterances.
The last of these utterances was the target utterance \ref{ex:item.conjunction.target}, in which the 1st person singular subject pronoun \textit{ich} was omitted from or realized in the prefield.
The (empty or filled) prefield was \ref{ex:item.conjunction.target.conj} or was not \ref{ex:item.conjunction.target.noconj} preceded by one of the three conjunctions \textit{und} (`and'), \textit{aber} (`but'), or \textit{denn} (parordinating `because').
As mentioned above, the conjunction was varied between items, so that eight token sets were built with \textit{und}, such as \ref{ex:item.conjunction}, another eight with \textit{aber}, and another eight with \textit{denn}.

\ex.\label{ex:item.conjunction}
\ag.A: Was habt ihr heute Abend geplant?\\
{} what have you.\textsc{2pl} today evening planned\\
A: `What do you have planned for tonight?'
\bg.B: Wir wollen uns den neuen Matrixfilm im Kino anschau- \newline en.\makebox[0pt][l]{\vspace*{-2.5cm}\raisebox{-0.35ex}{\includegraphics[scale=0.07]{Emojis/smiling-face-with-smiling-eyes_wa.png}}}\\
{} we want us the new matrix.movie in.the theater watch\\
\vspace{-\baselineskip}
B: `We want to watch the new Matrix movie in the theater.\makebox[0pt][l]{\vspace*{-2.5cm}\raisebox{-0.35ex}{\includegraphics[scale=0.07]{Emojis/smiling-face-with-smiling-eyes_wa.png}}}\phantom{B::}'
\c.\label{ex:item.conjunction.sg}
\ag.\label{ex:item.conjunction.sg.poss}B: Die ersten drei Filme mag ich total.\\
{} the first three movies like I totally\\
B: `I totally like the first three movies.' \hfill (subject gap possible)
\bg.\label{ex:item.conjunction.sg.imposs}B: Die ersten drei Filme gefallen mir total gut.\\
{} the first three movies please me.\textsc{dat} totally well\\
B: `I totally like the first three movies.' \\(Lit. `The first three movies please me totally well.') \\ \phantom{.}\hfill (subject gap impossible)
\z.
\d.\label{ex:item.conjunction.target}
\ag.\label{ex:item.conjunction.target.conj}B: Und (ich) bin jetzt  richtig gespannt auf den neuen Teil.\\
{} and I am now rightly keen on the new part\\
B: `And (I) am now really excited about the new part.' \\\phantom{.}\hfill(conjunction present)
\bg.\label{ex:item.conjunction.target.noconj}B: (Ich) bin jetzt  richtig gespannt auf den neuen Teil.\\
{} I am now rightly keen on the new part\\
B: `(I) am now really excited about the new part.'\\\phantom{.}\hfill(conjunction absent)

The penultimate utterance, \ref{ex:item.conjunction.sg}, was varied between whether it theoretically enabled an asymmetric coordination with a subject gap reading of the target utterance or not.
In the condition that was intended to enable such a reading, \ref{ex:item.conjunction.sg.poss}, the speaker appeared as a 1st person singular subject, just like in the target utterance.
In the other condition \ref{ex:item.conjunction.sg.imposs}, the speaker was referred to by means of an object pronoun in an impersonal construction with verbs such as \textit{jdm. gefallen} (`to please sb.'), \textit{jdm. gut gehen} (`to be well'), or \textit{jdm. wohl sein} (`to feel comfortable').
Examples \ref{ex:item.aber} and \ref{ex:item.denn} show the relevant manipulation also for a token set with \textit{aber} and \textit{denn} respectively.%
%% Footnote
\footnote{Note that the manipulation of \textsc{Subject Gap} is lost in the translation to English, but it can be seen from the glosses.}
%

\ex.\label{ex:item.aber}
\ag.Seit gestern kränkle ich ein bisschen\makebox[0pt][l]{\vspace*{-0.5cm}\raisebox{-0.35ex}{\includegraphics[scale=0.07]{Emojis/sneezing-face_wa.png}}}\phantom{mi} Aber $\Delta$ fühl mich bestimmt nächste Woche besser.\\
since yesterday ail I a bit but I feel \textsc{refl} surely next week better\\
`Since yesterday I have been a bit sick. But I'm sure I'll feel better next week.' \hfill (subject gap possible)
\bg.Seit gestern geht's mir nicht so gut\makebox[0pt][l]{\vspace*{-0.5cm}\raisebox{-0.35ex}{\includegraphics[scale=0.07]{Emojis/sneezing-face_wa.png}}}\phantom{mi} Aber $\Delta$ fühl mich bestimmt nächste Woche besser.\\
since yesterday goes.it me.\textsc{dat} not so good but I feel \textsc{refl} surely next week better\\
`Since yesterday I am not so well. But I'm sure I'll feel better next week.' \hfill (subject gap impossible)

\ex.\label{ex:item.denn}
\ag.So richtig wohl fühl ich mich dabei aber nicht.	Denn $\Delta$ kenn den Typ ja gar nicht.\\
so rightly well feel I \textsc{refl} thereby but not because I know the guy \textsc{part} at.all not\\
`But I don't really feel comfortable with it. Because I don't know the guy at all.' \hfill (subject gap possible)
\bg.So richtig wohl ist mir dabei aber nicht.	 Denn $\Delta$ kenn den Typ ja gar nicht.\\
so rightly well is me.\textsc{dat} thereby but not because I know the guy \textsc{part} at.all not\\
`But I don't really feel comfortable with it. Because I don't know the guy at all.' \hfill (subject gap impossible)

\subsubsubsection*{Fillers}
The items were mixed with a total of 72 fillers:
16 items from experiment \ref*{exp:top.q2}, which investigated the impact of topicality on topic drop (see \sectref{sec:exp.top.q2}) consisting of dialogues with four turns, 24 items on preposition omission in fragments with proper names vs. definite descriptions also consisting of four turns, 24 (potential) gapping structures with V2 vs. verb-final word order, half of which consisted of four and half of two turns, and eight slightly adjusted catch trials from experiment \ref*{exp:ex}, which also equally consisted of two and four turns.

\subsubsection{Procedure}\label{sec:ex.conj.prod}
Like the previously presented studies, experiment \arabic{expcounter} was an acceptability rating study conducted over the Internet using LimeSurvey \citep{limesurveygmbh}.
The experiment involved 72 native German speakers (age 18--40) who were again recruited from the crowdsourcing platform Clickworker \citep{clickworker2022} and who had not participated in any of my previous studies on topic drop.
They received a compensation of €4.00. 
Their task was again to read the dialogues and to rate the naturalness of the last utterance using a 7-point Likert scale (1 = completely unnatural, 7 = completely natural).
I distributed the materials across eight lists so that each subject saw each critical stimulus exactly once and in one condition.
The items and the fillers were mixed and presented as WhatsApp-like dialogues in individual pseudo-randomized order so that no two stimuli of the same type immediately followed each other.

\subsubsection{Results}\label{sec:ex.conj.res}
I excluded the data from 14 participants who had exceeded the threshold of having rated four or more of the eight catch trials with 6 or 7.
Table \ref{tab:descriptives.conjunctions} shows the mean ratings and standard deviations per condition calculated on the data from the remaining 58 participants.
In Figure \ref{fig:pl.conjunctions}, this information is also plotted.

\begin{table}
\caption{Mean ratings and standard deviations per condition for experiment \arabic{expcounter}}
\centering
\begin{tabular}{lllrr}
\lsptoprule
\textsc{Completeness} & \Centerstack[l]{\textsc{Presence of}\\\textsc{Conjunction}} & \textsc{Subject Gap} & \Centerstack{Mean\\rating} & \Centerstack{Standard\\deviation} \\
\midrule
Full form & Absent & Possible &  $6.03$ &  $1.13$\\
Topic drop & Absent & Possible &  $6.01$ & $1.20$ \\
Full form & Present & Possible &  $5.92$ & $1.20$ \\
Topic drop & Present & Possible &  $5.56$ & $1.54$ \\
Full form & Absent & Impossible & $6.12$ &  $1.12$ \\
Topic drop & Absent & Impossible & $6.06$ & $1.24$ \\
Full form & Present & Impossible &  $5.90$ &  $1.11$\\
Topic drop & Present & Impossible & $5.38$ & $1.72$ \\
\lspbottomrule
\end{tabular}
\label{tab:descriptives.conjunctions}
\end{table}

\begin{figure}
\centering
\includegraphics[scale=1]{Experimenteplots/PL_Conj_all.pdf}
\caption{Mean ratings and 95\% confidence intervals per condition for experiment \arabic{expcounter}}
\label{fig:pl.conjunctions} % pl for point line
\end{figure}

\noindent
While the ratings for full forms and utterances with topic drop were comparable when no conjunction was present, topic drop was degraded when it was preceded by a conjunction.
From visual inspection, it seems that this effect is even more pronounced when a subject gap reading was impossible.

I analyzed the rating data with CLMMs from the package ordinal \citep{christensen2019} in R, as described in \sectref{sec:data.analysis}.
The full model contained the binary predictors \textsc{Completeness}, \textsc{Presence of Conjunction}, and \textsc{Subject Gap} coded using deviation coding (full form, conjunction absent, and subject gap possible were coded as $0.5$, the other levels as $-0.5$ respectively), their three-way interaction and all two-way interactions between them, as well as the numeric scaled and centered \textsc{Position} of the trial in the experiment and all two-way interactions of the factors with \textsc{Position}.
As random effects, I included random intercepts for both subjects and items and by-item and by-subject random slopes for \textsc{Completeness}, \textsc{Presence of Conjunction}, \textsc{Subject Gap}, and their interactions which each other, as well as for \textsc{Position}.%
%% Footnote
\footnote{The formula of the full model was as follows: \texttt{Ratings \textasciitilde ~\textsc{Completeness} : \textsc{Presence of Conjunction} : \textsc{Subject Gap} + (\textsc{Completeness} + \textsc{Presence of Conjunction}  + \textsc{Subject Gap} + \textsc{Position})\textasciicircum2 + (1 + \textsc{Completeness} + \textsc{Presence of Conjunction}  + \textsc{Subject Gap})\textasciicircum2 + \textsc{Completeness} : \textsc{Presence of Conjunction} : \textsc{Subject Gap} + \textsc{Position} | Items) + (1 + \textsc{Completeness} + \textsc{Presence of Conjunction}  + \textsc{Subject Gap})\textasciicircum2 + \textsc{Completeness} : \textsc{Presence of Conjunction} : \textsc{Subject Gap} + \textsc{Position}  | Subjects)}.}
The fixed effects in the final model with flexible thresholds are shown in Table \ref{tab:conjunctions.model}.

\begin{table}
\caption{Fixed effects in the final CLMM of experiment \arabic{expcounter}}
\centering
\begin{tabular}{lrrrll}
\lsptoprule
Fixed effect & Est. & SE & $\chi^2$ & \textit{p}-value &   \\
\midrule
\textsc{Completeness} & $0.32$ & $0.22$ & $2.02$ & $> 0.15$ & \\
\textsc{Presence of Conjunction} & $1.00$  & $0.28$ & $10.98$ & $< 0.001$ & \***\\
\textsc{Position} & $0.08$ & $0.08$ & $1.00$ & $> 0.31$ & \\
\Centerstack[l]{\textsc{Completeness} $\times$ \\ \textsc{Presence of Conjunction}} & $-0.87$  & $0.35$ & $6.02$ & $< 0.05$ & *\\
\textsc{Completeness} $\times$ \textsc{Position} & $-0.27$  & $0.13$ & $4.63$ & $< 0.05$& *\\
\Centerstack[l]{\textsc{Presence of Conjunction}  $\times$ \\ \textsc{Position}} & $-0.32$  & $0.13$ & $6.45$ & $< 0.05$ & *\\
\lspbottomrule
\end{tabular}
\label{tab:conjunctions.model}
\end{table}

\noindent
There was a significant main effect of \textsc{Presence of Conjunction} ($\chi^2(1) = 10.98$, $p < 0.001$).
It indicates that utterances without preceding conjunctions were preferred over utterances with conjunctions.
The significant interaction  between \textsc{Completeness} and \textsc{Presence of Conjunction} ($\chi^2(1) = 6.02$, $p < 0.05$) shows that utterances with topic drop and preceding conjunctions were particularly degraded.
The \textsc{Position} of the trial in the experiment interacted significantly both with \textsc{Completeness} ($\chi^2~= 4.63$, $p < 0.05$) and with \textsc{Presence of Conjunction} ($\chi^2(1) = 6.45$, $p < 0.05$).
Throughout the experiment, participants gave better ratings to both stimuli with topic drop and to utterances with preceding conjunctions, which points toward a habituation effect.
The predictor \textsc{Subject Gap} was not involved in any significant effect.

A closer inspection of the data revealed large differences between token sets, in particular between the different conjunctions.
In particular, the ratings for \textit{denn} (parordinating `because') deviated strongly from those for \textit{und} (`and') and \textit{aber} (`but').
Figure \ref{fig:pl.conjunctions.3} shows the mean ratings and 95\% confidence intervals per conjunction and indicates that the ratings for \textit{denn} (parordinating `because') strongly deviate from the ratings for \textit{und} (`and') and \textit{aber} (`but').
While there was little to no difference between conditions for \textit{und} and \textit{aber}, for \textit{denn} already the full forms with preceding conjunction were rated worse but even more so the topic drop conditions.

\begin{figure}
\centering
\includegraphics[scale=0.9]{Experimenteplots/PL_Conj_3.pdf}
\caption{Mean ratings and 95\% confidence intervals per condition for experiment \arabic{expcounter} subdivided by conjunction}
\label{fig:pl.conjunctions.3}
\end{figure}

This motivated me to conduct a post hoc analysis in which I split the data into three sets, an \textit{und} (`and')-set, an \textit{aber} (`but')-set, and a \textit{denn} (parordinating `because')-set and repeated the analysis for each conjunction separately.
For \textit{aber} (`but') and \textit{denn} (parordinating `because'), I had to reduce the random effects structure by removing the by-items and by-subjects random slopes for the three-way interaction between \textsc{Completeness}, \textsc{Presence of Conjunction}, and \textsc{Subject Gap} for the model to converge.
Tables \ref{tab:conjunction.und.model}, \ref{tab:conjunction.aber.model}, and \ref{tab:conjunction.denn.model} show the fixed effects in the final models of each subanalysis.
The results confirm the visual inspection of Figure \ref{fig:pl.conjunctions.3}.

In the subanalysis of \textit{und} (`and'), there was only a significant interaction between \textsc{Presence of Conjunction} and \textsc{Position} ($\chi^2(1) = 11.32$, $p < 0.001 $).
It indicates that utterances with a preceding \textit{und} received better ratings when they occurred later in the experiment.

\begin{table}
\caption{Fixed effects in the final CLMM of the \textit{und} (`and')-subanalysis of experiment \arabic{expcounter}}
\centering
\begin{tabular}{lrrrll}
\lsptoprule
Fixed effect & Est. & SE & $\chi^2$ & \textit{p}-value &   \\
\midrule
\textsc{Presence of Conjunction} & $0.138$  & $0.43$ & $0.1096$ & $> 0.74$ & \\
\textsc{Position} & $0.001$ & $0.21$ & $0.0003$ & $> 0.99$ & \\
\Centerstack[l]{\textsc{Presence of Conjunction}  $\times$ \\\textsc{Position}} & $-1.015$ & $0.35$ & $11.3200$  & $< 0.001$ & ***\\
\lspbottomrule
\end{tabular}
\label{tab:conjunction.und.model}
\end{table}

\newpage
For \textit{aber} (`but'), there was a significant \textsc{Presence Of Conjunction} $\times$ \textsc{Subject Gap} interaction ($\chi^2(1) = 3.92$, $p < 0.05 $).
Utterances (both full forms and utterances with topic drop) that did not allow for a subject gap reading and contained a conjunction were rated as particularly bad.
Since this effect is independent of \textsc{Completeness}, it is, however, of no theoretical interest in this context.

\begin{table}
\caption{Fixed effects in the final CLMM of the \textit{aber} (`but')-subanalysis of experiment \arabic{expcounter}}
\centering
\begin{tabular}{lrrrll}
\lsptoprule
Fixed effect & Est. & SE & $\chi^2$ & \textit{p}-value &   \\
\midrule
\textsc{Presence of Conjunction} & $0.42$  & $0.38$ & $1.27$ & $> 0.25$ & \\
\textsc{Subject Gap} & $0.42$ & $0.34$ & $1.64$ & $> 0.20$ & \\
\Centerstack[l]{\textsc{Presence of Conjunction} $\times$ \\\textsc{Subject Gap}} & $-1.16$ & $0.61$ & $3.92$ & $< 0.05$ & *\\
\lspbottomrule
\end{tabular}
\label{tab:conjunction.aber.model}
\end{table}

In the subanalysis of \textit{denn} (parordinating `because'), there were significant main effects of \textsc{Completeness} ($\chi^2(1) = 8.6$, $p < 0.01$) and \textsc{Presence of Conjunction} ($\chi^2(1) = 17.4$, $p < 0.001$).
They indicate that full forms were generally preferred over utterances with topic drop and that utterances with \textit{denn} received worse ratings than utterances without.
A significant interaction between both predictors ($\chi^2(1) = 4.31$, $p < 0.05$) suggests that the latter effect was particularly pronounced for topic drop.
Utterances with topic drop and realized \textit{denn} received particularly bad ratings.
A significant \textsc{Completeness} $\times$ \textsc{Position}  interaction ($\chi^2(1) = 6.73$, $p < 0.01$) shows that the ratings for topic drop improved throughout the experiment indicating a slight habituation effect for the \textit{denn} materials.
\textsc{Subject Gap} had no effect.

\begin{table}
\caption{Fixed effects in the final CLMM of the \textit{denn} (parordinating `because')-subanalysis of experiment \arabic{expcounter}}
\centering
\begin{tabular}{lrrrll}
\lsptoprule
Fixed effect & Est. & SE & $\chi^2$ & \textit{p}-value &   \\
\midrule
\textsc{Completeness} & $1.40$ & $0.43$ & $8.60$ & $< 0.01$ & **\\
\textsc{Presence of Conjunction} & $2.88$  & $0.54$ & $17.40$ & $< 0.001$ & ***\\
\textsc{Position} & $0.36$ & $0.22$ & $2.53$ & $> 0.11$ & \\
\Centerstack[l]{\textsc{Completeness} $\times$ \\\textsc{Presence of Conjunction}} & $-1.82$  & $0.80$ & $4.31$ & $< 0.05$ & *\\
\textsc{Completeness} $\times$ \textsc{Position} & $-0.73$ & $0.30$ & $6.73$ & $< 0.01$ & **\\
\lspbottomrule
\end{tabular}
\label{tab:conjunction.denn.model}
\end{table}

\subsubsection{Discussion}
In experiment \arabic{expcounter}, I tested whether topic drop is restricted to a strictly sentence-initial position or whether elements, namely conjunctions, can be placed before topic drop.
The results suggest that generally elements can precede topic drop, i.e., that it is not restricted to the very first position.
The naturalness of utterances with \textit{und} (`and') or \textit{aber} (`but') followed by topic drop was comparable to the topic drop conditions without conjunctions and the corresponding full forms.

The fact that there were no effects of a possible subject gap reading on topic drop indicates that topic drop after \textit{und} (`and') and \textit{aber} (`but') is possible, irrespective of whether such a reading is licensed through a matching context sentence or not.
This suggests that the acceptability of topic drop after conjunctions does not depend on the readers reinterpreting such structures as (asymmetric) coordination structures with subject gaps across sentence boundaries.%
%% Footnote
\footnote{Recall that a subject gap interpretation would require analyzing the corresponding structure as asymmetric coordination and that according to the literature subject gaps in asymmetric coordination only function with \textit{und} (`and').
They are impossible with \textit{denn} (parordinating `because') and presumably require the conjunction to be positioned postverbally with \textit{aber} (`but'), which was not the case in my stimuli.
However, even for only the token sets with \textit{und}, which should allow for subject gaps in asymmetric coordination in any case, there was no effect of the predictor \textsc{Subject Gap}.
That is, the conditions that permit a subject gap interpretation and the conditions that do not, according to the literature, did not differ significantly in their acceptability.}
%

The results for \textit{denn} (parordinating `because') strongly deviated from those for \textit{und} (`and') and \textit{aber} (`but') in that (i) utterances with \textit{denn} were degraded across conditions and (ii) that in particular topic drop with \textit{denn} was rated as less natural.
A possible reason for (i) could be that the conjunction \textit{denn} is too formal for instant messages.
In spoken communication, \textit{denn} has been largely replaced by \textit{weil} (subordinating `because') \citep{selting1999,wegener2000} and text messages can be considered a conceptually spoken form of communication \citep{koch.oesterreicher1985}.
From (ii), we can conclude that while \textit{und} and \textit{aber} can precede topic drop, \textit{denn} cannot do so equally well.
It is obvious to attribute this difference to the difference between the conjunctions.
As a starting point, we can note that they differ with respect to their semantics and to the discourse relations they encode.
The \textit{Handbuch der deutschen Konnektoren 2} (`The handbook of German connectives 2') classifies \textit{und} as additive \citep[1211]{breindl.etal2014}, \textit{aber} as adversative or concessive \citep[1173]{breindl.etal2014}, and \textit{denn} as causal, marking the antecedent \is{Antecedent} of the causal relation \citep[1187]{breindl.etal2014}.
In the taxonomy of \citet{kehler2000,kehler2002}, \textit{und} usually either encodes a resemblance relation, specifically a parallel one, or a contiguity relation, specifically a narration.
\textit{Aber} encodes a contrast, a different subtype of a resemblance relation, whereas \textit{denn} encodes an explanation, a subtype of a cause-effect relation (see \cite[540--545]{kehler2000}, \cite[15--23]{kehler2002}).

A first hypothesis derived from this observation is that the conjunctions differ in how likely they make the subject of the following sentence.
In anticipation of the second part of this book, we can assume that topic drop is more felicitous if the omitted constituent is more predictable \is{Predictability} in context.
If a certain conjunction makes \textit{ich} a lot more likely, e.g., \textit{and} as additive or parallel connective, this may boost the acceptability/appropriateness of topic drop.
Coming back to \textit{denn}, we can specify the hypothesis in the following way:
Topic drop following \textit{denn} received degraded ratings in the experiment because the subject \textit{ich} is less likely to follow \textit{denn} than \textit{und} and \textit{aber}.
In a cloze study, \is{Production|(} where I collected continuations after the context sentence with or without the conjunction given, I found no evidence to support this hypothesis.%
% Footnote
\footnote{In a 2 $\times$ 2 design (\textsc{Presence of Conjunction} $\times$ \textsc{Subject Gap}), 60 participants provided continuations for the 24 items of experiment \arabic{expcounter} and 16 fillers. \is{Production|)}
The items were cut off after the context sentence, as shown in \ref{ex:conjunctions.cloze}, so that participants produced the target sentence.
It was varied between conditions whether the conjunction was already determined as the first word of this sentence or not.
I considered both the proportion of produced continuations with \textit{ich} (`I') as the subject and the proportion of continuations with \textit{ich} (`I') in the prefield.
For both measures, the variation between token sets was higher than the variation between conjunctions.

%\vspace{-0.5\baselineskip}
\ex.\label{ex:conjunctions.cloze}
\a.A: Was habt ihr heute Abend geplant?
\b.B: Wir wollen uns den neuen Matrixfilm im Kino anschauen\makebox[0pt][l]{\vspace*{-2cm}\raisebox{-0.35ex}{\includegraphics[scale=0.07]{Emojis/smiling-face-with-smiling-eyes_wa.png}}}
\c.B: Die ersten drei Filme (mag ich total | gefallen mir total gut)
\d.B: (Und) \underline{\phantom{wollen uns den neuen Matrixfilm im Kino anschauen}}
}
%
Whether participants produced a continuation with a 1st person singular subject differed more strongly between token sets than between conjunctions. 

A second hypothesis can be built based on the semantic and syntactic properties of \textit{denn}, which is considered to be a `syntactic loner' (\cite{pasch.etal2003, reich.reis2013, breindl.etal2014, breindl2017}, see also Footnote \ref{note:denn}).
\citet[565]{reich.reis2013} and \citet[866--867]{breindl.etal2014} state that a special property of \textit{denn} is its asymmetric behavior.
Semantically or discourse-structurally, \textit{denn} functions as a subordinating connective because it encodes an explanation relation between the first and the second conjunct, with the second conjunct as explanation being subordinate to the first.
Syntactically, however, \textit{denn} is a ``parordinating'' connective \citep[329]{hohle1986}, that neither coordinates nor subordinates its conjuncts \citep[590]{pasch.etal2003}.
\citet[870]{breindl.etal2014} state that the information expressed in the conjunct introduced by \textit{denn} is always asserted independently and therefore always in focus \is{Focus} so that there is a tendency to introduce new information through a \textit{denn} clause.
The fact that \textit{denn} seems to block topic drop may be a consequence of its hybrid status.
Given the semantic subordination of the second conjunct, \textit{denn} may indicate that the clause it introduces is not fully autonomous, therefore, its prefield is not available for ellipsis.
To verify this speculation, further research on \textit{denn} is needed.

In sum, experiment \arabic{expcounter} indicates that topic drop is not restricted to the very first element of an utterance.
Conjunctions like \textit{und} and \textit{aber} can precede it.
The corresponding structures are not ``hidden'' subject gap structures but genuine topic drop structures.
As discussed before the experiment, it is unclear whether the linear precedence translates itself into a syntactic superordination and whether this relation holds \textit{within} a clause, which may potentially affect \citeg{rizzi1994} \textit{specifier of the root}-account based on c-command \is{C-command} or \citeg{freywald2020} highest clause position account.
If we assume the coordination structure \is{Conjunction phrase|(} visualized in \ref{ex:tree.coord}, repeated here as \ref{ex:tree.coord.rep}, then the covert constituent in [Spec, CP] of S2 would be c-commanded \is{C-command} by the head \&.

\ex.\label{ex:tree.coord.rep}
\begin{forest}
for tree={s sep*=3, parent anchor=south, child anchor=north}
[\&P, baseline
	[S1
		[\phantom{SENT},roof]
	]
	[\&'
		[\&]
		[S2
			[\phantom{SENT},roof]	
		]
	]
]
\end{forest}\\
\phantom{.}\hfill\citep{reichtoappear} % XX Seiten

At first glance, this is not compatible with Rizzi's c-command \is{C-command} approach because the covert constituent in [Spec, CP] of S2 would be c-commanded \is{C-command} sentence-internally and, therefore, should not be available for topic drop.
There are two ways to counter this:
(i) As discussed above, there is the assumption that the conjunction in \& is not a genuine part of the following clause S2.
Therefore, it could be argued that Rizzi's principle is not violated because the c-command \is{C-command} relation between the conjunction in \& and the covert constituent in [Spec, CP] of S2 does not hold sentence-internally but sentence-externally.
(ii) Although there is a c-command \is{C-command} relation between the conjunction in \& and the covert constituent in [Spec, CP] of S2, the conjunction in \& cannot function as an identifier for the covert constituent as it is a functional head \citep{johannessen1998}, which cannot provide an antecedent \is{Antecedent} for a null pronoun.
Thus, Rizzi's principle is not violated because the covert constituent in [Spec, CP] of S2 may be c-commanded \is{C-command} but not by a potential identifier.

Concerning \citeg{freywald2020} proposal, we have to come back to the fact mentioned already in \sectref{sec:highest.theory} that she equates the concepts of autonomous clause and root clause.
However, it is, in particular, the difference between both concepts, which \citet{reich.reis2013} point out, that matters for the analysis of S2 in \ref{ex:tree.coord.rep}.
According to \citet[542]{reich.reis2013}, the conjuncts of an autonomous coordination are considered to be root clauses but not autonomous clauses.
This means that Freywald's account needs to be specified to allow for topic drop in S2.
It must be determined that the relevant category is the root clause and not the autonomous clause, i.e., that topic drop is enabled in the highest [Spec, CP] of a root clause.
This way, Freywald's approach becomes more general because root clauses are a superset of autonomous clauses.
Consequently, topic drop is permitted in more configurations. \is{Acceptability rating study|)} \is{Conjunction|)} \is{Conjunction phrase|)} \is{Subject gap|)}

\section{Summary: prefield restriction}\label{sec:top.pf.summary}
\largerpage[-1]
Before I turn to the final section of this chapter and address the question of how a suitable syntactic analysis may look in light of what we learned so far about the prefield restriction of topic drop, I summarize these findings and specify the prefield restriction.

In this chapter, I investigated in detail the most prominent licensing condition of topic drop:
its positional restriction to the prefield position.
As a first step, I looked at the role of the information structural category topic for topic drop.
I showed that the prefield in German is neither the only position in which topic expressions can be placed nor that all elements occurring in the prefield are topics.
Based on theoretical arguments and experimental data, I argued that the topicality of the omitted constituent is neither a (strictly) sufficient nor a necessary condition for topic drop.
Contrastive topics \is{Contrastive topic} and topics that cannot be recovered \is{Recoverability} question at least strict sufficiency indicating that there can be topic expressions in the prefield that cannot be omitted.
More importantly, the fact that expletive \is{Expletive} subjects, i.e., non-referential and, thus, non-topical elements, can be omitted calls necessity into question.
Apparently, elements can be targeted by topic drop without being topic expressions, which shows that the scope of topic drop goes beyond topics.
If one does not want to abandon the relevance of the notion of topic for topic drop altogether, one could argue that topics are at least the prototypical candidates for omission since they are usually easy to recover (see the discussions on recoverability \is{Recoverability} in Chapter \ref{ch:recover}) and predictable \is{Predictability} in context (see Chapter \ref{ch:factor.topicality}).
This means that the topic restriction, which is often regarded as a licensing condition in the literature, is only a favoring factor for topic drop.
I investigate the degree to which topicality is indeed such a factor in Chapter \ref{ch:factor.topicality} in the second part of this book.

In the second step, I showed experimentally that argument omissions in the middle field \is{Middle field} are degraded, which supports the validity of a structural prefield restriction of topic drop.
With two further experiments, I refined the properties of this restriction.
First, I showed that topic drop is not possible in every prefield position but only in a prefield position that is at least not c-commanded \is{C-command} sentence-internally by a potential identifier, as proposed by \citet{rizzi1994}, or even only in the highest prefield of an autonomous or root clause, according to \citet{freywald2020}.
Second, I was able to provide evidence that topic drop is not restricted to the very first element of a clause but that conjunctions \is{Conjunction} like \textit{und} (`and') and \textit{aber} (`but') can precede it.
I argued that this result does not question \citeg{rizzi1994} c-command-approach because \is{C-command} the conjunction is not really part of the second utterance, nor can it be an antecedent \is{Antecedent} for the covert constituent because it is a functional head. \is{Conjunction} \is{Conjunction phrase}
It makes it necessary, however, to specify \citeg{freywald2020} proposal in such a way that topic drop is restricted to the highest prefield of a root clause and not of an autonomous clause.
In this book, I do not distinguish between \citeg{rizzi1994} and \citeg{freywald2020} accounts but include both as options in the following specified definition of topic drop:

\begin{theorem}\label{def:topic2}
Topic drop is the omission of a constituent from the prefield of declarative verb-second (V2) clauses.
This prefield must at least not be c-commanded \is{C-command} sentence-internally by a potential identifier or it must even be the highest prefield of a root clause.
\end{theorem}

\section{Toward a syntactic analysis of topic drop}\label{sec:syntax}
In light of this sharpened prefield restriction, I turn to a potential syntactic analysis of topic drop.
Such an analysis faces the challenge of explaining the omission of both referential and non-referential constituents, as well as of subjects, objects, and adverbials, and of adequately modeling the prefield restriction as the central syntactic licensing condition.
I discuss the three main syntactic approaches to topic drop, shown in Figure \ref{fig:syntax.schema}: null operator analysis, \is{Operator analysis} \textit{pro}-analysis, \is{@\emph{pro}-analysis} and PF-deletion analysis. \is{PF-deletion analysis}

\begin{figure}
\centering
\begin{forest}
for tree={s sep*=1.5, l sep* = 2, parent anchor=south, child anchor=north, align = center}
[\textit{Is a special null element required?}
	[\textit{What kind of null element is required?}, edge label={node[midway,above]{yes}}
		[{\textsc{(null topic) operator}\\
		{\footnotesize (e.g., \cite{huang1984},}\\
		{\footnotesize\cite{sigurdsson1989},}\\
		{\footnotesize\cite{cardinaletti1990},}\\
		{\footnotesize\cite{rizzi1994})}
		}
		]
		[{\textsc{empty category \textit{pro}}\\
		{\footnotesize (e.g., \cite{cardinaletti1990},}\\
		{\footnotesize\cite{platzack1996},}\\
		{\footnotesize\cite{trutkowski2016},}\\
		{\footnotesize\cite{freywald2020})}
		}
		]
	]
	[{\textsc{pf-deletion}\\{\footnotesize
	(e.g., \cite{mornsjo2002},}\\
	{\footnotesize\cite{sigurdsson.maling2010},}\\
	{\footnotesize\cite{sigurdsson2011},} \\
	{\footnotesize\cite{nygard2018})}},
	edge label={node[midway,above]{no}}
	]
]
\end{forest}
\caption{Schematic overview of the three main syntactic analyses of topic drop proposed in the literature and of their central representatives  }
\is{@\emph{pro}-analysis}\is{Operator analysis} \is{PF-deletion analysis}
\label{fig:syntax.schema}
\end{figure}

\newpage
\noindent
I present their central representatives, as well as their respective advantages and disadvantages.%
%% Footnote
\footnote{See \citet[27--44]{nygard2018} for a more detailed overview covering among others the suggestions by \citet{huang1984}, \citet{sigurdsson1989}, \citet{cardinaletti1990}, \citet{rizzi1994}, \citet{haegeman1990}, \citet{mornsjo2002}, \citet{sigurdsson.maling2010}, and \citet{sigurdsson2011}.
I focus exclusively on generative approaches because they dominate the research.
But cf. \citet{helmer2017a}, who discusses the possibility of analyzing topic drop with construction grammar.}
%
As Figure \ref{fig:syntax.schema} illustrates, the null operator analysis, the analysis as null pronoun \textit{pro}, and the analysis as PF-deletion differ, first, in whether they require a specific null element and, if so, second, in what type of null element is assumed.

\subsection{Operator analysis}\label{sec:syntax.operator} \is{Operator analysis|(}
The first generative analysis proposed for topic drop that I am aware of is by \citet{huang1984}.
He suggests to analyze topic drop of both referential subjects and objects by means of a variable that is $\bar{A}$-bound by a null topic (or zero topic) operator in [Spec, CP], which in turn is linked to the discourse (\cite[543; 548]{huang1984}; see also \cite{sigurdsson1989} and \cite{haegeman1990} for similar approaches).%
%% Footnote
\footnote{\citet{cardinaletti1990} and, with slight modifications, also \citet{rizzi1994} adopt the operator analysis only for objects.
They argue that it follows from the alleged restriction of object topic drop in German to the 3rd person (see \sectref{sec:usage.function.theory} for a discussion) that the corresponding operator has intrinsic 3rd person properties (\cite[79]{cardinaletti1990}, \cite[161]{rizzi1994}).
For subject topic drop, they both propose a different analysis.
\citet{cardinaletti1990} advocates an analysis involving a null pronoun \textit{pro} (see below). \is{@\emph{pro}-analysis}
\citet{rizzi1994} argues that in V2 languages \is{V2 word order} [Spec, CP] can behave as an ${A}$- instead of an $\bar{A}$-position if the subject is moved there, which enables the null element in [Spec, CP] to bind an NP-trace in [Spec, IP] \citep[161]{rizzi1994}.
To prevent this null element, which is not identified sentence-internally, from violating the ECP, \citet{rizzi1994} proposes a relaxed ECP identification requirement that exempts the \textit{specifier of the root}, [Spec, CP] in V2 languages \is{V2 word order} \citep[162]{rizzi1994}, thereby providing a theory-driven justification for the prefield restriction of topic drop.}
%
It already hints at the main problem of the operator account that the operator is usually conceptualized as a null or zero topic operator \citep{huang1984, sigurdsson1989, cardinaletti1990, haegeman1990}.
The concept seems to be closely tied to topicality and, therefore, cannot straightforwardly account for the omission of non-topics, in particular of non-referential expletives. \is{Expletive|(}

The operator analysis of topic drop is similar to (i) the analysis by \citet{buering.hartmann1998} for subject gaps \is{Subject gap|(} in asymmetric coordinations and (ii) \citeg{reich2017} proposal for null copulas and null articles. \is{Article omission}\is{Copula omission}

\subsubsection{\citet[176--177]{buering.hartmann1998} }
% (i)
\citet[176--177]{buering.hartmann1998} assume for cases of asymmetric coordination like \ref{ex:buering.hartmann} that there is an empty element in the subject position of the second conjunct [Spec, IP] that is bound and identified by an empty operator in [Spec, CP] and that this operator, in turn, is bound by the subject of the first clause.%
%% Footnote
\footnote{It is worth noting that \citet[12]{hartmann1994} analyzes null subjects \is{Null subject} in coordinations as \textit{pro} \is{@\emph{pro}-analysis} bound by an empty operator, following a proposal by \citet{cinque1990}, and, earlier, \citet{chomsky1986}, for parasitic gaps and extractions out of NP islands.
She argues that the requirements imposed on the empty category involved in subject gaps are best fulfilled by \textit{pro} since it bears its own theta role and can appear in any grammatical person \citep[12--13]{hartmann1994}.
This analysis allows for combining operator and \textit{pro} (see below), but it does not solve the problem that non-referential expletives \is{Expletive} are not accounted for.}
%
In the appendix of their paper, they provide a semantic analysis of this idea \citep[191--198]{buering.hartmann1998}.

\exg.\label{ex:buering.hartmann}Gestern ging ich aus und $\Delta$ traf Olaf Thon.\\
yesterday went I out and I met Olaf Thon\\ 
`Yesterday, I went out and (I) met Olaf Thon.' \citep[176]{buering.hartmann1998}

Note that \citeg{buering.hartmann1998} variant of the operator analysis also does not cover the omission of non-referential expletive \is{Expletive} subjects.
For subject gaps, \is{Subject gap|)} it probably does not have to because expletives cannot occur in subject gap configurations according to \citet[53]{reich2009}.
For topic drop, on the other hand, it would be desirable, as mentioned above, if the omission of expletives \is{Expletive} were also explained by the syntactic account.
\citeg{buering.hartmann1998} approach could be adapted, though, for topic drop of referential constituents, with the following modifications:
First, to account for all attested cases of topic drop, we must assume that the empty element can function not only as a subject but also as an object or an adjunct.%S
%% Footnote
\footnote{Unlike the proposal by \citet{cardinaletti1990}, the operator for the objects does not need to be restricted to the 3rd person.
I argue that the apparent impossibility to drop 1st and 2nd person objects can be accounted for mainly by pragmatics (see \sectref{sec:usage.function.theory}).}
%
\is{Adjunct}
Second, the empty operator involved in topic drop constructions semantically works like a lambda abstraction, turning an open proposition of type $t$ into a property of type $\langle e,t\rangle$, provided the bound element is of type $e$.%
%% Footnote
\footnote{I thank Ingo Reich for the help with this semantic conceptualization.}
%
To become an interpretable proposition $t$, this unsaturated property $\langle e,t\rangle$ needs an argument of the object type $e$, on which it can operate.
Since the utterance itself does not supply such an argument, it needs to come from the previous linguistic or extralinguistic context, as in example \ref{ex:semantics}.

\ex.\label{ex:semantics}
\a. `What about Tino?'
\bg.\label{ex:semantics.td}$\Delta$ Schläft.\\
he sleeps. \\
`(He) is sleeping.'

Thus, the discourse orientation of [Spec, CP] that has often simply been postulated in previous research can be discourse-pragmatically motivated.
For the utterance with topic drop to denote a proposition, it needs to access an entity from the discourse.
This analysis has the severe drawback, though, that  it leads semantically to the assumption that declaratives cannot only be of type $t$ but also of type $\langle e,t\rangle$. 
That is, utterances with topic drop like \ref{ex:semantics.td} are semantically properties of type $\langle e,t\rangle$, but they behave as if they were declaratives of type $t$ in having a truth value (\ref{ex:semantics.td} is true if Tino sleeps and false otherwise).
While from a syntactic point of view, the operator analysis seems quite elegant, at least for referential constituents, it is therefore not very attractive from a semantic point of view.

\largerpage
\subsubsection{\citet{reich2017}}
% (ii)
\citet{reich2017} also proposes a null operator analysis of null copulas and null articles in newspaper headlines. \is{Article omission|(}\is{Copula omission|(}
He refers to the typological distinction between discourse-oriented and sentence-oriented languages \citep{tsao1977, huang1984} and the property that, according to \citet[545]{huang1984}, only discourse-oriented languages such as Chinese \il{Chinese} allow for a null or zero topic operator that binds a variable, while sentence-oriented languages such as \ili{English} do not \citep[192]{reich2017}.
\citet[193]{reich2017} argues that \ili{German} allows for a switch from ``normal'' sentence orientation to discourse orientation in certain registers (such as headlines), thus enabling the phenomena of null copulas and null articles, which he investigates.
He assumes in his analysis, exemplified for a null article in \ref{ex:null.article}, that null copulas and null articles are represented by variables bound by a null operator located in the specifier of a TopP, which forms the topmost phrase of the headline.
This null operator is a so-called \textit{event topic}, \textit{e}-topic for short, which is licensed in headlines because they are characterized by the fact that they report (newsworthy) events \citep[193]{reich2017}.
The event variable \textit{e\textsubscript{i}} is existentially bound by $\lambda$-abstraction.
\citet[194]{reich2017} argues that sentence-oriented and discourse-oriented registers differ in that ``[i]n \textit{sentence-oriented} registers covert variables are existentially quantified over at (some) propositional level of semantic interpretation.
In \textit{discourse-oriented} registers they are systematically bound by $\lambda$-abstraction'' (original emphasis).
It seems possible to extend this analysis also to topic drop by arguing that it is a further phenomenon that gets enabled by the switch to discourse orientation.
It must be noted, however, that null copulas and null articles are much more restricted to certain text types, \is{Text type} in particular to headlines, than topic drop.
Topic drop is particularly common in text messages and spoken language, but it is not entirely excluded in many other text types and registers (see \sectref{sec:corpus.texttype}). \is{Text type}
For this reason, it is questionable whether one can argue for a shift from sentence- to discourse-orientation in every case and for the existence of the postulated \textit{e}-topic.
\largerpage

\ex.\label{ex:null.article}
\ag.Kuh springt durch Fenster\\
cow jumps through window\\
`Cow jumps through window'
\b.
\begin{forest}
for tree={s sep*=3, parent anchor=south, child anchor=north}
[TopP, baseline
	[\emph{e}\textsubscript{\textsc{top}}]
	[Top\textquotesingle
		[Top]
		[TP\,:\,$\lambda$\emph{e\textsubscript{i}\,$\exists$\emph{f},\emph{g}}
			[DP
				[D
					[\emph{f}(\emph{e\textsubscript{i}})]
				]
				[NP
					[cow, roof]
				]
			]
			[T'
				[T
					[jumps]
				]
				[VP
					[through \emph{g}\,(\emph{e\textsubscript{i}})(window)\,\emph{t\textsubscript{v}}, roof]
				]
			]
		]		
	]
]
\end{forest}

%\vspace{-0.5em}
\phantom{.}\hfill\citep[197]{reich2017}\\
%\vspace{1em}

\is{Article omission|)}\is{Copula omission|)}
\noindent
In summary, an operator analysis of topic drop seems syntactically quite plausible and can motivate the prefield restriction discourse-pragmatically.
However, it has the clear disadvantages that, at least in its present form, it can only explain referential topic drop and that a precise semantic analysis would lead to a type problem of declaratives.
\is{Operator analysis|)}

\largerpage
\subsection{\textit{pro}-analysis}\label{sec:syntax.pro} \is{@\emph{pro}-analysis|(}

While I distinguished topic drop from \textit{pro}-drop in \sectref{sec:prodrop}, there are syntactic accounts of topic drop that analyze it in a very similar way.
They assume an empty pronoun \textit{pro} that is part of the lexicon, similar to the one that is assumed for \textit{pro}-drop languages \citep[e.g.,][]{cardinaletti1990, platzack1996, trutkowski2016, freywald2020}.
The traditional view is that Standard \ili{German} only allows for a non-referential expletive \is{Expletive} 3rd person \textit{pro} in subject function and in the nominative case \is{Nominative case} \citep[170--171]{sternefeld2015}, such as \ref{ex:pro.sternefeld} but, unlike \textit{pro}-drop languages such as Italian (see \sectref{sec:prodrop}),  not for a referential \textit{pro} with flexible person and case features and a theta role.

\exg.\label{ex:pro.sternefeld}weil \emph{pro} getanzt wird\\
because pro danced gets\\
`because there is dancing' \citep[170]{sternefeld2015}

To explain topic drop employing a \textit{pro}, we have to assume either a second \textit{pro} for \ili{German} or that the properties of the \ili{German} \textit{pro} are more flexible than usually described.%
%% Footnote
\footnote{In \ili{German} dialects such as Bavarian, we find cases of referential null subjects, \is{Null subject} which are often analyzed as \textit{pro} drop (see \sectref{sec:dialectal.null.subjects}).
Thus, at least for the dialects, a more flexible \textit{pro} is to be assumed anyway.}
%
Additionally, it is unclear whether this \textit{pro} is generally part of the lexicon or only in special text types \is{Text type} or registers (cf. the discussion of discourse-oriented vs. sentence-oriented registers in \sectref{sec:syntax.operator}).

If we additionally assume a \textit{pro} that can account for any occurrence of topic drop sketched in this book so far, it needs to be able to function at least as a referential and non-referential subject, an object, an adverbial, \is{Adverbial} and a part of prepositional adverbs with \textit{da}.
The last two cases in particular could be problematic since \citet[101]{platzack1996} assumes that \textit{pro} is an NP,%
%% Footnote
\footnote{\citet[101]{platzack1996} considers it an advantage of the \textit{pro}-analysis that it ensures that only nominal elements can be targeted by topic drop since \textit{pro} can only represent such elements.
What he considers an advantage, however, is criticized by \citet[38]{nygard2018}, who points out, 	with reference to \citet{mornsjo2002}, that in Swedish \il{Swedish} and Norwegian \il{Norwegian} also non-nominal adverbial \is{Adverbial} elements can be omitted.
The same is true for \ili{German}, as discussed in \sectref{sec:def.constituent}.}
%
but a \textit{pro} restricted to nominal elements does not apply to all cases of topic drop.%
%% Footnote
\footnote{In order to clarify this issue, it would be necessary to examine in detail, first, whether the \textit{pro} is in fact an NP, or  whether Chomsky's original definition of a ``pure pronominal'', an empty category with the features [+pronominal, -anaphor] \citep[81--82]{chomsky1982}, can be extended to other proforms that act as substitutes for adverbials,  \is{Adverbial} prepositional phrases, or prepositional adverbs. \is{Prepositional object}
Second, it needs to be clarified whether it is reasonable to assume that these categories have a nominal or at least pronominal part.
Cf. on the last point the remarks of \citet[162, footnote 6]{oppenrieder1991}, who contradicts \citeauthor{wunderlich1984}'s (\citeyear[88]{wunderlich1984}) claim according to which the pronominal part of prepositional adverbs is an NP.
}
%

In more recent literature, it is assumed that the \textit{pro} that represents topic drop is underspecified with respect to its person, number, and gender phi-features and receives the values from a matching corresponding discourse referent \citep[172]{freywald2020}.
\citet[172]{trutkowski2016} even assumes ``an internally structured \textit{pro}, i.e., an empty category that is of the category \textit{pro} and that contains the category \textit{pro}'', which is sufficiently underspecified to account for strict and sloppy readings and mismatches in syntactic identity \citep[173]{trutkowski2016}.
This conceptualization of \textit{pro} does not apply to topic drop of expletives, \is{Expletive} which is disregarded by \citet[150]{freywald2020} and only marginally considered by \citet[120--121]{trutkowski2016} (see also \sectref{sec:topicality.ness}).
The expletive \is{Expletive} \textit{es} is semantically empty, so it does not need a discourse referent to receive phi-features from.
Therefore, if we wanted to apply the \textit{pro}-analysis also to topic drop of non-referential constituents, we would need to assume two different \textit{pro}s, one for referential and one for non-referential constituents.
The latter, in turn, has similarities to the expletive \is{Expletive} \textit{pro} assumed for standard \ili{German} anyway.
Unlike the latter, however, it is restricted to the prefield in \ili{German}, like the referential \textit{pro}.
This results in the need for at least three different \textit{pro} types in \ili{German}.

To account for the prefield restriction of topic drop, \citet[94]{platzack1996} stipulates a feature A.
He argues that a marker [Repel A] can be assigned any phrase of a clause to force the correspondingly marked phrase to ``move to the first available node outside the highest A-position.''
According to his approach, the \textit{pro} representing the covert constituent is always marked [Repel A] and, thus, has to move to this position before spell out \citep[101]{platzack1996}.
The prefield restriction is hard-coded in the lexicon, so to speak.
In short, \citeg{platzack1996} account requires an additional feature that, stated in terms of Minimalism, can only be checked in [Spec, CP].
However, his solution does not explain the prefield restriction, i.e., there is no explanation for why this feature should exist.%
%% Footnote
\footnote{See also \citet[762--763]{sternefeld2009} for a critical discussion of topicalization as a feature-driven movement.}
%

\citet{trutkowski2016} and \citet{freywald2020}, on the other hand, assume that the prefield restriction of \textit{pro} results from it being only discourse accessible in this position and from the restriction of topic drop to topics that they assume (see \sectref{sec:topicality.ness} for counter-evidence).
Accordingly, \citet[170--172]{freywald2020} states that the ``topic drop \textit{pro}'' is restricted to the specifier of a ConTopP reserved for continuous topics, where it receives its corresponding features from discourse, as already outlined above.
Similarly, \citet[176]{trutkowski2016} formulates an identification condition for topic drop according to which ``[o]nly in the spec of the root-position an empty element is available for the identification by a discourse antecedent.'' \is{Antecedent}
She argues that the prefield restriction of topic drop is necessary to ensure that the antecedent \is{Antecedent} can be optimally accessed and that this is particularly important ``because a topic drop gap does not constitute a blind copy of its antecedent \is{Antecedent} but can accommodate semantic and syntactic variation (different interpretations, case mismatches)'' \citep[19]{trutkowski2016}.
However, this argumentation is weakened by the fact that other (more ``classic'') types of ellipsis such as sluicing \is{Sluicing} also allow for mismatches \citep{kroll.rudin2017,anand.etal2021} but are not positionally restricted to the left edge of an utterance.
Moreover, \citet{trutkowski2016} and \citet{freywald2020} can only explain the prefield restriction of referential and topical constituents.
Even in these cases, the question remains open as to why it should be precisely and exclusively the root or [Spec, ContTopP] position that enables discourse accessibility.
Overt pronouns, for instance, can also occur in the middle field \is{Middle field} and hearers can still establish the corresponding discourse reference (as shown in the trivial example \ref{ex:pronoun}).

\ex.\label{ex:pronoun}
\ag.Was ist mit Tino\textsubscript{i}?\\
what is with Tino\\
`What about Tino\textsubscript{i}?'
\bg.Am Montag hat er\textsubscript{i} gesagt, dass er\textsubscript{i} kommt.\\
on Monday has he said that he comes\\
`On Monday he\textsubscript{i} said that he\textsubscript{i} was coming.'

In summary, the \textit{pro}-analysis of topic drop follows similar existing analyses of \textit{pro}-drop languages by assuming a null pronoun \textit{pro} in the lexicon representing the omitted constituent.
However, since usually only a non-referential \textit{pro} is assumed for Standard \ili{German}, either another \textit{pro} must be postulated in the lexicon or the properties of the \ili{German} \textit{pro} must be adjusted.
Moreover, if we were to adopt \citeg{platzack1996} characterization of the \textit{pro} as an NP, the consequence would be that not all cases of topic drop could be explained by a \textit{pro}.
Finally, the prefield restriction can only be accounted for by postulated features or a discourse accessibility restriction that remains vague.
Thus, several additional assumptions are required for the \textit{pro}-analysis and still its explanatory force remains unsatisfactory.
\is{@\emph{pro}-analysis|)}

\subsection{PF-deletion analysis}\is{PF-deletion analysis|(}
The last major class of syntactic accounts to topic drop can be subsumed under the term \textit{PF-deletion}.
Instead of postulating a special null element, those accounts assume that utterances with topic drop only lack the spell-out of the prefield constituent while being otherwise (mainly) identical to the full form \citep{mornsjo2002,sigurdsson.maling2010,sigurdsson2011,nygard2018}:
``[T]he semantic and grammatical features of the phonetically non-realized element are present in Spec-CP, in order to feed the interpretation process. Lacking phonological features, this element cannot be spelled out. Consequently, the syntactic licensing of a phonetically non-realized element is identical to its visible alternative'' \citep[133--134]{mornsjo2002}.%
%% Footnote
\footnote{Note that \citet[66]{sigurdsson.maling2010} qualify this for their approach stating that ``[o]vert pronouns tend to be more specific or `bigger' than null-arguments in the sense that they express some additional properties like Focus \is{Focus} or Shifted Topic, not present in corresponding null-argument constructions.''}
%
This approach nicely captures \citeg{reis2000} claim that utterances with topic drop are functionally parallel to their overt counterparts.
It furthermore can straightforwardly account for the cases of omitted non-referential expletives \is{Expletive} that cannot be (easily) explained by the operator or \textit{pro}-analyses.

\citet{mornsjo2002} does not develop a detailed account of PF-deletion but suggests that it could be implemented using Distributed Morphology \citep{halle.marantz1993} and its concept of the late insertion of phonological features \citep[133]{mornsjo2002}.
\citet{nygard2018}  adopts \citeg{mornsjo2002} account of phonological non-realiza- tion, assuming that overt and covert elements are identical with respect to the syntax \citep[47]{nygard2018} but embeds it into a generative exoskeletal frame-based syntactic model.%
%% Footnote
\footnote{The idea of an exoskeletal frame-based syntax is that, unlike in Minimalism, the syntactic structure consists of abstract sentence frames which are ``generated independently of lexical items'' \citep[78]{nygard2018}.
This means that the lexical insertion takes place after the syntax and that the interpretation of the inserted lexical items with respect to their theta roles is determined by this frame (see \citeg{nygard2018} chapters 3 and 4, for details).
}
%

\citet{sigurdsson.maling2010} propose a sophisticated cross\hyp linguistic approach to null arguments \is{Argument omission|(} \citep[see also][]{sigurdsson2011}, arguing that any apparent differences between different types of null arguments result exclusively from restrictions at PF \citep[66]{sigurdsson.maling2010}.
Their approach also has similarities to the \textit{pro} \is{@\emph{pro}-analysis} accounts discussed in \sectref{sec:syntax.pro} because they consider null arguments to be pronouns and, in turn, pronominal arguments to be bundles of morpho-syntactic features for which spell-out at PF is principally optional \citep[68]{sigurdsson.maling2010}.
In their \textit{Context-Linking Generalization} \citep[61]{sigurdsson.maling2010} (\textit{C/Edge-Linking Generalization} in \citet[282]{sigurdsson2011}), they argue that any overt or covert pronominal argument must positively match a context-linking C-feature, which comprise at least topic, speaker, or hearer features located in the C-domain \citep[61]{sigurdsson.maling2010}.%
%% Footnote
\footnote{\citet[61]{sigurdsson.maling2010} refer back to \citet{sigurdsson2004b,sigurdsson2004a} by stating that ``the speaker/hearer features are referred to as the logophoric agent (speaker) and the logophoric patient (hearer), $\Lambda$A and $\Lambda$P for short.'' 
They sketch the features of the C-domain as follows: \is{Complementizer phrase|(}
%\vspace{-0.5\baselineskip}
\ex. [\textsubscript{CP}\,\dots\,\,Top\,\dots\,\,$\Lambda$A \dots\,\,$\Lambda$P\,\dots\,[\textsubscript{TP}\,\dots ~\citep[61]{sigurdsson.maling2010}

%\vspace{-0.5\baselineskip}
The context-linking is defined as a ``\,`transitive' matching relation (where A $\leftrightarrow$ B  reads `A is matched by B' or `B is interpreted in relation to A'):
%\vspace{-0.5\baselineskip}
\ex. Context $\leftrightarrow$ C-features $\leftrightarrow$ TP-internal elements'' \citep[61]{sigurdsson.maling2010}
%\vspace{-0.5\baselineskip}}
}
To account for the prefield restriction of Germanic topic drop, they propose a clause-internal operative \textit{Empty Left Edge Condition} \citep[62]{sigurdsson.maling2010}: 
``The left edge of a clause containing a silent referential argument must be phonetically empty.''
That is, topic drop is only possible if [Spec, CP] is phonetically empty.
This is explained as an intervention effect on feature matching at PF.
Overt material in [Spec, CP] blocks the null argument to raise and match the context-linking feature in the C-domain, \is{Complementizer phrase} as illustrated in \ref{ex:empty.left.edge}.

However, it remains unclear why an occupied [Spec, CP] blocks the context-linking of null arguments, given that overt pronouns can be easily context-linked in the middle field. \is{Middle field}
Even if one explains this restriction to be operative at PF rather than in the syntactic system, still the origin of the PF rule remains unclear.
\citet[294]{sigurdsson2011} states that it ``seems more promising to assume that it is movement (internal Merge) of more than one constituent across the finite verb in C that is blocked (for reasons that remain to be explicated [...]).''
Additionally, the account by \citet{sigurdsson.maling2010} in its outlined form seems to be restricted to referential pronominal arguments, which can be context-linked at all to the discussed topic, speaker, and hearer features.
To account for non-referential expletive \is{Expletive} subjects, linking them with a topic feature does not seem to be sensible (see again \sectref{sec:topicality.ness}).
We would probably either have to assume a new C-feature \is{Complementizer phrase|)} for expletives \is{Expletive} or, and this seems to be intuitively more plausible, to exempt expletives \is{Expletive} from context-linking since they are in fact not linked to the context in any form.

\ex.\label{ex:empty.left.edge}
%\vspace{-2.5em}
%\hspace{6em}
\begin{forest}
for tree={s sep*=4, parent anchor=south, child anchor=north}
[CP, baseline
	[\phantom{C}\dots\phantom{C}]
	[CP
		[\{Top\}]
		[CP
			[{$[\textsc{spec}]$}]
			[CP
				[C]
				[TP
					[$\varnothing$, name = src]		
					[\dots]
				]
			]
		]
	]
]
\draw[latex-latex] (2.95,-5.5) -- (2.95, -6) -- (0, -6) -- (0,-5.5);
\node[] at (1.5, -6.3) {*};
\end{forest}\\
\phantom{.}\hfill\citep[63]{sigurdsson.maling2010}
%\vspace{0.5\baselineskip}

This shows that the main challenge for the approach remains to adequately explain the prefield restriction of topic drop.
Either we could adopt \citeg{sigurdsson.maling2010} approach just described, but this implies that topicality has a central role to play, or we could solve it more technically, similarly to the \textit{pro} approach.
Then, we would have to assume a feature that sits in a [Spec, CP] position that is either not c-commanded \is{C-command} sentence-internally by a potential identifier or the highest [Spec, CP] of a root clause, and that allows for the phonological non-realization of the corresponding constituent only in this position.
In the latter case, however, as discussed above, the question of what would be an adequate motivation for such a feature still remains open. \is{Argument omission|)} \is{PF-deletion analysis|)}

\subsection{Summary: syntactic analysis}
In the following, I summarize the three syntactic approaches to a generative description of topic drop.
In the operator approach, \is{Operator analysis} the operator can be semantically conceptualized as a lambda abstraction that triggers the retrieval of a salient \is{Salience} discourse referent in order to pragmatically enrich the lambda abstraction to a proposition.
In this way, the operator approach provides a discourse-pragmatic explanation for the prefield restriction of topic drop.
The two major drawbacks of such an analysis are the following:
First, the assumption of an operator as a lambda abstraction leads to a type problem of declaratives.
It would have to be assumed that in addition to declaratives of type $t$, there are also declaratives of type $\langle e,t\rangle$.
This conflicts with standard assumptions concerning the semantics of declaratives (though it recalls cases of pragmatic or conceptual enrichment as argued for in the case of fragments \is{Fragment} by \cite{stainton2006} and others).
Second, an operator approach cannot be easily extended to non-referential constituents targeted by topic drop, which precisely have no linguistic antecedent or extralinguistic referent. \is{Antecedent}
Besides these two obvious disadvantages, a potential advantage of the operator account is that it provides a similar analysis of the related ellipsis types topic drop, subject gaps, \is{Subject gap} null copulas, and null articles. \is{Article omission}\is{Copula omission}
However, there are also obvious differences between these phenomena, which call into question whether a joint consideration is desirable at all.
Subject gaps, \is{Subject gap} as the name suggests, are restricted to subjects, unlike topic drop, and null copulas and null articles \is{Article omission} \is{Copula omission} are limited to headlines as a text type. \is{Text type} \is{Operator analysis}

\is{@\emph{pro}-analysis|(}
The \textit{pro} approach assumes the existence of a null element \textit{pro} that is underspecified with respect to its features and fixed in the lexicon.
This \textit{pro} is argued to receive its feature values from a discourse antecedent, \is{Antecedent} a mechanism that, to put it simply, syntactically anchors the felicity condition of recoverability \is{Recoverability} (see Chapter \ref{ch:recover}).
Again, the disadvantage arises that the cases of omitted non-referential constituents cannot be explained.
However, this could be solved by assuming another \textit{pro} for these cases, but this again complicates the lexicon and the grammatical system.
Another drawback of the \textit{pro}-analysis is that the prefield restriction can only be integrated into the proposal by assuming a special feature, i.e., mainly by stipulation. \is{@\emph{pro}-analysis|)}

Finally, the PF-deletion \is{PF-deletion analysis|(} approach has the advantage that relatively few additional assumptions have to be made.
We could say that, other things being equal, it is the null hypothesis for analyzing topic drop.
The lexicon and the grammatical system are not complicated, because this approach does without special null elements.
Since topic drop utterances are grammatically and functionally parallel to their full counterparts, it can also easily account for expletives. \is{Expletive}
However, it also has the disadvantage that the prefield restriction (at least of non-referential topic drop cases) does not follow directly from the theory but requires postulated features.

Despite this drawback, the PF-deletion approach seems to harmonize best with the empirical results I obtained.
Since it can explain as much as the other two approaches (or even more, i.e., the omission of expletives) \is{Expletive|)} without the assumption of additional null elements, I argue for giving it preference following Occam's Razor. \is{PF-deletion analysis|)}
\is{Prefield|)}

In the next chapter, I turn to recoverability, which has also been discussed as a factor impacting the licensing of topic drop.
