%*****************************************
\chapter{Definition and typological perspective}\label{ch:definition}
%*****************************************

The subject of this book is an ellipsis type in \ili{German|(}, where, roughly speaking, an element that can function at least as a subject \ref{ex:TD.subj} or as an object \ref{ex:TD.obj} is omitted from the preverbal, clause-initial position of a verb-second (V2) declarative clause.

\ex.\label{ex:TD}
\ag.\label{ex:TD.subj}$\Delta$ Hab das schon gemacht.\\
I have that already done\\
`(I) have done that already.'
\bg.\label{ex:TD.obj}$\Delta$ Hab ich schon gemacht.\\
that have I already done\\
`I have done (that) already.'

This phenomenon is known by a variety of terms, often implying a theoretical positioning on the part of the authors who invented or use them.
Initially, \citet{reis1982} called it \textit{ellipsis in telegraphese}, referring to the text type telegrams for which it is a characteristic. 
In the same year, \citet{ross1982} discussed it as \textit{pro}(\textit{noun}) \textit{zap} \citep[see also][]{fries1988} thereby characterizing it as pronoun omission.
Terms such as \textit{Vorfeld-Ellipse} (`prefield ellipsis') \is{Prefield}\citep{reis2000, frick2017} and \textit{Vorfeld-Analepse}%
% Footnote
\footnote{\citet{zifonun.etal1997} attribute the term \textit{Analepse} (`analepsis') to \citet{blatz1896} and use it to refer to elliptical phenomena, which involve a linguistic antecedent \is{Antecedent} as opposed to ellipsis where the omission can be reconstructed through the extralinguistic context \citep[569]{zifonun.etal1997}.\is{Antecedent}}
%
(`prefield analepsis') \citep{hoffmann1999} emphasize the positional restriction of the omission to the prefield (the preverbal position in V2 clauses, see below).
The terms (\textit{uneigentliche}) \textit{Verbspitzenstellung} (`improper verb top positioning') \citep{auer1993,sandig2000,guenthner2006,imo2013,ruppenhofer2018} or \textit{uneigentliche} \textit{Verberststellung} (`improper verb first positioning') \citep[§493--494]{duden2022}, which are mostly popular in spoken language research, describe the surface form of the phenomenon with the finite verb in clause-initial position.
However, the addition of `improper' suggests that this surface structure does not correspond to the real underlying structure.

In recent literature, the most popular term is \textit{topic drop} \citep{thrainson.hjartardottir1986,sigurdsson1989,haegeman1997,jaensch2005,ackema.neeleman2007,erteschik-shir2007,haegeman2007,reich2011,trutkowski2011,volodina2011,schalowski2015,trutkowski2016,helmer2016,helmer2017,ruppenhofer2018,freywald2020}.%
% Footnote
\footnote{The term seems to have been first used by \citet{thrainson.hjartardottir1986} in the title of their paper \textit{Pro-drop, topic-drop...: where do old and modern Icelandic fit in?}, presumably as an analogy to \textit{pro-drop}.
In the text itself, however, Thráinson and Hjartardóttir speak of \textit{zero topics}.
To the best of my knowledge, it is \citet[145]{sigurdsson1989} who first explicitly states that ``[t]he German pronoun zap [...] is actually a 'topic-drop'.''}
It suggests that ellipsis targets the topic of a clause and is therefore related to similar terms such as \textit{zero topic} \citep{huang1984}, \textit{null topic} or \textit{0-topic} \citep{fries1988, cardinaletti1990, eckert1998.diss}, \textit{Topik-Wegfall} (`topic omission') \citep{klein1993}, \textit{Topik-Tilgung} (`topic canceling') \citep{sternefeld1985}, and \textit{Themaanalepse} (`theme analepsis') \citep{zifonun.etal1997}.
As the term \textit{topic drop} is widely used, especially in international research, I use it in this book as a temporary label for the ellipsis type that is being investigated, without committing to its theoretical implications.
However, I also show that \textit{topic drop} may not be the most accurate term and propose that \textit{prefield ellipsis} would be a more suitable designation.\is{Prefield}

\section{Definition}\label{sec:definition}
For the ellipsis type that I henceforth term \textit{topic drop}, I propose the following preliminary definition:

\begin{theorem}\label{def:topic1} \is{Prefield}
Topic drop is the omission of a constituent from the prefield of declarative verb-second (V2) clauses.
\end{theorem}

\noindent
It is also a characteristic of topic drop in German, although not part of its definition, that it is mostly restricted to colloquial spoken language and to conceptually spoken \citep{koch.oesterreicher1985} text types. \is{Text type}
In the following, I briefly discuss the individual aspects of the provided definition, as well as the role of text type.
Most of the points are examined in greater depth in the course of this book.
I refer to the relevant chapters and sections as needed.
At the end of Chapter \ref{ch:topicality}, I present a refined definition of topic drop that captures the prefield restriction more precisely.

\subsection{Omission}\label{sec:def.omission}
By stating that topic drop involves the omission of a constituent, I take the view that utterances with topic drop are syntactically incomplete, i.e., they are cases of ellipsis \citep[e.g.,][]{reich2011,reich2018}.
I assume that the preverbal position, which can be described as the prefield \is{Prefield} in the topological field model \is{Topological field model} \citep{drach1937, hohle1986, wollstein2018} or as the specifier of the complementizer phrase \is{Complementizer phrase} [Spec, CP] in generative grammar \citep{thiersch1978, denbesten1983}, is still present in the clause but contains a covert or phonologically empty element instead of an overt one (see \sectref{sec:syntax} for different options how to represent this).
\citet[218]{reis2000} points out that, despite their syntactic incompleteness, utterances with topic drop are generally ``functional parallel'' to complete V2 structures.
\citet{helmer2016}, however, emphasizes functional differences between utterances with an overt prefield constituent and utterances with a covert prefield constituent.
I  address the question of when topic drop is actually used in the second part of this book.

\subsection{Prefield restriction}\label{sec:def.prefield}\is{Prefield|(}
Part of the above definition is the main licensing condition of topic drop, namely, its generally assumed restriction to the prefield \citep[e.g.,][]{fries1988,auer1993,zifonun.etal1997,reis2000,frick2017,freywald2020}, i.e., the position immediately to the left of the finite verb in declarative V2 clauses \ref{ex:MF}.
Omissions in the middle field \is{Middle field} are usually considered to be impossible \citep[but cf.,][and see \sectref{sec:topicality.prefield}]{helmer2016}.

The prefield restriction implies that topic drop can only occur in clauses that have a prefield.
This means that topic drop is possible in declarative main clauses but not in subordinate clauses with verb-final word order \ref{ex:subord}, or interrogative and imperative \is{Imperative} clauses with V1 word order.
It is also not possible to omit operator phrases from the prefield that have a specific function, such as relative or interrogative phrases, i.e., the question word in \textit{wh}-questions with V2 word \is{V2 word order} order cannot be omitted,%
%% Footnote
\footnote{
While it could be argued that \textit{wh}-words asking for adjuncts \is{Adjunct} (e.g., \textit{when}, \textit{where}, \textit{why}) cannot be omitted because they usually provide information about what the question is asking for and cannot be recovered \is{Recoverability} if omitted, this is not the case for \textit{who} or \textit{what}.
Those \textit{wh}-words that ask for arguments \is{Argument|(} can in fact be recovered from the sentence structure.
For instance, in \ref{ex:wh} it would generally be easy to infer that the omitted \textit{wh}-word needs to be the one asking for the subject, i.e., \textit{wer} (`who').
%\vspace{-0.5\baselineskip}
\exg.*$\Delta$ Hat angerufen?\label{ex:wh}\\
who has called\\
`(Who) has called?'

%\vspace{-0.5\baselineskip}
Since topic drop is still not possible in this case, it seems reasonable to assume the proposed ban on topic drop of operator phrases, as suggested above.
I thank Ingo Reich for this suggestion.
}
%
see \ref{ex:q}.%
% Footnote
\footnote{Note that the topic drop example is identical to the polar question \textit{Have you done that already?} but does not convey the intended meaning of asking for the time, place, or reason.}
%
The acceptability of topic drop in potentially embedded \is{Embedding} V2 clauses \ref{ex:emb} is disputed.
While \citet{jaensch2005} argues that it may be acceptable for some speakers and in restricted contexts, \citet{cardinaletti1990} rejects it altogether.
In \sectref{sec:highest}, I investigate this case in detail with experiment \ref*{exp:embedded}; the results of which motivate a specification of the prefield restriction.

From the observation that the prefield usually contains exactly one single constituent,%
%% Footnote
\footnote{An exception to this rule are well-known cases of so-called \textit{mehrfache Vorfeldbesetzung} (`multiple fronting') \citep{muller2005}, such as \ref{ex:multiple.fronting}, where two constituents occur in the prefield.
Since these cases are exceedingly rare and topic drop does not seem to be able to target both constituents simultaneously, I do not discuss them here.

%\vspace{-0.5\baselineskip}
\ex.\label{ex:multiple.fronting}
\ag. Alle Träume gleichzeitig lassen sich nur selten verwirklichen.\\
all dreams concurrently let themselves only seldom realize\\
`Only rarely can all dreams be realized concurrently.' \citep[299]{mueller2005}
\bg. Die Kinder nach Stuttgart sollst du bringen.\\
the kids to Stuttgart shall you bring\\
`You are to bring the kids to Stuttgart.' \citep[81]{engel1970}

%\vspace{-1.5\baselineskip}
}
it follows that the omitted element has to be a constituent%
%% Footnote
\footnote{In German, the constituent status of a word or phrase is often tested by inserting it into the prefield (the prefield test) \citep[33]{pittner.berman2021}.}
%
and that topic drop cannot target more than one constituent per clause \ref{ex:two} \citep{fries1988}.
In Chapter \ref{ch:topicality}, I examine the prefield restriction of topic drop in detail.

\ex.\label{ex:TDlimits}
\ag.*\label{ex:MF}Das hab $\Delta$ schon gemacht.\\
that have I already made\\
`(I) have done that already.'
\bg.*\label{ex:subord}Tino weiß, dass $\Delta$ das schon gemacht hab.\\
Tino knows that I that already made have\\
`Tino knows that (I) have done that already.'
\bg.*\label{ex:q}$\Delta$ hast du das schon gemacht?\\
{when/where/why/\dots} have you.\textsc{sg} that already made\\
`(When/where/why/\dots) have you done that already?'
\bg.?\label{ex:emb}Tino weiß, $\Delta$ hab das schon gemacht.\\
Tino knows I have that already made\\
`Tino knows (I) have done that already.'
\bg.*\label{ex:two}$\Delta$ Hab $\Delta$ schon gemacht.\\
I have that already made\\
`(I) have (that) already done.'

\is{Prefield|)}\il{German|)}

\subsection{Omitted constituent}\label{sec:def.constituent}
I stated in the definition above that topic drop is the omission of a constituent.
In the literature, this constituent is often equated with a pronoun \citep{klein1993,jaensch2005,reich2011,volodina.onea2012,duden2016}.
The Duden grammar specifies that topic drop is the omission of personal pronouns in subject function and of weakly stressed or unstressed demonstrative pronouns in subject, object, or predicative function (\cite[§1378]{duden2016}, \cite[§35]{duden2022}).%
%% Footnote
\footnote{The 2022 Duden adds that also the semantically empty expletive \is{Expletive} subject \textit{es} (`it') can be omitted \citep[§35]{duden2022}.
See also \citet[§268, §493--494]{duden2022} on topic drop.}
%
For instance, \citet[214]{volodina.onea2012} explicitly state that verb arguments \is{Argument} can only be omitted as definite pronouns because this ensures, in their view, that the referent is unambiguous and sufficiently activated in the discourse.
I discuss this restriction in detail in Chapter \ref{ch:recover} on recoverability.
While recoverability \is{Recoverability} is indeed an important condition not only for the omission of (referential) constituents but also for the resolution of pronouns, it is questionable whether topic drop can only target pronouns, as \citet{volodina.onea2012} postulate.

In any case, it is difficult to determine what exactly is omitted in topic drop.
In fact, we can only resolve the ellipsis by inserting a plausible prefield constituent depending on a possible antecedent. \is{Antecedent}
This prefield constituent is usually a pronoun.
However, there are cases where it is unclear which pronoun, if any, should be inserted.
For example, in \ref{ex:td.proper.name}, taken from the advertisement subcorpus of the fragment corpus FraC (\cite{horch.reich2017}, see \sectref{sec:corpus.frac} for details), the grammatical gender of the brand \textit{lavera} is unclear, which makes it hard to pick a pronoun to substitute it.

\exg.\label{ex:td.proper.name}lavera. $\Delta$ {Wirkt\footnotemark} natürlich schön.\\
lavera it/she/he works naturally beautiful\\
`lavera. (It/She/He) works naturally beautifully.' [FraC P1004--P1005]

%% Footnote
\footnotetext{The slogan can be seen as a play on words, using the double meaning of \textit{wirken} as `to appear' and `to work'.
As this is irrelevant to the gender discussion, I only focus on the `work'-meaning in the glossing and translation.}
%

\noindent Finally, a part of the literature argues that not only subjects and objects can be omitted but also temporal and local adverbials \is{Adverbial|(} in the form of what \citet[289, footnote 33]{sigurdsson2011} terms ``anaphoric light adverbials''. 
They usually must be reconstructed as the proforms \textit{da} (`there'/`here') or \textit{dann} (`then') instead of a ``classic'' pronoun referring to a determiner phrase (DP) \is{Determiner phrase} \citep{fries1988, sigurdsson2011, schalowski2015}.
In sum, the claim that only pronouns can be targeted by topic drop seems to be too strong.

The case of omitted adverbials leads us directly to the second point regarding the omitted constituent, that is, its syntactic function.
On the one hand, there are authors who assume that topic drop can only target verb arguments, \is{Argument} that is, only subjects and objects \citep[e.g.,][]{cardinaletti1990,auer1993,jaensch2005,volodina.onea2012, imo2013}.%
%% Footnote
\footnote{To my knowledge, there has been no discussion in the literature as to whether free datives \is{Dative case} \citep{hole2014}, which are not verb arguments, \is{Argument}can also be targeted by topic drop.
In the authentic example with a free dative in the prefield in \ref{ex:free.dative}, taken from the discussion section of a blog post, for instance, I would consider an omission of \textit{der} to be at least marked, if not impossible.
However, it is already controversial for dative objects whether they can be omitted (see \sectref{sec:usage.function.theory}).
\exg.\label{ex:free.dative}Also meine Gini, ein labbi ist 16 geworden, ?*(der) backe ich jedes Jahr eine Torte!\\
so my Gini a labrador.\textsc{dim} is 16 become \phantom{?*}her.\textsc{dat} bake I every year a cake\\
`So my Gini, a labrador turned 16, I bake (her) a cake every year!' (\url{https://www.fashion-kitchen.com/2015/09/hackfleisch-wurstchen-torte-fur-den.html}, visited on 01/02/2025)

%\vspace{-1\baselineskip}
}
However, there is no consensus on whether, first, only referential subjects or both referential and non-referential subjects can be omitted (see the discussion in \sectref{sec:topicality.ness}), and, second, whether all types of objects can be omitted (equally well) (see \sectref{sec:usage.function.theory}).
On the other hand, the view that only subjects and objects can be targeted by topic drop is challenged by \citet{fries1988} and \citet{schalowski2015}, mentioned above, who argue that adjuncts, \is{Adjunct|(} for instance in the form of adverbials, can also be omitted.
A corresponding example from the dialogue subcorpus%
%% Footnote
\footnote{This subcorpus contains data from the Tübinger Baumbank des Deutschen/Spontansprache TüBa-D/S (`Tübingen treebank of spoken German') \citep{hinrichs.etal2000}, which in turn stem from the Verbmobil project \citep{burger.etal2000}.}
%
of the FraC is given in \ref{ex:td.adverb}.
It also shows that resolving an omitted adverbial \is{Adverbial|)} is often more difficult and less clear than resolving an omitted verb argument. \is{Argument}
In this particular case, it is unclear whether a temporal element (replaceable by \textit{dann} (`then')) or a local element (replaceable by \textit{da} (`there')) has been omitted.

\ex.\label{ex:td.adverb}
\ag.und dann sind wir eben um zwölf Uhr spätestens in der Filiale. [...] ja, das ist sehr schön, genau zum zum [sic!] Mittagessen. [...]\\
and then are we \textsc{part} at twelve o'clock at.the.latest in the branch {} yes that is very nice exactly for.the for.the {} lunch {}\\
`And then we are just at twelve o'clock at the latest in the branch. [...] Yes, that's very nice, right at lunch time.'
\bg.ja, $\Delta$ können wir mit den Kollegen Mittagessen gehen\\
yes then/there can we with the colleagues lunch go\\
`Yes, (then/there) we can have lunch with the colleagues.' [FraC D348--D353]

A case similar to the omission of adverbs is the omission of \textit{da} as part of prepositional adverbs such as \textit{damit}, \textit{darauf}, \textit{davon}, \textit{daran} etc.
In colloquial speech in particular, \textit{da} can be separated from the preposition, placed in the prefield, and omitted from there, as shown in example \ref{ex:prepadverb} \citep[see][for details]{fries1988,reis2000}.%
%% Footnote
\footnote{Prepositional adverbs are generally formed by combining \textit{da} and the corresponding preposition into one lexeme (i.e., \textit{da} + \textit{mit} = \textit{damit}).
However, if the preposition starts with a vowel, an additional \textit{r} needs to be inserted for phonological reasons (i.e., \textit{da} + \textit{r} + \textit{an} = \textit{daran}).
If these cases are split, the remaining preposition does not only keep this \textit{r} but also receives an additional \textit{d} in the word-initial position, i.e., \textit{an} becomes \textit{ran} and then \textit{dran}, as shown in example \ref{ex:prepadverb}.}
%

\exg.\label{ex:prepadverb}ja, $\Delta$ hab ich nur gerade nicht dran gedacht.\\
yes there have I only now not on thought\\
`Yes, I just haven't thought about it right now.' [FraC S126]

While for prepositional adverbs the \textit{da} is relatively easy to recover, \is{Recoverability} for adverbs, as mentioned above, ellipsis resolution is often less straightforward.
For such adjuncts, \is{Adjunct|)} the question arises of where exactly to draw the line between a structure with omission and a possibly genuine V1 structure (see the discussion of V1 declaratives \is{V1 declarative} in \sectref{sec:def.v1}).
Therefore, I focus only on topic drop of verb arguments \is{Argument|)} in this book and leave the omission of adjuncts \is{Adjunct} and the placeholder \textit{es} to future research.
I mainly consider subjects, both referential and non-referential ones, as well as objects and, very marginally, predicatives.

\subsection{Modality, register, and text type}\label{sec:def.texttype} \is{Text type|(}
Another property of topic drop that is also often discussed in the literature is its frequent occurrence in certain modalities, registers \citep{halliday1978, biber1994}, or text types.
However, unlike null articles \is{Article omission} or null copulas, \is{Copula omission} which are usually ungrammatical outside the text type headlines, topic drop can occur in several registers and text types, even if it is sometimes more, sometimes less marked, or appropriate.
Therefore, this property of topic drop should not be considered as a defining criterion or a licensing condition.

It is important to note that the term \textit{text type} is a problematic one, not least because it is often not clearly distinguished from similar concepts.
For instance, in German, \textit{Textsorte}, \textit{Textklasse}, \textit{Texttyp}, and \textit{Textart} (all translatable as `text type') are often used interchangeably (\cite[see, e.g.,][65]{gansel.juergens2009} and, more broadly, the many contributions in \cite{habscheid2011}).
In this book, I try to use the term \textit{text type} in as general and theory-neutral a way as possible.
For instance, I use it to refer to the various subcorpora in the FraC, although, for at least some of them, it may be more reasonable to assume that they contain different (more fine-grained) text types (such as the chat subcorpus, which contains both group chats in a chat room and one-on-one chats), while others would be better classified as \textit{forms of communication} (see, for example, \cite{durscheid2016}).

In the literature, it is often stated that topic drop is a phenomenon of spoken German \citep{huang1984,cardinaletti1990,poitou1993,jaensch2005,dittmann.etal2007,volodina2011}, and even that it is exclusively restricted to spoken language \citep[219]{volodina.onea2012}.
This restriction is occasionally further specified to mean colloquial or informal spoken language \citep{huang1984,poitou1993,jaensch2005}.
\citet[219]{volodina.onea2012} argue that the more normative a text type is, the less frequent or less acceptable topic drop is.
According to both authors, topic drop occurs most often in spoken informal discourse, which is characterized by ``communicative proximity'' (intimacy and a high degree of familiarity of the communication partners, referential proximity, etc.) \citep[221]{volodina.onea2012}.

\largerpage[-2]
Here, \citet{volodina.onea2012} refer to the influential concept of ``language of immediacy and distance'' developed by \citet{koch.oesterreicher1985}.
\citet[23]{koch.oesterreicher1985} propose the term \textit{konzeptionelle Mündlichkeit} (`conceptual orality') to refer to text types that, regardless of whether they are phonically mediated, exhibit properties of the language of immediacy, such as the previously mentioned intimacy and high degree of familiarity of the communication partners, referential proximity, but also dialogue, face-to-face situation, affectivity, etc.
They assume a continuum between (conceptually) spoken and written texts and classify a (written) diary entry as more conceptually spoken than a (spoken) job interview \citep[18]{koch.oesterreicher1985}.
\citet[272]{volodina2011} and \citet[221]{volodina.onea2012} argue that topic drop occurs frequently in text types that they classify as conceptually spoken, for example, private letters, emails, chats, diaries, and chronicles (see \cite{ruppenhofer2018} and \sectref{sec:corpus.texttype} for empirical support).
The restriction to these and other text types is also mentioned in much of the remaining literature but without reference to \citeg{koch.oesterreicher1985} concept of conceptual orality. 
For instance, \citet[27]{fries1988} mentions telegrams, private letters, diaries, certain types of conversations, certain literary texts such as narratives, (audio) dramas, or speech bubbles in comics.
\citet[188]{trutkowski2016} talks about ``electronic [...] registers'', \citet[153, footnote 6]{imo2014} about ``computer-mediated communication such as chat, email or SMS communication''.
In particular, for the latter and their successors instant messages, topic drop is frequently described as a characteristic \citep{androutsopoulos.schmidt2002,doring2002,dittmann.etal2007,frick2017}. 
According to \citet[236]{frick2017}, speakers should be even more likely to use topic drop in instant messages than in text messages.
She argues that the permanent visibility of the precontext, as well as the displayed information of whether the interlocutor is online or typing enforces the joint context orientation and, therefore, facilitates the recovery \is{Recoverability} of omitted elements with linguistic or situational antecedents \is{Antecedent}(see also Chapter \ref{ch:recover}).

\citet[24--25]{dittmann.etal2007} propose to sharply separate topic drop in text messages, which they consider to be an instance of telegraphese, from topic drop in spoken language.
They refer to \citet{auer1993} and argue that the topic drop function of condensing adjacent utterances that he proposes (see also the corresponding discussions in my Sections \ref{sec:pusage.cohesion} and \ref{sec:pusage.effects}) is not possible turn-initially, whereas topic drop in text messages can (and does) occur in this position.
Since this book is mainly limited to topic drop in text and instant messages, I cannot contribute to an empirical differentiation of topic drop across text types. 
Still, it seems plausible to me that at least some of the effects I found for topic drop in text and instant messages also hold for other text types.
This hypothesis is supported by the results of my experiments \ref*{exp:1sg.2sg} and \ref*{exp:1sg.2sg.spoken}, which suggest that the difference between 1st and 2nd person singular in instant messages and spoken dialogues is equally irrelevant.

\subsection{Topic drop and text type in the fragment corpus FraC}\label{sec:corpus.texttype}
\largerpage[-2]
\is{Corpus|(}
\begin{figure}
\centering
\includegraphics[scale=1]{Korpusplots/Bar_Texttypes.pdf}
\caption[Proportion of topic drop across text types in the FraC (rounded to integers)]{Proportion of topic drop across text types in the FraC (rounded to integers)}
\label{fig:frac.texttypes}
\end{figure}

%\vspace{-0.5\baselineskip}
At this point, it seems reasonable to anticipate my corpus study (see \sectref{sec:corpus} for an overview), which also revealed the typical occurrence of topic drop in certain text types.
The FraC consists of 17 different text types. 
As Figure \ref{fig:frac.texttypes}%
%% Footnote
\footnote{All bar plots in this book were created using the package ggplot2 \citep{wickham2016} in R \citep{rcoreteam2021}.}
%
and Table \ref{tab:fraC.texttypes} show, the distribution of full forms and instances of topic drop is very different across these text types.
Two parameters are relevant here: (i) the omission rate by which the bars in Figure \ref{fig:frac.texttypes} are arranged and (ii) the number of instances of topic drop by which Table \ref{tab:fraC.texttypes} is ordered.
Text messages are far ahead in both parameters.
The text message subcorpus of the FraC contains the most instances of topic drop with 353 and has the highest omission rate of almost 64\%.
In news articles,%
% Footnote
\footnote{The following example from a news article published online shows that topic drop is not impossible even in this text type.
Possibly the more informal online setting and the lighter topic -- sports, more precisely soccer -- favor the occurrence of topic drop.

%\vspace{-0.5\baselineskip}
\exg.,,Ein Gegentor lag nicht in der Luft'', sagte Thomas Müller bei ``Amazon Prime''. ,,Ich weiß nicht, was ich sagen soll.'' $\Delta$ Hat man auch noch nicht oft erlebt.\\
\phantom{,,}a goal.against laid not in the air said Thomas Müller at Amazon Prime \phantom{,,}I know not what I say shall that has one also yet not often experienced\\
`\,`A goal against wasn't in the air,' Thomas Müller said on `Amazon Prime'. `I don't know what to say.' (That) hasn't been experienced very often either.' (tagesschau.de, 04/13/2022, \url{https://www.tagesschau.de/sport/sportschau/bayern-villareal-103.html}, visited on 03/13/2023)
%\vspace{-1.5\baselineskip}}%
}
legal texts, and weather reports, there are no instances of topic drop.


\begin{table}
\centering
\caption[Distribution of topic drop and full forms per text type in the FraC]{Distribution of topic drop and full forms across text types in the FraC, ordered by the number of topic drop occurrences}
\begin{tabular}{lrrrr}
\lsptoprule
Text type & Full form & Topic drop & Total & Omission rate\\
\midrule
Text message & $201$ & $353$ & $554$ & $63.72\%$\\
Classified ad & $150$ & $91$ & $241$ & $37.76\%$\\
Dialogue & $431$ & $85$ & $516$ & $16.47\%$\\
Online chat & $280$ & $74$ & $354$ & $20.90\%$\\
Tweet & $121$ & $66$ & $187$ & $35.29\%$ \\
Ad & $63$ & $49$ & $112$ & $43.75\%$ \\
Blog & $282$ & $44$ & $326$ & $13.50\%$\\
Email & $386$ & $38$ & $424$ & $8.96\%$\\
Radio transcript & $315$ & $33$ & $348$ & $9.48\%$\\
Interview & $368$ & $23$ & $391$ & $5.88\%$\\
User manual & $51$ & $9$ & $60$ & $15.00\%$\\
Opinion piece & $220$ & $4$ & $224$ & $1.79\%$\\
Recipe & $12$ & $3$ & $15$ & $20.00\%$\\
Headline & $36$ & $1$ & $37$ & $2.70\%$\\
News article & $165$ & $0$ & $165$ & $0.00\%$\\ 
Legal text & $67$ & $0$ & $67$ & $0.00\%$\\
Weather report & $63$ & $0$ & $63$ & $0.00\%$\\
\lspbottomrule
\end{tabular}
\label{tab:fraC.texttypes}
\end{table}

What we can see in Table \ref{tab:fraC.texttypes} and Figure \ref{fig:frac.texttypes} is only partially consistent with what is assumed about the role of text type for topic drop in the literature.
Two questions arise from this discrepancy:
first, why is topic drop so common in text messages, and, second, why is it not more common in other text types, especially spoken ones?
While an exhaustive answer to both questions must be left to future research, it is nevertheless permissible to present some initial thoughts here.

Regarding the first question, it is often discussed in the literature whether the frequent use of topic drop in text messages is a consequence of their technically imposed character limitation (e.g., \cite[100]{doring2002}, \cite[7--8]{durscheid.brommer2009}) and thus a mainly economic phenomenon (see \sectref{sec:pusage.economy}).
This assumption is questioned by the fact that omitting a personal pronoun has only little ``savings potential'' \citep[108]{doring2002} and by the observation of \citet[103--104]{doring2002} that text message writers usually do not fully exploit the character limit and even fall well below it (see also \cite[172]{thurlow.poff2013} for the same observation in \ili{English} and \cite[331]{hardafsegerstad2005} for Swedish). \il{Swedish}
In other words, the senders would still have enough space to make the prefield constituent explicit, and yet they do not do so.
The purely brevity-based explanation falls short here \citep[see also][45]{dittmann.etal2007}, at least synchronically.
It might be possible, however, that topic drop was at least originally motivated by brevity and has become a conventionalized pattern over time.
As an alternative explanation, \citet[172]{thurlow.poff2013} suggest that ``[t]he length and abbreviated linguistic forms of texts would seem to be more a function of the need for speed, ease of typing and, perhaps, other symbolic and pragmatic concerns, such as gender identity performance or a preference for more dialogic exchanges.''

This idea is supported by a further observation from the literature according to which the linguistic and stylistic devices found in text messages, including topic drop, are similar to those of other text types.
On the one hand, these are text types that have existed for a longer time, such as notes \citep[173]{thurlow.poff2013} and telegrams \citep{reis1982, barton1998, dittmann.etal2007} and for which brevity and conventionalization should also play a role.
Topic drop has already been discussed extensively in research in the context of telegrams under the term \textit{telegraphese}, especially in the work by Jürgen Tesak and colleagues \citep[e.g.,][]{tesak.dittmann1991,tesak.etal1995, tesak.niemi1997}.
On the other hand, \citet[230--237]{frick2017} and also \citet{stark.meier2017} have found that topic drop continues to be used in communication via instant messaging services such as WhatsApp, which can in some ways be considered the successor of text messaging (for similarities and differences, see \cite{durscheid.frick2014}).
These text types are likewise often characterized by demands of speed and simplification.
Future research on the role of text type for topic drop will face the task of systematically identifying similarities between text types that favor the occurrence of topic drop, such as the two previously mentioned characteristics but also the determinacy of speaker and hearer roles, as well as the general frequency of (personal) pronouns in the prefield.

\largerpage
The second question of why topic drop is not more frequent in some other text types in the FraC, concerns primarily the three spoken text types dialogues, radio transcripts, and interviews.
While topic drop is often said to be restricted to or at least to occur mainly in spoken German \citep{huang1984,cardinaletti1990,poitou1993,jaensch2005,dittmann.etal2007,volodina2011, volodina.onea2012}, these text types only have omission rates of 16\%, 9\%, and 6\% respectively, which are clearly below the rates of several written text types.
The relatively low omission rate in the radio transcripts may partially be explained by the fact that they are mainly based on pre-formulated texts that only mimic spontaneity and authenticity.
It is, therefore, possible that topic drop is rather a means of spontaneous speech and that radio personalities and announcers (unconsciously?) orient themselves more strongly to standard grammar when pre-formulating their texts than do speakers of spontaneous speech, which then leads to a lower frequency of topic drop in the radio transcripts.
However, this explanation does not apply to dialogues and interviews, since both corresponding subcorpora in the FraC do in fact contain transcripts of spontaneous speech.
The dialogue subcorpus consists of spontaneous but elicited conversations about organizing a business trip, collected as part of the Verbmobil project \citep{burger.etal2000}.
The interview subcorpus contains the transcripts of radio interviews conducted with politicians, scientists, prominent people, etc. conducted as part of the \textit{SWR1 Leute} podcast.

What all three subcorpora have in common is that they were recorded either for building a corpus or for broadcasting and that the interlocutors are aware of this fact.
Consequently, the conversations are mostly rather formal.
While the formality in the radio transcripts varies depending on the radio station and its potential target audience, the interlocutors in the Verbmobil corpus use the formal form to address each other%
% Footnote
\footnote{According to Florian Schiel (p.c.), the interlocutors in the Verbmobil corpus were explicitly instructed to do so, i.e., to use formal speech and to use the formal form to address each other even though some of them knew each other from their workplace.}
%
and presumably, the interviews mostly took place between people who did not know each other personally beforehand.
In these more formal settings, where the interlocutors are aware that they are being recorded, colloquial speech is likely less prevalent than in informal private conversations between friends.
This hypothesis awaits an empirical investigation in future research, but it would be consistent with the second claim often made in the literature, that topic drop is limited to informal contexts or colloquial speech \citep{huang1984,poitou1993,jaensch2005,volodina.onea2012,trutkowski2016}.

The explanation of formality may also apply to the text type emails in the FraC and explain why it does not exhibit a higher omission rate.
The literature also claims that topic drop is particularly frequent in personal letters and their ``successors'', private emails \citep{fries1988,volodina2011,volodina.onea2012}.
However, in the email subcorpus of the FraC, there are at most 36\% of private or at least informal emails, namely those between members of a band and between the same members and potential promoters. 
The remaining 64\% are formal emails between students and their professors.
While there are some cases of topic drop in these formal emails, as well (6 instances), most instances of topic drop stem from the more personal band emails (32 instances).

\largerpage
As discussed in \sectref{sec:def.texttype}, both publicity and unfamiliarity of the partners are conditions of communication that Koch and Oesterreicher associate with what they term \textit{Sprache der Distanz} (`language of distance') \citep[23]{koch.oesterreicher1985} and with conceptually written text types.
In contrast, a lack of publicity and familiarity characterizes \textit{Sprache der Nähe} (`language of immediacy') and conceptually spoken text types in which topic drop is argued to occur frequently according to \citet{volodina2011} and \citet{volodina.onea2012}.
This line of reasoning seems to be particularly fruitful for text messages, the text type with the highest omission rate and the largest number of topic drop in the FraC.
Text messages, in particular, the type of private text messages contained in the FraC%
% Footnote
\footnote{\citet[13]{spycher2004} points out that text messages are also used by organizations or companies to provide clients with information, e.g., concerning train timetables, stock market news, weather reports, concerts, etc.
Such text messages with a service function are not included in the FraC subcorpus.}
%
exhibit many of the exemplary features that \citet[23]{koch.oesterreicher1985} list as typical for the language of proximity: dialogue, familiarity, free development of topics, no publicity, spontaneity, involvement, expressiveness, and affectivity.
However, they do not meet the criteria of face-to-face communication and interleaving of situations.
Tweets, online chats, and blogs exhibit fewer of these features since they are public (except for the one-on-one chats) and less familiar, as the writer and the addressee do not necessarily know each other.
The text types ads, classified ads, recipes, and user manuals meet even fewer criteria for conceptually spoken text types, but they still exhibit high rates of topic drop.

In ads and instructions, topic drop often occurs as part of an enumeration, such as in \ref{ex:frac.enum}, or in the context of a combination between text and image where the object shown in a picture is left out in the corresponding linguistic description.
\ex.\label{ex:frac.enum}\ag.Die Spezialpflege mit Feuchtigkeit bindendem Urea\\
the special.care with moisture binding urea\\
`The special care with moisture-binding urea.'
\bg.- $\Delta$ glättet spürbar\\
{} it smoothes noticeably\\
`(it) smoothes noticeably'
\cg.- $\Delta$ lindert Juckreiz\\
{} it relieves itching\\
`(it) relieves the itching'
\dg.- $\Delta$ stärkt die Hautbarriere\\
{} it strengthens the skin.barrier\\
`(it) strengthens the skin barrier' [FraC P986--P989]

\largerpage[-1]
The four instances of topic drop in the recipe subcorpus stem from comments by the recipe author, i.e., add-ons to the actual instructions, as illustrated in \ref{ex:frac.recipe}.

\exg.\label{ex:frac.recipe}Ich streue immer etwas Petersilie darüber, $\Delta$ sieht gut aus und schmeckt.\\
I sprinkle always some parsley thereon that looks good \textsc{vpart} and tastes\\
`I always sprinkle some parsley on top, (that) looks good and tastes good.' [FraC R82]

In classified ads, topic drop occurs especially often in personal advertisements.
At least originally, it seems to have served an economic purpose (see \sectref{sec:pusage.economy}) there, as shown in example \ref{ex:frac.class.ad}.
Given the fact that the costs for personal advertisements often depend on their length \citep[e.g.,][98]{bachmann-stein2011}, topic drop can be considered a means to express more information in less space, similar to abbreviations such as the \textit{NR} for \textit{Nichtraucher} (`nonsmoker').
Further research is needed to decide whether topic drop is generally an expression of linguistic economy in this text type or whether the economic consideration led to a grammaticalization and/or conventionalization of topic drop in personal advertisements.

\ex.\label{ex:frac.class.ad}\ag.Kurz zu mir, $\Delta$ bin geschieden, 45, 1,78m, braune kurze Haare mit grauen Ansätzen, NT, NR, männlich.\\
briefly to me I am divorced 45 1.78m brown short hairs with grey roots nondrinker nonsmoker male\\
`Briefly about me, (I) am divorced, 45, 1.78 m, brown short hair with gray roots, nondrinker, nonsmoker, male.'
\bg.$\Delta$ Wohne und arbeite in Potsdam, $\Delta$ wäre cool wenn Du auch in Potsdam oder Umgebung Dein zu Hause hast.\\
I live and work in Potsdam it would.be cool if you.\textsc{2sg} also in Potsdam or surrounding your.\textsc{2sg} at home have\\
`(I) live and work in Potsdam, (it) would be cool if you also have your home in Potsdam or the surrounding area.' [FraC N1491--N1942]

These examples show that there are potentially several reasons why topic drop does or does not occur frequently in certain text types.
An explanation that only relies on (conceptually) spoken text types or economy falls short of accounting for the whole data.
Here, further research is needed (but see \cite{ruppenhofer2018} for a first corpus study of topic drop in several text types).
\is{Corpus|)} \is{Text type|)}

\section{Distinguishing topic drop from related phenomena}\label{sec:delimitation}
Having characterized topic drop as an ellipsis in predominantly (conceptually) spoken German, in this section, I distinguish it from (seemingly) similar phenomena to also determine what topic drop is \textit{not}.
This is of importance because it shows even more clearly what the defining or characteristic properties of topic drop are and how they may be appropriately modeled syntactically.

Following the seminal paper by \citet{huang1984}, topic drop is considered to be one of three types of referential null subjects \is{Null subject|(} that are often distinguished, each characteristic of different languages.
Besides the ``Germanic \textit{topic drop} type'', \citet[268]{sigurdsson2011} lists the ``Romance \textit{pro drop} type'', \is{@\emph{pro}-drop} and the ``Chinese \il{Chinese} \textit{discourse drop} \is{Discourse \emph{pro}-drop} type'', which is also called discourse \textit{pro}-drop or radical \textit{pro}-drop, referring to the most intensively discussed representatives.%
%% Footnote
\footnote{It is important to note that \citet{sigurdsson2011} himself argues for a unified approach to null arguments, \is{Argument omission} proposing the principle of \textit{C}/\textit{edge linking} as a requirement.
See \sectref{sec:syntax} for details.}
%
I discuss not only similarities and differences of topic drop in German to the two other null pronoun types \textit{pro}-drop and radical \textit{pro}-drop, but I also distinguish it from two phenomena relevant to German, null subjects \is{Null subject|)} in German dialects and V1 declaratives.
I argue that it is the positional restriction of topic drop and its syntactic incompleteness that justify a distinction from the other phenomena.

%*****************************************
% Pro Drop
%*****************************************
\subsection{\textit{pro}-drop}\label{sec:prodrop} \is{@\emph{pro}-drop|(}
At first glance, it may seem natural to treat topic drop as an instance of the similar phenomenon \textit{pro}-drop, i.e., the non-realization of subject pronouns such as the Spanish \il{Spanish|(} \textit{yo} (`I') in \ref{ex:prodrop} \citep{perlmutter1971}.

\exg.\label{ex:prodrop}$\Delta$ Ya lo he hecho.\\
I already this have done\\
`(I) have done that already.'

The concept of a so-called \emph{\emph{pro}-drop parameter} stems from generative grammar. 
In languages where the parameter has a positive setting, ``not only `weather verbs', such as \textit{rain} but also verbs that have subjects with a definite \textit{$\theta$}-role may appear at surface structure with no NP subject'' \citep[28]{chomsky1981}.
It is commonly assumed that \textit{pro}-drop is licensed by a rich inflectional system \citep{taraldsen1980, rizzi1982}.
\citet[82]{ackema.neeleman2007} state that ``[t]he rationale behind the hypothesis that rich agreement is necessary for pro drop is that the features of the empty pronoun would not be recoverable without it.'' \is{Verbal inflection}
In particular, Romance languages such as Italian \il{Italian} and Spanish, \il{Spanish} but also Slavic languages, Hebrew \il{Hebrew} and Basque \il{Basque} \citep[see][]{perlmutter1971} are characterized as \textit{pro}-drop languages and distinguished from non-\textit{pro}-drop languages such as \ili{English} and \ili{French} that normally require pronominal subjects to be realized.%
%% Footnote
\footnote{As \citet{wals-101} problematizes, the concept of \textit{pro}-drop is potentially Anglo-centric because it assumes that non-realization is a deviation from the norm of realizing the subject. 
However, in most languages of the world (\cite{wals-101} lists 437 out of 711 languages), pronominal subjects are expressed through affixes on the verb, while only 82 languages behave like English and German in that they have normally present subject pronouns.
Aware of the possibly Anglo-centric view, I nevertheless follow most of the previous literature and continue to use the term \textit{\emph{pro}-drop} and speak of the subject being unrealized or covert in these languages.}
%

Coming back to topic drop, we can observe that the utterance with \textit{pro}-drop in \ref{ex:prodrop} looks strikingly similar to the topic drop example \ref{ex:TD.subj}, repeated here as \ref{ex:two.rep}.
However, although \textit{pro}-drop and topic drop share the lack of realization of the subject in declarative main clauses, there are at the same time several differences that have led authors to explicitly argue against equating both phenomena \citep[e.g.,][]{fries1988,klein1993,trutkowski2011, volodina2011}.

\exg.\label{ex:two.rep}$\Delta$ Hab das schon gemacht.\\
I have that already done\\
`(I) have done that already.'

First, \textit{pro}-drop is restricted to subject pronouns in the nominative case, \is{Nominative case} whereas topic drop can also target objects and adverbs. 
Second, topic drop is positionally restricted to the prefield, i.e., the preverbal position of declarative V2 clauses ([Spec, CP] in generative terms),%
%% footnote
\footnote{Note that \citet{nygard2014} and \citet{helmer2016} consider topic drop to also be possible in the middle field. \is{Middle field}
See \sectref{sec:prefield.detail} for details.}
%
and to main clauses, whereas \textit{pro}-drop has no such restriction and can occur in several positions as well as in subordinate clauses, as in the Spanish \il{Spanish} example \ref{ex:prodrop.chars}.

\ex.\label{ex:prodrop.chars}
\ag.\label{ex:prodrop.spanish}{Boris Becker:} Si $\Delta$ fuiste un campeón, $\Delta$ siempre serás un \phantom{Boris Becker:} campeón \\
{} if you were a champion you always will.be a \phantom{Boris Becker:} champion\\
Boris Becker: `If (you) were a champion, (you) will always be a \\\phantom{Boris~Becker:} champion' \il{Spanish} (elpais.com, 08/27/2018)\footnotemark
\b.*\label{ex:prodrop:german}Wenn $\Delta$ ein Champion warst, wirst $\Delta$ immer ein Champion sein

\footnotetext{Source: \url{https://elpais.com/deportes/2018/08/27/actualidad/1535371649_689952.html} (visited on 12/29/2024).}

A further difference is that the use of overt subject pronouns in \textit{pro}-drop languages such as Spanish \il{Spanish|)} is often argued to fulfill special pragmatic functions such as emphasis or ambiguity avoidance \is{Ambiguity avoidance} \citep{davidson1996,peskova2013}.
This means that example \ref{ex:prodrop.spanish}, in which \textit{tú} is unrealized, is not functionally equivalent to a variant with overt subject pronouns.
This is different for topic drop, where according to \citet[218]{reis2000}, utterances with and without the omission are functionally parallel (but cf., \cite{eckert1998.diss} and \cite{helmer2016} for functional differences).
These major differences lead me to separate the two phenomena and to discuss topic drop independently of \textit{pro}-drop.
\is{@\emph{pro}-drop|)}

\subsection{Discourse (\textit{pro})-drop}\label{sec:discourse.drop} \is{Discourse \emph{pro}-drop|(}
Related to \textit{pro}-drop, \is{@\emph{pro}-drop} and sometimes even equated with it, is a phenomenon called discourse \textit{pro}-drop \citep{tomioka2003, barbosa2011}, discourse drop \citep{sigurdsson2011}, or radical \textit{pro}-drop \citep{neeleman.szendroi2005,neeleman.szendroi2007}.
In languages such as Chinese, \il{Chinese} \ili{Japanese}, and Korean \il{Korean} (see \cite{neeleman.szendroi2007} for the discussion of further languages) pronouns can be omitted rather freely as long as they are recoverable \is{Recoverability} from the discourse context \citep{huang1984, neeleman.szendroi2007}, as illustrated in the Chinese \il{Chinese} example \ref{ex:chinese.drop}.

\ex.\label{ex:chinese.drop}
\ag.  A: Zhangsan kanjian Lisi le ma?\\
{} Zhangsan see Lisi {\textsc{le}\footnotemark} \textsc{q}\\
A: `Did Zhangsan see Lisi?'
\footnotetext{\textsc{Le} is the perfective or inchoative aspect marker \citep[533, footnote 1]{huang1984}.}
\bg.\label{ex:chinese.drop.1c}B: $\Delta$ kanjian ta le.\\
{} he see he \textsc{le}\\
B: `(He) saw him.'
\cg.\label{ex:chinese.drop.2c}B: $\Delta$ kanjian $\Delta$ le.\\
{} he see he \textsc{le}\\
B: `(He) saw (him).'
\dg.\label{ex:chinese.drop.emb}B: Zhangsan shuo [$\Delta$ kanjian $\Delta$ le].\\
{} Zhangsan say he see he \textsc{le}\\
B: `Zhangsan said that (he) saw (him).' (\cite[533]{huang1984}, shortened)

Unlike the ``classic'' \textit{pro}-drop \is{@\emph{pro}-drop} languages discussed in the literature that have a rich inflectional system, the discourse (\textit{pro}) drop languages Chinese \il{Chinese} and Japanese \il{Japanese} completely lack agreement \citep[672]{neeleman.szendroi2007}.
This has led many authors to distinguish between both phenomena \citep[e.g.,][]{huang1984, dalessandro2015}, but there are also unitary approaches \citep[e.g.,][]{sigurdsson2011,duguine2014}.
Regardless of whether discourse (\textit{pro}) drop is considered to be a subtype of \textit{pro}-drop \is{@\emph{pro}-drop} or a phenomenon on its own, it is different from topic drop in German.
While example \ref{ex:chinese.drop.1c} again looks similar to the topic drop examples above -- a preverbal subject is omitted from a declarative main clause --, examples \ref{ex:chinese.drop.2c} and \ref{ex:chinese.drop.emb} differ remarkably from topic drop in German.
In both \ref{ex:chinese.drop.2c} and \ref{ex:chinese.drop.emb} not only the subject but also the clause-internal object is omitted, in \ref{ex:chinese.drop.emb} \il{Chinese} even from a subordinate clause.
The omission of two constituents per clause and the omission from a subordinate clause violate two main restrictions of topic drop in German.  
Therefore, I distinguish topic drop also from discourse (\textit{pro}) drop.
\is{Discourse \emph{pro}-drop|)}

\subsection{Dialectal null subjects}\label{sec:dialectal.null.subjects} \is{Null subject|(}
Example \ref{ex:prodrop:german} showed that Standard \ili{German|(} does not allow for \textit{pro}-drop.
However, in modern German dialects, there are instances of null subjects, such as \ref{ex:dialect.ns.bavarian} from Bavarian, that are regularly analyzed as \textit{pro}-drop \is{@\emph{pro}-drop} (\cite{rosenkvist2009,axel.weiss2011}, see also \cite[159--167]{frick2017} for Swiss German).

\exg.\label{ex:dialect.ns.bavarian}wennsd $\Delta$ moang wieda gsund bisd\\
if.\textsc{2sg} you.\textsc{2sg} tomorrow again healthy are.\textsc{2sg}\\
`if (you) are healthy again tomorrow' \citep[36]{axel.weiss2011}

Such null subjects do not only occur in German dialects but ``in all the major dialect groups within the Continental West-Germania'' \citep[21]{axel.weiss2011} and were also possible in earlier Old High German \citep{axel.weiss2011} and in other older Germanic languages \citep{rosenkvist2009}.%
%% Footnote
\footnote{There seems to be no consensus in the literature on whether null subjects in old Germanic languages follow the same principles as null subjects in modern dialects (see \cite{axel.weiss2011} for a unifying view; but cf. \cite{rosenkvist2009}).}
%

The properties that distinguish dialectal null subjects from topic drop are similar to those that were outlined for \textit{pro}-drop \is{@\emph{pro}-drop} in \sectref{sec:prodrop}.
Dialectal null subjects are not restricted to the preverbal position \citep[23]{fries1988}, and not only can they occur in main clauses with V2 word order but also in embedded clauses \is{Embedding} (\cite[23]{fries1988}, \cite[171]{rosenkvist2009}).
Moreover, they seem to be generally restricted to pronouns in the nominative case \is{Nominative case} \citep[23]{fries1988}, often require distinct inflectional marking \is{Verbal inflection} on the verb \citep[163]{rosenkvist2009}, and often demand or are accompanied by morphological changes of, e.g., complementizers, \is{Complementizer} i.e., so-called \textit{inflected complementizers} \is{Complementizer} such as \textit{wennsd} (`if-\textsc{2sg}') in \ref{ex:dialect.ns.bavarian} (\cite[23]{fries1988}, \cite[162]{rosenkvist2009}).
These characteristics clearly distinguish dialectal null subjects from topic drop.
\is{Null subject|)}

\subsection{V1 declaratives}\label{sec:def.v1}\is{V1 declarative|(}
In spoken German, as well as in certain written text types, \is{Text type} we find utterances such as \ref{ex:V1.witz} and \ref{ex:V1.da}.
Although they look identical to instances of topic drop because both start with a finite verb, they are in fact declarative utterances.

\exg.\label{ex:V1.witz}Kommt ein Mann in die Kneipe...\\
comes a man into the bar\\
`A man comes into the bar...' \citep[50, shortened]{oennerfors1997}

\exg.\label{ex:V1.da}Hab ich ihr ganz frech noch en Kuß gegeben.\\
have I her quite cheekily yet a kiss given\\
`I gave her another kiss quite cheekily.' \citep[99, shortened]{oennerfors1997}

In the literature, such cases are referred to as \textit{V1 declaratives} \citep{oennerfors1997, reis2000, schwitalla2012, schalowski2015} or \textit{eigentliche Verbspitzenstellung} (`proper verb top positioning') \citep{auer1993,imo2013}.
\citet[195]{auer1993} defines them as ``clauses in which all obligatory complements are present but occur after the finite verb.''%
%% Footnote
\footnote{My translation, the original: ``Sätze[], in denen sämtliche obligatorischen Ergänzungen vorhanden sind, aber nach dem finiten Verb stehen'' \citep[195]{auer1993}.}
%
\citet[637]{zifonun.etal1997} term them ``communicative minimal units in declarative mode with verb-first word order''%
%% Footnote
\footnote{My translation, the original: ``kommunikative Minimaleinheiten im Aussagemodus mit Verberststellung'' \citep[637]{zifonun.etal1997}.}
%
and state that they are complete because neither complements nor supplements, i.e., adjuncts, \is{Adjunct} are omitted.
This completeness is the main difference to topic drop and leads me (i) to use the term \textit{V1 declaratives} for utterances such as \ref{ex:V1.witz} and \ref{ex:V1.da}, and (ii) to distinguish them from topic drop.
In utterances with topic drop, arguments \is{Argument} required by the verb or adjuncts \is{Adjunct} are missing from the clause.
In contrast, V1 declaratives are syntactically complete and, thus, not an elliptical phenomenon \citep{oennerfors1997}.

It is worth noting that the completeness of V1 declaratives has been, and still is, disputed in the literature \citep[see][13--18, for an overview since the 19th century]{oennerfors1997}.
\citet[49--50]{oennerfors1997} notes that \citet[33]{altmann1987} discusses an example similar to \ref{ex:V1.witz}, which Önnerfors considers to be a V1 declarative, jointly with clear cases of topic drop such as \ref{ex:altmann.td}.
\citet[33]{altmann1987} treats both as ``confounding factors when determining the verb order''%
%% Footnote
\footnote{My translation, the original: ``Störfaktoren bei der Verbstellungsbestimmung'' \citep[33]{altmann1987}.}
%
and as cases of ``Vorfeldellipse'' (`prefield ellipsis'), thereby indirectly equating topic drop and V1 declaratives.

\exg.\label{ex:altmann.td}$\Delta$ Komme morgen. $\Delta$ Bleibe drei Tage.\\
I come tomorrow I stay three days\\
`(I) come tomorrow. (I) will stay three days.' \citep[33]{altmann1987}

\citet[300]{sandig2000} also treats V1 declaratives as ellipsis but as a separate phenomenon.
She argues that in V1 declaratives so-called ``cohesive devices'' such as (\textit{und}) \textit{da} or (\textit{und}) \textit{dann} are ``saved'', i.e., omitted.
Several authors, such as \citet{oppenrieder1987} and \citet{imo2013}, at least suggest that an element can be `inserted' into the presumably empty prefield, or that V1 declaratives can be `translated' into corresponding V2 \is{V2 word order|(} declaratives by inserting elements into the prefield \citep[e.g.,][]{auer1993}.
As candidates for these elements, the authors most often propose the semantically relatively empty elements \textit{da} (`there'), \textit{dann} (`then'), and \textit{es} (`it') (\cite{oppenrieder1987,auer1993, poitou1993,sandig2000,imo2013}; see \cite[][18]{oennerfors1997}) but also \textit{jetzt} (`now') \citep{oppenrieder1987} or even \textit{davon} (`thereof') or \textit{darüber} (`thereover') \citep{imo2013}.

In \citet{oennerfors1997}, all three proposals toward a more or less elliptical nature of V1 declaratives are rejected.
He compares the distribution of V1 declaratives and their apparent V2 equivalents with \textit{es} or \textit{da} in the prefield.
In several cases, he argues that the V2 variant is inadequate in context or in the text type \is{Text type} or leads to a shift in meaning compared to the original V1 utterance (\cite[54--59]{oennerfors1997}, see \ref{ex:V1.witz.ersetzt} vs. \ref{ex:V1.witz} and \ref{ex:V1.da.ersetzt} vs. \ref{ex:V1.da}).
Önnerfors concludes that, since neither V2 sentences with \textit{es} nor with \textit{da} can adequately replace the corresponding V1 declaratives in any usage case, V1 declaratives cannot be an instance of ellipsis but are complete sentences without a prefield \citep[4]{oennerfors1997}.

\ex.\label{ex:V1.witz.ersetzt}(?Es/?Da) kommt ein Mann in die Kneipe...

\ex.\label{ex:V1.da.ersetzt} (*Es/Da) hab ich ihr ganz frech noch en Kuß gegeben.

\largerpage
\citet{oennerfors1997} furthermore supports his claim with diachronic and cross\hyp linguistic data.
First, he argues that V1 declaratives are a very old word order option in Germanic languages.
They were already present in Proto-Indo-European and Proto-Germanic, albeit being more marked in comparison to the dominant verb-final word order \citep[8]{oennerfors1997}.
Even when several Germanic languages developed V2 word order, V1 remained available in all Germanic languages -- synchronically and diachronically \citep[9]{oennerfors1997}.
For instance, V1 word order was frequently used in Old High German and Early New High German but less so in Middle High German \citep[10]{oennerfors1997}.
Second, he states that the V1 declaratives, which several authors argue are derived by ellipsis from V2 structures, are older than their alleged `full forms'.
V2 declaratives with \textit{es} in the prefield developed in Middle High German, utterances with \textit{da} are attested earlier, but both types still occurred later than the first V1 declaratives \citep[52]{oennerfors1997}.
This makes an ellipsis interpretation of V1 declaratives impossible, at least in Old High German, and less likely for modern variants.
Following Occam's razor, \citet[52]{oennerfors1997} proposes to prefer the simpler option, i.e., to treat V1 declaratives as sentences without a prefield, which have been in existence in the German language since the Old High German period, over a more complex approach that treats V1 declaratives as elliptical utterances where a semantically empty element is first inserted and then deleted from the prefield.

Without ultimately concluding whether all cases treated by \citet{oennerfors1997} as V1 declaratives are in fact such, I distinguish them from topic drop.
In this book, I focus mostly on topic drop of verb arguments \is{Argument} because, for them, it can be clearly decided whether they have been omitted from a clause.
This decision has the practical consequence that I, unlike, e.g., \citet{frick2017}, do not consider cases where a placeholder \textit{es} could theoretically be placed in the prefield, as in \ref{ex:placeholder}.

\exg.\label{ex:placeholder}(Es) liest ja nicht jeder die Frankfurter Rundschau.\\
it reads \textsc{part} not everyone the Frankfurter Rundschau\\
`Not everyone reads the Frankfurter Rundschau.' [FraC B222]

\is{V1 declarative|)}\il{German|)}

\section{Topic drop in other Germanic V2 languages}\label{sec:td.germanic}
As \citet[11]{rohrbacher1999} points out, all Germanic languages except \ili{English} have V2 word order, i.e., in declarative main clauses the main verb is always in the second position following what Rohrbacher terms a ``clause-initial phrase unit such as the subject [...] or a topicalized XP'' \citep[11]{rohrbacher1999}.%
%% Footnote
\footnote{\citet[14]{rohrbacher1999} points out that in sentences with a subject in the initial position followed by an adverbial, \is{Adverbial} \ili{English} has verb-third order \ref{ex:eng.v3}, but that a finite auxiliary \is{Auxiliary} in the second position is required in direct complement questions \ref{ex:eng.q}.
He states that English is therefore often called a ``residual V2'' language, following \citet{rizzi1990a, rizzi1996}.

%\vspace{-0.5\baselineskip}
\ex.\label{ex:eng.v3}
\a. Mary never liked trashy movies. 
\b. *Mary liked never trashy movies. \citep[14]{rohrbacher1999}

%\vspace{-0.5\baselineskip}

%\vspace{-0.5\baselineskip}
\ex.\label{ex:eng.q}
\a. *What kind of movies Mary liked?
\b. *What kind of movies liked Mary?
\c. What kind of movies did Mary like? \citep[14]{rohrbacher1999}

%\vspace{-1.5\baselineskip}
}
In almost all Germanic V2 languages, this preverbal unit can be omitted from the clause, i.e., they allow for topic drop \citep[83]{ackema.neeleman2007} -- apparent exceptions of Germanic V2 languages without topic drop are the Flemish dialects of Dutch \citep[141]{haegeman1996} and some varieties of Danish \il{Danish} \citep[283]{rizzi2000}.

In this section, I briefly go through some other Germanic V2 languages \is{V2 word order|)} in alphabetical order and sketch the properties of topic drop in these languages.
Such an outline could in principle serve to further sharpen the picture of topic drop in German by revealing possible language-specific peculiarities.
However, in the following, an overall consistent picture of topic drop as a common ellipsis type of the Germanic languages under discussion emerges, despite several minor deviations.
Here, too, the prefield restriction and the typical occurrence in certain registers and/or text types \is{Text type} can be identified as characteristic features. 
Although this book is not typologically oriented but focuses on topic drop in German, this section is intended to open up a typological perspective.
It aims at stimulating reflection on the extent to which the results obtained in this work are transferable to topic drop in other languages, without actually being able to accomplish such a transfer.
I refer readers who are only interested in a summary of the typological results to \sectref{sec:typological.summary}.

\subsection{Danish} \il{Danish|(}
The research on topic drop in Danish, at least the one available in English, seems to be less extensive than in most other Germanic languages.
It even seems to be questionable whether topic drop is even possible in Danish.
On the one hand, \citet[141]{haegeman1996} claims that Danish is more liberal than Dutch \il{Dutch} or German because it allows not only for the omission of referential constituents but also of expletives \is{Expletive} such as \textit{det} in \ref{ex:td.da}.
In \sectref{sec:topicality.ness}, I argue that this is not a real difference to German, since German also allows for the omission of expletives \is{Expletive} and similar elements.

\exg.\label{ex:td.da}$\Delta$ Regnede meget igår\\
it rained much yesterday\\
`(It) rained a lot yesterday.' \citep[141]{haegeman1996}

On the other hand, \citet[283]{rizzi2000} claims that in some varieties of Danish, topic drop is not possible at all.
Similarly, \citet{hamann.plunkett1998} state that the existence of topic drop in Danish is unclear but that it seems to be ``not totally excluded on some stylistic levels and omission from the first position of the sentence is better than omission from other positions'' \citep[49]{hamann.plunkett1998}.
They support their claim with example \ref{ex:td.da.subj} with an omitted 3rd person singular referential subject and example \ref{ex:td.da.obj} with an omitted direct 3rd person singular object.

\setlength{\SubExleftmargin}{2.4em}
\ex.
\ag.???$\Delta$ Har ikke købot boge\label{ex:td.da.subj}\\
he has not bought book.the\\
`(He) has not bought the book.'
\bg.?$\Delta$ Kender jeg ikke\label{ex:td.da.obj}\\
that know I not\\
`I don't know (that).' \citep[49, their judgments]{hamann.plunkett1998}

\resetExdefaults
\il{Danish|)}

\subsection{Dutch} \il{Dutch|(}
Topic drop in Dutch is similar to topic drop in German.
It only targets elements in the preverbal position or, as \citet{thrift2001} puts it, in the topic position \citep[71]{thrift2001}, it is limited to \citep[57]{thrift2001} or, at least, common in spoken language \citep[54]{weerman1989}, and the omitted element must be recoverable \is{Recoverability} from the context (\cite[1329]{corver.broekhuis2016}; see also \cite[71]{thrift2001}).
Topic drop in Dutch can target ``subjects, (in)direct objects, complement prepositional phrases and the objects of prepositions'' \citep[49]{thrift2001}. \is{Prepositional object}
See \ref{ex:td.dutch} for an example of an omitted object.

\ex.\label{ex:td.dutch}
\ag.Wat heb jij met dat boek gedaan?\\
what have you.\textsc{2sg} with that book done\\
`What have you done with that book?'
\bg.$\Delta$ Heb ik aan Marie gegeven.\\
that have I to Marie given\\
`(That), I have given to Marie.' \citep[63]{thrift2001}

\noindent
\citet{jansen1981}, cited in \citet{thrift2001}, collected a corpus \is{Corpus} of spontaneous speech data and found that topic drop in Dutch is more common with objects, such as \ref{ex:td.dutch}, than with subjects \citep[58]{thrift2001}. 
A similar tendency seems to be present in German as well (see \sectref{sec:corpus.function}).
However, a systematic investigation with sufficient data is still pending for both German and Dutch.

\is{Acceptability rating study|(}
In a survey%
%% Footnote
\footnote{The informants first read a context question before the target utterance was read aloud to them in at least three conditions (full forms with different word orders and a version with topic drop), apparently one after the other. 
Their task was to judge each utterance as ``good'', ``ungrammatical'', or ``unsure'' \citep[59]{thrift2001}.
In addition to the problematic presentation of multiple conditions of the same token set in succession, which makes the manipulation evident, it remains unclear how many utterances each informant saw and how many utterances there were in total.
Therefore, the results reported below should be treated with some caution given the methodology.}
%
of 19 native speakers of Dutch, mostly students, \citet[62]{thrift2001} found that Dutch also patterns with German in that unstressed 1st and 2nd person objects cannot occur in the preverbal position and cannot be omitted (see \sectref{sec:usage.function.theory}).
Interestingly, she found an effect of animacy \is{Animacy|(} for 3rd person direct objects that, as far as I know, has not yet been attested for German.%
%% Footnote
\footnote{While \citet[190]{reis1982} mentions animacy \is{Animacy} as a potential factor for topic drop in German, she does so in a different context.
She speculates whether animate subjects can be better omitted in telegraphese.
This contrasts with \citeauthor{poitou1993}'s (\citeyear[116]{poitou1993}) corpus study, \is{Corpus} who found that most of the omitted subjects in his data set were inanimate.
}
The omission of inanimate objects, such as \ref{ex:td.dutch}, was rated as acceptable while animate direct objects received mixed judgments from  Thrift's informants \citep[63--64]{thrift2001}. \is{Acceptability rating study|)}
The results of my corpus study \is{Corpus} do not generally argue against the existence of such a restriction in German, since most of the omitted direct objects refer back to abstract and inanimate propositions and verb phrases \is{Verb phrase} (see \sectref{sec:frac.td.comp.person}).
However, a systematic empirical investigation of animacy \is{Animacy|)} is still pending for both languages.

An interesting difference between topic drop in Dutch and topic drop in German concerns 1st and 2nd person subjects.
In her survey, \citet[60]{thrift2001} found that the omission of unstressed 1st and 2nd person subject pronouns is generally impossible except in diary contexts, in which 1st person singular pronouns can be omitted.%
%% Footnote
\footnote{\citet[60]{thrift2001} states that this result is in agreement with \citet{jansen1981}.}
%
This result is surprising given the German data, where especially the 1st singular pronoun is frequently omitted in several text types \is{Text type} (see \sectref{sec:frac.td.comp.person}).
\citet[72]{thrift2001} attributes the Dutch pattern to the fact that the overt unstressed 1st and 2nd person pronouns cannot occur as topics, i.e., in the topic position, because their reference  constantly shifts during a conversation, just as the roles of speaker and hearer change.
At least cross\hyp linguistically this explanation cannot suffice because turn-taking and reference shift also occur in German and other languages, where 1st and 2nd person subject pronouns can easily be omitted.
Furthermore, when a speaker produces an utterance with an unrealized 1st or 2nd person subject pronoun, it is evident to whom they refer -- themselves in the case of the 1st person and their interlocutor in the case of the 2nd person.
Thus, it remains unclear how to explain this difference between German and Dutch.
Despite this divergence, both languages show many similarities concerning the properties of topic drop, such as the positional restriction and the syntactic functions that can be targeted.
\il{Dutch|)}

\subsection{Icelandic} \il{Icelandic|(}
As \citet{sigurdsson1989} points out, in Icelandic topic drop occurs in declarative main clauses.
Similar to the Germanic languages discussed so far, a subject can only be omitted from [Spec, CP] \ref{ex:td.ic}.
Although object drop seems to be less frequent than in German \citep[142]{sigurdsson1989}, it is possible in Icelandic, in particular with the 3rd person singular neuter `that', `it' or `this' \citep[156]{sigurdsson1989}.
Moreover, Icelandic also allows for a special type of null objects independent of the [Spec, CP] restriction that is heavily constrained and does not seem to be equally acceptable to all speakers in all contexts \citep[152--153]{sigurdsson1989}.

Similar to topic drop in the other Germanic languages, topic drop in Icelandic seems to be most common in colloquial speech and in the so-called \textit{telegraphic style}, which according to \citet[139]{sigurdsson1989} is distinctive of letters, diaries, postcards, and telegrams.
As a parallel not only to German but also Swedish \il{Swedish} (see below) and in contrast to Dutch, \il{Dutch} Icelandic topic drop is particularly frequent with the 1st person singular.
Ambiguous \is{Ambiguity} cases such as \ref{ex:td.ic} are most likely interpreted as 1st person singular, which Sigurðsson attributes to the easy identifiability of the speaker \citep[140]{sigurdsson1989}.

\exg.\label{ex:td.ic}$\Delta$ Veit það.\\
I/she/he know.(\textsc{1sg}/\textsc{3sg}) it\\
`(I) know it.' / `??(She/He) knows it.' \citep[140, his judgment]{sigurdsson1989}
%\vspace{-0.5\baselineskip}

\il{Icelandic|)}

\subsection{Norwegian} \il{Norwegian|(}
For Norwegian, \citet[333]{eide.sollid2011} state that topic drop is restricted to root clauses and topics.
They implicitly assume a restriction to the prefield or [Spec, CP] by stating that the resulting structures have V1 word order \citep[346]{eide.sollid2011}.
In contrast, \citet[10]{nygard2018} states that topic drop occurs also ``occasionally sentence-medially'' (similar to \cite{helmer2016} for German, see \sectref{sec:prefield.detail}).
In their corpus \is{Corpus} of ``2015 main clause declaratives from eight interviews with respondents from two dialect areas'' \citep[342]{eide.sollid2011}, \citet[346]{eide.sollid2011} found that 8.3\% of all clauses contained topic drop and that not only subjects, as in \ref{ex:td.nw}, but also non-subjects, as in \ref{ex:td.nw.o}, were omitted.
\citet[172--173]{nygard2014} states that topic drop in Norwegian can also target objects and expletives. \is{Expletive}

\ex.
\ag.\label{ex:td.nw}$\Delta$ Traff HAM igjen {i dag}.\\
I met HIM again today \\
`(I) met him again today.'  \citep[333, original emphasis]{eide.sollid2011}
%\vspace{-1\baselineskip}
\bg.\label{ex:td.nw.o}$\Delta$ va ho å kjøpte hus der.\\
then was she and bought house there \\
`(Then) she went and bought a house there.'  \citep[346]{eide.sollid2011}
%\vspace{-1\baselineskip}

Both \citet{eide.sollid2011} and \citet{nygard2014} emphasize the fact that topic drop typically occurs in certain registers and text types. \is{Text type}
\citet[333]{eide.sollid2011} list spoken informal language, headlines, and diary syntax.
\citet{nygard2014} extends the enumeration to include spontaneous speech and ```hybrid' registers such as sms, e-mails, online chats, Facebook updates, interviews and headlines'' \citep[172]{nygard2014}.
This is strikingly similar to the text types \is{Text type} in which topic drop occurs in German (see Sections \ref{sec:def.texttype} and \ref{sec:corpus.texttype}).
\il{Norwegian|)}

\subsection{Swedish} \il{Swedish|(}
According to \citet[50]{hakansson1994}, topic drop is a common phenomenon in colloquial Swedish that, like in German, is restricted to [Spec, CP] (\cite[55]{mornsjo2002}, \cite[27]{platzack2013}). 
Like topic drop in German, it targets not only subjects \ref{ex:td.sw} and direct objects but also indirect objects, objects of prepositions, expletives, \is{Expletive} and ``quasi-argument[s] or [...] frame topic[s] like här ‘here’, där ‘there’, då ‘then’, nu ‘now’ and the adjunctive så ‘so, then’'' (\cite[29]{platzack2013}, see also \cite{mornsjo2002}).
\citet[31]{mornsjo2002} analyzed a collection of V1 declaratives \is{V1 declarative} and instances of topic drop collected from spoken language corpora and radio and television broadcasts, among others.
She states that subjects and direct objects with propositional reference are the most common types of topic drop in her data, although objects with nominal reference were also omitted \citep[57--58]{mornsjo2002}.
However, she did not find any omitted indirect objects \citep[62]{mornsjo2002}.
This pattern is another similarity to topic drop in German, as \citet{mornsjo2002} points out as well (see also \sectref{sec:usage.function.theory}).
In addition, similarly to German and different from Dutch, \citet[279]{sigurdsson2011} claims that ambiguous \is{Ambiguity} utterances with topic drop and syncretic \is{Syncretism} verb forms, such as \ref{ex:td.sw}, are often interpreted as 1st person (singular) by default, whereas 2nd and 3rd person readings are more constrained, the latter requiring a context with speaker shift.

\exg.\label{ex:td.sw}$\Delta$ Kommer tillbaks imorgon.\\
{} come.$\emptyset$-\textsc{agr} back tomorrow\\
`(I/We/She, etc.) will be back tomorrow.' \citep[268]{sigurdsson2011}

One possible difference from German that \citet{mornsjo2002} addresses is the possibility of 1st and 2nd person object topic drop.
While topic drop of 1st and 2nd person objects is often considered to be ungrammatical in German (see the discussion in \sectref{sec:usage.function.theory}), \citet[70--73]{mornsjo2002}, admitting that she has not found any authentic examples, argues that in Swedish there is no actual ban on omitting them, but rather that pragmatic factors make it unlikely for them to be put into the prefield and omitted from there.
Such a pragmatic explanation may also be beneficial for German (see the discussions in Sections \ref{sec:usage.function.theory} and \ref{sec:info.theory.function}).
\il{Swedish|)}

\subsection{Yiddish} \il{Yiddish|(}
Yiddish, unlike German, is claimed to have no null objects \citep[342]{vanderwurff1996}.
\citet{rohrbacher1999} states that it allows for referential null subjects \ref{ex:td.yi} that have to be analyzed as topic drop because ``they are restricted to the utterance-initial position, thus being permitted in matrix clauses without topicalization [...] but being barred from questions [...], matrix clauses with topicalization [...], and embedded \is{Embedding} clauses'' \citep[253]{rohrbacher1999}.
Without providing numbers, \citet[437]{rosenkvist2012} emphasizes that topic drop is much more frequent in Yiddish than in other modern Germanic languages.
He states that all subjects can be omitted in the preverbal position if an antecedent \is{Antecedent} is available in discourse.

\exg.\label{ex:td.yi}$\Delta$ Horevet iber di koykhes.\\
she work over the strength\\
`(She) is working too hard.' (\cite[254]{rohrbacher1999}, following \cite[83]{prince1999})

Interestingly, \citet{rohrbacher1999}, following \citet{prince1999}, and \citet{rosenkvist2012} argue that Yiddish does not only allow for topic drop but also for a restricted version of referential pro drop but only in the 2nd person singular, which distinguishes it from German. \il{Yiddish|)}

\subsection{Summary}
The Germanic V2 languages discussed share with German the possibility of omitting a subject from the [Spec, CP] or prefield position of declarative main clauses.
This similarity, which is most likely rooted in the general V2 property, \is{V2 word order} suggests that it may be beneficial to also look at topic drop cross\hyp linguistically in a systematic manner.
This is not least due to the fact that topic drop seems to be restricted not only positionally in all these languages but also to spoken language and informal written text types. \is{Text type}
Besides these similarities, the differences between the languages, such as the apparent impossibility of object omission in Yiddish or the apparent impossibility of omitting the 1st and 2nd person in Dutch, Dutch may further motivate the joint consideration of topic drop in the Germanic languages to explain which language-specific differences lead to deviations from the cross\hyp linguistic pattern.

\section{Register-dependent omissions in other languages}\label{sec:null.subjects.other.languages} \il{English|(} \il{French|(} \is{Null subject|(}
Having distinguished topic drop from similar phenomena previously in this chapter, and having emphasized that it can be regarded as a common ellipsis type of numerous Germanic languages with the same basic properties, I would like to undertake another typological digression in the following, namely, to register-dependent argument omissions \is{Argument omission} in English, French, and Russian, Russian i.e., languages that are not regarded as classic V2 languages.
For these phenomena, I would also like to present the differences and similarities to topic drop, but it is more difficult to decide at this point whether a distinction from topic drop or a joint consideration with topic drop is the more useful strategy.
I do not commit myself to either option here, but I would like to suggest, as in the previous sections, a typological perspective on topic drop.
There may be cross\hyp linguistic tendencies favoring, for instance, left-peripheral positioning and/or the occurrence in certain registers or text types \is{Text type} as characteristics of different ellipsis types.

\subsection{Subject omission in English and French}\label{sec:english.french}
There are also register-dependent null arguments \is{Argument omission|(} in the non-V2 languages English \ref{ex:nullsubj.eng} and French \ref{ex:nullsubj.fr}.

\ex.\label{ex:nullsubj.eng}A very sensible day yesterday. $\Delta$ Saw no one. $\Delta$ Took the bus to Southwark Bridge... (\emph{The diary of Virginia Woolf}, vol. 5, 1936--41: 203; cited in \cite[167]{haegeman1990}, shortened)

\exg.\label{ex:nullsubj.fr}$\Delta$ M'accompagne au Mercure, puis à la gare... $\Delta$ s'est donné souvent l'illusion de l'amour à P... \\
she me.accompanies to.the Mercure then to the station... she herself.is given often the.illusion of the.love to P... \\
`(She) accompanies me to the Mercure, then to the station...  (She) often gave herself the illusion of love to P...' (Paul Léautaud, \emph{Le Fléau. Journal particulier} 1917--1930: 69--70; cited in \cite[167]{haegeman1990}, shortened)

The subject pronoun of a finite main clause can be omitted in spoken colloquial English \citep{haegeman1997, weir2012} and in certain ``abbreviated written registers'' \citep[233]{haegeman1997} of English  and French \citep[see also][]{haegeman1990,haegeman2000,weir2012}, such as diaries, postcards, informal letters, notes, instructional writing (e.g., recipes, stage directions), text messages, and emails \citep{haegeman1990, haegeman1997, haegeman2017, weir2012}.

These null subjects -- null objects are not possible \citep[172]{haegeman1990} -- are mostly restricted to main clauses, i.e., they cannot occur in subordinate clauses and only marginally in embedded \is{Embedding} clauses (\cite[168]{haegeman1990}, \cite[98]{haegeman2007}, \cite[108]{weir2012}).
Subjects of all grammatical persons can be omitted (\cite[238]{haegeman1997}, \cite[133]{haegeman2000}, \cite[95]{haegeman2007}, \cite[233]{haegeman2017}, \cite[106]{weir2012}), as well as both referential and non-referential subjects (\cite[236]{haegeman1997}, \cite[233]{haegeman2017}), provided they are not focused \is{Focus} \citep[108]{weir2012}.
Both \citet{weir2012} and \citet{haegeman2000} describe the English null subjects by and large as a left-edge phenomenon%
%% Footnote
\footnote{\citet[109]{weir2012} notes that in spoken English not only subjects but also additional elements such as auxiliaries, \is{Auxiliary} determiners, \is{Article} or syllables can be omitted at the left edge of a sentence.
A similar pattern has been observed in \ili{Norwegian}, where topic drop sometimes seems to `expand' so that the finite verb in the left bracket is also omitted, especially if it is an auxiliary \is{Auxiliary} or a modal verb \is{Modal verb} (\cite[333]{eide.sollid2011}, \cite[173]{nygard2014}).
For \ili{German}, \citet[161--164]{wilder1996} discusses similar cases as \textit{Vorwärtstilgung} (`forward canceling').
I remain agnostic as to whether these cases are in fact some kind of extension of left-peripheral null subjects or topic drop, or whether they should not be more uniformly described as fragments \is{Fragment} \citep[e.g.,][]{morgan1973,merchant2004,reich2007,lemke2021}.
 } 
%
 (\cite[109]{weir2012}, \cite[139]{haegeman2000}), i.e., the subject is usually  deleted in sentence-initial position.
However, written, but not spoken English, as well as written French, allow for preposed adjuncts \is{Adjunct} \ref{ex:nullsubj.preposed.adj} but not arguments \is{Argument} \ref{ex:nullsubj.preposed.arg}, so that the null subject is not necessarily the left-most element of the clause \citep[148]{haegeman2000}.

\ex.\label{ex:nullsubj.preposed} 
\a. \label{ex:nullsubj.preposed.adj} After Dr. Krook, $\Delta$ had good lunch at Eagle with Gary [Hamp]. \\(Sylvia Plath, 03/06/1959: 126; cited in \cite[149]{haegeman2000})
\b.\label{ex:nullsubj.preposed.arg}  *Dr. Krook, $\Delta$ met for lunch at Eagle. \citep[149]{haegeman2000}
	
This difference led Haegeman, who in previous work had discussed null subjects in spoken and written English jointly \citep{haegeman1997, haegeman2013}, to separate them in \citet{haegeman2017}.
With this distinction, she follows \citet{weir2012}  who lists another distributional difference.
In spoken English, null subjects cannot occur before unclitized variants of verbs that also have clitic forms such as \textit{am}, \textit{are}, and \textit{is}, see \ref{ex:clitized}.

\ex.*\label{ex:clitized}$\Delta$ am thinking of leaving tomorrow. \citep[107, his judgment]{weir2012}

\largerpage[2]
The preceding discussion has shown that null subjects in English and French share many similarities with topic drop.
In both cases, the omission is restricted to spoken language or special text types, \is{Text type} mostly occurs at the left periphery (except for preposed adjuncts \is{Adjunct} in written language) and in main clauses, and can target any grammatical person.
These similarities led \citet{haegeman1990} to analyze null subjects in French and English as instances of topic drop.
However, she corrects this view in \citet{haegeman1997} and \citet{haegeman2007}  rejecting a topic drop analysis for three reasons:
(i) In English and French only subjects but not objects can be null \citep{haegeman1997,haegeman2007}, unlike in most topic drop languages (see \sectref{sec:td.germanic}).
(ii) According to her argument, topic drop cannot target expletives \is{Expletive|(} and quasi-arguments as they cannot be topicalized, but these elements can be null in English and French \citep{haegeman1997,haegeman2007}. 
(iii) A topic drop analysis would incorrectly predict that embedded \is{Embedding} null subjects should only be possible in English if no overt complementizer \is{Complementizer|(} is present because subject extraction is possible in cases with null complementizers \is{Complementizer omission} \citep[103]{haegeman2007}.\clearpage

Concerning (i), the impossibility of omitting objects indeed seems to be the most notable difference between Germanic topic drop (but note the apparent ban on object omission in Yiddish) and the English and French left peripheral null arguments.
For example, \citet[178--179]{wilder1996} states that, under his analysis of topic drop as \textit{Vorwärtstilgung} (`forward deletion'), it is impossible to explain why a given, \is{Givenness} fronted object cannot be omitted in English.
However, it seems reasonable to look for the cause in the word order rules of English and French, which have SVO as their dominant word order \citep{dryer2013a}.
Therefore, the constituent that immediately precedes the verb is usually the subject and not the object in these languages.
Thus, if the register-dependent omission in English and French were restricted to the immediate preverbal position, as is the case with topic drop in German, this would explain the restriction to subjects and the impossibility of omitting objects.
The case of fronted objects addressed by \citet[178--179]{wilder1996} could then be treated analogously to cases of left dislocation \is{Left dislocation} in \ili{German}, which cannot be targeted by topic drop either (see \sectref{sec:intial.theory}).
In both cases, the left dislocated element \is{Left dislocation} is topicalized and, as, for example, \citet[161]{ward.birner1994} discuss for English, made salient \is{Salience} or emphasized, a function that would just no longer be fulfilled if it were omitted.

Regarding (ii), the literature review in \sectref{sec:td.germanic} made it clear that topic drop can also target expletives \is{Expletive} and quasi-arguments at least in Swedish \il{Swedish} and Danish.  \il{Danish}
In \sectref{sec:topicality.ness}, I show that omitting expletives is also possible in German and present corresponding corpus data, which I complement with experimental evidence in \sectref{sec:exp.ex}.
This poses a serious problem not only for the distinction between topic drop and English and French null subjects but also for the syntactic analysis of topic drop as presented in \citet{haegeman1990}, which incorrectly predicts that topic drop cannot target expletives. \is{Expletive|)}
I revisit this issue in \sectref{sec:topicality.ness}.

Finally, concerning (iii), \citet{haegeman.ihsane1999} did find instances of null subjects in embedded \is{Embedding} clauses in diary contexts not only without overt complementizers \is{Complementizer omission} such as \ref{ex:nullsubj.emb.cov} but also with overt complementizers \ref{ex:nullsubj.emb.ov}.%
%% Footnote
\footnote{Note that while example \ref{ex:nullsubj.emb.cov} stems from the authentic diary of the American poet Allen Ginsberg, example in \ref{ex:nullsubj.emb.ov} is taken from a fictional diary, Helen Fielding's \textit{Bridget Jones's diary}.
\citet[130--131]{haegeman.ihsane1999} note that embedded \is{Embedding} null subjects do occur more frequently in the fictional text by Fielding than in Ginsberg's real journals, but still, they do occur there as well.
\citet[131]{haegeman.ihsane1999} argue, referring to other fictional diary texts, that not all of them exhibit a particularly high rate of embedded \is{Embedding} null subjects.
They speculate that \textit{Bridget Jones's diary} is special because it originally appeared as a newspaper column and was subject to a word limit for print purposed ($1\,000$ for \textit{Bridget Jones's diary}) \citep[131]{haegeman.ihsane1999}.
At this point, the question remains as to how common subject omissions after overt complementizers \is{Complementizer} are in English, and to what extent they are a (deliberately chosen) stylistic device in diaries, or whether they also occur in authentic (spoken) speech data.
}
%
This represents a difference from topic drop in German, where the omission in embedded \is{Embedding} clauses with an overt complementizer \is{Complementizer|)} is ungrammatical, as discussed in \sectref{sec:def.prefield}.

\ex.\label{ex:nullsubj.emb.cov} also said $\Delta$ should make girls (Allen Ginsberg: \textit{Journals} \textit{1954--1958}: 99; cited in \cite[128]{haegeman.ihsane1999})

\ex.\label{ex:nullsubj.emb.ov}was worried that $\Delta$ might split (Helen Fielding: \textit{Bridget Jones's Diary}: 227; cited in \cite[128]{haegeman.ihsane1999})

In sum, the similarities between null subjects in English and French and topic drop in the Germanic languages, in particular, the restriction to the left periphery and the typical occurrence in certain registers or text types, are striking. \is{Text type}
Therefore, a joint consideration of the two phenomena seems quite promising, despite the differences that also exist concerning the omission of objects or the co-occurrence with an overt complementizer. \is{Complementizer}
\il{English|)}\il{French|)}\is{Null subject|)}

\subsection{Argument omission in Russian}
\is{Null subject|(} \il{Russian}
While most other Slavic languages are \textit{pro}-drop \is{@\emph{pro}-drop} languages, Russian \il{Russian} is discussed as a mixed type \citep[300--301]{franks1995}.
It shares with \textit{pro}-drop \is{@\emph{pro}-drop} languages the possibility of null arguments, i.e., subjects and objects can optionally be omitted.
However, in contrast to these languages, the overt realization of these arguments does not put a special emphasis on them. \is{Argument}
It is even the case that special pragmatic conditions are required to license null arguments in Russian \il{Russian} (\cite[188--189]{gordishevsky.avrutin2004}, see also \cite[300--301]{franks1995}).
\citet[307--308]{franks1995} defines argument omissions in Russian \il{Russian} \ref{ex:nullsubj.ru} as discourse ellipsis because they only target elements whose reference is clear from the discourse context (see also \cite[120--121]{zdorenko2010}).

\ex. \label{ex:nullsubj.ru}
\ag. A: My	vstret-im-s’a?\\
{} we 	meet-\textsc{pres}.\textsc{1pl}-\textsc{refl}?\\ 
A: `Shall we meet?'
\bg. B: Davajte,  kak $\Delta$ i dogovariv-al-i-s’.\\ 
{} let.us 	as we and agree-\textsc{past}-\textsc{pl}-\textsc{refl}\\
B: `Let’s do it, as (we) already agreed.' \citep[120]{zdorenko2010}

In her corpus \is{Corpus} study of null subjects in Russian,%
%% Footnote
\footnote{Zdorenko searched three written and three spoken subcorpora \is{Corpus} of the Russian \il{Russian} National Corpus for sentences with a verb, excluding imperatives and impersonal constructions.
She annotated the first 100 hits per subcorpus for whether the subject was a noun, a pronoun, or null \citep[124--125]{zdorenko2010}.}
%
\il{Russian} \citet[126]{zdorenko2010} found that, similar to topic drop and to null subjects in English and French, Russian \il{Russian} null subjects occur preferably in spoken colloquial speech where almost a third of the subject pronouns are omitted.%
%% Footnote
\footnote{In the other spoken text types, as well as in the written ones, it was at most 6\%.}
%
Unlike topic drop and the English and French null subjects, in Russian, \il{Russian} there is no restriction of this phenomenon to the left periphery and null subjects can also occur in subordinate clauses \citep[e.g.,][302, his example 30]{franks1995}.
In this respect, Russian \il{Russian} is more similar to the ``classical'' \textit{pro}-drop languages. \is{@\emph{pro}-drop}
It is primarily the typical occurrence in certain modalities, registers, or text types \is{Text type} that brings Russian \il{Russian} argument omission close to topic drop.
However, due to the lack of restriction to the left periphery, it seems more questionable here whether a joint consideration of the two ellipsis types would be useful. \is{Argument omission|)} \is{Null subject|)}

\section{Summary: a typological view on topic drop}\label{sec:typological.summary}
In \sectref{sec:delimitation}, I showed that there are distributional differences and diverging syntactic properties between topic drop in German on the one hand and the similar phenomena \textit{pro}-drop, discourse drop, dialectal null subjects, and V1 declaratives on the other hand that justify a distinction.
While its restriction to the left periphery of declarative main clauses distinguishes it from the positionally unrestricted cases of \textit{pro}-drop, discourse drop, and dialectal null subjects, \is{Null subject} it is also different from V1 declaratives. \is{@\emph{pro}-drop} \is{Discourse \emph{pro}-drop} \is{V1 declarative}
In V1 declaratives, all obligatory verb arguments \is{Argument} are overtly realized, whereas in most topic drop cases either the subject or the object is absent from the sentence. \is{V1 declarative}
It is therefore useful to consider topic drop as an ellipsis type in its own right.

This phenomenon is found not only in German but in almost all Germanic languages, where it is tied to the typical V2 positioning \is{V2 word order} with the resulting prefield restriction. \is{Prefield|(}
My brief overview of several Germanic languages suggests that the omission is always restricted to this position, that subjects can always be omitted and that (direct) objects can often be omitted as well. It also shows that omission occurs primarily in spoken language and conceptually spoken text types. \is{Text type}
Therefore, it seems reasonable to consider topic drop not only as an ellipsis type in its own right but as a common ellipsis type of the Germanic V2 languages. \is{V2 word order}
Consequently, many of the results obtained in this book for topic drop in German should be potentially transferable to topic drop in other languages.
Accordingly, it would be promising to empirically investigate whether topic drop in these languages is also generally unacceptable in the middle field, \is{Middle field} whether certain conjunctions can precede it, and whether its positional restriction can be described as a position that is either the highest [Spec, CP] of a root clause or a [Spec, CP] not c-commanded \is{C-command} sentence-internally by a potential identifier (see the discussion in Chapter \ref{ch:topicality}).
In a second step, it could be examined whether the factors that, according to the second part of this book, play a role in the usage of topic drop (see Chapters \ref{ch:usage.function} to \ref{ch:usage.verb} for corpus and experimental results), such as grammatical person and verb type, are also of relevance for topic drop usage in other languages.
If this is the case, the information-theoretic explanation that I discuss in Chapter \ref{ch:infotheory} might also be transferable, i.e., the usage of topic drop in the other Germanic languages might also be determined, at least in part, by distributing information in a sentence as uniformly as possible to reduce the processing effort for the hearer.

Similar considerations of transferability apply, at least in part, to the cases of argument omission \is{Argument omission} in English, French, and Russian discussed in \sectref{sec:null.subjects.other.languages}.
In all three languages, the omission seems to be particularly frequent in certain modalities, registers, or text types, \is{Text type} exactly as with topic drop.
In addition, at least for English and French, a syntactic restriction of the null subjects \is{Null subject} to the left periphery can be observed, similar to the prefield restriction of topic drop.
Since both languages are languages with a dominant SVO word order but not classic V2 languages, they do not have a prefield position. \is{Prefield|)}
Therefore, one could speculate whether the different word order properties could lead to different manifestations of what is actually a similar ellipsis type.
The impossibility of object omission in English and French may not be a categorical difference between topic drop and null subjects \is{Null subject} but merely the logical consequence of the fact that the immediate preverbal position in English and French is usually only occupied by the subject but not by the object.
Future research is needed to explore the possibilities of a joint analysis at least for Germanic topic drop and English and French register-dependent subject omissions. \is{Null subject}

This summary has again shown the central position of the prefield restriction for topic drop in German since it functions both as a separating  (from other ellipsis types) and as a unifying factor (within the group of Germanic languages).
In Chapter \ref{ch:topicality}, I take a closer look at this central syntactic property and specify it using theoretical discussions and four acceptability rating studies.
