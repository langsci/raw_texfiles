\chapter{Topicality}\label{ch:factor.topicality}
\is{Topic|(}  
In \sectref{sec:topicality}, I argued that topicality is not a (strictly) sufficient condition for topic drop, and, more importantly, that it is also not a necessary one.
From this, I concluded that topicality is not a licensing condition for topic drop.
Nevertheless, it remains an open issue whether topicality still plays a role in topic drop as a factor that impacts its usage.
The guiding question of the present chapter is whether topical constituents are more often targeted by topic drop than non-topical constituents.
This would be expected from an information-theoretic point of view because topics are often held constant across multiple utterances, and thus should be predictable \is{Predictability} and more likely to be targeted by topic drop.

Since I already provided an overview of the relevant theoretical literature on topicality in \sectref{sec:topicality}, I refrain from further theoretical discussion in this chapter.
Instead, I first present the predictions of my information-theoretic account and then turn to the four acceptability ratings experiments on topicality, which, to the best of my knowledge, are the first empirical studies that investigate the role of topicality for topic drop.
It was not possible to examine topicality with the corpus analysis because the FraC is not annotated for information structure.
Since for many utterances, no precontext is available, it would not be feasible to perform such an annotation myself.

\section{Information-theoretic predictions for topicality}\label{sec:info.theory.top}
The information-theoretic predictions regarding topicality follow from the \textit{avoid} \textit{troughs} principle (\sectref{sec:avoid.troughs}) according to which predictable \is{Predictability} constituents are more likely to be targeted by topic drop.
To describe the relationship between topicality and predictability, \is{Predictability} we can start with \citet{krifka2007}.
He points out, referring to \citet{givon1983}, that ``[t]here is a well-documented tendency to keep the topic constant over longer stretches of discourse'' \citep[43]{krifka2007}, i.e., to form so-called \textit{topic chains}.
Consequently, a hearer should expect that in an utterance following such a chain, the topic will remain constant instead of changing.\is{Prefield|(}
If we further realize that topics tend to occur in the prefield (but cf. \sectref{sec:topicality.prefield}), this means that a constituent referring to the topic of the previous utterance is not only likely to generally occur in the current utterance, but is particularly likely to occur in the prefield.
This means that the topic of an utterance in the prefield should generally be more predictable \is{Predictability}, and therefore more likely to be omitted if the speaker adheres to \textit{UID} \is{Uniform information density} and avoids troughs by omitting predictable constituents, where grammar allows it. \is{Prefield|)} 

\section{Experimental investigations of topicality}\label{sec:topicality.experiments}
In the following, I present four acceptability rating studies (experiments \ref*{exp:top.q1} to \ref*{exp:top.s.mv}), which investigated the impact of topicality on topic drop in a controlled setting using minimal pairs.
These studies used three different methods to manipulate the topic status of the prefield constituent: a simple question method, a complex question method, and a method using the subject function.
At the same time, in three of these studies, I also investigated the role of grammatical person experimentally.
The aim was to test the claim from the literature that topic drop of the 1st person singular is particularly frequent or acceptable by comparing it and the corresponding full form to 3rd person singular pronouns referring to humans.
I do not discuss the corresponding results in this chapter but in the next one, specifically in \sectref{sec:usage.person.exp}.

\refstepcounter{expcounter}\label{exp:top.q1}
\subsection{Experiment \arabic{expcounter}: topicality (simple question method) }
\is{Acceptability rating study|(}\label{sec:exp.top.q1.top}
In \sectref{sec:topicality} in the first part of this book, I argued that topicality is not a (strictly) sufficient condition for topic drop and showed by means of corpus and experimental data that it is not a necessary condition either.
In this experiment,%
%
\footnote{This experiment as well as experiments \ref*{exp:top.s.fv} and \ref*{exp:top.s.mv} were part of my bachelor thesis \citep{schafer2019}.
For this book, I reanalyzed the data considering further effects and using a different coding schema.
Additionally, I expanded and modified the conclusions drawn.
The materials and the analysis script can be accessed online: \url{https://osf.io/zh7tr}.}
%
I investigate whether it is at least a factor that impacts the usage of topic drop.
I examined whether topic drop is more acceptable when the omitted prefield constituent is coreferential with the discourse topic, \is{Discourse topic|(} which is set through a question beforehand.
As I discussed in \sectref{sec:info.theory.top}, my information-theoretic  account can explain such an effect.
The discourse topic \is{Discourse topic} is what the corresponding discourse is about and is often held constant across multiple utterances.
This makes it predictable \is{Predictability} and a good candidate for omitting it to avoid a surprisal minimum.

In addition to the effect of topicality, this experiment also investigated the influence of grammatical person on topic drop by comparing the 1st and the 3rd person singular (see \sectref{sec:exp.top.q1.person}).
It crossed the three binary predictors \textsc{Topicality}, grammatical \textsc{Person}, and \textsc{Completeness}, which resulted in a 2 $\times$ 2 $\times$ 2 within-subjects design.
For topicality, an interaction between \textsc{Topicality} and \textsc{Completeness} is predicted.
Topic drop should be more acceptable when the omitted prefield constituent is coreferential with the current discourse topic. \is{Discourse topic|)}

\subsubsection{Materials}\label{sec:exp.top.q1.mat}
\subsubsubsection*{Items}
The items were short instant messaging dialogues between two persons, A and B.
They all exhibited the same overall structure, illustrated in \ref{ex:item.exp.topic.q1}.

\ex.\label{ex:item.exp.topic.q1}
\a. 
\ag.\label{ex:item.exp.topic.q1.ai}A: Na, wie läuft's bei dir?\\
{} well how runs.it at you.\textsc{dat.2sg}\\
A: `Well, how's it going with you?'
\bg.\label{ex:item.exp.topic.q1.aii}B: Vorhin haben Anna und ich den Vertrag unterschrieben.\\
{} earlier have Anna and I the contract signed\\
B: `Earlier Anna and I signed the contract.'
\cg.\label{ex:item.exp.topic.q1.aiii}B: (Ich) befördere sie zur Abteilungsleiterin.\\
{} I promote her to.the head.of.the.department.\textsc{fem}\\
B: `(I) promote her to head of the department.' \\\phantom{.}\hfill (topic drop / full form, 1SG, topic)
\z.
\b. 
\ag.\label{ex:item.exp.topic.q1.bi}A: Na, wie läuft's bei Anna?\\
{} well how runs.it at Anna\\
A: `Well, how's it going with Anna?'
\bg.\label{ex:item.exp.topic.q1.bii}B: Vorhin haben Anna und ich den Vertrag unterschrieben.\\
{} earlier have Anna and I the contract signed\\
B: `Earlier Anna and I signed the contract.'
\cg.\label{ex:item.exp.topic.q1.biii}B: (Ich) befördere sie zur Abteilungsleiterin.\\
{} I promote her to.the head.of.the.department.\textsc{fem}\\
B: `(I) promote her to head of the department.'  \\\phantom{.}\hfill (topic drop / full form, 1SG, no topic)
\z.
\c. 
\ag.\label{ex:item.exp.topic.q1.ci}A: Na, wie läuft's bei Anna?\\
{} well how runs.it at Anna\\
A: `Well, how's it going with Anna?'
\bg.\label{ex:item.exp.topic.q1.cii}B: Vorhin haben Anna und ich den Vertrag unterschrieben.\\
{} earlier have Anna and I the contract signed\\
B: `Earlier Anna and I signed the contract.'
\cg.\label{ex:item.exp.topic.q1.ciii}B: (Sie) befördert mich zur Abteilungsleiterin.\\
{} she promotes me to.the head.of.the.department.\textsc{fem}\\
B: `(She) promotes me to head of the department.'  \\\phantom{.}\hfill (topic drop / full form, 3SG, topic)
\z.
\d. 
\ag.\label{ex:item.exp.topic.q1.di}A: Na, wie läuft's bei dir?\\
{} well how runs.it at you.\textsc{dat.2sg}\\
A: `Well, how's it going with you?'
\bg.\label{ex:item.exp.topic.q1.dii}B: Vorhin haben Anna und ich den Vertrag unterschrieben.\\
{} earlier have Anna and I the contract signed\\
B: `Earlier Anna and I signed the contract.'
\cg.\label{ex:item.exp.topic.q1.diii}B: (Sie) befördert mich zur Abteilungsleiterin.\\
{} she promotes me to.the head.of.the.department.\textsc{fem}\\
B: `(She) promotes me to head of the department.'  \\\phantom{.}\hfill (topic drop / full form, 3SG, no topic)

\largerpage
The conversation started with a question or a request by person A, who wants to get news about either person B, as in \ref{ex:item.exp.topic.q1.ai} and \ref{ex:item.exp.topic.q1.ci}, or a third person, e.g., Anna, as in \ref{ex:item.exp.topic.q1.bi} and \ref{ex:item.exp.topic.q1.di}.
This served the purpose of setting either person B or Anna as the (discourse) topic.
Next, person B provided the requested news by replying with two utterances.
The first utterance, namely \ref{ex:item.exp.topic.q1.aii}, \ref{ex:item.exp.topic.q1.bii}, \ref{ex:item.exp.topic.q1.cii}, or \ref{ex:item.exp.topic.q1.dii}, took up the topic, which was either the speaker, i.e., person B, or the third person, Anna.
Additionally, it introduced the other person, i.e., Anna or person B, by naming an event in which both are involved such as signing a contract.
In this utterance, Anna and person B always appeared as the coordinated subject in the middle field to keep the topic prominent: \textit{Anna und ich} (`Anna and I').
This ordering was held constant across conditions for reasons of politeness (it is considered impolite in German if the speaker names themselves first).
The prefield was always filled with a temporal adverbial, such as \textit{vorhin} (`earlier').

The second utterance of person B was the critical utterance to be rated by the participants.
There, either person B, as in \ref{ex:item.exp.topic.q1.aiii} and \ref{ex:item.exp.topic.q1.biii}, or Anna, as in \ref{ex:item.exp.topic.q1.ciii} and \ref{ex:item.exp.topic.q1.diii}, was picked up as the realized or omitted preverbal subject.
If the person about whom person A wants to get news was identical to the preverbal constituent, this was a \textit{topic} condition, otherwise, it was a \textit{not topic} condition.
The other referent appeared in a subordinate syntactic position as an object or adverbial.
Topic drop was always followed by the inflected form of a lexical verb \is{Lexical verb} in present indicative, which was distinctly marked in terms of grammatical person.

\subsubsubsection*{Fillers}
The 24 items were mixed with 84 fillers, which were also dialogues between two persons.
28 of these fillers were items for an experiment on gapping and right node raising consisting of three turns.
Thus, they also contained (the possibility of) optional omissions.
This was to prevent the items with topic drop from being the only elliptical materials in the experiment.
Seven of the fillers were dialogues with an ungrammatical target utterance containing two finite verbs in a single clause.
I included them as attention checks to be able to exclude the data from inattentive participants from the analysis.
The remaining 49 fillers were dialogues with three or four turns and were designed to represent a broad acceptability spectrum ranging from perfectly acceptable target utterances to marked ones, which should only be slightly more acceptable than the ungrammatical catch trials.
I ensured that none of the filler utterances started with a DP, which could potentially be targeted by topic drop.
This was to prevent participants from being primed by seeing more full forms than utterances with topic drop.

\subsubsection{Procedure}\label{sec:exp.top.q1.procedure}
48 native speakers of German between the ages of 18 and 50 who had not taken part in any of my other studies of topic drop participated in the experiment.
They were recruited from the crowdsourcing platform Clickworker \citep{clickworker2022} and received a compensation of €2.50.
The study was implemented with the LimeSurvey survey presentation software  \citep{limesurveygmbh}.
The participants' task was to rate the naturalness of the last italicized utterance of a small instant messaging dialogue on a 7-point Likert scale with labeled endpoints (1 was ``vollkommen unnatürlich'' (`completely unnatural') and 7 ``vollkommen natürlich'' (`completely natural')).
The items were distributed across eight lists according to a Latin square design so that each participant rated each of the 24 critical token sets exactly once and only in one of the eight conditions.
They were mixed with the fillers and shown to the participants in individual pseudo-randomized order ensuring that no two items or fillers of the same category immediately followed each other.
The utterances were presented as instant messages (similar to those in the previous experiments, see \sectref{sec:exp.ex.procedure}) and were fully lowercased to keep the conditions with and without topic drop parallel in terms of the spelling of the verb.%
% Footnote
\footnote{While writing everything in lowercase is a common stylistic device in text messages \citep[152]{schnitzer2012}, the majority of writers tries to adhere to the rules of capitalization. 
\citet[15--16]{dittmann.etal2007} report that in a corpus \is{Corpus} of 847 text messages by 115 persons collected in 2002, 60.1\% of the text messages exhibit correct capitalization.
The majority of capitalization violations were due to messages that were either completely capitalized or completely lowercased.
Therefore, I abandoned the lowercase spelling in later experiments, such as those presented in the first part of this book, in favor of greater naturalness.}
%


\subsubsection{Results}\label{sec:exp.top.q1.results}
I excluded six participants because they had rated four or more of the ungrammatical control utterances with a 6 or 7 on the Likert scale, which suggests that they had been inattentive.
Table \ref{tab:descriptives.top.q1} shows the mean ratings and standard deviations per condition for the remaining 42 participants.
In Figure \ref{fig:pl.top.q1}, the mean ratings and 95\% confidence intervals are plotted.

\begin{table}
\caption{Mean ratings and standard deviations per condition for experiment \arabic{expcounter}}
\centering
\begin{tabular}{lllrr}
\lsptoprule
\multicolumn{1}{c}{\textsc{Completeness}} & \multicolumn{1}{c}{\textsc{Person}} & \multicolumn{1}{c}{\textsc{Topicality}} & \multicolumn{1}{c}{\Centerstack{Mean\\rating}} & \multicolumn{1}{c}{\Centerstack{Standard \\deviation}} \\
\midrule
Full form & 1SG & Topic & $5.48$ & $1.61$ \\
Topic drop & 1SG & Topic & $4.76$ & $1.87$ \\
Full form  & 3SG & Topic & $5.71$ & $1.45$ \\
Topic drop & 3SG & Topic & $4.13$ & $1.88$ \\
Full form & 1SG & No topic  & $5.45$ & $1.69$ \\
Topic drop  & 1SG & No topic  & $4.90$ & $1.69$ \\
Full form & 3SG & No topic  & $5.60$ & $1.58$ \\
Topic drop &  3SG & No topic & $3.71$ & $1.88$ \\
\lspbottomrule
\end{tabular}
\label{tab:descriptives.top.q1}
\end{table}

\begin{figure}
\centering
\includegraphics[scale=1]{Experimenteplots/PL_TopFrage1.pdf}
 \caption{Mean ratings and 95\% confidence intervals per condition for experiment \arabic{expcounter}}
\label{fig:pl.top.q1}
\end{figure}

\largerpage[-1]
\noindent
The data were analyzed with cumulative link mixed models (CLMMs) \citep{christensen2019} in R \citep{rcoreteam2021}, following the procedure described in \sectref{sec:data.analysis}.
The full model contained the ordinal ratings as the dependent variable and as independent variables the binary predictors \textsc{Completeness}, \textsc{Person}, and \textsc{Topicality}, which were coded using deviation coding (full form, 1SG, and topic were coded as $0.5$, the other levels as $-0.5$), as well as the numeric scaled and centered \textsc{Position} at which the trial appeared in the experiment, the three-way interaction between \textsc{Completeness}, \textsc{Person}, and \textsc{Topicality}%
%% Footnote
\footnote{Although I have no concrete predictions about a joint influence of topicality and grammatical person on the acceptability of topic drop, I included it in my analysis because it could be of theoretical interest, unlike the other possible three-way interactions with \textsc{Position}.}
%
and all two-way interactions between the predictors.
The random effects structure consisted of random intercepts for participants and items and of by-item and by-subject random slopes for all three binary predictors and their two-way interactions, as well as for \textsc{Position}.%
% Footnote
\footnote{The formula of the full model was as follows: \texttt{Ratings \textasciitilde ~\textsc{Completeness} : \textsc{Topicality} : \textsc{Person} + (\textsc{Completeness} + \textsc{Topicality} + \textsc{Person} + \textsc{Position})\textasciicircum2 + (1 + (\textsc{Completeness} + \textsc{Topicality} + \textsc{Person})\textasciicircum2 + \textsc{Position} | Subjects) + (1 + (\textsc{Completeness} + \textsc{Topicality} + \textsc{Person})\textasciicircum2 + \textsc{Position} | Items).}}
I performed a backward model selection to obtain the final model, which in this analysis had symmetric thresholds.
Table \ref{tab:model.exp.top.q1} shows the fixed effects in this model.

\begin{table}
\caption{Fixed effects in the final CLMM of experiment \arabic{expcounter}}
\centering
\begin{tabular}{lrrrll}
\lsptoprule
Fixed effect & Est. & SE & $\chi^2$ & \textit{p}-value &   \\
\midrule
\textsc{Completeness} & $2.03$ & $0.34$ & $26.67$ & $< 0.001$ & ***\\
\textsc{Person} & $0.55$ & $0.23$ & $5.30$ & $< 0.05$ & *\\
\textsc{Completeness $\times$ Person} & $-1.57$ & $0.34$ & $15.07$ & $< 0.001$ & ***\\
\lspbottomrule
\end{tabular}
\label{tab:model.exp.top.q1}
\end{table}

\noindent
There were significant main effects of \textsc{Completeness} ($\chi^2(1) = 26.67$, $p < 0.001$) and \textsc{Person} ($\chi^2(1) = 5.30, p < 0.05$), as well as a significant interaction between them ($\chi^2(1) = 15.07, p < 0.001$).
Full forms received significantly better ratings than utterances with topic drop.
Utterances with the 1st person singular were preferred over utterances with the 3rd person singular.
Topic drop of the 3rd person was particularly degraded.
The main effect of \textsc{Topicality} was not significant ($\chi^2(1) = 1.47, p > 0.2$), nor was the interaction of \textsc{Topicality} with \textsc{Completeness} ($\chi^2(1) = 0.0001, p > 0.9$), nor the three-way interaction between \textsc{Completeness}, \textsc{Topicality}, and \textsc{Person} ($\chi^2(1) = 2.61, p > 0.1$), nor any of the other effects.

\subsubsection{Discussion}\label{sec:exp.top.q1.diss}
Experiment \arabic{expcounter} intended to test for effects of topicality and grammatical person on the acceptability of topic drop.
Whereas the results for grammatical person are discussed in \sectref{sec:exp.top.q1.person}, its results for topicality provide no evidence that this factor plays a role in the acceptability of topic drop.
The predictor \textsc{Topicality} was not involved in any significant effect, i.e., topic drop was not more acceptable when the omitted constituent was coreferential with the current discourse topic. \is{Discourse topic|(}
A possible concern with the design of the experiment is the context sentence, i.e., person B's second utterance.
There, the previously set discourse topic \is{Discourse topic} appeared as part of the coordinated subject \textit{Anna und ich}.
It could be that this subject overwrote the previous topic so that no longer the speaker \textit{or} Anna was the topic but both (see, e.g., \cite[42--43]{krifka2007} for the possibility of having more than one topic in a sentence).
This would cancel out the topicality manipulation because a part of the topic would be omitted in the target sentence in both \textsc{Topicality} conditions.
To circumvent this problem, I conducted a further experiment, namely experiment \ref*{exp:top.q2}.
 \is{Acceptability rating study|)}
 
\refstepcounter{expcounter}\label{exp:top.q2}
\subsection{Experiment \arabic{expcounter}: topicality (complex question method) }
\is{Acceptability rating study|(}\label{sec:exp.top.q2}
Experiment \arabic{expcounter} was an acceptability rating study that again focused on topicality, leaving aside grammatical person,%
% Footnote
\footnote{I restricted myself to topic drop of the 3rd person singular here because this allowed me to have two referents, one topical, the other non-topical, with the same grammatical person.}
%
and, thus, had the form of a 2 $\times$ 2 design (\textsc{Topicality} (topic vs. no topic) $\times$ \textsc{Completeness} (full form vs. topic drop)).%
% Footnote
\footnote{All items, fillers, and the analysis script can be accessed online: \url{https://osf.io/zh7tr}.}
%
Like in experiment \ref*{exp:top.q1}, I set the (discourse) topic with a question, but this time I used two questions instead of one:
the first to set the topic and the second to reinforce it after having introduced a competing referent.
This should reverse any overwriting of the topic as may have occurred in experiment \ref*{exp:top.q1}.
Even if a context sentence with a different subject (e.g., a competing referent or a coordination of competing referent and current topic) were to overwrite the discourse topic \is{Discourse topic} set by the first question, the second question would reestablish the original discourse topic. \is{Discourse topic}
Like in experiment \ref*{exp:top.q1}, the prediction is that if topicality is a favoring factor for topic drop, the conditions with topic drop should be preferred relative to the full forms if the omitted constituent is coreferential with the current discourse topic. \is{Discourse topic}
Experiment \arabic{expcounter} was part of the same study as experiment \ref*{exp:conjunctions}.

\largerpage
\subsubsection{Materials}
\subsubsubsection*{Items}
I constructed 16 token sets, such as \ref{ex:item.top.q2}, which were always short dialogues between two interlocutors A and B talking about two other persons of different gender, e.g., Jennifer and Marcel in \ref{ex:item.top.q2}.
These dialogues took the form of two consecutive question-answer pairs:

\ex.\label{ex:item.top.q2}
\a.
\ag.A: Was gibt's Neues von Jennifer?\label{ex:item.top.q2.a1}\\
{} what gives.it new from Jennifer\\
A: `What's new from Jennifer?'
\bg.\label{ex:item.top.q2.b1}B: Marcel ist inzwischen wieder Single.\\
{} Marcel is meanwhile again single\\
B: `Marcel is now single again.'
\cg.\label{ex:item.top.q2.c1}A: Und was hat das mit Jennifer zu tun?\\
{} and what has that with Jennifer to do\\
A: `And what does that have to do with Jennifer?'
\dg.\label{ex:item.top.q2.d1}B: (Sie) hat ihn nach einem Date gefragt.\\
{} she has him for a date asked\\
B: `(She) asked him out on a date.'\\\phantom{.}\hfill(topic drop / full form, topic)
\z.
\b.
\ag.A: Was gibt's Neues von Marcel?\label{ex:item.top.q2.a2}\\
{} what gives.it new from Marcel\\
A: `What's new from Marcel?'
\bg.\label{ex:item.top.q2.b2}B: Jennifer ist inzwischen wieder Single.\\
{} Jennifer is meanwhile again single\\
B: `Jennifer is now single again.'
\cg.\label{ex:item.top.q2.c2}A: Und was hat das mit Marcel zu tun?\\
{} and what has that with Marcel to do\\
A: `And what does that have to do with Marcel?'
\dg.\label{ex:item.top.q2.d2}B: (Sie) hat ihn nach einem Date gefragt.\\
{} she has him for a date asked\\
B: `(She) asked him out on a date.'\\\phantom{.}\hfill(topic drop / full form, no topic)

First, speaker A asked for information about either Jennifer in \ref{ex:item.top.q2.a1} or Marcel in \ref{ex:item.top.q2.a2}. 
Similar to experiment \ref*{exp:top.q1}, this should set one of them as the (discourse) topic.
Then, speaker B answered only indirectly by giving information about the other person, i.e., Marcel in \ref{ex:item.top.q2.b1} or Jennifer in \ref{ex:item.top.q2.b2}.
This way, Marcel or Jennifer were introduced into the discourse, presumably without changing the (discourse) topic.
In the third turn, A asked for the connection of B's answer to Jennifer in \ref{ex:item.top.q2.c1} or Marcel in \ref{ex:item.top.q2.c2} and, in this way, reinforced Jennifer or Marcel as the discourse topic. \is{Discourse topic}
Finally, B's answer, which established this connection, was the target utterance, i.e., \ref{ex:item.top.q2.d1} and \ref{ex:item.top.q2.d2}.
The omitted or realized feminine subject pronoun \textit{sie} (`she') was always placed in the prefield position, whereas the masculine referent was referred to by an object pronoun in the accusative or the dative case \textit{ihn}/\textit{ihm} (`him').

This means that in the conditions where A set Jennifer as the discourse topic, \is{Discourse topic} the prefield constituent and the discourse topic\is{Discourse topic}  were coreferential, whereas in the conditions where Marcel was set as the discourse topic \is{Discourse topic} they were not (predictor \textsc{Topicality}).
The target utterances were always in the perfect tense with the 3rd person singular form of the auxiliary \is{Auxiliary} \textit{haben} (`have') in the left bracket.
Similar to the 3rd person conditions in experiment \ref*{exp:top.q1}, the target utterances in the topic drop conditions were locally ambiguous \is{Ambiguity} until the participants encountered the gender-marked object pronoun, which allowed them to disambiguate between the two persons of different gender.
Since the structure of the stimuli with four turns including two questions was quite conspicuous and not as natural as the items used in my previous studies, I constructed and tested only 16 token sets to prevent a stronger habituation effect.

\subsubsubsection*{Fillers}
The 80 fillers consisted of the 24 items of experiment \ref*{exp:conjunctions} on topic drop after conjunctions and, as described in detail in \sectref{sec:exp.conj.mat}, of 24 items on preposition omission, 24 (potential) gapping structures, and eight ungrammatical catch trials.

\subsubsection{Procedure}
The procedure has been described in detail in \sectref{sec:ex.conj.prod}.

\subsubsection{Results}
As described already in \sectref{sec:ex.conj.res}, I excluded the data from 14 inattentive participants who had rated four or more of the eight catch trials with 6 or 7.
Table \ref{tab:descriptives.top.q2} shows the mean ratings and standard deviations per condition for the remaining 58 participants.
In Figure \ref{fig:pl.top.q2} the mean ratings and 95\% confidence intervals are plotted.

\begin{table}
\caption{Mean ratings and standard deviations per condition for experiment \arabic{expcounter}}
\centering
\begin{tabular}{llrr}
\lsptoprule
\multicolumn{1}{c}{\textsc{Completeness}} & \multicolumn{1}{c}{\textsc{Topicality}} & \multicolumn{1}{c}{\Centerstack{Mean\\ rating}} & \multicolumn{1}{c}{\Centerstack{Standard\\ deviation}} \\
\midrule
Full form & Topic &  $4.29$ & $1.89$ \\
Topic drop & Topic &  $4.06$ & $1.80$ \\
Full form & No topic &  $4.21$ &  $1.81$\\
Topic drop & No topic &  $3.99$ & $1.79$ \\
\lspbottomrule
\end{tabular}
\label{tab:descriptives.top.q2}
\end{table}

From visual inspection, it does not seem that topic drop of a non-topic was particularly degraded compared to the full forms.
The mean ratings were generally lower than the mean ratings for the items with topic drop after conjunctions (all above 5.3), which may indicate that the topicality items were indeed perceived as marked and less natural because of their unusual structure.
Still, they were rated considerably better than the ungrammatical catch trials (mean = 2.95, SD = 2.16).

\begin{figure}
\centering
\includegraphics[scale=1]{Experimenteplots/PL_TopFrage2.pdf}
\caption{Mean ratings and 95\% confidence intervals per condition for experiment \arabic{expcounter}}
\label{fig:pl.top.q2} % pl for point line
\end{figure}

\noindent I analyzed the data again with CLMMs from the package ordinal \citep{christensen2019} in R, as described in \sectref{sec:data.analysis}.
The full model contained the ordinal ratings as the dependent variable and as independent variables the two binary predictors \textsc{Completeness} and \textsc{Topicality}, which were again coded using deviation coding (full form and topic as $0.5$, the other levels as $-0.5$), as well as the numeric centered and scaled \textsc{Position} of the trial in the experiment.
As random effects, I included random intercepts for subjects and items, as well as by-subject and by-item random slopes for \textsc{Completeness}, \textsc{Topicality}, and their interactions.%
% Footnote
\footnote{Models with random slopes for \textsc{Position} did not converge.
Since this variable is of least theoretical interest, it seems to be relatively unproblematic for the analysis not to include the corresponding random slopes.
The formula of the full model was as follows:
\texttt{Ratings \textasciitilde ~(\textsc{Completeness} + \textsc{Topicality} + \textsc{Position})\textasciicircum2 + (1 + \textsc{Completeness} * \textsc{Topicality} | Subjects) + (1 + \textsc{Completeness} * \textsc{Topicality} | Items)}.
}
Table \ref{tab:model.exp.top.q2} shows the fixed effect in the final model, which had symmetric2 thresholds.

\begin{table}
\caption{Fixed effect in the final CLMM of experiment \arabic{expcounter}}
\centering
\begin{tabular}{lrrrll}
\lsptoprule
Fixed effect & Est. & SE & $\chi^2$ & \textit{p}-value &   \\
\midrule
\textsc{Position} & $-0.38$ & $0.07$ & $27.78$ & $< 0.001$ & ***\\
\lspbottomrule
\end{tabular}
\label{tab:model.exp.top.q2}
\end{table}

\noindent
There was only a theoretically uninteresting significant main effect of \textsc{Position} ($\chi^2(1) = 27.78, p < 0.001$), indicating that the ratings for the items generally decreased over the course of the experiment.
The main effect of \textsc{Completeness} was marginally significant ($\chi^2(1) = 3.49, p < 0.09$), indicating a tendency for the full forms to be preferred over the utterances with topic drop.
The main effect of \textsc{Topicality} was not significant ($\chi^2(1) = 1.04, p > 0.3$), nor were its interaction with \textsc{Completeness} ($\chi^2(1) = 0.0008, p > 0.9$) or any of the other effects.

\subsubsection{Discussion}
Experiment \arabic{expcounter} was designed to investigate again the impact of topicality on the acceptability of topic drop by employing the concept of a discourse topic. \is{Discourse topic|(}
By means of two questions, a referent was unambiguously set as the discourse topic, \is{Discourse topic} thereby circumventing the possible issue of overwriting the topic of experiment \ref*{exp:top.q1}, discussed above.
Similar to experiment \ref*{exp:top.q1}, the results of experiment \arabic{expcounter} provided no evidence that the omission of a constituent that is coreferential with the current discourse topic \is{Discourse topic|)} is more acceptable.
Therefore, there continues to be no support for topicality being a favoring factor for topic drop.
In the next two sections, I describe two further experiments that employed a different method of setting the topic. \is{Acceptability rating study|)}

\refstepcounter{expcounter}\label{exp:top.s.fv}
\subsection{Experiment \arabic{expcounter}: topicality (subject method, lexical verbs) }
\is{Acceptability rating study|(}\label{sec:exp.top.s.fv}
Experiment \arabic{expcounter} was similar to experiment \ref*{exp:top.q1} in that it investigated whether topicality and grammatical person influence the acceptability of topic drop in the form of a  2 $\times$ 2 $\times$ 2 acceptability rating study crossing \textsc{Completeness} (full form vs. topic drop), grammatical \textsc{Person} (1SG vs. 3SG), and \textsc{Topicality} (topic vs. no topic).%
% Footnote
\footnote{This experiment and experiment \ref*{exp:top.s.mv} were part of my bachelor thesis \citep{schafer2019} and were published in \citet{schafer2021}.
In this book, I revised their analyses and newly included a joint analysis of both experiments.
All materials and the analysis script can be found online: \url{https://osf.io/zh7tr}.
}
%
However, while the previous two studies used questions to set the (discourse) topic in the context preceding the critical utterance, in this experiment and experiment \ref*{exp:top.s.mv}, the topic was set via the subject function, as described below.
The prediction is identical to those of experiments \ref*{exp:top.q1} and \ref*{exp:top.q2}. 
If topicality is a favoring factor for topic drop, topic drop should be more acceptable when the omitted constituent is the topic.
For grammatical person, this experiment sought to replicate the effect found in experiment \ref*{exp:top.q1}, according to which topic drop of the 1st person singular is more acceptable than topic drop of the 3rd person singular (see \sectref{sec:exp.top.s.fv.person} for details).

\subsubsection{Background}\label{exp:top.s.fv.background}
In \sectref{sec:topicality.lit}, I already mentioned that subjects can be considered unmarked topics \citep[62]{reinhart1981} because there is a strong tendency across languages for the topic and subject of a sentence to coincide \citep[132]{lambrecht1994}.
Both \citet{reinhart1981} and \citet{lambrecht1994} stress the fact that the concurrence of topicality and subjecthood is only a tendency, i.e., that there are exceptions where subject and topic are distinct.
Nevertheless, it seems promising to exploit this tendency to set the topic experimentally. 

A further tendency for topics is to build so-called \textit{topic chains} in discourse, i.e., having the same topic over larger parts of a discourse \citep{givon1983}.
In combination, both tendencies predict (tentatively of course) that if a constituent is the subject of a sentence, such as Sabrina in \ref{ex:topic.subject.1}, it is also the unmarked topic and it should be more likely that a constituent referring to the same topic is also the topic expression in the following sentence, in particular, if this constituent is also the subject of this sentence, as in \ref{ex:topic.subject.2.ok} vs. \ref{ex:topic.subject.2.nok}.

\exg.\label{ex:topic.subject.1}{Vor Kurzem} ist Sabrina wieder bei mir eingezogen.\\
recently is Sabrina again at me in.moved\\
`Recently Sabrina moved back in with me.'

\ex.
\ag.\label{ex:topic.subject.2.ok}Sie gibt mir noch eine Chance.\\
she gives me yet a chance\\
`She gives me another chance.'
\bg.\label{ex:topic.subject.2.nok}Ich gebe ihr noch eine Chance.\\
I give her yet a chance\\
`I give her another chance.'

This prediction can be partially described through the so-called \textit{centering theory} \citep{grosz.etal1995, walker.etal1998}, a framework originally proposed to determine the reference of anaphora.
The basic idea of centering theory is that each utterance in a discourse segment except for the first one exhibits a so-called \textit{backward-looking center} $C_b$.
According to \citet[3]{walker.etal1998}, this backward-looking center $C_b$  corresponds to the topic of that utterance.
The $C_b$ of an utterance $U_n$ gets selected from the set of all referring expressions of the previous utterance $U_{n-1}$, which are called \textit{forward-looking centers} $C_f$.
It is assumed that these $C_f$ have a certain order, which for \ili{English} and \ili{German} is usually argued to be determined by their syntactic function (see \cite{walker.etal1998} for \ili{English}, \cite{speyer2007} for German, \il{German|(} and \cite{walker.etal1998} for a different assumed hierarchy in \ili{Japanese}), i.e., subjects > objects > adverbials.
It is the highest ranked element of the \textit{forward-looking centers} $C_f$ of the preceding utterance $U_{n-1}$ that occurs in the current utterance $U_n$ that is selected as \textit{backward-looking center} $C_b$, i.e., as topic, of this utterance $U_n$.

According to this mechanism, the $C_b$ of \ref{ex:topic.subject.2.ok} is the expression \textit{sie}, whereas in \ref{ex:topic.subject.2.nok} it is \textit{ihr}, both referring to Sabrina.
As the subject of the preceding utterance $U_{n-1}$ \ref{ex:topic.subject.1}, \textit{Sabrina} is the highest-ranked element of the forward-looking centers of this utterance $C_f = \{\textrm{\textit{Sabrina}} > \textrm{\textit{bei mir}}\}$ that occurs in the target utterances \ref{ex:topic.subject.2.ok} and \ref{ex:topic.subject.2.nok}.
From this, it follows that in \ref{ex:topic.subject.2.ok}, the $C_b$ occurs in the prefield, whereas in \ref{ex:topic.subject.2.nok} it occurs in the middle field. \is{Middle field}
Put differently, in \ref{ex:topic.subject.2.ok}, the prefield constituent is identical to the topic of the utterance and coreferential with the highest-ranked $C_f$-element of the previous utterance, whereas in \ref{ex:topic.subject.2.nok}, the prefield constituent is distinct from both.

Note that, under a different analysis than the one proposed by centering theory, one might assume that in \ref{ex:topic.subject.2.nok} the prefield constituent \textit{ich} is still the topic expression of the utterance, but it refers to a topic that is distinct from the topic of the previous utterance.
That means that one has to assume a shift in topics between both sentences, which ends a topic chain.
Under this analysis, topic drop both of \ref{ex:topic.subject.2.ok} and of \ref{ex:topic.subject.2.nok} would indeed target a topic but in one case a ``better'' topic, which is part of a topic chain.

While I adopt the interpretation derived from centering theory, according to which the prefield constituent in \ref{ex:topic.subject.2.nok} is not a topic expression, both interpretations predict that topic drop of \ref{ex:topic.subject.2.ok} is more acceptable than topic drop of \ref{ex:topic.subject.2.nok}, provided that one assumes that topicality affects the acceptability of topic drop.

It is important to reiterate at this point that topic continuity and subject continuity coincide in this manipulation of topicality.
It has been repeatedly argued that hearers have a preference for interpreting pronouns as coreferential with the subject of the previous utterance \citep[e.g.,][]{crawley.etal1990, jarvikivi.etal2005, kaiser2009}, in particular in German \citep[e.g.,][]{hemforth.etal2010, colonna.etal2012}.\il{German|)}
Therefore, hearers may also prefer elliptical utterances if the omitted constituent is coreferential with the subject of the preceding utterance, as proposed, for example, by \citet[200]{auer1993}.
With this experiment and experiment \ref*{exp:top.s.mv}, I cannot separate a potential effect of topic continuity from one of subject continuity.
From the perspective of the information-theoretic approach, however, such a distinction is not essential because both types of continuity should make the prefield constituent more likely and its omission thus more acceptable.

\largerpage
\subsubsection{Materials}\label{sec:exp.top.s.fv.materials}
\subsubsubsection*{Items}
I constructed 24 items such as \ref{ex:td.top.full.item} as short dialogues between two persons.

\ex.\label{ex:td.top.full.item}
\a.\label{ex:td.top.full.itema}
\ag.\label{ex:td.top.full.itemai}A: Hallo, wie sieht's aus?\\
{} hello how looks.it \textsc{vpart}\\
 A: `Hello, what's the situation?'
\bg.\label{ex:td.top.full.itemaii}B: {Vor Kurzem} bin ich wieder bei Thomas eingezogen.\\
{} recently am I again at Thomas in.moved\\
B: `Recently I moved back in with Thomas.'
\cg.\label{ex:td.top.full.itemaiii}B: (Ich) gebe ihm noch eine Chance.\\
{} I give him yet a chance\\
B: `(I) give him another chance.' \\\phantom{.}\hfill (topic drop / full form, 1SG, topic)
\z.
\b.\label{ex:td.top.full.itemb}
\ag.\label{ex:td.top.full.itembi}A: Hallo, wie sieht's aus?\\
{} hello how looks.it \textsc{vpart}\\
 A: `Hello, what's the situation?'
\bg.\label{ex:td.top.full.itembii}B: {Vor Kurzem} ist Claudia wieder bei mir eingezogen.\\
{} recently is Claudia again at me in.moved\\
B: `Recently Claudia moved back in with me.'
\cg.\label{ex:td.top.full.itembiii}B: (Ich) gebe ihr noch eine Chance.\\
{} I give her yet a chance\\
B: `(I) give her another chance.' \\\phantom{.}\hfill(topic drop / full form, 1SG, no topic)
\z.
\c.\label{ex:td.top.full.itemc}
\ag.\label{ex:td.top.full.itemci}A: Hallo, wie sieht's aus?\\
{} hello how looks.it \textsc{vpart}\\
 A: `Hello, what's the situation?'
\bg.\label{ex:td.top.full.itemcii}B: {Vor Kurzem} ist Sabrina wieder bei mir eingezogen.\\
{} recently is Sabrina again at me in.moved\\
B: `Recently Sabrina moved back in with me.'
\cg.\label{ex:td.top.full.itemciii}B: (Sie) gibt mir noch eine Chance.\\
{} she gives me yet a chance\\
B: `(She) gives me another chance.' \\\phantom{.}\hfill (topic drop / full form, 3SG, topic)
\z.
\d.\label{ex:td.top.full.itemd}
\ag.\label{ex:td.top.full.itemdi}A: Hallo, wie sieht's aus?\\
{} hello how looks.it \textsc{vpart}\\
 A: `Hello, what's the situation?'
\bg.\label{ex:td.top.full.itemdii}B: {Vor Kurzem} bin ich wieder bei Patrick eingezogen.\\
{} recently am I again at Patrick in.moved\\
B: `Recently I moved back in with Patrick.'
\cg.\label{ex:td.top.full.itemdiii}B: (Er) gibt mir noch eine Chance.\\
{} he gives me yet a chance\\
B: `(He) gives me another chance.' \\\phantom{.}\hfill (topic drop / full form, 3SG, no topic)

\largerpage[2]
The conversations had the following pattern:
Person A asked a non-specific question, e.g., \ref{ex:td.top.full.itemai}, which, unlike in experiments \ref*{exp:top.q1} and \ref*{exp:top.q2}, did not introduce any discourse referents. 
Person B answered with two utterances.
In the first utterance, the speaker, person B, mentioned themselves and a 3rd person in the form of a proper name such as \textit{Thomas}.%
%% Footnote
\footnote{Note that in this experiment and in experiment \ref{exp:top.s.mv}, the proper names of this 3rd person were varied within a token set.
Of course, there was always a match between the gender of the proper name in the context sentence and the gender of the personal pronoun in the target sentence.}
%
More specifically, person B uttered that they and this 3rd person participate in an action such as moving in.
It was varied whether the speaker, as in \ref{ex:td.top.full.itemaii} and \ref{ex:td.top.full.itemdii}, or the 3rd person, as in \ref{ex:td.top.full.itembii} and \ref{ex:td.top.full.itemcii}, is the subject of the utterance.
The other person was mentioned in a syntactically subordinate role as a prepositional object or adverbial and should therefore be less accessible as a topic (or as a backward-looking center) in the following sentence.
In this first utterance of person B, the subject always appeared in the middle field, while the prefield was filled with a temporal adverbial.
This way, structural parallelism effects similar to those found to impact pronoun resolution, i.e., that pronouns tend to be interpreted as referring to an element in the same syntactic position \citep{smyth1994, chambers.smyth1998}, should be alleviated.%
% Footnote
\footnote{Note that this strategy mainly prevents effects of the surface structure, i.e., between two constituents in the prefield.
It cannot rule out the possibility that there is an effect of subject continuity, as discussed above.}
%
The last utterance picked up both referents, i.e., the speaker and the 3rd person.
It was varied which of them appeared as the (realized or omitted) subject in the prefield and which, consequently, as the object in the middle field.
If the prefield constituent of the target utterance was identical to the subject of the previous utterance, as in \ref{ex:td.top.full.itema} and \ref{ex:td.top.full.itemc}, it is a \textit{topic} condition. If both were distinct, as in \ref{ex:td.top.full.itemb} and \ref{ex:td.top.full.itemd}, it is a \textit{no topic} condition.
In this experiment, the verb in the left bracket was always a lexical verb \is{Lexical verb} in the indicative present with a distinct inflectional marking.

\subsubsubsection*{Fillers}
The 24 items described above were mixed with 80 fillers, which were basically identical to those used in experiment \ref*{exp:top.q1} including seven ungrammatical attention checks.
The only difference was that I reduced the number of fillers containing (potential) instances of gapping and right node raising from 28 to 24 for simplification.

\largerpage[2]
\subsubsection{Procedure}\label{sec:exp.top.s.fv.procedure}
I recruited 48 self-reported native speakers of German between the ages 18 and 50 from Clickworker \citep{clickworker2022}, who had not participated in any of my other experiments on topic drop.
They received €2.50 to participate in the experiment, which I implemented again with LimeSurvey \citep{limesurveygmbh}.
The task of the participants was to rate the naturalness of the last italicized utterance on a 7-point Likert scale (7 = completely natural).
I used a Latin square design to distribute the materials among eight lists and showed them to the participants along with the fillers in an individual pseudo-randomized order, ensuring that no two items or fillers of the same type immediately followed each other.
Each participant saw each of the 24 critical items once and in only one condition. 
The materials were again presented in lowercase and as instant messaging dialogues (see \sectref{sec:exp.ex.procedure}).

\subsubsection{Results}\label{sec:exp.top.fv.top.results}
I excluded the data from five participants who had rated four or more of the seven ungrammatical attention checks with 6 or 7 points on the 7-point scale, i.e., as completely natural or almost completely natural, which suggests that they had been inattentive during the task.
The descriptive statistics for the rating data, i.e., the mean ratings and standard deviations per condition, are shown in Table \ref{tab:descriptives.top.full.gp}.

\begin{table}
\caption{Mean ratings and standard deviations per condition for experiment \arabic{expcounter}}
\centering
\begin{tabular}{lllrr}
\lsptoprule
\multicolumn{1}{c}{\textsc{Completeness}} & \multicolumn{1}{c}{\textsc{Person}} & \multicolumn{1}{c}{\textsc{Topicality}} & \multicolumn{1}{c}{\Centerstack{Mean \\rating}} & \multicolumn{1}{c}{\Centerstack{Standard\\ deviation}} \\
\midrule
Full form & 1SG & Topic & $5.64$ & $1.44$ \\
Topic drop & 1SG & Topic & $5.38$ & $1.40$ \\
Full form & 3SG & Topic & $5.58$ & $1.43$ \\
Topic drop & 3SG & Topic & $4.65$ & $1.60$ \\
Full form & 1SG & No topic & $5.58$ & $1.37$ \\
Topic drop & 1SG & No topic & $4.91$ & $1.46$ \\
Full form & 3SG & No topic & $5.67$ & $1.45$ \\
Topic drop & 3SG & No topic & $4.21$ & $1.67$ \\
\lspbottomrule
\end{tabular}
\label{tab:descriptives.top.full.gp}
\end{table}

\noindent
Figure \ref{fig:pl.top.full} illustrates the mean ratings also graphically, including 95\% confidence intervals.
They suggest that topic drop of a non-topical constituent was degraded compared to topic drop of a topical constituent.

\begin{figure}
\includegraphics[scale=1]{Experimenteplots/PL_TopFV.pdf}
\caption{Mean ratings and 95\% confidence intervals per condition for experiment \arabic{expcounter}}
\label{fig:pl.top.full} % pl for point line
\end{figure}

I analyzed the data using CLMMs from the package ordinal \citep{christensen2019} in R, following the procedure described in \sectref{sec:data.analysis} and similar to experiment \ref*{exp:top.q1}.
The participants' responses were modeled as a function of the three binary predictors \textsc{Completeness}, grammatical \textsc{Person}, and \textsc{Topicality}, coded using deviation coding (full form, 1SG and topic coded as $0.5$, topic drop, 3SG and no topic coded as $-0.5$), as well as the numeric scaled and centered \textsc{Position} at which the trial appeared in the experiment.
Additionally, I included the three-way interaction between \textsc{Completeness}, \textsc{Person}, and \textsc{Topicality}, as well as all two-way interactions between the four independent variables.
The random effects structure consisted of random intercepts for participants and items and by-item and by-subject random slopes for all independent variables, and the two-way interactions between \textsc{Completeness}, \textsc{Person}, and \textsc{Topicality}.%
% Footnote: model call
\footnote{The formula of the full model was as follows: \texttt{Ratings \textasciitilde ~\textsc{Completeness} : \textsc{Person} :\textsc{Topicality} + (\textsc{Completeness} + \textsc{Person} + \textsc{Topicality} + \textsc{Position})\textasciicircum2 + 
(1 + (\textsc{Completeness} + \textsc{Person} + \textsc{Topicality})\textasciicircum2 + \textsc{Position} | Subjects) + 
(1 + (\textsc{Completeness} + \textsc{Person} + \textsc{Topicality})\textasciicircum2 + \textsc{Position} | Items)}.}
%
The fixed effects in the final CLMM are shown in Table \ref{tab:model.exp.top.fv}.

\begin{table}
\caption{Fixed effects in the final CLMM of experiment \arabic{expcounter}}
\centering
\begin{tabular}{lrrrll}
\lsptoprule
Fixed effect & Est. & SE & $\chi^2$ & \textit{p}-value &   \\
\midrule
\textsc{Completeness} & $1.72$ & $0.35$ & $20.75$ & $< 0.001$ & ***\\
\textsc{Person} & $0.52$ & $0.20$ & $6.10$ & $< 0.05$ & *\\
\textsc{Topicality} & $0.38$ & $0.15$ & $6.69$ & $< 0.01$ & ** \\
\textsc{Completeness $\times$ Person} & $-1.33$ & $0.35$ & $12.85$ & $< 0.001$ & ***\\
\textsc{Completeness $\times$ Topicality} & $-0.81$ & $0.33$ & $5.95$ & $< 0.05$ & *\\
\lspbottomrule
\end{tabular}
\label{tab:model.exp.top.fv}
\end{table}

\noindent
The final model contained a significant interaction between \textsc{Completeness} and \textsc{Topicality} ($\chi^2(1) = 5.95, p < 0.05$), indicating that utterances with topic drop were rated better if the omitted constituent was the topic expression.
The significant interaction between \textsc{Completeness} and \textsc{Person} ($\chi^2(1) = 12.85, p < 0.001$) suggests that topic drop of the 1st person singular was more acceptable than topic drop of the 3rd person singular.
Additionally, there were significant main effects of all three binary predictors \textsc{Completeness} ($\chi^2(1) = 20.75, p < 0.001$), \textsc{Person} ($\chi^2 = 6.1, p < 0.05$), and \textsc{Topicality} ($\chi^2(1) = 6.69, p < 0.01$).
Utterances were more acceptable if they were syntactically complete, if the preverbal constituent was an overt or covert 1st person singular pronoun, and if it was the topic.

\subsubsection{Discussion}\label{sec:exp.top.fv.diss}
Experiment \arabic{expcounter} was designed to investigate the impact of topicality and grammatical person on the acceptability of topic drop in German.
In this respect, it was similar to experiment \ref*{exp:top.q1}, and, concerning topicality, also to experiment \ref*{exp:top.q2}.
However, the topic was set not using a question method as in these two experiments but through the subject function, exploiting the frequent concurrence of the topic and the subject.

Unlike the previous experiments, the results of experiment \arabic{expcounter} support an impact of topicality on topic drop.
While utterances with a topical prefield constituent were generally preferred, this effect was particularly strong for topic drop.
It seems that omitting a topic expression is more acceptable than omitting a non-topic expression if the topic was set using the subject function.
As pointed out in \sectref{sec:info.theory.top}, the information-theoretic account can explain this pattern.
Given the general tendency to build topic chains in discourse, the topic of an utterance can become predictable from the previous utterance and predictable \is{Predictability} constituents can better be omitted than unpredictable ones.

However, the conclusion that topicality is a favoring factor for topic drop has to be drawn cautiously.
On the one hand, the results of the following study question a general influence (see \sectref{sec:exp.top.s.mv}).
On the other hand, the topicality manipulation coincides with subject constancy in this experiment and the next one.
That is, in the conditions in which the (realized or omitted) subject was the topic of the target utterance, it was at the same time identical to the subject of the previous utterance.
As mentioned above, I tried to prevent a positional parallelism effect by placing the subject of the context utterance not in the prefield but in the middle field.
But possible parallelism effects due to the same syntactic function remained unaffected by this solution and could simply not be prevented here due to the kind of topic manipulation.
As mentioned above, my information-theoretic approach would also be able to explain the effect of subject continuity since subject continuity is also expected to increase the likelihood of the prefield constituent.
I revisit this discussion in \sectref{sec:exp.top.s.mv.diss}. \is{Acceptability rating study|)}

\largerpage
\refstepcounter{expcounter}\label{exp:top.s.mv}
\subsection{Experiment \arabic{expcounter}: topicality (subject method, modal verbs) }
\is{Acceptability rating study|(}\label{sec:exp.top.s.mv}
Experiment \arabic{expcounter} was a replication of experiment \ref*{exp:top.s.fv} with modal verbs \is{Modal verb} instead of lexical verbs.%
% Footnote
\footnote{All materials and the analysis scripts are available online: \url{https://osf.io/zh7tr}.}
%
Since the forms of the modal verbs \is{Modal verb} for the 1st and the 3rd person singular are syncretic \is{Syncretism} (e.g., \textit{ich muss} (`I must') and \textit{Tino muss} (`Tino must')), in this experiment, unlike in experiment \ref*{exp:top.s.fv}, there was no distinct inflectional ending on the verb that could serve as a cue to facilitate recovering the omitted preverbal subject. \is{Recoverability|(}
This omitted subject could, however, clearly be recovered based on the object pronoun referring to the competing referent that was also present in the utterances.
In other words, the corresponding utterances were only locally ambiguous. \is{Ambiguity|(}
While this experiment together with experiment \ref*{exp:top.s.fv} with the distinctly marked lexical verbs examined whether the factor of verbal inflection impacts the acceptability of topic drop (see \sectref{sec:exp.top.s.mv.person}), I also again investigated \textsc{Topicality}.
The intention was simply to replicate the result of experiment \ref*{exp:top.s.fv} according to which topic drop of a topical prefield constituent is preferred over topic drop of a non-topical one if the topic was set via the subject function.
This resulted again in a 2 $\times$ 2 $\times$ 2 design, which crossed  \textsc{Completeness}, grammatical \textsc{Person}, and \textsc{Topicality}.

In this section, in addition to analyzing experiment \arabic{expcounter} according to the established scheme, I also present a joint post hoc analysis of experiments \ref*{exp:top.s.fv} and \arabic{expcounter}, in which I considered \textsc{Verb Type} (full verb vs. modal verb) as a variable manipulated between-subjects.
This had the purpose of investigating whether the distinct inflectional marking has an impact on the acceptability of topic drop, in combination with grammatical person or as a factor on its own (see \sectref{sec:exp.top.s.mv.person}).

\largerpage[2]
\subsubsection{Materials}\label{sec:exp.top.s.mv.materials}
\subsubsubsection*{Items}
The items were identical to the 24 items of experiment \ref*{exp:top.s.fv}, except that the left bracket was filled with a modal verb \is{Modal verb} instead of a lexical verb.
The infinitive lexical verb was consequently moved to the right bracket.
The modal verbs \is{Modal verb} were varied between items so that the target utterance was still as natural as possible.
Eight utterances contained \textit{wollen} (`want'), five utterances \textit{dürfen} (`may'), four utterances \textit{sollen} (`shall'), three utterances \textit{können} (`can'), three utterances \textit{mögen} (`would like'), and one utterance \textit{müssen} (`must').
The comparison between \ref{ex:td.top.modal.item.1sg} and \ref{ex:td.top.modal.item.3sg} shows that both verb forms were identical.
When encountering the verb, the participants could not use the inflectional ending as a cue to recover \is{Recoverability|)} the omitted subject.
However, the items were only locally ambiguous \is{Ambiguity|)} since the object pronoun referring to one referent always clarified that the other referent had to be the omitted subject.%
% Footnote
\footnote{The acceptability rating study as an offline task is not suitable to observe the disambiguation \is{Ambiguity} process itself.
To accomplish this, an on-line method like self-paced reading or eye tracking would be necessary.
The effort \is{Processing effort} associated with disambiguating could be quantified by measuring reading times on the object pronoun and a spillover region.
If speakers have an a priori preference for a 1st person singular pronoun, they should slow down when encountering a 1st person singular object pronoun that rules out this reading compared to a 3rd person singular object pronoun.
}
%

\ex.\label{ex:td.top.modal.item}
\ag.\label{ex:td.top.modal.item.1sg}(Ich) will ihr noch eine Chance geben.\\
 I want her yet a chance give\\
`(I) want to give her another chance.'\hfill (topic drop / full form, 1SG)
\bg.\label{ex:td.top.modal.item.3sg}(Sie) will mir noch eine Chance geben.\\
she wants me yet a chance give\\
`(She) wants to give me another chance.'\hfill (topic drop / full form, 3SG)

\subsubsubsection*{Fillers}
The items were mixed with the same 80 fillers as in experiment \ref*{exp:top.s.fv}.

\subsubsection{Procedure}\label{sec:exp.top.s.mv.procedure}
48 self-reported native speakers of German between 18 and 40 years participated in the study for a reward of €2.50.
They had not taken part in any previous experiment on topic drop and were recruited from Clickworker \citep{clickworker2022}.
The study was implemented with LimeSurvey \citep{limesurveygmbh} and conducted online.
Just like in the previous experiments, the participants rated the naturalness of the target utterances on a 7-point Likert scale (7 = completely natural).
The 24 adapted items were distributed across eight lists according to a Latin square design and presented as instant messaging dialogues together with the fillers in individually pseudo-randomized order.

\subsubsection{Results}\label{sec:exp.top.s.mv.res}
\subsubsubsection{Analysis of experiment \arabic{expcounter}}
Since all participants passed the attention checks, the complete data was used for the analysis.
Table \ref{tab:descriptives.top.modal.gp} shows the mean ratings and standard deviations per condition.

\begin{table}
\caption{Mean ratings and standard deviations per condition for experiment \arabic{expcounter}}
\centering
\begin{tabular}{lllrr}
\lsptoprule
\multicolumn{1}{c}{\textsc{Completeness}} & \multicolumn{1}{c}{\textsc{Person}} & \multicolumn{1}{c}{\textsc{Topicality}} & \multicolumn{1}{c}{\Centerstack{Mean \\ rating}} & \multicolumn{1}{c}{\Centerstack{Standard\\ deviation}} \\
\midrule
Full form & 1SG & Topic & $5.41$ & $1.55$ \\
Topic drop & 1SG & Topic & $5.24$ &  $1.67$ \\
Full form & 3SG & Topic & $5.52$ & $1.43$ \\
Topic drop & 3SG & Topic & $4.43$ & $1.93$\\
Full form & 1SG & No topic & $5.35$ &  $1.67$\\
Topic drop & 1SG & No topic & $4.86$ & $1.77$\\
Full form & 3SG & No topic & $5.25$ &  $1.74$\\
Topic drop & 3SG & No topic &  $4.27$ & $1.77$\\
\lspbottomrule
\end{tabular}
\label{tab:descriptives.top.modal.gp}
\end{table}

In Figure \ref{fig:pl.top.modal}, the mean ratings are plotted including 95\% confidence intervals.
While omitting a non-topic seems to be degraded, also the full forms with a non-topical prefield constituent were worse than the conditions with a topical prefield constituent.

\begin{figure}
\centering
\includegraphics[scale=1]{Experimenteplots/PL_TopMV.pdf}
\caption{Mean ratings and 95\% confidence intervals per condition for experiment \arabic{expcounter}}
\label{fig:pl.top.modal} % pl for point line
\end{figure}

Again, I used CLMMs from the package ordinal \citep{christensen2019} for the inferential statistical analysis (see \sectref{sec:data.analysis} for details on the procedure).
The full model was identical to the one described for experiment \ref*{exp:top.s.fv}.
All predictors were coded as described in \sectref{sec:exp.top.fv.top.results}.%
% Footnote
\footnote{The formula of the full model was as follows: \texttt{Ratings \textasciitilde ~\textsc{Completeness} : \textsc{Person} : \textsc{Topicality} + (\textsc{Completeness} + \textsc{Person} + \textsc{Topicality} + \textsc{Position})\textasciicircum2 + 
(1 + (\textsc{Completeness} + \textsc{Person} + \textsc{Topicality})\textasciicircum2 + \textsc{Position} | Subjects) + 
(1 + (\textsc{Completeness} + \textsc{Person} + \textsc{Topicality})\textasciicircum2 + \textsc{Position} | Items)}.}
%
The fixed effects in the final model with flexible thresholds are shown in Table \ref{tab:model.exp.top.mv}.

\begin{table}
\caption{Fixed effects in the final CLMM of experiment \arabic{expcounter}}
\centering
\begin{tabular}{lrrrll}
\lsptoprule
Fixed effect & Est. & SE & $\chi^2$ & \textit{p}-value &   \\
\midrule
\textsc{Completeness} & $1.08$ & $0.23$ & $18.07$ & $< 0.001$ & ***\\
\textsc{Person} & $0.61$ & $0.17$ & $10.67$ & $< 0.01$ & **\\
\textsc{Completeness $\times$ Person} & $-0.98$ & $0.29$ & $9.43$ & $< 0.01$ & **\\
\lspbottomrule
\end{tabular}
\label{tab:model.exp.top.mv}
\end{table}

\noindent
There were significant main effects of \textsc{Completeness} ($\chi^2(1) = 18.07, p < 0.001$) and \textsc{Person} ($\chi^2(1) = 10.67, p < 0.01$) and a significant interaction between both predictors ($\chi^2(1) = 9.43, p < 0.01$).
Utterances with topic drop were degraded compared to complete utterances, utterances with the 3rd person singular were degraded compared to  utterances with the 1st person singular, and utterances with topic drop of a 3rd person singular prefield constituent were particularly degraded.
Neither the main effect of \textsc{Topicality} was significant ($\chi^2(1) = 1.18, p > 0.2$), nor its interaction with \textsc{Completeness} ($\chi^2(1) = 1.64, p > 0.1$), nor any other effect.

\subsubsubsection{Analysis of experiments \ref*{exp:top.s.fv} and \arabic{expcounter}}
In a post hoc analysis, I combined the data from experiments \ref*{exp:top.s.fv} and \arabic{expcounter} to look for a potential influence of the verb in the left bracket and its inflectional ending on topic drop.
I added \textsc{Verb Type} as a new predictor that distinguishes between the lexical verbs \is{Lexical verb} with a distinct inflectional ending used in experiment \ref*{exp:top.s.fv} and the modal verbs \is{Modal verb|(} with syncretic forms \is{Syncretism|(} used in experiment \arabic{expcounter}.
I coded this new predictor using deviation coding: $0.5$ for the lexical verbs and $-0.5$ for the modal verbs. 
I analyzed the combined data again with CLMMs \citep{christensen2019}.
Since the \textsc{Position} of the trial in the experiment was not involved in any significant effect in the analyses of experiments \ref*{exp:top.s.fv} and \arabic{expcounter}, I did not include it as a predictor in this post hoc analysis, which allowed me to use a simpler model.
I modeled the ordinal ratings as a function of the deviation-coded binary predictors \textsc{Completeness}, \textsc{Person}, \textsc{Topicality}, and \textsc{Verb Type}, including all three- and two-way interactions between them.
The random effects structure consisted of random intercepts for subjects and items and by-subject random slopes for \textsc{Completeness}, \textsc{Person}, \textsc{Topicality}, and their two-way interactions and by-item random slopes for \textsc{Completeness}, \textsc{Person}, \textsc{Topicality}, \textsc{Verb Type}, and their two-way interactions.%
% Footnote
\footnote{The formula of the full model was as follows:
\texttt{Ratings \textasciitilde ~(\textsc{Completeness} + \textsc{Person} + \textsc{Topicality} + \textsc{Verb Type})\textasciicircum3 + (1 + (\textsc{Completeness} + \textsc{Person} + \textsc{Topicality})\textasciicircum2 | Subjects) + (1 + (\textsc{Completeness} + \textsc{Person} + \textsc{Topicality} + \textsc{Verb Type})\textasciicircum2 | Items).}}
%

Table \ref{tab:model.exp.top.bv} shows the fixed effects in the final model, which had flexible thresholds and was obtained with a backward model selection.

\begin{table}
\caption{Fixed effects in the final CLMM of the joint analysis of experiments \ref*{exp:top.s.fv} and \arabic{expcounter}}
\centering
\begin{tabular}{lrrrll}
\lsptoprule
Fixed effect & Est. & SE & $\chi^2$ & \textit{p}-value &   \\
\midrule
\textsc{Completeness} & $1.10$ & $0.18$ & $30.41$ & $< 0.001$ & ***\\
\textsc{Person} & $0.38$ & $0.12$ &  $8.52$ & $< 0.01$ & **\\
\textsc{Topicality} & $0.24$ & $0.10$ & $5.73$ & $< 0.05$ & *\\
\textsc{Completeness $\times$ Person} & $-1.02$ & $0.19$ & $22.24$ & $< 0.001$ & ***\\
\textsc{Completeness $\times$ Topicality} & $-0.53$ & $0.20$ & $6.98$ & $< 0.01$ & **\\
\lspbottomrule
\end{tabular}
\label{tab:model.exp.top.bv}
\end{table}

\noindent
The predictor \textsc{Verb Type} was not involved in any significant effect.
In particular, neither the interaction with \textsc{Completeness} ($\chi^2(1) = 0.3, p > 0.5$), nor the interaction with \textsc{Completeness} and \textsc{Person} ($\chi^2(1) = 0.005$, $p > 0.9$) were significant.
Topic drop before a modal verb \is{Modal verb|)} was not significantly more or less acceptable than topic drop before a lexical verb, \is{Lexical verb} nor did topic drop of the 1st person singular have an advantage before a distinctly marked lexical verb \is{Lexical verb} in comparison to a syncretic form of a modal verb. \is{Modal verb}
The significant effects in the final model replicated those found in the analysis of experiment \ref*{exp:top.s.fv} rather than those in the analysis of experiment \arabic{expcounter}.
There were significant main effects of \textsc{Completeness} ($\chi^2(1) = 30.41, p < 0.001$), \textsc{Person} ($\chi^2(1) = 8.52, p < 0.01$), and \textsc{Topicality} ($\chi^2(1) = 5.73, p < 0.05$).
Furthermore, \textsc{Completeness} interacted significantly with \textsc{Person} ($\chi^2(1) = 22.24, p < 0.001$) and with \textsc{Topicality} ($\chi^2(1) = 6.98, p < 0.01$).
Full forms were preferred over topic drop, utterances with the 1st person singular were preferred over utterances with the 3rd person singular, and utterances with a topical prefield constituent were preferred over utterances with a non-topical prefield constituent.
Topic drop of the 3rd person singular and topic drop of a non-topical prefield constituent were particularly degraded.

\subsubsection{Discussion}\label{sec:exp.top.s.mv.diss}
Concerning topicality, experiment \arabic{expcounter} had the purpose of replicating the effect of topicality found in experiment \ref*{exp:top.s.fv}.
With a joint post hoc analysis of experiments \ref*{exp:top.s.fv} and \arabic{expcounter}, I furthermore investigated whether there is an impact of verbal inflection on topic drop and a potential interaction with topicality.

For topicality, a complex picture emerges from the analysis of experiment \arabic{expcounter} and the joint post hoc analysis of experiments \ref*{exp:top.s.fv} and \arabic{expcounter} (see \sectref{sec:exp.top.s.mv.person.diss} for a discussion with respect to grammatical person).
While the interaction between \textsc{Completeness} and \textsc{Topicality} was still significant in the joint analysis of both experiments, experiment \arabic{expcounter} alone failed to replicate this topicality pattern.
Neither were utterances with a topical prefield constituent generally rated as more acceptable, nor was topic drop of a topical constituent particularly preferred.
This suggests that the effect in the joint analysis was mainly driven by the ratings for the utterances with lexical verbs, \is{Lexical verb} but it also suggests that the ratings for the items with the modal verbs \is{Modal verb} are not diametrically opposed to this.
Given that the only factor that was modified between experiment \ref*{exp:top.s.fv} and experiment \ref*{exp:top.s.mv} was the type of the verb in the left bracket, any significant difference between the two experiments must be attributed to this modification.
That is, the lack of an interaction between \textsc{Topicality} and \textsc{Completeness} must in fact be related to the syncretic \is{Syncretism|)} forms of the modal verbs. \is{Modal verb}
The result can be interpreted as showing that topic drop of a topic constituent is more acceptable only if, at the same time, the reconstruction of this constituent is facilitated by a distinct inflectional marking on the following verb.
That is, topicality and inflectional marking seem to impact the acceptability of topic drop only in combination but not as factors on their own. \is{Acceptability rating study|)}

\section{Summary: topicality}
In the first part of this book, I argued both theoretically and empirically that topicality is neither a (strictly) sufficient nor a necessary condition for topic drop.
In this chapter, I investigated whether topicality favors topic drop, i.e., whether topic drop is more acceptable if it targets a topic.
Such an effect of topicality could be explained by the information-theoretic \textit{avoid troughs} principle (see \sectref{sec:info.theory.top}).
Given the fact that the topic is what the sentence is about and given the known tendency to have topic chains in discourse, i.e., to keep the topic constant across multiple utterances, a topic should generally be more predictable than a non-topic. \is{Predictability}
Therefore, topic drop should be more acceptable if it targets a more predictable topic than a less predictable non-topic. \is{Predictability}

In experiments \ref*{exp:top.q1} and \ref*{exp:top.q2}, I attempted to use questions to set the discourse topic \is{Discourse topic} and did not find any effect of topicality on topic drop.
In experiments \ref*{exp:top.s.fv} and \ref*{exp:top.s.mv}, I exploited the fact that subjects are often unmarked topics and used the subject function to set the (sentence) topic.
This had the consequence that topicality and subjecthood were intermingled so that I cannot finally decide which of them is the crucial factor.
In a future study, it may thus be beneficial to try to disentangle topicality and subjecthood without resorting to using questions to set the topic.%
%% Footnote
\footnote{Possibly, syntactic means of topic marking such as left dislocation \is{Left dislocation} and cleft clauses \citep[see, e.g.,][33--34]{musan2017}, which can also be used to mark syntactic objects as topics, would be suitable for this purpose.}
%
From the \textit{UID} \is{Uniform information density}perspective, however, what makes the antecedent constituent \is{Antecedent} more probable is ultimately irrelevant, be it its topic or its subject status.
The only decisive factor is that it is more predictable thus has a lower surprisal, and should be more likely omitted. \is{Predictability}

By using the subject method, I found an effect of topicality in the expected direction but only in experiment \ref*{exp:top.s.fv} and in the joint analysis of experiments \ref*{exp:top.s.fv} and \ref*{exp:top.s.mv} but not in experiment \ref*{exp:top.s.mv} in isolation.
This suggests that either the effect of topicality (or potentially of subjecthood) is not very robust, and/or that distinct verbal inflection (which was present in experiment \ref*{exp:top.s.fv} but not in experiment \ref*{exp:top.s.mv}) and topicality work together.
A subject can be better omitted if it is predictable \is{Predictability} as the topic and if the following verb provides a cue to recover it. \is{Recoverability}
From an information-theoretic perspective, if several factors come together to facilitate the resolution of the ellipsis, this could lower the processing effort \is{Processing effort} for the verb in the left bracket to the point where it falls below the channel capacity, \is{Channel capacity} whereas the influence of just one factor alone may not suffice.
Taken together, there is only sparse evidence of an impact of topicality on the usage of topic drop.
This impact could nevertheless be well captured by the \textit{avoid troughs} principle of my information-theoretic account of topic drop usage.

In the next chapter, I turn to the influencing factor that I investigate in most depth in this book, grammatical person, and to verbal inflection and ambiguity avoidance, which are closely related to grammatical person.
\is{Topic|)} 
