\documentclass[output=paper
,modfonts
,nonflat]{langsci/langscibook} 



\ChapterDOI{10.5281/zenodo.1402535}


\title{Referential cohesion in Bunun: A comparison of two genres} 
\author{Rik De Busser\affiliation{National Chengchi University}}

\abstract{This chapter investigates how referential expressions are involved in establishing and maintaining textual cohesion in Bunun, an Austronesian language of Taiwan, and how this behaviour varies across genres. Relying on a model of referential cohesion inspired by sys\-tem\-ic-func\-tion\-al grammar, it explores differences and similarities for encoding referential continuity across sentence boundaries in oral and narrative text. It concludes that, contrary to initial expectation, and despite considerable formal differences in how referential expressions are realized, at a more fundamental level the properties of referential cohesion are unexpectedly stable across genres.}

\begin{document}

\maketitle

\section{\label{s1}Introduction}

\subsection{\label{s1.1}Cohesion}

Now more than four decades ago, \citet{Halliday1976} published their seminal work on the linguistic subsystem that helps creating coherent text by establishing connections between related semantic elements in that text. More specifically, it is “a set of lexicogrammatical systems that have evolved specifically as a resource for making it possible to transcend the boundaries of the \isi{clause} — that is, the domain of the highest-ranking grammatical unit” (\citealt[532]{Halliday2004}). They referred to this subsystem as \isi{cohesion} and to the connections as cohesive ties, and described it as the set of “relations of meaning that exist within the text, and that define it as text” (\citealt[4]{Halliday1976}). This means that its realization is not confined by \isi{clause} or other grammatical boundaries, but typically operates on the scale of text or \isi{discourse}. 

It also implies that there is no isomorphic relationship between grammatical devices and cohesive effects; \isi{cohesion} pertains to semantic relationships within texts that “may take any one of various forms” (\citealt[13]{Halliday1976}). The original proposal, which has been integrated in Halliday’s systemic-functional grammar (\citealt{Halliday1994}; \citealt{Halliday2004}; \citealt{Halliday2014}), distinguishes four types of relationships: 

\begin{enumerate}
		\item \textsc{reference} establishes cohesive ties between linguistic elements through various forms of spatio-temporal and personal deixis, and through comparison. 
		\item \textsc{ellipsis} covers all phenomena that establish cohesive links by omitting a grammatical unit, or by swapping it for a placeholder element.\footnote{ \citet[88--141]{Halliday1976} called the latter substitution and originally considered it to be a separate cohesive category, meant to account for forms like \textit{one} in \ili{English} expressions such as \textit{You can choose the blue candy or the red one}. They acknowledged that both substitution and ellipsis established cohesive ties by replacement, either by zero or by a placeholder \citep[88]{Halliday1976} and both are subsumed under ellipsis in \citet{Halliday1994} and later publications.} 
		\item \textsc{conjunction} creates logical or spatio-temporal ties between propositions, typically through various grammatical mechanisms for \isi{clause} linking. 
		\item \textsc{lexical cohesion} is established between lexical elements through repetition and various semantic relationships.
\end{enumerate}

\noindent
The markers for each of the four types of cohesive relationship are indicated in the following examples.

\begin{exe}
	\ex\label{e1}
	\begin{xlist}
		\exi{English}
		\ex\label{e1a} Reference\\
		\textit{\textbf{That} man’s dog is \textbf{much larger than my} cat}.
		\ex\label{e1b} Ellipsis\\
		\textit{How many cookies are left? I took twelve \textbf{Ø}. \textbf{So did} you}.
		\ex\label{e1c} Conjunction\\
		\textit{\textbf{When} it shut down, something went wrong. \textbf{In short}, it caught fire}.
		\ex\label{e1d} Lexical organization\\
		\textit{\textbf{Emperor penguins protect} their \textbf{chicks} from the \textbf{cold} by \textbf{putting} the \textbf{little} \textbf{fluff balls} on their \textbf{feet}}.
	\end{xlist}
\end{exe}

\noindent
In (\ref{e1a}), \textit{that} points to a \isi{referent} that exists outside the text (\isi{exophoric} reference), the phrase \textit{much larger than} connects \textit{that man’s dog} and \textit{my cat}, and the possessive form \textit{my} creates an \isi{exophoric} link to the speaker. In (\ref{e1b}), ellipsis in the second \isi{clause} indicates that the head of \textit{twelve Ø} refers to the same set of referents as \textit{cookies} in the first \isi{clause}. The substitutive construction \textit{so did} in the third \isi{clause} indicates that its subject performed the same action, \textit{take [cookies]}, as the first person in the second \isi{clause}. In (\ref{e1c}), \textit{when} creates a relationship of simultaneity between the first and second \isi{clause}, and \textit{in short} indicates that the third \isi{clause} summarizes the previous \isi{discourse}. Finally, the penguin-related lexical items in (\ref{e1d}) arrange themselves in a complex of lexical cohesive relationships (see \figref{fig:1}). 


\begin{figure}[H]
% 	\noautomath
% % 	\includegraphics[width=\textwidth]{figures/DeBusserFINAL-img001}
        \begin{forest} for tree={grow=west,anchor=east,l=4cm,edge=-{Triangle[]}}
            [,l=0cm, s sep=2\baselineskip [fluff balls\vphantom{j},no edge,l=0cm [chicks\vphantom{j},edge label={node[above,midway,font=\itshape]{metaphor}} [Emperor penguins,name=penguin,edge label={node[midway,above,font=\itshape]{hyponym}}]]] [feet,name=feet,no edge,l=0cm]]
            \draw[-{Triangle[]}] (feet) -- ++(-7cm,0) node[above,midway,font=\itshape] {part-whole} -- (penguin.south);
        \end{forest}
	\caption{Schema of a cohesive chain}
	\label{fig:1}
\end{figure}

\noindent
As the examples illustrate, markers of \isi{cohesion} are highly heterogeneous in their grammatical and relational properties. What they have in common is that they establish cohesive ties, that is, semantic connections between linguistic elements (words, phrases, clauses, etc.) that are typically asymmetrical and express that the \isi{discourse} segments in which they occur are to be interpreted as part of a coherent whole. These ties, either in isolation or by combining into longer chains, weave through a text. Together with thematic structure (theme/rheme contrasts) and \isi{focus structure} (given/new), \isi{cohesion} thus creates ‘texture’ (\citealt[334]{Halliday1994}; \citealt[579]{Halliday2004}), the perception of a text or \isi{discourse} as a connected whole. Texture in turn is “one aspect of the study of coherence, which can be thought of as the process whereby a reading position is naturalized by texts for listener/readers” \citep[35]{Martin2001}.

Importantly, this implies that Hallidayan systemic-functional grammar “does \textit{not}\linebreak equate \isi{cohesion} with coherence” (\citealt[27]{Martin1992}; see also \citealt{Martin2001}). Cohesion is merely one of the linguistic systems responsible for textual coherence. Later work on coherence often merged the two concepts, and typically reduced the phenomenon to a semantic-pragmatic component responsible for combining clause-level propositions into larger rhetorical structures (see e.g. \citealt{Mann1987}; \citealt{Kehler2002}; \citealt{Kehler2004}).

In sum, \isi{cohesion} is an information-structuring device that, by establishing semantic connections between a heterogeneous set of linguistic units within a text, assists language users in interpreting that text as a cohesive, connected whole. In doing so, it is one of the subsystems responsible for structuring the distribution of information elements on a textual (supra-sentential) level.

Cohesion has been explored extensively in theoretical and applied linguistics, but overwhelmingly in the context of \ili{English} (\citealt{Halliday1976}; \citealt{Connor1984}; \citealt{Martin1992}; \citealt{Abadiano1995}; \citealt{Tanskanen2006}; \citealt{Crossley2012}) and occasionally other major languages (\citealt{Aziz1988} on \ili{Arabic}; \citealt{Hickmann1999} on \ili{English}, \ili{French}, \ili{German} and \ili{Mandarin}; \citealt{Kruger2000} on \ili{Afrikaans}; \citealt{Hassel2005} on \ili{English}, \ili{German} and \ili{Norwegian}). Work on minority languages is much less common. In the \ili{Austronesian} world, the only studies I am aware of are \citet{Ezard1978} on \ili{Tawala}, \citet{Flaming1983} on \ili{Wandamen}, and \citet{Benn1991} on Central Bontoc. The first two are literal applications of Halliday \& Hasan’s framework to their languages; Benn employs a number of frameworks, including Halliday \& Hasan’s, for his analysis of the discursive structure of Central Bontoc ritual texts.\largerpage[-1] 

This chapter adapts Halliday \& Hasan’s original model to fit the needs of analysing the role of referential expressions in establishing the cohesive texture of \ili{Bunun} texts. It investigates the role \isi{cohesion} plays in establishing genre distinctions through a small-scale pilot study.

\subsection{\label{s1.2}Genre and cohesion}

Genres or registers can be defined as specific types of texts or \isi{discourse} with sets of “relatively stable” properties associated with the “thematic content, style, and compositional structure” that reflects the specific needs of well-defined contexts in which they were realized \citep[60]{Bakhtin1986}.

Distinctions between genres are marked through various linguistic means. \citet[28]{Biber1995} makes a basic distinction between register markers and register features. The former are linguistic cues that are specific to a certain register or genre and therefore directly indicate that a text belongs to it. An anecdotal example is the phrase \textit{a long time ago in a land far, far away} introducing a fairy tale. Register features are linguistic elements that are not genre-specific, but whose frequency or distribution is in certain situations indicative of a specific register or genre. For instance, imperatives are relatively common in recipes, but they occur in many other genres as well. Cohesion falls into the latter category.

Research on the relationship between genres and their indicators has mainly focussed on the “relative distributions of surface linguistic features, such as adjectives, nominalizations, passives, and various \isi{clause} types” \citep[12]{Biber1995}. Even Biber, who went well beyond previous studies by focussing on complex feature bundles, mainly concentrates on morphosyntactic features that can be straightforwardly extracted from the surface realization of the text (see \citealt[94–104]{Biber1995}; also \citealt[217–226]{Biber2009}). Given that genre is associated with the global discursive and semantic features of texts, one should probably assume that these grammatical features serve as proxy indicators of certain structural elements of meaning, \isi{discourse} organisation and information structure.

Cohesion is an important determinant of the distribution of information in text, so it is reasonable to assume that it is interconnected with the global properties of text structure, and therefore contributes to (\citealt{Halliday1976}) or closely interacts with genre (\citealt{Martin1992}; \citealt{Martin2001}). There are a number of reasons why one would expect consistent correlations between \isi{cohesion} and genre, many linked to the accessibility of linguistic information (\citealt[74–116]{Lambrecht1994}; \citealt{Ariel1991}).

First, expectation patterns related to the nature and quantity of assumed background knowledge and explicitly expressed information are often genre-specific. Specialized genres assume a greater volume of background knowledge than more generalized genres. For example, the presupposed background knowledge in an informal conversation is different from that in an academic textbook (\citealt[14–15]{Biber2009}). This affects the need for explicitly expressing cohesive relationships between elements in a text.\largerpage[-1] 

Second, differences in genre commonly correlate with differences in modes of realization, which in term influences the options for realizing \isi{cohesion}. For instance, oral and written genres diverge in which cohesive strategies they employ (see e.g. \citealt{Fox1987}; \citealt{Givón1993}). Textual coherence in writing is partly realized through meta-linguistic means, such as writing conventions and punctuation, not available in oral \isi{discourse}. Because of the visual nature of the written medium, information is also more readily, and longer, accessible. All things being equal, one would therefore expect that oral genres tend to have a more dense cohesive structure (or a larger presence of other coherence-creating mechanisms) than their written equivalents, in order to reach an equal level of coherence.

For certain types of cohesive relationships, the link between \isi{cohesion} and genre is well understood. For instance, it is uncontroversial that “genre-specific conventions […] play a significant role in \isi{anaphoric} patterning in conversation and writing” \citep[2]{Fox1987}. Research explicitly comparing cohesive patterning across genres is scarce, but the influence of \isi{cohesion} on the realization of individual genres is the subject of a number of studies. The above-mentioned \citet{Benn1991} investigates \isi{cohesion} in single genre (written essays) in Central Bontoc. Another example are \citet{Malah2016} who, based on \citet{Hoey1991}, explore the role of content words in establishing the cohesive properties of \ili{English} language Nigerian newspaper texts.

One important question is in which manner exactly \isi{cohesion} indicates genre in text or – from a comparative perspective – how its realizations are indicative of differences between genres. Halliday \& Hasan suggest that “the possible differences among different genres and different authors [are] in the numbers and kinds of tie they typically employ” (\citealt[4]{Halliday1976}). In other words, one would expect that (1) \isi{cohesive density}, and (2) the nature of the connections between elements in cohesive relationships varies between genres.

This chapter investigates whether, and to what extent, these two hypotheses are true for two text genres, \isi{oral narrative} and biblical translations, in a \ili{Bunun} speech community. It compares the \isi{cohesive density} and the morphosyntactic and semantic-pragmatic properties of cohesive ties in these two types of texts.

Similar to \citet{Malah2016}, this chapter not discuss all aspects of \isi{cohesion} as they were introduced in \citet{Halliday1976}. Rather, the discussion focuses on \isi{referential cohesion}, the conceptually coherent subset of cohesive ties that is involved in establishing relationships between referential items. Its exact delineation is discussed in \sectref{s1.4}. Before this is possible, I first introduce the \ili{Bunun} language, its dialects and the genres involved in the present analysis.

\subsection{\label{s1.3}Bunun dialects}

\ili{Bunun} is one of around sixteen \ili{Austronesian} languages spoken on Taiwan \citep{Li2008}. It has five extant dialects that are classified into a Southern (\ili{Isbukun}), Central (\ili{Takbanuaz} and \ili{Takivatan}) and Northern branch (Takibakha and \ili{Takituduh}). Within the \ili{Isbukun} dialect, at least three distinct varieties are spoken in Kaohsiung, Taitung, and Nantou. Between dialects, especially between \ili{Isbukun} varieties and dialects of the other branches, there is a fair amount of phonological, lexical, and grammatical differentiation (see \citealt{Li1988} for an overview of phonological and lexical variation). Only the \ili{Takivatan} and \ili{Isbukun} dialects are relevant to the present discussion. 


\begin{exe}
	\ex\label{e2}
	\begin{xlist}
		\exi{Bunun}
		\ex\label{e2a} \langinfo{Takivatan}{}{fieldwork, observed}\\
		\gll mun-ʔisaq=ʔas  \\
		\textsc{all}-where-\textsc{2s}.\textsc{subj}\\
		\glt `Where are you going?’
		\ex\label{e2b} \langinfo{Isbukun}{}{Lilian Li, pers. comm.}\\
		\gll ku-ʔisaʔ  kasuʔ  ma-tuktuk  lukis  \\
		\textsc{all}-where  \textsc{2s}.\textsc{subj}  \textsc{dyn}-chop  wood\\
		\glt `Where do/did you go to chop wood?’
	\end{xlist}
\end{exe}

\noindent
Example (\ref{e2a}--\ref{e2b}) illustrates the degree of discrepancy between the two dialects.\footnote{The following changes were made to graphemic conventions: z > ð, ' > ʔ, ch > ʤ, ng > ŋ} The coda of the question word (/q/ in \ili{Takivatan}, /ʔ/ in \ili{Isbukun}) is illustrative of a systematic phonological contrast. In near-identical contexts, both dialects use different allative prefixes. Finally, whereas \ili{Takivatan} prefers a pronominal \isi{clitic} in subject positions like this, \ili{Isbukun} uses a \isi{free pronoun} that does not exist in \ili{Takivatan} (see \tabref{tab:debusser:1} and \tabref{tab:debusser:2}).

\ili{Bunun} dialects have a verb-initial \isi{constituent order} and what has been called a Western \ili{Austronesian} or Philippine-type \isi{voice system} (see \citealt{French1987}; \citealt{Foley2007voice}; \citealt{Riesberg2014} for general overviews), which in \ili{Bunun} distinguishes at minimum between actor (\textsc{av}), \isi{undergoer} (\textsc{uv}), and \isi{locative voice} (\textsc{lv}), marked by suffixes on the verb. In (\ref{e3a}), \textit{siða} is \isi{actor voice} and as a result unmarked; the \textsc{uv} in (\ref{e3b}) is indicated by a suffix -\textit{un}, and the \textsc{lv} in (\ref{e3c}) by -\textit{an}. 

\begin{exe}
	\ex{\ili{Takivatan} Bunun}\label{e3}
	\begin{xlist}
		\ex\label{e3a}  \langinfoverb{Actor voice}{}{(fieldwork, elicited)}\\
		\gll na-siða  qaimaŋsuð.\\
		\textsc{irr}-take  thing\\
		\glt `I will pick up things.’
		\ex\label{e3b}  \langinfoverb{Undergoer voice}{}{(fieldwork, text corpus)}\\
		\gll maŋmaŋ  ni  siða-un.  \\
		many  \textsc{neg}  take-\textsc{uv}\\
		\glt `... a lot were not caught.’
		\ex\label{e3c}  \langinfoverb{Locative voice}{}{(fieldwork, text corpus)}\\
		\gll maqtu  pa-siða-an-in  ŋabul  vanis.\\
		can  \textsc{caus}.\textsc{dyn}-take-\textsc{lv}-\textsc{prv}  deer  wild.boar\\
		\glt `... and we could catch deer and wild boar.’
	\end{xlist}
\end{exe}

\noindent
Certain analyses additionally include instrumental, beneficiary, and resultative object voices, but these forms are relatively uncommon and can be further ignored here.

The remainder of this section gives a short overview of various \isi{deictic} paradigms, since these are relevant to the discussion at hand. All five \ili{Bunun} dialects have sets of free and bound personal pronouns. Paradigmatic distinctions are largely equivalent, but the pronominal sets have formally diverged and have been analysed as expressing different grammatical distinctions in \ili{Takivatan} and \ili{Isbukun}. Tables \ref{tab:debusser:1} and \ref{tab:debusser:2} give the pronominal paradigms for both dialects.\footnote{The \ili{Takivatan} data on personal pronouns is from \citet[441]{DeBusser2009}; the \ili{Isbukun} data from the Kaohsiung variety in \citet[85]{Huang2016Bun}. The latter mark vowel length by grapheme doubling. This distinction is non-phonemic in \ili{Bunun}: generally, monosyllabic roots tend to have lengthened vowels, irrespective of the environment in which they occur. To make comparison easier, long vowels in the \ili{Isbukun} examples are represented by single vowel graphemes. Subject and non-subject forms are analysed and glossed differently in \citet{DeBusser2009} and \citet{Huang2016Bun}, in this might reflect subtle differences in the grammatical distribution of these forms. Again, to make comparison easier, this terminology has here been homogenized.}

\begin{table}
\begin{tabularx}{\textwidth}{XllllL{1.75cm}l}
	\lsptoprule
	\bfseries (a) & \multicolumn{2}{c}{\bfseries Subject}  & \multicolumn{2}{c}{\bfseries Non-subj.}  & \bfseries Poss. & \bfseries Loc.\\\cmidrule(lr){2-3}\cmidrule(lr){4-5}
	& \bfseries Free & \bfseries Bound & \bfseries Free & \bfseries Bound &  & \\
	\midrule
	\textsc{1s} & sak, saikin & {}-(ʔ)ak & ðaku, nak & {}-(ʔ)uk & inak, ainak, nak & ðakuʔan\\
	\textsc{2s} & — & {}-(ʔ)as & suʔu, su & — & isu, su & suʔuʔan\\
	\textsc{3s} & (see b) & {}-(ʔ)is & (see b) & — &  & \\
	\textsc{1i} & ʔata, inʔata & — & mita & — & imita & mitaʔan\\
	\textsc{1e} & ðamu, sam & {}-(ʔ)am & ðami, nam & — & inam, nam & ðamiʔan\\
	\textsc{2p} & amu & {}-(ʔ)am & muʔu, mu & — & imu, mu & muʔuʔan\\
	\textsc{3p} & (see b) & — & (see b) & — &  & \\
	\midrule
\end{tabularx}
\begin{tabularx}{\textwidth}{XXXX}
	\bfseries (b) & \multicolumn{3}{c}{\bfseries Subject \& Non-subject}\\\cmidrule(lr){2-4}
	& \bfseries \textsc{Prox} & \bfseries \textsc{Med} & \bfseries \textsc{Dist}\\
	\midrule
	\textsc{3s} & isti & istun & ista\\
	\textsc{3p} & inti & intun & inta\\
	\lspbottomrule
\end{tabularx}
	\caption{\label{tab:debusser:1}Personal pronouns in Takivatan Bunun}
\end{table}

\begin{table}
\begin{tabularx}{\textwidth}{XL{1.75cm}lL{1.75cm}lL{1.75cm}L{1.75cm}} 
	\lsptoprule
	& \multicolumn{2}{c}{\bfseries Subject}  & \multicolumn{2}{c}{\bfseries Non-subj.}  & \bfseries Poss. & \bfseries Loc.\\\cmidrule(lr){2-3}\cmidrule(lr){4-5}
	& \bfseries Free & \bfseries Bound & \bfseries Free & \bfseries Bound &  & \\
	\midrule
	\textsc{1s} & saikin & {}-ik & ðaku & {}-ku & inak & ðakuan\\
	\textsc{2s} & kasu(n) & {}-as & su & {}-su & isu & suan\\
	\textsc{3s} & saia, sai(n) & — & saiʤia & — & isaiʤia, isia & siʔaan ʤia\\
	\textsc{1i} & kata & {}-ta & ita, mita & {}-ta & imita & mitaan\\
	\textsc{1e} & kaimin & {}-im & ðami & — & inam & ðamian\\
	\textsc{2p} & kamu(n) & {}-am & mu & {}-mu & imu & muan\\
	\textsc{3p} & nai, nian (\textsc{vis}), naia (\textsc{nvis}) & — & nai (\textsc{vis}), naiʤia (\textsc{nvis}) & — & inai (\textsc{vis}), inaiʤia (\textsc{nvis}) & naian ʤia (\textsc{nvis})\\
	\lspbottomrule
\end{tabularx}
\caption{\label{tab:debusser:2}Personal pronouns in Isbukun Bunun}
\end{table}

\noindent
Some of the more systematic differences are worth mentioning. Third person pronouns in \ili{Takivatan} differentiate between proximal, medial and distal forms and do not have distinct subject and non-subject forms. They can therefore be interpreted as a subset of demonstratives (\tabref{tab:debusser:3}). In contrast, \ili{Isbukun} third person pronouns do not encode a \isi{deictic} contrast. Singular forms all appear derived from the stem \textit{sia}, which in \ili{Takivatan} is an \isi{anaphoric} form that appears in a number of grammatical positions (\citealt[467--474]{DeBusser2009}). Plural forms all derive from the stem \textit{nai}. \citet[72]{Zeitoun2000} suggests that variant forms within each category code a visibility distinction. The element \textit{ʤia} on third person forms is in all likelihood a distal determiner enclitic, making their status of as personal pronouns contentious.

Demonstrative pronouns vary substantially between dialects. \citet[95--97]{DeBusser2017} describes an elaborate paradigm for \ili{Takivatan}; see \tabref{tab:debusser:3}.

\begin{table}
\begin{tabularx}{\textwidth}{XXXXXX} 
	\lsptoprule
	&  & \bfseries \textsc{prox} & \bfseries \textsc{med} & \bfseries \textsc{dist} & \bfseries \textsc{uspec}\\
	\midrule
	\textsc{s} & \textsc{vis} & aipi & aipun & aipa & aip\\
	& \textsc{nvis} & naipi & naipun & naipa & naip\\
	\textsc{p} & \textsc{vis} & aiŋki & aiŋkun & aiŋka & —\\
	& \textsc{nvis} & naiŋki & naiŋkun & naiŋka & —\\
	\textsc{gnr} & \textsc{vis} & aiti & aitun & aita & —\\
	& \textsc{nvis} & naiti & naitun & naita & —\\
	\textsc{pauc} & \textsc{vis} & — & — & (ainta) & —\\
	& \textsc{nvis} & — & naintun & (nainta) & —\\
	\lspbottomrule
\end{tabularx}
\caption{\label{tab:debusser:3}Free demonstratives in Takivatan Bunun}
\end{table}

\noindent
None of these forms has so far been attested in \ili{Isbukun}. The \isi{demonstrative} forms described in \citet[95]{Huang2016Bun} distinguish case and distance, but not visibility. However, their paradigm consists of fully transparent combinations of the form \textit{sia} or the spatial adverbs \textit{di} and \textit{adi} ‘there’ with various bound determiners (see \tabref{tab:debusser:4}), which encode both distinctions mentioned above.

Finally, all \ili{Bunun} dialects have two sets of bound determiners, which encode a distance contrast and can occur on a variety of word classes including verbs (see \citealt[427--440]{DeBusser2009} for an explanation).

\begin{table}
\begin{tabularx}{\textwidth}{llXll} 
	\lsptoprule
	&  & \bfseries \ilit{Takivatan} & \bfseries \ilit{Isbukun} (Taitung \& Kaohsiung) & \bfseries \ilit{Isbukun} (Nantou)\\
	\midrule
	\textsc{d1} & \textsc{prox} & {}-ki & {}-in & {}-in\\
	& \textsc{med} & {}-kun & {}-an & {}-an\\
	& \textsc{dist} & {}-ka & {}-a & {}-a\\
	\textsc{d2} & \textsc{prox} & {}-ti & {}-ʤin & {}-tin\\
	& \textsc{med} & {}-tun & {}-tan & {}-tan\\
	& \textsc{dist} & {}-ta & {}-ʤia & {}-tia\\
	\lspbottomrule
\end{tabularx}
\caption{\label{tab:debusser:4}Determiners in Isbukun and Takivatan Bunun}
\end{table}

\noindent
Again, there are formal differences, this time even between different varieties of the \ili{Isbukun} dialect. The distinction between \textsc{d1} and \textsc{d2} appears to be fundamentally different in the two dialects. \citet[426--440]{DeBusser2009}; \citet{DeBusser2017} argues that in \ili{Takivatan} the difference between the two sets is semantic in nature; in \ili{Isbukun}, the difference is associated with case (\textsc{d1} = \textsc{nom}, \textsc{d2} = \textsc{obl}; see \citealt[95]{Huang2016Bun}; \citealt[76]{Zeitoun2000}). Bound determiners are optional, and are considerably more common in \ili{Takivatan} than in other dialects. 

The paradigms above serve as illustrations of the degree of differentiation between \ili{Takivatan} and \ili{Isbukun}, and give an overview of some of the paradigms that are relevant to the present analysis, as they directly influence the difference between different text genres in \ili{Bunun}.

\subsection{\label{s1.4}Narrative genres in Bunun}

This chapter compares two narrative genres, traditional \isi{oral narrative} text and biblical narrative, as they occur in a single \ili{Takivatan} \ili{Bunun} language community in the village of Bahuan at the East Coast of Taiwan.

Traditional narrative texts, despite being transmitted orally, are by nature not improvised. In traditional \ili{Bunun} communities, both expository and narrative texts in formal settings follow relatively strict conventions that govern amongst other things: (1) who has the right to speak and when; (2) how certain traditional knowledge should be presented; and (3) which formal aspects, such as formulae related to politeness and the veracity of the narration, should be included in specific oral genres. Many of these conventionalized aspects of stories appear to be the result of an ongoing consultation process between the elders, or a larger group of members, of the community. Transgressions of these rules are usually pointed out by authoritative members of the community, usually male elders.

A second genre with which many \ili{Bunun} people are confronted on a nearly daily basis are biblical narratives. Presbyterian and Catholic missionaries introduced Christianity after the end of the Second World War, and it is an important part of contemporary \ili{Bunun} culture. From the 1940s onwards, especially the Presbyterian Church, through the Bible Society of Taiwan, has been active in translating biblical texts into the \ili{Bunun} language. 

Bible translations are intricate undertakings that typically involve a translation team consisting of translators, native language consultants, and theologians. Especially in cultures that have little historical affinity with the Judaeo-Christian tradition, this process is more than simply translating texts: it requires the meticulous transposition of an alien conceptual universe with its associated lexical and grammatical framework (see \citealt{DeBusser2013}). This makes it nearly impossible to produce translations for every dialect of \ili{Bunun}.

The present \ili{Bunun} Bible translation \citep{Bible2000} will be referred to as the \ili{Bunun} Bible. It is the authoritative translation containing the full New Testament and an abbreviated Old Testament, and is heavily based on the \ili{Isbukun} dialect. Despite this, it is used in almost all \ili{Bunun} churches, irrespective of their denomination or the dialect area to which they belong. From a language planning perspective, this made a lot of sense: \ili{Isbukun} is the largest dialect, has the widest geographical spread, and has been studied most extensively. This is not to say that the \ili{Bunun} Bible is a written mirror of any specific \ili{Isbukun} variety: the translation process rather resulted in a supra-dialectal written standard for Christian texts in \ili{Bunun}, which is also used for other religious text genres such as hymns.

An interesting consequence is that, although many \ili{Bunun} are reading or listening to excerpts of the Bible on a regular basis, especially “to members of \ili{Takbanuaz}, \ili{Takivatan}, Takibakha and \ili{Takituduh} communities, the language of their Bible is not closely related to the common vernacular” (\citealt[67]{DeBusser2013}). Since the discrepancy between \ili{Isbukun} and the \ili{Takivatan} dialect is considerable, the result is a situation in which two dominant narrative genres in the \ili{Takivatan} language community have relatively divergent dialectal characteristics. 

This leads to a question: how and to what extent do language users in non-\ili{Isbukun} communities interpret the content of these texts that belong to related genres but have quite distinct lexical and grammatical characteristics? To an extent, this is a matter of lexical and grammatical overlap between dialects. However, an additional question concerns the cohesive fabric of these two types of texts. Given the differences between grammatical paradigms that are central to establishing \isi{cohesion}, such as personal pronouns, demonstratives, and determiners, how do language users keep tab of issues such as thematic integrity, topic continuity, and the general distribution of information in a narrative progression?

The present study will investigate this issue by looking at how these two different narrative genres realize \isi{cohesion}, and in a particular subset of cohesive relations that is here referred to as \isi{referential cohesion}.

\subsection{\label{s1.5}Referential cohesion}

The basic framework for my analysis of \isi{referential cohesion} is set out in \citet{DeBusser2017}. \citet{Halliday1976} consider \isi{cohesion} to be the aggregate set of cohesive ties, semantic relationships that exist between meaningful elements in a text. Cohesive ties are directional: they point from a textual source, which will be called the Reference (Rc), to a second element, which can exist inside (\ref{e4a}) or outside the text (\ref{e4b}). 

\begin{exe}
	\ex\label{e4}
	\begin{xlist}
		\ex{Rc points to text-external Rt}\label{e4a}\\
		\gll maq  \textbf{aipa}\\
		what  \textsc{dem.s.dist.vis}\\
		 Rc [→ Rt: object in external reality]\\
		\glt `What is that?’
		\ex{Rc points to text-internal Ta}\label{e4b}\\
		\gll maq-a  ainak-a  \textbf{tama-ka}  tu-tuða  tu  miqðiq  daiŋʔað  \textbf{aipa}.\\
		\textsc{defin-lnk}  \textsc{1s}.\textsc{poss}-\textsc{lnk}  father-\textsc{k}.\textsc{dist}  \textsc{intens}-real  \textsc{compl}  difficult  large  \textsc{dem}.\textsc{s}.\textsc{dist}.\textsc{vis}\\
		 Ta ← Rc\\
		\glt `As for my father, he really had a lot of difficulties.’
	\end{xlist}
\end{exe}

\noindent
When this element is linguistic in nature, I call it a Target (Ta). Targets are themselves References that point back to previous targets. In doing so, they create cohesive chains, referential strands of different length that ‘weave’ through a text. Eventually, the final Reference of each chain points to a concept that exists outside the textual universe; this is called the Referent (Rt) of the cohesive chain and is in effect its ultimate Target (\citealt[329]{Halliday1976}; \citealt[107--108]{DeBusser2017}). This is schematised in \figref{fig:2}.

\begin{figure}[H]
% % 	\noautomath
% % % 	\includegraphics[width=\textwidth]{figures/DeBusserFINAL-img002}
	\begin{tikzpicture} [every node/.style={minimum width=1.5cm,minimum height=1.5cm,thick}]
	\matrix [row sep=1em, column sep=1cm] {
	\node [draw,rounded corners=10pt] (Rt) {Rt}; &\\
	      & \node [rectangle split,rectangle split parts=2,rectangle split part align={left,right}] (RcTa1) {Rc\nodepart{two} Ta}; 
	      & \node (dots) {\ldots};
	      & \node [rectangle split,rectangle split parts=2,rectangle split part align={left,right}] (RcTa2) {Rc\nodepart{two} Ta}; 
	      & \node [draw] (Rc) {Rc};\\};
        \foreach \n in {1,2} \node [draw, inner sep=0pt, fit = (RcTa\n)] (RcTaF\n) {};
        \foreach \n in {1,2} \draw [thick] (RcTaF\n.60) -- (RcTaF\n.240);
        \draw [thick] (dots.120) -| (dots.180) |- (dots.240); \draw [thick] (dots.60) -| (dots.0) |- (dots.300);
        \path[thick,-{Triangle[]}] (Rc) edge (RcTaF2) (RcTaF2) edge (dots) (dots) edge (RcTaF1) (RcTaF1) edge (Rt); 
        \path[thick, dashed, -{Triangle[]}] (RcTaF2.north) edge (Rt.0) (Rc.north) edge (Rt.15);
	\end{tikzpicture}
	\caption{Schema of a cohesive chain}
% % 	\todo[inline]{redo in tikz}
	\label{fig:2}
\end{figure}

\noindent
Together with other linguistic mechanisms, such as \isi{prosody}, event expression, and contextual information, \isi{cohesion} allows language users to interpret a text as a coherent whole operating in a context. It does this by expressing “the continuity that exists between one part of the text and another” (\citealt[299]{Halliday1976}). In doing so, \isi{cohesion} forms an interface between the local and global distribution of information elements in a text. This interaction is given to be complex, but one specific example in which \isi{referential cohesion} interacts with clause-internal information structural devices is by the “Theme tell[ing] the hearer where to start from in the interpretation of a message, and the cohesive ties signal how the message latches on to other parts of the \isi{discourse}” \citep[68]{Hasselgård2004}.

A crucial aspect of \citegen{Halliday1976} model is that cohesive ties are inherently semantically motivated, and therefore do not correspond to a single grammatical mechanism or exist between fixed classes of words or other linguistic elements. Somewhat contradictorily, Halliday \& Hasan partly rely on grammatical criteria to distinguish different types of \isi{cohesion}: they make a basic distinction between grammatical \isi{cohesion}, which is expressed by grammatical means such as function words and grammatical constructions, and \isi{lexical cohesion}, which is expressed through content words. These two categories further break down in an assortment of subtypes, based on a combination of semantic and grammatical criteria (see \citealt[324]{Halliday1976}).

This chapter takes a different approach; it focuses exclusively on \isi{referential cohesion}, “the set of cohesive relations that create referring relationships between linguistic forms and referents” \citep[107]{DeBusser2017}. These are all linguistic expressions that can be targeted by deixis (or simply, that can be pointed at). In essence, this combines Halliday \& Hasan’s category of reference, with the exclusion of comparative reference, and their category of \isi{lexical cohesion}, with the exclusion of collocation.

The criterion for establishing \isi{referential cohesion} is semantic: all words and sub-lex\-i\-cal elements that are involved in establishing referential meaning are included in the \isi{cohesion} analysis irrespective of their \isi{word class} or grammatical status. Their involvement in reference is determined by their ability to be indicated by \isi{deictic} expressions. The only formal restriction is that (with the exception of lexical compounds) phrases and other multi-word units are excluded, to prevent the data selection process from becoming too arbitrary. Word classes that have so far been implicated in \isi{referential cohesion} in \ili{Bunun} are: (1) nouns; (2) personal pronouns (see \tabref{tab:debusser:1} and \tabref{tab:debusser:2}); (3) \isi{demonstrative} pronouns (see \tabref{tab:debusser:3}); (4) bound \isi{demonstrative} articles (see \tabref{tab:debusser:4}); (5) the \isi{anaphoric} marker \textit{sia} and its derivations; (6) numerals; (7) words expressing time, manner and location; (8) question words; and (9) certain verbal roots.

It is appealing to equate \isi{referential cohesion} to anaphora resolution, but this is only so in a very broad sense. Phoric reference is typically seen as a grammatical property of language that is involved in \isi{referent} tracking and uses fixed morphosyntactic strategies to establish relationships of \isi{identity} between expressions and their antecedents (see for instance \citealt{Huang2000}). On the other hand, \isi{referential cohesion}, though obviously involved in reference tracking, is a semantic property of language that creates meaning relations between two referential expressions. These are not always relations of \isi{identity} (although they can be), and neither do they necessarily have straightforward morphosyntactic correlates. For instance, the cohesive tie between \textit{bantas} ‘legs’ and the preceding word \textit{bunun} ‘man’ is meronymic in nature, and the fact that both are nouns is not grammatically determined.

\begin{exe}
		\ex{T3.8a}\label{e5}\\
		\gll itu  \textbf{bunun}-ʤia  \textbf{bantas}  mas  buhtuŋ  hai  ʤinsu  is-tamasað-an\\
		this.here  people-\textsc{dist}.\textsc{obl}  leg  \textsc{obl}  joint  \textsc{tpc}  immediately  \textsc{transfer}-strength-\textsc{lv}\\
		\glt `This man here, his legs and joints immediately became powerful, …’
\end{exe}

\noindent
The fact that \isi{cohesion} is involved in shaping the general structure of a text suggests that it varies between and is therefore indicative of genre \citep{Martin2001}. One of the goals of this research is to try to establish the nature of this variation. One possibility suggested by \citet[4]{Halliday1976} is that “among different genres and different authors in the numbers and kinds of tie they typically employ.” Another option is that genres vary in \isi{cohesive density}, the number of cohesive elements or cohesive chains relative to text length. I will investigate both possibilities in the next section by analysing three short text segments.

\section{\label{s2}Cohesion analysis}

\subsection{\label{s2.1}Data}

The present study is a small-scale comparison of oral and biblical narrative as it occurs in the \ili{Takivatan} \ili{Bunun} speech community in the village of Bahuan (\ili{Chinese} name: Mayuan) at the East Coast of Taiwan. It consists of the analysis of three text excerpts (T1, T2, and T3), which are given in their entirety with cohesive elements underlined in the Appendices. All three are part of larger narratives; segment boundaries were chosen to extract internally coherent sub-narratives.

The first two texts, T1 and T2, are traditional \isi{oral narrative} sequences. They feature two elderly men, both fluent speakers of \ili{Takivatan} \ili{Bunun} and both around 75 years old at the time of recording. Text T1 is an account of a hunt and is part of a long story in which the speaker, Vau Taisnunan, recounts his life story. In text T2, Tulbus Manququ recounts how the traditional \ili{Bunun} hunters used prophetic dreams to determine the appropriate time for the hunt.

Text T3 is an excerpt from the Acts of the Apostles in the \ili{Bunun} Bible \citep{Bible2000}, in which the apostles Peter and Paul heal a cripple. As mentioned in \sectref{s1.2}, it is a written text that is the product of a complex translation process. The spelling of the original text in the \ili{Bunun} Bible was adapted to make it consistent with the \ili{Takivatan} \ili{Bunun} texts and make it adhere to a one-grapheme-per-phoneme principle (see Footnote 1).

\subsection{\label{s2.2}Methodology}

All elements in these texts that could be unambiguously identified as having a referential function were marked for analysis. Importantly, this means that non-expressed (ellipted) elements are not included, despite having a referential value. In contrast with the coding scheme in \citet[329–355]{Halliday1976}, no prior assumption is made about the \isi{word class} (or morphological class) of the elements involved; so far only the nine linguistic classes mentioned in \sectref{s1.5} have been attested in referential cohesive relationships. 

In the data set the following information about each Reference, and the nature of its cohesive tie to its Target and Referent are encoded:

\begin{enumerate}
	\item The location of a Reference in the text;
	\item Its \isi{word class} (see \sectref{s1.5} and \tabref{tab:debusser:5});
	\item Its Target and the location of the Target in the text (this is not relevant for \isi{exophoric} links);
	\item The Referent of its cohesive chain, i.e. the text-external entity (or event) to which the Reference eventually refers;
	\item The ontological type of the Target, i.e. whether the immediate Target of a Reference refers an event, a location or time, or a textual element (see \tabref{tab:debusser:6});
	\item The phoric status of the cohesive tie, i.e. whether it is a \isi{anaphoric}, cataphoric, or \isi{exophoric} link (see \tabref{tab:debusser:7});
	\item The relationship between the concept indicated by a Reference and the concept indicated by its Target (see \tabref{tab:debusser:8}).
\end{enumerate}

\noindent
The first four data points provide information about the structural properties of cohesive chains; the information in 5--7 relates to conceptual and informational properties of individual cohesive ties. The Referent (data point 4) of referential expressions is indicated by unique names that allow us to track cohesive chains. Targets of cohesive ties can belong to a number of distinct ontological types (data point 5). Most commonly they are concrete or abstract entities in the real world, but they can also be reified events, physical or temporal locations, or textual elements; this is discussed in \sectref{s3.2}. The phoric status (data point 6) indicates whether a Reference points to a Target that precedes it (\isi{anaphoric} reference), follows it (cataphoric reference) or exists outside the textual universe (\isi{exophoric} reference).

Finally, data point 7 encodes the conceptual relationships between References and their Targets. Possible values are adapted from the set of relations subsumed under \citet[277--282]{Halliday1976} category of reiteration, complemented by Peirce’s fundamental semiotic relationships that exist between signifiers and semiotic objects (metaphor, metonym, symbol; see \citealt{Merrell2001}). The following relationships are distinguished:

\begin{description}
	\item[Identity:] Relations in which the Reference refers to the same concept as its Target. This can be because it is a literal repetition, a personal or \isi{demonstrative} \isi{deictic}, a synonym, or a near-synonym.
	\item[Hyponym/hyperonym/co-hyponym:] Relations dictated in terms of category membership. Hyponyms refer to other concepts of which they are a subclass; hyperonyms refer to concepts of which they are a superclass; co-hyponyms are terms that have the same immediate superclass.
	\item[Part/whole/co-part:] Relations that defined in terms of meronymy (see example \ref{e4} above). 
	\item[Antonym:] Relations based on conceptual opposition.
	\item[Metaphor:] Relations based on similarity, other than \isi{identity}.
	\item[Metonym:] Relations based on proximity or, more generally, contiguity.
	\item[Symbol:] Relations based on conventional semantic connections that cannot be reduced to any of the previous six relations.
\end{description}

\noindent
Originally envisaged to be applicable to \isi{lexical cohesion} alone, these relations here extend to all referential cohesive ties. In the present sample, no metaphoric and symbolic relations have been attested. The \isi{referential cohesion} analysis of the sample texts is included in the Appendix.

The next section compares the distribution of these data in the three text samples in order to investigate the following questions: 

\begin{enumerate}
	\item How similar or different are oral (T1 \& T2) and biblical narrative (T3) in terms of \isi{referential cohesion}?
	\item How do systematic differences manifest themselves?
	\item Given that biblical texts are strongly influenced by the \ili{Isbukun} dialect, to what extent are differences the result of dialect differentiation and to what extent of genre differentiation?
\end{enumerate}

\noindent
On a more fundamental level, these questions provide an insight in how \ili{Takivatan} speakers deal with the genre innovation that biblical narrative has brought to their literary repertoire. 

\sectref{s3.1} discusses the distribution of word classes in the three texts (data point 2), and \sectref{s3.2} that of various functional properties (data points 5--7). In \sectref{s3.3}, I look at the global properties of \isi{cohesion} in oral and biblical narrative text. Note again that this exploratory study uses a relatively small text sample.

\section{\label{s3}Discussion}

\subsection{\label{s3.1}Word class}

Let’s first have a look at how different word classes are involved in the expression of \isi{referential cohesion}. \tabref{tab:debusser:5} shows the distribution of word classes of References in the three text excerpts used in the present analysis (please take into account that these results are indicative only).

\begin{table}
\begin{tabular}{l *{4}{S[table-format=3.2]r}}
	\lsptoprule
	& \multicolumn{6}{c}{\bfseries Oral narrative} & \multicolumn{2}{c}{\bfseries Biblical narr.}\\\cmidrule(lr){2-7}\cmidrule(lr){8-9}
	& \multicolumn{2}{c}{\bfseries T1} & \multicolumn{2}{c}{\bfseries T2} & \multicolumn{2}{c}{\bfseries T1+T2} & \multicolumn{2}{c}{\bfseries T3}\\
	& \multicolumn{1}{c}{\bfseries \%} & \multicolumn{1}{c}{\bfseries \#} & \multicolumn{1}{c}{\bfseries \%} & \multicolumn{1}{c}{\bfseries \#} & \multicolumn{1}{c}{\bfseries \%} & \multicolumn{1}{c}{\bfseries \#} & \multicolumn{1}{c}{\bfseries \%} & \multicolumn{1}{c}{\bfseries \#}\\
	\midrule
	anaph. marker & 10.71 & 3 & 3.70 & 1 & 7.27 & 4 & 5.41 & 4\\
	article & 14.29 & 4 & 14.81 & 4 & 14.55 & 8 & 4.05 & 3\\
	dem. \isi{pronoun} & 10.71 & 3 & 0 & 0 & 5.45 & 3 & 0 & 0\\
	place word & 10.71 & 3 & 14.81 & 4 & 12.73 & 7 & 2.70 & 2\\
	manner word & 0 & 0 & 0 & 0 & 0 & 0 & 0 & 0\\
	noun & 21.43 & 6 & 29.63 & 8 & 25.45 & 14 & 44.59 & 33\\
	numeral & 3.57 & 1 & 0 & 0 & 1.82 & 1 & 2.70 & 2\\
	pers. \isi{pronoun} & 3.57 & 1 & 3.70 & 1 & 3.64 & 2 & 29.73 & 22\\
	time word & 14.29 & 4 & 7.41 & 2 & 10.91 & 6 & 1.35 & 1\\
	question word & 0 & 0 & 0 & 0 & 0 & 0 & 1.35 & 1\\
	verb & 10.71 & 3 & 25.93 & 7 & 18.18 & 10 & 8.11 & 6\\
	\midrule
	TOTAL & 100 & 28 & 100 & 27 & 100 & 55 & 100 & 74\\
	\lspbottomrule
\end{tabular}
\caption{\label{tab:debusser:5}Word class of References (Rc) of cohesive ties}
\end{table}

In line with expectation, nouns are the dominant \isi{word class} by a considerable margin in all texts and both text types: \isi{referential cohesion} prototypically involves reference to concrete or abstract entities, and cross-linguistically these are typically expressed by nouns. However, the relative proportion of nouns is significantly higher in biblical text than it is in oral narratives (44.59\% in T3 vs. 25.45\% in T1+T2). In oral narratives, this relative scarcity of nouns is offset by a relative abundance of place and time words. In the absence of any indication that these word classes behave differently in the \ili{Takivatan} and the \ili{Isbukun} dialect, the most likely explanation for these discrepancies is that it is a genre distinction. In oral narratives, especially when they concern historical accounts of a personally experienced past, the temporal and geographical anchoring of events is probably more important than in stories of a distant past that are mainly intended as moral lessons. Conversely, Biblical narrative often puts great emphasis on the symbolic significance of names and places; more than half of the nouns in T3 are proper names. This is evident when the distributions of ontological types of Targets are compared in both genres (\tabref{tab:debusser:6} below).

A second categorial inversion between the two genres can be observed in the distribution of \isi{demonstrative} (\isi{anaphoric} markers, determiners, and demonstratives) and personal deixis (personal pronouns). In oral narratives, \isi{demonstrative} reference accounts for 27.27\% of all referential expressions, and person pronouns for a mere 3.64\%. In the biblical excerpt, personal pronouns make up 29.73\% of all References, and the three \isi{demonstrative} classes combined only 9.46\%. It is not clear how this discrepancy can be explained as a genre distinction: oral narratives represent a highly speaker-centric form of storytelling and one would assume a relatively high proportion of personal pronominal reference. In this case, the difference is more likely due to dialect variation. For instance, as suggested in \sectref{s1.3}, bound determiners are much (the data suggests three times) more common in \ili{Takivatan} than in other dialects including \ili{Isbukun}. We also saw that, whereas \ili{Takivatan} has a highly developed free \isi{demonstrative} paradigm, putative demonstratives in \ili{Isbukun} are all complex forms involving \isi{deictic} determiners. Conversely, the \ili{Isbukun} pronominal paradigm is more complex than that in \ili{Takivatan}. In the sample, the most highly developed \isi{deictic} paradigm also has the highest relative frequency in each language variety.

Interestingly, verbs sometimes express \isi{referential cohesion}. This happens most commonly with verbs derived from nouns or locative words.

\begin{exe}
	\ex{T2.1b}\label{e6}\\
	\gll taŋus-aŋ  mati-\textbf{bahi}.\\
	first  \textsc{prog}-have.prophetic.dream\\
	\glt ‘‎‎… they interpreted a prophetic dream beforehand.’
\end{exe}

\begin{exe}
	\ex{T1.2b}\label{e7}\\
	\gll mina-\textbf{baʔav}  tupa  naip  tu:\\
	\textsc{abl}-high.location  say  \textsc{dem}.\textsc{s}.\textsc{nvis}  \textsc{compl}\\
	\glt ‘… Tiang had returned, he had come back from the mountain and told us: …’
\end{exe}

\subsection{\label{s3.2}Conceptual dimensions}

This section investigates the distribution of referential cohesive elements in terms of their conceptual-semantic properties (data points 5--7 in \sectref{s2.2}). It first looks at the ontological type of the Target. As mentioned above, referential expressions indicate concepts that can be targeted by deixis. One would assume that the prototypical Target of a referential cohesive expression is a material entity of some sort. \tabref{tab:debusser:6} shows that this is not always the case. 

\begin{table}
\begin{tabular}{l *{4}{S[table-format=3.2]r}}
	\lsptoprule
	& \multicolumn{6}{c}{\bfseries Oral narrative} & \multicolumn{2}{c}{\bfseries Biblical narr.}\\\cmidrule(lr){2-7}\cmidrule(lr){8-9}
	& \multicolumn{2}{c}{\bfseries T1} & \multicolumn{2}{c}{\bfseries T2} & \multicolumn{2}{c}{\bfseries T1+T2} & \multicolumn{2}{c}{\bfseries T3}\\
	& \multicolumn{1}{c}{\bfseries \%} & \multicolumn{1}{c}{\bfseries \#} & \multicolumn{1}{c}{\bfseries \%} & \multicolumn{1}{c}{\bfseries \#} & \multicolumn{1}{c}{\bfseries \%} & \multicolumn{1}{c}{\bfseries \#} & \multicolumn{1}{c}{\bfseries \%} & \multicolumn{1}{c}{\bfseries \#}\\
	\midrule
Entity & 39.29 & 11 & 29.63 & 8 & 34.55 & 19 & 78.38 & 58\\
Event & – & – & 18.52 & 5 & 9.09 & 5 & 1.35 & 1\\
Location / Time & 60.71 & 17 & 48.15 & 13 & 54.55 & 30 & 16.22 & 12\\
Text & – & – & 3.70 & 1 & 1.82 & 1 & 4.05 & 3\\
\midrule
TOTAL & 100 & 28 & 100 & 27 & 100 & 55 & 100 & 74\\
\lspbottomrule
\end{tabular}
\caption{\label{tab:debusser:6}Ontological type of the Target of the cohesive tie}
\end{table}

\noindent
In biblical narrative (T3) entities indeed make up more than two-thirds of the Targets of cohesive reference, the majority unsurprisingly people or concrete objects in the narrative world, e.g. \textit{Pitilu} ‘Peter’ (T3.1c), \textit{naiʤia} ‘they < Peter and John’ (T3.3c) or \textit{kim} ‘gold’ (T3.6b). However, in the \isi{oral narrative} sample this is only one-third. More than half of Targets in T1 and T2 refer to a spatial or temporal location, such as \textit{laqaiban} ‘route’ (T1.3b), \textit{ʔita} ‘there (distal)’ (T2.2a) or \textit{[tupa]-ka} ‘[say] at that time’ (T2.4g). In the previous section, I already explained that this discrepancy is genre-related. Traditional oral narratives in \ili{Bunun} culture are typically anchored in the immediate spatio-temporal context; in Biblical stories, on the other hand, identifying time and place is only of secondary importance relative to the need to keep track of people and objects that populate an unfamiliar narrative universe and commonly have a symbolic significance.

Counterintuitively, 9.09\% of referential expressions in T1 and T2 and 1.35\% in T3 refer to events. These generally are instances of event reification: events are reinterpreted as quantifiable objects, with a certain materiality and well-defined boundaries (\citealt{Quine1985}; \citealt{Zacks2001}). Finally, a number of referential cohesive ties have a meta-textual function: rather than referring to anyone or anything in the narrative universe, they point towards part of the text itself. This type of cohesive tie corresponds to what \citet{Himmelmann1996} and others refer to as \isi{discourse} deixis. In \ili{Bunun} dialects, these \isi{discourse} \isi{deictic} links are always expressed by \textit{sia}, which in \ili{Takivatan}, and possibly also in other dialects, is a specialized \isi{anaphoric} marker and typically refers back to a phrase, \isi{clause} or larger text segment in the immediate context (for shorter segments, typically the preceding sentence). For instance, in T3.3b the form \textit{sia} in \textit{sia masaniŋsiŋ pisvaŋduan} ‘the aforementioned Holy Temple’ refers back to an identical phrase in the previous sentence T3.2d, which in turn refers back to \textit{masaniŋsiŋ pisvaŋduan} ‘the Holy Temple’ (T3.1c), the first mention of this particular Referent in the story.

\begin{table}
\begin{tabular}{l *{4}{S[table-format=3.2]r}}
	\lsptoprule
	& \multicolumn{6}{c}{\bfseries Oral narrative} & \multicolumn{2}{c}{\bfseries Biblical narr.}\\\cmidrule(lr){2-7}\cmidrule(lr){8-9}
	& \multicolumn{2}{c}{\bfseries T1} & \multicolumn{2}{c}{\bfseries T2} & \multicolumn{2}{c}{\bfseries T1+T2} & \multicolumn{2}{c}{\bfseries T3}\\
	& \multicolumn{1}{c}{\bfseries \%} & \multicolumn{1}{c}{\bfseries \#} & \multicolumn{1}{c}{\bfseries \%} & \multicolumn{1}{c}{\bfseries \#} & \multicolumn{1}{c}{\bfseries \%} & \multicolumn{1}{c}{\bfseries \#} & \multicolumn{1}{c}{\bfseries \%} & \multicolumn{1}{c}{\bfseries \#}\\
	\midrule
Exophoric & 10.71 & 3 & 25.93 & 7 & 18.18 & 10 & 16.22 & 12\\
Anaphoric & 89.29 & 25 & 74.07 & 20 & 81.82 & 45 & 79.73 & 59\\
Cataphoric & – & – & – & – & – & – & 4.05 & 3\\
\midrule
TOTAL & 100 & 28 & 100 & 27 & 100 & 55 & 100 & 74\\
\lspbottomrule
\end{tabular}
\caption{\label{tab:debusser:7}Phoric function of the cohesive tie}
\end{table}

\noindent
\tabref{tab:debusser:7} gives an overview of the distribution of phoric functions of the cohesive ties in the sample. Anaphoric reference is dominant in all genres: most referents central to the text are introduced near the beginning and tend to persist throughout the story. This also explains why exophora are less common: they often occur towards the front of the text. Cataphoric reference is rare and in the present sample is only attested in biblical narrative.

Finally, \tabref{tab:debusser:8} gives a breakdown of the types of conceptual relationships between References and their Targets.\footnote{Totals in \tabref{tab:debusser:8} do not add up to 100\% because \isi{exophoric} cohesive ties have no associated conceptual relationship.} It is important to realize that these relationships are conceptual rather than lexical semantic distinctions: they hold between the concepts indicated by referential expressions, and not only lexemes, as is the case in \citet{Halliday1976}. This makes it possible, for instance, to establish a part-whole relationship between the noun \textit{ʔima} ‘hand’ (T3.7a) and the \isi{pronoun} \textit{isaiʤia} ‘\textsc{3s.poss}’ (T3.7a).

\begin{table}
\begin{tabular}{l *{4}{S[table-format=3.2]r}}
	\lsptoprule
	& \multicolumn{6}{c}{\bfseries Oral narrative} & \multicolumn{2}{c}{\bfseries Biblical narr.}\\\cmidrule(lr){2-7}\cmidrule(lr){8-9}
	& \multicolumn{2}{c}{\bfseries T1} & \multicolumn{2}{c}{\bfseries T2} & \multicolumn{2}{c}{\bfseries T1+T2} & \multicolumn{2}{c}{\bfseries T3}\\
	& \multicolumn{1}{c}{\bfseries \%} & \multicolumn{1}{c}{\bfseries \#} & \multicolumn{1}{c}{\bfseries \%} & \multicolumn{1}{c}{\bfseries \#} & \multicolumn{1}{c}{\bfseries \%} & \multicolumn{1}{c}{\bfseries \#} & \multicolumn{1}{c}{\bfseries \%} & \multicolumn{1}{c}{\bfseries \#}\\
	\midrule
Identity & 67.86 & 19 & 40.74 & 11 & 54.55 & 30 & 55.46 & 41\\
Hyponym & 7.14 & 2 & 11.11 & 3 & 9.09 & 5 & – & –\\
Hyperonym & – & – & 7.41 & 2 & 3.64 & 2 & – & –\\
Cohyponym & – & – & – & – & – & – & 2.70 & 2\\
Part & – & – & 3.70 & 1 & 1.82 & 1 & 8.11 & 6\\
Whole & 3.57 & 1 & – & – & 1.82 & 1 & 2.70 & 2\\
Copart & – & – & – & – & – & – & 1.35 & 1\\
Antonym & 7.14 & 2 & – & – & 3.64 & 2 & – & –\\
Metaphor & – & – & – & – & – & – & – & –\\
Metonym & 7.14 & 2 & 7.40 & 2 & 7.27 & 4 & 9.46 & 7\\
Symbol & – & – & – & – & – & – & – & –\\
\midrule
TOTAL & 92.86 & 26 & 70.37 & 19 & 81.82 & 45 & 83.78 & 59\\
\lspbottomrule
\end{tabular}
\caption{\label{tab:debusser:8}Cohesive relationship between Rc and Ta}
\end{table}

The introduction mentioned that \isi{referential cohesion} is not necessarily identificational and is therefore not exclusively “concerned with resources for tracking participants in \isi{discourse}” \citep[38]{Martin2001}. However, from the data it is clear that this is an important aspect of \isi{cohesion}: in both text genres, around 55\% of all cohesive ties establish relationships of \isi{identity}, and their function relates to reference tracking. Among the general semiotic relationships (metaphor, metonym, symbol), only metonymy is attested in the sample.\largerpage

One possible minor difference between genres is that \isi{oral narrative} appears to prefer hyponymic relationships, and biblical texts meronymy. However, this is in all likelihood an incidental difference resulting from the biblical story having as its main theme the miraculous healing of a physical handicap. Superficially, differences between oral and Biblical narratives appear almost non-existent, contrary to \posscitet[4]{Halliday1976} expectation that genres differ in the types of cohesive ties they employ.

In conclusion, despite differences between the frequency distribution of word classes in the two genres (see \sectref{s3.1}), and despite the fact that they have their origins in different dialects of \ili{Bunun}, oral and biblical narratives are largely similar in terms of the relative distribution of phoric properties and types of cohesive ties. The most conspicuous difference between the two genres is in the ontological type of the Target: cohesive ties in oral narratives have a higher tendency to refer to location or time, biblical narrative tends to refer more to material entities.

\subsection{\label{s3.3}Global properties}

The final section of this discussion examines the global properties of \isi{referential cohesion} in the three \ili{Bunun} text samples. As mentioned, it has been asserted that one of the ways in which \isi{cohesion} might exhibit genre-dependent variation is through consistent differences in its density. In other words, the “number and density of such networks is one of the factors which gives to any text its particular flavour or texture” \citep[52]{Halliday1976}. \citet[187--193]{Biber1995} suggests that this is indeed the case for \ili{Korean}: the degree to which cohesive relations, including \isi{referential cohesion}, are explicitly expressed varies widely between text genres. In this study, I measure density in \ili{Bunun} text in three different ways:

\begin{description}
	\item[Referential density:] The total number of words relative to the total number of References (or cohesive ties) in a text.\footnote{This measure is equivalent to \citegen[308]{Abadiano1995} \isi{cohesive density}.} Referential density gives a general impression of how much real estate cohesive referential expressions take up in a text. Note that it does not really measure which percentage of words are referential expressions, since References can be morphemes and a single word can therefore contain more than a single Reference (see e.g. \textit{daiða-ki} ‘there-\textsc{k}.\textsc{prox}’ in T1.2b).
	\item[Cohesive density:] The number of cohesive chains in a text relative to the total number of words. This is a proxy indicator of what in the quote by Halliday \& Hasan above is referred to as the density of the cohesive network, that is, how many cohesive chains weave themselves through a text of a normalized length.
	\item[Cohesive referential density:] The number of cohesive chains in a text relative to the number of References. This measure indicates the average length of cohesive chains in a text, in terms of its average number of referential expressions.
\end{description}

\noindent
\tabref{tab:debusser:9} calculates these three density metrics for the three texts and the two genres in the present sample. 

\begin{table}
\begin{tabular}{l*{4}{S[table-format=2.3]}}
\lsptoprule
& \multicolumn{3}{c}{\bfseries Oral} & \multicolumn{1}{c}{\bfseries Biblical}\\\cmidrule(lr){2-4}\cmidrule(lr){5-5}
& \multicolumn{1}{c}{\bfseries T1} & \multicolumn{1}{c}{\bfseries T2} & \multicolumn{1}{c}{\bfseries T1+T2} & \multicolumn{1}{c}{\bfseries T3}\\
\midrule
\# of words & 62 & 80 & 142 & 179\\
\# of referential expressions (Rc) & 28 & 27 & 55 & 74\\
\# of cohesive chains & 7 & 9 & 16 & 19\\
\midrule
Referential density (words / Rc) & 2.214 & 2.963 & 2.582 & 2.419\\
Cohesive density (chains / words) & 0.113 & 0.113 & 0.113 & 0.106\\
Referential \isi{cohesive density} (Rc / chains) & 4 & 3 & 3.438 & 3.895\\
\lspbottomrule
\end{tabular}
\caption{\label{tab:debusser:9}Global properties of the text segments}
\end{table}

While \isi{referential density} and referential \isi{cohesive density} both seem to be vacillating around a central value, the most surprising result is that the value for \isi{cohesive density} is almost completely equal (0.11) across texts and genres. Especially in a small sample, where a certain degree of instability is expected, it is not very likely that this is a spurious result. This is very much against initial expectation, as \isi{cohesive density} is one of the factors that one would most expect to vary across text types. For instance, in planned written text, such as our biblical narrative, tracking entities and spatio-temporal locations is cognitively less demanding than in oral narration, where visual cues that allow the listener to reaffirm the status of activated concepts are not available. The basic assumption would therefore be that written narration does not need to be as cohesively dense as oral narration.

Not only is this not the case, the present sample suggests that \isi{cohesive density} is a constant, at least in \ili{Bunun}. This is the opposite of “the possibility of cross-linguistic universals governing the patterns of \isi{discourse} variation across registers and text types” that \citet[359]{Biber1995} is looking for: what we have here is a property of the supra-clausal information structure of language that appears to be impervious to personal or genre-based variation. The reasons for the stability of this value are at present unclear. One possibility is that languages have a tendency to evolve towards a cohesive equilibrium, in which texts are as cohesive as necessary to make them coherent but not more so, an equivalent on a textual level of \posscitet{Haiman1983} competition between iconicity and expressiveness.

\section{\label{s4}Conclusion}

This leads us to an unexpected conclusion. Despite the evident grammatical differences between oral and biblical narratives in the sample, caused by dialect and genre differentiation, the conceptual properties of their underlying referential cohesive structures are surprisingly similar: against initial expectation, no major systematic differences can be observed in the phoricity or functional type of cohesive relationships. Even more so, the data suggests that, in defiance of lexical and grammatical variation in the two genres and dialects, the \isi{cohesive density} of the two genres under investigation is invariant. This may point towards a cohesive constant underlying the structure of \ili{Bunun} texts, though further research will need to verify this.

There are a small number of systematic differences between the two text genres. In terms of the referential type of the concepts they encode, referential cohesive ties in \isi{oral narrative} tend to refer more to spatial or temporal location and those in Biblical narrative more to entities. This corresponds to a predilection for place and time words in the former genre, and for nouns in the latter. I argued above that this distinction is in all likelihood due to genre-based informational demands. On the other hand, a contrast in the frequency of \isi{demonstrative} and personal deixis is probably rooted in dialect-related grammatical differences.

The present study is intended as a pilot, a fact-finding mission. Despite its modest data set, it has come up with interesting and unexpected results, but future research is necessary to test whether the present results will stand when tested against larger, statistically valid and more diversified data sets, and to find out whether regularities can be found in any of the lower-level categories. A number of questions regarding the invariance in \isi{cohesive density} need to be answered. Will the \isi{cohesive density} constant hold up in a larger sample with more genre distinctions and dialects? If so, how can it be explained? And does a similar phenomenon exist in other languages?

\section*{Acknowledgments}

The research in this chapter was made possible by grant 104-2410-H-004-139- from the Ministry of Science of Technology, Taiwan. Many thanks to the anonymous reviewers for their insightful comments and suggestions.

\section*{Abbreviations}

\begin{multicols}{2}
	\begin{tabbing}
		glossgloss \= \kill
		\textsc{?} \> function unknown\\
		\textsc{1e} \> first person exclusive\\
		\textsc{1s} \> first person singular\\
		\textsc{2s} \> second person singular\\
		\textsc{3p} \> third person plural\\
		\textsc{3s} \> third person singular\\
		\textsc{abl} \> ablative prefix expressing\\ \> movement from\\
		\textsc{all} \> allative prefix expression\\ \> movement toward\\
		\textsc{anaph} \> \isi{anaphoric} marker\\
		\textsc{art} \> article\\
		\textsc{assoc} \> associative\\
		\textsc{attr} \> attributive marker\\
		\textsc{av} \> \isi{actor voice}\\
		\textsc{caus} \> causative\\
		\textsc{compl} \> complementizer\\
		\textsc{coord} \> coordinator\\
		\textsc{cv} \> CV reduplication\\
		\textsc{d1} \> determiner paradigm 1\\
		\textsc{d2} \> determiner paradigm 2\\
		\textsc{defin} \> definitional marker\\
		\textsc{dem} \> \isi{demonstrative}\\
		\textsc{dist} \> distal\\
		\textsc{dyn} \> dynamic\\
		\textsc{emo} \> emotive\\
		\textsc{enum} \> enumerator\\
		\textsc{gnr} \> generic\\
		\textsc{hesit} \> hesitation marker\\
		\textsc{hum} \> human\\
		\textsc{inch} \> inchoative\\
		\textsc{intens} \> intensifier\\
		\textsc{inter} \> interjection\\
		\textsc{irr} \> irrealis\\
		\textsc{lnk} \> linker\\
		\textsc{loc} \> locational prefix expressing\\ \> position in or at\\
		\textsc{lv} \> \isi{locative voice}\\
		\textsc{med} \> medial\\
		\textsc{n} \> noun\\
		\textsc{neg} \> negator\\
		\textsc{nsubj} \> non-subject form\\
		\textsc{num} \> numeral\\
		\textsc{nvis} \> non-visible\\
		\textsc{obl} \> oblique case marker\\
		\textsc{p} \> plural\\
		\textsc{pauc} \> paucal\\
		\textsc{place} \> place word\\
		\textsc{poss} \> possessive\\
		\textsc{prog} \> progressive\\
		\textsc{pron} \> \isi{pronoun}\\ 
		\textsc{prox} \> proximal\\ 
		\textsc{prt} \> particle\\
		\textsc{prv} \> perfective\\
		\textsc{q} \> question word\\
		\textsc{recip} \> reciprocal\\
		\textsc{resobj} \> resultative object\\
		\textsc{s} \> singular\\
		\textsc{stat} \> stative\\
		\textsc{subj} \> subject form\\
		\textsc{subord} \> subordinator\\
		\textsc{time} \> time word\\
		\textsc{tpc} \> topicalizer\\
		\textsc{uspec} \> underspecified\\
		\textsc{uv} \> \isi{undergoer voice}\\
		\textsc{v} \> verb\\
		\textsc{vis} \> visual
	\end{tabbing}
\end{multicols}

\section*{Appendix}

\subsection*{T1: Segment oral narrative}

Source: \ili{Takivatan} \ili{Bunun} Corpus\\*
Corpus location: TVN-008-002:130-134\\*
Speaker: Vau Taisnunan, M, 75 y\\*
Location and time: Bahuan (Mayuan), 2006\\*
The excerpt below was previously published as example 22 in \citet{DeBusser2017}.

\subsubsection*{Text}

% \setcounter{exx}{0}
\begin{exe}
	\ex{\textit{Aupa tuða niaŋ tu nanu sanavan minsumina Tiaŋ, minabaʔav tupa naip tu}:}\label{tx1-1}
	\begin{xlist}
		\ex\label{tx1-1a}
		\gll aupa  tuða  ni-aŋ  tu  nanu  \textbf{sanavan}  min-suma-in-a  \textbf{Tiaŋ}\\
		thus  real  \textsc{neg}-\textsc{prog}  \textsc{compl}  really  evening  \textsc{inch}-return-\textsc{prv}-\textsc{lnk}  \textsc{T.}\\
		\ex\label{tx1-1b}
		\gll mina-\textbf{baʔav}  tupa  \textbf{naip}  tu\\
		\textsc{abl}-high.location  say  \textsc{dem}.\textsc{s}.\textsc{nvis}  \textsc{compl}\\
		\glt `But, when it wasn’t really evening yet, Tiang had returned, he had come back from the mountain and told us: …’
	\end{xlist}
\end{exe}

\begin{exe}
	\ex{\textit{Na, maqtu laqbiŋina, naʔasa dusa ta matiskun, maluʔumi han baʔav daiðaki,\linebreak pinkaunun isian baʔavta, ŋabul}.}\label{tx1-2}
	\begin{xlist}
		\ex\label{tx1-2a}
		\gll na  maqtu  \textbf{laqbiŋin}-a  na-asa  \textbf{dusa}-\textbf{ta}   ma-tiskun\\
		well  be.possible  tomorrow-\textsc{lnk}  \textsc{irr}-have.to  two-\textsc{t}.\textsc{dist}  \textsc{dyn}-in.a.group\\
		\ex\label{tx1-2b}
		\gll maluʔum-i  han  \textbf{baʔav}  \textbf{daiða}{}-\textbf{ki}\\
		disperse-\textsc{prt}  be.at  high.location  there-\textsc{k}.\textsc{prox}\\
		\ex\label{tx1-2c}
		\gll pinkaun-un  \textbf{i-sia-an}  \textbf{baʔav}{}-\textbf{ta}  \textbf{ŋabul}\\
		go.up-\textsc{uv}  \textsc{loc}{}-\textsc{anaph}{}-\textsc{lv}  high.location-\textsc{t}.\textsc{dist}  deer\\
		\glt `Well, tomorrow is possible, two of us will have to go together, and disperse when we get to this place, and we will climb upwards to the deer that is in that place above.’
	\end{xlist}
\end{exe}

\begin{exe}
	\ex{\textit{A, namaqaisaq dauka, saqnutai du sia ʔukai laqaiban}.}\label{tx1-3}
	\begin{xlist}
		\ex\label{tx1-3a}
		\gll a  na-ma-qaisaq  dau-\textbf{ka}\\
		\textsc{inter}  \textsc{irr}-\textsc{dyn}-in.that.direction  \textsc{emo}-\textsc{k}.\textsc{dist}\\
		\ex\label{tx1-3b}
		\gll saqnut-ai-du  \textbf{sia}  ʔuka-i  \textbf{laqaiban}\\
		get.stuck-\textsc{prt}-\textsc{emo}  \textsc{anaph}  \textsc{neg}.have-\textsc{prt}  route\\
		\glt `A, if he will go in that direction, he will get stuck there, without a way out.’
	\end{xlist}
\end{exe}

\begin{exe}
	\ex{\textit{Mei, mei kahaul duna ʔuka duduma laqaiban, aupa tuða, maupa tupina}. }\label{tx1-4}
	\begin{xlist}
		\ex\label{tx1-4a}
		\gll mei  mei  ka-\textbf{haul}  \textbf{dun}-a\\
		already  already  come.from.below  line-\textsc{lnk}\\
		\ex\label{tx1-4b}
		\gll ʔuka  du-duma  \textbf{laqaiban}\\
		\textsc{neg}.exist  \textsc{red}-other  route\\
		\ex\label{tx1-4c}
		\gll aupa  tuða  maupa  tupa-in-a\\
		thus  real  thus  say-\textsc{prv}-\textsc{lnk}\\
		\glt `The track is coming from below, and there is no other way out, it really is like that, thus he told us.’
	\end{xlist}
\end{exe}

\begin{exe}
	\ex{\textit{Ansaisaŋa Atul Daiŋ tu “nis, matiŋmutin tamudana madav.”}}\label{tx1-5}
	\begin{xlist}
		\ex\label{tx1-5a}
		\gll ansais-aŋ-a  \textbf{Atul} \textbf{daiŋ}  tu\\
		forbid-\textsc{prog}-\textsc{enum}  \textsc{A.}  large  \textsc{compl}\\
		\ex\label{tx1-5b}
		\gll ni-\textbf{is}  ma-\textbf{tiŋmut}-in  ta-mu-dan-a  maðʔav\\
		\textsc{neg}-\textsc{3s}.\textsc{f}  \textsc{stat}-morning-\textsc{prv}  ?-\textsc{all}-road-\textsc{lnk}  embarrassed\\
		\glt `But Big Atul forbade us: “no, when it has become morning, we will leave, it is embarrassing.’
	\end{xlist}
\end{exe}

\begin{exe}
	\ex{\textit{Na, ʔukin aipa ʔita namudanin, musbai naipa maqmut}.}\label{tx1-6}
	\begin{xlist}
		\ex\label{tx1-6a}
		\gll na  ʔuka-in  \textbf{aipa}  \textbf{ʔita}  na-mu-dan-in\\
		well  \textsc{neg}.have-\textsc{prv}  \textsc{dem}.\textsc{s}.\textsc{dist}.\textsc{vis}  there.\textsc{dist}  \textsc{irr}-\textsc{all}-go-\textsc{prv}\\
		\ex\label{tx1-6b}
		\gll mu-isbai  \textbf{naipa}  \textbf{maqmut}\\
		\textsc{all}-cause.to.move  \textsc{dem}.\textsc{s}.\textsc{dist}.\textsc{nvis}  night.time\\
		\glt `Well, it will not be there anymore, it will be gone, it will have run away during the night.’
	\end{xlist}
\end{exe}


\subsubsection*{Cohesion analysis}

\tabref{tab:debusser:10} contains an analysis of referential cohesive elements in text T1 above. Numbers in the headers refer to the data points referred to in \sectref{s2.2}.

\begin{sidewaystable} 
  \fittable{\small
\begin{tabular}{ll l lll lll}
	\lsptoprule
	\bfseries (1) & \bfseries Reference (Rc) & \bfseries Rc word  & \bfseries (3) & \bfseries Target (Ta) & \bfseries Referent (4) & \bfseries Functional  & \bfseries Phoric  & \bfseries Rel. \\
                    &                            & \bfseries class (2)\is{word class} & & & & \bfseries role Ta (5) & \bfseries function~(6)  & \bfseries Rc-Ta (7)\\
	\midrule
	1a & \textit{sanavan} ‘evening’ & \textsc{time} & ~ & ~ & time of day & Loc/Time & Exophoric & ~\\
	1a & \textit{Tiaŋ} ‘T.’ & \textsc{n} & ~ & (prev. text) & Tiaŋ & Entity & Anaphoric & Identity\\
	1b & \textit{[mina-]baʔav} ‘come from the mountain’ & \textsc{v} & ~ & (prev. text) & location deer & Loc/Time & Anaphoric & Identity\\
	1b & \textit{naip} ‘\textsc{dem.s.nvis}’ & \textsc{dem} & 1a & \textit{Tiaŋ} ‘T.’ & Tiaŋ & Entity & Anaphoric & Identity\\
	2a & \textit{laqbiŋin[-a]} ‘tomorrow’ & \textsc{time} & 1a & \textit{sanavan} ‘evening’ & time of day & Loc/Time & Anaphoric & Metonym\\
	2a & \textit{dusa-ta} ‘two’ & \textsc{num} & ~ & (prev. text) & we & Entity & Exophoric & Hyponym\\
	2a & \textit{[dusa]-ta} ‘\textsc{art.ent.dis}’ & \textsc{art} & 1b & \textit{[mina-]baʔav} ‘come from the mountain’ & location deer & Loc/Time & Anaphoric & Identity\\
	2b & \textit{baʔav} ‘high location’ & \textsc{v} & 2a & \textit{[dusa]-ta} ‘\textsc{art.ent.dis}’ & location deer & Loc/Time & Anaphoric & Identity\\
	2b & \textit{daiða[-ki]} ‘that place’ & \textsc{place} & 2b & \textit{baʔav} ‘high location’ & location deer & Event & Anaphoric & Identity\\
	2b & \textit{[daiða]-ki} ‘\textsc{art.evt.prox}’ & \textsc{art} & 2a & \textit{laqbiŋin} ‘tomorrow’ & time of day & Loc/Time & Anaphoric & Identity\\
	2c & \textit{i-sia-an} ‘the place of that one’ & \textsc{anaph} & 2b & \textit{daiða-ki} ‘that place’ & location deer & Loc/Time & Anaphoric & Identity\\
	2c & \textit{[i-]sia[-an]} ‘\textsc{anaph}’ & \textsc{anaph} & ~ & (\textit{dapana} ‘foot prints’) [008-002:125] & deer & Entity & Anaphoric & Whole\\
	2c & \textit{baʔav[-ta]} ‘high location’ & \textsc{place} & 2c & \textit{i-sia-an} ‘the place of that one’ & location deer & Loc/Time & Anaphoric & Identity\\
	2c & \textit{[baʔav]-ta} ‘\textsc{art.ent.dist}’ & \textsc{art} & 2c & \textit{sia} ‘\textsc{anaph}’ & location deer & Entity & Anaphoric & Identity\\
	2c & \textit{ŋabul} ‘deer’ & \textsc{n} & 2c &  \textit{{}-ta} ‘\textsc{art}’ & deer & Entity & Anaphoric & Identity\\
	3a & \textit{[na-ma-qaisaq-dau]-ka} ‘\textsc{art.evt.dist}’ & \textsc{art} & 2c & \textit{baʔav-ta} ‘high location’ & location deer & Loc/Time & Anaphoric & Identity\\
	3b & \textit{sia} ‘\textsc{anaph}’ & \textsc{anaph} & 2c & \textit{ŋabul} ‘deer’ & deer & Entity & Anaphoric & Identity\\
	3b & \textit{laqaiban} ‘route’ & \textsc{n} & ~ & ~ & route deer & Loc/Time & Exophoric & ~\\
	4a & [ka-]haul ‘below’ & \textsc{v} & 3a & \textit{[na-ma-qaisaq-dau]-ka} ‘\textsc{art.evt.dist}’ & location deer & Loc/Time & Anaphoric & Antonym\\
	4a & dun ‘line’ & \textsc{n} & 3b & \textit{laqaiban} ‘route’ & route deer & Loc/Time & Anaphoric & Identity\\
	4b & laqaiban ‘route’ & \textsc{n} & 4a & dun ‘line’ & route deer & Loc/Time & Anaphoric & Identity\\
	5a & \textit{Atul daiŋ} ‘Big Atul’ & \textsc{n} & ~ & (\textit{nas-Atul daiŋ} ‘the erstwhile Big Atul’) [008-002:126] & Atul & Entity & Anaphoric & Identity\\
	5b & \textit{[ni]-is} ‘\textsc{3s.f}’ & \textsc{pron} & 3b & \textit{sia} ‘anaph’ & deer & Entity & Anaphoric & Identity\\
	5b & \textit{[ma-]tiŋmut[-in]} ‘morning’ & \textsc{time} & 2b & \textit{[daiða]-ki} ‘\textsc{art.sit.prox}’ & time of day & Loc/Time & Anaphoric & Hyponym\\
	6a & \textit{aipa} ‘\textsc{dem.s.dist.vis}’ & \textsc{dem} & 3b & \textit{[ni]-is} ‘\textsc{3s.f}’ & deer & Entity & Anaphoric & Identity\\
	6a & \textit{ʔita} ‘there.\textsc{dist}’ & \textsc{place} & 4a & [ka-]haul ‘below’ & location deer & Loc/Time & Anaphoric & Antonym\\
	6b & \textit{naipa} ‘\textsc{dem.s.dist.nvis}’ & \textsc{dem} & 5a & \textit{aipa} ‘\textsc{dem.s.dist.vis}’ & deer & Entity & Anaphoric & Identity\\
	6b & \textit{maqmut} ‘night time’ & \textsc{time} & 4b & \textit{[ma-]tiŋmut[-in]} ‘morning’ & time of day & Loc/Time & Anaphoric & Metonym\\
	\lspbottomrule
\end{tabular}
}
\caption{\label{tab:debusser:10}Referential cohesion analysis T1} 
\end{sidewaystable}

\clearpage
\subsection*{T2: Segment oral narrative}

Source: \ili{Takivatan} \ili{Bunun} Corpus\\*
Corpus location: TVN-012-001:38-41\\*
Speaker: Tulbus Manququ, M, 75 y\\*
Location and time: Bahuan (Mayuan), 2006\\*

\subsubsection*{Text}

% \setcounter{exx}{0}
\begin{exe}
	\ex{\textit{Maqai maqabasi tupa tu madaiŋʔaði namuqumaka taŋusaŋ matibahi}.}\label{tx2-1}
	\begin{xlist}
		\ex\label{tx2-1a}
		\gll maqai  ma-\textbf{qabas}{}-i  tupa  tu  \textbf{ma-daiŋʔað}{}-i  na-mu-\textbf{quma}{}-\textbf{ka}\\
		if  \textsc{dyn}{}-in.former.times-\textsc{prt}  say  \textsc{compl}  \textsc{stat}{}-old-\textsc{prt}  \textsc{irr}{}-\textsc{all}{}-field-\textsc{k.dist}\\
		\ex\label{tx2-1b}
		\gll taŋus-aŋ  mati-\textbf{bahi}\\
		first  \textsc{prog}{}-have.prophetic.dream\\
		\glt `‎‎If in the old days the elders said they wanted to work on the land, they interpreted a prophetic dream beforehand.’
	\end{xlist}
\end{exe}

\begin{exe}
	\ex{\textit{Namaqun ʔita maqai masihala bahia, tudip, na, sintupadu tu maqai ʔitun asa namasihal kakaunun}.}\label{tx2-2}
	\begin{xlist}
		\ex\label{tx2-2a}
		\gll na-maqun  \textbf{ʔita}\\
		\textsc{irr}-cut.off  there.\textsc{dist}\\
		\ex\label{tx2-2b}
		\gll maqai  ma-sihal-a  \textbf{bahi}-a  \textbf{tudip}\\
		if  \textsc{stat}-good-\textsc{subord}  prophetic.dream-\textsc{subord}  that.time\\
		\ex\label{tx2-2c}
		\gll na  sin-tupa-du  tu  maqai  \textbf{ʔitun}\\
		well  \textsc{res}.\textsc{obj}-say-\textsc{emo}  \textsc{compl}  if  there.\textsc{med}\\
		\ex\label{tx2-2d}
		\gll asa  na-ma-sihal  ka-kaun-un\\
		be.able  \textsc{irr}-\textsc{stat}-good  \textsc{cv}-eat-\textsc{uv}\\
		\glt `And when they wanted to go there to harvest (lit: when they wanted to cut off things in that place), if the dream was good, that meant in those days that if you were there, you could eat very well.’
	\end{xlist}
\end{exe}\largerpage

\begin{exe}
	\ex{\textit{A maqai dipi madiqla bahia tupa tu asa ni ʔituni nalauq, nitu na … masihala\linebreak kakauna sanasia maqai, amin tu maqai ʔitun namuqða kuðaki madiqla bahi, na haiða matað}.}\label{tx2-3}
	\begin{xlist}
		\ex\label{tx2-3a}
		\gll a  maqai  \textbf{dip}{}-i  ma-diqla  \textbf{bahi}-a\\
		\textsc{inter}  if  then-\textsc{prt}  \textsc{stat}-bad  prophetic.dream-\textsc{lnk}\\
		\ex\label{tx2-3b}
		\gll tupa  tu  asa  ni  \textbf{ʔitun}{}-i\\
		say  \textsc{compl}  have.to  \textsc{neg}  there.\textsc{med}-\textsc{prt}\\
		\ex\label{tx2-3c}
		\gll nalauq  ni  tu  na  ma-sihal-a  ka-kaun-a\\
		otherwise  \textsc{neg}  \textsc{compl}  well  \textsc{stat}-good-\textsc{lnk}  \textsc{cv}-eat-\textsc{lnk}\\
		\ex\label{tx2-3d}
		\gll sana-\textbf{sia}  maqai\\
		\textsc{according.to}-\textsc{anaph}  if\\
		\ex\label{tx2-3e}
		\gll amin  tu  maqai  \textbf{ʔitun}  na-muqða  kuða-\textbf{ki}\\
		all  \textsc{compl}  if  there.\textsc{med}  \textsc{irr}-again  work-\textsc{k.prox}\\
		\ex\label{tx2-3f}
		\gll ma-diqla  \textbf{bahi}\\
		\textsc{stat}-bad  prophetic.dream\\
		\ex\label{tx2-3g}
		\gll na  haiða  matað\\
		well  have  die\\
		\glt `And if the dream was bad, then they said that you must not go there, because otherwise you would not eat well, if you followed the rule, but if anyone at all went back to that place to work, and there was a bad dream, people would die.’
	\end{xlist}
\end{exe}

\largerpage[2]
\begin{exe}
	\ex{\textit{A, maqai mataisaq … matataisaq a madadaiŋʔað tu, … maqai munʔitaʔa mavia mataisaq tu saduʔuki siatu, sinsusuað bunuað masmamua mavisqai, mavilasa tupaka madadaiŋʔað tu na maqtu munquma istaʔai nakasihalain kakaunun namasihala bunun}.}\label{tx2-4}
	\begin{xlist}
		\ex\label{tx2-4a}
		\gll a  maqai  ma-taisaq\\
		\textsc{inter}  if  \textsc{dyn}-dream\\
		\ex\label{tx2-4b}
		\gll ma-ta-taisaq  a  \textbf{madadaiŋʔað}  tu\\
		\textsc{dyn}-\textsc{cv}-dream  INTER  elder  \textsc{compl}\\
		\ex\label{tx2-4c}
		\gll maqai  mun-\textbf{ʔita}  a  ma-via  ma-taisaq  tu\\
		if  \textsc{all}-there.\textsc{dist}  \textsc{hesit}  \textsc{dyn}-why  \textsc{dyn}-dream  \textsc{compl}\\
		\ex\label{tx2-4d}
		\gll saduʔu-\textbf{ki}  \textbf{sia}  tu\\
		see-\textsc{k}.\textsc{prox}  \textsc{anaph}  \textsc{compl}\\
		\ex\label{tx2-4e}
		\gll sin-su-suað  \textbf{bunuað}  mas-ma-muav  ma-visqa-i\\
		\textsc{res.obj}-\textsc{cv}-sow  plum  \textsc{be}-\textsc{cv}-excessive  \textsc{stat}-abundant.with.fruit-\textsc{prt}\\
		\ex\label{tx2-4f}
		\gll mavi-\textbf{las}-a\\
		\textsc{contain}-fruits-\textsc{lnk}\\
		\ex\label{tx2-4g}
		\gll tupa-\textbf{ka}  \textbf{madadaiŋʔað}  tu\\
		tell-\textsc{k}.\textsc{dist}  elder  \textsc{compl}\\
		\ex\label{tx2-4h}
		\gll na  maqtu  mun-\textbf{quma}  \textbf{ista}-ai\\
		well  be.possible.to  \textsc{all}-field  \textsc{3s}.\textsc{dist}-\textsc{prt}\\
		\ex\label{tx2-4i}
		\gll na-ka-sihal-in  ka-kaun-un\\
		\textsc{irr}-\textsc{assoc}.\textsc{dyn}-good-\textsc{prv}  \textsc{cv}-eat-\textsc{un}\\
		\ex\label{tx2-4j}
		\gll na-ma-sihal-a  \textbf{bunun}\\
		\textsc{irr}-\textsc{stat}-good-\textsc{lnk}  people\\
		\glt `And if they dreamt… if the elders dreamt that, if they went over there, they suddenly dreamt that they saw that the plum tree had grown so that it was full of fruits and had large fruits, then the elders would say that it was permitted for them to the land to work, and they would produce good fruits, and the people would also be fine.’
	\end{xlist}
\end{exe}


 
\begin{sidewaystable} 
		\fittable{\small
\begin{tabular}{ll l lll lll}
	\lsptoprule
	\bfseries (1) & \bfseries Reference (Rc) & \bfseries Rc word  & \bfseries (3) & \bfseries Target (Ta) & \bfseries Referent (4) & \bfseries Functional  & \bfseries Phoric  & \bfseries Rel. \\
                    &                            & \bfseries class (2)\is{word class} & & & & \bfseries role Ta (5) & \bfseries function~(6)  & \bfseries Rc-Ta (7)\\
	\midrule
1a & \textit{ma-qabas[-i]} ‘in the old days’ & \textsc{v} &  &  & in former times & Loc/Time & Exophoric & \\
1a & \textit{ma-daiŋ-ʔað[-i]} ‘elders’ & \textsc{n} &  &  & elders & Entity & Exophoric & \\
1a & \textit{[na-mun-]quma[-ka]}’go work on the land’ & \textsc{v} &  &  & land & Loc/Time & Exophoric & \\
1a & \textit{[na-mun-quma]-ka} ‘over there’ & \textsc{art} &  &  & location land & Loc/Time & Exophoric & \\
1b & \textit{[mati-]bahi} ‘have a prophetic dream’ & \textsc{v} &  &  & dream & Event & Exophoric & \\
2a & \textit{ʔita} ‘there.\textsc{dist}’ & \textsc{place} & 1a & \textit{[na-mun-]quma[-ka]}’go work on the land’ & land & Loc/Time & Anaphoric & Identity\\
2b & \textit{bahi[-a]} ‘prophetic dream’ & \textsc{n} & 1b & \textit{[mati-]bahi} ‘have a prophetic dream’ & dream & Event & Anaphoric & Hyponym\\
2b & \textit{tudip} ‘that time’ & \textsc{time} & 1a & \textit{[ma-qabas]-i} ‘in the old days’ & in former times & Loc/Time & Anaphoric & Hyponym\\
2c & \textit{ʔitun} ‘there.\textsc{med}’ & \textsc{place} &  &  & village & Loc/Time & Exophoric & \\
3a & \textit{dip[-i]} ‘then’ & \textsc{time} & 2b & \textit{tudip} ‘that time’ & in former times & Loc/Time & Anaphoric & Metonym\\
3a & \textit{bahi[-a]} ‘prophetic dream’ & \textsc{n} & 2b & \textit{bahi[-a]} ‘prophetic dream’ & dream & Event & Anaphoric & Hyponym\\
3b & \textit{ʔitun[-i]} ‘there.\textsc{med}’ & \textsc{place} & 2a & \textit{ʔita} ‘there.\textsc{dist}’ & land & Loc/Time & Anaphoric & Identity\\
3d & \textit{[sana-]sia} ‘according to the aforementioned’ & \textsc{v} &  &  & text & Text & Anaphoric & \\
3e & \textit{ʔitun} ‘there.\textsc{med}’ & \textsc{place} & 3b & \textit{ʔitun[-i]} ‘there.\textsc{med}’ & land & Loc/Time & Anaphoric & Identity\\
3e & \textit{[kuða]-ki} ‘(the work) in this place’ & \textsc{art} & 3e & \textit{ʔitun} ‘there.\textsc{med}’ & land & Loc/Time & Anaphoric & Identity\\
3f & \textit{bahi} ‘prophetic dream’ & \textsc{n} & 3a & \textit{bahi[-a]} ‘prophetic dream’ & dream & Event & Anaphoric & Identity\\
4b & \textit{madadaiŋʔað} ‘elders’ & \textsc{n} & 1a & \textit{ma-daiŋ-ʔað[-i]} ‘elders’ & elders & Entity & Anaphoric & Identity\\
4c & \textit{[mun-]ʔita} ‘(go) over there’ & \textsc{v} & 3e & \textit{ʔitun} ‘there.\textsc{med}’ & land & Loc/Time & Anaphoric & Identity\\
4d & \textit{[saduʔu]-ki} ‘(see) here’ & \textsc{art} & 3f & \textit{bahi} ‘prophetic dream’ & dream & Event & Anaphoric & Metonym\\
4d & \textit{sia} ‘\textsc{anaph}’ & \textsc{anaph} & 4b & \textit{madadaiŋʔað} ‘elders’ & elders & Entity & Anaphoric & Identity\\
4e & \textit{bunuað} ‘plum’ & \textsc{n} &  &  & plum tree & Entity & Exophoric & \\
4f & \textit{[mavi-]las[-a]} ‘(be full of) fruits’ & \textsc{v} & 4e & \textit{bunuað} ‘plum’ & plum tree & Entity & Anaphoric & Hyperonym\\
4g & \textit{[tupa]-ka} ‘(say) at that time’ & \textsc{art} & 3a & \textit{dip[-i]} ‘then’ & in former times & Loc/Time & Anaphoric & Part\\
4g & \textit{madadaiŋʔað ‘elders’} & \textsc{n} & 4d & \textit{sia} ‘\textsc{anaph}’ & elders & Entity & Anaphoric & Identity\\
4h & \textit{[mun-]quma} ‘go to the field’ & \textsc{v} & 4c & \textit{[mun-]ʔita} ‘(go) over there’ & land & Loc/Time & Anaphoric & Identity\\
4h & \textit{ista[-ai]} ‘\textsc{3s.dist}’ & \textsc{pron} & 4g & \textit{madadaiŋʔað ‘elders’} & elders & Entity & Anaphoric & Identity\\
4j & \textit{bunun} ‘people’ & \textsc{n} & 4h & \textit{ista[-ai]} ‘\textsc{3s.dist}’ & elders & Entity & Anaphoric & Hyperonym\\
\lspbottomrule
\end{tabular}
}
\caption{\label{tab:debusser:11}Referential cohesion analysis T2} 
\end{sidewaystable}
\clearpage
\subsection*{T3: Segment Biblical narrative}

Source: \textit{Tama Dihanin tu Halinga. The \ili{Bunun} Bible in Today’s Taiwan \ili{Bunun} Version} \citep{Bible2000}\\*
Corpus location: Acts 3:1-10

\subsubsection*{Text}

% \setcounter{exx}{0}
\begin{exe}
	\ex{\textit{Aiða tu hanian, masa tauŋhuvalin tu ʤintau, Pitilu mas Iuhani hai kusia Masaniŋsiŋ Pisvaŋduan}.}\label{tx3-1}
	\begin{xlist}
		\ex\label{tx3-1a}
		\gll ʔaiða  tu  \textbf{hanian}\\
		exist  \textsc{compl}  day\\
		\ex\label{tx3-1b}
		\gll masa  \textbf{tauŋhuvali}-in  tu  \textbf{ʤin-ta}\\
		\textsc{when}  noon-\textsc{prv}  \textsc{attr}  \textsc{hour}-three\\
		\ex\label{tx3-1c}
		\gll \textbf{Pitilu}  mas  \textbf{Iuhani}  hai  kusia  \textbf{masaniŋsiŋ} \textbf{pisvaŋduan}\\
		Peter  \textsc{obl}  John  \textsc{tpc}  use  holy  temple\\
		\glt `There was a day, when it was three at noon, that Peter and John were using the Holy Temple.’
	\end{xlist}
\end{exe}

\begin{exe}
	\ex{\textit{Isia tupaun tu Manauað Ilav ʤia hai, aiða tu taʤini maisna tausʔuvaðun mapiha, kaupa hanian ansahanun mas bunun mapunsia ilav ʤia, makikisaiv mas\linebreak nakuŋadah sia Masaniŋsiŋ Pisvaŋduan tu bunun}.}\label{tx3-2}
	\begin{xlist}
		\ex\label{tx3-2a}
		\gll \textbf{i}-\textbf{sia}  tupa-un  tu  \textbf{manauʔað}  \textbf{ʔilav}-\textbf{ʤia}  hai\\
		\textsc{poss}-\textsc{anaph}  say-\textsc{uv}  \textsc{compl}  beautiful  door-\textsc{dist.obl}  \textsc{tpc}\\
		\ex\label{tx3-2b}
		\gll ʔaiða  tu  \textbf{taʤini}  maisna  tausʔuvað-un  \textbf{ma}-\textbf{piha}\\
		exist  \textsc{compl}  one.\textsc{hum}  from  give.birth-\textsc{uv}  \textsc{stat}-cripple\\
		\ex\label{tx3-2c}
		\gll kaupa  \textbf{hanian}  ansahan-un  mas  \textbf{bunun}  ma-pun-\textbf{sia}  \textbf{ʔilav}-\textbf{ʤia}\\
		each  day  carry.to-\textsc{uv}  \textsc{obl}  person  \textsc{dyn}-\textsc{caus}.\textsc{all}-\textsc{anaph}  door-\textsc{dist}.\textsc{obl}\\
		\ex\label{tx3-2d}
		\gll ma-ki-kisaiv  mas  na-ku-\textbf{ŋadah}  \textbf{sia}  \textbf{masaniŋsiŋ} \textbf{pisvaŋduan}  tu  \textbf{bunun}\\
		\textsc{dyn}-\textsc{red}-give  \textsc{obl}  \textsc{irr}-\textsc{assoc}.\textsc{all}-lower  \textsc{anaph}  holy  temple  \textsc{attr}  person\\
		\glt `At what was called the Beautiful Gate, there was one man who was cripple from birth, and people carried him every day and put them at that door, and he begged to people that went down into the Holy Temple.’
	\end{xlist}
\end{exe}

\begin{exe}
	\ex{\textit{Sadu saia tu Pitilu mas Iuhani hai nakuŋadah sia Masaniŋsiŋ Pisvaŋduan, at makisaiv naiʤia}.}\label{tx3-3}
	\begin{xlist}
		\ex\label{tx3-3a}
		\gll sadu  \textbf{saia}  tu\\
		see  \textsc{3s}.\textsc{nom}  \textsc{compl}\\
		\ex\label{tx3-3b}
		\gll \textbf{Pitilu}  mas  \textbf{Iuhani}  hai  na-ku-\textbf{ŋadah}  \textbf{sia}  \textbf{masaniŋsiŋ}  \textbf{pisvaŋduan}\\
		Peter  \textsc{obl}  John  \textsc{tpc}  \textsc{irr}-\textsc{assoc}.\textsc{all}-lower  \textsc{anaph}  holy  temple\\
		\ex\label{tx3-3c}
		\gll at  makisaiv  \textbf{naiʤia}\\
		and  make.give  \textsc{3p}.\textsc{obj}\\
		\glt `He saw that Peter and John were about to enter the Holy Temple and made them give (money) [tried to ask them for money].’
	\end{xlist}
\end{exe}

\begin{exe}
	\ex{\textit{Naia hai samantuk saiʤia tupa Pitilu tu: “Sadua kasu maðami!”}}\label{tx3-4}
	\begin{xlist}
		\ex\label{tx3-4a}
		\gll \textbf{naia}  hai  samantuk  \textbf{saiʤia}\\
		\textsc{3p}.\textsc{nom}  \textsc{tpc}  keep.close.watch.on  \textsc{3s}.\textsc{obl}\\
		\ex\label{tx3-4b}
		\gll tupa  \textbf{Pitilu}  tu\\
		say  Peter  \textsc{compl}\\
		\ex\label{tx3-4c}
		\gll sadu-a  \textbf{kasu}  ma-\textbf{ðami}\\
		see-\textsc{lnk}  \textsc{2s}.\textsc{nom}  \textsc{dyn}-\textsc{1e}.\textsc{obl}\\
		\glt `They looked straight at him, and Peter said as follows: “You look at us!”’
	\end{xlist}
\end{exe}

\begin{exe}
	\ex{\textit{Saia hai samantuk naiʤia, asa usiðan maðmað}.}\label{tx3-5}
	\begin{xlist}
		\ex\label{tx3-5a}
		\gll \textbf{saia}  hai  samantuk  \textbf{naiʤia}\\
		\textsc{3s.nom}  \textsc{tpc}  keep.close.watch.on  \textsc{3p.obl}\\
		\ex\label{tx3-5b}
		\gll asa  u-siða-an  \textbf{maðmað}\\
		want  \textsc{able.to}-take-\textsc{lv}  which.things\\
		\glt `He looked straigth at them, he wanted to be able to get something from them.’
	\end{xlist}
\end{exe}

\begin{exe}
	\ex{\textit{Pitilu hai tupa saiʤia tu: “Ukan saikin kim mas sui, haitu nasaivan ku kasu mas inak tu iskakaupa: Mapakasia saikin mas itu takisia Naðale tu Iesu Kilistu tu ŋan tupa masu tu, mindaŋkaða mudan!”}}\label{tx3-6}
	\begin{xlist}
		\ex\label{tx3-6a}
		\gll \textbf{Pitilu}  hai  tupa  \textbf{saiʤia}  tu\\
		Peter  \textsc{tpc}  say  \textsc{3s}.\textsc{obl}  \textsc{compl}\\
		\ex\label{tx3-6b}
		\gll ʔuka-an  \textbf{saikin}  \textbf{kim}  mas  \textbf{sui}\\
		\textsc{neg}.have-\textsc{lv}  \textsc{1s}.\textsc{top}.\textsc{ag}   gold  \textsc{coord}  money\\
		\ex\label{tx3-6c}
		\gll haitu  na-saiv-an-\textbf{ku}  \textbf{kasu}  mas  \textbf{i}-\textbf{nak}  tu  \textbf{iskakaupa}\\
		although  \textsc{irr}-give-\textsc{lv}-\textsc{1s}.\textsc{nsubj}  \textsc{2s}.\textsc{nom}  \textsc{obl}  \textsc{poss}-\textsc{1s}.\textsc{n}  \textsc{attr}  everything\\
		\ex\label{tx3-6d}
		\gll ma-paka-\textbf{sia}  \textbf{saikin}  mas  itu  taki-\textbf{sia}  \textbf{Naðale}  tu  \textbf{Iesu}  \textbf{Kilistu}  tu  \textbf{ŋan}\\
		\textsc{dyn}-\textsc{recip}-\textsc{anaph}  \textsc{1s}.\textsc{top}.\textsc{ag}  \textsc{obl}  this.here  \textsc{origin}-\textsc{anaph}  Nazareth  \textsc{attr}  Jezus  Christ  \textsc{attr}  name\\
		\ex\label{tx3-6e}
		\gll tupa  \textbf{masu}  tu\\
		say  \textsc{2s}.\textsc{obl}  \textsc{compl}\\
		\ex\label{tx3-6f}
		\gll mindaŋkað-a  mu-dan\\
		stand.up-\textsc{lnk}  \textsc{all}-go\\
		\glt `Peter told him: “I do not have gold or money here, but I will give you everything I have here. I use the name of Jesus Christ who comes from Nazareth to tell you: stand up and walk.”’
	\end{xlist}
\end{exe}

\begin{exe}
	\ex{\textit{Pitilu hai maʔalak mas isaiʤia tu tanaskaun ima, sidaŋkað saiʤia}.}\label{tx3-7}
	\begin{xlist}
		\ex\label{tx3-7a}
		\gll \textbf{Pitilu}  hai  ma-ʔalak  mas  \textbf{isaiʤia}  tu  tanaskaun  \textbf{ʔima}\\
		Peter  \textsc{tpc}  \textsc{dyn}-lead  \textsc{obl}  \textsc{3s}.\textsc{poss}  \textsc{attr}  right  hand\\
		\ex\label{tx3-7b}
		\gll si-daŋkað  \textbf{saiʤia}\\
		?-stand  \textsc{3s}.\textsc{obl}\\
		\glt `Peter led him by the right hand, and helped him to stand.’
	\end{xlist}
\end{exe}

\begin{exe}
	\ex{\textit{Itu bunun ʤia bantas mas buhtuŋ hai ʤinsu istamasaðan, at mataidaða, matuduldul, kitŋab mudadan}.}\label{tx3-8}
	\begin{xlist}
		\ex\label{tx3-8a}
		\gll itu  \textbf{bunun}-\textbf{ʤia}  \textbf{bantas}  mas  \textbf{buhtuŋ}  hai  ʤinsu  is-tamasað-an\\
		this.here  people-\textsc{dist}.\textsc{obl}  leg.and.foot  \textsc{obl}  joint  \textsc{tpc}  immediately  \textsc{transfer}-strength-\textsc{lv}\\
		\ex\label{tx3-8b}
		\gll at  mataidaða  matuduldul  kitŋab  mu-da-dan\\
		and  jump  stand  begin  \textsc{all}-\textsc{red}-road\\
		\glt `This man here, his legs and joints immediately became powerful, and he jumped up and stood, and he began to walk.’
	\end{xlist}
\end{exe}

\begin{exe}
	\ex{\textit{Saia hai taskun naiʤia kuŋadah sia Masaniŋsiŋ Pisvaŋduan, maʤishahainað mudadan, matumashiŋ mas Sasbinað Dihanin}.}\label{tx3-9}
	\begin{xlist}
		\ex\label{tx3-9a}
		\gll \textbf{saia}  hai  taskun  \textbf{naiʤia}  ku-\textbf{ŋadah}  \textbf{sia}  \textbf{masaniŋsiŋ  pisvaŋduan}\\
		\textsc{3s.nom  tpc}  do.together  \textsc{3p.nsubj}  \textsc{assoc.all}{}-lower  \textsc{anaph}  holy  temple\\
		\ex\label{tx3-9b}
		\gll maʤishahainað  mu-da-dan  matumashiŋ  mas  \textbf{Sasbinað Dihanin}\\
		gleeful  \textsc{all-red}{}-road  thank  \textsc{obl}  God\\
		\glt `Together with them he entered the Holy Temple, and gleefully walk over and he thanked God.’
	\end{xlist}
\end{exe}

\largerpage[2]
\begin{exe}
	\ex{\textit{\ili{Bunun} hai sadu saiʤiaa tu mudadan, at matumashiŋ mas Sasbinað Dihanin, at ʤiŋhuða, au pa sahal naia tu saia hai takisia Masaniŋsiŋ Pisvaŋduan tu Manauað Ilav malʔanuhu makisasaiv tu bunun}.}\label{tx3-10}
	\begin{xlist}
		\ex\label{tx3-10a}
		\gll \textbf{bunun}  hai  sadu  \textbf{saiʤia}{}-a  tu  mu-da-dan\\
		people  \textsc{tpc}  see  \textsc{3s.obl}{}-\textsc{lnk}  \textsc{compl}  \textsc{all}{}-\textsc{red}{}-road\\
		\ex\label{tx3-10b}
		\gll at  matumashiŋ  mas  \textbf{Sasbinað Dihanin}\\
		and  thank  \textsc{obl}  God\\
		\ex\label{tx3-10c}
		\gll at  ʤiŋhuða\\
		and  be.startled\\
		\ex\label{tx3-10d}
		\gll aupa  sahal  \textbf{naia}  tu\\
		because  clearly  \textsc{3p.nom}   \textsc{compl}\\
		\ex\label{tx3-10e}
		\gll \textbf{saia}  hai  taki-\textbf{sia}  \textbf{masaniŋsiŋ} \textbf{pisvaŋduan}  tu  \textbf{manauʔað} \textbf{ʔilav}  malʔanuhu  sa-makisaiv  tu  \textbf{bunun}\\
		\textsc{3s.nom}  \textsc{tpc}  have.origins.in-\textsc{anaph}  holy  temple  \textsc{attr}  beautiful  door  sit.down  \textsc{see}{}-beg  \textsc{attr}  person\\
		\glt `People saw him walk, and thank God, and they were startled, because they recognized him as the man that used to beg sitting down at the Beautiful Door that was the entrance to the Holy Temple.’
	\end{xlist}

\end{exe}

\clearpage
\begin{sidewaystable} 
		\fittable{\small
\begin{tabular}{ll l lll lll}
	\lsptoprule
	\bfseries (1) & \bfseries Reference (Rc) & \bfseries Rc word  & \bfseries (3) & \bfseries Target (Ta) & \bfseries Referent (4) & \bfseries Functional  & \bfseries Phoric  & \bfseries Rel. \\
                    &                            & \bfseries class (2)\is{word class} & & & & \bfseries role Ta (5) & \bfseries function~(6)  & \bfseries Rc-Ta (7)\\
	\midrule
	1a & \textit{hanian} ‘day’ & \textsc{n} & ~ & ~ & time of story & Loc/Time & Exophoric & ~\\
	1b & \textit{tauŋ}\textit{huvali[-in]} ‘having become noon’ & \textsc{v} & 1b & \textit{ʤin-ta} ‘three o’clock’ & time of day & Loc/Time & Cataphoric & Metonym\\
	1b & \textit{ʤin-ta} ‘three o’clock’ & \textsc{num} & 1a & \textit{hanian} ‘day’ & time of day & Loc/Time & Anaphoric & Part\\
	1c & \textit{Pitilu} ‘Peter’ & \textsc{n} & ~ & ~ & Peter & Entity & Exophoric & ~\\
	1c & \textit{Iuhani} ‘John’ & \textsc{n} & ~ & ~ & John & Entity & Exophoric & ~\\
	1c & \textit{masaniŋ}\textit{siŋ} \textit{pisvaŋ}\textit{duan} ‘the Holy Temple’ & \textsc{n} & ~ & ~ & temple & Entity & Exophoric & ~\\
	2a & \textit{i-sia} ‘\textsc{poss-anaph}’ & \textsc{anaph} & 1c & \textit{masaniŋ}\textit{siŋ} \textit{pisvaŋ}\textit{duan} ‘the Holy Temple’ & temple & Entity & Anaphoric & Identity\\
	2a & \textit{manauʔað ʔilav[-ʤia]} ‘beautiful door-\textsc{dist.obl}’ & \textsc{n} & 2a & \textit{i-sia} ‘\textsc{poss-anaph}’ & door & Entity & Anaphoric & Part\\
	2a & \textit{[manauʔað ʔilav]-ʤia} ‘beautiful door-\textsc{dist.obl}’ & \textsc{art} & 2a & \textit{i-sia} ‘\textsc{poss-anaph}’ & location temple & Loc/Time & Anaphoric & Metonym\\
	2b & \textit{taʤini} ‘one.\textsc{hum}’ & \textsc{num} & ~ & ~ & cripple & Entity & Exophoric & ~\\
	2b & \textit{mapiha} ‘\textsc{stat}{}-cripple’ & \textsc{n} & 2b & \textit{taʤini} ‘one.\textsc{hum}’ & cripple & Entity & Anaphoric & Identity\\
	2c & \textit{hanian} ‘day’ & \textsc{time} & ~ & ~ & every day & Loc/Time & Exophoric & ~\\
	2c & \textit{bunun} ‘person’ & \textsc{n} & ~ & ~ & people & Entity & Exophoric & ~\\
	2c & \textit{[ma-pun-]sia} ‘\textsc{dyn-caus.all-anaph}’ & \textsc{v} & 2a & \textit{manauʔað ʔilav[-ʤia]} ‘beautiful door-\textsc{dist.obl}’ & door & Entity & Anaphoric & Metonym\\
	2c & \textit{ʔilav[-ʤia]} ‘door-\textsc{dist.obl}’ & \textsc{n} & 2c & \textit{[ma-pun-]sia} ‘\textsc{dyn-caus.all-anaph}’ & door & Entity & Anaphoric & Metonym\\
	2c & \textit{[ʔilav]-ʤia} ‘door-\textsc{dist.obl}’ & \textsc{art} & 2a & \textit{[manauʔað ʔilav]-ʤia} & location temple & Loc/Time & Anaphoric & Identity\\
	2d & \textit{na-ku-ŋ}\textit{adah} ‘\textsc{irr-assoc.all}{}-lower’ & \textsc{place} & 2c & \textit{ʔilav[-ʤia]} ‘door-\textsc{dist.obl}’ & location temple & Loc/Time & Anaphoric & Metonym\\
	2d & \textit{sia} ‘\textsc{anaph}’ & \textsc{anaph} & 1c & \textit{masaniŋ}\textit{siŋ} \textit{pisvaŋ}\textit{duan} ‘the Holy Temple’ & temple & Text & Anaphoric & Identity\\
	2d & \textit{masaniŋ}\textit{siŋ} \textit{pisvaŋ}\textit{duan} ‘the Holy Temple’ & \textsc{n} & 2a & \textit{i-sia} ‘\textsc{poss-anaph}’ & temple & Entity & Anaphoric & Identity\\
	2d & \textit{bunun} ‘person’ & \textsc{n} & 2c & \textit{bunun} ‘person’ & people & Entity & Anaphoric & Cohyponym\\
	3a & \textit{saia} ‘\textsc{3s.nom}’ & \textsc{pron} & 2b & \textit{mapiha} ‘\textsc{stat}{}-cripple’ & cripple & Entity & Anaphoric & Identity\\
	3b & \textit{Pitilu} ‘Peter’ & \textsc{n} & 1c & \textit{Pitilu} ‘Peter’ & Peter & Entity & Anaphoric & Identity\\
	3b & \textit{Iuhani} ‘John’ & \textsc{n} & 1c & \textit{Iuhani} ‘John’ & John & Entity & Anaphoric & Identity\\
	\lspbottomrule
\end{tabular}}
\caption{\label{tab:debusser:12}Referential cohesion analysis T3} 
\end{sidewaystable}

\begin{sidewaystable} 
		\fittable{\small
\begin{tabular}{ll l lll lll}
	\lsptoprule
	\bfseries (1) & \bfseries Reference (Rc) & \bfseries Rc word  & \bfseries (3) & \bfseries Target (Ta) & \bfseries Referent (4) & \bfseries Functional  & \bfseries Phoric  & \bfseries Rel. \\
                    &                            & \bfseries class (2)\is{word class} & & & & \bfseries role Ta (5) & \bfseries function~(6)  & \bfseries Rc-Ta (7)\\
	\midrule
	3b & \textit{na-ku-ŋ}\textit{adah} ‘\textsc{irr-assoc.all}{}-lower’ & \textsc{v} & 2d & \textit{na-ku-ŋ}\textit{adah} ‘\textsc{irr-assoc.all}{}-lower’ & location temple & Loc/Time & Anaphoric & Identity\\
	3b & \textit{sia} ‘\textsc{anaph}’ & \textsc{anaph} & 2d & \textit{masaniŋ}\textit{siŋ} \textit{pisvaŋ}\textit{duan} ‘the Holy Temple’ & temple & Text & Anaphoric & Identity\\
	3b & \textit{masaniŋ}\textit{siŋ} \textit{pisvaŋ}\textit{duan} ‘the Holy Temple’ & \textsc{n} & 2d & \textit{masaniŋ}\textit{siŋ} \textit{pisvaŋ}\textit{duan} ‘the Holy Temple’ & temple & Entity & Anaphoric & Identity\\
	3c & \textit{naiʤia} ‘\textsc{3p.obj}’ & \textsc{pron} & 3b & \textit{Pitilu} ‘Peter’ + \textit{Iuhani} ‘John’ & Peter and John & Entity & Anaphoric & Whole\\
	4a & \textit{naia} ‘\textsc{3p.nom}’ & \textsc{pron} & 3c & \textit{naiʤia} ‘\textsc{3p.obj}’ & Peter and John & Entity & Anaphoric & Identity\\
	4a & \textit{saiʤia} ‘\textsc{3s.obl}’ & \textsc{pron} & 3a & \textit{saia} ‘\textsc{3s.nom}’ & cripple & Entity & Anaphoric & Identity\\
	4b & \textit{Pitilu} ‘Peter’ & \textsc{n} & 3b & \textit{Pitilu} ‘Peter’ & Peter & Entity & Anaphoric & Identity\\
	4c & \textit{kasu} ‘\textsc{2s.nom}’ & \textsc{pron} & 4a & \textit{saiʤia} ‘\textsc{3s.obl}’ & cripple & Entity & Anaphoric & Identity\\
	4c & \textit{[ma-]ðami} ‘\textsc{dyn-1e.obl}’ & \textsc{pron} & 4a & \textit{naia} ‘\textsc{3p.nom}’ & Peter and John & Entity & Anaphoric & Identity\\
	5a & \textit{saia} ‘\textsc{3s.nom}’ & \textsc{pron} & 4c & \textit{kasu} ‘\textsc{2s.nom}’ & cripple & Entity & Anaphoric & Identity\\
	5a & \textit{naiʤia} ‘\textsc{3p.obj}’ & \textsc{pron} & 4c & \textit{[ma-]ðami} ‘\textsc{dyn-1e.obl}’ & Peter and John & Entity & Anaphoric & Identity\\
	5b & \textit{maðmað} ‘which.things’ & \textsc{q} & ~ & ~ & possessions & Entity & Exophoric & ~\\
	6a & \textit{Pitilu} ‘Peter’ & \textsc{n} & 4b & \textit{Pitilu} ‘Peter’ & Peter & Entity & Anaphoric & Identity\\
	6a & \textit{saiʤia} ‘\textsc{3s.obl}’ & \textsc{pron} & 5a & \textit{saia} ‘\textsc{3s.nom}’ & cripple & Entity & Anaphoric & Identity\\
	6b & \textit{saikin} ‘\textsc{1s.nom}’ & \textsc{pron} & 6a & \textit{Pitilu} ‘Peter’ & Peter & Entity & Anaphoric & Identity\\
	6b & \textit{kim} ‘gold’ & \textsc{n} & 5b & \textit{maðmað} ‘which.things’ & possessions & Entity & Anaphoric & Part\\
	6b & \textit{sui} ‘money’ & \textsc{n} & 5b & \textit{maðmað} ‘which.things’ & possessions & Entity & Anaphoric & Part\\
	6c & \textit{[na-saiv-an]-ku} ‘-\textsc{1s.n}’ & \textsc{pron} & 6b & \textit{saikin} ‘\textsc{1s.nom}’ & Peter & Entity & Anaphoric & Identity\\
	6c & \textit{kasu} ‘\textsc{2s.nom}’ & \textsc{pron} & 6a & \textit{saiʤia} ‘\textsc{3s.obl}’ & cripple & Entity & Anaphoric & Identity\\
	6c & \textit{i-nak} ‘\textsc{poss-1s.n}’ & \textsc{pron} & 6c & \textit{[na-saiv-an]-ku} ‘-\textsc{1s.n}’ & Peter & Entity & Anaphoric & Identity\\
	6c & \textit{iskakaupa} ‘everything’ & \textsc{n} & 6b & \textit{sui} ‘money’ & possessions & Entity & Anaphoric & Whole\\
	6d & \textit{[ma-paka-]sia} ‘\textsc{dyn-recip-anaph}’ & \textsc{v} & 6b-6c & [entire sentence] & do & Event & Anaphoric & ~\\
	6d & \textit{saikin} ‘\textsc{1s.nom}’ & \textsc{pron} & 6c & \textit{i-nak} ‘\textsc{poss-1s.n}’ & Peter & Entity & Anaphoric & Identity\\
	6d & \textit{[taki-]sia} ‘\textsc{origin-anaph}’ & \textsc{v} & ~ & ~ & Nazareth & Loc/Time & Cataphoric & Identity\\
	6d & \textit{Naðale} ‘Nazareth’ & \textsc{n} & 6d & \textit{Naðale} ‘Nazareth’ & Nazareth & Loc/Time & Exophoric & Identity\\
	6d & \textit{Iesu Kilistu} ‘Jesus Christ’ & \textsc{n} &  & ~ & Jesus & Entity & Exophoric & ~\\
	6d & \textit{ŋan} ‘name’ & \textsc{n} & ~ & ~ & name & Entity & Exophoric & ~\\
	6e & \textit{masu} ‘\textsc{2s.obl}’ & \textsc{pron} & 6c & \textit{kasu} ‘\textsc{2s.nom}’ & cripple & Entity & Anaphoric & Identity\\
	7a & \textit{Pitilu} ‘Peter’ & \textsc{n} & 6d & \textit{saikin} ‘\textsc{1s.nom}’ & Peter & Entity & Anaphoric & Identity\\
	7a & \textit{isaiʤia} ‘\textsc{3s.poss}’ & \textsc{pron} & 6e & \textit{masu} ‘\textsc{2s.obl}’ & cripple & Entity & Anaphoric & Identity\\
	7a & \textit{ʔima} ‘hand’ & \textsc{n} & 7a & \textit{isaiʤia} ‘\textsc{3s.poss}’ & hand of cripple & Entity & Anaphoric & Part\\
	7b & \textit{saiʤia} ‘\textsc{3s.obl}’ & \textsc{pron} & 7a & \textit{isaiʤia} ‘\textsc{3s.poss}’ & cripple & Entity & Anaphoric & Identity\\
	\lspbottomrule
\end{tabular}} 
\end{sidewaystable}
	
	
\begin{sidewaystable} 
		\fittable{\small 
\begin{tabular}{ll l lll lll}
	\lsptoprule
	\bfseries (1) & \bfseries Reference (Rc) & \bfseries Rc word  & \bfseries (3) & \bfseries Target (Ta) & \bfseries Referent (4) & \bfseries Functional  & \bfseries Phoric  & \bfseries Rel. \\
                    &                            & \bfseries class (2)\is{word class} & & & & \bfseries role Ta (5) & \bfseries function~(6)  & \bfseries Rc-Ta (7)\\
	\midrule
	8a & \textit{bunun[-ʤia]} ‘people-\textsc{dist.obl}’ & \textsc{n} & 7b & \textit{saiʤia} ‘\textsc{3s.obl}’ & cripple & Entity & Anaphoric & Identity\\
	8a & \textit{[bunun]-ʤia} ‘people-\textsc{dist.obl}’ & \textsc{art} & 3b & \textit{na-ku-ŋ}\textit{adah} ‘\textsc{irr-assoc.all}{}-lower’ & location temple & Loc/Time & Anaphoric & Metonym\\
	8a & \textit{bantas} ‘leg’ & \textsc{n} & 8a & \textit{bunun[-ʤia]} ‘people-\textsc{dist.obl}’ & limb & Entity & Anaphoric & Part\\
	8a & \textit{buhtuŋ} ‘joint’ & \textsc{n} & 8a & \textit{bantas} ‘leg’ & limb & Entity & Anaphoric & Copart\\
	9a & \textit{saia} ‘\textsc{3s.nom}’ & \textsc{pron} & 8a & \textit{bunun[-ʤia]} ‘people-\textsc{dist.obl}’ & cripple & Entity & Anaphoric & Identity\\
	9a & \textit{naiʤia} ‘\textsc{3p.obj}’ & \textsc{pron} & 5a & \textit{naiʤia} ‘\textsc{3p.obj}’ & Peter and John & Entity & Anaphoric & Identity\\
	9a & \textit{[ku-]ŋ}\textit{adah} ‘\textsc{assoc.all}{}-lower’ & \textsc{place} & 8a & \textit{bunun[-ʤia]} ‘people-\textsc{dist.obl}’ & location temple & Loc/Time & Anaphoric & Metonym\\
	9a & \textit{sia} ‘\textsc{anaph}’ & \textsc{anaph} & 3b & \textit{masaniŋ}\textit{siŋ} \textit{pisvaŋ}\textit{duan} ‘the Holy Temple’ & temple & Text & Anaphoric & Identity\\
	9a & \textit{masaniŋ}\textit{siŋ} \textit{pisvaŋ}\textit{duan} ‘the Holy Temple’ & \textsc{n} & 3b & \textit{masaniŋ}\textit{siŋ} \textit{pisvaŋ}\textit{duan} ‘the Holy Temple’ & temple & Entity & Anaphoric & Identity\\
	9b & \textit{Sasbinað Dihanin} ‘God’ & \textsc{n} & ~ & ~ & God & Entity & Exophoric & ~\\
	10a & \textit{bunun} ‘person’ & \textsc{n} & 2d & \textit{bunun} ‘person’ & people & Entity & Anaphoric & Cohyponym\\
	10a & \textit{saiʤia[-a]} ‘\textsc{3s.obl-lnk}’ & \textsc{pron} & 9a & \textit{saia} ‘\textsc{3s.nom}’ & cripple & Entity & Anaphoric & Identity\\
	10b & \textit{Sasbinað} \textit{Dihanin} ‘God’ & \textsc{n} & 9b & \textit{Sasbinað Dihanin} ‘God’ & God & Entity & Anaphoric & Identity\\
	10d & \textit{naia} ‘\textsc{3p.nom}’ & \textsc{pron} & 10a & \textit{bunun} ‘person’ & people & Entity & Anaphoric & Identity\\
	10e & \textit{saia} ‘\textsc{3s.nom}’ & \textsc{pron} & 10a & \textit{saiʤia[-a]} ‘\textsc{3s.obl-lnk}’ & cripple & Entity & Anaphoric & Identity\\
	10e & \textit{[taki-]sia} ‘\textsc{origin-anaph}’ & \textsc{v} & 10e & \textit{masaniŋ}\textit{siŋ} \textit{pisvaŋ}\textit{duan} ‘the Holy Temple’ & temple & Entity & Cataphoric & Identity\\
	10e & \textit{masaniŋ}\textit{siŋ} \textit{pisvaŋ}\textit{duan} ‘the Holy Temple’ & \textsc{n} & 9a & \textit{masaniŋ}\textit{siŋ} \textit{pisvaŋ}\textit{duan} ‘the Holy Temple’ & temple & Entity & Anaphoric & Identity\\
	10e & \textit{manauʔað ʔilav} ‘beautiful door’ & \textsc{n} & 2d & \textit{na-ku-ŋ}\textit{adah} ‘\textsc{irr-assoc.all}{}-lower’ & door & Entity & Anaphoric & Identity\\
	10e & \textit{bunun} ‘person’ & \textsc{n} & 10e & \textit{saia} \textsc{‘3s.nom}’ & cripple & Entity & Anaphoric & Identity\\
	\lspbottomrule
\end{tabular}} 
\end{sidewaystable}
\clearpage
\sloppy
\printbibliography[heading=subbibliography,notkeyword=this]

\end{document}
