\addchap{Preface and Acknowledgements}
% \begin{refsection}        
\noindent
Information structure is a relatively new field to linguistics and has only recently been studied for smaller and less described languages. This book brings together contributions on information structure in Austronesian languages, covering all subgroups of the large Austronesian family (including Formosan, Central Malayo-Polynesian, South Halmahera-West New Guinea, and Oceanic). Its major focus, though, lies on Western Malayo-Polynesian languages. A number of chapters investigate two of the largest languages of the region, i.e. Tagalog and (different varieties of) Malay, others study information structural phenomena in smaller, underdescribed languages. The book emerged from a series of workshops on information structure in Austronesian languages between 2013 and 2016 that were organized by Atsuko Utsumi (Meisei University) and Asako Shiohara (Tokyo University of Foreign Studies) and which took place at the Research Institute for Languages and Cultures of Asia and Africa (ILCAA), Tokyo University of Foreign Studies. The book is divided into three major parts, which roughly reflect the different topics of these workshops. Part one subsumes chapters on the topic of NP marking and reference tracking devices. In part two, contributions investigate how syntactic constructions, such as, for example, cleft constructions or passives, may reflect different information structural categories. Finally, the third part studies the interaction of information structure and prosody.

We gratefully acknowledge the generous funding of the Linguistic Dynamics Science Project 2 (Research Institute for Languages and Cultures of Asia and Africa, Tokyo University of Foreign Studies (2013-2015AY), principal investigator: Toshihide Nakayama) which made it possible to hold the workshops in the first place and to invite its participants to come to Tokyo. During the process of editing this book, SR has been supported through the Volkswagen Foundation, the German Research Foundation (DFG) within the SFB 1252 “Prominence in Language” at the University of Cologne, and by the Australian Research Council (ARC) within the “Centre of Excellence for the Dynamics of Language” at the Australian National University. AS received funding through the Linguistic Dynamics Science 3 program (LingDy3) and the JSPS project “A collaborative network for usage-based research on less-studied languages.”\largerpage[-2]

We want to thank all of the contributors for the good collaboration. Special thanks to those who helped us with the internal reviewing process. For external reviewing we would like to express our gratitude to Sander Adelaar, Laura Arnold, Abigail Cohn, Carmen Dawuda, Michael Ewing, Katja Jasinskaja, Kurt Malcher, Gabriele Schwiertz, Aung Si, Volker Unterladstetter, and Akira Utsugi.
Maria Bardají Farré, Lena Rennert, and Katherine Walker have helped with some of the editorial work and proof reading, and Wataru Okubo has done a wonderful job in converting the manuscripts into LaTeX and preparing them for publication. A big thank you also to Sachiko Yoshida for doing all the admin work for the workshops and other volume-related issues.  

\begin{flushright}
Sonja Riesberg, Köln\\
Asako Shiohara, Tokyo\\
Atsuko Utsumi, Tokyo
\end{flushright}

% \printbibliography[heading=subbibliography]
% \end{refsection}

