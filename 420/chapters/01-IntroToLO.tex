\hypertarget{intro}{%
\chapter{Introduction to linguistics problems}\label{chap-intro}}
Linguistics problems are puzzles, logic games usually based on real languages, which allow the discovery of certain linguistic phenomena through logical reasoning. The problems usually consist of a corpus (a dataset) in an unfamiliar language accompanied by their English translations (either ordered or in random order). Solving the problem requires logical thinking and attention to detail in order to be able to decode certain aspects of the language, such as the meaning of certain words or some grammar rules.

It is important to note that all linguistics problems are internally consistent (self-consistent) and require no additional knowledge (self-sufficient), i.e., once certain rules are discovered, they will apply to all examples in the problem (of course, some rules might have exceptions) and all the information needed to solve the problem is found in the problem.

\section{Structure of linguistics problems}
All linguistics problems have the same structure, consisting of four parts (excluding the title and the author):
\begin{enumerate} 
\item Introduction

In the introduction, we learn about the language featured in the problem. For most problems, the introduction is simply a sentence like ``Given below are some [words/structures/constructions/sentences, etc.] in [Language] and their English translations (in random order)."

Some problems might have a more complex introduction which also includes information about the culture of the people speaking that language or information about certain characteristics of the language. Generally speaking, if the introduction is complex and contains additional information, this information is likely to be relevant to solving the problem.
\item Dataset

This part contains the examples based on which we should solve the tasks. This part is also known as the \textit{corpus}. If the corpus and the translations are given in order (it is known which translation corresponds to which structure), the problem is called a \textit{Rosetta stone problem}, while if the translations are given in random order, we are talking about a \textit{chaos-and-order problem}.
\item Tasks

The corpus is followed by the tasks. For chaos-and-order problems, the first task will always be “Determine the correct correspondences”. Therefore, an easy way to figure out whether the problem is Rosetta stone or chaos-and-order (besides reading the introduction) is by checking the first task.

The subsequent tasks will be “Translate into English”\ and “Translate into […]”. As a rule of thumb, the task asking to translate into English will precede the other one because (1) in order to translate into English, it is not always necessary to understand all the grammar rules, and (2) we might be able to use these additional examples in order to gather more information.

Some problems might also have special tasks, which usually offer important hints about the phenomena featured in the problem.
\item Notes

At the end of each problem, there will be some notes which provide three types of information. Firstly, there will be some data about the language featured in the problem, such as where it is spoken, how many people it is spoken by, what language family it belongs to, etc. In general, this information is not relevant to solving the problem, although, for experienced solvers, the family to which the language belongs might offer additional helpful information. In this book, this information has been removed from the notes and all the information regarding the languages can be found in Appendices \ref{appendix:1} and \ref{appendix:2}.

Relevant phonetic information follows, which offers details regarding how some letters, characters, symbols are pronounced. Moreover, it might also include details regarding the existence of diphthongs, or details on additional writing notations used in the problem. Depending on the problem, this kind of information may or may not be useful.

Finally, there might be some information about specific words, usually referring to types or species of animals or plants, traditional objects or garments, etc. Again, depending on the type of problem, this information may or may not be useful in solving it.
\end{enumerate}

\section{Classification of linguistics problems}
Generally, linguistics problems are classified based on the main phenomenon that is featured (which is closely related to a specific field of linguistics). In this book, the problems are divided into seven categories, as follows:

\begin{enumerate}
    \item Writing systems (\chapref{chap-writing})

    These problems feature words in an unfamiliar writing system together with their transliterations in Latin script. These problems can be either Rosetta stone or chaos-and-order.
    \item Phonetics and Phonology (Chapters~\ref{chap-phonetics} and~\ref{chap-phonology})

These problems are based on the sound changes that occur in different environments or contexts (e.g., two or more forms of a word -- such as singular--plural, different noun cases, declensions, etc. --, the way certain words change in different dialects, how words are transcribed phonetically, how words are stressed, etc.). Generally, these problems are Rosetta stone.
\item Morphology (Chapters~\ref{chap-noun} and \ref{chap-verb})

These problems are subdivided into two categories: morphology of the noun and its relation to other elements in the noun phrase (\chapref{chap-noun}) and morphology of the verb and its relation to other elements in the verb phrase (\chapref{chap-verb}). These problems can be either Rosetta stone or chaos-and-order.
\item Syntax (\chapref{chap-syntax})

Syntax problems contain sentences (or phrases), combining, to some extent, the morphology of the noun with the morphology of the verb. In most cases, these problems are Rosetta stone, mainly for simplicity. If the sentences were in random order, once the correspondences are made, all sentences need to be copied again – with their corresponding translations~– in order to better see all the data and extract all the rules and phenomena that occur; the copying of the sentences would require a lot of time.
\item Semantics (\chapref{chap-semantics})

Semantic problems are not based on grammar rules, but rather on associations of words which share different properties related to their meaning. These problems are always chaos-and-order problems.
\item Number systems (\chapref{chap-numbers})

This kind of problem contains numbers written out in a certain language and their numeric representation. The ``translations'' might or might not be ordered. Sometimes, the corpus might consist solely of some mathematical equalities, in which the numbers are written out in a certain language.
\item Other types of problems

This category is rather large and includes any problem that does not fit in the aforementioned categories. Nevertheless, even in this category, we can notice some types of problems that appear frequently in the linguistics olympiads. As a result, we can differentiate:
\begin{enumerate}[label = 7.\arabic*.]
    \item Metrics and prosody (\chapref{chap-phonetics}) – in which the corpus consists of lines from different poems and the purpose is discovering the general structure of the verse. 
	\item Time problems (\chapref{chap-numbers}) – these problems highlight how the calendar dates or the time are told in different languages. 
	\item Kinship problems (\chapref{chap-others}).
	\item Orientation system problems (\chapref{chap-others}) – which show how directionality is represented in different languages. 
\end{enumerate}
\end{enumerate}

Of course, there can also be \textit{mixed} problems, which combine two (or more) of the above categories.

Statistically speaking, based on 437 problems from different national and international linguistics olympiads, we notice that the most common type of problem is the syntax one (accounting for approximately 19.5\% of the problems). Indeed, syntax problems are usually pervasive in all olympiads and each olympiad will have at least one syntax problem. Morphology problems (both nominal and verbal) are almost as common as syntax ones, accounting for 19\% of the total. The other categories are phonetics and phonology (13.5\%), writing systems (12\%), number systems (9\%) and semantics (7\%).

\section{Understanding the problem}
When solving a linguistics problem, it is important to understand what exactly is expected from us. In most cases, the way the tasks are phrased (especially the special tasks, as mentioned above), but even the way the corpus is chosen, can offer important hints. For example, let us take a look at the way the following task is phrased (the whole problem is presented in \chapref{chap-noun}, Problem 5.13):

\begin{assgts}[label=Example task 1.1]
\item Fill in the blanks:
\begin{center}
\begin{tabular}{rll}
\setcounter{exx}{15}
\setcounter{pbblank}{6}
\sentlineonerow{baqra \pbblank}{blue cow}
\sentlineonerow{fjuri \pbblank}{red flowers}
\sentlineonerow{kelb \pbblank}{brown dog}
\sentlineonerow{kotba \pbblank}{yellow books}
\sentlineonerow{siġra \pbblank}{green tree}
\sentlineonerow{mwejjed \pbblank}{purple chairs}
\sentlineonerow{tuffieħa \pbblank}{yellow apple}
\end{tabular}
\end{center}
\end{assgts}

The thing that strikes us the most is that the first word is always given, and we only need to be concerned about the second. Normally, in a \textit{classical} problem, we would simply be asked to translate the structures \texttr{blue cow}, \texttr{red flowers}, etc. Since in this case the first word (which, from the full dataset -- not shown here --, we can see evidently represents the noun) is already given, we infer that, most likely, in this language (or at least based on the information given) the noun plural formation is irregular (or simply too complex) and does not follow specific rules, thus not being able to infer the singular form from the plural form or vice versa. Therefore, since the nouns are already mentioned, we know that the core phenomenon of the problem focuses on the adjective, rather than on the noun.

Let us consider the following (fictitious) example which would correspond to a writing system problem:
\begin{assgts}[resume, label=Example task 1.2]
    \item Write in the [...] script: \cmubdata{Mars}, \cmubdata{venus}, \cmubdata{JUPITER}, \cmubdata{NePtUne}.
\end{assgts}

 In this case, we deem unusual the writing of these words, some of them being just lowercase, others just uppercase, and others being written with a combination of the two. There is no plausible reason to do so unless the writing system differentiates between lower- and uppercase letters. Therefore, in this case, the choice of the tasks (and their form) offers us a valuable clue: most likely there is a difference between lower- and uppercase letters.

Another task might be:\footnote{From a problem by Bozhidar Bozhanov (UKLO 2010).}
\begin{assgts}[resume, label=Example task 1.3]
    \item Knowing that in Turkish \cmubdata{dil} = \texttr{language}, translate \texttr{linguist}, \texttr{mute}.
\end{assgts}

In this case, we are given a new word (\texttr{language}) and we are asked to translate two words which belong to the same semantic field. Therefore, in order to translate \texttr{linguist}, there must be a rule which allows the formation of an agent noun (the person which...) and similarly, in the case of \texttr{mute}, we need to find a “negative”-forming particle (which marks the impossibility\slash lack of\slash incapacity etc.), both of which must be inferred from the data given.

As a result, it is important to read all the data and tasks carefully and ask ourselves whenever we see something slightly peculiar: \textit{Why is it like this?}

\section{Solution writing}\largerpage[2]
Solution writing is a core part of problem-solving. It is important to write all the rules clearly and consistently and to cover all the phenomena featured, but, at the same time, to write them succinctly enough to not waste precious time during an official competition.

Officially, in the guidelines of the IOL (and in the case of most linguistics competitions) it is stated that: “Unless stated differently, you should describe any patterns or rules that you identified in the data. Otherwise, your solution will not be awarded full marks.”

The most important thing we need to understand is that we need to write \textit{the rules that we identified} and not \textit{how we found them}. Therefore, in the linguistics competitions, we are not asked to provide our reasoning for inferring the rules, but rather only to write the actual rules.

When writing the solution, \textit{we should}:
\begin{itemize}
    \item use tables, graphs, diagrams or any other kind of concise representation;
    \item use common abbreviations and symbols (we may also use less common abbreviations as long as we make a legend describing what they stand for);
    \item explain all the rules and phenomena that occur.
\end{itemize}

 Briefly, through orderly and concise writing of all the rules, we ought to try and \textit{tell the story of the language} we discovered.

At the same time, when writing the solution, \textit{we should not}:
\begin{itemize}
    \item explain how we inferred or deduced the rules and patterns;
    \item write dictionaries and explain the meaning of every single word (we will talk more about this in the next section);
    \item use connectors and excessive words, such as: ``I think", ``we deduce", ``it is obvious that", ``since", ``therefore", ``because", ``it is possible that", etc.
\end{itemize}

Therefore, when providing a solution, we should not write something like “Since examples 1, 3, and 5 all contain the word \cmubdata{mi} and all the English translations of these examples end in a question mark (thus being interrogative constructions), we most likely can infer that this word marks the fact that the structure is an interrogative one”\ since the same explanation can (and should) be briefly written as: “\cmubdata{mi} = question”.

\section{Dictionary vs. rules}
We mentioned above that a solution should not include the \textit{dictionary}. By dictionary we mean the base words (or stems) such as nouns, pronouns, adverbs, etc., \textit{whose form does not change}.

On the other hand, the \textit{rules} (which we need to write) explain the alternations that occur in the language. Therefore, they explain the word order, the way words change depending on their number, gender, tense, etc. We also include here all the words that do not have a direct English translation (usually they represent words that have a function rather than a meaning). For example, in Chinese, the character {\chinesetext{吗}} placed at the end of the sentence signals that it is a direct question (requiring a yes/no answer). Therefore, since this character has a function (marks the interrogation) and not a meaning (it does not mean anything and it would not be found in a dictionary), it must be included in the solution writing.

Let us consider the following dataset from the Turkish language, and imagine we are asked about how possession is marked in Turkish:
\begin{center}
\begin{tabular}{ll}
    \pbsv{babam}{my father} \\
    \pbsv{kedin}{your\sg~cat} \\
    \pbsv{kedimiz}{our cat} \\
    \pbsv{baban}{your\sg~father} \\
    \pbsv{kedi}{his cat} \\
\end{tabular}
\end{center}

 In this case, we can easily observe that the possessive is marked with a suffix (attached at the end of the word) as follows: \cmubdata{-m} for \texttr{my}, \cmubdata{-n} for \texttr{your\sg}, \cmubdata{-miz} for \texttr{our}, while for \texttr{his} nothing is added – in fact, it is important to specify that ``zero"\ is added or, in other words, that a null morpheme is used to mark the equivalent of \texttr{his} in English.

For this problem, the dictionary is: \cmubdata{baba} = \texttr{father} and \cmubdata{kedi} = \texttr{cat} (these words are invariable). Therefore, a correct and complete way of writing the solution is:

\ea\label{ex:1:turkish} possession suffixes: \cmubdata{-m} for \texttr{my}, \cmubdata{-n} for \texttr{your\sg}, \cmubdata{-miz} for \texttr{our}, $\varnothing$ for \texttr{his} \z

\note{The symbol $\varnothing$ marks the fact that nothing is added (it represents the null morpheme). This symbol does not need an explanation/legend when used.}

 Another, briefer, way to write the solution is:

 \begin{exe}
   \exp{ex:1:turkish}
 {possession suffixes: \cmubdata{-m} = 1\textsc{sg}, \cmubdata{-n} = 2\textsc{sg}, \cmubdata{-miz} = 1\textsc{pl}, $\varnothing$ = 3\textsc{sg}}
 \end{exe}

The simplest way to write these suffixes is by creating a table which includes the persons (1, 2, 3) in the rows and the numbers in the columns (singular, plural). Therefore, we can also write:

\begin{exe}
  \expp{ex:1:turkish}

possession suffixes: \quad
\begin{tabular}[t]{ccc}
\lsptoprule
    & \textsc{sg}            & \textsc{pl}  \\\midrule
  1 & \cmubdata{-m} & \cmubdata{-miz} \\
  2 & \cmubdata{-n} &  \\
  3 & $\varnothing$ &  \\
\lspbottomrule
\end{tabular}
\end{exe}

In this case, we can see the importance of using the null morpheme ($\varnothing$). Thanks to it, we can deduce (based on the table above) that there is a difference between the possessive suffix for 2\textsc{pl} (the cell is blank – therefore we cannot deduce it based on the data given) and for 3\textsc{sg} (the cell is not empty, it contains the symbol $\varnothing$, thus proving that we discovered the way it is marked).

Let us now consider the following three sentences in Spanish, and imagine we are asked to describe the word order:
\begin{center}
    \begin{tabular}{lcl}
        \cmubdata{Tu marido corre.} & = & \texttr{Your\sg~husband runs.}  \\
        \cmubdata{Él ve a tu marido.} & = & \texttr{He sees your\sg~husband.}  \\
        \cmubdata{Mi novio ve a él.} & = & \texttr{My boyfriend sees him.}  \\
    \end{tabular}
\end{center}

 The rules we need to write concern the order of subject, verb and object and the order of the possessor and the possessed. Therefore, the solution is:

 \ea\label{ex:1:spanish}
{Word order: S V (\cmubdata{a} O); Possessor – Possessed}
\z

 The rules above contain a lot of relevant information, written very briefly:

\begin{itemize}
    \item The word order is S(ubject), followed by V(erb), followed by the particle \cmubdata{a} and finally followed by the O(bject);
    \item The particle \cmubdata{a} only appears together with the object; if the sentence has no object, the particle \cmubdata{a} is not used, a fact marked by the use of brackets around the structure;
    \item In a possessive construction, the possessor (owner) is placed before the possessed object. 
\end{itemize}

We need not mention anything about the verbs since all of them are in the third person singular (3\textsc{sg}) present tense: therefore, we do not know how (or if) they change form in any way.

Let us now consider three more sentences (as a supplement to those above):

\begin{center}
    \begin{tabular}{lcl}
        \cmubdata{Él ve tu casa.} & = & \texttr{He sees your\sg~house.}  \\
        \cmubdata{Ella ve a su padre.} & = & \texttr{She sees her father.}  \\
        \cmubdata{Yo veo tu libro.} & = & \texttr{I see your\sg~book.}  \\
    \end{tabular}
\end{center}

 Based on these extra sentences, it is important to check, first of all, whether the rules we wrote before still hold for these examples as well. Therefore, we now see that \cmubdata{a} does not appear every time before the object, but it is only used when the object is human. Therefore, the rules become:

 \begin{exe}
 \exp{ex:1:spanish}
 {Word order: S V O; Possessor – Possessed}\\
 {If O = person, add \cmubdata{a} before it.}
\end{exe}

 Moreover, we see this time that the verb changes, having the pair \cmubdata{ve} (\texttr{he/she sees}) and \cmubdata{veo} (\texttr{I see}). Therefore, we also need to pay attention to verbal morphology. If we do so, we can deduce the conjugation rules for the verb: \cmubdata{-o} = 1\textsc{sg}, \mbox{$\varnothing$ = 3\textsc{sg}}.

Thus, the final rules are:

\begin{exe}
  \expp{ex:1:spanish}

Word order: S V O; Possessor – Possessed

If O = person, add \cmubdata{a} before it.

Verb: 1\textsc{sg} = \cmubdata{-o}, 3\textsc{sg} = $\varnothing$
\end{exe}

 Let us now consider the following possible task:

\begin{assgts}[label=Example task 1.4]
    \item There is an error in the following data. What is it?
\begin{center}
    \begin{tabular}{lcl}
        \cmubdata{Mi padre correo.} & = & \texttr{My father runs.}  \\
    \end{tabular}
\end{center}
\end{assgts}

Since we are told that there is an error in the sentence, we need to check the rules we have in order to see if any of them could justify this task. We remember that the \cmubdata{-o} ending of verbs is for 1\textsc{sg} subjects, but here the subject is 3\textsc{sg}, meaning there should be zero inflection on the verb, so the Spanish sentence should read \cmubdata{Mi padre corre}.

In the following chapters, each problem will be accompanied by a solution so that the reader can get accustomed to different ways of writing the solution.
