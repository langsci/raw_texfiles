\begin{refsection}
\hypertarget{syntax}{%
\chapter{Syntax}\label{chap-syntax}}

\section{Introduction}

 Syntax is concerned with the study of sentences and phrases. We remember that sentences can typically be considered to be formed from a noun phrase and a verb phrase (sentence $=$ NP $+$ VP)\footnote{The abbreviations used are: NP = noun phrase, VP = verb phrase.} (though not necessarily in that order). Therefore, syntax problems are nothing more than a noun morphology problem combined with a verb morphology problem.

Nevertheless, since NP and VP are related in the same structure, there can be some additional interactions between the two. For example, the noun phrase can  contain morphemes indicating the roles of nouns in the sentence (subject, object, agent, patient, etc.). Since the arguments can be expressed though nouns (not only pronouns, as was the case in the previous chapter), there can also be a wider variety of distinctions in the verb phrase. For verb phrase problems, the principal distinctions were person (1, 2, 3), number (singular, dual, plural) and, sometimes, gender (masculine, feminine), but for syntax problems we can also add distinctions such as $\pm$human (difference between nouns which refer to humans versus the rest).

\section{Word order}

 Word order is one of the most important phenomena in syntax problems. While in the NP and VP problems the number of words is relatively small, in sentences and phrases the number of words can increase considerably; therefore the order in which these words are placed plays a very important role.

Generally, when talking about the word order of a language, we refer to the order of the subject, verb, and object; therefore, we can have six possible patterns: SOV, SVO, VSO, VOS, OSV, OVS. Cross-linguistically, the SOV and SVO patterns are the most common; they account for patterns in around 90\% of the world's languages. The next one is VSO, which occurs in approx. 8\% of languages (such as Classical Arabic, Tagalog, or Celtic languages like Irish, Welsh, Breton, and Manx). The least common patterns are VOS, OVS and OSV, the latter being encountered in fewer than 0.5\% of languages.\footnote{Based on the data from \url{https://wals.info/chapter/81}.} If we look closely, we notice that the most common patterns (SOV, SVO, VSO) are those in which the subject is placed before the object, while the languages in which the subject is placed after the verb account for under 5\% of the languages. Thus, it is statistically probable that in a problem the subject is placed before the object.

Moreover, it is important to mention the dichotomy between languages with fixed or rigid word order and those with a flexible word order.

\textsc{Fixed word order} can be directly assigned to one of the six patterns above, i.e., when a language has fixed word order, (almost) every sentence follows the same word order. All the other word orders are either ungrammatical or rarely used. Languages with fixed word order often have a simpler (inflectional) morphology in subjects and objects since the role of the arguments is well determined by their position in the sentence. Thus, if we consider the English examples \cmubdata{The cat saw the boy} and \cmubdata{The boy saw the cat}, we notice that both sentences have the same structure and the only difference is the word order. Nevertheless, there is no ambiguity between subject and object (even though they both appear morphologically identical -- neither the subject nor the object is marked), since the fixed word order dictates that the subject is first (before the verb) and the object is placed after the verb.

\textsc{Flexible (variable) word order}, on the other hand, cannot be directly assigned to one of the six patterns above, i.e., when a language has flexible word order, not all sentences have the same word order. All six patterns (or most of them) are grammatical and commonly used. Nevertheless, there can be a \OlympiadNewTerm{dominant word order}, i.e., one word order that is favoured. Unlike the languages with fixed word order, those with variable word order tend to have a much richer inflectional system. This is due to the fact that, since word order is flexible, each argument needs to be morphologically marked in order to avoid any ambiguities. Let us consider the following three examples from Romanian:

\begin{center}
    \cmubdata{Uriașul îl dă \underline{{pe}} copil tatăl\textbf{ui}.}

    \cmubdata{Tatăl\textbf{ui} îl dă uriașul \underline{{pe}} copil.}

    \cmubdata{\underline{{Pe}} copil îl dă uriașul tatăl\textbf{ui}.}
\end{center}

All of these sentences have the same meaning (\texttr{The giant gives the child to the father.}), but the order of the arguments (subject, direct object, indirect object) differs in each case. Nevertheless, the meaning is clear and the role of each noun is well defined due to the morphological marking: the subject (\cmubdata{uriașul}) is unmarked, the direct object (\cmubdata{copil}) is preceded by \cmubdata{{pe}}, while the indirect object (\cmubdata{tatălui}) ends in the suffix \cmubdata{-ui}. In this instance, the suffix \cmubdata{-ui} is an example of the dative case, marking the indirect object.

Due to this morphological marking, the flexible word order does not create ambiguities in sentences such as:

\begin{center}
    \wordtrans{Tatăl îl dă pe copil uriașului.}{The father gives the child to the giant.}

    \wordtrans{Tatălui îl dă copilul pe uriaș.}{The child gives the giant to the father.}

    \wordtrans{Pe tată îl dă copilului uriașul.}{The giant gives the father to the child.}
\end{center}

Typologically speaking, the word order in a sentence is classified only based on subject, object, and verb. However, in writing the rules for linguistics problems, it is important to include all the components of the sentence. Thus, for the sentence \cmubdata{The child writes a letter with the pen}, it is not enough to write \fbox{S V O}, but rather we should write \fbox{S V O Instr.} (subject, verb, object, instrumental). Moreover, when describing the word order, we usually talk about the main constituents and not the internal structure of those constituents. This means we describe the relative positions of the subject, object, verb, instrumental, location, time, etc.
As for the internal structure of the constituents, this covers the order of the noun and its modifiers (adjectives, possessives, etc.), which we can write as {\smash{\fbox{Adj. -- Noun}}}, or, more generally, \fbox{Modifier -- Noun}. The reason why it is preferable to separate the two is that modifiers can occur together with different constituents (subject, object, instrumental, etc.), as in the sentence \cmubdata{The happy child writes a new letter with the black pen}

Thus, if we wanted to combine the two structures above (S V O Instr. and Adj. -- Noun), we would have to write:

\begin{center}\relax
[Adj -- S] V [Adj -- O] [Adj -- Instr]
\end{center}

\noindent So we would have to show that the adjective can be placed before the noun in each constituent (subject, object, instrumental). Moreover, the adjective is part of the noun phrase, so its position is strictly relative to the head of the phrase (the noun), independent of its role in the sentence (subject, object, etc.). Interestingly, in many languages the word order for the sentence is mirrored in the word order for the NP, assuming the verb/noun is the head of the VP/NP respectively. So if the sentence word order is, say verb-final, the NP order will also be head-final.

\begin{problem}{\langnameNung}{\nameAWade}{\LOYear{\UKLOAbbr}{2016}}
\IntroSentences{\langnameNung}\ \IntroAndEnglish:

\begin{tabular}{rl}
     \sentlinetworows{Cáu ca vửhn nhahng kíhn.}{I was about to continue to eat it.}
     \sentlinetworows{Cáu cháhn sl\`{ơ}ng páy mi?}{Do I truly want to go?}
     \sentlinetworows{Cáu mi slày kíhn.}{I don't have to eat it.}
     \sentlinetworows{Cáu ngám hẻht pehn t\'{ê}.}{I did it like that just now.}
     \sentlinetworows{Cáu tan đohc hảhn mưhng.}{I only see you.}
     \sentlinetworows{Cáu vửhn nhahng bô sạhm tảhng hẻht hơn.}{I also continue to build the house alone.}
     \sentlinetworows{Da kíhn!}{Don't eat it!}
     \sentlinetworows{Da khải hơn!}{Don't sell the house!}
     \sentlinetworows{Mưhn chớng ca cháhn fải khải.}{Then she truly was about to have to sell it.}
     \sentlinetworows{Mưhn mi cháhn đày non.}{She truly can't sleep.}
     \sentlinetworows{Mưhn náhc-thày chớng bô sạhm kíhn.}{Then she also just previously ate it.}
     \sentlinetworows{Mưhng náhc-thày slờng tảhng páy.}{You wanted to go alone just previously.}
\end{tabular}\largerpage

\begin{assgts}
\item \transinen
\begin{enumerate}[start = 13]
    \item \cmubdata{Cáu cháhn đày non.}
    \item \cmubdata{Da páy non!}
    \item \cmubdata{Mưhn bô sạhm mi slờng hẻht hơn mi?}
    \item \cmubdata{Mưhn ngám bô sạhm páy hơn.}
\end{enumerate}
\item \transinen[\langnameNung]
\begin{enumerate}[start = 17]
    \item \texttr{I wasn't about to eat it just previously.}
    \item \texttr{She didn't have to eat it alone like that just now.}
    \item \texttr{The house truly can't eat you.}
    \item \texttr{Then were you also about to go just previously?}
\end{enumerate}
\end{assgts}
\end{problem}

\begin{mysolution}

\begin{description}[labelwidth=\widthof{\bfseries Step 3.},leftmargin=!]\sloppy
\item[Step 1.] We notice the repetition of the first word, which we can easily correlate with the subject pronoun (\wordtrans{cáu}{I}, \wordtrans{mưhng}{you}, \wordtrans{mưhn}{she}). This is also confirmed by sentence 5 where we notice that the object has the same form (therefore, S = O).

Moreover, we notice that sentences 7 and 8 both start with \cmubdata{da} and both are in the (negative) imperative mood. We therefore deduce that \cmubdata{da} is the negative imperative marker.

\item[Step 2.] In sentence 7, we have only one word remaining to be translated (\cmubdata{kíhn}). This must be the verb (with semantic content), so \wordtrans{kíhn}{to eat}. This is also confirmed by sentences 1, 3, and 11.

In sentence 8, we have only two remaining words, so one must be the verb, while the other must represent the noun \texttr{house}. Comparing with sentence 9, we deduce that \wordtrans{khải}{to sell} and \wordtrans{hơn}{house}. Moreover, we infer that for negative imperative sentences the word order is \fbox{\cmubdata{Da} V O}. Furthermore, since in sentence 7 \texttr{it} is not translated, we infer that the 3\textsc{sg} pronominal object is unmarked.

Since we have already identified two verbs, we continue to focus on the verbs. From sentences 2 and 12, we deduce that \wordtrans{pȧy}{to go}.

\item[Step 3.] Based on sentence comparison, we can also identify most of the vocabulary, especially the adverbs: \wordtrans{chớng}{then}, \wordtrans{náhc-thày}{just previously}, \wordtrans{bô sạhm}{also}, \wordtrans{cháhn}{truly}, \wordtrans{tảhng}{alone}.

Based on the same principle, we can identify some modal verbs: \wordtrans{ca}{was about to}, \wordtrans{vửhn nhahng}{continue to}.

Lastly, since we already know that \wordtrans{cháhn}{truly}, we are left with \wordtrans{slờng}{want to}.

\item[Step 4.] Based on the above words, sentence 6 is left with only one untranslated word, mainly \wordtrans{hẻht}{to build}. On the other hand, we notice the same word in sentence 4, this time meaning \texttr{to do} (we can assume it also represents a verb). Therefore, we deduce that in Nung, similar to many languages derived from Latin, \texttr{to build a house} $=$ \texttr{to do (make) a house}.

Comparing sentences 3 and 10, we infer that \cmubdata{mi} is a negative marker. Nevertheless, the same marker occurs in sentence 2. Since the only thing left undiscovered in sentence 2 is the interrogation, we deduce that \cmubdata{mi} is also an interrogative marker. Moreover, we notice that if \cmubdata{mi} marks a question, it will be placed at the very end of the sentence (it is rather common that the interrogative marker is placed at the end of the sentence). Therefore, \cmubdata{mi} represents two distinct morphemes: an interrogative marker (in which case it is placed last in the sentence) or a negative marker (in which case it is placed directly after the subject).

Sentences 3 and 10 have only one untranslated phrase, \texttr{to have to}. Nevertheless, we notice that in Nung there are two distinct words: \cmubdata{slày} and \cmubdata{fải}. The simplest explanation for the variation between the two forms is that one is used in positive (affirmative) sentences, while the other is used in negative sentences. There can be many other explanations, such as one appears if the subject is 1\textsc{sg}, while the other if the subject is 3\textsc{sg}, but this explanation does not make sense linguistically.\footnote{The linguistic explanation for which the polarity distinction (affirmative vs. negative) is more likely to be the one that is responsible for the different verb form (rather than the subject of the sentence) is that, usually, suppletive forms are driven by intrinsic characteristics of the verb (polarity, TAM, etc.) and not by their interaction with the arguments.} Moreover, there are two possible interpretations regarding these two forms: either there are two completely different verbs (e.g., in English the negation of \cmubdata{must} is usually \cmubdata{not have to}), or there are two suppletive forms of the same verb, one being used in affirmative sentences and the other in negative ones. 

\item[Step 5.] In sentence 10, we are left with two words: \cmubdata{đày} and \cmubdata{non}, which mean (not necessarily in this order) \texttr{to sleep} and \texttr{can}. Comparing the examples we have got so far, we notice that the modal verb is always placed before the semantic verb. Thus, we deduce that \wordtrans{đày}{can} and \wordtrans{non}{to sleep}.

Using a similar thought process in sentence 4, in which we need to identify the words meaning \texttr{like that} and \texttr{just now}, we can assume that \texttr{just now} will behave similarly to \texttr{just previously} since they both have a similar form and meaning. Since \texttr{just previously} appears before the verb (and even immediately after the subject), it is likely that \wordtrans{ngám}{just now} and \wordtrans{pehn tế}{like that}.

The only words we have not identified yet are \texttr{only} and \texttr{to see}. Since we do not have enough details to determine which is which, we check task (a) to see whether either of them occurs in those examples, offering us additional information. Moreover, we also check task (b) to see whether we need these two words. Since these two words do not occur in any of the tasks, we can just ignore them and not try to assign them to their meaning. If, nevertheless, we would like to take a guess, we would base it on the fact that most of the adverbs are placed before the verb and that the verb is usually a single word. Therefore, we could assume that \wordtrans{hảhn}{to see} and \wordtrans{tan đohc}{only}.

\item[Step 6.] Since we have identified all the vocabulary, we can solve task (a). For each of the sentences, we will first write the translation of each word (in the order in which they appear in Nung) and afterwards we write the English translation. Thus:

\begin{enumerate}[start = 13]
    \item I -- truly -- can -- sleep \Rightarrow\ \texttr{I truly can sleep.}
    \item negative imperative -- go -- sleep \Rightarrow\ \texttr{Don't go to sleep!}
    \item She -- also -- not -- want -- do -- house -- question \Rightarrow\ \texttr{Does she also not want to build the house?}

    We take into account that \texttr{to do a house} is translated as \texttr{to build a house}. Moreover, we notice two things: the negation \cmubdata{mi} does not appear immediately after the subject as we assumed (but it still does appear before the verb). Moreover, another possible translation (and more adequate, grammatically) is \texttr{Doesn't she want to build a house either?}, since the meaning of the adverb \texttr{also} is changed in English when in the negative (compare: \cmubdata{She also came} -- \cmubdata{She didn't come either}).
    \item she -- just now -- also -- go -- house \Rightarrow\ \texttr{She also goes home just now}.

 In this case as well, the sentence could have been translated more literally as \texttr{She also goes to the house just now} (and still be awarded full marks), but, generally, it is preferable to use the more natural translation.
\end{enumerate}

 \item[Step 7.] The only thing left to do is figure out the word order. We already know that the subject comes first. Considering the verb has a fixed position, we notice that after the verb there are only three possible morphemes: the object, the question marker, and the adverb \texttr{like that}. We can hypothesise that the question marker is always last. Since we do not have any example in which the object and \texttr{like that} coexist, we cannot determine the order between them; therefore, we can assume that they occupy the same position. Consequently, we can consider the Nung word order:

\begin{itemize}
    \item[] Negative imperative: \fbox{\cmubdata{Da} V O}
    \item[] Indicative: \fbox{S [...] V O/\texttr{like that} Question}
\end{itemize}

 By ``[...]'' we mean all the adverbs and modal verbs that occur between subject and verb, whose order we are still to determine. Since we know the meaning of each of them and we are only interested in the relative order between them, we can just rewrite that part of the sentence (excluding the subject, the verb and everything after the verb). The negative imperative sentences can be excluded since their word order is clear and they do not contain any adverbs or modal verbs.
 
\begin{center}
    \begin{tabular}{ll}
         \pbsv{ca vửhn nhahng}{was about to continue} \\
         \pbsv{cháhn}{truly} \\
         \pbsv{mi slày}{not have to} \\
         \pbsv{ngám}{just now} \\
         \pbsv{tan đohc hảhn}{see only} \\
         \pbsv{vửhn nhahng bô sạhm tảhng}{also continue alone} \\
         \pbsv{chớng ca cháhn fải}{then truly was about to have to} \\
         \pbsv{mi cháhn đày}{truly can't} \\
         \pbsv{náhc-thày chớng bô sạhm}{then also just previously} \\
         \pbsv{náhc-thày slờng tảhng}{want alone just previously} \\
         \pbsv{cháhn đày}{truly can} \\
         \pbsv{bô sạhm mi slờng}{also not want} \\
         \pbsv{ngám bô sạhm}{also just now} \\
    \end{tabular}
\end{center}
\end{description}

The second, fourth, and fifth examples can be excluded: the second and the fourth because they each have only one word, and are thus not helpful in figuring out the relative positions of the adverbs; the fifth example since we have already discussed it in Step 5 and we cannot segment it.\largerpage

Moreover, since we already know the meaning of all the structures, we can just write their English translations in the order in which they appear in Nung. Thus, we get:

\begin{itemize}[noitemsep]
    \item[] \texttr{was about to -- continue to}
    \item[] \texttr{not -- have to}
    \item[] \texttr{continue to -- also -- alone}
    \item[] \texttr{then -- was about to -- truly -- have to}
    \item[] \texttr{not -- truly -- can}
    \item[] \texttr{just previously -- then -- also}
    \item[] \texttr{just previously -- want to -- alone}
    \item[] \texttr{truly -- can}
    \item[] \texttr{also -- not -- want to}
    \item[] \texttr{just now -- also}
\end{itemize}

 What is left now is no more than a logic puzzle in which we need to arrange all of these words into an overall order which is consistent with each example above. For convenience, we refer to the words by their translations. A simple method is to start with the first word (\texttr{was about to}). It cannot be the first one in the sentence since in the fourth row the word \texttr{then} appears before it. Next, \texttr{then} cannot be first since \texttr{just previously} appears before it in row 6. We notice that \texttr{just previously} always appears first, so we can consider it as occupying the first position. Moreover, we keep in mind that we assumed that \texttr{just previously} and \texttr{just now} behave similarly. Therefore, we can check whether \texttr{just now} also appears in first position. It occurs in only one example and it is indeed in the first position, so we can deduce that the first position is occupied by the temporal adverb \{\texttr{just now}\slash\texttr{just previously}\}. Now we can delete these two adverbs from the list, as well as all examples which now contain a single word. We get:\largerpage

\begin{itemize}[noitemsep]
    \item[] \texttr{was about to -- continue to}
    \item[] \texttr{not -- have to}
    \item[] \texttr{continue to -- also -- alone}
    \item[] \texttr{then -- was about to -- truly -- have to}
    \item[] \texttr{not -- truly -- can}
    \item[] \texttr{then -- also}
    \item[] \texttr{want to -- alone}
    \item[] \texttr{truly -- can}
    \item[] \texttr{also -- not -- want to}
\end{itemize}

Using a similar thought process, we notice that \texttr{then} appears only in one example (among those left), so we can consider it to be the next in line.

So far, we have: \{\texttr{just previously}\slash\texttr{just now}\} \rightarrow\ \{\texttr{then}\}. We are left with:

\begin{itemize}[noitemsep]
    \item[] \texttr{was about to -- continue to}
    \item[] \texttr{not -- have to}
    \item[] \texttr{continue to -- also -- alone}
    \item[] \texttr{was about to -- truly -- have to}
    \item[] \texttr{not -- truly -- can}
    \item[] \texttr{want to -- alone}
    \item[] \texttr{truly -- can}
    \item[] \texttr{also -- not -- want to}
\end{itemize}

 Now \texttr{was about to} appears first, so it can be the next in line.

Note that we get the same result if we start from another word. For example, if we start with the negation \texttr{not} (which appears in the second row), it cannot be the first one since, in the third row, \texttr{continue to} appears before it, while \texttr{continue to} cannot be first since \texttr{was about to} appears before it. Thus, we again end up with \texttr{was about to} as being next in line. Using the same process, we establish the overall order:

\begin{center}
    \begin{tabular}{@{}l@{~}l@{}}
    \{\texttr{just previously} / \texttr{just now}\} & \rightarrow\ \{\texttr{then}\} \rightarrow\ \{\texttr{was about to}\} \rightarrow\ \{\texttr{continue to}\} \\ 
                                                     & \rightarrow\ \{\texttr{also}\} \rightarrow\ \{\texttr{not}\}
    \end{tabular}
\end{center}

 We are left with:

\begin{itemize}[noitemsep]
    \item[] \texttr{truly -- have to}
    \item[] \texttr{truly -- can}
    \item[] \texttr{want to -- alone}
    \item[] \texttr{truly -- can}
\end{itemize}

We notice that \texttr{truly} appears in three out of the four remaining examples and after it we have \texttr{have to}\slash \texttr{can}\slash \texttr{want to}. We therefore deduce that the modal verbs follow it and the last word placed is the adverb \texttr{alone}. In order to get to this result, it is important to assume that \texttr{want to} and \texttr{can} behave similarly, both being modal verbs. Thus, the Nung order of adverbs\slash modal verbs is:

\begin{itemize}[noitemsep]
    \item[] \{\texttr{just previously} / \texttr{just now}\} 
    \item[] \rightarrow\ \{\texttr{then}\} 
    \item[] \rightarrow\ \{\texttr{was about to}\} 
    \item[] \rightarrow\ \{\texttr{continue to}\} 
    \item[] \rightarrow\ \{\texttr{also}\} 
    \item[] \rightarrow\ \{\texttr{not}\} 
    \item[] \rightarrow\ \{\texttr{truly}\} 
    \item[] \rightarrow\ \{\texttr{want to}, \texttr{have to}, \texttr{can}\} 
    \item[] \rightarrow\ \{\texttr{alone}\}
\end{itemize}

 One question that might arise is: why are the modal verbs at the end (\texttr{want to}, \texttr{can}, \texttr{have to}) separated from the verbs \texttr{was about to} and \texttr{continue to}, which we also referred to as modal verbs? In reality, the constructions \texttr{was about to} and \texttr{continue to} are aspectual verbs, rather than modal verbs. Therefore, in broad terms, the Nung order is: \OlympiadGrammar{time} \rightarrow\ \OlympiadGrammar{aspect} \rightarrow\ \texttr{also} \rightarrow\ \texttr{not} \rightarrow\ \texttr{truly} \rightarrow\ \OlympiadGrammar{modal} \rightarrow\ \texttr{alone}.

We need to observe that the adverb \wordtrans{tan đohc}{only} does not appear in the hierarchy above. This is explained by the fact that it appears in a single sentence and it is not accompanied by any other adverb, so it cannot be compared with any other word.

Now we can solve the task (b) and write the rules.

\rules

 Word order:

\begin{itemize}[noitemsep]
    \item[] Negative imperative: \fbox{\cmubdata{Da} V O}
    \item[] Indicative: \fbox{S [...] V O/\texttr{like that} (\cmubdata{mi} = Question)}
\end{itemize}

 [...] represents all the other adverbs\slash modals, which are written in the following order:
 
\begin{itemize}[noitemsep]
    \item[] \{\wordtrans{náhc-thày}{just previously}\slash \wordtrans{ngám}{just now}\}
    \item[] \rightarrow\ \{\wordtrans{chớng}{then}\}
    \item[] \rightarrow\ \{\wordtrans{ca}{was about to}\}
    \item[] \rightarrow\ \{\wordtrans{vửhn nhahng}{continue to}\}
    \item[] \rightarrow\ \{\wordtrans{bô sạhm}{also}\}
    \item[] \rightarrow\ \{\wordtrans{mi}{not}\}
    \item[] \rightarrow\ \{\wordtrans{cháhn}{truly}\}
    \item[] \rightarrow\ \{\wordtrans{slờng}{want to}\slash \wordtrans{fải}{have to}\slash \wordtrans{slày}{not have to}\slash \wordtrans{đày}{can}\} 
    \item[] \rightarrow\ \{\wordtrans{tảhng}{alone}\}
\end{itemize}

 Moreover, \{\wordtrans{tan đohc}{only}\} belongs to this category as well, but its place in the order cannot be determined.

\begin{assgts}
    \item
    \begin{enumerate}[start = 13]

        \item \texttr{I truly can sleep.}
        \item \texttr{Don't go to sleep!}
        \item \texttr{Doesn't she want to build the house either?}
        \item \texttr{She also goes to the house just now.}
    \end{enumerate}
    \item
    \begin{enumerate}[start = 17]

        \item \cmubdata{Cáu náhc-thày ca mi kíhn.}
        \item \cmubdata{Mưhn ngám mi slày tảhng kíhn pehn tế.}
        \item \cmubdata{Hơn mi cháhn đày kíhn mưhng.}
        \item \cmubdata{Mưhng náhc-thày chớng ca bô sạhm páy mi?}
    \end{enumerate}
\end{assgts}
\end{mysolution}

\section{Focusing}

 Focusing describes the syntactic process in which one part of the sentence is emphasized. In English, focusing can be done by intonation in speaking (compare: \cmubdata{He \textbf{hit} the dog} and \cmubdata{He hit \textbf{the dog}} In the first sentence, the emphasis is on the action of hitting, while in the latter the emphasis is on the object). Moreover, a common way to focus the object in English is by using a cleft sentence (e.g., \cmubdata{I saw a cat} vs. \cmubdata{It is a cat that I saw}). 
 
 In certain languages, focusing can be done by changing the word order (in most cases, this means that the focused part is moved nearer to the beginning of the sentence) or by using certain specific markers. For example, there might be definite or indefinite articles specific for the focused form or there can be specific morphemes which signal the fact that some words are focused. In Wolof,\footnote{This phenomenon was featured in a problem by \nameVNeacsu\ (RoLO 2019).} for example, there are four sets of pronouns: subject (1\textsc{sg} = \cmubdata{man}), object (1\textsc{sg} = \cmubdata{ma}), verb-focus (1\textsc{sg} = \cmubdata{damay}), and object-focus (1\textsc{sg} = \cmubdata{laa}). The first two types (subject and object) are used by default; the verb-focus pronoun is used for the subject if the verb is emphasized, while the object-focus form is used instead of the usual object pronoun, if it is emphasized. Moreover, if the verb or the object are focused, they are moved to the beginning of the sentence. Below the four forms for 1\textsc{sg} and 2\textsc{sg} are given:

\begin{table}[H]
    \begin{tabular}{l cccc}
    \lsptoprule
    & S & O & V-focus & O-focus \\ \midrule
    1\textsc{sg} & \cmubdata{man} & \cmubdata{ma} & \cmubdata{damay} & \cmubdata{laa}\\
    2\textsc{sg} & \cmubdata{yow} & \cmubdata{la} & \cmubdata{dangay} & \cmubdata{nga}\\
    \lspbottomrule
    \end{tabular}
\end{table}

 Let us consider the sentence \texttr{I saw you\sg} (in Wolof, the corresponding verb is \cmubdata{gisoon}). We can have the following three cases:

\begin{enumerate}
    \item Neutral sentence = no part of the sentence is focused. The word order is the default one, SOV.

    \centerline{\cmubdata{Man la gisoon.} (\texttr{I saw you\sg.})}
    \item Sentence with focused verb = the V-focus pronoun is used to replace the subject and it is placed at the beginning of the sentence (it can be considered a focus marker), being immediately followed by the verb.

    \centerline{\cmubdata{Damay gisoon la.} (\texttr{I \emph{saw} you\sg.})}
    \item Sentence with focused object = object becomes the first in the sentence, the rest of the order is preserved.

    \centerline{\cmubdata{Nga man gisoon.} (\texttr{I saw \emph{you}\sg.})}
\end{enumerate}

\section{Morphosyntactic alignment} \label{morphoalign}

 This refers to the way in which three verbal arguments (subject of intransitive verb, subject of transitive verb, object) behave. For simplicity, in this chapter we will use the following notation: Subject of intransitive verb = Subject = S, Subject of transitive verb = Agent = A, Direct object = Object = O.

In order to better illustrate this concept, let us consider the following problem:

\begin{problem}{Morphosyntactic alignments}{\nameVNeacsu}{\LOYear{\RoLOAbbr}{2016}}
The following 16 sentences represent the translations of four different English sentences into four different languages, \OlympiadRandomOrder{}:

\begin{multicols}{2}
\begin{enumerate}[noitemsep]
    \item \cmubdata{Bayi jugumbil baŋgul gúdaŋgu buŗan.}
    \item \cmubdata{'ua hi'o te tamari'i.}
    \item \cmubdata{Bayi yaŗa buŗan.}
    \item \cmubdata{Pol'eʔi hina nawta.}
    \item \cmubdata{Cíq'ămqalnim peexne 'áyatne.}
    \item \cmubdata{'ua hi'o te 'ūrī 'i te vahine.}
    \item \cmubdata{Háma peexne.}
    \item \cmubdata{Bayi gagara baŋgul yaŗaŋgu buŗan.}
    \item \cmubdata{Met'aii pol'eʔ nawta.}
    \item \cmubdata{'áyatnim peexne hámane.}
    \item \cmubdata{'ua hi'o te tamari'i 'i te 'āva'e.}
    \item \cmubdata{Pol'eʔi nawta.}
    \item \cmubdata{'ua hi'o te vahine 'i te tamari'i.}
    \item \cmubdata{Hámanim peexne hísemtuksne.}
    \item \cmubdata{Bayi yaŗa baŋgul jugumbilŋgu buŗan.}
    \item \cmubdata{Tsu'itsui met'ai nawta.}
\end{enumerate}
\end{multicols}

\begin{assgts}
\item Group the 16 sentences into four groups, based on the language they are in.
\item Group the 16 sentences into four groups, based on their meaning.
\item Here are eight more sentences:

\begin{enumerate}[start = 17]
    \item \cmubdata{'ua hi'o 'i te vahine.}
    \item \cmubdata{Met'ai pol'eʔi nawta.}
    \item \cmubdata{Bayi gúda buŗan.}
    \item \cmubdata{Cíq'ămqalnim peexne.}
    \item \cmubdata{'ua te hi'o 'i te tamari vahine.}
    \item \cmubdata{Bayi jugumbilŋgu baŋgul yaŗa buŗan.}
    \item \cmubdata{Pol'eʔi pol'eʔ nawta.}
    \item \cmubdata{Peexne 'áyatnim háma.}
\end{enumerate}
\item[] Out of these sentences, six are wrong. Which are these and why are they wrong?
\item Translate the two correct sentences from task (c) into the other three languages.
\end{assgts}
\end{problem}

\begin{mysolution}

\begin{description}[labelwidth=\widthof{\bfseries Step 3.},leftmargin=!]
\item[Step 1.] The sentence grouping based on language can be easily done considering that all sentences in Language 1 (\langnameDyirbal) contain the word \cmubdata{buŗan}, all sentences in Language 2 (\langnameTahitian) start with \cmubdata{'ua hi'o}, all sentences in Language 3 (\langnameNezPerce) contain the word \cmubdata{peexne}, and all sentences in Language 4 (\langnameWappo) end in \cmubdata{nawta}.

\item[Step 2.] Based on the previous grouping, we have the categories:
\begin{description}[font=\normalfont]
    \item[Lang. 1 (Dyirbal)]
      \begin{enumerate}
         \item[]
         \item[1.] \cmubdata{Bayi jugumbil baŋgul gúdaŋgu buŗan.} 
         \item[3.] \cmubdata{Bayi yaŗa buŗan.}                    
         \item[8.] \cmubdata{Bayi gagara baŋgul yaŗaŋgu buŗan.}   
         \item[15.] \cmubdata{Bayi yaŗa baŋgul jugumbilŋgu buŗan.}
      \end{enumerate}
     \item[Lang. 2 (Tahitian)]
       \begin{enumerate}
         \item[]
         \item[2.] \cmubdata{'ua hi'o te tamari'i.}
         \item[6.] \cmubdata{'ua hi'o te 'ūrī 'i te vahine.} 
         \item[11.] \cmubdata{'ua hi'o te tamari'i 'i te 'āva'e.} 
         \item[13.] \cmubdata{'ua hi'o te vahine 'i te tamari'i.} 
       \end{enumerate}
     \item[Lang. 3 (Nez-Perce)]
       \begin{enumerate}
         \item[]
         \item[5.] \cmubdata{Cíq'ămqalnim peexne 'áyatne.}
         \item[7.] \cmubdata{Háma peexne.}                
         \item[10.] \cmubdata{'áyatnim peexne hámane.}    
         \item[14.] \cmubdata{Hámanim peexne hísemtuksne.}
       \end{enumerate}
      \item[Lang. 4 (Wappo)]
        \begin{enumerate}
          \item[]
          \item[4.] \cmubdata{Pol'eʔi hina nawta.}
          \item[9.] \cmubdata{Met'aii pol'eʔ nawta.}
          \item[12.] \cmubdata{Pol'eʔi nawta.} 
          \item[16.] \cmubdata{Tsu'itsui met'ai nawta.}
        \end{enumerate}
\end{description}

 We notice that in each language there is one sentence which is shorter than the others. We assume that these sentences are translations of each other, so $3 = 2 = 7 = 12$.

	We notice that one part of these sentences occurs in all other sentences of that language (\cmubdata{buŗan}, \cmubdata{'ua hi'o}, \cmubdata{peexne}, \cmubdata{nawta}). This represents the verb.

\item[Step 3.] Looking at the sentences in each language, we notice that in Dyirbal \cmubdata{bayi} and \cmubdata{buŗan} appear in every sentence. Expanding on this, the structure of Dyirbal sentences is:

\begin{exe}
\sn[]{\cmubdata{Bayi $X$ buŗan.} \rightarrow~for short sentences}
\sn[]{\cmubdata{Bayi $X$ baŋgul $Y$-ŋgu buŗan.} \rightarrow~for long sentences}
\end{exe}
Doing the same for the other languages, we get:

\begin{center}
    \begin{tabular}{lll}
    \lsptoprule
    Language & Short sentence & Long sentence \\
    \midrule
    Dyirbal & \cmubdata{Bayi $X$ buŗan.} & \cmubdata{Bayi $Y$ baŋgul $Z$-ŋgu buŗan.} \\
    Tahitian & \cmubdata{'ua hi'o te $A$.} & \cmubdata{'ua hi'o te $B$ 'i te $C$.} \\
    Nez-Perce & \cmubdata{$M$ peexne.} & \cmubdata{$N$-nim peexne $P$-ne.} \\
    Wappo & \cmubdata{$R$-i nawta.} & \cmubdata{$S$-i $T$ nawta.}\\
    \lspbottomrule
\end{tabular}
\end{center}

 \item[Step 4.] By analysing the nouns in sentences 2, 3, 7, 12, we know that \cmubdata{yaŗa} = \cmubdata{tamari'i} = \cmubdata{pol'eʔi} = \cmubdata{háma}.

We notice that, in each language, these nouns appear in two more sentences. Therefore, each language has a sentence which does not contain these words, so $1 = 6 = 5 = 16$.

Each of these sentences contains two nouns: one which occurs in one more sentence, and one which does not occur in any other place. Checking the sentences in which that noun also occurs, we deduce $15 = 13 = 10 = 9$ and \cmubdata{jugumbil} = \cmubdata{vahine} = \cmubdata{met'ai} = \cmubdata{'áyat}.

We are left with the other noun from sentences $1=6=5=16$, so \cmubdata{gúda} = \cmubdata{'ūrī} = \cmubdata{cíq'ămqal} = \cmubdata{tsu'itsu}.

Now we are left with only the sentences $8 = 11 = 14 = 4$ and the nouns \cmubdata{gagara} = \cmubdata{'āva'e} = \cmubdata{hísemtuks} = \cmubdata{hina}.

Thus, the sentences are:

    \begin{description}[font=\normalfont]
    \item[Sentence A:]
    \begin{enumerate}
      \item[]
        \item[3.] \cmubdata{Bayi yaŗa buŗan.} \hfill(Dyirbal) 
        \item[2.] \cmubdata{'ua hi'o te tamari'i.} \hfill(Tahitian)
        \item[12.] \cmubdata{Pol'eʔi nawta.} \hfill(Wappo)
        \item[7.] \cmubdata{Háma peexne.} \hfill(Nez-Perce)
    \end{enumerate}
    \item[Sentence B:]
    \begin{enumerate}
        \item[]
        \item[1.] \cmubdata{Bayi jugumbil baŋgul gúdaŋgu buŗan.} \hfill(Dyirbal)
        \item[6.] \cmubdata{'ua hi'o te 'ūrī 'i te vahine.} \hfill(Tahitian)
        \item[16.] \cmubdata{Tsu'itsui met'ai nawta.} \hfill(Wappo)
        \item[5.] \cmubdata{Cíq'ămqalnim peexne 'áyatne.} \hfill(Nez-Perce)
    \end{enumerate}
    \item[Sentence C:]
    \begin{enumerate}
        \item[]
        \item[8.] \cmubdata{Bayi gagara baŋgul yaŗaŋgu buŗan.} \hfill(Dyirbal)
        \item[11.] \cmubdata{'ua hi'o te tamari'i 'i te 'āva'e.} \hfill(Tahitian)
        \item[4.] \cmubdata{Pol'eʔi hina nawta.} \hfill(Wappo)
        \item[14.] \cmubdata{Hámanim peexne hísemtuksne.} \hfill(Nez-Perce)
    \end{enumerate}
    \item[Sentence D:]
    \begin{enumerate}
        \item[]
        \item[15.] \cmubdata{Bayi yaŗa baŋgul jugumbilŋgu buŗan.} \hfill(Dyirbal)
        \item[13.] \cmubdata{'ua hi'o te vahine 'i te tamari'i.} \hfill(Tahitian)
        \item[9.] \cmubdata{Met'aii pol'eʔ nawta.} \hfill(Wappo)
        \item[10.] \cmubdata{'áyatnim peexne hámane.} \hfill(Nez-Perce)
    \end{enumerate}
    
    % \item[Dyirbal]
    % \begin{enumerate}
    %   \item[]
    %     \item[3.] \cmubdata{Bayi yaŗa buŗan.} (Dyirbal)     
    %     \item[1.] \cmubdata{Bayi jugumbil baŋgul gúdaŋgu buŗan.} (Dyirbal) 
    %     \item[8.] \cmubdata{Bayi gagara baŋgul yaŗaŋgu buŗan.} (Dyirbal)   
    %     \item[15.] \cmubdata{Bayi yaŗa baŋgul jugumbilŋgu buŗan.} (Dyirbal)
    % \end{enumerate}
    % \item[Tahitian]
    % \begin{enumerate}
    %   \item[]
    %     \item[2.] \cmubdata{'ua hi'o te tamari'i.} (Tahitian)              
    %     \item[6.] \cmubdata{'ua hi'o te 'ūrī 'i te vahine.} (Tahitian)     
    %     \item[11.] \cmubdata{'ua hi'o te tamari'i 'i te 'āva'e.} (Tahitian)
    %     \item[13.] \cmubdata{'ua hi'o te vahine 'i te tamari'i.} (Tahitian)
    % \end{enumerate}
    % \item[Wappo]
    % \begin{enumerate}
    %   \item[]
    %    \item[12.] \cmubdata{Pol'eʔi nawta.} (Wappo)         
    %    \item[16.] \cmubdata{Tsu'itsui met'ai nawta.} (Wappo)
    %    \item[4.] \cmubdata{Pol'eʔi hina nawta.} (Wappo)     
    %    \item[9.] \cmubdata{Met'aii pol'eʔ nawta.} (Wappo)   
    % \end{enumerate}
    % \item[Nez-Perce]
    % \begin{enumerate}
    %   \item[]
    %       \item[7.] \cmubdata{Háma peexne.} (Nez-Perce)
    %       \item[5.] \cmubdata{Cíq'ămqalnim peexne 'áyatne.} (Nez-Perce)
    %       \item[14.] \cmubdata{Hámanim peexne hísemtuksne.} (Nez-Perce)
    %       \item[10.] \cmubdata{'áyatnim peexne hámane.} (Nez-Perce)
    % \end{enumerate}
\end{description}

 \item[Step 5.] In the table at Step 3, we marked each noun with different letters, not knowing in which order they occur in the long sentences. Now, based on the correspondences, we deduce the final sentence structure:

\begin{table}[H]
    \begin{tabular}{lll}
    \lsptoprule
    Language & Short sentence & Long sentence \\
    \midrule
    Dyirbal & \cmubdata{Bayi $S$ buŗan.} & \cmubdata{Bayi $O$ baŋgul $A$-ŋgu buŗan.} \\
    Tahitian & \cmubdata{'ua hi'o te $S$.} & \cmubdata{'ua hi'o te $A$ 'i te $O$.} \\
    Nez-Perce & \cmubdata{$S$ peexne.} & \cmubdata{$A$-nim peexne $O$-ne.} \\
    Wappo & \cmubdata{$S$-i nawta.} & \cmubdata{$A$-i $O$ nawta.}\\
    \lspbottomrule
\end{tabular}
\end{table}
\end{description}
 We can easily solve tasks (c) and (d):

\begin{solutions}[start=3]
    \item
    \begin{enumerate}[start = 17]

        \item \cmubdata{'i} occurs
        \item Word order
        \item[20] Ending of the first word
        \item[21] Word order
        \item[22] Word order
        \item[24] Word order and ending of last word
    \end{enumerate}
    \item
    \begin{tabularx}{\linewidth}[t]{llQ}
    \lsptoprule
    & {19.} & {23.}\\
    \midrule
    Dyirbal & \cmubdata{Bayi gúda buŗan.} & \cmubdata{Bayi yaŗa baŋgul yaŗaŋgu buŗan.} \\
    Nez-Perce & \cmubdata{Cíq'ămqal peexne.} & \cmubdata{Hámanim peexne hámane.} \\
    Tahitian & \cmubdata{'ua hi'o te 'ūrī.} & \cmubdata{'ua hi'o te tamari'i 'i te tamari'i.} \\
    Wappo & \cmubdata{Tsu'itsui nawta.} & \cmubdata{Pol'eʔi pol'eʔ nawta.} \\
    \lspbottomrule
    \end{tabularx}
\end{solutions}

This problem allows us to better understand the fundamental difference between morphosyntactic alignments across languages. Returning to the general structure of the above sentences, we notice we have four situations:

\begin{enumerate}
    \item In Wappo, $S$ and $A$ are marked identically (using the suffix \cmubdata{-i}), but differently from $O$. In this case, we talk about a \OlympiadNewTerm{nominative-accusative alignment}. This is the most common type of alignment. In this case, $S$ and $A$ are the nominative arguments, while $O$ is the accusative argument.
    \item In Nez-Perce, $S$, $A$ and $O$ are all marked differently. This is, by definition, the \OlympiadNewTerm{tripartite alignment}.
    \item In Tahitian, there is no difference between $S$, $A$ and $O$ (all three are unmarked), so we say that this language features a \OlympiadNewTerm{direct alignment}.
    \item In Dyirbal, $S$ is marked like $O$ (in this case, unmarked or using a null morpheme), but differently from $A$ (which receives the suffix \cmubdata{-ŋgu}). This language exhibits an \OlympiadNewTerm{ergative-absolutive alignment}. The agent is the only ergative argument, while the absolutive arguments are the subject and the object.
    \item There is another type of alignment, extremely rarely used: the \textsc{transitive} alignment in which $A$ and $O$ are marked the same, while $S$ is marked differently.
\end{enumerate}

A schematic representation of the five types of alignments is shown below, where identically marked arguments are highlighted in the same colour:

\begin{table}[H]
  \begin{tabular}{cccccc}
  \lsptoprule
  Alignment & {Nom-Acc} & {Erg-Abs} & {Tripartite} & {Direct} & {Transitive} \\\midrule
  $S$ & \cellcolor[HTML]{aeaeae} & \cellcolor[HTML]{aeaeae} & \cellcolor[HTML]{aeaeae} & \cellcolor[HTML]{aeaeae} & \\
  $A$ & \cellcolor[HTML]{aeaeae} & & & \cellcolor[HTML]{aeaeae} & \cellcolor[HTML]{aeaeae} \\
  $O$ & & \cellcolor[HTML]{aeaeae} & \cellcolor[HTML]{878787} & \cellcolor[HTML]{aeaeae} & \cellcolor[HTML]{aeaeae}\\
  \lspbottomrule
\end{tabular}
\end{table}

It is important to understand that the morphosyntactic alignment is intrinsic to the language. Thus, in English we cannot talk about an ergative argument simply because English does not follow an ergative-absolutive alignment. Thus, the first step is determining the type of alignment that the language follows.

In the solution to Problem 6.13, we mentioned that the plurality of the verb is determined by the subject of the intransitive verb or the object of the transitive verb. In this problem, the two arguments have the same role (determining the verb plurality), so we can combine them under the specific of the absolutive case, stating that the marker \cmubdata{-pa} appears if the absolutive argument of the verb is plural.
\end{mysolution}

\section{Split alignment}

Certain languages can have two (or more) types of alignments, each of them appearing in a specific linguistic context. For example, Pashto has a split alignment: it follows a nominative-accusative alignment in the present tense, but an ergative-absolutive alignment in the past. We can compare the following examples:

\begin{exe}
\begin{multicols}{3}
\sn[]{
    \gll \textit{Ze} wlarrem.\\
          I went\\
     \glt \texttr{I went.}
}
\sn[]{
    \gll \textit{Ze} dzem.\\
         I go\\
     \glt \texttr{I go.}
}
\sn[]{
    \gll Dai \textit{me} woleed.\\
         him I saw\\
     \glt \texttr{I saw him.}
}
\sn[]{
    \gll \textit{Ze} yay weenem.\\
         I him see\\
    \glt \texttr{I see him.}
}
\sn[]{
    \gll \textit{Ze} yay woleedelem.\\
         me he saw\\
     \glt \texttr{He saw me.}
}
\sn[]{
    \gll Dai \textit{me} weenee.\\
         he me sees\\
    \glt \texttr{He sees me.}
}
\end{multicols}
\end{exe}

We notice that 1\textsc{sg} can be expressed in two ways: \cmubdata{ze} and \cmubdata{me}. In the past tense sentences (first row), \cmubdata{ze} is used for subject and object, while \cmubdata{me} is used for agent. Thus, the past follows an ergative-absolutive alignment, with \cmubdata{me} being used as the ergative form of 1\textsc{sg} and \cmubdata{ze} the absolutive form. At the same time, in the present tense sentences, we notice that \cmubdata{ze} is used for S and A, while \cmubdata{me} is used for O. Thus, the present tense follows a nominative-accusative alignment, with \cmubdata{ze} being the nominative form of 1\textsc{sg} and \cmubdata{me} the accusative form.

Another situation in which we can talk about the coexistence of two different alignments in the same language (but which is not considered split alignment) is that in which, historically speaking, a language had a certain alignment but, in time, due to changes in morphophonology, it came to use the direct alignment (completely unmarked). This can be easily observed in English, where we talk about a nominative-accusative alignment of the pronouns (\cmubdata{he} -- \cmubdata{him}, \cmubdata{I} -- \cmubdata{me}), but about a direct alignment of the nouns.

In linguistics problems the ergative-absolutive alignment is commonly found (together with the nominative-accusative one), while the split alignment is usually highlighted between these two types of alignment. Although in Pashto the context of the two alignments depends solely on the tense, the distinction is motivated for other reasons too: person, discourse prominence of arguments (noun vs pronoun), etc. Every time we see both transitive and intransitive sentences in a linguistics problem, we need to take into account the possibility that the language has a different alignment. Although nominative-accusative languages are more common, especially in the West, and therefore more familiar, ergative languages represent about a quarter of all world languages, and are particularly found in less known language families, and so are proportionally more likely to occur in linguistics problems! Another feature of ergative languages is that they are almost all verb-initial or verb-final, almost never SVO.

\hypertarget{practice-problems}{%
\section{Practice problems}}

\begin{problem}{\langnameLuiseno}{\nameRHudson}{\LOYear{\UKLOAbbr}{2012}}
\IntroSentences{\langnameLuiseno}\ written in the International Phonetic Alphabet \IntroAndEnglish:

\begin{center}
    \begin{tabular}{rll}
         \sentlineonerow{nawitmalqajwukalaqpoki:k}{The girl does not walk home.}
         \sentlineonerow{jaʔaʃpolo:v}{The man is good.}
         \sentlineonerow{hu:ʔunikatqajtʃipomkat}{The teacher is not a liar.}
         \sentlineonerow{haxʂuxetʃiqʂuŋa:li}{Who hits the woman?}
         \sentlineonerow{jaʔaʃwukalaq}{The man walks.}
    \sentlineonerow{to:wqʂuʂuŋa:lihu:ʔunikat}{Does the teacher see the woman?}
        \sentlineonerow{ʔiviʂuŋa:lnona:jixetʃiq}{This woman hits my father.}
    \sentlineonerow{nona:jiʂuxetʃiqʔiviʂuŋa:l}{Does this woman hit my father?}
    \sentlineonerow{ʔiviʂuŋa:lxetʃiqnona:ji}{This woman hits my father.}
         \sentlineonerow{hu:ʔunikattʃipomkat}{The teacher is a liar.}
         \sentlineonerow{ʔivihu:ʔunikatnona:jito:wq}{This teacher sees my father.}
         \sentlineonerow{hu:ʔunikatʂuto:wqʂuŋa:li}{Does the teacher see the woman?}
    \end{tabular}
\end{center}

\begin{assgts}
\item \transinen
\begin{enumerate}[start = 13]
    \item \cmubdata{jaʔaʃwukalaqpoki:k}
    \item \cmubdata{xetʃiqʂuʂuŋa:linona:j}
    \item \cmubdata{haxʂuqajtʃipomkat}
    \item \cmubdata{ʂuŋa:liʂuto:wqhu:ʔunikat}
\end{enumerate}
\item Translate into Luiseño. Use vertical lines to represent word spaces (\cmubdata{a|b}):
\begin{enumerate}[start = 17]
    \item \texttr{Is the teacher a liar?}
    \item \texttr{The teacher sees the woman.}
    \item \texttr{This girl does not see my father.}
    \item \texttr{Who is good?}
\end{enumerate}
\end{assgts}

\end{problem}

\begin{problem}{\langnameBeja}{\nameHSomers\ \& \nameRHudson}{\LOYear{\NACLOAbbr}{2018}}
Here are some Beja sentences and their English translations \OlympiadRandomOrder{}. Two Beja sentences have the same English translation.

\begin{center}
    \begin{tabular}{rl@{\hskip0.5in}cl}
         \chaosline{Tak rihan.}{I saw a man that is strong.}
         \chaosline{Yaas rihan.}{I know a man that I saw.}
         \chaosline{Akra tak rihan.}{I saw a man that is small.}
         \chaosline{Dabalo yaas rihan.}{I saw a small dog.}
         \chaosline{Tak akraab rihan.}{I saw a strong man.}
         \chaosline{Tak dabaloob rihan.}{I saw a dog.}
         \chaosline{Tak akteen.}{I saw a man.}
         \chaosline{Rihane tak akteen.}{I know a man.}
         9. & \cmubdata{Tak rihaneeb akteen.} & & \\
    \end{tabular}
\end{center}

\begin{assgts}
\item \detcorr
\item Here are some more words from the Beja language with their translations:
\begin{center}
    \cmubdata{araw} = \texttr{friend}, \cmubdata{mek} = \texttr{donkey}, \cmubdata{kwati} = \texttr{happy}
\end{center}
\item[] Translate the following sentences into Beja. If there are different ways to translate the sentence, show all the alternatives.

\begin{enumerate}[start = 10]
    \item \texttr{I saw a donkey.}
    \item \texttr{I saw a happy man.}
    \item \texttr{I know a strong donkey.}
    \item \texttr{I saw a friend that is happy.}
    \item \texttr{I know a dog that is small.}
    \item \texttr{I saw a donkey that I know.}
\end{enumerate}

\item Translate the following sentences into English. One of them has a mistake. Write the correct version of this sentence.
\begin{multicols}{2}
\begin{enumerate}[start = 16]
    \item \cmubdata{Kwati mek rihan.}
    \item \cmubdata{Akraab araw akteen.}
    \item \cmubdata{Akteene yaas rihan.}
    \item \cmubdata{Mek dabaloob akteen.}
\end{enumerate}
\end{multicols}
\end{assgts}
\end{problem}

\begin{problem}{\langnameMundari}{\namePArkadiev}{\LOYear{\MSKAbbr}{2014}}
\IntroSentences{\langnameMundari}\ \IntroAndEnglish:\largerpage

\begin{center}
    \begin{tabular}{rl}
        \sentlinetworows{senkena-ñ}{I left.}
        \sentlinetworows{koɽa-eʔ senkena}{The man left.}
        \sentlinetworows{otere-m dubkena}{You\sg\ sat on the ground.}
        \sentlinetworows{coke-ñ lelkiʔia}{I saw the frog.}
        \sentlinetworows{pulis honko-eʔ lelkedkoa}{The policeman saw the children.}
        \sentlinetworows{biŋ coke-ʔ huakiʔia}{The snake bit the frog.}
        \sentlinetworows{seta pulisko-eʔ huakedkoa}{The dog bit the policemen.}
        \sentlinetworows{biŋ setaʔre-m sabkiʔia}{You\sg\ caught the snake in the morning.}
        \sentlinetworows{pulisko kumbuɽu hola-ko sabkiʔia}{The policemen caught the thief yesterday.}
        \sentlinetworows{kuɽiko honko hature-ko ʈokoeʔkedkoa}{The women scolded the children in the village.}
    \end{tabular}
\end{center}

\begin{assgts}
\item \transinen
\begin{enumerate}[start = 11, leftmargin = 0.2in]
    \item \cmubdata{kumbuɽuko-ko dubkena}
    \item \cmubdata{hola-ñ senkena}
    \item \cmubdata{biŋko-m lelkedkoa}
    \item \cmubdata{hon seta setaʔre-ʔ ʈokoeʔkiʔia}
    \item \cmubdata{koɽa coke-ʔ sabkiʔia}
\end{enumerate}
\item \transinen[\langnameMundari]
\begin{enumerate}[resume, leftmargin = 0.2in]
    \item \texttr{They left.}
    \item \texttr{The woman sat on the ground.}
    \item \texttr{The thieves saw the men.}
    \item \texttr{The dogs bit the thief.}
    \item \texttr{He caught the frogs yesterday.}
\end{enumerate}
\end{assgts}
\end{problem}

\begin{problem}{\langnameSwahili}{\nameHSomers}{\LOYear{\NACLOAbbr}{2011}}
\IntroSentences{\langnameSwahili}\ \IntroAndEnglish:

\begin{center}
\begin{tabular}{rl}
    \sentlinetworows{Mtu ana watoto wazuri.}{The man has good children.}
    \sentlinetworows{Mto mrefu una visiwa vikubwa.}{The long river has large islands.}
    \sentlinetworows{Wafalme wana vijiko vidogo.}{The kings have small spoons.}
     \sentlinetworows{Watoto wabaya wana miwavuli midogo.}{The bad children have small umbrellas.}
     \sentlinetworows{Kijiko kikubwa kinatosha.}{The large spoon is enough.}
     \sentlinetworows{Mwavuli una mfuko mdogo.}{The umbrella has a small bag.}
     \sentlinetworows{Kisiwa kikubwa kina mfalme mbaya.}{The large island has a bad king.}
     \sentlinetworows{Watu wana mifuko mikubwa.}{The men have large bags.}
     \sentlinetworows{Viazi vibaya vinatosha.}{The bad potatoes are enough.}
     \sentlinetworows{Mtoto ana mwavuli mkubwa.}{The child has a large umbrella.}
     \sentlinetworows{Mito mizuri mirefu inatosha.}{The good long rivers are enough.}
     \sentlinetworows{Mtoto mdogo ana kiazi kizuri.}{The small child has a good potato.}
\end{tabular}
\end{center}

\begin{assgts}
\item \transinen[\langnameSwahili]\largerpage
\begin{enumerate}[start = 13]

    \item \texttr{The small children have good spoons.}
    \item \texttr{The long umbrella is enough.}
    \item \texttr{The bad potato has a good bag.}
    \item \texttr{The good kings are enough.}
    \item \texttr{The long island has bad rivers.}
    \item \texttr{The spoons have long bags.}
\end{enumerate}
\item If the Swahili word for \texttr{the prince} is \cmubdata{mkuu}, what do you think the word for \texttr{the princes} is? Explain.
\end{assgts}

\end{problem}

\begin{problem}{\langnameArabic}{\nameGDurnovo}{\LOYear{\MSKAbbr}{1997}}
\IntroSentences{\langnameArabic}\ \IntroAndEnglish:

\begin{center}
    \begin{tabular}{rl}
         \sentlinetworows{'aḥraqa lmudarrisu lḥayawāna}{The teacher burnt the monster.}
         \sentlinetworows{'abda'a lqāmūsa llaðī 'aḥraqtuhu}{He created the dictionary that I burnt.}
         \sentlinetworows{'aðlaltu lmudarrisa llaðī 'aṣammaka}{I beat [=did beat] the teacher who surprised you\sg.}
         \sentlinetworows{'axraǧtu lxādima llaðī 'aṣmamtahu}{I brought the servant whom you\sg\ surprised.}
         \sentlinetworows{'aðalla lḥayawānu lkalba llaðī 'afazzahu}{The monster beat [=did beat] the dog which scared him.}
    \end{tabular}
\end{center}

\begin{assgts}
\item One of the sentences above is ambiguous and can be translated into English in a different way. Which sentence is it and what is the alternative translation?
\item \transinen
\begin{enumerate}[start = 6]
    \item \cmubdata{'abda'tuhu}
    \item \cmubdata{'axraǧta lmudarrisa llaðī 'afazzaka}
    \item \cmubdata{'aṣamma lxādimu lkalba llaðī 'aðallahu lmudarrisu}
    \item \cmubdata{'aḥraqtu lḥayawāna llaðī 'aðalla lxādima}
\end{enumerate}
\item \transinen[\langnameArabic]
\begin{enumerate}[resume]
    \item \texttr{You\sg\ scared the servant who surprised the monster.}
    \item \texttr{The dog brought the teacher who beat [=did beat] you\sg.}
    \item \texttr{I burnt the dictionary that you\sg\ created.}
\end{enumerate}
\end{assgts}

\begin{tblsWarning}
\cmubdata{'}, \cmubdata{ð}, \cmubdata{ǧ}, \cmubdata{h}, \cmubdata{ḥ}, \cmubdata{q}, \cmubdata{ṣ}, \cmubdata{x} are consonants. A bar above a vowel denotes length.
\end{tblsWarning}
\end{problem}

\begin{problem}{\langnameWelsh}{\nameTMaisak}{\LOYear{\MSKAbbr}{1998}}
\IntroSentences{\langnameWelsh}\ \IntroAndEnglish:

\begin{center}
    \begin{tabular}{rl}
        \sentlinetworows{Mae tad canllaith gan ei fanon e.}{His queen has a good father.}
        \sentlinetworows{Mae banon ganllaith gan ei blentyn e.}{His child has a good queen.}
        \sentlinetworows{Mae brawd teg gan ei gyfaill e.}{His friend has a beautiful brother.}
        \sentlinetworows{Mae tywysoges deg gan 'y nhad i.}{My father has a beautiful princess.}
        \sentlinetworows{Mae cyfaill penffol gan 'y newynes i.}{My witch has a stupid friend.}
        \sentlinetworows{Mae plentyn talentog gan 'y manon i.}{My queen has a talented child.}
        \sentlinetworows{Mae dewynes gall gan 'y nghyfaill i.}{My friend has a wise witch.}
    \end{tabular}
\end{center}
\begin{assgts}
\item \transinen
\begin{enumerate}[start = 8]
    \item \cmubdata{Mae banon deg gan ei frawd e.}
    \item \cmubdata{Mae tywysoges gall gan ei ddewynes e.}
    \item \cmubdata{Mae cyfaill canllaith gan 'y nhywysoges i.}
\end{enumerate}
\item \transinen[\langnameWelsh]
\begin{enumerate}[start = 11]
    \item \texttr{His father has a stupid princess.}
    \item \texttr{His princess has a wise father.}
    \item \texttr{My child has a talented witch.}
\end{enumerate}
\end{assgts}

\begin{tblsWarning}
\cmubdata{c} = \texttr{c} in \texttr{car}.
\end{tblsWarning}
\end{problem}

\begin{problem}{\langnameTadaksahak}{\nameBBozhanov}{\LOYear{\UKLOAbbr}{2011}}
\IntroSentences{\langnameTadaksahak}\ \IntroAndEnglish:

\begin{tabular}{rl}
    \sentlinetworows{aɣagon cidi}{I swallowed the salt.}
    \sentlinetworows{atezelmez hamu}{He will have the meat swallowed (by someone).}
    \sentlinetworows{atedini a}{He will take it.}
    \sentlinetworows{hamu anetubuz}{The meat was not taken.}
    \sentlinetworows{jifa atetukuš}{The corpse will be taken out.}
    \sentlinetworows{amanokal anešukuš cidi}{The chief didn't have the salt taken out.}
    \sentlinetworows{aɣakaw hamu}{I took out the meat.}
    \sentlinetworows{itegzem}{They were slaughtered.}
    \sentlinetworows{aɣasezegzem a}{I'm not having him slaughtered.}
    \sentlinetworows{anešišu aryen}{He didn't have the water drunk (by anybody).}
    \sentlinetworows{feji abnin aryen}{The sheep is drinking the water.}
    \sentlinetworows{idumbu feji}{They slaughtered the sheep.}
    \sentlinetworows{cidi atetegmi}{The salt will be looked for.}
    \sentlinetworows{amanokal abtuswud}{The chief is being watched.}
    \sentlinetworows{cidi asetefred}{The salt is not being gathered.}
    \sentlinetworows{amanokal asegmi i}{The chief had them looked for.}
\end{tabular}

\pagebreak
\begin{assgts}
\item \transinen
\begin{enumerate}[start = 17]

\begin{multicols}{2}
    \item \cmubdata{aryen anetišu}
    \item \cmubdata{aɣasuswud feji}
    \item \cmubdata{cidi atetelmez}
    \item \cmubdata{asedini jifa}
    \end{multicols}
\end{enumerate}
\item If the stem of the verb \texttr{to walk} is \cmubdata{iʒuwenket}, translate into Tadaksahak:
\begin{enumerate}[start = 21]

    \item \texttr{He is having the water taken.}
    \item \texttr{I'm having them walked.}
    \item \texttr{The chief did not drink the water.}
    \item \texttr{The salt was not looked for.}
    \item \texttr{He will have the salt gathered.}
\end{enumerate}
\end{assgts}

\begin{tblsWarning}
\explainzh{ʒ}, \explainsh{š}, \cmubdata{ɣ} is a consonant.
\end{tblsWarning}
\end{problem}

\begin{problem}{\langnameSandawe}{\nameSCHuang}{\LOYear{\APLOAbbr}{2021}}
\IntroSentences{\langnameSandawe}\ \IntroAndEnglish:\largerpage[-1]

\begin{longtable}{rl}
    \sentlinetworows{!'ìnéỳsù kòŋkórìsà xéʔé̥wáá}{A hunter\fem\ brought roosters.}
    \sentlinetworows{thíméỳsù kókósà ǁ'èésú}{A cook\fem\ skinned a hen.}
    \sentlinetworows{múk'ùmè kókó xéʔé̥wáátshú}{A cow didn't bring hens.}
    \sentlinetworows{kòŋkórì múk'ùmèʔà khàású}{Roosters hit [=did hit] a cow.}
    \sentlinetworows{!'ìnéỳsò k'ámbà khàáyétshógé}{Apparently, hunters didn't hit a bull.}
    \sentlinetworows{thíméỳ !'ìnéỳ xééyétshèégé}{Apparently, a cook\masc\ didn't bring a hunter\masc.}
    \sentlinetworows{!'ìnéỳsò kókógéʔà ǁ'èésú}{Apparently, hunters skinned a hen.}
    \sentlinetworows{!'ìnéỳsò kókóʔà khǎʔḁ́wáá}{Hunters hit [=did hit] hens.}
    \sentlinetworows{kòŋkórì !'ìnéỳà xééyé}{A rooster brought a hunter\masc.}
    \sentlinetworows{thíméỳsù kókó khǎʔḁ́wáátshúgé}{Apparently, a cook\fem\ didn't hit hens.}
    \sentlinetworows{!'ìnéỳ thíméỳsògéà khàáʔíŋ}{Apparently, a hunter\masc\ hit [=did hit] cooks.}
    \sentlinetworows{!'ìnéỳsò thíméỳsò ǁ'èéʔíntshó}{Hunters didn't skin cooks.}
    \sentlinetworows{kòŋkórì !'ìnéỳsò xééʔíntshó}{Roosters didn't bring hunters.}
    \sentlinetworows{thíméỳ kòŋkórì khǎʔḁ́wáátshèé}{A cook\masc\ didn't hit roosters.}
\end{longtable}

Given below are some more words in Sandawe and their English translations:

\begin{exe}
\sn[]{\cmubdata{ŋ!àméỳ} = \texttr{blacksmith\masc}}
\sn[]{\cmubdata{bálóó} = \texttr{to herd}}  
\sn[]{\cmubdata{théká} = \texttr{leopard (any gender)}}
\end{exe}

\begin{assgts}
\item \transinen
\begin{enumerate}[start = 15]
    \item \cmubdata{thíméỳ kòŋkórìgéà ǁ'èéyé}
    \item \cmubdata{ŋ!àméỳsù thíméỳsùsà xéésú}
    \item \cmubdata{k'ámbà théká khàásútshógé}
    \item \cmubdata{múk'ùmè !'ìnéỳsòsà bálóóʔíŋ}
\end{enumerate}

\item \transinen[\langnameSandawe]
\begin{enumerate}[start = 19]
    \item \texttr{Cooks herded hens.}
    \item \texttr{Apparently, a blacksmith\fem\ didn't skin leopards.}
    \item \texttr{A leopard\fem\ didn't herd a rooster.}
    \item \texttr{Apparently, a bull didn't bring cooks.}
    \item \texttr{Apparently, a hunter\masc\ brought blacksmiths.}
\end{enumerate}
\end{assgts}

\begin{tblsWarning}
\cmubdata{x}, \cmubdata{th}, \cmubdata{tsh}, \cmubdata{kh}, \cmubdata{k'}, \cmubdata{ŋ}, \cmubdata{ŋ!}, \cmubdata{ʔ}, \cmubdata{!'}, and \cmubdata{ǁ'} are consonants. The marks {\char"25CC\char"301}, {\char"25CC\char"300}, and {\char"25CC\char"30C} above a vowel denote high, low and rising (low $\nearrow$ high) tones, respectively.
A circle under a vowel (e.g., \cmubdata{ḁ}) indicates a devoiced vowel.
\explainmascfem
\end{tblsWarning}
\end{problem}

\begin{problem}{\langnameBurushaski}{\nameDMysak}{\LOYear{\UkrLOAbbr}{2019}}
\IntroSentences{\langnameBurushaski}\ \IntroAndEnglish:

\begin{longtable}{rl}
     \sentlinetworows{khue gušiŋanc uwaran.}{These women will get tired.}
     \sentlinetworows{ise ṣiqar iγurci.}{That wasp will drown.}
     \sentlinetworows{biṭayue amin dasin musarkan?}{Which girl will the shamans let in?}
     \sentlinetworows{γeniṣ muwalo.}{The queen will fall.}
     \sentlinetworows{ue dasiwance šugulimuc usarkan.}{Those girls will let the friends\fem\ in.}
     \sentlinetworows{guse γurqune ṣiqarišo uγarki.}{This frog will catch the wasps.}
     \sentlinetworows{qhudaae ice \d{j}akuyo uyeeci.}{The god will see those donkeys.}
     \sentlinetworows{khine hilese belišo uγarki.}{This boy will catch the rams.}
     \sentlinetworows{hoolalase amic talabuudomuc uyeeci?}{Which spiders will the butterfly see?}
     \sentlinetworows{ue thamišue γeniṣanc uyaranan.}{Those kings will deceive the queens.}
     \sentlinetworows{hilešue šugulo isarkan.}{The boys will let the friend\masc\ in.}
     \sentlinetworows{γaṣepe khine biṭan iyarani.}{The magpie will deceive this shaman.}
\end{longtable}

\pagebreak
\begin{assgts}
\item \transinen
\begin{enumerate}[start = 13]
    \item \cmubdata{ice belišo uwalan.}
    \item \cmubdata{qhudaamuce tham iyaranan.}
    \item \cmubdata{talabuudue khine gus muyeeci.}
    \item \cmubdata{amin guse γurquyo uγarko?}
\end{enumerate}
\item \transinen[\langnameBurushaski]
\begin{enumerate}[start = 17]
    \item \texttr{Those shamans will drown.}
    \item \texttr{Which magpies will the women catch?}
    \item \texttr{The kings will see these butterflies.}
    \item \texttr{Which friend\masc\ will let the boys in?}
    \item \texttr{That boy will deceive the friend\fem.}
    \item \texttr{The queen will let that girl in.}
    \item \texttr{This girl will see the friends\masc.}
    \item \texttr{The wasp will deceive that frog.}
    \item \texttr{Which donkey will get tired?}
\end{enumerate}
\end{assgts}

\begin{tblsWarning}
\cmubdata{γ}, \cmubdata{\d{j}}, \cmubdata{ŋ}, \cmubdata{ṣ}, \cmubdata{š}, and \cmubdata{ṭ} are consonants. \explainmascfem
\end{tblsWarning}
\end{problem}

\hypertarget{solutions-of-practice-problems}{%
\section{Solutions of practice
problems}}

\begin{practiceproblemsolution}{ 7.3. \langnameLuiseno}

\begin{solutions}[label=Solution 7.3\alph*]
    \item
    \begin{enumerate}[start = 13]
        \item \texttr{The man walks home.}
        \item \texttr{Does my father hit the woman?}
        \item \texttr{Who is not a liar?}
        \item \texttr{Does the teacher see the woman?}
    \end{enumerate}
    \pagebreak
    \item \begin{enumerate}[resume]

        \item \cmubdata{hu:ʔunikat | ʂu | tʃipomkat}
        \item \cmubdata{hu:ʔunikat | to:wq | ʂuŋa:li}
        \item \cmubdata{ʔivi | nawitmal | qaj | to:wq | nona:ji}
        \item \cmubdata{hax | ʂu | polo:v}
    \end{enumerate}

\end{solutions}
 \note{Any other word order is accepted, as long as it follows the rules below.}

\rules
\begin{itemize}
    \item Flexible word order; Det. -- Noun; Neg. -- Verb;
    \item The interrogative particle is placed before the verb. If the verb is first in the sentence, the particle is placed after it; i.e., the interrogative particle is always second;
    \item \cmubdata{-i} = object marker.
\end{itemize}
\end{practiceproblemsolution}

\begin{practiceproblemsolution}{7.4. \langnameBeja}
\begin{solutions}[label=Solution 7.4\alph*]
    \item
    \begin{enumerate}
    \begin{multicols}{5}
        \item G
        \item F
        \item E
        \item D
        \item A
        \item C
        \item H
        \item B
        \item B
    \end{multicols}
    \end{enumerate}

    \item
    \begin{enumerate}[resume]
        \item \cmubdata{Mek rihan.}
        \item \cmubdata{Kwati tak rihan.}
        \item \cmubdata{Akra mek akteen.}
        \item \cmubdata{Araw kwatiib rihan.}
        \item \cmubdata{Yaas dabaloob akteen.}
        \item \cmubdata{Akteene mek rihan.} or \cmubdata{Mek akteeneeb rihan.}
    \end{enumerate}
    \pagebreak
    \item
    \begin{enumerate}[resume]
        \item \texttr{I saw a happy donkey.}
        \item Two options:
        \begin{itemize}

            \item \texttr{I know a friend that is strong.} (Correct: \cmubdata{Araw akraab akteen.})
            \item \texttr{I know a strong friend.} (Correct: \cmubdata{Akra araw akteen.})
        \end{itemize}
        \item \texttr{I saw a dog that I know.}
        \item \texttr{I know a donkey that is small.}
    \end{enumerate}
\end{solutions}

\rules
\begin{itemize}
    \item Three sentence patterns (\texttt{V} = Verb, \texttt{A} = Adjective, \texttt{N} = Noun):
    \begin{enumerate}
        \item \texttr{I} \texttt{V} \texttr{a} (\texttt{A}) \texttt{N}. \Rightarrow\ (\texttt{A}) \texttt{N} \texttt{V}.
        \item \texttr{I} \texttt{V} \texttr{a} \texttt{N} \texttr{that is} \texttt{A}. \Rightarrow\ \texttt{N} \texttt{A}-\cmubdata{V\textsuperscript{*}b} \texttt{V}, where \cmubdata{V\textsuperscript{*}} refers to the last vowel of the word.
        \item \texttr{I} \texttt{V} \texttr{a} \texttt{N} \texttr{that I} \texttt{V}$′$. \Rightarrow\ \texttt{V}$′$-\cmubdata{e} \texttt{N} \texttt{V} or \texttt{N} \texttt{V}$′$-\cmubdata{eeb} \texttt{V}.
    \end{enumerate}
    \item The parts separated by hyphen (-) are suffixes.
\end{itemize}
\end{practiceproblemsolution}

\begin{discussion}

 Word order: Object---Verb, Adjective---Noun. The adjective can also act like a verb (e.g., \texttr{small} -- \texttr{to be small}).

 The relative clause (introduced by \texttr{that}) can be placed:
\begin{itemize}

    \item immediately after the noun (in our case, between object and verb) – in which case it receives the suffix \cmubdata{--V\textsuperscript{*}b}, where \cmubdata{V\textsuperscript{*}} is the final vowel of the verb);
    \item before the noun – in which case it receives no suffix (but when the relative clause consists of a verb, the verb receives a suffix \cmubdata{-e}; compare, for example, sentences 6 and 8).
\end{itemize}

 If the relative clause contains a predicative expression, it can only be placed after the noun, otherwise it would simply be translated as an adjective (\texttr{I saw a \textit{small} donkey.} vs. \texttr{I saw a donkey \textit{that is small}.}).

 \end{discussion}



\begin{practiceproblemsolution}{7.5. \langnameMundari}

\begin{solutions}[label=Solution 7.5\alph*]
    \item
        \begin{enumerate}[start = 11]
            \item \texttr{The thieves sat.}
            \item \texttr{I left yesterday.}
            \item \texttr{You\sg\ saw the snakes.}
            \item \texttr{The child scolded the dog in the morning.}
            \item \texttr{The man caught the frog.}
        \end{enumerate}
    \item
        \begin{enumerate}[start = 16]
            \item \cmubdata{senkena-ko}
            \item \cmubdata{kuɽi otere-ʔ dubkena}
            \item \cmubdata{kumbuɽuko koɽako-ko lelkedkoa}
            \item \cmubdata{setako kumbuɽu-ko huakiʔia}
            \item \cmubdata{cokeko hola-eʔ sabkedkoa}
        \end{enumerate}
\end{solutions}

\rules
\begin{itemize}
    \item Word order: S O (Location/Time) V
    \item Plural: \cmubdata{-ko} added to the end of the noun (before the hyphen).
    \item Verbal suffixes:
    \begin{itemize}
        \item \cmubdata{-kena} = intransitive verb;
		\item \cmubdata{-kedkoa} = transitive verb, plural object;
		\item \cmubdata{-kiʔia} = transitive verb, singular object.
	\end{itemize}
 \item The agreement between verb and subject is marked through a suffix separated by a hyphen. It is attached to the word before the verb. If the sentence only contains a verb (one word), it is attached to the verb instead. The forms of this suffix are:
 \begin{itemize}
 \begin{multicols}{2}
     \item \cmubdata{-ñ} = 1\textsc{sg};
     \item \cmubdata{-m} = 2\textsc{sg};
     \item \cmubdata{-eʔ} = 3\textsc{sg} (\cmubdata{eʔ \rightarrow\ ʔ / e \_});
     \item \cmubdata{-ko} = 3\textsc{pl}.
     \end{multicols}
\end{itemize}
\end{itemize}
\end{practiceproblemsolution}

\begin{practiceproblemsolution}{7.6. \langnameSwahili}

\begin{solutions}[label=Solution 7.6\alph*]
    \item
    \begin{enumerate}[start = 13]

        \item \cmubdata{Watoto wadogo wana vijiko vizuri.}
        \item \cmubdata{Mwavuli mrefu unatosha.}
        \item \cmubdata{Kiazi kibaya kina mfuko mzuri.}
        \item \cmubdata{Wafalme wazuri wanatosha.}
        \item \cmubdata{Kisiwa kirefu kina mito mibaya.}
        \item \cmubdata{Vijiko vina mifuko mirefu.}
    \end{enumerate}
    \item \cmubdata{wakuu}. \texttr{Prince} belongs to Class 1 (human), so the plural is formed by replacing the singular prefix \cmubdata{m-} with the plural \cmubdata{wa-}.
\end{solutions}

\rules
\begin{itemize}
    \item Word order: SVO, Noun -- Adj.
    \item Nouns are grouped into three classes:
    \begin{enumerate}[label = Class \arabic*., leftmargin = 5em]
        \item Human nouns: \texttr{child}, \texttr{king}, \texttr{man};
        \item Non-human nouns: \texttr{umbrella}, \texttr{river}, \texttr{bag};
        \item Non-human nouns: \texttr{spoon}, \texttr{potato}, \texttr{island};
    \end{enumerate}
    \item The class and number are marked by a prefix on the noun. The adjective agrees with the noun in class and number, while the verb is conjugated according to the class and number of the subject, as follows:

\begin{table}[H]
    \begin{tabular}{lllllll}
    \lsptoprule
    & \multicolumn{2}{c}{Class 1} & \multicolumn{2}{c}{Class 2} & \multicolumn{2}{c}{Class 3} \\\cmidrule(lr){2-3}\cmidrule(lr){4-5}\cmidrule(lr){6-7}
    & \textsc{sg} & \textsc{pl} &\textsc{sg} & \textsc{pl} &\textsc{sg} & \textsc{pl}\\
    \midrule
    Noun\slash Adj. & \cmubdata{m-} & \cmubdata{wa-} & \cmubdata{m-} & \cmubdata{mi-} & \cmubdata{ki-} & \cmubdata{vi-}\\
    Verb            & \cmubdata{a-} & \cmubdata{wa-} & \cmubdata{u-} & \cmubdata{i-}  & \cmubdata{ki-} & \cmubdata{vi-}\\
    \lspbottomrule
\end{tabular}
\end{table}
\end{itemize}

\note{The identification of noun classes which are marked by a prefix is specific to the Bantoid languages. Depending on the language, the nouns can be classified based on certain semantic considerations (similar to the way in which classifiers work -- see \chapref{chap-noun}), but not necessarily. For example, in this problem, there is no semantic reason to discriminate between Classes 2 and 3. Moreover, we need not find a discriminator, since there are no new words whose class we need to determine. The only distinction we need to make is that Class 1 only includes human nouns, in order to be able to differentiate between Class 1 and Class 2, which use the same singular marker.
}
\end{practiceproblemsolution}

\begin{practiceproblemsolution}{7.7. \langnameArabic}

\begin{solutions}[label=Solution 7.7\alph*]
    \item Sentence 5. \texttr{The monster beat [=did beat] the dog which he scared.}
    \item
    \begin{enumerate}[start = 6]
        \item \texttr{I created him.}
        \item \texttr{You\sg\ brought the teacher who scared you\sg.}
        \item \texttr{The servant surprised the dog which the teacher beat [=did beat].}
        \item \texttr{I burnt the monster which beat [=did beat] the servant.}
    \end{enumerate}
    \item
    \begin{enumerate}[resume]
        \item \cmubdata{'afzazta lxādima llaðī 'aṣamma lḥayawāna}
        \item \cmubdata{'axraǧa lkalbu lmudarrisa llaðī 'aðallaka}
        \item \cmubdata{'aḥraqtu lqāmūsa llaðī 'abda'tahu}
    \end{enumerate}
\end{solutions}

\rules
\begin{itemize}
    \item Word order: VSO; the relative clause is introduced by \cmubdata{llaðī} (\texttr{which}\slash \texttr{who}\slash \texttr{whom}) and the word order inside it is identical (VSO).
    \item Noun: receives the suffixes \cmubdata{-u} (subject) or \cmubdata{-a} (object).
    \pagebreak
    \item Verb:
    \begin{itemize}
        \item Verb stem is represented by three consonants (C\textsubscript{1}-C\textsubscript{2}-C\textsubscript{3}), while the conjugation is done through transfixes (specific to Semitic languages). For example: \texttr{to create} = \cmubdata{b---d---'}, \texttr{to scare} = \cmubdata{f---z---z}, etc.
        \item Subject is marked by the following transfixes:
        \begin{itemize}
            \item 1\textsc{sg}: \cmubdata{'a}---C\textsubscript{1}C\textsubscript{2}---\cmubdata{a}---C\textsubscript{3}---\cmubdata{tu}
            \item 2\textsc{sg}: \cmubdata{'a}---C\textsubscript{1}C\textsubscript{2}---\cmubdata{a}---C\textsubscript{3}---\cmubdata{ta}
            \item 3\textsc{sg}: \cmubdata{'a}---C\textsubscript{1}C\textsubscript{2}---\cmubdata{a}---C\textsubscript{3}---\cmubdata{a} \quad\quad (if C\textsubscript{2} $\neq$ C\textsubscript{3})
            \item[] \hphantom{3\textsc{sg}: }\cmubdata{'a}---C\textsubscript{1}---\cmubdata{a}---C\textsubscript{2}C\textsubscript{3}---\cmubdata{a} \quad\quad (if C\textsubscript{2} = C\textsubscript{3})
\end{itemize}
        \item Object is marked as a suffix to the verb: 2\textsc{sg} = \cmubdata{-ka}, 3\textsc{sg} = \cmubdata{-hu} only if it is not already expressed by noun.
    \end{itemize}
\end{itemize}

\end{practiceproblemsolution}

\begin{practiceproblemsolution}{7.8. \langnameWelsh}

\begin{solutions}[label=Solution 7.8\alph*]
    \item
    \begin{enumerate}[start = 8]

        \item \texttr{His brother has a beautiful queen.}
        \item \texttr{His witch has a wise princess.}
        \item \texttr{My princess has a good friend.}
    \end{enumerate}
    \item
    \begin{enumerate}[resume]

        \item \cmubdata{Mae tywysoges benffol gan ei dad e.}
        \item \cmubdata{Mae tad call gan ei dywysoges e.}
        \item \cmubdata{Mae dewynes dalentog gan 'y mhlentyn i.}
    \end{enumerate}
\end{solutions}

\rules
\begin{itemize}
    \item Word order: \cmubdata{Mae} [O Adj.] \cmubdata{gan} S.
    \item The possessive surrounds the noun: \texttr{his $X$} = \cmubdata{ei $X$ e}, \texttr{my $X$} = \cmubdata{'y $X$ i}.
    \item Noun undergoes initial consonant mutation based on context:
\end{itemize}

\begin{table}[H]
    \begin{tabular}{ccc}
    \lsptoprule
    No possessive & 1\textsc{sg} poss. & 3\textsc{sg} poss. \\
    \midrule
    \cmubdata{b} & \cmubdata{m} & \cmubdata{f} \\
    \cmubdata{p} & \cmubdata{mh} & \cmubdata{b} \\
    \cmubdata{d} & \cmubdata{n} & \cmubdata{dd}\\
    \cmubdata{t} & \cmubdata{nh} & \cmubdata{d}\\
    \cmubdata{g} & \cmubdata{ng} & \cmubdata{}\\
    \cmubdata{c} & \cmubdata{ngh} & \cmubdata{g}\\
    
    \lspbottomrule
\end{tabular}
\end{table}

Thus, we observe the following rules: for 1\textsc{sg} poss., voiced stops become nasals, preserving the place of articulation (\cmubdata{b \rightarrow\ m}, \cmubdata{d \rightarrow\ n}, \cmubdata{g \rightarrow\ ng}), while voiceless stop become aspirated nasals with the same place of articulation (\cmubdata{p \rightarrow\ mh}, \cmubdata{t \rightarrow\ nh}, \cmubdata{c \rightarrow\ ngh}).

For 3\textsc{sg} poss., we cannot deduce the transformation rule for voiced stops, but in the case of the voiceless ones, they become voiced (\cmubdata{p \rightarrow\ b}, \cmubdata{t \rightarrow\ d}, \cmubdata{c \rightarrow\ g}).

The adjective undergoes an initial consonant mutation as well. In the masculine it will have a voiceless stop as the initial consonant, while if it is feminine, it will be voiced (e.g., \texttr{beautiful}: \cmubdata{teg} + \texttr{brother}, \cmubdata{deg} + \texttr{princess}).


\end{practiceproblemsolution}

\begin{practiceproblemsolution}{7.9. \langnameTadaksahak}

\begin{solutions}[label=Solution 7.9\alph*]
    \item
    \begin{enumerate}[start = 17]
        \item \texttr{The water was not drunk.}
        \item \texttr{I had the sheep watched.}
        \item \texttr{The salt will be swallowed.}
        \item \texttr{He is not taking the corpse.}
    \end{enumerate}
    \item
    \begin{enumerate}[resume]

        \item \cmubdata{abzubuz aryen}
        \item \cmubdata{aɣabʒiʒuwenket i}
        \item \cmubdata{amanokal anenin aryen}
        \item \cmubdata{cidi anetegmi}
        \item \cmubdata{atesefred cidi}
    \end{enumerate}
\end{solutions}

\rules
\begin{itemize}
    \item Word order: SVO. If the subject is a pronoun, it is omitted;
    \item Verb: \texttt{S}--\texttt{T}--\texttt{V}--\texttt{R};
    \begin{itemize}
        \item \texttt{S} = Subject: \cmubdata{a-} = 3\textsc{sg}, \cmubdata{i-} = 3\textsc{pl}, \cmubdata{aɣa-} = 1\textsc{sg};
        \item \texttt{T} = Tense (combined with negation):
                \begin{table}[H]
                    \begin{tabular}{cccc}
                    \lsptoprule
                    & Past & Present & Future \\\midrule
                    Affirmative & $\varnothing$ & \cmubdata{-b-} & \cmubdata{-te-} \\
                    Negative & \cmubdata{-ne-} & \cmubdata{-se-} & \\
                    \lspbottomrule
                    \end{tabular}
                \end{table}
    \item \texttt{V} = Voice:
    \begin{itemize}

        \item $\varnothing$ = active;
        \item \cmubdata{-t-} = passive;
        \item \cmubdata{-š-} / \cmubdata{-z-} / \cmubdata{-s-} / \cmubdata{-ʒ-} = causative (\texttr{to have someone do...}). If the stem contains any of these four sounds, the same sound is used here. Otherwise, \cmubdata{-s-} is used.
    \end{itemize}
    \item \texttt{R} = stem; the stem has two suppletive forms: one for active and another one for passive and causative.
    \end{itemize}
\end{itemize}
\end{practiceproblemsolution}

\begin{practiceproblemsolution}{7.10. \langnameSandawe}

\begin{solutions}[label=Solution 7.10\alph*]
    \item
    \begin{enumerate}[start = 15]

        \item \texttr{Apparently, a cook\masc\ skinned a rooster.}
        \item \texttr{A blacksmith\fem\ brought a cook\fem.}
        \item \texttr{Apparently, bulls didn't hit a leopard\fem.}
        \item \texttr{A cow herded hunters.}
    \end{enumerate}
    \item
    \begin{enumerate}[resume]

        \item \cmubdata{thíméỳsò kókóʔà bálóʔó̥wáá}
        \item \cmubdata{ŋ!àméỳsù théká ǁ'ěʔé̥wáátshúgé}
        \item \cmubdata{théká kòŋkórì bálóóyétshú}
        \item \cmubdata{k'ámbà thíméỳsò xééʔíntshèégé}
        \item \cmubdata{!'ìnéỳ ŋ!àméỳsògéà xééʔíŋ}
    \end{enumerate}
\end{solutions}

\rules
\begin{itemize}
    \item Human nouns receive the suffixes: $\varnothing$ (masc. \textsc{sg}), \cmubdata{-sù} (fem. \textsc{sg}), \cmubdata{-sò} (\textsc{pl});
    \item Sentence structure:
    \begin{itemize}
        \item Affirmative: S + O---(\cmubdata{gé})---X\textsubscript{S} + V---X\textsubscript{O};
        \item Negative: S + O + V---X\textsubscript{O}---Y\textsubscript{S}---(\cmubdata{gé}).
    \end{itemize}
    \item \cmubdata{-gé-} marks \texttr{Apparently} (non-witnessed evidential);
    \item X\textsubscript{S} and Y\textsubscript{S} agree with the subject, while X\textsubscript{O} agrees with the object, as follows:
\begin{table}[H]
    \begin{tabular}{lccc}
        \lsptoprule
        & X\textsubscript{S} & Y\textsubscript{S} & X\textsubscript{O} \\
        \midrule
        \textsc{sg} masc.     & \cmubdata{-à}   & \cmubdata{-tshèé} & \cmubdata{-yé} \\
        \textsc{sg} fem.      & \cmubdata{-sà}  & \cmubdata{-tshú}  & \cmubdata{-sú} \\
        \textsc{pl} human     & \cmubdata{-ʔà}  & \cmubdata{-tshó}  & \cmubdata{-ʔín\footnote{\cmubdata{-ʔín} \rightarrow \cmubdata{-ʔíŋ} / \_ \# (alternatively, in affirmative sentences).}} \\
        \textsc{pl} non-human &  \cmubdata{-ʔà} & \cmubdata{-tshó}  & \cmubdata{-ʔwáá\footnote{\cmubdata{\'{V}\'{V}} + \cmubdata{-ʔwáá} \rightarrow\ \cmubdata{\'{V}ʔ\'{V̥}wáá} and \cmubdata{\`{V}\'{V}} + \cmubdata{-ʔwáá} \rightarrow\ \cmubdata{\v{V}ʔ\'{V̥}wáá}.}} \\
        \lspbottomrule
    \end{tabular}
\end{table}
\end{itemize}

The morpheme \cmubdata{-ʔwáá} attracts tone change if V\textsubscript{2} has a high tone. In this case, V\textsubscript{2} will get devoiced and, if V\textsubscript{1} has low tone, it will become rising.
\end{practiceproblemsolution}

\begin{practiceproblemsolution}{7.11. \langnameBurushaski}
\begin{solutions}[label=Solution 7.11\alph*]
    \item
    \begin{enumerate}[start = 13]
        \item \texttr{Those rams will fall.}
        \item \texttr{The gods will deceive the king.}
        \item \texttr{The spider will see this woman.}
        \item \texttr{Which woman will catch the frogs?}
    \end{enumerate}
    \item
    \begin{enumerate}[resume]
        \item \cmubdata{ue biṭayo uγurcan.}
        \item \cmubdata{gušiŋance amic γaṣepišo uγarkan?}
        \item \cmubdata{thamišue guce hoolalašo uyeecan.}
        \item \cmubdata{amin šugulue hilešo usarki?}
        \item \cmubdata{ine hilese šuguli muyarani.}
        \item \cmubdata{γeniṣe ine dasin musarko.}
        \item \cmubdata{khine dasine šugulomuc uyeeco.}
        \item \cmubdata{ṣiqare ise γurqun iyarani.}
        \item \cmubdata{amis j̣akun iwari?}
    \end{enumerate}
\end{solutions}

\rules
\begin{itemize}
    \item Word order: SOV, Modifier -- Noun
    \item Modifiers:
    \begin{table}[H]
    \begin{tabular}{ *5{l} }
    \lsptoprule
    & \multicolumn{2}{c}{Human} & \multicolumn{2}{c}{Non-human}\\\cmidrule(lr){2-3}\cmidrule(lr){4-5}
    & \textsc{sg}  & \textsc{pl}      & \textsc{sg}    & \textsc{pl} \\\midrule
    \texttr{this}  & \cmubdata{khine} & \cmubdata{khue}& \cmubdata{guse}& \cmubdata{guce}\\
    \texttr{that}  & \cmubdata{ine}   & \cmubdata{ue}  & \cmubdata{ise}& \cmubdata{ice}\\
    \texttr{which} & \cmubdata{amin}  & \cmubdata{}    & \cmubdata{amis}& \cmubdata{amic}\\
    \lspbottomrule
    \end{tabular}
    \end{table}
    \item Noun plural:
    \begin{enumerate}

        \item Masculine (human) and non-human nouns:
        \begin{itemize}
            \item if singular ends in \cmubdata{n}: \cmubdata{-n \rightarrow\ -yo};
            \item if singular ends in \cmubdata{s}: \cmubdata{-s \rightarrow\ -šo};
            \item if singular ends in another consonant: \cmubdata{C \rightarrow\ -Cišo};
            \item if singular ends in a vowel: \cmubdata{-V \rightarrow\ -Vmuc};
        \end{itemize}
         \item Feminine (human):
        \begin{itemize}
            \item if singular ends in a vowel: \cmubdata{-V \rightarrow\ -Vmuc};
            \item if singular ends in a consonant --  nouns behave irregularly (they receive the suffix \cmubdata{-anc}, but some consonant alterations may occur). Nevertheless, the problem does not require us to infer any plural form from this category;
        \end{itemize}
    \end{enumerate}
    \item Ergative marker: \cmubdata{-e} added after the plural marker; \cmubdata{o \rightarrow\ u / \_ e};
    \item Verb: receives a prefix and a suffix. The prefix agrees with the absolutive argument of the verb, while the suffix agrees with the nominative argument of the verb, as follows:

    \begin{table}[H]
        \begin{tabular}{l ccc}
        \lsptoprule
        & non-human / &            &  \\
        & masc. human & fem. human & plural\\
        \midrule
        prefix & \cmubdata{i-} & \cmubdata{mu-} & \cmubdata{u-}\\
        suffix & \cmubdata{-i} & \cmubdata{-o} & \cmubdata{-an}\\
        \lspbottomrule
        \end{tabular}
    \end{table}
\end{itemize}
\end{practiceproblemsolution}

% %     \section{Further reading}
% \begin{enumerate}[{label=[\arabic{*}]}]
%     \item Carnie, Andrew. “Modern syntax: a coursebook.”\ \textit{Cambridge University Press}, Cambridge (2011).
%     \item Cinque, Guglielmo and Kayne, Richard S. (ed.). “The Oxford handbook of comparative syntax.”\ \textit{Oxford University Press}, Oxford (2008).
%     \item Coon, Jessica. “Aspects of split ergativity.”\ \textit{Oxford University Press}, Oxford (2013).
%     \item Coon, Jessica and Massam, Diane and Travis, Lisa Demena (ed.). “The Oxford handbook of ergativity.”\ \textit{Oxford University Press}, Oxford (2017).
%     \item Dixon, Robert M. W. “Ergativity.”\ \textit{Cambridge University Press}, Cambridge (1994).
% \end{enumerate}
\nocite{Carnie2011, CinqueKayne2008, Coon2013, CoonEtAl2017, Dixon1994}
% \printbibliography[heading=FurtherReading]
\FurtherReadingBox{}
\end{refsection}
