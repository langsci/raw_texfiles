\begin{refsection}
\hypertarget{noun}{%
\chapter{Noun and noun phrase}\label{chap-noun}}

\section{Introduction}

 Before presenting and analysing the main types of problems, it is important to get accustomed to the following general concepts. Traditional notional definitions of grammatical categories include the following:

\begin{description}
    \item[Noun:] words that name specific objects, places, beings, etc.
    \item[Proper noun:] refers to names of people or places (\cmubdata{Giovanni}, \cmubdata{London}, etc.)
    \item[Common noun:] describes a class of entities (\cmubdata{city}, \cmubdata{class}, \cmubdata{girl}, \cmubdata{cat}, etc.).
\end{description}

 We can also think about the ways in which words combine to form larger constituents and the roles these larger constituents play in syntactic structures. For instance:

\begin{description}
    \item[Noun phrase:] a phrase that has a noun as its head and performs various grammatical functions such as subject and direct object
\end{description}

 Thus, we can define a sentence (S) as a combination of a noun phrase (NP) and a verb phrase (VP). We can write: $S = NP + VP$. For example:

\ea
    \ea \cmubdata{Andrew eats.} (NP = Proper noun = \cmubdata{Andrew}, VP = V = \cmubdata{eats})
    \ex \cmubdata{A boy eats those sandwiches.} (NP = \cmubdata{A boy}, VP = V + NP = \cmubdata{eats} + \cmubdata{those sandwiches}).
    \z
\z

Other examples of noun phrases (the word in bold represents the head noun) can be: \cmubdata{the good \textbf{boy}}, \cmubdata{the \textbf{boy} inside the house}, \cmubdata{the five green \textbf{tables}}, etc.

We notice that the noun phrase includes the noun (the \textit{head} component) and other words subordinated to it (definite article, possessives, adjectives, etc.). We usually refer to all of these subordinated components as \textit{modifiers}.\footnote{Sometimes the term \textit{determiner} is also used in linguistics problems. In current theories, however, the term determiner refers strictly to the definite/indefinite article (the grammatical category of determination). Thus, the term modifier is usually preferred to refer to the elements of the noun phrase (adjectives, demonstratives, etc.).}

\begin{tblsfilledsymbol}{Note}{bulbon}
Some words can belong, depending on the sentence structure, to either the noun phrase or the verb phrase. For example, the sentences \cmubdata{The boy inside the house eats} and \cmubdata{The boy eats inside the house} differ only in terms of word order. In the former, \cmubdata{inside the house} is part of the noun phrase and modifies \cmubdata{boy} (the boy who is inside the house), while in the latter it refers to where the action takes place and modifies the verb, as part of the verb phrase.
    \end{tblsfilledsymbol}

\begin{description}
    \item[Morpheme:] the smallest linguistic unit with a meaning or a grammatical function. This is a broad term which includes different subtypes, such as stems (usually, the lexical stem of the word), prefixes, suffixes, etc. For example, the word \cmubdata{undesirability} is made up of four morphemes: the morpheme \cmubdata{desire} (which is also the stem), the morpheme \cmubdata{-able} (forming the adjective \cmubdata{desirable}), the morpheme \cmubdata{un-} (forming the adjective \cmubdata{undesirable}), and finally, the morpheme \cmubdata{-ity} (forming the noun \cmubdata{undesirability}).
    \item[Allomorph:] a variant of a morpheme. In some situations, a morpheme can have two or more similar forms, which are chosen based on phonological considerations. For example, in the words \cmubdata{impossible} and \cmubdata{intolerant}, the prefixes \cmubdata{im-} and \cmubdata{in-}, although they look different, serve the same function, for which reason we can consider them to be the same morpheme. Moreover, from a phonological point of view, \cmubdata{im-} appears when the nasal assimilates to the place of articulation of the following sound (\cmubdata{p} in \cmubdata{\textbf{p}ossible}).
    \item[Affix:]  a morpheme added to the stem. It is also a rather broad term which includes different types of affixes, like suffix or prefix.
    \item[Prefix:] affix placed at the beginning of the word.
    \item[Suffix:] affix placed at the end of the word.
\end{description}

 Although English has only two types of affixes (prefixes and suffixes), other languages can have a much greater variety, as follows:
\begin{description}
    \item[Infix:] affix placed inside the stem.
\end{description}

 Let us consider the following examples from Bontoc:

\begin{center}
    \begin{tabular}{ll@{\hskip3em}ll}
         \cmubdata{fikas} & \texttr{strong} & \cmubdata{fumikas} & \texttr{to be strong} \\
         \cmubdata{kilad} & \texttr{red} & \cmubdata{kumilad} & \texttr{to be red} \\
         \cmubdata{pusi} & \texttr{poor} & \cmubdata{pumusi} & \texttr{to be poor} \\
    \end{tabular}
\end{center}

 We can notice that the derivation of a verb from an adjective (e.g., \texttr{red} \rightarrow\ \texttr{to be red}) is made by adding the letters \cmubdata{um} after the first letter of the adjective. Thus, \cmubdata{-um-} is an example of an infix.

\begin{description}
    \item[Circumfix:] is a type of affix with two components, one part that is added to the beginning of the word and another that is added to the end of the word.
\end{description}

 To some extent, a circumfix can be considered as a combination of a prefix and a suffix. Nevertheless, there is a major difference between a circumfix and a prefix + suffix combination: the circumfix is indivisible (i.e., the part in the beginning has neither meaning nor function without the part at the end; both parts are needed in order to achieve a meaning or function), while a prefix + suffix combination represent two independent entities, each of them having their own function or meaning. An example of a circumfix can be found in Chickasaw, in which the negation is formed using the circumfix \cmubdata{ik- -o} added to the stem, with the additional property that the final vowel of the stem is dropped:
\begin{center}
    \begin{tabular}{ll@{\hskip3em}ll}
         \cmubdata{chokma} & \texttr{he is good} & \cmubdata{ikchokmo} & \texttr{he is not good} \\
         \cmubdata{tiwwi} & \texttr{he opens} & \cmubdata{iktiwwo} & \texttr{he does not open} \\
         \cmubdata{palli} & \texttr{it is hot} & \cmubdata{ikpallo} & \texttr{it is not hot} \\
    \end{tabular}
\end{center}
			
\begin{description}
    \item[Transfix:] is a type of discontinuous affix (resembling a combination of different infixes).
\end{description}

 Transfixes are generally associated with Semitic languages (Arabic, Hebrew, Maltese, etc.) in which, for example, the verb stems are discontinuous, consisting of two to four consonants. Thus, in Maltese, the stem of the verb \texttr{to write} is \cmubdata{k-t-b}. In order to conjugate it, we need to add different vowels before, between or after the consonants of the stem. For example, \cmubdata{kiteb} = \texttr{he wrote}. So, in order to obtain the 3\textsc{sg} masculine past simple of the verb, we need to use the transfix \cmubdata{-i-e-}.

\begin{description}
    \item[] Sometimes, part of the stem is deleted. We will refer to the part of the word that is deleted as a \OlympiadNewTerm{disfix}.
\end{description}

 Let us consider the following example in Alabama:

\exrule{\cmubdata{tipsali}\quad\texttr{to break}\hspace{5em}\cmubdata{tipli}\quad\texttr{to break up}}

 In this case, the second word is formed from the first through the elision of the penultimate syllable of the stem (\cmubdata{sa}). Based on this example, we could in fact consider that the second word is the stem and \cmubdata{sa} is merely an infix added to form the verb to break. Let us consider another example from the same language: \cmubdata{batatli} – \cmubdata{batli}. The formation of the second word can be easily explained using our first theory, in which the penultimate syllable is dropped; nevertheless, if we try to use our second explanation, in which an infix is added, we notice that in the first example, the infix added is \cmubdata{-sa-}, while the second example uses the infix \cmubdata{ta}. Moreover, there seems to be no connection or pattern regarding the choice of the infix. As a result, the disfix is the reasonable explanation, since the deleted morpheme is not predictable (cannot be deduced), but rather it is intrinsic to the stem.

 Most of the linguistics problems focused on the noun phrase will include different ways of forming new words (either singular – plural, etc.) or will focus on how nouns interact with different modifiers.

\section{Basic principle of morphological analysis}
\begin{enumerate}
    \item If a single phonetic form has two distinct meanings (or functions), it must be analysed as representing two different morphemes.
\end{enumerate}

 For example, the morpheme \cmubdata{-s} in the words \cmubdata{cats} and \cmubdata{sees} could be considered, at first sight, to be a single morpheme, since it looks the same. Nevertheless, in each of the two words it serves different functions (in the first case it forms the plural of the noun, while in the second it forms the 3\textsc{sg} present tense). Therefore it needs to be treated as two different morphemes (the morpheme \cmubdata{-s} used to form the plural of the nouns, and the morpheme \cmubdata{-s} used for the 3\textsc{sg} present tense).

\begin{enumerate}[start = 2]
    \item If the same function (or meaning) is associated with two or more phonetic forms, these different forms all represent the same morpheme and the choice of form in each case is usually predictable based on phonological or other considerations.
\end{enumerate}

For example, the prefixes \cmubdata{il-}, \cmubdata{im-}, \cmubdata{in-}, \cmubdata{ir-} (\cmubdata{\textbf{il}legal}, \cmubdata{\textbf{im}possible}, \cmubdata{\textbf{in}tolerant}, \cmubdata{\textbf{ir}\-re\-spon\-si\-ble}) serve the same function, so we can consider them to be allomorphs of the same morpheme. Moreover, the choice of the allomorph can be predicted based on phonological considerations: if the stem starts with a liquid, the prefix will be \cmubdata{iX-}, where \cmubdata{X} is the first consonant of the stem; if the stem starts with a stop, the prefix will be \cmubdata{iN-}, where \cmubdata{N} is a nasal consonant that assimilates to the place of articulation from the first consonant in the stem.


\begin{problem}{\langnameZulu}{\nameVNeacsu}{\wordoriginal}

\IntroWords{\langnameZulu} \IntroAndEnglish:

\begin{table}[H]
\begin{tabularx}{\textwidth}{@{} *3{l@{~~}Q} @{}}
\cmubdata{umdwebi}  & \texttr{painter}  &  \cmubdata{abazingeli}  & \texttr{hunters} &   \cmubdata{umbulali}  & \texttr{killer}\\
\cmubdata{abadwebi} & \texttr{painters} &  \cmubdata{zingela}     & \texttr{to hunt} &   \cmubdata{ababazi}   & \texttr{carvers}\\ 
\end{tabularx}
\end{table}

\begin{assgts}
\item \transinen[\langnameZulu] \texttr{to paint}, \texttr{hunter}, \texttr{killers}, \texttr{to kill}, \texttr{carver}, \texttr{to carve}.
\end{assgts}
\end{problem}
\begin{mysolution}

 We notice that we have three types of words: singular nouns, plural nouns and verbs which are semantically related to the nouns. Therefore, we can create the following table:

\begin{center}
    \begin{tabular}{llll}
        \lsptoprule
        & Noun \textsc{sg}  & Noun \textsc{pl} & Verb \\ 
        \midrule
        \texttr{painter} & \cmubdata{umdwebi}& \cmubdata{abadwebi} & \\
        \texttr{hunter} & & \cmubdata{abazingeli} & \cmubdata{zingela} \\
        \texttr{killer} & \cmubdata{umbulali} & & \\ 
        \texttr{carver} & & \cmubdata{ababazi} & \\
        \lspbottomrule
    \end{tabular}
\end{center}

From the table, we can easily deduce that the singular noun is formed using the circumfix \cmubdata{um- -i}, while the plural is formed with the circumfix \cmubdata{aba- -i}. The verb is formed by adding the suffix \cmubdata{-a}.

Another explanation, in order to avoid the idea of a circumfix, is that the singular and plural are formed using the prefixes \cmubdata{um-} and \cmubdata{aba-}, respectively, while the verb is formed by replacing the final vowel (\cmubdata{i}) with the vowel \cmubdata{a}.

Thus, the answers are:

\begin{solutions}
\item
    \begin{tabular}[t]{llll}
        \lsptoprule
        & Noun \textsc{sg}  & Noun \textsc{pl} & Verb \\ 
        \midrule
        \texttr{painter} & \cmubdata{umdwebi}& \cmubdata{abadwebi} & \cmubdata{\textbf{dweba}} \\
        \texttr{hunter} &\cmubdata{\textbf{umzingeli}} & \cmubdata{abazingeli} & \cmubdata{zingela} \\
        \texttr{killer} & \cmubdata{umbulali} &\cmubdata{\textbf{ababulali}} &\cmubdata{\textbf{bulala}} \\
        \texttr{carver} &\cmubdata{\textbf{umbazi}} & \cmubdata{ababazi} &\cmubdata{\textbf{baza}} \\
        \lspbottomrule
    \end{tabular}
\end{solutions}
\end{mysolution}

\begin{problem}{\langnameSwedish}{\nameVNeacsu}{\wordoriginal}
\IntroWords{\langnameSwedish}\ \IntroAndEnglish:

\begin{table}[H]
\begin{tabularx}{\textwidth}{@{} l@{~~}Q l@{~~}Q l@{~~}l @{}}
     \pbsv{en flaska}{a bottle} & \pbsv{hunden}{the dog} & \pbsv{hyllor}{shelves} \\
     \pbsv{en stol}{a chair} & \pbsv{flaskorna}{the bottles} & \pbsv{kattar}{cats} \\
     \pbsv{en hund}{a dog} & \pbsv{stolarna}{the chairs} & \pbsv{bilen}{the car} \\
     \pbsv{flaskor}{bottles} & \pbsv{hundarna}{the dogs} & \pbsv{hyllan}{the shelf} \\
     \pbsv{stolar}{chairs} & \pbsv{en bil}{a car} & \pbsv{katten}{the cat} \\
     \pbsv{hundar}{dogs} & \pbsv{en hylla}{a shelf} & \pbsv{bilarna}{the cars} \\
     \pbsv{flaskan}{the bottle} & \pbsv{en katt}{a cat} & \pbsv{hyllorna}{the shelves} \\
     \pbsv{stolen}{the chair} & \pbsv{bilar}{cars} & \pbsv{kattarna}{the cats} \\
\end{tabularx}
\end{table}

\begin{assgts}
\item Here are some more words in Swedish and their English translations:
\begin{center}
\cmubdata{en flicka} = \texttr{a girl} \qquad\quad\quad \cmubdata{bussarna} = \texttr{the buses}
\end{center}
\item[] \transinen[\langnameSwedish] \texttr{the girl}, \texttr{girls}, \texttr{the girls}, \texttr{a bus}, \texttr{the bus}, \texttr{buses}.
\end{assgts}
\end{problem}
\begin{mysolution}

 Just as in the previous problem, we first notice the forms of the given nouns. We realise that each noun is given in four different forms: definite singular, indefinite singular, definite plural, and indefinite plural. Therefore, in order to facilitate the analysis of the data and notice the similarities between them, we make the following table:

\begin{table}[H]
\begin{tabular}{ *5{l} }
\lsptoprule
Indef. \textsc{sg} & Def. \textsc{sg} & Indef. \textsc{pl} & Def. \textsc{pl} & Translation \\ \midrule
\cmubdata{en flaska} & \cmubdata{flaskan} & \cmubdata{flaskor} & \cmubdata{flaskorna} & \texttr{bottle} \\
\cmubdata{en stol} & \cmubdata{stolen} & \cmubdata{stolar} & \cmubdata{stolarna} & \texttr{chair} \\
\cmubdata{en hund} & \cmubdata{hunden} & \cmubdata{hundar} & \cmubdata{hundarna} & \texttr{dog} \\
\cmubdata{en bil} & \cmubdata{bilen} & \cmubdata{bilar} & \cmubdata{bilarna} & \texttr{car} \\
\cmubdata{en hylla} & \cmubdata{hyllan} & \cmubdata{hyllor} & \cmubdata{hyllorna} & \texttr{shelf} \\
\cmubdata{en katt} & \cmubdata{katten} & \cmubdata{kattar} & \cmubdata{kattarna} & \texttr{cat} \\
\lspbottomrule
\end{tabular}
\end{table}

Based on this table, we can consider the indefinite singular form as the stem (excluding the proclitic\footnote{The term \textit{proclitic} refers to the fact that the definite article is placed in front of the noun. This contrasts with the term \textit{enclitic}, meaning it is placed after the noun.} article \cmubdata{en}). Moreover, we notice that the definite plural is derived from the indefinite plural by adding the suffix \cmubdata{-na}. We are left to discover how to form the definite singular and the indefinite plural.

We notice that the def. \textsc{sg} is formed using the suffixes \cmubdata{-n} or \cmubdata{-en}. Therefore, we need to discover in which context each of them is used. It could be either a semantic context, in which the variation is driven by the meaning of the word, or a phonological context, in which the choice of the allomorph is dictated by the phonological structure of the word. In this case, we can easily notice that the suffix \cmubdata{-en} is used if the base form ends in a consonant, while \cmubdata{-n} is used if the base form ends in a vowel, thus the distinction is purely phonological. Another explanation for the definite singular endings can include a phonological process, an elision, by considering the suffix \cmubdata{-en} as the sole suffix for def. \textsc{sg}, with the additional feature that \cmubdata{en \rightarrow\ n / V \_}.

Applying the same thought process for the indefinite plural, we notice that the two suffixes are \cmubdata{-ar} and \cmubdata{-or}, the first being used if the stem ends in a consonant and the latter if the stem ends in a vowel. Moreover, if the stem ends in a vowel, the vowel is dropped when the suffix \cmubdata{-or} is added (alternatively: the suffix is \cmubdata{-ar} and \cmubdata{-V $+$ -ar \rightarrow\ -or} -- i.e., when the suffix \cmubdata{-ar} is added after a vowel, it merges with it and results in the suffix \cmubdata{-or}). Thus, we can write the rules that govern the formation of these noun forms in Swedish (we used the abbreviations $S$ = stem, $C$ = consonant, $V$ = vowel):

\begin{table}[H]
\begin{tabular}{ llll }
\lsptoprule
    Indef. \textsc{sg} & Def. \textsc{sg} & Indef. \textsc{pl} & Def. \textsc{pl} \\ \midrule
    \cmubdata{en $S$-$C$} & \cmubdata{$S$-$C$-en} & \cmubdata{$S$-$C$-ar} & \cmubdata{$S$-$C$-arna}  \\
    \cmubdata{en $S$-$V$} & \cmubdata{$S$-$V$-n} & \cmubdata{$S$-or} & \cmubdata{$S$-orna}  \\
\lspbottomrule
\end{tabular}
\end{table}

 Thus, the answers to the task are:

\begin{solutions}
\item
\begin{itemize}[leftmargin=0pt]
    \begin{multicols}{2}
    \item[] \texttr{the girl} = \cmubdata{flickan}
    \item[] \texttr{girls} = \cmubdata{flickor}
    \item[] \texttr{the girls} = \cmubdata{flickorna}
    \item[] \texttr{a bus} = \cmubdata{en buss}
    \item[] \texttr{the bus} = \cmubdata{bussen}
    \item[] \texttr{buses} = \cmubdata{bussar}
    \end{multicols}
\end{itemize}
\end{solutions}
\end{mysolution}

\begin{problem}{Māori}{\nameVNeacsu}{\wordoriginal}
Consider the following word forms in Māori:

\begin{table}[H]
\begin{tabular}{ lll }
    \lsptoprule
     Form I & Form II & Form III \\\midrule
     \pbpbpb{inu}{inumia}{inumaŋa}
     \pbpbpb{hopu}{hopukia}{hopukaŋa}
     \pbpbpb{eke}{ekeŋia}{ekeŋaŋa}
     \pbpbpb{ɸera}{ɸerahia}{ɸerahaŋa}
     \pbpbpb{aɸi}{aɸitia}{aɸitaŋa}
     \pbpbpb{tupu}{tupuria}{tupuraŋa}
     \lspbottomrule
\end{tabular}
\end{table}

\begin{assgts}
\item Explain how the forms are constructed.
\end{assgts}
\end{problem}
\begin{mysolution}

We can easily observe that, in order to obtain Form III from Form II we just replace the suffix \cmubdata{-ia} with \cmubdata{-aŋa}. Moreover, we notice that Form II derives from Form I by adding the suffix \cmubdata{-Cia}, where \cmubdata{C} is a consonant. The only thing left to do is figure out how the consonant is chosen.

The first thing we notice is that there are no two examples which use the same consonant; furthermore, the consonant does not seem to be related in any way to the structure of the word (to the phonological characteristics of the other sounds). Finally, we are sure that the choice of the consonant cannot be related to the meaning since the translations are not given in the data. Therefore, since the consonant does not seem to follow any pattern, we can consider it as being part of the stem, thus being a disfix.

If we consider that consonant as a disfix, we can easily figure out all the rules that generate the three forms:

\begin{solutions}
\item
\begin{description}[font=\normalfont]
    \item[Form I:] elision of the last consonant of the stem;
    \item[Form II:] add suffix \cmubdata{-ia};
    \item[Form III:] add suffix \cmubdata{-aŋa}.
\end{description}
\end{solutions}
\end{mysolution}
\note{Linguistics problems featuring disfixes are extremely rare. Before considering the possibility of a disfix, make sure there is absolutely no correlation between that sound/morpheme and the shape or meaning of the stem.}

\section{Variables of the noun}\label{sec:variables-of-the-noun}

When we talk about the \textit{variables}\footnote{By using the term noun \textit{variables}, we refer to the grammatical categories specific to English (number, case, etc.), as well as to other distinctions which might be relevant in other languages (usually, of a semantic nature).

Although all the grammatical categories of the noun are considered variables of the noun, not all variables are grammatical categories. For example, the distinction $\pm$human or $\pm$animate, to which we will refer in the following sections, is an important variable of the noun, but it cannot be considered a grammatical category.} of the noun, we mean those features of the noun which can change in a linguistics problem and which it is best to identify when starting to solve the problem, in order to form an idea regarding what types of morphemes we are looking for. These variables can also be marked on other elements in the phrase/clause, e.g., on adjectives. This phenomenon is known as \OlympiadNewTerm{agreement}.

The three common variables are: \OlympiadNewTerm{number} (singular, dual, plural), \OlympiadNewTerm{gender} (masculine, neuter, feminine etc.) which is a highly grammaticalised subtype of noun \OlympiadNewTerm{class}. The way in which we usually realise that gender is relevant in a linguistics problem is by identifying different allomorphs. Thus, if we notice that a set of allomorphs appears solely with some stems and another set with other stems, we can assume in that language there are two classes of nouns (two genders), each of them having their own characteristic morphemes. Class and classifiers will be further discussed in the next section.

The \emph{number} variable denotes how many items the noun refers to. The simplest distinction is singular (referring to one item) and plural (referring to more than one item). However, different languages might have other distinctions, among which the most common are dual (two items), trial (three items), and paucal (referring to a relatively small number of items. This would typically be translated into English as \texttr{a few} and usually refers to less than ten items). Of course, there are also languages which do not mark nouns for number.

An interesting case regarding noun plurals is encountered in Dagaare.\footnote{This phenomenon was featured in a problem by Ethan Chi (NACLO 2021).} This language features an \textit{inherent plurality} of the designated noun and, starting from it, it differentiates an unmarked form (a base form) and a marked form. Thus, if a noun represents an entity that is usually found alone (\texttr{forehead}, \texttr{hat}), the unmarked form will be represented by the singular (while the plural will be the marked form). However, if the noun is usually found in pairs (\texttr{leg}, \texttr{lung}, \texttr{shoe}) or groups (\texttr{bee}), then the unmarked form is the plural (and the singular is denoted by the marked form).

There is one more variable of the noun that we have not mentioned yet: \OlympiadNewTerm{grammatical case}. Case is an important feature of nouns and indicates the role the noun plays in the sentence. In some languages, nouns can be marked for many cases; for example, Uralic languages, which are known for their large number of cases, can have more than 20 cases -- nominative, genitive, partitive, accusative, inessive, elative, illative, adessive, ablative, additive, egressive, comitative, terminative, abessive, translative, allative, essive, instructive, instrumental, dative, causal, sublative, superessive, delative, temporal, sociative. Most of these cases will be translated into English using different prepositions, for example, the instrumental case (which shows the object with which the action is performed) is, usually, translated using the preposition \texttr{with} or \texttr{by} in English (\cmubdata{I write \textbf{with a pen}} or \cmubdata{I sew \textbf{by hand}}). When solving a linguistics problem, assuming the instrumental is marked by the suffix \cmubdata{-ok} in the target language, we do not necessarily have to write “Instrumental = \cmubdata{-ok}", but rather we can simply write; “\texttr{with $X$} = $X$\cmubdata{-ok}", without using the name of the case.

Moreover, most of the cases of the Uralic languages are locatives (they show the location). The elative case can be translated using \texttr{from $+M$}, illative = \texttr{in $+M$}, allative = \texttr{on $+M$}, adessive = \texttr{on $-M$}, etc, where $M$ denotes movement. Thus, the cases $+M$ are cases in which the object is moved in a direction (relative to the noun), while the $-M$ cases are those in which the objects are already in that location.

Let us consider, for example, the allative and adessive cases (both of them can be expressed in English using the preposition \texttr{on}, but they differ in terms of the parameter $\pm M$). As such, in the sentence \cmubdata{The apple is on the table}, \cmubdata{on} links to the concept of adessive, since it shows a location without any movement involved. However, in the sentence \cmubdata{He put the apple on the table}, the apple is moved towards the table: in the beginning, the apple was not on the table, but at the end of the action it was; therefore, in this case, \cmubdata{on} links to the concept of allative. Notice that, because English has very little in the way of inflectional morphology, the equivalents of nouns that appear in the adessive or allative cases in other languages such as Finnish appear as the complements of prepositions in English. In other words, what Finnish expresses via the morphological structure of the noun, English expresses in its syntactic structure (i.e. as a prepositional phrase).

Because case is relevant in both morphology and syntax, we will discuss specific examples when they arise. For example, nominative, accusative, ergative, and absolutive, which we will discuss in \sectref{morphoalign}.

\begin{problem}{\langnameBulgarian}{\nameKLow\ \& \nameMVasev}{\LOYear{\NACLOAbbr}{2017}}
\IntroSentences{\langnameBulgarian}\ (written in Latin script) \IntroAndEnglishRandom:

\begin{center}
\begin{tabular}{rlcl}
    \chaosline{Veshterǎt nahrani maymunata.}{Your son watched you.}
    \chaosline{Kamilata vǎrvya.}{The girl hugged the cat.}
    \chaosline{Momicheto pregǎrna kotkata.}{You dressed yourself.}
    \chaosline{Veshtitsata prokle kotkata.}{The cat scratched you.}
    \chaosline{Kotkata prokle tvoya sin.}{You fed the son.}
    \chaosline{Ti nahrani sina.}{The witch cursed the cat.}
    \chaosline{Kotkata te odraska.}{The camel walked.}
    \chaosline{Ti skochi.}{The cat cursed your son.}
    \chaosline{Tvoyat sin te gleda.}{The wizard fed the monkey.}
    \chaosline{Veshterǎt pregǎrna edna kamila.}{The son dressed your baby.}
    \chaosline{Ti se obleche.}{You jumped.}
    \chaosline{Sinǎt obleche tvoeto bebe.}{The wizard hugged a camel.}
\end{tabular}
\end{center}

\begin{assgts}
\item \detcorr
\item \transinen
\begin{enumerate}[start = 13]
    \item \cmubdata{Maymunata gleda tvoyata veshtitsa.}
    \item \cmubdata{Tvoyata kamila obleche edno momiche.}
    \item \cmubdata{Veshterǎt se prokle.}
    \item \cmubdata{Ti pregǎrna bebeto.}
    \item \cmubdata{Ti vǎrvya.}
    \item \cmubdata{Ti prokle edin veshter.}
\end{enumerate}
\item \transinen[\langnameBulgarian]
\begin{enumerate}[start = 19]
    \item \texttr{The witch dressed you.}
    \item \texttr{The baby watched the girl.}
    \item \texttr{The monkey jumped.}
    \item \texttr{You hugged a son.}
    \item \texttr{Your son dressed a baby.}
\end{enumerate}
\end{assgts}

\begin{tblsWarning}
\cmubdata{ǎ} $\thickapprox$ \texttr{u} in \texttr{but}.
\end{tblsWarning}
\end{problem}
\begin{mysolution}

 There are multiple possible starting points when approaching this problem, but one of the most common (and generally applicable) is to notice that there are only two sentences in Bulgarian which have two words (2 and 8). Therefore, we can assume that these are the simplest sentences and, most likely, correspond to the shortest sentences in English, i.e., those that have only a subject and a verb. Among all the English sentences, the only ones that follow this pattern are K and G. We have different ways to figure out which is which: we can use the similarity between the Bulgarian and English words (\cmubdata{kamilata} – \texttr{camel}) or we can notice that \cmubdata{ti} appears in two other sentences and \cmubdata{kamilata} does not occur in any other sentence, while in English we have got two other sentences that begin with \texttr{you}, but none that begin with \texttr{camel}. Since neither of the two verbs ever occurs again in the data and the first word does, we discover that this one is the subject and the last word is the verb. Hence, we deduce that 2-G and 8-K and the word order is S-V (Subject-Verb).

From here, we can continue with the other two sentences which contain the subject \wordtrans{ti}{you}, which are 6, 11 and C, E. In order to match them, we can notice that the last word in sentence 6 occurs (in a slightly changed form) as the first word of sentence 12, so we can deduce it is a noun (since it is a subject in sentence 12), thus it cannot mean \texttr{myself}. Therefore, 6-E and 11-C. Moreover, we understand that \wordtrans{sina}{son}, \wordtrans{nahrani}{to feed}, \wordtrans{obleche}{to dress}. Since \cmubdata{obleche} and \cmubdata{nahrani} occur one more time in another sentence, it follows that 1-I and 12-J (we can notice again the similar words \cmubdata{bebe} – \texttr{baby}). Based on this information, we can easily match all the other sentences. We get:

\renewcommand \chaosline [3]{\addtocounter{exx}{1}\arabic{exx}.&\cmubdata{#1}&#2.&\texttr{#3} \\[0.1em]}
\begin{center}
\begin{tabular}{rlcl}
\setcounter{exx}{0}
    \chaosline{Veshterǎt nahrani maymunata.}{I}{The wizard fed the monkey.}
    \chaosline{Kamilata vǎrvya.}{G}{The camel walked.}
    \chaosline{Momicheto pregǎrna kotkata.}{B}{The girl hugged the cat.}
    \chaosline{Veshtitsata prokle kotkata.}{F}{The witch cursed the cat.}
    \chaosline{Kotkata prokle tvoya sin.}{H}{The cat cursed your son.}
    \chaosline{Ti nahrani sina.}{E}{You fed the son.}
    \chaosline{Kotkata te odraska.}{D}{The cat scratched you.}
    \chaosline{Ti skochi.}{K}{You jumped.}
    \chaosline{Tvoyat sin te gleda.}{A}{Your son watched you.}
    \chaosline{Veshterǎt pregǎrna edna kamila.}{L}{The wizard hugged a camel.}
    \chaosline{Ti se obleche.}{C}{You dressed yourself.}
    \chaosline{Sinǎt obleche tvoeto bebe.}{J}{The son dressed your baby.}
\end{tabular}
\end{center}
\renewcommand \chaosline [2]{\addtocounter{exx}{1}\arabic{exx}.&\cmubdata{#1}&\addtocounter{exxx}{1}\AlphAlph{\value{exxx}}.&\texttr{#2} \\[0.3em]}

 Based on these correspondences, we can establish the structure of Bulgarian sentences. The word order is SVO (Subject-Verb-Object, if the object is a noun) or SOV (Subject-Object-Verb, if the object is a pronoun). Moreover, the determiner always precedes the noun (Det-Noun).

Next, we notice that the verb is invariable in these examples. The only phenomenon that we have not analysed yet is the structure of the noun phrase. We notice from the examples above that the noun can receive a suffix (equivalent to the definite article in the English examples) which can be \cmubdata{-ăt}, \cmubdata{-ta}, \cmubdata{-to} or \cmubdata{-a} or it can get a modifier which precedes it: indefinite article (\cmubdata{edin}, \cmubdata{edna}, \cmubdata{edno}) or 2\textsc{sg} possessive (\cmubdata{tvoyata}, \cmubdata{tvoyat}, \cmubdata{tvoya}, \cmubdata{tvoeto}).

Therefore, we can make a table with the different forms of each noun. Moreover, in order to get more data, we will also use the examples in task (b), which we can easily translate into English solely based on the word order and dictionary (we do not need to know the rules of noun declension). In order to cover all possibilities, we separate the subject and the object.

\begin{table}[H]
\begin{tabular}{lll}
\lsptoprule
Noun & Subject & Object \\ \midrule
\texttr{wizard} & \cmubdata{-ăt} & \cmubdata{edin} \\
\texttr{monkey} & \cmubdata{-ta} & \cmubdata{-ta} \\
\texttr{camel} & \cmubdata{-ta} / \cmubdata{tvoyata} & \cmubdata{edna} \\
\texttr{girl} & \cmubdata{-to} & \cmubdata{edno} \\
\texttr{cat} & \cmubdata{-ta} & \cmubdata{-ta} \\
\texttr{witch} & \cmubdata{-ta} & \cmubdata{tvoyata} \\
\texttr{son} & \cmubdata{tvoyat} / \cmubdata{-ăt} & \cmubdata{tvoya} / \cmubdata{-a} \\
\texttr{baby} &  & \cmubdata{tvoeto} / \cmubdata{-to} \\
\lspbottomrule
\end{tabular}
\end{table}

From the table, we notice that the definite suffix (the definite article, which is marked as a suffix) has only three forms, and the nouns meaning \texttr{camel} and \texttr{witch}, which use the suffix \cmubdata{-ta}, use the same form of 2\textsc{sg} possessive (\cmubdata{tvoyata}). Therefore, we can assume that there are three noun classes (in reality, they correspond to three genders: feminine, masculine, and neuter): Class 1 (using the definite suffix \cmubdata{-ăt} for the subject; it contains the nouns \texttr{wizard}, \texttr{son}), Class 2 (\cmubdata{-ta}: \texttr{camel}, \texttr{cat}, \texttr{witch}, \texttr{monkey}) and Class 3 (\cmubdata{-to}: \texttr{girl}). Based solely on the definite suffix, we cannot classify the noun \cmubdata{bebe} \texttr{baby}, but we notice that the object uses the same definite suffix. Therefore, we can assume that it also belongs to class 3. Now we can make a new table based on the noun class in order to see how different determiners are formed.

\begin{table}[H]
\begin{tabular}{lllll}
\lsptoprule
\multicolumn{2}{c}{Class} & Def. suff. & Indef. art. & 2\textsc{sg} poss. \\ \midrule
1 & S & \cmubdata{-ăt} &                 & \cmubdata{tvoyat} \\
1 & O & \cmubdata{-a}  & \cmubdata{edin} & \cmubdata{tvoya}  \\\midrule
2 & S & \cmubdata{-ta} &                 & \cmubdata{tvoyata}\\
2 & O & \cmubdata{-ta} & \cmubdata{edna} & \cmubdata{tvoyata}\\\midrule
3 & S & \cmubdata{-to} &                 &                   \\
3 & O & \cmubdata{-to} & \cmubdata{edno} & \cmubdata{tvoeto} \\
\lspbottomrule
\end{tabular}
\end{table}

 We can notice that for Classes 2 and 3 the definite suffix is the same for subject and object, and Class 2 uses the same possessive for subject and object as well. We can assume that for both Class 2 and Class 3 there is no difference between subject and object (thus deducing the Class 3 definite suffix for subject which we need for task 20). Moreover, we can notice some similarities between the definite suffix and the form of the possessive: it seems that the 2\textsc{sg} possessive is formed by adding the form of the definite suffix to the morpheme \cmubdata{tvoya} (with the additional feature that if the definite suffix begins with a vowel, it gets dropped and that Class 3 is an exception, since \cmubdata{tvoya} becomes \cmubdata{tvoe}). Nevertheless, in order to solve the problem, we need not understand how the possessive is formed and the table above is enough. Therefore, we can write the official solution and solve the tasks.

Another important thing to notice is that we have not attempted to find any rules based on which the nouns are split into the three classes. This should be done \textit{only} if new nouns are given in the tasks and we need to classify them into one of the three classes.

\rules
\begin{itemize}
\item Word order: SOV (O = pronoun) or SVO (O = noun), Det-Noun.
\item Noun is divided into three classes: Class 1 (\texttr{wizard}, \texttr{son}), Class 2 (\texttr{camel}, \texttr{cat}, \texttr{witch}, \texttr{monkey}), Class 3 (\texttr{girl}, \texttr{baby}).
\item Determiners:

\begin{table}[H]
    \begin{tabular}{lllll}
    \lsptoprule
    \multicolumn{2}{c}{Class} & Def. suff. & Indef. art. & 2\textsc{sg} poss. \\ \midrule
    1 & S & \cmubdata{-ăt}    &                 & \cmubdata{tvoyat} \\
    1 & O & \cmubdata{-a}     & \cmubdata{edin} & \cmubdata{tvoya} \\ \midrule
    2 & S & \cmubdata{-ta}    &                 & \cmubdata{tvoyata} \\
    2 & O & \cmubdata{-ta}    & \cmubdata{edna} & \cmubdata{tvoyata} \\ \midrule
    3 & S & \cmubdata{-to}    &                 &  \\
    3 & O & \cmubdata{-to}    & \cmubdata{edno} & \cmubdata{tvoeto} \\ 
    \lspbottomrule
\end{tabular}
\end{table}
\end{itemize}

\begin{solutions}
    \item
        \begin{enumerate}[leftmargin = 1em]
        \begin{multicols}{6}

            \item I.
            \item G.
            \item B.
            \item F.
            \item H.
            \item E.
            \item D.
            \item K.
            \item A.
            \item L.
            \item C.
            \item J.
       \end{multicols} \end{enumerate}
    \item \begin{enumerate}[start = 13]

        \item \texttr{The monkey watched your witch.}
        \item \texttr{Your camel dressed a girl.}
        \item \texttr{The wizard cursed himself.}
        \item \texttr{You hugged the baby.}
        \item \texttr{You walked.}
        \item \texttr{You cursed a wizard.}

    \end{enumerate}
    \item \begin{enumerate}[start = 19]

        \item \cmubdata{Veshtitsata te obleche.}
        \item \cmubdata{Bebeto gleda momicheto.}
        \item \cmubdata{Maymunata skochi.}
        \item \cmubdata{Ti pregărna edin sin.}
        \item \cmubdata{Tvoyat sin obleche edno bebe.}
        \item[]
    \end{enumerate}
\end{solutions}
\end{mysolution}

\section{Classifiers}

A classifier is a word (or an affix) which accompanies nouns and serves to classify them (thus talking about noun class). In most cases, the classification is made on semantic considerations. In Chinese, the classifier is mandatory between numeral and noun. Let us consider the following examples from Chinese:

\exrule{\texttr{three dogs} = {\chinesetext{三只狗}} \qquad\texttr{three cats} = {\chinesetext{三只猫}} \qquad \texttr{five cats} = {\chinesetext{五只猫}}}

 We notice that the first character represents the number ({\chinesetext{三}} = 3, {\chinesetext{五}} = 5), while the last one represents the noun ({\chinesetext{狗}} = \texttr{dog}, {\chinesetext{猫}} = \texttr{cat}). The middle word represents a classifier and, in this case, it refers to small animals. Other semantic considerations for the choice of the classifier are, usually, related to the shape: long objects, flat objects, round objects, etc.

\begin{problem}{\langnameJapanese}{\nameVNeacsu}{\LOYear{\RoLOAbbr}{2017}}
\IntroPhrases{\langnameJapanese}\ \IntroAndEnglish:
\begin{center}
\begin{longtable}{ll}
     \pbsv{isha kyūnin}{9 doctors} \\
     \pbsv{gakusei sannin}{3 students} \\
     \pbsv{hon yonsatsu}{4 books} \\
     \pbsv{inu kyūhiki}{9 dogs} \\
     \pbsv{kami hachimai}{8 sheets of paper} \\
     \pbsv{magajin nanasatsu}{7 magazines} \\
     \pbsv{neko nihiki}{2 cats} \\
     \pbsv{purēto yonmai}{4 plates} \\
     \pbsv{ratto gohiki}{5 rats} \\
     \pbsv{uma rokutō}{6 horses} \\
     \pbsv{zō rokutō}{6 elephants} \\
\end{longtable}
\end{center}
\begin{assgts}
\item \transinen \cmubdata{purēto rokumai}, \cmubdata{isha gonin}, \cmubdata{uma yontō}.

\item Here are some more Japanese words:

\begin{table}[H]
\begin{tabular}{@{} l@{~~}l @{\quad\quad} l@{~~}l  @{}}
\cmubdata{mangabon} & \texttr{comic books} & \cmubdata{piza}  & \texttr{pizzas}\\
\cmubdata{kaeru}    & \texttr{frogs}       & \cmubdata{ushi}  & \texttr{cows}  \\
\end{tabular}
\end{table}
\item[] \transinen[\langnameJapanese] \texttr{2 comic books}, \texttr{5 pizzas}, \texttr{7 frogs}, \texttr{9 cows}.
\end{assgts}
\end{problem}
\begin{mysolution}

 Comparing examples \texttr{9 doctors} and \texttr{9 dogs}, we notice that the only part that is repeated is the morpheme \cmubdata{kyū-} from the beginning of the second word. We can deduce that the number is represented by the first part of the second word. Moreover, comparing the last two examples, (\texttr{6 horses}, \texttr{6 elephants}), we notice that only the first word is different, so we can assume that this one represents the noun. Therefore, the phrase structure is \fbox{Noun Number-$X$}.

Based on this structure, we can infer all the numbers and nouns in Japanese, as follows:

\exrule{
Japanese numbers are:

\begin{enumerate}[start = 2, label = \arabic* =]
\begin{multicols}{3}
    \item \cmubdata{ni-}
    \item \cmubdata{san-}
    \item \cmubdata{yon-}
    \item \cmubdata{go-}
    \item \cmubdata{roku-}
    \item \cmubdata{nana-}
    \item \cmubdata{hachi-}
    \item \cmubdata{kyū-}
    \item[] \quad
\end{multicols}\end{enumerate}
}

 Note that in order to figure out the morpheme for \texttr{6}, we need to check the examples in task (a) in order to figure out which part corresponds to the number and which to the particle $X$.

 \exrule{
 Nouns are:

\begin{itemize}
\begin{multicols}{2}
    \item[] \wordtrans{isha}{doctor}
    \item[] \wordtrans{hon}{book}
    \item[] \wordtrans{kami}{sheet of paper}
    \item[] \wordtrans{neko}{cat}
    \item[] \wordtrans{ratto}{rat}
    \item[] \wordtrans{zō}{elephant}
    \item[] \wordtrans{gakusei}{student}
    \item[] \wordtrans{inu}{dog}
    \item[] \wordtrans{magajin}{magazine}
    \item[] \wordtrans{purēto}{plate}
    \item[] \wordtrans{uma}{horse}
    \item[] \quad
\end{multicols}\end{itemize}
}

 The only morpheme left to analyse is $X$. We notice that it can have five different forms: \cmubdata{-nin} (in the phrases \texttr{9 doctors}, \texttr{5 students}), \cmubdata{-satsu} (\texttr{4 books}, \texttr{7 magazines}), \cmubdata{-mai} (\texttr{8 sheets of paper}, \texttr{4 plates}), \cmubdata{-hiki} (\texttr{9 dogs}, \texttr{2 cats}, \texttr{5 rats}), and \cmubdata{-tō} (\texttr{6 horses}, \texttr{6 elephants}). Therefore, we deduce that they represent classifiers and correspond to: \cmubdata{-nin} for humans, \cmubdata{-satsu} for bound materials, \cmubdata{-mai} for flat objects, \cmubdata{-hiki} for small animals and \cmubdata{-tō} for large animals. Now we can write the rules and answer the tasks.

\rules
\begin{itemize}
    \item Structure: Noun Number--Class
    \item Class:
    \begin{itemize}
        \begin{multicols}{2}
            \item \cmubdata{-nin} \rightarrow\ humans
            \item \cmubdata{-satsu} \rightarrow\ prints
            \item \cmubdata{-mai} \rightarrow\ flat objects
            \item \cmubdata{-hiki} \rightarrow\ small animals
            \item \cmubdata{-tō} \rightarrow\ large animals
            \item[] \quad
        \end{multicols}
    \end{itemize}
\end{itemize}

\begin{solutions}
    \item \begin{enumerate}[leftmargin=0pt]
            \item[] \wordtrans{purēto rokumai}{6 plates}
            \item[] \wordtrans{isha gonin}{5 doctors}
            \item[] \wordtrans{uma yontō}{4 horses}
            \end{enumerate}
    \item \begin{enumerate}[leftmargin=0pt]
                \item[] \texttr{2 comic books} = \cmubdata{mangabon nisatsu}
                \item[] \texttr{5 pizzas} = \cmubdata{piza gomai}
                \item[] \texttr{7 frogs} = \cmubdata{kaeru nanahiki}
                \item[] \texttr{9 cows} = \cmubdata{ushi kyūtō}
        \end{enumerate}
\end{solutions}
\end{mysolution}

\section{Reduplication}

 Reduplication represents the formation of new words by partially or totally doubling a morpheme or a part of a word. An example of total reduplication is plural formation in Indonesian, as we can see in the following examples: \cmubdata{rumah} (\texttr{house}) -- \cmubdata{rumahrumah} (\texttr{houses}), \cmubdata{ibu} (\texttr{mother}) -- \cmubdata{ibuibu} (\texttr{mothers}), \cmubdata{lalat} (\texttr{fly}) -- \cmubdata{lalatlalat} (\texttr{flies}).

 We can notice that in order to form the plural we simply repeat the whole word, so we are talking about a process of total reduplication.

In linguistics problems, total reduplication is seldom encountered, since it is easy to spot and analyse; therefore, in most cases, partial reduplication is preferred. The following examples show two such processes:

\begin{center}
    \begin{tabular}{l@{\hskip3em}ll}
	Marshallese: &	\cmubdata{kagir} (\texttr{belt}) & \cmubdata{kagirgir} (\texttr{to wear a belt}) \\
    & \cmubdata{takin} (\texttr{socks}) & \cmubdata{takinkin} (\texttr{to wear socks})\\ 
    &&\\
	Samoan: &  	\cmubdata{savali} (\texttr{he walks}) & \cmubdata{savavali} (\texttr{they walk}) \\ 
 &     	\cmubdata{alofa} (\texttr{he loves})   & \cmubdata{alolofa} (\texttr{they love}) \\ 
    \end{tabular}
\end{center}

 We can easily notice that in the first example the last syllable is reduplicated, while in the second example, the second (or penultimate) syllable is reduplicated. Moreover, we notice that the reduplication process can have different roles, such as forming the plural of the noun (in Indonesian), forming the plural of the verb (transforming a verb from 3\textsc{sg} to 3\textsc{pl} in Samoan) or even transforming a noun into a verb (the pairs $X$ – \texttr{to wear} $X$, in Marshallese).

\note{An alternative explanation for the aforementioned reduplication process can be “repeating the last three letters”\ (in Marshallese). Nevertheless, “the last three letters”\ do not represent a phonological or morphological unit of any sort, so this explanation is not very rigorous from a linguistic perspective.}

\section{Suppletion}

 Another important phenomenon that might be featured in linguistics problems is suppletion. This generally refers to the situation in which one stem (so one morpheme with a semantic component) has two or more completely different forms (in most cases due to different etymologies). Suppletion can be noticed to a lesser extent in most languages, including English (\cmubdata{good} – \cmubdata{better}, \cmubdata{bad} – \cmubdata{worse}, \cmubdata{go} – \cmubdata{went}), French (\cmubdata{être} – \cmubdata{sommes}, \cmubdata{aller} – \cmubdata{vais}), Spanish (\cmubdata{ir} – \cmubdata{voy}), German (\cmubdata{gut} – \cmubdata{besser}), etc. We can notice that, in each of the cases, it is not simply a mutation of the stem (as for example in English: \cmubdata{goose} – \cmubdata{geese} or \cmubdata{mouse} – \cmubdata{mice}), but rather the two forms are completely distinct.

This phenomenon seldom appears in linguistics problems, which makes it extremely hard to notice and, similar to the disfixes, it must be used with caution since there might only be some phonological transformations.

\section{Expressing possession}

 A common way of marking possession in the world's languages is to use the genitive case. This is usually expressed by a morpheme attached to the noun (and, sometimes, its dependents) that functions as the possessor.

 In certain languages, often ones belonging to the Austronesian family, possession can be marked by an affix or an independent word. In these cases, the word used is a classifier (similar to those defined above). Moreover, in a multitude of languages, there is a dichotomy between alienable and inalienable possession. \textit{Inalienable} possession refers to those objects one owns and which cannot be borrowed/lent or taken (in most cases, they include body parts and family members), while \textit{alienable} possession refers to all the other cases. In most languages where this distinction is relevant, inalienable possession is formed by attaching the possessor morpheme directly to the noun, while alienable possession is formed using possession classes.

\begin{problem}{\langnameFijian}{\nameVPapp}{\LOYear{\UKLOAbbr}{2018}}
\IntroPhrases{\langnameFijian}\ \IntroAndEnglish:
\begin{center}
\begin{longtable}{rll}
     \sentlineonerow{na uluqu}{my head}
     \sentlineonerow{na nona wau}{her weapon (she owns)}
     \sentlineonerow{na memun\={i} bia}{your\pl\ beer}
     \sentlineonerow{na kemudrau itukutuku}{your\du\ story (about you two)}
     \sentlineonerow{na nona motokaa}{her car}
     \sentlineonerow{na meda t\={i}}{our\incl\ tea}
     \sentlineonerow{na kelemu}{your\sg\ belly}
     \sentlineonerow{na nona dio}{her oyster (she'll sell)}
     \sentlineonerow{na kequ uvi}{my yam}
     \sentlineonerow{na noqu itukutuku}{my story (I tell)}
     \sentlineonerow{na watiqu}{my spouse}
     \sentlineonerow{na kemun\={i} vuaka}{your\pl\ pig (you'll eat)}
     \sentlineonerow{na nomu kato}{your\sg\ basket}
     \sentlineonerow{na tamana}{her father}
     \sentlineonerow{na memudrau dio}{your\du\ oyster (you'll slurp)}
     \sentlineonerow{na nodra vuaka}{their pig (they raise)}
     \sentlineonerow{na keda wau}{our\incl\ weapon (we'll be hit with)}
     \sentlineonerow{na kedra raisi}{their rice}
\end{longtable}
\end{center}

\begin{assgts}\item Now the Fijian words are given to you. Your task is to translate the phrase in the table into Fijian:
\begin{center}
\begin{longtable}{rlll}\setcounter{exx}{18}
    & \langnameFijian & \langnameSolverese & English phrase to translate \\[0.3em]
     \fijiantask{uto}{heart}{my heart}
     \fijiantask{yaqona}{kava}{her kava (she's drinking)}
     \fijiantask{yaqona}{kava}{her kava (drunk in her honour)}
     \fijiantask{draunikau}{witchcraft}{my witchcraft (used on/against me)}
     \fijiantask{draunikau}{witchcraft}{your\du\ witchcraft (you're making)} 
     \fijiantask{dali}{rope}{your\sg\ rope (you own)}
     \fijiantask{dali}{rope}{your\pl\ rope (restraining you)}
     \fijiantask{ika}{fish}{your\du\ fish}
     \fijiantask{wai}{water}{your\pl\ water}
     \fijiantask{luve}{child}{her child}
     \fijiantask{waqa}{canoe}{our\incl\ canoe}
     \fijiantask{yapolo}{apple}{their apple (they'll sell)}
     \fijiantask{maqo}{mango}{their mango (for drinking)}
\end{longtable}
\end{center}

\item Explain your translation 21. (Why did you translate it this way?)
\item The word for \texttr{coconut} is \cmubdata{niu}. List all the ways to say \texttr{my coconut} and explain what they could mean. 
\end{assgts}

\begin{tblsWarning}
\texttr{Weapon} refers to a club-like tool. A \texttr{yam} is an edible starchy root. \texttr{Kava} is a ceremonial drink. 

\explainsgdupl\ \explainincl
\end{tblsWarning}
\end{problem}
\begin{mysolution}

 The first step we need to take is to determine the structure of the Fijian phrases. We can easily notice that all examples start with the word \cmubdata{na} followed by either one or two words. Separating the structures that contain a single word (i.e., the Fijian words for \texttr{my head}, \texttr{your\sg\ belly}, \texttr{my husband}, \texttr{her father}), we notice that all of them represent inalienable possessions. We therefore expect them to have the possessor marked directly on the noun, as an affix.

 Indeed, comparing the structures \texttr{my head} = \cmubdata{na uluqu} and \texttr{my husband} = \cmubdata{na watiqu}, we discover that they both share the suffix \cmubdata{-qu}, which we then can deduce to mean 1\textsc{sg} possession. Therefore, for the inalienable possession, the phrase structure is \fbox{\cmubdata{na} Noun--Poss}.

 In order to determine the structure of the rest of the phrases, we compare examples that contain the same noun (for example, 8 and 15, both containing the noun which means \texttr{oyster}). We notice that both of them have the last word in common (\cmubdata{dio}), so the last word represents the noun, the possessed. Moreover, comparing examples 9 and 10 (both containing the possessive \texttr{my}), we notice that they have in common the suffix \cmubdata{-qu} attached to the second word. Consequently, we can deduce that the structure of the alienable possession in Fijian is \fbox{\cmubdata{na} $X$--Poss Noun}. From here, we can easily determine the possession suffixes (applying similar processes as in Problem 5.5 where we had to separate the number from the classifier). We obtain:


\begin{itemize}
\begin{multicols}{4}
    \item[] 1\textsc{sg} = \cmubdata{-qu}
    \item[] 2\textsc{sg} = \cmubdata{-mu}
    \item[] 3\textsc{sg} = \cmubdata{-na}
    \item[] 1\textsc{pl} incl. = \cmubdata{-da}
    \item[] 2\textsc{du} = \cmubdata{-mudrau}
    \item[] 2\textsc{pl} = \cmubdata{-munī}
    \item[] 3\textsc{pl} = \cmubdata{-dra}
    \end{multicols}
\end{itemize}

 Moreover, we understand that we have only three possession classes, marked by \cmubdata{no-}, \cmubdata{-me-}, and \cmubdata{ke-} (excluding the inalienable possession). In order to see how each classifier is used, we split the given phrases according to the classifier they use:

\begin{table}[H]
\begin{tabular}{ lll }
\lsptoprule
 \cmubdata{ke-}& \cmubdata{me-} & \cmubdata{no-}\\
\midrule
  \texttr{my yam} &                                        \texttr{your\pl\ beer}                               &   \texttr{her weapon (she owns)} \\ 
  \texttr{their rice} &                                    \texttr{our tea}                                     &   \texttr{her car} \\ 
  `our weapon (we'll be  &                `your\du\ oyster  &   \texttr{her oyster (she'll sell)} \\ 
  \quad hit with)' & \quad (you'll slurp)'\\
  \texttr{your\pl\ pig (you'll eat)} &                                                                           &   \texttr{my story (I tell)} \\ 
  `your\du\ story (about you &                                                        &   \texttr{your\sg\ basket} \\ 
  \quad two)' & \\
\lspbottomrule
\end{tabular}
\end{table}

We can easily notice that the morpheme \cmubdata{me-} is used for things that are to be drunk. Moreover, it seems that \cmubdata{no-} represents the class for owned objects, in general. The last class is \cmubdata{ke-} and, based on the fact that we already have a class for things that are drunk, we expect to also have a class for things that are eaten.

Indeed, the class \cmubdata{ke-} includes \texttr{my yam}, \texttr{your\pl\ pig (for eating)}, and \texttr{their rice}, all of them being edible and meant to be eaten. On the other hand, we have two more structures (\texttr{our\incl\ weapon (we'll be hit with)} and \texttr{your\du\ story (about you two)}).

In order to figure out what these two have in common, we can also notice the pair of examples that contain the noun weapon: class \cmubdata{ke-} \texttr{we'll be hit with}, but class \cmubdata{no-} for \texttr{she owns}. Based on these, we realise that class \cmubdata{ke-} includes not only food, but also things that do not belong to us directly, but affect us (the weapon is not ours, but will be used against us; the story is not yours, but it is about you two, etc.).

Thus, we can write the rules and solve the tasks.

 \rules

 Structure:

 \begin{itemize}

     \item Inalienable possession (kinship, body parts): \fbox{\cmubdata{na} Noun--Poss}
     \item Alienable possession: \fbox{\cmubdata{na} $C$--Poss Noun}
 \end{itemize}


 Poss = possessive (suffix):

 \begin{itemize}
\begin{multicols}{4}
    \item[] 1\textsc{sg} = \cmubdata{-qu}
    \item[] 2\textsc{sg} = \cmubdata{-mu}
    \item[] 3\textsc{sg} = \cmubdata{-na}
    \item[] 1\textsc{pl} incl. = \cmubdata{-da}
    \item[] 2\textsc{du} = \cmubdata{-mudrau}
    \item[] 2\textsc{pl} = \cmubdata{-munī}
    \item[] 3\textsc{pl} = \cmubdata{-dra}
    \end{multicols}
\end{itemize}

 $C$ = class:

 \begin{itemize}

     \item \cmubdata{ke-} = food and objects we do not own, but affect us.
     \item \cmubdata{me-} = drinks
     \item \cmubdata{no-} = otherwise (owned objects)
 \end{itemize}

 \begin{assgts}
     \item \begin{enumerate}[start = 19, label = \arabic*.]

     \begin{multicols}{2}
         \item \cmubdata{na utoqu}
         \item \cmubdata{na mena yaqona}
         \item \cmubdata{na kena yaqona}
         \item \cmubdata{na kequ draunikau}
         \item \cmubdata{na nomudrau draunikau}
         \item \cmubdata{na nomu dali}
         \item \cmubdata{na kemunī dali}
         \item \cmubdata{na kemudrau ika}
         \item \cmubdata{na memunī wai}
         \item \cmubdata{na luvena}
         \item \cmubdata{na noda waqa}
         \item \cmubdata{na nodra yapolo}
         \item \cmubdata{na medra maqo}
         \item[] \quad
         \end{multicols}
     \end{enumerate}
     \item In example 21 we used the classifier \cmubdata{ke-}, since the action is indirectly reflected towards the person. It is not she who drinks the kava, but someone else drinks it in her honour.
     \item Since a coconut can clearly not be an inalienable possession, there are three different possible translations: \\
     \hphantom{Indent}\wordtrans{na kequ niu}{my coconut} (for eating / I'll be hit with)\\ \quad\quad
     \hphantom{Indent}\wordtrans{na mequ niu}{my coconut} (for drinking)\\\quad\quad
     \hphantom{Indent}\wordtrans{na noqu niu}{my coconut} (I own / I sell)
 \end{assgts}
\end{mysolution}

 \begin{problem}{\langnameAncientGreek}{\nameTTchervenkov}{\LOYear{\NACLOAbbr}{2007}}
\IntroPhrases{\langnameAncientGreek} (in a Latin-based transcription) \IntroAndEnglishRandom:
\begin{center}


\begin{tabular}{rl@{\hskip0.5in}cl}
     \chaosline{ho tōn hyiōn dulos}{the donkey of the master}
     \chaosline{hoi tōn dulōn cyrioi}{the brothers of the merchant}
     \chaosline{hoi tu emporu adelphoi}{the merchants of the donkeys}
     \chaosline{hoi tōn onōn emporoi}{the sons of the masters}
     \chaosline{ho tu cyriu onos}{the slave of the sons}
     \chaosline{ho tu oicu cyrios}{the masters of the slaves}
     \chaosline{ho tōn adelphōn oicos}{the house of the brothers}
     \chaosline{hoi tōn cyriōn hyioi}{the master of the house}
\end{tabular}
\end{center}
\begin{assgts}
\item \detcorr
\item \transinen[\langnameAncientGreek] \texttr{the houses of the merchants}, \texttr{the donkeys of the slave}.
\end{assgts}
\vskip-\baselineskip\largerpage[2]
\begin{tblsWarning}
\cmubdata{\={o}} denotes a long \cmubdata{o}.
\end{tblsWarning}
\end{problem}
\begin{mysolution}

\begin{description}
  \item[{Step 1.}] We notice that all Ancient Greek phrases have the structure [\cmubdata{ho}/\cmubdata{hoi}] [\cmubdata{tu}/\cmubdata{tōn}] $X$ $Y$, where $X$ and $Y$ represent the two nouns (the possessor and the possessed). Moreover, we notice that these nouns change their form. Furthermore, the first two words (which probably represent articles or possession markers) each have two forms. Checking the English translations, we notice that nouns appear both as singular and plural. Therefore, we can assume that the two markers from the beginning of the phrase change form, agreeing with the number of the noun.

\item[{Step 2.}] We make a frequency table based on the stem of the nouns (for now, we disregard the endings, which are variable).

\begin{center}
\begin{tabular}{ll ll}
    \lsptoprule
    \langnameGreek & Freq. &  \langnameSolverese & Freq. \\ \cmidrule(lr){1-2} \cmidrule(lr){3-4}
    \cmubdata{hyi-} & 2  &  \texttr{donkey} & 2 \\
    \cmubdata{dul-} & 2  &  \texttr{master} & 4 \\
    \cmubdata{cyri-} & 4  &  \texttr{brother} & 2 \\
    \cmubdata{empor-} & 2  &  \texttr{merchant} & 2 \\
    \cmubdata{adelph-} & 2  &  \texttr{son} & 2 \\
    \cmubdata{on-} & 2  &  \texttr{slave} & 2 \\
    \cmubdata{oic-} & 2  &  \texttr{house} & 2 \\
    \lspbottomrule
\end{tabular}
\end{center}

Since both in Greek and in English we have only one noun that appears four times (all the rest appearing only two times each), we infer that \wordtrans{cyri-}{master}.

\item[{Step 3.}] We separate all phrases which contain the noun master:

\begin{center}
    \begin{tabular}{lllll}
2. & \cmubdata{hoi tōn dulōn cyrioi} & \hphantom{Indent} & A. & \texttr{the donkey of the master} \\
5. & \cmubdata{ho tu cyriu onos} &  & D. & \texttr{the sons of the masters} \\
6. & \cmubdata{ho tu oicu cyrios} &  & F. & \texttr{the masters of the slaves} \\
8. & \cmubdata{hoi tōn cyriōn hyioi} &  & H. & \texttr{the master of the house}
\end{tabular}
\end{center}

In Greek, the nouns that appear together with \texttr{master} are \cmubdata{dul-}, \cmubdata{on-}, \cmubdata{oic-}, \cmubdata{hyi-}, while in English they are \texttr{donkey}, \texttr{son}, \texttr{slave}, \texttr{house}. The only two nouns that do not appear here are \texttr{merchant} and \texttr{brother}, so these must correspond to the two Greek nouns that do not occur: \cmubdata{empor-} and \cmubdata{adelph-}. Moreover, we notice that phrase 3 in Greek and phrase B in English contain both of these nouns. Therefore, we deduce the correspondence 3-B.

\item[{Step 4.}] We separate the sentences that contain the nouns \texttr{brother} or \texttr{merchant}.
\begin{center}
    \begin{tabular}{lllll}
3. & \cmubdata{hoi tu emporu adelphoi} & \hphantom{Ind} &  & \texttr{the brothers of the merchant} \\
4. & \cmubdata{hoi tōn onōn emporoi} &  & C. & \texttr{the merchants of the donkeys} \\
7. & \cmubdata{ho tōn adelphōn oicos} &  & G. & \texttr{the house of the brothers} \\
\end{tabular}
\end{center}

 \note{The fact that we didn't mention the letter in front of structure B. means that this phrase is already matched, its Greek correspondence being phrase 3.}

 Separating again the nouns which appear in these structures, we deduce that \texttr{donkey} and \texttr{house} are  \cmubdata{on-} and \cmubdata{oic-} though we do not know which is which, and similarly with other word pairs.

 \item[{Step 5.}] Recap.

 Based on the information above, we know that:

\begin{itemize}
    \item \wordtrans{cyri-}{master}
    \item\{\cmubdata{on-}/\cmubdata{oic-}\} = \{\texttr{donkey}/\texttr{house}\}
    \item\{\cmubdata{empor-}/\cmubdata{adelph-}\} = \{\texttr{merchant}/\texttr{brother}\}
    \item \{\cmubdata{dul-}/\cmubdata{hyi-}\} = \{\texttr{son}/\texttr{slave}\}.
\end{itemize}

 Moreover, we notice that there is one phrase in which the nouns \texttr{slave} and \texttr{son} co-occur, therefore 1-E.

\item[Step 6.] Based on this information, we split the phrases into subgroups:

\begin{center}
    \begin{tabular}{l@{~~}l l@{~~}l}
    1. & \cmubdata{ho tōn hyiōn dulos}     &     &  \texttr{the slave of the sons} \\
    3. & \cmubdata{hoi tu emporu adelphoi} &     &  \texttr{the brothers of the merchant} \\ \midrule
    4. & \cmubdata{hoi tōn onōn emporoi}   &  C. &  \texttr{the merchants of the donkeys} \\
    7. & \cmubdata{ho tōn adelphōn oicos}  &  G. &  \texttr{the house of the brothers} \\ \midrule
    5. & \cmubdata{ho tu cyriu onos}       &  A. &  \texttr{the donkey of the master} \\
    6. & \cmubdata{ho tu oicu cyrios}      &  H. &  \texttr{the master of the house} \\ \midrule
    2. & \cmubdata{hoi tōn dulōn cyrioi}   &  D. &  \texttr{the sons of the masters} \\
    8. & \cmubdata{hoi tōn cyriōn hyioi}   &  F. &  \texttr{the masters of the slaves} \\
    \end{tabular}
\end{center}

 We know that phrases 1 and 3 are already matched and that phrases 4 and 7 correspond to C and G (although not necessarily in this order), phrases 5/6 with A/H, and 2/8 with D/F.

Moreover, we notice that phrases 5/6 use the same articles (the first two words are identical), while in the English translations, all nouns are singular. Therefore, we deduce that \cmubdata{ho} and \cmubdata{tu} represent the singular form, while the other two (\cmubdata{hoi} and \cmubdata{tōn}) represent the plural form (which is confirmed by phrases 2 and 8 which use these articles, and in their English translations all nouns are plural).

\item[{Step 7.}]\sloppy We create a new frequency table for the four articles, knowing that they need to correspond to a combination of singular\slash plural and possessor\slash possessed.

\begin{center}
\begin{tabular}{lc lc}
\lsptoprule
\langnameGreek & Freq. & \langnameSolverese & Freq \\ \cmidrule{1-2} \cmidrule{3-4}
\cmubdata{ho} & 4  &  \textsc{sg}, possessor & 3 \\
\cmubdata{hoi} & 4  &  \textsc{pl}, possessor & 5 \\
\cmubdata{tu} & 3  &  \textsc{sg}, possessed & 4 \\
\cmubdata{tōn} & 5  &  \textsc{pl}, possessed & 4 \\
\lspbottomrule
\end{tabular}
\end{center}

 From the table, we can immediately deduce that \cmubdata{tu} is used for a singular possessor, and \cmubdata{tōn} when the possessor is plural. We are left with \cmubdata{ho}/\cmubdata{hoi} for the possessed. On the other hand, we already know that \cmubdata{ho} is used for the singular (from step 6). Therefore, we can make a table with all the forms of the article:

\begin{center}
\begin{tabular}{lll}
\lsptoprule
 & \textsc{sg} & \textsc{pl} \\ \midrule
possessor & \cmubdata{tu} & \cmubdata{tōn} \\
possessed & \cmubdata{ho} & \cmubdata{hoi} \\
\lspbottomrule
\end{tabular}
\end{center}

 Furthermore, knowing this, we can finish matching the phrases. Looking back at the table in step 6, in the pair 4-7 we have a singular possessed and a plural possessed (based on the articles), therefore 4 corresponds to C and 7-G. We get:

\begin{center}
    \begin{tabular}{l@{~~}l l@{~~}l}
    1. & \cmubdata{ho tōn hyiōn dulos}     &    & \texttr{the slave of the sons} \\
    3. & \cmubdata{hoi tu emporu adelphoi} &    & \texttr{the brothers of the merchant} \\
    4. & \cmubdata{hoi tōn onōn emporoi}   &    & \texttr{the merchants of the donkeys} \\
    7. & \cmubdata{ho tōn adelphōn oicos}  &    & \texttr{the house of the brothers} \\ \midrule
    5. & \cmubdata{ho tu cyriu onos}       & A. & \texttr{the donkey of the master} \\
    6. & \cmubdata{ho tu oicu cyrios}      & H. & \texttr{the master of the house} \\ \midrule
    2. & \cmubdata{hoi tōn dulōn cyrioi}   & D. & \texttr{the sons of the masters} \\
    8. & \cmubdata{hoi tōn cyriōn hyioi}   & F. & \texttr{the masters of the slaves} \\
\end{tabular}
\end{center}

 Now we can deduce, from 3 and 4, that \wordtrans{empor-}{merchant} and then we can easily make the rest of the correspondences. Moreover, we deduce that the possessed is placed after the possessor. Therefore, the word order in the Ancient Greek phrase is:

\exrule{[Art. possessed]  [Art. possessor]  Possessor  Possessed}

We notice an interesting phenomenon, that the word order between the noun and its corresponding article is “enclosed”\ (the possessor together with its article are placed in the middle and enclosed/surrounded by the possessor and its article).

\item[{Step 8.}] The last step is to figure out noun declension in Ancient Greek. To do so, we can make a table with the different forms of all nouns, based on number (singular/plural) and its role (possessor/possessed):

\begin{center}
\begin{tabular}{l l l l l}
\lsptoprule
     & \multicolumn{2}{c}{Possessor} & \multicolumn{2}{c}{Possessed} \\\cmidrule(lr){2-3}\cmidrule(lr){4-5}
Noun &  \textsc{sg} & \textsc{pl} & \textsc{sg} & \textsc{pl} \\ \midrule
\texttr{master} & \cmubdata{cyriu} & \cmubdata{cyriōn} & \cmubdata{cyrios} & \cmubdata{cyrioi} \\
\texttr{donkey} &  & \cmubdata{onōn} & \cmubdata{onos} &  \\
\texttr{brother} &  & \cmubdata{adelphōn} &  & \cmubdata{adelphoi} \\
\texttr{merchant} & \cmubdata{emporu} &  &  & \cmubdata{emporoi} \\
\texttr{son} &  & \cmubdata{hyiōn} &  & \cmubdata{hyioi} \\
\texttr{slave} &  & \cmubdata{dulōn} & \cmubdata{dulos} &  \\
\texttr{house} & \cmubdata{oicu} &  & \cmubdata{oicos} &  \\
\lspbottomrule
\end{tabular}
\end{center}

 From this table, we can easily notice that each form has its own characteristic suffix:

\begin{center}
\begin{tabular}{lll}
\lsptoprule
 & \textsc{sg} & \textsc{pl} \\ \midrule
possessor & \cmubdata{-u} & \cmubdata{-ōn} \\
possessed & \cmubdata{-os} & \cmubdata{-oi} \\
\lspbottomrule
\end{tabular}
\end{center}
\end{description}

 \note{Note that this approach is a basic one: it is based strictly on logical observations and not necessarily on linguistic intuition. Another more solid starting point would have been to notice the fact that there are similarities between the article and the noun suffix (\cmubdata{tōn} -- \cmubdata{-ōn}, \cmubdata{hoi} -- \cmubdata{-oi}, \cmubdata{tu} -- \cmubdata{-u}). This directly points towards the word order, which is the core phenomenon of the problem.}


Based on these, we can write the rules and solve the task.\pagebreak

\rules
\begin{itemize}
    \item Structure: \fbox{[Art. possessed]  [Art. possessor]  Possessor  Possessed}
    \item Articles: 
    \begin{tabular}[t]{lcc}
        \lsptoprule
         & \textsc{sg} & \textsc{pl} \\ \midrule
        possessor & \cmubdata{tu} & \cmubdata{tōn} \\
        possessed & \cmubdata{ho} & \cmubdata{hoi} \\
        \lspbottomrule
    \end{tabular}
    \item Noun endings:
    \begin{tabular}[t]{lcc}
    \lsptoprule
     & \textsc{sg} & \textsc{pl} \\ \midrule
    possessor & \cmubdata{-u} & \cmubdata{-ōn} \\
    possessed & \cmubdata{-os} & \cmubdata{-oi} \\
    \lspbottomrule
\end{tabular}
\end{itemize}
\begin{solutions}
    \item
        \begin{enumerate}[leftmargin = 1em]
        \begin{multicols}{8}

            \item E.
            \item F.
            \item B.
            \item C.
            \item A.
            \item H.
            \item G.
            \item D.
       \end{multicols} \end{enumerate}
       \item \texttr{the houses of the merchants} = \cmubdata{hoi tōn emporōn oicoi} \\
       \texttr{the donkeys of the slave} = \cmubdata{hoi tu dulu onoi}
\end{solutions}
\end{mysolution}

\section{Colour terms}

 A special focus is given to the problems that include noun phrases in which the adjectives are colour terms. For example, let us look at the following problem:

\begin{problem}{Colours}{\nameVNeacsu}{\LOYear{\RoLOAbbr}{2018}}
Brent Berlin and Paul Kay have studied colour terms of more than 100 languages by travelling the world and asking the locals to describe photos of different colours (\cite{BerlinandKay1969}). After examining the results, they concluded that colour perception in all of these languages is governed by the same law.

Here are the colour terms in 12 of the languages the two scientists examined:\footnote{\textit{Source:} Adapted from a problem by \nameKGilyarova\ (Elementy).}

\begin{table}[H]
\begin{tabular}{lllllll}
    \lsptoprule
                        & Fitzroy &                    & Upper &               &                  &  \\
     \langnameSolverese & River   &  \langnameNupe    & Pyramid & \langnameIbo & \langnameTzeltal & \langnameHanunoo \\\midrule
     \texttr{white} & \cmubdata{bura}& \cmubdata{bókùṇ}& \cmubdata{\pbblank}& \cmubdata{nzu}& \cmubdata{sak}& \cmubdata{(ma)biru}\\
     \texttr{blue} & \cmubdata{guru}& \cmubdata{dòfa}& \cmubdata{\pbblank}& \cmubdata{\pbblank}& \cmubdata{yaš}& \cmubdata{\pbblank}\\
     \texttr{yellow} & \cmubdata{kalmur}& \cmubdata{wọṇjiṇ}& \cmubdata{\pbblank}& \cmubdata{odo}& \cmubdata{k'an}& \cmubdata{(ma)raraʔ}\\
     \texttr{brown} & \cmubdata{\pbblank}& \cmubdata{dzúfú}& \cmubdata{mola}& \cmubdata{uhie}& \cmubdata{\pbblank}& \cmubdata{(ma)raraʔ}\\
     \texttr{black} & \cmubdata{\pbblank}& \cmubdata{ẓìkò}& \cmubdata{\pbblank}& \cmubdata{oji}& \cmubdata{ʔihk'}& \cmubdata{(ma)lagtiʔ}\\
     \texttr{red} & \cmubdata{kiran}& \cmubdata{dzúfú}& \cmubdata{\pbblank}& \cmubdata{\pbblank}& \cmubdata{cah}& \cmubdata{\pbblank}\\
     \texttr{green} & \cmubdata{\pbblank}& \cmubdata{álígà}& \cmubdata{muli}& \cmubdata{oji}& \cmubdata{yaš}& \cmubdata{(ma)latuy}\\
     \lspbottomrule
\end{tabular}
\end{table}

\begin{table}[H]
\begin{tabular}{lllllll}
    \lsptoprule
     \langnameSolverese & \langnameBari & \langnameJale & \langnameHausa & \langnameNasioi & \langnameDaza & \langnameIbibio \\\midrule
     \texttr{white} & \cmubdata{-kwe}& \cmubdata{hóló}& \cmubdata{fări}& \cmubdata{kakara}& \cmubdata{cuo}& \cmubdata{àfíá}\\
     \texttr{blue} & \cmubdata{-murye}& \cmubdata{siŋ}& \cmubdata{shuḍi}& \cmubdata{mutaŋa}& \cmubdata{zẹdẹ}& \cmubdata{\pbblank}\\
     \texttr{yellow} & \cmubdata{-forong}& \cmubdata{\pbblank}& \cmubdata{nawaya}& \cmubdata{\pbblank}& \cmubdata{mini}& \cmubdata{ńdàídàt}\\
     \texttr{brown} & \cmubdata{-jere}& \cmubdata{\pbblank}& \cmubdata{ja}& \cmubdata{\pbblank}& \cmubdata{maaḍo}& \cmubdata{\pbblank}\\
     \texttr{black} & \cmubdata{-rnö}& \cmubdata{siŋ}& \cmubdata{b\={a}\d{k}i}& \cmubdata{\pbblank}& \cmubdata{yasko}& \cmubdata{έbubit}\\
     \texttr{red} & \cmubdata{-tor}& \cmubdata{hóló}& \cmubdata{\pbblank}& \cmubdata{erereŋ}& \cmubdata{\pbblank}& \cmubdata{ńdàídàt}\\
     \texttr{green} & \cmubdata{-ngem}& \cmubdata{\pbblank}& \cmubdata{algashi}& \cmubdata{\pbblank}& \cmubdata{\pbblank}& \cmubdata{àwàwà}\\
    \lspbottomrule
\end{tabular}
\end{table}

\begin{assgts}
\item \fillblanks
\item Below are some colour terms in three other languages that follow the hypothesis of Berlin \& Kay:
\begin{enumerate}[label = \alph*.]
    \item \langnameUrhobo: \texttr{black} = \cmubdata{ɔbyibi}, \texttr{brown} = \cmubdata{ɔBaBare}, \texttr{green} = \cmubdata{ɔbyibi}, \texttr{yellow} = \cmubdata{5do};
    \item \langnameNgombe: \texttr{white} = \cmubdata{bopu}, \texttr{red} = \cmubdata{bopu};
    \item \langnameTanna: \texttr{yellow} = \cmubdata{laulau}, \texttr{black} = \cmubdata{rapen}, \texttr{brown} = \cmubdata{laulau}, \texttr{blue} = \cmubdata{ramimera}.
\end{enumerate}
\item[] For each language, specify which of the 12 languages above it most resembles. Explain your answer.
\item Here are some colour terms in two more languages: 
\begin{enumerate}[label = \alph*., start = 4]\sloppy
    \item \langnameAcehnese: \texttr{white} = \cmubdata{iĵu}, \texttr{red} = \cmubdata{pirã}, \texttr{green} = \cmubdata{prãna}, \texttr{yellow} = \cmubdata{iĵu}, \texttr{blue} = \cmubdata{prãna};
    \item \langnameAlabama: \texttr{green} = \cmubdata{okchakko}, \texttr{red} = \cmubdata{homma}, \texttr{brown} = \cmubdata{laana}, \texttr{blue} = \cmubdata{okchakko}, \texttr{yellow} = \cmubdata{laana}.
\end{enumerate}
\item[] Explain why they do not follow the hypothesis of Berlin \& Kay.
\item Formulate the hypothesis of Berlin \& Kay.
\end{assgts}
\end{problem}
\begin{mysolution}
\begin{description}
  \item [{Step 1.}] Looking closely at the table, we notice that almost all the languages in the table (save for English and Bari) have some colour terms which share the same name. Therefore, we deduce that, in order to fill in the blanks, we need to reuse some of the words in that language which are already given; in other words, we need to figure out which colour terms are translated identically in that language. This is also signalled by the fact that there does not seem to be any morphological process going on which allows us to derive some colour terms from others and there are few or no common morphemes.
  
  In order to do so, we classify each language based on the number of distinct colour terms (i.e., colour terms which have different names in that language) and reorder the table in descending order based on the number of distinct colour terms.\\

\begin{table}[H]
\begin{tabular}{ *6{l} }
\lsptoprule
        & 7 terms & \multicolumn{2}{c}{6 terms} & \multicolumn{2}{c}{5 terms} \\\cmidrule(lr){2-2}\cmidrule(lr){3-4}\cmidrule(lr){5-6}
\langnameSolverese & \langnameBari & \langnameNupe & \langnameHausa & \langnameTzeltal & \langnameDaza \\ \midrule 
\texttr{white}  &\cmubdata{-kwe}    &\cmubdata{bókùṇ}   &\cmubdata{fări} &\cmubdata{sak} &\cmubdata{cuo} \\
\texttr{blue}   &\cmubdata{-murye}  &\cmubdata{dòfa}    &\cmubdata{shuḍi} &\cmubdata{yaš}     &\cmubdata{zẹdẹ} \\
\texttr{yellow} &\cmubdata{-forong} &\cmubdata{wọṇjiṇ}  &\cmubdata{nawaya} &\cmubdata{k'an}    &\cmubdata{mini} \\
\texttr{brown}  &\cmubdata{-jere}   &\cmubdata{dzúfú}   &\cmubdata{ja} &                   &\cmubdata{maaḍo} \\
\texttr{black}  &\cmubdata{-rnö}    &\cmubdata{ẓìkò}    &\cmubdata{b\={a}\d{k}i} &\cmubdata{ʔihk'}   &\cmubdata{yasko} \\
\texttr{red}    &\cmubdata{-tor}    &\cmubdata{dzúfú}   & &\cmubdata{cah}     & \\
\texttr{green}  &\cmubdata{-ngem}   &\cmubdata{álígà}   &\cmubdata{algashi} &\cmubdata{yaš}     & \\
\lspbottomrule
\end{tabular}
\end{table}

\begin{table}[H]
\begin{tabular}{ *5{l} }
\lsptoprule
& \multicolumn{4}{c}{4 terms} \\\cmidrule(lr){2-5}
\langnameSolverese & \langnameFitzroy & {\langnameIbo} & {\langnameHanunoo} & {\langnameIbibio} \\\midrule
\texttr{white}  &\cmubdata{bura}    &\cmubdata{nzu}     &\cmubdata{(ma)biru}& \cmubdata{àfíá} \\
\texttr{blue}   &\cmubdata{guru}    &                   && \\
\texttr{yellow} &\cmubdata{kalmur}  &\cmubdata{odo}     &\cmubdata{(ma)raraʔ}& \cmubdata{ńdàídàt}\\
\texttr{brown}  & &\cmubdata{uhie}   &\cmubdata{(ma)raraʔ}& \\
\texttr{black}  &                   &\cmubdata{oji}     &\cmubdata{(ma)lagtiʔ}& \cmubdata{έbubit} \\
\texttr{red}    &\cmubdata{kiran}   &                   && \cmubdata{ńdàídàt} \\
\texttr{green}  &                   &\cmubdata{oji}     &\cmubdata{(ma)latuy}& \cmubdata{àwàwà} \\
\lspbottomrule
\end{tabular}
\end{table}

\begin{table}[H]
\begin{tabular}{ *4{l} }
\lsptoprule
        & 3 terms &  \multicolumn{2}{c}{2 terms} \\ \cmidrule(lr){2-2} \cmidrule(lr){3-4}
\langnameSolverese &  {\langnameNasioi} & {\langnameJale} & \langnameUpperPyramid\\ \midrule
\texttr{white}  &\cmubdata{kakara}  &\cmubdata{hóló}    & \\
\texttr{blue}   &\cmubdata{mutaŋa}  &\cmubdata{siŋ}     & \\
\texttr{yellow} &                   & & \\
\texttr{brown}  &                   &                   &\cmubdata{mola} \\
\texttr{black}  &                   &\cmubdata{siŋ}     & \\
\texttr{red}    & \cmubdata{erereŋ} &\cmubdata{hóló}    & \\
\texttr{green}  &                   &                   &\cmubdata{muli} \\
\lspbottomrule
\end{tabular}
\end{table}


\item[{Step 2.}] In Nupe (which has six distinct colours), we notice that the same word is used for both \texttr{brown} and \texttr{red}. In Tzeltal (which has five terms), the same word is used for both \texttr{blue} and \texttr{green}, etc.

We can assume that languages with the same number of colour terms will behave identically. We conclude that:

\begin{itemize}
    \item if a language has six colour terms, \texttr{red} = \texttr{brown}
    \item if a language has five colour terms, \texttr{red} = \texttr{brown} and \texttr{blue} = \texttr{green}
\end{itemize}

\item[{Step 3.}] Checking the languages with four colour terms, we notice that we have two options:

\begin{itemize}
    \item for Fitzroy River and Ibo: \texttr{red} = \texttr{brown}, \texttr{green} = \texttr{blue} = \texttr{black}
    \item for Ibobo and Hanunóo: \texttr{red} = \texttr{brown} = \texttr{yellow}, \texttr{green} = \texttr{blue}
    \end{itemize}

 So, in the case of languages with only four colour terms, we have two alternatives: \texttr{black} is named identically with \texttr{blue} and \texttr{green}, or \texttr{yellow} is named identically with \texttr{red} and \texttr{brown}.

\item[{Step 4.}] The rest of the problem becomes trivial and we deduce that:

\begin{itemize}
\item for languages with three colour terms: \texttr{black} = \texttr{green} = \texttr{blue}, \texttr{yellow} = \texttr{brown} = \texttr{red}
\item for languages with two terms: \texttr{black} = \texttr{green} = \texttr{blue}, \texttr{yellow} = \texttr{brown} = \texttr{red} = \texttr{white}
\end{itemize}
\end{description}

 Therefore, we can solve the tasks:

\begin{solutions}
    \item
    \begin{enumerate}[label = (\arabic*)]

    \begin{multicols}{4}
        \item \cmubdata{mola}
        \item \cmubdata{muli}
        \item \cmubdata{oji}
        \item \cmubdata{(ma)latuy}
        \item \cmubdata{mola}
        \item \cmubdata{kiran}
        \item \cmubdata{cah}
        \item \cmubdata{guru}
        \item \cmubdata{muli}
        \item \cmubdata{mola}
        \item \cmubdata{uhie}
        \item \cmubdata{(ma)raraʔ}
        \item \cmubdata{guru}
        \item \cmubdata{àwàwà}
        \item \cmubdata{hóló}
        \item \cmubdata{erereŋ}
        \item \cmubdata{hóló}
        \item \cmubdata{erereŋ}
        \item \cmubdata{ńdàídàt}
        \item \cmubdata{mutaŋa}
        \item \cmubdata{ja}
        \item \cmubdata{maaḍo}
        \item \cmubdata{siŋ}
        \item \cmubdata{mutaŋa}
        \item \cmubdata{zẹdẹ}
    \end{multicols}
    \end{enumerate}
    \item
    \begin{enumerate}[label = \alph*.]

        \item Urhobo is similar to Fitzroy River and Ibo.
        \item N'Gombe is similar to Upper Pyramid and Jalé.
        \item Tanna Island is similar to Hanunó'o and Ibibo.
    \end{enumerate}
    \item
    \begin{enumerate}[label = \alph*., start = 4]

        \item If the same word is used for both \texttr{white} and \texttr{yellow}, there should only be two colour terms in that language, so \texttr{red} should also be translated like \texttr{white} and \texttr{yellow}.
        \item If \texttr{green} and \texttr{blue} use the same word, the language must have five terms, so \texttr{brown} and \texttr{red} should be translated identically.
    \end{enumerate}
    \item Berlin \& Kay hypothesised that the name of the colour terms in each language can be deduced by the number of colour terms each language has.

If we use the notation \texttt{W} = \texttr{white}, \texttt{Bk} = \texttr{black}, \texttt{Br} = \texttr{brown}, \texttt{R} = \texttr{red}, \texttt{G} = \texttr{green}, \texttt{Y} = \texttr{yellow}, \texttt{Bl} = \texttr{blue}, Berlin \& Kay proposed six stages of evolution of colour terms in a language.

In the scheme below, the underlined colour terms exist in each language's vocabulary, while those following are perceived as being identical to the one underlined. For example, the notation \texttt{\uline{G}: Bl, Bk} means that in the respective language, there is a word for \texttr{green}, while blue and black share the same term and are perceived as shades of green.\\
\end{solutions}
\end{mysolution}

\begin{framed}
    \quad Stage I. \\ 
    \texttt{\uline{W}: Y, Br, R} \hfill \texttt{\uline{Bk}: Bl, G}
\end{framed}
\begin{framed}
    \quad Stage II. \\ 
    \texttt{\uline{W}} \hfill \texttt{\uline{R}: Y, Br} \hfill \texttt{\uline{Bk}: Bl, G}
\end{framed}
\begin{framed}
    \quad Stage IIIa. \\ 
    \texttt{\uline{W}} \hfill \texttt{\uline{R}: Br} \hfill \texttt{\uline{Y}} \hfill \texttt{\uline{Bk}: Bl, G}\\
    \noindent\hspace*{-\FrameSep}\rule{\textwidth + 2 \FrameSep}{\FrameRule}
    
    \quad Stage IIIb. \\ 
    \texttt{\uline{W}} \hfill \texttt{\uline{R}: Y, Br} \hfill \texttt{\uline{Bk}} \hfill \texttt{\uline{G}: Bl}
\end{framed}
\begin{framed}
    \quad Stage IV. \\ 
    \texttt{\uline{W}} \hfill \texttt{\uline{R}: Br} \hfill \texttt{\uline{Y}} \hfill \texttt{\uline{Bk}} \hfill \texttt{\uline{G}: Bl}
\end{framed}
\begin{framed}
    \quad Stage V. \\ 
    \texttt{\uline{W}} \hfill \texttt{\uline{R}: Br} \hfill \texttt{\uline{Y}} \hfill \texttt{\uline{Bk}} \hfill \texttt{\uline{G}} \hfill \texttt{\uline{Bl}}
\end{framed}
\begin{framed}
    \quad Stage VI. \\ 
    \texttt{\uline{W}} \hfill \texttt{\uline{R}} \hfill \texttt{\uline{Br}} \hfill \texttt{\uline{Y}} \hfill \texttt{\uline{Bk}} \hfill \texttt{\uline{G}} \hfill \texttt{\uline{Bl}}
\end{framed}


\pagebreak
We notice that the third stage has two variants: either \texttr{yellow} splits away (from \texttr{red} and \texttr{brown}), \emph{or} \texttr{blue} and \texttr{green} (together) split from \texttr{black}; while in the fourth stage, both splits take place: \texttr{yellow} splits (from \texttr{red}) \emph{and} \texttr{blue} and \texttr{green} from \texttr{black}.

This fact can be useful in linguistics problems, considering the following aspect: historically speaking, it is likely that some languages had a limited number of colour terms and, in time, due to contact with other nations, they might have borrowed terms for other colours. For this reason, it is likely that, in some languages, some colour terms behave differently from others. In other words, it is likely that there are some basic colours (inherent to the language, which, according to the aforementioned hypothesis, should contain \texttr{white}, \texttr{black}, and \texttr{red}), which follow the usual declension of the language and other colour terms (borrowings from other languages) with a different flexion (usually diminished, or even inflexible).

For example, in Romanian, the basic colour terms have a full set of inflected forms, with four different forms (\texttr{white}: \cmubdata{alb} -- \cmubdata{albă} -- \cmubdata{albi} -- \cmubdata{albe}, \texttr{black}: \cmubdata{negru} -- \cmubdata{neagră} -- \cmubdata{negri} -- \cmubdata{negre}), while other, new terms, are usually invariable (\texttr{pink}: \cmubdata{roz}, \texttr{brown}: \cmubdata{maro}, \texttr{turquoise}: \cmubdata{turcoaz}, etc.).

\hypertarget{practice-problems}{%
\section{Practice problems}}

\begin{problem}{\langnameUlwa}{\namePArkadiev}{\LOYear{\TurLomAbbr}{2003}}
\IntroWords{\langnameUlwa}\ \IntroAndEnglishRandom:

\begin{center}
    \cmubdata{suulu}, \cmubdata{suukilu}, \cmubdata{suumanalu}, \cmubdata{mismatu}, \cmubdata{miskatu}, \cmubdata{onkinayan}, \cmubdata{onkayan}, \cmubdata{onyan}
\end{center}

\begin{center}
    \texttr{bow}, \texttr{your\sg\ cat}, \texttr{my dog}, \texttr{our bow}, \texttr{his cat}, \texttr{dog}, \texttr{his bow}, \texttr{your\pl\ dog} 
\end{center}

\begin{assgts}
\item \detcorr
\item \transinen\ \cmubdata{suumalu} and \cmubdata{miskanatu}.
\item \transinen[\langnameUlwa] \texttr{cat}, \texttr{my cat}, \texttr{your\sg\ bow}, \texttr{their bow}.
\end{assgts}
\end{problem}

\begin{problem}{\langnamePalauan}{\nameMSalter}{\LOYear{\NACLOAbbr}{2018}}
\IntroPhrases{\langnamePalauan}\ \IntroAndEnglish: 
\begin{center}

\begin{tabular}{ll @{\hskip3em} ll}
     \pbsv{eru ęl buil}{2 months} & \pbsv{kltiu ęl hong}{9 books} \\[0.3em]
     \pbsv{ede ęl sils}{3 days} & \pbsv{kllolem ęl lius}{6 coconuts} \\[0.3em]
     \pbsv{tede ęl chad}{3 people} & \pbsv{teai ęl ngalęk}{8 children} \\[0.3em]
     \pbsv{kllolem ęl malk}{6 chickens} & \pbsv{ongeru ęl buil}{February} \\[0.3em]
     \pbsv{teim ęl sensei}{5 teachers} & \pbsv{ongede ęl ureor}{Wednesday} \\[0.3em]
     \pbsv{eim ęl rak}{5 years} & \pbsv{etiu ęl klębęse}{9 nights} \\[0.3em]
     \multicolumn{3}{r}{\cmubdata{tęruich me a tede ęl buik}}&\texttr{13 boys}\\[0.3em]
     \multicolumn{3}{r}{\cmubdata{tęruich me a euid ęl sikang}}&\texttr{17 hours}\\
\end{tabular}
    
\end{center}
\begin{assgts}
\item \transinen
\begin{multicols}{2}
\begin{enumerate}
    \item \cmubdata{telolem ęl sensei}
    \item \cmubdata{tęruich me a etiu ęl buil}
    \item \cmubdata{tęruich me a ongeru ęl buil}
    \item \cmubdata{ongeim ęl ureor}
\end{enumerate}
\end{multicols}
\item \transinen[\langnamePalauan]
\begin{multicols}{3}
\begin{enumerate}[start = 5]
    \item \texttr{8 days}
    \item \texttr{19 people}
    \item \texttr{7 teachers}
    \item \texttr{June}
    \item \texttr{August}
\end{enumerate}
\end{multicols}
\item For each of the following, write the Palauan word that would be used to translate the word \texttr{3}:
\begin{multicols}{3}
\begin{enumerate}[start = 10]
    \item \texttr{\textbf{3} hours}
    \item \texttr{\textbf{3} girls}
    \item \texttr{\textbf{3} dolphins}
\end{enumerate}
\end{multicols}
\end{assgts}
\end{problem}

\begin{problem}{\langnameNorwegian}{\nameBNewsome}{\LOYear{\NACLOAbbr}{2017}}
\IntroSentences{\langnameNorwegian}\ \IntroAndEnglishRandom:
\begin{center}
    
\renewcommand \chaosline [2]{\addtocounter{exx}{1}\arabic{exx}.&\cmubdata{#1}&\addtocounter{exxx}{1}\AlphAlph{\value{exxx}}.&\texttr{#2} \\[0em]}

\begin{tabular}{rl@{\hskip0.5in}cl}
    \chaosline{Bussen stanser her.}{A woman has the apple.}
    \chaosline{Jeg har en bil.}{I have an apple.}
    \chaosline{Bilen stanser her.}{The bus stops here.}
    \chaosline{Jeg har eplet.}{The woman has cars.}
    \chaosline{Jeg har et eple.}{The car stops here.}
    \chaosline{En kvinne har eplet.}{I have buses.}
    \chaosline{Kvinna har biler.}{The woman has the cars.}
    \chaosline{Kvinna har bilene.}{I have the apple.}
    \chaosline{Jeg har busser.}{The women stop here.}
    \chaosline{Kvinnene stanser her.}{I have a car.}
\end{tabular}
\end{center}
\begin{assgts}
\item \detcorr
\item Nouns in Norwegian can belong to one of three classes: masculine, feminine, or neuter. The class determines how the noun can be used with determiners (words such as \texttr{the}, \texttr{a}, \texttr{an}) and be made plural. The nouns you encountered above are all regular and feature examples of all three classes:


\begin{center}
    \cmubdata{kvinne} -- feminine, \cmubdata{bil} -- masculine, \cmubdata{eple} -- neuter
\end{center}

\item[] Here are three more regular Norwegian nouns and their translations:
\begin{center}
\cmubdata{jente} (feminine) = \texttr{girl}, \cmubdata{hund} (masculine) = \texttr{dog}, \cmubdata{hotell} (neuter) = \texttr{hotel}
\end{center}
\item[] \transinen[\langnameNorwegian]

\begin{multicols}{2}
\begin{enumerate}[start = 11]

    \item \texttr{The girl stops here.}
    \item \texttr{A girl has a hotel.}
    \item \texttr{I have the dogs.}
    \item \texttr{The girl has dogs.}
\end{enumerate}
\end{multicols}

\item Here are some more Norwegian words without any information about the classes the slightly irregular nouns belong to:\largerpage
\begin{center}
\cmubdata{sko} = \texttr{shoe}, \cmubdata{mann} = \texttr{man}, \cmubdata{ikke} = \texttr{not}
\end{center}
\item[] \transinen

\begin{multicols}{2}
\begin{enumerate}[start = 15]

    \item \cmubdata{Mennene har epler.}
    \item \cmubdata{Kvinna har ikke skoene.}
    \item \cmubdata{Jeg har ikke eplene.}
\end{enumerate}
\end{multicols}

\end{assgts}
\end{problem}

\renewcommand \chaosline [2]{\addtocounter{exx}{1}\arabic{exx}.&\cmubdata{#1}&\addtocounter{exxx}{1}\AlphAlph{\value{exxx}}.&\texttr{#2} \\[0.3em]}

\begin{problem}{\langnameAfrihili}{\nameMSalter\ \& \nameABlackwell}{\LOYear{\UKLOAbbr}{2019}}
\IntroWords{\langnameAfrihili}\ \IntroAndEnglish:

\begin{multicols}{3}
\begin{tabbing}
\cmubdata{mmmmmm} \= \texttr{tooth}\kill
\cmubdata{adu} \> \texttr{tooth} \\
\cmubdata{afidi} \> \texttr{machine} \\
\cmubdata{ajamuri} \> \texttr{republic} \\
\cmubdata{akalini} \> \texttr{pen} \\
\cmubdata{amadu} \> \texttr{dentist} \\
\cmubdata{amkate} \> \texttr{bread} \\
\cmubdata{amola} \> \texttr{children} \\
\cmubdata{amukamo} \> \texttr{kingdom} \\
\cmubdata{aturesine} \> \texttr{bouquet} \\
\cmubdata{emelisini} \> \texttr{fleet} \\
\cmubdata{emeli} \> \texttr{ship} \\
\cmubdata{enti} \> \texttr{date tree} \\
\cmubdata{eshuli} \> \texttr{principal} \\
\cmubdata{eture} \> \texttr{flowers} \\
\cmubdata{ijamura} \> \texttr{president} \\
\cmubdata{ikalini} \> \texttr{pens} \\
\cmubdata{ilengi} \> \texttr{horses} \\
\cmubdata{imukazi} \> \texttr{girls} \\
\cmubdata{isabamatu} \> \texttr{cobbler} \\
\cmubdata{ishule} \> \texttr{school} \\
\cmubdata{olengi} \> \texttr{horse} \\
\cmubdata{oluganda} \> \texttr{dialect} \\
\cmubdata{omola} \> \texttr{child} \\
\cmubdata{omukazi} \> \texttr{girl} \\
\cmubdata{omuntundu} \> \texttr{dwarf} \\ 
\cmubdata{omuntu} \> \texttr{man} \\
\cmubdata{uruzindi} \> \texttr{stream} \\
\cmubdata{uruzi} \> \texttr{river} 
\end{tabbing}
\end{multicols}

\begin{assgts}\sloppy
\item \transinen\ \cmubdata{ajamura}, \cmubdata{amkamate}, \cmubdata{oluga}.
\item \transinen[\langnameAfrihili] \texttr{machinist}, \texttr{ships}, \texttr{flower}, \texttr{group of girls}, \texttr{date fruit}, \texttr{shoe}, \texttr{king}.
\item Below are three more Afrihili words and three options for a likely translation of the word:

\begin{tabular}{@{}l lll@{}}
     \cmubdata{imulenzi} & a. \texttr{fruit} & b. \texttr{boys} & c. \texttr{bridge} \\[0.3em]
     \cmubdata{aposino} & a. \texttr{baggage} & b. \texttr{classroom} & c. \texttr{parent} \\[0.3em]
     \cmubdata{iwelemase} & a. \texttr{book} & b. \texttr{library} & c. \texttr{librarian} \\
\end{tabular}
\item[] Pick the translation most likely to be correct and explain your choice. 
\end{assgts}
\end{problem}

\begin{problem}{\langnameMaltese}{\nameSStrizhevskaya}{\LOYear{\LLOAbbr}{2020}}
\IntroPhrases{\langnameMaltese}\ \IntroAndEnglish:\largerpage

\begin{longtable}{@{}ll ll@{}}
     \pbsv{serp aħdar}{green snake} & \pbsv{mogħża sewda}{\pbblank\ \pbblank} \\[0.3em]
     \pbsv{mogħżiet bojod}{white goats} & \pbsv{ktieb aħmar}{\pbblank\ \pbblank} \\[0.3em]
     \pbsv{kowt blu}{blue coat} & \pbsv{mejda bajda}{\pbblank\ \pbblank} \\[0.3em]
     \pbsv{mejda kannella}{brown chair} & \pbsv{baqra \pbblank}{blue cow} \\[0.3em]
     \pbsv{żiemel iswed}{black horse} & \pbsv{fjuri \pbblank}{red flowers} \\[0.3em]
     \pbsv{fjura vjola}{purple flower} & \pbsv{kelb \pbblank}{brown dog} \\[0.3em]
     \pbsv{karozza ħamra}{red car} & \pbsv{kotba \pbblank}{yellow books} \\[0.3em]
     \pbsv{rħula ħodor}{green settlments} & \pbsv{siġra \pbblank}{green tree} \\[0.3em]
     \pbsv{kowtijiet roża}{pink coats} & \pbsv{mwejjed \pbblank}{purple chairs} \\[0.3em]
     \pbsv{qomos blu}{blue shirts} & \pbsv{tuffieħa \pbblank}{yellow apple} \\[0.3em]
     \pbsv{fenek isfar}{yellow rabbit} & \pbsv{\pbblank\ \pbblank}{red snake} \\[0.3em]
     \pbsv{qomos sowod}{black shirts} &  \\[0.3em]
\end{longtable}

\begin{assgts}
\item \fillblanks{} Each blank corresponds to \textit{a single} word.
\item Based on the data given, one cannot translate \texttr{white book}. Why not?
\end{assgts}
\end{problem}

\begin{problem}{\langnameLatvian}{\nameMRubinstein}{\LOYear{\MSKAbbr}{1999}}
\IntroPhrases{\langnameLatvian}\ \IntroAndEnglish:
\begin{center}

\begin{longtable}{rll}
     \sentlineonerow{augsts ozols}{tall oak}
     \sentlineonerow{vecs grāmatu veikals}{old bookstore}
     \sentlineonerow{veca meža}{of the old wood}
     \sentlineonerow{stikla galds}{glass table}
     \sentlineonerow{bruņinieka cimds}{knight's glove}
     \sentlineonerow{sudraba ābols}{silver apple}
     \sentlineonerow{autora teksts}{author's text}
     \sentlineonerow{operāciju galds}{surgery table}
     \sentlineonerow{pretīgu piena ēdienu}{of disgusting dairy products}
     \sentlineonerow{labs institūts}{good institute}
     \sentlineonerow{laba bērnu ārsta}{of the good paediatrician}
     \sentlineonerow{grāmatu veikala}{\pbblank}
     \sentlineonerow{\pbblank kolektīvs}{authors' group}
     \sentlineonerow{\pbblank turnīrs}{knight tournament}
     \sentlineonerow{pretīgs \pbblank}{\pbblank child}
     \sentlineonerow{balts \pbblank}{white silver}
     \sentlineonerow{\pbblank zara}{of the oak branch}
     \sentlineonerow{\pbblank}{of the oak wood}
     \sentlineonerow{\pbblank ārstu}{of the institute's doctors}
\end{longtable}
\end{center}

\begin{assgts}
\item \fillblanks{} Some blanks may correspond to multiple words. If you think some blanks can be filled in different ways, write all possibilities and explain the difference between them.
\end{assgts}
\end{problem}

\begin{problem}{\langnameIlocano}{\namePLittell}{\LOYear{\NACLOAbbr}{2008}}
Below are 12 Ilocano words written in the Baybayin script, as well as their English translations given \OlympiadRandomOrder{}:

\begin{center}
\begin{tabular}{@{}rc@{\hskip3em}rc@{}}
     1. & \baytext{\char"1703\char"1712\char"1706} & 
     7. & \baytext{\char"170D\char"1713\char"170D\char"1713\char"1704\char"1714} \\[0.5em]
     
     2. & \baytext{\char"1703\char"1712\char"1706\char"1714\char"1703\char"1712\char"1706} &
     8. & \baytext{\char"170D\char"1713\char"170D\char"1714\char"170D\char"1713\char"170D\char"1713\char"1704\char"1714} \\[0.5em]
     
     3. & \baytext{\char"1703\char"1713\char"170B\char"1712\char"1706} & 
     9. & \baytext{\char"170D\char"1713\char"170B\char"1713\char"170D\char"1714\char"170D\char"1713\char"170D\char"1713\char"1704\char"1714} \\[0.5em]
     
     4. & \baytext{\char"1703\char"1713\char"170B\char"1712\char"1706\char"1714\char"1703\char"1712\char"1706} & 
     10. & \baytext{\char"1704\char"1713\char"170B\char"1706\char"1705\char"1714} \\[0.5em]
     
     5. & \baytext{\char"170D\char"1704\char"1714\char"1710\char"1703\char"1714} & 
     11. & \baytext{\char"1710\char"1709\char"1706\char"1714} \\[0.5em]
     
     6. & \baytext{\char"170D\char"1713\char"170B\char"1704\char"1714\char"170D\char"1704\char"1714\char"1710\char"1703\char"1714} & 
     12. & \baytext{\char"1710\char"1713\char"170B\char"1709\char"1706\char"1714} \\
\end{tabular} 
\end{center}

\begin{center}
    \texttr{to look}, \texttr{is skipping with joy}, \texttr{is becoming a skeleton}, \texttr{a skeleton}, \texttr{to buy}, \texttr{various skeletons}, \texttr{various appearances}, \texttr{to reach the top}, \texttr{is looking}, \texttr{appearance}, \texttr{summit}, \texttr{happiness}, \texttr{skeleton}
\end{center}

\begin{assgts}
\item \detcorr
\item \fillblanks
\begin{center}

\begin{tabular}{cc}
\pbsv{\baytext{\char"170D\char"1713\char"170B\char"1713\char"170D\char"1713\char"1704\char"1714}}{\pbblank} \\[0.3em]
\pbsv{\baytext{\char"1710\char"1709\char"1714\char"1710\char"1709\char"1706\char"1714}}{\pbblank} \\[0.3em]
\pbsv{\baytext{\char"1710\char"1713\char"170B\char"1709\char"1714\char"1710\char"1709\char"1706\char"1714}}{\pbblank} \\[0.3em]
\pbsv{\pbblank}{(a) purchase} \\[0.3em]
\pbsv{\pbblank}{is buying} \\[0.3em]
\end{tabular}
\end{center}
\end{assgts}
\end{problem}

\begin{problem}{\langnameIrish}{\nameTPayne}{\LOYear{\UKLOAbbr}{2011}}
Below are some number phrases in Irish and their English equivalents:
\begin{center}
    

\begin{tabular}{rll}
     \sentlineonerow{garra amháin}{1 garden}
     \sentlineonerow{gasúr déag}{11 boys}
     \sentlineonerow{ocht mballa is dhá fichid}{48 walls}
     \sentlineonerow{dhá gharra déag is ceithre fichid}{92 gardens}
     \sentlineonerow{trí bhád}{3 boats}
     \sentlineonerow{seacht ndoras déag}{17 doors}
     \sentlineonerow{seacht mbád déag is dhá fichid}{57 boats}
     \sentlineonerow{naoi nduine déag is fiche}{39 people}
     \sentlineonerow{ceithre fichid doras}{80 doors}
     \sentlineonerow{cúig bhalla}{5 walls}
     \sentlineonerow{sé ghasúr is trí fichid}{66 boys}
     \sentlineonerow{deich mbád}{10 boats}
     \sentlineonerow{sé dhuine}{6 people}
     \sentlineonerow{trí dhoras is dhá fichid}{43 doors}
     \sentlineonerow{garra is ceithre fichid}{81 gardens}
\end{tabular}
\end{center}
\begin{assgts}
\item \transinen
\begin{multicols}{2}
\begin{enumerate}[start = 16]
    \item \cmubdata{naoi mbád déag is ceithre fichid}
    \item \cmubdata{sé dhuine déag}
    \item \cmubdata{naoi nduine}
    \item \cmubdata{fiche gasúr}
    \item \cmubdata{garra déag is fiche}
\end{enumerate}
\end{multicols}

\item \transinen[Irish]
\begin{multicols}{3}
\begin{enumerate}[start = 21]
    \item \texttr{2 boys}
    \item \texttr{38 walls}
    \item \texttr{14 walls}
    \item \texttr{71 doors}
    \item \texttr{21 boats}
    \item \texttr{90 people}
\end{enumerate}
\end{multicols}
\end{assgts}
\end{problem}

\begin{problem}{\langnameIaai}{\nameRGandhi}{\LOYear{\APLOAbbr}{2020}}
Here are some phrases spoken by two speakers of Iaai and their English translations, \OlympiadRandomOrder{}:

\subsubsection*{Speaker 1 (male, 50 years old):}
\begin{center}

\begin{tabular}{rl@{\hskip0.5in}cl}
     \chaosline{hoom hu}{your fire}
     \chaosline{belem mââng}{my water}
     \chaosline{waau haalee Aiawa}{my bus}
     \chaosline{uutap taben than}{your mango}
     \chaosline{tabik kar}{your boat}
     \chaosline{anyik sawakiny}{the chief's chair}
     \chaosline{anyim meic}{my necklace}
     \chaosline{belik köiö}{Aiawa's cat}
\end{tabular}
\end{center}

\subsubsection*{Speaker 2 (male, 12 years old):}

\begin{center}    

\begin{tabular}{rl@{\hskip0.7in}cl}\setcounter{exx}{8}\setcounter{exxx}{8}
     \chaosline{belen koka}{your goat}
     \chaosline{anyik tang}{his yam}
     \chaosline{anyim karopëë}{Kua's mother}
     \chaosline{haaleem nani}{his watermelon}
     \chaosline{an koko}{his car}
     \chaosline{hinyö anyi Kua}{my basket}
     \chaosline{belik nu}{his Coke}
     \chaosline{anyin loto}{my coconut}
     \chaosline{an waajem}{your dugout}
\end{tabular}
\end{center}
\begin{assgts}
\item For each speaker, determine the correct correspondences.
\item \transinen
\begin{multicols}{2}
\begin{enumerate}[start = 18]
    \item \cmubdata{belem waajem}
    \item \cmubdata{karopëë hoon hinyö}
\end{enumerate}
\end{multicols}
\item[] Mention any meaning that is not reflected in the literal translation.
\item Given below are some English words and their Iaai translations:\largerpage

\begin{multicols}{3}
\begin{itemize}
    \item[] \texttr{dog} = \cmubdata{kuli}
    \item[] \texttr{tea} = \cmubdata{trii}
    \item[] \texttr{canoe} = \cmubdata{ok}
    \end{itemize}
\end{multicols}
\pagebreak
\item[] \transinenall[\langnameIaai]

\begin{multicols}{2}
\begin{enumerate}[start = 20]
    \item \texttr{the cat's tea}
    \item \texttr{Kua's coconut}
    \item \texttr{his canoe}
    \item \texttr{my dog}
\end{enumerate}
\end{multicols}
\end{assgts}

\begin{tblsWarning}
\cmubdata{â}, \cmubdata{ë}, \cmubdata{ö} are vowels. A \texttr{dugout} is a long, narrow canoe made of a tree trunk. A \texttr{yam} is a starchy vegetable, similar to a sweet potato. \texttr{Coke} is a carbonated beverage sold by The Coca-Cola Company, an American multinational company. \texttr{Aiawa} and \texttr{Kua} are names of people.
\end{tblsWarning}
\end{problem}

\hypertarget{solutions-of-practice-problems}{%
\section{Solutions of practice problems}}

\begin{practiceproblemsolution}{5.9. \langnameUlwa}

\begin{solutions}[label=Solution 5.9\alph*]
    \item 
    \begin{itemize}[leftmargin = 1em]

    \begin{multicols}{2}
        \item[] \wordtrans{suulu}{dog}
        \item[] \wordtrans{suukilu}{my dog}
        \item[] \wordtrans{suumanalu}{your\pl\ dog}
        \item[] \wordtrans{onyan}{bow}
        \item[] \wordtrans{miskatu}{his cat}
        \item[] \wordtrans{onkinayan}{our bow}
        \item[] \wordtrans{mismatu}{your\sg\ cat}
        \item[] \wordtrans{onkayan}{his bow}
    \end{multicols}
    \end{itemize}
    \item 
    \begin{itemize}[leftmargin = 1em]

    \begin{multicols}{2}
        \item[] \wordtrans{suumalu}{your\sg\ dog}
        \item[] \wordtrans{miskanatu}{their cat}
    \end{multicols}
    \end{itemize}
    \item 
    \begin{itemize}[leftmargin = 1em]

    \begin{multicols}{2}
        \item[] \texttr{cat} = \cmubdata{mistu}
        \item[] \texttr{my cat} = \cmubdata{miskitu}
        \item[] \texttr{your\sg\ bow} = \cmubdata{onmayan}
        \item[] \texttr{their bow} = \cmubdata{onkanayan}
    \end{multicols}
    \end{itemize}
\end{solutions}

\rules
\begin{itemize}
    \item The possessive is marked by an infix placed before the last syllable. The structure of the infix is \fbox{Person--Number}.
    \item Person: \qquad\quad \cmubdata{-ki-} = 1	\qquad\quad \hphantom{i} \cmubdata{-ma-} = 2	\qquad\quad \cmubdata{-ka-} = 3
	\item Number: \qquad\quad $\varnothing$ = \textsc{sg} \qquad\quad \cmubdata{-na-} = \textsc{pl}
\end{itemize}

\end{practiceproblemsolution}
\pagebreak
\begin{practiceproblemsolution}{5.10. \langnamePalauan}

\begin{solutions}[label=Solution 5.10\alph*]
    \item 
    \begin{enumerate}

    \begin{multicols}{2}
        \item \texttr{6 teachers}
        \item \texttr{19 months}
        \item \texttr{December}
        \item \texttr{Friday}
    \end{multicols}
    \end{enumerate}
    \item 
    \begin{enumerate}[start = 5]

    \begin{multicols}{2}
        \item \cmubdata{eai e̜l sils}
        \item \cmubdata{te̜ruich me a tetiu e̜l chad}
        \item \cmubdata{teuid e̜l sensei}
        \item \cmubdata{ongelolem e̜l buil}
        \item \cmubdata{ongeai e̜l buil}
        \item[] \quad 
    \end{multicols}
    \end{enumerate}
    \item 
    \begin{enumerate}[start = 10]

    \begin{multicols}{3}
        \item \cmubdata{ede}
         \item \cmubdata{tede}
          \item \cmubdata{klde}
    \end{multicols}
    \end{enumerate}
\end{solutions}

\rules
\begin{itemize}
    \item Structure: \fbox{Numeral + \cmubdata{e̜l} + Noun}
    \item Numerals have the following prefixes:
    \begin{itemize}

        \item \cmubdata{e-} = time periods (days, months, years)
        \item \cmubdata{te-} = people
        \item \cmubdata{kl-} = non-human nouns (which do not refer to persons)
        \item \cmubdata{onge-} = ordinal numerals
\end{itemize}
\item For numbers higher than 10, the prefix is attached to the units. 
\item 10 $+ X =$ \cmubdata{tęruich me a} $X$
\end{itemize}


\end{practiceproblemsolution}
\begin{practiceproblemsolution}{5.11. \langnameNorwegian}

\begin{solutions}[label=Solution 5.11\alph*]
    \item
    \begin{multicols}{5}
        \begin{enumerate}

            \item C.
            \item J.
            \item E.
            \item H.
            \item B.
            \item A.
            \item D.
            \item G.
            \item F.
            \item I.
        \end{enumerate}
    \end{multicols}
    \item 
\begin{enumerate}[start = 11]
\begin{multicols}{2}
    \item \cmubdata{Jenta stanser her.}
    \item \cmubdata{En jente har et hotell.}
    \item \cmubdata{Jeg har hundene.}
    \item \cmubdata{Jenta har hunder.}
\end{multicols}\end{enumerate}
    \item 
\begin{enumerate}[start = 15]
    \item \texttr{The men have apples.}
    \item \texttr{The woman does not have the shoes.}
    \item \texttr{I do not have the apples.}
\end{enumerate}
\end{solutions}

\rules

 Sentence structure: SOV (Subject-Object-Verb)

\begin{table}[H]
    \begin{tabular}{llllll}
    \lsptoprule
    & Last & \textsc{sg} indef. & \textsc{sg} def. & \textsc{pl} def. & \textsc{pl} indef. \\ 
    & letter & \texttr{a dog} & \texttr{the dog} & \texttr{the dogs} & \texttr{dogs} \\
    \midrule 
    {Neuter} & \cmubdata{e}       & \cmubdata{et $X$-e} & \cmubdata{$X$-et} & \cmubdata{$X$-ene} & \cmubdata{$X$-er}\\ 
             & $\neq$ \cmubdata{e} & \cmubdata{et $X$}  & \cmubdata{$X$-et} & \cmubdata{$X$-ene} & \cmubdata{$X$-er} \\ \midrule 

    {Feminine} & \cmubdata{e}        & \cmubdata{en $X$-e} & \cmubdata{$X$-a} & \cmubdata{$X$-ene} & \\ 
               & $\neq$ \cmubdata{e} & \cmubdata{en $X$}   & \cmubdata{$X$-a} & \cmubdata{$X$-ene} & \\ \midrule 

    {Masculine} & \cmubdata{e}        & \cmubdata{en $X$-e} & \cmubdata{$X$-en} & \cmubdata{$X$-ene} & \cmubdata{$X$-er} \\ 
                & $\neq$ \cmubdata{e} & \cmubdata{en $X$}   & \cmubdata{$X$-en} & \cmubdata{$X$-ene} & \cmubdata{$X$-er}\\
    \lspbottomrule
    \end{tabular}
\end{table}

\begin{table}[H]
\begin{tabular}{lll}
\lsptoprule
Neuter & Feminine & Masculine \\ \midrule
\texttr{apple}  & \texttr{woman} & \texttr{bus} \\
\texttr{hotel}  & \texttr{girl}  & \texttr{car}\\
                &                & \texttr{dog}\\
\lspbottomrule
\end{tabular}
\end{table}

\note{The nouns for \texttr{man} and \texttr{shoe} are not included in the table, since their gender cannot be determined.}
\end{practiceproblemsolution}

\begin{practiceproblemsolution}{5.12. \langnameAfrihili}

\begin{solutions}[label=Solution 5.12\alph*]
    \item \begin{itemize}[leftmargin = 1em]

    \begin{multicols}{2}
        \item[] \wordtrans{ajamura}{presidents}
        \item[] \wordtrans{amkamate}{baker}
        \item[] \wordtrans{oluga}{language}
    \end{multicols}
    \end{itemize}
    \pagebreak
    \item 
    \begin{itemize}[leftmargin = 1em]

    \begin{multicols}{2}
        \item[] \texttr{machinist} = \cmubdata{afimadi}
        \item[] \texttr{ships} = \cmubdata{imeli}
        \item[] \texttr{flower} = \cmubdata{ature}
        \item[] \texttr{group of girls} = \cmubdata{omukazisini}
        \item[] \texttr{date fruit} = \cmubdata{entindi}
        \item[] \texttr{shoe} = \cmubdata{isabatu}
        \item[] \texttr{king} = \cmubdata{omukama} 
         \end{multicols}
    \end{itemize}
\item 
\begin{tabularx}{\linewidth}[t]{llQ}
     \cmubdata{imulenzi} & = b. & \texttr{boys} (first and last vowels are identical \Rightarrow\ plural) \\
     \cmubdata{aposino} & = a. & \texttr{baggage} (affix \cmubdata{-sin-} \Rightarrow\ collective noun) \\
     \cmubdata{iwelemase} & = c. & \texttr{librarian} (affix \cmubdata{-ma-} \Rightarrow\ profession) \\
\end{tabularx}
\end{solutions}

\rules
\begin{itemize}
    \item All nouns begin and end with a vowel ($V_1RV_2$)
    \item Singular: $V_1 \neq V_2$, plural: $V_1 \rightarrow\ V_2$ ($V_1$ becomes $V_2$ \Rightarrow $V_2RV_2$)
    \item Derived nouns (from a basic noun $V_1RV_2$):
    \begin{itemize}

        \item head of an organisation: $V_2RV_1$ (first and last vowels switch places);
        \item profession: infix \cmubdata{-ma-} is inserted before the last syllable of the stem ($V_1C...CV_2$ \Rightarrow\ $V_1C...\text{\cmubdata{ma}}CV_2$);
        \item collective noun: $V_1RV_2\text{\cmubdata{sin}}V_2$ (add suffix \cmubdata{-sin$V$}, where $V$ is the last vowel of the stem);
        \item diminutive: $V_1RV_2\text{\cmubdata{nd}}V_2$
    \end{itemize}
\end{itemize}


\end{practiceproblemsolution}
\begin{practiceproblemsolution}{5.13. \langnameMaltese}

\begin{solutions}[label=Solution 5.13\alph*]
    \item \begin{enumerate}[label = (\arabic*)]
        \begin{multicols}{3}
            \item \texttr{goat}
            \item \texttr{black}
            \item \texttr{book}
            \item \texttr{red}
            \item \texttr{chair}
            \item \texttr{white}
            \item \cmubdata{blu}
            \item \cmubdata{ħomor}
            \item \cmubdata{kannella}
            \item \cmubdata{sofor}
            \item \cmubdata{ħadra}
            \item \cmubdata{vjola}
            \item \cmubdata{safra}
            \item \cmubdata{serp}
            \item \cmubdata{aħmar}
        \end{multicols}
    \end{enumerate}
    \item We do not know the first vowel of the stem for \texttr{white} ($V_1$).
\end{solutions}

\begin{sloppypar}
\rules There are two types of colours: invariable (\texttr{pink}, \texttr{purple}, \texttr{brown}, \texttr{blue}) -- which have the same form in all contexts -- and variable (\texttr{green}, \texttr{white}, \texttr{black}, \texttr{red}, \texttr{yellow}).
\end{sloppypar}

Nouns are divided into three categories:
\begin{enumerate}[label = Cat. \Roman*., leftmargin = 7em,noitemsep]
    \item Singular, end in a consonant;
    \item Singular, end in \cmubdata{a};
    \item Plural. 
\end{enumerate}

Based on these categories, the adjective is declined as follows:

\begin{enumerate}[label = Cat. \Roman*., leftmargin = 7em, noitemsep]
\item $V_1C_1C_2V_2C_3$ 
\item $C_1V_2C_2C_3$\cmubdata{a}
\item $C_1$\cmubdata{o}$C_2$\cmubdata{o}$C_3$
\end{enumerate}

\noindent where $C$ and $V$ refer to a consonant and a vowel, respectively.

 We notice here that $V_1$ appears only for Cat. I. Therefore, knowing the forms for Cat. II and III is not sufficient to deduce the form for Cat. I; similarly, only knowing the form for Cat. III is not enough to deduce the forms for Cat. I and II.




\end{practiceproblemsolution}
\note{ We refer here to problem 5.8 and the discussion thereafter. We notice that the five colours that can be inflected (\texttr{green}, \texttr{white}, \texttr{black}, \texttr{red}, \texttr{yellow}) are the first to enter the language, according to the hypothesis of Berlin \& Kay, while the invariable colours are those from a later stage (\texttr{blue}, \texttr{pink}, \texttr{purple}, \texttr{brown}) -- which, in this case, are loan words.
}
\begin{practiceproblemsolution}{5.14. \langnameLatvian}

\begin{solutions}[label=Solution 5.14\alph*]
    \item \begin{enumerate}[label = (\arabic*), leftmargin = 2em]
        \begin{multicols}{2}
            \item \texttr{of the bookshop}
            \item \cmubdata{autoru}
            \item \cmubdata{bruņinieku}
            \item \cmubdata{bērns}
            \item \texttr{disgusting}
            \item \cmubdata{sudrabs}
            \item \cmubdata{ozola}
            \item \cmubdata{ozolu}
        \end{multicols}\end{enumerate}

        \begin{enumerate}[label = (\arabic*), start = 9]
        \item \cmubdata{institūta} or \cmubdata{institūtu} (the former is used if we refer to the doctors from a single institute, while the latter is used to refer to the doctors from different institutes)
    \end{enumerate}
\end{solutions}


\begin{itemize}
    \item Suffixes of the noun head:

 \cmubdata{-s} = nominative \hfill  \cmubdata{-a} = genitive \textsc{sg} \hfill \cmubdata{-u} = genitive \textsc{pl}
    \item Determiners can be split into three groups: 
    \begin{itemize}\sloppy
        \item Those which agree with the noun head (receive the same suffix). In this category, we include all the qualifying adjectives (\texttr{tall}, \texttr{old}, \texttr{good}, \texttr{disgusting}).
        \item Those which receive the ending \cmubdata{-a} -- they are represented by Latvian nouns in genitive singular (\texttr{dairy products} = products \emph{of milk}).
        \item Those receiving the ending \cmubdata{-u} -- they are Latvian nouns in genitive plural (\texttr{paediatrician} = doctor \emph{of children}, \texttr{bookshop} = shop \emph{of books}).
    \end{itemize}\end{itemize}

The difference between the last two groups is purely semantic, depending on whether they refer to a singular or a plural noun (a \texttr{surgery table} is a table for \emph{surgeries} since multiple surgeries are performed on the same table; a \texttr{paediatrician} is a doctor of \emph{children} because they treat more children, not a single one, etc.)\\



\end{practiceproblemsolution}
\begin{practiceproblemsolution}{5.15. \langnameIlocano}

\begin{solutions}[label=Solution 5.15\alph*]
    \item 
    \begin{enumerate}

    \begin{multicols}{2}
    \item \texttr{appearance}
    \item \texttr{various appearances}
    \item \texttr{to look}
    \item \texttr{is looking}
    \item \texttr{happiness}
    \item \texttr{is skipping for joy}
    \item \texttr{skeleton}
    \item \texttr{various skeletons}
    \item \texttr{is becoming a skeleton}
    \item \texttr{to buy}
    \item \texttr{summit}
    \item \texttr{to reach the top}
    \end{multicols}
        \end{enumerate}
    \item \begin{enumerate}[start = 1, label = (\arabic*)]

    \begin{multicols}{2}
    \item \texttr{to become a skeleton}
    \item \texttr{various summits}
    \item \texttr{is reaching the top}
    \item \baytext{\char"1704\char"1706\char"1705\char"1714}
    \item \baytext{\char"1704\char"1713\char"170B\char"1706\char"1714\char"1704\char"1706\char"1705\char"1714}
    \end{multicols}
        \end{enumerate}
\end{solutions}

\rules
\begin{itemize}
    \item Stems: 
    \begin{itemize}

    \begin{multicols}{2}\raggedcolumns
        \item \texttr{appearance} = \baytext{\char"1703\char"1712\char"1706}
        \item \texttr{skeleton} = \baytext{\char"170D\char"1713\char"170D\char"1713\char"1704\char"1714}
        \item \texttr{purchase} = \baytext{\char"1704\char"1706\char"1705\char"1714}
        \item \texttr{happiness} = \baytext{\char"170D\char"1704\char"1714\char"1710\char"1703\char"1714}
        \item \texttr{summit / top} = \baytext{\char"1710\char"1709\char"1706\char"1714}
        \item[] \quad
    \end{multicols} 
    \end{itemize}
    \item Derivation processes:
    \begin{enumerate}[label = \alph*.]

        \item \emph{Partial reduplication}: copy the first two symbols and add them to the beginning of the word. The first symbol retains its diacritic, while the diacritic of the second symbol is replaced by a plus (\baytext{\char"25CC\char"1714}).
        \item \emph{Epenthesis}: insert \baytext{\char"170B} after the first symbol. The diacritic of the first symbol moves to this one and the first symbol receives an underdot (\baytext{\char"25CC\char"1713}).
    \end{enumerate}
\end{itemize}

 With these two processes, we can obtain three different transformations starting from the singular noun:

\begin{itemize}
    \item Plural noun (\texttr{various...}) (process a)
    \item Infinitive verb (process b)
    \item 3\textsc{sg} present cont. verb (process a, followed by b)
\end{itemize}



\end{practiceproblemsolution}
\begin{practiceproblemsolution}{5.16. \langnameIrish}

\begin{solutions}[label=Solution 5.16\alph*]
    \item  \begin{enumerate}[start = 16]
        \begin{multicols}{2}
        \item \texttr{99 boats}
        \item \texttr{16 people}
        \item \texttr{9 people}
        \item \texttr{20 boys}
        \item \texttr{31 gardens}
        \item[] \quad 
        \end{multicols}\end{enumerate}
        \item  \begin{enumerate}[start = 21]
        \begin{multicols}{2}
        \item \cmubdata{dhá ghasúr}
        \item \cmubdata{ocht mballa déag is fiche}
        \item \cmubdata{ceithre bhalla déag}
        \item \cmubdata{doras déag is trí fichid}
        \item \cmubdata{bád is fiche}
        \item \cmubdata{deich nduine is ceithre fichid}
        \end{multicols}\end{enumerate}
\end{solutions}

\rules

 Irish uses base 20; numbers are written as:

\exrule{($U$) + (10) + (20$X$) \Leftrightarrow\ ($U$) (\cmubdata{déag}) (\cmubdata{is} $X$ \cmubdata{fichid})}
\begin{itemize}
    \item $U$ is between 2 and 9
    \item number 10 has two forms: \cmubdata{déag} -- used if there is a unit number (from 2 to 9) -- and \cmubdata{deich} -- used only for 10 and multiples of 10 (30, 50, etc.)
    \item if $X = 1$, \cmubdata{is $X$ fichid} becomes \cmubdata{is fiche} 
\end{itemize}

Phrase structure: we can consider each structure to have four parts: I + II + III + IV.

\begin{itemize}
\item Part I is for the units (or \cmubdata{deich}).
\item Part II is always the noun.
\item Part III can only be filled by two words: \cmubdata{amháin} (meaning \cmubdata{1}, only for a singular noun) or \cmubdata{déag} (but not \cmubdata{deich}).
\item Part IV is always a multiple of 20 (structures \cmubdata{is $X$ fichid}\slash\cmubdata{is fiche}).
\end{itemize}

 The only exception takes place when the number of objects is a multiple of 20 (20, 40, etc.). In this case, Part I is occupied by $X$ \cmubdata{fichid}\slash\cmubdata{fiche} and the noun is placed at the end (in this case the particle \cmubdata{is} is not used). Therefore, the examples in the problem (and task (a)) can be analysed as follows:

\begin{longtable}{rlllll}
\lsptoprule
& \multicolumn{1}{c}{I} & \multicolumn{1}{c}{II} & \multicolumn{1}{c}{III} & \multicolumn{1}{c}{IV} & \multicolumn{1}{c}{Translation} \\ \midrule\endfirsthead
\midrule
& \multicolumn{1}{c}{I} & \multicolumn{1}{c}{II} & \multicolumn{1}{c}{III} & \multicolumn{1}{c}{IV} & \multicolumn{1}{c}{Translation} \\ \midrule\endhead
1. &  & \cmubdata{garra} & \cmubdata{amháin} &  & \texttr{1 garden} \\
2. &  & \cmubdata{gasúr} & \cmubdata{déag} &  & \texttr{11 boys} \\
3. & \cmubdata{ocht} & \cmubdata{mballa} &  & \cmubdata{is dhá fichid} & \texttr{48 walls} \\
4. & \cmubdata{dhá} & \cmubdata{gharra} & \cmubdata{déag} & \cmubdata{is ceithre fichid} & \texttr{92 gardens} \\
5. & \cmubdata{trí} & \cmubdata{bhád} &  &  & \texttr{3 boats} \\
6. & \cmubdata{seacht} & \cmubdata{ndoras} & \cmubdata{déag} &  & \texttr{17 doors} \\
7. & \cmubdata{seacht} & \cmubdata{mbád} & \cmubdata{déag} & \cmubdata{is dhá fichid} & \texttr{57 boats} \\
8. & \cmubdata{naoi} & \cmubdata{nduine} & \cmubdata{déag} & \cmubdata{is fiche} & \texttr{39 people} \\
10. & \cmubdata{cúig} & \cmubdata{bhalla} &  &  & \texttr{5 walls} \\
11. & \cmubdata{sé} & \cmubdata{ghasúr} &  & \cmubdata{is trí fichid} & \texttr{66 boys} \\
12. & \cmubdata{deich} & \cmubdata{mbád} &  &  & \texttr{10 boats} \\
13. & \cmubdata{sé} & \cmubdata{dhuine} &  &  & \texttr{6 people} \\
14. & \cmubdata{trí} & \cmubdata{dhoras} &  & \cmubdata{is dhá fichid} & \texttr{43 doors} \\
15. &  & \cmubdata{garra} &  & \cmubdata{is ceithre fichid} & \texttr{81 gardens} \\
16. & \cmubdata{naoi} & \cmubdata{mbád} & \cmubdata{déag} & \cmubdata{is ceithre fichid} & \texttr{99 boats} \\
17. & \cmubdata{sé} & \cmubdata{dhuine} & \cmubdata{déag} &  & \texttr{16 people} \\
18. & \cmubdata{naoi} & \cmubdata{nduine} &  &  & \texttr{9 people} \\
20. &  & \cmubdata{garra} & \cmubdata{déag} & \cmubdata{is fiche} & \texttr{31 gardens} \\ \midrule 
9. & \cmubdata{ceithre fichid} & \cmubdata{doras} &  &  & \texttr{80 doors} \\
19. & \cmubdata{fiche} & \cmubdata{gasúr} &  &  & \texttr{20 boys} \\
\lspbottomrule
\end{longtable}

 We separated examples 9 and 19 to highlight the case in which the number of objects is a multiple of 20.

 Moreover, we can notice from the table that the difference between, for example, 20 and 21 (or 40/41, 60/61 etc.) is only based on the position of the noun.

Lastly, we notice that the noun has a variable form. In this case, it undergoes an initial consonant mutation. We notice three types of initial consonants:
\begin{itemize}
    \item simple consonants (\cmubdata{b}, \cmubdata{d}, \cmubdata{g}) -- used if the number of units is 1 or if the number of objects is a multiple of 20 (in other words, if position I is empty or occupied by a multiple of 20);
    \item consonants followed by \cmubdata{h} (\cmubdata{bh}, \cmubdata{dh}, \cmubdata{gh}) -- used if the number of units is between 2 and 6 (or if position I is occupied by 2--6);
    \item consonants preceded by a nasal (\cmubdata{mb}, \cmubdata{nd}) – used if the unit number is 7--10 (10 refers only to the form \cmubdata{deich}, which appears in the first position). In the problem, there are no examples and we are not asked to write what happens with the consonant \cmubdata{g} in this context, but we can notice that the nasal added before assimilates to the place of articulation.
\end{itemize}



\end{practiceproblemsolution}
\begin{practiceproblemsolution}{5.17. \langnameIaai}

\begin{solutions}[label=Solution 5.17\alph*]
    \item
    \begin{multicols}{5}
        \begin{enumerate}
            \item E.
            \item D.
            \item H.
            \item F.
            \item C.
            \item G.
            \item A.
            \item B.
            \item O.
            \item N.
            \item Q.
            \item I.
            \item J.
            \item K.
            \item P.
            \item M.
            \item L.
        \end{enumerate}
    \end{multicols}
     \item
        \begin{enumerate}[start = 18]
            \item \texttr{your watermelon (for drinking)}
            \item \texttr{the mother's dugout}
        \end{enumerate}
    \item
    \begin{multicols}{2}
        \begin{enumerate}[start = 20]

            \item \cmubdata{trii belen waau}
            \item \cmubdata{nu a Kua} or \cmubdata{nu bele Kua}
            \item \cmubdata{hoon ok} or \cmubdata{anyin ok}
            \item \cmubdata{haaleik kuli}
        \end{enumerate}
    \end{multicols}
\end{solutions}

\rules

 Structure:
\begin{itemize}

    \item[] $X$'s $Y$ = 
    \begin{tabular}[t]{ll}
        $C$--Poss $Y$ & ($X$ = pronoun) \\
        $Y$ $C$--Poss $X$&($X$ = common noun)\\
        $Y$ $C$ $X$ &($X$ = proper noun) \\
    \end{tabular} 
    \item[] Poss: 
    \begin{tabular}[t]{ll}
         \cmubdata{-k} & ($X$ = 1\textsc{sg}) \quad\quad \cmubdata{e \rightarrow\ i / \_ k}\\
       \cmubdata{-m} & ($X$ = 2\textsc{sg})\\
        \cmubdata{-n} & ($X$ = 3\textsc{sg})
    \end{tabular}
    \item[] $C$: 
    \begin{tabular}[t]{ll}
         \cmubdata{a} & $Y$ = food*  \\
         \cmubdata{bele} & $Y$ = drinks* \\
         \cmubdata{haalee} & $Y$ = animals \\
         \cmubdata{hoo} & $Y$ = boats\textsuperscript{\dag} \\
         \cmubdata{tabe} & $Y$ = things you can sit on/in\textsuperscript{\dag} \\
         \cmubdata{anyi} & otherwise \\
    \end{tabular}
\end{itemize}

 * Fruits can be either ‘food’\ or ‘drink’\ depending on how the speaker intends them to be consumed.

 \textsuperscript{\dag} In the case of younger generations (Speaker 2), these types of noun also fall into the \cmubdata{anyi} category (new generations tend to simplify the classifier system and give up on very specific classifiers, preferring to use the general classifier). \\


\end{practiceproblemsolution}

\nocite{AikhenvaldDixon2013, Booij2012, CabredoHofherrZribi-Hertz2013, Rijkhoff2002}
% \printbibliography[heading=FurtherReading]
\FurtherReadingBox{}

% \begin{enumerate}[{label=[\arabic{*}]}]
%
%     \item Aikhenvald, Alexandra Y. and Dixon, Robert M. W. (ed). “Possession and ownership.”\ \textit{Oxford University Press}, Oxford (2013).
%     \item Booij, Geert. “The grammar of words.”\ \textit{Oxford University Press}, Oxford (2012).
%     \item  Hofherr, Patricia Cabredo and Zribi-Hertz, Anne (ed.). “Crosslinguistic studies on noun phrase structure and reference.”\ \textit{Brill Academic Publishers}, Leiden (2013).
%     \item Rijkhoff, Jan. “The noun phrase.”\ \textit{ Oxford University Press}, Oxford (2002).
% \end{enumerate}
\end{refsection}
