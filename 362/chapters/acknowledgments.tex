\addchap{\lsAcknowledgementTitle}\largerpage

My M.A.\ and Ph.D.\ advisor Stefan Müller provided my very first introduction into the wonderland of linguistics, back in 2007 when I took his class on morphology. Meeting him literally changed my life, and I owe him so much that it is ridiculous to even attempt to make a list (which I'm attempting nevertheless): He has been my gateway to formal linguistics, to formal syntax and to HPSG; has been an example of scientific discipline and intellectual honesty; and financed my Ph.D.\ thesis by hiring me on his scientific staff (which, trivial as it sounds, was not the least important). My gratitude to him is +$\infty$.

I am so honored that Anne Abeillé agreed to be my thesis advisor in 2017, despite my uttermost ignorance of French linguistics at that time. She is the origin of the strong bent of this book toward an experimental approach. I am immensely grateful for all the time she invested in me and this piece of work, as well as for the long hours of discussions she never seemed in a hurry to end. I have been able to ask her any silly question that came to my mind, and always got an answer. 

She was also the one who suggested I apply for a binational cotutelle. Thanks to that, I had the opportunity to be part of the great Laboratoire de Linguistique Formelle at the Université Paris Cité. From this team, I want to thank especially Olivier Bonami, who directed the lab during my stay and always made me feel welcome and a full member of the lab. I am grateful for having met and worked with Barbara Hemforth, who taught me a lot about the methodology of experiments and statistical analyses. 
Many thanks to Danièle Godard and Berthold Crysmann, who gave me so much good advice regarding my research and the formalization of my analysis. I also wish to thank all members of the lab who supported my research: Céline Pozniak, Caterina Donati, Heather Burnett, Yair Haendler, Pascal Asmili; and the cheering group of Ph.D.\ students who helped me survive through the process\footnote{Mostly, I must say, because they fed me. A lot.}: Suzanne Lesage, Gabriel Thiberge, Bujar Rushiti, Quentin Dénigot, Saida Loucif and Antoine Simoulin.

At the Humboldt-Universität zu Berlin, which I joined in 2016, I wish to thank all the members of the Syntax Group for their support, especially Marc Felfe and Chenyuan Deng. I also thank  Elisabeth Verhoefen and Paula Fritz Huechante, as well as Pia Knoeferle and her whole staff, with whom I really enjoyed talking about our respective research. 
Most of this work was funded by the Deutsche Forschungsgemeinschaft (DFG) as I worked in the project \emph{Long-distance dependencies in French}, with Guido Mensching, Franziska Werner and Esma Tanis. I thank them all for their help and advice.

For some part of my time as a Ph.D.\ student, I was also employed at the Freie Universität Berlin. I have met so many people there during my years as a MA and as a Ph.D.\ student, who were of so much importance in my life, that I can't make a complete list. First of all I wish to thank Horst Simon and his whole crew (especially Tanja Ackermann and Christian Zimmer) for their warmth and support. I am very grateful to Felix Bildhauer, who had supervised my M.A.\ thesis with much consideration and whose doctoral dissertation has been an important early inspiration for my work. At the FU, I was also lucky enough to meet my dear colleague and friend Naomi Truan, who gave me very good advice for finding my way in the cotutelle maze. Finally, I wish to express a (very) big thank-you to Viola Auermann for her support and patience.\largerpage

A special elbow bump goes to my Dr.uder Antonio Machicao y Priemer and to Aixiu An, my thesis ``siblings''. They made the journey so much more interesting and fun. Thank you for the countless hours of chatting. And just in general for everything.

A great deal of this work relies on the valuable data that I was able to collect. I cannot thank the French language itself for being so amazing,  but I can at least thank all of those who participated in the (mostly unpaid) experiments, shared the experiments in order to help me, and did not even complain that rating strange sentences was such a boring task to do. I also thank the staff and colleagues who helped set up the studies.

\begin{sloppypar}
Two thesis reporters, Rui Chaves and Dominique Sportiche, and two anonymous reviewers made excellent comments and remarks, which forced me to re-think several sections of the book, and helped improve the final manuscript. I wish to express my gratitude to the people who proofreaded this book: Maryvonne Gerin, Simon Luck, Carla Bombi, Janina Rado, Hannah Schleupner, Amir Ghorbanpour, Elliott Pearl, Brett Reynolds, Wilson Lui, Annie Zaenen and Felix Kopecky. It goes without saying that all remaining errors are mine.
\end{sloppypar}

Writing this book was not only a great intellectual adventure but also a tough time that I managed to overcome thanks to three very special people. Stephanie Gagne was always virtually here with me from the very start and until the very end in some way, and Zina Cohen was literally always here with me (talking about food and statistics, my two favorite topics). And of course, above all, I thank Antoine Laslier for his unconditional love and unconditional support.\footnote{Example (\ref{ex:d2000-subj-unerg}) is dedicated to you.}  

Finally, I want to thank my mother, because she's just the best. I'm convinced that no one has ever had such a wonderful, patient and supportive mother as I do. As my mother is never far from the rest of my family, I have to express my thanks going around the table: Jean-Jacques, Michèle, Dédé, Jeannine, Olivier, Ingrid, Jean-Christophe, Béatrice, Mathieu, Élise, Alexandre, Sophie, Julien, Aurélie, Paul, Mathias, Charlotte, Sarah, Paul, Emma, Jules, Antoine, Hortense, Victor, Claire, Raphaël and Benjamin. I love y'all to the moon and back.

%``I hope that in my criticisms of the three conditions proposed by Chomsky I have not given the impression that I wish to belittle them, merely because they can be proven to be wrong today; for the contrary is true: these conditions, in particular the A-over-A principle, provide the basis for the present work.'' \citep[35--36]{Ross.1967}

The experiments reported in this book were approved by the CERES, (Conseil d'évaluation éthique pour les recherches en santé) of the University of Paris Descartes. 

My work over the last year has been supported by the German Research Foundation (DFG) as part of the research program ``Long-distance Dependencies in French: Comparative Analyses (HPSG and the Minimalist Program)'', Guido Mensching, Göttingen (ME 1252/14-1)\slash Stefan Müller, Berlin (MU 2822/9-1). It was also supported by a public grant overseen by the French National Research Agency (ANR) as part of the program ``Investissements d'Avenir'' (reference: ANR-10-LABX-0083). It contributed to the IdEx Université Paris Cité (ANR-18-IDEX-0001). Finally, some of this work has been conducted as part of a research position at the Freie Universität Berlin.

\addchap{Collaborative work}

The corpus studies presented in this book have all been conducted with my supervisor Anne Abeillé. Barbara Hemforth also collaborated on the corpus study that we published in \citet{Abeille.2016}.

The extraction of the data from the corpus was carried out by Anne Abeillé for the French Treebank, and by myself for Frantext. We agreed on the guidelines for the annotation. The annotation was almost entirely my work, but Anne Abeillé would help me decide uncertain cases. I am completely responsible for the statistical analysis, but I received useful comments from Barbara Hemforth, Anne Abeillé and many other researchers who discussed my work with me during many conferences and workshops. The final decision on the statistical methods and interpretation of the results remained my own.

The experiments have all been conducted with my supervisor Anne Abeillé, and many of them in collaboration with Barbara Hemforth and Ted Gibson. 

For most studies, the materials were first constructed by me, and then discussed and modified with Anne Abeillé before we agreed on a final version. Sometimes, Ted Gibson constructed English material, which was then discussed and modified with our remarks, and translated into French by me or Anne Abeillé to construct a parallel experiment in French. I only present the French experiments in this work, but the interested reader will find a short discussion of the English experiments in Section \ref{ch:exp-conclu-eng} and is otherwise referred to \citet{Abeille.2020.Cognition}.

Internet experiments with acceptability judgment tasks were set up on the Ibex Platform by one of the Ph.D.\ students at the Laboratoire de Linguique Formelle (Aixiu An, Céline Pozniak or myself)  \citep{Ibex}. Lab experiments were set up by Etienne Riou (speeded acceptability judgment), Céline Pozniak (eye tracking) and Aixiu An (self-paced reading).

I am completely responsible for the statistical analysis, but I received useful comments from Anne Abeillé, Ted Gibson, Barbara Hemforth, Céline Pozniak, Yair Haendler and many other researchers who discussed my work with me at many conferences and workshops. The final decision on the statistical methods and interpretation of the results remained my own.
