I have presented a series of experiments on relative clauses and a series of experiments on interrogatives that show a strong cross-construction difference. Usually, approaches to subject islands expect all extractions to have a similar behavior. Only the FBC constraint predicts the results we have observed in the previous experiments. This chapter is about another construction: \emph{it}-cleft sentences (\emph{c'est}-cleft in French).

I will discuss the discourse status of the different elements in \emph{it}-clefts, and explain why \emph{it}-clefts are a good way to test the predictions of the FBC constraint. I will then report on an experiment:

\begin{description}

\item[Experiment 14:] In this acceptability judgment study, we crossed extraction type (subextraction\slash non-subextraction) with syntactic function (subject\slash object) and tested \emph{c'est}-clefts. The extraction out of the subject received significantly lower ratings than extraction out of the object, but significantly higher than ungrammatical controls. There was no significant interaction.

\end{description}

\section{Information structure of \emph{c'est}-clefts}
\label{ch:is-clefts}

% difference c'est cleft / il y a cleft (presentational). 

French \emph{c'est}-clefts are similar to English \emph{it}-clefts at the level of syntax, but show some subtle differences from their English counterpart with respect to pragmatics. Two parts can be identified in this construction, as illustrated in (\ref{ex:cleft}): the pivot and the \emph{that}-clause (or \emph{que}-clause in French). 

\eal\label{ex:cleft}
\ex[]{[It is tomorrow$_i$]\textsubscript{pivot} [that we have to leave~\trace{}$_i$]\textsubscript{that-clause}.}
\ex[]{\gll [C' est demain$_i$]\textsubscript{pivot} [que nous devons partir~\trace{}$_i$]\textsubscript{que-clause}.\\
it is tomorrow that we must leave\textsc{.inf}\\}
\zl 

\emph{It}-clefts are a special kind of filler-gap dependency. The pivot comprises an expletive \emph{c(e)} `it', the copula \emph{est} `is' and an argument: it is a XP that I will designate in the following as the ``filler'' of the \emph{it}-clefted structure, even though strictly speaking it is not a filler from the syntactic point of view. In an \emph{it}-cleft construction, the filler is focused \citep{Lambrecht.1994}. 
The extracted element is then presupposed to be exhaustive relative to the proposition expressed by the \emph{que}-clause.

The kind of focus involved in \emph{it}-clefts has been discussed extensively (see \citet{Destruel.2019} for an overview). \citet{Prince.1978} said that the focus in \emph{it}-clefts is contrastive. However, \citet{Destruel.2012} and \citet{Destruel.2019} have shown experimentally that the focus in English \emph{it}-clefts and French \emph{c'est}-clefts expresses what they call ``contrariness'': ``clefts signal a commitment on the part of the interlocutor to a proposition that conflicts with the one the cleft expresses and [\dots] express opposition to that commitment.'' As such, it is a corrective focus more than a contrastive focus. We can illustrate this with the example (\ref{ex:material-destruel-2019}) below: the commitment \emph{k} in question here is that Alice told Amy about her surprise party, a commitment to which speaker B is opposed (because speaker B knows that Ken told her about it). The strength of the commitment varies in (\ref{ex:material-destruel-2019}) depending on speaker A's utterance. In (\ref{ex:material-destruel-2019}a), it is non-existent: there is no reason for speaker B to assume that speaker A is committed to \emph{k}. In (\ref{ex:material-destruel-2019}b), the commitment is present, but weak: speaker A does not seem strongly committed to \emph{k}. In (\ref{ex:material-destruel-2019}c), the commitment is strong, stronger than in (\ref{ex:material-destruel-2019}b): speaker B has every reason to assume that A has a strong commitment to \emph{k}.\largerpage

\eal\label{ex:material-destruel-2019}
\ex Non-contradictory:\\
Speaker A: We were planning Amy's surprise party for weeks. I can't believe she found out about it. I guess someone from the staff told her.\\
Speaker B: Actually, it's Ken who told her about it.
\ex Weak commitment:\\
Speaker A: We were planning Amy's surprise party for weeks. I can't believe she found out about it. I guess Alice must have told her.\\
Speaker B: Actually, it's Ken who told her about it.
\ex Strong commitment:\\
Speaker A: We were planning Amy's surprise party for weeks. I can't believe she found out about it. Alice told her about it, you know.\\
Speaker B: Actually, it's Ken who told her about it.\\
(examples from \citealt[5]{Destruel.2019})
\zl 

The results of an experiment conducted by \citet{Destruel.2019} using materials similar to (\ref{ex:material-destruel-2019}) show that the strength of the commitment \emph{k} is a good predictor for the acceptability of the \emph{it}-cleft (speaker B's utterance): the stronger the commitment, the higher the acceptability. \citet{Destruel.2019} conclude that the focus of \emph{it}-clefts (at least in French and English) does not merely express contradiction: both (\ref{ex:material-destruel-2019}b) and (\ref{ex:material-destruel-2019}c) contradict A's statement, so if this were the only factor then they should not differ in acceptability. Contrariness, however, incorporates strength of commitment in addition to contradiction. 

% Definition of contrariness in Zimmermann 2008
        % Zimmermann, M. (2008). “Contrastive focus and emphasis,” in Acta Linguistica Hungarica is a Quarterly Hungarian Peer-reviewed Academic Journal in the     Field of Linguistics, ed F. Kiefer (Akadémiai Kiadó), 347–360.
% Definition of informational focus in Kiss 1998
        % Kiss, K. É. (1998). Identificational focus versus information focus. Language 74, 245–273. doi: 10.1353/lan.1998.0211
% Definition of corrective focus in Gussenhoven 2008
        % Gussenhoven, C. (2008). “Types of focus in english,” in Topic and Focus, eds C. Lee, M. Gordon, and D. Büring (Heidelberg; New York, NY; London: Springer), 83–100.

\citet{Destruel.2012} also shows that French \emph{c'est}-clefts can mark information focus. This kind of focus can range over the whole sentence (broad focus) or over one element (narrow focus). Having main stress on the beginning of an intonational phrase is strongly dispreferred in French. Therefore, it is dispreferred -- though not at all infelicitous -- to mark narrow information focus on the subject via intonation. Instead, a frequent and preferred option in French is to use clefted subjects (i.e.\ fillers in the pivot corresponding to a subject gap in the \emph{que}-clause) for this purpose. % I could quote Lambrecht and Hamlaoui 2008, and reproduce syntactic structure from Hamlaoui 2008 
        % Hamlaoui, F. (2008). “Focus, contrast, and the syntax-phonology interface: the case of French cleft-sentences,” in Current Issues in Unity and Diversity of Languages: Collection of the Papers Selected From the 18th International Congress of Linguists (Seoul: Linguistic Society of Korea).

For the same reason, there is a preference for clefted subjects over clefted objects (as experimentally shown by \citet{Destruel.2012} and \citet{Destruel.2019}), even though the latter are also fully acceptable. In English we don't see a similar preference, subjects can be in situ and receive main stress without a problem. 

% "French bans prosodic marking on heavy NPs in sentence-initial position" Destruel 2012, 105

\section[head=Experiment 14]{Experiment 14: Acceptability judgment study on \emph{c'est}-clefts}
\label{ch:exp14}

If \emph{it}-clefts are focalization, extraction out of the subject by means of \emph{it}-clefting should be more similar to interrogatives than to relative clauses. This is an important test of the FBC constraint, because \emph{it}-clefts are at the same time syntactically closer to relative clauses than to interrogatives. That is why I close this series of experiments on subextraction from subject NPs by testing \emph{c'est}-clefts.

\subsection{Design and materials}

To construct the stimuli for Experiment~14, we used the items already tested in Experiments 4, 10, 11 and 13 and turned the items into \emph{c'est}-clefts. We tested \emph{c'est}-clefts with a short-distance dependency. Subextraction from the subject placed the \emph{de}-complement into the pivot.

\eal 
\ex[]{{Condition subject + PP-extracted:}\nopagebreak\\
\gll [C' est [de cette innovation]$_i$ que [l' originalité~\trace{}] enthousiasme mes collègues sans aucune raison.\\
it is of this innovation that the uniqueness excites my colleagues without any reason\\
\glt `It is of this innovation that the uniqueness excites my colleagues for no reason.'}
\label{ex:exp14-subj-pp}
\ex[]{{Condition object + PP-extracted:}\nopagebreak\\
\gll C' est [de cette innovation]$_i$ que mes collègues admirent [l' originalité~\trace{}$_i$] sans aucune raison~?\\
it is of this innovation that my colleagues admire the uniqueness without any reason\\
\glt `It is of this innovation that my colleagues admire the uniqueness for no reason.'}
\label{ex:exp14-obj-pp}
\zl 

It is not possible to form \emph{c'est}-clefts without extraction, so we used as felicitous control extraction of the whole subject (\ref{ex:exp14-subj-no}) or of the whole object (\ref{ex:exp14-obj-no}). 

\eal \label{ex:exp14-no}
\ex[]{{Condition subject + noextr:}\\
\gll C' est l‘ originalité de cette innovation qui enthousiasme mes collègues sans aucune raison.\\
it is the uniqueness of this innovation that\textsc{.subj} excites my colleagues without any reason\\
\glt `It is the uniqueness of this innovation that excites my colleagues for no reason.'}
\label{ex:exp14-subj-no}
\ex[]{{Condition object + noextr:}\\
\gll C' est l' originalité de cette innovation que mes collègues admirent sans aucune raison.\\
it is the uniqueness of this innovation that my colleagues admire without any reason\\
\glt `It is the uniqueness of this innovation that my colleagues admire for no reason.'}
\label{ex:exp14-obj-no}
\zl 

The ungrammatical controls were constructed by leaving out the preposition of the extracted element in  subextraction conditions.

\eal 
\ex[]{{Condition subject + PP-extracted:} \nopagebreak\\
\gll [C' est cette innovation que l' originalité enthousiasme mes collègues sans aucune raison.\\
it is this innovation that the uniqueness excites my colleagues without any reason\\
\glt `It is this innovation that the uniqueness excites my colleagues for no reason.'}
\label{ex:exp14-subj-un}
\ex[]{{Condition object + PP-extracted:}\nopagebreak\\
\gll C' est cette innovation que mes collègues admirent l' originalité sans aucune raison~?\\
it is this innovation that my colleagues admire the uniqueness without any reason\\
\glt `It is this innovation that my colleagues admire the uniqueness for no reason.'}
\label{ex:exp14-obj-un}
\zl 

We tested the same 24 items as in Experiment~4, 10, 11 and 13, each manipulated according to the six conditions described above. In addition, the experiment included 24 distractors. The distractors were declarative sentences. About two third of the experimental items and distractors were followed by a comprehension question. The sample item above was paired with the comprehension question \emph{Est-ce que les collègues ont raison d'être enthousiastes~?} (`Are the colleagues right to be enthusiastic?').

\subsection{Predictions}

As discussed before, the FBC constraint is the only approach that assumes a difference between constructions. \emph{C'est}-clefts imply the focalization of the extracted element, and for this reason, under the FBC constraint, we expect the results of Experiment~14 to be similar to the findings on the interrogatives (at least interrogatives with an extraction). 

The predictions of the other accounts remain the same, except for the processing account based on linear distance. The non\hyp subextraction conditions in this experiment involve extraction. The contrast between extraction of the subject and extraction of the object has been studied extensively (especially for relative clauses): extraction of the subject is rated higher, read faster and processed more easily than extraction of the object (\citealt{Gibson.1998}, \citealt{Pozniak.2015} to name just a few references on this topic). 

This subject preference is compatible with processing accounts based on linear distance, and sometimes motivated these accounts in the first place. However, in the present experiment, it should not lead to the interaction effect we have previously observed:  There is one referent between the filler and the gap in the subject subextraction condition (sg.) (\ref{ex:exp14-subj-pp}) and none in the extraction of the subject (\ref{ex:exp14-subj-no}), while there are three referents between the filler and the gap in the object subextraction condition (sg.)  (\ref{ex:exp14-obj-pp}) and two in the extraction of the object (\ref{ex:exp14-obj-no}).

All predictions for \emph{c'est}-clefts are summarized in Table~\ref{tab:exp14-predictions}.

% http://www.tablesgenerator.com/#

\begin{sidewaystable}
\oneline{%
\begin{tabular}{llllll}
\lsptoprule
 &
  \multicolumn{5}{c}{{Predictions}} \\ \cmidrule(lr){2-6} 
 &
  \multicolumn{3}{c}{{“subject island” accounts}} &
  \multicolumn{2}{c}{{no-island accounts}} \\ \cmidrule(lr){2-6} 
 &
  {\begin{tabular}[c]{@{}l@{}}“traditional” \\ syntactic account\end{tabular}} &
  {\begin{tabular}[c]{@{}l@{}}processing account \\ with surprisal \\ due to subject \\ complexity\end{tabular}} &
  {\begin{tabular}[c]{@{}l@{}}BCI account\\ (Goldberg 2006)\end{tabular}} &
  {\begin{tabular}[c]{@{}l@{}}account based\\ on linear distance\\ (DG, DLT)\end{tabular}} &
  {\begin{tabular}[c]{@{}l@{}}FBC constraint \\ account\end{tabular}} \\ \midrule
\multicolumn{1}{l}{{\begin{tabular}[c]{@{}l@{}}extractions \\ out of the subject\\ vs. extractions \\ out of the object\end{tabular}}} &
  (a) \textless (b) &
  (a) \textless (b) &
  (a) \textless (b) &
  (a) \textgreater (b) &
  (a) \textless (b) \\ \midrule
\multicolumn{1}{l}{{\begin{tabular}[c]{@{}l@{}}extractions vs. \\ non-extractions\end{tabular}}} &
  \begin{tabular}[c]{@{}l@{}}main effect of extraction\\ + interaction effect \\ such that (a) \textless (b,c,d)\end{tabular} &
  \begin{tabular}[c]{@{}l@{}}main effect of extraction\\ + interaction effect \\ such that (a) \textless (b,c,d)\end{tabular} &
  \begin{tabular}[c]{@{}l@{}}main effect of extraction\\ + interaction effect \\ such that (a) \textless (b,c,d)\end{tabular} &
  \begin{tabular}[c]{@{}l@{}}interaction effect,\\ such that (b) \textless (a,c,d)\end{tabular} &
  \begin{tabular}[c]{@{}l@{}}main effect of extraction\\ + interaction effect \\ such that (a) \textless (b,c,d)\end{tabular} \\ \midrule
\multicolumn{1}{l}{{\begin{tabular}[c]{@{}l@{}}extractions \\ out of the subject\\ vs. ungrammatical\\ controls\end{tabular}}} &
  (a) $\simeq$ (e) &
  (a) \textgreater (e) &
  (a) \textgreater (e) &
  (a) \textgreater (e) &
  (a) \textgreater (e) \\ \midrule
\multicolumn{1}{l}{{\begin{tabular}[c]{@{}l@{}}extractions vs.\\ vs. ungrammatical\\ controls\end{tabular}}} &
  \begin{tabular}[c]{@{}l@{}}interaction effect\\ such that (b) \textgreater (a,e,f)\end{tabular} &
  \begin{tabular}[c]{@{}l@{}}main effect\\ of grammaticality\end{tabular} &
  \begin{tabular}[c]{@{}l@{}}main effect\\ of grammaticality\end{tabular} &
  \begin{tabular}[c]{@{}l@{}}main effect\\ of grammaticality\end{tabular} &
  \begin{tabular}[c]{@{}l@{}}main effect\\ of grammaticality\end{tabular} \\ \lspbottomrule
\end{tabular}}
\caption{Predictions of the different accounts for Experiments 10, 12 and 13 (interrogatives with extraction). \emph{Notes}: (a) Condition subject + PP-extracted (b) Condition object + PP-extracted (c) Condition subject + no extraction (d) Condition object + no extraction (e) Condition subject + ungrammatical (f) Condition object + ungrammatical.}
\label{tab:exp10-predictions}
\end{sidewaystable}


\subsection{Procedure} 

We conducted the experiment on the Ibex platform \citep{Ibex}. The procedure was similar to the procedure used in the previous acceptability judgment experiments (see Section \ref{ch:methodo-AJ}). Participants rated the sentences on a Likert scale from 0 to 10, 0 being labeled as ``bad'' and 10 being labeled as ``good''. They also answered comprehension questions after some of the sentences.

The experiment took approximately 20 minutes to complete. 

\subsection{Participants}

The study was run between April and May 2019. 
Participants were recruited on the R.I.S.C.\ website (\url{http://experiences.risc.cnrs.fr/}) and on social media (e.g.\ Facebook).
They received no financial compensation.

24 participants took part in the experiment. 
The analysis presented here is based on the data from the 21 participants who satisfied all criteria.
They were aged 22 to 76 years. 17 of them self-identified as women, 4 self-identified as men. Six of them (12.77\%) indicated having an educational background related to language.

\subsection{Results and analysis}

\figref{fig:exp14-boxplot} shows the results of the acceptability judgment task. The ratings for the subextraction conditions were relatively low. The extraction out of the subject (\ref{ex:exp14-subj-pp}) received a mean  rating of 2.60, which is lower than the mean rating of 3.75 in the extraction out of the object (\ref{ex:exp14-obj-pp}). The grammatical controls had high  ratings: 7.93 in the subject control condition (\ref{ex:exp14-subj-no}), and 7.70 in the object control condition (\ref{ex:exp14-obj-no}). The ungrammatical controls received the lowest ratings: 2.15 in the subject condition (\ref{ex:exp14-subj-un}), and 2.07 in the object condition (\ref{ex:exp14-obj-un}). 

\begin{figure}
    \centering
    \includegraphics[width=\textwidth]{chapters/part2-Empirical/Exp14-clefts/boxplots.jpeg}
    \caption{Acceptability judgments by condition in Experiment~14. The grey box plots indicate the median and quartiles of the results. Black points are outliers. Mean and confidence intervals are indicated in white.}
    \label{fig:exp14-boxplot}
\end{figure}

\figref{fig:exp14-boxplot} suggests a potential ceiling effect in the grammatical controls and a potential floor effect in the ungrammatical controls. \figref{fig:exp14-repartition} indicates the same, but the distribution in the subextraction conditions seems relatively normal.

\begin{figure}
    \centering
    \includegraphics[width=\textwidth]{chapters/part2-Empirical/Exp14-clefts/repartition.jpeg}
    \caption{Density of the ratings across conditions for Experiment~14}
    \label{fig:exp14-repartition}
\end{figure}

Another representation of the results is provided by the ROC and zROC curves of the results in \figref{fig:exp14-ROC} on page \pageref{fig:exp14-ROC}. The ROC curves show that participants discriminated between the ungrammatical baselines and the other conditions, even though the subextraction from subject condition is close to the baseline. Corroborating what we see on \figref{fig:exp14-boxplot}, the non-subextraction conditions build larger curves than the subextraction conditions. The zROC curves are relatively straight and parallel to the baseline. 

\begin{figure}
    \centering
    \includegraphics[width=\textwidth]{chapters/part2-Empirical/Exp14-clefts/ROC.jpeg}
    \includegraphics[width=\textwidth]{chapters/part2-Empirical/Exp14-clefts/zROC.jpeg}
    \caption{ROC curves (top) and zROC curves (bottom) of the non-extraction conditions compared to their respective subextraction conditions, represented by the dotted grey baseline (\citealt{Dillon.2019}'s method) in Experiment~14.}
    \label{fig:exp14-ROC}
\end{figure}

The ROC and zROC curves in \figref{fig:exp14-ROC-subj} on page \pageref{fig:exp14-ROC-subj} show the discrimination between the subject and object conditions. The curves for the object condition are barely discriminated from the baseline, except for the subextraction condition, where the difference between the syntactic functions seems more important. The zROC curves indicate that the distribution is not completely normal, except for the subextraction condition that has a straight line parallel to the baseline.

\begin{figure}
    \centering
    \includegraphics[width=\textwidth]{chapters/part2-Empirical/Exp14-clefts/ROC-subject.jpeg}
    \includegraphics[width=\textwidth]{chapters/part2-Empirical/Exp14-clefts/zROC-subject.jpeg}
    \caption{ROC curves (top) and zROC curves (bottom) of the object conditions compared to their respective subject conditions represented by the dotted grey baseline (\citealt{Dillon.2019}'s method) in Experiment~14.}
    \label{fig:exp14-ROC-subj}
\end{figure}

\subsubsection{Habituation}\largerpage

The habituation effects in the course of the experiment are given in \figref{fig:exp14-habituation} on page \pageref{fig:exp14-habituation}. We see some habituation effect in the subject variants of the two controls, but not in their object variants. Subextraction, on the other hand, received similar ratings throughout the whole experiment.

\begin{figure}
    \centering
    \includegraphics[width=\textwidth]{chapters/part2-Empirical/Exp14-clefts/habituation.jpeg}
    \caption{Changes in the mean acceptability ratings ($z$-scored by participant) by condition in the course of Experiment~14}
    \label{fig:exp14-habituation}
\end{figure}

\subsubsection{Comparing subextraction from the subject with subextraction from the object}

We fitted a first model to compare the extractions out of the subject and out of the object on their own (mean centered with subject coded negative and object coded positive). We included trial number as a covariate, and random slopes for the fixed effect and the covariates grouped by participants and items. The results of the model are reported in Table \ref{tab:exp14-m1}. 
There is a significant effect of the syntactic function, such that the object condition received significantly higher ratings than the subject condition. As we saw in \figref{fig:exp14-habituation}, there is no significant effect of habituation.

% latex table generated in R 3.6.3 by xtable 1.8-4 package
% Fri Apr 24 21:31:19 2020
\begin{table}
\begin{tabular}{l S[table-format=1.3] S[table-format=1.4] c S[table-format=<1.3] S[table-format=1.2]}
  \lsptoprule
 & {Estimate} & {SE} & {$z$} & {$\text{Pr}(>|z|)$} & {OR} \\ 
  \midrule
  syntactic function & 0.459 & 0.144 & 3 & <.005 & 1.58 \\ 
  trial              & 0.016 & 0.010 & 2 & 0.1025 & 1.02 \\ 
   \lspbottomrule
\end{tabular}
\caption{Results of the Cumulative Link Mixed Model (model n$^{\circ}$2)}
\label{tab:exp12-m2}
\end{table}


In a second model, we compared the subextractions with the non-subextractions. The model crossed syntactic function and extraction type (mean centered with extraction coded positive, non-subextraction coded negative). We included trial number as a covariate, and random slopes for all fixed effects and covariates grouped by participants and items. The results of the model are reported in Table \ref{tab:exp14-m2}.
There is a significant main effect of syntactic function (in favor of the object) and a significant main effect of extraction type (non-extractions are rated higher). There is also a significant interaction effect. 
\figref{fig:exp14-interaction} illustrates this effect: we see a decrease of acceptability for the extraction out of subjects. However, if we compare the AUCs (green and red curves on \figref{fig:exp14-ROC-subj}), the difference is not significant. The two methods thus lead to different results as far as the interaction effect is concerned. 

% latex table generated in R 3.6.3 by xtable 1.8-4 package
% Mon Apr 13 17:44:16 2020
\begin{table}
\begin{tabular}{l S[table-format=-1.3] S[table-format=1.3] S[table-format=1] S[table-format=1.4] S[table-format=1.2]}
  \lsptoprule
 & {Estimate} & {SE} & {$z$} & {$\text{Pr}(>|z|)$} & {Odd.ratio} \\ 
  \midrule
  syntactic function & -0.070 & 0.324 & -0 & 0.8286 & 1.07 \\ 
  trial              & 0.024 & 0.018 & 1 & 0.1717 & 1.02 \\ 
   \lspbottomrule
\end{tabular}
\caption{Results of the Cumulative Link Mixed Model (model n$^{\circ}$1)}
\label{tab:exp06-m1}
\end{table}


\begin{figure}
    \centering
    \includegraphics[width=\textwidth]{chapters/part2-Empirical/Exp14-clefts/interaction.jpeg}
    \caption{Interaction between syntactic function and extraction type in Experiment~14}
    \label{fig:exp14-interaction}
\end{figure}

\subsubsection{Comparing subextraction from the subject with ungrammatical controls}

We fitted a third model to compare extraction out of the subject and the subject ungrammatical controls on their own (mean centered with subextraction coded positive and ungrammatical coded negative). We included trial number as a covariate, and random slopes for the fixed effects and the covariates grouped by participants and items. The results of the model are reported in \tabref{tab:exp14-m3}. There is a significant effect of extraction type, such that the ratings for extraction out of the subject are significantly higher than for its ungrammatical control.

% latex table generated in R 3.6.3 by xtable 1.8-4 package
% Thu Apr 23 00:04:53 2020
\begin{table}
\begin{tabular}{l S[table-format=1.3] S[table-format=1.3] c S[table-format=<1.3] S[table-format=2.2]}
  \lsptoprule
 & {Estimate} & {SE} & {$z$} & {$\text{Pr}(>|z|)$} & {OR}\\ 
  \midrule
  extraction type & 2.342 & 0.395 & 6 & <.001 & 10.40 \\ 
  trial           & 0.039 & 0.012 & 3 & <.005 & 1.04 \\ 
   \lspbottomrule
\end{tabular}
\caption{Results of the Cumulative Link Mixed Model (model n$^{\circ}$3)}
\label{tab:exp10-m3}
\end{table}


In a fourth model, we compared the subextraction with the ungrammatical controls. We fitted a model crossing syntactic function (mean centered with object coded positive, subject coded negative) and extraction type (grammaticality). We included trial number as a covariate, and random slopes for all fixed effects and covariates grouped by participants and items. The results of the model are reported in Table \ref{tab:exp14-m4}. 
There is a significant main effect of extraction type (in favor of the extraction conditions) and a significant interaction, such that extractions out of the object are rated higher than all other conditions.

% latex table generated in R 3.6.3 by xtable 1.8-4 package
% Sat Apr 25 19:15:30 2020
\begin{table}
\begin{tabular}{l S[table-format=1.3] S[table-format=1.3] c S[table-format=<1.4] S[table-format=1.2]}
  \lsptoprule
 & {Estimate} & {SE} & {$z$} & {$\text{Pr}(>|z|)$} & {OR} \\ 
  \midrule
  syntactic function & 0.134 & 0.091 & 1 & 0.1405 & 1.14 \\ 
  extraction type & 0.631 & 0.133 & 5 & <.001 & 1.88 \\ 
  trial & 0.025 & 0.005 & 5 & <.001 & 1.03 \\ 
  syntactic function:extraction type & 0.009 & 0.088 & 0 & 0.9142 & 1.01 \\ 
   \lspbottomrule
\end{tabular}
\caption{Results of the Cumulative Link Mixed Model (model n$^{\circ}$5)}
\label{tab:exp13-m5}
\end{table}


\subsubsection{Comparing extraction of the subject with extraction of the object}

A fifth model compared the extractions out of the subject and out of the object on their own (mean centered with subject coded negative and object coded positive). We included trial number as a covariate, and random slopes for all fixed effects grouped by participants and items. The results of the model are reported in Table \ref{tab:exp14-m5}. 
There is a significant effect of habituation, but no significant effect of syntactic function. Extractions of the subject (\ref{ex:exp14-subj-no}) did not get significantly higher ratings than (\ref{ex:exp14-obj-no}). 

% latex table generated in R 3.6.3 by xtable 1.8-4 package
% Sun Jul 19 14:56:20 2020
\begin{table}
\begin{tabular}{l S[table-format=-1.4] S[table-format=1.4] S[table-format=-1.4] S[table-format=<1.4] S[table-format=2.2]}
  \lsptoprule
 & {Estimate} & {SE} & {$z$} & {$\text{Pr}(>|z|)$} & {OR} \\ 
  \midrule
(Intercept) & -1.5653 & 0.4991 & -3.1362 & <.005 & 4.78 \\ 
  extraction type & -0.1756 & 0.0673 & -2.6073 & <.01 & 1.19 \\ 
  distance & 0.0849 & 0.067 & 1.2673 & 0.2051 & 1.09 \\ 
  length & 0.0273 & 0.0408 & 0.6683 & 0.504 & 1.03 \\ 
  extraction type:distance & -0.197 & 0.0678 & -2.907 & <.005 & 1.22 \\ 
   \lspbottomrule
\end{tabular}
\caption{Results of the Regression Mixed Model (model n$^{\circ}$6)}
\label{tab:exp03-m6}
\end{table}


\subsection{Discussion}
\label{ch:exp14-discussion}

In this experiment, extraction out of the subject received significantly lower ratings than extraction out of the object (model n$^{\circ}$1), but this did not lead to a significant interaction (comparison of AUCs). Extraction out of the subject was nevertheless significantly better than its ungrammatical control (model n$^{\circ}$3). 

The findings that extractions out of the subject have lower ratings than extractions out of the object, and also that there is a significant main effect in favor of objects contradict the expectations of processing accounts based on linear distance. On the other hand, the fact that extractions out of the subject were judged better than the ungrammatical controls contradicts the predictions of the syntactic account. 

All other accounts predict a significant interaction effect. The results are contradictory: the interaction is significant with the less conservative method (model n$^{\circ}$2), but not with the more conservative one (comparison of AUCs). We cautiously conclude that the null-hypothesis (i.e.\ that there is no interaction effect) is not falsified, and therefore that the study is inconclusive in this respect. Accounts that predict an interaction effect are not falsified, but it is noticeable that the degradation in extraction out of the subject is less strong in these \emph{c'est}-clefts than in the interrogatives of the previous studies. Indeed, in the corpora, we found extractions out of the subject in \emph{c'est}-clefts (especially an undeniable focalization with example (\ref{ex:d1900-clefts-subj-1}) page \pageref{ex:d1900-clefts-subj-1}), while we did not find any in interrogatives.

We were not able to reproduce the findings of \citet{Destruel.2012} for extractions of the subject (model  n$^{\circ}$5). This is surprising, because the subject preference has been attested repeatedly for relative clauses. It may be an indication that our experiment is not very powerful (it may not have enough participants), or some unidentified factor affected our stimuli. 
