\section{\emph{De quel} in Frantext}
\label{ch:deq-corpus}

\subsection{Motivation}

This corpus study is the interrogative counterpart to the study on \emph{duquel} (Section \ref{ch:duquel}). \emph{Duquel} and \emph{de quel} + N are synonymous, but whereas \emph{duquel} is almost exclusively used in relative clauses, \emph{de quel} can only be used in interrogatives. 
If the contrast between relative clauses and interrogatives found in the corpus studies on \emph{de qui} is robust, we should find no extraction out of the subject in \emph{de quel} interrogatives. 

\subsection{Procedure}

Because this corpus study was intended as a counterpart to the previous one, we used the same corpus: Frantext 2000--2013. We looked for the combination of the two lemmas `de'' and ``quel'', that can be realized in four forms: \emph{de quel} (masculine singular), \emph{de quelle} (feminine singular), \emph{de quels} (masculine plural or masculine+feminine plural), \emph{de quelles} (feminine plural).

This query yielded 445 occurrences of \emph{de quel}, that we annotated as in the previous corpus studies. The results confirm that \emph{duquel} is not used in relative clauses, and most of the time occurs in interrogatives. There were 426 direct or indirect questions and 14 exclamatives. The five remaining occurrences were noise.

There was no extraction out of the subject in exclamatives. Nine of the exclamatives were fragments, without a verb. In two of them, \emph{de quel} + N is in situ, like (\ref{ex:deq-excl-insitu}). The three remaining ones were: one extraction of the complement of a verb (\ref{ex:deq-excl-verb}), one extraction out of an adjective and one adjunct.

\eal 
\ex (Dans la main du diable, Anne-Marie Garat, 2006)\\
\gll Vous m' avez trahi, abusé [de	quelle façon]~!\\
you me\textsc{.acc} have betryed abused of which manner\\
\glt `You betrayed me, abused me in such a way!'
\label{ex:deq-excl-insitu}
\ex (Journal sous l'Occupation en Périgord~: 1942-1945, Jeanne Pouquet, 2006)\\
\gll [De quelle tendresse infinie]$_i$ [mon cher mari m' a entourée~\trace{}$_i$]~!\\
of which gentleness endless my dear husband me\textsc{.acc} has surrounded\\
\glt `With which endless gentleness did my dear husband surround me!'
\label{ex:deq-excl-verb}
\zl 

The following section presents the results for the interrogatives.

\subsection{Results and analysis}

Of the 426 interrogatives with \emph{de quel} + N, 259 were direct questions and 167 indirect questions. After excluding the 109 verbless interrogatives and 13 \emph{de qui} in situ, there remain 304 interrogatives with a gap: 159 direct questions and 145 indirect questions.

The distribution of the \emph{de quel} + N interrogatives with one gap is given in Table~\ref{tab:deq-wh} and on \figref{fig:deq-wh}.

\begin{table}
    \begin{tabular}{lrr}
         \lsptoprule
         & Frequency & \% \\
         \midrule
         Verb & 197 & 64.80 \\
         Noun & & \\
             \quad Object & 6 & 1.97 \\
             \quad Predicate & 15 & 4.93 \\
             \quad Cplt of Preposition & 13 & 4.28 \\
         Adjective & 13 & 4.28 \\
         Preposition & 4 & 1.32 \\
         Adjunct & 56 & 18.42 \\
         \lspbottomrule
    \end{tabular}
    \caption{Distribution of \emph{de quel} + N interrogatives in Frantext 2000-2013}
    \label{tab:deq-wh}
\end{table}

\begin{figure}
    \centering
    \includegraphics[width=\textwidth]{chapters/part2-Empirical/dequel-Frantext/distribution-qu.jpeg}
    \caption[Distribution of \emph{de quel} + N interrogatives in Frantext 2000--2013]{Distribution of \emph{de quel} + N interrogatives in Frantext 2000--2013. See page~\pageref{ch:conf-intervals-binomial} for the confidence intervals (here seven comparisons).}
    \label{fig:deq-wh}
\end{figure}

\emph{De quel} + N can be the complement of a verb (\ref{ex:deq-verb-qu}), of a noun (\ref{ex:deq-noun-qu}), of an adjective (\ref{ex:deq-adjective-qu}) or of an preposition (\ref{ex:deq-adjective-prep}), or be an adjunct (\ref{ex:deq-adjective-adjunct}). All categories had a frequency significantly greater than zero.

\ea An example of \emph{de quel} + N as verb complement\\
(Une vie brève, Michèle Audin, 2012)\\
\gll Je ne sais pas [de quelles « revues d' analyses~»]$_i$ [disposait~\trace{}$_i$ la bibliothèque de la Faculté des sciences]~[\dots].\\
I \textsc{neg} know not of which {} journal of analyses possessed the library of the Faculty of.the Sciences\\
\glt `I don't know which analysis journal the library of the Faculty of Sciences possessed.'
\label{ex:deq-verb-qu}
\z 
\pagebreak
\begin{exe}
    \ex Examples of \emph{de quel} + N as noun complement\label{ex:deq-noun-qu}
\begin{xlist}
\ex Object noun:\\
(L'enfant des ténèbres, Anne-Marie Garat, 2008)\\
\gll [De quelle infortune] portait - il [l' infirmité~\trace{}$_i$]~[\dots]~?\\
of which misfortune carried {} he the infirmity\\
\glt `Of which misfortune did he carry the infirmity?'
\ex Predicate noun:\\
(Formation, Pierre Guyotat, 2007)\\
\gll [[D]e quel royaume minuscule -- le plus petit d' Europe ---]$_i$ est - il [le petit roi~\trace{}$_i$]~?\\
of which kingdom tiny {} the most small of Europe {} is {} he the small king\\
\glt `Of which tiny kingdom -- the smallest in Europe -- is he the small king?'
\ex Noun complement of a preposition:\\
(La Vie sauve, Lydie Violet, 2005)\\
\gll [Au terme de quelle histoire]$_i$ [a -t- il choisi cette part - là de l' humanité~\trace{}$_i$]~?\\
at.the end of which events has 0\footnotemark{} he chosen this side {} there of the humanity\\
\glt `At the end of which events did he choose this side of humanity?'
\end{xlist}
\end{exe}
\footnotetext{In subject-verb inversion, a so-called euphonic \emph{t} appears between the verb and the subject to avoid adjacent vowels. This euphonic \emph{t} has no semantic content and is glossed as \textit{0}, following the glossing guideline in \citet[79]{Blaszczak.2007}.}

\ea An example of \emph{de quel} + N as adjective complement\\
(Comment j'ai vidé la maison de mes parents, Lydia Flem, 2004)\\
\gll [De quel impensable maternel et paternel]$_i$ étais - je [issue~\trace{}$_i$]~?\\
of which unthinkable maternal and paternal was {} I originating\\
\glt `From which maternal and paternal unthinkable was I coming?'
\label{ex:deq-adjective-qu}
\z 

\ea An example of \emph{de quel} + N as the complement of a preposition\\
(Autoportrait, Édouard Levé, 2005)\\
\gll Je ne sais pas [vis-à-vis de quels artistes] [j' ai des dettes~\trace{}$_i$].\\
I \textsc{neg} know not toward of which artists I have some debt\\
\glt `I don't know toward which artists I have debts.'
\label{ex:deq-adjective-prep}
\z 

\ea An example of \emph{de quel} + N as an adjunct\nopagebreak\\
(Et le jour pour eux sera comme la nuit, Ariane, Bois, 2009)\nopagebreak\\
\gll [De quel droit] lui assure -t- on que son chagrin cessera un jour~?\\
of which right her\textsc{.dat} assures 0 one that her pain will.cease one day\\
\glt `By what right do they assure her that her pain will be over one day?'
\label{ex:deq-adjective-adjunct}
\z 

We can see that the most common usage of \textit{de quel} is to extract the complement of the verb, and that even pied-piping cases are relatively rare compared to \emph{duquel} relative clauses. There are very few extractions out of an NP (6.91\%, half of them being from Anne-Marie Garat). 

\subsection{Conclusion}

If we compare \emph{duquel} and \emph{de quel} + N, we can see that relative clauses and interrogatives show very different patterns. This corroborates our findings for \emph{de qui} relative clauses and interrogatives.

The dominant usage in \emph{de qui} and \emph{de quel} + N interrogatives is to extract the complement of the verb, and extractions out of NPs are rare. There is also no attested extraction out of the subject.
In relative clauses, on the other hand, we find extractions out of NPs, and among them extractions out of subject NPs.

This cross-construction difference is only expected if we take into account the discourse status of these two constructions as in the FBC constraint. 
