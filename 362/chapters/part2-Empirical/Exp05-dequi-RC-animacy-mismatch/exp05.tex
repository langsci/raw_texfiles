\section[head=Experiment 5]{Experiment 5: Acceptability judgment study on \emph{de qui} relative clauses with an animate antecedent and animacy mismatch between subject and object}

The corpus results presented in Section \ref{ch:de-qui-corpus} suggest that relative clauses with \emph{dont} and \emph{de qui} are similar with respect to extraction out of the subject. However, there is a long tradition of assuming that extractions out of the subject with \emph{dont} are exceptional because \emph{dont} is not a pronoun, or because it has genitive case. This was first proposed by \citet{Tellier.1990,Tellier.1991}, and then echoed by \citet{Stepanov.2007} and \citet{Heck.2009}, even though it had already been criticized by \citet{Godard.1988} (see the whole discussion in Section~\ref{ch:intro-disscussion-French}). On these grounds we considered useful to test \emph{de qui} relative clauses with a methodology similar to the one adopted in the previous experiments for \emph{dont} relative clauses. 

\subsection{Design and materials}
The experiment was an acceptability judgment task with a 2*2 design. The design was very similar to the one in Experiment~4, but there were no ungrammatical controls. We compared extractions out of subjects (\ref{ex:exp05-subj-pp}) with extractions out of objects (\ref{ex:exp05-obj-pp}), and paired each with a non-extraction control including coordination (\ref{ex:exp05-subj-no}) and (\ref{ex:exp05-obj-no}).

\eal 
\ex[]{{Condition subject + PP-extracted:}\\
\gll  J' ai exclu un garçon [de qui]$_i$ [l’ arrogance~\trace{}$_i$] rebute mes collègues.\\
I have excluded a boy of who the arrogance repels my colleagues\\
\glt `I excluded a boy whose arrogance repels my colleagues.'}
\label{ex:exp05-subj-pp}
\ex[]{{Condition object + PP-extracted:}\\
\gll  J' ai exclu un garçon de qui mes collègues détestent l’ arrogance.\\
I have excluded a boy of who my colleagues hate the arrogance\\
\glt `I excluded a boy of who my colleagues hate the arrogance.'}
\label{ex:exp05-obj-pp}
\zl 

\eal \label{ex:exp05-no}
\ex[]{{Condition subject + noextr:}\nopagebreak\\
\gll J' ai exclu un garçon et son arrogance rebute mes collègues.\\
I have excluded a boy and his arrogance repels my colleagues\\
\glt `I excluded a boy and his arrogance repels my colleagues.'}
\label{ex:exp05-subj-no}
\ex[]{{Condition object + noextr:}\\
\gll J' ai exclu un garçon et mes collègues détestent son arrogance.\\
I have excluded a boy and my colleagues hate his arrogance\\
\glt `I excluded a boy and my colleagues hate his arrogance.'}
\label{ex:exp05-obj-no}
\zl 

It was not possible to use the same stimuli as in Experiment~4 because they contained inanimate antecedents and \emph{de qui} requires animate antecedents. But as in Experiment~4, the relation between \emph{de qui} and the gap expressed a quality (e.g.\ \emph{arrogance} `arrogance', \emph{violence} `violence') and we used psych verbs that come in pairs (e.g.\ \emph{rebuter} `repel' and \emph{détester} `hate'). 

We tested 20 items, each appearing in the four conditions already described. In addition, the experiment included 42 distractors. Each item and distractor was followed by a comprehension question. The item presented here as an example was paired with the comprehension question \emph{Qui est arrogant~?} (`Who is arrogant?'). 

\subsection{Predictions} 

The predictions for this experiment (as well as for all other experiments on relative clauses that follow) are similar to the predictions already discussed in Experiment~4. They are summarized in Table~\ref{tab:exp05-predictions}. 

% http://www.tablesgenerator.com/#

\begin{sidewaystable}
\oneline{%
\begin{tabular}{llllll}
\lsptoprule
 &
  \multicolumn{5}{c}{{Predictions}} \\ \cmidrule(lr){2-6} 
 &
  \multicolumn{3}{c}{{“subject island” accounts}} &
  \multicolumn{2}{c}{{no-island accounts}} \\ \cmidrule(lr){2-6} 
 &
  {\begin{tabular}[c]{@{}l@{}}“traditional” \\ syntactic account\end{tabular}} &
  {\begin{tabular}[c]{@{}l@{}}processing account \\ with surprisal \\ due to subject \\ complexity\end{tabular}} &
  {\begin{tabular}[c]{@{}l@{}}BCI account\\ (Goldberg 2006)\end{tabular}} &
  {\begin{tabular}[c]{@{}l@{}}account based\\ on linear distance\\ (DG, DLT)\end{tabular}} &
  {\begin{tabular}[c]{@{}l@{}}FBC constraint \\ account\end{tabular}} \\ \midrule
\multicolumn{1}{l}{{\begin{tabular}[c]{@{}l@{}}extractions \\ out of the subject\\ vs. extractions \\ out of the object\end{tabular}}} &
  (a) \textless (b) &
  (a) \textless (b) &
  (a) \textless (b) &
  (a) \textgreater (b) &
  (a) \textless (b) \\ \midrule
\multicolumn{1}{l}{{\begin{tabular}[c]{@{}l@{}}extractions vs. \\ non-extractions\end{tabular}}} &
  \begin{tabular}[c]{@{}l@{}}main effect of extraction\\ + interaction effect \\ such that (a) \textless (b,c,d)\end{tabular} &
  \begin{tabular}[c]{@{}l@{}}main effect of extraction\\ + interaction effect \\ such that (a) \textless (b,c,d)\end{tabular} &
  \begin{tabular}[c]{@{}l@{}}main effect of extraction\\ + interaction effect \\ such that (a) \textless (b,c,d)\end{tabular} &
  \begin{tabular}[c]{@{}l@{}}interaction effect,\\ such that (b) \textless (a,c,d)\end{tabular} &
  \begin{tabular}[c]{@{}l@{}}main effect of extraction\\ + interaction effect \\ such that (a) \textless (b,c,d)\end{tabular} \\ \midrule
\multicolumn{1}{l}{{\begin{tabular}[c]{@{}l@{}}extractions \\ out of the subject\\ vs. ungrammatical\\ controls\end{tabular}}} &
  (a) $\simeq$ (e) &
  (a) \textgreater (e) &
  (a) \textgreater (e) &
  (a) \textgreater (e) &
  (a) \textgreater (e) \\ \midrule
\multicolumn{1}{l}{{\begin{tabular}[c]{@{}l@{}}extractions vs.\\ vs. ungrammatical\\ controls\end{tabular}}} &
  \begin{tabular}[c]{@{}l@{}}interaction effect\\ such that (b) \textgreater (a,e,f)\end{tabular} &
  \begin{tabular}[c]{@{}l@{}}main effect\\ of grammaticality\end{tabular} &
  \begin{tabular}[c]{@{}l@{}}main effect\\ of grammaticality\end{tabular} &
  \begin{tabular}[c]{@{}l@{}}main effect\\ of grammaticality\end{tabular} &
  \begin{tabular}[c]{@{}l@{}}main effect\\ of grammaticality\end{tabular} \\ \lspbottomrule
\end{tabular}}
\caption{Predictions of the different accounts for Experiments 10, 12 and 13 (interrogatives with extraction). \emph{Notes}: (a) Condition subject + PP-extracted (b) Condition object + PP-extracted (c) Condition subject + no extraction (d) Condition object + no extraction (e) Condition subject + ungrammatical (f) Condition object + ungrammatical.}
\label{tab:exp10-predictions}
\end{sidewaystable}


\subsection{Procedure} 

We conducted the Experiment~on the Ibex platform \citep{Ibex}. The procedure for acceptability judgment tasks is described in Section \ref{ch:methodo-AJ}. Participants rated the sentences on a Likert scale from 1 to 10, 1 being labeled as ``bad'' and 10 being labeled as ``good''. After each sentence, participants had to answer a comprehension question, for example \emph{Qui est arrogant~?} (`Who is arrogant?') which appeared on the screen together with two possible answers. Participants had to click on the appropriate answer in order to proceed to the following sentence.

The experiment took approximately 20 minutes to complete. We recruited the participants on the R.I.S.C. website (\url{http://experiences.risc.cnrs.fr/}) and on social media (e.g.\ Facebook).

\subsection{Participants}

The study was run between April and October 2017.  
75 participants took part in the experiment. Data from 60 participants were included in the analysis based on our inclusion criteria.\footnote{In addition to the usual criteria, we excluded three participants who did not use the whole Likert scale. To calculate accuracy, we excluded not only the answers to comprehension questions of the practice items, but also one series of distractors that had an overall accuracy rate of 71\% only.}
The 60 participants were aged 19 to 79 years. 41 of them self-identified as women, and 19 as men. Four participants (6.67\%) indicated having an educational background related to language.

\subsection{Results and analysis}



\figref{fig:exp05-boxplot} summarizes the results of the acceptability judgment task.\footnote{We involuntarily introduced a typo in one condition of one experimental item, which was therefore excluded from the results and treated as a distractor. The results presented here are therefore based on 19 experimental items.}
In the subextraction condition, extraction out of the subject (\ref{ex:exp05-subj-pp}) received a mean acceptability rating of 5.02, slightly lower than extraction out of the object (\ref{ex:exp05-obj-pp}) with a mean rating of 5.36. The control conditions without extraction were rated better overall: 7.80 in the subject condition (\ref{ex:exp05-subj-no}) and 7.62 in the object condition (\ref{ex:exp05-obj-no}). 

\begin{figure}
    \centering
    \includegraphics[width=\textwidth]{chapters/part2-Empirical/Exp05-dequi-RC-animacy-mismatch/boxplots.jpeg}
    \caption{Acceptability judgments by condition in Experiment~5. The grey box plots indicate the median and quartiles of the results. Black points are outliers. Mean and confidence intervals are indicated in white.}
    \label{fig:exp05-boxplot}
\end{figure}

\figref{fig:exp05-boxplot} suggests potential ceiling effects in the non-extraction conditions. Unlike the extractions out of NPs with \emph{dont}, extractions out of NPs with \emph{de qui} do not display ceiling effects. This is corroborated by \figref{fig:exp05-repartition}: there is a clear ceiling effect for non-extraction conditions, but not for extraction conditions. However, the distribution of the subject + extraction condition is not fully normal, it is flat. 

\begin{figure}
    \centering
    \includegraphics[width=\textwidth]{chapters/part2-Empirical/Exp05-dequi-RC-animacy-mismatch/repartition.jpeg}
    \caption{Density of the ratings across conditions for Experiment~5}
    \label{fig:exp05-repartition}
\end{figure}

Another representation of the results is given by the ROC and zROC curves of the results in \figref{fig:exp05-ROC}. There is strong discrimination between the non-extraction conditions (grey baseline) and the extraction condition (blue lines). The zROC curve for the object condition is a straight line parallel to the baseline, but the zROC for the subject condition is not. This corroborates the impression given by \figref{fig:exp05-repartition}: the distribution of the results in the subject condition is not fully normal.

\begin{figure}
    \centering
    \includegraphics[width=\textwidth]{chapters/part2-Empirical/Exp05-dequi-RC-animacy-mismatch/ROC.jpeg}
    \includegraphics[width=\textwidth]{chapters/part2-Empirical/Exp05-dequi-RC-animacy-mismatch/zROC.jpeg}
    \caption{ROC curves (top) and zROC curves (bottom) of the non-extraction conditions compared to their respective subextraction conditions, represented by the dotted grey baseline (\citealt{Dillon.2019}'s method) in Experiment~5.}
    \label{fig:exp05-ROC}
\end{figure}

Indeed, if we look at the results for the early trials on \figref{fig:exp05-repartition-begin-end} (page \pageref{fig:exp05-repartition-begin-end}), we see that the distribution looks more normal. I think that the non-normal distribution of the subextraction ratings is due to the high degree of habituation to subextractions (see below). 

\begin{figure}
    \centering
    \includegraphics[width=\textwidth]{chapters/part2-Empirical/Exp05-dequi-RC-animacy-mismatch/repartition-begin.jpeg}
    
    \includegraphics[width=\textwidth]{chapters/part2-Empirical/Exp05-dequi-RC-animacy-mismatch/repartition-end.jpeg}
    \caption{Density of the ratings across conditions for the beginning (first quartile, top) and the end (fourth quartile, bottom) of Experiment~5}
    \label{fig:exp05-repartition-begin-end}
\end{figure}

The ROC and zROC curves in \figref{fig:exp05-ROC-subj} depict the discrimination between the subject and object conditions. The ROC curves show that the participants barely discriminate between the subject and object conditions. The zROC curves are relatively straight.

\begin{figure}
    \centering
    \includegraphics[width=\textwidth]{chapters/part2-Empirical/Exp05-dequi-RC-animacy-mismatch/ROC-subject.jpeg}
    \includegraphics[width=\textwidth]{chapters/part2-Empirical/Exp05-dequi-RC-animacy-mismatch/zROC-subject.jpeg}
    \caption{ROC curves (top) and zROC curves (bottom) of the object conditions compared to their respective subject conditions, represented by the dotted grey baseline (\citealt{Dillon.2019}'s method) in Experiment~5.}
    \label{fig:exp05-ROC-subj}
\end{figure}

\subsubsection{Habituation} 

\figref{fig:exp05-habituation} shows the habituation effects in the course of the experiment. The non-extraction condition does not display any habituation effects, there is even a slight decrease of acceptability for the subject + non-extraction condition. However, we can see strong habituation effects in the extraction conditions. 

\begin{figure}
    \centering
    \includegraphics[width=\textwidth]{chapters/part2-Empirical/Exp05-dequi-RC-animacy-mismatch/habituation.jpeg}
    \caption{Changes in the average acceptability ratings ($z$-scored by participant) for each condition of Experiment~5 in the course of the experiment}
    \label{fig:exp05-habituation}
\end{figure}

\subsubsection{Comparing subextraction from the subject with subextraction from the object}

We fitted a first model to compare the extractions out of the subject and out of the object on their own (mean centered with subject coded negative and object coded positive). We included trial number as a covariate, and random slopes for the fixed effect grouped by participants and items. The results of the model are reported in \tabref{tab:exp05-m1}. There is a significant effect of syntactic function: the object condition received significantly higher ratings than the subject condition. There is also a significant effect of trial (habituation).

% latex table generated in R 3.6.3 by xtable 1.8-4 package
% Mon Apr 13 17:44:16 2020
\begin{table}
\begin{tabular}{l S[table-format=-1.3] S[table-format=1.3] S[table-format=1] S[table-format=1.4] S[table-format=1.2]}
  \lsptoprule
 & {Estimate} & {SE} & {$z$} & {$\text{Pr}(>|z|)$} & {Odd.ratio} \\ 
  \midrule
  syntactic function & -0.070 & 0.324 & -0 & 0.8286 & 1.07 \\ 
  trial              & 0.024 & 0.018 & 1 & 0.1717 & 1.02 \\ 
   \lspbottomrule
\end{tabular}
\caption{Results of the Cumulative Link Mixed Model (model n$^{\circ}$1)}
\label{tab:exp06-m1}
\end{table}


In a second model, we compared the subextractions with the non-extractions. We fitted a model crossing syntactic function and extraction type (mean centered with extraction coded positive, non-extraction coded negative). We included trial number as a covariate, and random slopes for all fixed effects and covariates grouped by participants and items. The results of the model are reported in \tabref{tab:exp05-m2}. There is a significant main effect of extraction type in favor of the non-extraction controls. There is no significant main effect of syntactic function, no significant main effect of trial (habituation) and no significant interaction effect. \figref{fig:exp05-interaction1} shows the interaction: we see a weak tendency toward an interaction effect. Furthermore, if we compare the AUC (see \figref{fig:exp05-ROC-subj} on page~\pageref{fig:exp05-ROC-subj}), the difference is not significant, either.

% latex table generated in R 3.6.3 by xtable 1.8-4 package
% Fri Apr 10 17:22:54 2020
\begin{table}
\begin{tabular}{l S[table-format=-1.3] S[table-format=1.3] S[table-format=3.2] S[table-format=-1] S[table-format=<1.4] S[table-format=1.2]}
  \lsptoprule
 & {Estimate} & {SE} & {df} & {$t$} & {$\text{Pr}(>|t|)$} & {OR} \\ 
  \midrule
(Intercept) & 0.856 & 0.577 & 152.52 & 1 & 0.1403 & 2.35 \\ 
  extraction type & -0.131 & 0.061 & 23.29 & -2 & <.05 & 1.14 \\ 
  distance & 0.169 & 0.105 & 25.66 & 2 & 0.1213 & 1.18 \\ 
  length & 0.557 & 0.014 & 521.85 & 39 & <.001 & 1.74 \\ 
  extraction type:distance & -0.042 & 0.073 & 29.31 & -1 & 0.5741 & 1.04 \\ 
   \lspbottomrule
\end{tabular}
\caption{Results of the Linear Mixed Model (model n$^{\circ}$1)}
\label{tab:exp03-m1}
\end{table}


\begin{figure}
    \centering
    \includegraphics[width=\textwidth]{chapters/part2-Empirical/Exp05-dequi-RC-animacy-mismatch/interaction.jpeg}
    \caption{Interaction between syntactic function and extraction type in Experiment~5}
    \label{fig:exp05-interaction1}
\end{figure}

\subsection{Discussion}

The results of Experiment~5 are puzzling. On the one hand they seem more compatible with accounts that expect a superadditivity effect, because there is a significant difference between extractions out of the subject vs.\ the object: extractions out of the object were rated better than extractions out of the subject (model n$^{\circ}$1). On the other hand, there is no significant interaction effect. This latter fact can be due to a weakness of the experiment: perhaps the number of participants was not sufficient to allow a small effect to become apparent. In that case, the effect size of the interaction must be rather small. A small-sized effect is not expected by a traditional syntactic account, but can be compatible with processing accounts based on subject complexity. 

The results are in contradiction with predictions of a processing account based on memory costs, because the extraction out of the subject should be better than the extraction out of the object. The results are also unexpected under the FBC constraint. 

The next experiment tested the same stimuli with \emph{dont} instead of \emph{de qui}, in order to see if we reproduce the same results.
