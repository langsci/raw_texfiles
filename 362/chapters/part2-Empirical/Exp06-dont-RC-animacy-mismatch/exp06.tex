\section[head=Experiment 6]{Experiment 6: Acceptability judgment study on \emph{dont} relative clauses with an animate antecedent and an animacy mismatch between subject and object}

In this study, we tested the same stimuli as in Experiment~5, but replaced the relative word \emph{de qui} by \emph{dont}. We did this in reaction to the results of that study, where we saw extractions out of the subject having lower ratings than extractions out of the object, contrary to our previous experiments on \emph{dont} (Experiments~1 to 4). In the previous experiments, we always tested \emph{dont} relative clauses with an inanimate antecedent. This study aims to be parallel to Experiment~5.

\subsection{Design and materials}

This experiment reproduces the design of the previous one: it is an acceptability judgment task with a 2*2 design, as shown below:
\pagebreak

\eal 
\ex[]{{Condition subject + PP-extracted:}\\
\gll  J' ai exclu un garçon dont$_i$ [l’ arrogance~\trace{}$_i$] rebute mes collègues.\\
I have excluded a boy of.which the arrogance repels my colleagues\\
\glt `I excluded a boy whose arrogance repels my colleagues.'}
\label{ex:exp06-subj-pp}
\ex[]{{Condition object + PP-extracted:}\\
\gll  J' ai exclu un garçon dont mes collègues détestent l’ arrogance.\\
I have excluded a boy of.which my colleagues hate the arrogance\\
\glt `I excluded a boy of which my colleagues hate the arrogance.'}
\label{ex:exp06-obj-pp}
\zl 

\eal \label{ex:exp06-no}
\ex[]{{Condition subject + noextr:}\nopagebreak\\
\gll J' ai exclu un garçon et son arrogance rebute mes collègues.\\
I have excluded a boy and his arrogance repels my colleagues\\
\glt `I excluded a boy and his arrogance repels my colleagues.'}
\label{ex:exp06-subj-no}
\ex[]{{Condition object + noextr:}\\
\gll J' ai exclu un garçon et mes collègues détestent son arrogance.\\
I have excluded a boy and my colleagues hate his arrogance\\
\glt `I excluded a boy and my colleagues hate his arrogance.'}
\label{ex:exp06-obj-no}
\zl 

We tested the same 20 items and 42 distractors as in Experiment~5. Each item and distractor was also followed by the same comprehension question as in Experiment~5 (e.g.\ for the example item: \emph{Qui est arrogant~?}, `Who is arrogant?').

\subsection{Predictions} 

The predictions for this experiment are similar to the ones summarized in \tabref{tab:exp05-predictions} on page \pageref{tab:exp05-predictions}. 

However, notice that under a syntactic approach like \citegen{Tellier.1991}  (the hypothesis that extractions out of the subject with \emph{dont} are an exception because \emph{dont} is special compared to \emph{de qui}) extraction out of the subject should not be worse than extraction out of the object, and there should be no interaction between extraction type and syntactic function.

On the other hand, if this experiment reproduces the disadvantage of extractions out of the subject, that could indicate that the results of Experiment~5 are due to the animate antecedent.

\subsection{Procedure}

We conducted the experiment on the Ibex platform \citep{Ibex}. The procedure was exactly the same as in Experiment~5. Participants rated the sentences on a Likert scale from 1 to 10, 1 being labeled as ``bad'' and 10 being labeled as ``good''. After each sentence, participants had to answer a comprehension question.
The experiment took approximately 20 minutes to complete.

\subsection{Participants}

The study was run between August and September 2017. We recruited the participants on the R.I.S.C. website (\url{http://experiences.risc.cnrs.fr/}) and on social media (e.g.\ Facebook).

28 participants took part in the experiment. We present the data of the 25 participants who satisfied all inclusion criteria.{\interfootnotelinepenalty=10000\footnote{To calculate accuracy, we excluded not only the answers to comprehension questions of the practice items, but also one series of distractors that had an overall accuracy rate of 75\% only (the same distractors that had been mentioned in the previous experiment).}}
The 25 participants were aged 18 to 71 years. 15 of them self-identified as women, and 10 as men. Four participants (16\%) indicated having an educational background related to language.

\subsection{Results and analysis}\largerpage[2.25]

\figref{fig:exp06-boxplot} shows the results of the acceptability judgment task.{\interfootnotelinepenalty=10000\footnote{We had the same typo as in Experiment~5, so we again had to exclude the whole item from the results and treat it as a distractor. The results provided here are therefore based on 19 experimental items.}}
Acceptability ratings were high in all experimental conditions, with a mean rating of 9.10 for extraction out of the subject (\ref{ex:exp06-subj-pp}), 8.73 for extractions out of the object (\ref{ex:exp06-obj-pp}), 8.60 for the subject control (\ref{ex:exp06-subj-no}) and 8.67 for the object control (\ref{ex:exp06-obj-no}). 

\begin{figure}
    \centering
    \includegraphics[width=\textwidth]{chapters/part2-Empirical/Exp06-dont-RC-animacy-mismatch/boxplots.jpeg}
    \caption{Acceptability judgments by condition in Experiment~6. The grey box plots indicate the median and quartiles of the results. Black points are outliers. Mean and confidence intervals are indicated in white.}
    \label{fig:exp06-boxplot}
\end{figure}

However, we clearly have ceiling effects in all conditions, and this is corroborated by \figref{fig:exp06-repartition}. The zROC curves in \figref{fig:exp06-ROC} on page \pageref{fig:exp06-ROC} also show that the distribution is not normal, as indicated by the fact that the lines are far from being straight.

\begin{figure}
    \centering
    \includegraphics[width=\textwidth]{chapters/part2-Empirical/Exp06-dont-RC-animacy-mismatch/repartition.jpeg}
    \caption{Density of the ratings across conditions for Experiment~6}
    \label{fig:exp06-repartition}
\end{figure}

\begin{figure}
    \centering
    \includegraphics[width=\textwidth]{chapters/part2-Empirical/Exp06-dont-RC-animacy-mismatch/ROC.jpeg}
    \includegraphics[width=\textwidth]{chapters/part2-Empirical/Exp06-dont-RC-animacy-mismatch/zROC.jpeg}
    \caption{ROC curves (top) and zROC curves (bottom) of the non-extraction conditions compared to their respective subextraction condition, represented by the dotted grey baseline (\citealt{Dillon.2019}'s method) in Experiment~6.}
    \label{fig:exp06-ROC}
\end{figure}

The by-participant ratings for the extraction conditions are displayed in \figref{fig:exp06-byparticipant} on page \pageref{fig:exp06-byparticipant}: it is obvious that a large proportion of the participants gave the maximum rating for the extraction conditions, regardless of the syntactic function. 

\begin{figure}
    \centering
    \includegraphics[width=\textwidth]{chapters/part2-Empirical/Exp06-dont-RC-animacy-mismatch/by-participant.jpeg}
    \caption{Acceptability judgments for the extraction conditions (out of subjects and out of objects) for each participant of Experiment~6. The blue box plots indicate the first and third quartiles of the results. Black points are mean ratings, and blue points are outliers.}
    \label{fig:exp06-byparticipant}
\end{figure}

\subsubsection{Habituation} 

The habituation effects in the course of the experiment are shown in \figref{fig:exp06-habituation} on page \pageref{fig:exp06-habituation}. As in the previous experiment, the extraction conditions show a larger habituation effect than the non-extraction conditions, but with little difference between extractions out of subjects vs.\ objects.

\begin{figure}
    \centering
    \includegraphics[width=\textwidth]{chapters/part2-Empirical/Exp06-dont-RC-animacy-mismatch/habituation.jpeg}
    \caption{Changes in the average acceptability ratings ($z$-scored by participant) for each condition of Experiment~6 in the course of the experiment}
    \label{fig:exp06-habituation}
\end{figure}

\subsubsection{Comparing subextraction from the subject with subextraction from the object}

We present here the results of two models. However, we have to bear in mind that the distribution is not normal, as shown above, thus the results are not very reliable.

We fitted a first model to compare the extractions out of the subject and out of the object on their own (mean centered with subject coded negative and object coded positive). We included trial number as a covariate, and random slopes for fixed effects and covariates grouped by participants and items. The results of the model are reported in Table \ref{tab:exp06-m1}. There is no significant difference between the two extractions.

% latex table generated in R 3.6.3 by xtable 1.8-4 package
% Mon Apr 13 17:44:16 2020
\begin{table}
\begin{tabular}{l S[table-format=-1.3] S[table-format=1.3] S[table-format=1] S[table-format=1.4] S[table-format=1.2]}
  \lsptoprule
 & {Estimate} & {SE} & {$z$} & {$\text{Pr}(>|z|)$} & {Odd.ratio} \\ 
  \midrule
  syntactic function & -0.070 & 0.324 & -0 & 0.8286 & 1.07 \\ 
  trial              & 0.024 & 0.018 & 1 & 0.1717 & 1.02 \\ 
   \lspbottomrule
\end{tabular}
\caption{Results of the Cumulative Link Mixed Model (model n$^{\circ}$1)}
\label{tab:exp06-m1}
\end{table}


In a second model, we compared subextraction with non-extraction. We fitted a model crossing syntactic function and extraction type (mean centered with extraction coded positive, non-extraction coded negative). We included trial number as a covariate, and random slopes for all fixed effects and covariates grouped by participants and items. The results of the model are reported in Table \ref{tab:exp06-m2}. There is a significant main effect of trial (habituation), but no other significant effect. \figref{fig:exp06-interaction} on page \pageref{fig:exp06-interaction} shows the interaction: we see a weak tendency toward an interaction effect. Furthermore, the difference in the AUC is not significant, either.

% latex table generated in R 3.6.3 by xtable 1.8-4 package
% Fri Apr 10 17:22:54 2020
\begin{table}
\begin{tabular}{l S[table-format=-1.3] S[table-format=1.3] S[table-format=3.2] S[table-format=-1] S[table-format=<1.4] S[table-format=1.2]}
  \lsptoprule
 & {Estimate} & {SE} & {df} & {$t$} & {$\text{Pr}(>|t|)$} & {OR} \\ 
  \midrule
(Intercept) & 0.856 & 0.577 & 152.52 & 1 & 0.1403 & 2.35 \\ 
  extraction type & -0.131 & 0.061 & 23.29 & -2 & <.05 & 1.14 \\ 
  distance & 0.169 & 0.105 & 25.66 & 2 & 0.1213 & 1.18 \\ 
  length & 0.557 & 0.014 & 521.85 & 39 & <.001 & 1.74 \\ 
  extraction type:distance & -0.042 & 0.073 & 29.31 & -1 & 0.5741 & 1.04 \\ 
   \lspbottomrule
\end{tabular}
\caption{Results of the Linear Mixed Model (model n$^{\circ}$1)}
\label{tab:exp03-m1}
\end{table}


\begin{figure}
    \centering
    \includegraphics[width=\textwidth]{chapters/part2-Empirical/Exp06-dont-RC-animacy-mismatch/interaction.jpeg}
    \caption{Interaction between syntactic function and extraction type in Experiment~6}
    \label{fig:exp06-interaction}
\end{figure}

\subsection{Discussion}

Experiment~6 is really problematic, because all ratings are very high. We can see ceiling effects (Figures \ref{fig:exp06-boxplot} and \ref{fig:exp06-repartition}), and visual cues of a non-normal distribution (\figref{fig:exp06-ROC}). The results of the models do not show any significant effect relevant for the hypotheses at hand. However, there is a main effect of habituation (model n$^{\circ}$2) indicating that the model is powerful enough to identify some effects, even though the habituation effect is not very strong (odds ratio of 1.02). 

Thus Experiment~6 fails to answer the question it was supposed to answer. Consequently we adopted an alternative strategy for the next experiment. 
