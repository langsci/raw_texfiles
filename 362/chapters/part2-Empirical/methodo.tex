\label{ch:exp-methodo}
It is impossible to design any meaningful theory (about the subject island, any other kind of island or any linguistic phenomenon in general) without agreeing on the data in the first place. But as far as extraction out of the subject is concerned, scholars do not agree on the acceptability status of many examples. For example: Is extraction out of the subject of passives acceptable or not? Is extraction out of the subject with \emph{dont} in French better than extraction out of the subject with \emph{de qui}? In the former case, different authors have given different judgements on the same language. In the latter case, \citegen{Godard.1992} intuition, as well as my own, differs from \citegen{Tellier.1990}. 
This disagreement between linguists, even between native speakers, is far from exceptional. Experimental data have shown for a long time that native speakers disagree on such matters as soon as a sentence structure is complex, or the situation it describes infrequent \citep{Chaves.2019.Frequency}. Obviously, filler-gap dependencies are a complex phenomenon in themselves and, unsurprisingly, not all speakers feel equally comfortable with all of them. Often enough, the constructed examples are also at least slightly (and sometimes very) unnatural, and the appropriate context is missing (see discussion in Section \ref{ch:discourse}). Furthermore, linguists are not naive speakers on metalinguistic questions, and a linguist's intuition (including mine), albeit highly valuable, may be biased.
% linguist disease

Quantitative data are by nature more resistant to individual preferences. Provided that we have a sufficient amount of data and that we treat outliers as needed, individual preferences disappear in the statistical result. What remains are general tendencies.
For this reason, nowadays quantitative investigations play an important role  in the linguistic debate.
There are several ways to collect quantitative data, and in this work, I will use two methods that I judge equally important: corpus studies and experiments. 

Islands in the languages of the world have been the focus of many experimental studies, probably starting with \citet{Kluender.1991}. His work itself was the logical extension of the first online experiments on filler-gap dependencies that took place in the 80s \citep[a.o.][]{Tanenhaus.1985,Stowe.1986}. Since then there have been a number of experiments on islands. In Chapter~\ref{ch:previous-exp}, I will present some experiments on subject islands in particular, a list that is  necessarily by no means exhaustive. At the same time, there was little corpus investigation on the subject before our own work \citep{Abeille.2016,Abeille.2020.JFLS}, except for \citet{Candito.2012.ldd} which I will mention below, but which deals with subextractions from the subject only incidentally. It is thus puzzling that extraction out of the subject were considered impossible without even considering whether speakers produced it. 

Corpus data are productions with non-metalinguistic aims, it is therefore useful to look at them before constructed experimental items. The experimental items should describe familiar situations and an appropriate context should be easy to imagine. It is thus important to look at spontaneous productions of native speakers in order to know how filler-gap dependencies are actually used by speakers, what kind of situations are described and what kind of vocabulary is used. 

Here, we should also pay attention to the relative frequencies of different structures based on different factors, especially the relative frequencies of extractions out of subjects and out of objects. My expectations are as follows: 

\begin{enumerate}
    \item If extraction out of subjects is indeed grammatically ruled out, then it should be absent from the corpora (or be accidental~-- I come back to this issue later on). Notice that the opposite is not necessarily true: the fact that a certain structure does not appear in a corpus does not imply nor prove that this structure is absent from the specific language, let alone that it is ruled out by syntax. No corpus, however large it might be, can be expected to include all possible structures of a certain language. 
    \item If extraction out of the subject is not ruled out, but only more difficult to process than extraction out of the object, then we expect the former to be less frequent in the corpora than the latter. Several previous corpus studies have shown that complex subjects are less frequent than complex objects. \citet{Kluender.2004}, among various other scholars, argues that it is because they are harder to process (see Section \ref{ch:processing-kluender}). Extraction out of subjects, per definition, requires complex subjects to begin with, and should consequently be less frequent than extraction out of objects. 
    \item The Focus-Background Conflict constraint (\ref{rule:FBC}) predicts different results for different constructions. Because subjects are usually topics, focalization of a subpart of the subject should be dispreferred. For this reason, extraction out of subjects should be less frequent in interrogatives and \emph{c'est}-clefts. In these constructions, we also predict that the subjects which allow for subextraction can be interpreted as focus (e.g.\ with a contrastive meaning). In relative clauses, there is no constraint on the subextraction from subject: we expect to see different distributions between the different constructions.
\end{enumerate}

As the corpus data cannot provide negative evidence, it is essential to conduct controlled experiments as well. If subextractions from subjects are not found in the corpora, we would need to verify that the speakers do not accept them. Without this proof, their absence in the corpora may be coincidental. Moreover, as we shall see, some constructions (\textit{c'est} clefts and infinitival subjects) are too infrequent to allow a relevant statistical analysis in our corpus studies. Experimental data can provide us with more information on these constructions. 

Since early experimental work on islands by \citet{Kluender.1993.Bridging,Kluender.1993.Subjacency} and then by \citeauthor{Sprouse.2007.PhD} and his colleagues \citep[e.g.][]{Sprouse.2011,Sprouse.2012,Sprouse.2016,Sprouse.2017.design}, it has become common practice to use factorial designs in order to test island hypotheses, and to expect superadditive effects as a result of island constraints. The factorial design is usually a 2*2 design, with a double comparison. The first comparison is between gap sites, comparing the ``island'' gap site with another gap site, similar but not expected to be an island for extraction. For example, the comparison often used in the literature on subject islands is between extraction out of NP subjects versus extraction out of NP objects. The second comparison is between two maximally similar constructions, one expected to create the island under investigation, and the other not expected to do so. For example, we can compare extraction of the whole NP with subextraction out of the NP, or~-- as we often did in the experiments that I will present~-- non-extraction with subextraction.\footnote{We favor non-extraction instead of extraction of the whole NP, because there is a well-known preference in processing (and acceptability judgments) for subject relative clauses over object relative clauses \citep[a.o.][]{Wanner.1978,Traxler.2002,Pozniak.2015}. This preference can create interaction effects in the factorial design, that would not be related to the island phenomenon.} This double comparison ensures that we isolate the factor leading to the ``island effect'', ensuring that it does not come from the gap site independently of the extraction type, or from the extraction type independently of the gap site. Island phenomena are expected to cause a superadditive effect, i.e.\ a statistically significant interaction effect between gap site and extraction type. If subjects are islands, the difference between extraction out of the object and extraction out of the subject is expected to be greater than between the two control conditions (e.g.\ extraction of the object and extraction of the subject). \figref{fig:superadditivity} illustrates the expected difference between a linear additivity and superadditivity.

\begin{figure}[ht]
	\centering
	\includegraphics[width=\textwidth]{figures/superadditivity.jpeg}
	\caption{Prediction for linear additivity vs.\ superadditivity (inspired by \citealt[86]{Sprouse.2012}). Extraction out of the subject (subject + subextraction) is the island condition.}
	\label{fig:superadditivity}
\end{figure}

I will present a series of 16 experiments, most of which follow a 2*3 factorial design. In addition to the usual 2*2 design, there is indeed a comparison between the subextraction condition and two ungrammatical controls. 

In this work, I test the expectations of the various linguistic accounts of subject islands. As it is not possible to take them all into account, I identify five general categories of accounts: (i) ``traditional'' syntactic accounts (syntactic island hypothesis), (ii) processing accounts based on the hypothesis that complex subjects are unexpected, (iii) functional accounts based on the hypothesis that backgrounded constituents are islands to extraction, (iv) processing accounts based on the hypothesis that the distance between dependents should be as short as possible, and (v) functional accounts based on the hypothesis that focalizing a part of a backgrounded constituent results in a discourse clash. There are of course many variations within these categories, especially among the proponents of a syntactic account of subject island, as laid out in Section~\ref{ch:syntax}. However, they give rise to similar general expectations with respect to the experiments I will present. In a nutshell, we can say that accounts in the categories (i), (ii) and (iii) predict superadditive effects such that extractions out of the subjects are less acceptable (for offline experiments) or read more slowly (for online experiments) than the other grammatical conditions. By contrast, accounts from category (iv) predict an advantage for extractions out of the subject. Under the Focus-Background Conflict constraint, in category (v), superadditive effects with extraction out of the subject should only occur in focalizing constructions. 

%concrete guidelines for experimental practice (Cowart 1997; Schütze 1996/22016; Schütze & Sprouse 2013) 

The quantitative analyses reported in this work were conducted in R \citep[version~3.6.3]{R}, and graphs are all created using the \texttt{ggplot2} package \citep{Wickham.2016}. According to the broad consensus in statistics, I consider a result significant if and only if the confidence level is at least 95\%, i.e.\ if and only if the probability value (\emph{p} value) is at most 5\% ($p<0.05$). I indicate the $p$ value relative to six levels of significance: $p<0.05$, $p<0.01$, $p<0.005$, $p<0.001$, $p<0.0005$ or $p<0.0001$. 

The data and R code of all the corpus studies and experiments presented below are available online at \url{https://osf.io/5qhxa/}.
