\section[head=Experiment 4]{Experiment 4: Acceptability judgment study on \emph{dont} relative clauses with long-distance dependencies}

In this experiment, we tested similar stimuli but with long-distance dependencies: the \emph{dont} relative clause contains an embedded clause, and the NP out of which extraction takes place is either subject or object of this embedded clause. For this experiment, we wanted to have a design similar to the long-distance dependencies tested by \citet{Sprouse.2016} in English and Italian. Using long-distance dependencies also ensures that extractions out of the subject are ``real'' extractions and not a movement inside the DP as proposed by \citet{Heck.2009} (see Section \ref{ch:intro-disscussion-French}). 

\subsection{Design and materials}
The experiment was an acceptability judgment task with a 2*3 design. 
In this experiment, we do not test three different distances between \emph{dont} and the gap, but only compare extractions out of subjects (\ref{ex:exp04-subj-pp}) with extractions out of objects (\ref{ex:exp04-obj-pp}), as in \citet{Sprouse.2016}.

\eal 
\ex[]{{Condition subject + PP-extracted:}\nopagebreak\\
\gll Google présente une innovation dont$_i$ on suppose [que [l' originalité~\trace{}$_i$] enthousiasme mes collègues sans aucune raison].\\
Google presents an innovation of.which one suspects that the uniqueness excites my colleagues without any reason\\
\glt `Google presents an innovation of which we suspect that the uniqueness excites my colleagues for no reason.'}
\label{ex:exp04-subj-pp}
\ex[]{{Condition object + PP-extracted:}\\
\gll Google présente une innovation dont$_i$ on suppose [que mes collègues admirent [l' originalité~\trace{}$_i$] sans aucune raison].\\
Google presents an innovation of.which one suspects that my colleagues admire the uniqueness without any reason\\
\glt `Google presents an innovation of which we suspect that my colleagues admire the uniqueness for no reason.'}
\label{ex:exp04-obj-pp}
\zl 

Apart from these subextraction conditions, there is also a non-extraction condition (\ref{ex:exp04-no}) that serves as a baseline. This condition parallels the extraction one, but includes a coordination instead of a relative clause.

\eal \label{ex:exp04-no}
\ex[]{{Condition subject + noextr:}\\
\gll Google présente une innovation, et on suppose que son originalité enthousiasme mes collègues sans aucune raison.\\
Google presents an innovation and one suspects that its uniqueness excites my colleagues without any reason\\
\glt `Google presents an innovation, and we suspect that its uniqueness excites my colleagues for no reason.'}
\label{ex:exp04-subj-no}
\ex[]{{Condition object + noextr:}\\
\gll Google présente une innovation, et on suppose que mes collègues admirent son originalité sans aucune raison.\\
Google presents an innovation and one suspects that my colleagues admire its uniqueness without any reason\\
\glt `Google presents an innovation, and we suspect that my colleagues admire its uniqueness for no reason.'}
\label{ex:exp04-obj-no}
\zl 

These two extraction types are very similar to the design in \citet{Sprouse.2016}. We also included a third extraction type: the relative word \emph{que} instead of \emph{dont}. Like \emph{dont}, \emph{que} is a complementizer, but it is used to extract direct NP objects instead of \emph{de}-PPs. Therefore, the sentences are ungrammatical, and serve as a baseline for low judgments. Notice that the switch from \emph{dont} to \emph{que} is relatively frequent in informal French, but strongly stigmatized as ``bad'' French. It is therefore not a control with a strong ungrammaticality, but is nevertheless a good low baseline for a reading task.

\eal \label{ex:exp04-un}
\ex[]{{Condition subject + ungramm:}\nopagebreak\\
\gll Google présente une innovation qu' on suppose que l' originalité enthousiasme mes collègues sans aucune raison.\\
Google presents an innovation that one suspects that the uniqueness excites my colleagues without any reason\\
\glt `Google presents an innovation that we suspect that the uniqueness excites my colleagues for no reason.'}
\label{ex:exp04-subj-un}
\ex[]{{Condition object + ungramm:}\\
\gll Google présente une innovation qu' on suppose que mes collègues admirent l' originalité sans aucune raison.\\
Google presents an innovation that one suspects that my colleagues admire the uniqueness without any reason\\
\glt `Google presents an innovation of which we suspect that my colleagues admire the uniqueness for no reason.'}
\label{ex:exp04-obj-un}
\zl 

The materials were very similar to the materials used in the first three experiments, but with small changes. We planned this experiment (and some of the following ones) as the French pendant to a cross-linguistic series of studies (French-English) and tried to keep the French and English materials as close to each other as possible. Since some of the items of the previous experiments did not transfer well into English, a few changes were introduced in the French materials.

As in the previous experiments, the relation between \emph{dont} and the gap always expressed a quality (e.g.\ \emph{originalité} `uniqueness', \emph{beauté} `beauty'). The noun out of which the extraction takes place was always inanimate. We used psych verbs that come in reversible pairs (e.g.\ \emph{apprécier} `value' and \emph{émerveiller} `delight'), but also some transitive non-psych verbs (e.g.\ \emph{commenter} `comment'). 

We tested 24 items, each manipulated according to the six conditions already described. In addition, the experiment included 36 distractors. Half of the experimental items and distractors were followed by a comprehension question. The item presented here as an example was followed by the comprehension question \emph{Est-ce que les collègues ont raison d'être enthousiastes~?} (`Are the colleagues right to be enthusiastic?'). 

\subsection{Predictions}

From the different accounts presented in the first part of this work, four big different patterns of predictions emerge, which I briefly discuss here.

A traditional syntactic account predicts a superadditivity effect in extraction out of the subject. For this reason the first expectation is that the acceptability of (\ref{ex:exp04-subj-pp}) will be degraded compared to (\ref{ex:exp04-obj-pp}). Long-distance dependencies should be less acceptable than the non-extraction conditions because they are more difficult. Additionally an interaction effect is expected such that (\ref{ex:exp04-obj-pp}) should be worse than the three non-island conditions (\ref{ex:exp04-obj-pp}), (\ref{ex:exp04-subj-no}) and (\ref{ex:exp04-obj-no}). The island condition (\ref{ex:exp04-subj-pp}) should not be significantly better than the ungrammatical controls (\ref{ex:exp04-subj-un}) and (\ref{ex:exp04-obj-un}), and an interaction effect is expected as well, such that the object control (\ref{ex:exp04-obj-pp}) is better than these three ungrammatical conditions. 

A processing account based on surprisal due to subject complexity \citep[such as][]{Kluender.2004} makes similar predictions. The only difference is that, even though extraction out of the subject should be degraded, it is not necessarily expected to be as bad as  ungrammatical controls. Subextractions in general should be rated higher than ungrammatical controls (main effect). A discourse-based account like the one expressed by \citet{Erteschik-Shir.1973} or \citet{Goldberg.2006} makes essentially the same prediction.

A processing account based on memory costs (like Dependency Grammar or the DLT) expects extraction out of the subject  (\ref{ex:exp04-subj-pp}) to receive better ratings than extraction out of the object (\ref{ex:exp04-obj-pp}), the distance between \emph{dont} and the gap being longer in the latter case. When we compare the non-extraction conditions with the subextractions, an interaction effect is expected, because extraction out of the object like (\ref{ex:exp04-obj-pp}) should give rise to stronger processing difficulties. But subextractions are expected to be better than the ungrammatical controls. 

Our discourse-based account with the FBC constraint only predicts a main effect of extraction types; the control without extraction should be better than the extraction condition, which in turn should be better than the ungrammatical controls. The account is neutral as to whether there should be a main effect of syntactic function (subject conditions better than object conditions or inversely), but it crucially does not expect to see any significant interaction effect. 

For the sake of simplicity, we can summarize these different expectations in two major predictions: some accounts predict a ``subject island'' effect, whereas others do not predict it, with some minor differences between the accounts.  

\subsection{Procedure}

We conducted the experiment on the Ibex platform \citep{Ibex}. The procedure used in acceptability judgment tasks is explained in Section \ref{ch:methodo-AJ}. Participants had to rate the sentences on a Likert-scale from 0 to 10, 0 being labeled as ``bad'' and 10 being labeled as ``good''.

The experiment took approximately 20 minutes to complete. The participants were recruited through FouleFactory (\url{https://www.foulefactory.com}) and paid 5€ for their participation. The payment was not contingent on the participants' responses to the questions about native language or place of birth.

\subsection{Participants}

The study was conducted in October 2019. 
57 participants took part in the experiment. We present the analysis based on the answers of the 51 participants who satisfied all inclusion criteria.\footnote{In addition to the five usual criteria, we excluded two participants who did not use the Likert scale appropriately. To calculate accuracy, we excluded not only the answers to comprehension questions of the practice items and ungrammatical controls like (\ref{ex:exp04-un}), but also to some ungrammatical distractors.}
The 51 participants were 21 to 67 years old. 31 participants self-identified as women and 20 as men. None of them indicated having an educational background related to language.

\subsection{Results and analysis}

\figref{fig:exp04-boxplot} summarizes the results of the acceptability judgment task. 
In the subextraction condition, extraction out of the subject (\ref{ex:exp04-subj-pp}) received a mean acceptability rating of 7.63, slightly higher than extraction out of the object (\ref{ex:exp04-obj-pp}) with a mean rating of 7.42. The control conditions without extraction were judged better overall: the subject condition (\ref{ex:exp04-subj-no}) received a mean rating  of 8.10, the object condition (\ref{ex:exp04-obj-no}) 7.48. The ungrammatical controls were rated low: 3.64 in the subject condition (\ref{ex:exp04-subj-un}), and 3.21 in the object condition (\ref{ex:exp04-obj-un}).

\begin{figure}
    \centering
    \includegraphics[width=\textwidth]{chapters/part2-Empirical/Exp04-dont-RC-ldd/boxplots.jpeg}
    \caption{Acceptability judgments by condition in Experiment 4. The grey box plots indicate the median and quartiles of the results. Black points are outliers. Mean and confidence intervals are indicated in white.}
    \label{fig:exp04-boxplot}
\end{figure}

\figref{fig:exp04-boxplot} suggests potential ceiling effects on the extraction conditions and the non-extraction conditions, and a potential floor effect on the ungrammatical control. Indeed, \figref{fig:exp04-repartition} shows indications of ceiling effects and floor effects, especially for the subject + non-extraction condition. 

\begin{figure}
    \centering
    \includegraphics[width=\textwidth]{chapters/part2-Empirical/Exp04-dont-RC-ldd/repartition.jpeg}
    \caption{Density of the ratings across conditions for Experiment 4}
    \label{fig:exp04-repartition}
\end{figure}

Another representation of the results is given by their ROC and zROC curves in \figref{fig:exp04-ROC}. There is strong discrimination between the ungrammatical baseline and the other conditions, but no strong discrimination between coordination and subextraction. The zROC curves are slightly convex, which could be the sign for bimodality \citep[21--22]{Dillon.2019}. This may be due to the strong habituation effect on the ungrammatical controls (see below).

\begin{figure}
    \centering
    \includegraphics[width=\textwidth]{chapters/part2-Empirical/Exp04-dont-RC-ldd/ROC.jpeg}
    \includegraphics[width=\textwidth]{chapters/part2-Empirical/Exp04-dont-RC-ldd/zROC.jpeg}
    \caption{ROC curves (top) and zROC curves (bottom) of the grammatical conditions compared to their respective ungrammatical condition, represented by the dotted grey baseline (\citealt{Dillon.2019}'s method) in Experiment 4.}
    \label{fig:exp04-ROC}
\end{figure}

The ROC and zROC curves in \figref{fig:exp04-ROC-subj} illustrate the discrimination between the subject and object conditions. The ROC curves show that the participants barely discriminate between the subject and object conditions. The most important distinction is between the non-extraction and the subextraction conditions. As we show below, there is indeed a small tendency toward an interaction, but it is not significant. The zROC curves show straight lines, which is a visual cue that the distribution is normal.

\begin{figure}
    \centering
    \includegraphics[width=\textwidth]{chapters/part2-Empirical/Exp04-dont-RC-ldd/ROC-subject.jpeg}
    \includegraphics[width=\textwidth]{chapters/part2-Empirical/Exp04-dont-RC-ldd/zROC-subject.jpeg}
    \caption{ROC curves (top) and zROC curves (bottom) of the subject conditions compared to their respective object condition, represented by the dotted grey baseline (\citealt{Dillon.2019}'s method) in Experiment 4.}
    \label{fig:exp04-ROC-subj}
\end{figure}

\subsubsection{Habituation} 

\figref{fig:exp04-habituation} shows the habituation effects in the course of the experiment. All conditions undergo habituation during the experiment, but to different degrees. The ffect is strong for the ungrammatical controls, even though the judgments remain very low until the end of the experiment. We can also see an important habituation effect for the object + non-extraction condition that I cannot explain. 

\begin{figure}
    \centering
    \includegraphics[width=\textwidth]{chapters/part2-Empirical/Exp04-dont-RC-ldd/habituation.jpeg}
    \caption{Changes in the average acceptability ratings ($z$-scored by participant) for each condition of Experiment 4 in the course of the experiment}
    \label{fig:exp04-habituation}
\end{figure}

\subsubsection{Comparing subextraction from the subject with subextraction from the object}

We fitted a first model to compare extractions out of the subject and out of the object on their own (mean centered with subject coded negative and object coded positive). We included trial number as a covariate, and random slopes for all fixed effects and covariates grouped by participants and items. The results of the model are reported in \tabref{tab:exp04-m1}. 
There is a significant effect of the syntactic function: the subject condition received significantly higher ratings than the object condition. There is also a significant effect of trial (habituation), which corroborates the impression given by \figref{fig:exp04-habituation}.

% latex table generated in R 3.6.3 by xtable 1.8-4 package
% Wed May 20 14:00:46 2020
\begin{table}
\begin{tabular}{l S[table-format=1.3] S[table-format=1.3] S[table-format=1] S[table-format=<1.4] S[table-format=1.2]}
  \lsptoprule
 & {Estimate} & {SE} & {$z$} & {$\text{Pr}(>|z|)$} & {Odd.ratio} \\ 
  \midrule
(Intercept) & 1.127 & 0.239 & 5 & <.001 & 3.09 \\ 
  syntactic function & 0.044 & 0.091 & 0 & 0.6283 & 1.04 \\ 
  trial & 0.012 & 0.005 & 2 & <.05 & 1.01 \\ 
   \lspbottomrule
\end{tabular}
\caption{Results of the Logistic regression model (model n$^{\circ}$2)}
\label{tab:exp02-m2}
\end{table}


In a second model, we compared subextraction with non-extraction. We fitted a model crossing syntactic function and extraction type (mean centered with extraction coded positive, non-extraction coded negative). We included trial number as a covariate, and random slopes for all fixed effects grouped by participants and items. The results of the model are reported in \tabref{tab:exp04-m2}.
There are significant main effects of the syntactic function (in favor of the subject condition), of extraction type (in favor of the non-extraction controls), and of trial (habituation). There is no significant interaction effect. \figref{fig:exp04-interaction1} illustrates the interaction: we see a weak tendency toward an interaction effect, but the confidence intervals overlap. Furthermore, if we compare the Area Under the Curve (AUC) for the ROC curves of the two grammatical conditions (see \figref{fig:exp04-ROC-subj}) the difference is not significant, either.

% latex table generated in R 3.6.3 by xtable 1.8-4 package
% Sun Apr 26 23:02:02 2020
\begin{table}
\begin{tabular}{l S[table-format=1.3] S[table-format=1.3] c S[table-format=<1.3] S[table-format=1.2]}
  \lsptoprule
                     & {Estimate} & {SE} & {$z$} & {$\text{Pr}(>|z|)$} & {OR} \\ 
  \midrule
  syntactic function & 1.328 & 0.488 & 3 & <.01 & 3.77 \\ 
  trial              & 0.069 & 0.023 & 3 & <.005 & 1.07 \\ 
 \lspbottomrule
\end{tabular}
\caption{Results of the Cumulative Link Mixed Model (model n$^{\circ}$1)}
\label{tab:exp16-m1}
\end{table}


\begin{figure}
    \centering
    \includegraphics[width=\textwidth]{chapters/part2-Empirical/Exp04-dont-RC-ldd/interaction.jpeg}
    \caption{Interaction between syntactic function and extraction type in Experiment 4}
    \label{fig:exp04-interaction1}
\end{figure}

\pagebreak
A third model compared the extractions out of the subject and the ungrammatical subject condition on their own (mean centered with extraction coded negative and ungrammatical coded positive). We included trial number as a covariate, and random slopes for fixed effects and covariates grouped by participants and items. The results of the model are reported in \tabref{tab:exp04-m3}. 
There is a significant effect of extraction type, such that the subextraction condition is better than the non-extraction condition (with a strong effect size: odds ratio~= 6.58) and a significant habituation effect. 

% latex table generated in R 3.6.3 by xtable 1.8-4 package
% Thu Apr 23 00:04:53 2020
\begin{table}
\begin{tabular}{l S[table-format=1.3] S[table-format=1.3] c S[table-format=<1.3] S[table-format=2.2]}
  \lsptoprule
 & {Estimate} & {SE} & {$z$} & {$\text{Pr}(>|z|)$} & {OR}\\ 
  \midrule
  extraction type & 2.342 & 0.395 & 6 & <.001 & 10.40 \\ 
  trial           & 0.039 & 0.012 & 3 & <.005 & 1.04 \\ 
   \lspbottomrule
\end{tabular}
\caption{Results of the Cumulative Link Mixed Model (model n$^{\circ}$3)}
\label{tab:exp10-m3}
\end{table}


In a fourth model, we compared subextraction with the ungrammatical control. We fitted a model crossing syntactic function and extraction type (mean centered with extraction coded positive, ungrammatical coded negative). We included trial number as a covariate, and random slopes for all fixed effects grouped by participants and items. The results of the model are reported in \tabref{tab:exp04-m4}. 
There are main effects of syntactic function (in favor of the subject condition), of extraction type (in favor of the subextraction, with a strong effect size: odds ratio = 6.82) and a main effect of trial (habituation), but no interaction effect (see also \figref{fig:exp04-interaction1}).

% latex table generated in R 3.6.3 by xtable 1.8-4 package
% Sun Apr 26 23:02:15 2020
\begin{table}
\begin{tabular}{l S[table-format=1.3] S[table-format=1.3] c S[table-format=<1.3] S[table-format=1.2]}
  \lsptoprule
 & {Est.} & {SE} & {$z$} & {$\text{Pr}(>|z|)$} & {OR} \\ 
  \midrule
  syntactic function & 0.474 & 0.465 & 1 & 0.307 & 1.61 \\ 
  trial & 0.034 & 0.014 & 2 & <.05 & 1.03 \\ 
  syntactic function:trial & 0.014 & 0.015 & 1 & 0.342 & 1.01 \\ 
   \lspbottomrule
\end{tabular}
\caption{Results of the Cumulative Link Mixed Model (model n$^{\circ}$2)}
\label{tab:exp16-m2}
\end{table}


\subsection{Discussion}

The data from Experiment 4 seem similar to what we saw in Experiment~1. If we compare the results with what is expected under superadditivity, we can first notice that extractions out of the subject are significantly better than extractions out of the object (model n$^{\circ}$1, estimate has a negative value), contrary to what is predicted. Furthermore, there is no interaction effect to corroborate the prediction (model n$^{\circ}$2). Extraction out of subject is significantly better than its ungrammatical control (model n$^{\circ}$3) contrary to what the traditional syntactic approach predicts, and here again there is no interaction effect (model n$^{\circ}$4).

If we now compare the results to what is expected under accounts that do not predict a superadditivity effect, we can see that the data do not falsify the predictions. The fact that extraction out of the subject is better than extraction out of the object (model n$^{\circ}$1) is in line with processing accounts based on memory load. However, this subject preference is present for all extraction types: syntactic function is a main effect in models n$^{\circ}$2 and n$^{\circ}$4, with no interaction effect. This means that extraction out of the subject is not facilitated beyond the effect observed in the baseline conditions, whereas a facilitation should be expected under a processing account based on memory costs. But we cannot rule out that the ceiling effect on the non-extraction conditions is responsible for cancelling out potential interaction effects.

The data are perfectly in line with the predictions of the FBC constraint. There is a main effect of extraction type (models n$^{\circ}$2 and n$^{\circ}$4), and no interaction effect, as expected. Undeniably, Experiment 4 alone does not do much to support or weaken the FBC constraint hypothesis. A recurring problem concerning relative clauses is that the FBC constraint predicts null effects. Statistical analyses can only reject a null hypothesis, never confirm it. The results of the experiments on relative clauses will gain importance, though, when we compare them with the experiments on interrogatives (Experiment~10 to 13) and on \emph{c'est}-clefts (Experiment~14). However, we can already see at this point that all accounts that predict extractions out of the subject to be degraded compared to extractions out of the object are falsified.

%The habituation graph (\figref{fig:exp04-habituation} on page \pageref{fig:exp04-habituation}) shows a strong habituation effect for the object + non-extraction condition. The reader may recall that Experiment 1 showed a strong decrease of acceptability ratings during the course of the experiment for the medium distance, i.e. the object+clitic condition (\figref{fig:exp01-habituation} on page \pageref{fig:exp01-habituation}).
