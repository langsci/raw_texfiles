\section[head=Experiment 10]{Experiment 10: Acceptability judgment study on \emph{de quel} \emph{wh}-questions with \emph{wh}-extraction}

In Section \ref{ch:deq-corpus}, I presented a corpus study on \emph{de quel} interrogatives. We saw that, in contrast to the corpus studies on relative clauses, we cannot find any extraction out of the subject in Frantext involving \emph{de quel} + N. The aim of this study was to test this kind of interrogative with an experimental design.

\subsection{Design and materials}
To construct the stimuli for Experiment~10, we took the materials of Experiment~4 as a starting point and turned the items into interrogatives, without the long-distance dependency, which was tested in Experiment~13. For the subextraction conditions, we used \emph{est-ce que} interrogatives (lit.\ `is it that') because they allowed us to avoid subject-verb inversion and are natural in written and spoken language.

\eal 
\ex[]{{Condition subject + PP-extracted:}\\
\gll [De quelle innovation]$_i$ est - ce que [l' originalité~\trace{}$_i$] enthousiasme mes collègues sans aucune raison~?\\
of which innovation is {} it that the uniqueness excites my colleagues without any reason\\
\glt `Of which innovation does the uniqueness excite my colleagues for no reason?'}
\label{ex:exp10-subj-pp}
\ex[]{{Condition object + PP-extracted:}\\
\gll [De quelle innovation]$_i$ est - ce que mes collègues admirent [l' originalité~\trace{}$_i$] sans aucune raison~?\\
of which innovation is {} it that my colleagues admire the uniqueness without any reason\\
\glt `Of which innovation do my colleagues admire the uniqueness for no reason?'}
\label{ex:exp10-obj-pp}
\zl 

For the non-extraction conditions, we used polar questions (\ref{ex:exp10-no}), also constructed with \emph{est-ce que}.\largerpage

\eal \label{ex:exp10-no}
\ex[]{{Condition subject + noextr:}\nopagebreak\\
\gll Est - ce que l‘ originalité de cette innovation enthousiasme mes collègues sans aucune raison~?\\
is {} it that the uniqueness of this innovation excites my colleagues without any reason\\
\glt `Does the uniqueness of this innovation excite my colleagues for no reason?'}
\label{ex:exp10-subj-no}
\ex[]{{Condition object + noextr:}\\
\gll Est - ce que mes collègues admirent l' originalité de cette innovation sans aucune raison~?\\
is {} it that my colleagues admire the uniqueness of this innovation without any reason\\
\glt `Do my colleagues admire the uniqueness of this innovation for no reason?'}
\label{ex:exp10-obj-no}
\zl 

For the ungrammatical controls we modified the subextraction conditions by leaving out the preposition of the extracted element. 

\eal \label{ex:exp10-un}
\ex[]{{Condition subject + ungramm:}\\
\gll [Quelle innovation]$_i$ est - ce que [l' originalité~\trace{}$_i$] enthousiasme mes collègues sans aucune raison~?\\
which innovation is {} it that the uniqueness excites my colleagues without any reason\\
\glt `Which innovation does the uniqueness excite my colleagues for no reason?'}
\label{ex:exp10-subj-un}
\ex[]{{Condition object + ungramm:}\\
\gll [Quelle innovation]$_i$ est - ce que mes collègues admirent [l' originalité~\trace{}$_i$] sans aucune raison~?\\
which innovation is {} it that my colleagues admire the uniqueness without any reason\\
\glt `Which innovation do my colleagues admire the uniqueness for no reason?'}
\label{ex:exp10-obj-un}
\zl 

The items were very similar to the ones used in Experiment~4: the relation between \emph{de quel}~+ N and the gap always expressed a quality (e.g.\ \emph{originalité} `uniqueness', \emph{beauté} `beauty'), and extraction always took place out of an NP headed by an inanimate noun. We used pairs of psych verbs (e.g.\ \emph{apprécier} `value' and \emph{émerveiller} `delight'), but also some transitive non-psych verbs (e.g.\ \emph{commenter} `comment'). 

We tested 24 items, each manipulated according to the six conditions described above. In addition, the experiment included 32 distractors, also interrogatives. Each experimental item and distractor was followed by a comprehension question. The comprehension questions did not have the form of an interrogative, in order to distinguish them from the sentences to be rated. Instead, they were statements, and participants were asked to respond by clicking on \emph{Oui} (`Yes') or \emph{Non} (`No'). The sample item presented above was followed by the comprehension question \emph{Les collègues ont raison d'être enthousiastes.} (`The colleagues are right to be enthusiastic.'). Accuracy was very high in all conditions, indicating that participants understood the task.

\subsection{Predictions}

The predictions are summarized in Table~\ref{tab:exp10-predictions}. Because the design was very similar to the experiments we presented on relative clauses, and because only the FBC constraint predicts a cross-construction difference for extractions, all other accounts make the same predictions for this study (and for all following acceptability judgment studies on interrogatives) as they did for relative clauses.

As the questions in this experiment are presented in isolation, without any specific context, participants should treat the subject as the topic of the utterance. This leads to a violation of the FBC constraint, because questions, unlike relative clauses, focalize the the extracted element. We should then see a superadditivity effect 
disfavoring extraction out of the subject: extraction out of the subject (\ref{ex:exp10-subj-pp}) should be degraded compared to extraction out of the object (\ref{ex:exp10-obj-pp}), and should also have lower ratings than the non-extraction controls (\ref{ex:exp10-no}), leading to an interaction effect. However, a violation of the FBC constraint should not lead to ungrammaticality, thus extraction out of the subject (\ref{ex:exp10-subj-pp}) should still be rated higher than the ungrammatical controls (\ref{ex:exp10-subj-un}). To summarize, the account based on the FBC constraint makes similar predictions as \citet{Kluender.2004} or \citet{Goldberg.2006} as far as \emph{wh}-questions are concerned. 

% http://www.tablesgenerator.com/#

\begin{sidewaystable}
\oneline{%
\begin{tabular}{llllll}
\lsptoprule
 &
  \multicolumn{5}{c}{{Predictions}} \\ \cmidrule(lr){2-6} 
 &
  \multicolumn{3}{c}{{“subject island” accounts}} &
  \multicolumn{2}{c}{{no-island accounts}} \\ \cmidrule(lr){2-6} 
 &
  {\begin{tabular}[c]{@{}l@{}}“traditional” \\ syntactic account\end{tabular}} &
  {\begin{tabular}[c]{@{}l@{}}processing account \\ with surprisal \\ due to subject \\ complexity\end{tabular}} &
  {\begin{tabular}[c]{@{}l@{}}BCI account\\ (Goldberg 2006)\end{tabular}} &
  {\begin{tabular}[c]{@{}l@{}}account based\\ on linear distance\\ (DG, DLT)\end{tabular}} &
  {\begin{tabular}[c]{@{}l@{}}FBC constraint \\ account\end{tabular}} \\ \midrule
\multicolumn{1}{l}{{\begin{tabular}[c]{@{}l@{}}extractions \\ out of the subject\\ vs. extractions \\ out of the object\end{tabular}}} &
  (a) \textless (b) &
  (a) \textless (b) &
  (a) \textless (b) &
  (a) \textgreater (b) &
  (a) \textless (b) \\ \midrule
\multicolumn{1}{l}{{\begin{tabular}[c]{@{}l@{}}extractions vs. \\ non-extractions\end{tabular}}} &
  \begin{tabular}[c]{@{}l@{}}main effect of extraction\\ + interaction effect \\ such that (a) \textless (b,c,d)\end{tabular} &
  \begin{tabular}[c]{@{}l@{}}main effect of extraction\\ + interaction effect \\ such that (a) \textless (b,c,d)\end{tabular} &
  \begin{tabular}[c]{@{}l@{}}main effect of extraction\\ + interaction effect \\ such that (a) \textless (b,c,d)\end{tabular} &
  \begin{tabular}[c]{@{}l@{}}interaction effect,\\ such that (b) \textless (a,c,d)\end{tabular} &
  \begin{tabular}[c]{@{}l@{}}main effect of extraction\\ + interaction effect \\ such that (a) \textless (b,c,d)\end{tabular} \\ \midrule
\multicolumn{1}{l}{{\begin{tabular}[c]{@{}l@{}}extractions \\ out of the subject\\ vs. ungrammatical\\ controls\end{tabular}}} &
  (a) $\simeq$ (e) &
  (a) \textgreater (e) &
  (a) \textgreater (e) &
  (a) \textgreater (e) &
  (a) \textgreater (e) \\ \midrule
\multicolumn{1}{l}{{\begin{tabular}[c]{@{}l@{}}extractions vs.\\ vs. ungrammatical\\ controls\end{tabular}}} &
  \begin{tabular}[c]{@{}l@{}}interaction effect\\ such that (b) \textgreater (a,e,f)\end{tabular} &
  \begin{tabular}[c]{@{}l@{}}main effect\\ of grammaticality\end{tabular} &
  \begin{tabular}[c]{@{}l@{}}main effect\\ of grammaticality\end{tabular} &
  \begin{tabular}[c]{@{}l@{}}main effect\\ of grammaticality\end{tabular} &
  \begin{tabular}[c]{@{}l@{}}main effect\\ of grammaticality\end{tabular} \\ \lspbottomrule
\end{tabular}}
\caption{Predictions of the different accounts for Experiments 10, 12 and 13 (interrogatives with extraction). \emph{Notes}: (a) Condition subject + PP-extracted (b) Condition object + PP-extracted (c) Condition subject + no extraction (d) Condition object + no extraction (e) Condition subject + ungrammatical (f) Condition object + ungrammatical.}
\label{tab:exp10-predictions}
\end{sidewaystable}


\subsection{Procedure} 

We conducted the experiment on the Ibex platform \citep{Ibex}. The procedure was similar to the procedure used in the previous acceptability judgment experiments (see Section \ref{ch:methodo-AJ}). Participants rated the sentences on a Likert scale from 1 to 10, 1 being labeled as ``bad'' and 10 being labeled as ``good''. They also answered comprehension questions after each sentence.

The experiment took approximately 20 minutes to complete. 

\subsection{Participants}

The study was run in October 2017. 
Participants were recruited on the R.I.S.C.\ website (\url{http://experiences.risc.cnrs.fr/}) and on social media (e.g.\ Facebook).
They received no financial compensation.

55 participants took part in the experiment. The analysis presented here is based on the data from the 47 participants who satisfied all criteria.\footnote{To calculate accuracy, we excluded not only the answers to comprehension questions of the practice items and of the ungrammatical controls, but also of some distractors that had a low accuracy rate.}
The 47 participants were aged 18 to 76 years. 34 of them self-identified as women, 13 self-identified as men. Six participants (12.77\%) indicated having an educational background related to language.

\subsection{Results and analysis}

\figref{fig:exp10-boxplot} shows the results of the acceptability judgment task.
In the subextraction conditions, extraction out of the subject (\ref{ex:exp10-subj-pp}) received a mean rating of 4.53, lower than extraction out of the object (\ref{ex:exp10-obj-pp}), which had a mean rating of 6.39. The non-extraction conditions were rated higher, with a mean rating of 8.41 in the subject control condition (\ref{ex:exp10-subj-no}), and 7.95 in the object control condition (\ref{ex:exp10-obj-no}). The ungrammatical controls received very low acceptability ratings: 2.27 in the subject condition (\ref{ex:exp10-subj-un}), and 2.73 in the object condition (\ref{ex:exp10-obj-un}). 

\begin{figure}
    \centering
    \includegraphics[width=\textwidth]{chapters/part2-Empirical/Exp10-dequel-whq/boxplots.jpeg}
    \caption{Acceptability judgments by condition in Experiment~10. The grey box plots indicate the median and quartiles of the results. Black points are outliers. Mean and confidence intervals are indicated in white.}
    \label{fig:exp10-boxplot}
\end{figure}

\figref{fig:exp10-boxplot} suggests a potential ceiling effect in the non-extraction conditions and a potential floor effect in the ungrammatical controls. The ratings for the subextraction conditions, however, appear to be distributed along the whole scale. The exact distribution of the ratings is illustrated by \figref{fig:exp10-repartition}: it shows especially a floor effect in the ungrammatical subject condition. The ratings for the subextraction conditions seem to be distributed almost evenly along the scale, with a small peak at the bottom of the scale for the extractions out of subject and at the top for the extractions out of object. 

\begin{figure}
    \centering
    \includegraphics[width=\textwidth]{chapters/part2-Empirical/Exp10-dequel-whq/repartition.jpeg}
    \caption{Density of the ratings across conditions for Experiment~10}
    \label{fig:exp10-repartition}
\end{figure}

Another representation of the results is given by the ROC and zROC curves in \figref{fig:exp10-ROC}. The ROC curves show that participants discriminated between the ungrammatical baselines and the other conditions. Corroborating what we see in \figref{fig:exp10-boxplot}, the non-extraction conditions build larger curves than the subextraction conditions. The zROC curves are relatively straight and parallel to the baseline. 

\begin{figure}
    \centering
    \includegraphics[width=\textwidth]{chapters/part2-Empirical/Exp10-dequel-whq/ROC.jpeg}
    \includegraphics[width=\textwidth]{chapters/part2-Empirical/Exp10-dequel-whq/zROC.jpeg}
    \caption{ROC curves (top) and zROC curves (bottom) of the non-extraction conditions compared to their respective subextraction condition, represented by the dotted grey baseline (\citealt{Dillon.2019}'s method) in Experiment~10.}
    \label{fig:exp10-ROC}
\end{figure}

The ROC and zROC curves in \figref{fig:exp10-ROC-subj} show the discrimination between the subject and object conditions. We see that it is in favor of the object condition for subextractions and the ungrammatical controls (the curves are above the baseline), and in favor of the subject for the controls without extraction. The zROC curves are slightly convex.

\begin{figure}
    \centering
    \includegraphics[width=\textwidth]{chapters/part2-Empirical/Exp10-dequel-whq/ROC-subject.jpeg}
    \includegraphics[width=\textwidth]{chapters/part2-Empirical/Exp10-dequel-whq/zROC-subject.jpeg}
    \caption{ROC curves (top) and zROC curves (bottom) of the object conditions compared to their respective subject conditions, represented by the dotted grey baseline (\citealt{Dillon.2019}'s method) in Experiment~10.}
    \label{fig:exp10-ROC-subj}
\end{figure}

\subsubsection{Habituation} 

The habituation effects in the course of the experiment are shown in \figref{fig:exp10-habituation}. We can see clearly that the ratings are grouped by extraction type on the graph: non-extraction at the top, subextractions in the middle, ungrammatical controls at the bottom. A discrimination between subject and object is only clear for the subextraction conditions. All conditions except the non-extraction controls undergo habituation. Despite the habituation, the extractions out of the subject never reach the acceptability ratings that the extractions out of the object display at the beginning of the experiment. They also do not seem to undergo a stronger habituation than the extractions out of the object. These facts indicate that the reduced acceptability of extractions out of the subject compared to extractions out of the object is very robust.

\begin{figure}
    \centering
    \includegraphics[width=\textwidth]{chapters/part2-Empirical/Exp10-dequel-whq/habituation.jpeg}
    \caption{Changes in the mean acceptability ratings ($z$-scored by participant) by condition in the course of Experiment~10}
    \label{fig:exp10-habituation}
\end{figure}

\subsubsection{Comparing subextraction from the subject with subextraction from the object}

We fitted a first model to compare the extractions out of the subject and out of the object on their own (mean centered with subject coded negative and object coded positive). We included trial number as a covariate, and random slopes for the fixed effects grouped by participants and items. The results of the model are reported in Table \ref{tab:exp10-m1}. 
There is a significant effect of the syntactic function, such that the object condition received significantly higher ratings than the subject condition. There is also a significant effect of trial (habituation).

% latex table generated in R 3.6.3 by xtable 1.8-4 package
% Fri Apr 24 21:31:19 2020
\begin{table}
\begin{tabular}{l S[table-format=1.3] S[table-format=1.4] c S[table-format=<1.3] S[table-format=1.2]}
  \lsptoprule
 & {Estimate} & {SE} & {$z$} & {$\text{Pr}(>|z|)$} & {OR} \\ 
  \midrule
  syntactic function & 0.459 & 0.144 & 3 & <.005 & 1.58 \\ 
  trial              & 0.016 & 0.010 & 2 & 0.1025 & 1.02 \\ 
   \lspbottomrule
\end{tabular}
\caption{Results of the Cumulative Link Mixed Model (model n$^{\circ}$2)}
\label{tab:exp12-m2}
\end{table}


In a second model, we compared the subextractions with the non-extractions. We fitted a model crossing syntactic function and extraction type (mean centered with extraction coded positive, non-extraction coded negative). We included trial number as a covariate, and random slopes for all fixed effects and covariates grouped by participants and items. The results of the model are reported in Table \ref{tab:exp10-m2}. 
There is a significant main effect of syntactic function (in favor of the object), a significant main effect of extraction type (non-extractions were rated higher), and a significant main effect of trial (habituation). There is also a significant interaction effect. \figref{fig:exp10-interaction} indeed shows considerably lower ratings for extractions out of subjects compared to the other conditions. The difference is also significant ($p < 0.005$) if we compare the the AUCs (green and red curves on \figref{fig:exp10-ROC-subj} on page \pageref{fig:exp10-ROC-subj}). 

% latex table generated in R 3.6.3 by xtable 1.8-4 package
% Mon Apr 13 17:44:16 2020
\begin{table}
\begin{tabular}{l S[table-format=-1.3] S[table-format=1.3] S[table-format=1] S[table-format=1.4] S[table-format=1.2]}
  \lsptoprule
 & {Estimate} & {SE} & {$z$} & {$\text{Pr}(>|z|)$} & {Odd.ratio} \\ 
  \midrule
  syntactic function & -0.070 & 0.324 & -0 & 0.8286 & 1.07 \\ 
  trial              & 0.024 & 0.018 & 1 & 0.1717 & 1.02 \\ 
   \lspbottomrule
\end{tabular}
\caption{Results of the Cumulative Link Mixed Model (model n$^{\circ}$1)}
\label{tab:exp06-m1}
\end{table}


\begin{figure}
    \centering
    \includegraphics[width=\textwidth]{chapters/part2-Empirical/Exp10-dequel-whq/interaction.jpeg}
    \caption{Interaction between syntactic function and extraction type in Experiment~10}
    \label{fig:exp10-interaction}
\end{figure}

\subsubsection{Comparing subextraction from the subject with ungrammatical controls}

A third model compared extraction out of the subject and the subject ungrammatical controls on their own (mean centered with subextraction coded positive and ungrammatical coded negative). We included trial number as a covariate, and random slopes for all fixed effects and covariates grouped by participants and items. The results of the model are reported in Table \ref{tab:exp10-m3}. There is a significant effect of extraction type, such that ratings for extraction out of the subject are significantly higher than for its ungrammatical control. There is also a significant effect of trial (habituation).

% latex table generated in R 3.6.3 by xtable 1.8-4 package
% Thu Apr 23 00:04:53 2020
\begin{table}
\begin{tabular}{l S[table-format=1.3] S[table-format=1.3] c S[table-format=<1.3] S[table-format=2.2]}
  \lsptoprule
 & {Estimate} & {SE} & {$z$} & {$\text{Pr}(>|z|)$} & {OR}\\ 
  \midrule
  extraction type & 2.342 & 0.395 & 6 & <.001 & 10.40 \\ 
  trial           & 0.039 & 0.012 & 3 & <.005 & 1.04 \\ 
   \lspbottomrule
\end{tabular}
\caption{Results of the Cumulative Link Mixed Model (model n$^{\circ}$3)}
\label{tab:exp10-m3}
\end{table}


In a fourth model, we compared subextraction with the ungrammatical controls. We fitted a model crossing syntactic function (mean centered with object coded positive, subject coded negative) and extraction type (grammaticality). We included trial number as a covariate, and random slopes for all fixed effects grouped by participants and items. The results of the model are reported in Table \ref{tab:exp10-m4}. 
There is a significant main effect of syntactic function (in favor of the object), of extraction type (in favor of the extraction conditions) and of trial (habituation). There is also a significant interaction: extraction out of the object was rated higher than all other conditions.

% latex table generated in R 3.6.3 by xtable 1.8-4 package
% Sat Apr 25 19:15:30 2020
\begin{table}
\begin{tabular}{l S[table-format=1.3] S[table-format=1.3] c S[table-format=<1.4] S[table-format=1.2]}
  \lsptoprule
 & {Estimate} & {SE} & {$z$} & {$\text{Pr}(>|z|)$} & {OR} \\ 
  \midrule
  syntactic function & 0.134 & 0.091 & 1 & 0.1405 & 1.14 \\ 
  extraction type & 0.631 & 0.133 & 5 & <.001 & 1.88 \\ 
  trial & 0.025 & 0.005 & 5 & <.001 & 1.03 \\ 
  syntactic function:extraction type & 0.009 & 0.088 & 0 & 0.9142 & 1.01 \\ 
   \lspbottomrule
\end{tabular}
\caption{Results of the Cumulative Link Mixed Model (model n$^{\circ}$5)}
\label{tab:exp13-m5}
\end{table}


\subsection{Discussion}

The results of Experiment~10 strikingly differ from the results of all previous experiments on relative clauses. When extracting out of subjects by means of a \emph{wh}-question, we observe an ``island effect'': the ratings are lower than those for extraction out of the object (model n$^{\circ}$1), and there is a significant interaction (model n$^{\circ}$2). The interaction is significant even in the more conservative analysis of the AUCs.

However, even though extraction out of the subject receives low ratings, it remains significantly more acceptable than the ungrammatical controls with a preposition missing (model n$^{\circ}$3). 

These results are expected under the FBC constraint. They are also compatible with processing accounts based on surprisal as well as with other discourse-based accounts, but these accounts cannot explain why we did not find similar results for the relative clauses. The traditional syntactic account is falsified by the significant difference between subextraction from the subject (\ref{ex:exp10-subj-pp}) and ungrammatical controls (\ref{ex:exp10-subj-un}). If extracting out of the subject is ruled out for syntactic reasons, there is no explanation why it should be rated higher than another syntactic violation (preposition missing). Lastly, the results are completely unexpected under an account based on memory costs.
