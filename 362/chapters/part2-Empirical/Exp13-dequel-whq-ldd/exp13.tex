\section[head=Experiment 13]{Experiment 13: Acceptability judgment study on \emph{de quel} \emph{wh}-questions with long-distance dependencies}

We tested \emph{de quel} questions with a long-distance dependency. This allowed us to compare our results on the one hand with our experiment on relative clauses with a long-distance dependency (Experiment~4) and on the other hand with \citegen{Sprouse.2016} experiments on Italian and English.

\subsection{Design and materials}

We used as a starting point the stimuli of Experiment~10, and introduced the same long-distance dependencies as in Experiment~4. As in Experiment~10, we tested extractions out of the subject and out of the object with \emph{est-ce que} questions. The embedding verbs were non-factive (e.g. \emph{supposer} `suppose' or \emph{croire} `believe') to make sure that the content of the embedded clause is not presupposed. 

\eal 
\ex[]{{Condition subject + PP-extracted:}\\
\gll [De quelle innovation]$_i$ est - ce qu' on suppose [que [l' originalité~\trace{}$_i$] enthousiasme mes collègues sans aucune raison]~?\\
of which innovation is {} it that one supposes that the uniqueness excites my colleagues without any reason\\
\glt `Of which innovation do we suppose that the uniqueness excites my colleagues for no reason?'}
\label{ex:exp13-subj-pp}
\ex[]{{Condition object + PP-extracted:}\\
\gll [De quelle innovation]$_i$ est - ce qu' on suppose [que mes collègues admirent [l' originalité~\trace{}$_i$] sans aucune raison]~?\\
of which innovation is {} it that one supposes that my colleagues admire the uniqueness without any reason\\
\glt `Of which innovation do we suppose that my colleagues admire the uniqueness for no reason?'}
\label{ex:exp13-obj-pp}
\zl 

As in Experiment~10, we used polar questions (\ref{ex:exp13-no}) for the non-extraction conditions, also with \emph{est-ce que}. 

\eal \label{ex:exp13-no}
\ex[]{{Condition subject + noextr:}\nopagebreak\\
\gll Est - ce qu' on suppose que l' originalité de cette innovation enthousiasme mes collègues sans aucune raison~?\\
is {} it that one supposes that the uniqueness of this innovation excites my colleagues without any reason\\
\glt `Do we suppose that the uniqueness of this innovation excites my colleagues for no reason?'}
\label{ex:exp13-subj-no}
\ex[]{{Condition object + noextr:}\nopagebreak\\
\gll Est - ce qu' on suppose que mes collègues admirent l' originalité de cette innovation sans aucune raison~?\\
is {} it that we suppose that my colleagues admire the uniqueness of this innovation without any reason\\
\glt `Do we suppose that my colleagues admire the uniqueness of this innovation for no reason?'}
\label{ex:exp13-obj-no}
\zl 

As in Experiment~10, the ungrammatical controls were like the subextraction conditions without the preposition of the extracted element. 

\eal 
\ex[]{{Condition subject + ungramm:}\\
\gll Quelle innovation est - ce qu' on suppose que l' originalité enthousiasme mes collègues sans aucune raison]~?\\
which innovation is {} it that one supposes that the uniqueness excites my colleagues without any reason\\
\glt `Which innovation do we suppose that the uniqueness excites my colleagues for no reason?'}
\label{ex:exp13-subj-un}
\ex[]{{Condition object + ungramm:}\nopagebreak\\
\gll Quelle innovation est - ce qu' on suppose que mes collègues admirent l' originalité sans aucune raison~?\\
which innovation is {} it that one supposes that my colleagues admire the uniqueness without any reason\\
\glt `Which innovation do we suppose that my colleagues admire the uniqueness for no reason?'}
\label{ex:exp13-obj-un}
\zl 

We tested the same 24 items as in Experiments~4 and 10. Each item appeared in the six conditions just described. In addition, the experiment included 36 distractors. The distractors were declaratives, and some of them were ungrammatical. Half of the experimental items and distractors were followed by a comprehension question. The item presented here as an example was paired with the comprehension question \emph{Est-ce que les collègues ont raison d'être enthousiastes~?} (`Are the colleagues right to be enthusiastic?'). 

\subsection{Predictions}

Predictions for long-distance dependencies do not differ essentially from the general predictions for interrogatives summarized in \tabref{tab:exp10-predictions} on page~\pageref{tab:exp10-predictions}. 

Processing accounts expect to see more overall processing costs associated with the extraction itself because of the increased distance between the filler and the gap. 

By contrast, under the discourse-based account based on the FBC constraint, extraction may be facilitated because embedded clauses may be more backgrounded.

\subsection{Procedure} 

We conducted the experiment on the Ibex platform \citep{Ibex}. The procedure was similar to that used in the previous acceptability judgment experiments (see Section \ref{ch:methodo-AJ}). Participants rated the sentences on a Likert scale from 0 to 10, 0 being labeled as ``bad'' and 10 being labeled as ``good''. They also answered comprehension questions after some of the sentences.

The experiment took approximately 20 minutes to complete. 

\subsection{Participants}

The study was run between October and November 2019. 
Participants were recruited through FouleFactory (\url{https://www.foulefactory.com}), and paid 5€ for their participation. The payment was not contingent on the participants' responses to the questions about native language or place of birth.

65 participants took part in the experiment. 
The analysis presented here is based on the data from the 51 participants who satisfied all criteria.\footnote{To calculate accuracy, we excluded not only the answers to comprehension questions of the practice items and of the ungrammatical controls, but also of the ungrammatical distractors.}
They were aged 21 to 78 years. 31 of them self-identified as women, 20 self-identified as men. Two participants (3.92\%) indicated having an educational background related to language.

\subsection{Results and analysis}

\figref{fig:exp13-boxplot} shows the results of the acceptability judgment task.
In the subextraction conditions, the extraction out of the subject (\ref{ex:exp13-subj-pp}) had a mean  rating of 3.07, lower than extraction out of the object (\ref{ex:exp13-obj-pp}) with a mean rating of 3.39. The difference between the two conditions is not as manifest as in Experiment~10 (without the long-distance dependency). The non-extraction conditions were rated higher than the subextraction conditions: the subject control condition (\ref{ex:exp13-subj-no}) had a mean  rating of 6.25, the object control condition (\ref{ex:exp13-obj-no}) of 6.09. The ungrammatical controls received very low ratings: 1.99 in the subject condition (\ref{ex:exp13-subj-un}), and 2.35 in the object condition (\ref{ex:exp13-obj-un}). 

\begin{figure}
    \centering
    \includegraphics[width=\textwidth]{chapters/part2-Empirical/Exp13-dequel-whq-ldd/boxplots.jpeg}
    \caption{Acceptability judgments by condition in Experiment~13. The grey box plots indicate the median and quartiles of the results. Black points are outliers. Mean and confidence intervals are indicated in white.}
    \label{fig:exp13-boxplot}
\end{figure}

\figref{fig:exp13-boxplot} suggests potential floor effects in the ungrammatical controls. \figref{fig:exp13-repartition} confirms this impression: The ungrammatical controls display a floor effect, especially in the subject variant, but the other conditions seem to have a normal distribution.

\begin{figure}
    \centering
    \includegraphics[width=\textwidth]{chapters/part2-Empirical/Exp13-dequel-whq-ldd/repartition.jpeg}
    \caption{Density of the ratings across conditions for Experiment~13}
    \label{fig:exp13-repartition}
\end{figure}

Another representation of the results is given by the ROC and zROC curves of the results in \figref{fig:exp13-ROC} on page \pageref{fig:exp13-ROC}. The ROC curves show that participants discriminated between ungrammatical baselines and the other conditions. As expected based on the data in \figref{fig:exp13-boxplot}, the non-extraction conditions build larger curves than the subextraction conditions. The zROC curves are relatively straight and parallel to the baseline. 

\begin{figure}
    \centering
    \includegraphics[width=\textwidth]{chapters/part2-Empirical/Exp13-dequel-whq-ldd/ROC.jpeg}
    \includegraphics[width=\textwidth]{chapters/part2-Empirical/Exp13-dequel-whq-ldd/zROC.jpeg}
    \caption{ROC curves (top) and zROC curves (bottom) of the non-extraction conditions compared to their respective subextraction conditions, represented by the dotted grey baseline (\citealt{Dillon.2019}'s method) in Experiment~13.}
    \label{fig:exp13-ROC}
\end{figure}

The ROC and zROC curves in \figref{fig:exp13-ROC-subj} on page \pageref{fig:exp13-ROC-subj} show the discrimination between the subject and object conditions. We see that participants barely discriminated between the two. The zROC curves are relatively straight and parallel to the baseline.

\begin{figure}
    \centering
    \includegraphics[width=\textwidth]{chapters/part2-Empirical/Exp13-dequel-whq-ldd/ROC-subject.jpeg}
    \includegraphics[width=\textwidth]{chapters/part2-Empirical/Exp13-dequel-whq-ldd/zROC-subject.jpeg}
    \caption{ROC curves (top) and zROC curves (bottom) of the object conditions compared to their respective subject conditions, represented by the dotted grey baseline (\citealt{Dillon.2019}'s method) in Experiment~13.}
    \label{fig:exp13-ROC-subj}
\end{figure}

\subsubsection{Habituation}\largerpage

The habituation effects in the course of the experiment are given in \figref{fig:exp13-habituation} on page \pageref{fig:exp13-habituation}. The subextraction conditions and the ungrammatical controls show strong habituation. 

Habituation is stronger in extraction out of the object than in extraction out of the subject (but not significantly: see model n$^{\circ}$2 below).

Acceptability of the ungrammatical controls at the end of the experiment was not lower than the subextraction conditions at the beginning of the experiment. Given that the ungrammaticality comes from the lack of a preposition, it can be overlooked relatively easily if participants are not paying enough attention. The habituation could therefore be due to ``noisy channel'' effects, i.e.\ participants' attention dwindling throughout the experiment.

\begin{figure}
    \centering
    \includegraphics[width=\textwidth]{chapters/part2-Empirical/Exp13-dequel-whq-ldd/habituation.jpeg}
    \caption{Changes in the mean acceptability ratings ($z$-scored by participant) by condition in the course of Experiment~13}
    \label{fig:exp13-habituation}
\end{figure}

\subsubsection{Comparing subextraction from the subject with subextraction from the object}

We fitted a first model to compare extractions out of the subject and out of the object on their own (mean centered with subject coded negative and object coded positive). We included trial number as a covariate, and random slopes for all fixed effects and covariates grouped by participants and items. The results of the model are reported in \tabref{tab:exp13-m1}. 
There is a significant effect of habituation, but no main effect of syntactic function: extractions out of the subject do not have significantly lower ratings than extractions out of the object.

% latex table generated in R 3.6.3 by xtable 1.8-4 package
% Fri Apr 24 21:31:19 2020
\begin{table}
\begin{tabular}{l S[table-format=1.3] S[table-format=1.4] c S[table-format=<1.3] S[table-format=1.2]}
  \lsptoprule
 & {Estimate} & {SE} & {$z$} & {$\text{Pr}(>|z|)$} & {OR} \\ 
  \midrule
  syntactic function & 0.459 & 0.144 & 3 & <.005 & 1.58 \\ 
  trial              & 0.016 & 0.010 & 2 & 0.1025 & 1.02 \\ 
   \lspbottomrule
\end{tabular}
\caption{Results of the Cumulative Link Mixed Model (model n$^{\circ}$2)}
\label{tab:exp12-m2}
\end{table}


A second model compared extractions out of the subject and out of the object on their own, crossing syntactic function with trial number. We included participants and items as random factors. The results of the model are reported in \tabref{tab:exp13-m2}. 
The results for the main effects are similar to model n$^{\circ}$1, but we can see that the interaction is not significant. The stronger habituation effect on extractions out of the object that we can see in \figref{fig:exp13-habituation} is not significant, either. 

% latex table generated in R 3.6.3 by xtable 1.8-4 package
% Sat Apr 25 22:24:53 2020
\begin{table}
\begin{tabular}{l S[table-format=-1.3] S[table-format=1.3] S[table-format=-1.0] S[table-format=<1.4] S[table-format=1.2]}
  \lsptoprule
 & {Estimate} & {SE} & {$z$} & {$\text{Pr}(>|z|)$} & {OR} \\ 
  \midrule
  syntactic function       & -0.165 & 0.240 & -1 & 0.4918 & 1.18 \\ 
  trial                    & 0.022 & 0.006 & 3 & <.001 & 1.02 \\ 
  syntactic function:trial & 0.008 & 0.007 & 1 & 0.2153 & 1.01 \\ 
   \lspbottomrule
\end{tabular}
\caption{Results of the Cumulative Link Mixed Model (model n$^{\circ}$2)}
\label{tab:exp13-m2}
\end{table}


In a third model, we compared subextraction with non-extraction. We fitted a model crossing syntactic function and extraction type (mean centered with extraction coded positive, non-extraction coded negative). We included trial number as a covariate, and random slopes for all fixed effects and covariates grouped by participants and items. The results of the model are reported in \tabref{tab:exp13-m3}. 
There is a significant main effect of extraction type (non-extractions are rated higher). There is no significant main effect of syntactic function, no significant main effect of habituation, and no significant interaction effect. The results are the same if we compare the the AUCs (green and red curves on \figref{fig:exp13-ROC-subj}): the difference is not significant. 
\figref{fig:exp13-interaction} illustrates this: we see a slight tendency toward an interaction effect, but it is very small.

% latex table generated in R 3.6.3 by xtable 1.8-4 package
% Mon Apr 13 17:44:16 2020
\begin{table}
\begin{tabular}{l S[table-format=-1.3] S[table-format=1.3] S[table-format=1] S[table-format=1.4] S[table-format=1.2]}
  \lsptoprule
 & {Estimate} & {SE} & {$z$} & {$\text{Pr}(>|z|)$} & {Odd.ratio} \\ 
  \midrule
  syntactic function & -0.070 & 0.324 & -0 & 0.8286 & 1.07 \\ 
  trial              & 0.024 & 0.018 & 1 & 0.1717 & 1.02 \\ 
   \lspbottomrule
\end{tabular}
\caption{Results of the Cumulative Link Mixed Model (model n$^{\circ}$1)}
\label{tab:exp06-m1}
\end{table}


\begin{figure}
    \centering
    \includegraphics[width=\textwidth]{chapters/part2-Empirical/Exp13-dequel-whq-ldd/interaction.jpeg}
    \caption{Interaction between syntactic function and extraction type in Experiment~13}
    \label{fig:exp13-interaction}
\end{figure}

\subsubsection{Comparing subextraction from the subject with ungrammatical controls}

We fitted a fourth model to compare the extractions out of the subject and the subject ungrammatical controls on their own (mean centered with subextraction coded positive and ungrammatical coded negative). We included trial number as a covariate, and random slopes for all fixed effects grouped by participants and items. The results of the model are reported in \tabref{tab:exp13-m4}. There is a significant effect of extraction type: ratings for extraction out of the subject are significantly higher than for its ungrammatical control. There is also a significant effect of trial (habituation).

% latex table generated in R 3.6.3 by xtable 1.8-4 package
% Thu Apr 23 00:04:53 2020
\begin{table}
\begin{tabular}{l S[table-format=1.3] S[table-format=1.3] c S[table-format=<1.3] S[table-format=2.2]}
  \lsptoprule
 & {Estimate} & {SE} & {$z$} & {$\text{Pr}(>|z|)$} & {OR}\\ 
  \midrule
  extraction type & 2.342 & 0.395 & 6 & <.001 & 10.40 \\ 
  trial           & 0.039 & 0.012 & 3 & <.005 & 1.04 \\ 
   \lspbottomrule
\end{tabular}
\caption{Results of the Cumulative Link Mixed Model (model n$^{\circ}$3)}
\label{tab:exp10-m3}
\end{table}


In a fifth model, we compared subextraction with the ungrammatical controls. We crossed syntactic function (mean centered with object coded positive, subject coded negative) and extraction type (grammaticality). Trial number as also included as an explanatory variable, as well as random slopes for syntactic function and extraction type grouped by participants and items. The results of the model are reported in \tabref{tab:exp13-m5}. 
There is a significant main effect of extraction type (in favor of the extraction conditions) and of trial (habituation). There is, however, no significant main effect of syntactic function and no significant interaction effect.

% latex table generated in R 3.6.3 by xtable 1.8-4 package
% Sat Apr 25 19:15:30 2020
\begin{table}
\begin{tabular}{l S[table-format=1.3] S[table-format=1.3] c S[table-format=<1.4] S[table-format=1.2]}
  \lsptoprule
 & {Estimate} & {SE} & {$z$} & {$\text{Pr}(>|z|)$} & {OR} \\ 
  \midrule
  syntactic function & 0.134 & 0.091 & 1 & 0.1405 & 1.14 \\ 
  extraction type & 0.631 & 0.133 & 5 & <.001 & 1.88 \\ 
  trial & 0.025 & 0.005 & 5 & <.001 & 1.03 \\ 
  syntactic function:extraction type & 0.009 & 0.088 & 0 & 0.9142 & 1.01 \\ 
   \lspbottomrule
\end{tabular}
\caption{Results of the Cumulative Link Mixed Model (model n$^{\circ}$5)}
\label{tab:exp13-m5}
\end{table}


\subsection{Discussion}

In this experiment on long-distance dependencies, we see no typical ``island'' effect when extracting out of the subject. The mean acceptability rating for extraction out of the subject was slightly under the mean acceptability rating for objects, but the difference is not significant (model n$^{\circ}$1) and there is no interaction when comparing with non-extraction controls (model n$^{\circ}$3). The results are more similar to the results on long-distance dependencies in relative clauses (Experiment~4) than to the results for the other interrogatives with an extraction (Experiment~10 and 12). 

As such, the results do not falsify any account. Only the finding that the extraction out of the subject received significantly higher ratings than its ungrammatical control goes against the predictions of the syntactic account. 

The ratings in the non-extraction conditions are relatively low compared to the other experiments. This can of course be an accident, be related to the items, or to the choice of distractors, but it may well be due to the embedded structure itself. However, we did not see a similar effect in the relative clauses with long-distance dependencies.
% It may well be that extractions costs for the extraction are more important in the interrogatives than in the relative clauses. Focalization as a processing cost?

