\section[head=Experiment 1]{Experiment 1: Acceptability judgment study on \emph{dont} relative clauses with different linear distances}
\label{ch:exp01}
\largerpage[2.25]

The goal of this first experiment was to test the opposite predictions of two approaches. We pitted the traditional syntactic approach against a processing approach based on memory costs. We compared extraction out of the subject with extraction out of the object: the traditional syntactic approach predicts a subject island, whereas approaches based on memory load predict a subject advantage.\footnote{We replicated Experiment~1 with a slightly different 3*2 design in \citet{Abeille.2020.Cognition}. The results of the replication study do not differ in any major way from those I present in this section.}

\subsection{Design and materials}

The experiment used an acceptability judgment task with a $3\times 2$ design. 
We compared extractions out of the subject (\ref{ex:exp1-subj-pp}) and extractions out of the object (\ref{ex:exp1-obj-pp}). In this experiment, we manipulated not only different gap sites, but also the distance between the relative word and the gap, which has an impact on the memory load. For this reason, I call the subject condition the ``narrow-distance'' condition and the object condition the ``wide-distance'' condition.

\eal 
\ex[]{{Condition narrow-distance + PP-extracted:}\nopagebreak \\
\gll Ils présentent une innovation dont$_i$ [l' originalité~\trace{}$_i$]
émerveille mes collègues.\\
they present an innovation of.which the uniqueness delights my colleagues\\
\glt `They present an innovation of which the uniqueness delights my colleagues.'}
\label{ex:exp1-subj-pp}
\ex[]{{Condition wide-distance + PP-extracted:}\\
\gll Ils présentent une innovation dont$_i$ mes collègues apprécient [l' originalité~\trace{}$_i$].\\
they present an innovation of.which my colleagues value the uniqueness\\
\glt `They present an innovation of which my colleagues value the uniqueness.'}
\label{ex:exp1-obj-pp}
\zl 

The relation between \emph{dont} and the gap always expressed a quality (e.g.\ \emph{originalité} `uniqueness', \emph{beauté} `beauty'). The noun out of which extraction takes place was always inanimate. We used subject/object experiencer psych verbs in pairs with similar semantics (e.g., \emph{apprécier} `value' and \emph{émerveiller} `delight') in order to have transitive verbs but also in order to compare sentences whose content is as similar as possible. In all extraction conditions, the extraction took place out of the stimulus argument of the verb. Using transitive verbs was crucial, because some syntactic accounts only expect a subject island for subjects of transitive verbs. 
% the worst results in Polinsky et al. 
It was also important to keep the content maximally similar between the subject and object conditions. We saw in Chapter~\ref{ch:discourse-relevance} that the relevance of the extracted element for the main proposition is central for the acceptability of filler-gap dependencies. By keeping the content the same, we ensure that the relevance for the main proposition remains constant.

In the narrow-distance condition (extraction out of the subject) in (\ref{ex:exp1-subj-pp}), one new referent is introduced into the discourse (\emph{originalité}). In the wide-distance condition (extraction out of the object) in (\ref{ex:exp1-obj-pp}), the number of refernts is three (\emph{collègues}, \emph{apprécier} and \emph{originalité}). We added a third intermediate category, with two referents between \emph{dont} and the gap, by using a clitic subject instead of a nominal one in the extraction out of the object, as in (\ref{ex:exp1-obj-clitic-pp}). 

\ea[]{{Condition medium-distance + PP-extracted:}\\
\gll Ils présentent une innovation dont$_i$ nous apprécions [l' originalité~\trace{}$_i$].\\
they present an innovation of.which we value the uniqueness\\
\glt `They present an innovation of which we value the uniqueness.'}
\label{ex:exp1-obj-clitic-pp}
\z 

To create a grammatical baseline, the three distance conditions were used in a coordination construction which contains no extraction. The material in (\ref{ex:exp1-subj-no}), (\ref{ex:exp1-obj-clitic-no}) and (\ref{ex:exp1-obj-no}) are thus the respective controls for (\ref{ex:exp1-subj-pp}), (\ref{ex:exp1-obj-clitic-pp}) and (\ref{ex:exp1-obj-pp}).

\eal 
\ex[]{{Condition narrow-distance + noextr:}\\
\gll Ils présentent une innovation et son originalité émerveille mes collègues.\\
they present an innovation and its uniqueness delights my colleagues\\
\glt `They present an innovation and its uniqueness delights my colleagues.'}
\label{ex:exp1-subj-no}
\ex[]{{Condition medium-distance + noextr:}\\
\gll Ils présentent une innovation et nous apprécions son originalité.\\
they present an innovation and we value its uniqueness\\
\glt `They present an innovation and we value its uniqueness.'}
\label{ex:exp1-obj-clitic-no}
\ex[]{{Condition wide-distance + noextr:}\nopagebreak\\
\gll Ils présentent une innovation et mes collègues apprécient son originalité.\\
they present an innovation and my colleagues value its uniqueness\\
\glt `They present an innovation and my colleagues value its uniqueness.'}
\label{ex:exp1-obj-no}
\zl 

We tested 30 items, each manipulated according to the six conditions described above. In addition, the experiment included 36 distractors. 

\subsection{Predictions}

The aim of this experiment was to test the predictions of the traditional syntactic account and of processing accounts based on memory costs.

The traditional syntactic account predicts a subject superadditivity effect when extracting out of the subject. First of all (\ref{ex:exp1-subj-pp}) should be degraded compared to (\ref{ex:exp1-obj-pp}). The expected interaction effect is presumably that (\ref{ex:exp1-subj-pp}) is worse than all three conditions (\ref{ex:exp1-obj-pp}), (\ref{ex:exp1-subj-no}) and (\ref{ex:exp1-obj-no}). At the same time, under syntactic accounts there should be no difference between the two conditions of extraction out of an object. Therefore they also predict that (\ref{ex:exp1-subj-pp}) should be degraded compared to (\ref{ex:exp1-obj-clitic-pp}), and that there should be an interaction in that (\ref{ex:exp1-obj-pp}) is worse than all three conditions (\ref{ex:exp1-obj-clitic-pp}), (\ref{ex:exp1-subj-no}) and (\ref{ex:exp1-obj-clitic-no}).

By contrast, processing accounts based on memory load predict that extraction out of the subject should be better than extraction out of the object: there is only one intervening new referent in discourse between \emph{dont} and the gap in (\ref{ex:exp1-subj-pp}), but two referents for (\ref{ex:exp1-obj-clitic-pp}) and three referents for (\ref{ex:exp1-obj-pp}). Therefore, these processing approaches expect (\ref{ex:exp1-subj-pp}) to be more acceptable than (\ref{ex:exp1-obj-clitic-pp}) and (\ref{ex:exp1-obj-pp}), and (\ref{ex:exp1-obj-clitic-pp}) to be more acceptable than (\ref{ex:exp1-obj-pp}). The extraction baseline is also expected to be less costly than the extraction conditions, and thus a main effect of extraction type is predicted with higher acceptability for non-extraction than for extraction. 

\subsection{Procedure}

\label{ch:methodo-AJ}
Thirteen of the experiments presented in this book used the acceptability judgment task with non-binary responses on a Likert scale. Acceptability judgment tasks are a widespread formal method for linguistic experiments\footnote{\citet{Tonhauser.Matthewson.Ms} surveyed 40 papers on meaning and report that 3/4 of them use acceptability judgment tasks.}: they are technically easy to set up, are conducted on online platforms by untrained participants, often on a volunteer basis which makes them very cheap to run. They also seem highly robust, results are reproducible, and in line with other experimental paradigms: \citet{Pechmann.1994} carried out a comparison with offline acceptability judgment tasks and conclude that acceptability judgment tasks provide reliable results \citep[see also][]{Keller.2001,Sorace.2005,Gibson.2013}.
Provided that there is a careful experimental design, it is a reliable method to compare conditions \citep{Schutze.2016.ex1996,Tonhauser.Matthewson.Ms}.

The acceptability judgment tasks were set up on the online platform Ibex \citep{Ibex}. For every experiment, the homepage consisted of a short welcome text with an explanation of the task and of the duration of the experiment. Participants were told that they were free to close their browser and thus delete their answers at any time, and that the data collected do not enable researchers to identify them. They had to give their consent before proceeding to the experiment. The experimental items (test items and distractors in at least a 1:1 ratio) were presented one sentence at a time on the screen. Each session began with 3 practice items which were explicitly identified as such and gave the participants the opportunity to get used to the judgment task. Participants had to judge each sentence on a Likert-scale from 1 to 10, 1 being labeled as ``bad'' and 10 being labeled as ``good''.\footnote{The main difference between a 0 to 10 Likert scale and a 1 to 10 Likert scale is that the former provides a value in the middle of the scale, whereas the latter does not. On the contrary, a 0 to 10 Likert scale forces the participants to choose between more than average and less than average. I used both scales, depending on the experiment, and never noticed any difference in the results.} The description explicitly mentioned that there was no right or wrong answer. Typically, 7-point (sometimes 5-point) Likert scales are used in psychology and psycholinguistics for this kind of experiments. However, the Laboratoire de Linguistique Formelle of the Université Paris Cité traditionally uses a 10-point Likert scale for experiments with French native speakers. This is because the French school system (including preschool and university) makes extensive use of a 10-point (alternatively 20-point) grade system, thus French participants are very familiar with this scale, more so than with a 7-point scale. After the rating, some experiments included a comprehension question to stimulate the participants' attention.
The order of presentation of the items can have an impact on the participants' judgments (e.g.\ because of fatigue, see \citealt[180--181]{Schutze.2016.ex1996}). To balance this out, the sentences were pseudo-randomized for each participant. Pseudorandomization ensured that participants would not see two consecutive sentences in the same condition. I used a Latin square design so participants saw each test item only once and in only one condition, but they saw each condition equally often. 

This experiment took approximately 20 minutes to complete. Participants were recruited on the R.I.S.C. website (\url{http://experiences.risc.cnrs.fr/}) and on social media (e.g.\ Facebook); they did not receive any financial compensation. 

\subsection{Participants}

The study was conducted in January and February 2016. 55 participants took part in the experiment. We present here the analysis of the data from the 44 participants who satisfied all inclusion criteria. Exclusion critera were the same for all experiments I present in this book; they are described in Appendix~\ref{ch:participants-criteria}.
28 participants self-identified as women, 16 self-identified as men; their age range from 19 to 73 years. Seven participants (15.91\%) indicated having an educational background related to language.

\subsection{Results and analysis}

\figref{fig:exp01-boxplot} displays the results of the acceptability judgment task. In the subextraction conditions, acceptability ratings were highest in the narrow-distance condition (mean rating: 8.41), followed by the medium-distance condition (mean: 8.30) and finally, the wide-distance condition (mean rating: 7.88). The average acceptability for the control conditions without extraction was lower: acceptability ratings were highest in the narrow-distance condition (mean rating: 7.23), followed by the medium-distance condition (mean: 7.08) and the wide-distance condition (mean rating: 7.00).

\begin{figure}
    \centering
    \includegraphics[width=\textwidth]{chapters/part2-Empirical/Exp01-dont-RC-AJ/boxplots.jpeg}
    \caption{Acceptability judgments for each condition in Experiment~1. The grey box plots indicate the median and quartiles of the results. Black points are outliers. Mean and confidence intervals are indicated in white.}
    \label{fig:exp01-boxplot}
\end{figure}

\figref{fig:exp01-boxplot} suggests a ceiling effect in the extraction conditions. \figref{fig:exp01-repartition} shows a normal distribution of the ratings with a strong ceiling effect for the extraction conditions (on the right). There also appears to be a ceiling effect, albeit smaller, in the non-extraction conditions (on the left). 

\begin{figure}
    \centering
    \includegraphics[width=\textwidth]{chapters/part2-Empirical/Exp01-dont-RC-AJ/repartition.jpeg}
    \caption{Density of the ratings across conditions for Experiment~1}
    \label{fig:exp01-repartition}
\end{figure}

Another representation of the results is given by the ROC (Receiver Operating Characteristic) and zROC curves in \figref{fig:exp01-ROC}.\footnote{See the methodology for Receiver Operating Characteristic curves below.} The ROC curve shows that the participants discriminated between the \emph{dont}-relative clauses (extraction) and the coordinations (non-extraction). We can also see that the narrow-distance and medium-distance conditions are rather similar, while the wide-distance condition receives slightly lower judgments. However, the difference is not large. The zROC curves for the wide-distance and narrow-distance conditions are straight lines, which, following \citet[21--22]{Dillon.2019}, constitutes a visual cue that the underlying acceptability distribution is normally distributed. The line of the medium-distance condition is slightly convex, which can be a visual cue of bimodality. Bimodality could be due to a strong change of acceptability during the experiment (decreasing ``habituation'' effect) that I present below.

\begin{figure}
    \centering
    \includegraphics[width=\textwidth]{chapters/part2-Empirical/Exp01-dont-RC-AJ/ROC.jpeg}
    \includegraphics[width=\textwidth]{chapters/part2-Empirical/Exp01-dont-RC-AJ/zROC.jpeg}
    \caption{ROC curves (top) and zROC curves (bottom) of the extraction conditions compared to their respective non-extraction condition, represented by the dotted grey baseline (\citeauthor{Dillon.2019}'s  method) in Experiment 1. For the zROC curve, I had to exclude level 2 of the scale from the graph. The proportion of hits for this level in the PP-extracted + narrow-distance condition was 100\%, which leads to a $z$-score of +$\infty$.}
    \label{fig:exp01-ROC}
\end{figure}

\subsubsection{Cumulative Link Mixed Models}\label{ch:cumulative-link-model}
In acceptability judgment tasks, participants are asked to rate sentences on a Likert scale. 
%The general assumption we make when working with acceptability judgments is that certain factors (e.g.\ morphosyntactic factors, pragmatic factors) are good predictors for the rating that the participant will give. Depending on the hypothesis, a series of items in condition A with factor $\alpha$ is expected by the hypothesis to lead to a higher acceptability than the same items in condition B with factor $\beta$. For island, we expect the judgments to reflect a superadditive effect, such that the island condition is rated significantly lower than the other control conditions (usually in a 2*2 design, see above). 
Until recently, ANalysis Of VAriance (ANOVA) or linear models were typically used to identify interaction effects in acceptability judgements.\footnote{This includes my earlier work \citep{Abeille.2016,Abeille.2020.Cognition,Abeille.2020.JFLS}.} But these models assume continuous numeric variables, thus they are problematic for ordinals like acceptability judgments, as \citet{Dillon.2019} illustrate with some simulations. Standardized ratings ($z$-scores) are somewhat more appropriate in this respect, but they are not ideal, because we cannot be certain that the choice e.g.\ between a rating of 2 vs.\ 3 in a participant's judgment represents the same ``size'' difference as the choice between 5 and 6.

\begin{sloppypar}
Cumulative Link Mixed Models, on the other hand, avoid these problems and are well-suited for analyzing ordinal variables. 
They are now easily available on R. The analyses in this book were carried out using the function \texttt{clmm()} from the \texttt{ordinal} package \citep{clmm}.
\end{sloppypar}
\largerpage

Following the ``best practices'' recommendations by \citet[275--277]{Barr.2013} for linear mixed models, I included random slopes for all fixed effects grouped by participants and items in my Cumulative Link Mixed Models (because including only random intercepts can be anti-conservative, at least in linear models) and fitted a maximal model whenever convergence was achievable. If the model did not converge, I would step back to non-maximal models. This procedure has flaws and using a Bayesian inference method would be more adequate to avoid false positive results, but I leave this for future research.

There is, to my knowledge, no method of residual diagnostics for Cumulative Link Mixed Models.
% No residual diagnostic: https://stat.ethz.ch/pipermail/r-sig-mixed-models/2013q4/021281.html

\subsubsection{Signal Detection Theory \& Receiver Operating Characteristic curve} \label{ch:ROC}
\citet{Dillon.2019} proposed an alternative way to analyse the results of acceptability judgments by using tools of Signal Detection Theory. Following the methodology in that paper, I built Receiver Operating Characteristic (ROC) curves for the different conditions. The ROC curve is constructed by comparing the cumulative probability of hits at every point on the scale for one condition with the respective cumulative hits for another condition: the probability that participants will click on 0 for point 0, the probability for participants to click on 0 or 1 for point 1, the probability for participants to click on 0, 1 or 2 for point 2, and so on. The ``better'' condition is then plotted with respect to the ``worse'' one, which serves as a baseline. The Area Under the Curve (AUC) provides a graphical representation of how well the participants were able to discriminate between the two conditions: the larger the AUC, and especially the further away it is from the baseline (AUC = 0.5), the more the participants discriminated between the two conditions. 

I used the function \texttt{roc()} from the \texttt{pROC} package \citep{pROC} to calculate the AUCs. For the graphical representation, I carried out the calculations and plotted the curves with \texttt{ggplot2} \citep{Wickham.2016}, because the \texttt{pROC} package does not provide a way to plot multiple ROC curves.

I also constructed the respective zROC curves, following the same methodology, except that the probabilities of hits were transformed into $z$-scores. \citet{Dillon.2019} explain how the visual inspection of the zROC curve allows first conclusions about the distribution of the underlying ratings and helps identify bimodality in the data.  

The AUC is a measure of the difference between two conditions. By comparing two AUCs in a 2x2 design, we can see whether there is a difference between two differences, which is basically the definition of an interaction effect. I used the function \texttt{roc.test()} from the \texttt{pROC} package \citep{pROC} for the analysis. Even though I examined one-sided hypotheses (i.e., hypotheses that expect one specific AUC to be significantly larger than the other AUC), I used two-sided tests for the difference in AUCs, to reduce the number of statistical tests, as different hypotheses with opposite predictions were tested simultaneously. Notice that I did not correct the confidence level for multiple comparisons. 

A significant difference between two AUCs is an indicator of an interaction effect, and can corroborate the results of the Cumulative Link Mixed Models. In my experience, the methodology from  Signal Detection Theory proposed by \citet{Dillon.2019} is more conservative than Cumulative Link Mixed Models. It is less likely to refute the null hypothesis with a significant probability than CLMM.


\subsubsection{Habituation} 

\figref{fig:exp01-habituation} depicts the habituation effects in the course of the experiment. In general, the acceptability remains unchanged during the experiment, except for the two extractions out of the object whose acceptability declines. Especially extraction out of the object with a clitic subject (medium-distance) shows a strong decrease.

\begin{figure}
    \centering
    \includegraphics[width=\textwidth]{chapters/part2-Empirical/Exp01-dont-RC-AJ/habituation.jpeg}
    \caption{Changes in the average acceptability ratings ($z$-scored by participant) for each condition of Experiment 1 in the course of the experiment}
    \label{fig:exp01-habituation}
\end{figure}

\subsubsection{Comparing the narrow-distance condition with the wide-distance condition}\largerpage[2]

The first model was fitted to compare extraction out of the subject and out of the object on its own (mean centered with subject coded negative and object coded positive). We included trial number as a covariate, and random slopes for fixed effects and covariates grouped by participants and items. The results of the model are reported in \tabref{tab:exp1-m1}. 
There is a significant effect of the distance (or grammatical function), but not of trial (habituation). This is expected under the processing account, and displays the opposite pattern than the one predicted by the syntactic account (the value is expected to be positive, but it is negative in the results). 

\begin{table}
\centering
\begin{tabular}{l S[table-format=-1.3] S[table-format=1.3] c S[table-format=<1.4] S[table-format=1.2] }
\lsptoprule
 & {Estimate} & {SE} & {$z$} & {$\text{Pr}(>|z|)$} & {Odd. ratio} \\ 
\midrule
  distance & -0.341 & 0.064 & $-5$ & <0.001 & 1.41 \\ 
  Trial    & -0.001 & 0.004 & $-0$ & 0.7113 & 1.00 \\ 
 \lspbottomrule
\end{tabular}
\caption{Results of the Cumulative Link Mixed Model (model $n^{\circ}1$)}\label{tab:exp1-m1}
\end{table}
% % % % % latex table generated in R 3.6.3 by xtable 1.8-4 package
% Wed May 20 14:00:46 2020
\begin{table}
\begin{tabular}{l S[table-format=1.3] S[table-format=1.3] S[table-format=1] S[table-format=<1.4] S[table-format=1.2]}
  \lsptoprule
 & {Estimate} & {SE} & {$z$} & {$\text{Pr}(>|z|)$} & {Odd.ratio} \\ 
  \midrule
(Intercept) & 1.127 & 0.239 & 5 & <.001 & 3.09 \\ 
  syntactic function & 0.044 & 0.091 & 0 & 0.6283 & 1.04 \\ 
  trial & 0.012 & 0.005 & 2 & <.05 & 1.01 \\ 
   \lspbottomrule
\end{tabular}
\caption{Results of the Logistic regression model (model n$^{\circ}$2)}
\label{tab:exp02-m2}
\end{table}


We fitted a second model crossing distance and extraction type (mean centered with extraction coded positive, non-extraction coded negative). We included trial number as covariate, and random slopes for all fixed effects and covariates grouped by participants and items. The results of the model are shown in \tabref{tab:exp1-m2}. 
There is a significant main effect of the distance (or grammatical function) and of extraction type, but no main effect of trial (habituation). There is also a significant interaction: the difference between the extraction and non-extraction conditions is larger in the narrow-distance conditions than in the wide-distance conditions. The interaction is showed in \figref{fig:exp01-interaction1}. 

% latex table generated in R 3.6.3 by xtable 1.8-4 package
% Sun Apr 26 23:02:02 2020
\begin{table}
\begin{tabular}{l S[table-format=1.3] S[table-format=1.3] c S[table-format=<1.3] S[table-format=1.2]}
  \lsptoprule
                     & {Estimate} & {SE} & {$z$} & {$\text{Pr}(>|z|)$} & {OR} \\ 
  \midrule
  syntactic function & 1.328 & 0.488 & 3 & <.01 & 3.77 \\ 
  trial              & 0.069 & 0.023 & 3 & <.005 & 1.07 \\ 
 \lspbottomrule
\end{tabular}
\caption{Results of the Cumulative Link Mixed Model (model n$^{\circ}$1)}
\label{tab:exp16-m1}
\end{table}


\begin{figure}
    \centering
    \includegraphics[width=\textwidth]{chapters/part2-Empirical/Exp01-dont-RC-AJ/interaction1.jpeg}
    \caption{Interaction between distance and extraction type in Experiment 1. The graph only shows the narrow-distance and wide-distance conditions.}
    \label{fig:exp01-interaction1}
\end{figure}

However, if we compare the Area Under the Curve (AUC) for the ROC curves of the two distance conditions (see \figref{fig:exp01-ROC} on page \pageref{fig:exp01-ROC}), the difference is not significant.

\subsubsection{Comparing the narrow-distance condition with the medium-distance condition}

A third model was fitted to compare extraction out of the subject and out of the object on its own (mean centered with subject coded negative and object coded positive). We included trial number as a covariate, and random slopes for fixed effects and covariates grouped by participants and items. The results of the model are given in \tabref{tab:exp01b-m3}. The difference is not significant.

% latex table generated in R 3.6.3 by xtable 1.8-4 package
% Thu Apr 23 00:04:53 2020
\begin{table}
\begin{tabular}{l S[table-format=1.3] S[table-format=1.3] c S[table-format=<1.3] S[table-format=2.2]}
  \lsptoprule
 & {Estimate} & {SE} & {$z$} & {$\text{Pr}(>|z|)$} & {OR}\\ 
  \midrule
  extraction type & 2.342 & 0.395 & 6 & <.001 & 10.40 \\ 
  trial           & 0.039 & 0.012 & 3 & <.005 & 1.04 \\ 
   \lspbottomrule
\end{tabular}
\caption{Results of the Cumulative Link Mixed Model (model n$^{\circ}$3)}
\label{tab:exp10-m3}
\end{table}


A fourth model crossed distance and extraction type (mean centered with extraction coded positive, non-extraction coded negative). We included trial number as a covariate, and random slopes for all fixed effects and covariates grouped by participants and items. The results of the model are reported in \tabref{tab:exp1-m4}. 
There is a main effect of extraction type, but no main effect of distance or trial and no interaction effect. The interaction is illustrated by \figref{fig:exp01-interaction2}. 

% latex table generated in R 3.6.3 by xtable 1.8-4 package
% Sun Apr 26 23:02:15 2020
\begin{table}
\begin{tabular}{l S[table-format=1.3] S[table-format=1.3] c S[table-format=<1.3] S[table-format=1.2]}
  \lsptoprule
 & {Est.} & {SE} & {$z$} & {$\text{Pr}(>|z|)$} & {OR} \\ 
  \midrule
  syntactic function & 0.474 & 0.465 & 1 & 0.307 & 1.61 \\ 
  trial & 0.034 & 0.014 & 2 & <.05 & 1.03 \\ 
  syntactic function:trial & 0.014 & 0.015 & 1 & 0.342 & 1.01 \\ 
   \lspbottomrule
\end{tabular}
\caption{Results of the Cumulative Link Mixed Model (model n$^{\circ}$2)}
\label{tab:exp16-m2}
\end{table}


\begin{figure}
    \centering
    \includegraphics[width=\textwidth]{chapters/part2-Empirical/Exp01-dont-RC-AJ/interaction2.jpeg}
    \caption{Interaction between distance and extraction type in Experiment 1. The graph only shows the narrow-distance and medium-distance conditions.}
    \label{fig:exp01-interaction2}
\end{figure}

The comparison of the Area Under the Curve (AUC) for the ROC curves of the two distance conditions also yields a non-significant difference (see \figref{fig:exp01-ROC} on page \pageref{fig:exp01-ROC}).

\subsubsection{Comparing the medium-distance condition with the wide-distance condition}

A fifth model was fitted to compare extraction out of the object with a clitic subject vs.\ a nominal subject on its own (mean centered with clitic coded negative and nominal coded positive). We again included trial number as a covariate, and random slopes for fixed effects and covariates grouped by participants and items. The results of the model are reported in \tabref{tab:exp1-m5}. 
The difference is not significant.

% latex table generated in R 3.6.3 by xtable 1.8-4 package
% Sun Jul 19 15:34:20 2020
\begin{table}
\begin{tabularx}{\textwidth}{Q S[table-format=-1.3] 
                  S[table-format=1.3] 
                  S[table-format=3.2] 
                  S[table-format=-1] 
                  S[table-format=<1.4] 
                  S[table-format=3.2]}
  \lsptoprule
 & {Estimate} & {SE} & {df} & {$t$} & {$\text{Pr}(>|t|)$} & {OR} \\ 
  \midrule
(Intercept) & 5.468 & 0.086 & 403.36 & 64 & <.001 & 236.91 \\ 
  extraction type & 0.012 & 0.017 & 515.96 & 1 & 0.4713 & 1.01 \\ 
  distance & -0.014 & 0.018 & 514.96 & -1 & 0.4151 & 1.01 \\ 
  length & -0.006 & 0.009 & 520.02 & -1 & 0.5205 & 1.01 \\ 
  frequency & -0.000 & 0.000 & 522.58 & -1 & 0.1397 & 1.00 \\ 
  extr.\ type:distance & 0.018 & 0.017 & 515.64 & 1 & 0.2937 & 1.02 \\ 
   \lspbottomrule
\end{tabularx}
\caption{Results of the Linear Mixed Model (model n$^{\circ}$7)}
\label{tab:exp03-m7}
\end{table}


The last model crossed distance and extraction type (mean centered with extraction coded positive, non-extraction coded negative). As before, we included trial number as a covariate, and random slopes for all fixed effects and covariates grouped by participants and items. The results of the model appear in Table~\ref{tab:exp1-m6}. 
There is a main effect of extraction type and a main effect of distance (higher ratings in the medium-distance than in the wide-distance condition). There is however no effect of trial (habituation) and no significant interaction. The interaction is illustrated by \figref{fig:exp01-interaction3}. 

% latex table generated in R 3.4.4 by xtable 1.8-4 package
% Sat Mar 28 14:47:01 2020
\begin{table}
\begin{tabular}{l S[table-format=1.3] S[table-format=1.3] S[table-format=1] S[table-format=<1.4] S[table-format=2.2]}
  \lsptoprule
 & {Estimate} & {SE} & {$z$} & {$\text{Pr}(>|z|)$} & {Odd.ratio} \\ 
  \midrule
  extraction type          & 0.629 & 0.124 & 5 & <.001  & 1.88 \\ 
  distance                 & 0.213 & 0.076 & 3 & <.01   & 1.24 \\ 
  trial                    & 0.001 & 0.005 & 0 & 0.8325 & 1.00 \\ 
  extraction type:distance & 0.173 & 0.090 & 2 & 0.0553 & 1.19 \\ 
   \lspbottomrule
\end{tabular}
\caption{Results of the Cumulative Link Mixed Model (model n$^{\circ}$6)}
\label{tab:exp1-m6}
\end{table}


\begin{figure}
    \centering
    \includegraphics[width=\textwidth]{chapters/part2-Empirical/Exp01-dont-RC-AJ/interaction3.jpeg}
    \caption{Interaction between distance and extraction type in Experiment 1. The graph only represents the medium-distance and wide-distance conditions.}
    \label{fig:exp01-interaction3}
\end{figure}

If we compare the Area Under the Curve (AUC) for the ROC curves of the two distance conditions (see \figref{fig:exp01-ROC} on page \pageref{fig:exp01-ROC}), the difference is not significant, either.

\subsection{Discussion}

We start by comparing the results of the analysis with the predictions of the traditional syntactic approach: The expected degradation in subextraction from subject (\ref{ex:exp1-subj-pp}) compared to subextraction from object conditions (\ref{ex:exp1-obj-pp}, \ref{ex:exp1-obj-clitic-pp}) was not manifested in the data: the difference between subextraction from a subject and from an object with a nominal subject is significant, but with subextraction from the object being worse (model n$^{\circ}$1). We detected an interaction effect (model n$^{\circ}$2) with a rather small effect size (odds ratio of 1.19), but, again, the pattern goes against the predictions of syntactic accounts: The interaction is caused by extraction out of the object being worse, not better, than extraction of out the subject.

Notice that our results are also problematic for other kinds of accounts that expect a penalty for extracting out of the subject. With 30 items for six conditions, participants only saw the extraction out of subject condition five times during the experiment: according to \citet{Chaves.2019.Frequency}, this is not sufficient to obtain improvement of ratings due to habituation. 

We can now compare the results to the predictions of processing approaches based on memory costs. Here, the acceptability means are at least pointing in the right direction: the mean ratings for the narrow-distance extraction were higher than for the medium-distance extraction, which, in turn, were higher than for the wide-distance extraction. However, this difference was only significant for the two extremes, i.e.\ between narrow-distance and wide-distance (model n$^{\circ}$1); the medium-distance extraction condition did not significantly differ from the other two (models n$^{\circ}$3 and  n$^{\circ}$5). Even between the two extremes, the effect size was rather small (odds ratio 1.39), but this is expected if the effect is caused by processing factors. The interaction between narrow and wide (model n$^{\circ}$2) is also compatible with a superadditive effect. But notice that this interaction was only significant in the Cumulative Link Mixed Models, not in the AUCs. To conclude, even though the data did not exactly reproduce the predictions of the DLT (or any other model based on memory load), they at least did not run counter to these predictions. 

The habituation graph (\figref{fig:exp01-habituation} on page \pageref{fig:exp01-habituation}) shows a strange pattern for the medium-distance condition (extraction and non-extraction alike), whose acceptability strongly decreased in the course of the experiment. I cannot find any explanation for this phenomenon. Notice however that there is no main effect of trial in model n$^{\circ}$5, therefore this decrease of acceptability was not significant.

Finally, some remarks are in order on the low acceptability of the non-extraction condition: I do not know of any study on long-distance dependencies in any other language in which the non-extraction control was found to be less acceptable than the extraction condition (island or non-island). This conflicts with the assumption that acceptability reflects processing costs, since extractions are more costly. The reason might be the high frequency of \emph{dont} relative clauses in French and French speakers' attitude to stylistic properties of sentences. Our corpus results revealed that \emph{dont} relative clauses were very frequent (especially with extraction out of an NP). At the same time, the control conditions involve a certain anaphoric repetition (\emph{an innovation... its originality}) which could be considered redundant because the \emph{dont} relative clause alternative is permitted. \citet[133]{Armstrong.2001} points out the ``still highly normative and formal teaching methods employed in French schools to teach the language''. Unnecessary repetitions are strongly stigmatized by the French education system: they get corrected almost systematically in written productions, and are described as ``unelegant'' or ``heavy''. In a task involving written stimuli, participants may be influenced by their school experience and give lower acceptability ratings to the coordination conditions. This phenomenon is known as social desirability \citep{Edwards.1957}: a distortion of the participants' results toward metalinguistic judgments that reflect the (socially favored) norm instead of the actual grammatical competence.

On the other hand, it must be acknowledged that better ratings for the relative clauses than for coordinations is consistent with an analysis of \emph{dont} as a ``hanging topic'': the \emph{dont} relative clauses would then be coordinate-like structures, but without an anaphoric determiner as in the baseline condition, hence their higher acceptability (reflecting lower processing costs). This does not explain the interaction found between extraction out of the subject and out of the object, though.

The fact that the extraction conditions were perceived to be better than the non-extraction conditions had an unfortunate and unintended consequence for our results: \figref{fig:exp01-repartition} on page \pageref{fig:exp01-repartition} indicates a ceiling effect in the extraction conditions, which may flatten the results of the models.

