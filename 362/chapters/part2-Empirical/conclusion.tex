Contrary to an idea strongly anchored in syntax, extraction out of the subject is possible in French. The empirical data in Part~\ref{part:2} show that extraction out of the subject can be found in the productions of native speakers, and that speakers do not reject it in acceptability judgment tasks.

I was not able to confirm experimentally the observation made by Chaves and his colleagues \citep{Chaves.2014,Chaves.2019.Frequency,Chaves.2020.UDC} that extraction out of the subject undergoes strong habituation and that the ratings increase in the course of the experiment. But this is not very surprising, because the number of items in the present experiments is below the threshold considered by these authors as sufficient for observing habituation effects. In general, the habituation patterns seem to be relatively inconsistent, and this factor is rarely significant in my models. 

\section{Extraction out of the NP and extraction out of the subject}

In general, what we can observe in production data is that extraction out of subjects is very frequent where extraction out of NPs is frequent. In written French, there is no fixed usage of \textit{dont} or any other relative word for extracting the \textit{de}-PP complement of the verb, unlike what  \citet{Blanche-Benveniste.1990} observed for spoken French. 
By contrast, extraction of the \textit{de}-complement of the noun out of NPs is frequent in relative clauses. However, in interrogatives, we see a different pattern, namely the \textit{wh}-word is used almost exclusively to extract the complement of the verb and extraction out of NPs is rare. There is also no extraction out of the subject NP in interrogatives. I do not think that this is a coincidence. Indeed we see something similar when we look at \emph{avec qui} relative clauses. On the one hand, almost all occurrences of \emph{avec qui} relative clauses are extractions of the complement of the verb or adjuncts, and extraction out of NPs is very rare, while on the other hand we observe very few examples of extraction out of subjects. 

This first observation is compatible with what \citet{Kluender.2004} says about extraction out of the subject: Whenever extracting out of an NP is not a frequent option, there seems to be an additional disadvantage for extracting out of the subject compared to extracting out of the object.

This may explain a difference between the results of Experiment~12 and those of Experiments~10, 13, and 14. In Experiments~10, 13, and 14, extraction out of the subject was rated higher than the ungrammatical controls, whereas in Experiment~12 it was not significant. There are several ways to interpret this result. Syntactic accounts would take this as evidence that they are ungrammatical. But the predictions of the syntactic accounts have been falsified on many other occasions in the other experiments. Another option is to appeal to specificity. Indeed, many studies on English so-called ``\emph{wh}-islands'' show that \emph{which} + N interrogatives are more acceptable than \emph{who/what} interrogatives when the structure is complex. But I propose a different explanation: For some reason, \emph{de qui} is not considered a suitable interrogative phrase for NP subextractions. The evidence for this is that we find almost no \emph{de qui} interrogatives with an NP extraction in corpus studies.\footnote{Of course, it remains to be explained why \emph{de qui} is not suitable to extract out of an NP, while \emph{dont} is. I have no answer to this question at the moment.} Wherever extraction out of the NP is unacceptable, extraction out of the subject is even less acceptable.

But even if we take into account the frequency/acceptability of extraction out of the NP, there is definitely a difference between the constructions. \emph{Avec qui} relatives and \emph{de qui} interrogatives are both used rarely to extract out of the NP, but we find a few examples of extraction out of the subject with the former and none with the latter. This second observation shows that \citegen{Kluender.2004} hypothesis is not sufficient to explain the data, because it does not predict a cross-construction difference. For this, we need the Focus-Background Conflict constraint.

\section{Cross-construction difference}
\label{ch:exp-conclu-cross-construction}

The results of the experiments and of the corpus studies show a clear contrast between non-focalization constructions (the relative clauses) and focalization constructions (interrogatives and \emph{c'est}-clefts) as far as extraction out of a subject is concerned.\footnote{That the results of the experiments are perfectly compatible with the results of the corpus studies  provides further support for \citegen{Bosch.2020} observations that the two methodologies yield similar results.} This contrast is expected according to the FBC constraint, but not expected under any other account that I am aware of. My conclusion is that the FBC constraint predicts the results of these different experiments best, even though in itself it cannot account for all results.



\subsection{Relative clauses}

Production data from the French Treebank and Frantext show that speakers produce extraction out of the subject frequently. In \textit{dont} and \textit{de qui} relative clauses, extracting out of the subject is the most frequent usage of the relative word. In \textit{duquel} relative clauses, it is used more frequently than extraction out of the object, even though it is not the most frequent usage overall. In the experiments, the ratings for extraction out of the subject were significantly higher than those for extraction out of the object in Experiments~1, 4, 7 and 8, with a significant interaction in Experiment~1. In all experiments that include ungrammatical controls (Experiments 4, 7 and 8), the subextraction from the subject was rated significantly higher than its ungrammatical counterpart. Also, reading patterns in Experiment~3 indicated a disadvantage for extraction out of the object, if anything. Overall, the empirical data corroborate \citegen{Godard.1988} claim about extraction out of subjects in relative clauses in French. 

The lower ratings for extraction out of the subject observed in Experiment~5 are probably an artefact of the material, which had an animacy mismatch between subject and object. As we saw, the disadvantage disappeared when the animacy mismatch was removed in Experiment~7. 

The significant interaction in Experiment~1 in favor of the subextraction from the subject should be considered with caution. As I pointed out in the discussions of the experiments, these effects may be an artefact of the stimuli (especially of the choice of coordination with an anaphoric possessive article for the control conditions), but also due to the ceiling effects observed almost systematically for \emph{dont} relative clauses. The preference for extraction out of the subject is expected under accounts based on memory costs, because the filler-gap dependency is shorter in this case. It is not expected according to the FBC constraint. Of course, the two approaches are not incompatible, and my conviction is that a combination of the two predicts the data for relative clauses best. However, my studies do not provide enough evidence for me to conclude that processing accounts based on memory costs are confirmed by the data.

We conducted two corpus studies on Frantext 1900--1913 to see whether there had been a shift in usage over time, but this was not the case: Extraction out of the subject was also frequent at the beginning of the 20th century as well.

The corpus data therefore falsify accounts that predict any form of subject island effect. The experimental results are compatible with them, nevertheless I want to underline once again that null effects can neither support nor falsify any theory or prediction. As a matter of fact, what we observe in the experiments on relative clauses are mostly null effects. These results are therefore only meaningful because they contrast so undeniably with the results of the experiments on other constructions. On the other hand, the fact that potential ``subject island'' effects (i.e.\ interactions between extraction type and extraction site) are systematically absent is a strong cue~-- even though it is not evidence~-- that syntactic accounts that expect extraction out of the subject to be completely ruled out by the grammar, cannot be on the right track. 

\subsection{Focalizations (interrogatives and \emph{c'est}-clefts)}

The results on the interrogatives and clefts are directly opposite to those on the relative clauses. In the corpus studies, we find no extraction out of subjects in interrogatives, and the examples of extraction out of subjects in clefts may be debatable. In the experiments, there is much evidence suggesting a disadvantage for extraction out of the subject. First, it received significantly lower ratings than extraction out of the object in interrogatives with a short-distance dependency (Experiments~10 and 12) and in the \emph{c'est}-clefts (Experiment~14). Second, there is a significant interaction between extraction type and extraction site in interrogatives with a short-distance dependency. Whether there is also a significant interaction in the \emph{c'est}-clefts is not clear: the result is significant in the Cumulative Link Mixed Model, but not when comparing the AUCs (the latter method being more conservative). 

These results falsify an approach based on memory costs, which expects that extractions out of the subject should be rated better than extractions out of the object. Even though the other accounts expect a degradation when extracting out of the subject, no account except the FBC constraint can explain the contrast with relative clauses. Furthermore, the syntactic accounts are falsified, because extractions out of the subject should be ruled out by grammar and therefore be as unacceptable as ungrammatical controls. Yet, in Experiments~10, 13 and 14, extraction out of the subject was rated higher than the ungrammatical controls.

\subsubsection{Interrogatives worse than \emph{c'est}-clefts}

Interrogatives have their own pragmatic constraints. One of these constraints is that the presupposed part of the answers must be part of the Common Ground \citep{Simonenko.2015}. We can illustrate this requirement with example (\ref{ex:conceivable-answer-pair}).

\eal \label{ex:conceivable-answer-pair}
\ex[]{Question: Of which innovation does the uniqueness excite my colleagues for no reason? \label{ex:conceivable-answer-question}}
\ex[]{One answer alternative: [The uniqueness of the digital bracelet for social distancing]$_{B}$ excites my colleagues for no reason. \label{ex:conceivable-answer}}
\zl 

By default, the whole subject in (\ref{ex:conceivable-answer}) is presupposed and backgrounded. The pragmatic constraint of questions thus states that the subject of (\ref{ex:conceivable-answer}) must be part of the Common Ground in (\ref{ex:conceivable-answer-question}). Then we are faced with a contradiction: the inquirer asks for information that is presupposed as being part of the Common Ground (a presupposition that they introduced themselves). \citealt{Simonenko.2015} makes a similar argument about another kind of island. This mechanism may reinforce the FBC constraint and explain why the ``island effect'' is stronger in interrogatives than in \emph{c'est}-clefts.

\subsubsection{Exceptions}

The interaction showing a dispreference for extraction out of the subject was only found in interrogatives with a short-distance dependency. There are therefore two exceptions: interrogatives with the \emph{wh}-phrase in situ (Experiment~11), and interrogatives with a long-distance dependency (Experiment~13).

In Experiment~11, no significant effect was found, except for a main effect of extraction type. We can conclude that in-situ interrogatives do not behave like interrogatives with extraction as far as the ``subject island'' is concerned. This is not expected under the FBC constraint, in which extraction plays no role. One way to deal with the issue would be to add extraction to the constraint as a necessary factor. 
On the other hand, I have shown in the discussion of this experiment that the functional status of in-situ questions is not clear. Possibly the structure is not focalization by default, so the FBC constraint is not violated in in situ questions in French.
There is, for the time being, not enough strong evidence to decide between these two possibilities. Therefore, I leave this question open, and will assume the FBC constraint as defined in (\ref{rule:FBC}). 

In Experiment~13, no significant difference was observed between extraction out of the subject and out of the object. This contrasts with Experiment~10 and directly contradicts syntax-based and processing-based accounts. Higher complexity and an increased length of the extraction in long-distance dependencies is predicted by these accounts to lead to a stronger ``island effect'' than in short-distance dependencies. 
Is the contrast between Experiment~10 and Experiment~13 problematic for the FBC constraint as well?  I think not. First, Experiment~13 may not have been strong enough to reveal a discourse clash due to some factor that we did not take into account. More importantly, I can easily imagine that the discourse clash becomes weaker, or even disappears in a long-distance dependency. We view information structure as a relation between a constituent and a clause (see Section~\ref{ch:is-internal}). In (\ref{ex:interr-ldd-eng}), \emph{of who} is by default focused with respect to the matrix clause. But it is not certain that it is focused with respect to the embedded clause. 

\ea[]{[Of who]$_i$ do you think [that [the daughter~\trace{}$_i$] plays the piano]?}
\label{ex:interr-ldd-eng}
\z 

If \emph{of who} is not the focus of the embedded clause, then there is no discourse clash, and the FBC constraint is not violated. If \emph{of who} is the focus of the embedded clause, then it is likely to be focused to a lesser degree, based on the common assumption that elements may be focused, topic, or backgrounded to a greater or lesser extent. I will come back to this question in Section \ref{ch:analysis-ldd}. 


\subsection{Extraction out of a verbal subject}
\label{ch:exp-conclusion-CP-subject}

In Experiments~15 and~16, we looked at extraction out of infinitive subjects. Using \emph{que}, in Experiment~16 we observed an interaction effect, but with \emph{où} in  Experiment~15 we did not. Yet these results are not necessarily in conflict. Experiment~15 appeared to have strong ceiling effects, which may mask a potential interaction. 

Experiments~15 and~16 pose a problem with respect to the FBC constraint. They are relativizations, so no degradation is expected in the extraction out of the subject. Thus, it seems that the FBC constraint alone cannot explain the results we observe.

Let us consider the other approaches. Distance-based processing accounts do not predict the pattern we observe in Experiments~15 and~16 and can therefore be set aside. Overall, extraction out of the subject received high ratings, which is at odds with the ungrammaticality of these extractions assumed by syntactic accounts. They should therefore be set aside as well. Finally, functional accounts of the type ``Backgrounded Constituents are Islands'' are inconsistent with the other empirical results I have presented.

\citegen{Kluender.2004} proposal that retrieving complex topics induces higher processing costs seems to account for the findings, and it can be combined with the FBC constraint. We can indeed assume that processing costs are higher when the subject is more complex. NPs including a \emph{de}-PP in French seem relatively simple, NPs with another kind of PP are probably more complex. Verbal subjects are even more complex, but infinitival ones probably less so than sentential ones. This results in a hierarchy of complexity. The more complex the subject, the less expected it is, and the less likely that the addressee predicts a gap inside it. Unexpected gaps create stronger processing costs. As long as these processing costs are below a certain threshold, they have virtually no effect on the addressee's perception (as in our experiments on relative clauses). In Experiment~16, where the relative word is less specific than in Experiment~15, the threshold is crossed, and we observe a significant interaction, probably caused by more difficulty in extracting out of the subject.

\section{Discourse clash}

We cannot reduce the difference between the constructions to an incompatibility in information structure between the extracted element and its head. Although it seems problematic to focalize out of a topic, topicalizing out of focus (or non-topic) is not a problem. This is attested by the examples of extraction out of postverbal subjects that we find in the corpus studies. Following \citet{Lahousse.2011}, postverbal subjects are less topical than preverbal subjects. In the different corpus studies presented above, we find in total 12 cases of extraction out of a postverbal subject.\footnote{\citet[261]{Lahousse.2011} notes that most of the examples of postverbal subjects in the literature on subject-verb inversions are restrictive. In my corpus studies, all but one cases of extraction out of a postverbal subject are non-restrictive (and the last one reported page \pageref{ex:avecq-subj-incise}, (\ref{ex:avecq-subj-incise}) is ambiguous, even though it is annotated as restrictive~-- see our guidelines for annotation in Appendix~\ref{ch:corpus-methodo-annotation}). \citeauthor{Lahousse.2011} claims that inversion is always acceptable in restrictive relative clauses, but not in non-restrictive relative clauses. In a non-restrictive relative clause, she says, there must be a clear indication that the subject is non-topical (e.g., through the presence of a framing topic). I cannot confirm this assertion, and find no special indication that the subject is non-topical in these 12 relative clauses. In four of them, using a preverbal subject would have as a consequence that the verb would stand in the last position, and this is dispreferred in French \citep[still following][261]{Lahousse.2011}.} This means that topicalization out of a less topical element does not lead to a discourse clash. The reason why extraction out of a preverbal subject is much more frequent may be that speakers tend to minimize the distance between the filler and the gap. 

\section{Verb types}

\subsection{Extraction out of the subject of transitive verbs}

\citet{Chomsky.2008} and \citet{Polinsky.2013}, among other scholars, claim that extracting out of the subject of a transitive verb is degraded compared to extracting out of an ``underlying object'', i.e. out of the subject of a passive or unaccusative verb. In the majority of the corpus studies, there is indeed a significant difference between extraction out of the subject on the one hand and the other relative clauses on the other hand: in extraction out of the subject transitive verbs are significantly less frequent. Although this could mean that extraction out of the subject of transitive verbs is degraded, I will argue that this is not the case, for several reasons. First, the number of transitive verbs found in extractions out of subject NPs is far from marginal. Second, as already mentioned, the results may be biased by the fact that, ``other relative clauses'' include extractions out of object NPs, which by definition only occur with transitive verbs. But a third argument should also be mentioned.

In Section \ref{ch:processing-kluender}, we mentioned several corpus studies on English, Italian and Japanese which show that speakers tend to produce fewer complex subjects when the sentence contains a direct object than when it does not. 
The general rule in production data seems to be: the longer the rest of the VP, the shorter the subject.
If we assume that this is holds for French as well, then transitive verbs will have more clitic subjects than the other verb types. 
As it is not possible to extract out of a clitic, it is not surprising that extraction out of subjects involves transitive verbs less often than other types of extractions. We used the corpus study on \emph{duquel} in Frantext to reproduce on a small scale the corpus studies done in other languages. 
\figref{fig:duq-diversesubjtype-verbtype} shows the distribution of the subject types for the different verb types: the results are given only for pied-piping structures (i.e., \emph{duquel} is the complement of a preposition or the complement of a noun complement of a preposition). Except for a few cases, these pied-piping structures are extractions of the complement of a verb, or are extractions of an adjunct. We can clearly see that transitive verbs have a higher proportion of clitic subjects compared to the other kinds of verbs, as expected.

\begin{figure}
        \centering
        \includegraphics[width=\textwidth]{chapters/part2-Empirical/duquel-Frantext/subj-verb-type.jpeg}
        \caption{Subject type for every verb type in \emph{duquel} relative clauses with a pied-piping structure in Frantext 2000--2013}
        \label{fig:duq-diversesubjtype-verbtype}
\end{figure}

\subsection{A remark on psychological verbs}
\label{ch:psych-verbs-internal-arguments}

A common concern voiced by reviewers and conference audiences about the experiments presented in Part~\ref{part:2} has been our use of psychological verbs. The issue is that the subjects of some psych verbs are considered to be underlying objects in some theories. This idea, which only applies to experiencer-object psych verbs, goes back to \citet{Belletti.1988}, and the various debates have been summarized by \citet{Landau.2010}. The original claim made by \citet{Belletti.1988} was that these verbs are unaccusative, but \citet{Legendre.1989} and \citet{Herschensohn.1992} have brought some evidence against this for French, summarized as follows by \citeauthor{Landau.2010}: 

\begin{quote}
    ``On too many points -- for example, auxiliary selection, passivization, lexical operations referring to external arguments, compatibility with pure expletives (e.g.\ \emph{il} vs. \emph{cela} in French) -- class II verbs do not pattern with unaccusatives, their subject (a causer rather than a theme) behaving like a normal external argument [...].'' \citep[38--39]{Landau.2010}
\end{quote}

Nevertheless, \citet{Legendre.1989}, \citet{Herschensohn.1992} and \citet{Landau.2010} assume that the stimulus argument of such verbs is a derived subject.

\subsubsection{Arguments in the literature}

 \citet{Landau.2010} mentions three core arguments in favor of the derived subject hypothesis: experiencer-object psych verbs do not reflexivize, cannot be embedded as infinitival complements of a causative verb, and do not passivize. A careful study of these arguments would go beyond the scope of this work, but here are a few remarks which lead us to think that psych verbs behave like transitives.\footnote{Many thanks to Elisabeth Verhoeven for her input on this topic.}

\citet{Herschensohn.1992} offers the following example to illustrate the impossibility of reflexivizing experiencer-object psych verbs:

\eal 
\ex[]{\gll Les enfants se lavent les uns les autres.\\
the children \textsc{refl} wash the ones the others\\
\glt `The children wash one another.'}
\ex[*]{\gll Les enfants se préoccupent les uns les autres.\\
the children \textsc{refl} worry the ones the others\\
\glt `The children worry one (about) another.'}
\label{ex:psych-verb-refl-preoccuper}
\zl 

But, as mentioned by \citet{Herschensohn.1992} herself, verbs of this class have pronominal variants. For example, \emph{préoccuper} (`worry') has a ditransitive variant \emph{se préoccuper} (`worry about') with an indirect object, so that there is a straightforward way to express (\ref{ex:psych-verb-refl-preoccuper}), namely (\ref{ex:psych-verb-refl-sepreoccuper}):

\ea[]{\gll Les enfants se préoccupent les uns des autres.\\
the children \textsc{refl} worry the ones of.the others\\
\glt `The children worry one about another.'}
\label{ex:psych-verb-refl-sepreoccuper}
\z 

The second argument relies on causative constructions. The verb \emph{faire} (`to do') can be used to form a causative construction in French, as illustrated by example (\ref{ex:causative-basis}). The referent of the causative adverbial in (\ref{ex:causative-basis-1}) can be the subject of a \emph{faire} + Vinf structure like (\ref{ex:causative-basis-2}). 

\eal\label{ex:causative-basis}
\ex[]{\gll Jean porte un appareil auditif à cause de son grand âge.\\
Jean wears a device hearing at reason of his old age\\
\label{ex:causative-basis-1}
\glt `Jean wears a hearing device because of his advanced age.'}
\ex[]{\gll Son grand âge fait porter à Jean un appareil auditif.\\
his old age makes wear\textsc{.inf} at Jean a device hearing\\
\label{ex:causative-basis-2}
\glt `His advanced age makes Jean wear a hearing device.'}
\zl 

The transposition from (\ref{ex:causative-basis-1}) to (\ref{ex:causative-basis-2}) is not possible for experiencer-object verbs:

\eal 
\ex \gll La télévision dégoûte Jean à cause de son bruit déplaisant.\\
the television disgusts Jean at reason of its noise annoying\\
\glt `The television disgusts Jean because of its annoying noise.'
\ex \citep[252]{Kayne.1975}\\
* \gll Son bruit déplaisant fait dégoûter Jean à la télévision.\\
its noise annoying makes disgust\textsc{inf} Jean at the television.\\
\glt `Its annoying noise makes Jean disgusted by the television.'
\zl 

According to \citet{Burzio.1986}, causative constructions can be used as a test to see whether a subject is a derived one. Clauses with derived subjects cannot be embedded as infinitival complements of causative constructions. However, notice that this kind of embedding is infelicitous for any non-agentive subject of a transitive verb, as the following example shows:

\eal 
\ex[]{La grange abrite Jean grâce à son toit solide.\\
the barn protects Jean thanks at its roof robust\\
\glt `The barn protects Jean thanks to its robust roof.'}
\ex[*]{Son toit solide fait abriter Jean à la grange.\\
its roof robust makes protect\textsc{.inf} Jean at the barn\\
\glt `Its robust roof makes Jean protected by the barn.'}
\zl 

\citeauthor{Kayne.1975} himself notes that this restriction is lifted in agentive contexts, as reported by \citet[38]{Landau.2010}. Experiencer-object psych verbs are thus not different from any other non-agentive verbs in this respect. One may argue that all non-agentive subjects are internal arguments, but this goes beyond the analysis of \citet{Landau.2010} that we discuss here.\footnote{For a discussion of this specific point, see Section~\ref{ch:psych-verb-no-consequence} below.}

The last claim, made by \citeauthor{Legendre.1989} and reported by \citeauthor{Landau.2010}, is that the passive of psych verbs is stative, i.e.\ that \emph{choqué} in (\ref{ex:psych-verb-adjective}) is an adjective and not a real passive form.

\ea[]{\gll Pierre est choqué par le film.\\
Pierre is shocked by the movie\\
\glt `Pierre is shocked by the movie.'
\label{ex:psych-verb-adjective}}
\z 

\citeauthor{Legendre.1989}'s demonstration relies on two arguments: (i) the impossibility of embedding psych verbs in a causative construction with \emph{faire} and (ii) the impossibility of applying \emph{re}-prefixation. I have just shown that (i) is not a good argument for French. Argument (ii) is the following: the prefix \emph{re-} in French can be used productively to form new verbs, expressing the idea that the event has been repeated. According to \citeauthor{Legendre.1989}, past participles of experiencer-object verbs cannot undergo \emph{re}-prefixation, which shows that they are adjectives and not verbs. This is illustrated by the contrast between the
``real'' participle in (\ref{ex:psych-verb-adjective-rechoquer}) and the alleged adjective in  (\ref{ex:psych-verb-adjective-rechoque}) (the judgments on the examples are in concordance with \citeauthor{Legendre.1989}'s account):

\eal
\ex[]{\gll Ce film a rechoqué Pierre.\\
this movie has shocked.again Pierre\\
\glt `This movie shocked Pierre again.'
\label{ex:psych-verb-adjective-rechoquer}}
\ex[*]{\gll Pierre a été rechoqué par ce film.\\
Pierre has been shocked.again by this movie\\
\glt `Pierre has been shocked again by this movie.'
\label{ex:psych-verb-adjective-rechoque}}
\zl 

My intuition differs from \citeauthor{Legendre.1989}'s in this respect. Both (\ref{ex:psych-verb-adjective-rechoquer}) and (\ref{ex:psych-verb-adjective-rechoque}) are relatively infelicitous for me, probably because of the neologism, but I do not have the impression of a strong contrast between the two. 
Furthermore, if the passive of psych verbs were stative, a construction with \emph{en train de} (equivalent of V+ing in English) should be ruled out. My intuition is that (\ref{ex:psych-verb-adjective-entrainde}) is strange, but possible:

\ea[?]{\gll Pierre est en train d’ être choqué par ce film.\\
Pierre is in pace of be\textsc{.inf} shocked by the movie\\
\glt `Pierre is being shocked by the movie.'}
\label{ex:psych-verb-adjective-entrainde}
\z 

To conclude, the arguments brought in favor of analyzing the subjects of experiencer-object psych verbs as underlying objects are not sufficiently well-grounded. I acknowledge that dedicated work should be done on this topic, but this is not the aim of the present book.

% maybe N Ruwet 1993 Les verbes dits psychologiques: trois theories et quelques questions, Recherches linguistiques de Vincennes

\subsubsection{Consequence for the experiments in this book}
\label{ch:psych-verb-no-consequence}

Even if the subjects of experiencer-object psych-verbs were not ``real'' subjects, this would not have a big impact on my experiments and would not account for the results.

In Experiments~1--6, 10, 11, 13 and 14 systematically involve extraction out of the stimulus argument. 
Therefore, extraction out of subjects in these experiments would be extraction out of underlying objects under the above hypothesis. This could account for the results of the relative clauses (i.e., that native speakers do not reject these sentences), but not for the interaction effects that I find in focalizing constructions (i.e., that extraction out of the subject is rated worse than extraction out of the object). If we inadvertently tested extraction out of objects in all conditions, where does the ``subject island'' effect come from?

In Experiments~7, 8, 9 and 11, some test sentences contained experiencer-object psych verbs, but they only represent around 1/3 of the items. The other items were extraction out of ``real'' subjects under \citegen{Landau.2010} analysis. Grouping the results of Experiment~9 according to verb types did not reveal any obvious contrast between the experiencer-object verbs and the other verbs.\footnote{It would be possible to run additional experiments without any experiencer-object psych verbs. But there is a certain correlation between the type of psych verb (experiencer subject vs.\ object) and the implicit causativity of the verb. Counterbalancing this factor would be very complicated, as the list of common psych verbs is relatively short.}

\section{A remark on hanging topics}
\label{ch:hanging-topics}

Another recurrent concern about the interpretation of the experimental results is a certain suspicion that relative clauses (or at least some relative clauses) may not involve ``real'' extraction.
This is the argumentation presented by \citet[82--85]{Giorgi.1991}, and developed extensively by \citet[Section 5]{Jurka.2010} and \citet[Section 2.3]{Uriagereka.2012}. 
In a nutshell, their claim is that extraction of a PP-complement out of a subject NP is a bad counterexample to the subject island constraint, because such PP-complements are possible hanging topics (or ``aboutness proleptics'' in \citegen{Uriagereka.2012} terms). An example of fronted aboutness proleptics in Spanish is given in (\ref{ex:proleptic}), the relevant elements appear in italics. 

\ea \citep[94]{Uriagereka.2012}\\
\gll \WinckelEmph{De} \WinckelEmph{los} \WinckelEmph{árboles} \WinckelEmph{frutales} me gusta el melocotón, y \WinckelEmph{de} \WinckelEmph{los} \WinckelEmph{reyes} \WinckelEmph{de} \WinckelEmph{España} Alfonsito de Borbón.\\
of the trees fruit-growing me pleases the peach and of the kings of Spain Alphonso de Borbón \\
\glt `Concerning fruit trees, I like peaches ; and concerning the kings of Spain, I like Alphonso de Borbón.'
\label{ex:proleptic}
\z

The English translation makes the aboutness relation of these proleptics transparent. Evidence that the fronted elements are not extracted out of the NPs \emph{el melocotón} and \emph{Alphonso de Borbón} is that neither [\emph{el melocotón de los árboles}] nor [\emph{Alfonsito de Borbón de los reyes de España}] is a correct complete NP. \citeauthor{Uriagereka.2012} and \citeauthor{Jurka.2010} assume that all examples of felicitous extraction out of NPs in the literature can be considered to be only  apparent extraction, and actually showing aboutness proleptics. Hence, following their argument, acceptable cases of pied-piped PPs in English, as in (\ref{ex:urigareka-picture-there}), can also be analyzed by treating the fronted element (here \emph{of which politician}) as a proleptic, similarly to what happens in (\ref{ex:urigareka-picture-paraphrase}) which obviously does not involve extraction.\footnote{For \citet{Uriagereka.2012}, \emph{a picture} in (\ref{ex:urigareka-picture-there}) is an underlying subject. Example (\ref{ex:urigareka-picture-there}) is therefore a potential English counter-example to the subject island constraint.}

\begin{exe}
\ex \citep[95]{Uriagereka.2012}
\begin{xlist}
	\ex Of which politician is there a picture on the wall? \label{ex:urigareka-picture-there}
	\ex I often think of politicians that there is a picture of them on every wall.\label{ex:urigareka-picture-paraphrase}
\end{xlist}
\end{exe}

Notice that \citeauthor{Uriagereka.2012}'s argument is not restricted to relative clauses, as example (\ref{ex:urigareka-picture-there}) shows. His objection, then, does not hold for our studies, because we used similar materials across experiments, hence with the same potential of having proleptical topics, and nevertheless we found a contrast between relative clauses and interrogatives.

But the original remark in \citeauthor{Giorgi.1991} concerns specifically relative clauses, and topicalizing constructions like relativization seem indeed more prone to be associated with aboutness proleptics or hanging topics. If so, this could explain the cross-construction difference that we observed in our experiments.

The possibility of analyzing relativization out of an NP as non-extraction has been discussed by \citet[87--88]{Haegeman.2014}. They argue that the relative phrases are not hanging topics based on the following evidence: (i) the head noun constrains the preposition of the relative phrase, and (ii) the relative phrase is sensitive to island constraints.

In this respect, we agree with \citeauthor{Haegeman.2014}. Hanging topics in relatives are possible in French, though restricted to colloquial French. But they involve a gapless relative clause and are restricted to \emph{que} relatives, hence to NPs.

\ea (\citealt[42]{Abeille.2007.Relatives}; attributed to Fran\c{c}oise Gadet)\\
\gll Vous avez une figure que vous devez avoir de la température.\\
you have a face that you must have\textsc{.inf} of the temperature\\
\glt `You have a face that you must have fever.'
\z 

% True hanging topic in spoken French (Abeille.2008, 308)
% Le cinéma alors on se décide ? (the movie then we make a decision?)
% b.euh la mairie de Saintes on connaît le le candidat socialiste qui vient de se déterminer [CRFP] (hum the town council of Saintes we know the the socialist candidate who has just made his decision)

The complementizer \emph{dont} cannot be used in such an aboutness relation. It can introduce gapless relative clauses, but must be coindexed with a resumptive pronoun.

\begin{exe}
\ex \citep[21--22]{Godard.1985}
\begin{xlist}
\ex[]{\gll un argument dont$_i$ on pense que personne ne l$_i$' a utilisé\\
an argument of.which one thinks that noone \textsc{neg} it has used\\
\glt `an argument that we think that noone has used it'}
\ex[*]{\gll un argument dont on pense que personne n' a utilisé\\
an argument of.which one thinks that noone \textsc{neg} has used\\
\glt `an argument that we think that noone has used it'}
\end{xlist}
\end{exe}

This condition is not met in our experimental stimuli, in which there is no element that can serve as a resumptive pronoun in the relative. And yet, replacing \emph{dont} by \emph{que} in our relatives decreases the acceptability: it is the way we constructed our ``ungrammatical'' controls in the experiments on \emph{dont} relative clauses. Even though, as just explained, the construction is only colloquial and not ruled out, these controls received very low acceptability ratings.\footnote{English seems less strict as far as subextraction out of an NP is concerned, as shown by the following examples:

\begin{itemize}
    \item[(i) *] This is the city that I've always wanted to go.
    \item[(ii) ] This is the city that I've always wanted to visit the capital. \citep[examples from][59]{Chaves.2020.UDC}
\end{itemize}

A contrast like the one between (i) and (ii) does not exist in French.}

The relative phrase \emph{de qui} cannot introduce gapless relatives. Moreover, the head noun of the subject/object of the relative clause selects for the the preposition \emph{de}. Omitting the preposition or replacing it with a different one would be ungrammatical. The ungrammatical controls in our experiments on \emph{de qui} relative clauses were created by omitting the preposition, and these conditions received very low ratings. 

I conclude that the relative clauses tested in our experiments (Experiment~1 to 9) are ``real'' instances of filler-gap dependencies. The relative phrases do not express loose relations, but are constrained by syntactic rules (e.g.\ argument selection). 

\section{A cross-linguistic perspective}
\label{ch:exp-conclu-eng}

As mentioned previously, the results of our experiments on French are compatible with the experiments by \citet{Sprouse.2016} on Italian, even though they do not match exactly.

In \citet{Abeille.2020.Cognition}, we reproduced their results on English: there was a significant ``island effect'' (interaction) disfavoring extraction out of the subject with preposition stranding, both for interrogatives and relative clauses like (\ref{ex:Cognition2020-stranding}). 

\ea The dealer sold a sportscar, which [the color of~\trace{}] delighted the baseball player because of its surprising luminance.
\label{ex:Cognition2020-stranding}
\z 

Our experimental items also contain extractions with the \emph{of} preposition pied-piped, thus in a configuration more similar to French or Italian. In this case, a cross-construction difference emerges in that English does not behave differently than Italian or French. We therefore find a significant ``island effect'' in interrogatives, but not in relative clauses like (\ref{ex:Cognition2020-ppied}).

\ea The dealer sold a sportscar, of which [the color~\trace{}] delighted the baseball player because of its surprising luminance.
\label{ex:Cognition2020-ppied}
\z 

This is not surprising, because the cognitive principles responsible for the FBC constraint should be cross-linguistically valid. 

The obvious question that arises is: Why did we observe an interaction effect in English relative clauses with preposition stranding? I am not able to give a satisfactory answer, at this time. Many questions remain open about preposition stranding in English that may help us to understand English extractions out of the subject. 

There have been some corpus studies comparing the use of preposition stranding vs.\ pied-piping in English: \citet{Johansson.1998} on the London-Lund Corpus (spoken British English), the Birmingham Corpus (spoken component), the British National Corpus (spoken component) and the London-Oslo/Bergen Corpus, \citet{Trotta.2000} on the Brown University Corpus (written American English), \citet{Gries.2002} on the British National Corpus, \citet{Hoffmann.2005,Hoffmann.2008,Hoffmann.2011} on the International Corpus of English (British English component) and \citet{Hoffmann.2011} on the International Corpus of English (Kenyan English component). Two main contrasts have been reported by many authors. (a) Spoken data contains more preposition stranding than written data (\citealt[70]{Johansson.1998}; \citealt{Gries.2002}). \citet[280--284]{Hoffmann.2005} attributes this to a distinction in formality. Indeed, \citet[88]{Haegeman.2014} claim that pied-piping constructions are ``unnatural'' in colloquial English. 
(b) A cross-construction difference is noticeable: there is a preference for preposition stranding in interrogatives, but a preference for pied-piping in (non-free) relative clauses (\citet{Johansson.1998}, \citealt{Trotta.2000}; \citealt{Hoffmann.2005,Hoffmann.2011}).
% JP Koenig?

In the Brown Corpus, \citet[57]{Trotta.2000} finds that 63.7\% of the interrogatives use preposition stranding instead of pied-piping. \citet{Hoffmann.2008} reports much higher proportions, with 96\% of direct interrogatives and 92\% of indirect interrogatives displaying preposition stranding. \citet[148 Kenyan English]{Hoffmann.2011} has similar figures: 81\% of direct and 91.7\% of indirect interrogatives show preposition stranding. Pied-piping is more present in formal contexts, but the effect of formality is minor (\citealt[152--155]{Hoffmann.2011}; see also \citealt[64--65]{Trotta.2000}).
%But \citegen[230]{Gries.2002} statement that ``P[reposition] S[tranding] in interrogatives is prescriptively considered ungrammatical''. 

In relative clauses, pied-piping is only possible with \textit{wh}-words (\textit{that} and null relativizers only allow for preposition stranding) and almost impossible in free relative clauses \citep{Hoffmann.2005,Hoffmann.2008,Hoffmann.2011}. However, preposition stranding is not an option in some circumstances: with certain antecedents like \textit{way} or \textit{extent}, or with prepositions like \textit{beyond} or \textit{during} \citep[74--76]{Johansson.1998}. In the written modality, \citet{Trotta.2000} reports that 98.9\% of \textit{wh-} relative clauses had pied-piping; in \textit{which} relative clauses, \citet[70]{Johansson.1998} found 97\% and \citet{Hoffmann.2005} found  92\%  of pied piping. In \citet{Hoffmann.2008}'s data, 69\% of \textit{wh}- relative clauses displayed pied-piping, but this proportion goes up to 91\% when only the formal register is considered. Similarly, 86.4\% of the relative clauses in \citeauthor{Hoffmann.2011}'s British English data, and 84.1\% of the relative clauses in \citeauthor{Hoffmann.2011}'s Kenyan English data have pied-piping \citep[148]{Hoffmann.2011}. 
% In Trotta 2000, 186: 
    % "diachronic, dialectal and standard English show that stranding is not really an option with Wh- OP relativess: not now and not in the past" Van den Eynden 1996, 444
    % "Wh- OP relatives occur with pied-piping only" Van den Eynden 1996, 444

Even in spoken data, \citet{Johansson.1998} report that, depending on the corpus, 69\%--86\% of \textit{which} relative clauses use pied-piping instead of preposition stranding. 
The proportion of pied-piping is higher in more formal settings. Here, the classification adopted by the scholars leads to different results. \citet[71]{Johansson.1998} uses the %brod 
classifications of their corpora. They report no less than 66\% of pied-piping in \textit{which} relative clauses in the category ``Leisure-Dialogue'' of the BNC. \citet{Hoffmann.2011} distinguished three levels of formality. He observes that the proportion of pied-piping increases strongly with the degree of formality. In the ``informal'' level, barely more than 20\% of the \textit{wh}- relative clauses show pied-piping.

\citet[150]{Hoffmann.2011} distinguishes relative clauses from clefts. In the other studies, the proportions given above for relative clauses probably include clefts as well. In British English, the distribution is similar in clefts and relative clauses, as 86.0\% of the clefts use pied-piping instead of preposition stranding.\footnote{In Kenyan English, we observe a preference for preposition stranding (only 40.0\% of the clefts have pied-piping). However, this distribution is based on only 5 occurrences in total.} 

Furthermore, all extractions out of NPs studied by \citet{Trotta.2000} are instances of pied-piping.\footnote{One occurrence is a question \citep[61]{Trotta.2000}, 56 occurrences are relative clauses \citep[184]{Trotta.2000}.} In \citegen{Hoffmann.2011} data, this proportion is 80.9\% (75.6\% in British English; 88.9\% in Kenyan English). Interestingly, \citeauthor{Hoffmann.2011} found 17 instances of extraction out of the subject in the British English component and 4 instances in the Kenyan English component, which are not included in his analysis, since "stranding is not an option in these cases'' \citep[119;fn.\ 3]{Hoffmann.2011}. 

The English relative clauses with preposition stranding tested by \citet{Abeille.2020.Cognition} employed preposition stranding (i) in a relative clause, (ii) in a written and relatively formal context, and (iii) with extraction out of an NP. As we have just seen, such relative clauses are unlikely to occur in spontaneous production. This may not be sufficient to account for the interaction effect that was found disfavoring extraction out the subject by \citeauthor{Abeille.2020.Cognition}, but it still shows that there are more factors at play than a subject/object distinction. As the corpus studies reported in this book show, there is a contrast between configurations in which extraction out of NPs is frequent and configurations in which it is rare: In the former case (e.g., \textit{dont} or \textit{de} \textit{qu}- relative clauses), extraction out of the subject is the majority, in the latter case (e.g., \textit{avec} \textit{qu}- relative clauses, interrogatives), it is virtually absent in the corpora.
This resembles the pattern that we observe here: extraction out of the subject is acceptable in \textit{which} relative clauses with pied-piping (a frequent configuration for extraction out of NPs) and inacceptable in relative clauses with preposition stranding (in which extraction out of NPs is hardly attested). 

\citet{Gries.2002} noticed that transitive verbs show a preference for pied-piping compared to other verb types. Since we used transitive verbs, this remark is interesting, but does not explain the contrast between subject and object.
\citeauthor{Gries.2002} thinks that this factor is actually correlated with another one, also significant in predicting the choice of the structure: the syllabic length of the material between the filler and the gap. He observes that the longer the distance, the more often the sentences show pied-piping. This observation conflicts with our results, because the distance between the filler and the gap is shorter in extraction out of the subject than in extraction out of the object. This indicates that the factor that causes the interaction effect in our results and in \citegen{Sprouse.2016} findings must be strong enough to override the attested preference for shorter distances.


