\section[head=Experiment 12]{Experiment 12: Acceptability judgment study on \emph{de qui} \emph{wh}-questions with \emph{wh}-extraction}

% Do we see a difference between no-extraction control and subextraction from subject as far as accuracy is concerned? Do people understand those sentences?

This experiment aimed to reproduce the results of Experiment~10 (interrogatives without long-distance dependency) while being parallel to Experiment~7 (\emph{de qui} relative clauses). The materials were based on Experiment~7, but turned into interrogatives. 

\subsection{Design and materials}

To construct the stimuli for Experiment~10, we took the materials of Experiment~7 as a starting point and turned the items into interrogatives, following the same design as in Experiment~10. Interrogatives were formed with \emph{est-ce que}, with polar questions as non-extraction controls, and questions with a missing preposition as ungrammatical controls. 

\eal 
\ex[]{{Condition subject + PP-extracted:}\nopagebreak\\
\gll [De qui]$_i$ est - ce que [l' associé~\trace{}$_i$] aide mon cousin sans contrepartie financière~?\\
of who is {} it that the associate helps my cousin without counterpart financial\\
\glt `Of who does the associate help my cousin without financial compensation?'}
\label{ex:exp12-subj-pp}
\ex[]{{Condition object + PP-extracted:}\\
\gll [De qui]$_i$ est - ce que mon cousin aide [l' associé~\trace{}$_i$] sans contrepartie financière~?\\
of who is {} it that my cousin helps the associate without counterpart financial\\
\glt `Of who does my cousin help the associate without financial compensation?'}
\label{ex:exp12-obj-pp}
\ex[]{{Condition subject + noextr:}\\
\gll Est - ce que l' associé de l' avocat aide mon cousin sans contrepartie financière ?\\
is {} it that the associate of the lawyer helps my cousin without counterpart financial\\
\glt `Does the associate of the lawyer help my cousin without financial compensation?'}
\label{ex:exp12-subj-no}
\ex[]{{Condition object + noextr:}\\
\gll Est - ce que mon cousin aide l' associé de l' avocat sans contrepartie financière ?\\
is {} it that my cousin helps the associate of the lawyer without counterpart financial\\
\glt `Does my cousin help the associate of the lawyer without financial compensation?'}
\label{ex:exp12-obj-no}
\ex[]{{Condition subject + ungramm:}\\
\gll Qui est - ce que l' associé aide mon cousin sans contrepartie financière~?\\
who is {} it that the associate helps my cousin without counterpart financial\\
\glt `Who does the associate help my cousin without financial compensation?'}
\label{ex:exp12-subj-un}
\ex[]{{Condition object + ungramm:}\nopagebreak\\
\gll Qui est - ce que mon cousin aide l' associé sans contrepartie financière~?\\
who is {} it that my cousin helps the associate without counterpart financial\\
\glt `Who does my cousin help the associate without financial compensation?'}
\label{ex:exp12-obj-un}

\zl 

We tested the same 20 items as in Experiment~7, each manipulated according to the six conditions just described. In addition, the experiment included 36 distractors. The distractors were declaratives, and some of them were ungrammatical. Half of the experimental items and distractors were followed by a comprehension question. The item presented here as an example was paired with the comprehension question \emph{Est-il question d'un associé~?} (`Is this about an associate?').

\subsection{Predictions}

The predictions for this experiment, and all experiments on interrogatives with extraction, were the same as for Experiment~10. They are summarized in \tabref{tab:exp10-predictions} on page~\pageref{tab:exp10-predictions}.

\subsection{Procedure} 

We conducted the experiment on the Ibex platform \citep{Ibex}. The procedure was similar to that in the previous acceptability judgment experiments (see Section \ref{ch:methodo-AJ}). Participants rated the sentences on a Likert scale from 0 to 10, 0 being labeled as ``bad'' and 10 being labeled as ``good''. They also answered comprehension questions after each sentence.

The experiment took approximately 20 minutes to complete. 

\subsection{Participants}

The study was run in October 2019. 
Participants were recruited through FouleFactory (\url{https://www.foulefactory.com}), and paid 5€ for their participation. The payment was not contingent on the participants' responses to the questions about native language or place of birth.

57 participants took part in the experiment. 
The analysis presented here is based on the data from the 48 participants who satisfied all criteria.\footnote{To calculate accuracy, we excluded not only the answers to comprehension questions of the practice items and of the ungrammatical controls, but also of some distractors that had a low accuracy rate.}
They were aged 23 to 69 years. 36 of them self-identify as women, 12 self-identified as men. One participant (2.08\%) indicated having an educational background related to language.

\subsection{Results and analysis}

\figref{fig:exp12-boxplot} shows the results of the acceptability judgment task.
The subextraction conditions received very low rating. Extraction out of the subject (\ref{ex:exp12-subj-pp}) had a mean acceptability rating of 1.49, lower than extraction out of the object (\ref{ex:exp12-obj-pp}) with a mean rating of 2.50. The non-extraction conditions were rated high, as expected: the mean acceptability rating was 8.16 in the subject control condition (\ref{ex:exp12-subj-no}), and 7.59 in the object control condition (\ref{ex:exp12-obj-no}). The ungrammatical controls received very low acceptability ratings, comparable to the subextraction conditions: 2.03 in the subject condition (\ref{ex:exp12-subj-un}), and 1.66 in the object condition (\ref{ex:exp12-obj-un}). 

\begin{figure}
    \centering
    \includegraphics[width=\textwidth]{chapters/part2-Empirical/Exp12-dequi-whq/boxplots.jpeg}
    \caption{Acceptability judgments by condition in Experiment~12. The grey box plots indicate the median and quartiles of the results. Black points are outliers. Mean and confidence intervals are indicated in white.}
    \label{fig:exp12-boxplot}
\end{figure}

Our conditions received either very high or very low ratings, with potential ceiling and floor effects. \figref{fig:exp12-repartition} corroborates this: it is very probable that we see ceiling and floor effects here.

\begin{figure}
    \centering
    \includegraphics[width=\textwidth]{chapters/part2-Empirical/Exp12-dequi-whq/repartition.jpeg}
    \caption{Density of the ratings across conditions for Experiment~12}
    \label{fig:exp12-repartition}
\end{figure}

Another representation of the results is given by the ROC and zROC curves of the results in \figref{fig:exp12-ROC} on page \pageref{fig:exp12-ROC}. The ROC curves show that participants discriminated between the ungrammatical baseline and the non-extraction conditions, but barely discriminated between the ungrammatical baseline and the subextraction conditions. The zROC curves are convex for the object conditions: this could be due to the strong floor effect of the ungrammatical controls that serve as a baseline. 

\begin{figure}
    \centering
    \includegraphics[width=\textwidth]{chapters/part2-Empirical/Exp12-dequi-whq/ROC.jpeg}
    \includegraphics[width=\textwidth]{chapters/part2-Empirical/Exp12-dequi-whq/zROC.jpeg}
    \caption{ROC curves (top) and zROC curves (bottom) of the non-extraction conditions compared to their respective subextraction conditions, represented by the dotted grey baseline (\citealt{Dillon.2019}'s method) in Experiment~12.}
    \label{fig:exp12-ROC}
\end{figure}

The ROC and zROC curves in \figref{fig:exp12-ROC-subj} on page \pageref{fig:exp12-ROC-subj} show the discrimination between the subject and object conditions. Participant barely discriminated between the subject and the object variants, however, there is more discrimination in the subextraction conditions. The zROC curves are relatively straight, but the curve for subextraction deviates from the parallel line. This could also be due to the floor effect in the extractions out of the subject: there is much less variance because all ratings are concentrated around the bottom of the scale. 

\begin{figure}
    \centering
    \includegraphics[width=\textwidth]{chapters/part2-Empirical/Exp12-dequi-whq/ROC-subject.jpeg}
    \includegraphics[width=\textwidth]{chapters/part2-Empirical/Exp12-dequi-whq/zROC-subject.jpeg}
    \caption{ROC curves (top) and zROC curves (bottom) of the object conditions compared to their respective subject conditions, represented by the dotted grey baseline (\citealt{Dillon.2019}'s method) in Experiment~12.}
    \label{fig:exp12-ROC-subj}
\end{figure}

\subsubsection{Habituation} 

The habituation effects in the course of the experiment are given in \figref{fig:exp12-habituation} on page~\pageref{fig:exp12-habituation}. Subextraction and ungrammatical conditions are gathered at the bottom of the scale, but we can see an considerable habituation in the subextraction from object condition. 

\begin{figure}
    \centering
    \includegraphics[width=\textwidth]{chapters/part2-Empirical/Exp12-dequi-whq/habituation.jpeg}
    \caption{Changes in the mean acceptability ratings ($z$-scored by participant) by condition in the course of Experiment~12}
    \label{fig:exp12-habituation}
\end{figure}

We fitted a first model to explore the habituation effect in extraction out of the object. We included random slopes for the fixed factor grouped by participants and items. The results of the model are reported in Table \ref{tab:exp12-m1}. 
The results are not significant, showing that, despite the strong habituation effect seen in \figref{fig:exp12-habituation}, the trial number is not a good predictor for the rating.

% latex table generated in R 3.4.4 by xtable 1.8-4 package
% Sat Mar 28 14:47:01 2020
\begin{table}
\begin{tabular}{l S[table-format=1.3] S[table-format=1.3] S[table-format=1] S[table-format=<1.4] S[table-format=2.2]}
  \lsptoprule
 & {Estimate} & {SE} & {$z$} & {$\text{Pr}(>|z|)$} & {Odd.ratio} \\ 
  \midrule
  extraction type          & 0.629 & 0.124 & 5 & <.001  & 1.88 \\ 
  distance                 & 0.213 & 0.076 & 3 & <.01   & 1.24 \\ 
  trial                    & 0.001 & 0.005 & 0 & 0.8325 & 1.00 \\ 
  extraction type:distance & 0.173 & 0.090 & 2 & 0.0553 & 1.19 \\ 
   \lspbottomrule
\end{tabular}
\caption{Results of the Cumulative Link Mixed Model (model n$^{\circ}$6)}
\label{tab:exp1-m6}
\end{table}


\subsubsection{Comparing subextraction from the subject with subextraction from the object}

We fitted a second model to compare the extractions out of the subject and out of the object on their own (mean centered with subject coded negative and object coded positive). We included trial number as a covariate, and random slopes for all the fixed effects and the covariates grouped by participants and items. The results of the model are reported in Table \ref{tab:exp12-m2}. 
There is a significant effect of the syntactic function, such that the object condition has significantly higher ratings than the subject condition.

% latex table generated in R 3.6.3 by xtable 1.8-4 package
% Wed May 20 14:00:46 2020
\begin{table}
\begin{tabular}{l S[table-format=1.3] S[table-format=1.3] S[table-format=1] S[table-format=<1.4] S[table-format=1.2]}
  \lsptoprule
 & {Estimate} & {SE} & {$z$} & {$\text{Pr}(>|z|)$} & {Odd.ratio} \\ 
  \midrule
(Intercept) & 1.127 & 0.239 & 5 & <.001 & 3.09 \\ 
  syntactic function & 0.044 & 0.091 & 0 & 0.6283 & 1.04 \\ 
  trial & 0.012 & 0.005 & 2 & <.05 & 1.01 \\ 
   \lspbottomrule
\end{tabular}
\caption{Results of the Logistic regression model (model n$^{\circ}$2)}
\label{tab:exp02-m2}
\end{table}


In a third model, we compared subextraction with non-extraction. We fitted a model crossing syntactic function and extraction type (mean centered with extraction coded positive, non-extraction coded negative). We included trial number as a covariate, and random slopes for all fixed effects and covariates grouped by participants and items. The results of the model are reported in Table \ref{tab:exp12-m3}. 
There is a significant main effect of extraction type (non-extractions are rated higher), and a significant main effect of trial (habituation). There is also a significant interaction effect. Indeed, \figref{fig:exp12-interaction} shows a disadvantage for extractions out of the subject compared to the other conditions. The difference is also significant ($p < 0.005$) if we compare the the AUCs (green and red curves on \figref{fig:exp12-ROC-subj}). 

% latex table generated in R 3.6.3 by xtable 1.8-4 package
% Sun Apr 26 23:02:02 2020
\begin{table}
\begin{tabular}{l S[table-format=1.3] S[table-format=1.3] c S[table-format=<1.3] S[table-format=1.2]}
  \lsptoprule
                     & {Estimate} & {SE} & {$z$} & {$\text{Pr}(>|z|)$} & {OR} \\ 
  \midrule
  syntactic function & 1.328 & 0.488 & 3 & <.01 & 3.77 \\ 
  trial              & 0.069 & 0.023 & 3 & <.005 & 1.07 \\ 
 \lspbottomrule
\end{tabular}
\caption{Results of the Cumulative Link Mixed Model (model n$^{\circ}$1)}
\label{tab:exp16-m1}
\end{table}


\begin{figure}
    \centering
    \includegraphics[width=\textwidth]{chapters/part2-Empirical/Exp12-dequi-whq/interaction.jpeg}
    \caption{Interaction between syntactic function and extraction type in Experiment~12}
    \label{fig:exp12-interaction}
\end{figure}

Most participants rated both subextraction conditions equally low, see \tabref{fig:exp12-byparticipant}.
Participants 11, 29, 35 and 83 show a very strong preference for extraction out of the object (rated at the top of the scale). Their ratings create the advantage for extraction out of the object
Only one participant, 27, has a relatively weak preference for  extraction out of the subject. 

\begin{figure}
    \centering
    \includegraphics[width=\textwidth]{chapters/part2-Empirical/Exp12-dequi-whq/by-participant.jpeg}
    \caption{Ratings of the subextraction conditions for each participant in Experiment~12.}
    \label{fig:exp12-byparticipant}
\end{figure}

We observed similar patterns in Experiment~10. \figref{fig:exp10-byparticipant} on page \pageref{fig:exp10-byparticipant} shows the results by participants for Experiment~10, and, as in Experiment~12, some of them strongly disliked subextractions from subjects while most rated them on a relatively same level of the scale, but only two participants (participants 46 and 48) showed a preference for subextractions from subjects. The conclusion is that the significant interaction effect is probably due to only a few participants, but that the disadvantage of subextraction from subjects is still robust among participants, and can by no means be considered an artefact of some participants being outliers. 

\begin{figure}
    \centering
    \includegraphics[width=\textwidth]{chapters/part2-Empirical/Exp10-dequel-whq/by-participant.jpeg}
    \caption{Ratings of the subextraction conditions for each participant in Experiment~10.}
    \label{fig:exp10-byparticipant}
\end{figure}

\subsubsection{Comparing subextraction from the subject with the ungrammatical controls}

We fitted a fourth model to compare the extractions out of the subject and the subject ungrammatical controls on their own (mean centered with subextraction coded positive and ungrammatical coded negative). We included trial number as a covariate, and random slopes for the fixed effect grouped by participants and items. The results of the model are reported in Table \ref{tab:exp12-m4}. 
The difference between the two conditions is not significant.

% latex table generated in R 3.6.3 by xtable 1.8-4 package
% Thu Apr 23 00:04:53 2020
\begin{table}
\begin{tabular}{l S[table-format=1.3] S[table-format=1.3] c S[table-format=<1.3] S[table-format=2.2]}
  \lsptoprule
 & {Estimate} & {SE} & {$z$} & {$\text{Pr}(>|z|)$} & {OR}\\ 
  \midrule
  extraction type & 2.342 & 0.395 & 6 & <.001 & 10.40 \\ 
  trial           & 0.039 & 0.012 & 3 & <.005 & 1.04 \\ 
   \lspbottomrule
\end{tabular}
\caption{Results of the Cumulative Link Mixed Model (model n$^{\circ}$3)}
\label{tab:exp10-m3}
\end{table}


In a fifth model, we compared the subextractions with the ungrammatical controls. We fitted a model crossing syntactic function (mean centered with object coded positive, subject coded negative) and extraction type (grammaticality). We included trial number as a covariate, and random slopes for all fixed effects grouped by participants and items. The results of the model are reported in Table \ref{tab:exp12-m5}. 
There is a significant main effect of extraction type (in favor of the extraction conditions) and of trial (habituation). There is also a significant interaction: extractions out of the object are rated higher than all other conditions.

% latex table generated in R 3.6.3 by xtable 1.8-4 package
% Sun Apr 26 23:02:15 2020
\begin{table}
\begin{tabular}{l S[table-format=1.3] S[table-format=1.3] c S[table-format=<1.3] S[table-format=1.2]}
  \lsptoprule
 & {Est.} & {SE} & {$z$} & {$\text{Pr}(>|z|)$} & {OR} \\ 
  \midrule
  syntactic function & 0.474 & 0.465 & 1 & 0.307 & 1.61 \\ 
  trial & 0.034 & 0.014 & 2 & <.05 & 1.03 \\ 
  syntactic function:trial & 0.014 & 0.015 & 1 & 0.342 & 1.01 \\ 
   \lspbottomrule
\end{tabular}
\caption{Results of the Cumulative Link Mixed Model (model n$^{\circ}$2)}
\label{tab:exp16-m2}
\end{table}


\subsection{Discussion}

We see in this experiment the expected pattern of a ``subject island'' effect: extraction out of the subject was rated lower than extraction out of the object (model n$^{\circ}$2) and there was a significant interaction compared to non-extraction controls (model n$^{\circ}$3). 

The low acceptability of both subextraction conditions is striking, however. Subextractions with \emph{de quel} + N, even though degraded compared to the non-extraction conditions, were not so disfavored. Even though there was a main effect of extraction type such than subextractions received higher ratings than ungrammatical controls (model n$^{\circ}$5), this main effect comes from subextraction from objects. The mean rating for the subextraction from the subject is actually lower than for its ungrammatical control (model n$^{\circ}$4) and there is no significant difference between the two.
As far as model n$^{\circ}$4 is concerned, the results are closer to the expectations under the traditional syntactic approach than to the other accounts, even though null-effects are not able to falsify the predictions of the other accounts. The significant interaction in the comparison of subextractions with ungrammatical controls (model n$^{\circ}$5) is also only predicted by the syntactic account. However, the very low ratings for extracting out of objects with \emph{de qui} is surprising under all accounts.

To make sense of this finding, recall the difference in acceptability between relative clauses with \emph{dont} and with \emph{de qui}. The subextraction conditions seemed much degraded in the \emph{de qui} relative clauses, regardless of the syntactic function. We see a similar contrast between \emph{de quel} and \emph{de qui} interrogatives. The significant interaction found in model n$^{\circ}$5 can be explained as a product of superadditivity: if it is inherently difficult to extract out of a NP with \emph{de qui}, and this processing difficulty is compounded with difficulties discussed earlier (clash in discourse status following the FBC constraint, processing difficulty of extraction on its own, weak specificity of the pronoun), the parser may reach a point where processing the sentence is not possible anymore (at least for some parsers). Assuming that \emph{de qui} subextractions are difficult per se is actually consistent with the results of the corpus studies. We have seen that \emph{de qui} is not very frequent in general, that there are fewer subextractions from NP with \emph{e qui} than with \emph{dont} (the occurrences are overwhelmingly by one single author for 2000--2013) and that subextraction from NPs in general, and not just subextraction from subjects, is almost completely absent in interrogatives. A mere frequency effect could lead to surprisal and therefore to processing difficulties. Furthermore, this infrequency could in fact be the symptom of another reason that makes \emph{de qui} ill-suited for subextractions.
