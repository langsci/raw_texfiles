\section[head=Experiment 8]{Experiment 8: Acceptability judgment study on \emph{dont} relative clauses with an animate antecedent and animate subject and object}

We built Experiment~8 as a parallel experiment to Experiment~7, in order to make sure that the results are similar when the relative word is \emph{dont}. 

\subsection{Design and materials}
We used the same stimuli and the same design as in Experiment~7, but with \emph{dont} instead of \emph{de qui} for the subextraction conditions. The ungrammatical controls were constructed with \emph{que}, as in Experiment~4. 

\eal 
\ex[]{{Condition subject + PP-extracted:}\nopagebreak\\
\gll  J' ai pris un avocat, dont l' associé aide mon cousin sans contrepartie financière.\\
I have taken a lawyer of.which the associate helps my cousin without counterpart financial\\
\glt `I took a lawyer of which the associate helps my cousin without financial compensation.'}
\label{ex:exp08-subj-pp}
\ex[]{{Condition object + PP-extracted:}\\
\gll  J' ai pris un avocat, dont mon cousin aide l' associé sans contrepartie financière.\\
I have taken a lawyer of.which my cousin helps the associate without counterpart financial\\
\glt `I took a lawyer of which my cousin helps the associate without financial compensation.'}
\label{ex:exp08-obj-pp}
\pagebreak
\ex[]{{Condition subject + noextr:}\\
\gll J' ai pris un avocat, et l' associé de cet avocat aide mon cousin sans contrepartie financière.\\
I have taken a lawyer and the associate of this lawyer helps my cousin without counterpart financial\\
\glt `I took a lawyer and the associate of this lawyer helps my cousin without financial compensation.'}
\label{ex:exp08-subj-no}
\ex[]{{Condition object + noextr:}\nopagebreak\\
\gll J' ai pris un avocat, et mon cousin aide l' associé de cet avocat sans contrepartie financière.\\
I have taken a lawyer and my cousin helps the associate of this lawyer without counterpart financial\\
\glt `I took a lawyer and my cousin helps the associate of this lawyer without financial compensation.'}
\label{ex:exp08-obj-no}
\ex[]{{Condition subject + ungrammatical:}\\
\gll J' ai pris un avocat, que l' associé aide mon cousin sans contrepartie financière.\\
I have taken a lawyer that the associate helps my cousin without counterpart financial\\
\glt `I took a lawyer that the associate helps my cousin without financial compensation.'}
\label{ex:exp08-subj-un}
\ex[]{{Condition object + ungrammatical:}\\
\gll J' ai pris un avocat, que mon cousin aide l' associé sans contrepartie financière.\\
I have taken a lawyer that my cousin helps the associate without counterpart financial\\
\glt `I took a lawyer that my cousin helps the associate without financial compensation.'}
\label{ex:exp08-obj-un}
\zl 

The 20 experimental items were the same as in Experiment~7, but the distractors were different. We used 36 distractors, some of which were ungrammatical. Half of the experimental items and distractors were followed by a comprehension question. The item presented here as an example was paired with the comprehension question \emph{Cet avocat a-t-il un associé~?} (`Does this lawyer have an associate?'). 

\subsection{Predictions}

The predictions for this experiment are similar to those for the previous experiment. They are summarized in Table \ref{tab:exp05-predictions} on page \pageref{tab:exp05-predictions}.

\subsection{Procedure} 

We conducted the experiment on the Ibex platform \citep{Ibex}. The procedure was similar to the procedure used in the previous acceptability judgment experiments (see Section \ref{ch:methodo-AJ}). Participants rated the sentences on a Likert scale from 0 to 10, 0 being labeled as ``bad'' and 10 being labeled as ``good''. They also answered comprehension questions after some of the sentences.

The experiment took approximately 20 minutes to complete.

\subsection{Participants}

The study was run in October 2019.  
Participants were recruited through FouleFactory (\url{https://www.foulefactory.com}), and paid 5€ for their participation. The payment was not contingent on the participants' responses to the questions about native language or place of birth.

61 participants took part in the experiment. The analysis presented here is based on the data from the 52 participants who satisfied all inclusion criteria.\footnote{To calculate accuracy, we excluded not only the answers to comprehension questions of the practice items and of the ungrammatical controls, but also of the ungrammatical distractors.}
The 52 participants were aged 23 to 73 years. 29 of them self-identified as women, and 22 as men. One participant (1.92\%) indicated having an educational background related to language.

\subsection{Results and analysis}

\figref{fig:exp08-boxplot} shows the results of the acceptability judgment task.
In the subextraction conditions, extraction out of the subject (\ref{ex:exp08-subj-pp}), with a mean rating of 7.68, was rated higher than extraction out of the object (\ref{ex:exp08-obj-pp}) with a mean rating of 6.14. With a mean rating of 7.43, the subject control condition without extraction (\ref{ex:exp08-subj-no}) was judged a worse than the corresponding extraction. The object control condition without extraction (\ref{ex:exp08-obj-no}) was judged a bit better than the corresponding extraction with a mean  rating of 6.55. The ungrammatical controls were rated very low: 1.77 in the subject condition (\ref{ex:exp08-subj-un}) and 1.98 in the object condition (\ref{ex:exp08-obj-un}).\largerpage[2.25]

\figref{fig:exp08-boxplot} suggests potential ceiling effects in the extraction and non-ex\-trac\-tion conditions, especially in the subject variant. There is also a possible floor effect in the ungrammatical controls. The ratings for \emph{dont} relative clauses are again very high, but participants seem to have used a wider range of the scale than in Experiment~6. The distribution of the ratings is illustrated by \figref{fig:exp08-repartition}: we see a clear floor effect for the ungrammatical controls, and some ceiling effects in the other conditions, but especially in the subject variants.\pagebreak

\begin{figure}
    \centering
    \includegraphics[width=\textwidth]{chapters/part2-Empirical/Exp08-dont-RC-animacy-match/boxplots.jpeg}
    \caption{Acceptability judgments by condition in Experiment~8. The grey box plots indicate the median and quartiles of the results. Black points are outliers. Mean and confidence intervals are indicated in white.}
    \label{fig:exp08-boxplot}
\end{figure}

\begin{figure}
    \centering
    \includegraphics[width=\textwidth]{chapters/part2-Empirical/Exp08-dont-RC-animacy-match/repartition.jpeg}
    \caption{Density of the ratings across conditions for Experiment~8}
    \label{fig:exp08-repartition}
\end{figure}

Another representation of the results is given by the ROC and zROC curves of the results in \figref{fig:exp08-ROC} on page \pageref{fig:exp08-ROC}. The ROC curves show that participants discriminate between ungrammatical baselines and the other conditions. We observe that the lines group by syntactic function rather than by extraction type: the subject variants build larger curves than the object variants. The zROC curves are relatively straight and parallel to the baseline. 

\begin{figure}
    \centering
    \includegraphics[width=\textwidth]{chapters/part2-Empirical/Exp08-dont-RC-animacy-match/ROC.jpeg}
    \includegraphics[width=\textwidth]{chapters/part2-Empirical/Exp08-dont-RC-animacy-match/zROC.jpeg}
    \caption{ROC curves (top) and zROC curves (bottom) of the non-extraction conditions compared to their respective subextraction conditions, represented by the dotted grey baseline (\citealt{Dillon.2019}'s method) in Experiment~8}
    \label{fig:exp08-ROC}
\end{figure}

The ROC and zROC curves in \figref{fig:exp08-ROC-subj} on page \pageref{fig:exp08-ROC-subj} show the discrimination between the subject and object conditions. Discrimination is almost non-existent in the ungrammatical controls, but more important for the two other conditions. The zROC curves are relatively straight and parallel to the baseline.

\begin{figure}
    \centering
    \includegraphics[width=\textwidth]{chapters/part2-Empirical/Exp08-dont-RC-animacy-match/ROC-subject.jpeg}
    \includegraphics[width=\textwidth]{chapters/part2-Empirical/Exp08-dont-RC-animacy-match/zROC-subject.jpeg}
    \caption{ROC curves (top) and zROC curves (bottom) of the object conditions compared to their corresponding subject condition, represented by the dotted grey baseline (\citealt{Dillon.2019}'s method) in Experiment~8}
    \label{fig:exp08-ROC-subj}
\end{figure}

\subsubsection{Habituation} 

The habituation effects in the course of the experiment are depicted in \figref{fig:exp08-habituation} on page~\pageref{fig:exp08-habituation}. We can see a slight decrease of the ratings during the experiment for extractions out of subjects. Extractions out of objects, by contrast, show a strong habituation effect. Habituation was also strong for the ungrammatical controls, but their acceptability remained very low during the whole experiment. 

\begin{figure}
    \centering
    \includegraphics[width=\textwidth]{chapters/part2-Empirical/Exp08-dont-RC-animacy-match/habituation.jpeg}
    \caption{Changes in the mean acceptability ratings ($z$-scored by participant) by condition in Experiment~8 in the course of the experiment}
    \label{fig:exp08-habituation}
\end{figure}

\subsubsection{Comparing subextraction from the subject with subextraction from the object}

We fitted a first model to compare extractions out of the subject and out of the object on their own (mean centered with subject coded negative and object coded positive). We included trial number as a covariate, and random slopes for the fixed effects and covariates grouped by participants and items. The results of the model are reported in Table~\ref{tab:exp08-m1}. 
There is a significant effect of syntactic function, such that the ratings in the subject condition are significantly higher than in the object condition. There is also a significant effect of trial (habituation).

% latex table generated in R 3.6.3 by xtable 1.8-4 package
% Fri Apr 24 21:31:19 2020
\begin{table}
\begin{tabular}{l S[table-format=1.3] S[table-format=1.4] c S[table-format=<1.3] S[table-format=1.2]}
  \lsptoprule
 & {Estimate} & {SE} & {$z$} & {$\text{Pr}(>|z|)$} & {OR} \\ 
  \midrule
  syntactic function & 0.459 & 0.144 & 3 & <.005 & 1.58 \\ 
  trial              & 0.016 & 0.010 & 2 & 0.1025 & 1.02 \\ 
   \lspbottomrule
\end{tabular}
\caption{Results of the Cumulative Link Mixed Model (model n$^{\circ}$2)}
\label{tab:exp12-m2}
\end{table}


A second model compared extractions out of the subject and out of the object on their own (mean centered with subject coded negative and object coded positive), but this time crossing syntactic function with trial number. We added random slopes for all fixed effects grouped by participants and items. The results of the model are reported in Table \ref{tab:exp08-m1b}. 
The results are in line with model n$^{\circ}$1, and there is no significant interaction between syntactic function and trial number.

% latex table generated in R 3.6.3 by xtable 1.8-4 package
% Thu Jul 16 18:01:08 2020
\begin{table}
\begin{tabular}{l S[table-format=-1.3] S[table-format=1.3] S[table-format=-1] S[table-format=<1.3] S[table-format=1.2]}
  \lsptoprule
 & {Estimate} & {SE} & {$z$} & {$\text{Pr}(>|z|)$} & {Odd.ratio} \\ 
  \midrule
  syntactic function & -1.164 & 0.257 & -5 & <.001 & 3.20 \\ 
  trial & 0.015 & 0.007 & 2 & <.05 & 1.01 \\ 
  syntactic function:trial & 0.013 & 0.007 & 2 & 0.0523 & 1.01 \\ 
   \lspbottomrule
\end{tabular}
\caption{Results of the Cumulative Link Mixed Model (model n$^{\circ}$2)}
\label{tab:exp08-m1b}
\end{table}


In a third model, we compared subextraction with non-extraction. We fitted a model crossing syntactic function and extraction type (mean centered with extraction coded positive, non-extraction coded negative). We included trial number as a covariate, and random slopes for all fixed effects and covariates grouped by participants and items. The results of the model are reported in Table \ref{tab:exp08-m2}. 
There is a significant main effect of syntactic function (in favor of the subject), and of trial (habituation) as well as a significant interaction effect. 
Indeed, \figref{fig:exp08-interaction} shows that the lines cross, even though the confidence intervals overlap. However, if we compare the AUCs (green and red curves on \figref{fig:exp08-ROC-subj}), the difference is not significant. 

% latex table generated in R 3.6.3 by xtable 1.8-4 package
% Mon Apr 13 17:44:16 2020
\begin{table}
\begin{tabular}{l S[table-format=-1.3] S[table-format=1.3] S[table-format=1] S[table-format=1.4] S[table-format=1.2]}
  \lsptoprule
 & {Estimate} & {SE} & {$z$} & {$\text{Pr}(>|z|)$} & {Odd.ratio} \\ 
  \midrule
  syntactic function & -0.070 & 0.324 & -0 & 0.8286 & 1.07 \\ 
  trial              & 0.024 & 0.018 & 1 & 0.1717 & 1.02 \\ 
   \lspbottomrule
\end{tabular}
\caption{Results of the Cumulative Link Mixed Model (model n$^{\circ}$1)}
\label{tab:exp06-m1}
\end{table}


\begin{figure}
    \centering
    \includegraphics[width=\textwidth]{chapters/part2-Empirical/Exp08-dont-RC-animacy-match/interaction.jpeg}
    \caption{Interaction between syntactic function and extraction type in Experiment~8}
    \label{fig:exp08-interaction}
\end{figure}

\subsubsection{Comparing subextraction from the subject with the ungrammatical controls}\largerpage

We fitted a fourth model to compare extractions out of the subject and the ungrammatical subject controls on their own (mean centered with subextraction coded positive and ungrammatical coded negative). We included trial number as a covariate, and random slopes for the fixed effects and covariates grouped by participants and items. The results of the model are reported in Table~\ref{tab:exp08-m3}. There is a significant effect of extraction type (grammaticality), such that ratings for extraction out of the subject are significantly higher than for its ungrammatical control. There is no significant effect of trial.

% latex table generated in R 3.6.3 by xtable 1.8-4 package
% Thu Apr 23 00:04:53 2020
\begin{table}
\begin{tabular}{l S[table-format=1.3] S[table-format=1.3] c S[table-format=<1.3] S[table-format=2.2]}
  \lsptoprule
 & {Estimate} & {SE} & {$z$} & {$\text{Pr}(>|z|)$} & {OR}\\ 
  \midrule
  extraction type & 2.342 & 0.395 & 6 & <.001 & 10.40 \\ 
  trial           & 0.039 & 0.012 & 3 & <.005 & 1.04 \\ 
   \lspbottomrule
\end{tabular}
\caption{Results of the Cumulative Link Mixed Model (model n$^{\circ}$3)}
\label{tab:exp10-m3}
\end{table}


In a fifth model, we compared subextraction with the ungrammatical controls. We fitted a model crossing syntactic function (mean centered with object coded positive, subject coded negative) and extraction type (grammaticality). We included trial number as a covariate, and random slopes for all fixed effects and covariates grouped by participants and items. The results of the model are reported in Table \ref{tab:exp08-m4}. 
There is a significant main effect of syntactic function (in favor of the subject), of extraction type (in favor of the extraction conditions) and of trial (habituation). There is also a significant interaction.

% latex table generated in R 3.6.3 by xtable 1.8-4 package
% Sat Apr 25 19:15:30 2020
\begin{table}
\begin{tabular}{l S[table-format=1.3] S[table-format=1.3] c S[table-format=<1.4] S[table-format=1.2]}
  \lsptoprule
 & {Estimate} & {SE} & {$z$} & {$\text{Pr}(>|z|)$} & {OR} \\ 
  \midrule
  syntactic function & 0.134 & 0.091 & 1 & 0.1405 & 1.14 \\ 
  extraction type & 0.631 & 0.133 & 5 & <.001 & 1.88 \\ 
  trial & 0.025 & 0.005 & 5 & <.001 & 1.03 \\ 
  syntactic function:extraction type & 0.009 & 0.088 & 0 & 0.9142 & 1.01 \\ 
   \lspbottomrule
\end{tabular}
\caption{Results of the Cumulative Link Mixed Model (model n$^{\circ}$5)}
\label{tab:exp13-m5}
\end{table}


\subsection{Discussion}

The results of Experiment~8 are generally in line with the results of Experiment~7. The general advantage for the subject variants was confirmed (main effect of syntactic function in model n$^{\circ}$3 and n$^{\circ}$5). This is surprising given that complex subjects are claimed to be dispreferred compared to complex objects, especially with transitive verbs \citep{Kluender.2004}. 

Participants rated extractions out of the subject significantly better than extractions out of the object in this experiment. This is again in contradiction with what we might expect from a subject island. 

Whether extraction out of the object is more difficult to process is not very clear. We found a significant interaction in model n$^{\circ}$3, but the more conservative method of comparing the AUCs did not yield a significant difference. 

Once again, we find some variability between participants and between items. No item showed a strong preference for extraction out of the object over extraction out of the subject, but many items showed a strong preference in the other direction: the mean rating for extractions out of the subject was more than 2 points higher than the mean rating for extractions out of the object). These items, however, were not necessarily the ones that showed a strong preference for extraction out of the subject in Experiment~7. I conclude that there is no evidence that some items are biased toward one or the other subextraction type, and that the variation between items is probably random. 
