\section[head=Experiment 15]{Experiment 15: Acceptability judgment study on subextraction from infinitivals with \emph{où}}
\label{ch:exp15}

We chose to test infinitival clauses, because extraction out of sentential clauses poses additional problems, as noted by \citet{Kluender.2004}. 
\citet[63]{Godard.1988} argues that extraction out of finite clauses may be ruled out in general. In their corpus study (see Section \ref{ch-candito-2012}), \citet{Candito.2012.ldd} also find that extraction out of non-finite clauses is much more frequent than extraction out of finite clauses: ``We noted that extraction out of finite verbal clause is totally absent from the corpora we've annotated (one case only for over 15000 sentences), but
extraction out of infinitival clause accounts for one third of the occurrences of the relative pronoun \emph{que}''.

Infinitival subjects are also somewhat more frequent, and therefore more familiar. In the French Treebank \citep{Abeille.2019.FTB}, there are only 24 sentential subjects, but 99 infinitival subjects. For comparison, there are 26000 nominal subjects in the same corpus.

\subsection{Design and materials}

The experiment used an acceptability judgment task with a 2*2 design that I explain here in detail. 

We based our design on felicitous examples from English, such as the following, which \citet[105]{Chaves.2020.UDC} attribute to \citet[72]{Grosu.1981}:
% check grosu

\eal
\ex[]{The `Hunan' restaurant is a place [where]$_i$ [having dinner~\trace{}$_i$] promises
to be most enjoyable.}
\ex[]{The pre-midnight hours are the time [when]$_i$ [sleeping soundly~\trace{}$_i$] is
most beneficial to one’s health.}
\zl 

For this reason, we used verbs that select a locative complement (e.g., \emph{flâner} `wander' or \emph{habiter} `reside') and extract this locative complement with \emph{où} (`where'). The verb appeared in an infinitival subject clause in the subject condition (\ref{ex:exp1-subj-pp}), and in an impersonal construction in the object condition (\ref{ex:exp1-obj-pp}). 
%(Boons, Guillet, Leclère constructions intransitives en français, Droz?): construction transitive locative

\eal 
\ex[]{{Condition subject + \emph{où} extracted:}\\
\gll Amsterdam est connue pour son centre-ville, où$_i$ [flâner~\trace{}$_i$] est charmant lorsqu' il fait beau.\\
Amsterdam is known for its city.center where wander\textsc{.inf} is pleasant when it does nice\\
\glt `Amsterdam is well-known for its city center, where to wander is pleasant when the weather is nice.'}
\label{ex:exp15-subj-pp}
\ex[]{{Condition object + \emph{où} extracted:}\\
\gll Amsterdam est connue pour son centre-ville, où$_i$ il est charmant [de flâner~\trace{}$_i$] lorsqu' il fait beau.\\
Amsterdam is known for its city.center where it is pleasant of wander\textsc{.inf} when it does nice\\
\glt `Amsterdam is well-known for its city center, where it is pleasant to wander when the weather is nice.'}
\label{ex:exp15-obj-pp}
\zl 

We constructed non-extraction controls by using coordination instead of extraction.

\eal \label{ex:exp15-no}
\ex[]{{Condition subject + noextr:}\nopagebreak\\
\gll Amsterdam est connue pour son centre-ville, et y flâner est charmant lorsqu' il fait beau.\\
Amsterdam is known for its city.center and there wander\textsc{.inf} is charming when it does nice\\
\glt `Amsterdam is well-known for its city center, and to wander there is charming when the weather is nice.'}
\label{ex:exp15-subj-no}
\ex[]{{Condition object + noextr:}\\
\gll Amsterdam est connue pour son centre-ville, et il est charmant d' y flâner lorsqu' il fait beau.\\
Amsterdam is known for its city.center and it is charming of there wander\textsc{.inf} when it does nice\\
\glt `Amsterdam is well-known for its city center, and it is charming to wander there when the weather is nice.'}
\label{ex:exp15-obj-no}
\zl 

We created 12 items, each manipulated according to the four conditions just described. In addition, the experiment included 36 distractors. About a third of the experimental items and distractors were followed by a comprehension question. The item presented here as an example was followed by the comprehension question \emph{Est-ce que j'aime me promener dans Amsterdam~?} (`Do I like to take a walk in Amsterdam?').

\subsection{Predictions}

The predictions for this experiment are similar to the predictions for relative clauses in general. They are summarized in Table~\ref{tab:exp05-predictions}, which is reproduced here as \tabref{tab:exp15-predictions}.

% http://www.tablesgenerator.com/#

\begin{sidewaystable}
\oneline{%
\begin{tabular}{llllll}
\lsptoprule
 &
  \multicolumn{5}{c}{{Predictions}} \\ \cmidrule(lr){2-6} 
 &
  \multicolumn{3}{c}{{“subject island” accounts}} &
  \multicolumn{2}{c}{{no-island accounts}} \\ \cmidrule(lr){2-6} 
 &
  {\begin{tabular}[c]{@{}l@{}}“traditional” \\ syntactic account\end{tabular}} &
  {\begin{tabular}[c]{@{}l@{}}processing account \\ with surprisal \\ due to subject \\ complexity\end{tabular}} &
  {\begin{tabular}[c]{@{}l@{}}BCI account\\ (Goldberg 2006)\end{tabular}} &
  {\begin{tabular}[c]{@{}l@{}}account based\\ on linear distance\\ (DG, DLT)\end{tabular}} &
  {\begin{tabular}[c]{@{}l@{}}FBC constraint \\ account\end{tabular}} \\ \midrule
\multicolumn{1}{l}{{\begin{tabular}[c]{@{}l@{}}extractions \\ out of the subject\\ vs. extractions \\ out of the object\end{tabular}}} &
  (a) \textless (b) &
  (a) \textless (b) &
  (a) \textless (b) &
  (a) \textgreater (b) &
  (a) \textless (b) \\ \midrule
\multicolumn{1}{l}{{\begin{tabular}[c]{@{}l@{}}extractions vs. \\ non-extractions\end{tabular}}} &
  \begin{tabular}[c]{@{}l@{}}main effect of extraction\\ + interaction effect \\ such that (a) \textless (b,c,d)\end{tabular} &
  \begin{tabular}[c]{@{}l@{}}main effect of extraction\\ + interaction effect \\ such that (a) \textless (b,c,d)\end{tabular} &
  \begin{tabular}[c]{@{}l@{}}main effect of extraction\\ + interaction effect \\ such that (a) \textless (b,c,d)\end{tabular} &
  \begin{tabular}[c]{@{}l@{}}interaction effect,\\ such that (b) \textless (a,c,d)\end{tabular} &
  \begin{tabular}[c]{@{}l@{}}main effect of extraction\\ + interaction effect \\ such that (a) \textless (b,c,d)\end{tabular} \\ \midrule
\multicolumn{1}{l}{{\begin{tabular}[c]{@{}l@{}}extractions \\ out of the subject\\ vs. ungrammatical\\ controls\end{tabular}}} &
  (a) $\simeq$ (e) &
  (a) \textgreater (e) &
  (a) \textgreater (e) &
  (a) \textgreater (e) &
  (a) \textgreater (e) \\ \midrule
\multicolumn{1}{l}{{\begin{tabular}[c]{@{}l@{}}extractions vs.\\ vs. ungrammatical\\ controls\end{tabular}}} &
  \begin{tabular}[c]{@{}l@{}}interaction effect\\ such that (b) \textgreater (a,e,f)\end{tabular} &
  \begin{tabular}[c]{@{}l@{}}main effect\\ of grammaticality\end{tabular} &
  \begin{tabular}[c]{@{}l@{}}main effect\\ of grammaticality\end{tabular} &
  \begin{tabular}[c]{@{}l@{}}main effect\\ of grammaticality\end{tabular} &
  \begin{tabular}[c]{@{}l@{}}main effect\\ of grammaticality\end{tabular} \\ \lspbottomrule
\end{tabular}}
\caption{Predictions of the different accounts for Experiments 10, 12 and 13 (interrogatives with extraction). \emph{Notes}: (a) Condition subject + PP-extracted (b) Condition object + PP-extracted (c) Condition subject + no extraction (d) Condition object + no extraction (e) Condition subject + ungrammatical (f) Condition object + ungrammatical.}
\label{tab:exp10-predictions}
\end{sidewaystable}


\subsection{Procedure} 

We conducted the experiment on the Ibex platform \citep{Ibex}. The procedure was similar to the previous acceptability judgment experiments (see Section \ref{ch:methodo-AJ}). Participants rated the sentences on a Likert scale from 0 to 10, 0 being labeled as ``bad'' and 10 being labeled as ``good''. They also answered comprehension questions after some of the sentences.

The experiment took approximately 20 minutes to complete. 

\subsection{Participants}

The study was run between October 2018 and January 2019. 
Participants were recruited on the R.I.S.C.\ website (\url{http://experiences.risc.cnrs.fr/}) and through social media (e.g.\ Facebook).
They received no financial compensation. 
 
37 participants took part in the experiment. 
The analysis presented here is based on the data from the 27 participants who satisfied all criteria.
They were aged 18 to 90 years. 15 of them self-identified as women and 11 as men. 10 of them (37.04\%) indicated having an educational background related to language.

\subsection{Results and analysis}

\figref{fig:exp15-boxplot} shows the results of the acceptability judgment task.
All experimental conditions received very high ratings. In the subextraction conditions, the extraction out of the subject (\ref{ex:exp15-subj-pp}) had a mean rating of 7.58, lower than extraction out of the object (\ref{ex:exp15-obj-pp}) with a mean rating of 8.33. The subject control condition (\ref{ex:exp15-subj-no}) has a mean  rating of 8.12, the object control condition (\ref{ex:exp15-obj-no}) a mean  rating of 8.41.

\begin{figure}
    \centering
    \includegraphics[width=\textwidth]{chapters/part2-Empirical/Exp15-infinitival-ou/boxplots.jpeg}
    \caption{Acceptability judgments by condition in Experiment~15. The grey box plots indicate the median and quartiles of the results. Black points are outliers. Mean and confidence intervals are indicated in white.}
    \label{fig:exp15-boxplot}
\end{figure}

Unfortunately, these high ratings suggest that we may have ceiling effects in all conditions. \figref{fig:exp15-repartition} suggests the same, even though the ceiling effects seem more substantial in the object conditions.

\begin{figure}
    \centering
    \includegraphics[width=\textwidth]{chapters/part2-Empirical/Exp15-infinitival-ou/repartition.jpeg}
    \caption{Density of the ratings across conditions for Experiment~15}
    \label{fig:exp15-repartition}
\end{figure}

Another representation of the results is given by the ROC and zROC curves of the results in \figref{fig:exp15-ROC}. The ROC curves show that participants barely discriminated between the subextraction condition (grey baseline) and the non-extraction controls. The zROC curves are relatively straight, but the curve for subjects deviates from the baseline. Following \citet{Dillon.2019}, this can be a visual cue that there is more variance in one condition. This is in line with the box plots in \figref{fig:exp15-boxplot} that show more variability in the ratings for the subextraction from subject condition.\pagebreak

\begin{figure}[ph]
    \centering
    \includegraphics[width=\textwidth]{chapters/part2-Empirical/Exp15-infinitival-ou/ROC.jpeg}
    \includegraphics[width=\textwidth]{chapters/part2-Empirical/Exp15-infinitival-ou/zROC.jpeg}
    \caption{ROC curves (top) and zROC curves (bottom) of the non-extraction conditions compared to their respective subextraction conditions, represented by the dotted grey baseline (\citealt{Dillon.2019}'s method) in Experiment~15.}
    \label{fig:exp15-ROC}
\end{figure}
\pagebreak

The ROC and zROC curves in \figref{fig:exp15-ROC-subj} show the discrimination between the subject and object conditions. We see a weak preference for the two object conditions (curve above the baseline). The zROC curves are relatively straight and parallel to the baseline.

\begin{figure}[h]
    \centering
    \includegraphics[width=\textwidth]{chapters/part2-Empirical/Exp15-infinitival-ou/ROC-subject.jpeg}
    \includegraphics[width=\textwidth]{chapters/part2-Empirical/Exp15-infinitival-ou/zROC-subject.jpeg}
    \caption{ROC curves (top) and zROC curves (bottom) of the object conditions compared to their respective subject conditions, represented by the dotted grey baseline (\citealt{Dillon.2019}'s method) in Experiment~15.}
    \label{fig:exp15-ROC-subj}
\end{figure}

\pagebreak
\subsubsection{Habituation} 

The habituation effects in the course of the experiment are given in \figref{fig:exp15-habituation} on page \pageref{fig:exp15-habituation}. All conditions except the subextraction from objects show some weak habituation. 

\begin{figure}
    \centering
    \includegraphics[width=\textwidth]{chapters/part2-Empirical/Exp15-infinitival-ou/habituation.jpeg}
    \caption{Changes in the mean acceptability ratings ($z$-scored by participant) by condition in the course of Experiment~15}
    \label{fig:exp15-habituation}
\end{figure}

\subsubsection{Comparing subextraction from the subject with subextraction from the object}

We fitted a first model to compare extraction out of the subject and out of the object on their own (mean centered with subject coded negative and object coded positive). We included trial number as a covariate, and random slopes for the fixed effect and covariates by participants and items. The results of the model are reported in Table \ref{tab:exp15-m1}. 
There is a significant main effect of the syntactic function, such that the object condition gets significantly higher ratings than the subject condition.

% latex table generated in R 3.6.3 by xtable 1.8-4 package
% Mon Apr 13 17:44:16 2020
\begin{table}
\begin{tabular}{l S[table-format=-1.3] S[table-format=1.3] S[table-format=1] S[table-format=1.4] S[table-format=1.2]}
  \lsptoprule
 & {Estimate} & {SE} & {$z$} & {$\text{Pr}(>|z|)$} & {Odd.ratio} \\ 
  \midrule
  syntactic function & -0.070 & 0.324 & -0 & 0.8286 & 1.07 \\ 
  trial              & 0.024 & 0.018 & 1 & 0.1717 & 1.02 \\ 
   \lspbottomrule
\end{tabular}
\caption{Results of the Cumulative Link Mixed Model (model n$^{\circ}$1)}
\label{tab:exp06-m1}
\end{table}


In a second model, we compared subextraction with non-extraction. We fitted a model crossing syntactic function and extraction type (mean centered with extraction coded positive, non-extraction coded negative). We included trial number as a covariate, and random slopes for all fixed effects grouped by participants and items. The results of the model are reported in Table \ref{tab:exp15-m2}. 
The results corroborate the observation based on the zROC curves in \figref{fig:exp15-ROC-subj}: there is a significant main effect of syntactic function (in favor of the object). 
However, there is no main effect of extraction type and no significant interaction, even though \figref{fig:exp15-interaction} shows a weak tendency toward an interaction effect. If we compare the AUCs (green and red curves on \figref{fig:exp15-ROC-subj}), the difference is not significant, either. Trial number is also not a significant factor, as \figref{fig:exp15-habituation} shows. 

% latex table generated in R 3.6.3 by xtable 1.8-4 package
% Fri Apr 10 17:22:54 2020
\begin{table}
\begin{tabular}{l S[table-format=-1.3] S[table-format=1.3] S[table-format=3.2] S[table-format=-1] S[table-format=<1.4] S[table-format=1.2]}
  \lsptoprule
 & {Estimate} & {SE} & {df} & {$t$} & {$\text{Pr}(>|t|)$} & {OR} \\ 
  \midrule
(Intercept) & 0.856 & 0.577 & 152.52 & 1 & 0.1403 & 2.35 \\ 
  extraction type & -0.131 & 0.061 & 23.29 & -2 & <.05 & 1.14 \\ 
  distance & 0.169 & 0.105 & 25.66 & 2 & 0.1213 & 1.18 \\ 
  length & 0.557 & 0.014 & 521.85 & 39 & <.001 & 1.74 \\ 
  extraction type:distance & -0.042 & 0.073 & 29.31 & -1 & 0.5741 & 1.04 \\ 
   \lspbottomrule
\end{tabular}
\caption{Results of the Linear Mixed Model (model n$^{\circ}$1)}
\label{tab:exp03-m1}
\end{table}


\begin{figure}
    \centering
    \includegraphics[width=\textwidth]{chapters/part2-Empirical/Exp15-infinitival-ou/interaction.jpeg}
    \caption{Interaction between syntactic function and extraction type in Experiment~15}
    \label{fig:exp15-interaction}
\end{figure}

\subsection{Discussion}

The results of the experiment on infinitival subjects show first of all that this kind of subextractions receives very high ratings. 

In fact, these high ratings cause a problem in the statistical analysis: given all the ceiling effects we cannot be sure that they do not conceal other significant effects. The ratings for extractions out of the subject are significantly lower than those for extractions out of the object, but there is no detectable interaction effect. Quite possibly, adding more participants (the analysis was based on only 27 participants) or additional complexity to the sentences to reduce ceiling effects would make the interaction significant. This is indeed what we did in the following study (Section~\ref{ch:exp16}).

However, the mere fact that we are confronted with ceiling effects in extractions out of infinitival subjects is in my opinion strong evidence that the constraint, if there is one, has nothing to do with a syntactic island. There were no ungrammatical controls in this experiment, but there is no doubt that extraction out of the infinitival subject would have been rated higher than ungrammatical controls.\footnote{For example, the ``ungrammatical'' distractors received a mean rating of 4.17. They are relatively acceptable, because they all involve structures with potential agreement attraction like `\emph{Le commandant des armées ont attaqué la Biélorussie.} `The commander(.\textsc{sg}) of the troops(.\textsc{pl}) attacked(.\textsc{pl}) Belarus'.}


%\citet{Prince.1978}: sentential subjects may be focussed in an it-cleft so long as they are potentiel surface subjects :
%\eal 
%\ex[]{it was that he would say such a thing that surprised me}
%\ex[*]{it was that he was sick that seemed to me}
%\zl 

% If we see a contrast between these extractions and extractions out of finite phrasal subjects, then Ross's rule \ref{rule:Ross-general-output-condition-on-performance} may help to understand what's going on, in addition to Kluender's explanations. He says "gerund" in the rule, but I think this means non-finite. In any case, he doesn't give any examples with infinitives.
% But it doesn't even say that you get interaction in a factorial design, so I'm not talking about it here. 
