\section[head=Experiment~2]{Experiment~2: Speeded acceptability judgment study on \emph{dont} relative clauses with different linear distances}
\label{ch:exp02}

As we just saw, in Experiment~1 the \emph{dont} relative clauses received very high ratings across the board, thus ceiling effects may hide potential interactions. Standard acceptability judgment experiments are untimed and participants can go back and reread earlier parts of the sentence if necessary. This is different in speeded acceptability judgment tasks, where words are presented one at a time. We therefore hoped that this technique would increase processing difficulty in the extraction conditions, thereby making the differences between the (narrow-, medium- and wide-)distance conditions more visible. 
We use stimuli very similar to the materials of Experiment~1 in a new experiment.

\subsection{Design and materials}

In this speeded acceptability judgment experiment, we used the same 3*2 design as in Experiment~1 but slightly changed the materials.\footnotetext{The reason for this change is that the materials were mixed with stimuli from another experiment, which served as distractor items. The other experiment presented question/answer pairs, and we wanted to mimic these structures to ensure that the participants would not be able to distinguish the two sets of items from each other.} For the subextraction conditions, the item began with an open question, whose answer was an NP with a relative clause. Like in Experiment~1, the narrow-distance condition (\ref{ex:exp02-subj-pp}) contained extraction out of the subject, the medium-distance condition (\ref{ex:exp02-obj-clitic-pp}) extraction out of the object with a clitic subject, and the wide-distance condition (\ref{ex:exp02-obj-pp}) extraction out of the object with a nominal subject.

\eal 
\ex[]{{Condition narrow-distance + PP-extracted:}\nopagebreak\\
\gll Quel genre d' innovation ont - ils présentée hier~? Une innovation dont$_i$ [l' originalité~\trace{}$_i$] émerveille mes collègues.\\
which kind of innovation have {} they presented yesterday an innovation of.which the uniqueness delights my colleagues\\
\glt `What kind of innovation did they present yesterday? An innovation of which the uniqueness delights my colleagues.'}
\label{ex:exp02-subj-pp}
\ex[]{{Condition medium-distance + PP-extracted:}\\
\gll Quel genre d' innovation ont - ils présentée hier~? Une innovation dont$_i$ nous apprécions [l' originalité~\trace{}$_i$].\\
which kind of innovation have {} they presented yesterday an innovation of.which we value the uniqueness\\
\glt `What kind of innovation did they present yesterday? An innovation of which we value the uniqueness.'}
\label{ex:exp02-obj-clitic-pp}
\ex[]{{Condition wide-distance + PP-extracted:} \nopagebreak\\
\gll Quel genre d' innovation ont - ils présentée hier~? Une innovation dont$_i$ mes collègues apprécient [l' originalité~\trace{}$_i$].\\
which kind of innovation have {} they presented yesterday an innovation of.which my colleagues value the uniqueness\\
\glt `What kind of innovation did they present yesterday? An innovation of which my colleagues value the uniqueness.'}
\label{ex:exp02-obj-pp}
\zl 

Like in Experiment~1, the non-extraction controls involved coordinations. They were followed by a yes/no question, whose answer was always ``yes''. The distractors were all polar questions, followed by ``yes'' or ``no'' answers in a 1:1 ratio.


\eal 
\ex[]{{Condition narrow-distance + noextr:} \nopagebreak \\
\gll Ils ont présenté une innovation hier~? Oui, et son originalité émerveille mes collègues.\\
they have presented an innovation yesterday yes and its uniqueness delights my colleagues\\
\glt `Did they present an innovation yesterday? Yes, and its uniqueness delights my colleagues.'}
\label{ex:exp02-subj-no}
\ex[]{{Condition medium-distance + noextr:} \nopagebreak\\
\gll Ils ont présenté une innovation hier~? Oui, et nous apprécions son originalité.\\
they have presented an innovation yesterday yes and we value its uniqueness\\
\glt `Did they present an innovation yesterday? Yes, and we value its uniqueness.'}
\label{ex:exp02-obj-clitic-no}
\ex[]{{Condition wide-distance + noextr:}\\
\gll Ils ont présenté une innovation hier~? Oui, et mes collègues apprécient son originalité.\\
they have presented an innovation yesterday yes and my colleagues value its uniqueness\\
\glt `Did they present an innovation yesterday? Yes, and my colleagues value its uniqueness.'}
\label{ex:exp02-obj-no}
\zl 

We tested the same 30 items as in Experiment~1, each manipulated according to the six conditions described above. In addition, the experiment included 28 distractors. 

\subsection{Experimental method}

Speeded acceptability judgment tasks differ from usual acceptability judgment tasks in two ways: the experimental items are displayed on the screen word by word (for a duration between 300 and 400ms per word), and judgments must be made as fast as possible. The decision is binary: participants have to either accept or reject the sentence. 

This kind of task forces participants to rely on their intuition. Indeed, the time at their disposal during reading and the judgment is too short for metalinguistic thinking to take place \citep[5]{Riou.2018}. \citet{Bader.2010} compare speeded acceptability judgment tasks with other methodologies (acceptability judgment task and magnitude estimation) and conclude that it yields compelling results. 

\subsection{Predictions}

Because this experiment used the same design as the previous one, the predictions were also the same. Again, our aim was to compare the predictions of the traditional syntactic account with the predictions of a processing account that only relates on memory costs.

Unlike the the standard acceptability judgment task, participants were asked to make a binary yes/no decision. The underlying assumption is that the more acceptable a condition is the less likely participants will be to reject the items in this condition.

The traditional syntactic account predicts a difference between extraction out of the subject (\ref{ex:exp02-subj-pp}) and extraction out of objects (\ref{ex:exp02-obj-clitic-pp}) and (\ref{ex:exp02-obj-pp}), such that the former should receive significantly more rejections.

By contrast, a processing account based on memory load would expect acceptability to increase as the filler-gap distance decreases. This means that (\ref{ex:exp02-subj-pp}) should receive significantly fewer rejections than (\ref{ex:exp02-obj-clitic-pp}), and (\ref{ex:exp02-obj-clitic-pp}) should in turn receive significantly fewer rejections than (\ref{ex:exp02-obj-pp}). 

%Participants had only a short period of time to either accept of reject the sentence, for this reason, the proportion of fails may also be an interesting indication.

\subsection{Procedure}

We constructed six lists with a Latin square design such that each participant saw each item and distractor in only one condition. The items on each list were in a pseudo-randomized order, to ensure that no two experimental items and no two distractors occurred in a row.

The experiment was conducted at the Laboratoire de Linguistique Formelle (LLF) of the Université Paris Cité.\footnote{When the experiment was run, the university was named Université Paris-Diderot -- Paris VII.} Participants were tested individually in a soundproof room. They received instructions from the experimenter and provided informed consent.

The experiment was run on a computer using the E-Prime software \citep{Eprime}.\footnote{I wish to thank Etienne Riou, who set up the experiment and conducted it at the Université Paris Cité, and Doriane Gras, who also helped with the use of E-Prime.} Experimental items were presented on the screen in a word-by-word fashion. Words were presented for 250ms + 25ms for each character (such that longer words were displayed longer). After the last word of the sentence, participants had to either accept or reject it by pressing the S or the L key, respectively. If no response was provided within 2 seconds, the trial was aborted and the program moved on to the next sentence. Participants could pause between two experimental items.

\subsection{Participants}

The study was conducted in June 2016.  33 participants took part in the experiment, all native speakers and monolinguals. Their were recruited on the R.I.S.C.\ website (\url{http://experiences.risc.cnrs.fr/}). Participants' age ranged from 18 to 75 years old. They received a financial retribution of 5€ for their participation.

\subsection{Results and analysis}

\figref{fig:exp02-boxplot} shows the results of the speeded acceptability judgment task. We can see that all conditions were judged acceptable in more than 3/4 of the cases. In the subextraction conditions, the narrow-distance condition (\ref{ex:exp02-subj-pp}) was accepted in 77\% of the cases, the medium-distance condition (\ref{ex:exp02-obj-clitic-pp}) in 79\% of the cases, and the wide-distance condition (\ref{ex:exp02-obj-pp}) in 79\% of the cases. The control conditions had a slightly higher acceptability: 78\% in the narrow-distance condition (\ref{ex:exp02-subj-no}), 85\% in the medium-distance condition (\ref{ex:exp02-obj-clitic-no}) and 79\% in the wide-distance condition (\ref{ex:exp02-obj-no}). 

\begin{figure}
    \centering
    \includegraphics[width=\textwidth]{chapters/part2-Empirical/Exp02-dont-RC-SAJ/acceptability.jpeg}
    \caption{Mean acceptability judgments (in percentage) by condition of Experiment~2.}
    \label{fig:exp02-boxplot}
\end{figure}

The fail rate is relatively similar across conditions: participants failed to answer in time in less than 5\% of the cases, as shown in \figref{fig:exp02-fails}. The lowest fail rate (1\%) was found in extraction out of the subject (\ref{ex:exp02-subj-pp}).

\begin{figure}
    \centering
    \includegraphics[width=\textwidth]{chapters/part2-Empirical/Exp02-dont-RC-SAJ/fails.jpeg}
    \caption{Mean answer rate by condition of Experiment~2. Non-answers are NA, i.e.\ participants failed to answer in time.}
    \label{fig:exp02-fails}
\end{figure}

\subsubsection{Logistic regression models}\label{ch:logistic-regression-binary}
Unlike acceptability judgments on a Likert scale, the data from this experiment are binary. In order to predict a binary variable, we ran logistic regression models, using the \texttt{glm()} function under R \citep{R}. One prerequisite is the validity of the Gaussian model for the data: it is generally assumed that the regression model is valid if and only if the number of data points is at least 5 times the number of explanatory variables, and the residuals are normally distributed. For all models, we validated the model by performing a residual diagnostic using the R package DHARMa \citep{DARMa}. We only considered the model valid and report it in this work if the residual diagnostic is compelling.

As in the Cumulative Link Mixed Models (see page \pageref{ch:cumulative-link-model}), I include random slopes for all fixed effects grouped by participants and items whenever convergence was achievable, and fit a non-maximal model otherwise.

\subsubsection{Habituation} 

The habituation effects in the course of the experiment are depicted in \figref{fig:exp02-habituation}. Recall that in Experiment~1, the two extractions out of the object show a ``reversed'' habituation, with a decline in acceptability in the course of the experiment (especially the medium condition), which however was not significant. For Experiment~2, the results are very different, because the medium-distance conditions (with and without extraction) display strong habituation. The habituation pattern of the narrow-distance conditions is particularly striking: whereas habituation is small in the non-extraction condition, it seems much stronger in the subextraction condition. 

\begin{figure}
    \centering
    \includegraphics[width=\textwidth]{chapters/part2-Empirical/Exp02-dont-RC-SAJ/habituation.jpeg}
    \caption{Changes in the average acceptability for each condition of Experiment~2 in the course of the experiment.}
    \label{fig:exp02-habituation}
\end{figure}

We fitted a first model to compare the subject conditions (narrow-distance) on their own (mean centered with subextraction coded positive, no extraction coded negative) crossing extraction type with trial number. We included participants and items as random variables. The results of the model are reported in \tabref{tab:exp02-m1}. 
There is no significant main effect or interaction effect: the difference in habituation seen in \figref{fig:exp02-habituation} is not significant.\footnote{Additional models that I do not report here show that there is a significant interaction between syntactic function (distance) and trial number when we compare the narrow-distance and the wide-distance subextraction condition (p $<$ .05), but not when we compare the medium-distance and the wide-distance subextraction condition. This corroborates the idea that extractions out of the subject show habituation effects \citep{Chaves.2019.Frequency}.}

% latex table generated in R 3.6.3 by xtable 1.8-4 package
% Sun Jul 19 15:33:52 2020
\begin{table}
\begin{tabularx}{\textwidth}{Q S[table-format=-1.3] 
                  S[table-format=1.3] 
                  S[table-format=3.2] 
                  S[table-format=-1] 
                  S[table-format=<1.4] 
                  S[table-format=3.2]}
  \lsptoprule
 & {Estimate} & {SE} & {df} & {$t$} & {$\text{Pr}(>|t|)$} & {OR} \\ 
  \midrule
(Intercept) & 5.748 & 0.125 & 82.27 & 46 & <.001 & 313.66 \\ 
  extraction type & -0.031 & 0.022 & 493.48 & -1 & 0.1545 & 1.03 \\ 
  distance & 0.024 & 0.023 & 513.06 & 1 & 0.3118 & 1.02 \\ 
  length & 0.014 & 0.013 & 60.08 & 1 & 0.2921 & 1.01 \\ 
  frequency & -0.001 & 0.000 & 155.47 & -2 & 0.1356 & 1.00 \\ 
  extr.\ type:distance & -0.042 & 0.022 & 494.78 & -2 & 0.0585 & 1.04 \\ 
   \lspbottomrule
\end{tabularx}
\caption{Results of the Linear Mixed Model (model n$^{\circ}$8)}
\label{tab:exp03-m8}
\end{table}


\subsubsection{Comparing the narrow-distance condition with the wide-distance condition}

We fitted a second model to compare the extractions out of the subject and out of the object on their own (mean centered with subject coded negative and object coded positive). Trial number was included as a covariate, and participants and items as random variables. The results of the model are reported in \tabref{tab:exp02-m2}. 
There is a significant main effect of trial (habituation), but no main effect of syntactic function.

% latex table generated in R 3.6.3 by xtable 1.8-4 package
% Fri Apr 24 21:31:19 2020
\begin{table}
\begin{tabular}{l S[table-format=1.3] S[table-format=1.4] c S[table-format=<1.3] S[table-format=1.2]}
  \lsptoprule
 & {Estimate} & {SE} & {$z$} & {$\text{Pr}(>|z|)$} & {OR} \\ 
  \midrule
  syntactic function & 0.459 & 0.144 & 3 & <.005 & 1.58 \\ 
  trial              & 0.016 & 0.010 & 2 & 0.1025 & 1.02 \\ 
   \lspbottomrule
\end{tabular}
\caption{Results of the Cumulative Link Mixed Model (model n$^{\circ}$2)}
\label{tab:exp12-m2}
\end{table}


A third model crossed distance and extraction type (mean centered with extraction coded positive, non-extraction coded negative). We included trial number as a covariate, and participants and items as random variables. The results of the model are reported in \tabref{tab:exp02-m3}. 
There is no significant main effect, and no interaction effect. The interaction is shown in \figref{fig:exp02-interaction1} and indeed all conditions seem to have a similar acceptability rate.

% latex table generated in R 3.6.3 by xtable 1.8-4 package
% Mon Apr 13 17:44:16 2020
\begin{table}
\begin{tabular}{l S[table-format=-1.3] S[table-format=1.3] S[table-format=1] S[table-format=1.4] S[table-format=1.2]}
  \lsptoprule
 & {Estimate} & {SE} & {$z$} & {$\text{Pr}(>|z|)$} & {Odd.ratio} \\ 
  \midrule
  syntactic function & -0.070 & 0.324 & -0 & 0.8286 & 1.07 \\ 
  trial              & 0.024 & 0.018 & 1 & 0.1717 & 1.02 \\ 
   \lspbottomrule
\end{tabular}
\caption{Results of the Cumulative Link Mixed Model (model n$^{\circ}$1)}
\label{tab:exp06-m1}
\end{table}


\begin{figure}
    \centering
    \includegraphics[width=\textwidth]{chapters/part2-Empirical/Exp02-dont-RC-SAJ/interaction1.jpeg}
    \caption{Interaction between distance and extraction type in Experiment~2. The graph only shows the narrow-distance and wide-distance conditions.}
    \label{fig:exp02-interaction1}
\end{figure}

\subsubsection{Comparing the narrow-distance condition with the medium-distance condition}
\largerpage[2.25]

We fitted a fourth model to compare the extractions out of the subject and out of the object on their own (mean centered with subject coded negative and object coded positive). Trial number was included as a covariate, and participants and items as random variables. The results of the model are reported in \tabref{tab:exp02-m4}. 
There is a main effect of habituation (trial), but no significant difference between subject and object.

% latex table generated in R 3.6.3 by xtable 1.8-4 package
% Thu Apr 23 00:04:53 2020
\begin{table}
\begin{tabular}{l S[table-format=1.3] S[table-format=1.3] c S[table-format=<1.3] S[table-format=2.2]}
  \lsptoprule
 & {Estimate} & {SE} & {$z$} & {$\text{Pr}(>|z|)$} & {OR}\\ 
  \midrule
  extraction type & 2.342 & 0.395 & 6 & <.001 & 10.40 \\ 
  trial           & 0.039 & 0.012 & 3 & <.005 & 1.04 \\ 
   \lspbottomrule
\end{tabular}
\caption{Results of the Cumulative Link Mixed Model (model n$^{\circ}$3)}
\label{tab:exp10-m3}
\end{table}


A fifth model crossed distance and extraction type (mean centered with extraction coded positive, non-extraction coded negative). As in the previous analyses, we included trial number as a covariate, and participants and items as random variables. The results of the model are reported in \tabref{tab:exp02-m5}. 
There is again a main effect of habituation (trial), but no other main effect, and no interaction effect. The interaction is shown in \figref{fig:exp02-interaction2}: there is a slight tendency toward a penalty for extracting out of the object (medium-distance), but it is not significant.

% latex table generated in R 3.6.3 by xtable 1.8-4 package
% Sat Apr 25 19:15:30 2020
\begin{table}
\begin{tabular}{l S[table-format=1.3] S[table-format=1.3] c S[table-format=<1.4] S[table-format=1.2]}
  \lsptoprule
 & {Estimate} & {SE} & {$z$} & {$\text{Pr}(>|z|)$} & {OR} \\ 
  \midrule
  syntactic function & 0.134 & 0.091 & 1 & 0.1405 & 1.14 \\ 
  extraction type & 0.631 & 0.133 & 5 & <.001 & 1.88 \\ 
  trial & 0.025 & 0.005 & 5 & <.001 & 1.03 \\ 
  syntactic function:extraction type & 0.009 & 0.088 & 0 & 0.9142 & 1.01 \\ 
   \lspbottomrule
\end{tabular}
\caption{Results of the Cumulative Link Mixed Model (model n$^{\circ}$5)}
\label{tab:exp13-m5}
\end{table}


\begin{figure}
    \centering
    \includegraphics[width=\textwidth]{chapters/part2-Empirical/Exp02-dont-RC-SAJ/interaction2.jpeg}
    \caption{Interaction between distance and extraction type in Experiment~2. The graph only shows the narrow-distance and medium-distance conditions.}
    \label{fig:exp02-interaction2}
\end{figure}

\subsubsection{Comparing the medium-distance condition with the wide-distance condition}
\begin{sloppypar}
We fitted a sixth model to compare the extractions out of the object with a clitic subject and with a nominal subject on their own (mean centered with clitic subject coded negative and nominal subject coded positive). We included trial number as a covariate, and random slopes for all fixed effects and covariates grouped by participants and items. The results of the model are reported in \tabref{tab:exp02-m6}.  There is a main effect of habituation (trial) but the difference between subject and object is not significant.
\end{sloppypar}

% latex table generated in R 3.4.4 by xtable 1.8-4 package
% Sat Mar 28 14:48:29 2020
\begin{table}
\begin{tabular}{l S[table-format=-1.3] S[table-format=1.3] S[table-format=-1] S[table-format=1.4] S[table-format=1.2]}
  \lsptoprule
 & {Estimate} & {SE} & {$z$} & {$\text{Pr}(>|z|)$} & {Odd.ratio} \\ 
  \midrule
  distance & -0.003 & 0.063 & -0 & 0.9604 & 1.00 \\ 
  trial    & -0.001 & 0.004 & -0 & 0.8379 & 1.00 \\ 
 \lspbottomrule
\end{tabular}
\caption{Results of the Cumulative Link Mixed Model (model n$^{\circ}$5)}
\label{tab:exp1-m5}
\end{table}


The last model crossed distance and extraction type (mean centered with extraction coded positive, non-extraction coded negative). We included trial number as a covariate, and random slopes for all fixed effects and covariates grouped by participants and items. The results of the model are reported in \tabref{tab:exp02-m7}. As in model n$^{\circ}$5, there is a significant main effect of habituation (trial), but no other significant main effect or interaction effect. The interaction is shown in \figref{fig:exp02-interaction3}: there is again a slight tendency toward a penalty for extracting out of the medium-distance object, but it is not significant.

% latex table generated in R 3.6.3 by xtable 1.8-4 package
% Sun Jul 19 14:56:20 2020
\begin{table}
\begin{tabular}{l S[table-format=-1.4] S[table-format=1.4] S[table-format=-1.4] S[table-format=<1.4] S[table-format=2.2]}
  \lsptoprule
 & {Estimate} & {SE} & {$z$} & {$\text{Pr}(>|z|)$} & {OR} \\ 
  \midrule
(Intercept) & -1.5653 & 0.4991 & -3.1362 & <.005 & 4.78 \\ 
  extraction type & -0.1756 & 0.0673 & -2.6073 & <.01 & 1.19 \\ 
  distance & 0.0849 & 0.067 & 1.2673 & 0.2051 & 1.09 \\ 
  length & 0.0273 & 0.0408 & 0.6683 & 0.504 & 1.03 \\ 
  extraction type:distance & -0.197 & 0.0678 & -2.907 & <.005 & 1.22 \\ 
   \lspbottomrule
\end{tabular}
\caption{Results of the Regression Mixed Model (model n$^{\circ}$6)}
\label{tab:exp03-m6}
\end{table}


\begin{figure}
    \centering
    \includegraphics[width=\textwidth]{chapters/part2-Empirical/Exp02-dont-RC-SAJ/interaction3.jpeg}
    \caption{Interaction between distance and extraction type in Experiment~2. The graph only shows the medium-distance and wide-distance conditions.}
    \label{fig:exp02-interaction3}
\end{figure}

\subsection{Discussion}

Thanks to the speeded nature of the task, we managed to reduce the ceiling effects that we observed in Experiment~1. Unfortunately, all we see in the results are null effects that do not allow us to falsify any predictions. The results of Experiment~2 are overall compatible with any account of extraction out of NPs. In light of Experiment~2, we might even consider the possibility that the advantage for extractions out of the subject in Experiment~1 is an artefact of the ceiling effects.

Nevertheless, the absence of any strong degradation when extracting out of subjects raises suspicion against a syntactic approach, in which clear and categorical judgments are expected: if extraction out of the subject is ruled out by grammar, we expect participants to reject it strongly, and not to accept this type of subextraction in 77\% of the cases. 

The habituation patterns in \figref{fig:exp02-habituation} are compatible with \citegen{Chaves.2014} conclusions. According to them, extractions out of the subject become better over time, but the effects can only be seen after enough exposure to the structure, and our experiment does not have enough items for habituation to be a significant factor. However, this is not compatible with the FBC constraint, because it predicts no special behavior with extractions out of subjects in relative clauses.
