\subsection{Results for \emph{c'est}-clefts}

\textit{C'est}-clefts are only found in Frantext 1900--1913. Of the five hits, three are pied-piping cases like (\ref{ex:dq1900-cleft-piedpiping}) where \emph{de qui} is the complement of a noun complement of a preposition, and two are extractions out of an object NP like (\ref{ex:dq1900-cleft-object}). Two of the pied-piping cases are presentationals.

\eal 
\ex (A.O. Barnabooth, ses oeuvres complètes : le Pauvre chemisier ; Poésies ; Journal intime, Valery Larbaud, 1913)\\
\gll C' était lui, l' ennemi [sur la tête de qui]$_i$ [je devais mettre les charbons ardents~\trace{}$_i$]~!\\
it was him the enemy on the head of who I must put\textsc{.inf} the coal lighted\\
\glt `It was him, the enemy on whose head I had to pour the lighted coal!'
\label{ex:dq1900-cleft-piedpiping}
\ex (La Leçon d'amour dans un parc, René Boylesve, 1902)\\
\gll Alors il inclinait l' entretien sur Châteaubedeau, et c' était celui - là [de qui]$_i$, dans l' ombre, il étranglait [le fantôme~\trace{}$_i$].\\
then he led the conversation on Châteaubedeau and it was this {} there of who in the shadow he choked the ghost\\
\glt `He then led the conversation on Châteaubedeau, and it was this one that he secretly choked the ghost of.' (intended: he secretly choked the ghost of Châteaubedeau)
\label{ex:dq1900-cleft-object}
\zl 