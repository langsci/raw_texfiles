\subsection{Procedure}

In parallel to the two corpus studies conducted on \emph{dont} in Frantext (see Section~\ref{ch:dont-corpus}), we searched two subcorpora of \citep{Frantext}: those for 1900--1913 and for 2000--2013. We found 449 occurrences of \emph{de qui} for 2000--2013 and 271 for 1900--1913. These corpus studies were conducted with Anne Abeillé, the results are published in \citet{Abeille.2020.JFLS}.

Table \ref{tab:dequi-total} shows the distribution of \textit{de qui} in different constructions. In the majority of cases, it is used in a relative with an antecedent. It also frequently occurs in interrogatives, and we can find a few examples of free relative clauses and \emph{c'est} clefts. 

\begin{table}
    \begin{tabular}{lrr}
         \lsptoprule
               & \multicolumn{2}{c}{Frantext}\\\cmidrule(lr){2-3}
               & \multicolumn{1}{c}{2000--2013} & \multicolumn{1}{c}{1900--1913} \\
         \midrule
         relative clauses with an antecedent & 201 & 172 \\
         free relative clauses & 0 & 3 \\
         \emph{c'est} clefts & 0 & 5 \\
         direct and indirect questions & 129 & 70 \\
         noise & 119 & 21 \\
         Total & 449 & 271 \\
         \lspbottomrule
    \end{tabular}
    \caption{Occurrences of \emph{de qui} in Frantext}
    \label{tab:dequi-total}
\end{table}

The three free relatives are all extractions of the complement of the verb. One example is shown in (\ref{ex:dq1900-freerel}).

\ea (Connaissance de l'Est, Paul Claudel, 1907)\\
\gll Heureux [[de qui]$_i$ une parole nouvelle jaillit avec violence~\trace{}$_i$]~!\\
blessed of who a speech new flows.out with violence\\
\glt `Blessed (be the one) from who a new speech flows out violently.'
\label{ex:dq1900-freerel}
\z 

The other occurrences were noise, i.e.\ \emph{qui} free relatives like (\ref{ex:Fr2000-freerel}), and free choice uses like (\ref{ex:Fr2000-freechoice}).

\eal
\ex (Programme sensible, Anne-Marie Garat, 2012)\\
\gll le geste craintif de [qui cherche secours]\\
the gesture fearful of who seeks help\\
\glt `the fearful gesture of who is seeking help'
\label{ex:Fr2000-freerel}
\ex (Signes de vie, le pacte autobiographique 2, Philippe Lejeune, 2005)\\
\gll ne jamais être spécialiste de Proust, ni de [qui que ce soit]\\
\textsc{neg} never be.\textsc{inf} specialist of Proust or of who that it may.be\\
\glt `to never be a specialist of Proust, or of whoever it may be'
\label{ex:Fr2000-freechoice}
\zl 

I will first present the results for the relative clauses, and then for interrogatives and \emph{c'est}-clefts.
