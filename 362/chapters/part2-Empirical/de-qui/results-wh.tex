\subsection{Results for interrogatives}

After excluding verbless and gapless interrogatives, we found 75 interrogatives in Frantext 2000--2013 (33 direct, 42 indirect)\footnote{26 direct questions (78.79\%) and 13 indirect questions (30.95\%) were again from Anne-Marie Garat. She is therefore also relatively overrepresented in the interrogatives, albeit less so than in the relative clauses.}, and 51 interrogatives in Frantext 1900--1913 (32 direct, 19 indirect), see Table \ref{tab:dequi-wh}.

\begin{table}
    \begin{tabular}{lrr}
         \lsptoprule
         & \multicolumn{2}{c}{Frantext}\\\cmidrule(lr){2-3}
         & \multicolumn{1}{c}{2000--2013} & \multicolumn{1}{c}{1900--1913} \\
         \midrule
         direct questions        & 78 & 47 \\
         \quad - with a verb and a gap & 33  & 32  \\
         \quad - verbless & 25  & 13  \\
         \quad - \emph{de qui} in situ & 20  & 2  \\
         indirect questions & 51 & 23 \\
         \quad - with a verb and a gap & 42  & 19  \\
         \quad - verbless & 5  & 4  \\
         \quad - \emph{de qui} in situ & 4  & 0  \\
         Total & 129 & 70 \\
         \lspbottomrule
    \end{tabular}
    \caption{\emph{De qui} interrogatives in Frantext}
    \label{tab:dequi-wh}
\end{table}

Table \ref{tab:dq2000-wh} and \figref{fig:dequi-d2000+1900-wh} on page \pageref{fig:dequi-d2000+1900-wh} summarize the different functions of \emph{de qui} in the corpus. \emph{De qui} can be the complement of a verb (\ref{ex:dq2000-verb-qu}), of a noun (\ref{ex:dq2000-noun-qu}), of an adjective (\ref{ex:dq2000-adjective-qu}) or of a preposition (\ref{ex:dq2000-prep-qu}). 

\begin{table}
    \begin{tabular}{l *2{r@{~}r}}
         \lsptoprule
                   & \multicolumn{4}{c}{Frantext}\\\cmidrule(lr){2-5}
         Frequency & \multicolumn{2}{c}{2000--2013} & \multicolumn{2}{c}{1900-1913} \\\midrule
         Verb & 49 & (65.33\%) & 38 & (74.51\%) \\
         Noun & & \\
         \quad Object              & 2  & (2.67\%) & 0 & \\
         \quad Indirect object     & 3  & (4.00\%) & 0 & \\
         \quad Predicate           & 11 & (14.67\%) & 6 & (11.76\%) \\
         \quad Cplt of Preposition & 7  & (9.33\%)  & 6 & (11.76\%) \\
         Adjective & 1 & (1.33\%) & 1 & (1.96\%) \\
         Preposition & 2 & (2.67\%) & 0 &\\
         \lspbottomrule
    \end{tabular}
    \caption{Distribution of \emph{de qui} interrogatives in Frantext}
    \label{tab:dq2000-wh}
\end{table}

\begin{figure}
    \begin{subfigure}[b]{.49\textwidth}
    \includegraphics[width=\linewidth]{chapters/part2-Empirical/de-qui/dequi-Frantext-2000/distribution-qu.jpeg}
    \caption{Frantext 2000--2013}
    \end{subfigure}
    \begin{subfigure}[b]{.49\textwidth}
    \includegraphics[width=\linewidth]{chapters/part2-Empirical/de-qui/dequi-Frantext-1900/distribution-qu.jpeg}
    \caption{Frantext 1900--1913}
    \end{subfigure}
    \caption{Distribution of \emph{de qui} interrogatives in Frantext}
    \label{fig:dequi-d2000+1900-wh}
\end{figure}

\begin{exe}
\ex Some examples of \emph{de qui} as verb complement\label{ex:dq2000-verb-qu}
\begin{xlist}
   \ex (Les Bienveillantes, Jonathan Littell, 2006)\\
\gll [De qui]$_i$ devons - nous recevoir nos ordres~\trace{}$_i$, à la fin~?\\
of who must {} we receive\textsc{.inf} our orders at the end\\
\glt `From whom should we really get our orders?'
\pagebreak
\ex (Jean-Christophe : Dans la maison, Romain Rolland, 1909)\\
\gll Et lorsqu' il lui demanda [de qui]$_i$ [elle tenait ces détails~\trace{}$_i$], elle lui dit que c' était de Lucien Lévy-coeur~[\dots].\\
and when he her\textsc{.dat} asked of who she hold these details she him\textsc{.dat} said that it was of Lucien Lévy-coeur\\
\glt `And when he asked her from whom she got these details, he said that it was from Lucien Lévy-coeur.' 
\end{xlist}
\end{exe}

\begin{exe}
    \ex Some examples of \emph{de qui} as noun complement\label{ex:dq2000-noun-qu}
    \begin{xlist}
        \ex Object noun:\\
(La vie possible de Christian Boltanski, Christian Boltanski \& Catherine Grenier, 2007)\\
\gll [\dots] [de qui]$_i$ as - tu utilisé [les voix~\trace{}$_i$]~?\\
{} of who have {} you used the voices \\
\glt `Whose voices did you use?'
\ex Indirect object:\\
(La voix des mauvais jours et des chagrins rentrés, Jean-Luc Benoziglio, 2004)\\
\gll [\dots] il a vu [de qui]$_i$ [\dots] cette morne et interminable liste avait le chagrin de faire part de [la soudaine et tragique disparition~\trace{}$_i$]~[\dots].\\
{} he has seen of who {} this bleak and endless list had the grief of do\textsc{.inf} announcement of the sudden and tragic loss\\
\glt `He saw whose sudden and tragic loss this endless, bleak list was very sorry to announce.' \label{ex:frantext2000-dequi-indirectobject}
\ex Predicate noun:\\
(Mon évasion, Benoîte Groult, 2008)\\
\gll [De qui]$_i$ est - ce [la faute~\trace{}$_i$]~?\\
of who is {} it the mistake\\
\glt `Whose mistake is it?'
\ex (Jean-Christophe : L'Adolescent, Romain Rolland, 1905)\\
\gll [De qui] était - il [la proie~\trace{}$_i$]~?\\
of who was {} he the prey\\
\glt `Whose prey was he?'
\ex Noun complement of a preposition:\\
(Dans la main du diable, Anne-Marie Garat, 2006)\\
\gll [\dots] [au service de qui]$_i$ se mettait - il~\trace{}$_i$~? \\
{} at.the service of who \textsc{refl} put {} he\\
\glt `At whose disposal did he put himself?'
\ex (Aimé Pache, peintre vaudois, Charles-Ferdinand Ramuz, 1911)\\
\gll [De la part de qui]$_i$ venez - vous~\trace{}$_i$~?\\
of the behalf of who come {} you\\
\glt `On behalf of whom are you coming?'
    \end{xlist}
\end{exe}

\begin{exe}
    \ex Some examples of \emph{de qui} as adjective complement \label{ex:dq2000-adjective-qu}
    \begin{xlist}
        \ex (L'événement, Annie Ernaux, 2000)\\
        \gll Il voulait savoir [de qui]$_i$ j' étais [enceinte~\trace{}$_i$], depuis quand.\\
he wanted know\textsc{.inf} of who I was pregnant since when\\
\glt `He wanted to know by whom I was pregnant, since when.'
\ex (Le Journal d'une femme de chambre, Octave Mirbeau, 1900)\\
\gll Et [de qui]$_i$ pourriez - vous être [enceinte~\trace{}$_i$], Marianne~?\\
and of who could {} you be\textsc{.inf} pregnant Marianne\\
\glt `And of whom could you be pregnant, Marianne?'
    \end{xlist}
\end{exe}
\pagebreak\largerpage[2]
\ea An example of \emph{de qui} as preposition complement\\
(Un roman russe, Emmanuel Carrère, 2007)\\
\gll Sergueï Sergueïevitch, [à côté de qui]$_i$ es - tu assis~\trace{}$_i$~?\\
Sergueï Sergueïevitch at next of who are {} you sitting\\
\glt `Sergueï Sergueïevitch, next to whom are you sitting?'
\label{ex:dq2000-prep-qu}
\z 

Of course, the absence of any extraction out of the subject among interrogatives is striking, especially because it is the most common use in relative clauses. Such a difference between relative clauses and interrogatives is not expected under \citegen{Tellier.1990} proposal. In general, there is very little extraction out of an NP: 16 occurrences (21.33\% of all interrogatives) in Frantext 2000--2013 and 6 occurrences (11.76\%) in Frantext 1900--1913.

Notice that in Frantext 2000--2013, extraction out of a direct object, out of an indirect object, out of an adjective, and \emph{de qui} as the complement of a preposition are all statistically not higher than zero. In Frantext 1900--1913, extraction out of an adjective is not statistically higher than zero; moreover, there are no examples of extraction out of a direct or indirect object, and no case in which \emph{de qui} is the complement of a preposition.

The by far most common usage for \emph{de qui} in interrogatives is with verb complements. The presence of extractions out of the indirect object is surprising, but all three occurrences are from the same sentence (and no other occurrence can be found in the other corpus studies we conducted). Hence, we can consider this an exceptional case, even though it is an indicator that extractions out of a PP are not utterly ruled out by syntax.
