\subsection{Motivation}
The main aim of this study was to find out whether \emph{de qui} relative clauses exhibit the contrast between subextraction from the subject and subextraction from the object claimed by \citet{Tellier.1990,Tellier.1991}, which \citet{Stepanov.2007} and \citet{Heck.2009} subsequently took for granted in their own analyses. Furthermore, as \emph{de qui} is not restricted to relative clauses and \emph{c'est}-clefts, we can look at another kind of very common extraction: interrogatives. Hence, we wanted to see whether we find any extractions out of the subject with \emph{de qui} in these different constructions, and, if so, whether there is at least a large difference in frequency when compared to extraction out of the object. Two time periods were compared to establish if there has been a change in this usage over the last century, since we want to consider the possibility that \citegen{Tellier.1990} acceptability judgments reflect an older usage of \emph{de qui} and that extraction out of the subject is a rather new innovation in formal French. 

Another aim was to determine if extraction out of subjects is restricted to certain verb types, as assumed by \citet{Chomsky.2008}. If this were the case for \emph{de qui} but not for \emph{dont}, that could indicate that \citeauthor{Chomsky.2008} is right and examples with \emph{dont} are not real cases of subextraction but rather some kind of hanging topic (as claimed by \citealt{Uriagereka.2012}).

Because \emph{de qui} appears in various filler-gap dependencies, we expect differences across constructions. Under a view of islands based on information structure, relative clauses and interrogatives should show different patterns with respect to extraction out of the subject, because the filler in a relative clause is background or topic, whereas the filler in an interrogative is more similar to a focus.