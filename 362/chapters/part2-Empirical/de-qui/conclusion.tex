\subsection{General conclusion on the corpus studies on \emph{de qui}}

Both corpus studies on \emph{de qui} show that extraction out of the subject is frequent in relative clauses. It is not a recent development: in both time periods, extracting out of the subject is more frequent than extracting out of the object (even though this difference is not statistically significant). In this respect, \emph{de qui} as a relative word does not differ from \emph{dont} in our data.

Extraction out of a subject NP is not restricted to a certain verb type. Transitive verbs are found frequently in the construction; however, chi-square tests and regression analyses show that they are significantly less frequent in extraction out of a subject NP than in other usages of \emph{dont} and \emph{de qui}. There are also more passives in extraction out of the subject than in the other kinds of extraction. This difference in frequency may explain the intuition reported in the literature that extraction out of the subject is less natural with transitive verbs and more natural with passives. However, it does not explain why extraction out of the subject of a transitive verb is marked as ungrammatical by some scholars.

\begin{sloppypar}
Both relative words occur in long-distance dependencies, even though they seem to be very rare in Frantext.
In general, \emph{dont} is used far more often than \emph{de qui}, but this seems to be one of the few differences between the two relative words. Extraction out of the subject also tends to involve part-whole relations (especially for body parts)  more often than extraction out of the object. More importantly, the occurrences of \emph{de qui} are due to just a few authors in Frantext, whereas \emph{dont} seems to be more common. Using \emph{de qui} may thus have a stylistic flavor, and this could be one reason for the diverging intuitions about its use.
\end{sloppypar}

We observed a big difference in the usage of \emph{de qui} in relative clauses vs.\ interrogatives. In the latter, \emph{de qui} is used almost exclusively for extraction of the complement of the verb. Extraction out of NPs is rare (whereas it is very frequent in relative clauses), and we did not find any extraction out of a subject NP. This may be related to the fact that subjects tend to be topics, and that focusing a part of a topic would create a discourse clash, making it dispreferred and unlikely to occur in well-edited production data like the texts we can find in Frantext. 
