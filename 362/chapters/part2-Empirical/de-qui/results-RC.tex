\subsection{Results and analysis for relative clauses}

It is remarkable that 149 of the 201 relative clauses with \emph{de qui} (73.13\% of the total) in Frantext 2000--2013 are from one single author, Anne-Marie Garat. This may be a sign that using \emph{de qui} in relative clauses was stylistically marked, at least during this time period. This provided additional motivation to examine whether our results for \emph{de qui} over the period 2000--2013 were independently confirmed for the period 1900--1913, which obviously did not include the author in question. Since the results are consistent, we think that the overrepresentation of \textit{de qui} by this one author does not disqualify the study. 

Table \ref{tab:dq2000} summarizes the different functions of \emph{de qui} in the corpus which are also shown in \figref{fig:dequi-d2000+1900}.\footnote{Data missing in the table are relative clauses without a gap, verbless relative clauses and relative clauses with different gap sites, which we excluded.}

\begin{table}
    \begin{tabular}{l *2{r@{~}r}}
         \lsptoprule
                   & \multicolumn{4}{c}{Frantext}\\\cmidrule(lr){2-5}
         Frequency &  \multicolumn{2}{c}{2000--2013} &  \multicolumn{2}{c}{1900–1913} \\\midrule
         Verb & 33 & (17.01\%) & 31 & (18.45\%) \\
         Noun & & \\
         \quad Subject             & 54 & (27.84\%) & 38 & (22.62\%) \\
         \quad Object              & 30 & (15.46\%) & 15 & (8.93\%) \\
         \quad Predicate           & 8  & (4.12\%)  & 4  & (2.38\%) \\
         \quad Cplt of Preposition & 37 & (19.07\%) & 42 & (25.00\%) \\
         Adjective                 & 2  & (1.03\%)  & 1  & (0.60\%) \\
         Preposition               & 30 & (15.46\%) & 37 & (22.02\%) \\
         \lspbottomrule
    \end{tabular}
    \caption{Distribution of \emph{de qui} relative clauses in Frantext}
    \label{tab:dq2000}
\end{table}

\begin{figure}
    \begin{subfigure}[b]{.5\linewidth}
    \includegraphics[width=\linewidth]{chapters/part2-Empirical/de-qui/dequi-Frantext-2000/distribution-rel.jpeg}
    \caption{Frantext 2000--2013}
    \end{subfigure}\begin{subfigure}[b]{.5\linewidth}
    \includegraphics[width=\linewidth]{chapters/part2-Empirical/de-qui/dequi-Frantext-1900/distribution-rel.jpeg}
    \caption{Frantext 1900--1913}
    \end{subfigure}
    \caption{Distribution of \emph{de qui} relative clauses in Frantext}
    \label{fig:dequi-d2000+1900}
\end{figure}

With the exception of adjuncts, all possible functions of a \emph{de}-PP are found in the subcorpus, pied-piping included. \emph{De qui} can be the complement of a verb (\ref{ex:dq2000-verb}), of a noun (\ref{ex:dq2000-noun}), of an adjective (\ref{ex:dq2000-adjective}) or of a preposition (\ref{ex:dq2000-prep}). 


\ea Some examples of \emph{de qui} as verb complement\label{ex:dq2000-verb}
\ea (L'Enfant d'Austerlitz, Paul Adam, 1902)\\
\gll Sa majesté et monsieur, frère du roi, [de qui] [dépend~\trace{}$_i$ surtout l' octroi du privilège], verraient avec faveur le neveu du postulant près d' entrer au séminaire.\\
His majesty and Monsieur brother of.the king of who relies especially the granting of.the privilege would.see with preference the nephew of.the applicant near of enter\textsc{.inf} at.the seminary\\
\glt `His majesty and Monsieur, brother of the king, on whom especially the granting of the privilege relies, would view favourably the applicant's nephew being about to enter the seminary.'
\ex (Pense à demain, Anne-Marie Garat, 2010)\\
\gll Un homme responsable de ses actes est un homme [de qui]$_i$ [on peut tout craindre~\trace{}$_i$].\\
a man responsible of his actions is a man of who one can everything fear\\
\glt `A man accountable for his actions is a man from whom anything can be feared.'
\z 

\ex Some examples of \emph{de qui} as noun complement\label{ex:dq2000-noun}
\ea Subject noun:\\
(Mes Cahiers : t. 3 : 1902-1904, Maurice Barrès, 1904)\\
\gll Et en effet, il a l' air, maintenant, d' un vieux pigeon [de qui]$_i$ [le coeur~\trace{}$_i$] bat.\\
and in effect he has the air now of an old pigeon of who the heart beats\\
\glt `And indeed he looks now like an old pigeon whose heart is beating.'
\label{ex:dq1900-subj}
\ex (Pense à demain, Anne-Marie Garat, 2010)\\
\gll Elle y a rejoint, un temps, l' exil de sa soeur, [de qui]$_i$ [le vieux mari~\trace{}$_i$] a eu la délicatesse de s' éclipser rapidement.\\
she there has joined a while the exile of her sister of who the old husband has had the
thoughtfulness of \textsc{refl} vanish quickly\\
\glt `She joins there her sister's exile for a while, whose old husband was thoughtful enough to vanish quickly.'
\label{ex:dq2000-subj}
\ex Object noun:\\
(Mes Cahiers : t. 9 : 1911-1912, Maurice Barrès, 1912)\\
\gll Ne sommes - nous pas~[\dots] le peuple [de qui]$_i$ saint Bernard a exprimé [l' âme~\trace{}$_i$], le pays de la chevalerie.\\
\textsc{neg} are {} we not the folk of who saint Bernard has expressed the soul the country of the knighthood\\ 
\glt `Aren't we [\dots] the people whose soul St.\ Bernard gave expression to, the land of knighthood?'
\ex (Vie et mort de Paul Gény, Philippe Artières, 2013)\\
\gll De là je fus conduit à la prison de Regina Coeli où je fus interrogé par le juge instructeur [de qui]$_i$ je rejetai [la première déclaration~\trace{}$_i$].\\
from there I was brought at the prison of Regina Coeli where I was questioned by the judge instructor of who I rejected the first statement\\
\glt `I was brought from there to the Regina Coeli prison where I was questioned by the investigating judge whose first statement I rejected.'
\ex Predicate noun:\\
(Claudine à l'école, Colette, 1900)\\
\gll Mais que j' aime vous entendre et vous voir, vous~[\dots] [de qui]$_i$ je me sens, à chaque instant, [la soeur aînée~\trace{}$_i$]~!\\
but how I love you\textsc{.acc} hear\textsc{.inf} and you\textsc{.acc} see\textsc{.inf} you of who I \textsc{refl} feel at every moment the sister older\\
\glt `But how (much) I love to hear and see you, you of whom I feel every second as (if I were) the oldest sister!'
\ex (Pense à demain, Anne-Marie Garat, 2010)\\
\gll C' est elle qui l' a posée, non sa bru, [de qui]$_i$ c' est pourtant [la dernière trouvaille~\trace{}$_i$].\\
it is her who it\textsc{.acc} has installed not her daughter-in-law of who it is though the last idea\\
\glt `She put it there, not her daughter-in-law, whose latest idea it was, though.'
\ex Noun complement of a preposition:\\
(La Ville [2e version], Paul Claudel, 1901)\\
\gll Vous tous, voyez celui [aux pieds de qui]$_i$ [je me suis mise~\trace{}$_i$]~!\\
you all watch this.one at.the feet of who I \textsc{refl} am put\\
\glt `You all, watch the man at the feet of whom I placed myself.'
\ex (D'autres vies que la mienne, Emmanuel Carrère, 2009)\\
\gll Dans dix ans, Amélie serait une jeune fille [[dans la vie de qui]$_i$ j' aurais peut-être un rôle~\trace{}$_i$]~[\dots]. \\
in ten years Amélie would.be a young woman in the life of who I would.have maybe a role\\
\glt `Ten years from now, Amélie may be a young woman in the life of whom I would maybe play a role.'
\z

\ex Some examples of \emph{de qui} as adjective complement\label{ex:dq2000-adjective}
\ea (Mes Cahiers : t. 2 : 1898-1902, Maurice Barrès, 1902)\\
\gll [\dots] je vois chez lui la haine de l' étranger, du ``monsieur'' [de qui]$_i$ son père semblait [inférieur~\trace{}$_i$].\\
{} I see by him the hatred of the stranger of.the sir of who his father seemed inferior\\
\glt `In him, I see the hatred against the stranger, against this ``sir'' to whom his father seemed inferior.''
\ex (Programme sensible, Anne-Marie Garat, 2012)\\
\gll À quoi sert - il de [\dots] classer la dynastie des criminels [de qui]$_i$ nous sommes [issus~\trace{}].\\
at what helps {} it of {} class\textsc{.inf} the dynasty of.the criminals of who we are originating\\
\glt `What does it help to classify the dynasty of criminals from whom we originate.'
\z 

\ex Some examples of \emph{de qui} as complement of a preposition \label{ex:dq2000-prep}
\ea (Histoire de l'art : L'Art médiéval, Elie Faure, 1912)\\
\gll Il fut la petite église des campagnes [autour de qui]$_i$ [s' assemblaient~\trace{}$_i$ quelques chaumes]~[\dots].\\
it was the small church of.the country around of who \textsc{refl} gathered some thatched.cottages\\
\glt `It was the small country church around which some thatched cottages gathered.'
\ex (À défaut de génie, Fran\c{c}ois Nourissier, 2000)\\
\gll [\dots] que venait - il faire parmi les étudiants anémiés [[autour de qui]$_i$ rôdaient~\trace{}$_i$ les idées noires et les BK]~?\\
{} what came {} he do\textsc{.inf} among the students anemic around of who lurked the thoughts black and the tuberculosis\\
\glt `What was he looking for among the anemic students around whom black thoughts and tuberculosis were lurking?'
\z
\z

We observe a large number of extractions from the NP (subject, object or predicate), many of them extractions from the subject. Pied-piping is frequent as well. However, it is important to note that in Frantext 2000--2013, the majority of these extractions from NPs, including all instances of extractions from the subject (the most common usage), originate from Anne-Marie Garat. Extraction out of the subject is only the second most common usage for 1900--1913, it is nevertheless quite widespread.
Thus, we can be confident that the corpus findings contradict \citegen{Tellier.1990}'s claim that extraction out of the subject is impossible with \emph{de qui}. 

\subsubsection{Subject position}\largerpage

As in the case of \emph{dont}, extraction out of a postverbal subject is very rare, but attested. We find three occurrences, all in the subcorpus 1900--1913, and all involving body parts. One example can be seen in (\ref{ex:dq1900-subj-inv}).  

\ea (L'enfant d'Austerlitz, Paul Adam, 1902)\\
\gll le père Anselme, [de qui]$_i$ voltigeaient [les boucles angéliques~\trace{}$_i$] sur un col gras\\
the father Anselme of who fluttered the curls angelic on a collar oily\\
\glt `Father Anselme, whose angelic curly hair fluttered down on an oily collar'
\label{ex:dq1900-subj-inv}
\z 

Among all extractions out of the subject, there are no long distance dependencies. In three cases, there is a parenthetical adjunct between \emph{de qui} and the subject. One example is reproduced in (\ref{ex:dq2000-subj-incise}). 

\ea (Pense à demain, Anne-Marie Garat, 2010)\\
\gll Eliot Kidman	[de qui]$_i$, se prévalait - il, [la mère~\trace{}$_i$] était une authentique Cheyenne\\
Eliot Kidman of who \textsc{refl} prided {} he the mother was an authentic Cheyenne\\
\glt `Eliot Kidman whose mother, so he prided himself, was an authentic Cheyenne'
\label{ex:dq2000-subj-incise}
\z 

\subsubsection{Verb types}

Table~\ref{tab:FTB-verbtype} shows the verb types involved in extraction out of the subject. Just as in \emph{dont} relative clauses, all kind of verbs are attested. Transitives (\ref{ex:dq2000-subj-trans}), unergatives (\ref{ex:dq1900-subj-unerg}) and state verbs (\ref{ex:dq1900-subj-state}) are frequent. There are only a few occurrences of passives (\ref{ex:dq2000-subj-passive}), unaccusatives (\ref{ex:dq1900-subj-unacc}) and mediopassives (\ref{ex:dq1900-subj-medio}).

\begin{table}
    \begin{tabular}{l *2{r@{~}r}}
         \lsptoprule
                      & \multicolumn{4}{c}{Frantext}\\\cmidrule(lr){2-5} 
         Verb type    & \multicolumn{2}{c}{2000--2013} & \multicolumn{2}{c}{1900--1913} \\\midrule
         Passive      & 3  &  (5.56\%) &  2 & (5.26\%) \\
         Unaccusative & 2  &  (3.70\%) &  2 & (5.26\%) \\
         Mediopassive & 4  &  (7.41\%) &  3 & (7.89\%) \\
         Transitive   & 23 & (42.59\%) & 14 & (36.84\%) \\
         Unergative   & 11 & (20.37\%) & 10 & (26.32\%) \\
         State        & 11 & (20.37\%) &  7 & (18.42\%) \\
         \lspbottomrule
    \end{tabular}
    \caption{Verb types in \emph{dont} relative clauses with extraction out of the subject}
    \label{tab:FTB-verbtype}
\end{table}

\eal 
%\ex[]{\gll l' homme aux yeux durs, [de qui] [les grands pas~\trace{}$_i$] traversaient les pièces\\
%the man at.the eyes hard of who the big steps crossed the rooms\\
%(L'enfant d'Austerlitz, Paul Adam, 1902)\\
%\glt `the man with the hard eyes, whose big steps crossed the rooms'}
%\label{ex:dq1900-subj-trans}
\ex (Pense à demain, Anne-Marie Garat, 2010)\\
\gll lui [de qui]$_i$ [la trogne~\trace{}$_i$] inspire la caricature\\
him of who the face inspires the caricature \\
\glt `him, whose face inspires caricature' (i.e.\ makes people want to caricature it)
\label{ex:dq2000-subj-trans}
\ex (La Leçon d'amour dans un parc, René Boylesve, 1902)\\
\gll [des] femmes de cet âge, [de qui]$_i$ [les charmes~\trace{}$_i$]~[\dots] ont grandi d' année en année\\
\textsc{det} women of this age of who the charms have grown of year in year\\
\glt `women as old, whose charms have grown with the years'
\label{ex:dq1900-subj-unerg}
%\ex[]{\gll son mari excentrique, [de qui]$_i$ [la collection orientale~\trace{}$_i$] dormait au musée de l' Homme.\\
%her husband excentric of who the collection oriental slept at.the Musée de l' Homme\\
%(Pense à demain, Anne-Marie Garat, 2010)\\
%\glt `her excentric husband, whose oriental collection slept in the Musée de l'Homme'}
%\label{ex:dq2000-subj-unerg}
\ex (Le Journal d'une femme de chambre, 1900)\nopagebreak\\
\gll toi [de qui] [l' âme~\trace{}$_i$] est si merveilleusement jumelle de la mienne\\
you of who the soul is so wonderfully twin of the mine\\
\glt `you whose soul is such a wonderful twin of mine'
\label{ex:dq1900-subj-state}
%\ex[]{\gll cette enfant [de qui]$_i$ [l' a\"{i}eule~\trace{}$_i$] était assise près de lui\\
%this girl of who the grandmother was sitting next of him \\
%(L'enfant des ténèbres, Anne-Marie Garat, 2008)\\
%\glt `this child, whose grandmother was sitting next to him'}
%\label{ex:dq2000-subj-state}
%\ex[]{\gll le garçon~[\dots] [de qui]$_i$ [la piété ambitieuse~\trace{}$_i$] était méconnue par le rigorisme de l' église\\
%the boy of who the piety ambitious was disregarded by the rigidity of the church\\
%(L'enfant d'Austerlitz, Paul Adam, 1902)\\
%\glt `the boy whose ambitious piety was disregarded by the rigidity of the Church'}
%\label{ex:dq1900-subj-passive}
\ex (Pense à demain, Anne-Marie Garat, 2010)\\
\gll ces gens [de qui]$_i$ [le nom~\trace{}$_i$] n' a plus été prononcé\\
these people of who the name \textsc{neg} has not.more been spoken \\
\glt `these people whose name was not spoken anymore'
\label{ex:dq2000-subj-passive}
\ex (De Goupil à Margot : histoire de bêtes, Louis Pergaud, 1910)\\
\gll [des] serpents géants~[\dots] [de qui]$_i$ [la tête~\trace{}$_i$] et [la queue~\trace{}$_i$] seraient restées enfouies\\
\textsc{det} snakes giant of who the head and the tail would.be stayed buried\\
\glt `giant snakes whose head and tail would still be buried'
\label{ex:dq1900-subj-unacc}
%\ex[]{\gll ce grand-oncle~[\dots], [de qui]$_i$ [l' unique fille~\trace{}$_i$] vient de temps en temps chez ses parents\\
%this grand-uncle of who the single daughter comes of time in time by his parents\\
%(Pense à demain, Anne-Marie Garat, 2010)\\
%\glt `this grand-uncle whose single daughter comes from time to time to his parents'}
%\label{ex:dq2000-subj-unacc}
\ex (La Vie unanime, Jules Romains, 1908)\\
\gll nous [de qui] [le vouloir~\trace{}$_i$] s' étale dru comme la crinière des bêtes\\
us of who the will \textsc{refl} spread thick like the mane of.the beasts\\
\glt `we whose will is thick like a beast's mane'
\label{ex:dq1900-subj-medio}
%\ex[]{\gll un grand-oncle maternel~[\dots], [de qui]$_i$ [l' unique photo sépia~\trace{}$_i$] finit de se craqueler dans un album\\
%a great-uncle motherly of who the single picture sepia ends of \textsc{refl} crack\textsc{.inf} in an album\\
%(Pense à demain, Anne-Marie Garat, 2010)\\
%\glt `a great-uncle on the mother side, of who the single sepia picture ends to crack in an album' (intended: the sepia picture of the grand-oncle lies almost completely cracked in an album)}
%\label{ex:dq2000-subj-medio}
\zl 

We can compare the verb types attested in extraction out of the subject with those in other kinds of extractions. \figref{fig:dequi-Frantext1900-verbtype} illustrates this for Frantext 1900--1913, but the results are similar for Frantext 2000--2013. There is no significant difference between the two groups, both of which include many transitives and unergatives. Notice that in both groups the frequency of passives and mediopassives is not significantly above zero. Moreover, the frequency of unaccusatives in the relatives with extraction out of the subject is not significantly above zero. This goes against the idea that extraction out of the subject is restricted to passives and unaccusatives: in fact, these verb types are very rare. While I cannot explain this difference compared to \emph{dont} relative clauses (see \figref{fig:dont-FTB-trans}), it seems that the decisive factor is the relative word, not extractions out of the subject.


\begin{figure}
    \includegraphics[width=.7\textwidth]{chapters/part2-Empirical/de-qui/dequi-Frantext-1900/verbtype.jpeg}
    \caption{Frantext 1900--1913: Distribution of the verb types in extraction out of the subject, compared to other extraction types in \emph{de qui} relative clauses. See page~\pageref{ch:conf-intervals-binomial} for the confidence intervals (here six comparisons). The percentage is given for each group (extraction out of the subject vs.\ other extraction types).}
    \label{fig:dequi-Frantext1900-verbtype}
\end{figure}

\begin{figure}
    \includegraphics[width=1\textwidth]{chapters/part2-Empirical/de-qui/verbtype-bin.jpeg}
    \caption{Distribution of the transitive verbs in \emph{de qui} relative clauses. See page~\pageref{ch:conf-intervals-binomial} for the confidence intervals (here two comparisons). The percentage is given for each group (a group = one kind of extraction in one corpus).}
    \label{fig:dequi-trans}
\end{figure}

\figref{fig:dequi-trans} shows the distribution of transitive verbs and other verb types, and we can see that, although trasitive verbs are numerically less frequent in extraction out of the subject than in the other kinds of \emph{de qui} relative clauses, the difference is not significant.\footnote{Pearson's chi-squared Tests performed on each subcorpus confirm that the differences are not significant. Pearson's chi-squared Tests are performed using the function \texttt{chisq.test()} from the R Stats Package \citep{R}.} This contrasts with \textit{dont} relative clauses, where non-transitive verbs were more frequent than transitive verbs in this type of extraction.

\subsubsection{Other factors}

Number and definiteness of the antecedent do not seem to follow any specific pattern. Appendix~\ref{ch:other-factors} provides the interested reader with more details. 

Extraction out of the subject is restrictive in most cases, in contrast with \emph{dont} relative clauses, and contrary to our expectations.

Regarding the most common semantic relations holding between the relative phrase and its head noun, we can see that \emph{de qui} and \emph{dont} relative clauses in Frantext 1900--1913 are remarkably similar. By contrast, the distribution varies in Frantext 2000--2013 for extractions out of the subject: there are more part-whole relations in \emph{dont} relative clauses, more quality (e.g.\ beauty) or relational (e.g.\ mother) relations in \emph{de qui} relative clauses. Given the wide variety of possible semantic relationships, more data would be needed to draw any clear conclusions, but it seems that the use of \emph{dont} and \emph{de qui} has become specialized in the 21st century. 

In \citet{Abeille.2020.JFLS} we attempted to predict the usage of \emph{dont} or \emph{de qui} by applying comprehensive statistical models to our data. 
These models were very exploratory and did not clearly identify one or more decisive factors. I invite the reader to consult this article to learn more. 
