\section{\emph{Avec} + \emph{wh-} in Frantext}
\subsection{Motivation}

All previous corpus studies in this section deal with the extraction of \emph{de}-PPs, and especially with \emph{de}-PP complements of subject head nouns. We showed that extraction out of subjects is attested (and frequent) in relative clauses, and not attested in questions. 

However, it has been argued that the relative word in such relative clauses is not extracted out of the subject, but is instead a sort of hanging topic \citep{Jurka.2010,Uriagereka.2012}. Under such analyses, \emph{de}-PPs are good candidates for being topics because the relation they express is imprecise and can thus be an ``aboutness'' relation. 
% other analysis by Guido Mensching: the \emph{de}-phrase is not a PP, and because of that it has genitive case and can escape the \theta{}-assignment constraints.
If this is true, then other prepositions with a more specific semantic content should not show the same pattern as \emph{de qui}, \emph{duquel} or \emph{dont} in our corpus, and be more similar to interrogatives. 

For this reason, we conducted a last corpus study on \emph{avec} (`with') + \emph{wh-}. Its use should be more restricted than the previous relative and \emph{wh}-words, because it can only be a complement to a small set of nouns. We thus expect to have fewer occurrences overall. The aim of this corpus study was to see if we find the same pattern with \emph{avec} + \emph{-wh} as with \emph{de qui} (i.e.\ extraction out of the subject in relative clauses, with a frequency similar to that of extraction out of the object, and no extraction out of the subject in interrogatives).

\subsection{Procedure}

Again, the corpus Frantext 2000--2013 was used. Because a low frequency of \emph{avec} + \emph{wh} was expected overall, several queries were run: we looked for \emph{avec qui} (`with who'), for \emph{avec quoi} (`with what') and for the lemmas \emph{avec lequel} (`with which'; and its feminine and plural derivates \emph{avec laquelle, avec lesquels} and \emph{avec lesquelles}). In the following, the results of these three queries are combined as a single corpus study.\largerpage

There were 1058 occurrences of \emph{avec} + \emph{wh-} in total, which were annotated as in the previous corpus studies.{\interfootnotelinepenalty=10000\footnote{Esma Tanis, a student assistant in the research program ``Long-distance Dependencies in French: Comparative Analyses (HPSG and the Minimalist Program)'', helped with the annotation.}} We identified 930 relative clauses with an antecedent, five free relatives, 100 direct and indirect questions and eight \emph{c'est}-clefts. The 15 remaining occurrences are noise, i.e.\ \emph{qui} free relatives like (\ref{ex:avecq-freerel}), and free choice uses like (\ref{ex:avecq-freechoice}).

\eal
\ex (Qu'as-tu fait de tes frères~?, Claude Arnaud, 2010)\\
\gll Je veux pouvoir danser jusqu' à l' aube et dormir avec [qui me plaît]~[\dots].\\
I want can\textsc{.inf} dance\textsc{.inf} until at the dawn and sleep\textsc{.inf} with who me\textsc{.acc} appeal\\
\glt `I want to be able to dance until dawn and sleep with whom I like.'
\label{ex:avecq-freerel}
\ex (Ceux qui savent comprendront, Anna Gavalda, 2000)\\
\gll Je déteste me fâcher avec [qui que ce soit].\\
I hate \textsc{refl} quarrel with who that it it\\
\glt `I hate quarreling with whoever it might be.'
\label{ex:avecq-freechoice}
\zl 

I will first present the results for the relative clauses, and then the results of interrogatives and \emph{c'est}-clefts.

\subsection{Results and analysis for relative clauses}

We excluded two gapless relative clauses and one verbless relative clause (a fragment) from the results presented in this section. For the remaining 927 relative clauses, the distribution of \emph{avec} + \emph{wh-} is given in Table \ref{tab:avecq} and in \figref{fig:avecq}.

\begin{table}
    \begin{tabular}{lrr}
         \lsptoprule
              & Frequency & \% \\\midrule
         Verb & 334 & 36.03 \\
         Noun & & \\
         \quad Subject & 15 & 1.62 \\
         \quad Object & 72 & 7.77 \\
         \quad Predicate & 32 & 3.45 \\
         Adjective & 20 & 2.16 \\
         Adjunct & 454 & 48.98 \\
         \lspbottomrule
    \end{tabular}
    \caption{Distribution of \emph{avec} + \emph{wh-} relative clauses in Frantext 2000--2013}
    \label{tab:avecq}
\end{table}

\begin{figure}
    \centering
    \includegraphics[width=\textwidth]{chapters/part2-Empirical/avecqui-Frantext/distribution-rel.jpeg}
    \caption[Distribution of \emph{avec} + \emph{wh-} relative clauses in Frantext 2000--2013]{Distribution of \emph{avec} + \emph{wh-} relative clauses in Frantext 2000--2013. See page~\pageref{ch:conf-intervals-binomial} for the confidence intervals (here six comparisons).}
    \label{fig:avecq}
\end{figure}

\emph{Avec} + \emph{wh-} can thus serve as complement of a verb (\ref{ex:avecq-verb}), of a noun (\ref{ex:avecq-noun}) or of an adjective (\ref{ex:avecq-adjective}), or be an adjunct (\ref{ex:avecq-adjunct}). 

\ea An example of \emph{avec} + \emph{wh-} as verb complement\\
(Retour à Reims, Didier Eribon, 2009)\nopagebreak\\
\gll Le jeune homme [avec qui]$_i$ [elle avait « fauté~»~\trace{}$_i$] ne devait pas être beaucoup plus âgé.\\
the young man with who she had {} sinned \textsc{neg} must\textsc{.past} not be\textsc{.inf} much more old\\
\glt `The young man with whom she had ``sinned'' was probably not much older.'
\label{ex:avecq-verb}
\z 

\begin{exe}
    \ex Examples of \emph{avec} + \emph{wh-} as noun complement\label{ex:avecq-noun}
    \begin{xlist}
        \ex Subject noun:\nopagebreak \\ 
(L'arrivée de mon père en France, Martine Storti, 2008)\\
\gll Pas la même chose d' avoir son beau-frère pour patron plutôt que quelqu'un	[avec lequel]$_i$ [aucun lien~\trace{}$_i$] n' existe~!\\
not the same thing of have\textsc{.inf} his brother-in-law for boss instead that someone with the.which no link \textsc{neg} exists\\
\glt `(It is) not the same thing to have your brother-in-law as a boss instead of someone with whom no link exists!'
\label{ex:avecq-subj}
\ex Object noun:\nopagebreak \\ 
(Impératif catégorique : récit, Jacques Roubaud, 2008)\\
\gll Pourquoi ai - je choisi cette ville, [avec laquelle] je n' avais [aucune attache familiale~\trace{}$_i$]~?\\
why have {} I chosen this city with the.which I \textsc{neg} had no attachment familial\\
\glt `Why did I choose this city, with which I had no family attachment?'
\ex Predicate:\nopagebreak \\ 
(Ils sont votre épouvante et vous êtes leur crainte, Thierry Jonquet, 2006)\\
\gll Tel était l' avis de Samira, [avec laquelle]$_i$ Fatoumata était [en total désaccord~\trace{}$_i$].\\
so was the opinion of Samira with the.which Fatouma was in complete disagreement\\
\glt `This was Samira's opinion, with which Fatouma was completely at odds.'
\label{ex:avecq-pred}
    \end{xlist}
\end{exe}

\ea An example of \emph{avec} + \emph{wh-} as adjective complement\nopagebreak\\
(Fenêtres, Jean-Bertrand Pontalis, 2000)\\
\gll Ce qui est refoulé, c' est~[\dots] ce [avec quoi]$_i$ il est [relié~\trace{}$_i$]~[\dots].\\
it who is repressed it is it with what he is linked\\
\glt `The repressed, it's that with which he is linked.'
\label{ex:avecq-adjective}
\z 

\begin{exe}
\ex Examples of \emph{avec} + \emph{wh-} as an adjunct \label{ex:avecq-adjunct}
    \begin{xlist}
        \ex (Tigre en papier, Olivier Rolin, 2002)\nopagebreak\\
\gll Il passait ses nuits~[\dots] à manipuler un énorme et antique poste de radio	[avec	lequel]$_i$ [il écoutait les ondes révolutionnaires du monde entier~\trace{}$_i$].\\
he spent his nights at manipulate\textsc{.inf} a huge and ancient station of radio with the.which he listened the waves revolutionary of.the world whole\\
\glt `He spent his nights manipulating a huge and ancient radio station with which he listened to the revolutionary waves from the whole world.'
\label{ex:avecq-adjunct-instru}
\ex (Qu'as-tu fait de tes frères~?, Claude Arnaud, 2010)\\
\gll Michel Dalberto, [avec qui]$_i$ [j' allais à l' école enfant~\trace{}$_i$], a donné un premier récital de piano remarqué~[\dots].\\
Michel Dalberto with who I went at the school child has given a first concert of piano notable\\
\glt `Michel Dalberto, with whom I went to school as a child, gave a notable first piano concert.'
\label{ex:avecq-adjunct-attend}
\ex (Ensemble, c'est tout, Anna Gavalda, 2004)\\
\gll Et quand Camille s' étonnait de la rapidité [avec laquelle]$_i$ [ils s' étaient engagés~\trace{}$_i$], ils la regardaient bizarrement.\\
and when Camille \textsc{refl} wonder of the rapidity with the.which they \textsc{refl} were engaged they her\textsc{.acc} looked weirdly\\
\glt `And when Camille wondered about the rapidity with which they got engaged, they looked at her weirdly.'
\label{ex:avecq-adjunct-manner}
    \end{xlist}
\end{exe}

The first noticeable difference between \emph{de}-PP and \emph{avec}-PP is that the latter is mostly used as an adjunct. Indeed, \emph{avec} (`with') is often used either to introduce an instrument (\ref{ex:avecq-adjunct-instru}), a co-attendant to the event (\ref{ex:avecq-adjunct-attend}), or an adverbial of manner (\ref{ex:avecq-adjunct-manner}). 

The second noticeable difference is that there are few extractions out of the NP: only 12.83\% of the relative clauses are of this type.\footnote{Anne-Marie Garat also uses few extractions out of NPs. 13.33\% of the extractions out of a subject, 11.11\% of the extractions out of an object and 3.12\% of the extractions out of a predicate come from her work.} We see for the first time in our corpus studies a significantly lower proportion of extractions out of a subject NP compared to extractions out of an object NP (the confidence intervals do not overlap).

The most common usage \emph{de}-PPs extraction in relative clauses was to extract out of the subject NP. That is not the case for \emph{avec}-PP extraction. However, the frequency of extraction out of subjects is significantly greater than zero, and not significantly lower than extraction from predicates or adjectives (which have never been claimed to be islands to extraction). Notice also, as illustrated by \figref{fig:avecq-subjtype}, that in nominal subjects (i.e.\ subjects that allow extraction), we find more extractions out of the subject than out of the object or the predicate\footnote{All but two are cases of extraction out of a predicate and have a structure similar to (\ref{ex:avecq-pred}), i.e. \emph{en} (`in') + N: \emph{en relation} (`in relation'), \emph{en contact} (`in contact'), \emph{en guerre} (`in war'), etc. The two exceptions are \emph{à l'aise} (`at ease') and \emph{amie} (`friend').}. 

\begin{figure}
    \centering
    \includegraphics[width=\textwidth]{chapters/part2-Empirical/avecqui-Frantext/subjtype2.jpeg}
    \caption{Distribution of \emph{avec} + \emph{wh-} by subject types}
    \label{fig:avecq-subjtype}
\end{figure}

\emph{Avec}-PPs are, as expected, only complement to a small subset of nouns. The ones involved in extraction out of subjects in the corpus are: \emph{combat} (`fight'), \emph{contact} (`contact'), \emph{démarche} (`procedure'), \emph{entente} (`understanding'), \emph{lien} (`link'), \emph{pourparlers} (`negotiations'), \emph{rapport} (`relationship'), \emph{relation} (`relation'), \emph{sympathie} (`sympathy') and \emph{vie} (`life'). Most of them also figure in extractions out of objects. 


\subsubsection{Subject position}

\figref{fig:avecq-inv} 
    \begin{figure}
    \centering
    \includegraphics[width=\textwidth]{chapters/part2-Empirical/avecqui-Frantext/inversion.jpeg}
    \caption{Proportion of subject-verb inversion in \emph{avec} + \emph{wh-} relative clauses}
    \label{fig:avecq-inv}
    \end{figure}
shows the proportion of postverbal subjects in the \emph{avec} + \emph{wh-} relative clauses. Among the extractions out of NPs, there are only two postverbal subjects, and both are extractions out of the subject:

\eal\label{ex:avecq-subj-distance}
\ex (L'enfant des ténèbres, Anne-Marie Garat, 2008)\\
\gll M. Jouvet en personne~[\dots], [avec qui]$_i$ se menaient [les pourparlers~\trace{}$_i$ pour adapter La Machine infernale]\\
Mr Jouvet in person with who \textsc{refl} held the negotiations for adapt\textsc{.inf} the machine infernal\\
\glt `Mr Jouvet in person, with whom negotiations to adapt The Infernal Machine were ongoing.'
\label{ex:avecq-subj-inv}
\ex (La vie sexuelle de Catherine M. précédé de Pourquoi et Comment, Catherine Millet, 2001)\\
\gll ces étrangers [avec lesquels]$_i$ -- c' est le paradoxe de la situation -- pouvait s' engager [une relation~\trace{}$_i$ plus confiante, plus intime, plus intense qu' avec nos amis]\\
these strangers with the.which {} it is the paradox of the situation {} could \textsc{refl} initiate a relation more confident more intimate more intense than with our friends\\
\glt `these strangers with whom -- that's the paradox of this situation -- a more confident, more intimate, more intense relationship could be initiated than with our friends'
\label{ex:avecq-subj-incise}
\zl 

Notice that in example (\ref{ex:avecq-subj-incise}), there is in addition a parenthetical clause between the filler and the rest of the relative clause, increasing the distance between filler and gap. 

\subsubsection{Verb types}

The low number of hits does not allow a statistical analysis of the verb types involved in these relative clauses. We can mention, however, that there are no passives among the hits, but five mediopassives, as in (\ref{ex:avecq-subj-distance}), and one unaccusative. We found two transitives: one example is given in (\ref{ex:avecq-subj-trans}). 


\ea (La vie sexuelle de Catherine M. précédé de Pourquoi et Comment, Catherine Millet, 2001)\\
\gll les personnes [avec qui]$_i$ [la relation~\trace{}$_i$] pouvait prendre un tour sexuel\\
the persons with who the relation could take\textsc{.inf} a turn sexual\\
\glt `the persons with whom the relation could take on a sexual character.'
\label{ex:avecq-subj-trans}
\z  

\subsubsection{Other factors}

Number, definiteness and restrictiveness do not seem to play an important role in distinguishing subjects from objects. Extraction out of NPs seems to be non-restrictive in a higher proportion than in the other kinds of extractions. I report more figures in Appendix~\ref{ch:other-factors}.

\subsection{Results for interrogatives}

There are 100 \emph{avec} + \emph{wh-} interrogatives in Frantext 2000--2013, 66 direct questions and 34 indirect questions. If we exclude the 53 verbless interrogatives and six \emph{avec} + \emph{wh-} in situ, we are left with 41 interrogatives with one gap site, 22 direct questions and 19 indirect questions.

The functions of \emph{avec} + \emph{wh-} in the 41 interrogatives with one gap are given in \tabref{fig:avecq-wh} and on \figref{tab:avecq-wh} page \pageref{tab:avecq-wh}. 

\begin{table}
    \begin{tabular}{l S[table-format=2] S[table-format=2.2]}
         \lsptoprule
                   & {Frequency} & {\%} \\\midrule
         Verb      & 25 & 60.98 \\
         Adjective & 1 & 2.44 \\
         Adjunct   & 15 & 36.59 \\
         \lspbottomrule
    \end{tabular}
    \caption{Distribution of \emph{avec} + \emph{wh-} interrogatives in Frantext 2000--2013}
    \label{tab:avecq-wh}
\end{table}

\begin{figure}
    \centering
    \includegraphics[width=\textwidth]{chapters/part2-Empirical/avecqui-Frantext/distribution-qu.jpeg}
    \caption[Distribution of \emph{avec} + \emph{wh-} interrogatives in Frantext 2000--2013]{Distribution of \emph{avec} + \emph{wh-} interrogatives in Frantext 2000--2013. See page~\pageref{ch:conf-intervals-binomial} for the confidence intervals (here three comparisons).}
    \label{fig:avecq-wh}
\end{figure}

Only three functions are attested: the \emph{avec}-PP is either the complement of a verb (\ref{ex:avecq-verb-qu}) or an adjective (\ref{ex:avecq-adjective-qu}), or is an adjunct (\ref{ex:avecq-verb-adjunct}). Extraction out of NPs does not occur.

\ea  An example of \emph{avec} + \emph{wh-} as verb complement\\
(Un roman russe, Emmanuel Carrère, 2007)\\
\gll [\dots] je ne sais pas [avec qui]$_i$ [tu es~\trace{}$_i$], mais tu n' es pas avec Véro.\\
{} I \textsc{neg} know not with whom you are but you \textsc{neg} are not with Véro\\
\glt `I don't know with whom you are, but you're not with Véro.'
\label{ex:avecq-verb-qu}
\ex An example of \emph{avec} + \emph{wh-} as adjective complement\\
(Mécanique, François Bon, 2001)\\
\gll [\dots] [avec quoi]$_i$ il ne serait pas [d'accord~\trace{}$_i$]~[\dots]~?\\
{} with what he \textsc{neg} would.be not in.agreement\\
\glt `With what would he not be in agreement?'
\label{ex:avecq-adjective-qu}
\pagebreak
\ex An example of \emph{avec} + \emph{wh-} as an adjunct\\
(Les carnets blancs, Mathieu Simonet, 2010)\\
\gll [Avec qui]$_i$ [tu fêtes Noël~\trace{}$_i$]~?\\
with who you celebrate Christmas\\
\glt `With whom are you celebrating Christmas?'
\label{ex:avecq-verb-adjunct}
\z 

\subsection{Results for \emph{c'est}-clefts}

Seven out of the eight \emph{c'est}-clefts are focalizations: the \emph{avec}-PP is a complement of the verb (\ref{ex:avecq-cleft-verb}) or of the object (\ref{ex:avecq-cleft-object}). The only extraction out of a subject is a presentational \emph{c'est}-cleft (\ref{ex:avecq-cleft-subject}).

\eal
\ex (Un roman russe, Emmanuel Carrère, 2007)\\
\gll C' est le genre d' idée [avec quoi]$_i$ [on joue~\trace{}$_i$]~[\dots].\\
it is the kind of idea with what one plays.\\
\glt `It's the kind of ideas with which people play.'
\label{ex:avecq-cleft-verb}
\ex (La vie possible de Christian Boltanski, Christian Boltanski, 2007)\nopagebreak\\
\gll C' est quelqu'un	[avec qui]$_i$ tu peux faire [une sorte de ping-pong mental~\trace{}$_i$]~[\dots].\\
it is someone with who you can do\textsc{.inf} a kind of ping-pong mental\\
\glt `It is someone with whom you can play some kind of mental ping-pong.'
\label{ex:avecq-cleft-object}
\ex (La vie sexuelle de Catherine M. précédé de Pourquoi et Comment, Catherine Millet, 2001)\\
\gll C' était un homme vif et perspicace, [avec qui]$_i$ [les conversations~\trace{}$_i$] allaient bon train~[\dots].\\
it was a man bright and perspicacious with who the discussions went well train\\
\glt `It was a bright and perspicacious man, with whom the discussions were vivid.'
\label{ex:avecq-cleft-subject}
\zl 

\subsection{Conclusion}

This corpus study demonstrates that \emph{avec} + \emph{wh-} is used differently in relative clauses and in interrogatives. In both, speakers use it mostly to construct adjuncts or to extract the complement of a verb, but they only extract out of NPs in relative clauses. In particular, we found no extraction out of a subject NP in interrogatives. In this respect, \emph{avec} + \emph{wh-} patterns with the previous corpus studies.

On the other hand, the results of this study differ from the results of those studies in that there are more occurrences of \emph{avec} + \emph{wh-} extracting out of object NPs than out of subject NPs in relative clauses. There is no clear indication in the corpus for the underlying reason behind this important difference between \emph{avec}-PPs and \emph{de}-PPs. It cannot be satisfactorily explained by syntactic accounts based on subject islands because there are still 15 felicitous cases of extraction out of subject; they are not degraded or especially odd compared to the extractions out of objects. Postulating a subject island would also fail to account for the difference across constructions. 

A more promising explanation is provided by \citet{Kluender.2004}, who relies on several corpus and experimental studies to show that complex subjects are dispreferred in many languages.
% Problem with the definition of complex subjects
Therefore, I propose the following: PP complements of head nouns are perceived as more complex to process when their preposition is semantically more contentful. The preposition \emph{de} is semantically light, and thus an NP containing a \emph{de}-PP is not perceived as complex. For this reason, extracting a \emph{de}-PP does not cause much processing overload in addition to the processing load of the extraction itself. Thus we find the usual subject preference in extraction, i.e. extraction out of the subject is as frequent or more frequent than extraction out of the object. NPs with \textit{avec} complements do not differ syntactically from NPs with \textit{de} complements. However,because the preposition has more semantic content, it is perceived as more complex. The prediction then is that the more complex the NP, the less it tends to be realized as a subject. This could be tested in a corpus study with annotation of the complexity of subjects and objects (regardless of subextraction).\footnote{A complexity scale could be the following: clitics $<$ nouns without complement $<$ nouns with a \emph{de}-complement $<$ nouns with another PP complement $<$ infinitival subjects $<$ sentential subjects.} 
% a possible support for this: Wasow, T. (1997). Remarks on grammatical weight. Language Variation and Change, 9, 81–105.
Such a study is beyond the scope of this work.
