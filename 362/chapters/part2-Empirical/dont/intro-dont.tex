\section{Corpus studies on \emph{dont}}
\label{ch:dont-corpus}
This section summarizes the results of three corpus studies previously published in \citet{Abeille.2016} and \citet{Abeille.2020.JFLS} looking at relative clauses introduced by the relative word (and complementizer) \emph{dont}. I call them \emph{dont} relative clauses in this work. Previously, \citet{Godard.1988} claimed that these relative clauses allow for extraction out of the NP subject. The corpus studies support \citeauthor{Godard.1988}'s claim and go beyond it, showing that this is a common and frequently attested phenomenon in French.

\subsection{Motivation}

\emph{Dont} is very frequent in written (and spoken) French. That is probably the reason why the discussion on extraction out of the subject in French started with examples of \emph{dont}-relative clauses.
As mentioned in Section~\ref{ch:intro-disscussion-French}, \citet{Godard.1988} has shown that \emph{dont} may be the complement of a subject noun, despite the general subject island constraint assumed in the syntactic tradition. However, to the best of my knowledge, \citet{Candito.2012.ldd} is the only corpus study providing quantitative results on \emph{dont} in production data, even though this was not their primary concern. I will briefly present the relevant points of this study in Section~\ref{ch-candito-2012}. 

Our corpus studies on \textit{dont} pursued seven main goals. First, we wanted to have a detailed description of its usage. \citet{Blanche-Benveniste.1990} noticed that \emph{dont} has what she calls a ``fixed usage'' in contemporary spoken French because it occurs almost exclusively as the complement of a verb. If this were the case in written French as well, and if extractions out of an NP were a rare usage of \emph{dont}, any comparison between the frequency of extraction out of the subject NP and the object NP would be irrelevant. We thus wanted to see whether extraction out of NPs is frequent with \emph{dont}, if extractions out of the subject are present, and in what proportion. 

Second, we wanted to describe extraction out of subjects more precisely in order better understand the way it is processed. In particular, we wanted to annotate the position of subjects containing a gap with respect to the verb. Even though French is a SVO language, subject-verb inversion is common in relative clauses. % cite Pozniak here ?
\citet[263]{Lahousse.2011} reports that postverbal subjects are less topical than preverbal ones. Following an approach to filler-gap dependencies based on memory load like the DLT or Dependency Grammar (see Section~\ref{ch:dlt+dg}), one would expect postverbal subjects to be dispreferred when extracting out of the subject, because this construction has an additional intervening referent between the relative word and the gap compared to a preverbal subject, as illustrated in (\ref{ex:ldt-subject}). On the other hand, one would expect a preference for postverbal subjects when extracting the complement of the verb, because this way there is one fewer intervening referent between the relative word and its gap compared to a preverbal subject, as illustrated in (\ref{ex:ldt-verb}).

\eal\label{ex:ldt-subject}
\ex Extraction out of a subject with preverbal subject (one intervening referent):\\
\gll the cat \textit{of} \textit{which} [the owner~\trace{}] disappeared \\
{} {} {} {} {} 1 {} \\
\ex Extraction out of a subject with postverbal subject (two intervening referents):\\
\gll the cat \textit{of} \textit{which} disappeared [the owner~\trace{}]  \\
{} {} {} {} 1 {} 1\\
\zl 

\eal\label{ex:ldt-verb}
\ex Extraction out of a verb with preverbal subject (two intervening referents):\\
\gll the cat \textit{of} \textit{which} Ernest thinks~\trace{} \\
{} {} {} {} 1 1  \\
\ex Extraction out of a verb with postverbal subject (one intervening referent):\\
\gll the cat \textit{of} \textit{which} thinks~\trace{} Ernest \\
{} {} {} {} 1 \\
\zl 


Furthermore, looking for postverbal subjects containing a gap also allows us to test \citegen{Heck.2009} analysis of \emph{dont}. In an attempt to reconciliate \citegen{Godard.1988} counterexamples with the minimalist account of subject islands, \citeauthor{Heck.2009} proposes that \emph{dont} is not an extracted element when it is the complement of the subject, but a specifier of the subject DP. Under his proposal, the whole DP is pied-piped to the edge of the relative clause, as shown in example (\ref{ex-Heck-transitive-subject}). This implies that there are two different usages of \emph{dont} for \citet{Heck.2009}, because there is a filler-gap dependency when \emph{dont} is the complement of a verb or of an object noun (\ref{ex-Heck-transitive-object}).

\eal
\ex \citep[101]{Heck.2009}\\
\gll la fille [dont le frère]$_{\textsc{dp}}$ t' a rencontré\\
the girl of.which the brother you.\textsc{acc} has met \\
\glt `the girl whose brother met you'
\label{ex-Heck-transitive-subject}
\ex \gll la fille dont tu as rencontré [le frère~\trace{}]$_{\textsc{dp}}$\\
the girl of.which you have met the brother\\
\glt `the girl whose brother you met'
\label{ex-Heck-transitive-object}
\zl 

According to Heck's analysis, no material may intervene between \emph{dont} and the subject when \emph{dont} is a specifier as in (\ref{ex-Heck-transitive-subject}): no subject inversion like (\ref{ex-Heck-sv-inversion}) and no long distance dependency like (\ref{ex-Heck-long-distance}) is supposed to be possible when a complement of the subject is relativized. However, such constraints have not been tested empirically.

\eal\label{ex-Heck}
\ex \citep{Heck.2009}\\
* \gll Colin, dont choque la coiffure blonde peroxydée\\
Colin of.which surprises the hair blond bleached\\
\glt `Colin, whose bleached blond hair is shocking'
\label{ex-Heck-sv-inversion}
\ex \citep{Tellier.1991} \\
?? \gll un homme dont je refuse que le fils vous fréquente\\
a man of.which I refuse that the son you.\textsc{acc} dates\\
\glt `a man of whom I refuse that the son dates you'
\label{ex-Heck-long-distance}
\zl 

For this reason, our third goal was to see if we can find extractions out of the subject that are long-distance dependencies. This will be a way to test \citegen{Heck.2009} prediction. 

Fourth, by annotating the kinds of verbs in the corpus, we wanted to see whether extraction out of the subject NP is predominantly found with subjects of passives or unaccusative verbs, or ``internal objects'' as they are sometimes called in the literature. Accounts along the line of \citet{Chomsky.2008} predict that speakers only produce extractions out of ``internal objects'', and not out of subjects which are base-generated in a specifier position.\footnote{See fn.\ \ref{fn:unaccusative} on page \pageref{fn:unaccusative}.}

Fifth, we wanted to examine if extractions out of the subject differ from other uses of \emph{dont} with respect to restrictiveness. We do not know of any accounts predicting that extraction out of the subject is sensitive to this factor, but there are different predictions as far as information structure is concerned for restrictive and non-restrictive relative clauses. According to \citet[181--186]{Song.2017}, the antecedent of a non-restrictive relative clause is the ``aboutness topic'' of its main verb. However, there is ``no additional clue''  for identifying the relation that holds between the antecedent and the main verb of a restrictive relative clause. Indeed, non-restrictive relative clauses like (\ref{ex:song-restr-nr}) can be paraphrased as (\ref{ex:song-restr-paraphrase}) using the test for aboutness topic, whereas restrictive relative clauses like (\ref{ex:song-restr-r}) cannot. 

\begin{exe}
\ex \citep[181]{Song.2017}
\begin{xlist}
\ex[]{Kim chases the dog that likes Lee. \label{ex:song-restr-r}}
\ex[]{Kim chases the dog, which likes Lee. \label{ex:song-restr-nr}}
\ex[]{Kim chases the dog, and speaking of the dog, it likes Lee. \label{ex:song-restr-paraphrase}}
\end{xlist}
\end{exe}

\begin{sloppypar}
Hence, it may be the case that the extracted element in restrictive relative clauses is something other than a topic. If extraction out of the subject requires that the extracted element be non-focal, then we expect to find a higher proportion of non-restrictive relative clauses among extractions out of the subject than in other extraction types. 
% "non-restrictive relatives are almost equivalent to coordinated clauses which clearly involve root phenomena (Heycock 2007: 177) --> (Heycock, Caroline. 2007. Embedded root phenomena. In Martin Everaert & Henk van Riemsdijk (eds.), The blackwell companion to syntax, 174–209. Wiley Online Library.)
% Chaves & Putnam say the opposite : filler of RESTRICTIVE RC is topical
\end{sloppypar}

Sixth, we wanted to look at the distribution of extractions out of the subject in terms of their semantics, specifically, the meaning of the relation expressed by the extracted \textit{de}-complement (spatial or temporal relations, property, possessives, etc.). This is again a more exploratory aspect of the corpus study, given that the various approaches do not make any predictions in this respect.

Our final aim was to distinguish relative clauses with an antecedent from \emph{c'est}-cleft sentences. The latter do involve focalization, and in terms of discourse status, they are closer to \textit{wh}-questions than to relative clauses. We will develop this idea more extensively in Section~\ref{ch:is-clefts}. 


\subsection{A previous exploration of \emph{dont} in the FTB (Candito \& Seddah)}
\label{ch-candito-2012}

\citet{Candito.2012.ldd} examined two corpora, the French Treebank (\citealt{Abeille.2019.FTB}, see our corpus study below) and the Sequoia treebank \citep{Candito.2012.Sequoia}, which contains sentences from the French Wikipedia, from medical texts, from \textit{Europarl} and from the regional newspaper \emph{L’Est Républicain}. They looked for words involved in unbounded dependencies: the clitic \emph{en} (`of it') and \textit{wh}-words (relative and interrogative pronouns and determiners). The relative word \emph{dont} is the second most frequent word in their results after the relative word \emph{que}. They found 501 extractions with \emph{dont}, the three most frequent types being extractions out of subject NPs (251 cases), extractions out of object NPs (29 cases) and extractions out of predicative complements (27 cases). This means that half of the extractions with \emph{dont} are extractions out of subject NPs, a surprisingly high number if one expects these extractions to be banned by syntax. In general, \citeauthor{Candito.2012.ldd} note, using terminology from Dependency Grammar (see Section~\ref{ch:dlt+dg}), that around one third of their extractions (259 out of 618 dependencies) are projective: ``This is because most [long distance dependencies] are extractions from a subject NP [\dots].''
% no page number in Candito & Seddah

Even though \citeauthor{Candito.2012.ldd} do not provide much more detail on their results (the usage of \emph{dont} was not what they were interested in), we can already see that extractions out of the subject, especially with \emph{dont}, are very frequent in the two corpora, given that they represent the majority of unbounded dependencies. 
\subsection{Procedure}

One of the corpus studies, published in \citet{Abeille.2016}, was carried out on the French Treebank (FTB version 1.0, \citealt{Abeille.2019.FTB}). The corpus is a tagged newspaper corpus of 21 550 sentences (about 664 500 words) from the French newspaper \emph{Le Monde} (articles from 1990 to 1993). We explored it with TIGERsearch \citep{Tiger}. For the two other studies, published in \citet{Abeille.2020.JFLS}, we used a larger corpus, Frantext \citep{Frantext}~--- an online collection of texts by various authors in French literature. We explored it with the online search tool of the corpus. Both the FTB and Frantext collect well-edited written productions\footnote{My colleagues and I also presented a corpus study on spontaneous spoken French in \citet{Abeille.2016}.}, but they differ in their typology, since the FTB contains many texts focusing on economical issues, while Frantext often describes protagonists and their interactions. 

The ultimate aim of the two studies in Frantext was to compare \emph{dont} and \emph{de qui} in order to confirm \citeauthor{Tellier.1990}'s (\citeyear{Tellier.1990,Tellier.1991}) intuitions (cf.\ Section~\ref{ch:intro-disscussion-French}). To do this, and because the \emph{de qui} relative clauses only have animate antecedents, we only selected animate antecedents for the \emph{dont} relatives as well. In order to detect changes over time, we selected two similar periods in Frantext: texts published between 1900 and 1913 (179 texts, about 7.8 million words), and texts published between 2000 and 2013 (222 texts, about 13.2 million words).

We looked for occurrences of \emph{dont} in the corpus. Since Frantext contains too many occurrences of this word (see \figref{tab:dont-total}), we only annotated a random subset of the output.

\begin{table}[ht]
\small
    \begin{tabular}{lrrr}
         \lsptoprule
                           &                 & \multicolumn{2}{c}{Frantext}\\\cmidrule(lr){3-4}
                           & French Treebank &  2000--2013 & 1900--1913 \\
         \midrule
         total occurrences of \emph{dont} & 632 & more than 13~000 & close to 10~000 \\
         total annotated         & 632 & 500 & 1300 \\
         among which: & & & \\
         - relative clauses & &  & \\
         with a verb a one gap\footnote{For Frantext: only with animate antecedents, as explained above.} 
                                 & 382 & 123 & 176 \\
         - \emph{c'est} clefts   & 2 & 1 & 3 \\
         \lspbottomrule
    \end{tabular}
    \caption{Occurrences of \emph{dont} in the French Treebank and Frantext}
    \label{tab:dont-total}
\end{table}

These corpora are annotated for part of speech and lemmas. All other annotations had to be done manually. First, it was necessary to remove occurrences that are ``noise'' (i.e.\ false positives of a given query that are not what we were looking for). Second, we wanted to test the impact of certain factors on the results, e.g.\ whether there were any occurrences of extractions out of a subject. To this end, it was necessary to annotate these factors for each occurrence.

We found only 6 \emph{c'est}-clefts among the occurrences we annotated. The two clefts in (\ref{ex:FTB-clefts}) are extractions of the complement of a verb. 

\eal\label{ex:FTB-clefts}
\ex {}[FTB - flmf7ajlep-212]\\
\gll C' est de semi- retraite dont parle M. Tapie.\\
it is of semi retirement of.which talks Mr Tapie\\
\glt `It is semi-retirement that Mr Tapie is talking about.
\ex {}[FTB - flmf3\_11000\_11499ep-11113]\\
\gll C' est de l' indépendance tout court dont il a été finalement question.\\
it is of the independence all short of.which it has been finally question.\\
\glt `In the end, it was simply a matter of independence.'
\zl 

Example (\ref{ex:d1900-clefts-obj}) is extraction out of the object. However, it is presentational, and thus more similar to English \emph{there}-clefts than English \emph{it}-clefts.

\ea (La Mort de Philæ, Pierre Loti, 1909)\nopagebreak\\
\gll C' est la momie d' un embryon humain, dont on avait dans les temps orné [le visage~\trace{}] d' une belle couche d' or~[\dots].\\
it is the mummy of an embryo human of.which one had in the times decorated the face of a nice layer of gold\\
\glt `This is the mummy of a human embryo, of whom someone had decorated the face with a nice layer of gold in a timely manner.'
\label{ex:d1900-clefts-obj}
\z 

The three remaining \emph{c'est}-clefts, reproduced in (\ref{ex:dont-corpus-clefts}), are interesting because they display extraction out of the subject. However, (\ref{ex:d2000-clefts}) is presentational and (\ref{ex:d1900-clefts-subj-2}) is probably presentational as well. Only (\ref{ex:d1900-clefts-subj-1}) is actual focalization by means of extraction. 

\eal \label{ex:dont-corpus-clefts}
\ex(Jean-Christophe : Le Buisson ardent, Romain Rolland, 1911)\\
\gll C' était lui maintenant, dont [les yeux~\trace{}] évitaient les yeux de l' autre.\\
it was him now of.which the eyes avoided the eyes of the other\\
\glt `Now it was him whose eyes avoided the other's eyes.'
\label{ex:d1900-clefts-subj-1}
\ex(Le protocole compassionnel, Hervé Guibert, 2007)\\
\gll C' était donc le jeune homme dont [le livre de chevet~\trace{}] était resté longtemps \emph{Des} \emph{aveugles}.\\
it was then the young man of.which the book of bedside was stayed long of.the blinds\\
\glt `So this is the young man whose bedtime reading had been for a long time \emph{About the blinds}.'
\label{ex:d2000-clefts}
\ex(Sermons, fragments et lettres, Horace Monod, 1911)\\
\gll [\dots] c' est un mourant dont [les traits~\trace{} creusés par la souffrance] s' éclairent d' une céleste joie~[\dots]\\
{} it is a dying of.which the face drilled by the suffering \textsc{refl} shine of an unearthly joy\\
\glt `It's a dying man, whose face, deformed by the suffering light up with an unearthly joy~[\dots].'
\label{ex:d1900-clefts-subj-2}
\zl 

Extraction out of the subject is therefore attested in clefts. Unfortunately, with so little data we cannot draw any further meaningful conclusions. The analysis below does not take the \emph{c'est}-clefts into account.

Furthermore, I only present the analysis for relative clauses with a verb and one gap. The partitive verbless \emph{dont} relative clauses like (\ref{ex:dont-sans-verbe-FTB}) of the French Treebank have already been described by \citet{Bilbiie.2010}. In the present analyses, we ignore gapless \emph{dont} relative clauses like (\ref{ex:dont-pronom-FTB}) and relatives with different gap sites.\footnote{We distinguish between several gaps and different gap sites. If there are several gaps, but with the same syntactic function, like in (\ref{ex:several-gaps-good}), they are included in the results below. If the different gaps have different syntactic functions and are therefore at different gap sites, like in (\ref{ex:several-gaps-bad}), then they are not considered in the result. This holds for for all corpus studies in this work.

\begin{exe}\ex
\begin{xlist}
\ex the car of which [the wheels~\trace{}] and [the brakes~\trace{}] are defective
\label{ex:several-gaps-good}
\ex the car of which [the driver~\trace{}] broke [the wheels~\trace{}]
\label{ex:several-gaps-bad}
\end{xlist}
\end{exe}
}

\ea {}[FTB - flmf7aa1ep-328]\\
\gll En Amérique latine, 23 journalistes ont trouvé la mort, dont 9 en Colombie et 7 au Pérou.\\
in America Latin 23 journalists have found the death of.which 9 in Colombia and 7 in Peru.\\
\glt `In Latin America, 23 journalists have died, among which 9 in Colombia and 7 in Peru.'
\label{ex:dont-sans-verbe-FTB}
\z 

\ea {}[FTB - flmf7ak1ep-272]\\
\gll Un  bel  effort,  dont  l' avenir  dira  s' il  persuade  les  consommateurs,  s' il  suscite  des  imitateurs [\dots]\\
a nice effort of.which the future will.say if it persuades the consumers if it generates \textsc{det} imitators\\
\glt `A fine effort, and the future will tell us whether consumers are persuaded and whether it will be imitated.'
\label{ex:dont-pronom-FTB}
\z 

In order to present all corpus studies in this work in a uniform and consistent way, some minor corrections have been made to the annotation (see the guidelines in Appendix A). The values may therefore vary slightly from those reported in the respective publications.

\subsection{Results and analysis}

Table \ref{tab:dont-FTB} summarizes the functions of \emph{dont} in the three corpora. They are also displayed in Figures~\ref{fig:dont-FTB} and~\ref{fig:dont-d2000+1900}. 

\begin{table}
\begin{tabular}{ll *3{r@{~}r}}
     \lsptoprule
     \multicolumn{2}{l}{}          & \multicolumn{2}{c}{French}   & \multicolumn{2}{c}{Frantext}   & \multicolumn{2}{c}{Frantext}\\
     \multicolumn{2}{l}{Frequency} & \multicolumn{2}{c}{Treebank} & \multicolumn{2}{c}{2000--2013} & \multicolumn{2}{c}{1900--1913}\\\midrule
     \multicolumn{2}{l}{Verb}      & 74 & (19.42\%) & 25 & (20.33\%) & 20 & (12.20\%) \\\addlinespace
     \multicolumn{2}{l}{Noun}      & & & \\
                       & Subject   & 216 & (56.69\%) & 60 & (48.78\%) & 99 & (60.37\%) \\
                       & Object    & 48  & (12.60\%) & 31 & (25.20\%) & 35 & (21.34\%) \\
                       & Predicate & 26  & (6.82\%)  & 4  & (3.25\%)  & 4  & (2.44\%) \\
     \addlinespace
     \multicolumn{2}{l}{Adjective} & 6   & (1.57\%) & 3   & (2.44\%) & 6  & (3.66\%) \\
     \addlinespace
     \multicolumn{2}{l}{Adjunct} & 11 & (2.89\%) & 0 & 0 \\
     \lspbottomrule
\end{tabular}
\caption{Distribution of \emph{dont} relative clauses in the French Treebank and Frantext}
\label{tab:dont-FTB}
\end{table}

\begin{figure}
    \centering
    \includegraphics[width=.8\textwidth]{chapters/part2-Empirical/dont/dont-FTB/distribution.jpeg}
    \caption[Distribution of \emph{dont} relative clauses in the French Treebank]{Distribution of \emph{dont} relative clauses in the French Treebank}
    \label{fig:dont-FTB}
\end{figure}



\begin{figure}
    \begin{multicols}{2}
    \centering
    \textbf{Frantext 2000--2013}
    \includegraphics[width=0.49\textwidth]{chapters/part2-Empirical/dont/dont-Frantext-2000/distribution-rel.jpeg}
    
    \columnbreak
    
    \textbf{Frantext 1900--1913}
    \includegraphics[width=0.49\textwidth]{chapters/part2-Empirical/dont/dont-Frantext-1900/distribution-rel.jpeg}
    \end{multicols}
    \caption[Distribution of \emph{dont} relative clauses in Frantext 2000--2013]{Distribution of \emph{dont} relative clauses in Frantext}
    \label{fig:dont-d2000+1900}
\end{figure}

\subsubsection{Confidence intervals for frequency} \label{ch:conf-intervals-binomial}
As mentioned before, the advantage of quantitative data is that they show the broader tendencies in the usage of a language while concealing individual preferences. In order to achieve this, it is necessary to identify and treat the outliers accordingly. Outliers are exceptional outcomes, which are present in quantitative data but do not reflect the general tendency. Let us assume that extraction out of subjects is indeed completely out in French. Let us assume in addition that, in a million occurrences of a given filler in a given corpus, one is in fact a case of extraction out of a subject: this fact should not suffice to falsify the hypothesis that such structures are ruled out. Indeed, even in a well-edited text, an error may have been overlooked by proofreaders, or made intentionally to create a feeling of strangeness. Hence, we need statistical validation that few occurrences are more than marginal. In this respect, the size of the corpus is also important: one out of ten occurrences for 1000 occurrences is a more reliable rate than one out of ten occurrences for 20 occurrences. In order to take this into account, I perform an exact test of a simple null hypothesis about the probability of success in a Bernoulli experiment with the function \texttt{binom.test()} from the R Stats Package \citep{R}: this test gives us confidence intervals with a probability of 95\% where the nmber of occurrences of each structure is compared to the total size of our subcorpus.\footnote{The package documentation indicates that the ``confidence intervals are obtained by a procedure first given in \citet{Clopper.Pearson.1934.The}''.} This confidence level of 95\% is corrected for multiple comparisons when needed. For example, in  \figref{fig:dont-FTB}, it is corrected for 6 comparisons, and in \figref{fig:dont-d2000+1900} for 5 comparisons. 
% Clopper, C. J. & Pearson, E. S. (1934). The use of confidence or fiducial limits illustrated in the case of the binomial. Biometrika, 26, 404–413.

If the lower bound of the confidence interval for one structure is smaller than 0.5 occurrences, we consider the frequency of this structure to be not significantly above 0. However, the fact that a given structure does not occur -- or that it occurs with a frequency that is not significantly above zero -- does not necessarily mean that the structure is ruled out by the grammar of the language. We can only say that the corpus data do not contradict the hypothesis that this structure is out in the language.

\subsubsection{Functions of \emph{dont} attested in the corpus}
As expected, \emph{dont} can have any function of a \emph{de}-PP except being the complement of a preposition (or of a noun complement of a preposition). It can therefore either be complement of a verb (\ref{ex:dont-corpus-verb}), of a noun (\ref{ex:FTB-noun}) or of an adjective (\ref{ex:FTB-adjective}) or be an adjunct (\ref{ex:FTB-adjunct}).

\ea Some examples of \emph{dont} as verb complement\label{ex:dont-corpus-verb}
\ea (FTB - flmf7af2ep-602)\\
\gll Le gouvernement n' avait ni écrit ni choisi cet accord [dont nous avons hérité~\trace{}].\\
the government \textsc{neg} had neither written nor chosen this agreement of.which we have inherited\\
\glt `The government had neither written nor chosen this agreement that we inherited.'
\label{ex:FTB-verb}
%\ex[]{\gll Aujourd'hui je me dis que ce doit être terrible d' être l' enfant de quelqu'un [dont on a honte~\trace{} sans savoir pourquoi] [\dots].\\
%today I \textsc{REFL} say that it must be\textsc{.inf} terrible of be\textsc{.inf} the child of someone of.which one has shame without know\textsc{.inf} why\\
%(L'été du sureau, Marie Chaix, 2005)\\
%\glt `Today, I think that it must be terrible to be the child of someone that you are ashamed of without knowing why.'}
%\label{ex:d2000-verb}
\ex (L'enfant d'Austerlitz, Paul Adam, 1902)\\
\gll De nouvelles figures s' imposaient bientôt, dont [il attendait plus de charmes~\trace{}].\\
of new faces \textsc{refl} imposed soon of.which he expected more of charms\\
\glt `New faces soon establish themselves, from which he expected more charms.'
\label{ex:d1900-verb}
\z
\pagebreak
\ex Some examples of \emph{dont} as noun complement\label{ex:FTB-noun}
\ea Subject noun (FTB - flmf7al1ep-66):\\
\gll C' est ce qu' a annoncé récemment M. Georges Fillioud, PDG, dont [le mandat~\trace{}] devrait être reconduit .\\
it is this that has announced recently Mr Georges Fillioud CEO of.which the mandate should be renewed\\
\glt `This is the recent announcement of Mr Georges Fillioud, CEO, whose mandate should be renewed.'
\label{ex:FTB-subj}
\ex (Demain il fera beau~: journal d'une adolescente (novembre 1939-1944), Denise Domenach-Lallich, 2001)\\
\gll Je suis partie avec Georges Lesèvre, étudiant en Lettres dont [le père~\trace{}], [la mère~\trace{}] et [le frère~\trace{}] avaient été arrétés puis déportés.\\
I am left with Georges Lesèvre student in Literature of.which the father the mother and the brother had been arrested then deported\\
\glt `I left with Georges Lesèvre, a literature student whose mother, father and brother had been arrested, and then deported.'
\label{ex:d2000-subj}
%\ex[]{\gll Olivier, dont [le regard lucide~\trace{}] pénétrait l' arrière-pensée des gens, était attristé par leur médiocrité~[\dots].\\
%Olivier of.which the vision lucid penetrated the hidden.thoughts of.the people was saddened by their mediocrity\\
%(Jean-Christophe : Le Buisson ardent, Romain Rolland, 1911)\\
%\glt `Olivier, whose lucid vision penetrated people's hidden thoughts, was made sad by their mediocrity.'}
%\label{ex:d1900-subj}
\ex Object noun:\\
%\gll Une  mutation est en  marche,~[\dots] dont on se  garde  bien de préciser~[\dots] [les tenants et aboutissants \trace{}].\\
%a mutation is in march of.which one \textsc{refl} beware well to specify the ins and outs\\
%(FTB - flmf7ag1exp-245)\\
%\glt `A mutation is about to come, of which one avoids carefully to specify the ins and outs.'}
%\ex[]{
(Un peu de désir sinon je meurs, Marie Billetdoux, 2006)\\
\gll Et chaque pore de ma peau, alors, est un chien qui se redresse, dont on a touché [la laisse~\trace{}]\dots'\\
and every pore of my skin then is a dog who \textsc{refl} get.up of.which one has touched the leash\\
\glt `And every pore of my skin is then like a dog who stands up, the leash of whom someone touched\dots'
\ex (Dingley, l'illustre écrivain, Jérôme Tharaud, 1906)\\
\gll Elle ne bougea pas, afin de ne pas réveiller son mari, dont elle redoutait [la violence~\trace{}].\\
she \textsc{neg} moved not so.that of \textsc{neg} not wake.up her husband of.which she feared the violence.\\
\glt `She did not move, so as not to wake up her husband, the violence of whom she feared.'
\pagebreak
\ex Predicate noun (FTB - flmf7ag1exp-449):\\
\gll Le premier n' avait pas les faveurs de France Télécom, dont Matra Communication est [l' un des tout premiers fournisseurs~\trace{}].\\
the first \textsc{neg} had not the support of France Télécom of.which Matra Communication is the one of.the very first suppliers\\
\glt `The first one did not have the support of France Télécom, of which Matra Communication is one of the very first suppliers.'
\ex (La légende des cycles, Jean-No\"{e}l Blanc, 2003)\nopagebreak\\
\gll Fignon se rebelle contre Hinault, dont il a été [lieutenant~\trace{}]~: provocation~[\dots].\\
Fignon \textsc{refl} rebel against Hinault of.which he has been lieutenant provocation\\
\glt `Fignon rebels against Hinault, who he has been the lieutenant of: (that's a) provocation.'
%\ex[]{\gll L' Avenir~[\dots] garda un visage massif comme les personnes dont c' est [métier~\trace{}] de donner des conseils.\\
%the Future kept a face massive like the persons of.which it is profession of give\textsc{.inf} \textsc{det} advises.\\
%(La Mére et l'enfant, Charles-Louis Philippe, 1900)\\
%\glt `The Future kept a massive face like someone of who it's the profession to give advises.'}
\z 

\ex Some examples of \emph{dont} as an adjective complement 
\ea (FTB - flmf7aa2ep-513)\\
\gll Ils ôtent à la politique monétaire ses références chiffrées dont les  membres  du  directoire  sont [friands~\trace{}].\\
they remove from the policy monetary its references quantitative, of.which the members of.the directorate are fond\\
\glt `They deprive the monetary policy of its quantitative benchmarks, which the members of the management board are fond of.'
\label{ex:FTB-adjective}
\ex (La vie après, Virginie Linhart, 2012)\\
\gll Sans doute n' imaginait - il plus sa vie sans ma grand-mère dont il était [fort épris~\trace{}]~?\\
without doubt \textsc{neg} imagined {} he anymore his life without my grandmother of.which he was very in.love\\
\glt `Without doubt, he could not imagine anymore to live without my grandmother, who he was in love with?'
\label{ex:d2000-adjective}
%\ex[]{\gll Un esclave placé entre les deux brancards compléta l' équipage, dont Kalj parut [enchanté~\trace{}].\\
%a slave placed between the two stretcher enhanced the crew, of.which Kalj seemed delighted\\
%(Impressions d'Afrique, Raymond Roussel, 1910)\\
%\glt `A slave placed between the strechers rounded off the crew, by which Kalj seemed delighted.'}
%\label{ex:d1900-adjective}
\z

\ex An example of \emph{dont} as an adjunct (FTB - flmf7an2co-919)\\
\gll La GSA conteste la manière dont [l' armée de l' air américaine a mené l' évaluation des différentes propositions~\trace{}]~[\dots].\\
the GSA contests the manner of.which the army of the air American has led the evaluation of.the different proposals\\
\glt `The GSA contests the manner in which the American air force led the different proposals' evaluation.'
\label{ex:FTB-adjunct}
\z

There is a very high number of extractions out of NPs. For the 21st century (FTB and Frantext), they represent around 3/4 of usages of \emph{dont}, but the proportion is even higher in Frantext 1900--1913. In all cases, most of them are extractions out of the subject. Except for Frantext 2000-2013, extractions out of the subject are significantly more frequent than extractions out of the object. This provides further support that extractions out of the subject are possible with \emph{dont} \citep{Godard.1988}, and are in fact the most frequent case \citep{Candito.2012.ldd}. 

The occurrences of \textit{dont} as an adjunct in FTB are almost exclusively \emph{la manière dont} or \emph{la fa\c{c}on dont} (`the way how'). Such usages were not found in our Frantext corpus, because we only considered relative clauses with animate antecedents. For this reason, there are no occurrences of \textit{dont} as an adjunct in our Frantext results.

\subsubsection{Subject position}

Although possible, subject-verb inversion is very rare in our results, as can be seen in Table~\ref{tab:dont-inversions}. It is strikingly more frequent in the FTB, probably for stylistic reasons that have to do with journalistic writing.\footnote{In \citet{Abeille.2016}, we show that, most of the time, these postverbal subjects appear when there is extraction out of the verb. We argue that, in this case, the cost for subject-verb inversion is counterbalanced by a reduction of filler-gap dependency length.} Extraction out of the subject is very rare as well, but is attested.

\begin{table}
\begin{tabularx}{\textwidth}{Qccc}
     \lsptoprule
     Nb of occurrences & French Treebank & \multicolumn{2}{c}{Frantext} \\\cmidrule(lr){3-4}
                       &                 & 2000--2013 & 1900--1913 \\
     \midrule
     postverbal subjects in total & 36 & 6 & 5 \\
     extractions out of  a postverbal subject & 0 & 4 & 1 \\
     \lspbottomrule
\end{tabularx}
    \caption{Subject-verb inversions in the French Treebank and Frantext}
    \label{tab:dont-inversions}
\end{table}

For extraction out of a subject, using a postverbal subject instead of a preverbal one increases the linear distance between the relative word and its gap. Processing accounts (DLT or UD) therefore predict that this configuration will be avoided.

All five occurrences of extraction out of a postverbal subject are reported in (\ref{ex:d2000-subj-inv}).\largerpage[2]

\eal \label{ex:d2000-subj-inv}
\ex (Dans la main du diable, Anne-Marie Garat, 2006)\\
\gll les morts, dont se dissout dans l' air [la présence~\trace{}]\\
the deads of.which \textsc{refl} dissolves in the air the presence\\
\glt `the dead ones, whose presence vanishes in the air'
\ex (Dans la main du diable, Anne-Marie Garat, 2006)\\
\gll Millie, dont grandissait [l' angoisse~\trace{}]\\
Millie of.which grew the anxiety\\
\glt `Millie, whose anxiety was growing'
\ex (Entretiens et conférences II [1979--1981], Georges Perec, 2003)\\
\gll Pierre Getzler, dont est reproduite [une gravure~\trace{}]\\
Pierre Getzler of.which is reproduced an engraving\\
\glt `Pierre Getzler, an engraving of whom is reproduced'
\ex (Voix off, Denis Podalydès, 2008)\\
\gll Éric Elmosnino lui-même, dont nous charment [la voix~\trace{}], [la présence~\trace{}], [le mouvement~\trace{}], [la malice~\trace{}].\\
Éric Elmosnino himself of.which us\textsc{.acc} charm the voice the presence the movement the craftiness\\
\glt `Éric Elmosnino himself, whose voice, presence, movement and craftiness charm us.'
\ex (L'Inde (sans les Anglais), Pierre Loti, 1903)\\
\gll [l]es monstres cabrés, dont se reconnaissent déjà [les silhouettes~\trace{}].\\
the monsters rearing of.which \textsc{refl} recognise already the shapes\\
\glt `the rearing monsters, whose shapes are already recognizable'
\label{ex:d1900-subj-inv}
\zl 

Postverbal subjects are counterexamples to \citegen{Heck.2009} analysis of \emph{dont} as being inside the subject NP. As a native speaker, I find these sentences unproblematic and well-formed.

I can report a few more cases challenging \citegen{Heck.2009} account. The first one is questionable: in (\ref{ex:d2000-subj-particle}), a negative conjunction (\emph{ni\dots ni\dots}) stands between \emph{dont} and the subject. One could possibly argue that both the negative particle and \emph{dont} occupy the specifier position of D, but this would be an unusual analysis.

\ea (La vie sexuelle de Catherine M.\ précédé de Pourquoi et Comment, Catherine Millet, 2001)\nopagebreak\\
\gll ces êtres privés~[\dots] de l' usage de leurs membres et de celui de la parole, mais	dont	\textbf{ni} [l' intelligence~\trace{}] \textbf{ni} [le besoin~\trace{} de communiquer] ne sont altérés\\
these beings deprived of the use of their limbs and of the.one of the speech but of.which neither the intelligence nor the need of communicate\textsc{.inf} \textsc{neg} are modified\\
\glt `these beings, deprived of the use of their limbs and of the ability to speak, but of whom neither the intelligence nor the need to communicate have been modified'
\label{ex:d2000-subj-particle}
\z\largerpage[2]

Another problematic case is given in (\ref{ex:FTB-double-de}). Here, the extracted \emph{de}-PP is the complement of the \emph{de}-PP complement of the subject noun. If \emph{dont}  in (\ref{ex:FTB-double-de}) is situated in the specifier position of DP, then it cannot be in the specifier position of the DP of its head noun, but it must be in the specifier position of the DP of the head of its head noun. Such a configuration is not explicitly expected by \citet{Heck.2009}, but perhaps it is not entirely incompatible with his hypothesis (e.g.\ in assuming cyclic movement inside the subject NP). 

\ea (FTB - flmf7ag2ep-663)\\
\gll [les] chômeurs~[\dots] dont [les durées [d' affiliation~\trace{}]] sont les plus courtes\\
the unemployed of.whom the period of affiliation are the most short\\
\glt `the unemployed whose period of affiliation are the shortest'
\label{ex:FTB-double-de}
\z 

Finally, there is an indisputable case of a long-distance dependency in (\ref{ex:d2000-subj-ldd}), a possibility which was explicitly ruled out by \citegen{Heck.2009} analysis.

\ea (Voix off, Podalydes, 2008)\\
\gll madame Segond-Weber, la grande tragédienne, dont \textbf{il} \textbf{aime} \textbf{rappeler} \textbf{que} [les répliques~\trace{}] tombaient de sa bouche «~comme des fûts de colonne~».\\
Madame Segond-Weber the great tragedian of.whom he likes recall\textsc{.inf} that the lines felt of her mouth like some shafts of column\\
\glt `Madame Segond-Weber, the great tragedian, of who he enjoys recalling that the lines fall out of her lips « like pillars ».'
\label{ex:d2000-subj-ldd}
\z 

The attested examples in (\ref{ex:d2000-subj-inv}) and (\ref{ex:d2000-subj-ldd}) show that \citegen{Heck.2009} explanation cannot hold. His analysis of \emph{dont} is contradicted by the empirical data. Even cases such as (\ref{ex:d2000-subj-particle}) and (\ref{ex:FTB-double-de}) would be hard to conciliate with his approach. 

\subsubsection{Verb types}

% VOICI was excluded

Following some generativist approaches on syntax (notably, \citealp{Chomsky.2008}), extracting out of the subject of a passive or unaccusative verb is extraction out of the (underlying) direct object.
% unergative ? mediopassive ? state ?
Table \ref{tab:FTB-verbtype-dont} shows the verb types involved in extractions out of the subject in our corpora. We can see that all types are attested. Transitives (\ref{ex:d1900-subj-trans}), unergatives (\ref{ex:d2000-subj-unerg}) and state verbs (\ref{ex:FTB-subj-state}) are frequent. Passives (\ref{ex:d2000-subj-passive}) are more frequent in the FTB than in Frantext, and more frequent in the 21st century than in the 20th century. Unaccusatives (\ref{ex:d2000-subj-unacc}) and mediopassives (\ref{ex:d1900-subj-medio}) are attested, but not frequent. Mediopassives are generally rare in our relative clauses. 

\begin{table}
    \begin{tabular}{l *3{r@{~}r}}
         \lsptoprule
                   & \multicolumn{2}{c}{French}   & \multicolumn{2}{c}{Frantext}   & \multicolumn{2}{c}{Frantext} \\
         Verb type & \multicolumn{2}{c}{Treebank} & \multicolumn{2}{c}{2000--2013} & \multicolumn{2}{c}{1900--1913} \\\midrule
         Passive      & 53 & (24.54\%) & 8  & (13.33\%) & 9  & (9.09\%) \\
         Unaccusative & 17 & (7.87\%)  & 7  & (11.67\%) & 9  & (9.09\%) \\
         Mediopassive & 4  & (1.85\%)  & 4  & (6.67\%)  & 5  & (5.05\%) \\
         Transitive   & 49 & (22.69\%) & 15 & (25.00\%) & 33 & (33.33) \\
         Unergative   & 29 & (13.43\%) & 12 & (20.00\%) & 22 & (22.22\%) \\
         State        & 64 & (29.63\%) & 14 & (23.33\%) & 21 & (21.21\%) \\
         \lspbottomrule
    \end{tabular}
    \caption{Verb types in extractions out of the subject among \emph{dont} relative clauses}
    \label{tab:FTB-verbtype-dont}
\end{table}

\eal 

%\ex[]{\gll Les  premiers  étaient  des  coopératives  dont  [les  membres~\trace{}]  exploitaient  sous  forme  privée  des lopins de terre.\\
%the first were \textsc{det} cooperatives of.which the members exploited under form private \textsc{det} pieces of land \\
%(FTB - flmf7ab1co-440)\\
%\glt `The first ones were cooperatives, whose members exploited privately some pieces of land.'}
%\label{ex:FTB-subj-trans}
%\ex[]{\gll les apaches,	dont [les bandes~\trace{}] portent des noms évocateurs\\
%the Apaches of.which the gangs bear \textsc{det} names evocative\\
%(Porte de Champerret, Évelyne Bloch-Dano, 2013)\\
%\glt `the Apaches (criminals) whose gangs bear evocative names'}
%\label{ex:d2000-subj-trans}
\ex(Tête d'or [2e version], Paul Claudel, 1901)\\
\gll Les soldats, dont [quelques-uns~\trace{}] portent des drapeaux, envahissent la salle.\\
the soldiers of.which some carry \textsc{det} flags overrun the room\\
\glt `The soldiers, of which some are carrying flags, run into the room.'
\label{ex:d1900-subj-trans}

%\ex[]{\gll Les entreprises dont [l' effectif~\trace{}] se situe entre 50 et 400 salariés acquittent leurs cotisations.\\
%the companies of.which the headcount \textsc{refl} situates between 50 and 400 employees pay their contributions \\
%(FTB - flmf7ab2ep-853)\\
%\glt `Companies whose headcount lies between 50 and 400 employees pay their contributions.'}
%\label{ex:FTB-subj-unerg}
\ex (Besoin de vélo, Paul Fournel, 2001)\\
\gll [l]es tatanes de Greg LeMond dont [les pieds~\trace{}] ne ressemblaient à rien de connu dans le peloton~[\dots]\\
the big.shoes of Greg LeMond of.which the feet \textsc{neg} resembled at nothing of known in the peloton\\
\glt `the big shoes of Greg LeMond, whose feet were like nothing else that was known in the peloton [\dots]'
\label{ex:d2000-subj-unerg}
%\ex[]{\gll un Goupil dont [le museau pointu~\trace{}] seul vivait\\
%a Goupil of.which the snout pointed alone lived\\
%(De Goupil à Margot : histoire de bêtes, Louis Pergaud, 1910)\\
%\glt `a Goupil (fox), of who only the pointed snout was alive'}
%\label{ex:d1900-subj-unerg}

\ex (FTB - flmf3\_08500\_08999ep-8561)\nopagebreak\\
\gll La fumée contenait en effet du oxyde cyanhydrique dont [les effets~\trace{}] sont immédiats.\\
the smoke contained in effect some oxide hydrocyanic of.which the effects are immediate\\
\glt `The smoke indeed contained  some hydrogen cyanide, whose effects are immediate.'
\label{ex:FTB-subj-state}
%\ex[]{\gll Enrique, dont [la bande démo~\trace{}] est formidable\\
%Enrique of.which the reel demo is wonderful\\
%(99 francs, Frédéric Beigbeder, 2000)\\
%\glt `Enrique, whose demo reel is wonderful'}
%\label{ex:d2000-subj-state}
%\ex[]{\gll Ginisty dont [les yeux~\trace{}] sont comme des fentes de porte-plume pour mettre la plume\\
%Ginisty of.which the eyes are like some slot of penholder for put\textsc{.inf} the pen\\
%(Journal : 1887-1910, Jules Renard, 1910)\\
%\glt `Ginisty, whose eyes are like two slots in a penholder where to put the pen'}
%\label{ex:d1900-subj-state}

%\ex[]{\gll Cette  vente, dont [le  montant~\trace{}] est  gardé  secret, concrétise les rumeurs courant  sur Biba depuis l' automne.\\
%this sale of.which the cost is kept secret materialises the rumor going on Biba since the autumn \\
%(FTB - flmf7al1ep-464)\\
%\glt `This sale, whose cost is kept secret, makes the rumor about Biba since autumn more concrete.'}
%\label{ex:FTB-subj-passive}
\ex (L'arrivée de mon père en France, Martine Storti, 2008)\\
\gll les morts de Lampedusa dont [les noms~\trace{}] sont illustrés de photos\\
the dead.\textsc{pl} of Lampedusa of.which the names are illustrated of photographs\\
\glt `The dead of Lampedusa whose names are illustrated with photographs'
\label{ex:d2000-subj-passive}
%\ex[]{\gll [l]es auteurs dont [le nom~\trace{}] est souvent cité avec éloge dans les journaux\\
%the authors of.which the name is often cited with praise in the newspapers\\
%(Réflexions sur la violence, Georges Sorel, 1912)\\
%\glt `the authors whose name is often praised in the newspapers'}
%\label{ex:d1900-subj-passive}

%\ex[]{\gll le courant dit ``progressiste'', dont [le vocabulaire spécifique~\trace{}] vient se mélanger au précédent \\
%the approach called progressive of.which the vocabulary specific comes \textsc{refl} mix to.the preceding \\
%(FTB - flmf7af1ep-186)\\
%\glt `le so-called ``progressive'' approach, whose specific terminology mixes with the preceding one'}
%\label{ex:FTB-subj-unacc}
\ex (Rendez-vous, Christine Angot, 2006)\\
\gll une actrice dont [le fils~\trace{}] voulait devenir metteur en scène~[\dots]\\
an actress of.which the son wanted become\textsc{.inf} director of stage\\
\glt `an actress whose son wanted to become a stage director [\dots]'
\label{ex:d2000-subj-unacc}
%\ex[]{\gll sa mère, dont [la figure~\trace{}] redevenait plus sereine\\
%his mother of.which the face became more peaceful\\
%(Jean-Christophe : L'Adolescent, Romain Rolland, 1905)\\
%\glt `his mother, whose face became more peaceful'}
%\label{ex:d1900-subj-unacc}

%\ex[]{\gll L' arrivée du Peso ne signifie pas l' élimination automatique des Australs, dont [le remplacement~\trace{}] devrait se faire progressivement, en moins de six mois.\\
%the arrival of.the Peso \textsc{neg} means not the abolition automatic of.the Australs of.which the replacement should \textsc{refl} make progressively in less than six months\\
%(FTB - flmf7aa1ep-58)\\
%\glt `The Peso's arrival does not mean the automatic abolition of Australs, whose replacement should take place gradually in less than six months.'}
%\label{ex:FTB-subj-medio}
%\ex[]{\gll tous ceux dont [le mode de vie~\trace{}] ne s' alignait pas sur le sien\\
%all these of.which the way of life \textsc{neg} \textsc{refl} aligned not on the his\\
%La solitude de la fleur blanche, Annelise Roux, 2009)\\
%\glt `all of those whose way of life didn't align with his own'}
%\label{ex:d2000-subj-medio}
\ex (Terres lorraines, Emile Moselly, 1907)\\
\gll Pierre, dont [la haute taille~\trace{}] s' encadrait dans la fenêtre\\
Pierre of.which the high size \textsc{refl} frame in the window\\
\glt `Pierre, whose high figure framed itself in the window' (intended: the window formed a frame around his high figure)
\label{ex:d1900-subj-medio}
\zl 

\begin{figure}
    \includegraphics[width=0.7\textwidth]{chapters/part2-Empirical/dont/dont-FTB/verbtype.jpeg}
    \caption[Distribution on the French Treebank of verb types involved among extractions out of the subject, compared to other extraction types in \emph{dont} relative clauses]{FTB: Distribution of the type of verb involved among extractions out of the subject, compared to other type of extractions in \emph{dont} relative clauses. See page~\pageref{ch:conf-intervals-binomial} for the confidence intervals (here six comparisons). The percentage is given for each group (extraction out of the subject vs.\ other extraction).}
    \label{fig:dont-FTB-verbtype}
\end{figure}
    
We can compare the verb types involved in extraction out of the subject with the verb types in other kinds of extractions. \figref{fig:dont-FTB-verbtype} illustrates this for the French Treebank. Taking into account the confidence intervals, the only difference seems to be that, in extraction out of subjects, passives are more frequent and unergatives are less frequent than in other kinds of extractions. Similar observations can be made for the two corpus studies on Frantext. However, this has a simple explanation: extraction out of objects nouns or predicate nouns cannot involve a passive, and we have seen that extraction out of nouns is very frequent in the corpus. The second most frequent case are extractions of the complement of the verb, and since most \emph{de}-PP complement of verbs are complements of an unergative verb, it is also clear why the category ``other kinds of extractions'' contains such a high proportion of unergatives. We do not need to assume that extraction out of the subject is especially felicitous with passives or especially infelicitous with unergatives. 

\begin{figure}
    \includegraphics[width=1\textwidth]{chapters/part2-Empirical/dont/verbtype-dont.jpeg}
    \caption[Distribution of transitive verbs in \emph{dont} relative clauses]{Distribution of transitive verbs in \emph{dont} relative clauses. See page~\pageref{ch:conf-intervals-binomial} for the confidence intervals (here two comparisons). The percentage is given for each group (a group = one kind of extraction in one corpus).}
    \label{fig:dont-FTB-trans}
\end{figure}

Crucially, a non-marginal number or extractions out of subjects contain transitive verbs: \figref{fig:dont-FTB-trans} shows a binary distinction between transitive verbs and the other verb types. Transitives are around 1/4 of the verbs involved in extraction out of the subject: 22.69\% in the French Treebank, 25\% in Frantext 2000--2013 and 33.33\% in Frantext 1900--1913. We note that this correlates with the proportion of transitives in the other extraction types. Thus, the small number of transitive verbs is not specific to extraction out of the subject but rather a general tendency in the French Treebank, maybe as a consequence of the text genre. There are more transitive verbs in the group ``other kinds of extractions'', but this does not mean that the number of transitives is especially low among extractions out of the subject:  the group ``other kinds of extractions'' contains extractions out of a direct object, which necessarily imply the presence of a transitive verb.


To sum up, we can indeed confirm that there are more non-transitive than transitive verbs among extractions out of the subject. And indeed, we find a considerable number of passives and state verbs among them. This may explain why many of the examples we find in the literature are extractions out of the subject of a passive or state verb. This could even be the reason for the sense in the literature that extraction out of the subject sounds particularly good with these kinds of verbs. However, this does not mean that extraction out of the subject is degraded~-- let alone ungrammatical~-- with transitive verbs, since such extractions are still common.

\subsubsection{Other factors}

We annotated the relative clauses for several other factors: the number (singular/plural) and definiteness of the antecedent, the restrictiveness of the relative clauses and the semantic relation holding between \emph{dont} and its head noun. I briefly summarize the relevant findings in this section, and include more details in Appendix~\ref{ch:other-factors}.

Number and definiteness of the antecedent do not seem to follow any specific pattern.

We find a clear tendency for extraction out of the subject to be non-restrictive relatives. This result is expected under the FBC constraint (\ref{rule:FBC}): if we assume that the extracted element in relative clauses may be non-topic (thus potentially focus) and that extraction out of the subject requires that the extracted element be non-focal, then extraction out of the subject should display a preference for non-restrictiveness. In line with this, there are more non-restrictive relative clauses among extractions out of the subject than in other extraction types: presumably speakers opt for an alternative construction when they wish to express a restrictive relative clause with a focused extracted element.  

As for the semantic relation holding between \emph{dont} and its head noun, extraction type plays at best a minor role. More important are (i) the text genre of the corpora (newspaper article vs.\ literary text) and (ii) whether or not we restrict our results to animate antecedents. However, by comparing extractions out of subjects vs.\ of objects in Frantext 2000--2013 and 1900--1913, the kind of relation was close to being a good predictor for the gap site. In general, we found many part-whole relations (especially for body parts) in extractions out of the subject, and many patient-event relations in extractions out of the object.


\subsection{General conclusion on the corpus studies on \emph{dont}}

All these corpus results converge on one major point: extracting out of the subject with \emph{dont} is not only possible, but is in fact its most common usage. In Frantext 2000--2013, it does not significantly differ from extraction out of the object, but it is significantly different from all other kinds of extractions in FTB and Frantext 1900--1913. This corroborates \citet{Godard.1988}, but also any account based on minimizing the distance between the relative word and the gap. 
In general, extraction out of NPs is very frequent (over 3/4 of all extractions in all three corpora). 

Extraction out of subjects of transitive verbs is attested in all corpora, and is actually a very frequent case. This means that extraction out of subject cannot be claimed to be restricted to cases of underlying objects, but is also found for what all theories on syntax consider as ``real'' subjects. Even if one assumes subject movement as in Minimalism, the data show clear evidence that there is no freezing effect blocking the extraction.

However, we can also observe in all three corpora that passives are more frequent among extractions out of subjects than among all other kinds of extractions. We can also see that transitive are less frequent among extractions out of subjects than among all other kinds of extractions. This does not mean that there are more passives than transitives among the extractions out of the subject: in fact, Frantext 1900--1913 has significantly more transitives, and in the two other corpora, the proportions of passives and transitives are very similar. This may however explain French scholars' intuitions that extractions out of the subject are especially good with passives, and especially bad with transitives.\footnote{It is also important to take into account some parameters that may affect our results: 
\begin{itemize}
    \item With a passive verb, there are few possible gap sites for a \emph{de}-PP besides one inside the subject itself. 
    %The only other possibilities are to extract an adverb (for \emph{dont} relatives with an inanimate antecedent only) or extract out of the oblique object (this leads to an extraction out of a PP, that is syntactically not impossible, but seem subject to some strong constraints). 
    This fact alone can explain why there are more passives in this group than in other kinds of extractions.
    \item Extraction out of an object NP necessarily involves transitive verbs, no other verb type is possible. This fact alone can explain why there are more transitives in the ``other kinds of extractions'' group.
    \item Extraction of the \emph{de}-complement of a verb can only involve unergative verbs, and no other verb type. This fact alone can explain why there are so much unergatives in the ``other kinds of extractions'' group.
\end{itemize}
In this respect the factors are not completely independent of each other. This weakens the conclusion that we can draw from the comparison.}

In \citet{Godard.1988}, all felicitous examples of extraction out of the subject contain the state verb \emph{être} (`be'). Indeed, extraction out of the subject of state verbs tends to be frequent, but possibly \citeauthor{Godard.1988} used these in order to construct simple sentences with a very straightforward semantic content. Similarly, the reason for using passives to illustrate extraction from subjects in the literature may be that such sentences are relatively simple. No other conclusion should be drawn from this choice, at least as far as production data are concerned.

In Frantext 2000--2013, we find evidence that material can stand between \emph{dont} and the subject NP it is extracted from. This contradicts \citet{Heck.2009}. The fact that intervening material rarely appears after \emph{dont} is easy to understand under accounts assuming that speakers tend to minimize the linear distance between \emph{dont} and the gap. These eccounts can also explain why constructed counterexamples have been judged as degraded in the literature \citep[e.g.,][]{Tellier.1990}. But the corpus data provide evidence that extractions out of subjects with \emph{dont} are ``real'' extractions.  
