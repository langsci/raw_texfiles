\subsection{A previous exploration of \emph{dont} in the FTB (Candito \& Seddah)}
\label{ch-candito-2012}

\citet{Candito.2012.ldd} examined two corpora, the French Treebank (\citealt{Abeille.2019.FTB}, see our corpus study below) and the Sequoia treebank \citep{Candito.2012.Sequoia}, which contains sentences from the French Wikipedia, from medical texts, from \textit{Europarl} and from the regional newspaper \emph{L’Est Républicain}. They looked for words involved in unbounded dependencies: the clitic \emph{en} (`of it') and \textit{wh}-words (relative and interrogative pronouns and determiners). The relative word \emph{dont} is the second most frequent word in their results after the relative word \emph{que}. They found 501 extractions with \emph{dont}, the three most frequent types being extractions out of subject NPs (251 cases), extractions out of object NPs (29 cases) and extractions out of predicative complements (27 cases). This means that half of the extractions with \emph{dont} are extractions out of subject NPs, a surprisingly high number if one expects these extractions to be banned by syntax. In general, \citeauthor{Candito.2012.ldd} note, using terminology from Dependency Grammar (see Section~\ref{ch:dlt+dg}), that around one third of their extractions (259 out of 618 dependencies) are projective: ``This is because most [long distance dependencies] are extractions from a subject NP [\dots].''
% no page number in Candito & Seddah

Even though \citeauthor{Candito.2012.ldd} do not provide much more detail on their results (the usage of \emph{dont} was not what they were interested in), we can already see that extractions out of the subject, especially with \emph{dont}, are very frequent in the two corpora, given that they represent the majority of unbounded dependencies. 