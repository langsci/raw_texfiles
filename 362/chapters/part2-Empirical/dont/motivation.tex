\subsection{Motivation}

\emph{Dont} is very frequent in written (and spoken) French. That is probably the reason why the discussion on extraction out of the subject in French started with examples of \emph{dont}-relative clauses.
As mentioned in Section~\ref{ch:intro-disscussion-French}, \citet{Godard.1988} has shown that \emph{dont} may be the complement of a subject noun, despite the general subject island constraint assumed in the syntactic tradition. However, to the best of my knowledge, \citet{Candito.2012.ldd} is the only corpus study providing quantitative results on \emph{dont} in production data, even though this was not their primary concern. I will briefly present the relevant points of this study in Section~\ref{ch-candito-2012}. 

Our corpus studies on \textit{dont} pursued seven main goals. First, we wanted to have a detailed description of its usage. \citet{Blanche-Benveniste.1990} noticed that \emph{dont} has what she calls a ``fixed usage'' in contemporary spoken French because it occurs almost exclusively as the complement of a verb. If this were the case in written French as well, and if extractions out of an NP were a rare usage of \emph{dont}, any comparison between the frequency of extraction out of the subject NP and the object NP would be irrelevant. We thus wanted to see whether extraction out of NPs is frequent with \emph{dont}, if extractions out of the subject are present, and in what proportion. 

Second, we wanted to describe extraction out of subjects more precisely in order better understand the way it is processed. In particular, we wanted to annotate the position of subjects containing a gap with respect to the verb. Even though French is a SVO language, subject-verb inversion is common in relative clauses. % cite Pozniak here ?
\citet[263]{Lahousse.2011} reports that postverbal subjects are less topical than preverbal ones. Following an approach to filler-gap dependencies based on memory load like the DLT or Dependency Grammar (see Section~\ref{ch:dlt+dg}), one would expect postverbal subjects to be dispreferred when extracting out of the subject, because this construction has an additional intervening referent between the relative word and the gap compared to a preverbal subject, as illustrated in (\ref{ex:ldt-subject}). On the other hand, one would expect a preference for postverbal subjects when extracting the complement of the verb, because this way there is one fewer intervening referent between the relative word and its gap compared to a preverbal subject, as illustrated in (\ref{ex:ldt-verb}).

\eal\label{ex:ldt-subject}
\ex Extraction out of a subject with preverbal subject (one intervening referent):\\
\gll the cat \textit{of} \textit{which} [the owner~\trace{}] disappeared \\
{} {} {} {} {} 1 {} \\
\ex Extraction out of a subject with postverbal subject (two intervening referents):\\
\gll the cat \textit{of} \textit{which} disappeared [the owner~\trace{}]  \\
{} {} {} {} 1 {} 1\\
\zl 

\eal\label{ex:ldt-verb}
\ex Extraction out of a verb with preverbal subject (two intervening referents):\\
\gll the cat \textit{of} \textit{which} Ernest thinks~\trace{} \\
{} {} {} {} 1 1  \\
\ex Extraction out of a verb with postverbal subject (one intervening referent):\\
\gll the cat \textit{of} \textit{which} thinks~\trace{} Ernest \\
{} {} {} {} 1 \\
\zl 


Furthermore, looking for postverbal subjects containing a gap also allows us to test \citegen{Heck.2009} analysis of \emph{dont}. In an attempt to reconciliate \citegen{Godard.1988} counterexamples with the minimalist account of subject islands, \citeauthor{Heck.2009} proposes that \emph{dont} is not an extracted element when it is the complement of the subject, but a specifier of the subject DP. Under his proposal, the whole DP is pied-piped to the edge of the relative clause, as shown in example (\ref{ex-Heck-transitive-subject}). This implies that there are two different usages of \emph{dont} for \citet{Heck.2009}, because there is a filler-gap dependency when \emph{dont} is the complement of a verb or of an object noun (\ref{ex-Heck-transitive-object}).

\eal
\ex \citep[101]{Heck.2009}\\
\gll la fille [dont le frère]$_{\textsc{dp}}$ t' a rencontré\\
the girl of.which the brother you.\textsc{acc} has met \\
\glt `the girl whose brother met you'
\label{ex-Heck-transitive-subject}
\ex \gll la fille dont tu as rencontré [le frère~\trace{}]$_{\textsc{dp}}$\\
the girl of.which you have met the brother\\
\glt `the girl whose brother you met'
\label{ex-Heck-transitive-object}
\zl 

According to Heck's analysis, no material may intervene between \emph{dont} and the subject when \emph{dont} is a specifier as in (\ref{ex-Heck-transitive-subject}): no subject inversion like (\ref{ex-Heck-sv-inversion}) and no long distance dependency like (\ref{ex-Heck-long-distance}) is supposed to be possible when a complement of the subject is relativized. However, such constraints have not been tested empirically.

\eal\label{ex-Heck}
\ex \citep{Heck.2009}\\
* \gll Colin, dont choque la coiffure blonde peroxydée\\
Colin of.which surprises the hair blond bleached\\
\glt `Colin, whose bleached blond hair is shocking'
\label{ex-Heck-sv-inversion}
\ex \citep{Tellier.1991} \\
?? \gll un homme dont je refuse que le fils vous fréquente\\
a man of.which I refuse that the son you.\textsc{acc} dates\\
\glt `a man of whom I refuse that the son dates you'
\label{ex-Heck-long-distance}
\zl 

For this reason, our third goal was to see if we can find extractions out of the subject that are long-distance dependencies. This will be a way to test \citegen{Heck.2009} prediction. 

Fourth, by annotating the kinds of verbs in the corpus, we wanted to see whether extraction out of the subject NP is predominantly found with subjects of passives or unaccusative verbs, or ``internal objects'' as they are sometimes called in the literature. Accounts along the line of \citet{Chomsky.2008} predict that speakers only produce extractions out of ``internal objects'', and not out of subjects which are base-generated in a specifier position.\footnote{See fn.\ \ref{fn:unaccusative} on page \pageref{fn:unaccusative}.}

Fifth, we wanted to examine if extractions out of the subject differ from other uses of \emph{dont} with respect to restrictiveness. We do not know of any accounts predicting that extraction out of the subject is sensitive to this factor, but there are different predictions as far as information structure is concerned for restrictive and non-restrictive relative clauses. According to \citet[181--186]{Song.2017}, the antecedent of a non-restrictive relative clause is the ``aboutness topic'' of its main verb. However, there is ``no additional clue''  for identifying the relation that holds between the antecedent and the main verb of a restrictive relative clause. Indeed, non-restrictive relative clauses like (\ref{ex:song-restr-nr}) can be paraphrased as (\ref{ex:song-restr-paraphrase}) using the test for aboutness topic, whereas restrictive relative clauses like (\ref{ex:song-restr-r}) cannot. 

\begin{exe}
\ex \citep[181]{Song.2017}
\begin{xlist}
\ex[]{Kim chases the dog that likes Lee. \label{ex:song-restr-r}}
\ex[]{Kim chases the dog, which likes Lee. \label{ex:song-restr-nr}}
\ex[]{Kim chases the dog, and speaking of the dog, it likes Lee. \label{ex:song-restr-paraphrase}}
\end{xlist}
\end{exe}

\begin{sloppypar}
Hence, it may be the case that the extracted element in restrictive relative clauses is something other than a topic. If extraction out of the subject requires that the extracted element be non-focal, then we expect to find a higher proportion of non-restrictive relative clauses among extractions out of the subject than in other extraction types. 
% "non-restrictive relatives are almost equivalent to coordinated clauses which clearly involve root phenomena (Heycock 2007: 177) --> (Heycock, Caroline. 2007. Embedded root phenomena. In Martin Everaert & Henk van Riemsdijk (eds.), The blackwell companion to syntax, 174–209. Wiley Online Library.)
% Chaves & Putnam say the opposite : filler of RESTRICTIVE RC is topical
\end{sloppypar}

Sixth, we wanted to look at the distribution of extractions out of the subject in terms of their semantics, specifically, the meaning of the relation expressed by the extracted \textit{de}-complement (spatial or temporal relations, property, possessives, etc.). This is again a more exploratory aspect of the corpus study, given that the various approaches do not make any predictions in this respect.

Our final aim was to distinguish relative clauses with an antecedent from \emph{c'est}-cleft sentences. The latter do involve focalization, and in terms of discourse status, they are closer to \textit{wh}-questions than to relative clauses. We will develop this idea more extensively in Section~\ref{ch:is-clefts}. 

