\subsection{Procedure}

One of the corpus studies, published in \citet{Abeille.2016}, was carried out on the French Treebank (FTB version 1.0, \citealt{Abeille.2019.FTB}). The corpus is a tagged newspaper corpus of 21 550 sentences (about 664 500 words) from the French newspaper \emph{Le Monde} (articles from 1990 to 1993). We explored it with TIGERsearch \citep{Tiger}. For the two other studies, published in \citet{Abeille.2020.JFLS}, we used a larger corpus, Frantext \citep{Frantext}~--- an online collection of texts by various authors in French literature. We explored it with the online search tool of the corpus. Both the FTB and Frantext collect well-edited written productions\footnote{My colleagues and I also presented a corpus study on spontaneous spoken French in \citet{Abeille.2016}.}, but they differ in their typology, since the FTB contains many texts focusing on economical issues, while Frantext often describes protagonists and their interactions. 

The ultimate aim of the two studies in Frantext was to compare \emph{dont} and \emph{de qui} in order to confirm \citeauthor{Tellier.1990}'s (\citeyear{Tellier.1990,Tellier.1991}) intuitions (cf.\ Section~\ref{ch:intro-disscussion-French}). To do this, and because the \emph{de qui} relative clauses only have animate antecedents, we only selected animate antecedents for the \emph{dont} relatives as well. In order to detect changes over time, we selected two similar periods in Frantext: texts published between 1900 and 1913 (179 texts, about 7.8 million words), and texts published between 2000 and 2013 (222 texts, about 13.2 million words).

We looked for occurrences of \emph{dont} in the corpus. Since Frantext contains too many occurrences of this word (see \figref{tab:dont-total}), we only annotated a random subset of the output.

\begin{table}[ht]
\small
    \begin{tabular}{lrrr}
         \lsptoprule
                           &                 & \multicolumn{2}{c}{Frantext}\\\cmidrule(lr){3-4}
                           & French Treebank &  2000--2013 & 1900--1913 \\
         \midrule
         total occurrences of \emph{dont} & 632 & more than 13~000 & close to 10~000 \\
         total annotated         & 632 & 500 & 1300 \\
         among which: & & & \\
         - relative clauses & &  & \\
         with a verb a one gap\footnote{For Frantext: only with animate antecedents, as explained above.} 
                                 & 382 & 123 & 176 \\
         - \emph{c'est} clefts   & 2 & 1 & 3 \\
         \lspbottomrule
    \end{tabular}
    \caption{Occurrences of \emph{dont} in the French Treebank and Frantext}
    \label{tab:dont-total}
\end{table}

These corpora are annotated for part of speech and lemmas. All other annotations had to be done manually. First, it was necessary to remove occurrences that are ``noise'' (i.e.\ false positives of a given query that are not what we were looking for). Second, we wanted to test the impact of certain factors on the results, e.g.\ whether there were any occurrences of extractions out of a subject. To this end, it was necessary to annotate these factors for each occurrence.

We found only 6 \emph{c'est}-clefts among the occurrences we annotated. The two clefts in (\ref{ex:FTB-clefts}) are extractions of the complement of a verb. 

\eal\label{ex:FTB-clefts}
\ex {}[FTB - flmf7ajlep-212]\\
\gll C' est de semi- retraite dont parle M. Tapie.\\
it is of semi retirement of.which talks Mr Tapie\\
\glt `It is semi-retirement that Mr Tapie is talking about.
\ex {}[FTB - flmf3\_11000\_11499ep-11113]\\
\gll C' est de l' indépendance tout court dont il a été finalement question.\\
it is of the independence all short of.which it has been finally question.\\
\glt `In the end, it was simply a matter of independence.'
\zl 

Example (\ref{ex:d1900-clefts-obj}) is extraction out of the object. However, it is presentational, and thus more similar to English \emph{there}-clefts than English \emph{it}-clefts.

\ea (La Mort de Philæ, Pierre Loti, 1909)\nopagebreak\\
\gll C' est la momie d' un embryon humain, dont on avait dans les temps orné [le visage~\trace{}] d' une belle couche d' or~[\dots].\\
it is the mummy of an embryo human of.which one had in the times decorated the face of a nice layer of gold\\
\glt `This is the mummy of a human embryo, of whom someone had decorated the face with a nice layer of gold in a timely manner.'
\label{ex:d1900-clefts-obj}
\z 

The three remaining \emph{c'est}-clefts, reproduced in (\ref{ex:dont-corpus-clefts}), are interesting because they display extraction out of the subject. However, (\ref{ex:d2000-clefts}) is presentational and (\ref{ex:d1900-clefts-subj-2}) is probably presentational as well. Only (\ref{ex:d1900-clefts-subj-1}) is actual focalization by means of extraction. 

\eal \label{ex:dont-corpus-clefts}
\ex(Jean-Christophe : Le Buisson ardent, Romain Rolland, 1911)\\
\gll C' était lui maintenant, dont [les yeux~\trace{}] évitaient les yeux de l' autre.\\
it was him now of.which the eyes avoided the eyes of the other\\
\glt `Now it was him whose eyes avoided the other's eyes.'
\label{ex:d1900-clefts-subj-1}
\ex(Le protocole compassionnel, Hervé Guibert, 2007)\\
\gll C' était donc le jeune homme dont [le livre de chevet~\trace{}] était resté longtemps \emph{Des} \emph{aveugles}.\\
it was then the young man of.which the book of bedside was stayed long of.the blinds\\
\glt `So this is the young man whose bedtime reading had been for a long time \emph{About the blinds}.'
\label{ex:d2000-clefts}
\ex(Sermons, fragments et lettres, Horace Monod, 1911)\\
\gll [\dots] c' est un mourant dont [les traits~\trace{} creusés par la souffrance] s' éclairent d' une céleste joie~[\dots]\\
{} it is a dying of.which the face drilled by the suffering \textsc{refl} shine of an unearthly joy\\
\glt `It's a dying man, whose face, deformed by the suffering light up with an unearthly joy~[\dots].'
\label{ex:d1900-clefts-subj-2}
\zl 

Extraction out of the subject is therefore attested in clefts. Unfortunately, with so little data we cannot draw any further meaningful conclusions. The analysis below does not take the \emph{c'est}-clefts into account.

Furthermore, I only present the analysis for relative clauses with a verb and one gap. The partitive verbless \emph{dont} relative clauses like (\ref{ex:dont-sans-verbe-FTB}) of the French Treebank have already been described by \citet{Bilbiie.2010}. In the present analyses, we ignore gapless \emph{dont} relative clauses like (\ref{ex:dont-pronom-FTB}) and relatives with different gap sites.\footnote{We distinguish between several gaps and different gap sites. If there are several gaps, but with the same syntactic function, like in (\ref{ex:several-gaps-good}), they are included in the results below. If the different gaps have different syntactic functions and are therefore at different gap sites, like in (\ref{ex:several-gaps-bad}), then they are not considered in the result. This holds for for all corpus studies in this work.

\begin{exe}\ex
\begin{xlist}
\ex the car of which [the wheels~\trace{}] and [the brakes~\trace{}] are defective
\label{ex:several-gaps-good}
\ex the car of which [the driver~\trace{}] broke [the wheels~\trace{}]
\label{ex:several-gaps-bad}
\end{xlist}
\end{exe}
}

\ea {}[FTB - flmf7aa1ep-328]\\
\gll En Amérique latine, 23 journalistes ont trouvé la mort, dont 9 en Colombie et 7 au Pérou.\\
in America Latin 23 journalists have found the death of.which 9 in Colombia and 7 in Peru.\\
\glt `In Latin America, 23 journalists have died, among which 9 in Colombia and 7 in Peru.'
\label{ex:dont-sans-verbe-FTB}
\z 

\ea {}[FTB - flmf7ak1ep-272]\\
\gll Un  bel  effort,  dont  l' avenir  dira  s' il  persuade  les  consommateurs,  s' il  suscite  des  imitateurs [\dots]\\
a nice effort of.which the future will.say if it persuades the consumers if it generates \textsc{det} imitators\\
\glt `A fine effort, and the future will tell us whether consumers are persuaded and whether it will be imitated.'
\label{ex:dont-pronom-FTB}
\z 

In order to present all corpus studies in this work in a uniform and consistent way, some minor corrections have been made to the annotation (see the guidelines in Appendix A). The values may therefore vary slightly from those reported in the respective publications.
