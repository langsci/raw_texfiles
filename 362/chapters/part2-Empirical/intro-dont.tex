\label{ch:exp-dont}

I now turn to the experimental and corpus results from my own work that form the empirical basis of this book. In this chapter, I present the results of three corpus studies and four experiments on \emph{dont} relative clauses with inanimate antecedents. Other experiments on \emph{dont} with an animate antecedent will be discussed in the next chapter. The organization of the chapter is the following:

\begin{description}
\item[Corpus studies on \emph{dont}:] In these studies, I annotated occurrences of \emph{dont} in two different written corpora for two time periods (20th century and 21st century). The results show that extractions out of the subject are not only very frequent in written French, but in fact the most frequent use of \emph{dont} in relative clauses. This is not restricted to subjects of passive or unaccusative verbs and is attested in both time periods.

\item[Experiment 1:] In this acceptability judgment study, we cross extraction type (extraction\slash non-extraction) with three different distances between \emph{dont} and the gap (one new referent\slash two new referents\slash three new referents). The shortest distance is extraction out of the subject, the other ones are extractions out of the object. The results show that the longer the distance, the lower the acceptability. Contrary to what is expected from a subject island, we find that extraction out of the subject is rated significantly higher than extraction out of the object.

\item[Experiment 2:] In this speeded acceptability judgment study, we reproduce the design of Experiment 1 with slightly different stimuli. Sentences are presented on the screen one word at a time relatively quickly, and participants have to accept or reject the sentence within two seconds. This technique allow us to reduce the ceiling effects seen in Experiment 1, but we only obtain null effects: all conditions are acceptable to the same degree. Besides a hint that extractions out of the subject are not ruled out by grammar, Experiment 2 does not give any statistical evidence for it and does not falsify any prediction.

\item[Experiment 3:] In this eye tracking study, we test the same experimental material as in Experiment 1. We find relatively little variation in the participants' reading patterns, except that relative clauses seem easier to read than coordinations (non-extractions). If we compare extractions out of the subject with extractions out of the object with a nominal subject (the low and high distance conditions the data indicate at best an increase in processing difficulty for extractions out of the object.

\item[Experiment 4:] In this acceptability judgment study, we cross extraction type (extraction\slash non-extraction\slash ungrammatical controls) with syntactic function (subject\slash object). Extraction out of the subject receives significantly higher ratings than extraction out of the object. There is no interaction effect between extraction type and grammatical function.

\end{description}
