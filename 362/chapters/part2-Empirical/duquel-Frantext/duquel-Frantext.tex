\section{Corpus study on \emph{duquel}}
\label{ch:duquel}

\subsection{Motivation}

We have seen that extraction out of the subject is frequent in \emph{de qui} relative clauses, but not attested in \emph{de qui} interrogatives. The filler \emph{duquel} (lit.\ `of the which') can also be used for extracting \emph{de}-PPs. If the contrast between relative clauses and interrogatives is robust, we expect to find many cases of extraction out of the subject in \emph{duquel} relative clauses as well. 

\subsection{Procedure}

As in the previous studies, we used the Frantext corpus \citep{Frantext}. As we found no evidence for a major change since 1900, we only looked at texts in Frantext published between 2000 and 2013 (222 texts, about 13.3 million tokens). 

The lemma \emph{duquel} can be realized in four forms: \emph{duquel} (masculine singular), \emph{de laquelle} (feminine singular), \emph{desquels} (masculine plural or masculine+feminine plural) and \emph{desquelles} (feminine plural). In Frantext, \emph{duquel, desquels} and \emph{desquelles} are tagged under the lemma ``duquel'', but \emph{de laquelle} is tagged as two different lemmas ``de'' and ``lequel'' (lit.\ `the which'). For this reason, we ran two different queries, one with the lemma ``duquel'' and one with the words ``de laquelle''. In the following, we combine the results of the two queries as a single corpus study.

The two queries combined yield 955 occurrences of \emph{duquel}, which we annotated the same way as in the previous corpus studies. The results confirm that \emph{duquel} is mostly used in relative clauses, but also show that it may occur in interrogatives. There are 941 relative clauses with an antecedent, seven direct and indirect questions and four \emph{c'est}-clefts. The three remaining occurrences are noise, i.e.\ passages written in non-contemporary French.

The seven interrogatives (three direct questions and four indirect questions) are too few to draw any meaningful conclusions. Notice however that there is no example of extraction out of the subject (see \figref{fig:duq-wh}).

\begin{figure}
    \includegraphics[width=0.7\textwidth]{chapters/part2-Empirical/duquel-Frantext/distribution-qu.jpeg}
    \caption{Distribution of \emph{duquel} interrogatives in Frantext 2000--2013 (cplt = complement). See page~\pageref{ch:conf-intervals-binomial} for the confidence intervals (here three comparisons).}
    \label{fig:duq-wh}
\end{figure}

The four \emph{c'est}-clefts are all pied-piping cases like (\ref{ex:duq-cleft-piedpiping}), where \emph{de qui} is either the complement of a preposition or of a noun complement of a preposition. The pied-piped PP is an adjunct with respect to the verb of the relative clause. All \emph{c'est}-clefts except (\ref{ex:duq-cleft-piedpiping}) are presentational.

\ea (Journal : 1928, Christian Lazard, 2012)\\
\gll C’ est le résultat du Congrès radical [à la suite duquel]$_i$ [Herriot, Sarraut, Queuille et Perrier ont démissionné~\trace{}$_i$].\\
it is the result of.the congress radical at the following of.the.which Herriot Sarraut Queuille and Perrier have resigned\\
\glt `It is the result of the Congress of the Radical Party following which Herriot, Sarraut, Queuille and Perrier resigned.'
\label{ex:duq-cleft-piedpiping}
\z  

The following section discusses the results for the relative clauses.

\subsection{Results and analysis}

We excluded five relative clauses without a gap and 11 verbless relative clauses. The functions of \emph{de qui} in the remaining 925 relative clauses with an antecedent and a verb are given in Table \ref{tab:duq} and in \figref{fig:duq}.

\begin{table}[ht]
\centering
    \begin{tabular}{llrr}
         \lsptoprule
         & & Frequency & \% \\\midrule
         \multicolumn{2}{l}{Verb} & 46 & 4.97 \\
         \multicolumn{2}{l}{Noun} & & \\
                                  & Subject & 7 & 0.76 \\
                                  & Object & 10 & 1.08 \\
                                  & Predicate & 2 & 0.22 \\
                                  & Cplt of Preposition & 567 & 61.30 \\
         \multicolumn{2}{l}{Adjective}                  & 4 & 0.43 \\
         \multicolumn{2}{l}{Preposition}                & 286 & 30.92 \\
         \multicolumn{2}{l}{Adjunct}                    & 3 & 0.32 \\
         \lspbottomrule
    \end{tabular}
    \caption{Distribution of \emph{duquel} relative clauses in Frantext 2000--2013}
    \label{tab:duq}
\end{table}

\begin{figure}
    \centering
    \includegraphics[width=0.7\textwidth]{chapters/part2-Empirical/duquel-Frantext/distribution-rel.jpeg}
    \caption[Distribution of \emph{duquel} relative clauses in Frantext 2000--2013]{Distribution of \emph{duquel} relative clauses in Frantext 2000--2013. See page~\pageref{ch:conf-intervals-binomial} for the confidence intervals (here eight comparisons).}
    \label{fig:duq}
\end{figure}

\emph{Duquel} relative clauses differ substantially from \emph{de qui} relative clauses. More than 90\% of the occurrences are pied-piped. Even when \emph{duquel} is the \emph{de}-complement of a subject or object noun, the whole NP is sometimes pied-piped, as illustrated in example (\ref{ex:duq-noun-piedpiping}). This phenomenon, while very common in English, is possible but stylistically marked in French. Notice that both examples are from the same author, and my subjective impression is that both relative clauses (especially \ref{ex:duq-noun-piedpiping-subject}) are nearly infelicitous, unlike the relative clauses cited so far. 

\eal \label{ex:duq-noun-piedpiping}
\ex (La dissolution, Jacques Roubaud, 2008)\\
\gll une publication traînante autour de mon lit où un film avec cet acteur, [la tête duquel]$_i$ [\trace{}$_i$ était reproduite], que je reconnus\\
a work lying around of my bed where a movie with this actor the head of.the.which {} was reproduced that I recognized\\
\glt `a work lying under my bed where a film with this actor, whose head was reproduced, I recognized it/him(?)'
\label{ex:duq-noun-piedpiping-subject}
\ex (La Bibliothèque de Warburg : version mixte, Jacques Roubaud, 2002)\\
\gll j' ai switché un moment dans la lignée temporelle, moment [la durée duquel]$_i$ [je ne peux préciser~\trace{}$_i$]\\
I have switched a moment in the line timely moment the duration of.the.which I \textsc{neg} can specify\textsc{.inf}\\
\glt `I switch for a moment in the timeline, a moment whose duration I can't specify.'
\label{ex:duq-noun-piedpiping-object}
\zl 

Extraction out of NPs is generally very rare with \emph{de quel} + N, so a quantitative analysis is not very appropriate in this case. However, extraction out of the subject is attested, as is extraction out of the object. Extraction out of the predicate is the only kind of extraction whose frequency statistically does not differ from zero. 

There are six cases of extraction out of a subject NP. Five of them are from Anne-Marie Garat, three of them involve subjects of transitive verbs, like (\ref{ex:duq-subject-garat}). 

\ea (La Première fois, Anne-Marie Garat, 2013)\\
\gll ce livre sans images, carte ni gravure~[\dots], duquel [la reliure~\trace{}] tache les doigts de moisissure et [les feuilles~\trace{}] sentent l' amande amère\\
this book without illustration map nor engraving of.the.which the binding stains the finders of mold and the pages smell the almond bitter\\
\glt `this book without any illustration, map or engraving, whose binding stains the fingers with mold and whose pages smell like bitter almond'
\label{ex:duq-subject-garat}
\z 

The last one is a long-distance dependency with extraction out of the subject of an embedded question. Hence, it is a violation of two alleged islands: subject island and \textit{wh}-island. 

\ea (Mécanique, François Bon, 2001)\\
\gll l' ordinateur de plastique tout neuf, duquel$_i$ il vous avait demandé [à quoi servaient [les prises de branchement~\trace{}$_i$, là ,sur le côté]]\\
the computer of plastic all new of.the.which he you\textsc{.dat} had asked at what are.used the plugs of connection there on the side\\
\glt `the brand-new plastic computer, of which he had asked you what the connection plugs there on the side were good for'
\z 

The corpus also includes other cases of extraction out of the subject, even though this is not directly reflected by Table \ref{tab:duq} and \figref{fig:duq}. Example (\ref{ex:duq-subject-complex-embedding}) is extraction out of a subject, albeit \emph{duquel} itself is not a complement of the subject noun (it is annotated as complement of a noun complement of a preposition).

\ea (Programme sensible, Anne-Marie Garat, 2012)\\
\gll la poubelle du verre usagé, [au sujet de laquelle]$_i$ circula [une pétition~\trace{}$_i$ de riverains ulcérés]\\
the trash of.the glass used at.the subject of the.which circulates a petition of residents upset\\
\glt `the glass recycle bin, about which a petition of upset residents circulated'
\label{ex:duq-subject-complex-embedding}
\z 

Notice, however, that \emph{au sujet de laquelle} in (\ref{ex:duq-subject-complex-embedding}) could also potentially be analyzed as an adjunct:
%Because the verb of the relative clause is an unergative, a reanalysis of the complement of \emph{une pétition} as a complement of the verb is not excluded either. 

\ea[]{\gll Une pétition de riverains ulcérés circule [au sujet de la poubelle de verre usagé].\\
a petition of residents upset circulates at.the subject of.the trash of the glass used\\
\glt `A petition of upset residents circulated about the glass recycle bin.'}
\z 

The same does not hold for example (\ref{ex:duq-subject-sentential}), an extraction out of a infinitival subject (\emph{duquel} is annotated as complement of a preposition).

\ea (Pense à demain, Anne-Marie Garat, 2010)\\
\gll une vieillesse tissée de filaments du passé [au-devant duquel]$_i$ [revenir~\trace{}$_i$] fatigue.\\
an old.age woven of filament of.the past toward of.the.which come.back\textsc{.inf} tires\\
\glt `An old age woven with filament from the past back to which to go is tiring' (intended: It is tiring to go back to the past.)
\label{ex:duq-subject-sentential}
\z

%\ex[]{\gll Son idéal au nom duquel vivre et mourir n' est plus la révolution, mais la lutte contre le nazisme.\\
%his ideal at.the name of.the.which live\textsc{.inf} and die\textsc{.inf} \textsc{neg} is no.more the revolution but the fight against the Nazism\\
%(Histoire des grands-parents que je n'ai pas eus~: une enquête, Ivan Jablonka, 2012)}

\subsubsection{Subject position}

\figref{fig:duq-inv} 
    \begin{figure}
    \centering
    \includegraphics[width=\textwidth]{chapters/part2-Empirical/duquel-Frantext/inversion.jpeg}
    \caption{Proportion of subject-verb inversion in \emph{duquel} relative clauses}
    \label{fig:duq-inv}
    \end{figure}
shows the proportion of postverbal subjects in the \emph{duquel} relative clauses. The distribution is similar to that for \emph{de qui}: Postverbal subject are frequent when \emph{duquel} is a verb complement or in pied-piping structures, and there are some cases of extraction out of an inverted subject (even though the inversion increases the distance between the filler and the gap). 

\subsubsection{Other factors}

As already mentioned, the numebr of extractions out of the subject (and of anything else than pied-piping in general) is too low to allow any meaningful comparison. It is, for example, not possible to compare the different verb types, but extraction out of the subject of transitive verbs is attested, cf.\ example (\ref{ex:duq-subject-garat}). 

Further annotation of number, definiteness and restrictiveness shows that most of the cases of extraction out of the subject are non-restrictive. The same holds for extraction of the complement of the verb, but not for other kinds of extraction out of NPs. The details are reported in Appendix~\ref{ch:other-factors}.

\subsection{Conclusion}

The filler \emph{duquel} is used almost exclusively for pied-piping. Just like \emph{de qui} relative clauses, the usage for extraction out of NPs seems to be stylistically marked (it is found with a very small subset of authors), except when it occurs  in pied-piping constructions.

However, extraction out of the subject is attested with statistically non-zero frequency, comparable to extraction out of the object. There is therefore no strong evidence in favor of the subject island hypothesis. 

Additionally, extraction out of the subject is found with long-distance dependencies and inverted subjects. Our corpus also included one example of extraction out of an infinitival subject. Of course, one data point is not much evidence, but corpus studies are probably not ideal for investigating infinitival and sentential subjects, as they are rare. I will come back to infinitival and sentential subjects in Sections~\ref{ch:exp15} and \ref{ch:exp16} in two experimental studies.

Thus the results of the corpus study on relative clauses with \emph{de qui} tell us more about extraction out of subjects than the present study, but \emph{duquel} relative clauses do not contradict the previous findings.
