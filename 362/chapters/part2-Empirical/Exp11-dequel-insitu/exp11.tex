\section[head=Experiment 11]{Experiment 11: Acceptability Judgment study on \emph{de quel} \emph{wh}-questions with the \emph{wh}-word in situ}
\label{ch:exp11}

In general, a distinction is made between languages with \textit{wh}-ex-situ and with \textit{wh}-in-situ. But even languages with \textit{wh}-ex-situ usually allow the \textit{wh}-element to remain in situ in order to express echo questions or mirative questions.

As we have seen, French belongs to the category of \textit{wh}-ex-situ languages, as do all Romance languages. Most Romance languages rule out the in-situ option for information seeking questions. This is not the case in French \citep{Cheng.2000} or Portuguese \citep{Ambar.2002}, where the in-situ variant is acceptable, even beyond echo questions and mirative questions.

\eal
\ex French\\ 
\gll Jean a acheté quoi~?\\
Jean has bought what\\
\glt `What did Jean buy?'
\label{ex:in-situ-French}
\ex Portuguese \citep{Kaiser.2015}\\
\gll O Jo\~{a}o comprou o quê?\\
\textsc{det} Jo\~{a}o bought \textsc{det} what\\
\glt `What did Jo\~{a}o buy?'
\label{ex:in-situ-Portugese}
\zl 

Echo and mirative questions differ in French from in-situ information seeking questions in prosodic and syntactic\footnote{Generally speaking, information seeking in-situ questions are more syntactically constrained than other in-situ questions. According to \citet{Schlonsky.2012}, the \textit{wh}-element cannot be within the scope of negation. Some authors also argue that it cannot be inside an embedded clause introduced by a non-factive verb, but the evidence seems weak. For a summary of these debates, see \citet{Kaiser.2015}.} respect. They also differ from a pragmatic point of view, but we will come back to this point in the discussion below.

Interrogatives with in-situ \emph{wh}-words are interesting for our central question. Syntactic accounts usually assume covert movement (in order to check a \textit{wh-} feature): If this is the case, interrogatives with an in-situ \emph{wh}-word should pattern just like interrogatives with extraction (like in Experiment~11). Under processing and discourse-based accounts, on the other hand, there should be no island effect without extraction. Interrogatives with an in-situ \emph{wh}-word should therefore not show any penalty when extracting out of the subject. Finally, extraction is not a relevant factor for the FBC constraint, so the constraint should apply the same way, except if in-situ questions in French have a special information structure.

\subsection{Design and materials}

In this experiment, we adapted the materials from the previous experiment. The subextraction condition appeared without extraction, but with \emph{de quel} + N in situ inside the subject NP and the object NP, respectively:

\eal 
\ex[]{{Condition subject + \emph{wh} in-situ:}\\
\gll [L‘ originalité de quelle innovation] enthousiasme mes collègues sans aucune raison~?\\
the uniqueness of which innovation excites my colleagues without any reason\\
\glt `The uniqueness of which innovation excite my colleagues for no reason?'}
\label{ex:exp11-subj-pp}
\ex[]{{Condition object + \emph{wh} in-situ:}\\
\gll Mes collègues admirent [l' originalité de quelle innovation] sans aucune raison~?\\
my colleagues admire the uniqueness of which innovation without any reason\\
\glt `My colleagues admire the uniqueness of which innovation for no reason?'}
\label{ex:exp11-obj-pp}
\zl 

We also used the polar conditions from Experiment~10 as `non-island' controls:

\eal \label{ex:exp11-no}
\ex[]{{Condition subject + no-wh:}\nopagebreak\\
\gll Est - ce que l' originalité de cette innovation enthousiasme mes collègues sans aucune raison~?\\
is {} it that the uniqueness of this innovation excites my colleagues without any reason\\
\glt `Does the uniqueness of this innovation excite my colleagues for no reason?'}
\label{ex:exp11-subj-no}
\ex[]{{Condition object + no-wh:}\\
\gll Est - ce que mes collègues admirent l' originalité de cette innovation sans aucune raison~?\\
is {} it that my colleagues admire the uniqueness of this innovation without any reason\\
\glt `Do my colleagues admire the uniqueness of this innovation for no reason?'}
\label{ex:exp11-obj-no}
\zl 

Unlike Experiment~10, we did not include ungrammatical controls, because we though they would make the experiment unnecessarily long.

We tested the same 24 items than in Experiment~10, each manipulated according to the four conditions I just described. In addition, the experiment included 36 distractors. They were a mixture of declaratives and interrogatives, some of them ungrammatical. 
Around 60\% of the experimental items and distractors were followed by a comprehension question that could be answered by selecting \emph{Oui} (`Yes') or \emph{Non} (`No').
The comprehension questions sometimes had the form of an interrogative, and sometimes of a declarative. The sample item above did not have a corresponding comprehension question; a sentence like \emph{La couleur de quelle fleur charme les vieilles dames durant leur promenade matinale~?} (`The color of which flower delights the old ladies during their morning walk?') was followed by the comprehension question \emph{Les vieilles dames se promènent le matin~?} (`The old ladies go for a walk in the morning?'). Accuracy was very high in all conditions, indicating that participants correctly understood the task.

\subsection{Predictions}

% difference ex-situ vs. in-situ: Aoun et al. 1981 + Lai-Shen Cheng and Rooryck 2000

Movement in Minimalism is usually motivated by the need for the extracted element to check a \emph{wh}-feature high in the syntactic structure. Movement is thus necessary for interpretation purposes, and in order for the element to have its \emph{wh}-form. Consequently, scholars like \citet{Watanabe.1992} or \citet{Boskovic.1998} assume covert movement for in-situ questions: the element moves in deep structure, but its phonological realization remains in situ. Chinese has frequently been used as evidence that there is indeed covert movement: even though there is no (overt) extraction in Chinese, interrogatives seem to be subject to island constraints \citep{Huang.1982,Cheng.1991}. 

Hence, such a syntactic account predicts a subject island effect in French: \emph{wh}-words inside the subject as in (\ref{ex:exp11-subj-pp}) should be degraded compared to \emph{wh}-words inside the object (\ref{ex:exp11-obj-pp}), and an interaction effect is expected, such that \emph{wh}-words inside the subject (\ref{ex:exp11-subj-pp}) are also worse than the non-extraction controls (\ref{ex:exp11-subj-no}) and (\ref{ex:exp11-obj-no}).

In this case, the FBC constraint also predicts the same results. If in-situ \emph{wh}-words are focalization, then the FBC constraint is violated when the \emph{wh}-word is in a subject, which is presumably backgrounded.\largerpage

In Minimalism, the movement hypothesis for in-situ questions has been challenged by \citet{Comorovski.1996}, \citet{Reinhart.1997} and \citet{Adli.2006}, among others. According to them, there is no evidence of covert movement and thus of the necessity of feature checking.{\interfootnotelinepenalty=10000\footnote{\citet{Cheng.2000} propose a somewhat different view, in which a \emph{wh}-morpheme checks the \emph{wh}-feature thus there is no need for movement of the \emph{wh} word. See \citet{Adli.2006} for criticism of their main argument (intonation in in-situ questions) in French.}} In particular, \citet{Reinhart.1997} presents data from Chinese and Korean that contradict \citet{Huang.1982} and \citet{Cheng.1991}. This variant of the syntactic approach predicts null effects for this experiment. Processing accounts, or discourse-based accounts other than the FBC, make the same predictions.

In any case, a main effect of question type is expected, for independent reasons: previous studies have shown that in-situ \emph{wh}-questions receive low ratings in French \citep{Thiberge.2018}.\footnote{However, \citet{Adli.2006} compares in-situ questions with ex-situ questions in French through a graded acceptability judgment task and a self-paced reading task, and finds no significant difference in either acceptability or reading times. In the acceptability judgment task, the experimental item follows a context sentence evoking a colloquial situation. This may be the reason why he does not see a contrast between question types.}
This is probably a distortion effect caused by social desirability \citep{Edwards.1957}, given that in-situ \emph{wh}-questions are usually considered to belong to non-standard colloquial French. \citet[98]{Coveney.1996} shows that prescriptive French grammars strongly stigmatize in-situ questions as colloquial speech. This question type is also associated with a lower social status. 
\citet{Thiberge.2018} conducted a between-subject study in which participants read an interview and gave their impression of the journalist. The only difference between the conditions was the type of question used by the journalist. The results show that the journalist was judged overall less educated and from a lower sociological background when they produced in-situ questions. They were also judged younger and less ``parisian'' than with an ex-situ question with subject-verb inversion.\footnote{Ex-situ questions without inversion are also somewhat stigmatized.}\citet{Thiberge.2018} concludes that participants are negatively biased against in-situ questions.

\subsection{Procedure} 

We conducted the experiment on the Ibex platform \citep{Ibex}. The procedure was similar to that used in the previous acceptability judgment experiments (see Section \ref{ch:methodo-AJ}). Participants rated the sentences on a Likert scale from 0 to 10, 0 being labeled as ``bad'' and 10 being labeled as ``good''. They also answered comprehension questions after some of the sentences.

The experiment took approximately 20 minutes to complete. 

\subsection{Participants}

The study was run between July and September 2018. 
Participants were recruited on the R.I.S.C.\ website (\url{http://experiences.risc.cnrs.fr/}) and on social media (e.g.\ Facebook).
They received no financial compensation. 

30 participants took part in the experiment. 
The analysis presented here is based on the data from the 24 participants who satisfied all criteria.\footnote{To calculate accuracy, we excluded not only the answers to comprehension questions of the practice items and of the ungrammatical controls, but also of the ungrammatical distractors.}
They were aged 21 to 73 years. 19 of them self-identified as women, three self-identified as men. Three participants (12.50\%) indicated having an educational background related to language.

\subsection{Results and analysis}

\figref{fig:exp11-boxplot} shows the results of the acceptability judgment study.
For the \emph{wh}-word in-situ inside the subject (\ref{ex:exp11-subj-pp}), the mean rating was 6.25, slightly higher than for the \emph{wh}-word in-situ inside the object (\ref{ex:exp11-obj-pp}) with 6.09. The conditions without a \emph{wh}-word were rated higher: the subject control condition (\ref{ex:exp11-subj-no}) had a mean acceptability rating of 8.97, the object control condition (\ref{ex:exp11-obj-no}) of 8.54. 

\begin{figure}
    \centering
    \includegraphics[width=\textwidth]{chapters/part2-Empirical/Exp11-dequel-insitu/boxplots.jpeg}
    \caption{Acceptability judgments by condition in Experiment~11. The grey box plots indicate the median and quartiles of the results. Black points are outliers. Mean and confidence intervals are indicated in white.}
    \label{fig:exp11-boxplot}
\end{figure}

\figref{fig:exp11-boxplot} suggests potential ceiling effects in the control conditions. This is confirmed by the distribution in \figref{fig:exp11-repartition}. The ratings for the conditions with the \emph{wh}-word in situ seem to be distributed almost evenly along the scale, with a small peak at the top of the scale. 

\begin{figure}
    \centering
    \includegraphics[width=\textwidth]{chapters/part2-Empirical/Exp11-dequel-insitu/repartition.jpeg}
    \caption{Density of the ratings across conditions for Experiment~11}
    \label{fig:exp11-repartition}
\end{figure}

Another representation of the results is given by the ROC and zROC curves of the results in \figref{fig:exp11-ROC}. The ROC curves show that participants discriminated between the conditions with the \emph{wh}-word in-situ (the baseline) and the polar questions. The object and subject curves are very close. The zROC curve for the subject condition is not parallel to the baseline, which, following \citet{Dillon.2019}, can be a sign that there is more variance in one condition. We used the function \texttt{var.text()} from the package \texttt{stats} \citep{R} for a two-by-two comparison of the variances and, indeed, there are significant differences for every condition except when we compare the two conditions with a \emph{wh}-word in situ. 

\begin{figure}
    \centering
    \includegraphics[width=\textwidth]{chapters/part2-Empirical/Exp11-dequel-insitu/ROC.jpeg}
    \includegraphics[width=\textwidth]{chapters/part2-Empirical/Exp11-dequel-insitu/zROC.jpeg}
    \caption{ROC curves (top) and zROC curves (bottom) of the non-extraction conditions compared to their respective subextraction conditions, represented by the dotted grey baseline (\citealt{Dillon.2019}'s method) in Experiment~11.}
    \label{fig:exp11-ROC}
\end{figure}

The ROC and zROC curves in \figref{fig:exp11-ROC-subj} show the discrimination between the subject and object conditions. We can see that participants hardly discriminate between the subject and the object variants. Again, the zROC curve for the polar question is not parallel to the baseline.

\begin{figure}
    \centering
    \includegraphics[width=\textwidth]{chapters/part2-Empirical/Exp11-dequel-insitu/ROC-subject.jpeg}
    \includegraphics[width=\textwidth]{chapters/part2-Empirical/Exp11-dequel-insitu/zROC-subject.jpeg}
    \caption{ROC curves (top) and zROC curves (bottom) of the object conditions compared to their respective subject conditions, represented by the dotted grey baseline (\citealt{Dillon.2019}'s method) in Experiment~11.}
    \label{fig:exp11-ROC-subj}
\end{figure}

\subsubsection{Habituation} 

\figref{fig:exp11-habituation} on page \pageref{fig:exp11-habituation} shows the habituation effects in the course of the experiment. Clearly, the ratings are grouped by question type: the polar questions are at the top, the questions with a \emph{wh}-word at the bottom. Subject and object variants are relatively similar: the polar questions show almost no habituation effect, while there is strong habituation in the questions with a \emph{wh}-word in-situ. At the end of the experiment, ratings for the in-situ questions were very close to the polar ones. 

\begin{figure}
    \centering
    \includegraphics[width=\textwidth]{chapters/part2-Empirical/Exp11-dequel-insitu/habituation.jpeg}
    \caption{Changes in the mean acceptability ratings ($z$-scored by participant) by condition in the course of Experiment~11}
    \label{fig:exp11-habituation}
\end{figure}

\pagebreak
\subsubsection{Age} 

Previous experiments have shown that older participants reject in-situ interrogatives more strongly than younger participants \citep[72]{Thiberge.2018}. \figref{fig:exp11-age} on page \pageref{fig:exp11-age} displays the effect of age on participants' rating ($z$-scored ratings) in this experiment. The results are surprising: object conditions are not affected by the factor of age, but the subject conditions are, especially when the \emph{wh}-word is inside the subject. Older participants seem to accept in-situ \emph{wh}-word better in subjects than in objects, and better than younger participants. 

We therefore fitted a first model to predict the ratings of the questions with an in-situ \emph{wh}-word inside the subject with trial number and age as explanatory variables. We included random slopes for trial number grouped by participants and items. The results of the model are reported in Table \ref{tab:exp11-m1}. There is no significant main effect of age, even though there is a significant main effect of habituation. Hence, the impression given by \figref{fig:exp11-age} is not corroborated by the model.

\begin{figure}
    \centering
    \includegraphics[width=\textwidth]{chapters/part2-Empirical/Exp11-dequel-insitu/age.jpeg}
    \caption{Mean acceptability ratings ($z$-scored by participant) by condition in  Experiment~11 depending on the participants' age.}
    \label{fig:exp11-age}
\end{figure}

% latex table generated in R 3.6.3 by xtable 1.8-4 package
% Sun Apr 26 23:02:15 2020
\begin{table}
\begin{tabular}{l S[table-format=1.3] S[table-format=1.3] c S[table-format=<1.3] S[table-format=1.2]}
  \lsptoprule
 & {Est.} & {SE} & {$z$} & {$\text{Pr}(>|z|)$} & {OR} \\ 
  \midrule
  syntactic function & 0.474 & 0.465 & 1 & 0.307 & 1.61 \\ 
  trial & 0.034 & 0.014 & 2 & <.05 & 1.03 \\ 
  syntactic function:trial & 0.014 & 0.015 & 1 & 0.342 & 1.01 \\ 
   \lspbottomrule
\end{tabular}
\caption{Results of the Cumulative Link Mixed Model (model n$^{\circ}$2)}
\label{tab:exp16-m2}
\end{table}


\subsubsection{Comparing subject in-situ questions with object in-situ questions}

We fitted a second model to compare the in-situ questions on their own (mean centered with subject coded negative, object coded positive). We included trial number and age as covariates, and random slopes for the fixed effects and trial numbers grouped by participants and items. The results of the model are reported in Table \ref{tab:exp11-m2}. 
There is a significant effect of habituation, but no significant effect of syntactic function or age.

% latex table generated in R 3.6.3 by xtable 1.8-4 package
% Sun Apr 26 23:02:02 2020
\begin{table}
\begin{tabular}{l S[table-format=1.3] S[table-format=1.3] c S[table-format=<1.3] S[table-format=1.2]}
  \lsptoprule
                     & {Estimate} & {SE} & {$z$} & {$\text{Pr}(>|z|)$} & {OR} \\ 
  \midrule
  syntactic function & 1.328 & 0.488 & 3 & <.01 & 3.77 \\ 
  trial              & 0.069 & 0.023 & 3 & <.005 & 1.07 \\ 
 \lspbottomrule
\end{tabular}
\caption{Results of the Cumulative Link Mixed Model (model n$^{\circ}$1)}
\label{tab:exp16-m1}
\end{table}


In a third model, we compared all four conditions. The model crossed syntactic function and question type (mean centered with in-situ coded positive, no \emph{wh}-word coded negative). We included trial number and age as covariates, and random slopes for all fixed effects and for trial number grouped by participants and items. The results of the model are reported in Table \ref{tab:exp11-m3}. 
There is a significant main effect of question type, such that the polar questions are rated significantly higher than the questions with a \emph{wh}-word in-situ. There is also a significant main effect of habituation, but no significant main effect of age, and no significant interaction. This is corroborated by the AUCs: if we compare the the AUCs (green and red curves on \figref{fig:exp11-ROC-subj}), the difference is not significant. \figref{fig:exp11-interaction} illustrates the interaction effect: the lines are parallel (no interaction).  

% latex table generated in R 3.6.3 by xtable 1.8-4 package
% Fri Apr 10 17:22:54 2020
\begin{table}
\begin{tabular}{l S[table-format=-1.3] S[table-format=1.3] S[table-format=3.2] S[table-format=-1] S[table-format=<1.4] S[table-format=1.2]}
  \lsptoprule
 & {Estimate} & {SE} & {df} & {$t$} & {$\text{Pr}(>|t|)$} & {OR} \\ 
  \midrule
(Intercept) & 0.856 & 0.577 & 152.52 & 1 & 0.1403 & 2.35 \\ 
  extraction type & -0.131 & 0.061 & 23.29 & -2 & <.05 & 1.14 \\ 
  distance & 0.169 & 0.105 & 25.66 & 2 & 0.1213 & 1.18 \\ 
  length & 0.557 & 0.014 & 521.85 & 39 & <.001 & 1.74 \\ 
  extraction type:distance & -0.042 & 0.073 & 29.31 & -1 & 0.5741 & 1.04 \\ 
   \lspbottomrule
\end{tabular}
\caption{Results of the Linear Mixed Model (model n$^{\circ}$1)}
\label{tab:exp03-m1}
\end{table}


\begin{figure}
    \centering
    \includegraphics[width=\textwidth]{chapters/part2-Empirical/Exp11-dequel-insitu/interaction.jpeg}
    \caption{Interaction between syntactic function and extraction type in Experiment~11}
    \label{fig:exp11-interaction}
\end{figure}

\subsection{Discussion}

Direct interrogatives with the \emph{wh}-word in situ received lower ratings in general that the polar controls (model n$^{\circ}$3). This was expected given previous studies on in-situ questions in French.

However, there was no evidence for the interaction characteristic of ``subject island'' effects (model n$^{\circ}$3). There was also no significant difference in ratings between a \emph{wh}-word inside the subject and a \emph{wh}-word inside the object (model n$^{\circ}$2). 

The predictions of syntactic accounts that assume covert movement even with \textit{wh}-in-situ are therefore not borne out. At the same time, the results of this study are similar to the data on Chinese and Korean presented by \citet{Reinhart.1997}, in which she observes no ``subject island'' effect in in-situ questions. 

The results also do not align with the predictions of the FBC constraint. From this, I see two possible conclusions to be drawn:
\begin{enumerate}
    \item The current formulation of the FBC constraint is incorrect and should include the existence of a gap  as a necessary condition for discourse clash. However, there is no clear a priori reason why this syntactic factor should play a role in a discourse-based constraint.
    \item The FBC constraint is correct, but the above predictions were based on the~-- wrong~-- assumption that in-situ \emph{wh}-words denote focalization. The main problem with this explanation is that it is post hoc. However, it finds support on the literature on in-situ questions.
\end{enumerate}

It could also be the case that the sentences in the experiment were not interpreted as information seeking questions. We expected that reading based on our intuitions concerning the stimuli, and given the lack of context that would license a reading as an echo question or mirative question. But since the reading was not supported by either prosodic markers or context, we may have made an incorrect assumption.

I will now pursue point 2. and outline the arguments that seem to indicate that in-situ questions differ in their discursive status from ex-situ questions.

\subsection{The discursive status of in-situ questions}

There is an ongoing debate in the literature about the exact pragmatic status of in-situ interrogatives in French. 

\citet{Coveney.1996}, \citet{Boeckx.1999}, \citet{Cheng.2000} and \citet{Beyssade.2006} (a.o.) claim that in-situ questions are used in French whenever the proposition of the utterance is presupposed, or whenever it has been activated in the preceding discourse. However, \citet{Adli.2006} correctly underlines that in-situ questions can be used out of the blue:

\ea \citep[184]{Adli.2006}\nopagebreak\\
\gll Pardon, il est quelle heure~?\\
sorry it is which hour\\
\glt `Sorry, what's the time?'
\z 

\citet{Larrivee.2016} conducted several corpus searches and compared the results with corpus studies by other scholars; he comes to the conclusion that in-situ questions in French were restricted to ask for discourse-given (activated) referents from the 15th to the middle of the 19th century. During that period, in-situ questions represent less that 0.25\% of the questions in written texts. From the end of the 19th century to the present, the usage of in-situ questions has increased drastically in written French, while the restriction to discourse given referents has vanished. They are still mostly used to mark orality in written texts.  

We can therefore say that the debate is ongoing, and we do not have the last word yet on the pragmatic status of in-situ questions. It seems clear, however, that this status is not standard. Consequently, it is possible that the default reading is not focalization. This could explain the results (or lack of effects) in our experiment.
