\section[head=Experiment~3]{Experiment~3: Eye tracking study on \emph{dont} relative clauses with different linear distances}
\label{ch:exp03}

In this experiment, we tested the materials from Experiments~1 and 2 with another experimental method: eye tracking. Eye tracking is a useful method to identify processing difficulties (see Section~\ref{ch:method-eye-tracking} below).

\subsection{Design and materials}

In this eye tracking experiment, the whole sentence appeared on the screen and participants had to read it before going on to the next sentence. Therefore, we defined regions of interest on the sentence and measured fixations, i.e.\ the time the participants' eyes stayed on each region. Reading times on the first region are not very informative because they reflect not only the actual reading but also the reaction time to the beginning of the reading task, as well as potential correction of fixations for the beginning of the line. Therefore, reading times in this region cannot be compared to reading times of other regions. The last region also is not very informative, because reading times will be long regardless of the material presented in the region~-- this is known as the wrap-up effect \citep{Rayner.1995}. We therefore want to define regions of interest in such a way that the first and last regions are not relevant for the outcome of the experiment.

In our case, we wanted to compare the reading times for subjects with a gap to the reading times for objects with a gap. For this reason, we used the same stimuli as in Experiment~1, but introduced some material after the direct object, resulting in sentences such as the following (square brackets indicate regions for explanatory purposes, but there was no indication of regions on the screen):

\eal 
\ex[]{{Condition narrow-distance + PP-extracted:}\\
\gll [Ils présentent une innovation]$_1$ [dont]$_2$ [l' originalité]$_3$
[émerveille]$_4$ [mes collègues]$_5$ [sans aucune raison]$_6$.\\
\sbar{}they present an innovation \sbar{}of.which \sbar{}the uniqueness \sbar{}delights \sbar{}my colleagues \sbar{}without any reason\\
\glt `They present an innovation of which the uniqueness delights my colleagues for no reason.'}
\label{ex:exp03-subj-pp}
\ex[]{{Condition medium-distance + PP-extracted:}\\
\gll [Ils présentent une innovation]$_1$ [dont]$_2$ [nous]$_3$ [apprécions]$_4$ [l' originalité]$_5$ [sans aucune raison]$_6$.\\
\sbar{}they present an innovation \sbar{}of.which \sbar{}we \sbar{}value \sbar{}the uniqueness \sbar{}without any reason\\
\glt `They present an innovation of which we value the uniqueness for no reason.'}
\label{ex:exp03-obj-clitic-pp}
\ex[]{{Condition wide-distance + PP-extracted:} \nopagebreak \\
\gll [Ils présentent une innovation]$_1$ [dont]$_2$ [mes collègues]$_3$ [apprécient]$_4$ [l' originalité]$_5$ [sans aucune raison]$_6$.\\
\sbar{}they present an innovation \sbar{}of.which \sbar{}my colleagues \sbar{}value \sbar{}the uniqueness \sbar{}without any reason\\
\glt `They present an innovation of which my colleagues value the uniqueness for no reason.'}
\label{ex:exp03-obj-pp}
\zl 

\eal 
\ex[]{{Condition narrow-distance + noextr:}\nopagebreak\\
\gll [Ils présentent une innovation]$_1$ [et]$_2$ [son originalité]$_3$ [émerveille]$_4$ [mes collègues]$_5$ [sans aucune raison]$_6$.\\
\sbar{}they present an innovation \sbar{}and \sbar{}its uniqueness \sbar{}delights \sbar{}my colleagues \sbar{}without any reason\\
\glt `They present an innovation and its uniqueness delights my colleagues for no reason.'}
\label{ex:exp03-subj-no}
\ex[]{{Condition medium-distance + noextr:}\nopagebreak\\
\gll [Ils présentent une innovation]$_1$ [et]$_2$ [nous]$_3$ [apprécions]$_4$ [son originalité]$_5$ [sans aucune raison]$_6$.\\
\sbar{}they present an innovation \sbar{}and \sbar{}we \sbar{}value \sbar{}its uniqueness \sbar{}without any reason\\
\glt `They present an innovation and we value its uniqueness for no reason.'}
\label{ex:exp03-obj-clitic-no}
\ex[]{{Condition wide-distance + noextr:}\nopagebreak\\
\gll [Ils présentent une innovation]$_1$ [et]$_2$ [mes collègues]$_3$ [apprécient]$_4$ [son originalité]$_5$ [sans aucune raison]$_6$.\\
\sbar{}they present an innovation \sbar{}and \sbar{}my colleagues \sbar{}value \sbar{}its uniqueness \sbar{}without any reason\\
\glt `They present an innovation and my colleagues value its uniqueness for no reason.'}
\label{ex:exp03-obj-no}
\zl 

We tested 30 items, each appearing in the six conditions already described. One item was excluded from the results because of a typo in one of the conditions. In addition, the experiment included 32 distractors. 

\subsection{Experimental method}
\label{ch:method-eye-tracking}

% Reaction time research: Jiang 2012, ch 2
% Jiang 2012, Conducting reaction time research in second Language Studies
% Colantoni et al. 2015 ch 3
% Colantoni, Steele & Escudero 2015, Second Language Speech: Theory and Practice

Unlike standard acceptability judgment tasks, which record offline measures, eye tracking (as well as self-paced reading, a method used in Experiment~9) provides online measures that are assumed to reflect the ongoing processing of sentences. This assumption is based on the eye-mind hypothesis \citep{Just.1980}, namely that readers look at the area on the screen currently processed. Under this hypothesis, longer reading times reflect higher processing difficulties (\citealt{Staub.2007}; \citealt[65]{Conklin.2018}).

In an eye tracking study, participants typically face a screen with visual stimuli, while their eye movements are recorded by the eye tracking device \citep[33--35]{Conklin.2018}. In our case, the stimuli were static, and the participant's task was to read one sentence at a time.

The eye-tracker records fixations and saccades of the participant's eye. While early measures such as first fixation duration usually reflect the reader's lexical access to the words in the critical region, intermediate and late measures such as regression path or second pass reading times are more likely to reflect processing difficulties linked to discourse or contextual factors \citep[Section 3.2.1]{Conklin.2018}. However, syntactic processing difficulties can be reflected by virtually any of these measures (\citealt{Clifton.2007}; \citealt[90]{Conklin.2018}). Here is a list of the different measures that were used in the present experiment:

\begin{description}
    \item[First fixation duration:] The first fixation is the first time a fixation is recorded inside the critical region. It does not include the duration of another fixation inside the same region, not even a second fixation inside the region without any saccade outside the region between the two. \\ 
    The first fixation is usually associated with lexical retrieval, but it is also relevant in the so-called spillover region, i.e.\ the region after the critical region. Spillover effects occur when processing difficulties in one region induce an increase in reading time in the following region. The reader may for example look forward in the sentence to find some piece of information that may help the processing of the critical region.
    \item[Regression path duration:] ``Regression'' refers to the whole time spent between the first fixation inside the critical region and the first recorded fixation on a region to the right. Therefore, it may include several saccades and fixations backward to the left of the critical region. It is a good indicator of syntactic or pragmatic difficulties. For example, the reader may need to go back to a previous region in order to confirm that they correctly interpreted previous words, or to correct a false interpretation. I also looked at regression path duration in the spillover region in order to identify potential spillover effects.
    \item[Regression out:] the number of backward regressions from the critical region to previous regions in the sentence.
    \item[Regression in:] the number of backward regressions to the critical region from a later region in the sentence.
    \item[Total reading time:] the sum of all fixation durations in the critical region.
\end{description}

The experiment was designed to isolate the two factors of interest: linear distance of the dependency and extraction type. In addition, the length (= number of characters) of the regions 3 and 5 was kept equal as much as possible for nominal NPs, and was included in the models as explanatory factors. Another factor that can impact reading time on a word is participants' familiarity with the word; familiar words are typically read faster than unfamiliar words. For this reason, we also took into account word frequency as explanatory variable. We assigned a frequency value to the region 3, 4 and 5 based on the frequency of the head noun for region 4 and 6, and on the frequency of the verb for region 5. These values were taken from \url{lexique.org}.\footnote{The Lexique database was implemented by Boris New and Christophe Pallier. We used the frequency of the lemma, called \texttt{freqlemfilm2} and based on the frequency in French subtitles.}
%The distribution of the frequency is not normal, and we are using for this reason here the log-transformed frequency, that is more close to a normal distribution. 
Of course corpus frequencies are only an approximation of the participant's familiarity with a certain word, but they are a good predictor for the influence of lexical access on reading times \citep{Rayner.1986}. Finally, the predictability of a word also affects its reading time \citep{Ehrlich.1981}. We relied on our intuition that relational nouns were all equally predictable and plausible in the test items, but did not perform a pretest to control for this factor.

\subsection{Predictions}
\label{ch:exp03-predictions}

As in Experiment~1 and Experiment~2, the aim of this experiment was to compare the predictions of the traditional syntactic account with the predictions of a processing account based on memory load.

\subsubsection{Subject (region 3)}

Focusing on the subject region, the pre-critical region is region 2 (the conjunction or the relative word), and the spillover region is region 4. We can only compare condition (\ref{ex:exp03-subj-pp}) with condition (\ref{ex:exp03-obj-pp}), since (\ref{ex:exp03-obj-clitic-pp}) is too short.

Accounts that treat subjects as islands predict processing difficulties when integrating the gap with the subject. This difficulty may be reflected on the subject itself, but it is more likely to create a spillover effect on the next region (the verb). We therefore examined regression path durations on region 3 and first fixation and regression path durations on region 4. These measures should be longer for (\ref{ex:exp03-subj-pp}) than for the other conditions. If we assume that the gap inside the subject is unexpected, then we should see a higher rate of regression out  for extractions out of the subject (\ref{ex:exp03-subj-pp}) (e.g., the reader going back to the relative word to check its requirement for a \emph{de}-PP). For the same reason, we also expect more regressions in (e.g., once readers reach the object and realize that the gap is filled by the possessive article, they go back to the subject to check its compatibility with a gap). 

A processing account based on memory costs predicts the exact opposite. Under such an account, the reader is more likely to posit a gap inside the subject than at any other site. Reading times should then be shorter for subextraction out of the subject, and filled-gap effects should occur in subextraction out of the object. Regression path durations on region 3 as well as first fixation and regression path durations on region 4 are therefore expected to be longer for extraction out of the object (\ref{ex:exp03-subj-pp}) than in the other conditions.

Notice however that some scholars, like \citet{Yoshida.2014}, take even a decrease in reading time on the subject in extractions out of the subject to be compatible with the subject island hypothesis. They argue that a gap is never posited in the subject, because that would be ungrammatical. Therefore, the reader ``gives up'' on integrating the filler. There are reasons to suspect that this is a post hoc explanation which was proposed to deal with a decrease of reading times in their data. More importantly, I judge this argumentation more adequate for self-paced reading than for eye tracking~--- and, indeed, \citet{Yoshida.2014} apply the explanation to results from self-paced reading. In my opinion, as far as the reader is able to go back and try to make sense of the sentence during an eye tracking experiment, they will do so before ``giving up''. Hence, I put this line of explanation aside for the present experiment, even though it casts doubt on the predictions of traditional syntactic accounts that I just described. I will consider the ``giving up'' scenario in Experiment~9. 

\subsubsection{Object (region 5)}

Focusing on the object region, the pre-critical region is region 4 (the verb), and the spillover region is region 6. Unfortunately, in this configuration, the spillover region is also the last region, and potential wrap-up effects may interfere with identifying spillover effects. Another potentially interesting pre-critical region is region 3 (the conjunction or the relative word). 

We can first compare condition (\ref{ex:exp03-subj-pp}) with condition (\ref{ex:exp03-obj-pp}). Assuming that subjects are islands, we expect longer regression path durations and more regressions out in subextraction out of the subject: the reader realizes that the expected gap is filled by the possessive article and goes back in order to reanalyze the relative. Overall, there should also be longer total reading times on regions 3+4+5 in this condition. 

By contrast, if we assume that shorter dependency lengths are easier to process and that subjects are not islands, then we expectlonger regression path durations for subextraction out of the object. Furthermore, we also expect longer total reading times on regions 3+4+5 in this condition.

We can also compare conditions (\ref{ex:exp03-obj-pp}) with conditions (\ref{ex:exp03-obj-clitic-pp}). The subject island approach does not make any predictions for different extractions out of the object.\footnote{Relativized minimality probably does because of the intervening subject NP in (\ref{ex:exp03-obj-pp}), but for reasons that are orthogonal to the subject island discussion.} The DLT, on the other hand, predicts less processing difficulty when the subject is a clitic, consequently, regression path durations should be longer in (\ref{ex:exp03-obj-pp}), and there should be more regressions out in this region, e.g.\ the reader going back to the relative word. 

\subsubsection{Relative word (region 2)}

It is also interesting to look at regressions in for the relative word. In general, we expect a higher rate in the extraction conditions than in the coordination conditions, because it should be more necessary to check the form of the relative word. The syntactic accounts predict that extractions out of the subject trigger more regressions in than the other conditions. The processing accounts predict a gradation such that less regression in should be observed in condition (\ref{ex:exp03-subj-pp}) than in condition (\ref{ex:exp03-obj-clitic-pp}), which in turn should be less than in condition (\ref{ex:exp03-obj-pp}).

\subsection{Procedure}

The experiment was conducted in the eye tracking lab of the Laboratoire de Linguistique Formelle (LLF) in the Université Paris Cité. The investigators were Céline Pozniak, Aoi Shiraishi and myself. The experiment was run on Eyelink II and recorded the participant's dominant eye (following \citeauthor{Miles.1930}'s \citeyear{Miles.1930} test). Testing was done individually.

The participants received written instructions and gave informed consent. Before the actual experiment, participants provided information on their linguistic background. These information forms were treated anonymously during data processing.

Sentences were presented one at a time on a computer screen and participants were instructed to read them as fast as possible while maintaining comprehension.
They would then press a button to proceed to the following sentence. 
In some trials, a comprehension question would appear on the screen related to the sentence just read. Participants responded to it by choosing one of two possible answers on the screen. 
We used a Latin square design, such that each participant saw each item and distractor in only one condition.
% Celine is not sure that that the items were pseudo-randomized.

After an initial calibration phase, participants first went through a practice block of three sentences and had the opportunity to ask questions. Then the investigator would leave the room and the experimental items and distractors were presented in three blocks. 
Each block began with a short calibration phase. Participants could take a break between the blocks as needed. The experiment lasted approximately one hour. At the end, participants were debriefed and they received a payment of 10€.

\subsection{Participants}

The study was conducted in July 2016. 
32 participants took part in the experiment. 
They were recruited on the R.I.S.C.\ website (\url{http://experiences.risc.cnrs.fr/}) and on social media (e.g.\ Facebook).

One of them turned out to be bilingual and was excluded from the analysis.  The data presented here come from the remaining 31 participants. 
Their age ranged from 18 to 57 years. None of them had any educational background or occupation related to language.

\subsection{Results and analysis}

Reaction times typically have a non-normal distribution with a very long tail for longer reaction times. For this reason, following the usual methodology in reading time studies, the results presented here are based on log-transformed reading times, whose distribution is closer to normal. The results in ms are log-transformed using the function \texttt{log()} under R \citep{R}.

I suppressed data for skipped regions, i.e.\ regions with no fixation at all. Outliers, i.e.\ measurements that were more than 3 standard deviations away from each participant's mean for a given region in a given condition, were eliminated.

\figref{fig:exp03-all-TRT-main} shows the total reading times, and \figref{fig:exp03-all-RPD-main} shows the regression path durations on the experimental items.

\begin{figure}
    \centering
    \includegraphics[width=\textwidth]{chapters/part2-Empirical/Exp03-dont-RC-ET/all-TRT.jpeg}
    \caption[Region means and 95\% confidence intervals for the log-transformed total reading times of all conditions in Experiment~3]{Region means and 95\% confidence intervals for the log-transformed total reading time of all conditions in Experiment~3\\
    (Regions = 1: Matrix clause; 2: Relative word/conjunction; 3: Subject; 4: Verb; 5: Object; 6:AdvP)}
    \label{fig:exp03-all-TRT-main}
\end{figure}

\begin{figure}
    \centering
    \includegraphics[width=\textwidth]{chapters/part2-Empirical/Exp03-dont-RC-ET/all-RPD.jpeg}
    \caption[Region means and 95\% confidence intervals for the log-transformed regression path durations of all conditions in Experiment~3]{Region means and 95\% confidence intervals for the log-transformed regression path duration of all conditions in Experiment~3\\
    (Regions = 1: Matrix clause; 2: Relative word/conjunction; 3: Subject; 4: Verb; 5: Object; 6:AdvP)}
    \label{fig:exp03-all-RPD-main}
\end{figure}

Linear Mixed-Effects Models:
\label{ch:linear-models}
Log-transformed reaction times can be considered a continuous variable.  
To see how well a set of explanatory variables predicts a (continuous) variable A, we ran Linear Mixed-Effects Models, using the \texttt{lmer()} function under R \citep{R}. One prerequisite for Linear models is that the Gaussian model must be valid for the variable A. I performed a visual inspection of the residuals using the function \texttt{qqnorm()} from the R Stats Package \citep{R}, and only report the results of the models if the residuals diagnostic is compelling.
As in the Cumulative Link Mixed Models (see page \pageref{ch:cumulative-link-model}), I included random slopes for all fixed effects grouped by participants and items whenever convergence was achievable, and fitted a non-maximal model otherwise.


In order to test every prediction listed in Section \ref{ch:exp03-predictions} above, it was necessary to run several mixed models. The interested reader is referred to Appendix~\ref{ch:exp03-appendix}, in which every step of the statistical analysis is described in detail. In the present section, I only highlight the main findings relative to the predictions. Appendix~\ref{ch:exp03-appendix} also provides more figures that illustrate the results.

Table~\ref{tab:exp03-models} summarizes the results of all models.\footnote{Notice that model n$^{\circ}$14 is missing, because it did not converge, cf.\ Appendix~\ref{ch:exp03-appendix}.} All models include participants and items as random variables. We ran maximal models whenever possible, and added length (number of characters) and frequency as covariates whenever possible (it was sometimes necessary to drop either length or frequency because of singularities in the model). Notice that the results must be taken with caution, because we computed a large number of models and did not correct the confidence level for multiple comparisons. 

\begin{sidewaystable}
\begin{tabular}{ll *5{S[table-format=<1.4]} c}
\lsptoprule
Model & Dependent variable & \multicolumn{3}{c}{Fixed effects}    & \multicolumn{2}{c}{Covariate}  & Max?\\\cmidrule(lr){3-5}\cmidrule(lr){6-7}
      &                    & {dist.} & {extr.} & {dist.:extr.} & {length} & {frequ.} & \\ \midrule
n$^{\circ}$1   & total reading time on region 3+4+5           & 0.1213 & < .05  & 0.5741  & < .001  &        & yes  \\
n$^{\circ}$2   & regression path duration  on region 3        & 0.9238 & < .01  & 0.4538  & < .005  & 0.8217 & yes  \\
n$^{\circ}$3   & regression out  on region 3                  & 0.9906 & 0.7051 & 0.1673  & < .005  & 0.1890 & no   \\
n$^{\circ}$4   & amount of regression out  on region 3        & 0.8411 & 0.996  & 0.4062  & < .01   &        & no   \\
n$^{\circ}$5   & regression in  on region 3                   & 0.2928 & < .05  & < .005  & 0.7446  &        & no   \\
n$^{\circ}$6   & amount of regression in  on region 3         & 0.2051 & < .01  & < .005  & 0.5040  &        & no   \\
n$^{\circ}$7   & first fixation duration  on region 4         & 0.4151 & 0.4713 & 0.2937  & 0.5205  & 0.1397 & no   \\
n$^{\circ}$8   & regression path duration  on region 4        & 0.3118 & 0.1545 & 0.0585  & 0.2921  & 0.1356 & no   \\
n$^{\circ}$9   & regression path duration  on region 5        & 0.1482 & 0.1038 & 0.4454  & 0.0536  & 0.2014 & no   \\
n$^{\circ}$10  & regression out  on region 5                  & 0.4724 & 0.4600 & 0.2344  & 0.5704  &        & no   \\
n$^{\circ}$11  & amount of regression out  on region 5        & 0.6853 & 0.4875 & 0.7646  & 0.1805  &        & no   \\
n$^{\circ}$12  & regression path duration  on region 5        & 0.0934 & < .01  & 0.0677  & < .01   & 0.8521 & no   \\
n$^{\circ}$13  & regresion out  on region 5                   & 0.1147 & < .05  & 0.4702  & 0.1396  &        & no   \\
n$^{\circ}$15  & regression path duration  on extraction site & <.05   & < .005 & 0.3036  & < .005  & < .05  & no   \\
n$^{\circ}$16  & regression in  on region 2                   & 0.7502 &         &         &         & 0.7668 & no   \\
n$^{\circ}$17  & regression in  on region 2                   & 0.6061 &         &         &         & 0.2283 & yes  \\
n$^{\circ}$18  & regression in  on region 2                   & 0.0682 &         &         &         & 0.6732 & no   \\
n$^{\circ}$19  & amount of regression in  on region 2         & 0.3377 & 0.4641 & 0.7308  &         & 0.6652 & no   \\
n$^{\circ}$20  & amount of regression in  on region 2         & 0.6888 & 0.7167 & 0.6466  & 0.6044  & 0.2536 & yes  \\
n$^{\circ}$21  & amount of regression in  on region 2         & 0.1288 & 0.4496 & 0.4508  & 0.3650  &        & no   \\
\lspbottomrule
\end{tabular}
\caption{Results with \emph{p} values of the models. dist: distance; extr: extraction type; “Max?” indicates whether the model is maximal.}
% % % (LMM = Linear Mixed Model; BRMM = Binomial Regression Mixed Model; PRMM = Poisson Regression Mixed Model)
\label{tab:exp03-models}
\end{sidewaystable}


There is usually no significant main effect of distance in the models, except when we compare the regression path durations at extraction sites (results for region 3 in the subject conditions and for region 5 in the object conditions) in model n$^{\circ}$15. Indeed, as illustrated by \figref{fig:exp03-35-RPD-main}, the subject conditions have longer reading times. This is not surprising given that region 3 appears to the left of region 5. There is however no significant interaction effect between distance and extraction type in model n$^{\circ}$15. 

\begin{figure}
    \centering
    \includegraphics[width=\textwidth]{chapters/part2-Empirical/Exp03-dont-RC-ET/35-RPD.jpeg}
    \caption{Regression path durations for region 3 (subject) on the one hand and region 5 (object) on the other hand in Experiment~3 with 95\% confidence intervals.}
    \label{fig:exp03-35-RPD-main}
\end{figure}

In the models, we often observe a significant main effect of extraction type. In each of these cases, the non-ex\-trac\-tion conditions lead to longer reading times, more regressions in or more regressions out than the subextraction conditions. This is also clearly visible in \figref{fig:exp03-35-RPD-main}, as well as \figref{fig:exp03-3-Rin-prop-main}. 

\begin{figure}
    \centering
    \includegraphics[width=\textwidth]{chapters/part2-Empirical/Exp03-dont-RC-ET/3-Rin.jpeg}
    \caption{Mean amount of regression in in Region 3 in Experiment~3}
    \label{fig:exp03-3-Rin-prop-main}
\end{figure}

The most interesting effect for the predictions in Section \ref{ch:exp03-predictions} is the interaction between distance and extraction type. There is no evidence from the data that regression path durations are longer on the subject (model N$^{\circ}$2) or the object (model n$^{\circ}$9) when there is extraction out of the subject~-- nor is there evidence to the contrary. There is also no indication that there are more regressions out from the subject or from the object potentially due to the need to re-read the filler or the antecedent (models n$^{\circ}$3 and n$^{\circ}$4 for subject, models n$^{\circ}$10 and n$^{\circ}$11 for object), nor is there an indication of the opposite. The total reading times for the whole section from subject to object of the relative clause are not longer for extractions out of the subject, nor are they longer for extractions out of the object when the distance is the highest (model n$^{\circ}$1). 

However, there is a significant interaction if we compare the narrow- and the wide-distance with respect to regressions back to the subject: extractions out of the subject generate fewer regressions in than the other conditions, as illustrated by \figref{fig:exp03-3-Rin-prop-main}. But there do not seem to be more regressions back to the relative word (models n$^{\circ}$16 to n$^{\circ}$21).

There is also a small hint of a pullover effect on the verb, but the interaction effect is only marginal. Indeed, if we compare narrow- and wide-distances, regression path durations on region~4 show a small tendency such that extractions out of the subject generate shorter reading times than the other conditions (model n$^{\circ}$8), as shown in \figref{fig:exp03-4-RPD-main}. 

\begin{figure}
    \centering
    \includegraphics[width=\textwidth]{chapters/part2-Empirical/Exp03-dont-RC-ET/4-RPD.jpeg}
    \caption{Regression path durations by condition for region 4 in Experiment~3 with 95\% confidence intervals.}
    \label{fig:exp03-4-RPD-main}
\end{figure}

All the results above concern a comparison of the narrow-distance with the wide-distance conditions. None of the effects predicted by the DLT (or other processing accounts based on memory costs) were significant when comparing the narrow-distance with the  medium-distance conditions (models n$^{\circ}$16 and n$^{\circ}$19) or the medium-distance with the wide-distance conditions (models n$^{\circ}$12, n$^{\circ}$13, n$^{\circ}$18 and n$^{\circ}$21).

\subsection{Discussion}

Overall, we did not observe many of the expected effects described in Section~\ref{ch:exp03-predictions}. Whenever we found a significant interaction effect, it was in general more in line with the predictions of processing accounts based on memory costs than with the predictions of the traditional syntactic account. The increase of regressions in on the subject when there is extraction out of the object could be evidence that readers posit a gap inside the subject and need to come back to it to ensure that there is no dependency between the relative phrase and the subject. The tendency toward a spillover effect on the verb also goes in this direction: it could be caused by a filled-gap effect, in that the reader has to readjust their expectation that the gap is inside the subject.

However, based on \figref{fig:exp03-3-Rin-prop-main}, I think that the interaction effects are caused by another factor. My impression is that the interaction effect comes from an increase of regressions in for the subject control condition. The reason can be related to what I pointed out in the discussion of the two previous experiments: \emph{dont} relative clauses seemed to be preferred over the anaphoric possessive article. Perhaps this possessive article is even less expected when its referent is close, as is the case in the narrow-distance non-extraction condition (there is no new referent introduced between \emph{innovation} and \emph{son} in (\ref{ex:exp03-subj-no})). 

The main effect of extraction type observed several times in the models seems to contradict the fact that there are additional costs associated with the integration of the gap \citep{Momma.2019}. My only explanation is that such short-distance dependencies do not induce enough memory costs to slow down reading times, especially in a construction very frequently encountered by French speakers. Moreover, the control conditions also induce memory costs due to a different kind of syntactic binding. These two memory costs may cancel each other out in my experiment. As noticed by \citet{Keshev.2019}, experiments on cataphoric constructions display interaction effects similar to those attributed to weak islands. Since experiments on islands typically do not control for binding, at least some of the results usually attributed to islands could be caused by binding.
