\chapter{Experiment 3, detailed results and analysis}
\label{ch:exp03-appendix}

I report here the detailed steps of the statistical analysis for the eye tracking study described in Section~\ref{ch:exp03} (Experiment 3). 
Note that I only report results from statistical models that satisfied the validity criteria and whose residual analysis was compelling (see on page~\pageref{ch:linear-models}).

Reaction times typically have a non-normal distribution with a very long tail for longer reaction times. For this reason, following the usual methodology in reading time studies, the results presented here are based on log-transformed reading times, whose distribution is closer to normal. 

I suppressed data for skipped regions, i.e.\ regions for which there was no fixation at all. 

\section{Total reading times}

In order to look at total reading times, I built a subset of the data in which I suppressed outliers, i.e.\ total reading time measurements that were more than 3 standard deviations away from each participant's mean reading time (each region in each condition considered separately). 

\figref{fig:exp03-all-TRT} shows the total reading times for the experimental items and \figref{fig:exp03-345-TRT} shows in more detail the distribution of the reading times for regions 3+4+5\footnote{Note that skipped regions are then necessarily counted as having a reading time of 0.}. A visual inspection of the graph does not reveal strong increase of reading times for the subextraction from subject or the subextraction from object. Only the medium condition with a clitic subject is read faster, which is not surprising. 

\begin{figure}
    \centering
    \includegraphics[width=\textwidth]{chapters/part2-Empirical/Exp03-dont-RC-ET/all-TRT.jpeg}
    \caption{Region means and 95\% confidence intervals for the log-transformed total reading times of all conditions in Experiment 3}
    \label{fig:exp03-all-TRT}
\end{figure}

\begin{figure}
    \centering
    \includegraphics[width=\textwidth]{chapters/part2-Empirical/Exp03-dont-RC-ET/345-TRT.jpeg}
    \caption{Sum of the mean log-transformed total reading times for the regions 3, 4 and 5 for each condition of Experiment 3 with 95\% confidence intervals.}
    \label{fig:exp03-345-TRT}
\end{figure}

We fitted a first linear mixed model to predict log-transformed total reading times on the three regions by comparing extraction out of the subject and out of the object with nominal subject (mean centered with subject coded positive and object coded negative) crossed with extraction type (mean centered with no extraction coded negative and subextraction coded positive). We included the length of the region (number of characters) as covariate\footnote{Frequency is not an appropriate covariate, because we are looking at several regions at the same time.}, and random slopes for all fixed effects grouped by participants and items. 
The results of the model are reported in \tabref{tab:exp03-m1}. There is a significant effect of extraction type such that coordination controls were read more slowly than the extraction conditions, but there is no significant interaction. 

Thus the data do not confirm any of the predictions. Extraction out of the subject does not lead to increased total reading times, and nor does extraction out of the object. Surprisingly, total reading times are even shorter in extraction than in non-extraction. 

% latex table generated in R 3.6.3 by xtable 1.8-4 package
% Fri Apr 10 17:22:54 2020
\begin{table}
\begin{tabular}{l S[table-format=-1.3] S[table-format=1.3] S[table-format=3.2] S[table-format=-1] S[table-format=<1.4] S[table-format=1.2]}
  \lsptoprule
 & {Estimate} & {SE} & {df} & {$t$} & {$\text{Pr}(>|t|)$} & {OR} \\ 
  \midrule
(Intercept) & 0.856 & 0.577 & 152.52 & 1 & 0.1403 & 2.35 \\ 
  extraction type & -0.131 & 0.061 & 23.29 & -2 & <.05 & 1.14 \\ 
  distance & 0.169 & 0.105 & 25.66 & 2 & 0.1213 & 1.18 \\ 
  length & 0.557 & 0.014 & 521.85 & 39 & <.001 & 1.74 \\ 
  extraction type:distance & -0.042 & 0.073 & 29.31 & -1 & 0.5741 & 1.04 \\ 
   \lspbottomrule
\end{tabular}
\caption{Results of the Linear Mixed Model (model n$^{\circ}$1)}
\label{tab:exp03-m1}
\end{table}


\section{Regressions in region 3 (subject)}

I built a different subset of the data in which I suppressed outliers, i.e.\ regression path duration measurements that were more than 3 standard deviations away from each participant's mean reading time (each region in each condition considered separately). 
\figref{fig:exp03-all-RPD} shows the regression path durations for the experimental items.

\figref{fig:exp03-3-RPD} displays in more detail the distribution of the regression path durations for regions 3. 

\begin{figure}
    \centering
    \includegraphics[width=\textwidth]{chapters/part2-Empirical/Exp03-dont-RC-ET/all-RPD.jpeg}
    \caption{Region means and 95\% confidence intervals for the log-transformed regression path durations of all conditions in Experiment 3}
    \label{fig:exp03-all-RPD}
\end{figure}

\begin{figure}
    \centering
    \includegraphics[width=\textwidth]{chapters/part2-Empirical/Exp03-dont-RC-ET/3-RPD.jpeg}
    \caption{Regression path durations for region 3 for each condition of Experiment 3 with 95\% confidence intervals.}
    \label{fig:exp03-3-RPD}
\end{figure}

We fitted a second linear mixed model to predict log-transformed regression path duration by comparing extraction out of the subject and out of the object with nominal subject (mean centered with subject coded positive and object coded negative) crossed with extraction type (mean centered with no extraction coded negative and subextraction coded positive). We included the length of the region (number of characters) and the frequency of the lemma as covariates, and random slopes for all fixed effects grouped by participants and items. 
The results of the model are reported in \tabref{tab:exp03-m2}. There is a significant effect of extraction type: coordination controls were read more slowly than the extraction conditions, but there is no significant interaction.

% latex table generated in R 3.6.3 by xtable 1.8-4 package
% Sun Apr 26 23:02:02 2020
\begin{table}
\begin{tabular}{l S[table-format=1.3] S[table-format=1.3] c S[table-format=<1.3] S[table-format=1.2]}
  \lsptoprule
                     & {Estimate} & {SE} & {$z$} & {$\text{Pr}(>|z|)$} & {OR} \\ 
  \midrule
  syntactic function & 1.328 & 0.488 & 3 & <.01 & 3.77 \\ 
  trial              & 0.069 & 0.023 & 3 & <.005 & 1.07 \\ 
 \lspbottomrule
\end{tabular}
\caption{Results of the Cumulative Link Mixed Model (model n$^{\circ}$1)}
\label{tab:exp16-m1}
\end{table}


\figref{fig:exp03-3-Rout-prop} shows the rate and number of regressions out from the subject. 

\begin{figure}
    \centering
    \includegraphics[width=\textwidth]{chapters/part2-Empirical/Exp03-dont-RC-ET/3-Rout-prop.jpeg}
    
    \includegraphics[width=\textwidth]{chapters/part2-Empirical/Exp03-dont-RC-ET/3-Rout.jpeg}
    \caption{Rate (top) and mean number (bottom) of regressions out in Region 3 in Experiment 3}
    \label{fig:exp03-3-Rout-prop}
\end{figure}

We fitted a third binominal regression model to predict regressions out (yes coded 1, no coded 0). The explanatory variables were distance (mean centered with narrow distance coded positive and wide distance coded negative) crossed with extraction type (mean centered with no extraction coded negative, subextraction coded positive). We included the length of the region (number of characters) and the frequency of the lemme as covariates, and participants and items as random factors. 
The results of the model are reported in \tabref{tab:exp03-m3}. Only length is a significant predictor.\largerpage[-1]

% latex table generated in R 3.6.3 by xtable 1.8-4 package
% Sun Apr 26 23:02:15 2020
\begin{table}
\begin{tabular}{l S[table-format=1.3] S[table-format=1.3] c S[table-format=<1.3] S[table-format=1.2]}
  \lsptoprule
 & {Est.} & {SE} & {$z$} & {$\text{Pr}(>|z|)$} & {OR} \\ 
  \midrule
  syntactic function & 0.474 & 0.465 & 1 & 0.307 & 1.61 \\ 
  trial & 0.034 & 0.014 & 2 & <.05 & 1.03 \\ 
  syntactic function:trial & 0.014 & 0.015 & 1 & 0.342 & 1.01 \\ 
   \lspbottomrule
\end{tabular}
\caption{Results of the Cumulative Link Mixed Model (model n$^{\circ}$2)}
\label{tab:exp16-m2}
\end{table}


We fitted a fourth poisson regression model to predict the number of regressions out. The explanatory variables were distance (mean centered with narrow distance coded positive and wide distance coded negative) crossed with extraction type (mean centered with no extraction coded negative and subextraction coded positive). We included the length of the region (number of characters) as a covariate{\interfootnotelinepenalty=10000\footnote{In this and some following models, I did not add the frequency of the lemma as a covariate, because it makes the model fail to converge.}}, and participants and items as random factors. 
The results of the model are reported in \tabref{tab:exp03-m4} and are similar to model n$^{\circ}$3.

% latex table generated in R 3.6.3 by xtable 1.8-4 package
% Wed May 20 14:00:46 2020
\begin{table}
\begin{tabular}{l S[table-format=1.3] S[table-format=1.3] S[table-format=1] S[table-format=<1.4] S[table-format=1.2]}
  \lsptoprule
 & {Estimate} & {SE} & {$z$} & {$\text{Pr}(>|z|)$} & {Odd.ratio} \\ 
  \midrule
(Intercept) & 1.127 & 0.239 & 5 & <.001 & 3.09 \\ 
  syntactic function & 0.044 & 0.091 & 0 & 0.6283 & 1.04 \\ 
  trial & 0.012 & 0.005 & 2 & <.05 & 1.01 \\ 
   \lspbottomrule
\end{tabular}
\caption{Results of the Logistic regression model (model n$^{\circ}$2)}
\label{tab:exp02-m2}
\end{table}


\figref{fig:exp03-3-Rin-prop} shows the rate and number of regressions in to the subject. We observe that there are fewer regressions in with extraction out of the subject than in both the coordination controls and in extraction out of the object. 

\begin{figure}
    \centering
    \includegraphics[width=\textwidth]{chapters/part2-Empirical/Exp03-dont-RC-ET/3-Rin-prop.jpeg}
    
    \includegraphics[width=\textwidth]{chapters/part2-Empirical/Exp03-dont-RC-ET/3-Rin.jpeg}
    \caption{Rate (top) and mean number (bottom) of regressions in in Region 3 in Experiment 3}
    \label{fig:exp03-3-Rin-prop}
\end{figure}

We fitted a fifth binominal regression model to predict regressions in (yes coded 1, no coded 0). The explanatory variables were distance (mean centered with narrow distance coded positive and wide distance coded negative) crossed with extraction type (mean centered with no extraction coded negative and subextraction coded positive). We included the length of the region (number of characters) as a covariate, and participants and items as random factors. 
The results of the model are reported in \tabref{tab:exp03-m5}. There is a significant main effect of extraction type, such that regressions in back to the subject are more frequent in the non-extraction conditions. There is also a significant interaction, such that the subject non-extraction condition has more regressions in.\largerpage[1.5]

% latex table generated in R 3.6.3 by xtable 1.8-4 package
% Thu Apr 23 00:04:53 2020
\begin{table}
\begin{tabular}{l S[table-format=1.3] S[table-format=1.3] c S[table-format=<1.3] S[table-format=2.2]}
  \lsptoprule
 & {Estimate} & {SE} & {$z$} & {$\text{Pr}(>|z|)$} & {OR}\\ 
  \midrule
  extraction type & 2.342 & 0.395 & 6 & <.001 & 10.40 \\ 
  trial           & 0.039 & 0.012 & 3 & <.005 & 1.04 \\ 
   \lspbottomrule
\end{tabular}
\caption{Results of the Cumulative Link Mixed Model (model n$^{\circ}$3)}
\label{tab:exp10-m3}
\end{table}


We fitted a sixth poisson regression model to predict the number of regressions in. The explanatory variables were distance (mean centered with narrow distance coded positive and wide distance coded negative) crossed with extraction type (mean centered with no extraction coded negative and subextraction coded positive). We included the length of the region (number of characters) as a covariate, and participants and items as random factors. 
The results of the model are reported in \tabref{tab:exp03-m6} and corroborate the results of model n$^{\circ}$5. 

% latex table generated in R 3.4.4 by xtable 1.8-4 package
% Sat Mar 28 14:47:01 2020
\begin{table}
\begin{tabular}{l S[table-format=1.3] S[table-format=1.3] S[table-format=1] S[table-format=<1.4] S[table-format=2.2]}
  \lsptoprule
 & {Estimate} & {SE} & {$z$} & {$\text{Pr}(>|z|)$} & {Odd.ratio} \\ 
  \midrule
  extraction type          & 0.629 & 0.124 & 5 & <.001  & 1.88 \\ 
  distance                 & 0.213 & 0.076 & 3 & <.01   & 1.24 \\ 
  trial                    & 0.001 & 0.005 & 0 & 0.8325 & 1.00 \\ 
  extraction type:distance & 0.173 & 0.090 & 2 & 0.0553 & 1.19 \\ 
   \lspbottomrule
\end{tabular}
\caption{Results of the Cumulative Link Mixed Model (model n$^{\circ}$6)}
\label{tab:exp1-m6}
\end{table}


There is therefore no evidence in the data that extraction out of the subject leads to longer regression path duration, or to more regressions. Actually, there are more regressions in back to the subject when extraction is out of the object than when it is out of the subject. 

\section{First fixations and regression path durations in region 4 (verb)}

\figref{fig:exp03-4-RPD} shows the distribution of the first fixations and regression path durations for region 4. For first fixation durations, I built a different subset of the data in which I suppressed outliers, i.e.\ first fixation duration measurements that were more than 3 standard deviations away from each participant's mean reading time (each region in each condition considered separately). 

\begin{figure}
    \centering
    \includegraphics[width=\textwidth]{chapters/part2-Empirical/Exp03-dont-RC-ET/4-FFD.jpeg}
    
    \includegraphics[width=\textwidth]{chapters/part2-Empirical/Exp03-dont-RC-ET/4-RPD.jpeg}
    \caption{First fixation durations (top) and regression path durations (bottom) in region 4 for each condition of Experiment 3 with 95\% confidence intervals.}
    \label{fig:exp03-4-RPD}
\end{figure}

We fitted a seventh linear mixed model to predict log-transformed first fixation durations by comparing extraction out of the subject and out of the object with nominal subject (mean centered with subject coded positive and object coded negative) crossed with extraction type (mean centered with no extraction coded negative and subextraction coded positive). We included the length of the region (number of characters) and the frequency of the lemma as covariates, and participants and items as random variables. 
The results of the model are reported in \tabref{tab:exp03-m7}. There is a significant main effect of extraction type: the coordination controls were read more slowly than the extraction conditions, but there is no significant interaction.

% latex table generated in R 3.6.3 by xtable 1.8-4 package
% Sun Jul 19 15:34:20 2020
\begin{table}
\begin{tabularx}{\textwidth}{Q S[table-format=-1.3] 
                  S[table-format=1.3] 
                  S[table-format=3.2] 
                  S[table-format=-1] 
                  S[table-format=<1.4] 
                  S[table-format=3.2]}
  \lsptoprule
 & {Estimate} & {SE} & {df} & {$t$} & {$\text{Pr}(>|t|)$} & {OR} \\ 
  \midrule
(Intercept) & 5.468 & 0.086 & 403.36 & 64 & <.001 & 236.91 \\ 
  extraction type & 0.012 & 0.017 & 515.96 & 1 & 0.4713 & 1.01 \\ 
  distance & -0.014 & 0.018 & 514.96 & -1 & 0.4151 & 1.01 \\ 
  length & -0.006 & 0.009 & 520.02 & -1 & 0.5205 & 1.01 \\ 
  frequency & -0.000 & 0.000 & 522.58 & -1 & 0.1397 & 1.00 \\ 
  extr.\ type:distance & 0.018 & 0.017 & 515.64 & 1 & 0.2937 & 1.02 \\ 
   \lspbottomrule
\end{tabularx}
\caption{Results of the Linear Mixed Model (model n$^{\circ}$7)}
\label{tab:exp03-m7}
\end{table}


We fitted an eighth linear mixed model to predict log-transformed regression path durations by comparing extraction out of the subject and out of the object with nominal subject (mean centered with subject coded positive and object coded negative) crossed with extraction type (mean centered with no extraction coded negative and subextraction coded positive). We included the length of the region (number of characters) and the frequency of the lemma as covariates, and participants and items as random variables. 
The results of the model are reported in \tabref{tab:exp03-m8}. There is no significant factor in the model, even though there is a small hint of an interaction effect (marginally significant), which we can also identify in \figref{fig:exp03-4-RPD}: this effect rather disfavors extraction out of the object, where the verb is read slightly more slowly, especially compared to the non-extraction control. 

% latex table generated in R 3.6.3 by xtable 1.8-4 package
% Sun Jul 19 15:33:52 2020
\begin{table}
\begin{tabularx}{\textwidth}{Q S[table-format=-1.3] 
                  S[table-format=1.3] 
                  S[table-format=3.2] 
                  S[table-format=-1] 
                  S[table-format=<1.4] 
                  S[table-format=3.2]}
  \lsptoprule
 & {Estimate} & {SE} & {df} & {$t$} & {$\text{Pr}(>|t|)$} & {OR} \\ 
  \midrule
(Intercept) & 5.748 & 0.125 & 82.27 & 46 & <.001 & 313.66 \\ 
  extraction type & -0.031 & 0.022 & 493.48 & -1 & 0.1545 & 1.03 \\ 
  distance & 0.024 & 0.023 & 513.06 & 1 & 0.3118 & 1.02 \\ 
  length & 0.014 & 0.013 & 60.08 & 1 & 0.2921 & 1.01 \\ 
  frequency & -0.001 & 0.000 & 155.47 & -2 & 0.1356 & 1.00 \\ 
  extr.\ type:distance & -0.042 & 0.022 & 494.78 & -2 & 0.0585 & 1.04 \\ 
   \lspbottomrule
\end{tabularx}
\caption{Results of the Linear Mixed Model (model n$^{\circ}$8)}
\label{tab:exp03-m8}
\end{table}


There is therefore no evidence in the data for region 4 that extraction out of the subject leads to longer reading times. If anything, there is a reverse tendency, possibly indicating a small spillover effect on the verb for extractions out of the object.

\section{Regressions in region 5 (object)}

\figref{fig:exp03-5-RPD} shows in more detail the distribution of the regression path durations for region 5, and \figref{fig:exp03-5-Rout-prop} shows the rate and number of regressions out from the object.  

\begin{figure}
    \centering
    \includegraphics[width=\textwidth]{chapters/part2-Empirical/Exp03-dont-RC-ET/5-RPD.jpeg}
    \caption{Regression path durations for region 5 for each condition of Experiment 3 with 95\% confidence intervals.}
    \label{fig:exp03-5-RPD}
\end{figure}

\begin{figure}
    \centering
    \includegraphics[width=\textwidth]{chapters/part2-Empirical/Exp03-dont-RC-ET/5-Rout-prop.jpeg}
    
    \includegraphics[width=\textwidth]{chapters/part2-Empirical/Exp03-dont-RC-ET/5-Rout.jpeg}
    \caption{Rate (top) and mean number (bottom) of regressions out in Region 5 in Experiment 3}
    \label{fig:exp03-5-Rout-prop}
\end{figure}

I will first compare the narrow vs.\ wide distance conditions, then the medium vs.\ wide distance conditions.

We fitted a ninth linear mixed model to predict log-transformed regression path durations by comparing extraction out of the subject and out of the object with nominal subject (mean centered with subject coded positive and object coded negative) crossed with extraction type (mean centered with no extraction coded negative and subextraction coded positive). We included the length of the region (number of characters) and the frequency of the lemma as covariates, and participants and items as random variables. 
The results of the model are reported in \tabref{tab:exp03-m9}. There is no significant effect in the model.

% latex table generated in R 3.6.3 by xtable 1.8-4 package
% Sun Jul 19 16:50:04 2020
\begin{table}
\begin{tabularx}{\textwidth}{Q 
                  S[table-format=-1.3] 
                  S[table-format=1.3] 
                  S[table-format=3.2] 
                  S[table-format=-1] 
                  S[table-format=<1.4] 
                  S[table-format=3.2]}
  \lsptoprule
 & {Estimate} & {SE} & {df} & {$t$} & {$\text{Pr}(>|t|)$} & {OR} \\ 
  \midrule
(Intercept) & 5.590 & 0.197 & 38.29 & 28 & <.001 & 267.78 \\ 
  extraction type & -0.037 & 0.022 & 509.65 & -2 & 0.1038 & 1.04 \\ 
  distance & 0.034 & 0.023 & 534.32 & 1 & 0.1482 & 1.03 \\ 
  length & 0.033 & 0.016 & 33.51 & 2 & 0.0536 & 1.03 \\ 
  frequency & 0.001 & 0.000 & 175.50 & 1 & 0.2014 & 1.00 \\ 
  extr.\ type:distance & 0.017 & 0.022 & 510.91 & 1 & 0.4454 & 1.02 \\ 
   \lspbottomrule
\end{tabularx}
\caption{Results of the Linear Mixed Model (model n$^{\circ}$9)}
\label{tab:exp03-m9}
\end{table}


We fitted a tenth binominal regression model to predict regressions out (yes coded 1, no coded 0). The explanatory variables were distance (mean centered with narrow distance coded positive and wide distance coded negative) crossed with extraction type (mean centered with no extraction coded negative and subextraction coded positive). We included the length of the region (number of characters) as a covariate, and participants and items as random factors. 
The results of the model are reported in \tabref{tab:exp03-m10}. Once again, there is no significant effect.

% latex table generated in R 3.6.3 by xtable 1.8-4 package
% Sun Jul 19 17:58:01 2020
\begin{table}
\begin{tabular}{l
                S[table-format=-1.4]
                S[table-format=1.4]
                S[table-format=-1.4]
                S[table-format=<1.4]
                S[table-format=1.2]}
  \lsptoprule
 & {Estimate} & {SE} & {$z$} & {$\text{Pr}(>|z|)$} & {OR} \\ 
  \midrule
(Intercept) & -2.1957 & 1.0002 & -2.1951 & <.05 & 8.99 \\ 
  extraction type & -0.082 & 0.111 & -0.7388 & 0.46 & 1.09 \\ 
  distance & 0.0799 & 0.1112 & 0.7186 & 0.4724 & 1.08 \\ 
  length & 0.0481 & 0.0848 & 0.5675 & 0.5704 & 1.05 \\ 
  extraction type:distance & -0.1322 & 0.1111 & -1.1891 & 0.2344 & 1.14 \\ 
   \lspbottomrule
\end{tabular}
\caption{Results of the Regression Mixed Model (model n$^{\circ}$10)}
\label{tab:exp03-m10}
\end{table}


We fitted an eleventh poisson regression model to predict the number of regressions out. The explanatory variables were distance (mean centered with narrow distance coded positive and wide distance coded negative) crossed with extraction type (mean centered with no extraction coded negative and subextraction coded positive). We included the length of the region (number of characters) as a covariate, and participants and items as random factors. 
The results of the model are reported in \tabref{tab:exp03-m11} and there is again no significant effect.

% latex table generated in R 3.6.3 by xtable 1.8-4 package
% Sun Jul 19 17:58:08 2020
\begin{table}
\begin{tabular}{l
                S[table-format=-1.4]
                S[table-format=1.4]
                S[table-format=-1.4]
                S[table-format=<1.4]
                S[table-format=2.2]}
  \lsptoprule
 & {Estimate} & {SE} & {$z$} & {$\text{Pr}(>|z|)$} & {OR} \\ 
  \midrule
(Intercept) & -2.6385 & 0.7112 & -3.71 & <.001 & 13.99 \\ 
  extraction type & -0.0626 & 0.0901 & -0.6942 & 0.4875 & 1.06 \\ 
  distance & 0.0366 & 0.0903 & 0.4052 & 0.6853 & 1.04 \\ 
  length & 0.0799 & 0.0596 & 1.3393 & 0.1805 & 1.08 \\ 
  extraction type:distance & -0.0271 & 0.0904 & -0.2995 & 0.7646 & 1.03 \\ 
   \lspbottomrule
\end{tabular}
\caption{Results of the Regression Mixed Model (model n$^{\circ}$11)}
\label{tab:exp03-m11}
\end{table}


There is therefore no evidence in the data that extraction out of the subject leads to longer regression path durations, or to more regressions, and also no evidence to the contrary. I now compare the medium and wide conditions, where the DLT expects a processing difference.

We fitted a twelfth linear mixed model to predict log-transformed regression path durations by comparing extraction out of the object with clitic subject and with nominal subject (mean centered with clitic subject coded positive and nominal subject coded negative) crossed with extraction type (mean centered with no extraction coded negative and subextraction coded positive). We included the length of the region (number of characters) and the frequency of the lemma as covariates, and participants and items as random variables. 
The results of the model are reported in \tabref{tab:exp03-m12}. There is a main effect of extraction type, such that the extraction conditions have shorter regression path durations than the control conditions. There is no interaction effect. 

% latex table generated in R 3.6.3 by xtable 1.8-4 package
% Sun Jul 19 18:57:11 2020
\begin{table}
\begin{tabularx}{\textwidth}{Q 
                  S[table-format=-1.3] 
                  S[table-format=1.3] 
                  S[table-format=3.2] 
                  S[table-format=-1] 
                  S[table-format=<1.4] 
                  S[table-format=3.2]}
  \lsptoprule
 & {Estimate} & {SE} & {df} & {$t$} & {$\text{Pr}(>|t|)$} & {OR} \\ 
  \midrule
(Intercept) & 5.431 & 0.190 & 41.56 & 29 & <.001 & 228.28 \\ 
  extraction type & -0.061 & 0.023 & 520.45 & -3 & <.01 & 1.06 \\ 
  distance & -0.040 & 0.024 & 541.89 & -2 & 0.0934 & 1.04 \\ 
  length & 0.047 & 0.016 & 34.74 & 3 & <.01 & 1.05 \\ 
  frequency & 0.000 & 0.000 & 142.43 & 0 & 0.8521 & 1.00 \\ 
  extr.\ type:distance & -0.042 & 0.023 & 522.26 & -2 & 0.0677 & 1.04 \\ 
   \lspbottomrule
\end{tabularx}
\caption{Results of the Linear Mixed Model (model n$^{\circ}$12)}
\label{tab:exp03-m12}
\end{table}


We fitted a thirteenth binominal regression model to predict regressions out (yes coded 1, no coded 0). The explanatory variables were distance (mean centered with medium distance coded positive and wide distance coded negative) crossed with extraction type (mean centered with no extraction coded negative and subextraction coded positive). We included the length of the region (number of characters) as a covariate, and participants and items as random factors. 
The results of the model are reported in \tabref{tab:exp03-m13}. In line with model n$^{\circ}$12, there is a main effect of extraction type, such that there are more regressions out in the control consitions than the subextraction conditions.

% latex table generated in R 3.6.3 by xtable 1.8-4 package
% Sun Jul 19 19:05:55 2020
\begin{table}
\begin{tabular}{l
                S[table-format=-1.4]
                S[table-format=1.4]
                S[table-format=-1.4]
                S[table-format=<1.4]
                S[table-format=2.2]}
  \lsptoprule
 & {Estimate} & {SE} & {$z$} & {$\text{Pr}(>|z|)$} & {OR} \\ 
  \midrule
(Intercept) & -2.8826 & 0.8373 & -3.4426 & <.001 & 17.86 \\ 
  extraction type & -0.2885 & 0.1139 & -2.5326 & <.05 & 1.33 \\ 
  distance & -0.1803 & 0.1143 & -1.5775 & 0.1147 & 1.20 \\ 
  length & 0.1031 & 0.0698 & 1.4773 & 0.1396 & 1.11 \\ 
  extraction type:distance & -0.0822 & 0.1139 & -0.7221 & 0.4702 & 1.09 \\ 
   \lspbottomrule
\end{tabular}
\caption{Results of the Regression Mixed Model (model n$^{\circ}$13)}
\label{tab:exp03-m13}
\end{table}


A fourteenth poisson regression model trying to predict the number of regressions out failed to converge.

There is therefore no evidence in the data that extraction out of the object with a nominal subject leads to longer regression path durations, or to more regressions, than extraction out of the object with a clitic subject.

\section{Regressions at the extraction site (subject/object)}

\figref{fig:exp03-35-RPD} shows the regression path durations at the extraction site, i.e.\ in region 3 for the subject conditions and in region 5 for the object conditions (here we only consider the cases with a nominal subject). 

\begin{figure}
    \centering
    \includegraphics[width=\textwidth]{chapters/part2-Empirical/Exp03-dont-RC-ET/35-RPD.jpeg}
    \caption{Regression path durations for region 3 on the one hand (subject) and region 5 on the other hand (object) of Experiment 3 with 95\% confidence intervals.}
    \label{fig:exp03-35-RPD}
\end{figure}

We fitted a fifteenth linear mixed model to predict log-transformed regression path durations by comparing extraction out of the subject and extraction out of the object with nominal subject (mean centered with subject coded positive and object coded negative) crossed with extraction type (mean centered with no extraction coded negative and subextraction coded positive). We included the length of the region (number of characters) and the frequency of the lemma as covariates, and participants and items as random variables. 
The results of the model are reported in \tabref{tab:exp03-m15}. There is a main effect of extraction type, such that the extraction conditions have shorter regression path durations than the control conditions. There is also a main effect of distance, such that the subject conditions have longer regression path durations. However, there is no significant interaction effect. 

% latex table generated in R 3.6.3 by xtable 1.8-4 package
% Sun Jul 19 19:37:36 2020
\begin{table}
\begin{tabularx}{\textwidth}{Q 
                  S[table-format=-1.3] 
                  S[table-format=1.3] 
                  S[table-format=3.2] 
                  S[table-format=-1] 
                  S[table-format=<1.4] 
                  S[table-format=3.2]}
  \lsptoprule
 & {Estimate} & {SE} & {df} & {$t$} & {$\text{Pr}(>|t|)$} & {OR} \\ 
  \midrule
(Intercept) & 5.424 & 0.170 & 31.62 & 32 & <.001 & 226.87 \\ 
  extraction type & -0.076 & 0.023 & 516.04 & -3 & <.005 & 1.08 \\ 
  distance & 0.054 & 0.024 & 531.36 & 2 & <.05 & 1.05 \\ 
  length & 0.048 & 0.014 & 26.63 & 3 & <.005 & 1.05 \\ 
  frequency & 0.001 & 0.000 & 119.86 & 2 & <.05 & 1.00 \\ 
  extr.\ type:distance & -0.024 & 0.023 & 508.31 & -1 & 0.3036 & 1.02 \\ 
   \lspbottomrule
\end{tabularx}
\caption{Results of the Linear Mixed Model (model n$^{\circ}$15)}
\label{tab:exp03-m15}
\end{table}


Hence, when we compare extraction sites, there is no evidence that extraction out of the subject leads to longer regression path durations. 

\section{Regressions in region 2 (relative word)}\largerpage

I will first look at the rate of regressions in, comparing the three distance conditions in pairs, and then turn to the amount of regressions in. 

\figref{fig:exp03-2-Rin-prop} shows the rate of regressions in to the relative word or coordination word. There are hardly any regressions in in the coordinations, therefore the following three models only compare the extraction conditions. We observe an increase of regressions in as the distance increases.

\begin{figure}
    \centering
    \includegraphics[width=\textwidth]{chapters/part2-Empirical/Exp03-dont-RC-ET/2-Rin-prop.jpeg}
    \caption{Rate of regressions in in Region 2 of Experiment 3}
    \label{fig:exp03-2-Rin-prop}
\end{figure}

I first compared the narrow vs.\ medium distance. We fittes a sixteenth binominal regression model to predict regressions in (yes coded 1, no coded 0). The explanatory variable was the distance (mean centered with narrow distance coded positive and medium distance coded negative). We included the frequency of the lemma as a covariate\footnote{In models 16--19, I did not add the length of the region as a covariate, because it makes the model fail to converge.}, and participants and items as random factors. 
The results of the model are reported in \tabref{tab:exp03-m16}. There is no significant effect in the model.


I then compared the narrow vs.\ wide distance. We fitted a seventeenth binominal regression model to predict regressions in (yes coded 1, no coded 0). The explanatory variable was the distance (mean centered with narrow distance coded positive and wide distance coded negative). We included the frequency of the lemma as a covariate, and participants and items as random factors. 
The results of the model are reported in \tabref{tab:exp03-m17}. There is no significant effect in the model.\pagebreak

% latex table generated in R 3.6.3 by xtable 1.8-4 package
% Sun Jul 19 20:41:20 2020
\begin{table}
\begin{tabular}{l
                S[table-format=-1.4]
                S[table-format=1.4]
                S[table-format=-1.4]
                S[table-format=<1.4]
                S[table-format=1.2]}
  \lsptoprule
 & {Estimate} & {SE} & {$z$} & {$\text{Pr}(>|z|)$} & {OR} \\ 
  \midrule
(Intercept) & -1.6939 & 0.2925 & -5.7917 & <.001 & 5.44 \\ 
  distance &   0.0789 & 0.2478 & 0.3184 & 0.7502 & 1.08 \\ 
  frequency & -0.001 & 0.0033 & -0.2966 & 0.7668 & 1.00 \\ 
   \lspbottomrule
\end{tabular}
\caption{Results of the Regression Mixed Model (model n$^{\circ}$16)}
\label{tab:exp03-m16}
\end{table}
   
% latex table generated in R 3.6.3 by xtable 1.8-4 package
% Sun Jul 19 20:41:26 2020
\begin{table}
\begin{tabular}{l
                S[table-format=-1.4]
                S[table-format=1.4]
                S[table-format=-1.4]
                S[table-format=<1.4]
                S[table-format=2.2]}
  \lsptoprule
 & {Estimate} & {SE} & {$z$} & {$\text{Pr}(>|z|)$} & {OR} \\ 
  \midrule
(Intercept) & -2.7473 & 1.0553 & -2.6034 & <.01 & 15.60 \\ 
  distance & 0.3616 & 0.7011 & 0.5157 & 0.6061 & 1.44 \\ 
  frequency & 0.0087 & 0.0072 & 1.2047 & 0.2283 & 1.01 \\ 
   \lspbottomrule
\end{tabular}
\caption{Results of the Regression Mixed Model (model n$^{\circ}$17)}
\label{tab:exp03-m17}
\end{table}


Finally, I compared the medium vs.\ wide distance. We fitted an eighteenth binominal regression model to predict regressions in (yes coded 1, no coded 0). The explanatory variable was the distance (mean centered with the medium distance coded positive and the wide distance coded negative). We included the frequency of the lemma as a covariate, and participants and items as random factors. 
The results of the model are reported in \tabref{tab:exp03-m18}. There is no significant effect in the model, but a slight tendency such that extraction out of the object with clitic subject shows fewer regressions in back to the relative word.

% latex table generated in R 3.6.3 by xtable 1.8-4 package
% Sun Jul 19 20:41:29 2020
\begin{table}
\begin{tabular}{l
                S[table-format=-1.4]
                S[table-format=1.4]
                S[table-format=-1.4]
                S[table-format=<1.4]
                S[table-format=2.2]}
  \lsptoprule
 & {Estimate} & {SE} & {$z$} & {$\text{Pr}(>|z|)$} & {OR} \\ 
  \midrule
(Intercept) & -1.8922 & 0.45 & -4.2053 & <.001 & 6.63 \\ 
  distance & -0.4382 & 0.2403 & -1.8234 & 0.0682 & 1.55 \\ 
  frequency & 0.0012 & 0.0028 & 0.4218 & 0.6732 & 1.00 \\ 
   \lspbottomrule
\end{tabular}
\caption{Results of the Regression Mixed Model (model n$^{\circ}$18)}
\label{tab:exp03-m18}
\end{table}


\figref{fig:exp03-2-Rin} shows the amount of regression in to the relative word or coordination word. What is striking here is the higher number of regressions in for coordinations, given that there were only very few occurrences (\figref{fig:exp03-2-Rin-prop}). This is especially true for the wide distance condition.

\begin{figure}
    \centering
    \includegraphics[width=\textwidth]{chapters/part2-Empirical/Exp03-dont-RC-ET/2-Rin.jpeg}
    \caption{Mean number of regressions in in Region 2 of Experiment 3}
    \label{fig:exp03-2-Rin}
\end{figure}

I first compared the narrow vs.\ medium distance. We fitted a nineteenth poisson regression model to predict the number of regressions in. The explanatory variables were distance (mean centered with narrow distance coded positive, medium coded negative) crossed with extraction type (mean centered with no extraction coded negative and subextraction coded positive). We included the frequency of the lemma as a covariate, and participants and items as random factors. 
The results of the model are reported in \tabref{tab:exp03-m19}. There is no significant effect in the model.

% latex table generated in R 3.6.3 by xtable 1.8-4 package
% Sun Jul 19 21:09:46 2020
\begin{table}
\begin{tabular}{l
                S[table-format=-1.4]
                S[table-format=1.4]
                S[table-format=-1.4]
                S[table-format=<1.4]
                S[table-format=1.2]}
  \lsptoprule
 & {Estimate} & {SE} & {$z$} & {$\text{Pr}(>|z|)$} & {OR} \\ 
  \midrule
(Intercept) & -1.7567 & 0.2936 & -5.9841 & <.001 & 5.79 \\ 
  extraction type & 0.1731 & 0.2365 & 0.7321 & 0.4641 & 1.19 \\ 
  distance & 0.1922 & 0.2005 & 0.9587 & 0.3377 & 1.21 \\ 
  frequency & -0.0012 & 0.0029 & -0.4327 & 0.6652 & 1.00 \\ 
  extraction type:distance & -0.074 & 0.215 & -0.3441 & 0.7308 & 1.08 \\ 
   \lspbottomrule
\end{tabular}
\caption{Results of the Regression Mixed Model (model n$^{\circ}$19)}
\label{tab:exp03-m19}
\end{table}


I then compared the narrow vs.\ wide distance. We fitted a twentieth poisson regression model to predict the number of regressions in. The explanatory variables were distance (mean centered with narrow distance coded positive and wide distance coded negative) crossed with extraction type (mean centered with no extraction coded negative and subextraction coded positive). We included the frequency of the lemma and the length of the region (number of characters) as covariates, and random slopes for all fixed effects grouped by participants and items. 
The results of the model are reported in \tabref{tab:exp03-m20}. There is no significant effect in the model.

% latex table generated in R 3.6.3 by xtable 1.8-4 package
% Sun Jul 19 21:13:06 2020
\begin{table}
\begin{tabular}{l
                S[table-format=-1.3]
                S[table-format=1.3]
                S[table-format=-1.4]
                S[table-format=1.4]
                S[table-format=2.2]}
  \lsptoprule
 & {Estimate} & {SE} & {$z$} & {$\text{Pr}(>|z|)$} & {OR} \\ 
  \midrule
(Intercept) & -3.521 & 2.880 & -1.2224 & 0.2216 & 33.81 \\ 
  extraction type & -0.240 & 0.661 & -0.3629 & 0.7167 & 1.27 \\ 
  distance & -0.149 & 0.371 & -0.4005 & 0.6888 & 1.16 \\ 
  frequency & 0.003 & 0.003 & 1.1416 & 0.2536 & 1.00 \\ 
  length & 0.393 & 0.758 & 0.5181 & 0.6044 & 1.48 \\ 
  extraction type:distance & 0.233 & 0.508 & 0.4586 & 0.6466 & 1.26 \\ 
   \lspbottomrule
\end{tabular}
\caption{Results of the Regression Mixed Model (model n$^{\circ}$20)}
\label{tab:exp03-m20}
\end{table}


Finally I compared the medium vs.\ wide distance. We fitted a twenty-first poisson regression model to predict the number of regressions in. The explanatory variables were distance (mean centered with the medium distance coded positive and the wide distance coded negative) crossed with extraction type (mean centered with no extraction coded negative and subextraction coded positive). We included the length of the region  (number of characters) as a covariate, and participants and items as random factors. 
The results of the model are reported in \tabref{tab:exp03-m21}. There is no significant effect in this model, either.

% latex table generated in R 3.6.3 by xtable 1.8-4 package
% Sun Jul 19 21:16:09 2020
\begin{table}
\begin{tabular}{l
                S[table-format=-1.3]
                S[table-format=1.3]
                S[table-format=-1.4]
                S[table-format=1.4]
                S[table-format=2.2]}
  \lsptoprule
 & {Estimate} & {SE} & {$z$} & {$\text{Pr}(>|z|)$} & {OR} \\ 
  \midrule
(Intercept) & -3.938 & 2.399 & -1.6416 & 0.1007 & 51.29 \\ 
  extraction type & -0.321 & 0.424 & -0.7560 & 0.4496 & 1.38 \\ 
  distance & -0.273 & 0.180 & -1.5187 & 0.1288 & 1.31 \\ 
  length & 0.573 & 0.633 & 0.9059 & 0.3650 & 1.77 \\ 
  extraction type:distance & 0.157 & 0.208 & 0.7540 & 0.4508 & 1.17 \\ 
   \lspbottomrule
\end{tabular}
\caption{Results of the Regression Mixed Model (model n$^{\circ}$21)}
\label{tab:exp03-m21}
\end{table}


Therefore, apart from a very small tendency with a marginally significant effect of distance in model n$^{\circ}$18, there is no evidence that distance has an impact on regressions back to the relative word. 
