\section[head=Experiment 16]{Experiment 16: Acceptability judgment study on subextraction from infinitivals with \emph{que}}
\label{ch:exp16}

\citet{Chaves.2013} reports numerous felicitous examples of extraction of the direct object from an infinitival subject:

\eal\label{ex:eng-extr-out-of-infinite-subj}
\ex \citep[1094, fn.\ 27]{Huddleston.2002}\\
The eight dancers and their caller, Laurie Schmidt, make up the
Farmall Promenade of nearby Nemaha, a town$_i$ that [[to describe~\trace{}$_i$
as tiny] would be to overstate its size].
\ex \citep[471]{Chaves.2013}\\
In his bedroom, [which]$_i$ [to describe~\trace{}$_i$ as small] would be a gross
understatement, he has an audio studio setup.
\zl 

According to corpus results by \citet{Candito.2012.ldd}, it is very frequent to extract with \emph{que} out of a non-finite clause in French. Our intuition, however, was that sentences like (\ref{ex:eng-extr-out-of-infinite-subj}) are less felicitous than the extraction of a PP tested in Experiment~15. This is why in Experiment~16 we investigated the extraction of an object NP, assuming that a potential contrast between extraction out of subjects vs.\ objects would be stronger.

\subsection{Design and materials}

The experiment replicated the 2*2 design of the previous experiment, but used verbs that select a direct object. Whenever possible, we tried to stay close to the items of Experiment~15, and described similar situations. For this reason, many direct objects also denote locations (e.g.\ \emph{explorer une forêt} `explore a forest').

Again, the subextraction took place from a infinitival subject in the subject condition (\ref{ex:exp1-subj-pp}), and from the complement of an impersonal construction in the object condition (\ref{ex:exp1-obj-pp}). Non-extraction controls (\ref{ex:exp1-subj-no}) and (\ref{ex:exp1-obj-no}) involved coordination. 

\eal 
\ex[]{{Condition subject + \emph{que} extracted:}\nopagebreak\\
\gll Amsterdam est connue pour ses péniches, qu$_i$’ [observer~\trace{}$_i$] est charmant lorsqu' il fait beau.\\
Amsterdam is known for its barges that observe\textsc{.inf} is pleasant when it does nice\\
\glt `Amsterdam is well-known for its barges, which to observe is pleasant when the weather is nice.'}
\label{ex:exp16-subj-pp}
\ex[]{{Condition object + \emph{que} extracted:}\\
\gll Amsterdam est connue pour ses péniches, qu$_i$' il est charmant [d' observer~\trace{}$_i$] lorsqu' il fait beau.\\
Amsterdam is known for its barges that it is pleasant of observe\textsc{.inf} when it does nice\\
\glt `Amsterdam is well-known for its barges, which it is pleasant to observe when the weather is nice.'}
\label{ex:exp16-obj-pp}
\ex[]{{Condition subject + noextr:}\\
\gll Amsterdam est connue pour ses péniches, et les observer est charmant lorsqu' il fait beau.\\
Amsterdam is known for its barges and them\textsc{.acc} observe\textsc{.inf} is pleasant when it does nice\\
\glt `Amsterdam is well-known for its barges, and to observe them is pleasant when the weather is nice.'}
\label{ex:exp16-subj-no}
\ex[]{{Condition object + noextr:}\\
\gll Amsterdam est connue pour ses péniches, et il est charmant de les observer lorsqu' il fait beau.\\
Amsterdam is known for its barges and it is pleasant of them\textsc{.acc} observe\textsc{.inf} when it does nice\\
\glt `Amsterdam is well-known for its barges, and it is pleasant to observe them when the weather is nice.'}
\label{ex:exp16-obj-no}
\zl 

We constructed 12 items, based on the items in Experiment~15. Each item was manipulated according to the four conditions just described. In addition, the experiment included 36 distractors, some of which were ungrammatical. Around 40\% of the experimental items and 70\% of the distractors were followed by a comprehension question. The item presented here as an example was followed by the comprehension question \emph{Il est question de Hambourg.} (`This is about Hamburg.').

\subsection{Predictions}

The predictions for this experiment are the same as the predictions for the previous experiment. These were summarized in Table \ref{tab:exp15-predictions} on page \pageref{tab:exp15-predictions}.

\subsection{Procedure} 

We conducted the experiment on the Ibex platform \citep{Ibex}. The procedure was similar to the one used in the previous acceptability judgment experiments (see Section \ref{ch:methodo-AJ}). Participants rated the sentences on a Likert scale from 0 to 10, 0 being labeled as ``bad'' and 10 being labeled as ``good''. They also answered comprehension questions after some of the sentences.

The experiment took approximately 20 minutes to complete. 

\subsection{Participants}

The study was run between May and August 2019. 
Participants were recruited on the R.I.S.C.\ website (\url{http://experiences.risc.cnrs.fr/}) and through social media (e.g.\ Facebook).
They received no financial compensation. 

29 participants took part in the experiment. 
The analysis presented here is based on the data from the 19 participants who satisfied all criteria.
They were aged 19 to 76 years. 12 of them self-identified as women, and seven as men. Three of them (15.79\%) indicated they had an educational background related to language.

\subsection{Results and analysis}


\figref{fig:exp16-boxplot} shows the results of the acceptability judgment task.
In the subextraction conditions, the extraction out of the subject (\ref{ex:exp16-subj-pp}) had an mean  rating of 4.47, lower than the extraction out of the object (\ref{ex:exp16-obj-pp}) which had a mean rating of 6.73. The non-extraction conditions received higher ratings: a mean rating of 8.00 in the the subject control condition (\ref{ex:exp16-subj-no}), and of 7.96 in the object control condition (\ref{ex:exp16-obj-no}). 

\begin{figure}
   \centering
    \includegraphics[width=\textwidth]{chapters/part2-Empirical/Exp16-infinitival-que/boxplots.jpeg}
    \caption{Acceptability judgments by condition in Experiment~16. The grey box plots indicate the median and quartiles of the results. Black points are outliers. Mean and confidence intervals are indicated in white.}
    \label{fig:exp16-boxplot}
\end{figure}

\figref{fig:exp16-boxplot} suggests potential ceiling effects in the non-extraction conditions. However, the subextraction conditions seem distributed along the whole scale in this experiment. This is corroborated by \figref{fig:exp16-repartition}.

\begin{figure}
    \centering
    \includegraphics[width=\textwidth]{chapters/part2-Empirical/Exp16-infinitival-que/repartition.jpeg}
    \caption{Density of the ratings across conditions for Experiment~16}
    \label{fig:exp16-repartition}
\end{figure}

Another representation of the results is given by the ROC and zROC curves of the results in \figref{fig:exp16-ROC} on page \pageref{fig:exp16-ROC}. The ROC curves show that participants discriminated between the subextraction conditions and the non-extraction controls. The zROC curve for the subject condition is convex.

\begin{figure}
    \centering
    \includegraphics[width=\textwidth]{chapters/part2-Empirical/Exp16-infinitival-que/ROC.jpeg}
    \includegraphics[width=\textwidth]{chapters/part2-Empirical/Exp16-infinitival-que/zROC.jpeg}
    \caption{ROC curves (top) and zROC curves (bottom) of the non-extraction conditions compared to their respective subextraction conditions, represented by the dotted grey baseline (\citealt{Dillon.2019}'s method) in Experiment~16.}
    \label{fig:exp16-ROC}
\end{figure}

The ROC and zROC curves in \figref{fig:exp16-ROC-subj} on page \pageref{fig:exp16-ROC-subj} show the discrimination between the subject and object conditions. We see that the discrimination is more pronounced for the subextraction than in the non-extraction controls. The zROC curve for the subextraction is slightly convex.

\begin{figure}
    \centering
    \includegraphics[width=\textwidth]{chapters/part2-Empirical/Exp16-infinitival-que/ROC-subject.jpeg}
    \includegraphics[width=\textwidth]{chapters/part2-Empirical/Exp16-infinitival-que/zROC-subject.jpeg}
    \caption{ROC curves (top) and zROC curves (bottom) of the object conditions compared to their respective subject conditions, represented by the dotted grey baseline (\citealt{Dillon.2019}'s method) in Experiment~16.}
    \label{fig:exp16-ROC-subj}
\end{figure}

\subsubsection{Habituation} 

The habituation effects in the course of the experiment are given in \figref{fig:exp16-habituation} on page \pageref{fig:exp16-habituation}. Habituation effects are absent (or very small) in the non-extraction conditions. But the subextraction conditions show a habituation effect, especially in extractions out of the object (although there is no interaction, see below). 

\begin{figure}
    \centering
    \includegraphics[width=\textwidth]{chapters/part2-Empirical/Exp16-infinitival-que/habituation.jpeg}
    \caption{Changes in the mean acceptability ratings ($z$-scored by participant) by condition in the course of Experiment~16}
    \label{fig:exp16-habituation}
\end{figure}

\subsubsection{Comparing subextraction from the subject with subextraction from the object}

We fitted a first model to compare extractions out of the subject and out of the object on their own (mean centered with subject coded negative and object coded positive). We included trial number as a covariate, and random slopes for the fixed effect and covariates grouped by participants and items. The results of the model are reported in Table \ref{tab:exp16-m1}. 
There is a significant effect of the syntactic function, such that the object condition has significantly higher ratings than the subject condition. There is also a significant effect of trial (habituation).

% latex table generated in R 3.6.3 by xtable 1.8-4 package
% Mon Apr 13 17:44:16 2020
\begin{table}
\begin{tabular}{l S[table-format=-1.3] S[table-format=1.3] S[table-format=1] S[table-format=1.4] S[table-format=1.2]}
  \lsptoprule
 & {Estimate} & {SE} & {$z$} & {$\text{Pr}(>|z|)$} & {Odd.ratio} \\ 
  \midrule
  syntactic function & -0.070 & 0.324 & -0 & 0.8286 & 1.07 \\ 
  trial              & 0.024 & 0.018 & 1 & 0.1717 & 1.02 \\ 
   \lspbottomrule
\end{tabular}
\caption{Results of the Cumulative Link Mixed Model (model n$^{\circ}$1)}
\label{tab:exp06-m1}
\end{table}


We fitted a second model to compare extractions out of the subject and out of the object on their own, and crossed this factor with trial number in order to see if they differ in terms of habituation. We included participants and items as random variables. The results of the model are reported in Table \ref{tab:exp16-m2}. 
The main effect of syntactic function which is seen in model n$^{\circ}$1 disappears in this model. There is no significant interaction, contrary to the impression we may get based on \figref{fig:exp16-habituation}. 

% latex table generated in R 3.6.3 by xtable 1.8-4 package
% Sat Apr 25 19:15:30 2020
\begin{table}
\begin{tabular}{l S[table-format=1.3] S[table-format=1.3] c S[table-format=<1.4] S[table-format=1.2]}
  \lsptoprule
 & {Estimate} & {SE} & {$z$} & {$\text{Pr}(>|z|)$} & {OR} \\ 
  \midrule
  syntactic function & 0.134 & 0.091 & 1 & 0.1405 & 1.14 \\ 
  extraction type & 0.631 & 0.133 & 5 & <.001 & 1.88 \\ 
  trial & 0.025 & 0.005 & 5 & <.001 & 1.03 \\ 
  syntactic function:extraction type & 0.009 & 0.088 & 0 & 0.9142 & 1.01 \\ 
   \lspbottomrule
\end{tabular}
\caption{Results of the Cumulative Link Mixed Model (model n$^{\circ}$5)}
\label{tab:exp13-m5}
\end{table}


In a third model, we compared the subextractions with the non-extractions. We fitted a model crossing syntactic function and extraction type (mean centered with extraction coded positive, non-extraction coded negative). We included trial number as a covariate, and random slopes for all fixed effects grouped by participants and items. The results of the model are reported in Table \ref{tab:exp16-m3}. 
There is a significant main effect of syntactic function (in favor of the object), a significant main effect of extraction type (non-extraction has higher ratings), and a significant main effect of trial (habituation). There is also a significant interaction effect. \figref{fig:exp16-interaction} indeed shows a strong decrease in the ratings for the extraction out of subjects compared to the other conditions. The difference is also significant ($p < 0.05$) if we compare the AUCs (green and red curves on \figref{fig:exp16-ROC-subj}). 

% latex table generated in R 3.6.3 by xtable 1.8-4 package
% Fri Apr 10 17:22:54 2020
\begin{table}
\begin{tabular}{l S[table-format=-1.3] S[table-format=1.3] S[table-format=3.2] S[table-format=-1] S[table-format=<1.4] S[table-format=1.2]}
  \lsptoprule
 & {Estimate} & {SE} & {df} & {$t$} & {$\text{Pr}(>|t|)$} & {OR} \\ 
  \midrule
(Intercept) & 0.856 & 0.577 & 152.52 & 1 & 0.1403 & 2.35 \\ 
  extraction type & -0.131 & 0.061 & 23.29 & -2 & <.05 & 1.14 \\ 
  distance & 0.169 & 0.105 & 25.66 & 2 & 0.1213 & 1.18 \\ 
  length & 0.557 & 0.014 & 521.85 & 39 & <.001 & 1.74 \\ 
  extraction type:distance & -0.042 & 0.073 & 29.31 & -1 & 0.5741 & 1.04 \\ 
   \lspbottomrule
\end{tabular}
\caption{Results of the Linear Mixed Model (model n$^{\circ}$1)}
\label{tab:exp03-m1}
\end{table}


\begin{figure}
    \centering
    \includegraphics[width=\textwidth]{chapters/part2-Empirical/Exp16-infinitival-que/interaction.jpeg}
    \caption{Interaction between syntactic function and extraction type in Experiment~16}
    \label{fig:exp16-interaction}
\end{figure}

\subsection{Discussion}

The contrast between extraction out of the subject and extraction out of the object is more noticeable in this experiment than in the previous one. In general, subextraction out of an infinitival complement seems more felicitous with \emph{où} than with \emph{que}, because in this experiment we see a main effect of extraction type (model n$^{\circ}$3) that we did not observe in Experiment~15. 

We see a ``subject island effect'': The extraction out of the subject receives significantly lower ratings than the extraction out of the object (model n$^{\circ}$1), and there is a significant interaction such that extractions out of the subject had lower ratings than all other conditions (model n$^{\circ}$3). However, notice that the significant difference between extraction out of the subject and out of the object disappears if we cross this factor with trial number. Nevertheless, the interaction effect seems strong: it is also significant with the more conservative method of comparing the AUCs.

The ratings for extraction out of the infinitival subject are not very low: 4.47 on a scale of 1 to 10. For comparison, the ungrammatical distractors (wrong subject-verb agreement) received a mean rating of 1.72. Traditional syntactic accounts would not be able to explain such large difference between extraction out of the subject and ungrammatical controls.

