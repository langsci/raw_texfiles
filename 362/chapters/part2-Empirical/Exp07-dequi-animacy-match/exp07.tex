\section[head=Experiment 7]{Experiment 7: Acceptability judgment study on \emph{de qui} relative clauses with animate subject and object}
\label{ch:exp07}

In the sentences tested in Experiments~5 and 6, there was an animacy mismatch between the subject and the object of the relative clause. The main verb was a psych verb, with either an experiencer object (extraction out of the subject) or an experiencer subject (extraction out of the object), and the extraction always took place out of the NP denoting the stimulus. Since the stimulus was always a quality noun, it was inanimate, while the experiencer was always animate. 

To my knowledge, the effect of animacy on subextractions from NPs has never been studied. However, there is some work on the effect of animacy on extraction of NPs. 

For example, \citet{Gennari.2008} ran a self-paced reading study, crossing extraction of the stimulus vs.\ the experiencer of psych verbs with active vs.\ passive voice. The object was extracted in active sentences, and the subject in passive sentences. (\ref{ex:Gennari-2008}) shows a sample item. After each trial, participants had to answer a comprehension question, targeting mostly the content of the relative clause.

\eal \label{ex:Gennari-2008}
\ex[]{{Extraction of the experiencer (animate) + active (object)}\\ The director that the movie pleased~\trace{} had received a prize at the film festival.}
\ex[]{{Extraction of the stimulus (inanimate) + active (object)}\nopagebreak\\ The movie that the director watched~\trace{} had received a prize at the film festival.}
\ex[]{{Extraction of the experiencer (animate) + passive (subject)}\\ The director that~\trace{} was pleased by the movie had received a prize at the film festival.}
\ex[]{{Extraction of the stimulus (inanimate) + passive (subject)}\\ The movie that~\trace{} was watched by the director had received a prize at the film festival.}
\zl 

They observed that reading times slows down (after \emph{that} and until the end of the sentence) when the animate experiencer was extracted. They also observed slower reading times on the regions after the relative clause for extractions of the object experiencer (with a significant interaction effect). These results correlated with the results of comprehension questions: participants made more mistakes in extractions of the object animate experiencer (mean accuracy 69\%) than in the other conditions (mean accuracy between 81\% and 84\%). Animacy therefore seems to play a role in extraction, especially in an inanimate subject + animate object configuration.

We do not know how \citegen{Gennari.2008} results transfer to subextraction. But the animacy mismatch between subject and object could have played a role in our results of Experiments~5 and 6. For this reason, Experiment~7 replicated Experiment~5 without the animacy mismatch. The results of this study have already been published in \citet{Abeille.2020.JFLS} in a less detailed fashion. %We also ran a replication of this study: Experiment 7B can be found in the Annex, Section \ref{ch:exp07b}.

\subsection{Design and materials}
In Experiment~5, in contrast to all experiments with \emph{dont} relative clauses, extractions out of the NP were rated relatively low compared to the baseline. However, Experiment~5 did not include ungrammatical controls to see how bad the subextraction was considered. For this reason, in Experiment~7, we used a 2*3 design, i.e.\ a design very similar to Experiment~4. We compared extractions out of subjects (\ref{ex:exp07-subj-pp}) with extractions out of objects (\ref{ex:exp07-obj-pp}), and paired each with a non-extraction control with coordination (\ref{ex:exp07-subj-no};\ref{ex:exp07-obj-no}) and an ungrammatical control with the preposition \emph{de} missing (\ref{ex:exp07-subj-un};\ref{ex:exp07-obj-un}).

\eal 
\ex[]{{Condition subject + PP-extracted:}\nopagebreak\\
\gll  J' ai pris un avocat, de qui l' associé aide mon cousin sans contrepartie financière.\\
I have taken a lawyer of who the associate helps my cousin without counterpart financial\\
\glt `I took a lawyer of whom the associate helps my cousin without financial compensation.'}
\label{ex:exp07-subj-pp}
\ex[]{{Condition object + PP-extracted:}\\
\gll  J' ai pris un avocat, de qui mon cousin aide l' associé sans contrepartie financière.\\
I have taken a lawyer of who my cousin helps the associate without counterpart financial\\
\glt `I took a lawyer of whom my cousin helps the associate without financial compensation.'}
\label{ex:exp07-obj-pp}
\ex[]{{Condition subject + noextr:}\nopagebreak\\
\gll J' ai pris un avocat, et l' associé de cet avocat aide mon cousin sans contrepartie financière.\\
I have taken a lawyer and the associate of this lawyer helps my cousin without counterpart financial\\
\glt `I took a lawyer and the associate of this lawyer helps my cousin without financial compensation.'}
\label{ex:exp07-subj-no}
\ex[]{{Condition object + noextr:}\\
\gll J' ai pris un avocat, et mon cousin aide l' associé de cet avocat sans contrepartie financière.\\
I have taken a lawyer and my cousin helps the associate of this lawyer without counterpart financial\\
\glt `I took a lawyer and my cousin helps the associate of this lawyer without financial compensation.'}
\label{ex:exp07-obj-no}
\ex[]{{Condition subject + ungrammatical:}\\
\gll J' ai pris un avocat, qui l' associé aide mon cousin sans contrepartie financière.\\
I have taken a lawyer who the associate helps my cousin without counterpart financial\\
\glt `I took a lawyer who the associate helps my cousin without financial compensation.'}
\label{ex:exp07-subj-un}
\ex[]{{Condition object + ungrammatical:}\\
\gll J' ai pris un avocat, qui mon cousin aide l' associé sans contrepartie financière.\\
I have taken a lawyer who my cousin helps the associate without counterpart financial\\
\glt `I took a lawyer who my cousin helps the associate without financial compensation.'}
\label{ex:exp07-obj-un}
\zl 

Because both subject and object are animate, we used the same verb in all conditions. Many of the verbs in the relative clause were psych-verbs (e.g., \emph{effrayer} `scare') but not all of them (e.g., \emph{aider} `help'). Subject and object NPs were relational (e.g., \emph{cousin} `cousin', \emph{associé} `associate'). Complements of relational nouns are highly relevant in the sense of \citet[Section 3.1.2]{Chaves.2012} because the existence of the referent they denote is presupposed by the relational noun: one can only be ``cousin'' with respect to someone else. The NP that should not be interpreted as containing the gap always contained a possessive determiner \emph{mon/ma} (`my').   

We tested 20 items, each containing in the six conditions already described. In addition, the experiment included 24 distractors. 
About one third of the items and distractors were followed by a comprehension question. The item presented here as an example was paired with the comprehension question \emph{Cet avocat a-t-il un associé~?} (`Does this lawyer have an associate?'). 

\subsection{Predictions} 

Animacy was not expected to have any impact on the acceptability of subject island structures. Therefore, the predictions were the same as those summarized in Table \ref{tab:exp05-predictions} on page \pageref{tab:exp05-predictions}.

\subsection{Procedure} 

We conducted the experiment on the Ibex platform \citep{Ibex}. The procedure was similar to the procedure used in the previous acceptability judgment experiments (see Section \ref{ch:methodo-AJ}). Participants rated the sentences on a Likert scale from 0 to 10, 0 being labeled as ``bad'' and 10 being labeled as ``good''. They also answered comprehension questions after some of the sentences.

The Experiment~took approximately 20 minutes to complete. 

\subsection{Participants}

The study was run between September and October 2018.  
35 participants took part in the experiment. Participants were recruited on the R.I.S.C.\ website (\url{http://experiences.risc.cnrs.fr/}) and on social media (e.g.\ Facebook). They received no financial compensation for taking part in the experiment. 

The analysis presented here is based on the data from the 26 participants who satisfied all inclusion criteria.\footnote{To calculate accuracy, we excluded not only the answers to comprehension questions of the practice items, but also the extractions out of the object, which had an overall accuracy rate of 69\% only.}
The 26 participants were aged 18 to 75 years. 19 of them self-identified as women, and six as men. Five participants (20\%) indicated having an educational background related to language.

\subsection{Results and analysis}

\figref{fig:exp07-boxplot} shows the results of the acceptability judgment task.\footnote{We involuntarily introduced a typo in one condition of one experimental item. For this reason the item was excluded from the results and treated as a distractor. The results reported here are therefore based on 19 experimental items.}
In the subextraction conditions, the mean rating for extraction out of the subject (\ref{ex:exp07-subj-pp}) was 5.15, slightly higher than extraction out of the object (\ref{ex:exp07-obj-pp}) with a mean rating of 4.64. The control conditions without extraction received better acceptability judgements: 7.90 in the subject condition (\ref{ex:exp07-subj-no}) and 7.65 in the object condition (\ref{ex:exp07-obj-no}). The ungrammatical controls were rated lower: the subject condition (\ref{ex:exp07-subj-un}) has a mean rating of 3.53, and the object condition (\ref{ex:exp07-obj-un}) 3.41. 

\begin{figure}
    \centering
    \includegraphics[width=\textwidth]{chapters/part2-Empirical/Exp07-dequi-animacy-match/boxplots.jpeg}
    \caption{Acceptability judgments by condition in Experiment~7. The grey box plots indicate the median and quartiles of the results. Black points are outliers. Mean and confidence intervals are indicated in white.}
    \label{fig:exp07-boxplot}
\end{figure}

\figref{fig:exp07-boxplot} suggests potential ceiling effects, but only in the non-extraction conditions. There is also a possible floor effect for the ungrammatical controls. Thus we have further evidence that \emph{de qui} extractions out of NPs are judged in the middle of the scale, unlike \emph{dont} extractions. The distribution of the ratings is illustrated by \figref{fig:exp07-repartition}: we observe a clear ceiling effect in the non-extraction conditions and a small floor effect in the ungrammatical controls. The ungrammatical object controls may show a small bimodality, with some items rated relatively high in the scale. There is no ceiling or floor effect in the subextraction conditions, but the ratings do not seem normally distributed either, as the curve is relatively flat. The $z$-scored ratings ($z$-score for each participant) in \figref{fig:exp07-repartition-z} show two peaks for both subextractions, suggesting bimodality. The peaks are situated to the right and left of 0. Participants seemed to classify the subextractions either as very good or very bad, but not in the middle of their scale. 

\begin{figure}
    \centering
    \includegraphics[width=\textwidth]{chapters/part2-Empirical/Exp07-dequi-animacy-match/repartition.jpeg}
    \caption{Density of the ratings across conditions for Experiment~7}
    \label{fig:exp07-repartition}
\end{figure}

\begin{figure}
    \centering
    \includegraphics[width=\textwidth]{chapters/part2-Empirical/Exp07-dequi-animacy-match/repartition-z.jpeg}
    \caption{Density of the z-transformed ratings across conditions for Experiment~7}
    \label{fig:exp07-repartition-z}
\end{figure}

Another representation of the results is given by the ROC and zROC curves of the data in \figref{fig:exp07-ROC}. The ROC curves show that participants discriminated between the ungrammatical baselines and the other conditions. Unsurprisingly, the discrimination is stronger for the non-extraction conditions than for the subextraction conditions. The zROC curves are not very straight, which corroborates the data in \figref{fig:exp07-repartition}. Several conditions seem to have a bimodal distribution. 

\begin{figure}
    \centering
    \includegraphics[width=\textwidth]{chapters/part2-Empirical/Exp07-dequi-animacy-match/ROC.jpeg}
    \includegraphics[width=\textwidth]{chapters/part2-Empirical/Exp07-dequi-animacy-match/zROC.jpeg}
    \caption{ROC curves (top) and zROC curves (bottom) for the non-extraction conditions compared to their respective subextraction conditions, represented by the dotted grey baseline (\citealt{Dillon.2019}'s method) in Experiment~7.}
    \label{fig:exp07-ROC}
\end{figure}

The ROC and zROC curves in \figref{fig:exp07-ROC-subj} depict the discrimination between the subject and object conditions. The ROC curves show that the participants hardly discriminate between the subject and object conditions, but there seems to be a slight preference for the subject conditions (curves below the baseline). The zROC curves are relatively straight.

\vfill
\begin{figure}[H]
    \centering
    \includegraphics[width=\textwidth]{chapters/part2-Empirical/Exp07-dequi-animacy-match/ROC-subject.jpeg}
    \includegraphics[width=\textwidth]{chapters/part2-Empirical/Exp07-dequi-animacy-match/zROC-subject.jpeg}
    \caption{ROC curves (top) and zROC curves (bottom) for the object conditions compared to their respective subject conditions, represented by the dotted grey baseline (\citealt{Dillon.2019}'s method) in Experiment~7}
    \label{fig:exp07-ROC-subj}
\end{figure}
\vfill
\pagebreak

\subsubsection{Habituation} 

\figref{fig:exp07-habituation} displays the habituation effects in the course of the experiment. The non-extraction conditions do not display a habituation effect, but the subextractions do. Habituation was stronger for the ungrammatical + subject control: this condition was rated very low in the early trials of the experiment, but received ratings close to those in extractions out of the object at the end of the experiment. 

\begin{figure}
    \centering
    \includegraphics[width=\textwidth]{chapters/part2-Empirical/Exp07-dequi-animacy-match/habituation.jpeg}
    \caption{Changes in the mean acceptability ratings ($z$-scored by participant) for each condition of Experiment~7 in the course of the experiment}
    \label{fig:exp07-habituation}
\end{figure}

\subsubsection{Comparing subextraction from the subject with subextraction from the object}

Our first model compared extractions out of the subject and out of the object on their own (mean centered with subject coded negative and object coded positive). We included trial number as a covariate, and random slopes for the fixed effects grouped by participants and items. The results of the model are reported in Table~\ref{tab:exp07-m1}. There is a significant effect of syntactic function, such that ratings for the subject condition are significantly higher than for the object condition. There is also a significant effect of trial (habituation).

% latex table generated in R 3.6.3 by xtable 1.8-4 package
% Fri Apr 24 21:31:19 2020
\begin{table}
\begin{tabular}{l S[table-format=1.3] S[table-format=1.4] c S[table-format=<1.3] S[table-format=1.2]}
  \lsptoprule
 & {Estimate} & {SE} & {$z$} & {$\text{Pr}(>|z|)$} & {OR} \\ 
  \midrule
  syntactic function & 0.459 & 0.144 & 3 & <.005 & 1.58 \\ 
  trial              & 0.016 & 0.010 & 2 & 0.1025 & 1.02 \\ 
   \lspbottomrule
\end{tabular}
\caption{Results of the Cumulative Link Mixed Model (model n$^{\circ}$2)}
\label{tab:exp12-m2}
\end{table}


In a second model, we compared subextraction with non-extraction. We fitted a model crossing syntactic function and extraction type (mean centered with extraction coded positive, non-extraction coded negative). We included trial number as a covariate, and participants and items as random factors. The results of the model are reported in Table \ref{tab:exp07-m2}. 
There is a significant main effect of syntactic function (in favor of the subject), of extraction type (in favor of the non-extraction controls) and of trial (habituation), but no interaction effect. If we compare the AUC (green and red curves in \figref{fig:exp07-ROC-subj}), the difference is not significant, either. Indeed, in \figref{fig:exp07-interaction} all lines seem almost perfectly parallel (and may be perfectly parallel without the ceiling effect in the non-extraction conditions). 

% latex table generated in R 3.6.3 by xtable 1.8-4 package
% Mon Apr 13 17:44:16 2020
\begin{table}
\begin{tabular}{l S[table-format=-1.3] S[table-format=1.3] S[table-format=1] S[table-format=1.4] S[table-format=1.2]}
  \lsptoprule
 & {Estimate} & {SE} & {$z$} & {$\text{Pr}(>|z|)$} & {Odd.ratio} \\ 
  \midrule
  syntactic function & -0.070 & 0.324 & -0 & 0.8286 & 1.07 \\ 
  trial              & 0.024 & 0.018 & 1 & 0.1717 & 1.02 \\ 
   \lspbottomrule
\end{tabular}
\caption{Results of the Cumulative Link Mixed Model (model n$^{\circ}$1)}
\label{tab:exp06-m1}
\end{table}


\begin{figure}
    \centering
    \includegraphics[width=\textwidth]{chapters/part2-Empirical/Exp07-dequi-animacy-match/interaction.jpeg}
    \caption{Interaction between syntactic function and extraction type in Experiment~7}
    \label{fig:exp07-interaction}
\end{figure}

\subsubsection{Comparing subextraction from the subject with the ungrammatical controls}

In our third model, we compared extractions out of the subject and the ungrammatical subject controls on their own (mean centered with subextraction coded positive and ungrammatical coded negative). We included trial number as a covariate, and random slopes for the fixed effects grouped by participants and items. The results of the model are reported in Table \ref{tab:exp07-m3}. There is a significant effect of extraction type, such that the ratings are significantly higher for extraction out of the subject than for its ungrammatical control. There is also a significant effect of trial (habituation).

% latex table generated in R 3.6.3 by xtable 1.8-4 package
% Thu Apr 23 00:04:53 2020
\begin{table}
\begin{tabular}{l S[table-format=1.3] S[table-format=1.3] c S[table-format=<1.3] S[table-format=2.2]}
  \lsptoprule
 & {Estimate} & {SE} & {$z$} & {$\text{Pr}(>|z|)$} & {OR}\\ 
  \midrule
  extraction type & 2.342 & 0.395 & 6 & <.001 & 10.40 \\ 
  trial           & 0.039 & 0.012 & 3 & <.005 & 1.04 \\ 
   \lspbottomrule
\end{tabular}
\caption{Results of the Cumulative Link Mixed Model (model n$^{\circ}$3)}
\label{tab:exp10-m3}
\end{table}


In a fourth model, we compared the subextraction with the ungrammatical controls. The model crossed syntactic function (mean centered with object coded positive, subject coded negative) and extraction type (grammaticality). We included trial number as a covariate, and random slopes for all fixed effects grouped by participants and items. The results of the model are reported in Table \ref{tab:exp07-m4}. 
There is a significant main effect of syntactic function (in favor of the subject), of extraction type (in favor of the extraction conditions) and of trial (habituation) but no interaction effect.

% latex table generated in R 3.6.3 by xtable 1.8-4 package
% Sat Apr 25 19:15:30 2020
\begin{table}
\begin{tabular}{l S[table-format=1.3] S[table-format=1.3] c S[table-format=<1.4] S[table-format=1.2]}
  \lsptoprule
 & {Estimate} & {SE} & {$z$} & {$\text{Pr}(>|z|)$} & {OR} \\ 
  \midrule
  syntactic function & 0.134 & 0.091 & 1 & 0.1405 & 1.14 \\ 
  extraction type & 0.631 & 0.133 & 5 & <.001 & 1.88 \\ 
  trial & 0.025 & 0.005 & 5 & <.001 & 1.03 \\ 
  syntactic function:extraction type & 0.009 & 0.088 & 0 & 0.9142 & 1.01 \\ 
   \lspbottomrule
\end{tabular}
\caption{Results of the Cumulative Link Mixed Model (model n$^{\circ}$5)}
\label{tab:exp13-m5}
\end{table}


\subsection{Discussion}

In Experiment~5, extraction out of the subject received lower ratings than extraction out of the object, with a significant interaction effect. In this experiment, extraction out of the subject received higher ratings than extraction out of the object, but there was no interaction effect. The factor that changed between Experiment~5 and the present experiment is the animacy mismatch between subject and object. Indeed, in line with  \citegen{Gennari.2008} findings, we can see that extraction of the object is judged better when the object is inanimate and the subject animate than the other way around. This could indicate a general preference for a configuration in which the subject is animate and the object inanimate, a pattern very often observed with agentive verbs, and therefore very frequent. Since extraction involves processing difficulties, it reflects this preference which is probably less apparent in ratings for easier sentences, like our grammatical controls. The significant difference that we saw in Experiment~5 can therefore be explained as a superadditive effect resulting from the processing difficulty linked to extraction on the one hand and the processing difficulty linked to the low frequency of the configuration (subject inanimate and object animate) on the other hand.

This explanation seems more adequate than one based on a syntactic subject island: if extraction out of the subject were indeed ungrammatical, the decrease in acceptability in Experiment~7 should have been much stronger. An explanation based on a  superadditive processing effect linked to complex subjects is not satisfactory, either, because if that were the case then Experiment~7 should replicate the results of Experiment~5. 

As far as Experiment~7 is concerned, the fact that extraction out of the subject received significantly better ratings than extraction out of the object is in contradiction with the expectations of most accounts (based on syntax and processing) that predict a superadditivity effect. The results also do not display the expected interaction effect. The fact that extraction out of the subject is significantly better than ungrammatical controls is unexpected if subjects are syntactic islands. 

Both processing accounts based on memory costs and the discourse-based FBC constraint predict better acceptability ratings for extraction out of the subject than for extraction out of the object, but only processing accounts expect a significant interaction as well. The results of Experiment~7 do not falsify these two kinds of accounts.

Where does the significant difference between extraction out of the subject and out of the object come from? \figref{fig:exp07-byparticipant} shows the ratings for each participant. Out of 25 participants, only five have a higher mean rating for extractions out of objects. But the difference between subject and object is not very large, and most participants treated them similarly: The mean difference between each participant's mean ratings for extractions out of object vs.\ out of subject is only 0.42 (standard deviation: 1.2). Thus the participants' behavior is relatively homogeneous. 

\begin{figure}
    \centering
    \includegraphics[width=\textwidth]{chapters/part2-Empirical/Exp07-dequi-animacy-match/by-participant.jpeg}
    \caption{Ratings of the subextraction conditions for each participant in Experiment~7}
    \label{fig:exp07-byparticipant}
\end{figure}

There is a bit more variability between items. Three items show a strong preference for extraction out of the object: the mean rating for extractions out of the object is more than 2 points higher than the mean rating for extractions out of the subject. One such item is shown in (\ref{ex:exp07-good-object}):\pagebreak

\eal\label{ex:exp07-good-object}
\ex[]{{Condition subject + PP-extracted:}\nopagebreak\\
\gll J' ai un dentiste, de qui l' assistante aime bien ma mère malgré ses plaintes continuelles.\\
I have a dentist of who the assistant likes well my mother despite her complaints perpetual\\
\glt `I have a dentist, of who the assistant likes my mother despite her perpetual complaints.'}
\ex[]{{Condition object + PP-extracted:}\nopagebreak\\
\gll J' ai un dentiste, de qui ma mère aime bien l' assistante malgré ses plaintes continuelles.\\
I have a dentist of who my mother likes well the assistant despite her complaints perpetual\\
\glt `I have a dentist, of who my mother likes the assistant despite her perpetual complaints.'}
\zl 

Five other items, including (\ref{ex:exp07-good-subject}), show a strong preference for extraction out of the subject: the mean rating for extractions out of the subject is more than 2 points higher than the mean rating for extractions out of the object.

\eal\label{ex:exp07-good-subject}
\ex[]{{Condition subject + PP-extracted:}\\
\gll Il y a ce collègue, de qui le stagiaire impressionne mon stagiaire pendant la préparation d’ une conférence.\\
it there has this colleague of who the trainee impresses my trainee during the preparation of a conference\\
\glt `There is this colleague, of who the trainee impresses my trainee during a conference preparation.'}
\ex[]{{Condition object + PP-extracted:}\\
\gll Il y a ce collègue, de qui mon stagiaire impressionne le stagiaire pendant la préparation d’ une conférence.\\
it there has this colleague of who my trainee impresses the trainee during the preparation of a conference\\
\glt `There is this colleague, of who my trainee impresses the trainee during a conference preparation.'}
\zl\largerpage

Most of the time, however, the difference between the two conditions is small. I was not able to identify any clear parameter to account for the variability between items.{\interfootnotelinepenalty=10000\footnote{In (\ref{ex:exp07-good-subject}), the subject and the object are the same noun, but this does not seem to play a role: there are several items with the same noun for subject and object, and not all of them show a strong subject preference: in fact, one of them shows a strong object preference.}}

In general, the preference for extraction out of the subject over extraction out of the object seems to be a general tendency, and not an effect created by some specific items or participants. The effect size is not large (model n$^{\circ}$1 gives an odds ratio of 1.42 for the syntactic function), as could be expected from an effect that reflects small processing preferences.
