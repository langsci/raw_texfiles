\section{Attested extractions out of the subject in interrogatives}
\label{ch:attested-questions-subextraction}

Even though there were no case of extraction out of subjects with \emph{de qui} and \emph{de quel} in interrogatives in our corpus studies, it is possible to find some on the Internet. I will present here a few examples we found, will comment on them and will show how they are compatible with a discourse-based analysis.

\subsection{Examples with a pre-defined set of alternatives}

The following example can be found in a quiz on a website where visitors can create and take quizzes on any possible subject. Here, the subject of the quiz is the ``Kids United''.\footnote{\url{https://www.quizz.biz/quizz-1027073.html, las access 14/03/2020}}

\ea[]{\gll De qui l' anniversaire tombe -t- il le 27 ?\\
of who the birthday falls 0 it the 27\\
\glt `Of who is the birthday on the 27th?'}
\label{ex:internet-kidsunited}
\z 

The Kids United was a French music band consisting of five children when this quiz was released (28/12/2006). The question in (\ref{ex:internet-kidsunited}) is followed by a set of possible responses: the names of the five members of the group at that time, Gabriel, Gloria, Esteban, Erza and Nilusi. 
Many of the previous and following questions in the quiz have the same set of possible responses since the aim of the quiz is to allow participants to test their knowledge about the members of the group: how old they are, what countries they are from, what musical instrument they can play, what are their favorite colors or superheroes, etc.

We can therefore conclude that the extracted element \emph{de qui} (`of who') in (\ref{ex:internet-kidsunited}) only superficially opens the set of alternatives from which the right answer has to be chosen, because this set of alternatives was already defined beforehand, if not by the topic of the quiz then by the previous questions. This probably makes the extraction less focalized, and as a consequence it is more felicitous. 

% https://satires18.univ-st-etienne.fr/texte/ci-g%C3%AEt-m-le-duc-de-prie-fr%C3%A8res-p%C3%A2ris/epitaphe-de-m-le-duc-de-bourbon

The second example is similar in this respect. It can be found on a tutorial website\footnote{\url{https://fr.wikihow.com/jouer-\%C3\%A0-\%C2\%AB-devine-ce-que-je-vois-\%C2\%BB}, last access 14/03/2020} explaining how to play a game called ``guess what I see'':

\ea[]{\gll Vous pourrez choisir le premier espion de différentes façons. Par exemple, vous pourrez~[\dots] demander [de qui]$_i$ [l’ anniversaire\trace{}$_i$] est le plus proche~[\dots].\\
you will.can choose the first spy of different manners for example you will.can ask\textsc{.inf} of who the birthday is the most close\\
\glt `There are different ways to choose the first spy. For example, you can ask whose birthday is the closest.'}
\label{ex:internet-espion}
\z 

In this case, participants in the game must choose a first player (or first ``spy''), and asking for their birthday is one method of selecting the first player. It is clear beforehand that the first player must be chosen from the players of the game. Again, the set of alternatives was known and part of the Common Ground. This could make the extraction less focalized, and thus make the discourse clash less pronounced. 

\subsection{Questions with a pre-defined answer (rhetorical question)}

The first example comes from a political speech given by the Iranian president Mahmoud Ahmadinejad.\footnote{\url{https://www.voltairenet.org/article171526.html}, last access 14/03/2020}

\ea[]{\gll [De quel pays]$_i$ [la dépense militaire~\trace{}$_i$] dépasse annuellement mille milliards de dollars~[\dots]~?\\
of which country the budget military exceeds yearly thousand billion of dollars\\
\glt `Of which country does the military budget exceed 100 B.\ dollars?'}
\label{ex:internet-ahmadinejad}
\z\vskip-.5\baselineskip

This is obviously a translation from the original speech, and we have to bear in mind that the syntax of the sentence in the original text may have an influence on the syntax of the translation. This question follows a series of other questions very clearly targeting the USA, with explicit mentions of the nuclear bombs in Hiroshima and Nagasaki and the events of the 11th September 2001 among others. As such, the answer to the preceding series of questions has systematically been ``the USA'', and the hearer/reader is primed to assume the same answer to this question as well. In this case, we can even consider the extracted element \emph{de quel pays} (`of which country') to be more of a topic continuation than focus, and the rest of the utterance can be seen as the focal domain with new pieces of information. It can be paraphrased as: \emph{speaking of this country, its military budget exceeds 100 B.\ dollars}. The question is a rhetorical question, and the particular status of the extracted element can explain why extraction is felicitous and facilitated. 

The second example is part of a piece of poetry from 1741, an epitaph for the ``Duke of Bourbon'', found on a website\footnote{\url{https://satires18.univ-st-etienne.fr/texte/ci-g\%C3\%AEt-m-le-duc-de-prie-fr\%C3\%A8res-p\%C3\%A2ris/epitaphe-de-m-le-duc-de-bourbon}, last access 14/03/2020} that collects satirical poetry of the 18th century. I assume that Louis IV Henri de Bourbon-Condé (1692--1740) is the intended referent. The author of the poem is unknown.\largerpage[2.25]\vskip-.5\baselineskip

\ea[]{\gll Au fond de ce noir monument\\
at.the depth of this black memorial\\ \newline
\gll Sais - tu [de qui]$_i$ [le corps~\trace{}$_i$] repose~?\\
know {} you of how the body rests\\ \newline
\gll C’ est d’ un Condé, non pas le Grand.\\
it is of a Condé \textsc{neg} not the Great\\
\glt `In the depth of this memorial\\
Do you know whose body rests?\\
It is the one of a Condé, not the Great\footnote{This is a reference to Louis II de Bourbon, Prince of Condé (1621--1686), also called the ``Great Condé'', the most famous member of the Condé family.} one'}
\z

In this case also I consider the question to be a rhetorical one. 
%If the epitaph is written on the grave of Louis IV Henri de Bourbon-Condé, the reader knows already who is the corpse buried in this memorial. 
The title ``Epitaphe de M.\ le duc de Bourbon'' already revealed who the epitaph is for and hence whose body is meant. The question is a rhetorical way for highlighting the disgrace of the Duke (he fell into disgrace and was exiled by the king at the end of his life) and hints that he may not be worthy of being known (hence also the allusion to his famous ancestor, with whom the Duke cannot compare). 

\subsection{Conclusion on the extractions out of subjects found on the Internet}

These four examples are the only ones that I have been able to find on the Internet so far.\footnote{I thank Anne Abeillé who discovered examples (\ref{ex:internet-kidsunited}) and (\ref{ex:internet-ahmadinejad}).} There is no good way to formulate a query that would systematically identify similar examples, because there would be too much noise: As shown in our corpus studies, extraction out of the subject in interrogatives is very rare.

In all of the examples, the alternative set opened by the focalization of the extracted element has a very particular status: it was already present in the Common Ground (and sometimes reduced to a set of one element). In fact, I am not sure that example (\ref{ex:internet-ahmadinejad}) is a focalization at all.

We have assumed that the difficulty of extracting out of the subject in interrogatives comes from a discourse clash caused by focalizing part of a backgrounded element. Now we can explain why the attested extractions are felicitous: the element extracted in these special examples is less focal than in prototypical interrogatives, because it does not really introduce an alternative set to the Common Ground. For this reason, we have to assume that being focus is not categorical, but continuous: an element can be more or less focused, and can be more or less backgrounded. I will come back to this important point in Section \ref{ch:general-analysis-gradability}.
