% Please add the following required packages to your document preamble:
% \usepackage[table,xcdraw]{xcolor}
% If you use beamer only pass "xcolor=table" option, i.e. \documentclass[xcolor=table]{beamer}
% \usepackage{lscape}
% \usepackage{longtable}
% Note: It may be necessary to compile the document several times to get a multi-page table to line up properly
\begin{landscape}
	\begin{longtable}{llllll}
		\caption{Studies on English \textit{wh}-questions}\label{tab:previous-english-wh}\\
		\lsptoprule
		Publication &
		{\begin{tabular}[c]{@{}l@{}}Subject\\ type\end{tabular}} &
		{\begin{tabular}[c]{@{}l@{}}Filler\\ (island condition)\end{tabular}} &
		{Task} &
		{Design} &
		{Results} \\ \midrule
		\endfirsthead
		\midrule
		\endhead
		%
		\begin{tabular}[c]{@{}l@{}}\citet{Phillips.2006}, \\ Experiment 1\end{tabular} &
		NP &
		\begin{tabular}[c]{@{}l@{}}Direct object\\ (of the complement / \\ of the relative clause)\end{tabular} &
		\begin{tabular}[c]{@{}l@{}}Acceptability \\ ratings,\\ Likert scale\end{tabular} &
		\begin{tabular}[c]{@{}l@{}}Tested embedded \textit{wh}-question, crossing extractee type \\ (direct object\slash complement of the subject NP\slash both) \\ with finiteness (non-finite complement of NP\slash finite \\ relative clause of NP).\end{tabular} &
		\begin{tabular}[c]{@{}l@{}}- interaction extractee type:finiteness ($p < 0.0001$)\\ - no effect of finiteness for direct objects\\ - extraction out of the subject rated higher in the\\ non-finite condition ($p < 0.05$)\\ - double gap rated higher in the non-finite condition\\ ($p < 0.0001$)\end{tabular} \\ \midrule
		\begin{tabular}[c]{@{}l@{}}\citet{Phillips.2006}, \\ Experiment 2\end{tabular} &
		NP &
		\cellcolor[HTML]{C0C0C0} &
		\begin{tabular}[c]{@{}l@{}}Self-paced \\ reading\end{tabular} &
		\begin{tabular}[c]{@{}l@{}}Tested embedded \textit{wh}-questions, crossing plausibility\\ (filler is plausible as subject complement\slash implausible \\ as subject complement) and finiteness (non-finite\\ complement of subject\slash finite relative clause of subject).\end{tabular} &
		\begin{tabular}[c]{@{}l@{}}Accuracy on comprehension questions:\\ - no interaction\\ \\ Reading times:\\ - (marginal) interaction plausibility:finiteness\\ ($p < 0.05$ in the participant analysis ; $p = 0.086$ in the \\ item analysis), in that reading times increase only \\ in the implausible + non-finite condition.\end{tabular} \\ \midrule
		\begin{tabular}[c]{@{}l@{}}\citet{Sprouse.2007.PhD}, \\ Experiment in \\ section 3.2\end{tabular} &
		NP &
		\begin{tabular}[c]{@{}l@{}}PP-complement\\ (with preposition\\ stranding)\end{tabular} &
		\begin{tabular}[c]{@{}l@{}}Acceptability \\ ratings,\\ magnitude \\ estimation\end{tabular} &
		\begin{tabular}[c]{@{}l@{}}Crossing function (subject\slash object), extraction type \\ (extraction of NP\slash extraction out of NP) and context\\ (with\slash without supporting context).\end{tabular} &
		\begin{tabular}[c]{@{}l@{}}- interaction function:extraction type ($p < 0.001$; \\ Cohen's d = .521)\\ - no main effect of context\\ - no 3-way interaction\end{tabular} \\ \midrule
		\begin{tabular}[c]{@{}l@{}}\citet{Jurka.2010}, \\ Experiment 8\end{tabular} &
		NP &
		\begin{tabular}[c]{@{}l@{}}Of-complement\\ (with and without\\ preposition \\ stranding)\end{tabular} &
		\begin{tabular}[c]{@{}l@{}}Acceptability \\ ratings,\\ Likert scale\end{tabular} &
		\begin{tabular}[c]{@{}l@{}}Crossing function (subject\slash object), extraction type\\ (no extraction\slash extraction out of the NP) and\\ preposition stranding (with\slash without preposition\\ stranding).\end{tabular} &
		\begin{tabular}[c]{@{}l@{}}- for preposition stranding, interaction function:\\ extraction type, in that extraction out of the\\ subject is rated lowest\\ - without preposition stranding, interaction\\ function: extraction type, in that extraction\\ out of the subject is rated lowest\\ - interaction for preposition stranding larger\\ than without preposition stranding\end{tabular} \\ \midrule
		\begin{tabular}[c]{@{}l@{}}\citet{Jurka.2010}, \\ Experiment 9\end{tabular} &
		NP &
		\begin{tabular}[c]{@{}l@{}}Of-complement\\ (with preposition \\ stranding)\end{tabular} &
		\begin{tabular}[c]{@{}l@{}}Acceptability \\ ratings,\\ Likert scale\end{tabular} &
		\begin{tabular}[c]{@{}l@{}}Crossing extraction type (no extraction\slash extraction out\\ of the subject) and Exceptional Case Marking (simple\\ verb\slash embedding with ECM).\end{tabular} &
		\begin{tabular}[c]{@{}l@{}}- interaction extraction type:ECM, mostly due to the\\ condition ECM with no extraction being much lower than\\ ECM with subextraction\\ - for extraction out of the subject, the simple verb is\\ rated higher than the ECM ($p = 0.031$)\end{tabular} \\ \midrule
		\begin{tabular}[c]{@{}l@{}}\citet{Jurka.2010}, \\ Experiment 10\end{tabular} &
		NP &
		\begin{tabular}[c]{@{}l@{}}Of-complement\\ (with preposition \\ stranding)\end{tabular} &
		\begin{tabular}[c]{@{}l@{}}Acceptability \\ ratings,\\ Likert scale\end{tabular} &
		\begin{tabular}[c]{@{}l@{}}Tested embedded \textit{wh}-questions, crossing extraction type\\ (no extraction\slash extraction out of the subject) and verb\\ type (unergative\slash passive).\end{tabular} &
		\begin{tabular}[c]{@{}l@{}}- interaction extraction type:verb type, in that the\\ extraction out of the subject of the unergative verb is\\ rated lowest, but the effect size is very small\end{tabular} \\ \midrule
		\begin{tabular}[c]{@{}l@{}}\citet{Sprouse.2012},\\ Experiment 1\end{tabular} &
		NP &
		\begin{tabular}[c]{@{}l@{}}PP-complement\\ (with preposition \\ stranding)\end{tabular} &
		\begin{tabular}[c]{@{}l@{}}Acceptability \\ ratings,\\ Likert scale\\ \\ +\\ \\ Serial recall \\ task\end{tabular} &
		\begin{tabular}[c]{@{}l@{}}Crossing function (subject\slash object) and extraction type\\ (extraction of NP\slash extraction out of NP). A measure of\\ working memory is also included.\end{tabular} &
		\begin{tabular}[c]{@{}l@{}}- interaction function:extraction type ($p < 0.0001$)\\ - main effect of serial recall ($p < 0.02$)\end{tabular} \\ \midrule
		\begin{tabular}[c]{@{}l@{}}\citet{Sprouse.2012},\\ Experiment 2\end{tabular} &
		NP &
		\begin{tabular}[c]{@{}l@{}}PP-complement\\ (with preposition\\ stranding)\end{tabular} &
		\begin{tabular}[c]{@{}l@{}}Acceptability \\ ratings,\\ magnitude \\ estimation\\ \\ +\\ \\ Serial recall \\ task and\\ n-back task\end{tabular} &
		\begin{tabular}[c]{@{}l@{}}Crossing function (subject\slash object) and extraction type\\ (extraction of NP\slash extraction out of NP). Two measures \\ of working memory are also included.\end{tabular} &
		\begin{tabular}[c]{@{}l@{}}- interaction function:extraction type ($p < 0.0001$)\\ - no effect of serial recall ($p = 0.7$)\\ - no effect of n-back ($p = 0.66$)\end{tabular} \\ \midrule
		\begin{tabular}[c]{@{}l@{}}\citet{Fukuda.2012},\\ Experiment 1\end{tabular} &
		NP &
		\begin{tabular}[c]{@{}l@{}}Of-complement\\ (with preposition\\ stranding)\end{tabular} &
		\begin{tabular}[c]{@{}l@{}}Acceptability \\ ratings,\\ yes/no\end{tabular} &
		\begin{tabular}[c]{@{}l@{}}Crossing extraction type (no extraction\slash extraction out\\ of NP) and extraction site (nominal subject\slash nominal\\ object\slash \textit{wh}-subject).\end{tabular} &
		\begin{tabular}[c]{@{}l@{}}- interaction extraction type:extraction site\\ ($p = 0.0218$)\\ - extraction out of nominal object rated higher \\ than extraction out of nominal subject ($p = 0.0052$)\\ - extraction out of nominal object rated higher\\ than extraction out of \textit{wh}-subject ($p < 0.001$)\\ - no difference in rating between extraction out of\\ nominal subject and extraction out of \textit{wh}-subject\\ ($p = 0.997$)\end{tabular} \\ \midrule
		\begin{tabular}[c]{@{}l@{}}\citet{Fukuda.2012},\\ Experiment 2\end{tabular} &
		NP &
		\begin{tabular}[c]{@{}l@{}}Of-complement\\ (with preposition\\ stranding)\end{tabular} &
		\begin{tabular}[c]{@{}l@{}}Acceptability \\ ratings,\\ Likert scale\end{tabular} &
		Same as previous &
		\begin{tabular}[c]{@{}l@{}}- interaction extraction type:extraction site\\ ($p = 0.0027$)\\ - extraction out of nominal object rated higher\\ than extraction out of nominal subject ($p = 0.0001$)\\ - extraction out of nominal object rated higher\\ than extraction out of \textit{wh}-subject ($p = 0.0001$)\\ - no difference in rating between extraction out of \\ nominal subject and extraction out of \textit{wh}-subject\\ ($p = 0.7504$)\end{tabular} \\ \midrule
		\begin{tabular}[c]{@{}l@{}}\citet{Fukuda.2012},\\ Experiment 3\end{tabular} &
		NP &
		\begin{tabular}[c]{@{}l@{}}Of-complement\\ (with preposition\\ stranding)\end{tabular} &
		\begin{tabular}[c]{@{}l@{}}Acceptability \\ ratings,\\ magnitude\\ estimation\end{tabular} &
		Same as previous &
		\begin{tabular}[c]{@{}l@{}}- marginal interaction extraction type:extraction \\ site ($p = 0.0582$)\\ - extraction out of nominal object rated higher\\ than extraction out of nominal subject ($p = 0.0036$)\\ - extraction out of nominal object rated higher \\ than extraction out of \textit{wh}-subject ($p = 0.0001$)\\ - no difference in rating between extraction out of\\ nominal subject and extraction out of \textit{wh}-subject\\ ($p = 0.2554$)\end{tabular} \\ \midrule
		\begin{tabular}[c]{@{}l@{}}\citet{Polinsky.2013},\\ Experiment 1a\end{tabular} &
		NP &
		\begin{tabular}[c]{@{}l@{}}Of-complement\\ (with preposition \\ stranding)\end{tabular} &
		\begin{tabular}[c]{@{}l@{}}Acceptability \\ ratings,\\ Likert scale\end{tabular} &
		\begin{tabular}[c]{@{}l@{}}Tested embedded \textit{wh}-questions, crossing extraction \\ type (extraction of the subject\slash extraction out of the \\ subject) and verb type (unaccusative\slash unergative\slash \\ transitive).\end{tabular} &
		\begin{tabular}[c]{@{}l@{}}- marginal effect of verb type, only between\\ unaccusatives and unergatives ($p < 0.1$)\\ - no interaction effect\end{tabular} \\ \midrule
		\begin{tabular}[c]{@{}l@{}}\citet{Polinsky.2013},\\ Experiment 1b\end{tabular} &
		NP &
		\begin{tabular}[c]{@{}l@{}}Of-complement\\ (with preposition\\ stranding)\end{tabular} &
		\begin{tabular}[c]{@{}l@{}}Self-paced \\ reading\end{tabular} &
		Same as previous &
		\begin{tabular}[c]{@{}l@{}}- the verb and the NP following the verb are read \\ slower with transitives than the other verb types \\ ($p < 0.05$)\\ - the PP following the verb is read slower with \\ unergatives than with unaccusatives ($p < 0.005$)\\ - the NP/PP following the verb is read slower in the\\ subextraction than in the extraction ($p < 0.0001$)\end{tabular} \\ \midrule
		\citet{Bianchi.2015} &
		NP &
		\begin{tabular}[c]{@{}l@{}}Of-complement\\ (with and without\\ preposition \\ stranding)\end{tabular} &
		\begin{tabular}[c]{@{}l@{}}Acceptability \\ ratings,\\ continuous \\ scale\end{tabular} &
		\begin{tabular}[c]{@{}l@{}}Crossing predicate (stage-level\slash individual-level) and \\ extraction type (with\slash without preposition stranding)\end{tabular} &
		\begin{tabular}[c]{@{}l@{}}- interaction predicate:extraction type ($p = 0.036$)\\ - for s-level predicates, preposition stranding is\\ rated lower ($p = \num{6.127e-13}$)\\ - for i-level predicates, preposition stranding is\\ rated lower ($p = \num{2.719e-05}$)\end{tabular} \\ \midrule
		\begin{tabular}[c]{@{}l@{}}\citet{Sprouse.2016},\\ English Experiment 2\end{tabular} &
		NP &
		\begin{tabular}[c]{@{}l@{}}PP-complement\\ (with preposition\\ stranding)\end{tabular} &
		\begin{tabular}[c]{@{}l@{}}Acceptability \\ ratings,\\ Likert scale\end{tabular} &
		\begin{tabular}[c]{@{}l@{}}Crossing function (subject\slash object) and extraction type\\ (extraction of NP\slash subextraction from NP)\end{tabular} &
		- marginal interaction ($p < 0.062$) \\ \midrule
		\begin{tabular}[c]{@{}l@{}}\citet{Chaves.2019.Frequency},\\ Experiment 4\end{tabular} &
		NP &
		\cellcolor[HTML]{C0C0C0} &
		\begin{tabular}[c]{@{}l@{}}Self-paced \\ reading\end{tabular} &
		\begin{tabular}[c]{@{}l@{}}Compared plausibility (filler is plausible\slash implausible \\ as direct object of the relative clause inside the subject)\end{tabular} &
		\begin{tabular}[c]{@{}l@{}}- reading times increase for implausible fillers on\\ the verb of the relative clause\end{tabular} \\ \midrule
		\begin{tabular}[c]{@{}l@{}}\citet{Chaves.2020.UDC},\\ Experiment 6\end{tabular} &
		NP &
		\begin{tabular}[c]{@{}l@{}}Of-complement\\ (with preposition \\ stranding)\end{tabular} &
		\begin{tabular}[c]{@{}l@{}}Acceptability \\ ratings,\\ Likert scale\end{tabular} &
		\begin{tabular}[c]{@{}l@{}}Compared relevance (the extracted element is relevant /\\ less relevant for the situation described by the sentence).\\ The relevance scores are based on a previous norming\\ experiment.\end{tabular} &
		\begin{tabular}[c]{@{}l@{}}- more relevant fillers rated higher than less relevant\\ fillers ($p = 0.02$)\end{tabular} \\ 
		\lspbottomrule
	\end{longtable}
\end{landscape}



%Compared relevance (the extracted element is relevant /


%- more relevant fillers rated higher than less relevant

