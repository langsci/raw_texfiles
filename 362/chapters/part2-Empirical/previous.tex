\label{ch:previous-exp}

In this chapter, I attempt to give an overview of the previous work on subject islands, reporting the results that I judged most interesting. I only mention experiments published in peer-reviewed publications or in Ph.D.\ theses and leave aside unpublished conference talks and posters. Nevertheless, given the growing interest of experimental linguistics in islands, the following is likely to be incomplete.

The experiments on French presented in this book, some of which have been published, are not included in this chapter, nor are a few experiments on English that we carried out in parallel with the French ones \citep{Abeille.2019.Quasy,Abeille.2020.Cognition}. They will be taken up later in the book.

I deliberately leave aside the work done by several scholars on a phenomenon called ``satiation'' or ``habituation''. \citet{Sprouse.2007.PhD,Sprouse.2007.Acceptability,Sprouse.2009}, \citet{Francom.2011}, \citet{Goodall.2011}, \citet{Chaves.2014,Chaves.2019.Frequency} and \citet{Chaves.2020.UDC} investigated whether the acceptability of subject islands increases after repeated exposure. All these experiments examined extraction out of subjects in \textit{wh}-questions. The subjects are NP subjects as well as non-finite and finite sentential subjects. The results are mixed, but the latest evidence suggests that the acceptability of extraction out of subjects improves after at least eight exposures.

I also omit studies on non-native speakers because the research questions they target are very different from my own. \citet{Kush.2022} provide a current overview of the state of the art.

I start with studies on extractions out of NP subjects, which are the most common ones, before presenting studies on extractions out of sentential subjects.

\section{Interrogatives}
\label{ch:previous-wh}

Unsurprisingly, the most studied language in these studies is English. Maybe more surprisingly, a large majority of the investigations concentrate on extraction in \textit{wh}-questions (mostly direct \textit{wh}-interrogatives, sometimes also indirect ones). An overview of these studies is found in Table~\ref{tab:previous-english-wh}.

{\tiny% Please add the following required packages to your document preamble:
% \usepackage[table,xcdraw]{xcolor}
% If you use beamer only pass "xcolor=table" option, i.e. \documentclass[xcolor=table]{beamer}
% \usepackage{lscape}
% \usepackage{longtable}
% Note: It may be necessary to compile the document several times to get a multi-page table to line up properly
\begin{landscape}
	\begin{longtable}{llllll}
		\caption{Studies on English \textit{wh}-questions}\label{tab:previous-english-wh}\\
		\lsptoprule
		Publication &
		{\begin{tabular}[c]{@{}l@{}}Subject\\ type\end{tabular}} &
		{\begin{tabular}[c]{@{}l@{}}Filler\\ (island condition)\end{tabular}} &
		{Task} &
		{Design} &
		{Results} \\ \midrule
		\endfirsthead
		\midrule
		\endhead
		%
		\begin{tabular}[c]{@{}l@{}}\citet{Phillips.2006}, \\ Experiment 1\end{tabular} &
		NP &
		\begin{tabular}[c]{@{}l@{}}Direct object\\ (of the complement / \\ of the relative clause)\end{tabular} &
		\begin{tabular}[c]{@{}l@{}}Acceptability \\ ratings,\\ Likert scale\end{tabular} &
		\begin{tabular}[c]{@{}l@{}}Tested embedded \textit{wh}-question, crossing extractee type \\ (direct object\slash complement of the subject NP\slash both) \\ with finiteness (non-finite complement of NP\slash finite \\ relative clause of NP).\end{tabular} &
		\begin{tabular}[c]{@{}l@{}}- interaction extractee type:finiteness ($p < 0.0001$)\\ - no effect of finiteness for direct objects\\ - extraction out of the subject rated higher in the\\ non-finite condition ($p < 0.05$)\\ - double gap rated higher in the non-finite condition\\ ($p < 0.0001$)\end{tabular} \\ \midrule
		\begin{tabular}[c]{@{}l@{}}\citet{Phillips.2006}, \\ Experiment 2\end{tabular} &
		NP &
		\cellcolor[HTML]{C0C0C0} &
		\begin{tabular}[c]{@{}l@{}}Self-paced \\ reading\end{tabular} &
		\begin{tabular}[c]{@{}l@{}}Tested embedded \textit{wh}-questions, crossing plausibility\\ (filler is plausible as subject complement\slash implausible \\ as subject complement) and finiteness (non-finite\\ complement of subject\slash finite relative clause of subject).\end{tabular} &
		\begin{tabular}[c]{@{}l@{}}Accuracy on comprehension questions:\\ - no interaction\\ \\ Reading times:\\ - (marginal) interaction plausibility:finiteness\\ ($p < 0.05$ in the participant analysis ; $p = 0.086$ in the \\ item analysis), in that reading times increase only \\ in the implausible + non-finite condition.\end{tabular} \\ \midrule
		\begin{tabular}[c]{@{}l@{}}\citet{Sprouse.2007.PhD}, \\ Experiment in \\ section 3.2\end{tabular} &
		NP &
		\begin{tabular}[c]{@{}l@{}}PP-complement\\ (with preposition\\ stranding)\end{tabular} &
		\begin{tabular}[c]{@{}l@{}}Acceptability \\ ratings,\\ magnitude \\ estimation\end{tabular} &
		\begin{tabular}[c]{@{}l@{}}Crossing function (subject\slash object), extraction type \\ (extraction of NP\slash extraction out of NP) and context\\ (with\slash without supporting context).\end{tabular} &
		\begin{tabular}[c]{@{}l@{}}- interaction function:extraction type ($p < 0.001$; \\ Cohen's d = .521)\\ - no main effect of context\\ - no 3-way interaction\end{tabular} \\ \midrule
		\begin{tabular}[c]{@{}l@{}}\citet{Jurka.2010}, \\ Experiment 8\end{tabular} &
		NP &
		\begin{tabular}[c]{@{}l@{}}Of-complement\\ (with and without\\ preposition \\ stranding)\end{tabular} &
		\begin{tabular}[c]{@{}l@{}}Acceptability \\ ratings,\\ Likert scale\end{tabular} &
		\begin{tabular}[c]{@{}l@{}}Crossing function (subject\slash object), extraction type\\ (no extraction\slash extraction out of the NP) and\\ preposition stranding (with\slash without preposition\\ stranding).\end{tabular} &
		\begin{tabular}[c]{@{}l@{}}- for preposition stranding, interaction function:\\ extraction type, in that extraction out of the\\ subject is rated lowest\\ - without preposition stranding, interaction\\ function: extraction type, in that extraction\\ out of the subject is rated lowest\\ - interaction for preposition stranding larger\\ than without preposition stranding\end{tabular} \\ \midrule
		\begin{tabular}[c]{@{}l@{}}\citet{Jurka.2010}, \\ Experiment 9\end{tabular} &
		NP &
		\begin{tabular}[c]{@{}l@{}}Of-complement\\ (with preposition \\ stranding)\end{tabular} &
		\begin{tabular}[c]{@{}l@{}}Acceptability \\ ratings,\\ Likert scale\end{tabular} &
		\begin{tabular}[c]{@{}l@{}}Crossing extraction type (no extraction\slash extraction out\\ of the subject) and Exceptional Case Marking (simple\\ verb\slash embedding with ECM).\end{tabular} &
		\begin{tabular}[c]{@{}l@{}}- interaction extraction type:ECM, mostly due to the\\ condition ECM with no extraction being much lower than\\ ECM with subextraction\\ - for extraction out of the subject, the simple verb is\\ rated higher than the ECM ($p = 0.031$)\end{tabular} \\ \midrule
		\begin{tabular}[c]{@{}l@{}}\citet{Jurka.2010}, \\ Experiment 10\end{tabular} &
		NP &
		\begin{tabular}[c]{@{}l@{}}Of-complement\\ (with preposition \\ stranding)\end{tabular} &
		\begin{tabular}[c]{@{}l@{}}Acceptability \\ ratings,\\ Likert scale\end{tabular} &
		\begin{tabular}[c]{@{}l@{}}Tested embedded \textit{wh}-questions, crossing extraction type\\ (no extraction\slash extraction out of the subject) and verb\\ type (unergative\slash passive).\end{tabular} &
		\begin{tabular}[c]{@{}l@{}}- interaction extraction type:verb type, in that the\\ extraction out of the subject of the unergative verb is\\ rated lowest, but the effect size is very small\end{tabular} \\ \midrule
		\begin{tabular}[c]{@{}l@{}}\citet{Sprouse.2012},\\ Experiment 1\end{tabular} &
		NP &
		\begin{tabular}[c]{@{}l@{}}PP-complement\\ (with preposition \\ stranding)\end{tabular} &
		\begin{tabular}[c]{@{}l@{}}Acceptability \\ ratings,\\ Likert scale\\ \\ +\\ \\ Serial recall \\ task\end{tabular} &
		\begin{tabular}[c]{@{}l@{}}Crossing function (subject\slash object) and extraction type\\ (extraction of NP\slash extraction out of NP). A measure of\\ working memory is also included.\end{tabular} &
		\begin{tabular}[c]{@{}l@{}}- interaction function:extraction type ($p < 0.0001$)\\ - main effect of serial recall ($p < 0.02$)\end{tabular} \\ \midrule
		\begin{tabular}[c]{@{}l@{}}\citet{Sprouse.2012},\\ Experiment 2\end{tabular} &
		NP &
		\begin{tabular}[c]{@{}l@{}}PP-complement\\ (with preposition\\ stranding)\end{tabular} &
		\begin{tabular}[c]{@{}l@{}}Acceptability \\ ratings,\\ magnitude \\ estimation\\ \\ +\\ \\ Serial recall \\ task and\\ n-back task\end{tabular} &
		\begin{tabular}[c]{@{}l@{}}Crossing function (subject\slash object) and extraction type\\ (extraction of NP\slash extraction out of NP). Two measures \\ of working memory are also included.\end{tabular} &
		\begin{tabular}[c]{@{}l@{}}- interaction function:extraction type ($p < 0.0001$)\\ - no effect of serial recall ($p = 0.7$)\\ - no effect of n-back ($p = 0.66$)\end{tabular} \\ \midrule
		\begin{tabular}[c]{@{}l@{}}\citet{Fukuda.2012},\\ Experiment 1\end{tabular} &
		NP &
		\begin{tabular}[c]{@{}l@{}}Of-complement\\ (with preposition\\ stranding)\end{tabular} &
		\begin{tabular}[c]{@{}l@{}}Acceptability \\ ratings,\\ yes/no\end{tabular} &
		\begin{tabular}[c]{@{}l@{}}Crossing extraction type (no extraction\slash extraction out\\ of NP) and extraction site (nominal subject\slash nominal\\ object\slash \textit{wh}-subject).\end{tabular} &
		\begin{tabular}[c]{@{}l@{}}- interaction extraction type:extraction site\\ ($p = 0.0218$)\\ - extraction out of nominal object rated higher \\ than extraction out of nominal subject ($p = 0.0052$)\\ - extraction out of nominal object rated higher\\ than extraction out of \textit{wh}-subject ($p < 0.001$)\\ - no difference in rating between extraction out of\\ nominal subject and extraction out of \textit{wh}-subject\\ ($p = 0.997$)\end{tabular} \\ \midrule
		\begin{tabular}[c]{@{}l@{}}\citet{Fukuda.2012},\\ Experiment 2\end{tabular} &
		NP &
		\begin{tabular}[c]{@{}l@{}}Of-complement\\ (with preposition\\ stranding)\end{tabular} &
		\begin{tabular}[c]{@{}l@{}}Acceptability \\ ratings,\\ Likert scale\end{tabular} &
		Same as previous &
		\begin{tabular}[c]{@{}l@{}}- interaction extraction type:extraction site\\ ($p = 0.0027$)\\ - extraction out of nominal object rated higher\\ than extraction out of nominal subject ($p = 0.0001$)\\ - extraction out of nominal object rated higher\\ than extraction out of \textit{wh}-subject ($p = 0.0001$)\\ - no difference in rating between extraction out of \\ nominal subject and extraction out of \textit{wh}-subject\\ ($p = 0.7504$)\end{tabular} \\ \midrule
		\begin{tabular}[c]{@{}l@{}}\citet{Fukuda.2012},\\ Experiment 3\end{tabular} &
		NP &
		\begin{tabular}[c]{@{}l@{}}Of-complement\\ (with preposition\\ stranding)\end{tabular} &
		\begin{tabular}[c]{@{}l@{}}Acceptability \\ ratings,\\ magnitude\\ estimation\end{tabular} &
		Same as previous &
		\begin{tabular}[c]{@{}l@{}}- marginal interaction extraction type:extraction \\ site ($p = 0.0582$)\\ - extraction out of nominal object rated higher\\ than extraction out of nominal subject ($p = 0.0036$)\\ - extraction out of nominal object rated higher \\ than extraction out of \textit{wh}-subject ($p = 0.0001$)\\ - no difference in rating between extraction out of\\ nominal subject and extraction out of \textit{wh}-subject\\ ($p = 0.2554$)\end{tabular} \\ \midrule
		\begin{tabular}[c]{@{}l@{}}\citet{Polinsky.2013},\\ Experiment 1a\end{tabular} &
		NP &
		\begin{tabular}[c]{@{}l@{}}Of-complement\\ (with preposition \\ stranding)\end{tabular} &
		\begin{tabular}[c]{@{}l@{}}Acceptability \\ ratings,\\ Likert scale\end{tabular} &
		\begin{tabular}[c]{@{}l@{}}Tested embedded \textit{wh}-questions, crossing extraction \\ type (extraction of the subject\slash extraction out of the \\ subject) and verb type (unaccusative\slash unergative\slash \\ transitive).\end{tabular} &
		\begin{tabular}[c]{@{}l@{}}- marginal effect of verb type, only between\\ unaccusatives and unergatives ($p < 0.1$)\\ - no interaction effect\end{tabular} \\ \midrule
		\begin{tabular}[c]{@{}l@{}}\citet{Polinsky.2013},\\ Experiment 1b\end{tabular} &
		NP &
		\begin{tabular}[c]{@{}l@{}}Of-complement\\ (with preposition\\ stranding)\end{tabular} &
		\begin{tabular}[c]{@{}l@{}}Self-paced \\ reading\end{tabular} &
		Same as previous &
		\begin{tabular}[c]{@{}l@{}}- the verb and the NP following the verb are read \\ slower with transitives than the other verb types \\ ($p < 0.05$)\\ - the PP following the verb is read slower with \\ unergatives than with unaccusatives ($p < 0.005$)\\ - the NP/PP following the verb is read slower in the\\ subextraction than in the extraction ($p < 0.0001$)\end{tabular} \\ \midrule
		\citet{Bianchi.2015} &
		NP &
		\begin{tabular}[c]{@{}l@{}}Of-complement\\ (with and without\\ preposition \\ stranding)\end{tabular} &
		\begin{tabular}[c]{@{}l@{}}Acceptability \\ ratings,\\ continuous \\ scale\end{tabular} &
		\begin{tabular}[c]{@{}l@{}}Crossing predicate (stage-level\slash individual-level) and \\ extraction type (with\slash without preposition stranding)\end{tabular} &
		\begin{tabular}[c]{@{}l@{}}- interaction predicate:extraction type ($p = 0.036$)\\ - for s-level predicates, preposition stranding is\\ rated lower ($p = \num{6.127e-13}$)\\ - for i-level predicates, preposition stranding is\\ rated lower ($p = \num{2.719e-05}$)\end{tabular} \\ \midrule
		\begin{tabular}[c]{@{}l@{}}\citet{Sprouse.2016},\\ English Experiment 2\end{tabular} &
		NP &
		\begin{tabular}[c]{@{}l@{}}PP-complement\\ (with preposition\\ stranding)\end{tabular} &
		\begin{tabular}[c]{@{}l@{}}Acceptability \\ ratings,\\ Likert scale\end{tabular} &
		\begin{tabular}[c]{@{}l@{}}Crossing function (subject\slash object) and extraction type\\ (extraction of NP\slash subextraction from NP)\end{tabular} &
		- marginal interaction ($p < 0.062$) \\ \midrule
		\begin{tabular}[c]{@{}l@{}}\citet{Chaves.2019.Frequency},\\ Experiment 4\end{tabular} &
		NP &
		\cellcolor[HTML]{C0C0C0} &
		\begin{tabular}[c]{@{}l@{}}Self-paced \\ reading\end{tabular} &
		\begin{tabular}[c]{@{}l@{}}Compared plausibility (filler is plausible\slash implausible \\ as direct object of the relative clause inside the subject)\end{tabular} &
		\begin{tabular}[c]{@{}l@{}}- reading times increase for implausible fillers on\\ the verb of the relative clause\end{tabular} \\ \midrule
		\begin{tabular}[c]{@{}l@{}}\citet{Chaves.2020.UDC},\\ Experiment 6\end{tabular} &
		NP &
		\begin{tabular}[c]{@{}l@{}}Of-complement\\ (with preposition \\ stranding)\end{tabular} &
		\begin{tabular}[c]{@{}l@{}}Acceptability \\ ratings,\\ Likert scale\end{tabular} &
		\begin{tabular}[c]{@{}l@{}}Compared relevance (the extracted element is relevant /\\ less relevant for the situation described by the sentence).\\ The relevance scores are based on a previous norming\\ experiment.\end{tabular} &
		\begin{tabular}[c]{@{}l@{}}- more relevant fillers rated higher than less relevant\\ fillers ($p = 0.02$)\end{tabular} \\ 
		\lspbottomrule
	\end{longtable}
\end{landscape}



%Compared relevance (the extracted element is relevant /


%- more relevant fillers rated higher than less relevant

}
%\caption{Studies on English wh-questions}
%\label{tab:previous-english-wh}

Most studies on English \textit{wh}-questions employ acceptability rating tasks with slightly different methodologies, whose results tend to be roughly similar \citep{Fukuda.2012}. The ``subject island effects'' are attested and seem robust \citep{Sprouse.2007.PhD,Jurka.2010,Sprouse.2012,Fukuda.2012,Sprouse.2016}. All the studies which extracted a PP-complement out of the subject used preposition stranding, except \citet[Experiment 8]{Jurka.2010} and \citet{Bianchi.2015} which test cases both with and without preposition stranding: preposition stranding seems to lower the acceptability of the extraction out of the subject. Some studies compared different verb types \citep{Jurka.2010,Polinsky.2013,Bianchi.2015}, and their results indeed reveal some differences between verbs. But the studies do not systematically measure whether island effects disappear based on those factors. The presence of a quantifier might ameliorate extraction out of the subject, but the effect sizes are very small, thus the results are difficult to interpret \citep[Experiment 11]{Jurka.2010}. Supporting context does not seem to improve extraction out of subjects \citep{Sprouse.2007.PhD} but relevance does \citet{Chaves.2020.UDC}. The results on working memory are mixed: working memory may have an impact on the acceptability ratings of extractions out of subjects, but it does not explain them \citep{Sprouse.2012}. Two online studies measures investigated  whether or not readers postulate a gap in the subject \citep{Phillips.2006,Chaves.2019.Frequency}. Their results go in the same direction, though the authors draw opposite conclusions from them.

Table~\ref{tab:previous-languages-wh} presents studies on \textit{wh}-questions carried out in other languages (only examining direct interrogatives).

{\tiny% Please add the following required packages to your document preamble:
% \usepackage{lscape}
% \usepackage{longtable}
% Note: It may be necessary to compile the document several times to get a multi-page table to line up properly
\begin{landscape}
	\begin{longtable}{lllllll}
		\caption{Studies on \textit{wh}-questions in other languages}\label{tab:previous-languages-wh}\\
		\lsptoprule
		Publication &
		{Language} &
		{\begin{tabular}[c]{@{}l@{}}Subject\\ type\end{tabular}} &
		{\begin{tabular}[c]{@{}l@{}}Filler\\ (island condition)\end{tabular}} &
		{Task} &
		{Design} &
		{Results} \\ \midrule
		\endfirsthead
		\midrule
		\endhead
		%
		\begin{tabular}[c]{@{}l@{}}\citet{Jurka.2010},\\ Experiment 1\end{tabular} &
		German &
		NP &
		\begin{tabular}[c]{@{}l@{}}Specifier\\ (\textit{was-für} split)\end{tabular} &
		\begin{tabular}[c]{@{}l@{}}Acceptability\\ ratings,\\ Likert scale\end{tabular} &
		\begin{tabular}[c]{@{}l@{}}Crossing function (subject / object), position\\ (NP moved before adverb / NP in situ after adverb)\\ and extraction type (extraction of NP / extraction\\ out of NP).\end{tabular} &
		\begin{tabular}[c]{@{}l@{}}- for the subextraction, interaction function :\\ position ($p < 0.001$), in that extraction out of\\ the ``moved'' subject is rated worse\end{tabular} \\ \midrule
		\begin{tabular}[c]{@{}l@{}}\citet{Jurka.2010},\\ Experiment 2\end{tabular} &
		German &
		NP &
		\begin{tabular}[c]{@{}l@{}}Specifier\\ (\textit{was-für} split)\end{tabular} &
		\begin{tabular}[c]{@{}l@{}}Acceptability\\ ratings,\\ Likert scale\end{tabular} &
		\begin{tabular}[c]{@{}l@{}}Tested passive verbs, crossing function (subject /\\ dative object), markedness (unmarked with dative\\ object before subject / marked with dative object\\ after subject) and extraction (no extraction /\\ extraction out of NP).\end{tabular} &
		\begin{tabular}[c]{@{}l@{}}- extraction out of the subject rated higher than\\ extraction out of the dative object\\ - for extracted conditions, no interaction\\ function : markedness\end{tabular} \\ \midrule		
		\begin{tabular}[c]{@{}l@{}}\citet{Jurka.2010},\\ Experiment 3A\end{tabular} &
		German &
		NP &
		\begin{tabular}[c]{@{}l@{}}Specifier\\ (\textit{was-für} split)\end{tabular} &
		\begin{tabular}[c]{@{}l@{}}Acceptability\\ ratings,\\ Likert scale\end{tabular} &
		\begin{tabular}[c]{@{}l@{}}Tested ``in situ'' subjects (after the adverb),\\ crossing argument type (subject of unaccusative /\\ passive / transitive / unergative / object of\\ transitive) and extraction type (no extraction /\\ extraction out of the NP).\end{tabular} &
		\begin{tabular}[c]{@{}l@{}}- for transitives, extraction out of object rated\\ higher than extraction out subjects ($p < 0.001$)\\ - for transitives (two-by-two comparison with all\\ other verb types), significant interactions\\ argument : extraction type, in that extraction\\ out of the subject is worse than the other\\ conditions\\ - for other extractions out of the subject, no\\ other interaction argument : extraction type\end{tabular} \\ \midrule
		\begin{tabular}[c]{@{}l@{}}\citet{Jurka.2010},\\ Experiment 3B\end{tabular} &
		German &
		NP &
		\begin{tabular}[c]{@{}l@{}}Specifier\\ (\textit{was-für} split)\end{tabular} &
		\begin{tabular}[c]{@{}l@{}}Acceptability\\ ratings,\\ Likert scale\end{tabular} &
		\begin{tabular}[c]{@{}l@{}}Tested ``in situ'' subjects (after the adverb),\\ crossing verb type (unaccusative / unergative)\\ and extraction type (extraction of the subject /\\ extraction out of the subject).\end{tabular} &
		\begin{tabular}[c]{@{}l@{}}- interaction verb type : extraction type ($p < 0.001$)\end{tabular} \\ \midrule
		\begin{tabular}[c]{@{}l@{}}\citet{Jurka.2010},\\ Experiment 4\end{tabular} &
		German &
		NP &
		\begin{tabular}[c]{@{}l@{}}Specifier\\ (\textit{was-für} split)\end{tabular} &
		\begin{tabular}[c]{@{}l@{}}Acceptability\\ ratings,\\ Likert scale\end{tabular} &
		\begin{tabular}[c]{@{}l@{}}Tested embedded \textit{wh}-questions, crossing function\\ (subject / object), extraction type (extraction\\ of the NP / out of the NP) and scrambling (object\\ scrambled and before the adverb / not scrambled\\ and after the adverb).\end{tabular} &
		\begin{tabular}[c]{@{}l@{}}- extraction of subject rated higher than\\ extraction of object ($p = 0.0014$)\\ - extraction out of subject with object\\ scrambled rated lower than extraction out of\\ object ($p < 0.004$)\\ - extraction out of subject without object\\ scrambled rated lower than extraction out of\\ object ($p = 0.041$)\\ - for subextractions, no interaction function :\\ scrambling\end{tabular} \\ \midrule
		\begin{tabular}[c]{@{}l@{}}\citet{Jurka.2010},\\ Experiment 13a\end{tabular} &
		Serbian &
		NP &
		Specifier &
		\begin{tabular}[c]{@{}l@{}}Acceptability\\ ratings,\\ Likert scale\end{tabular} &
		\begin{tabular}[c]{@{}l@{}}Crossing function (subject / object) and extraction\\ type (no extraction / extraction out of the NP).\end{tabular} &
		\begin{tabular}[c]{@{}l@{}}- no interaction function : extraction type\end{tabular} \\ \midrule
		\begin{tabular}[c]{@{}l@{}}\citet{Jurka.2010},\\ Experiment 13b\end{tabular} &
		Serbian &
		NP &
		PP-complement &
		\begin{tabular}[c]{@{}l@{}}Acceptability\\ ratings,\\ Likert scale\end{tabular} &
		\begin{tabular}[c]{@{}l@{}}Crossing function (subject / object) and\\ extraction type (extraction of the NP / \\ extraction out of the NP).\end{tabular} &
		\begin{tabular}[c]{@{}l@{}}- interaction function : extraction type, in that \\extraction out of the subject is rated lowest\end{tabular} \\ \midrule
		\begin{tabular}[c]{@{}l@{}}\citet{Jurka.2011},\\ Experiment in 2.1\end{tabular} &
		German &
		NP &
		\begin{tabular}[c]{@{}l@{}}Specifier\\ (\textit{was-für} split)\end{tabular} &
		\begin{tabular}[c]{@{}l@{}}Acceptability\\ ratings,\\ Likert scale\end{tabular} &
		\begin{tabular}[c]{@{}l@{}}Crossing function (subject / object), extraction \\ type (extraction of NP / extraction out of NP) \\ and freezing (scrambled NP / NP in situ).\end{tabular} &
		\begin{tabular}[c]{@{}l@{}}- interaction function : freezing ($p < 0.001$), the \\ condition with extraction out of the scrambled \\ NP being rated lowest\end{tabular} \\ \midrule
		\begin{tabular}[c]{@{}l@{}}\citet{Polinsky.2013},\\ Experiment 2\end{tabular} &
		Russian &
		NP &
		Specifier &
		\begin{tabular}[c]{@{}l@{}}Acceptability\\ ratings,\\ Likert scale\end{tabular} &
		\begin{tabular}[c]{@{}l@{}}Crossing extraction type (extraction of NP /\\ extraction out of the NP), extraction site \\ (subject of unaccusative / unergative / \\ transitive / object of transitive) and subject \\ position (preverbal / postverbal).\end{tabular} &
		\begin{tabular}[c]{@{}l@{}}- sentences with transitive verbs rated lower \\ than other verb types ($p < 0.005$ for postverbal; \\ $p < 0.005$ for preverbal)\\ - sentences with an unergative verb rated lower \\ than other verb types ($p < 0.05$ for postverbal; \\ $p < 0.01$ for preverbal)\\ - for transitive verbs, extraction out of the \\ subject rated lower than extraction out of the \\ object ($p < 0.001$ for postverbal; $p < 0.05$ for \\ preverbal)\end{tabular} \\ \midrule
		\begin{tabular}[c]{@{}l@{}}\citet{Bianchi.2014},\\ Experiment 1\end{tabular} &
		Italian &
		NP &
		Di-complement &
		\begin{tabular}[c]{@{}l@{}}Acceptability\\ ratings,\\ continuous\\ scale\end{tabular} &
		\begin{tabular}[c]{@{}l@{}}Tested state verbs and indefinite NPs, crossing\\ predicate (identity-denoting / stage-denoting)\\ and subject position (preverbal / postverbal).\end{tabular} &
		\begin{tabular}[c]{@{}l@{}}- interaction predicate : subject position\\ ($p < 0.003$)\\ - for i-level, no effect of subject position\\ ($p = 0.6$)\\ - for s-level, postverbal is rated higher than\\ preverbal ($p = 0.003$)\end{tabular} \\ \midrule
		\begin{tabular}[c]{@{}l@{}}\citet{Bianchi.2014},\\ Experiment 2\end{tabular} &
		Italian &
		NP &
		Di-complement &
		\begin{tabular}[c]{@{}l@{}}Acceptability\\ ratings,\\ continuous\\ scale\end{tabular} &
		\begin{tabular}[c]{@{}l@{}}Tested extraction out of definite postverbal \\ subject, comparing verb types (unergative / \\ unecacusative).\end{tabular} &
		- no difference between verb types ($p = 0.932$) \\ \midrule
		\begin{tabular}[c]{@{}l@{}}\citet{Sprouse.2016},\\ Italian Experiment 2\end{tabular} &
		Italian &
		NP &
		Di-complement &
		\begin{tabular}[c]{@{}l@{}}Acceptability\\ ratings,\\ Likert scale\end{tabular} &
		\begin{tabular}[c]{@{}l@{}}Crossing function (subject / object) and\\ extraction type (extraction of NP / extraction\\ out of NP).\end{tabular} &
		\begin{tabular}[c]{@{}l@{}}- interaction function : extraction type \\ ($p < 0.001$)\end{tabular} \\ \midrule
		\begin{tabular}[c]{@{}l@{}}\citet{Greco.2017},\\ Experiment 1\end{tabular} &
		Italian &
		NP &
		Di-complement &
		\begin{tabular}[c]{@{}l@{}}Acceptability\\ ratings,\\ Likert scale\end{tabular} &
		\begin{tabular}[c]{@{}l@{}}Compared extractions (extraction out of the\\ subject / object / extraction of adjunct).\end{tabular} &
		\begin{tabular}[c]{@{}l@{}}- extractions out of objects rated higher than \\ extractions out of subjects ($p < 0.0001$)\\ - extractions of adverbs rated higher than \\ extractions out of objects ($p < 0.05$)\end{tabular} \\ \midrule
		\begin{tabular}[c]{@{}l@{}}\citet{Greco.2017},\\ Experiment 2\end{tabular} &
		Italian &
		NP &
		Di-complement &
		\begin{tabular}[c]{@{}l@{}}Acceptability\\ ratings,\\ Likert scale\end{tabular} &
		\begin{tabular}[c]{@{}l@{}}Crossing verb type (passive / unaccusative / \\ unergative / transitive), extraction types\\ (extraction out of the subject / extraction of \\ adjunct) and subject position (preverbal / \\ postverbal).\end{tabular} &
		\begin{tabular}[c]{@{}l@{}}- for transitive verbs, preverbal subjects rated\\ higher than postverbal ($p = 0.0539$), but no\\ effect of extraction type\\ - for passive verbs, preverbal subject rated\\ lower than postverbal ($p = 0.0176$), but no\\ effect of extraction type\\ - for extractions out of the subject, unergatives\\ rated lower than other verb types\end{tabular} \\ \midrule
		\begin{tabular}[c]{@{}l@{}}\citet{Kush.2018},\\ Experiment 1\end{tabular} &
		Norwegian &
		NP &
		\begin{tabular}[c]{@{}l@{}}PP-complement\\ (with preposition\\ stranding)\end{tabular} &
		\begin{tabular}[c]{@{}l@{}}Acceptability\\ ratings,\\ Likert scale\end{tabular} &
		\begin{tabular}[c]{@{}l@{}}Tested bare fillers (similar to \textit{who}/\textit{what}),\\ crossing extraction site (extraction in the main\\ clause / out of embedded clause) and complexity\\ (subject of the embedded clause simple /\\ complex).\end{tabular} &
		\begin{tabular}[c]{@{}l@{}}- interaction extraction site : complexity\\ ($p < 0.001$), in that extraction out of the complex\\ subject is rated lowest.\end{tabular} \\ \midrule
		\begin{tabular}[c]{@{}l@{}}\citet{Kush.2018},\\ Experiment 2\end{tabular} &
		Norwegian &
		NP &
		\begin{tabular}[c]{@{}l@{}}PP-complement\\ (with preposition\\ stranding)\end{tabular} &
		\begin{tabular}[c]{@{}l@{}}Acceptability\\ ratings,\\ Likert scale\end{tabular} &
		Same as previous. &
		\begin{tabular}[c]{@{}l@{}}- interaction extraction site : complexity\\ ($p < 0.001$), in that extraction out of the complex\\ subject is rated lowest.\end{tabular} \\ \midrule
		\begin{tabular}[c]{@{}l@{}}\citet{Kush.2018},\\ Experiment 3\end{tabular} &
		Norwegian &
		NP &
		\begin{tabular}[c]{@{}l@{}}PP-complement\\ (with preposition\\ stranding)\end{tabular} &
		\begin{tabular}[c]{@{}l@{}}Acceptability\\ ratings,\\ Likert scale\end{tabular} &
		\begin{tabular}[c]{@{}l@{}}Tested complex filler (similar to \textit{which NP}),\\ otherwise same as previous.\end{tabular} &
		\begin{tabular}[c]{@{}l@{}}- interaction extraction site : complexity\\ ($p < 0.001$), in that extraction out of the complex\\ subject is rated lowest\end{tabular} \\ \midrule
		\citet{Paneda.2020} &
		Spanish &
		NP &
		De-complement &
		\begin{tabular}[c]{@{}l@{}}Speeded\\ acceptability\\ ratings,\\ yes/no\end{tabular} &
		\begin{tabular}[c]{@{}l@{}}Crossing extraction site (extraction in the main\\ clause / out of embedded clause) and complexity\\ (subject of the embedded clause simple /\\ complex).\end{tabular} &
		\begin{tabular}[c]{@{}l@{}}- interaction extraction site : complexity, so \\ that extraction out of the complex subject is\\ rated lowest\end{tabular} \\ \midrule
		\begin{tabular}[c]{@{}l@{}}\citet{Kobzeva.2022},\\ \textit{wh}-question\end{tabular} &
		Norwegian &
		NP &
		\begin{tabular}[c]{@{}l@{}}PP-complement\\ (with preposition\\ stranding)\end{tabular} &
		\begin{tabular}[c]{@{}l@{}}Acceptability\\ ratings,\\ Likert scale\end{tabular} &
		\begin{tabular}[c]{@{}l@{}}Same as previous. Another construction is also\\ tested (see Table \ref{tab:previous-other-cx} below).\end{tabular} &
		\begin{tabular}[c]{@{}l@{}}- interaction extraction site * complexity\\ ($p < 0.0001$), in that extraction out of the\\ complex subject is rated lowest\end{tabular} \\ \lspbottomrule
	\end{longtable}
\end{landscape}

% Tested state verbs and indefinite NPs, crossing

% - sentences with a transitive verbs rated lower
}
%\caption{Studies on wh-questions in other languages}
%\label{tab:previous-languages-wh}


Some studies employed a design that makes it possible to detect  island effects (\citealt[Experiment 13b]{Jurka.2010}, \citealt{Jurka.2011,Sprouse.2016} \citealt[Experiment 2]{Greco.2017}, \citealt{Kush.2018,Paneda.2020,Kobzeva.2022}). In those studies, there is almost systematically a significant interaction effect pointing to a potential ``subject island'' (unfortunately, even though \citet[Experiment 1 \& 2]{Jurka.2010} and \citet{Polinsky.2013} used such a design, they do not report the results for the interaction). However, all except one \citep{Greco.2017} use extraction of the whole subject as a baseline to test subextraction. This is not ideal, given that extraction of the subject displays processing advantages. One study that stands out is \citet[Experiment 13a]{Jurka.2010} on Serbian, which does not show a superadditive effect. Some experiments explore different verb types \citep{Jurka.2010,Polinsky.2013,Bianchi.2014} and their results differ from the English ones: in German, Russian and Italian, different verb types seem to behave similarly as far as extraction out of the subject is concerned. The position of the subject affects the acceptability of the subject island condition \citep{Jurka.2010,Jurka.2011,Polinsky.2013,Bianchi.2014}. However, the findings could be due to general preferences that have nothing to do with subextraction, since this is not controlled for with a baseline (or it is not reported).

Several experiments test extraction of the specifier  in languages that allow such extraction \citep{Jurka.2010,Jurka.2011,Polinsky.2013}. The possibility of specifiers behaving differently than complements in subextraction is only explored in \citet[Experiment 13a\&b]{Jurka.2010}. Indeed, \citet[Experiment 13a]{Jurka.2010} does not find superadditive effects when the specifier is extracted, while \citet[Experiment 13b]{Jurka.2010} observes superadditive effects in the extraction of the complement. Notice, however, that the two experiments used different baselines, namely ``no extraction'' (Experiment 13a) and ``extraction of the NP'' (Experiment 13b). Unfortunately there is no test reported for a three-way interaction.

\citeauthor{Sprouse.2007.PhD} and colleagues also conducted a series of experiments on in-situ questions. An overview is provided in Table~\ref{tab:previous-in-situ}.

Since in-situ questions in English are uncommon and restricted to specific contexts, \citet{Sprouse.2007.PhD} and \citet{Sprouse.2011} tested them by using double interrogatives. In these constructions, when a first \textit{wh}-phrase is extracted, a second \textit{wh}-phrase can be in situ. The results are mixed: depending on the baseline, a superadditive effect may be observed. In Japanese, interrogatives are in situ by default, so \citet{Sprouse.2011} was able to test  the \textit{wh}-phrase directly inside the subject. They found no interaction effects, which goes against an analysis of in-situ questions as covert movement. I will develop the issue of in-situ questions in more detail in Section~\ref{ch:exp11}.

{\tiny\input{chapters/part2-Empirical/Previous-experiments/in-situ.tex}}
%\caption{Studies on in situ questions}
%\label{tab:previous-in-situ}

\section{Other constructions}

Few studies have tested non-interrogative constructions. Table~\ref{tab:previous-other-cx} presents five such studies.

The distinction between focalizing and non-focalizing extractions is very important for the Focus-Background Conflict constraint (see Section~\ref{ch:discourse}). In the literature, apart from \textit{wh}-questions, the other constructions tested are topicalization \citep{Kush.2019} and relative clauses \citep{Sprouse.2016,Kobzeva.2022}, both of them probably non-focalizing constructions. There is a caveat with \citegen{Kobzeva.2022} study, which examine Norwegian: as the authors note, the Norwegian demonstrative relative clause could be read as a cleft. It is thus difficult to know whether the construction tested by \citeauthor{Kobzeva.2022} is a focalizing construction or not. 

All the studies observe superadditive effects, except the Italian experiment \citep{Sprouse.2016}. As previously discussed, the baseline in these studies might cause a problem (if there is a subject advantage in the baseline, as in extraction of the subject in relative clauses, the interaction might be caused by this subject preference rather than by the subextraction). The Italian experiment in \citep{Sprouse.2016} is also the only one without preposition stranding. 

Only \citet{Kobzeva.2022} looked for a three-way interaction to compare different constructions: they found that the superadditive effect in relative clauses (or clefts) is smaller than in \textit{wh}-questions.

\section{Sentential subjects}

Even though the subject island constraint was first formulated as the sentential subject island constraint \citep{Ross.1967}, there are relatively few studies on extraction out of sentential subjects. These studies are presented in Table~\ref{tab:previous-sentential-subject}.

The studies examined finite and non-finite sentential subjects, sometimes in the same experiment. In English, extractions out of sentential subjects are one of the worse island violation (in raw ratings), and worse for non-finite subjects than finite ones \citep{Sprouse.2007.PhD}. In German, \citet{Jurka.2010} found superadditive effects, but again the baseline condition showed an advantage for the subject condition (meaning that the baseline increases the interaction). Extraposition does not seem to play a role. Finally, it appears that Japanese does not have a ``subject island effect'' \citep{Jurka.2010,Jurka.2011,Fukuda.2014}, in line with what was reported in the literature.

{\tiny% Please add the following required packages to your document preamble:
% \usepackage{lscape}
% \usepackage{longtable}
% Note: It may be necessary to compile the document several times to get a multi-page table to line up properly
\begin{landscape}
	\begin{longtable}{llllllll}
	\caption{Studies on other constructions}\label{tab:previous-other-cx}\\
		\lsptoprule
		Publication &
		{Language} &
		{Construction} &
		{\begin{tabular}[c]{@{}l@{}}Subject\\ type\end{tabular}} &
		{\begin{tabular}[c]{@{}l@{}}Filler\\ (island condition)\end{tabular}} &
		{Task} &
		{Design} &
		{Results} \\ \midrule
		\endfirsthead
		\midrule
		\endhead
		%
		\begin{tabular}[c]{@{}l@{}}\citet{Sprouse.2016},\\ English Experiment 1\end{tabular} &
		English &
		\begin{tabular}[c]{@{}l@{}}Relative\\ clause\end{tabular} &
		NP &
		\begin{tabular}[c]{@{}l@{}}PP-complement\\ (with preposition\\ stranding)\end{tabular} &
		\begin{tabular}[c]{@{}l@{}}Acceptability\\ ratings,\\ Likert scale\end{tabular} &
		\begin{tabular}[c]{@{}l@{}}Crossing function (subject\slash object) and \\extraction type (extraction of NP\slash extraction\\ out of NP).\end{tabular} &
		\begin{tabular}[c]{@{}l@{}}- interaction function:\\ extraction type ($p < 0.001$)\end{tabular} \\ \midrule
		\begin{tabular}[c]{@{}l@{}}\citet{Sprouse.2016},\\ Italian Experiment 1\end{tabular} &
		Italian &
		\begin{tabular}[c]{@{}l@{}}Relative\\ clause\end{tabular} &
		NP &
		Di-complement &
		\begin{tabular}[c]{@{}l@{}}Acceptability\\ ratings,\\ Likert scale\end{tabular} &
		\begin{tabular}[c]{@{}l@{}}Crossing function (subject\slash object) and \\extraction type (extraction of NP\slash extraction\\ out of NP).\end{tabular} &
		\begin{tabular}[c]{@{}l@{}}- no interaction function:\\ extraction type ($p < 0.84$)\end{tabular} \\ \midrule
		\begin{tabular}[c]{@{}l@{}}\citet{Kush.2019},\\ Experiment 1\end{tabular} &
		Norwegian &
		Topicalization &
		NP &
		\begin{tabular}[c]{@{}l@{}}PP-complement\\ (with preposition\\ stranding)\end{tabular} &
		\begin{tabular}[c]{@{}l@{}}Acceptability\\ ratings,\\ Likert scale\end{tabular} &
		\begin{tabular}[c]{@{}l@{}}Tested sentences without context, crossing extraction\\ site (extraction in the main clause\slash out of embedded\\ clause) and complexity (suject of the embedded clause\\ simple\slash complex).\end{tabular} &
		\begin{tabular}[c]{@{}l@{}}- interaction extraction site:\\ complexity ($p < 0.001$)\end{tabular} \\ \midrule
		\begin{tabular}[c]{@{}l@{}}\citet{Kush.2019},\\ Experiment 2\end{tabular} &
		Norwegian &
		Topicalization &
		NP &
		\begin{tabular}[c]{@{}l@{}}PP-complement\\ (with preposition\\ stranding)\end{tabular} &
		\begin{tabular}[c]{@{}l@{}}Acceptability\\ ratings,\\ Likert scale\end{tabular} &
		\begin{tabular}[c]{@{}l@{}}Tested sentences with supporting context, crossing\\ extraction site (extraction in the main clause /\\ out of embedded clause) and complexity (suject of\\ the embedded clause simple\slash complex).\end{tabular} &
		\begin{tabular}[c]{@{}l@{}}- interaction extraction site: \\complexity ($p < 0.001$)\end{tabular} \\ \midrule
		\begin{tabular}[c]{@{}l@{}}\citet{Kobzeva.2022},\\ relative clause\end{tabular} &
		Norwegian &
		\begin{tabular}[c]{@{}l@{}}Demonstrative\\ relative clause\\ (or cleft)\end{tabular} &
		NP &
		\begin{tabular}[c]{@{}l@{}}PP-complement\\ (with preposition\\ stranding)\end{tabular} &
		\begin{tabular}[c]{@{}l@{}}Acceptability\\ ratings,\\ Likert scale\end{tabular} &
		\begin{tabular}[c]{@{}l@{}}Crossing extraction site (extraction in the main\\ clause\slash out of embedded clause) and complexity\\ (suject of the embedded clause simple\slash complex).\\ \textit{Wh}-questions are also tested.\end{tabular} &
		\begin{tabular}[c]{@{}l@{}}- interaction extraction site \\: complexity ($p = 0.0090$)\\- three-way interaction \\($p < 0.0003$), in that the\\ interaction effect is larger\\ for \textit{wh}-questions than for\\ relative clauses.\end{tabular} \\ 
		\lspbottomrule
	\end{longtable}
\end{landscape}

% (p = 0.0090) is the result reported in Kobzeva et al 2022


% \textless 

%Tested embedded wh-question, crossing extractee type




%- double gap rated higher in the non-finite condition
%- double gap rated higher in






}
%\caption{Studies on other constructions}
%\label{tab:previous-other-cx}

{\tiny% Please add the following required packages to your document preamble:
% \usepackage{lscape}
% \usepackage{longtable}
% Note: It may be necessary to compile the document several times to get a multi-page table to line up properly
\begin{landscape}
	\begin{longtable}{llllllll}
		\caption{Studies on sentential subjects}\label{tab:previous-sentential-subject}\\
		\lsptoprule
		Publication &
		{Language} &
		{Construction} &
		{\begin{tabular}[c]{@{}l@{}}Subject\\ type\end{tabular}} &
		{\begin{tabular}[c]{@{}l@{}}Filler\\ (island condition)\end{tabular}} &
		{Task} &
		{Design} &
		{Results} \\ \midrule
		\endfirsthead
		%
		\midrule
		\endhead
		%
		\begin{tabular}[c]{@{}l@{}}\citet{Sprouse.2007.PhD},\\ Experiment in Section 2.2\end{tabular} &
		English &
		\begin{tabular}[c]{@{}l@{}}\textit{Wh}-question\end{tabular} &
		\begin{tabular}[c]{@{}l@{}}Non-finite\\ sentential\\ subject\end{tabular} &
		\begin{tabular}[c]{@{}l@{}}Direct\\ object\end{tabular} &
		\begin{tabular}[c]{@{}l@{}}Acceptability\\ ratings,\\ magnitude\\ estimation\end{tabular} &
		\begin{tabular}[c]{@{}l@{}}Tested extraction out of the subject, \\ compared with other island violations.\end{tabular} &
		\begin{tabular}[c]{@{}l@{}}- rated worse than any other island\\ violation\end{tabular} \\ \midrule
		
		\begin{tabular}[c]{@{}l@{}}\citet{Sprouse.2007.PhD},\\ Experiment in Section 2.2\end{tabular} &
		English &
		\begin{tabular}[c]{@{}l@{}}\textit{Wh}-question\end{tabular} &
		\begin{tabular}[c]{@{}l@{}}Finite\\ sentential\\ subject\end{tabular} &
		\begin{tabular}[c]{@{}l@{}}Direct\\ object\end{tabular} &
		\begin{tabular}[c]{@{}l@{}}Acceptability\\ ratings,\\ magnitude\\ estimation\end{tabular} &
		\begin{tabular}[c]{@{}l@{}}Tested extraction out of the subject, \\ compared with other island violations.\end{tabular} &
		\begin{tabular}[c]{@{}l@{}}- rated higher than the extraction out of\\ non-finite sentential subjects,\\ sometimes rated higher than extraction\\ out of adjuncts (``adjunct island'')\end{tabular} \\ \midrule
		
		\begin{tabular}[c]{@{}l@{}}\citet{Jurka.2010},\\ Experiment 5\end{tabular} &
		German &
		\begin{tabular}[c]{@{}l@{}}\textit{Wh}-question\end{tabular} &
		\begin{tabular}[c]{@{}l@{}}Non-finite\\ sentential\\ subject\end{tabular} &
		\begin{tabular}[c]{@{}l@{}}Direct\\ object\end{tabular} &
		\begin{tabular}[c]{@{}l@{}}Acceptability\\ ratings,\\ Likert scale\end{tabular} &
		\begin{tabular}[c]{@{}l@{}}Crossing function (subject\slash object),\\ extraction (no extraction\slash extraction out\\ of sentential VP) and extraposition\\ (sentential VP extraposed with expletive\\ \textit{es}\slash not extraposed).\end{tabular} &
		\begin{tabular}[c]{@{}l@{}}- interaction function:extraction, so that\\ the extraction out of the sentential\\ subject is rated worse\\ - no interaction extraposition:extraction\\ - for non-extraposed sentential VP,\\ interaction function:extraction ($p < 0.001$)\\ - for non-extraposed sentential VP, subject\\ rated higher than object ($p < 0.001$)\\ - for extraposed sentential VP, subject\\ rated higher than object ($p < 0.001$)\\ - for sentential subject, interaction\\ extraposition:extraction\end{tabular} \\ \midrule
		
		\begin{tabular}[c]{@{}l@{}}\citet{Jurka.2010},\\ Experiment 6\end{tabular} &
		German &
		\begin{tabular}[c]{@{}l@{}}\textit{Wh}-question\end{tabular} &
		\begin{tabular}[c]{@{}l@{}}Non-finite\\ sentential\\ subject\end{tabular} &
		\begin{tabular}[c]{@{}l@{}}Direct\\ object\end{tabular} &
		\begin{tabular}[c]{@{}l@{}}Acceptability\\ ratings,\\ Likert scale\end{tabular} &
		\begin{tabular}[c]{@{}l@{}}Crossing function (sentential subject /\\ sentential object) and V2 (auxiliary with\\ past participle\slash main verb with present\\ tense).\end{tabular} &
		\begin{tabular}[c]{@{}l@{}}- no interaction function:V2\end{tabular} \\ \midrule
		
		\begin{tabular}[c]{@{}l@{}}\citet{Jurka.2010},\\ Experiment 7\end{tabular} &
		German &
		\begin{tabular}[c]{@{}l@{}}\textit{Wh}-question\end{tabular} &
		\begin{tabular}[c]{@{}l@{}}Non-finite\\ sentential\\ subject\end{tabular} &
		\begin{tabular}[c]{@{}l@{}}Direct\\ object\end{tabular} &
		\begin{tabular}[c]{@{}l@{}}Acceptability\\ ratings,\\ Likert scale\end{tabular} &
		\begin{tabular}[c]{@{}l@{}}Crossing extraction (extraction out of the\\ sentential subject\slash no extraction) and\\ verb type (particle verb\slash verb without\\ article).\end{tabular} &
		\begin{tabular}[c]{@{}l@{}}- no interaction extraction:verb type\end{tabular} \\ \midrule
		
		\begin{tabular}[c]{@{}l@{}}\citet{Jurka.2011},\\ Experiment in Section 2.2\end{tabular} &
		German &
		\begin{tabular}[c]{@{}l@{}}\textit{Wh}-question\end{tabular} &
		\begin{tabular}[c]{@{}l@{}}Non-finite\\ sentential\\ subject\end{tabular} &
		\begin{tabular}[c]{@{}l@{}}Direct\\ object\end{tabular} &
		\begin{tabular}[c]{@{}l@{}}Acceptability\\ ratings,\\ Likert scale\end{tabular} &
		\begin{tabular}[c]{@{}l@{}}Crossing extraction type (no extraction /\\ extraction) and function (subject\slash object).\end{tabular} &
		\begin{tabular}[c]{@{}l@{}}- interaction extraction type:function\\ ($p < 0.001$)\end{tabular} \\ \midrule	
		
		\begin{tabular}[c]{@{}l@{}}\citet{Jurka.2010},\\ Experiment 12\\ \& \\ \citet{Jurka.2011},\\ Japanese Experiment\end{tabular} &
		Japanese &
		\begin{tabular}[c]{@{}l@{}}Scrambling \&\\ clefts\end{tabular} &
		\begin{tabular}[c]{@{}l@{}}Finite\\ sentential\\ subject\end{tabular} &
		\begin{tabular}[c]{@{}l@{}}Direct\\ object\end{tabular} &
		\begin{tabular}[c]{@{}l@{}}Acceptability\\ ratings,\\ Likert scale\end{tabular} &
		\begin{tabular}[c]{@{}l@{}}Crossing function (subject\slash object) and\\ construction (no extraction\slash scrambling /\\ cleft).\end{tabular} &
		\begin{tabular}[c]{@{}l@{}}- interaction function:construction\\ ($p < 0.001$)\\ - for scrambling, no difference between\\ subject and object\\ - for clefts, no difference between\\ object and subject\\ - for no extraction, object rated lower\\ than subject ($p = 0.002$)\end{tabular} \\ \midrule
		
		\begin{tabular}[c]{@{}l@{}}\citet{Fukuda.2014},\\ Japanese Experiment\end{tabular} &
		Japanese &
		Scrambling &
		\begin{tabular}[c]{@{}l@{}}Finite\\ sentential\\ subject\end{tabular} &
		\begin{tabular}[c]{@{}l@{}}Direct\\ object\end{tabular} &
		\begin{tabular}[c]{@{}l@{}}Acceptability\\ ratings,\\ Likert scale\end{tabular} &
		\begin{tabular}[c]{@{}l@{}}Crossing function (subject\slash object) and\\ scrambling (no scrambling\slash scrambling).\end{tabular} &
		\begin{tabular}[c]{@{}l@{}}- no interaction function:scrambling\end{tabular} \\ \lspbottomrule
	\end{longtable}
\end{landscape}

% \textless 

%Tested embedded wh-question, crossing the
}
%\caption{Studies on sentential subjects}
%\label{tab:previous-sentential-subject}
