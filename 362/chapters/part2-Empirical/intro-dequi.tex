In this chapter, I present two corpus studies and five experiments. The chapter is organized as follows:

\begin{description}
\sloppy
\item[Corpus studies on \emph{de qui}:] Data from Frantext show that \emph{de qui} relative clauses behave very similarly to \emph{dont} relative clauses. The most common usage of \emph{de qui} in relative clauses is extracting out of the subject. Interrogatives, however, are very different, and extraction out of the subject is not attested either in direct or direct questions. This may indicate that there is a cross-construction difference with respect to extracting out of the subject. 

\item[Experiment 5:] In this acceptability judgment study, we tested \emph{de qui} relative clauses, crossing extraction type (extraction\slash non-extraction) with syntactic function (subject\slash object). The extraction takes place out of quality denoting NP (e.g.\ \emph{violence} `violence'). Contrary to what we saw in previous experiments, extraction out of the subject received significantly lower ratings than extraction out of the object, but there is no interaction effect between extraction type and grammatical function.

\item[Experiment 6:] In this acceptability judgment study, we tested the material of Experiment~5 but with the relative word \emph{dont} instead of \emph{de qui}. We also used the same distractors. The ratings for extraction out of the subject and out of the object did not differ significantly, but the results of this experiment are questionable because we observe very strong ceiling effects. 

\item[Experiment 7:] In this acceptability judgment study, we tested \emph{de qui} relative clauses again, crossing extraction type (extraction\slash non-extraction\slash ungrammatical controls) with syntactic function (subject\slash object). Extraction took place out of NPs that denote human relations (e.g.\ \emph{associé} `associate'). The preference for extractions out of the object observed in Experiment 5 disappeared, and extractions out of the subject were significantly better than extractions out of the object. We conclude that the significant difference observed in Experiment~5 was due to an animacy mismatch between subject and object, and not to extraction out of the subject as such.

\item[Experiment 8:] In this acceptability judgment study, we tested the materials of Experiment~7 but with the relative word \emph{dont} instead of \emph{de qui}. Again, extractions out of the subject were judged to be significantly better than extractions out of the object.

\item[Experiment 9:] In this self-paced reading experiment, we investigated the online processing of \emph{de qui} relative clauses, crossing extraction type (extraction\slash non-extraction\slash ungrammatical controls) with syntactic function (subject\slash object). The results do not show any obvious difficulty with extractions out of the subject. Subjects were read more quickly than objects, regardless of the extraction type. Surprisingly, we did not observe any processing cost associated with subextraction, either, which may be due to our grammatical controls, or to the experiment lacking power (e.g.\ not enough participants).

\end{description}
