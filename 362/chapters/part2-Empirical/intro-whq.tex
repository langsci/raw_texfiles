In this chapter, I present four experiments on interrogatives. All are acceptability judgment tasks, and are designed to replicate the previous experiments on relative clauses with similar material but for a different construction. To the best of my knowledge, no account except our proposal based on the FBC constraint expects different results across constructions. But we will see that, just like in the corpus studies, there is a sharp contrast between relative clauses and interrogatives as far as extraction out of subjects is concerned.

\begin{description}
\item[Experiment 10:] In this acceptability judgment study, we crossed extraction type (extraction\slash non-extraction\slash ungrammatical controls) with syntactic function (subject\slash object) and tested \emph{de quel} + N interrogatives. The extraction took place out of quality denoting NP (e.g.\ \emph{originalité} `uniqueness'), using stimuli similar to Experiment~4. The extraction out of the subject received significantly lower ratings than the extraction out of the object, and there was a significant interaction such that ratings were lower for extraction out of the subject than for the non-extraction controls; however, they were better than for the ungrammatical controls. 

\item[Experiment 11:] In this acceptability judgment study, we used the same materials as in Experiment~10, but the \emph{wh}-word in the subextraction condition was in situ. We crossed question type (with \emph{wh}-word\slash polar question) with syntactic function (subject\slash object) and tested \emph{de quel} + N interrogatives. There was no significant difference between the conditions with the \emph{wh}-word inside the subject NP vs.\ inside the object NP. There was also no significant interaction. This result is unexpected under the FBC constraint, unless in situ questions have a different information structure than questions with the \emph{wh}-word extracted.

\item[Experiment 12:] In this acceptability judgment study, we crossed extraction type (extraction\slash non-extraction\slash ungrammatical controls) with syntactic function (subject\slash object) and tested \emph{de qui} interrogatives. Extraction took place out of an NP denoting a social relation (e.g.\ \emph{associé} `associate'), using stimuli similar to Experiment~7. Ratings were significantly lower for extraction out of the subject than for extraction out of the object, and there was a significant interaction such that extraction out of the subject has lower ratings than the non-extraction controls. The results would be very similar to the results of Experiment~10 but for the very low acceptability of \emph{de qui} subextractions: extraction out of the subject was actually not significantly better than the ungrammatical controls. 

\item[Experiment 13:] In this acceptability judgment study, we crossed extraction type (extraction\slash non-extraction\slash ungrammatical controls) with syntactic function (subject\slash object) and tested \emph{de quel} + N interrogatives in a long-distance dependency. Extraction out of the subject received lower ratings than extraction out of the object, but the difference was not significant. Overall, the results were relatively similar to the long-distance dependencies in relative clauses.
\end{description} 
