\chapter{Number, definiteness and restrictiveness in relative clauses (corpus studies)}
\label{ch:other-factors}

In the corpus studies described in this book, I have annotated several properties of the relative clauses that are not reported in Part~\ref{part:2} of the book, since these factors do not provide any conclusive explanation. The results are, however, worth mentioning, if only because they show that the factors in question do not play a dominant role.

\begin{sloppypar}
I annotated two factors regarding the antecedent of the relative clauses: whether the antecedent was singular or plural, and definite or indefinite. \figref{fig:number-corpus} illustrates the results for the number of the antecedent, while \figref{fig:definiteness-corpus} on page~\pageref{fig:definiteness-corpus} shows the results for definiteness of the antecedent. Antecedents are often singular and often definite. Extraction type seems to be irrelevant, so extraction out of the subject does not stand out.
\end{sloppypar}

Finally, I annotated the restrictiveness of the relative clauses, following the guidelines explained above (Appendix A). The results of this annotation are displayed in \figref{fig:restrictiveness corpus}.

In general, we can say that the corpus contains about as many restrictive as non-restrictive relative clauses. Extraction of the complement of the verb stands out in that there seem to be a particularly high number of restrictive occurrences (also in extractions of the complement of the adjective, but the observation is based on few occurrences, thus it is not very reliable). 


\begin{sloppypar}
Following \citet{Song.2017}, the extracted element in a non-restrictive relative clause is necessarily a topic, but not in a restrictive relative clause. For this reason, we expected to find more non-restrictive relative clauses in extraction out of the subject than in the other kinds of extraction. The data show that relativization out of the subject is quite often non-restrictive, but not necessarily more than relativization out of other kinds of NPs. Relativization out of the subject with \emph{duquel} appears to be an outlier with very few restrictive relative clauses.
\end{sloppypar}

Table~\ref{tab:recap-corpus-restrictivity} compares extraction out of subjects with extraction out of objects and extraction of the complement of the verb, because these types of extraction always have a frequency significantly above zero. The + sign indicates that there are more non-restrictive relative clauses in extraction out of a subject, and the -- sign indicates that there are fewer. A doubled sign signals that the difference is large. 

We can see a clear tendency that confirms our expectations. However, the difference with extraction out of the object is not as strong as expected. Notice, however, that our annotation criteria for restrictiveness were somewhat rudimentary, and that it would be desirable to take into account the whole context in determining the value of the restrictiveness. The internal information structure of restrictive and non-restrictive relative clauses is not well understood in general and should be studied on its own.

\begin{figure}
        \centering
        \includegraphics[width=\textwidth]{chapters/partAnnex/Other factors graphs/number.jpeg}
        \caption{Number of the antecedent for all relative clauses in the corpus studies}
        \label{fig:number-corpus}
\end{figure}

\begin{figure}
        \centering
        \includegraphics[width=\textwidth]{chapters/partAnnex/Other factors graphs/definition.jpeg}
        \caption{Definiteness of the antecedent for all relative clauses in the corpus studies}
        \label{fig:definiteness-corpus}
\end{figure}

\begin{figure}
        \centering
        \includegraphics[width=\textwidth]{chapters/partAnnex/Other factors graphs/restrictiveness.jpeg}
        \caption{Restrictiveness of all relative clauses from the corpus studies}
        \label{fig:restrictiveness corpus}
\end{figure}

\begin{table}
    \begin{tabular}{lcc}
         \lsptoprule
          & \multicolumn{2}{p{6cm}}{\raggedright\arraybackslash More non-restrictive relative clauses in extractions out of the subject than in extractions\dots}\\\addlinespace
          & \multicolumn{1}{p{3cm}}{\raggedright\arraybackslash \dots of the comple- ment of the verb} & \multicolumn{1}{p{3cm}}{\raggedright\arraybackslash \dots out of the object}\\
         \midrule
         \emph{dont} in FTB & ++ & + \\
         \emph{dont} in Frantext 2000--2013 & ++ & ++ \\
         \emph{dont} in Frantext 1900--1913 & + & + \\
         \emph{de qui} in Frantext 2000--2013 & + & − − \\
         \emph{de qui} in Frantext 1900--1913 & ++ & + \\
         \emph{duquel} in Frantext 2000--2013 & + & ++ \\
         \emph{avec}\,+\,\emph{wh} in Frantext 2000--2013 & ++ & + \\
         \lspbottomrule
    \end{tabular}
    \caption{Comparison of the amount of non-restrictive relative clauses in the different corpus studies}
    \label{tab:recap-corpus-restrictivity}
\end{table}
