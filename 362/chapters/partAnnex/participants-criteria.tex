\chapter{Exclusion criteria for participants' data}
\label{ch:participants-criteria}

The decision on whether or not to include a participant's response in the statistical analysis was made by applying the same criteria for all experiments.

\begin{enumerate}
    \item The responses of the participants that did not complete the entire experiment were deleted altogether, as were the data from participants who did not give their explicit consent.
    \item For legal reasons, I only kept the data from participants over 18 (age of legal majority in France).
    \item I only kept data from native monolingual speakers of French. Participants were asked to name their native language, but also to mention the other languages their speak and their level of proficiency on a scale of 1 to 10. I considered as non-native speakers participants who indicated a language other than French as their native language, did not answer the question, or gave an unrelated answer. I considered as bilingual or potentially bilingual participants who either indicated two languages as their native languages or indicated an L2 with a proficiency of 10. The data from non-native speakers, bilinguals and potential bilinguals are excluded from the statistical analysis.
    \item I only kept only data from participants who grew up in a French-speaking country. I considered as francophone country any country in which French is one of the official languages, as well as Algeria and Morocco. The data from participants who did not grow up in a francophone country according to this criterion are excluded from the statistical analysis. Overall, most likely due to the way the participants were recruited  for internet experiments (over social media per snowballing effects), or to the fact that the experiments run in the lab were conducted at the Université Paris Cité, the great majority of participants grew up in Metropolitan France.
    \item Based on the answers to comprehension questions (if there were any), I computed each participant's accuracy. For this, the answers to the comprehension questions related to practice items and ungrammatical controls were not taken into account. If I noticed that a condition or a particular set of distractors received an unusually high number of incorrect answers overall, then these answers were also excluded from computing the participants' accuracy rate. I mention these details in the relevant sections. Data from participants with an accuracy rate below 75\% were excluded from the statistical analysis.
    \item Finally, I sometimes had to exclude participants that did not discriminate between conditions in my test items, and always gave the same rating (or almost always with one or two outliers). I mention these cases in the relevant sections.
\end{enumerate}
