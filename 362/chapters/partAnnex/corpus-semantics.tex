\chapter{Semantic relations in relative clauses (corpus studies)}
In addition to the factors already mentioned in Part~\ref{part:2}, we annotated the type of semantic relationship between the extracted element and its head. We performed this annotation both for extraction out of the subject and out of the direct object for comparison.

We identified seven different kinds of relations: 
\begin{itemize}
    \item[(i)] agent or cause, where the extracted element fulfills any kind of proto-agentive role with respect to the (often deverbal) head noun
    \item[(ii)] part of whole, the whole being expressed by the head noun
    \item[(iii)] patient, where the extracted element fulfills any kind of proto-patient role\footnote{Experiencers are annotated as `patient' (\emph{l'angoisse de Grégoire}, `Grégoire's anxiety') and so are topics in depicted-topic relations (\emph{la peinture de la Vierge'}, `the painting of the Virgin Mary') and information (\emph{la biographie de Françoise Sagan}, `the biography of Françoise Sagan').} with respect to the (often deverbal) head noun
    \item[(iv)] possession
    \item[(v)] quality, where the head noun denotes a quality of the extracted element
    \item[(vi)] quantifier, where the extracted element is quantificational with scope over the head noun
    \item[(vii)] relational, where the (mostly animate) head noun denotes a social or family relation with respect to the extracted element
\end{itemize}
The following examples, taken from the corpus studies mentioned above, are examples of extraction out of the subject illustrating these semantic relations.\largerpage[-1]\pagebreak

\begin{exe} \label{ex:agens-cause}
\ex agent/cause:
\begin{xlist}
\ex (FTB - flmf7ah2ep-760)\\
\gll l' Afghanistan, dont [la participation~\trace{}] avait été suspendue à la suite de l' invasion soviétique\\
the Afghanistan of.which the participation had been suspended at the result of the invasion Soviet\\
\glt `Afghanistan, whose participation had been suspended after the Soviet invasion'
\ex (Pense à demain, Anne-Marie Garat, 2010)\\
\gll un blondinet boutonneux, de qui [le geste~\trace{}] l' électrisa\\
a {blond boy} pimply of who the gesture her.\textsc{acc} electrified\\
\glt `the pimply blonde boy, whose gesture electrified her'
\ex (Jean Barois, Roger Martin Du Gard, 1913)\\
\gll ceux, du moins, de qui [le jugement~\trace{}] garde une activité propre\\
these of least of who the judgment retains an activity own\\
\glt `these, at least, whose judgment retains an activity of its own
\end{xlist}

\ex part-whole: \label{ex:part-of-whole}
\begin{xlist}
\ex (FTB - flmf7af2ep-559)\\
\gll un groupe dont [les unités~\trace{}] sont très autonomes\\
a group of.which the units are very autonomous\\
\glt `a group whose units are very independent'
\ex (Mécanique, François Bon, 2001)\\
\gll l' ordinateur de plastique tout neuf, duquel il vous avait demandé à quoi servaient [les prises de branchement~\trace{}]\\
the computer of plastic all new of.which he you\textsc{.acc} had asked at what used the plugs of connection\\
\glt `the new plastic computer, of which he had asked you what the plugs  were for'
\ex (Partage de midi [1re version], Paul Claudel, 1906)\\
\gll une danseuse écoutante, dont [les petits pieds jubilants~\trace{}] sont cueillis par la mesure irrésistible~! \\
a dancer.\textsc{fem} listening of.which the little feet jubilant are caught by the beat irresistible\\
\glt `a listening dancer, whose jubilant little feet are caught by the irresistible beat!'
\end{xlist}

\ex patient: \label{ex:patient}
\begin{xlist}
\ex (FTB - flmf3\_03500\_03999ep-3699)\\
\gll cette arme dont [la gestion~\trace{}] relève du ministère de la défense\\
this weapon of.which the management belongs of.the Ministry of the Defense\\
\glt `this weapon whose management belongs to the Ministry of Defense'
\ex (Pense à demain, Anne-Marie Garat, 2010)\\
\gll Viviane, de qui [l' euphorie~\trace{}] augmentait sa tristesse\\
Viviane of who the euphoria increased her sadness\\
\glt `Vivian, whose euphoria increased her sadness'
\ex (Dans la main du diable, Anne-Marie Garat, 2006)\\
\gll Millie, dont grandissait [l' angoisse~\trace{}]\\
Millie of.which grew the anxiety\\
\glt `Millie, whose anxiety was growing'
\end{xlist}

\ex possession: \label{ex:possession-relation}
\begin{xlist}
\ex (FTB - flmf3\_11000\_11499ep-11199)\\
\gll American et United, dont [les flottes~\trace{}] dépassent les cinq cents avions\\
American and United of.which the fleets surpass the five hundred aircraft\\
\glt `American and United, whose fleets contain more than five hundred aircrafts'
\ex(L'Île des pingouins, Anatole France, 1908)\\
\gll un vieux compagnon d' armes dont [les états de service~\trace{}] étaient superbes\\
an old comrade of arms of.which the states of service were superb\\
\glt `an old comrade-in-arms whose service record was superb'
\end{xlist}

\ex quality: \label{ex:quality}
\begin{xlist}
\ex (FTB - flmf7am2ep-661)\\
\gll l' économie américaine, dont [le poids~\trace{}] est lourd\\
the economy american of.which the weight is heavy\\
\glt `the American economy, whose weight is high'
\ex (Le Voyage de Sparte, Maurice Barrès, 1906)\\
\gll ce fameux sire de Caritena, de qui [le courage~\trace{}], [la courtoisie envers les dames~\trace{}] et [l' absurde frivolité~\trace{}] éclatent dans le livre de la conqueste publié par Buchon.\\
this famous sire of Caritena of who the courage the courtesy towards the ladies and the absurd frivolity burst in the book of the conqueste published by Buchon\\
\glt `this famous sire of Caritena whose courage, courtesy towards the ladies and absurd frivolity shine through in the book of the conqueste published by Buchon.'
\end{xlist}


\ex quantifier: \label{ex:quantifier}
\begin{xlist}
\ex (FTB - flmf7ao1ep-492)\\
\gll des journalistes dont [certains~\trace{}] ont connu le chômage\\
some journalists of.which several have known the unemployment \\
\glt `journalists among which several have experienced unemployment'
\ex (Jean-Christophe : La Foire sur la place, Romain Rolland, 1908)\\
\gll ses {vingt et un} enfants, dont [treize~\trace{}] moururent avant lui\\
his twenty-one children of.which thirteen died before him\\
\glt `his twenty-one children, out of which thirteen died before him'
\end{xlist}


\ex relational: \label{ex:relational}
\begin{xlist}
\ex(FTB - flmf7ai2ep-886)\\
\gll Banexi und partner,~[\dots] dont bon nombre [de clients~\trace{}] sont des petits patrons\\
Banexi und partner of.which good amount of clients are \textsc{det} small bosses\\
\glt `Banexi und partner, of which a good number of the clients own small businesses'
\ex (L'enfant des ténèbres, Anne-Marie Garat, 2008)\\
\gll Un modeste papetier de qui [l' épouse~\trace{}]~[\dots] avait constitué le premier rayon d' ouvrages pour dames\\
a modest papermaker of who the wife had established the first section of books for ladies\\
\glt `the modest papermaker, whose wife had set up the first section for ladies' books'
\ex (Le corps incertain, Vanessa Gault, 2006)\\
\gll quelqu'un dont [la fille~\trace{}] a une copine de classe qui a une {sclérose en plaques}\\
someone of.which the daughter has a friend of class who has a {multiple sclerosis}\\ 
\glt `someone whose daughter has a classmate who has multiple sclerosis'
\end{xlist}
\end{exe}

\section{\emph{Dont} relatives in the French Treebank}


\begin{figure}
        \centering
        \includegraphics[width=1\textwidth]{chapters/part2-Empirical/dont/dont-FTB/relations.jpeg}
        \caption{Semantic relations in \emph{dont} relative clauses in the French Treebank (subject vs.\ object subextractions). See page~\pageref{ch:conf-intervals-binomial} for the confidence intervals (here seven comparisons for subject and six for object). The percentage is given for each group (extraction out of the subject vs.\ extraction out of the object).}
        \label{fig:dont-FTB-relations}
\end{figure}

\figref{fig:dont-FTB-relations} gives the proportion of every category in extraction out of subjects in the FTB, comparing it to extraction out of direct objects as a baseline. The most common kind are \emph{de}-PPs denoting possession in both kinds of extraction. We find almost no relationals (given the confidence intervals, their number is not significantly higher than zero).

We fitted a logistic regression model predicting the source of extraction (subject~= 1; object =~0) for the kind of relation, and we performed a residual diagnostic to test the predictions of the model. The detailed results of the model are given in \tabref{tab:FTB-relations}.\footnote{Validation of the model: The regression model is valid iff the number of data points is at least equal to 5 times the number of explanatory variables. Here, it must be at least equal to 35, and there are 263 data points. Furthermore, the residual diagnostics are compelling. The model is therefore valid.} No category of relation is a good predictor for the variable to be explained, which corroborates what we can see in \figref{fig:dont-FTB-relations} (confidence intervals overlap pairwise). 

\begin{table}
\begin{tabular}{l S[table-format=-2.3] S[table-format=3.3] S[table-format=-1.4] S[table-format=1.4] S[table-format=6.2]}
  \lsptoprule
         & {Estimate} & {SE} & {$z$} & {$p$} & {Odd.ratio} \\ 
  \midrule
(Intercept)   & 1.350 & 0.424 & 3.1827 & 0.0015 & 3.86 \\ 
   part-whole & 1.289 & 1.119 & 1.1524 & 0.2491 & 3.63 \\ 
   patient    & 0.596 & 0.749 & 0.7958 & 0.4261 & 1.81 \\ 
   possession & -0.090 & 0.491 & -0.1841 & 0.8539 & 1.09 \\ 
   quality    & 0.241 & 0.538 & 0.4484 & 0.6539 & 1.27 \\ 
   quantifier & 0.483 & 0.685 & 0.7041 & 0.4814 & 1.62 \\ 
   relational & 13.216 & 882.744 & 0.0150 & 0.9881 & 549135.22 \\ 
   \lspbottomrule
\end{tabular}
\caption{Results of the logistic regression}
        \label{tab:FTB-relations}
\end{table}

We can therefore say that in this corpus, no semantic relation type seems to increase or decrease the ability to form an extraction out of the subject. But some semantic relation types are more common than others (although we cannot say whether this is due to the extraction).

\section{\emph{Dont} relatives in Frantext 2000--2013}


\begin{figure}
        \centering
        \includegraphics[width=1\textwidth]{chapters/part2-Empirical/dont/dont-Frantext-2000/relations.jpeg}
        \caption{Semantic relations in Frantext 2000 \emph{dont} relative clauses (subject vs.\ object subextractions) See page~\pageref{ch:conf-intervals-binomial} for the confidence intervals (here seven comparisons for subject and five for object). The percentage is given for each group (extraction out of the subject vs.\ extraction out of the object).}
        \label{fig:dont-d2000-relations}
\end{figure}

\figref{fig:dont-d2000-relations} gives the proportion of every category in extraction out of subjects in Frantext 2000--2013, comparing it to extraction out of direct objects as a baseline. 
Part-whole is the most common relation in extraction out of the subject, agent and patient are equally the most common in extraction out of the object. 
While there is almost no relational in the FTB (see above), 10\% of the extractions out of the subject are relational subject nouns in Frantext 2000--2013. 
In extraction out of the subject, the frequency of the categories patient and quantifier is not significantly different from zero. 
In extraction out of the object, there are no instances of the categories quantifier and relational, and the frequency of the category part-whole is not significantly different from zero.

We fitted a logistic regression model predicting the source of extraction (subject~= 1; object =~0) for the kind of relation, and we performed a residual diagnostic to test the predictions of the model. The detailed results of the model are given in \tabref{tab:d2000-relations}.\footnote{Validation of the model: The regression model is valid iff the number of data points is at least equal to 5 times the number of explanatory variables. Here, it must be at least equal to 35, and there are 91 data points. Furthermore, the residual diagnostics are compelling. The model is therefore valid.} 

\begin{table}
\begin{tabular}{l S[table-format=-2.3] S[table-format=4.3] S[table-format=-1.4] S[table-format=1.4] S[table-format=8.2]}
  \lsptoprule
         & {Estimate} & {SE} & {$z$} & {$p$} & {Odd.ratio} \\ 
  \midrule
(Intercept) & 0.560 & 0.443 & 1.2627 & 0.2067 & 1.75 \\ 
   part-whole & 1.232 & 0.765 & 1.6105 & 0.1073 & 3.43 \\ 
   patient & -1.946 & 0.906 & -2.1470 & 0.0318 & 7.00 \\ 
   possession & 0.229 & 0.698 & 0.3278 & 0.7431 & 1.26 \\ 
   quality & -0.714 & 0.711 & -1.0035 & 0.3156 & 2.04 \\ 
   quantifier & 17.006 & 2284.102 & 0.0074 & 0.9941 & 24312471.33 \\ 
   relational & 17.006 & 1615.104 & 0.0105 & 0.9916 & 24312471.33 \\ 
   \lspbottomrule
\end{tabular}
\caption{Results of the logistic regression}
        \label{tab:d2000-relations}
\end{table}

No category of relation is a good predictor for the variable to be explained, but if we drop all possible single terms (the different relations),\footnote{In order to do this, I used the function \texttt{drop1()} from the R package \emph{stats} \citep{R}.} the effect of the variable relation in predicting the source of the extraction becomes significant ($p<0.005$). Overall, there is a difference with respect to the kind of relation between \emph{dont} and its head noun in extraction out of the subject compared to extraction out of the object.

There is thus a difference between FTB and Frantext, which is probably due to both the corpora and to our searches. On the one hand, the corpora diverge in the kind of texts they contain. Newspaper articles have a large number of quantifiers because their texts often aim to provide objective descriptions of a situation (numbers being considered objective facts), and often deal with economics. Frantext, by contrast, contains a lot of autobiographical texts, with an introspective dimension. On the other hand, because we only looked at relatives with an animate antecedent, there are many social or familty relations (brother, mother, uncle, etc.) and many body parts (part-whole) in our results for Frantext. In that corpus we observe an asymmetry in the subextractions. Part-whole and relational are rare (or completely absent) in extraction out of objects.

\section{\emph{Dont} relatives in Frantext 1900--1913}

\begin{figure}
        \centering
        \includegraphics[width=1\textwidth]{chapters/part2-Empirical/dont/dont-Frantext-1900/relations.jpeg}
        \caption[Semantic relations in Frantext 1900 \emph{dont} relative clauses (subject vs.\ object subextractions)]{Semantics relations in Frantext 1900 \emph{dont} relative clauses (subject vs.\ object subextractions). See page~\pageref{ch:conf-intervals-binomial} for the confidence intervals (here seven comparisons for subject and five for object). The percentage is given for each group (extraction out of the subject vs.\ extraction out of the object).}
        \label{fig:dont-d1900-relations}
\end{figure}

\figref{fig:dont-d1900-relations} gives the proportion of every kind of relation between \emph{dont} and its head noun in extraction out of subjects in Frantext 1900--1913, comparing it to extraction out of direct objects as a baseline. The results corroborate to a certain extent what we see in Frantext 2000--2013. Part-whole is the most common relation in extraction out of the subject (similar in Frantext 2000--2013) and agent\slash cause and quality are similarly the most common relations in extraction out of the object (in Frantext 2000--2013, agent/cause and patient were the most common relations). In extraction out of the subject, the frequency of the category relational is not significantly above zero. In extraction out of the object, the categories relational, quantifier and possession have frequencies not significantly different from zero.

We fitted a logistic regression model predicting the source of extraction (subject~= 1; object~= 0) for the kind of relation, and we performed a residual diagnostic to test the predictions of the model. The detailed results of the model are given in \tabref{tab:d1900-relations}.\footnote{Validation of the model: The regression model is valid iff the number of data points is at least equal to 5 times the number of explanatory variables. Here, it must be at least equal to 35, and there are 134 data points. Furthermore, the residual diagnostics are compelling. The model is therefore valid.} 

\begin{table}
\begin{tabular}{l S[table-format=-1.3] S[table-format=1.3] S[table-format=-1.4] S[table-format=1.4] S[table-format=1.2]}
  \lsptoprule
         & {Estimate} & {SE} & {$z$} & {$p$} & {Odd.ratio} \\ 
  \midrule
(Intercept) & 0.693 & 0.408 & 1.6979 & 0.0895 & 2.00 \\ 
   part-whole & 1.224 & 0.629 & 1.9445 & 0.0518 & 3.40 \\ 
   patient & -0.827 & 0.659 & -1.2541 & 0.2098 & 2.29 \\ 
   possession & 1.012 & 0.870 & 1.1622 & 0.2451 & 2.75 \\ 
   quality & 0.000 & 0.577 & 0.0000 & 1.0000 & 1.00 \\ 
   quantifier & 1.504 & 1.130 & 1.3307 & 0.1833 & 4.50 \\ 
   relational & -0.000 & 1.291 & -0.0000 & 1.0000 & 1.00 \\ 
   \lspbottomrule
\end{tabular}
\caption{Results of the logistic regression}
        \label{tab:d1900-relations}
\end{table}

In this model, part-whole is close to the threshold for being a good predictor for the variable to be explained (but still p$>$0.05). If we drop all possible single terms in the model, the effect of the variable relation in explaining the source of the extraction becomes significant ($p<0.05$). Overall, there is a difference with respect to the kind of relation between \emph{dont} and its head noun in extraction out of the subject compared to extraction out of the object.

\emph{Dont} relative clauses are similar in Frantext in the periods 2000--2013 and 1900--1913. Generally speaking, the relation between the head subject noun and the extracted \emph{dont} is often a part-whole relation, whereas the one between the head object noun and \emph{dont} is often a patient-event relation.

\section{\emph{De qui} relatives in Frantext 2000--2013}

\begin{figure}
        \centering
        \includegraphics[width=\textwidth]{chapters/part2-Empirical/de-qui/dequi-Frantext-2000/relations.jpeg}
        \caption[Semantic relations in Frantext 2000 \emph{de qui} relative clauses (subject vs.\ object subextractions)]{Semantics relations in Frantext 2000 \emph{de qui} relative clauses (subject vs.\ object subextractions). See page~\pageref{ch:conf-intervals-binomial} for the confidence intervals (here seven comparisons). The percentage is given for each group (extraction out of the subject vs.\ other extraction).}
        \label{fig:dq2000-relations}
\end{figure}

\figref{fig:dq2000-relations} illustrates the different semantic relations between the filler \emph{de qui} and the head noun. The most frequent relation for both subject and object is agent/cause. We found no instances of the quantifier relation for subjects and the number of instances with quantifier and relational for objects is not significantly above zero.

We fitted a logistic regression model predicting the source of extraction (subject~= 1; object =~0) for the kind of relation, and we performed a residual diagnostic to test the predictions of the model. The detailed results of the model are given in \tabref{tab:dq2000-relations}.\footnote{Validation of the model: The regression model is valid iff the number of data points is at least equal to 5 times the number of explanatory variables. Here, it must be at least equal to 35, and there are 83 data points. Furthermore, the residual diagnostics are compelling. The model is therefore valid.} 

\begin{table}
\begin{tabular}{l S[table-format=-2.3] S[table-format=4.3] S[table-format=-1.4] S[table-format=1.4] S[table-format=8.2]}
  \lsptoprule
         & {Estimate} & {SE} & {$z$} & {$p$} & {Odd.ratio} \\ 
  \midrule
(Intercept)   & 0.4855 & 0.4494 & 1.0804 & 0.2799 & 1.63 \\ 
   part-whole & 0.3618 & 0.8235 & 0.4393 & 0.6604 & 1.44 \\ 
   patient    & -1.1787 & 0.8378 & -1.4068 & 0.1595 & 3.25 \\ 
   possession & -0.4855 & 0.6983 & -0.6953 & 0.4869 & 1.63 \\ 
   quality    & 0.9008  & 0.7865 & 1.1453 & 0.2521 & 2.46 \\ 
   quantifier & -17.052 & 1696.734 & -0.0100 & 0.9920 & 25434065.63 \\ 
   relational & 1.9124  & 1.1370 & 1.6819 & 0.0926 & 6.77 \\ 
   \lspbottomrule
\end{tabular}
\caption{Results of the logistic regression}
        \label{tab:dq2000-relations}
\end{table}

No single category is a significant predictor for the source of the extraction. If we drop all possible single terms in the model, the effect of the variable relation in explaining the source of the extraction becomes significant ($p<0.05$).

In Frantext 2000--2013, we do not observe any difference between the use of \emph{dont} and the use of \emph{de qui}.\footnote{An additional model (crossing the semantic relation with the distinction between subject and object) with data from both \emph{dont} and \emph{de qui} in Frantext 2000--2013 shows that the semantic relation is not a good predictor of the choice for one relative phrase over the other ($p=0.37889$), though there is a significant interaction of semantic role with subject vs.\ object ($p= 0.03278$). This interaction is difficult to interpret. The diagnostics for this model are not good, therefore this result should be observed with caution.}

\section{\emph{De qui} relatives in Frantext 1900--1913}


\begin{figure}
        \centering
        \includegraphics[width=\textwidth]{chapters/part2-Empirical/de-qui/dequi-Frantext-1900/relations.jpeg}
        \caption[Semantics relations in Frantext 1900 \emph{de qui} relative clauses (subject vs.\ object subextractions)]{Semantic relations in Frantext 1900 \emph{de qui} relative clauses (subject vs.\ object subextractions). See page~\pageref{ch:conf-intervals-binomial} for the confidence intervals (here six comparisons for subjects and four comparisons for objects). The percentage is given for each group (extraction out of the subject vs.\ other extraction).}
        \label{fig:dq1900-relations}
\end{figure}

\figref{fig:dq1900-relations} illustrates the different semantic relations between the filler and the head noun, and their proportions, comparing extraction out of the subject and out of the object. Just as in \emph{dont} relative clauses in Frantext~1900--1913, part-whole is the most common relation in extraction out of the subject and agent\slash cause and quality are the most common in extraction out of the object. There are no relationals with objects and not significantly more than zero with subjects. Moreover, there are no instances of the patient relation for extraction out of object, which is probably the most striking difference with respect to \emph{dont} relative clauses (compare with \figref{fig:dont-d1900-relations}).  

We fitted a logistic regression model predicting the source of extraction (subject~= 1; object = 0) for the kind of relation, and we performed a residual diagnostic to test the predictions of the model. The detailed results of the model are given in \tabref{tab:dq1900-relations}.\footnote{Validation of the model: The regression model is valid iff the number of data points is at least equal to 5 times the number of explanatory variables. Here, it must be at least equal to 35, and there are 53 data points. Furthermore, the residual diagnostics are compelling. The model is therefore valid.} 

\begin{table}
\begin{tabular}{l S[table-format=-2.3] S[table-format=4.3] S[table-format=-1.4] S[table-format=1.4] S[table-format=8.2]}
  \lsptoprule
         & {Estimate} & {SE} & {$z$} & {$p$} & {Odd.ratio} \\ 
  \midrule
(Intercept) & 0.788 & 0.539 & 1.4618 & 0.1438 & 2.20 \\ 
   part-whole & 1.003 & 0.935 & 1.0730 & 0.2833 & 2.73 \\ 
   patient & 16.778 & 2284.102 & 0.0073 & 0.9941 & 19338335.90 \\ 
   possession & -0.095 & 1.020 & -0.0934 & 0.9256 & 1.10 \\ 
   quality & -0.634 & 0.775 & -0.8186 & 0.4130 & 1.89 \\ 
   relational & 16.778 & 3956.180 & 0.0042 & 0.9966 & 19338335.90 \\ 
   \lspbottomrule
\end{tabular}
\caption{Results of the logistic regression}
        \label{tab:dq1900-relations}
\end{table}

No category of relation is a good predictor for the variable to be explained, even if we drop all possible single terms in the model. The semantic relations are therefore not a good predictor for the extraction site. 

The main difference between these corpus results and the previous ones from Frantext has to do with the relation patient: it is absent in this corpus, whereas it is the most typical semantic relation in the other three case. We cannot generalize it to all extractions out of the object with \emph{de qui}, given the results for Frantext~2000--2013. We notice a tendency when comparing Frantext 2000 and Frantext 1900, namely that there are few relationals in Frantext 1900 (and many relationals in extraction out of the subject in Frantext 2000), and this without distinction between \emph{dont} and \emph{de qui}. 

\section{\emph{Duquel} relatives in Frantext 2000--2013}


\begin{figure}
        \centering
        \includegraphics[width=\textwidth]{chapters/part2-Empirical/duquel-Frantext/relations.jpeg}
        \caption[Semantics relations in Frantext 2000 \emph{duquel} relative clauses (subject vs.\ object subextractions)]{Semantic relations in Frantext 2000 \emph{duquel} relative clauses (subject vs.\ object subextractions). See page~\pageref{ch:conf-intervals-binomial} for the confidence intervals (here three comparisons for subjects and five comparisons for objects). The percentage is given for each group (extraction out of the subject vs.\ other extraction).}
        \label{fig:duquel-relations}
\end{figure}

\figref{fig:duquel-relations} illustrates the different semantic relations between the filler and the head noun, and their proportions, comparing extraction out of the subject and out of the object. There are not enough occurrences to run a meaningful regression model (hence the very large confidence intervals in the Figure). Still, the observations we made previously for Frantext hold: there are many part-whole relations in extraction out of the subject and many patient relations in extraction out of the object. 

\section{Conclusion}

As far as semantic relations are concerned, the results for FTB and Frantext are very different. In FTB, there is a predominance of possessive relations, which is not found in Frantext. I would imagine that the genre of the texts that the respective corpora are built on plays a major role in this distinction. 

If we look only at the Frantext results, we can see a difference between subject and object. In extraction out of the subject we find many part-whole and relational relations (the latter only in Frantext 2000). In extraction out of the object we mainly find patient, quality and agent relations. In general, however, the filler itself does not seem to be a factor. 

I do not know what is responsible for the diachronic difference between Frantext 2000--2013 and Frantext 1900--1913 (the latter having almost no relationals). As for the difference between subjects and objects, I suspect that it can be linked to the type of nouns that are most often used as subject or as object: Are relationals such as \textit{father}, \textit{sister} more often used as subjects/agent while events such as \textit{management}, \textit{death} and qualities such as \textit{beauty}, \textit{colour} are more often used as objects? The scope of these corpus studies does not allow me to answer this question.
