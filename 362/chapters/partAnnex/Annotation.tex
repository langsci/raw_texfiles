\chapter{Corpus annotation guidelines}
\label{ch:corpus-methodo-annotation}
In annotating the corpus results presented in Part~\ref{part:2} of this work, we followed the guidelines below:

\begin{itemize}
    \item Verb types: 
\begin{itemize}
    \item \textit{state}: exclusively for the verb \emph{être} (`be')
    \item \textit{transitive}: for any verb with a realized direct object, also including transitive verbs whose direct object is realized as a reflexive pronoun (sometimes called `true reflexives')
    \item \textit{mediopassive}: verb with a reflexive; its prototypical form is transitive, but here the prototypical object is the grammatical subject of the verb
    \item \textit{passive}
    \item \textit{unaccusative}: verb without a realized direct object and building its \textit{participe passé} (past participle) with the auxiliary \emph{être} (`be'); this category also includes verbs with a reflexive that are neither true reflexives nor mediopassive and have a non-agentive subject
    \item \textit{unergative}: verb without a realized direct object and building its \textit{participe passé} (past participle) with the auxiliary \emph{avoir} (`have'); this category also includes verbs with a reflexive that are neither true reflexives nor mediopassive and have an agentive subject
    %\item \textbf{impersonal}
\end{itemize}
    For the sake of simplicity, we use the auxiliary to distinguish between unaccusative and unergative verbs \citep{Labelle.1992}, except for reflexives \citep[206--208]{Legendre.2003}. Notice that with this method, the number of unergative verbs may be overestimated \citep{Legendre.2003}.
    \item Restrictiveness: Whether a relative clause is restrictive or non-restrictive. We used the following criterion: the relative clause is considered appositive (restrictive = no) if the antecedent is a proper noun or if the relative clause is enclosed between commas; otherwise, it is considered restrictive (restrictive = yes). This annotation rule has obvious drawbacks, because the use of commas around a non-restrictive relative clause is not mandatory, but it enabled us to stay as objective as possible while annotating.
\end{itemize}
