In this first part of the book, I presented three main approaches to the subject island phenomenon, and to the contrast between (some?) extractions out of the subject and extractions out of the object. The traditional and still most well-known approach is a syntactic approach. I have outlined different syntactic accounts that provide slightly different predictions, especially about whether extractions out of subjects of passives fall under the subject island constraint or not, and why some languages do not display a subject island effect. I also showed that all syntactic accounts predict that French is not an exception to the subject island constraint and that extraction out of the subject is ruled out. Relative clauses with \emph{dont} may be an exception, according to \citet{Tellier.1990,Tellier.1991}. In the following parts, I will simply use ``syntactic accounts'' as an umbrella term to refer to accounts that predict an important degradation when extracting out of subjects: extraction out of the subject is not part of the grammar of French. 

There are a number of non-syntactic proposals concerning the subject island phenomenon. I presented accounts based on processing and on information structure. I identified two main trends in processing-based accounts. Looking at extraction from the point of view of memory load, shorter dependencies should be easier to process and extraction out of the subject is actually expected to be better than extraction out of the object. This is predicted by the DLT, and, to some extent, can be derived from Dependency Grammar. Another possible approach is to say that extractions out of the subject are unexpected (because subjects are complex or because there is not enough information yet when the addressee reaches the gap inside the subject) and therefore are surprising for the reader or hearer. Such surprisal-based accounts predict a degradation when extracting out of subjects. The degradation is not necessarily large, because other factors may counterbalance the effect of surprisal and ameliorate the processing difficulties caused by these subextractions.

Finally, I made a distinction between two main trends in discourse-based accounts. Previous discourse-based accounts predict a degradation when extracting out of subjects, regardless of the extraction type. As in processing-based accounts, the impact is not necessarily big. The problem with subextraction from subjects is caused by the difficulty to imagine a context in which the extraction would be felicitous, and this means that an appropriate context can improve the acceptability of the utterance. The FBC constraint that I put forth here is different. It states that the degradation in extraction out of subjects is caused by a discourse clash, but predicts extraction out of the subject to be acceptable in topicalizing constructions (e.g., relative clauses). 

Many of these accounts are mutually exclusive, but not all are. Processing-based accounts and discourse-based accounts can be compatible with each other. I will argue in this book that the FBC can best account for the data that I present, in particular for the cross-construction contrast that I found. But the FBC constraint is compatible with accounts based on relevance because being relevant or at-issue is certainly necessary in order to create a context that supports the focalization or topicalization of some phrase. The FBC constraint is also compatible with accounts based on memory load, and whenever there is no violation of the constraint, a preference for shorter dependencies is expected. I also posit that the complexity of subjects plays a role, but that this factor has a smaller impact than has been claimed by \citet{Kluender.2004}. 

% subjacency refers to cognitive constraints
% erteschik-shir refers to processing constraints