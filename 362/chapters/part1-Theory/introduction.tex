French is a language with a relatively fixed word order. Declarative sentences in French typically display a SVO order. Though in some respects it is more free than English word order, it is less free than German word order~-- to name just a few neighboring languages which otherwise have many similarities. Obviously, however, word order in any language is necessarily constrained in some ways. In this work, I will discuss some constraints on a non-declarative word order that is usually called ``extraction'', or more technically referred to as ``unbounded dependency''. In the 1960s, John R.\ ``Haj'' \citeauthor{Ross.1967} came up with a very pictorial and poetic word to refer to this kind of constraints, calling them ``islands''. The idea underlying this concept is that some structures are units out of which it is very difficult, if not utterly impossible, for constituents to escape. Islands are nowadays one of the most important concepts in syntax. The questions that I am addressing in this work are the following: Do the constraints on locality really have a syntactic origin? Are these constraints universal, i.e.\ cross-linguistically valid, and if so, what does this reveal about human language?

In this work, the discussion mostly concentrates on subject islands, that is on non-declarative positions of elements that syntactically depend on the subject and may or may not have the possibility to appear outside of the subject. The empirical studies that I present are on French. But the implications of what I discuss go beyond the scope of French, or even of subject islands. I will argue that the mechanisms at play in extractions out of the subject in French have deep roots in our cognitive capacities and in the way we as humans treat information. Therefore it is not specific to a particular language or to syntactic subjects. 

\section{Some definitions}

Before we turn to the topic of subextraction from subjects, it is necessary to define some core concepts used in this work. In ``extractions'', one element occupies a position at the leftmost edge of a clause, as in (\ref{ex:extraction-def}), where the relevant element is italicized. In this respect, extraction differs from scrambling, which is the free permutation of (verbal) dependents inside a clause, as shown in the German example (\ref{ex:scrambling-def}) and in the French example (\ref{ex:scrambling-fr}).

\eal \label{ex:extraction-def}
\ex SVO declarative: \\ I recently saw \WinckelEmph{a woman} with a parrot on her shoulder.
\ex extraction: \\ \WinckelEmph{Who} did you recently see with a parrot on her shoulder?
\ex extraction: \\ \WinckelEmph{Who} did Mark pretend he saw with a parrot on her shoulder?
\zl 

\eal \label{ex:scrambling-def}
\ex[]{\gll Gestern hielt Kristin eine bewegende Rede bei der Demo.\\
yesterday held Kristin a moving speech at the protest\\}
\ex[]{\gll Gestern hielt Kristin bei der Demo eine bewegende Rede.\\
yesterday held Kristin at the protest a moving speech \\}
\ex[]{\gll Gestern hielt eine bewegende Rede Kristin bei der Demo.\\
yesterday held a moving speech Kristin at the protest\\}
\ex[]{\gll Gestern hielt bei der Demo Kristin eine bewegende Rede.\\
yesterday held at the protest Kristin a moving speech\\}
\ex[]{\gll Gestern hielt eine bewegende Rede bei der Demo Kristin.\\
yesterday held a moving speech at the protest Kristin\\}
\glt `Yesterday, Kristin gave a moving speech during the protest.'
\zl 

\eal  \label{ex:scrambling-fr}
\ex[]{\gll (Hier) Capucine a (hier) donné (hier) un livre (hier) à son fils (hier).\\ 
yesterday Capucine has yesterday given yesterday a book yesterday at her son yesterday\\ 
\glt `Capucine gave a book to her son (yesterday).'}
\ex[]{\gll (Hier) Capucine a (hier) donné (hier) à son fils (hier) un livre (hier).\\ 
yesterday Capucine has yesterday given yesterday at her son yesterday a book yesterday\\ 
\glt `Capucine gave her son a book (yesterday).}
\zl 

Extraposition, i.e.\ the non-canonical position of an element at the rightmost edge of a clause, is also treated as extraction by some scholars, but it is bound to a clause and cannot form long-distance dependencies \citep{Gueron.1980}, that is why I will not talk about these cases in the present work. Some other phenomena are sometimes treated as ``movement'', and therefore as a kind of extraction (e.g.\ the canonical position of verbs in main clauses in German), but again, it is not what I will call ``extraction'' in the following sections.

I will refer to the canonical position of the extracted element as the ``gap'', and identify it in the examples with an underscore (\trace{}). This notation is common in linguistic works, and very practical in helping the reader identify the kind of dependency that is meant without too much explanation, especially in very complicated examples. Additionally, the gap will be coindexed with the extracted element, which helps identify the actual extraction at play, especially when there are multiple extractions.\footnote{Most of the time, the coindexation also means that both the missing element and the extracted element refer to the same semantic variable. However, as we will see later, complementizers are not per se extracted. Nevertheless, I will use coindexation in examples involving complementizers for the sake of readability.} However, this annotation should not be interpreted as presupposing that extraction of a constituent leaves anything at its canonical position. Traditional accounts of generative grammar assume that the extracted element leaves behind a ``trace'' at the position where it is taken to be base-generated in the deep structure. Similarly, in HPSG, extraction is sometimes analyzed with empty categories, and thus the assumption is that there is an empty element at the canonical position of the extracted element. Nevertheless, I wish to make clear that it is not what the notation means here. At the end of this work, I propose an HPSG analysis that does not use empty categories \citep{Sag.1994.Fodor,Sag.1994.Godard,Sag.2007,Sag.2010}.

I will refer to the position of the extracted element as the ``filler''. As said previously, this position is at the leftmost position of a clause. The relation between the filler and the gap will be termed ``filler-gap dependency''. This term will sometimes be applied to structures in which the filler is not realized. What ``filler-gap dependency'' means in these cases is that the sequence of words provides cues to the addressee that they have to identify a missing element, i.e.\ a gap, in the rest of the utterance. For example, in (\ref{ex:filler-gap-dependency}), the presence of the word \emph{you} signals the beginning of a relative clause, and the reader will start looking for the gap, even though there is no actual filler in the relative clause.\footnote{Notice that \emph{the woman} is not the filler, but the antecedent of the relative clause. These details will be discussed extensively in the rest of the book, and especially in the formal analysis.}

\ea the woman you ... saw with a parrot on her shoulder \label{ex:filler-gap-dependency}
\z  

I will make a distinction between ``short-distance'' and ``long-distance'' filler-gap dependencies. Short-distance dependencies do not cross the boundary of the clause in which the gap is directly situated. For example, (\ref{ex:extraction-sdd}) is a short-distance dependency. In long-distance dependencies on the other hand, the dependency crosses one or more clause boundaries. Example (\ref{ex:extraction-ldd}) is a long-distance dependency, because the filler is not at the leftmost position of the embedded clause, but at the leftmost position of the matrix clause. By definition, long-distance dependencies hence involve at least a matrix and an embedded clause. 

\eal 
\ex[]{Who$_i$ did you recently see~\trace{}$_i$ with a parrot on her shoulder? \label{ex:extraction-sdd}}
\ex[]{Who$_i$ did Mark pretend [that he saw~\trace{}$_i$ with a parrot on her shoulder]? \label{ex:extraction-ldd}}
\zl 

When extraction takes place out of a phrase, I will refer to it as ``subextraction''. For example, (\ref{ex:subextraction-def}) is subextraction from a direct object. The main topic of this work is subextraction out of subjects, i.e. extractions in which only part of the NP subject, part of the verbal subject or part of the sentential subject is extracted.

\ea[]{the parrot [whom$_i$ I saw [the owner of~\trace{}$_i$]$_{\text{NP}}$ running away]}
\label{ex:subextraction-def}
\z 

\section{Extractions in French}
\begin{sloppypar}
Three extraction constructions are discussed in depth in this work: relative clauses, interrogatives, and \emph{it}-clefts. In French, relative clauses are noun modifiers that are introduced either by a relative word as in (\ref{ex:relative-clause-relative-word}) or by a filler phrase containing a relative word as in (\ref{ex:relative-clause-relative-phrase}). 
\end{sloppypar}

\eal 
\ex canonical word order: \\ 
\gll Nous avons parlé \WinckelEmph{de} \WinckelEmph{Gaetan} hier.\\
we have talked of Gaetan yesterday\\
\glt `We talked about Gaetan yesterday'
\ex relative clause: \\
\gll Gaetan, [dont$_i$ nous avons parlé~\trace{}$_i$ hier]\\
Gaetan of.which we have talked yesterday\\
\glt `Gaetan, whom we talked about yesterday'
\label{ex:relative-clause-relative-word}
\zl 

\eal
\ex canonical word order: \\
\gll Nous avons parlé \WinckelEmph{du} \WinckelEmph{perroquet} \WinckelEmph{d'} \WinckelEmph{Agate} hier.\\
we have talked of.the parrot of Agate yesterday\\
\glt `We talked about Agate's parrot yesterday'
\ex relative clause: \\\
\gll Agate, [[du perroquet de qui]$_i$ nous avons parlé~\trace{}$_i$ hier]\\
Agate of.the parrot of who we have talked yesterday\\
\glt `Agate, whose parrot we talked about yesterday'
\label{ex:relative-clause-relative-phrase}
\zl 

The use of a relative word to build a relative clause is common in European languages, though not very common cross-linguistically. \citet{wals-s8} list only 12 to 13 languages using this morphosyntactic strategy. This, however, is the only option in French. The gap strategy exemplified in (\ref{ex:filler-gap-dependency}) is not available:

\ea[*]{\gll le perroquet [tu as vu~\trace{} hier]\\
the parrot you have seen yesterday\\}
\z 

Clefts in French are either presentational (and similar to \emph{here}-clefts in English) or focalizing (and similar to \emph{it}-clefts in English). I will refer to them as \emph{c'est}-clefts. Both are constructions following the pattern [\emph{ce} (`it') + copula + XP + S]. The XP is either the referent introduced in presentationals, like (\ref{ex:clefts-presentational}), or the focused element in focalizing \emph{c'est}-clefts, as in (\ref{ex:clefts-focus}).

\eal 
\ex[]{\gll C' est le perroquet [dont$_i$ nous avons parlé~\trace{}$_i$ hier].\\
it is the parrot of.which we have talked yesterday\\
(\textit{pointing to the parrot})
\glt `Here is the parrot whom we talked about yesterday.'}
\label{ex:clefts-presentational}
\ex[]{\gll C' est du perroquet d' Agate [que$_i$ nous avons parlé~\trace{}$_i$ hier] (, pas du perroquet de Gaetan).\\
it is of.the parrot of Agate that we have talked yesterday {} not of.the parrot of Gaetan\\
\glt `It's Agate's parrot whom we talked about yesterday (not Gaetan's parrot).'}
\label{ex:clefts-focus}
\zl 

The last element of the pattern (S) is the one showing extraction, with a relative word or a relative phrase at its left edge. It is hence very similar to a relative clause, but I will argue in my HPSG analysis that it is not always one. I leave aside constructions with \emph{il y a}, like (\ref{ex:cleft-ilya}), which are often referred to as clefts as well in the literature \citep{Lambrecht.1994,Doetjes.2004,Karssenberg.2018} and also involve an extraction. 

\ea \citep[517]{Karssenberg.2018} \nopagebreak \\
\gll Il y a des enfants [qui$_i$~\trace{}$_i$ aiment le fromage].\\
it there has \textsc{det} children who like the cheese\\
\glt `There are some children who like cheese.'
\label{ex:cleft-ilya}
\z 

I also leave aside \emph{wh}-clefts (\ref{ex:wh-cleft-eng}) and their French counterparts (\ref{ex:wh-cleft-fr}), which are different in terms of syntax and function.

\eal 
\ex[]{What the woman had on her shoulder was a parrot. \label{ex:wh-cleft-eng}}
\ex[]{\gll Ce que mes enfants aiment, c' est le fromage.\\ 
it that my children like it is the cheese\\ 
\glt `What my children like is cheese.'
\label{ex:wh-cleft-fr}}
\zl 

Whereas extraction in relative clauses and \emph{it}-clefts is mandatory, it is optional in interrogatives \citep{Obenauer.1976}, but see Section~\ref{ch:exp11} for the potential pragmatic factors implied by extracted vs.\ in situ \emph{wh}-words.

\eal 
\ex[]{\gll [De qui]$_i$ avez - vous parlé~\trace{}$_i$?\\
of who have {} you talked\\
\glt `Whom did you talk about?'}
\ex[]{\gll Vous avez parlé \WinckelEmph{de} \WinckelEmph{qui}?\\
you have talked about who\\
\glt `You talked about whom?'}
\zl 

Further constructions involving extraction are not discussed in this work, even though of course they are assumed to be affected by the island constraints, just like any extraction. For example, French also has complement fronting (\ref{ex:complement-fronting}), exclamatives (\ref{ex:exclamative}), comparative correlatives (\ref{ex:comparative}), topicalization (\ref{ex:topicalization}), concessives (\ref{ex:concessive}) and so-called \emph{tough} constructions (\ref{ex:tough-cx-fr}).%
\footnote{For an HPSG analysis of complement fronting in French, see \citet{Abeille.2008.NP-preposing}. For a typology and HPSG analysis of exclamatives in French, see \citet{Marandin.2008}. For an HPSG analysis of comparative correlatives in English, French and other languages, see \citet{Abeille.2008.Comparative-correlatives}. See also \citet[40--42]{Godard.1988} and \citet[106--107]{Cinque.1990} on some infinitival complements in French that involve an unbounded dependency without extraction in French (and an overview of this same construction in other Romance languages in \citealt{Mensching.2000}):
\ea
    \citep[76]{Mensching.2000}\\ \nopagebreak
    \gll ~ Qui crois - tu être intelligent?\\
         ~  who think {} you be\textsc{.inf} intelligent\\
    \glt ~ `Who do you believe to be intelligent?'
    \ex[*]{ \gll Tu crois Richard être intelligent.\\
    you believe Richard be\textsc{.inf} intelligent\\
    \glt `You believe Richard to be intelligent.'}
\z}

\eal 
\ex\citep[306]{Abeille.2008.NP-preposing} \\ \nopagebreak
\gll [Huit ans]$_i$ je devais avoir~\trace{}$_i$.\\
eight years I must\textsc{.past} have\textsc{.inf}\\
\glt `Eight years, I must have had.'
\label{ex:complement-fronting}
\ex \citep[438]{Marandin.2008}\\
\gll [Quel chapeau]$_i$ il portait~\trace{}$_i$~!\\ 
what hat he wore\\ 
\glt `What a hat he was wearing!'
\label{ex:exclamative}
\ex\citep[1148]{Abeille.2008.Comparative-correlatives} \\
\gll Plus je lis~\trace{}, plus je comprends~\trace{}.\\ 
more I read more I understand\\ 
\glt `The more I read, the more I understand.'
\label{ex:comparative}
\ex\gll [De tout cela]$_i$, nous reparlerons~\trace{}$_i$ demain.\\ 
of all that we talk\textsc{.future} tomorrow\\ 
\glt `About all this, we will talk tomorrow.'
\label{ex:topicalization}
\ex\gll Aussi difficile [que$_i$ ce problème soit~\trace{}$_i$], tu pourras le résoudre.\\ 
as difficult that this problem be you can\textsc{.future} \textsc{masc.acc} resolve\\ 
\glt `As difficult as this problem may be, you'll be able to solve it.'
\label{ex:concessive}
\ex\gll une nouvelle$_i$ [difficile à croire~\trace{}$_i$]$_{\text{AdjP}}$\\ 
a news hard at believe\textsc{.inf}\\ 
\glt `news (that is) hard to believe'
\label{ex:tough-cx-fr}
\zl 

%English has further constructions involving extractions, like topicalizations (\ref{ex:eng-topicalization}) and dangling modifiers (\ref{ex:eng-dangling-modifiers}). Other structures display unbounded dependencies without extraction, like Missing object constructions (\ref{ex:eng-missing-obkect-cx}).

An overview of all unbounded dependencies can be found in \citet[Section~1.2.2]{Godard.1988} for French and in \citet[Chapter~2]{Sag.2010} and \citet[Chapter~2]{Chaves.2020.UDC} for English.

%\eal 
%\ex[]{[That proposal]$_i$ we discussed~\trace{}$_i$ at length. \citep[35]{Chaves.2020.UDC}} % topicalization
%\label{ex:eng-topicalization}
%\ex[]{Difficult$_i$ though the puzzle was~\trace{}$_i$, Robin managed to solve it. \citep[43]{Chaves.2020.UDC}} % dangling modifier
%\label{ex:eng-dangling-modifiers}
%\ex[]{Kin$_i$ is [easy to please~\trace{}$_i$]. \citep[44]{Chaves.2020.UDC}} % Missing-object construction
%\label{ex:eng-missing-obkect-cx}
%\zl 

Notice that combinations are of course possible: a sentence may display several extractions, and even extraction out of extracted elements~-- although this is subject to constraints (see below). Also, one single filler (or equivalent) may correspond to several gaps, see example (\ref{ex:multiple-gap}).  

One filler may be coindexed with more than one gap (which can, but must not, be coindexed with each other). 

\ea \citep[7]{Chaves.2020.UDC}\\
There's no engine [[which]$_i$ Geoff can't disassemble \trace{}$_i$, clean \trace{}$_i$, and put \trace{}$_i$ back together without disparaging \trace{}$_i$ or complaining about \trace{}$_i$]. 
\label{ex:multiple-gap}
\z 

\section{Some remarks on subject extraction}

The subject is extracted in French with the relative word \emph{qui}. Subject-\emph{qui} and \emph{que} are two relative words that are traditionally considered complementizers rather than pronouns. On page~\pageref{p:qui-que-are-complementizers}, I come back to the distinction between complementizers and pronouns.

Because French is an SVO language with a relatively fixed subject-verb order (subject-verb inversion is only allowed in very specific cases, see \citealt{Bonami.2001}), the filler in short-distance dependencies is located just before the subject. When the \textit{wh}-phrase is the subject, it is impossible to know whether it is extracted (i.e., the \textit{wh}-phrase is followed by a gap) or in situ. In the context of this book, I assume, however, that the subject undergoes extraction, as illustrated by (\ref{ex:relative-clause-subject-extracted}), for several reasons.

\ea[]{\gll les enfants [qui$_i$~\trace{}$_i$ aiment le fromage]\\
the children who like the cheese\\
\glt `children who like cheese'}
\label{ex:relative-clause-subject-extracted}
\z 

I mention below some arguments that apply to French, but the interested reader should refer to \citet[28--32]{Chaves.2020.UDC} for further cross-linguistic arguments.

\subsection{Extraction of the subject in relative clauses}

First, there is no Indo-European language with postnominal relative clauses with relative words in situ.\footnote{For \citet{Downing.1978}, the fact that relative pronouns are never realized in situ belongs to the universal properties of relative clauses. However, \citet[37]{Vries.2002} mentions a few languages from West Africa in which this rule is not true. If I correctly understand their findings, both are talking about relative words (in contrast to resumptives) in general (e.g.\ not about relative pronouns in contrast to complementizers).}
Why would subjects be an exception to this rule? Hence I assume that extraction in relative clauses is obligatory.

The second argument comes from the so-called \emph{que-qui}-rule \citep[see a.o.][]{Kayne.1974.1975,Kayne.1976,Pesetsky.1982,Koopman.2014}. The relative word \emph{qui} is used to relativize the subject, as in (\ref{ex:relative-clause-subject-extracted}). But in long-distance dependencies with an intervening complementizer \emph{que}, it becomes obvious that \emph{qui} here is only a variant of \emph{que} that appears before an extracted subject. In example (\ref{ex:relative-clause-subject-extracted-ldd}), the relative word is \emph{que} (and not \emph{qui}, even though the subject is relativized), while \emph{qui} can only be introducing the sentential complement of \emph{pense} (`think'). Sentential complements are introduced by \emph{que} if there is no subject extraction involved.

\eal
\ex[]{\gll Je pense que tes enfants aiment le fromage.\\
I think that your children like the cheese\\
\glt `I think that your children like cheese'}
\ex[]{\gll les enfants [que$_i$ je pense [qui~\trace{}$_i$ aiment le fromage]]\\
the children that I think that like the cheese\\
\glt `the children who I think that (they) like cheese'}
\label{ex:relative-clause-subject-extracted-ldd}
\zl 

Because \emph{qui} is the relative word in short-distance dependencies, we must assume that these dependencies involve extraction. The \emph{que-qui} rule will be described and analyzed in more detail in Section~\ref{ch:hpsg-extraction}.  

\subsection{Extraction of the subject in interrogatives}

One type of French interrogatives is formed with \emph{est-ce que} (lit.\ `is it that'), which is analyzed by \citet{Abeille.2012} as a complementizer. Interrogatives in \emph{est-ce que} cannot have their interrogative word in situ, as the following example shows:

\eal 
\ex[]{\gll Tu vas où?\\
you go where\\}
\ex[]{\gll Où$_i$ est - ce que tu vas~\trace{}$_i$?\\
where is {} it that you go\\}
\ex[*]{\gll Est - ce que tu vas où?\\
is {} it that you go where\\}
\ex[*]{\gll Tu vas où$_i$ est - ce (que)?\\
you go where is {} it that\\}
\glt `Where are you going?'
\zl 

Since interrogatives with \emph{est-ce que} can be used to question the subject, there must be extraction in (\ref{ex:interrogative-estceque-sujet}). And if extraction is possible in these interrogatives, it is presumably possible in interrogatives in general. 

\ea[]{Qui$_i$ est - ce qui~\trace{}$_i$ aime le fromage?\\
who is {} it that likes the cheese\\
\glt `Who likes cheese?'}
\label{ex:interrogative-estceque-sujet}
\z 

Since in situ questions are allowed in French, it follows that they must be possible to question the subject. Therefore, we must assume that the example (\ref{ex:interrogative-subject-ambig}) is syntactically ambiguous. 

\eal \label{ex:interrogative-subject-ambig}
\ex[]{\gll Qui aime le fromage?\\
who likes the cheese\\}
\ex[]{\gll Qui$_i$~\trace{}$_i$ aime le fromage?\\
who likes the cheese\\}
\glt `Who likes cheese?'
\zl 

\section{A definition of ``islands''}

There are lots of ``islands'' or ``syntactic islands'' identified in the literature -- rightly or wrongly. I invite the reader who wants a broader overview of all these island types to consult \citet{Chaves.2020.UDC}. In the present book, I focus exclusively on a particular type called subject island. This being said, I believe that many of the findings I present have some significance for the general theory on islands. Limiting the scope of this work to a particular case of islands was necessary in order to account for its complexities and nuances. Even so, and as the reader will realize, I am far from exhausting the question of extractions out of the subject.

I will first define what I mean by islands, before turning to the particular case investigated in this work.

\subsection{The notion of ``islands'' over time}

It has been noticed that extraction cannot cross certain boundaries, even though there is virtually no limit to the linear length of filler-gap dependencies. For example, in coordinations, extraction of one of the conjuncts is impossible, as illustrated by (\ref{ex:coordination-island}).

\eal \label{ex:coordination-island}
\ex[*]{Who did [your father and~\trace{}$_i$] buy a parrot?}
\ex[*]{\gll Qu$_i$' est - ce que tes enfants aiment [le fromage et~\trace{}$_i$]?\\
what is {} it that your children like the cheese and\\
\glt `What do your children like cheese and?'}
\zl

This led linguists, starting with the seminal work of \citet{Ross.1967}\footnote{An anonymous reviewer traces this discussion back to \citet{Chomsky.1964}.}, to postulate syntactic constraints on extraction, called ``islands''. The constraint illustrated by example (\ref{ex:coordination-island}) is referred to as the Coordinate Structure Constraint.

There is currently still a debate regarding island effects in constructions without extraction, especially in languages in which \textit{wh}-questions do not necessarily require filler-gap dependencies (like French), or never involve filler-gap dependencies (like Mandarin Chinese, see \citealt{Huang.1982,Aoun.1993,Lu.2020} a.m.o.). Although in the present book I focus on islands in the context of filler-gap dependencies, I come back to the question of \textit{wh}-in situ interrogatives in Section~\ref{ch:exp11}.

In their first definition by \citeauthor{Ross.1967}, island constraints were syntactic in nature, and a violation of island constraints was considered to lead to ungrammaticality.\footnote{Notice that, for \citet{Ross.1967}, islands did not necessarily reflect universal constraints. In his original definition, islands could be language-specific.} Even though the discussion about islands mostly originates from observations concerning extraction constructions, islandhood did not only play a role in extraction, but was ``the maximal areas in which syntactic process of a designated sort could apply'' \citep[258]{Ross.1987}. Very soon after however, \citet{Erteschik-Shir.1973} proposed that islands are not caused by syntactic factors, but have a functional (discourse-based) explanation. This proposal has been followed by processing \citep{Kluender.1998} or semantic accounts \citep{Szabolcsi.1990}. This island constraints are no longer considered to be only a matter of syntax. Furthermore, the lively discussion of the phenomenon brought to the fore novel data from various languages, including exceptions to some of \citeauthor{Ross.1967}'s original islands, and to some other islands added later to list of island structures. \citet{Cinque.1990} and \citet{Rizzi.1990} introduced a distinction between ``strong'' and ``weak'' islands \citep{Kluender.1998,Szabolcsi.2006}. In a nutshell, strong islands correspond to the original definition of islands that block any kind of extraction. Weak islands, on the other hand, are cases in which only the extraction of some constituents is ruled out.\footnote{For \citet{Cinque.1990}, the distinction between strong and weak islands was that extraction (without a resumptive pronoun) was never allowed out of strong islands while extraction of PPs (but not of DPs) was allowed out of weak islands. Another definition can be found in \citet{Huang.1982} and \citet{Chomsky.1986}, who draw the line between argument extraction (weak islands allow the extraction of arguments, but strong islands do not) and adjunct extraction (extraction of adjuncts is never allowed out of any island). In general, though, scholars agree that strong island are ``absolute'' (they are always valid) while weak islands are ``selective'' (some elements are sensitive to the island and some are not). See \citet{Szabolcsi.2006} for an overview of the debate around weak islands.} Some also consider violations of weak islands to be more acceptable than violations of strong islands, a conception that diverges a lot from the original definition of islands. \citet{Almeida.2014} even talks about ``subliminal islands''~-- (weak) island configurations that do not lead to unacceptability in a given language but still display a disadvantage when compared to another similar non-island structure. However, multiplying the degrees of islandhood with notions like ``weak'' or ``subliminal'' islands is a move away from what motivated the notion of islands in the first place, which is that some configurations are not possible even though nothing seems to prevent them. The issue is not the preference for one configuration over another, but the fact that not everything is possible in a given language and that some of these impossible things are particular to extractions (and perhaps also to binding).

\subsection{Definition adopted in this work}

In this work, I adopt a definition of islands closer to the one formulated by \citeauthor{Ross.1967}, but without postulating that islands are necessarily syntactic in origin. A more complete defense of this definition can be found in \citet{Liu.Y.2022.Structural}.

An island is a extraction construction whose constructs, in some systemic manner, are considered unacceptable by most speakers. Systematicity is an important factor in this definition: it is not an isolated example, rather it is possible to identify a group of examples that are similar in form and similarly unacceptable. Extraction is an important factor as well: it must be possible to construct acceptable examples that differ from the unacceptable ones only by the absence of extraction. If this condition is not met, the unacceptability is not due to the extraction, but can come from any feature of the sentence, and it is not caused by the existence of an island \citep[77--78]{Chaves.2020.UDC}.

Thus, stating that subjects are islands means that examples involving extractions out of a subject are systematically considered unacceptable by most speakers, even though examples with a close propositional and lexical content with no extraction would be judged acceptable. 

Islands can be language-specific, although most researchers implicitly (and sometimes explicitly) take them to be universal constraints.

\subsection{Identifying islands through empirical work}

The main issue, therefore, is to find out what makes it possible to identify an island. It is generally assumed that the ungrammaticality of a sentence goes hand in hand with its unacceptability. This is why great importance is given in this work to experimental data, under the assumption that the speakers of a language are the real experts as to what structures are allowed or not in a language. There are of course well-known exceptions in experimental linguistics, such as grammatical illusions: sentences that many speakers spontaneously perceive as acceptable, even though they are ungrammatical (e.g., from an agreement point of view in the case of agreement attraction). It seems to me, however, that a sentence cannot be both an island violation and a grammatical illusion, since by definition, islands are configurations that are systematically ill-formed even though nothing seems to prevent them from being well-formed. In grammatical illusions, the reason for their ill-formedness is easy to identify (so easy that we judge it surprising that speakers accept them).

From the experimental point of view, \citet{Sprouse.2007.PhD} and subsequent work by \citeauthor{Sprouse.2007.PhD} and his colleagues have proposed a superadditivity design as a diagnostic for islandhood. I will describe this design in detail in Section~\ref{ch:exp-methodo}, but the idea, in a nutshell, is to construct and test minimal pairs comparing island structures with non-island structures, in extraction and non-extraction conditions. Participants should give lower ratings to sentences with an extraction in an island configuration than to sentences in a non-island configuration or without extraction. Although very useful, this diagnostic only applies to experimental data. I will indeed use \citegen{Sprouse.2007.PhD} superadditivity design in a series of experiments that I present in Part~\ref{part:2} and describe the superadditivity effects observed in these experiments as ``island effects''. Of course, this experimental method can only deliver a good diagnostic for islandhood if the investigator was successful in eliminating the impact of unrelated factors. Thus a superadditivity effect is not necessarily synonymous with islandhood. 

I should add that we expect native speakers not to produce sentences that violate an island constraint (or very marginally, as performance errors). This is why, in addition to experimental data, I look at islands in speech production. In Part~\ref{part:2}, I present a series of corpus studies on well-edited written French (newspaper articles and texts from French literature), in which I expect errors to be rare.

\section{The structure of this book}

Part~\ref{part:1} of the book introduces the previous theoretical approaches to what is generally known as ``subject islands''. In Part~\ref{part:2}, we move on to the empirical data on French: I present a series of corpus studies and experiments that help understand the phenomenon. Finally, Part~\ref{part:3} proposes a formal HPSG analysis of extractions, short- and long-distance dependencies and extraction out of the subject in French.

\begin{description}
\item[Part I:] In Chapter~\ref{ch:syntax}, I present the three general kinds of explanations based on syntax. I explain that, for traditional syntactic accounts, extraction out of the subject is not only dispreferred but is ruled out by syntax. Scholars in the syntactic tradition do not agree on the reasons that cause subject islands, but they agree that these reasons are universal and are based on innate properties of language. Some syntactic accounts treat subject islands and adjunct islands (the ban on subextraction out of adjuncts) as a single phenomenon \citep[e.g.][]{Huang.1982} and claim that only extraction out of complements is acceptable. In other syntactic accounts, subjects are considered special because they are in the Specifier position, which is already at the edge of the phrase \citep[e.g.][]{Chomsky.2008}. Finally, a third kind of syntactic account proposes that subjects are the result of the movement of some elements to the subject position, and that any movement out of a moved element is ruled out \citep[e.g.][]{Uriagereka.2012}. I also explore the French data and previous proposals of syntactic analyses for French subject islands.  In Chapter~\ref{ch:processing-accounts}, I present accounts based on processing. In general, work on the processing of filler-gap dependencies shows that it is easier to parse shorter dependencies between the extracted phrase and the gap. If this is the case, then subextraction from (preverbal) subjects should be easier to process than subextraction from objects because the distance is shorter. But there have also been processing-based accounts of subject islands: \citet{Kluender.2004} have theorized that complex subjects are dispreferred for processing reasons and that therefore subextraction is unexpected. In Chapter~\ref{ch:discourse}, I discuss discourse-based accounts of islands in general. In a nutshell, these accounts assume that extraction makes an element more salient, and propose that some constituents cannot be made salient and thus cannot be extracted \citep[e.g.][]{Erteschik-Shir.1973}. I then present a new proposal based on information structure, called the Focus-Background Conflict (FBC) constraint, which states that part of a backgrounded constituent cannot be focalized. Contrary to previous discourse-based accounts, this constraint predicts that extraction out of the subject will show cross-constructions differences based on the function of the construction. Some constructions are indeed focalizing (like interrogatives and \emph{it}-clefts) and some are not (like relative clauses). 

\item[Part II:]
After this, I present eight corpus studies and 16 experiments on extraction out of the subject. The corpus studies are based on two different corpora of written French: the French Treebank \citep{Abeille.2003.FTB,Abeille.2019.FTB} and Frantext (\url{https://www.frantext.fr/}). Most of the experiments used acceptability judgment tasks, one is a speeded acceptability judgment task, one is a self-paced reading task and one is an eye tracking experiment. The empirical data on relative clauses confirm \citegen{Godard.1988} intuitions: relativization out of the subject is very frequent and generally accepted by native speakers. In the corpus we observe a clear distinction between subextraction out of subjects in relatives and in interrogatives. In fact, there is not a single example of extraction out of the subject in interrogatives, a result supported by the experimental data on interrogatives in which participants rejected extractions out of the subject. One experiment on \emph{c'est}-clefts shows more nuanced results, but extractions out of the subject seem problematic there as well. Two experiments investigate extraction out of infinitival subjects. Subextractions from infinitival subjects received surprisingly high acceptability judgments, even though they were dispreferred compared to extractions out of infinitival complements. Comparing this empirical evidence with the different accounts on subject islands, I conclude that the FBC constraint explains the data best, especially the strong distinction between relative clauses on one hand and interrogatives and \emph{it}-clefts on the other hand.

\item[Part III:]
Part~\ref{part:3} offers an HPSG analysis of the FBC constraint. In Chapter~\ref{ch:fbc-generals}, I first discuss the FBC constraint's implications in some detail. In Chapter~\ref{ch:hpsg-fragment}, I describe a small fragment for the analysis of French sentences in HPSG, explaining how syntax, semantics and information structure are represented in HPSG and interact with each other. I then present the three main constructions involving extraction that I explored in the empirical parts: interrogatives, relative clauses and \emph{c'est}-clefts. In Chapter~\ref{ch:hpsg-fbc}, I formalize the FBC constraint within my HPSG fragment for French. Finally, in Chapter~\ref{ch:hpsg-sent-subj}, I present the analysis of verbal and sentential subjects and of subextractions out of verbal and sentential subjects. 


\end{description}
