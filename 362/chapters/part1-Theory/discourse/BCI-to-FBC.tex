\section[General principles: from Erteschik-Shir (1973) to Goldberg's BIC]{General principles: from Erteschik-Shir's dominance constraint on extraction to Goldberg's ``Backgrounded Constituents are Islands''}

In this section, I will present some accounts based on the discourse function of extractions that offer an analysis of many different islands. The next section will be devoted to the subject island in particular.
 
\subsection{The Focus approach}

\citeauthor{Erteschik-Shir.1973} proposed another alternative to syntactic accounts of islands \citep{Erteschik-Shir.1973,Erteschik-Shir.1997,Erteschik-Shir.2006}. Her proposal is based on information structure\footnote{Which she calls the f(ocus)-structure since \citet{Erteschik-Shir.1997}.} and maintains as a general principle that extraction can only occur out of the ``potential focus domain''. The focus domain consists of the focus and the elements it c-commands (including traces). In her early works, this concept was defined as the semantically dominant phrase or clause.\footnote{She defines ``semantic dominance'' as such: ``A clause or phrase is semantically dominant if it is not presupposed and does not have contextual reference.'' \citep[22]{Erteschik-Shir.1973}.
In later works \citep[e.g.][]{Erteschik-Shir.2006}, she says that the two formulations are equivalent.}
Her original constraint is reproduced in (\ref{ex:rule-ES}):

\ea The dominance condition on extraction \citep[27]{Erteschik-Shir.1973}:\\ 
Extraction can occur only out of clauses or phrases which can be considered dominant in some context. 
\label{ex:rule-ES}
\z 

\citeauthor{Erteschik-Shir.1973} uses the liar test in order to identify the potential focus domain in a specific context. Example (\ref{ex:liar-test-rhinoceros}) makes clear that the context is very important in determining the focus domain: in both examples, the liar test targets the complement of the noun in the NP [\emph{a book about Nixon}]. However, the test shows that \emph{about Nixon} is in the focus domain only in (\ref{ex:liar-test-rhinoceros-wrote}), not in (\ref{ex:liar-test-rhinoceros-destroy}).

\eal\label{ex:liar-test-rhinoceros}
\ex[]{Sam said: John wrote a book about Nixon. Which is a lie -- it was about a rhinoceros.} \label{ex:liar-test-rhinoceros-wrote}
\ex[]{Sam said: John destroyed a book about Nixon. \#{}Which is a lie -- it was about a rhinoceros.} \label{ex:liar-test-rhinoceros-destroy}
\zl 
% example cited in Erteschik-shir 1981
%Erteschik-Shir, N.: 1981, ‘On Extraction from Noun Phrases (picture noun phrases)', Annali della Scoula Normale Superiore di Pisa, special issue, Pisa, Italy.

This distinction explains the contrast in (\ref{ex:book-about-write-destroy}): in (\ref{ex:book-about-write}), the extracted element belongs to the potential focus domain, and can therefore be extracted following (\ref{ex:rule-ES}), whereas in (\ref{ex:book-about-destroy}), it does not.

\begin{exe}
\ex \citep[272]{Bach.1976}
\label{ex:book-about-write-destroy}
\begin{xlist}
\ex[]{What did they write a book about? \label{ex:book-about-write}}
\ex[*]{What did they destroy a book about? \label{ex:book-about-destroy}}
\end{xlist}
\end{exe}

Context also plays a role in the fact that acceptability varies for islands: ``the positive response of informants is conditional on their ability to contextualize in such a way that the clause from which extraction has occurred is interpreted as a focus domain" \citep[320]{Erteschik-Shir.2006}.

\subsection{The Topic approach}\largerpage[-2]

In contrast to \citeauthor{Erteschik-Shir.1973}, \citet{Kuno.1987} proposes an account of islands based on Topic. He notices that \citeauthor{Erteschik-Shir.1973}'s proposal is not able to account for the contrast between (\ref{ex:marys-portrait-a-portrait}) and (\ref{ex:marys-portrait}). The context remains the same (hence with the same potential focus domain), and the liar test gives similar results but extracting \emph{the actress} out of the NP [\emph{Mary's portrait of this actress}] is not felicitous. The reason cannot be the presence of the  genitive \emph{Mary's} alone, because the extraction in (\ref{ex:marys-version}) is felicitous. 
\pagebreak
\begin{exe}
\ex \citep[13]{Kuno.1987} 
\label{ex:marys-portrait-version}
\begin{xlist}
\ex[]{Yesterday, I met the actress who I had bought a portrait of. \label{ex:marys-portrait-a-portrait}}
\ex[*]{Yesterday, I met the actress who I had bought Mary's portrait of. \label{ex:marys-portrait}}
\ex[]{This is the story that I haven't been able to get Mary's version of.}
\label{ex:marys-version}
\end{xlist}
\end{exe}

\citegen{Kuno.1987} proposal is that topics, and not dominant (or focused) elements, are extracted. His definition of topichood is somewhat broader than the one I gave previously, because in his proposal not only utterances but NPs can have a topic as well. 
% Kuno also says there's a problem with this definition of topic and isn't very happy with it: no test (continuation topic not very conclusive), and he has to assume a topic/comment structure within the NP. 
The utterance (\ref{ex:marys-version}) implies that the speaker has heard the version of this story from at least one other person. It thus opens an alternative set: \emph{Mary's} is interpreted as contrastive, and therefore as focus. \emph{The actress} can be interpreted as the topic of the NP in (\ref{ex:marys-portrait-a-portrait}), while in (\ref{ex:marys-portrait}) \emph{Mary} is more naturally the topic, and \emph{the actress} the focus in the NP. 
\citeauthor{Kuno.1987} formulates this constraint as follows:

\ea[]{Topichood Condition for Extraction \citep[23]{Kuno.1987}:\\
Only those constituents in a sentence that qualify as the topic of the sentence can undergo extraction processes (i.e.\ Wh-Q Movement, Wh-Relative Movement, Topicalization, and It-Clefting).}
\z 

% yet Bailard (1981:17) says that extraction emphasizes a noun by adding information and speaks of a "focusing function" (same with clefts). 
% see also Schachter 1973: "foreground function" (clefts and relative clauses)
% Bailard, Joëlle. 1981. A functional approach to subject inversion. Studies in Language 5. 1–29.
% Schachter, Paul. 1973. Focus and Relativization. Language. Linguistic Society of America 49(1). 19–46. https://doi.org/10.2307/412101.

\subsection{The salience approach (reconciling the Focus and Topic approaches)}
\label{ch:salience-and-BCI}\largerpage[-2]

The Focus approach and the Topic approach are not mutually exclusive, and \citet{Kuno.1987} sees the Topic Condition as an extension of \citeauthor{Erteschik-Shir.1973}'s rule. What is missing in both accounts, however, is an explanation of how a syntactic factor like extraction and discursive factors like topic and focus interact. 

\citet{Deane.1991} answers this concern and provides a unifying account based on the management of cognitive resources. He suggests that extraction requires simultaneous consideration of two separate parts, the filler and its head, which we need to link together in order to obtain the appropriate syntactic structure. The longer the distance, the stronger this division of attention taxes our cognitive resources: we have limited space in our short-term memory. If the two parts are cognitively salient, however, it is easier to keep them active. Focus and topic are the two most salient elements in the sentence: the focus is salient because it is the important part of the discourse, and the topic is salient because it is the center of interest in the sentence. They are therefore the best candidates for extraction. 

Building on this idea, and adopting the constraint-based counterpart of  \citet{Erteschik-Shir.1973} dominance condition on extraction, \citeauthor{Goldberg.2006} proposed the BCI constraint \citep[see also][]{Ambridge.2008,Goldberg.2013,Cuneo.2023}:

\ea Backgrounded constructions are islands (BCI) \citep[2]{Cuneo.2023}:\\
Constructions are islands to long-distance dependency constructions to the
extent that their content is backgrounded within the domain of
the long-distance dependency construction. 
\label{rule:BCI}
\z 

The BIC and the dominance condition on extraction make the same predictions: extraction out of non-focus (hence backgrounded) constituents is infelicitous. 
%The difference is that \citeauthor{Erteschik-Shir.1973} assumes that extraction is ruled out by default and needs rules that license it, while \citeauthor{Goldberg.2006} assumes that extractions are possible by default, and that only certain constructions can constrain them.  
Both constraints are discourse-based, but this is not reflected explicitly by their respective formulation.

\section{The subject island constraint from a functional perspective}

\citet{Erteschik-Shir.1973} shows that sentential subjects are presupposed in the utterance. Consider first the sentential complement in (\ref{ex:ES-sentential-object}):

\begin{exe} 
\ex \citep[157]{Erteschik-Shir.1973}\\
Bill said `It's likely that Sheila knew all along.'
\label{ex:ES-sentential-object}
\begin{xlist}
\ex[]{, which is a lie -- it isn't.}
\ex[]{, which is a lie -- she didn't.}
\end{xlist}
\end{exe}

Targeting the sentential complement with the liar test seems to be felicitous. We can conclude that the it is not backgrounded, hence part of the potential focus domain. This clearly contrasts with the sentential subject in (\ref{ex:ES-sentential-subject}).

\begin{exe}
\ex \citep[157]{Erteschik-Shir.1973}\\
Bill said `That Sheila knew all along is likely.'
\label{ex:ES-sentential-subject}
\begin{xlist}
\ex[]{, which is a lie -- it isn't.}
\ex[*]{, which is a lie -- she didn't.}
\end{xlist}
\end{exe}

Most scholars in the functional approach agree that the ``subject island contraint'' for an NP subject is caused by the subject being the default topic of the utterance. One piece of evidence is that topics have a preference for being expressed as subjects. Indeed, when \emph{John} is the topic, the answer in (\ref{ex:topic-subject}) is more natural than the one in (\ref{ex:topic-object}). 

\begin{exe} 
\ex \citep[323]{Erteschik-Shir.2006}\\
Tell me about John.
\begin{xlist}
\ex[]{-- He is in love with Mary. \label{ex:topic-subject}}
\ex[]{-- Mary is in love with him. \label{ex:topic-object}}
\end{xlist}
\end{exe}

% Even all-focus structures (out of the blue) have a covert topic before subject: Erteschik-Shir 1997
% Eteschik-Shir 1997 The Dynamics of Focus Structure. Cambridge: Cambridge University Press.

% Grosu 1978 there are "natural" or "preferred" topic-comment structures. Subjects ted to be topic of the clause.
% Webelhuth, 2007: subject = topic

This is not to say that subject are always toppics. We can see a counterexample in (\ref{ex:subject-not-topic-1}). Here, the subject is more likely to be the new or unpredicted information in the sentence, and thus the focus, which is why (\ref{ex:subject-not-topic-2}) is a good paraphrase for it. 

\begin{exe}
\ex \citep{Kuno.1987}
\label{ex:subject-not-topic}
\begin{xlist}
\ex[]{[This person alone]$_F$ [passed the test]$_B$. \label{ex:subject-not-topic-1}}
\ex[]{The only person who passed the test was this person. \label{ex:subject-not-topic-2}}
\end{xlist}
\end{exe}

\citet[324]{Erteschik-Shir.2006} assumes that extraction is allowed only in what she calls ``canonical f-structures'', in which the subject is the topic \citep[see also][186]{Erteschik-Shir.1997}. The reason is that it is harder for the addressee to identify the dependents in an utterance with a non-canonical f-structure like (\ref{ex:subject-not-topic-1}), and it is therefore harder to identify the gap. Because extraction has to take place from the potential focus domain, as stated in (\ref{ex:rule-ES}), extraction out of subjects is ruled out.

As could be expected, \citet{Goldberg.2006} makes a similar proposal. With subjects being default topics~-- what she calls ``primary topics''~--, and topics being backgrounded, extraction out of the subject violates the BCI (\ref{rule:BCI}).
It is possible to extract a primary topic as a whole, but not part of it. She explains: ``It is pragmatically
anomalous to treat an element as at once backgrounded and discourse-prominent.'' Hence, according to \citeauthor{Goldberg.2006}, the subject island is caused by a discourse clash.

\section{The BCI revisited: the Focus-Background Conflict constraint}

The previous functional approaches to islands did not pay much attention to the fact that not all filler-gap dependencies have the same discourse function. Indeed, \emph{wh}-questions and \emph{it}-clefts focus the extracted element \citep{Lambrecht.1994}, while relativization and topicalization topicalize it \citep[15]{Kuno.1987}.\footnote{The idea that the relationship is a topic-comment relationship is not new; it was probably first proposed by \citet{Kuno.1973} for Japanese (see also the Thematic Constraint on Relative Clauses by \citealt[420]{Kuno.1976}): ``On the basis of the pervasive parallelism between topicalization and relativization, I proposed that in Japanese what is relativized is the theme of the relative clause.'' \citep[15]{Kuno.1987}. \citet[25]{Schachter.1973} provides evidence from Ilonggo based on case marking that supports this claim. Several authors assume that the topic-comment relationship applies to English relative clauses as well (\citealt{Gundel.1974}; \citealt[79]{Gundel.1988}; \citealt[15]{Kuno.1987}). In general, many assume that it is a universal property of relative clauses (even though \citet{Lambrecht.1994} suggests that it may only be true for languages with post-antecedent relative clauses like French or English).}
Even though \citeauthor{Erteschik-Shir.1973}, \citeauthor{Kuno.1972} and \citeauthor{Goldberg.2006} explain constraints on extractions in terms of discourse status, they all take for granted that
``topicalization processes (Topicalization and Relativization) and focusing processes (Wh-Q Movement and It-Clefting) are subject to the same constraint'' \citep[27]{Kuno.1987}.
Probably for the same reason, the constraints they proposed (Subject Condition, BCI, Topichood Condition for Extraction) rely on discursive factors, but are not explained in terms of discursive mechanisms. 

% Intuition in Ross 1971: ‘Variable Strength’, MS, MIT
% Cinque 1990 answers that this is due to referentiality of relative clauses. But Erteschik-Shir (2006,318) criticizes this answer: Cinque does not explain how referenciality influences something here.
% Cinque, G. (1990) Types of A’-Dependencies. Cambridge, MA: MIT Press.

Notably, these proposals all assume that extraction is a key factor in the constraint. But extraction, and word order more generally, is only one of many tools used in human languages to encode specific discourse status. There is no reason to believe that discourse clash cannot lead to infelicitous sentences independently of extraction.
There are actually several examples of subject/object asymmetries present in \textit{wh}-questions and not in relative clauses: in Kihung’an \citet{Takizala.1973}, in Chiche\^wa \citep{Bresnan.1987}, in Kaqchikel Mayan \citep{Heaton.2016} or in Tagalog \citet{PizarroGuevara.2020}. For example, in Chiche\^wa (a language from the Bantu family), in which object marking (\textsc{om}) on the verb is otherwise optional, the presence of an object marker is ruled out for object interrogatives. This is illustrated by the contrast in (\ref{ex:chichewa}). 

\begin{exe}
\ex \citep[759--760]{Bresnan.1987}
\label{ex:chichewa}
\begin{xlist}
\ex[]{\gll Mu-ku-fún-á chiyâni?\\
you-\textsc{pres}-want-\textsc{indic} what\\
\glt `What do you want?'}
\ex[*]{\gll Mu-ku-chí-fún-á chiyâni?\\
you-\textsc{pres}-\textsc{om}-want-\textsc{indic} what\\
\glt `What do you want?'}
\label{ex:chichewa-bad}
\end{xlist}
\end{exe}

In (\ref{ex:chichewa-bad}), the verbal object marker \emph{-chí-} seems incompatible with the object interrogative word \emph{chiyâni}. According to \citet[758--760]{Bresnan.1987}, the reason is that the verbal object marker \emph{-chí-} is an anaphoric pronoun that signals that the object is the topic. Since the object cannot be topic and focus of the utterance at the same time, the sentence is ruled out.
\begin{quote}
    Because the topic designates what is under discussion (whether previously mentioned or assumed in discourse), it is presupposed. The interrogative focus designates what is \textsc{not} presupposed as known, and is contrasted with presupposed material. Hence, allowing the same constituent to be both topic and focus of the same clause leads to inconsistent presupposition. \citep[758]{Bresnan.1987}
\end{quote}
Extraction here plays no role, because the interrogative word is in situ.

Furthermore, as already discussed in Section~\ref{ch:processing}, the contrast between the subject island on the one hand and the greater preference for subject relatives over object relatives on the other hand is very surprising and remains unexplained under the previous discourse-based accounts. The subject island seems to directly contradict \citegen{Keenan.1977} Accessibility hierarchy. 

Based on experimental data from English and French, we proposed in \citet[rule (8)]{Abeille.2020.Cognition} that the penalty observed in extraction out of the subject known as ``subject island''
is caused by a discourse clash: the degradation results from the attempt to focalize some part of a backgrounded element. Indeed, it seems reasonable to assume that we cannot simultaneously identify an individual \emph{x} as part of the Common Ground and open a set of alternatives about some property inherent to this same individual. 
We therefore reformulated the BCI and call this the Focus-Background Conflict constraint, which we define as:

\ea Focus-background conflict (FBC) constraint:\\
A focused element should not be part of a backgrounded constituent.
\label{rule:FBC}
\z 

We agree with previous discourse-based approaches in assuming that subjects are default topics (and thus backgrounded). Subextraction out of the subject that leads to focalization of the extracted phrase thus violates the FBC constraint, and this, we claim, is why it is degraded compared to a similar subextraction out of the object. Complements have a tendency to belong to the focus, and for this reason subextraction out of the object is more often felicitous. 

Notice that this constraint explicitly presents focusing processes as the cause of the degradation. The straightforward consequence is that only focusing extractions like \emph{wh}-questions and \emph{it}-clefts can violate the FBC constraint. In a relative clause, the extraction is topicalization: the referent denoted by the antecedent of the relative clause (the noun modified by the relative clause) is the topic of the relative clause. In other words, the relative clause ``is about'' the noun it modifies. The subject in relative clauses is preferably backgrounded, but since the extracted element is not focused, the FBC constraint is not violated. 

% NB: Is there a distinction of topicalization of the extracted element between restrictive vs. non-restrictive relatives ? 
% Doug Arnold on restrictive relatives

The scope of the FBC constraint (\ref{rule:FBC}) extends beyond extraction. Focalization of part of a backgrounded constituent that does not involve extraction would violate the FBC constraint as well.

Moreover, the FBC constraint (\ref{rule:FBC}) is not expected to apply to all subjects. Even though subjects are topics by default, they may also be focus. This means that extraction out of a focus subject by means of an interrogative or \emph{it}-cleft should be possible, because this does not lead to a discourse clash. 
