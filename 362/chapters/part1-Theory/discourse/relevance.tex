\label{ch:discourse-relevance}

It is important to note, however, that there are also other pragmatic considerations at play. Constructions involving extractions are costly from a processing point of view, but they are used in order to fulfill specific communicative goals (which differ from one construction to another). If this were not the case, according to \citegen{Grice.1975} Cooperative Principle, the speaker would resort to a simpler construction. It is therefore essential that the use of extraction is relevant: this is the central idea of \citet{Chaves.2020.UDC} who attribute the first formulation of it to \citet{Kuno.1987}. 

\citeauthor{Kuno.1987} notices that the contrast in (\ref{ex:book-about-write-destroy}) cannot be attributed to a semantic distinction between the two verbs involved.

\begin{exe}
\ex \citep[23]{Kuno.1987}
\begin{xlist}
\ex[]{What did you see pictures of?}
\ex[*]{What did you see a book about?} 
\label{ex:non-relevant-book-about}
\end{xlist}
\end{exe}

The explanation, \citeauthor{Kuno.1972} claims, relies on the fact that seeing a picture is necessarily synonymous with seeing what this picture portrays. For that reason, what the picture portrays is a relevant aspect of the event of seeing a picture. By contrast, seeing a book does not necessarily imply seeing what the book is about. Reading a text and seeing the physical object book are two distinct events. Consequently, it is not obvious that the theme of the book is relevant to the event of seeing a book. Because the theme of the book is not relevant, there is no compelling reason to ask questions about it. Hence, it is difficult for the addressee to imagine a context in which the question would be needed in the first place, and the sentence (\ref{ex:non-relevant-book-about}) is perceived as unacceptable. 

\citet{Chaves.2013} shows with numerous examples that extraction of non-relevant elements leads to an important degradation of the sentences. Example (\ref{ex:owner-cat}) is an extreme case. It is indeed very difficult to imagine a situation in which the ownership is at-issue in a sneezing event.

\ea \citep[12]{Chaves.2013}\\
* What$_i$ did [the owner of~\trace{}$_i$] sneeze?
\label{ex:owner-cat}
\z 

%The filler of the gap must be relevant for the proposition (the assertion) and the subject. cf: ``shadow arguments'' in Pustejovsky (1995,ch.5).
% Pustejovsky, James. 1995. The Generative Lexicon. MIT Press, Cambridge, MA, USA.

Expanding on \citegen{Kuno.1976} idea, \citet[327]{Chaves.2020.UDC} posit that the extracted element must be relevant to ``the main action that the sentence describes''. 