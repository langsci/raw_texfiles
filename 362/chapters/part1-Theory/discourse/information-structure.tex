\section{Information structure}
\label{ch:is}

The discourse-based approach takes into account an undoubtedly crucial parameter in communication, a parameter so complex that it is very difficult to formalize it entirely: the fact that communication is an interchange of information between several participants in a discourse event. Phatic discourse, i.e.\ discourse in which no exchange of information is involved, is possible, but it is the exception rather than the rule. An interchange of information between several participants requires from each of them a capacity for what \citet{Kuno.1976} calls ``empathy'' (and which could also be labeled ``Theory of Mind''): They need to recall which pieces of information the other participant(s) have and which ones they do not have. The sentence ``\emph{Mary is a good scholar.}'' only succeeds in its informative role if all participants in the discourse know who Mary is \citep[309]{Kuno.1972}. In order for the communication to be efficient, it is also necessary that each participant keeps in mind the information that they have already been given in the previous part of the discourse.

In this work, we assume a formalization of how individual participants manage discourse information based on ``information packaging'' \citep{Chafe.1976} and on the notion of Common Ground \citep{Krifka.2007}.\footnote{The notion of ``Common Ground'' was probably first discussed by \citeauthor{Stalnaker.1978} under the notion of ``common knowledge'' and especially ``common background knowledge'' \citep[86]{Stalnaker.1978}.} % also in Grice 1981:190
Common Ground covers information exchanged in a particular discourse. The Cooperative Principle defined by \citet{Grice.1975} also implies that every utterance in a discourse entails a proposition which augments the Common Ground. For this reason, example (\ref{ex:kuno-presupposition-vet-cat}) is inappropriate in the specific discourse situation. The first utterance, \emph{I had to bring my cat to the vet}, entails at least two pieces of information: the first one is presupposed and is that a cat exists and that the speaker is the owner of this cat; the second piece of information, \emph{I have a cat}, brings no new information to the Common Ground, because the information that it entails is redundant to the information presupposed in the previous utterance. This is not the case with (\ref{ex:kuno-presupposition-cat-vet}), where each part of the sentence brings new information to the Common Ground. 

\eal 
\label{ex:kuno-presupposition}
\ex I have a cat, and I had to bring my cat to the vet.
\label{ex:kuno-presupposition-cat-vet}
\ex  \citep[16]{Krifka.2007}\\
\# I had to bring my cat to the vet and I have a cat.
\label{ex:kuno-presupposition-vet-cat}
\zl 

\subsection{Topic and comment}

\citet{Reinhart.1982} proposes a helpful metaphor to describe the integration of new information in the Common Ground: we can imagine the information that conversation partners store as a collection of index cards. Each card has a title, its index: this is the topic. On each card, under the title, participants keep record of the relevant information: this is the comment, i.e.\ what the utterance states about the topic.

We use the subscript $_T$ and square brackets to identify the topic in our examples if needed. Similarly, we use the subscript $_C$ and square brackets to identify the comment. 

For example, in (\ref{ex:basic-topic}), the first utterance introduces the individual Geneva to the Common Ground; this individual is then the topic of the second utterance. The comment is the information about her love for parrots.

\ea This is Geneva Howell. [Geneva]$_T$ loves parrots.
\label{ex:basic-topic}
\z 

Example (\ref{ex:basic-topic}) illustrates an \emph{aboutness topic}. There is a second kind of topic, called \emph{frame-setting topic} \citep[45--46]{Krifka.2007}. A frame-setting topic ``acts as a restrictor as to when, where or with respect to who or what, the truth value of the predication is to be evaluated'' \citep[130]{Erteschik-Shir.1997}. In example (\ref{ex:körperlich}), the adverb is the topic and restricts the domain for which it is true that Peter is well (implicitly implying that Peter is not well regarding other domains of his life).

\ea \citep[655]{Jacobs.2001}\\
\gll [Körperlich]$_T$ geht es Peter gut.\\
physically goes it Peter well\\
\glt `Physically, Peter is well.'
\label{ex:körperlich}
\z 

As we will see, the aboutness topic will be the most relevant one in the study of subject islands. Very often, the aboutness topic of an utterance is anaphoric, hence uses ``given'' information. ``Given'' means here that it already entered the Common Ground at some point of the discourse. However, givenness is not required for topicalization: Corpus studies show that new information can become the topic of an utterance \citep[41--42]{Krifka.2007}.  

\eal
\ex \citep[42]{Krifka.2007}\\
{} [A good friend of mine]$_T$ [married Britney Spears last year]$_C$.
\ex \citep[66]{Reinhart.1982}\\%(exemple (21a)
Because they wanted to know more about the ocean's current, [students in the science club at Mark Twain Junior High School of Coney Island]$_T$ gave ten bottles with return address cards inside to crewmen of one of New York City's sludge barges.
\zl 

\citet[Section~3.2]{Reinhart.1982} proposes a way to test whether X is the aboutness topic of a given utterance by using a paraphrase such as ``\emph{as for} X ...'', ``\emph{speaking about} X ...'', or ``\emph{about} X ...''. If the paraphrase is pragmatically identical with the original utterance, then X is the aboutness topic. 
Based on this idea, \citeauthor{Goetze.2007} propose the following test for aboutness topics:

\ea An NP X is the aboutness topic of a sentence S containing X if: S would be the natural continuation of the announcement \emph{Let me tell you something about X}. \citep[19]{Goetze.2007}
\z 

Lastly, I should add that some sentences do not contain a topic. Topicless sentences are called thetic sentences \citep{Kuroda.1976,Ladusaw.1994}. 

\subsection{Focus}

The notion of ``focus'' is very common in linguistics and at the same time usually poorly defined. Different authors employ the term in different ways, without explicitly specifying which definition they are using. In this work I adopt the definition given by Alternative Semantics \citep{Rooth.1992}, which has the advantage of being a formal definition. In Alternative Semantics, focus signals the importance of alternatives to the focused element for the interpretation of the utterance. For this reason, the most straightforward example of focus is an answer to a \emph{wh}-question. An interrogative word like \emph{which} in (\ref{ex:def-focus-qu}) signals a set of alternatives (here the set of speaker B's siblings, let us assume the set \{Jennifer, Karen, Brandon\}). The focus of speaker B's answer in (\ref{ex:def-focus-ans}) is the most informative element of the utterance (namely here the individual Karen).

% Collingwood (1940) is credited with the original idea in Anglo-Saxon literature: "every statement that anybody ever makes is made in answer to a question" (cited in Fintel 2004 : 147).
%     Collingwood, 1940, An essay on metaphysics. Oxford. 
%    Fintel von K., 2004, A minimal theory of adverbial quantification, [Kamp H. & Partee B., eds] Context-dependence in the analysis of linguistic meaning, Elsevier. 

\eal 
\ex Speaker A: Which one of your siblings is the oldest? \label{ex:def-focus-qu}
\ex Speaker B: [Karen]$_F$ is the oldest. \label{ex:def-focus-ans}
\zl 

We use the subscript $_F$ and square brackets to identify the focus in our examples if needed, as can be seen in (\ref{ex:def-focus-ans}). All other elements of the utterance -- i.e.\ \emph{be the oldest} for (\ref{ex:def-focus-ans}) -- are backgrounded (see Section~\ref{ch:background}).  

Any kind of constituent can be focused: a single word like in (\ref{ex:def-focus-ans}), whole sentences like in (\ref{ex:whole-sentence-focus}), as well as everything in between.\footnote{Even contrastive focus on one syllable is possible in order to stress metalinguistic information.
\begin{itemize}
    \item[(i)] I did not say that he had been a pathetic help, but that he had been a SYMpathetic help! 
\end{itemize}}  

\eal 
\ex Speaker A: What happened?
\ex Speaker B: [Karen bought a parrot]$_F$. \label{ex:whole-sentence-focus}
\zl 


This leads to a distinction between narrow and broad focus based on the type of constituent being focused: 
the whole sentence (broad focus) or some constituent(s) (narrow focus). 

Though other definitions have been proposed in previous literature, I will assume \citegen{Goetze.2007} definition of focus:

\ea Typically, focus on a subexpression indicates that it is selected from possible alternatives that are either implicit or given explicitly, whereas the background can be derived from the context of the utterance.\label{ex:definition-focus}
\z 

The focused element is also mostly the one bearing the main stress of the sentence (at least in languages like English or French).
\citet{Krifka.2007} notes, however, that stress is only one possible way to signal focus, and not the very definition of focus. Prosody is at best a useful tool to identify certain kinds of focus. In the written language, however, we can only stipulate the place of the main stress. As the empirical part of this work is based almost exclusively on written French (research in written corpus and experiments based on reading tasks), I will not say much about the intonational aspect of focus.

% on correlation between IS and pitch: \citet{Gregory.2001}

Focus is also sometimes described as the most ``important'' part of the utterance. \citet{Krifka.2007} criticizes this formulation for being vague and subjective. In his opinion, importance, as well as pertinence or main stress, only correlates with focus, but none of these aspects are criteria to define it. 

% Zimmerman and Onea 2011 for a distinction between informative and contrastive focus
        % Zimmermann, M., and Onea, E. (2011). Focus marking and focus realization. Lingua 121, 1651–1670. doi: 10.1016/j.lingua.2011.06.002

Finally, it is also useful to say a word on the relation between focus and new information, or between focus and the topic/comment distinction. The focus, unlike the background as stated in definition (\ref{ex:definition-focus}), cannot be derived from the context of the utterance, it is new information. This does not mean, however, that the semantic referent has not been mentioned in the discourse, only that this part of the proposition is new. In (\ref{ex:not-new-focus}), the answer selects one of the alternatives previously mentioned in the discourse. What is new is that the destination was the beach. 

\eal \label{ex:not-new-focus}
\ex Speaker A: Did you go to the beach or to the museum yesterday?
\ex Speaker B: We went to the [beach]$_F$.
\zl 

In (\ref{ex:old-focus}), the focus contains an anaphoric pronoun, i.e.\ the referent has already been mentioned, but it is still a felicitous answer to speaker A's question, because it is selected from other possible alternative answers. 

\begin{exe}
\ex \citep{Marandin.2007}
\label{ex:old-focus}
\begin{xlist}
\ex Speaker A: Who did Felix praise?
\ex Speaker B: Felix praised [himself]$_F$.
\end{xlist}
\end{exe}

Because of this, focus is often part of the comment, but contrastive, corrective or confirmative focus (see below) is also possible on the topic, which is then usually called a ``contrastive topic''. 

Many kinds of focus have been identified in the literature \citep[6--34]{Krifka.2007}. I will now define the kinds of focus which are useful in this work. This list is by no means exhaustive.

\subsubsection{Information focus}

Information focus is the prototypical kind of focus, and is also called \emph{semantic focus}. Informational focus occurs when new information is added to the Common Ground; it is the element that answers the implicit or explicit question. 

% called semantic focus (= non-contrastive, the new information, hence informational focus) in Song 2017, this terminology comes from Gundel 1999
% Gundel, Jeanette K. 1999. On different kinds of focus. In Peter Bosch & Rob van der Sandt (eds.), focus: linguistic, cognitive, and computational perspectives, 293–305. Cambridge, UK: Cambridge University Press.

\subsubsection{Contrastive focus}
An utterance containing a contrastive focus reacts to a proposition which just entered the Common Ground. The focus signals an element that the speaker wants to correct or wants to provide additional information on. One example of the latter is given in~(\ref{ex:contrastive-focus}).

\eal \label{ex:contrastive-focus}
\ex Speaker A: Karen has a child.
\ex Speaker B: [Brandon]$_F$ has a child too.
\zl 

\subsubsection{Corrective (or confirmative) focus} An utterance with a corrective or confirmative focus also reacts to a proposition which just entered the Common Ground. In (\ref{ex:def-focus-corrective}), the focus element corrects the alternative previously mentioned in the discourse (here: Karen) and excludes it: this alternative makes the proposition false. In confirmative focus like (\ref{ex:def-focus-confirmative}), the alternative previously mentioned in the discourse is pertinent, and other potential alternatives are excluded: the proposition with this alternative is true.

\begin{exe}
\ex Speaker A: Karen is the oldest.
\begin{xlist}
\ex Speaker B: No, [Brandon]$_F$ is the oldest. \label{ex:def-focus-corrective}
\ex Speaker B: Yes, [Karen]$_F$ is the oldest. \label{ex:def-focus-confirmative}
\end{xlist}
\end{exe}

\subsubsection{A topic with focus properties: Contrastive topic}

The answer in (\ref{ex:contrastive-topic-ans}) contains two topics: \emph{Karen} and \emph{Brandon}. Both are continuation topics that add more information to the topic introduced in the question (\ref{ex:contrastive-topic-qu}), \emph{your siblings}. Since \emph{siblings} refers to several individuals, there is potentially a need to distinguish between them.

\eal 
\ex Speaker A: What do your siblings do? \label{ex:contrastive-topic-qu}
\ex Speaker B: [Karen]\textsubscript{CT} is a writer and [Brandon]\textsubscript{CT} is a life guard. \label{ex:contrastive-topic-ans}
\zl 

In this case, we talk about contrastive topics \citep[44--45]{Krifka.2007}. Contrastive topics have some properties of focus, because they signal a set of sets of propositions (whereas focus signals a set of propositions). 


\subsection{Background (and presuppositional content)}
\label{ch:background}

The background~-- already defined in (\ref{ex:definition-focus})~-- is the part of the utterance that is presupposed, following the definition of presupposition given by \cite{Lambrecht.1994}. 

% Background :
% Büring 1999 => topic is not backgrounded
% Büring, Daniel. 1999. Topic. In P. Bosch and R. van der Sandt, eds., Focus, Studies in Natural Language Processing, pages 142-165. Cambridge, UK: Cambridge University Press.
% Vallduvi => topic is part of background
% von Stechow 1981,101
% von Stechow, Arnim. 1981. Topic, focus, and local relevance. In W. Klein and W. Levelt, eds., Crossing the Boundaries in Linguistics , pages 95-130. Dordrecht: Reidel.
% Le Draoulec 1997:9-44

\ea A proposition P is a pragmatic presupposition of a speaker in a given context just in case the speaker assumes or believes that P, assumes or believes that his addressee assumes or believes that P, and assumes or believes that this addressee recognizes that he is making these assumptions, or has these beliefs. \citep[51]{Lambrecht.1994}
\z 

Any information can be backgrounded because it is old information in the Common Ground, but it can also be new information that is not at issue and that must be taken for granted in order for the propositional content of the utterance to be true \citep[54]{Lambrecht.1994}. Focus and Background are in complementary distribution, such that an element in the utterance must be either focused or backgrounded. 

In Alternative Semantics, the background is regarded as introducing a set of only one element. 

Presuppositions differ from implicatures, because implicatures can be negated while the negation of a presupposition is infelicitous. Consider (\ref{ex:implicature}), whose implicature is that the speaker has only one child. Yet, the conversation in (\ref{ex:implicature-neg}) is felicitous, event though this implicature is contradicted in the next sentence.

\eal 
\ex[]{I have a child.} \label{ex:implicature}
\ex[]{Speaker A: I wish I were a father. What about you, do you have a child?\\ Speaker B: Yes, I have a child. Actually, I have three children.} \label{ex:implicature-neg}
\zl 

This contrasts with the presupposition of (\ref{ex:presupposition}), which is that Jennifer has two siblings (the verb \emph{to know} takes as a complement an S whose propositional content is presupposed). Contradicting this presupposition as in (\ref{ex:presupposition-neg}) is infelicitous.

\eal 
\ex[]{The landlord does not know that Jennifer has two siblings.} \label{ex:presupposition}
\ex[\#]{The landlord does not know that Jennifer has two siblings. Actually, she's an only child.} \label{ex:presupposition-neg}
\zl 

This property of presuppositions led \citet{Erteschik-Shir.1973} to propose a test of backgroundedness, called the ``liar test''. The test consists in reporting that someone has said the utterance, and then adding that this person was lying about a particular part of the utterance. Example (\ref{ex:liar-test}) is a liar test for the utterance in (\ref{ex:presupposition}). The continuation in (\ref{ex:liar-test-good}) shows that the elements [\emph{does not know}] are not backgrounded (they bear the informational focus), while the continuation in (\ref{ex:liar-test-bad}) shows that the sentential complement [\emph{Jennifer has two siblings}] is backgrounded. 

\begin{exe}
\ex Ash said: The landlord does not know that Jennifer has two siblings...
\label{ex:liar-test}
\begin{xlist}
\ex[]{... which is a lie: he does know.} \label{ex:liar-test-good}
\ex[\#]{... which is a lie: she's an only child.} \label{ex:liar-test-bad}
\end{xlist}
\end{exe}

\citet[51]{Lambrecht.1994}, \citet{Ambridge.2008} and \citet{Cuneo.2023} propose similar tests that take advantage of the same property of presupposition in order to identify the backgrounded elements. It must be noted that the liar test targets only the backgroundedness with respect to the main clause. The internal information structure of an embedded clause cannot be targeted directly; the embedded sentence has to be tested in isolation.

% Lahousse 2011: difference between assertive and non-assertive (non-assertive = presuppositional)
% caution: the lier test applies to main clauses, not subordination

% other test: assertive epistemic adverbs "perhaps" and "probably" possible only if the statement is assertive Lahousse 2001:236

% presupposition / assertion
% Kuroda 1992:66, an assertion is "an expression of epistemic commitment" (I'm not sure whether there is a connection with clefts and Destruel's commitment???)
