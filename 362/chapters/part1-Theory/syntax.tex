\label{ch:syntax}

The first and still best-known approaches to the phenomenon known as ``subject island'' posit that the ban on extraction out of subjects~-- sentential and verbal subjects and/or NP subjects~-- is caused by their syntactic properties. In this chapter, I will present \citegen{Ross.1967} original constraint for sentential subjects, and the subsequent early analysis by \citet{Chomsky.1973} for all subjects, which have been very influential. It is unfortunately impossible to review all syntactic approaches to the subject island constraint: there have been an abundance of different analyses. I will therefore only discuss three main lines of explanation that have been adopted over the years. Each one is based on one of the properties of subjects: not being a complement (those approaches usually treat the subject island together with the adjunct island\footnote{The adjunct island was not part of \citeauthor{Ross.1967}'s original islands, but it has been associated with the island constraints principally by \citet{Cattell.1976} (as the ``Adjunct Island Condition''). It is claimed to be a strong island. 

\eal 
\ex \citep[168]{Longobardi.1985}\\
* a person$_i$ who$_i$ I think that [while informing~\trace{}$_i$ about my work] I could be embarrassed.
\label{longobardi-adjunct-island}
\ex \citep[43]{Godard.1988}\\
\gll * [une] décision$_i$ [que$_i$ vous savez bien que [s' ils prenaient~\trace{}$_i$] nous n' aurions d' autre choix que de partir d' ici]\\
{} \sbar{}a decision \sbar{}that you know well that \sbar{}if they take\textsc{.cond} we \textsc{neg} have\textsc{.cond} of other choice that of leave\textsc{.inf} of here\\ 
\glt `a decision that you know very well that if they make (it) we would have no other choice than departing'
\zl} as a single phenomenon), being a specifier, and having to move to Spec,TP/IP. Then I will turn to the major criticisms that have been brought against syntactic approaches. The major one is that it has been well documented that extraction out of the subject is possible in many languages \citep{Stepanov.2007}. Of particular interest for this study is the work on French by \citet{Godard.1988}, which has shown that subextraction from subjects is possible with the relative word \emph{dont}. I will present the French data, the debate around French, and explain how \citet{Tellier.1990,Tellier.1991} and \citet{Heck.2009} have tried to account for these data while still maintaining the subject island constraint for French. Finally, I will give a brief historical survey on how the subject island constraint has been treated in HPSG. 

%------------------------------------------------

\section{The Sentential Subject Constraint of Ross (1967)}

\citeauthor{Ross.1967} appears to have been the first linguist to notice that extraction\footnote{In his terminology -- following a transformational approach -- \emph{reordering}.} out of subjects is limited by certain constraints in English. His rule is reproduced in (\ref{rule:SSC}). \citep[243]{Ross.1967}

\ea No element dominated by an S may be moved out of that S if that node S is dominated by a NP which itself is immediately dominated by S.
\label{rule:SSC}
\z 


The Sentential Subject Constraint accounts for the visible contrast between extraction out of sentential subjects and out of sentential complements, as illustrated in (\ref{ex:Ross-battleax}).

\begin{exe}\ex\label{ex:Ross-battleax}
\citep[241]{Ross.1967}
\begin{xlist}
    \ex[]{The teacher [who$_i$ the reporters expected [that the principal would fire~\trace{}$_i$]] is a crusty old battleax.} \label{ex:Ross-battleax-O}
    \ex[*]{The teacher [who$_i$ [that the principal would fire~\trace{}$_i$] was expected by the reporters] is a crusty old battleax.}  \label{ex:Ross-battleax-S}
\end{xlist}
\end{exe}

In order to understand why (\ref{rule:SSC}) correctly rules out (\ref{ex:Ross-battleax-S}), it is necessary to know that \citeauthor{Ross.1967} analyzes sentential complements as NPs directly embedding an S, as shown in \figref{fig:Ross-battleax-tree-base-O} for a sentential complement and in \figref{fig:Ross-battleax-tree-base-S} for a sentential subject.

\begin{figure}[p]
\small
\begin{forest}
where n children=0{tier=word}{}
[S 
    [NP
        [The reporters, roof]
    ]
    [VP
    	[V 
    		    [expected]
    	]
    	[NP
    	    [S
    	        [C
    	            [that]
    	       ]
    	       [NP
                    [the principal, roof]
        	   ]
        	   [VP
        	       [V
            	       [would fire, roof]
                    ]
            	   [NP
            	       [some teacher, roof]
            	   ]
                ]
        	]
        ]
    ]
]
\end{forest}

\caption{Syntactic tree for ``The reporters expected [that the principal would fire some teacher].'' (Baseline of extraction in (\ref{ex:Ross-battleax-O})) according to \citeauthor{Ross.1967}'s analysis}
    \label{fig:Ross-battleax-tree-base-O}
\end{figure} 

\begin{figure}[p]
\small
\resizebox{\linewidth}{!}{\begin{forest}
where n children=0{tier=word}{}
[S 
	[NP
    	    [S
    	        [C
    	            [That]
    	       ]
    	       [NP
                    [the principal, roof]
        	   ]
        	   [VP
        	       [V
            	       [would fire, roof]
                    ]
            	   [NP
            	       [some teacher, roof]
            	   ]
                ]
        	]
        ]
	[VP
	    [V 
	        [was expected, roof]
	    ]
    	[PP
    	    [by the reporters, roof]
        ]
    ]
]
\end{forest}}
\caption{Syntactic tree for ``[That the principal would fire some teacher] was expected by the reporters.'' (Baseline of extraction in (\ref{ex:Ross-battleax-S})) according to \citeauthor{Ross.1967}'s analysis}
    \label{fig:Ross-battleax-tree-base-S}
\end{figure} 

In (\ref{ex:Ross-battleax-O}), whose syntactic tree is given in \figref{fig:Ross-battleax-tree-O} on page~\pageref{fig:Ross-battleax-tree-O}, the complement (the NP embedding the \emph{that}-clause) is embedded under VP, thus not directly embedded under the S of the relative clause. Therefore, the Sentential Subject Constraint does not apply, the sentence is grammatical and thus felicitous.

\begin{figure}[ht]
\oneline{
\begin{forest}
where n children=0{tier=word}{}
[NP
    [NP
        [the teacher, roof]
    ]
    [S
        [Wh
            [who, name=filler]
        ]
        [NP
            [the reporters, roof]
        ]
        [VP
        	[V 
    		    [expected]
        	]
        	[NP
    	        [S
    	            [C
    	                [that]
        	       ]
        	       [NP
                        [the principal, roof]
            	   ]
        	       [VP
        	           [V
            	           [would fire, roof]
                        ]
                	   [NP
                	       [\emph{t}, name=gap]{\draw[<-,dashed] (.east) -| ++(1.5em,10ex) node[anchor=south,align=center]{moves out\\of S};}
                    	]
                    ]
                ]{\draw[<-,dashed] (.east) -| ++(5em, 2ex) node[anchor=south,align=center]{embedded under NP};}
            ]{\draw[<-,dashed] (.east) -| ++(4em, 2ex) node[anchor=south,align=center]{not immediately\\ embedded under S};}
        ]
    ]
]
\draw[->,densely dotted] (gap.south) |- ++(-1ex, -.75\baselineskip) -| (filler); 
\end{forest}}
\caption{Syntactic tree for ``the teacher [who$_i$ the reporter expected [that the principal would fire~\trace{}$_i$]]'' according to \citeauthor{Ross.1967}'s analysis}
    \label{fig:Ross-battleax-tree-O}
\end{figure} 

In (\ref{ex:Ross-battleax-S}), whose syntactic tree is given in \figref{fig:Ross-battleax-tree-S} on page~\pageref{fig:Ross-battleax-tree-S}, the subject (again an NP embedding the \emph{that}-clause) is directly embedded under the S of the relative clause. The Sentential Subject Constraint is therefore violated, and the sentence is ungrammatical.

\begin{figure}[ht]
\oneline{
\begin{forest}
where n children=0{tier=word}{}
[NP
    [NP
        [the teacher, roof]
    ]
    [S
        [Wh
            [who, name=filler]
        ]
        [S
        	[NP
    	        [S
    	            [C
    	                [that]
        	       ]
        	       [NP
                        [the principal, roof]
            	   ]
        	       [VP
        	           [V
            	           [would fire, roof]
                        ]
                	   [NP
                	       [\emph{t}, name=gap]{\draw[<-,dashed] (.285) |- ++(1em, -.75\baselineskip) node[anchor=west,align=center]{moves out of S};}
                    	]
                    ]
                ]{\draw[<-,dashed] (.east)--++(1em,0pt) node[anchor=west,align=center,inner sep=1pt]{embedded under NP};}
            ]{\draw[<-,dashed] (.west)--++(-1em,0pt)    node[anchor=east,align=center,inner sep=1pt]{immediately\\ embedded \\ under S};}
        	[VP
        	    [V 
        	        [was expected, roof]
        	    ]
            	[PP
            	    [by the reporters, roof]
                ]
            ]
        ]
    ]
]
%\draw[->,densely dotted] (gap) to[out=south west,in=south west] (filler) node[anchor=east,align=center];
%\draw[->,densely dotted] (gap) to[in=south, out=south west] (-1.7,-6) to[out=north,in=south west] (filler);
\draw[->,densely dotted] (gap.255) |- ++(-1ex, -.75\baselineskip) -| (filler.south); 
\end{forest}}
\caption{Syntactic tree for ``the teacher [who$_i$ the reporters expected [that the principal would fire~\trace{}$_i$]]'' according to \citeauthor{Ross.1967}'s analysis}
    \label{fig:Ross-battleax-tree-S}
\end{figure} 


\citeauthor{Ross.1967} also notices that the Sentential Subject Constraint is in competition with a ``general output condition on performance'' reproduced in (\ref{rule:Ross-general-output-condition-on-performance}).

\ea\label{rule:Ross-general-output-condition-on-performance} Grammatical sentences containing an internal NP which exhaustively dominates S are unacceptable unless the main verb of that S is a gerund. \citep[251]{Ross.1967}
\z 

In (\ref{rule:Ross-general-output-condition-on-performance}), ``internal'' means that the element is neither at the beginning nor final. In this respect, [[\emph{that the principal would fire}]$_S$]\textsubscript{NP} in (\ref{ex:Ross-battleax-S}) violates (\ref{rule:Ross-general-output-condition-on-performance}), because it is neither at the beginning nor end of the sentence, and because its verb is finite. Any subextraction from a finite sentential subject will per definition fall within this case, because Ross analyses sentential subjects as being S immediately dominated by an NP, and because the sentential subject will necessarily stand between the filler and a verb. However, Ross makes a clear distinction between the unacceptability caused by the violation of (\ref{rule:Ross-general-output-condition-on-performance}) and the ungrammaticallity caused by the violation of (\ref{rule:SSC}). For him, ungrammatical sentences, unlike unacceptable ones, are ``beyond intonational help'' \citep[247]{Ross.1967}. Rule (\ref{rule:Ross-general-output-condition-on-performance}) was designed to account for the unacceptability of examples like (\ref{ex:Ross-internal-that}).\largerpage

\eal 
\ex[]{I told [a man who had a kind face] [that we were in trouble].}
\ex \citep[53]{Ross.1967} \\
? I told [that we were in trouble] [a man who had a kind face].
\label{ex:Ross-internal-that}
\zl 

It is not clear why Ross does not consider (\ref{ex:Ross-battleax-O}) unacceptable, given that it seems to violate (\ref{rule:Ross-general-output-condition-on-performance}) too, but the rule in (\ref{rule:Ross-general-output-condition-on-performance}) certainly does not account for the contrast noticed between (\ref{ex:Ross-battleax-S}) and (\ref{ex:Ross-battleax-O}).


%------------------------------------------------

\section{The Subject Condition: subextraction from subject NPs}

\citeauthor{Ross.1967}'s Sentential Subject Constraint was very soon extended to all kinds of subjects, including NP subjects.\footnote{Yet, \citet{Ross.1967} explicitly disagreed with this, because it made false predictions in his opinion. He gives the following example:

\begin{itemize}
    \item[(i)] \citep[242]{Ross.1967} 
    \item[] [Of which car]$_i$ were [the hoods~\trace{}$_i$] damaged by the explosion? 
\end{itemize}
This evidence has been dismissed by \citet{Chomsky.2008} as being extraction out of the subject of a passive, thus out of an underlying object, see below.}
This is what is usually meant by ``the Subject Island'' constraint in the literature. According to \citet[158]{Erteschik-Shir.1973}, the first proposal to extend the Sentential Subject Constraint to subject NPs was made by \citet{Horn.1972}. 

% and yet, Ross says that the sentential subject constraint is already too strong

The contrast between subject NPs and object NPs is often brought up as evidence that there is an ``island effect'' when subextracting from subjects. The extracted element is a PP dependent of the noun.\footnote{In Section~\ref{ch:previous-wh}, I present a handful of studies that looked at extraction of the specifier. Here is an example of the material tested by \citet{Jurka.2011}:
\begin{itemize}
    \item[(i)] \citep[125]{Jurka.2011}
    \item[] \gll Was$_i$ hat~[\trace{}$_i$ für ein Käfer] denn den Beamten gebissen?\\
    what has for a beetle indeed the clerk bitten\\
    \glt `What kind of beetle bit the clerk?'
\end{itemize}
} In languages allowing preposition stranding like English, the NP may be extracted alone, leaving the preposition in situ inside the subject or object NP. Example (\ref{ex:stories-about}) illustrates this contrast between subextraction from the subject NP (\ref{ex:stories-about-S}) and from the object NP (\ref{ex:stories-about-O}). \citegen{Chomsky.1973} introspective judgement is that the former is ungrammatical whereas the latter is grammatical. Most linguists working on this topic agree that the former is at least degraded compared to the latter.

\eal\label{ex:stories-about}
\ex \citep[248]{Chomsky.1973}\\
Who$_i$ did you hear [stories about~\trace{}$_i$]? \label{ex:stories-about-O}
\ex \citep[249]{Chomsky.1973}\\ \nopagebreak
 * Who$_i$ did [stories about~\trace{}$_i$] terrify John? \label{ex:stories-about-S}
\zl

It is traditionally assumed that the very same contrast holds when the whole PP complement is extracted, like in (\ref{ex:Chomsky-driver}). We will refer to this kind of extraction as pied-piping extraction, as opposed to preposition stranding extraction like in (\ref{ex:stories-about}). Again, there is agreement in the literature that (\ref{ex:Chomsky-driver-S}) is at least degraded compared to (\ref{ex:Chomsky-driver-O}). In Section~\ref{ch:previous-wh}, I present studies that have tested this contrast in \textit{wh}-questions.

\eal\label{ex:Chomsky-driver}
\ex \citep[147]{Chomsky.2008}\\
{ } [Of which car]$_i$ did they find [the driver~\trace{}$_i$]?
\label{ex:Chomsky-driver-O}
\ex \citep[153]{Chomsky.2008}\\
* [Of which car]$_i$ did [the driver~\trace{}$_i$] cause a scandal?
\label{ex:Chomsky-driver-S}
\zl 

Still, \citet[32]{Chomsky.1986} acknowledges that subextraction from the subject is ``more acceptable'' with pied-piping, as in (\ref{ex:pied-piping-better}). This observation is barely addressed by minimalists working on subject islands, and \citeauthor{Chomsky.1986} does not mention it in his subsequent works. 

\ea \citep[32]{Chomsky.1986}\\
He is the person [[of whom]$_i$ [pictures~\trace{}$_i$] are one the table].
\label{ex:pied-piping-better}
\z 

In their chapter about the possible location of gap, \citet[1093--1094]{Huddleston.2002} mention that gaps in a subject NP are only allowed as pied-piping, and give (\ref{ex:huddleston-felicitous}) as an example.

\ea \citep[1093]{Huddleston.2002}\\
They have eight children [[of whom]$_i$ [five~\trace{}$_i$] are still living at home]. 
\label{ex:huddleston-felicitous}
\z 

Notice that (\ref{ex:pied-piping-better}) and (\ref{ex:huddleston-felicitous}) are relative clauses, and not interrogatives like the examples cited before. \citet[32]{Chomsky.1986} notes that violations are ``less severe'' in relative clauses than in interrogatives, ``for unclear reasons''.\footnote{\citeauthor{Chomsky.1986} attributes the remarks concerning amelioration through pied-piping and relativization to \citet{Kuno.1972}, even though I can find no reference to it in that paper.} This remark was never addressed again by \citeauthor{Chomsky.1986} or any syntactic account of subject islands~-- probably because the violation of the Subject Condition is considered to be present in (\ref{ex:pied-piping-better}), and the effect therefore negligible. I suppose that example (\ref{ex:pied-piping-better}) should be marked as barely acceptable in \citeauthor{Chomsky.1986}'s view. 

\citet{Chomsky.1973} assumes that \textsc{pro} subjects are not available for subextraction, as in (\ref{ex:Chomsky-control}). In this example, \emph{stories about who}, although grammatically the direct object of \emph{expect}, is the underlying subject of \emph{terrify}. It also fills the role of the stimulus of \emph{terrify}, the semantic role associated with the subject for this experiencer-object psych verb. Being an underlying subject, this NP is an island to extraction, according to \citeauthor{Chomsky.1973}'s analysis.

\ea \citep[249]{Chomsky.1973}\\
 * Who$_i$ do you expect [stories about~\trace{}$_i$] to terrify John?
\label{ex:Chomsky-control}
\z 
% NB: actually, Chomsky 1973 (p 250) assumes that there is "no raising rule" in this case

Subject-to-object raising verbs (or verbs with Exceptional Case Marking in minimalist terms) should, however, allow extraction out of the raised argument NP (\citealt[20]{Chomsky.2005}; \citealt{Gallego.2007}; \citealt{Jimenez-Fernandez.2009}).

\ea \citep[109]{Jimenez-Fernandez.2009}\\
{}[Of which car]$_i$ did they believe [the driver~\trace{}$_i$] to have caused a scandal?
\z 
% Gallego & Uriagereka 2007: these kind of sentences are not {$\varphi$}-complete

Existential constructions like (\ref{ex:existential-cx}) involve an expletive, a copula and a predicative. The predicative is often treated as a subject in minimalist works.\footnote{Even though the details of the analyses differ from one scholar to the next, the general idea is that (i) and (\ref{ex:existential-cx}) share the same deep structure:
\begin{itemize}
    \item[(i)] [A picture of Grace Kelly] is on the wall.
\end{itemize}
In (i), the subject is base-generated in VP (or vP) and moves to Spec,TP (or Spec,IP). In (\ref{ex:existential-cx}), the subject does not move, and a dummy pronoun occupies Spec,TP (or Spec,IP) in order to fulfil the Extended Projection Principle (the requirement that every verb has a subject).}  
Some scholars \citep[e.g.][]{Stepanov.2007,Uriagereka.2012} assume that subextraction out of this predicative NP is grammatical, cf.\  (\ref{ex:existential-cx-subextraction}), but they disagree on why the subject island constraint does not apply in this case.

\eal \label{ex:passives-are-good}
\ex There is [a picture of Grace Kelly] on the wall. \label{ex:existential-cx}
\ex \citep[102]{Stepanov.2007}\\
Who$_i$ is there [a picture of \trace{}$_i$] on the wall?
\label{ex:existential-cx-subextraction}
\zl 

The status of the subject of passives is controversial. For some authors, it is more felicitous~-- or completely acceptable~-- to extract out of NP subjects of passives than, for example, out of subjects of transitives. Following \citet{Chomsky.2008}, this is because they are underlying objects and therefore not bound by the subject island constraint.

\eal \label{ex:passives-are-bad}
\ex \citep[268]{Kluender.1998}\\
? What$_i$ were [pictures of~\trace{}$_i$] seen around the globe?
\ex \citep[147]{Chomsky.2008}\\
{} It was the CAR (not the TRUCK) [of which]$_i$ [the driver~\trace{}$_i$] was found.  \label{ex:passiv-chomsky-cleft}
\ex \citep[147]{Chomsky.2008}\\
{} [Of which car]$_i$ was [the driver~\trace{}$_i$] awarded a prize?
\label{ex:passiv-chomsky}
\zl 

But others disagree, and hold that extraction out of the subject of a passive is ungrammatical:

\eal
\ex \citep[157]{Erteschik-Shir.1973}\\
 * Who$_i$ was [a picture of~\trace{}$_i$] painted by Picasso?
\ex \citep[85]{Stepanov.2007}\\
?* Who$_i$ was [a friend of~\trace{}$_i$] arrested?
\ex \citep[325]{Wexler.1980}\\
* It's of success [that$_i$ [our hopes~\trace{}$_i$] would be well rewarded].
\zl 

There is a similar disagreement about acceptability judgments for extraction out of subjects of unaccusatives. These disagreements seem to indicate that the data are borderline as far as acceptability is concerned. In general, syntactic approaches in which subjects are islands to extraction treat extraction out of subjects as ungrammatical (the reason why the sentences are unacceptable). Depending on the particular analysis of ``subject islands'', the theory predicts subjects of passives or unaccusatives to fall under the constraint or not (see below). Since the 2010s, there have been several attempts to gather empirical data in order to test the theory's predictions. I will present some of them in Section~\ref{ch:previous-exp}. But first, I will sketch the different lines of analyses that propose a syntactic reason for the contrast in (\ref{ex:stories-about}). 

%------------------------------------------------
\subsection{The start: \citet{Chomsky.1973}}

As far as subextraction from the subject is concerned, \citet{Chomsky.1973} differentiates between long-distance dependencies and short-distance dependencies. In his proposal, subextraction from the subject in short-distance dependencies like (\ref{ex:stories-about-S}) is ruled out by the Subject Condition, whereas subextraction from the subject in long-distance dependencies like (\ref{ex:Chomsky-control}) is ruled out by Subjacency. 

The Subject Condition is one of a series of rules on transformations. It is quite straightforward: no XP embedded in a subject may be subextracted from the subject (even though the term ``extraction'' is not used yet in \citet{Chomsky.1973}, for the phenomenon is conceived in terms of transformation). 

More precisely, the Subject Condition states that there can be no transformational dependency if the element to be transformed -- what we will call in this work the ``gap position'' -- is L-contained in a subject. For an XP to be L-contained in a YP, there must be at least one lexical element in YP that is not in XP: this ensures on the one hand that the extraction of a whole NP subject is possible, and at the same time that, for example, the NP-complement of a preposition does not fall under the Subject Condition (for reasons that do not concern me in this work).

The Subject Condition seems ad hoc, because it only accounts  for the Subject Island phenomenon and is not based on any independent evidence. It also seems arbitrary, because it is never justified in terms of cognitive processes. 

Extraction out of the subject in long-distance dependencies is ruled out by a more general rule based on Subjacency. Here is the definition of Subjacency given by \citeauthor{Chomsky.1973}: 
\pagebreak
\largerpage

\begin{quote}
 [I]f X is superior to Y in a phrase marker P, then Y is ``subjacent'' to X if there is at most one cyclic category C $\neq$ Y such that C contains Y and C does not contain X. \citep[247]{Chomsky.1973}
\end{quote}

That X is superior to Y means that it is higher in the syntactic tree: therefore, X is what we will call in this work the filler position, whereas Y is what we will call the gap position. Cyclic nodes are maximal projections in a sentence where syntactic information is processed and passed to PF and LF. For \citet{Chomsky.1973}, S and NP are the two kinds of cyclic nodes in human language. To paraphrase \citeauthor{Chomsky.1973} with our terminology, if it is true for more than one of such nodes in the sentence that it contains the gap position but not the filler position, then the gap is not subjacent to its filler. 

One of the general rules on transformations states that extraction can only take place if the gap is subjacent to its filler, otherwise the extraction is blocked.
Let me illustrate this constraint with the example of subextraction from the subject of an embedded clause we saw in (\ref{ex:Chomsky-control}). The tree in \figref{fig:stories-about-tree-S} gives the underlying (or deep) structure of (\ref{ex:Chomsky-control}), namely (\ref{ex:stories-about-tree-S}), in which cyclic nodes are circled. Notice that, following the traditional view in Transformation Grammar, the subject is considered to be base-generated outside of VP \citep{Chomsky.1965}. In order to form an interrogative, the \emph{wh}-word \emph{who} should occupy the position under the leftmost \textsc{comp}. There are two cyclic nodes containing the \emph{wh}-word but not the landing site, they are indicated in dashed circles. The NP \emph{who} is therefore not subjacent to the highest \textsc{comp}, and the transformation is blocked: (\ref{ex:Chomsky-control}) is ruled out.

\ea Underlying structure of (\ref{ex:Chomsky-control}):\\
you expect stories about who to terrify John
\label{ex:stories-about-tree-S}
\z 

\begin{figure}[ht]
\centering
\oneline{\begin{forest}
where n children=0{tier=word}{}
    [S,circle,draw
        [COMP$_1$,name=filler [{}, no edge]]
        [S'
        	[NP,circle,draw
    		    [you]
        	]
        	[VP
    	        [V
    	            [expect]
    	       ]
    	       [S,circle,draw,dashed
    	            [COMP$_2$, name=stop [{}, no edge]]
    	            [S'
    	                [NP,circle,draw,dashed,name=controlpointNP
        	        [N,name=controlpointN
        	            [stories]
        	       ]
                    [PP
                        [P
                            [about]
                        ]
                        [NP
                            [who, name=gap]
                        ]
                    ]
    	                ]
    	                [VP
    	                    [V
    	                        [to terrify]
    	                   ]
    	                   [NP, circle, draw
    	                        [John]
    	                   ]
    	                ]
    	            ]
    	       ]
            ]
        ]
    ]
\coordinate (controlpointone) at ($ (controlpointNP) !.5! (controlpointN) $);
\path [draw, densely dotted, ->] (gap.135) to [bend right] (controlpointone) to [bend left] (stop.south);
%\draw[->,densely dotted] ([xshift= -5pt,yshift = 2pt]gap.south) .. controls +(south west:1cm)  and +(south west:6cm)  .. ([xshift= -9pt,,yshift = 5pt]stop.south);
\draw[->,densely dotted] (stop) to[out=west,in=south,looseness=1.25] (filler);
\end{forest}}
\caption{Syntactic tree for ``you expect stories about who to terrify John'' according to \citet{Chomsky.1973}}
    \label{fig:stories-about-tree-S}
\end{figure}
 

Notice, however, that the transformation would be ruled out for any subextraction from an NP in the embedded clause, so this is not strictly speaking a ``subject island'' effect (there would be no contrast with subextraction out of an NP object). 

In contrast to the Subject Condition, constraints on Subjacency are ultimately explained in terms of cognitive capacities (the cyclic nodes  reduce the memory burden). %``biolinguistics''
They are also independently motivated, because the same mechanism is used to account for other putative islands, including subextraction from embedded questions. 
Embedded questions are often called ``\textit{wh}-islands''. \citet{Ross.1967} was the first to remark that these constructions were special with respect to extraction, but he also noticed that the restrictions were not absolute. For this reason, they belong to the weak islands for scholars who adopt this distinction \citep[e.g.][]{Cinque.1990}.

The constraints in \citet{Chomsky.1973} are problematic, because they are too restrictive.\largerpage{} The so-called \emph{wh}-islands have been shown to have many exceptions, depending on several different factors: the function of the extracted element (extraction of an adjunct is degraded compared to the extraction of an indirect or direct object, see (\ref{ex:wh-island-weak})), the nature of the filler of the embedded interrogative \citep{Kluender.1993.Subjacency}, the finiteness of the embedded question, and the specificity of the extracted element or of the \emph{wh}-filler of the embedded question \citep{Kluender.1998}.\footnote{For a discussion of French data and the difference in acceptability between extraction of an adjunct, an indirect object or a direct object on the one hand, and extraction out of \emph{wh}-embedded questions and \emph{si}-embedded questions (similar to English \emph{if}-embedded questions) on the other hand, see \citet{Hirschbuehler.1992}.}
And yet, all extractions out of an embedded question should be excluded by Subjacency to the same degree. 

\eal \label{ex:wh-island-weak}
\ex \citep[1]{Cinque.1990}\\
* How$_i$ did they ask you [who$_j$~\trace{}$_j$ behaved~\trace{}$_i$]?
\ex \citep[1]{Cinque.1990}\\
{} [To whom]$_i$ didn't they know [when to give their present~\trace{}$_i$]?
\ex \citep[494]{Szabolcsi.2006}\\
{} [Which problem]$_i$ did John ask [how to phrase~\trace{}$_i$]?
\zl 

\begin{sloppypar}
Subjacency also rules out subextraction out of a dependent of the dependent of a noun like (\ref{ex:dep-of-dep-bad}).
\end{sloppypar}


\ea \citep[248]{Chomsky.1973}\\
* What$_i$ do you receive [requests for [articles about \trace{}$_i$]]?
\label{ex:dep-of-dep-bad}
\z 

This is problematic, because felicitous cases of such structures are very easy to find. \citet{Ross.1967} cites example (\ref{ex:dep-of-dep-good-ross}) for English. 
I was also able to find many such examples for French in the corpus frWaC \citep{Baroni.2009}. Example (\ref{ex:dep-of-dep-good-frwac}) is one of them. 

\eal\label{ex:dep-of-dep-good}
\ex \citep[15]{Ross.1967}\\  % also: \citep[248]{Chomsky.1973}
{} [What books]$_i$ does the government prescribe [the height of [the lettering of \trace{}$_i$]]?
\label{ex:dep-of-dep-good-ross}
\ex \gll Il présente cependant les défauts traditionnels des textes internationaux [dont$_i$ [la réalité de [l' application~\trace{}$_i$]] est très rarement contrôlée].\\
it has however the flaws traditional of.the texts international \sbar{}of.which \sbar{}the reality of \sbar{}the application is very rarely controlled\\
\glt `It has the traditional flaws of international texts, the reality of the application of which is rarely controlled.'
\label{ex:dep-of-dep-good-frwac}
\zl 

Since \citet{Chomsky.1973}, there have been a vast number of syntactic approaches to the subject island constraint. In my view, they can be divided into three principal groups, which I will now present briefly.

%(Chomsky, 1957, 1975, 1977, 1989, 1995, 1998).
% also 1973, 1981 and 1986
% subjacency is first categorial (good/bad), with 1986 it becomes continuous : cumulative effect of crossing barriers 

%------------------------------------------------
\subsection{The subject is a non-complement: The CED and its successors}

\subsubsection{The Condition on Extraction Domain}

\citet{Huang.1982} introduces another island into the list proposed by \citet{Ross.1967}: the adjunct island. For him, the ban on extraction out of adjuncts like (\ref{ex:adjunct-island-intro}) is parallel to the ban on extraction out of subjects. 

\ea[*]{Who$_i$ did Mary cry [after John hit~\trace{}$_i$]?}
\label{ex:adjunct-island-intro}
\z 

He considers the Subject Condition to be only ``a special case of an even more general asymmetry between complements on the one hand and non-complements (subjects and adjuncts) on the other'' \citep[503]{Huang.1982}. He subsumes \citegen{Chomsky.1973} Subject Condition with \citegen{Kayne.1981}  Empty Category Principle in a constraint called the ``Condition on Extraction Domain'', which is reproduced in (\ref{rule:CED}). 
% Kayne ECP: an empty category must be properly governed and c-commanded by its antecedent, which also must be governed by a lexical object (my formulation) 


\ea Condition on Extraction Domain \citep[505]{Huang.1982}\\
A phrase A may be extracted out of a domain B only if B
is properly governed.
\label{rule:CED}
\z 

Since objects are complements of V, they are c-commanded by a lexical head, and thus ``properly governed''. For this reason, extraction out of NP objects or sentential complements is possible. Subjects and adjuncts are not c-commanded by V, therefore subextraction out of subjects and adjuncts is blocked, following \citeauthor{Huang.1982}.

\citegen{Huang.1982} CED inspired a lot of subsequent work on subject islands, and started the tradition of considering the Subject Island and the adjunct island under the same constraint. An important successor of \citegen{Huang.1982} CED is the concept of ``Barriers'' in \citet{Chomsky.1986}. 


\subsubsection{Barriers}
\label{ch:barriers}

In the 80's, \citeauthor{Chomsky.1986} model of sentence construction had changed compared to \citet{Chomsky.1973}. During this period, the operation \textsc{move} had been introduced. In \citet{Chomsky.1986}, he assumes that a moved element leaves a trace (\emph{t}) at the position it occupies at deep structure, and at any position occupied during movement. The movement of \emph{wh}-words (\textit{wh}-movement) is cyclic, and the landing site of each cyclic movement must be a specifier position. Notice also that the subject is considered to be base-generated in Spec,IP (this view has been abandoned since, see below).
The tree in \figref{fig:stories-about-tree-0-barrier} illustrates how (felicitous) movement works in \citegen{Chomsky.1986} account: \emph{who} first moves from its initial position to the specifier position of VP 
% the specifier of NP is not available for movement?
and from that position to the specifier position of CP (notice that the specifier of IP is not available for movement, \citealt[32]{Chomsky.1986}). There is also a second movement involved here: head-movement of \emph{did} from I to C, but this is not \emph{wh}-movement. 

\ea[]{Who did you hear stories about? [see (\ref{ex:stories-about-O})]}
\label{ex:stories-about-tree-0-barrier}
\z 

\begin{figure}[ht]
\centering
\begin{forest}
where n children=0{tier=word}{}
[CP
    [Spec [who$_i$, name = filler]]
    [C'
        [C [did$_j$]]
        [IP, circle, draw, dashed
            [NP$_1$ [you]]
            [I'
                [I [\emph{t}$_j$]]
                [VP
                    [Spec [\emph{t}$_i$, name = stop]]
                    [V'
                        [V [hear]]
                        [NP$_2$
                            [N [stories]]
                            [PP
                                [P [about]]
                                [NP$_3$ [\emph{t}$_i$, name = gap]]
                            ]
                        ]
                    ]
                ]
            ]
        ]
    ]
]
\draw[->,densely dotted] (gap.south) |- ++(-1ex, -.75\baselineskip) -| (stop.285);
\draw[->,densely dotted] (stop.255) |- ++(-1ex, -.75\baselineskip) -| (filler);
\end{forest}
\caption{Syntactic tree for ``Who did you hear stories about?'' according to \citet{Chomsky.1986}}
    \label{fig:stories-about-tree-0-barrier}
\end{figure} 

``Cyclic categories'' of \citet{Chomsky.1973} have now been replaced by ``blocking categories'' and ``barriers''. Any maximal projection which is not directly assigned a $\theta$-role by a lexical category is a blocking category for any element it contains. A blocking category becomes a barrier for movement\footnote{Other rules apply to block other kinds of government that we will not discuss here.} under certain conditions given in \ref{rule:barrier}.% even a non-blocking category can become a barrier, because of (a).

\ea\label{rule:barrier}
$\gamma$ is a barrier for $\beta$ iff (a) or (b):\\
(a) $\gamma$ immediatly dominates $\delta$, $\delta$ a [blocking category] for $\beta$\\
(b) $\gamma$ is a [blocking category] for $\beta$, $\gamma$ $\neq$ IP
\z 

Subjacency receives a new definition based on the concept of barriers as defined in (\ref{rule:barrier}): an element is 0-subjacent to a (dominating) landing site if there is no barrier between them; it is 1-subjacent if there is one barrier crossed, 2-subjacent if there are two barriers, etc. 

Let us illustrate subject islands as explained by subjacency by comparing (\ref{ex:stories-about-tree-0-barrier}) whose analysis is shown in \figref{fig:stories-about-tree-0-barrier} and (\ref{ex:stories-about-tree-2-barrier}) whose analysis is shown in \figref{fig:stories-about-tree-2-barrier}. Blocking categories are circled; we used a plain line for those that are barriers, and a dashed line for those that are not. 

\ea[]{Who did stories about terrified John? [see (\ref{ex:stories-about-S})]}
\label{ex:stories-about-tree-2-barrier}
\z 

In \figref{fig:stories-about-tree-0-barrier}, NP$_2$ is directly assigned a $\theta$-role by V and thus not a blocking category; the VP is per definition lexical and thus not a blocking category; and the IP, even though it is a blocking category, is not a barrier because of (\ref{rule:barrier}b). The initial position of \emph{who} is hence 0-subjacent to its landing position.

\begin{figure}[ht]
\centering
\begin{forest}
where n children=0{tier=word}{}
[CP
    [Spec [who$_i$, name = filler]]
    [C'
        [C [did$_j$]]
        [IP, circle, draw
            [NP$_1$, circle, draw
                [N [stories]]
                [PP
                    [P [about]]
                    [NP$_2$ [\emph{t}$_i$, name = gap]]
                ]
            ]
            [I'
                [I [\emph{t}$_j$]]
                [VP
                    [V [terrified]]
                    [NP$_3$
                        [John]
                    ]
                ]
            ]
        ]
    ]
]
%\draw[->,densely dotted] ([xshift= -5pt,yshift = 2pt]gap.south) .. controls +(south west:2cm)  and +(south west:5.5cm)  .. ([xshift= -9pt,,yshift = 5pt]filler.south);
\draw[->,densely dotted] (gap) |- ++(-1ex, -1.5\baselineskip) -| (filler);
\end{forest}
\caption{Syntactic tree for ``Who did stories about terrified John?'' according to \citet{Chomsky.1986}}
    \label{fig:stories-about-tree-2-barrier}
\end{figure} 


The tree in \figref{fig:stories-about-tree-2-barrier}, on the other hand, illustrates what happens when extracting out of a subject. The PP is directly assigned a $\theta$-role by N and is therefore not a blocking category. But NP$_1$, being base-generated outside of VP, is not directly assigned a $\theta$-role\footnote{Subjects are only indirectly $\theta$-marked \citep[cf.][13]{Chomsky.1986}, because they receive their $\theta$-marking from the VP and not from the head V \citep[37]{Chomsky.1981}. Only lexical words can L-mark their arguments, phrases cannot.\label{fn:theta-indirect-marking-subject}} and is a blocking category and a barrier because of (\ref{rule:barrier}b). Furthermore, the IP is a blocking category, and also a barrier because of (\ref{rule:barrier}a). The initial position of \emph{who} is hence 2-subjacent to its landing position. 

The link between subjacency and acceptability is explicitly stated: the more subjacent the trace is relative to its landing position, the less acceptable the sentence. Thus \citet{Chomsky.1986} introduces a notion of gradient acceptability in his theory, although the distinction used to be categorial. In general, if we follow acceptability judgments in his work, it seems that 0-subjacent dependencies are acceptable, 1-subjacent dependencies are awkward (?) and 2-subjacent dependencies ungrammatical or unacceptable (*).

\citet[82]{Stepanov.2007} criticizes the theory of barriers, because ``$\theta$-theory and bounding theory (responsible for the locality of movement including extraction) are different modules of core grammar, driven by separate sets of principles''. 

\citet{Deane.1991}, on the other hand, noticed that the definition of Barriers, like Subjacency in \citet{Chomsky.1973}, is too strong. It excludes acceptable examples like (\ref{ex:Deane-Nixon}), in which the relative word crosses one IP and two NPs. Example (\ref{ex:Deane-Nixon}) is reminiscent of example (\ref{ex:dep-of-dep-good-ross}) cited before.

\ea \citep[10]{Deane.1991}\\
Nixon was one president that$_i$ [they had no trouble getting [votes for [the impeachment of~\trace{}$_i$]\textsubscript{DP}]\textsubscript{DP}]\textsubscript{IP}. 
\label{ex:Deane-Nixon}
\z 

For French, counterexamples similar to (\ref{ex:Deane-Nixon}) have been produced by \citet{Godard.1988}. Yet none of them is subextraction from a subject or an adjunct. 

\eal 
\ex \citep[38]{Godard.1988}\nopagebreak\\
\gll La pièce$_i$ que$_i$ [l' évolution de la situation politique donnait [l' impression [qu' on pouvait enfin monter~\trace{}$_i$]\textsubscript{IP}]\textsubscript{NP} n' a pourtant pas été autorisée.\\
the play that \sbar{}the evolution of the situation political gave \sbar{}the impression \sbar{}that one could finally create \textsc{neg} has yet not been allowed\\
\glt `The play that the political development gave the impression that it would be finally possible to create (it) has yet not been allowed.' 
\label{ex:Godard-situation-politique}
\pagebreak
\ex \citep[59]{Godard.1988}\\
\gll un problème$_i$ auquel$_i$ il me semble [qu' on m' a dit [que tu t' étais attaqué \trace{}$_i$]\textsubscript{IP}]\textsubscript{IP}\\
a problem at.which it \textsc{1.sg.dat} seems that one \textsc{1.sg.acc} has told that you \textsc{refl} were tackle\\
\glt `a problem that I believe that someone told me that you tackled'
\label{ex:Godard-auquel-ldd}
\zl 

%Successors: Chomsky and Lasnik 1993 (p79) a barrier is ``an XP that is not a complement''. 
% CHOMSKY, N. & H. LASNIK. 1993. The theory of principles and parameters. In Syntax: An international handbook of contemporary research, Vol. 1, ed. J. Jacobs, A. von Stechow, W. Sternefeld & T. Venneman, 506–569. Berlin: Mouton de Gruyter.+

%------------------------------------------------

\subsection{The subject is a specifier: The Connectedness Condition and other ``specifier''-based analyses}

\subsubsection{The Connectedness Condition}

In contrast to \citeauthor{Chomsky.1973} and \citeauthor{Huang.1982}, \citet{Kayne.1983} does not assume that the subject island constraint holds in all languages. Indeed, as already noticed by \citet{Ross.1967}, Japanese seems to allow extraction out of subjects. Instead of treating the constraint as language-specific, as \citeauthor{Ross.1967} did, \citeauthor{Kayne.1983} proposes that it is a consequence of the canonical government configuration which differs between English and French on the one hand and Japanese on the other hand. In English and French, which are SVO languages, the verb canonically governs its complement on the right, while in Japanese, which is an SOV language, the verb canonically governs its complement on the left.

Following \citet[225]{Kayne.1983}, an extracted element must be inside an XP that is canonically governed by the verb. In Japanese, the subject is on the left of the V, therefore it is canonically governed, and subextraction out of the subject is possible. In English and French, the subject is on the left of the V, and is therefore not canonically governed, so subextraction out of the subject is banned.

With this rule, \citet{Kayne.1983} does not only account for the subject island, but also for the Left Branch Condition in \citet[207--217]{Ross.1967}, which stated that no leftmost element of an NP may be extracted out of the NP. 

\begin{exe} \ex \citep[208]{Ross.1967}
\begin{xlist}
\ex[]{We elected [the boy's guardian's employer]\textsubscript{NP} president.}
\ex[*]{The boy [[whose guardian's]$_i$ we elected~[\trace{}$_i$ employer] president] ratted on us.}
\ex[*]{The boy [whose$_i$ we elected~[\trace{}$_i$ guardian's employer] president] ratted on us.}
\end{xlist}
\end{exe}

Indeed, elements in the specifier position of NPs are not canonically governed following the Connectedness Condition. 

\citet{Longobardi.1985} takes over \citegen{Kayne.1983} Connectedness Condition and applies it to adjunct islands. It is relatively straightforward that extraction out of an AdvP that is situated on the left of V is ruled out by the Connectedness Condition. But even AdvP on the right of the verb are analyzed by \citet{Longobardi.1985} as being sisters to the VP, and thus not c-commanded by the verb \citep[see ex.\ 14 in][157]{Longobardi.1984}. Subextraction would then be ruled out. 

\subsubsection{Phase Impenetrability}

To understand phrase impenetrability, we have to start with \citegen{Chomsky.2001} decision to make the notion of ``phases'' a core concept of his analysis. This new turn is known as the Phase Theory. Phases are comparable to barriers. I will summarize here the analysis in \citet{Chomsky.2008}, but, even though the theory has evolved since 2001, the general idea of the phase impenetrability has remained unchanged. 

Some maximal projections, namely v*P and CP, are phases. Sentence formation proceeds in a bottom-up fashion, thus the lower phases are formed before the higher ones, one at a time. At the Edge of a given phase, the information inside this maximal projection is transferred to the phonetic and semantic interface. 
After the transfer, the phase remains impenetrable for heads higher in the tree, and no element can move out of the phase. The only way an element can undergo movement higher in the syntactic tree is by moving to the Edge of the maximal projection prior to the Transfer.\footnote{\citet{Chomsky.2008}, to my knowledge, never explicitly describes what the Edge is. In the following, I assume that the Edge is a sister to the XP (or X') situated on its left, some sort of second specifier above the specifier.}
\figref{fig:phase-theory-object} illustrates a simple extraction out of the object following the Phase Theory. Phases are identified with circles.\footnote{It is generally assumed that at least the agentive subjects are base-generated under Spec,vP (or VoiceP, which is roughly similar to vP). For example, \citet{Kratzer.1996} proposes that there are two different VoicePs, one for licensing agentive subjects and one for licensing holder subjects, and \citet[Chapter 2]{Alexiadou.2015} distinguish three different VoicePs, so that causer subjects are also external arguments. This implies that some subjects, such as experiencer subjects, would be internal arguments. Does this mean that these analyses do not predict an island effect for non-agentive/non-holder/non-causative subjects? As far as I know, this question has not been addressed by these authors. In any case, the answer has no bearing on the interpretation of the empirical data that I present in Part~\ref{part:2}. I will argue this point in Section~\ref{ch:psych-verb-no-consequence}, in which I briefly return to the issue of experiencer subjects.}

Since \citet{Chomsky.1995}, following a proposal by \citet{Kitagawa.1986} and \citet{Koopman.1991}, \citeauthor{Chomsky.1995} assumes that (some) subjects are base-generated in the specifier position of the functional projection vP (or v*P).
No movement is allowed from a specifier to the Edge of the same maximal projection. That is why extraction out of these subjects is not allowed: as illustrated in \figref{fig:phase-theory-subject}, the complement of the subject noun cannot move to the Edge of the v*P, and is therefore transferred to the phonetic and semantic interfaces where it is no longer accessible for C. Hence the subextraction is ungrammatical.

\begin{figure}
\begin{forest}
where n children=0{tier=word}{}
[CP, circle, draw
    [Spec\\{[}+wh{]} [who$_i$, name = filler]]
    [C'
        [C [did]]
        [v*P, circle, draw
            [Edge [\emph{t}$_i$, name = stop]]
            [v*P
                [Spec [you]]
                [VP
                    [V [hear]]
                    [NP
                        [stories about \emph{t}$_i$, roof, name = gap]
                    ]
                ]
            ]
        ]
    ]
]
%\draw[->,densely dotted] ([xshift= -50pt]gap) to[out=south west,in=south east] (stop);
\draw[->,densely dotted] (gap.340) |- ++(-1ex,-.75\baselineskip)  -| (stop.285);
\draw[->,densely dotted] (stop.255) |- ++(-1ex,-.75\baselineskip)  -| (filler);
\end{forest}
\caption{Syntactic tree for ``Who$_i$ did you hear [stories about~\trace{}$_i$]?'' according to \citet{Chomsky.2008}}
    \label{fig:phase-theory-object}
\end{figure} 

\begin{figure}
\begin{forest}
where n children=0{tier=word}{}
[CP, circle, draw
    [Spec\\{[}+wh{]} [-, name = filler]]
    [C'
        [C [did]]
        [v*P, circle, draw
            [Edge [-, name = stop]]
            [v*P
                [Spec [stories about \emph{x}, roof, name = gap]]
                [VP
                    [V [terrify]]
                    [NP
                        [John]
                    ]
                ]
            ]
        ]
    ]
]
\draw[->,densely dotted] (gap.south) |- ++(-1ex,-.75\baselineskip) -| node[near start] {\ding{56}} (stop);
\end{forest}
\caption{Syntactic tree for ``Who$_i$ did [stories about~\trace{}$_i$] terrify John?'' according to \citet{Chomsky.2008}}
    \label{fig:phase-theory-subject}
\end{figure} 

The verbal projection v*P is only projected for verbs that require an external argument, such as transitive verbs. When subjects are internal arguments (underlying objects), there is no v*P, and therefore no phase. That is why subextraction out of the subject of passives, like (\ref{ex:passiv-phase-theory}), is acceptable. The analysis is shown in \figref{fig:passiv-phase-theory} on page \pageref{fig:passiv-phase-theory}. For the same reason, extraction out of subjects or unaccusatives is predicted to be grammatical.\footnote{This prediction of Phase Theory is not addressed by \citeauthor{Chomsky.2008}, but has been underlined by \citet{Polinsky.2013} and \citet{Haegeman.2014}. \citet{Perlmutter.1978} proposed a distinction between unaccusative and unergative verbs, based on their distinct syntactic behavior that correlates with distinct semantic properties. This hypothesis has been formalized in the GB framework by \citet{Burzio.1986}. In his approach, at deep structure, unergative verbs take an external argument (subject) and no internal argument, while unaccusative verbs take an internal argument (object) and no external argument. With both kinds of verbs, at the surface, the argument appears outside of VP, in the subject position. The consequence for Phase Theory, as \citeauthor{Polinsky.2013} notice, is that extraction out of subjects of unergatives should be ungrammatical, whereas extraction out of subjects of unaccusatives should be grammatical. They conducted an experiment to test this prediction, see Section~\ref{ch:previous-exp}.\label{fn:unaccusative}}

\ea Who$_i$ were [stories about~\trace{}$_i$] written down?
\label{ex:passiv-phase-theory}
\z 

\begin{figure}
\begin{forest}
where n children=0{tier=word}{}
[CP, circle, draw
    [Spec\\{[}+wh{]} [who$_i$, name = filler]]
    [C'
        [C [were]]
        [vP
            [Edge [\emph{t}$_i$, name = stop]]
            [vP
                [Spec [{[}stories about \emph{t}$_i${]}$_j$, roof, name = sujet]]
                [VP
                    [V [written down, roof]]
                    [NP
                        [\emph{t}$_j$, name = gap]
                    ]
                ]
            ]
        ]
    ]
]
\draw[->,densely dotted] (sujet.340) |- ++(-1ex,-0.75\baselineskip) -| (stop.285);
\draw[->,densely dotted] (stop.255)  |- ++(-1ex,-0.75\baselineskip) -| (filler);
\draw[->,densely dotted] (gap.south) |- ++(-1ex,-0.75\baselineskip) -| (sujet.343);
\end{forest}
\caption{Syntactic tree for ``Who$_i$ were [stories about~\trace{}$_i$] written down?'' according to \citet{Chomsky.2008}}
    \label{fig:passiv-phase-theory}
\end{figure} 

\subsubsection{Spell out}

The analysis of Spell out developed in \citet{Nunes.2000} and \citet{Uriagereka.2012} (a.o.) borrows some elements from Phase Theory, while taking into account the incremental processing of sentences. I will summarize here the analysis in \citet{Uriagereka.2012}, which incorporates many of the elements developed in previous work by \citeauthor{Uriagereka.2012} and his colleagues. The starting point of this analysis is the Linear Correspondence Theorem (LCT), which states, in a nutshell, that asymmetrical branching (i.e.\ a pair of branches with one terminal node and one non-terminal node) \citep[53]{Uriagereka.2012} and a configuration in which the terminal node precedes the non-terminal one \citep[56]{Uriagereka.2012} is easier to parse than other combinations.\footnote{The Binary Principle \citep{Culicover.1977} was a very similar idea.}
This is illustrated by \figref{fig:LCT} on page \pageref{fig:LCT}: the tree in (b) first branches into two non-terminal nodes and thus violates the first part of the LCT (Finite State Limit); in the tree in (c), the non-terminal nodes precede the terminal ones, which violates the second part of the LCT (Linear Correspondence Axiom). 
%\citeauthor{Uriagereka.2012} underlines that the leftward movement of extracted elements has as a consequence that the \emph{Wh} operator appears before the variable that is bound by the operator. 

Whenever we have to parse a syntactic structure that does not follow the LCT, the tree is chunked into different sub-trees that are in accordance with the LCT and are then parsed in parallel. A sentence with a complex subject is similar to the tree (b) in \figref{fig:LCT}. The NP subject is then parsed as a unit: this is the operation called Spell out. After Spell out, the structure is no longer phrasal, but it is treated as a word. The mechanism is illustrated by \figref{fig:Spell-out}. The internal structure of the subject is hardly accessible, because of Spell out.

\begin{figure}
\begin{subfigure}[b]{.25\textwidth}\centering
\begin{forest}
[A
    [B]
    [C
        [D]
        [E
            [F]
            [\dots]
        ]
    ]
]
\end{forest}
\caption{}
\end{subfigure}\begin{subfigure}[b]{.5\textwidth}\centering
\begin{forest}
[A
    [B
        [C]
        [D]
    ]
    [E
        [F]
        [G
            [H]
            [\dots]
        ]
    ]
]
\end{forest}
\caption{}
\end{subfigure}\begin{subfigure}[b]{.25\textwidth}\centering
\begin{forest}
[A
    [B
        [C
            [\dots]
            [D]
        ]
        [E]
    ]
    [F]
]
\end{forest}
\caption{}
\end{subfigure}
\caption{Illustration of the Linear Correspondence Theorem; Tree (a) is in accordance with the Linear Correspondence Theorem; Tree (b) violates the Finite State Limit; Tree (c) violates the Linear Correspondence Axiom}
    \label{fig:LCT}
\end{figure} 

\begin{figure}[ht]
\centering
\begin{minipage}[c]{.35\textwidth}
\begin{forest}
		sn edges,
		[VP
			[DP, name=Ph1
				[D]
				[NP]
			]
			[V'
				[V]
				[DP
					[D]
					[NP]
				]
			]
		]
\phase[dashed,very thick]{Ph1}
\end{forest} 
\end{minipage}
\quad \begin{minipage}[c]{2em} $\rightarrow$ \end{minipage} \quad 
\begin{minipage}[c]{.35\textwidth}
\begin{forest}
sn edges,
[VP
	[wordlike, name=Ph1
		[D+NP, roof]
	]
	[V'
	    [V]
	    [DP
	        [D]
	        [NP]
	    ]
	]
]
\phase[dashed,very thick]{Ph1}
\end{forest}
\end{minipage}
\caption{Spell out for a sentence with a complex subject}
    \label{fig:Spell-out}
\end{figure} 

Subextracting out of the subject implies that the subject is complex, and thus subextraction is made very difficult by Spell out. The same applies to adjuncts, which are considered to modify maximal projections: the VP and the adjunct XP are built in parallel. 

Ultimately, in \citegen{Uriagereka.2012} account, the cause of the subject island constraint is the fact that the subject is in a specifier position, but for reasons different from the proposal of \citet{Kayne.1983} or of \citet{Chomsky.2008}. Also, unlike these previous analyses, the subject is not completely opaque for subextraction. It is merely hard to interpret, because ``any extraction from there [the subject] will involve material within something that is not a syntactic object'' \citep[92--93]{Uriagereka.2012}. Furthermore, subextraction out of some subjects may face an additional difficulty which I explore further in the next section.

%Problem pointed out by \citet[94]{Stepanov.2007}: There is [SC [DP a picture of who] [on the wall]]: here [a picture of who] should be linearized and not accessible anymore, but \emph{Who is there a picture of on the wall?} is felicitous.

%------------------------------------------------

\subsection{The subject is moved: Freezing analyses}

A historical overview of the analysis of Freezing effects can be found in \citet{Corver.2006}. He finds the origin of these analyses in \citegen{Ross.1967} constraint on extraposition  and Heavy NP shift: no subextraction can take place out of an element that has moved to the right. I present here three influential approaches that attribute the difficulty to extract out of a subject to the fact that the subjects typically move to Spec,IP, a view commonly adopted in Minimalism (see above). 

\subsubsection{The  Freezing  Principle}

The Freezing Principle was originally formalized in \citet{Wexler.1980}. The part of the Freezing Principle that is relevant for our topic is the Raising Principle, reproduced in (\ref{rule:raising-principle}).

\ea If a node A is raised, then no node that A dominates may be used to fit a transformation.\footnote{Or, in other words: ``If a node has been raised, it is
frozen; that is, further transformations may not analyze nodes below the raised node.'' \citep[28]{Wexler.1980}.} \citep[341]{Wexler.1980}
\label{rule:raising-principle}
\z 

 \citeauthor{Wexler.1980} used the Raising Principle to account for a ban on extraction out of extraposed elements like (\ref{ex:extraposition-freezing}).\footnote{The extraposed sentential complement in (\ref{ex:extraposition-freezing}) is ``raised'' because \citet{Wexler.1980} define raising as movement from inside a cyclic category (NP or S) to inside another cyclic category higher in the syntactic tree. In (\ref{ex:extraposition-freezing-nomove}), the cyclic category of the sentential complement is the subject NP. In (\ref{ex:extraposition-freezing}), \citet{Wexler.1980} assume that the sentential complement has moved to a position sister to V; its cyclic category is hence the matrix S, which is higher in the tree.} 

\begin{exe}
\ex \citep[341--342]{Wexler.1980}
    \begin{xlist}
    \ex[]{[A suspicion~\trace{}$_i$] has arisen [that you have been holding back on the IRS]$_i$.} \label{ex:extraposition-freezing-nomove}
    \ex[*]{The IRS is the government agency [[that]$_j$ [a suspicion~\trace{}$_i$] has arisen [that you are holding back on~\trace{}$_j$]$_i$]. } \label{ex:extraposition-freezing}
    \end{xlist}
\end{exe}

Based on some evidence from German and Dutch, \citet{Mueller.G.1998} proposed to generalize the Freezing Principle to any kind of movement. For example, in German, extraction out of an element that has scrambled rightward from its canonical position in the middlefield is degraded compared to the same subextraction from the canonical position. Compare extraction out of the scrambled object NP in (\ref{ex:scrambling-freezing}) with extraction out of the object NP in situ in (\ref{ex:noscrambling-freezing}). 

\begin{exe}
\ex \citep[20]{Mueller.G.1998}
\begin{xlist}
\ex[*]{\gll Worüber$_i$ hat [ein Buch~\trace{}$_i$]$_j$ keiner~\trace{}$_j$ gelesen?\\
about.what has a book nobody read\\} \label{ex:scrambling-freezing}
\ex[]{\gll Worüber$_i$ hat keiner [ein Buch~\trace{}$_i$] gelesen?\\
about.what has nobody a book read\\
\label{ex:noscrambling-freezing}
\glt `About what did nobody read a book?'}        
\end{xlist}
\end{exe}

\citet{Gallego.2007} discuss what they consider to be a similar contrast in English:

\eal\label{ex:stepanov-picture-on-the-wall}
\ex \citep[80]{Stepanov.2007}\\
?* Who$_i$ does [a picture of \trace{}$_i$] hang on the wall?  \label{ex:stepanov-picture-on-the-wall-bad}
\ex \citep[102]{Stepanov.2007}\\
Who$_i$ is there [a picture of \trace{}$_i$] on the wall?  \label{ex:stepanov-picture-on-the-wall-good}
\zl 

According to them, in (\ref{ex:stepanov-picture-on-the-wall-bad}), the NP \emph{a picture of who} first undergoes movement to Spec,TP, and for this reason the subsequent movement from \emph{who} from Spec,TP to Spec,CP is ruled out. By contrast, the subject NP \emph{a picture of who} in (\ref{ex:stepanov-picture-on-the-wall-good}) does not move to Spec,TP. Instead, Spec,TP is occupied by \emph{there} in order to fulfil the EPP requirement, and movement of \emph{who} to Spec,CP is felicitous. 

\citet[103]{Boeckx.2003} explains Freezing by the principle of feature checking. Movement is triggered by the need of an element to check its features: the subject moves because of the Case-feature. Once an element has checked its features, it becomes ``inert''. For this reason, agreement and extraction are no longer possible. 

Notice however that \citet{Mueller.G.1998} still assumes Subjacency, because he considers extraction out of in-situ subjects to be ungrammatical:

\ea \citep[30]{Mueller.G.1998}\\
\gll * [Über wen]$_i$ hat [ein Buch~\trace{}$_i$] den Karl beeindruckt?\\
{} about who has a book the\textsc{.acc} Karl impressed\\
\glt `About whom did a book impress Karl?'
\label{ex:german-subject-insitu}
\z\largerpage[2.25]

In example (\ref{ex:german-subject-insitu}), the subject was not scrambled, so Freezing cannot be the reason for the ungrammaticality.{\interfootnotelinepenalty=10000\footnote{According to \citet{Mueller.G.1998}, subjects of passives  are underlying objects and do not need to move to a subject position in German. Extraction out of them is felicitous:
\ea \citep[124]{Mueller.G.1998}\\
    \gll Worüber$_i$ ist von keinem [ein Buch~\trace{}$_i$] gelesen worden?\\
    about.what \textsc{aux} by nobody a book read been\\
    \glt `About what was a book read by nobody?'
\z Such subjects are not frozen as they are internal arguments.}}
\citegen{Gallego.2007} explanation for the grammaticality of (\ref{ex:stepanov-picture-on-the-wall-good}) is therefore not compatible with \citegen{Mueller.G.1998} analysis.

\subsubsection{Chain Uniformity}

\citet{Takahashi.1994} defines two main constraints on movement, which are reproduced in (\ref{rule:chain-uniformity}).

\eal\label{rule:chain-uniformity} 
\ex Shortest Movement Condition (SMC): Make the Shortest Movement. \citep[8]{Takahashi.1994}
\ex Uniformity Corollary on Adjunction (UCA): Adjunction is impossible to a proper subpart of a uniform group where a uniform group is a nontrivial chain or a coordination. \citep[25]{Takahashi.1994}
\zl 

Let me add some explanation in order to better understand the constraints in (\ref{rule:chain-uniformity}). In his definition of SMC, \citeauthor{Takahashi.1994} assumes cyclic movement, in which the extracted element is copied into the nearest appropriate position, then into the next one, and so on until the final position. Lower copies of an element are deleted at surface structure. \emph{Wh}-movement is seen as movement to a specifier position (cyclically from specifier to specifier). Furthermore, \emph{wh}-movement (extraction) is analyzed as adjunction rather than substitution, and a nontrivial chain means that the element has a copy, hence that it has moved. This is what makes the UCA ultimately similar to Freezing, except that it also accounts for the Coordinate Structure Constraint.

In order to extract out of a subject that has moved from Spec,VP to Spec,IP\footnote{\citeauthor{Takahashi.1994} assumes this movement for English and French subjects \citep[28]{Takahashi.1994}, but not for Japanese subjects \citep[65]{Takahashi.1994}, hence the acceptability of extracting out of the subject in Japanese, see below.}, there are two options: (i) the closest specifier position is Spec,DP of the subject DP~-- but this is disallowed by the UCA~-- or (ii) the copy is adjoined to Spec,IP, which violates the SMC. Hence, it is not possible to have fully acceptable subextraction from a subject.

\citet{Uriagereka.2012} combines the constraints on Spell out (see above) with \citegen{Takahashi.1994} SMC and UCA in order to account for subject island effects. Recall that the effects of Spell out are not seen as categorical: extracting out of a complex NP in specifier position makes parsing more complicated, but not impossible. Shortest Move is also considered by \citeauthor{Uriagereka.2012} as a preference, rather than a rule. According to him, this explains a contrast found in Spanish in extraction out of preverbal vs.\ postverbal subjects. 

\eal 
\judgewidth{??}
\ex[??]{\gll [De qué artistas]$_i$ han herido tu sensibilidad [las obras~\trace{}$_i$]?\\
		of what artists have.\textsc{3pl} hurt your sensibility the works\\}
		\label{ex:artistas-postverbal}
\ex[*]{\gll [De qué artistas]$_i$ [las obras~\trace{}$_i$] han herido tu sensibilidad?\\
			of what artists the works have.\textsc{3pl} hurt your sensibility\\}
		\label{ex:artistas-preverbal}
\zl 

In both (\ref{ex:artistas-postverbal}) and (\ref{ex:artistas-preverbal}), the subject is in a specifier position, and the subextraction is degraded because of Spell out. But in (\ref{ex:artistas-postverbal}), the subject is in situ in Spec,VP, while in (\ref{ex:artistas-preverbal}), the subject has moved from Spec,VP to Spec,TP. The subextraction in (\ref{ex:artistas-preverbal}) hence violates two principles, thus it is degraded compared to (\ref{ex:artistas-postverbal}), which only violates one principle. As the reader can see, \citet{Uriagereka.2012} and \citet{Mueller.G.1998} have very similar views, even though the details of their analyses differ. 


%------------------------------------------------

\section{The subject island constraint cross-linguistically}

Ross himself asserted that the Sentential Subject Constraint in (\ref{rule:SSC}) only applies in some languages.

\begin{quote}
    This constraint [i.e.\ (\ref{rule:SSC})], though operative in the grammars of many languages other than English, cannot be stated as a universal, because there are languages whose rules are not subject to it. \citep[243]{Ross.1967}
\end{quote}

But subsequent accounts based on syntax explicitly or implicitly assume that the constraints that cause subject island effects are universal. Of course, many counterexamples have been brought up and discussed over the years. That is one of the reasons why the syntax-based analyses became more and more sophisticated, taking advantage of cross-linguistic properties that would explain why some languages can ``escape'' the subject island constraint.

\subsection{The discussion around French}
\label{ch:intro-disscussion-French}

Most of the discussion on extraction out of NPs in French addresses the extraction of a specific kind of French complement: \emph{de}-phrases (or \emph{de}-PPs). These PPs are introduced by the preposition \emph{de} (`of') and can express almost any kind of relation between the NP inside the PP and the N head. For example, the most direct reading for (\ref{ex:possession}) is one with the \emph{de}-phrase expressing possession, but it could also be a less unspecific and more context-dependent relation (for example, the house Thérèse always talks about). Part-whole relations for body parts like (\ref{ex:partofwhole}) are also possible, etc.

\eal
\ex[]{\gll la maison de Thérèse\\
the house of Thérèse\\
\label{ex:possession}}
\ex[]{\gll la main d' Adrien\\
the hand of Adrien\\
\label{ex:partofwhole}}
\zl 

These \emph{de}-PPs are not only used as complements of nouns, but also as complements verbs, adjectives, etc. All \emph{de}-PPs can be extracted with \emph{dont} (``of which'), \emph{de qui} (`of who'), \emph{de quoi} (`of what'), \emph{duquel} (`of which' [+masculine,+sg]), \emph{desquels} (`of which' [+masculine,+pl]), \emph{de laquelle} (`of which' [+feminine,+sg]), \emph{desquelles} (`of which' [+feminine,+pl]) or \emph{de quel(le)(s)} + N (`of which +N').

The French relative word \emph{dont} is used exclusively to introduce relative clauses with an antecedent or \emph{c'est}-clefts, see examples (\ref{ex:dont-rel}), (\ref{ex:dont-freerel}), (\ref{ex:dont-qu}) and (\ref{ex:dont-cleft}). It cannot be in situ and may not be the complement of a preposition: it therefore does not allow pied-piping, as illustrated in (\ref{ex:dont-piedpiping}) where \emph{loin} (`far') is a French preposition.

\begin{exe}\ex\label{ex:intro-dont}
    \begin{xlist}
    \ex[]{\gll cet homme [dont$_i$ on m' a dit du bien~\trace{}$_i$]\\
    this man \sbar{}of.which one me\textsc{.acc} has told some good\\
    \glt intended: `this man who I heard good things about'}
    \label{ex:dont-rel}
    \ex[*]{\gll Je me souviens [dont$_i$ on m' a dit du bien~\trace{}$_i$].\\
    I myself remember \sbar{}of.which one me\textsc{.acc} has told some good\\
    \glt intended: `I remember who(ever) I heard good things about.'}
    \label{ex:dont-freerel}
    \ex[*]{\gll Dont$_i$ as~- tu dis du bien~\trace{}$_i$?\\
    of.which have you told some good\\
    \glt intended: `About who did you say good things?'}
    \label{ex:dont-qu}
    \ex[]{\gll C' est cet homme [dont$_i$ on m' a dit du bien~\trace{}$_i$].\\
    it is this man \sbar{}of.which one me\textsc{.acc} has told some good\\
    \glt intended: `It's this man who I heard good things about'}
    \label{ex:dont-cleft}
    \ex[*]{\gll cet homme [[loin dont]$_i$ je suis~\trace{}$_i$]\\
    this man far \ssbar{}of.which I am\\
    \glt intended: `the man who I am far from'}
    \label{ex:dont-piedpiping}
    \end{xlist}
\end{exe}

In some circumstances, the relative word \emph{dont} is linked to to a resumptive pronoun rather than a gap  \citep{Godard.1988,Abeille.2007.Relatives}. In this work, we will concentrate on \emph{dont} relative clauses containing a gap. In these cases, \emph{dont} must occur with a \emph{de}-PP gap. In example (\ref{ex:dont-only-de}), the verb \emph{assister} (`to attend') requires a PP complement introduced by \emph{à}, and \emph{dont} cannot be used for the extraction. 

\eal\label{ex:dont-only-de}
\ex[]{\gll Jeannine assiste à une conférence.\\
           Jeannine attends at a conference\\
    \glt `Jeannine attends a conference.'}
\ex[*]{\gll la conférence [dont$_i$ Jeannine assiste~\trace{}$_i$]\\
           the conference \sbar{}of.which Jeannine attends\\
    \glt intended: `the conference that Jeannine attends'}
\zl 

In this usage, \emph{dont} can roughly be translated as `of which' (`of whom' with human antecedents), and will systematically be glossed \texttt{of.which}. But \emph{dont} also can be used in verbless partitive relative clauses like (\ref{ex:partitive-dont}), where it means something like `among which'.\footnote{See \citet{Bilbiie.2010} for a detailed description of these  verbless partitive relative clauses.} In these cases, \emph{dont} is not linked to a gap.

\ea[]{\gll Pierre a plusieurs passions, dont le ski.\\
           Pierre has several hobbies of.which the skiing\\
    \glt `Pierre has many hobbies, among which skiing.'}
\label{ex:partitive-dont}
\z 

The combination \emph{de} (`of') + \emph{qui} (`who') is also available to extract a \emph{de}-PP, provided that the head noun is animate. 

\eal
\ex[]{\gll l' homme [[de qui]$_i$~/ dont$_i$ il parle~\trace{}$_i$]\\
           the man \ssbar{}of who of.which he talks\\
    \glt `the man he talks about'}
\ex[]{\gll l' ordinateur [[*de qui$_i$]~/ dont$_i$ il parle~\trace{}$_i$]\\
           the computer \hspace{4mm}of who of.which he talks\\
    \glt `the computer he talks about'}
\label{ex:dequi-semantic-contraint}
\zl 

Apart from that, it is less restricted than \emph{dont}: it can be used in relative clauses with an antecedent like (\ref{ex:dequi-rel}) or free relative clauses (like \ref{ex:dequi-freerel}), but also in interrogatives like (\ref{ex:dequi-qu}), \emph{c'est}-clefts like (\ref{ex:dequi-cleft}) or exclamatives. It may also serve as complement to a preposition and thus appears in pied-piping constructions, as (\ref{ex:dequi-piedpiping}) illustrates. 

\begin{exe}\ex\label{ex:intro-dequi}
    \begin{xlist}
    \ex[]{\gll cet homme [[de qui]$_i$ on m' a dit du bien~\trace{}$_i$]\\
    this man \ssbar{}of who one me\textsc{.acc} has told some good\\
    \glt `this man who I heard good things about'}
    \label{ex:dequi-rel}
    \ex[]{\gll Je me souviens [[de qui]$_i$ on m' a dit du bien~\trace{}$_i$].\\
    I myself remember \ssbar{}of who one me\textsc{.acc} has told some good\\
    \glt `I remember who(ever) I heard good things about.'}
    \label{ex:dequi-freerel}
    \ex[]{\gll [De qui]$_i$ as~- tu dis du bien~\trace{}$_i$?\\
    \sbar{}of who have you told some good\\
    \glt `About whom did you say good things?'}
    \label{ex:dequi-qu}
    \ex[]{\gll C' est cet homme [[de qui]$_i$ on m' a dit du bien~\trace{}$_i$].\\
    it is this man \ssbar{}of  who one me\textsc{.acc} has told some good\\
    \glt `It's this man who I heard good things about.'}
    \label{ex:dequi-cleft}
    \ex[]{\gll cet homme [[loin de qui]$_i$ je suis~\trace{}$_i$]\\
    this man \ssbar{}far of who I am\\
    \glt `the man who I am far from'}
    \label{ex:dequi-piedpiping}
    \end{xlist}
\end{exe}

\label{p:qui-que-are-complementizers}

Following a proposal by \citet{Godard.1988}, there is general consensus that French \emph{dont} is a complementizer, as are \emph{que} and the subject-sub\-or\-di\-nate variant of \emph{qui}\footnote{In French, \citet{Abeille.2007.Relatives} assume two lexical entries for \emph{qui} (`who'): one is a complementizer and is used only in subordinate clauses when followed by a subject gap, the other is a pronoun and is used in all other cases.}, whereas the other fillers used in relative clauses, including \emph{de qui}, are pronouns \citep{Tellier.1990,Abeille.2007.Relatives,LeGoffic.2007}. The distinction between relative pronouns and complementizers relies on four main contrasts. First, complementizers are invariable, whereas most relative pronouns have to agree with their antecedent in gender and number. Second, relative pronouns may semantically constrain their antecedent (like \emph{de qui}, which does not allow for inanimate antecedents), whereas complementizers do not. Third, complementizers cannot be used in pied-piping constructions (as stated before, \emph{dont} cannot but \emph{de qui} can). Finally, relative pronouns may introduce an infinite relative clause, but complementizers require a tensed verb: this is only marginally true for \emph{dont}, which allows infinitivals to some extent.
% also Kayne 1974, 1975 ; Benveniste 1980

\eal
	\ex[?]{\gll une personne [dont dire du bien] \\
	         a person \sbar{}of.which tell\textsc{.inf} some good\\}
	\ex[]{\gll une personne [de qui dire du bien] \\
	          a person \sbar{}of who tell\textsc{.inf} some good\\
	     \glt ‘a person one can speak well about’}
	\ex[*]{\gll un restaurant [qu' apprécier] \\
	           a restaurant \sbar{}that appreciate\textsc{.inf} \\
	      \glt ‘a restaurant one can appreciate’}
\zl

\citet{Godard.1988} cited several examples with relativization out of the subject using \emph{dont} such as those in (\ref{ex:Godard-dont-felicitous}), challenging the general tradition on subject islands. 

\eal \label{ex:Godard-dont-felicitous}
\ex \citep[109]{Godard.1988}\\
\gll J' ai rencontré Paul, [dont$_i$ [la maison~\trace{}$_i$] est à vendre].\\
I have met Paul \sbar{}of.which \sbar{}the house is at sell\textsc{.inf}\\
\glt `I met Paul, whose house is for sale.'
\ex \citep[109]{Godard.1988}\\
\gll J' ai rencontré Paul, [dont$_i$ il semblerait [que [la maison~\trace{}$_i$] est à vendre]].\\
I have met Paul \sbar{}of.which it would.seem \sbar{}that \sbar{}the house is at sell\textsc{.inf}\\
\glt `I met Paul, of whom it seems that the house is for sale.'
\ex \citep[63]{Godard.1996}\\
\gll la jeune femme [dont$_i$ [le portrait \trace{}$_i$] est à la fondation Barnes] \\
the young woman \sbar{}of.which \sbar{}the portrait {} is at the foundation Barnes \\
\glt `this young lady, of which the portrait is at the Barnes foundation'
\label{ex:GS-french} 
\zl 

Her conclusion is that the subject island constraint does not apply to NP subjects in French. However, she still thinks that extracting out of a verbal subject is impossible, as illustrated by (\ref{ex:Godard-out-of-S}). (\ref{ex:Godard-out-of-S-infinite}) is extraction out of an infinitival, (\ref{ex:Godard-out-of-S-finite}) is extraction out of a finite sentential subject.

\begin{exe}
\ex \citep[43]{Godard.1988}
\label{ex:Godard-out-of-S}
\begin{xlist}
\ex[*]{\gll Paul [[à qui]$_i$ il apparaissait que [confier ce qui s' était réellement
passé~\trace{}$_i$] était malheureusement impossible], serait certainement
très en colère.\\
Paul \ssbar{}at who it appeared that \sbar{}confide it that \textsc{refl} was really happened was unfortunately impossible would.be certainly very in rage\\
\glt `Paul, to whom it appeared that, unfortunately, to confide what really happened was impossible, would certainly be very upset.'\\
(i.e.\ It appeared that, unfortunately,  to confide what really happened to Paul was impossible, and Paul would certainly be very upset.)
\label{ex:Godard-out-of-S-infinite}}
\ex[*]{\gll Strasbourg [où$_i$ on sait que [la décision de fixer le siège du
Parlement européen~\trace{}$_i$] n' est pas encore acquise] a pourtant
effectué de grands travaux dans cette perspective.\\
Strasbourg \sbar{}where one knows that \sbar{}the decision of settle\textsc{.inf} the seat of.the Parliament European \textsc{neg} is not yet definitive has  nevertheless made of big works in this respect\\
\glt `Strasbourg, where it is known that the decision to settle the seat of the European Parliament is not yet definitive has  nevertheless already undertaken big steps in that respect.'\\
(i.e.\ It is known that the decision to settle the seat of the European Parliament in Strasbourg is not yet definitive, but Strasbourg has nevertheless already undertaken big steps in that respect.)
\label{ex:Godard-out-of-S-finite}}
\end{xlist}
\end{exe}

As \citet[56]{Godard.1988} points out, the fact that extraction out of NP subjects is allowed in French is a major problem for the general theory: ``It is explicitly expected that the [Subject Island] Constraint has a general scope and applies to French''.\footnote{Translation. Original quote: ``Or, il est explicitement prévu que la Contrainte ait une portée générale et s'applique au français.'' \citep[56]{Godard.1988}}

In response to \citet{Godard.1988}, \citet{Tellier.1990,Tellier.1991} hypothesizes that the reason why extraction out of the subject is possible with \emph{dont} is that it is a complementizer. Consequently, according to \citeauthor{Tellier.1991}, subextraction out of the subject is not allowed with \emph{de qui}, as examples in (\ref{ex:Tellier-secretaire}) show. 

\begin{exe}
\ex \citep[307]{Tellier.1990}
\label{ex:Tellier-secretaire}
\begin{xlist}
\ex[]{\gll le diplomate [dont$_i$ [la secrétaire~\trace{}] t' a téléphoné]\\
the diplomat \sbar{}of.which \sbar{}the secretary you\textsc{.acc} has called\\
\glt `the diplomat of who the secretary called you'
\label{ex:Tellier-secretaire-dont}}
\ex[*]{\gll le diplomate [[de qui]$_i$ [la secrétaire~\trace{}] t' a téléphoné]\\
the diplomat \ssbar{}of who \sbar{}the secretary you\textsc{.acc} has called\\
\glt `the diplomat of who the secretary called you'
\label{ex:Tellier-secretaire-dequi}}
\end{xlist}
\end{exe}

The asterisk in (\ref{ex:Tellier-secretaire}) reflects \citeauthor{Tellier.1990}'s judgements. The traditional indication of a subject island effect, namely the contrast between subextraction from subject vs.\ object, would arise if we replace \emph{dont} with \emph{de qui}, as expected by the subject island constraint. This is illustrated by (\ref{ex:Tellier-linguiste}), again with \citeauthor{Tellier.1991}'s acceptability judgements. 

\begin{exe}
\ex \citep[89--90]{Tellier.1991}
\label{ex:Tellier-linguiste}\judgewidth{?*}
    \begin{xlist}
    \ex[?*]{\gll C' est un linguiste  [[de qui]$_i$ [les parents~\trace{}$_i$] ont déménagé à Chartres].\\
    it is a linguist \ssbar{}of who \sbar{}the parents have moved at Chartres\\
    \glt `this is a linguist of whom the parents have moved to Chartres'}
    \label{ex:Tellier-linguiste-sujet}
    \ex[]{\gll C' est un linguiste [[de qui]$_i$ vous avez rencontré [les parents~\trace{}$_i$]].\\
    it is a linguist \ssbar{}of who you have met \sbar{}the parents\\
    \glt `this is a linguist of who you have met the parents'}
    \label{ex:Tellier-linguiste-objet}
\end{xlist}
\end{exe}

% underlying idea is probably that French, because it's not a pro-drop language, should not escape the constraint on subject island 

\citegen{Tellier.1990} analysis is based on the notion of ``barriers'', as defined in \citet{Chomsky.1986} (this notion as explained in Section~\ref{ch:barriers} page \pageref{ch:barriers}ff.). In her analysis, extraction out of the subject with \emph{de qui} violates the subject island constraint, illustrated in \figref{fig:Tellier-secretaire-dequi-tree}, similar to the English example (\ref{ex:stories-about-tree-2-barrier}).

\begin{figure}[ht]
\centering
\begin{forest}
where n children=0{tier=word}{}
[CP
    [Spec [de qui$_i$\\of who, name = filler]]
    [C'
        [C [$\emptyset$]]
        [IP, circle, draw,
            [NP, circle, draw
                [D [la\\the]]
                [N'
                    [N [secrétaire\\secretary]]
                    [PP [\emph{t}$_i$, name = gap]]
                ]
            ]
            [I'
                [I [a\\has]]
                [VP
                    [téléphoné\\called]
                ]
            ]
        ]
    ]
]
%\draw[->,densely dotted] ([xshift= -5pt,yshift = 2pt]gap.south) .. controls +(south west:2cm)  and +(south west:5.5cm)  .. ([xshift= -9pt,,yshift = 5pt]filler.south);
\draw[->,densely dotted] (gap) |- ++(-2ex,-2.5\baselineskip) -| (filler);
\end{forest}
\caption{Syntactic tree for ``[de qui]$_i$ [la secrétaire~\trace{}] a téléphoné'' (`of who the secretary called') according to \citeauthor{Tellier.1991}.}
    \label{fig:Tellier-secretaire-dequi-tree}
\end{figure} 


Recall that following \citet{Chomsky.1986} the NP is not directly $\theta$-marked by the V\footnote{See fn.\ \ref{fn:theta-indirect-marking-subject}.} and thus is a blocking category and a barrier for the PP-complement of the noun. The IP is not assigned a $\theta$-role either, and because it contains a blocking category (the NP), it is also a barrier for the PP. \emph{Wh}-movement of \emph{de qui} crosses two barriers, and is therefore unacceptable.

\citet[308--309]{Tellier.1990} says that the genitive PP-complement of the subject noun moves to the specifier of CP, where it agrees with the head (i.e., \emph{dont}) and is then deleted. This is how \emph{dont} receives genitive case. Because of this genitive case, the complementizer \emph{dont} is L-marked (i.e.\ it is lexical).\footnote{She assumes that the overt genitive case ``assigns to the complementizer sufficent lexical weight'' for it to be treated as a lexical word \citep[309]{Tellier.1990}.}  \citeauthor{Tellier.1990} stipulates (i) that an L-marked C head ``$\theta$-marks its complement IP'' and (ii) that a lexical $\alpha$ ``L-marks $\beta$ iff $\beta$ agrees with the head of $\tau$ that is $\theta$-governed by $\alpha$". Her analysis is shown in \figref{fig:Tellier-secretaire-dont-tree}.

\begin{figure}[ht]
\begin{forest}
where n children=0{tier=word}{}
[CP
    [\emph{wh}$_i$, name = filler [{}, no edge]]
    [C'
        [C [dont\\of.which]]
        [IP
            [NP
                [D [la\\the]]
                [N'
                    [N [secrétaire\\secretary]]
                    [PP [\emph{t}$_i$, name = gap]]
                ]
            ]
            [I'
                [I [a\\has]]
                [VP
                    [téléphoné\\called]
                ]
            ]
        ]
    ]
]
\draw[->,densely dotted] (gap) |- ++(-2ex,-2.5\baselineskip) -| (filler);
\end{forest}
\caption{Syntactic tree for ``dont$_i$ [la secrétaire~\trace{}$_i$] a téléphoné'' (`of who the secretary called') according to \citeauthor{Tellier.1991}}
    \label{fig:Tellier-secretaire-dont-tree}
\end{figure}

Because \emph{dont} is lexical, C assigns a $\theta$-role to IP (i), which is thus not a blocking category. Additionally, \emph{dont} L-marks the IP (ii) and the IP in turn assigns a $\theta$-role to the NP (i), which is thus not a blocking category, either. It then follows that there is no barrier in \figref{fig:Tellier-secretaire-dont-tree} (the movement is 0-subjacent), and extraction out of the subject with \emph{dont} is acceptable.\largerpage

There are several problems with \citegen{Tellier.1990} analysis. The proposal that \emph{dont} can assign a $\theta$-role to the IP seems very stipulative and ad hoc, it only applies to subextraction from the subject. There is also no explanation for why the complementizer can only receive genitive and no other case, for example when extracting out of a sentential subject. Furthermore the premises are problematic as well. 
The judgements on the (un)acceptability of examples in (\ref{ex:Tellier-secretaire}) and (\ref{ex:Tellier-linguiste}) are \citeauthor{Tellier.1991}'s, and have not been confirmed by quantitative empirical data so far. In particular, \citet[56]{Godard.1988} cites some felicitous examples of extraction out of a subject NP with \emph{de qui} and dismisses a possible analysis along the lines of \citet{Tellier.1990}.{\interfootnotelinepenalty=10000\footnote{``Il ne sert à rien de dire qu'il s'agit là d'une particularité de la forme \emph{dont} par opposition aux mots \emph{qu}'' \citep[56]{Godard.1988} (`There is no use in saying that this is a peculiarity of the form \emph{dont} as opposed to \emph{wh}-words.')}}

\begin{exe}
\ex \citep[56]{Godard.1988}
\begin{xlist}
\ex \gll un homme [de qui]$_i$ [la force de travail~\trace{}$_i$] est étonnante\\
a man of who the power of work is astonishing\\
\glt `a man whose work power is astonishing'
\ex \gll [De qui]$_i$ te semblait - il que [la force de travail~\trace{}$_i$] est étonnante~?\\
of who you\textsc{.dat} seemed {} it that the power of work is astonishing\\
\glt `Of who did it seem to you that the work power is astonishing?'
\end{xlist}
\end{exe}
\largerpage

%\citet[42]{Godard.1988} proposes to extend Ross's constraint.

To conclude, French challenges the subject island in Minimalist accounts in various ways. First, the subject is by definition not a complement, so in accounts based on a distinction between complement and non-complement, extraction out of the subject should be impossible~-- except if extraction is allowed out of underlying objects, and this possibility should be tested empirically. Second, French is an SVO language, and the subject is considered to be base-generated in a specifier position (of IP, VP, vP or v*P depending on the analysis), which means that specifier-based accounts do not expect subextraction from the subject in French to be acceptable~-- except from subjects that are underlying objects, as just mentioned. Third, these accounts assume movement of the subject from Spec,VP to Spec,IP so analyses based on Freezing cannot explain why subextraction out of the subject is acceptable~-- except if extraction out of the subject happened to be restricted to postverbal subjects, but the examples discussed by \citet{Godard.1988}, \citet{Tellier.1990,Tellier.1991} or \citet{Heck.2009} are not.



\subsection{Other cross-linguistic counterexamples}\largerpage

French is not the only language that challenges the syntax-based accounts, and I now briefly present some other interesting data from the literature. The goal of this section is not to provide a complete list of languages that are relevant with respect to the Subject island Condition, but only to show that French is not exceptional.

\subsubsection{The subject as a non-complement}

As mentioned before, \citeauthor{Ross.1967} never intended the Sentential Subject Constraint as a universal constraint, because of data from Japanese, in which subextraction out of sentential subjects seems to be felicitous.

\ea[]{Japanese \citep[244]{Ross.1967}\\
\gll Kore~wa [[Mary~ga~\trace{} kabutte ita koto]~ga akiraka na] boosi da.\footnotemark\\
this \ssbar{}Mary wearing was thing obvious is hat is\\
\label{ex:Japanese}
\glt `This is the hat which that Mary was wearing (it) was obvious.'}
\z 
\footnotetext{There is no relative pronoun in Japanese relative clauses, hence the lack of a filler coindexed with the gap in (\ref{ex:Japanese}).}

\citet{Huang.1982} accounted for cases like Japanese and Chinese by proposing that Infl in these languages is lexical. The subject would then be properly governed, which would explain why it is possible to extract out of it.

More cross-linguistic counterexamples to the subject island constraint, have been included in the debate showing felicitous extraction out of subject NPs, infinitival subjects and sentential subjects. \citet{Stepanov.2007} offers a very complete collection of these counterexamples. His main goal in doing this is to show that accounts based on a distinction between complements and non-complements such as the CED cannot be correct. As he claims, languages that allow extraction out of the subject still exclude extraction out of adjuncts. If \citegen{Huang.1982} account were right, then adjuncts should be properly governed to the same degree as subjects, and subextraction from adjuncts should be acceptable.
% Malagasy (Sabel 2001), Tagalog (Nakamura 1998), Icelandic (Kitahara 1994; but contradicted in Zaenen 1985), Hindi (Mahajan 1992, Stepanov 2004), German (Haider 1993, 1997, 2000, Müller 1995, Fanselow 1987, Grewendorf 1989), Japanese (Lasnik & Saito 1992, Yatsushiro 1999), Russian (King 1994), Palauan (Georgopoulos 1991).

Since then, the existence of adjunct islands has been challenged as well \citep{Truswell.2011}, but \citegen{Stepanov.2007} argument against these accounts remains valid, because extraction out of subjects (and of adjuncts) should be completely ungrammatical. In particular, felicitous examples from English like (\ref{ex:huddleston-felicitous}) cited on page \pageref{ex:huddleston-felicitous} are a problem. \citeauthor{Santorini.2017} provides many attested examples of extraction out of NP subjects and \citeauthor{Huddleston.2002} and \citet[17--18]{Chaves.2012} give attested examples of extraction out of infinitival subjects.
\ea English \label{ex:English}
\begin{xlist}
\ex (Jane Austen, \citetitle{Austen1981} (\citeyear[p. 84]{Austen1981}), cited by \citealt{Santorini.2017})\\
a letter [[of which]$_i$ [every line \trace{}$_i$] was an insult] 
\ex (David Quammen, \citetitle{Quammen1985} (\citeyear[p. 176]{Quammen1985}), cited by \citealt{Santorini.2017})\\
virginity and sans serif typeface, [[of which]$_i$ [the definition~\trace{}$_i$] must begin with negatives] 
\ex \citep[1094]{Huddleston.2002}\\
The eight dancers and their caller, Laurie Schmidt, make up the Farmall Promenade of nearby Nemaha, a town [that$_i$ [to describe~\trace{}$_i$ as tiny] would be to overstate its size]. 
\ex (internet example cited by \citealt[18]{Chaves.2012})\\
The [\dots] brand has just released their S/S 2009 , [which$_i$ [to describe~\trace{}$_i$ as noticeable] would be a sore understatement]. 
\end{xlist}
\z 

\subsubsection{The subject as a specifier}

As previously noted, \citegen{Kayne.1983} Connectedness Condition was explicitly designed to account for the Japanese data, and for extraction out of the subject in SOV languages in general.

But felicitous counterexamples from Romance languages like (\ref{ex:romance-languages}) are problematic for the specifier-based accounts, because Romance languages are considered SVO.

\eal \label{ex:romance-languages}
\ex Italian \citep[61]{Rizzi.1982}\\ 
\gll questo autore, [[di cui]$_i$ so [che [il primo libro~\trace{}$_i$] è stato pubblicato recentemente]]\\ 
this author \ssbar{}of which know.\textsc{1sg} \sbar{}that \sbar{}the first book has been published recently\\ 
\glt `this author, of which I know that the first book has been published recently'
\label{ex:rizzi-it}
\ex Spanish \citep[103]{Jimenez-Fernandez.2009}\\
\gll \textquestiondown{}[De qué cantante]$_i$ parece [que [algunas fotos~\trace{}$_i$] les han escandalizado]?\\
\ssbar{}of which singer seems \sbar{}that \sbar{}some photos them\textsc{.acc} have shocked\\
\glt `Of which singer does it seem that some photos have shocked them?'
\label{ex:spanish-preverbal-subject-good}
\zl 

This has led to the proposal that extraction out of subjects in Romance languages and the other languages listed by \citet{Stepanov.2007} is grammatical because no real extraction is involved \citep{Rizzi.1990,Uriagereka.2012}. In fact, all of these languages allow a null subject. Felicitous extraction out of the subject in Romance languages (\ref{ex:romance-languages}), Japanese (\ref{ex:Japanese}), Turkish and many other languages would then be special instances of a null subject.\footnote{\citet{Huang.1984} also proposes something similar for Mandarin. He assumes that acceptable extraction out of an island is not actual extraction, and the gap is not an actual gap, but rather a null pronoun, which is licensed because itsreferent is the nearest preceding NP. His analysis works particularly well for subject islands, since the (false) gap is by definition near the filler. For a refutation of \citet{Huang.1984}, see \citet[2--3]{Dong.2021}.} French is thus an interesting case, because it is a Romance language without a null subject.

Notice also that the data from English in (\ref{ex:English}) are a problem for these approaches as well. 

\subsubsection{The moved subject (Freezing analyses)}

\citet{Lasnik.1992} adopt an analysis based on Freezing and account for Japanese data by assuming that the subject in Japanese is always in situ and does not leave the VP.\footnote{As pointed out by \citet{Stepanov.2007}, $\theta$-role assignment must then be completely reconsidered.} 

As mentioned previously, \citet{Gallego.2007} claim that in Spanish, extraction from preverbal subjects is ruled out, while extraction from postverbal subjects is felicitous. They take this as evidence that the subject island phenomenon is caused by Freezing effects: movement of the subject to Spec,TP blocks the subsequent \emph{wh}-movement from the subject to Spec,CP. \citet{Jimenez-Fernandez.2009} criticizes this distinction between preverbal and postverbal subjects in Spanish, on the basis of examples like (\ref{ex:spanish-preverbal-subject-good}).\footnote{Notice that the extracted element in (\ref{ex:spanish-preverbal-subject-good}) is specific, and that the subject is an indefinite, two factors that contribute to make the sentence more acceptable. I will come back to this aspect in Section~\ref{ch:specificity-definiteness}.} Thus there is disagreement about the data from Romance languages and about a possible asymmetry between preverbal and postverbal subjects.

Also, the analyses based on Freezing predict that extraction out of subjects of passives is ungrammatical because these subjectsundergo movement (except in some languages like German where the movement is optional). Under many of the other analyses, however, extraction out of subjects of passives is grammatical. This results in disagreement about the data for passives as well: see examples (\ref{ex:passives-are-good}) and (\ref{ex:passives-are-bad}) on page \pageref{ex:passives-are-good}.

%Lasnik and Saito 1992: The subject merges at a sister position of VP (and not in Spec,vP as in English). Some languages do not have a vP (Lasnik 1999, Sabel 2001), but then the whole $\theta$-role assignment must be reconsidered. V is combined simultaneously (in a flat structure, see Kiss 1987) with subject and object DP, and thus none of them needs to be linearized, and extraction out of both is possible. However, if the subject moves (like in passives), then extraction out of the subject is out.
% Flat structure for Japanese
% LASNIK, H. 1999a. Chains of arguments. In Working minimalism, ed. S. D. Epstein & N. Hornstein, 189–215. Cambridge, Mass.: MIT Press.
% SABEL, J. 2001. \emph{wh}-questions and extraction asymmetries in Malagasy. Paper presented at the AFLA VIII, MIT, Cambridge, Mass.
% KISS, K. É. 1987. Configurationality in Hungarian. Dordrecht: D. Reidel.
% But the problem is that such a flat structure is not properly governed. (because of non-asymetry?) For this reason, Chomsky 1995 proposes that the subject leaves VP, but then the subject moves and again there should be a subject island.
% CHOMSKY, N. 1995a. Bare phrase structure. In Government and binding theory and the Minimalist Program, ed. G. Webelhuth, 383–439. Cambridge: Blackwell.

Finally, an interesting point is raised by \citet{Chaves.2020.UDC}: not only subextraction from moved subjects, but also subextraction out of extracted elements should be impossible according to Freezing accounts.  But this is not always the case, as the following example shows:

\ea \citep[52]{Chaves.2020.UDC}\nopagebreak\\
This is the handout [which$_i$ I can't remember [[how many copies of~\trace{}$_i$]$_j$ we have to print~\trace{}$_j$]]. 
\z 

% Levine and Huraki discuss Freezing theories too


\section{Criticisms of the syntactic approach}

We already saw that there is some disagreement concerning the data about the subject island cross-linguistically as well as concerning extraction out of certain kinds of subjects. This is not only due to a lack of empirical data, but alsoto a problem of interpreting the data, when available. As a matter of fact, it is hard to draw the line between ``good'' extractions and ``bad'' extractions, because there are numerous ways to improve an infelicitous subextraction from a subject. But precisely this point is problematic, for several reasons that I elaborate on in this section.

\subsection{Non-syntactic factors increasing the acceptability}

Especially researchers working on sentence processing have been criticizing syntactic analyses of islands for a long time. Without going into details, as it will be the focus of Chapter~\ref{ch:processing-accounts}, their principal claim is that an island caused by a syntactic constraint should not increase in acceptability unless we manipulate syntactic factors. If non-syntactic factors ameliorate an island violation, this means that the superadditivity effect is not caused by a syntactic constraint. This criticism of islands in general was formulated by \citeauthor{Kluender.1991} (\citeyear{Kluender.1991}, and later works), then adopted by \citeauthor{Hofmeister.2010} (\citeyear{Hofmeister.2010}, and later works) and is now supported by \citeauthor{Chaves.2013} (\citeyear{Chaves.2013}, and later works). 

Regarding the subject island in particular, \citeauthor{Chaves.2013} has shown that many non-syn\-tac\-tic factors can improve extraction out of subjects (especially, but not exclusively, subject NPs). 

\begin{enumerate}
    \item Semantic factors like definiteness and specificity help improve subextraction from NPs in general, not only for subjects (\citealt{Jimenez-Fernandez.2009,Chaves.2013,Simonenko.2015}, see Section~\ref{ch:specificity-definiteness}). Notice that some proponents of syntax-based approaches posit different syntactic structures for definite and indefinite NPs, so these factors can be addressed by a syntactic analysis. 
    \item Appropriate prosody can help identify the gap more easily and make sentences with extraction out of the subject more acceptable \citep{Chaves.2014}.
    \item\sloppy Change in lexical material without modifying the syntactic structure can help change the proposition expressed by the utterance and make extraction more felicitous, because the purpose of the utterance (e.g.\ why one may need to ask such a question) is easier to understand \citep{Chaves.2019.Frequency}.
    \item With repeated exposure to extraction out of the subject, naive speakers increasingly accept the structure \citep{Chaves.2014,Do.2017}. This effect is known as ``satiation'', or ``habituation''. Ungrammatical sentences usually do not show habituation effects \citep{Sprouse.2007.Acceptability}. 
\end{enumerate}

\subsection{``Parasitic'' gaps}
\label{ch:parasitic-gaps}


An early observation made about subextraction from subjects in English is that the extraction of the subject is far more acceptable if the gap is related to another gap situated in a non-island environment. For example, in (\ref{damage-repair-car}), we see that subextracting out of the subject is more felicitous if there is a second gap in the object.\footnote{This is supposed to hold cross-linguistically and for all constructions that involve extraction. \citeauthor{Tellier.1991} gives the following contrast for French:
\ea \citep[122]{Tellier.1991}
    \ea[*]{\gll un enfant [[de qui]$_i$ [les parents~\trace{}$_i$] ont déménagé]\\ 
            a child of who the parents have moved\\ 
           \glt `a child whose parents have moved'
          }
    \ex[]{\gll un enfant [[de qui]$_i$ [les parents~\trace{}$_i$] se méfient~\trace{}$_i$] \\ 
          a child of who the parents \textsc{refl} beware\\
          \glt `a child whom the parents of beware'
         }
\z
\z
Whether the contrast between (ia) and (ib) is real has not been tested empirically so far. See also \citet[117--119]{Godard.1988} on so-called ``parasitic gaps'' in French.}

\begin{exe}
	\ex \citep[303]{Chaves.2013}
 \label{damage-repair-car}
 \begin{xlist}
		\ex[*]{What$_i$ did the attempt to repair \trace{}$_i$ ultimately damaged the car?} 
		\ex[]{What$_i$ did the attempt to repair \trace{}$_i$ ultimately damaged \trace{}$_i$? } \label{damage-repair-car-good}
	\end{xlist}
\end{exe}


Syntactic-based accounts of subject islands have all addressed this contrast. They usually assume that only the gap in the non-island environment is a ``real'' gap. The first missing element in the subject phrase is a null pronominal which is only felicitous because of the actual extraction. This phenomenon is known as a ``parasitic gap'': the gap in the subject is ``parasitic'' as it takes advantage of the presence of the legitimate gap. Parasitic gaps are supposedly allowed in adjuncts for similar reasons. 
% Cinque, Gugliemo. 1990. Types of A-Dependencies. Cambridge, Mass.: MIT Press.
%Postal, Paul M. 1993. Parasitic Gaps and the Across-the-board Phenomenon. Linguistic Inquiry 24, 735–754.
%Postal, Paul M. 1994. Parasitic and pseudoparasitic gaps. Linguistic Inquiry 25, 63–117.
%Postal, Paul M. 1998. Three investigations of extraction. MIT: Cambridge, Massachusetts.
%Postal, Paul M. 2001. Islands. In Mark Baltin and Chris Collins (eds.), The Handbook of Syntactic Theory, Oxford: Blackwell.

%This analysis is challenged however by extractions from non-nominals.
% Levine, Robert D., Hukari, Thomas E. and Calcagno, Michael. 2001. Parasitic gaps in English: some overlooked cases and their theoretical implications. In Peter Culicover and Paul M. Postal (eds.), Parasitic Gaps, pages 181–222, Cambridge, MA: MIT Press.
% Levine, Robert D. 2001. The extraction riddle: just what are we missing? Journal of Linguistics 37, 145–174.
% Levine, Robert D. and Sag, Ivan A. 2003. Some empirical issues in the grammar of extraction. In Stefan M¨uller (ed.), Proceedings of the HPSG-2003 Conference, Michigan State University, East Lansing, pages 236–256, Stanford: CSLI Publications.

\citeauthor{Chaves.2013} has worked extensively on so-called parasitic gaps and presents many examples that are problematic for syntax-based accounts. I reproduce here only a few of them, but I invite the interested reader to consult \citet{Chaves.2013} and \citet{Chaves.2020.UDC} for further discussion of the phenomenon. 

It is true that a second coindexed gap can improve the acceptability of an unacceptable gap. For example, in (\ref{ex:true-parasitic-good}), the second gap requires an NP, while the filler is a PP, but since the filler is appropriate for the first gap, the sentence is unproblematic. As the contrast with (\ref{ex:true-parasitic-bad}) shows, though, it makes a difference whether the filler-gap mismatch is at the first or at the second gap. 

\begin{exe}
\sloppy
\ex \citep[62, my emphasis]{Chaves.2020.UDC}
\begin{xlist}
\ex[]{It was \WinckelEmph{on} \WinckelEmph{Sue} [that$_i$ I think Sam relied~\trace{}$_i$ the most but didn't thank~\trace{}$_{i?}$ nearly enough in his speech].} \label{ex:true-parasitic-good}
\ex[*]{It was \WinckelEmph{on} \WinckelEmph{Sue} [that$_i$ I think Sam thanked~\trace{}$_{i?}$ the most but didn't rely~\trace{}$_i$ nearly enough in his speech]. } \label{ex:true-parasitic-bad}
\end{xlist}
\end{exe}

Given that subjects precede objects, the subject gap is the first one, and hence unlikely to be parasitic. When perceivers first encounter the subject, they would probably not realize that it contains a gap. This would only become clear when they reach the second gap, which would then be too late to posit one in the subject ``after the fact'' (this is what happens in \ref{ex:true-parasitic-bad}). Such a process would require reanalysis, and online measurements would show a slowdown. According to \citet{Chaves.2019.Frequency}, this is not the case.

The contrast between (\ref{ex:parasitic-subject-adjunct-good}) and  (\ref{ex:parasitic-subject-adjunct-subject}) is the same as the contrast in (\ref{damage-repair-car}). Again, extraction out of the subject is improved by the presence of a second gap. However, in (\ref{ex:parasitic-subject-adjunct-good}), the second gap is inside an adjunct. Since extraction out of adjuncts is not felicitous (and supposedly a violation of an island constraint), as shown by example (\ref{ex:parasitic-subject-adjunct-adjunct}), none of the gaps can be ``parasitic'' on another legitimate gap. This example shows that what must be happening is reactivation of the filler, which makes the structure easier to process. It cannot be explained by parasitic licensing. 

\begin{exe}
\judgewidth{??}
\ex \citep[305]{Chaves.2013}
\label{ex:parasitic-subject-adjunct}
\begin{xlist}
\ex[*]{[What kinds of books]$_i$ do [authors of~\trace{}$_i$] argue about royalties after writing malicious pamphlets?} \label{ex:parasitic-subject-adjunct-subject}
\ex[??]{[What kinds of books]$_i$ do authors of malicious pamphlets argue about royalties [after writing~\trace{}$_i$]?} \label{ex:parasitic-subject-adjunct-adjunct}
\ex[]{[What kinds of books]$_i$ do [authors of~\trace{}$_i$] argue about royalties [after writing~\trace{}$_i$]?} \label{ex:parasitic-subject-adjunct-good} 
\end{xlist}
\end{exe}

In example (\ref{ex:people-sensitive}), again, extraction out of the object causes the extraction out of the subject to become more acceptable. But since the two gaps are not coindexed, the one in the subject cannot be a parasitic gap.


\ea \citep[305]{Chaves.2013}\\
{} [People that sensitive]$_i$, I never know [[which topics]$_j$ [jokes about~\trace{}$_j$] are likely to offend~\trace{}$_i$].
\label{ex:people-sensitive}
\z 

Instead of parasitic licensing, \citeauthor{Chaves.2013} proposes an analysis based on cognitive principles \citep{Chaves.2013,Chaves.2014,Chaves.2019.Frequency}. When first encountered, the filler needs to be kept in memory. The comprehender then accesses this memory representation when they identify a gap and need to fill it. 
This process reactivates the referent of the filler, so it becomes cognitively more salient than it was before the first gap. 
At the second gap, the referent of the filler is still salient, so the process of accessing this piece of information is facilitated. This view is supported by independent evidence from processing in \citet{Vasishth.2006}.\footnote{Furthermore, having a second gap where the gap is most expected, as in (\ref{damage-repair-car-good}), helps avoid a potential filled-gap effect, see Section~\ref{ch:processing}.} 

\subsection{Gradient grammaticality}

Finally, I cannot avoid the sensitive topic of gradient acceptability of island structures in general \citep{Hofmeister.2010,Hofmeister.2013}, and of subject islands in particular (see Part~\ref{part:2} of this book). It seems that judgments on subextraction from the subject vary from case to case, and range from completely unacceptable to perfectly acceptable with everything in between. But this variation has been interpreted in different ways.

The problem is rooted in the relation between acceptability and grammaticality.\footnote{I will not go into detail about the exact definition of grammaticality and acceptability, which has been discussed at length by other scholars \citep[a.o.][]{Schutze.2016.ex1996}. Let us only assume that acceptability is the graded subjective judgment that reflects a native speaker's perception of a given utterance, regardless of the reason why they may find the utterance good or bad. 
Grammaticality, on the other hand, is a theoretical construct: a given sentence may or may not follow the syntactic, semantic and pragmatic rules of a given language and thus be part of the (potentially infinite) set of sentences that belong to this language.}
Everyone agrees that there must be at least a correlation between acceptability and grammaticality. Generally speaking, a sentence that (i) is licensed by syntactic rules, (ii) is not semantically incoherent, and (iii) is consistent with pragmatic principles such as Grice's maxims, is also an acceptable sentence in all but exceptional cases~-- like center-embedded structures, whose unacceptability can likely be explain by cognitive processing limitations. However, is the reverse also true? And in particular: Is a sentence that is ungrammatical from the syntactic point of view also unacceptable? And where is the threshold between acceptability and unacceptability? When is a sentence acceptable enough to falsify a syntactic hypothesis?

A common criticism against syntactic approaches is that they are not able to explain acceptability judgments in the grey area between acceptability and non-acceptability. Syntax is discrete: a sentence either belongs to a given language or it does not. \citeauthor{Erteschik-Shir.2006} expresses it in the following terms:

\begin{quote}
    Violations of syntactic constraints necessarily
cause strong grammaticality infractions, thus resulting in ungrammatical
sentences.\\
\hbox{}\hfill\hbox{\citep[335]{Erteschik-Shir.2006}}
\end{quote}

\citeauthor{Erteschik-Shir.2006}'s opinion is shared by many others \citep[e.g.][]{Chaves.2013,Hofmeister.2010,Abeille.2020.Cognition}, and even by some supporters of the syntactic approaches \citep{Sprouse.2007.Acceptability}. Since judgments on (some) islands are gradient, their conclusion is that syntactic approaches to these islands cannot account for the gradience of the actual data.\footnote{As far as experimental data are concerned, \citet{Sprouse.2007.Acceptability}, who argues in favor of syntax-based accounts of subject islands, denies the gradience of the data, and says that a distinction between acceptability and unacceptability judgments can be observed if one uses adequate methodology.}
Alternatives are processing-based and discourse-based accounts, which can straightforwardly explain this gradience (see Chapters~\ref{ch:processing-accounts} and \ref{ch:discourse}). 

And yet, the question of the gradience of grammar is very often explicitly addressed by the linguists who develop the syntactic accounts. % Huang, Chomsky 1986, Heageman
%For example, Huang says that ECP violations are stronger than Subjacency violations. 
For example, Subjacency was first categorical \citep{Chomsky.1973} but was soon formulated as a constraint that explicitly expects graded grammaticality proportional to the number of barriers crossed during the movement \citep{Chomsky.1986}. 
%‘‘an  adequate  linguistic  theory  will  have  to  recognize  degrees  of grammaticalness’’ Chomsky  (1975:  131) % Chomsky, N., 1975. The Logical Structure of Linguistic Theory, Plenum Press, New York.

\begin{quote}
    Rules of grammar do not simply apply or fail to apply; rather they apply to a degree.~[\dots] Grammatical constructions are not simply islands or non-islands; rather they may be islands to a degree. \citep[271]{Lakoff.1973}
\end{quote}

The real problem is hence the fact that these linguists have not managed to explain convincingly how their syntactic rules allow ungrammatical structures to be acceptable, or how violation of a constraint can lead to half-grammaticality. In syntactic theories that make a distinction between syntax, LF and PF, syntactic constraints should block the transfer to LF. It follows that a sentence that violates a syntactic rule cannot be interpreted at all, since the transfer that would enable the interpretation has failed. If this were the case, then acceptable violations of a syntactic rule could only be considered to be a grammatical illusion, and native speakers should not be able to understand such sentences. We will see that this is not the case in extraction out of the subject. Furthermore, a theory that postulates the innateness of syntactic constraints needs to explain why there is a distinction in strength between these innate rules. In my view, \citet[Chapter~1]{Uriagereka.2012} takes this problem seriously and dedicates some effort to explaining the link between processing and syntactic constraints. He sees constraints as merely preferences~-- strong preferences, but ones that can still be subverted. Similarly, in Optimality Theory, constraints are seen as criteria, and the structure that violates the fewest criteria ``wins'' in being the most acceptable \citep{Keller.2000}.

Of course, there is also an opposite view, according to which discrete grammaticality can nevertheless lead to non-discrete acceptability judgements ``as by-products of interactions of grammatical knowledge with the behavioral systems required to perceive, comprehend, and intuit'' \citep[871--872]{Carroll.1979}. The gradience of data is attributed to the human capacity to find a solution~-- any solution~-- to a problem. Since the purpose of verbal communication is the exchange of information, the comprehender finds a way to interpret sentences, even if they do not belong to the language.\footnote{A concrete example is the common experience that we are able to correctly interpret utterances of non-native speakers that contain grammatical errors, as an anonymous reviewer rightly pointed out.} 
%If the sentence is ungrammatical but comprehensible, then it is more acceptable than an ungrammatical and incomprehensible sentence. Carroll (1979) + Katz (1964) --> but what makes it comprehensible if not grammar???
%``Where there are conflicts between communicative and computational efficiency, it seems that the latter prevails (...)'' (Chomsky, 2013, 41)
% Jackendoff, 1983: 155–122 -> fuzziness and family resemblances should play a role in linguistic studies (he's an exception)

According to \citet{Schutze.2016.ex1996}, there is actually no empirical way to distinguish between the two views:

\begin{quote}
    It might be that the nature of the particular tasks used by prototype theorists
(and linguists) inherently induces graded behavior, independent of the nature of
the underlying knowledge. If this is so, the status of that underlying knowledge
as discrete or continuous must be demonstrated by other means. But how could
we ever know whether a grammar, if it exists independent of performance mechanisms, classifies sentences dichotomously? If performance mechanisms induce
graded structure by themselves, and if (as I argue) they can never be circumvented because competence is not directly accessible, then it might not be possible to investigate empirically how a grammar itself classifies sentences. \citep[69]{Schutze.2016.ex1996}
\end{quote}

In the empirical work I present later, I have done my best to take into account this difficult issue. First, I postulate that an ungrammatical structure should appear in well-edited written production only in extremely exceptional cases (if at all). The problem then is to define "extremely exceptional". I therefore establish an objective threshold; constructions with fewer occurrences are considered marginal or hardly existent. In the experiments, I employed a methodology proposed by \citet{Sprouse.2007.PhD} in which a factorial design is used in order to detect superadditivity effects. This design does not allow us to identify the reason of the superadditivity~-- whether it is caused by syntax, processing or pragmatic factors~-- but it enables us at least to clearly identify a contrast in acceptability. In most experiments, I added an ungrammatical baseline. I chose the baseline in such a way that the ungrammatical sentences are nevertheless somewhat comprehensible. Thus, it is possible to see whether participants discriminate between a violation of the subject island constraint and ungrammatical-but-interpretable sentences. 

%------------------------------------------------

\section{The subject island constraint in HPSG}\largerpage
% \citet[91]{Chomsky.1981} says that non-movement accounts are ``virtually indistinguishable'' from accounts based on movement.

Before presenting the processing-based and discourse-based accounts of subject islands, let me briefly sketch the way this phenomenon has been analyzed in HPSG. The first approaches were indeed syntax-based, and some islands  (extraction of a conjunct and subextraction from relative clauses) are still treated in terms of syntactic constraints. But the analysis of subextraction from subjects has changed over the years.

In HPSG, filler-gap dependencies are not conceived of as movement. The information that the verbal phrase is missing an element is stored in a nonlocal feature \textsc{slash}, so that the missing element can be saturated on the clausal level. The first formalization of the \textsc{slash} feature is in \citet{Gazdar.1981}, where the information that an element is missing is percolated from daughter to mother in the same way as other kinds of information. This analysis is compatible with online processing data on filler-gap dependencies: The resolution of filler-gap dependencies seems to take place at the subcategorizer, and not at the gap site \citep{Boland.1995,Traxler.1996}.
In current HPSG analyses, there are two ways to treat these missing elements: either as empty categories or through a lexical rule.

Early GPSG had an equivalent to the Subject Island Condition. 
In \citet{Pollard.1984}, the Binding Inheritance Principle makes sure that mother nodes cannot inherit an element in \textsc{slash} from a specifier.
\citet{Pollard.1994} have a Subject Condition (\ref{rule:Subject-Condition-HPSG}) that allows parasitic gaps in the subject but no other gaps. However, the authors explicitly state that this constraint is not universal, but possibly only belongs to the grammar of English. Even this is not certain, given that ``many [English] speakers'' consider extraction out of the subject acceptable \citep[183]{Pollard.1994}.

\ea\label{rule:Subject-Condition-HPSG} A lexical head's \textsc{subcat} list may contain a slashed subject only if it also contains another slashed element. \citep[200]{Pollard.1994}
\z 

Hence, what \citeauthor{Pollard.1994} ultimately say is that a syntactic island constraint is possible in the HPSG formalism, but they leave open the question of whether it should be used in the case of subject islands. 
Subsequent work on extraction has abandoned this Subject Condition \citep{Godard.1996,Sag.1997,Bouma.2001,Sag.2010,Chaves.2020.UDC}, which is probably motivated by the lack of evidence that a subject island constraint exists. \citeauthor{Godard.1988} and her remarks on French \emph{dont} relative clauses may have played an important role in this change. Still, \citet{Levine.2003} go back to the idea of an English-specific syntactic constraint for subject islands, but their approach is the exception rather than the rule.

% LFG (Kaplan.1989) : a filler with a certain discourse function (eg. focus) searches a path to a gap with a certain grammatical function (eg. direct object), which should have the same index value than the gap (and be phonologically empty ?). Only certain gaps are allowed (therefore quite close to the idea of Chomsky with barriers/phases).
% Role and Reference Grammar (Van-valin.2005) : islands are completely pragmatic

In conclusion, as far as the subject island is concerned, we can see that there is a de facto difference between HPSG accounts and analyses proposed in a ``Chomskyan'' tradition of generative grammar. The HPSG analysis I develop in Part~\ref{part:4} of this book is in line with a general tendency of HPSG because I adopt a discourse-based approach to the phenomenon. This is motivated by empirical evidence that I present in the rest of this work. However, this does not mean that a syntax-based analysis of the subject island is inherently impossible in HPSG. Syntax-based accounts have been proposed in the past. Simpler Syntax borrows from HPSG the analysis of filler-gap dependencies (``discontinuous dependencies'') through a feature \textsc{slash}, but has a syntactic constraint for extraction out of subjects \citep[332]{Culicover.2005}. To my knowledge, this is also the state of the art in Lexical Functional Grammar. 
At the same time, \citet{Erteschik-Shir.1973} has shown that a discourse-based analysis of islands phenomena is possible in the framework of Transformation Grammar.
