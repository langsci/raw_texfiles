\label{ch:discourse}

It is time to address the last type of approach, the one that takes into account pragmatics and discourse. In this chapter I introduce the working hypothesis formulated with colleagues during the years of research that led to the writing of this book: the Focus-Background Conflict constraint. Thus this chapter focuses on aspects of information structure.

\label{ch:discourse-relevance}

It is important to note, however, that there are also other pragmatic considerations at play. Constructions involving extractions are costly from a processing point of view, but they are used in order to fulfill specific communicative goals (which differ from one construction to another). If this were not the case, according to \citegen{Grice.1975} Cooperative Principle, the speaker would resort to a simpler construction. It is therefore essential that the use of extraction is relevant: this is the central idea of \citet{Chaves.2020.UDC} who attribute the first formulation of it to \citet{Kuno.1987}. 

\citeauthor{Kuno.1987} notices that the contrast in (\ref{ex:book-about-write-destroy}) cannot be attributed to a semantic distinction between the two verbs involved.

\begin{exe}
\ex \citep[23]{Kuno.1987}
\begin{xlist}
\ex[]{What did you see pictures of?}
\ex[*]{What did you see a book about?} 
\label{ex:non-relevant-book-about}
\end{xlist}
\end{exe}

The explanation, \citeauthor{Kuno.1972} claims, relies on the fact that seeing a picture is necessarily synonymous with seeing what this picture portrays. For that reason, what the picture portrays is a relevant aspect of the event of seeing a picture. By contrast, seeing a book does not necessarily imply seeing what the book is about. Reading a text and seeing the physical object book are two distinct events. Consequently, it is not obvious that the theme of the book is relevant to the event of seeing a book. Because the theme of the book is not relevant, there is no compelling reason to ask questions about it. Hence, it is difficult for the addressee to imagine a context in which the question would be needed in the first place, and the sentence (\ref{ex:non-relevant-book-about}) is perceived as unacceptable. 

\citet{Chaves.2013} shows with numerous examples that extraction of non-relevant elements leads to an important degradation of the sentences. Example (\ref{ex:owner-cat}) is an extreme case. It is indeed very difficult to imagine a situation in which the ownership is at-issue in a sneezing event.

\ea \citep[12]{Chaves.2013}\\
* What$_i$ did [the owner of~\trace{}$_i$] sneeze?
\label{ex:owner-cat}
\z 

%The filler of the gap must be relevant for the proposition (the assertion) and the subject. cf: ``shadow arguments'' in Pustejovsky (1995,ch.5).
% Pustejovsky, James. 1995. The Generative Lexicon. MIT Press, Cambridge, MA, USA.

Expanding on \citegen{Kuno.1976} idea, \citet[327]{Chaves.2020.UDC} posit that the extracted element must be relevant to ``the main action that the sentence describes''. 

It seems indisputable to me that this factor is highly pertinent. I refer the reader to \citet{Chaves.2020.UDC} for a more exhaustive discussion. My own work, however, has addressed a discourse factor that is different from relevance, but which is not in contradiction with it. Before going into more detail about my working hypothesis, I need to define the concepts of information structure that I use in this book (Section \ref{ch:is}), and then present previous works that have explored these notions in the context of islands.

\section{Information structure}
\label{ch:is}

The discourse-based approach takes into account an undoubtedly crucial parameter in communication, a parameter so complex that it is very difficult to formalize it entirely: the fact that communication is an interchange of information between several participants in a discourse event. Phatic discourse, i.e.\ discourse in which no exchange of information is involved, is possible, but it is the exception rather than the rule. An interchange of information between several participants requires from each of them a capacity for what \citet{Kuno.1976} calls ``empathy'' (and which could also be labeled ``Theory of Mind''): They need to recall which pieces of information the other participant(s) have and which ones they do not have. The sentence ``\emph{Mary is a good scholar.}'' only succeeds in its informative role if all participants in the discourse know who Mary is \citep[309]{Kuno.1972}. In order for the communication to be efficient, it is also necessary that each participant keeps in mind the information that they have already been given in the previous part of the discourse.

In this work, we assume a formalization of how individual participants manage discourse information based on ``information packaging'' \citep{Chafe.1976} and on the notion of Common Ground \citep{Krifka.2007}.\footnote{The notion of ``Common Ground'' was probably first discussed by \citeauthor{Stalnaker.1978} under the notion of ``common knowledge'' and especially ``common background knowledge'' \citep[86]{Stalnaker.1978}.} % also in Grice 1981:190
Common Ground covers information exchanged in a particular discourse. The Cooperative Principle defined by \citet{Grice.1975} also implies that every utterance in a discourse entails a proposition which augments the Common Ground. For this reason, example (\ref{ex:kuno-presupposition-vet-cat}) is inappropriate in the specific discourse situation. The first utterance, \emph{I had to bring my cat to the vet}, entails at least two pieces of information: the first one is presupposed and is that a cat exists and that the speaker is the owner of this cat; the second piece of information, \emph{I have a cat}, brings no new information to the Common Ground, because the information that it entails is redundant to the information presupposed in the previous utterance. This is not the case with (\ref{ex:kuno-presupposition-cat-vet}), where each part of the sentence brings new information to the Common Ground. 

\eal 
\label{ex:kuno-presupposition}
\ex I have a cat, and I had to bring my cat to the vet.
\label{ex:kuno-presupposition-cat-vet}
\ex  \citep[16]{Krifka.2007}\\
\# I had to bring my cat to the vet and I have a cat.
\label{ex:kuno-presupposition-vet-cat}
\zl 

\subsection{Topic and comment}

\citet{Reinhart.1982} proposes a helpful metaphor to describe the integration of new information in the Common Ground: we can imagine the information that conversation partners store as a collection of index cards. Each card has a title, its index: this is the topic. On each card, under the title, participants keep record of the relevant information: this is the comment, i.e.\ what the utterance states about the topic.

We use the subscript $_T$ and square brackets to identify the topic in our examples if needed. Similarly, we use the subscript $_C$ and square brackets to identify the comment. 

For example, in (\ref{ex:basic-topic}), the first utterance introduces the individual Geneva to the Common Ground; this individual is then the topic of the second utterance. The comment is the information about her love for parrots.

\ea This is Geneva Howell. [Geneva]$_T$ loves parrots.
\label{ex:basic-topic}
\z 

Example (\ref{ex:basic-topic}) illustrates an \emph{aboutness topic}. There is a second kind of topic, called \emph{frame-setting topic} \citep[45--46]{Krifka.2007}. A frame-setting topic ``acts as a restrictor as to when, where or with respect to who or what, the truth value of the predication is to be evaluated'' \citep[130]{Erteschik-Shir.1997}. In example (\ref{ex:körperlich}), the adverb is the topic and restricts the domain for which it is true that Peter is well (implicitly implying that Peter is not well regarding other domains of his life).

\ea \citep[655]{Jacobs.2001}\\
\gll [Körperlich]$_T$ geht es Peter gut.\\
physically goes it Peter well\\
\glt `Physically, Peter is well.'
\label{ex:körperlich}
\z 

As we will see, the aboutness topic will be the most relevant one in the study of subject islands. Very often, the aboutness topic of an utterance is anaphoric, hence uses ``given'' information. ``Given'' means here that it already entered the Common Ground at some point of the discourse. However, givenness is not required for topicalization: Corpus studies show that new information can become the topic of an utterance \citep[41--42]{Krifka.2007}.  

\eal
\ex \citep[42]{Krifka.2007}\\
{} [A good friend of mine]$_T$ [married Britney Spears last year]$_C$.
\ex \citep[66]{Reinhart.1982}\\%(exemple (21a)
Because they wanted to know more about the ocean's current, [students in the science club at Mark Twain Junior High School of Coney Island]$_T$ gave ten bottles with return address cards inside to crewmen of one of New York City's sludge barges.
\zl 

\citet[Section~3.2]{Reinhart.1982} proposes a way to test whether X is the aboutness topic of a given utterance by using a paraphrase such as ``\emph{as for} X ...'', ``\emph{speaking about} X ...'', or ``\emph{about} X ...''. If the paraphrase is pragmatically identical with the original utterance, then X is the aboutness topic. 
Based on this idea, \citeauthor{Goetze.2007} propose the following test for aboutness topics:

\ea An NP X is the aboutness topic of a sentence S containing X if: S would be the natural continuation of the announcement \emph{Let me tell you something about X}. \citep[19]{Goetze.2007}
\z 

Lastly, I should add that some sentences do not contain a topic. Topicless sentences are called thetic sentences \citep{Kuroda.1976,Ladusaw.1994}. 

\subsection{Focus}

The notion of ``focus'' is very common in linguistics and at the same time usually poorly defined. Different authors employ the term in different ways, without explicitly specifying which definition they are using. In this work I adopt the definition given by Alternative Semantics \citep{Rooth.1992}, which has the advantage of being a formal definition. In Alternative Semantics, focus signals the importance of alternatives to the focused element for the interpretation of the utterance. For this reason, the most straightforward example of focus is an answer to a \emph{wh}-question. An interrogative word like \emph{which} in (\ref{ex:def-focus-qu}) signals a set of alternatives (here the set of speaker B's siblings, let us assume the set \{Jennifer, Karen, Brandon\}). The focus of speaker B's answer in (\ref{ex:def-focus-ans}) is the most informative element of the utterance (namely here the individual Karen).

% Collingwood (1940) is credited with the original idea in Anglo-Saxon literature: "every statement that anybody ever makes is made in answer to a question" (cited in Fintel 2004 : 147).
%     Collingwood, 1940, An essay on metaphysics. Oxford. 
%    Fintel von K., 2004, A minimal theory of adverbial quantification, [Kamp H. & Partee B., eds] Context-dependence in the analysis of linguistic meaning, Elsevier. 

\eal 
\ex Speaker A: Which one of your siblings is the oldest? \label{ex:def-focus-qu}
\ex Speaker B: [Karen]$_F$ is the oldest. \label{ex:def-focus-ans}
\zl 

We use the subscript $_F$ and square brackets to identify the focus in our examples if needed, as can be seen in (\ref{ex:def-focus-ans}). All other elements of the utterance -- i.e.\ \emph{be the oldest} for (\ref{ex:def-focus-ans}) -- are backgrounded (see Section~\ref{ch:background}).  

Any kind of constituent can be focused: a single word like in (\ref{ex:def-focus-ans}), whole sentences like in (\ref{ex:whole-sentence-focus}), as well as everything in between.\footnote{Even contrastive focus on one syllable is possible in order to stress metalinguistic information.
\begin{itemize}
    \item[(i)] I did not say that he had been a pathetic help, but that he had been a SYMpathetic help! 
\end{itemize}}  

\eal 
\ex Speaker A: What happened?
\ex Speaker B: [Karen bought a parrot]$_F$. \label{ex:whole-sentence-focus}
\zl 


This leads to a distinction between narrow and broad focus based on the type of constituent being focused: 
the whole sentence (broad focus) or some constituent(s) (narrow focus). 

Though other definitions have been proposed in previous literature, I will assume \citegen{Goetze.2007} definition of focus:

\ea Typically, focus on a subexpression indicates that it is selected from possible alternatives that are either implicit or given explicitly, whereas the background can be derived from the context of the utterance.\label{ex:definition-focus}
\z 

The focused element is also mostly the one bearing the main stress of the sentence (at least in languages like English or French).
\citet{Krifka.2007} notes, however, that stress is only one possible way to signal focus, and not the very definition of focus. Prosody is at best a useful tool to identify certain kinds of focus. In the written language, however, we can only stipulate the place of the main stress. As the empirical part of this work is based almost exclusively on written French (research in written corpus and experiments based on reading tasks), I will not say much about the intonational aspect of focus.

% on correlation between IS and pitch: \citet{Gregory.2001}

Focus is also sometimes described as the most ``important'' part of the utterance. \citet{Krifka.2007} criticizes this formulation for being vague and subjective. In his opinion, importance, as well as pertinence or main stress, only correlates with focus, but none of these aspects are criteria to define it. 

% Zimmerman and Onea 2011 for a distinction between informative and contrastive focus
        % Zimmermann, M., and Onea, E. (2011). Focus marking and focus realization. Lingua 121, 1651–1670. doi: 10.1016/j.lingua.2011.06.002

Finally, it is also useful to say a word on the relation between focus and new information, or between focus and the topic/comment distinction. The focus, unlike the background as stated in definition (\ref{ex:definition-focus}), cannot be derived from the context of the utterance, it is new information. This does not mean, however, that the semantic referent has not been mentioned in the discourse, only that this part of the proposition is new. In (\ref{ex:not-new-focus}), the answer selects one of the alternatives previously mentioned in the discourse. What is new is that the destination was the beach. 

\eal \label{ex:not-new-focus}
\ex Speaker A: Did you go to the beach or to the museum yesterday?
\ex Speaker B: We went to the [beach]$_F$.
\zl 

In (\ref{ex:old-focus}), the focus contains an anaphoric pronoun, i.e.\ the referent has already been mentioned, but it is still a felicitous answer to speaker A's question, because it is selected from other possible alternative answers. 

\begin{exe}
\ex \citep{Marandin.2007}
\label{ex:old-focus}
\begin{xlist}
\ex Speaker A: Who did Felix praise?
\ex Speaker B: Felix praised [himself]$_F$.
\end{xlist}
\end{exe}

Because of this, focus is often part of the comment, but contrastive, corrective or confirmative focus (see below) is also possible on the topic, which is then usually called a ``contrastive topic''. 

Many kinds of focus have been identified in the literature \citep[6--34]{Krifka.2007}. I will now define the kinds of focus which are useful in this work. This list is by no means exhaustive.

\subsubsection{Information focus}

Information focus is the prototypical kind of focus, and is also called \emph{semantic focus}. Informational focus occurs when new information is added to the Common Ground; it is the element that answers the implicit or explicit question. 

% called semantic focus (= non-contrastive, the new information, hence informational focus) in Song 2017, this terminology comes from Gundel 1999
% Gundel, Jeanette K. 1999. On different kinds of focus. In Peter Bosch & Rob van der Sandt (eds.), focus: linguistic, cognitive, and computational perspectives, 293–305. Cambridge, UK: Cambridge University Press.

\subsubsection{Contrastive focus}
An utterance containing a contrastive focus reacts to a proposition which just entered the Common Ground. The focus signals an element that the speaker wants to correct or wants to provide additional information on. One example of the latter is given in~(\ref{ex:contrastive-focus}).

\eal \label{ex:contrastive-focus}
\ex Speaker A: Karen has a child.
\ex Speaker B: [Brandon]$_F$ has a child too.
\zl 

\subsubsection{Corrective (or confirmative) focus} An utterance with a corrective or confirmative focus also reacts to a proposition which just entered the Common Ground. In (\ref{ex:def-focus-corrective}), the focus element corrects the alternative previously mentioned in the discourse (here: Karen) and excludes it: this alternative makes the proposition false. In confirmative focus like (\ref{ex:def-focus-confirmative}), the alternative previously mentioned in the discourse is pertinent, and other potential alternatives are excluded: the proposition with this alternative is true.

\begin{exe}
\ex Speaker A: Karen is the oldest.
\begin{xlist}
\ex Speaker B: No, [Brandon]$_F$ is the oldest. \label{ex:def-focus-corrective}
\ex Speaker B: Yes, [Karen]$_F$ is the oldest. \label{ex:def-focus-confirmative}
\end{xlist}
\end{exe}

\subsubsection{A topic with focus properties: Contrastive topic}

The answer in (\ref{ex:contrastive-topic-ans}) contains two topics: \emph{Karen} and \emph{Brandon}. Both are continuation topics that add more information to the topic introduced in the question (\ref{ex:contrastive-topic-qu}), \emph{your siblings}. Since \emph{siblings} refers to several individuals, there is potentially a need to distinguish between them.

\eal 
\ex Speaker A: What do your siblings do? \label{ex:contrastive-topic-qu}
\ex Speaker B: [Karen]\textsubscript{CT} is a writer and [Brandon]\textsubscript{CT} is a life guard. \label{ex:contrastive-topic-ans}
\zl 

In this case, we talk about contrastive topics \citep[44--45]{Krifka.2007}. Contrastive topics have some properties of focus, because they signal a set of sets of propositions (whereas focus signals a set of propositions). 


\subsection{Background (and presuppositional content)}
\label{ch:background}

The background~-- already defined in (\ref{ex:definition-focus})~-- is the part of the utterance that is presupposed, following the definition of presupposition given by \cite{Lambrecht.1994}. 

% Background :
% Büring 1999 => topic is not backgrounded
% Büring, Daniel. 1999. Topic. In P. Bosch and R. van der Sandt, eds., Focus, Studies in Natural Language Processing, pages 142-165. Cambridge, UK: Cambridge University Press.
% Vallduvi => topic is part of background
% von Stechow 1981,101
% von Stechow, Arnim. 1981. Topic, focus, and local relevance. In W. Klein and W. Levelt, eds., Crossing the Boundaries in Linguistics , pages 95-130. Dordrecht: Reidel.
% Le Draoulec 1997:9-44

\ea A proposition P is a pragmatic presupposition of a speaker in a given context just in case the speaker assumes or believes that P, assumes or believes that his addressee assumes or believes that P, and assumes or believes that this addressee recognizes that he is making these assumptions, or has these beliefs. \citep[51]{Lambrecht.1994}
\z 

Any information can be backgrounded because it is old information in the Common Ground, but it can also be new information that is not at issue and that must be taken for granted in order for the propositional content of the utterance to be true \citep[54]{Lambrecht.1994}. Focus and Background are in complementary distribution, such that an element in the utterance must be either focused or backgrounded. 

In Alternative Semantics, the background is regarded as introducing a set of only one element. 

Presuppositions differ from implicatures, because implicatures can be negated while the negation of a presupposition is infelicitous. Consider (\ref{ex:implicature}), whose implicature is that the speaker has only one child. Yet, the conversation in (\ref{ex:implicature-neg}) is felicitous, event though this implicature is contradicted in the next sentence.

\eal 
\ex[]{I have a child.} \label{ex:implicature}
\ex[]{Speaker A: I wish I were a father. What about you, do you have a child?\\ Speaker B: Yes, I have a child. Actually, I have three children.} \label{ex:implicature-neg}
\zl 

This contrasts with the presupposition of (\ref{ex:presupposition}), which is that Jennifer has two siblings (the verb \emph{to know} takes as a complement an S whose propositional content is presupposed). Contradicting this presupposition as in (\ref{ex:presupposition-neg}) is infelicitous.

\eal 
\ex[]{The landlord does not know that Jennifer has two siblings.} \label{ex:presupposition}
\ex[\#]{The landlord does not know that Jennifer has two siblings. Actually, she's an only child.} \label{ex:presupposition-neg}
\zl 

This property of presuppositions led \citet{Erteschik-Shir.1973} to propose a test of backgroundedness, called the ``liar test''. The test consists in reporting that someone has said the utterance, and then adding that this person was lying about a particular part of the utterance. Example (\ref{ex:liar-test}) is a liar test for the utterance in (\ref{ex:presupposition}). The continuation in (\ref{ex:liar-test-good}) shows that the elements [\emph{does not know}] are not backgrounded (they bear the informational focus), while the continuation in (\ref{ex:liar-test-bad}) shows that the sentential complement [\emph{Jennifer has two siblings}] is backgrounded. 

\begin{exe}
\ex Ash said: The landlord does not know that Jennifer has two siblings...
\label{ex:liar-test}
\begin{xlist}
\ex[]{... which is a lie: he does know.} \label{ex:liar-test-good}
\ex[\#]{... which is a lie: she's an only child.} \label{ex:liar-test-bad}
\end{xlist}
\end{exe}

\citet[51]{Lambrecht.1994}, \citet{Ambridge.2008} and \citet{Cuneo.2023} propose similar tests that take advantage of the same property of presupposition in order to identify the backgrounded elements. It must be noted that the liar test targets only the backgroundedness with respect to the main clause. The internal information structure of an embedded clause cannot be targeted directly; the embedded sentence has to be tested in isolation.

% Lahousse 2011: difference between assertive and non-assertive (non-assertive = presuppositional)
% caution: the lier test applies to main clauses, not subordination

% other test: assertive epistemic adverbs "perhaps" and "probably" possible only if the statement is assertive Lahousse 2001:236

% presupposition / assertion
% Kuroda 1992:66, an assertion is "an expression of epistemic commitment" (I'm not sure whether there is a connection with clefts and Destruel's commitment???)

\section[General principles: from Erteschik-Shir (1973) to Goldberg's BIC]{General principles: from Erteschik-Shir's dominance constraint on extraction to Goldberg's ``Backgrounded Constituents are Islands''}

In this section, I will present some accounts based on the discourse function of extractions that offer an analysis of many different islands. The next section will be devoted to the subject island in particular.
 
\subsection{The Focus approach}

\citeauthor{Erteschik-Shir.1973} proposed another alternative to syntactic accounts of islands \citep{Erteschik-Shir.1973,Erteschik-Shir.1997,Erteschik-Shir.2006}. Her proposal is based on information structure\footnote{Which she calls the f(ocus)-structure since \citet{Erteschik-Shir.1997}.} and maintains as a general principle that extraction can only occur out of the ``potential focus domain''. The focus domain consists of the focus and the elements it c-commands (including traces). In her early works, this concept was defined as the semantically dominant phrase or clause.\footnote{She defines ``semantic dominance'' as such: ``A clause or phrase is semantically dominant if it is not presupposed and does not have contextual reference.'' \citep[22]{Erteschik-Shir.1973}.
In later works \citep[e.g.][]{Erteschik-Shir.2006}, she says that the two formulations are equivalent.}
Her original constraint is reproduced in (\ref{ex:rule-ES}):

\ea The dominance condition on extraction \citep[27]{Erteschik-Shir.1973}:\\ 
Extraction can occur only out of clauses or phrases which can be considered dominant in some context. 
\label{ex:rule-ES}
\z 

\citeauthor{Erteschik-Shir.1973} uses the liar test in order to identify the potential focus domain in a specific context. Example (\ref{ex:liar-test-rhinoceros}) makes clear that the context is very important in determining the focus domain: in both examples, the liar test targets the complement of the noun in the NP [\emph{a book about Nixon}]. However, the test shows that \emph{about Nixon} is in the focus domain only in (\ref{ex:liar-test-rhinoceros-wrote}), not in (\ref{ex:liar-test-rhinoceros-destroy}).

\eal\label{ex:liar-test-rhinoceros}
\ex[]{Sam said: John wrote a book about Nixon. Which is a lie -- it was about a rhinoceros.} \label{ex:liar-test-rhinoceros-wrote}
\ex[]{Sam said: John destroyed a book about Nixon. \#{}Which is a lie -- it was about a rhinoceros.} \label{ex:liar-test-rhinoceros-destroy}
\zl 
% example cited in Erteschik-shir 1981
%Erteschik-Shir, N.: 1981, ‘On Extraction from Noun Phrases (picture noun phrases)', Annali della Scoula Normale Superiore di Pisa, special issue, Pisa, Italy.

This distinction explains the contrast in (\ref{ex:book-about-write-destroy}): in (\ref{ex:book-about-write}), the extracted element belongs to the potential focus domain, and can therefore be extracted following (\ref{ex:rule-ES}), whereas in (\ref{ex:book-about-destroy}), it does not.

\begin{exe}
\ex \citep[272]{Bach.1976}
\label{ex:book-about-write-destroy}
\begin{xlist}
\ex[]{What did they write a book about? \label{ex:book-about-write}}
\ex[*]{What did they destroy a book about? \label{ex:book-about-destroy}}
\end{xlist}
\end{exe}

Context also plays a role in the fact that acceptability varies for islands: ``the positive response of informants is conditional on their ability to contextualize in such a way that the clause from which extraction has occurred is interpreted as a focus domain" \citep[320]{Erteschik-Shir.2006}.

\subsection{The Topic approach}\largerpage[-2]

In contrast to \citeauthor{Erteschik-Shir.1973}, \citet{Kuno.1987} proposes an account of islands based on Topic. He notices that \citeauthor{Erteschik-Shir.1973}'s proposal is not able to account for the contrast between (\ref{ex:marys-portrait-a-portrait}) and (\ref{ex:marys-portrait}). The context remains the same (hence with the same potential focus domain), and the liar test gives similar results but extracting \emph{the actress} out of the NP [\emph{Mary's portrait of this actress}] is not felicitous. The reason cannot be the presence of the  genitive \emph{Mary's} alone, because the extraction in (\ref{ex:marys-version}) is felicitous. 
\pagebreak
\begin{exe}
\ex \citep[13]{Kuno.1987} 
\label{ex:marys-portrait-version}
\begin{xlist}
\ex[]{Yesterday, I met the actress who I had bought a portrait of. \label{ex:marys-portrait-a-portrait}}
\ex[*]{Yesterday, I met the actress who I had bought Mary's portrait of. \label{ex:marys-portrait}}
\ex[]{This is the story that I haven't been able to get Mary's version of.}
\label{ex:marys-version}
\end{xlist}
\end{exe}

\citegen{Kuno.1987} proposal is that topics, and not dominant (or focused) elements, are extracted. His definition of topichood is somewhat broader than the one I gave previously, because in his proposal not only utterances but NPs can have a topic as well. 
% Kuno also says there's a problem with this definition of topic and isn't very happy with it: no test (continuation topic not very conclusive), and he has to assume a topic/comment structure within the NP. 
The utterance (\ref{ex:marys-version}) implies that the speaker has heard the version of this story from at least one other person. It thus opens an alternative set: \emph{Mary's} is interpreted as contrastive, and therefore as focus. \emph{The actress} can be interpreted as the topic of the NP in (\ref{ex:marys-portrait-a-portrait}), while in (\ref{ex:marys-portrait}) \emph{Mary} is more naturally the topic, and \emph{the actress} the focus in the NP. 
\citeauthor{Kuno.1987} formulates this constraint as follows:

\ea[]{Topichood Condition for Extraction \citep[23]{Kuno.1987}:\\
Only those constituents in a sentence that qualify as the topic of the sentence can undergo extraction processes (i.e.\ Wh-Q Movement, Wh-Relative Movement, Topicalization, and It-Clefting).}
\z 

% yet Bailard (1981:17) says that extraction emphasizes a noun by adding information and speaks of a "focusing function" (same with clefts). 
% see also Schachter 1973: "foreground function" (clefts and relative clauses)
% Bailard, Joëlle. 1981. A functional approach to subject inversion. Studies in Language 5. 1–29.
% Schachter, Paul. 1973. Focus and Relativization. Language. Linguistic Society of America 49(1). 19–46. https://doi.org/10.2307/412101.

\subsection{The salience approach (reconciling the Focus and Topic approaches)}
\label{ch:salience-and-BCI}\largerpage[-2]

The Focus approach and the Topic approach are not mutually exclusive, and \citet{Kuno.1987} sees the Topic Condition as an extension of \citeauthor{Erteschik-Shir.1973}'s rule. What is missing in both accounts, however, is an explanation of how a syntactic factor like extraction and discursive factors like topic and focus interact. 

\citet{Deane.1991} answers this concern and provides a unifying account based on the management of cognitive resources. He suggests that extraction requires simultaneous consideration of two separate parts, the filler and its head, which we need to link together in order to obtain the appropriate syntactic structure. The longer the distance, the stronger this division of attention taxes our cognitive resources: we have limited space in our short-term memory. If the two parts are cognitively salient, however, it is easier to keep them active. Focus and topic are the two most salient elements in the sentence: the focus is salient because it is the important part of the discourse, and the topic is salient because it is the center of interest in the sentence. They are therefore the best candidates for extraction. 

Building on this idea, and adopting the constraint-based counterpart of  \citet{Erteschik-Shir.1973} dominance condition on extraction, \citeauthor{Goldberg.2006} proposed the BCI constraint \citep[see also][]{Ambridge.2008,Goldberg.2013,Cuneo.2023}:

\ea Backgrounded constructions are islands (BCI) \citep[2]{Cuneo.2023}:\\
Constructions are islands to long-distance dependency constructions to the
extent that their content is backgrounded within the domain of
the long-distance dependency construction. 
\label{rule:BCI}
\z 

The BIC and the dominance condition on extraction make the same predictions: extraction out of non-focus (hence backgrounded) constituents is infelicitous. 
%The difference is that \citeauthor{Erteschik-Shir.1973} assumes that extraction is ruled out by default and needs rules that license it, while \citeauthor{Goldberg.2006} assumes that extractions are possible by default, and that only certain constructions can constrain them.  
Both constraints are discourse-based, but this is not reflected explicitly by their respective formulation.

\section{The subject island constraint from a functional perspective}

\citet{Erteschik-Shir.1973} shows that sentential subjects are presupposed in the utterance. Consider first the sentential complement in (\ref{ex:ES-sentential-object}):

\begin{exe} 
\ex \citep[157]{Erteschik-Shir.1973}\\
Bill said `It's likely that Sheila knew all along.'
\label{ex:ES-sentential-object}
\begin{xlist}
\ex[]{, which is a lie -- it isn't.}
\ex[]{, which is a lie -- she didn't.}
\end{xlist}
\end{exe}

Targeting the sentential complement with the liar test seems to be felicitous. We can conclude that the it is not backgrounded, hence part of the potential focus domain. This clearly contrasts with the sentential subject in (\ref{ex:ES-sentential-subject}).

\begin{exe}
\ex \citep[157]{Erteschik-Shir.1973}\\
Bill said `That Sheila knew all along is likely.'
\label{ex:ES-sentential-subject}
\begin{xlist}
\ex[]{, which is a lie -- it isn't.}
\ex[*]{, which is a lie -- she didn't.}
\end{xlist}
\end{exe}

Most scholars in the functional approach agree that the ``subject island contraint'' for an NP subject is caused by the subject being the default topic of the utterance. One piece of evidence is that topics have a preference for being expressed as subjects. Indeed, when \emph{John} is the topic, the answer in (\ref{ex:topic-subject}) is more natural than the one in (\ref{ex:topic-object}). 

\begin{exe} 
\ex \citep[323]{Erteschik-Shir.2006}\\
Tell me about John.
\begin{xlist}
\ex[]{-- He is in love with Mary. \label{ex:topic-subject}}
\ex[]{-- Mary is in love with him. \label{ex:topic-object}}
\end{xlist}
\end{exe}

% Even all-focus structures (out of the blue) have a covert topic before subject: Erteschik-Shir 1997
% Eteschik-Shir 1997 The Dynamics of Focus Structure. Cambridge: Cambridge University Press.

% Grosu 1978 there are "natural" or "preferred" topic-comment structures. Subjects ted to be topic of the clause.
% Webelhuth, 2007: subject = topic

This is not to say that subject are always toppics. We can see a counterexample in (\ref{ex:subject-not-topic-1}). Here, the subject is more likely to be the new or unpredicted information in the sentence, and thus the focus, which is why (\ref{ex:subject-not-topic-2}) is a good paraphrase for it. 

\begin{exe}
\ex \citep{Kuno.1987}
\label{ex:subject-not-topic}
\begin{xlist}
\ex[]{[This person alone]$_F$ [passed the test]$_B$. \label{ex:subject-not-topic-1}}
\ex[]{The only person who passed the test was this person. \label{ex:subject-not-topic-2}}
\end{xlist}
\end{exe}

\citet[324]{Erteschik-Shir.2006} assumes that extraction is allowed only in what she calls ``canonical f-structures'', in which the subject is the topic \citep[see also][186]{Erteschik-Shir.1997}. The reason is that it is harder for the addressee to identify the dependents in an utterance with a non-canonical f-structure like (\ref{ex:subject-not-topic-1}), and it is therefore harder to identify the gap. Because extraction has to take place from the potential focus domain, as stated in (\ref{ex:rule-ES}), extraction out of subjects is ruled out.

As could be expected, \citet{Goldberg.2006} makes a similar proposal. With subjects being default topics~-- what she calls ``primary topics''~--, and topics being backgrounded, extraction out of the subject violates the BCI (\ref{rule:BCI}).
It is possible to extract a primary topic as a whole, but not part of it. She explains: ``It is pragmatically
anomalous to treat an element as at once backgrounded and discourse-prominent.'' Hence, according to \citeauthor{Goldberg.2006}, the subject island is caused by a discourse clash.

\section{The BCI revisited: the Focus-Background Conflict constraint}

The previous functional approaches to islands did not pay much attention to the fact that not all filler-gap dependencies have the same discourse function. Indeed, \emph{wh}-questions and \emph{it}-clefts focus the extracted element \citep{Lambrecht.1994}, while relativization and topicalization topicalize it \citep[15]{Kuno.1987}.\footnote{The idea that the relationship is a topic-comment relationship is not new; it was probably first proposed by \citet{Kuno.1973} for Japanese (see also the Thematic Constraint on Relative Clauses by \citealt[420]{Kuno.1976}): ``On the basis of the pervasive parallelism between topicalization and relativization, I proposed that in Japanese what is relativized is the theme of the relative clause.'' \citep[15]{Kuno.1987}. \citet[25]{Schachter.1973} provides evidence from Ilonggo based on case marking that supports this claim. Several authors assume that the topic-comment relationship applies to English relative clauses as well (\citealt{Gundel.1974}; \citealt[79]{Gundel.1988}; \citealt[15]{Kuno.1987}). In general, many assume that it is a universal property of relative clauses (even though \citet{Lambrecht.1994} suggests that it may only be true for languages with post-antecedent relative clauses like French or English).}
Even though \citeauthor{Erteschik-Shir.1973}, \citeauthor{Kuno.1972} and \citeauthor{Goldberg.2006} explain constraints on extractions in terms of discourse status, they all take for granted that
``topicalization processes (Topicalization and Relativization) and focusing processes (Wh-Q Movement and It-Clefting) are subject to the same constraint'' \citep[27]{Kuno.1987}.
Probably for the same reason, the constraints they proposed (Subject Condition, BCI, Topichood Condition for Extraction) rely on discursive factors, but are not explained in terms of discursive mechanisms. 

% Intuition in Ross 1971: ‘Variable Strength’, MS, MIT
% Cinque 1990 answers that this is due to referentiality of relative clauses. But Erteschik-Shir (2006,318) criticizes this answer: Cinque does not explain how referenciality influences something here.
% Cinque, G. (1990) Types of A’-Dependencies. Cambridge, MA: MIT Press.

Notably, these proposals all assume that extraction is a key factor in the constraint. But extraction, and word order more generally, is only one of many tools used in human languages to encode specific discourse status. There is no reason to believe that discourse clash cannot lead to infelicitous sentences independently of extraction.
There are actually several examples of subject/object asymmetries present in \textit{wh}-questions and not in relative clauses: in Kihung’an \citet{Takizala.1973}, in Chiche\^wa \citep{Bresnan.1987}, in Kaqchikel Mayan \citep{Heaton.2016} or in Tagalog \citet{PizarroGuevara.2020}. For example, in Chiche\^wa (a language from the Bantu family), in which object marking (\textsc{om}) on the verb is otherwise optional, the presence of an object marker is ruled out for object interrogatives. This is illustrated by the contrast in (\ref{ex:chichewa}). 

\begin{exe}
\ex \citep[759--760]{Bresnan.1987}
\label{ex:chichewa}
\begin{xlist}
\ex[]{\gll Mu-ku-fún-á chiyâni?\\
you-\textsc{pres}-want-\textsc{indic} what\\
\glt `What do you want?'}
\ex[*]{\gll Mu-ku-chí-fún-á chiyâni?\\
you-\textsc{pres}-\textsc{om}-want-\textsc{indic} what\\
\glt `What do you want?'}
\label{ex:chichewa-bad}
\end{xlist}
\end{exe}

In (\ref{ex:chichewa-bad}), the verbal object marker \emph{-chí-} seems incompatible with the object interrogative word \emph{chiyâni}. According to \citet[758--760]{Bresnan.1987}, the reason is that the verbal object marker \emph{-chí-} is an anaphoric pronoun that signals that the object is the topic. Since the object cannot be topic and focus of the utterance at the same time, the sentence is ruled out.
\begin{quote}
    Because the topic designates what is under discussion (whether previously mentioned or assumed in discourse), it is presupposed. The interrogative focus designates what is \textsc{not} presupposed as known, and is contrasted with presupposed material. Hence, allowing the same constituent to be both topic and focus of the same clause leads to inconsistent presupposition. \citep[758]{Bresnan.1987}
\end{quote}
Extraction here plays no role, because the interrogative word is in situ.

Furthermore, as already discussed in Section~\ref{ch:processing}, the contrast between the subject island on the one hand and the greater preference for subject relatives over object relatives on the other hand is very surprising and remains unexplained under the previous discourse-based accounts. The subject island seems to directly contradict \citegen{Keenan.1977} Accessibility hierarchy. 

Based on experimental data from English and French, we proposed in \citet[rule (8)]{Abeille.2020.Cognition} that the penalty observed in extraction out of the subject known as ``subject island''
is caused by a discourse clash: the degradation results from the attempt to focalize some part of a backgrounded element. Indeed, it seems reasonable to assume that we cannot simultaneously identify an individual \emph{x} as part of the Common Ground and open a set of alternatives about some property inherent to this same individual. 
We therefore reformulated the BCI and call this the Focus-Background Conflict constraint, which we define as:

\ea Focus-background conflict (FBC) constraint:\\
A focused element should not be part of a backgrounded constituent.
\label{rule:FBC}
\z 

We agree with previous discourse-based approaches in assuming that subjects are default topics (and thus backgrounded). Subextraction out of the subject that leads to focalization of the extracted phrase thus violates the FBC constraint, and this, we claim, is why it is degraded compared to a similar subextraction out of the object. Complements have a tendency to belong to the focus, and for this reason subextraction out of the object is more often felicitous. 

Notice that this constraint explicitly presents focusing processes as the cause of the degradation. The straightforward consequence is that only focusing extractions like \emph{wh}-questions and \emph{it}-clefts can violate the FBC constraint. In a relative clause, the extraction is topicalization: the referent denoted by the antecedent of the relative clause (the noun modified by the relative clause) is the topic of the relative clause. In other words, the relative clause ``is about'' the noun it modifies. The subject in relative clauses is preferably backgrounded, but since the extracted element is not focused, the FBC constraint is not violated. 

% NB: Is there a distinction of topicalization of the extracted element between restrictive vs. non-restrictive relatives ? 
% Doug Arnold on restrictive relatives

The scope of the FBC constraint (\ref{rule:FBC}) extends beyond extraction. Focalization of part of a backgrounded constituent that does not involve extraction would violate the FBC constraint as well.

Moreover, the FBC constraint (\ref{rule:FBC}) is not expected to apply to all subjects. Even though subjects are topics by default, they may also be focus. This means that extraction out of a focus subject by means of an interrogative or \emph{it}-cleft should be possible, because this does not lead to a discourse clash. 

