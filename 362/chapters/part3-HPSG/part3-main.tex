\chapter{General discussion}
\label{ch:fbc-generals}
The main finding of the corpus studies and experiments that I have presented in this book is that there is a cross-construction difference as far as extracting out of the subject is concerned, and that the constraints on locality known as ``subject island'' since the foundational work of \citet{Ross.1967} depend on discourse functions. The empirical data also show that extraction out of the subject depends on the type of subjects: sentential or infinitival subjects are not like NP subjects, and embedded subjects are not like subjects of the matrix clause. This was my motivation for proposing the Focus-Background Conflict constraint in order to account for this cross-construction difference. As I argued in part~II, other processing factors are also useful for understanding the data, but the FBC constraint relies on information structure, at the interface between semantics and pragmatics, and is therefore part of the grammar. 
In this section, I will discuss the FBC constraint in more detail, leaving aside the other processing factors at play for extractions (e.g.\ surprisal caused by infrequent and complex structures). In the following sections, I will propose a formalisation of the FBC constraint in the framework of HPSG.

The FBC constraint in (\ref{rule:FBC}), reproduced here in (\ref{rule:FBC-bis}) for convenience, states the following:

\ea Focus-background conflict (FBC) constraint:\\
A focused element should not be part of a backgrounded constituent.
\label{rule:FBC-bis}
\z 

\section{Clarifying the FBC constraint}

Intuitively, it seems reasonable to assume that something cannot simultaneously be backgrounded and in the foreground, presupposed and unknown, highlighted and able to be elided. Focus and background come in complementary distribution, and an element is either focused or belongs to the background. 

The novelty of the FBC constraint is to state that a constituent cannot be partly focused if it is also backgrounded. Some assumptions are necessary in order to make the constraint work. First, the FBC does not specify what kind of focus is not allowed as part of a backgrounded constituent. So far, my main concern has been informational focus, because that is the type involved in interrogatives and clefts. However, because focus and background are in complementary distribution, I will assume that all kinds of focus fall under the FBC. 

Second, we assume that a constituent is backgrounded if its head is backgrounded. Otherwise relativizing out of an object NP (or any NP) that is part of the focus domain would violate the FBC constraint. In (\ref{ex:object-subextraction-rc}), the relative clause is restrictive and backgrounded, and thus the NP object of the matrix clause is partly backgrounded and partly focused, but we assume that the the direct object is focused, because its head is focused. Hence, (\ref{ex:object-subextraction-rc}) does not violate the FBC constraint. 

\ea I can't see [the man [you mentioned yesterday]$_{\text{Bg}}$]$_{\text{F}}$.
\label{ex:object-subextraction-rc}
\z 

Third, \label{ch:is-internal} 
I assume with other scholars \citep[a.o.][]{Lahousse.2011,Song.2017} that embedded clauses have an internal information structure, independent of the information structure role the clause itself may play with respect to the matrix clause meaning. In the formalism that I adopt in the following sections, information structure is expressed as a binary relation between a constituent and a clause meaning. For example, a restrictive relative clause is generally considered as backgrounded with respect to the meaning of the clause that embeds it, but this does not mean that its antecedent cannot play the role of a topic with respect to the relative clause itself. In example (\ref{ex:internal-is}), I use a graphical representation of information structure in which the binary relation is represented by an arrow starting from the constituent that bears the information structure role and pointing to the clause (represented by its verb) with respect to which the relation holds.

\ea \label{ex:internal-is}
\attop{\oneline{\begin{forest}
[[The, no edge] [mice, no edge, name = mouse] [{[}to whom, no edge] [we, no edge] [gave, no edge, name = gave] [only, no edge] [glucose{]}, no edge, name = glucose] [were, no edge, name = were] [in, no edge] [lipogenesis., no edge, name = lipo]]
\draw[->] ([xshift= 0pt]mouse.north) .. controls +(up:6mm)  and +(up:6mm)  .. node[above] {Topic}  ([xshift= -3pt]gave.north);
\draw[->] ([xshift= 0pt]glucose.north) .. controls +(up:6mm)  and +(up:6mm)  .. node[above] {Focus}  ([xshift= 3pt]gave.north);
\draw[->] ([xshift= -3pt]mouse.north) .. controls +(up:22mm)  and +(up:22mm)  .. node[above] {Topic}  ([xshift= 0pt]were.north);
\draw[->] ([xshift= 0pt]gave.north) .. controls +(up:13mm)  and +(up:13mm)  .. node[above] {BG}  ([xshift= -3pt]were.north);
\draw[->] ([xshift= 0pt]lipo.north) .. controls +(up:6mm)  and +(up:6mm)  .. node[above] {Focus}  ([xshift= 3pt]were.north);
\end{forest}}}
\z 

A possible context for (\ref{ex:internal-is}) is that of an experiment in which one group of mice are fed glucose, and the other ones are fed glucose and fat. Given this context, \emph{mice} is at the same time the default topic of the matrix clause (its subject)\footnote{In this particular context, it is probably best interpreted as a contrastive topic.} and the default topic of the relative clause (its antecedent). The relative clause is interpreted as a restrictive relative, and is therefore backgrounded with respect to the matrix clause meaning. However, the presence of the focus particle \emph{only} signals that \emph{glucose} bears a focus. The focused element [\emph{only glucose}] is hence ``part of'' the backgrounded relative clause, but this does not violate the FBC constraint, because the former is focused with respect to the embedded clause meaning and the latter is backgrounded with respect to the matrix clause meaning. 
By contrast, (\ref{ex:internal-is-clash}) does violate the FBC constraint, because [\emph{which car}] is focus with respect to the matrix clause meaning, and [\emph{the driver of (the car)}] is topic with respect to the same clause meaning. The former is therefore both topic and focus of the clause meaning at the same time.

\ea \label{ex:internal-is-clash}
\begin{forest}
[
[{[}Which, no edge] [car{]}, no edge, name = car] [did, no edge] [{[}the, no edge] [driver, no edge, name = driver] [of{]}, no edge] [cause, no edge, name = cause] [a, no edge] [scandal?, no edge]
]
\draw[->] ([xshift= 0pt]car.north) .. controls +(up:18mm)  and +(up:18mm)  .. node[above] {Focus}  ([xshift= 0pt]cause.north);
\draw[->] ([xshift= 0pt]driver.north) .. controls +(up:6mm)  and +(up:6mm)  .. node[above] {Topic}  ([xshift= -3pt]cause.north);
\end{forest}
\z 

Fourth, S or VP are considered constituents only with respect to the clause that embeds them.\footnote{Here and in the following, I am distinguishing between an S (or CP) like [\emph{David read the book}] and a VP like [\emph{read the book}].} The FBC constraint does not apply to their internal structure. 
In example (\ref{ex:focus-part-of-VP-answer}), the V is backgrounded, but this does not mean that the NP object cannot be focus. 
Thus the VP is not backgrounded, even though its head is.

\eal \label{ex:focus-part-of-VP}
\ex A: What did David read?
\ex B: [David read]$_{\text{Bg}}$ [this book]$_{\text{F}}$. \label{ex:focus-part-of-VP-answer}
\zl 

Similarly, the subject or an adverb can be the focus of the clause meaning even when the verb is backgrounded. On the other hand, if the meaning of the embedded clause is backgrounded with respect to the matrix clause meaning, as in (\ref{ex:internal-is}), all elements in the embedded clause are backgrounded with respect to the matrix clause meaning.\footnote{Relative clauses belong to the syntactic island defined by \citet{Ross.1967} as ``Complex NP Constraint''. 

\begin{itemize}
    \item[(i)] \citep[119]{Ross.1967}
    \item[] * The man [who$_i$ I read a statement [which$_j$~\trace{}$_j$ was about~\trace{}$_i$]] is sick. 
\end{itemize}

\citet{Goldberg.2013} argues that extracting out of relative clauses is ruled out because relative clauses are backgrounded (according to the BCI constraint, see Section~\ref{ch:salience-and-BCI}). The FBC constraint predicts that focalization out of a relative clause should lead to a discourse clash, but it does not constrain relativizations (topicalization) out of a relative clause. 

This seems at odds with examples like (i), but research on extraction out of relative clauses shows that acceptability increases drastically if the words in the matrix clause are semantically weakly defined, i.e.\ with indefinite antecedents \citep{Kluender.1998} and main verbs almost devoid of meaning \citep[91--92]{Chaves.2020.UDC}. Furthermore, \citet{Erteschik-shir.1979} have shown that extraction out of presentational relative clauses is acceptable. Notice that most of the time, the examples given in this literature are relativizations out of a relative clause.

Furthermore, there are languages that allow some extractions out of a relative clause (with or without resumptive pronouns, see \citet{Crysmann.2012} for a discussion and analysis). Example (ii) is an extraction out of a relative clause in Hausa, showing relativization.

\begin{itemize}
    \item[(ii)] (\citealt[84]{Tuller.1986} cited by \citealt[55]{Crysmann.2012})
    \item[] \gll ?g\`{\=a} m\`{\=a}tar [dà$_i$ ka b\=a nì litt\=afin [dà$_j$ m\`{\=a}l\=amai sukà san mùtumìn [d\=a$_k$~\trace{}$_i$ ta rub\`{\=u}t\=a w\=a~\trace{}$_k$~\trace{}$_j$]]].\\
\sbar{}here.is woman \sbar{}\textsc{rel} 2\textsc{sg.mas.cpl} give me book \sbar{}\textsc{rel} teachers 3\textsc{.pl.cpl} know man \sbar{}\textsc{rel} 3\textsc{.sg.fem.cpl} write for\\
\glt `Here is the woman that you gave me the book that the teachers know the man (she) wrote (it) for (him).'
\end{itemize}

This evidence challenges the idea that extraction out of relative clauses is altogether unacceptable. Therefore, the fact that relativizing out of a relative clause does not violaate the FBC constraint makes correct predictions. But it still remains to be demonstrated that there is a cross-construction difference between relativizations and interrogatives.}

Finally, \label{ch:general-analysis-gradability}
an important aspect of the FBC constraint is that it assumes a gradience of discourse status:
``The more focused an element, the more focused the constituent
it is part of''  \citep[21]{Abeille.2020.Cognition}. The gradient nature of information structure is assumed by many scholars. \citet{Kuno.1987} defines focus as the ``highly unpredictable'' information in the sentence; since information can be more or less predictable, it follows that focalization is graded. \citet[310--311]{Webelhuth.2007} defines a ``more thematic than'' relation that holds between different arguments, such that some are topics (themes) to a greater or lesser degree than the others. 
% Webelhuth, p 310-311: he defines a "more thematic than" relationship (the arguments for this are in Bresnan et al 2007 according to him, but I can't identify them)
% Bresnan, Joan, Cueni, Anna, Nikitina, Tatiana and Baayen, Harald. 2007. Predicting the Dative Alternation. In I. Kraemer G. Boume and J. Zwarts (eds.), Cognitive Foundations of Interpretation, pages 69–94, Amsterdam: Royal Netherlands Academy of Science.
\citet[369]{Ambridge.2008} assume that a constituent can be more or less backgrounded, and that this is the factor that explains the gradient acceptability of extractions out of backgrounded constituents.

\section{The FBC in long-distance dependencies}
\label{ch:analysis-ldd}

In Section \ref{ch:exp-conclu-cross-construction}, I discussed the results of focalization out of backgrounded constructions in long-distance dependencies. In (\ref{ex:interr-ldd-eng-bis}), the extraction gives rise to focalization of the \emph{wh}-phrase. More precisely: The \emph{wh}-phrase is focused with respect to the matrix clause meaning.

\ea \label{ex:interr-ldd-eng-bis}
\attop{\oneline{%
\begin{forest}
[
[Of, no edge] [who, no edge, name = who] [do, no edge] [you, no edge] [think, no edge, name = think] [that, no edge] [the, no edge] [daughter, no edge, name = daughter] [plays, no edge, name = play] [the, no edge] [piano?, no edge]
]
\draw[->] ([xshift= 0pt]who.north) .. controls +(up:6mm)  and +(up:6mm)  .. node[above] {Focus}  ([xshift= 0pt]think.north);
\draw[->] ([xshift= -3pt]who.north) .. controls +(up:24mm)  and +(up:24mm)  .. node[above] {?}  ([xshift= 3pt]play.north);
\draw[->] ([xshift= 0pt]daughter.north) .. controls +(up:6mm)  and +(up:6mm)  .. node[above] {Topic?}  ([xshift= -3pt]play.north);
\end{forest}}}
\z 

Two questions arise in this configuration: What is the status of the \emph{wh}-phrase with respect to the embedded clause meaning? And: Is the subject of the embedded clause also a default topic?\footnote{\citet{Portner.1998} claim that topics have wide scope over the whole utterance. For them, this follows logically from the definition of topics: if the topic is ``the thing which the sentence is about'' \citep[127]{Portner.1998}, then the whole utterance can have only one focus. This would, however, mean that my assumption in the previous section that antecedents are the topic of (restrictive) relative clauses cannot hold.} In view of the FBC constraint, these two questions are interrelated: If we assume that the subject is the default topic of the embedded clause meaning, and that interrogation makes the \emph{wh}-phrase the focus of the embedded clause meaning, then (\ref{ex:interr-ldd-eng-bis}) should violate the FBC constraint. The question is therefore: is there evidence that a discourse clash takes place in embedded structures?

The results of Experiment~13 on \emph{wh}-questions with a long distance dependency were not conclusive. Extraction out of the subject was not rated significantly lower than extraction out of the object, and there was no significant interaction. 

\begin{sloppypar}
Even though Experiment~13 shows null results, other researchers testing similar stimuli in English \citep{Sprouse.2007.PhD,Sprouse.2012,Sprouse.2016} and in Italian \citep{Sprouse.2016} found significant interaction effects.\footnote{I leave aside \citet{Sprouse.2011}, as they did not cross extraction site and extraction type. \citet{Sprouse.2016} reported only a marginally significant interaction ($p<0.062$) in English. However, reanalyzing their data, I found that the interaction became significant ($p<0.05$) using cumulative link mixed models instead of the authors' original linear mixed effects models. Thus, according to the criteria adopted in this book, the interaction for \textit{wh}-question in English in \citet{Sprouse.2016} would be significant. I think we can safely assume that the interaction is robust with their material, although the effect may not be very large.}
\end{sloppypar}

Unfortunately, all these experiments used subject and object relative clauses as a baseline. This is problematic because it is well-known that there is a strong cross-linguistic preference for subject relative clauses. \figref{fig:results-sprouse-2016} illustrates the interaction effects found for English relative clauses and \emph{wh}-questions in \citet{Sprouse.2016}. Even a purely visual inspection of the interaction plot shows a clear subject preference in the non-island baseline (green line). Reanalysis of their data revealed a significant difference ($p<0.005$ and $p<0.05$ respectively) between extraction out of the subject and extraction out of the object (i.e. the slope of the red line is significantly different from 0). However, we cannot tell whether the interaction effect would also appear with another grammatical baseline or is artificially produced by the subject relative clause preference.

\begin{figure}
    \includegraphics[width=\textwidth]{figures/interaction-English-sprouse2016}
    \caption{Interaction plot of \citegen{Sprouse.2016} experiments on subject island in English}
    \label{fig:results-sprouse-2016}
\end{figure}

I conclude that there is enough evidence concerning long-distance dependencies to affirm that the focused element is focus also with respect to the meaning of the embedded clauses. There is certainly a persistent tendency for long-distance \emph{wh}-questions pointing to this, and it is also supported by the intuitions of many scholars. But further research is needed.

\section{The puzzle of \emph{it}-clefts}
\label{ch:analysis-clefts}

In this section, I will concentrate on \emph{it}-clefts (or \emph{c'est}-clefts for French) with narrow focus, i.e.\ with focus on the pivot. As explained in Section~\ref{ch:is-clefts}, all-focus \emph{c'est}-clefts are possible and common in French \citep{Lambrecht.1994,Doetjes.2004}, but they differ in prosody \citep[541--549]{Doetjes.2004} and therefore should be analyzed differently.

\subsection{The information structure of the \textit{that}-clause}

The information structure of the pivot was discussed previously in Section~\ref{ch:is-clefts}, but we have not considered the information structure of the \textit{that}-clause (or \textit{que}-clause in French). The approach that seems to predominate in the literature is \citegen{Prince.1978}. \citeauthor{Prince.1978} describes the content of the \emph{that}-clause in English \emph{it}-clefts as being presupposed and containing ``known'', but not necessarily ``given'' information (her terminology). Actually, in some variants of \emph{it}-clefts ``frequent in historical naratives'', the speaker assumes the information to be unknown to the hearer (hence not given in discourse) but known in general terms as an indisputable (for example historical) fact. One example is given in (\ref{ex:narrative-cleft}).

\ea \citep[900]{Prince.1978}\\
It was in this year that Yekuno Amlak, a local chieftain in the
Amba-Sel area, acceded to the so-called Solomonic Throne. 
\label{ex:narrative-cleft}
\z 

In the other and most common variant of \emph{it}-clefts, the information is presupposed to be known by both speaker and hearer, thus the information in the \emph{that}-clause does not play an important role in the development of the discourse. Sometimes this information may even be given. For this reason, this part may easily be omitted in most of the cases, as shown in (\ref{ex:cleft-that-deleted}).

\ea \citep[897]{Prince.1978}\\
Who made this mold? Was it the teachers? 
\label{ex:cleft-that-deleted}
\z 

% Prince 1978 : ``not the theme'' (not sure if she means by that ``not the topic''?)

That the \emph{that}-clause is presupposed (and therefore backgrounded) can be shown using the negation test: whereas (\ref{ex:cleft-presupposition-test-comp-base}) and (\ref{ex:cleft-presupposition-test-base}) convey the same information, the negation in (\ref{ex:cleft-presupposition-test}) only targets the pivot, whereas John having lost something remains true. This is not the case in (\ref{ex:cleft-presupposition-test-comp}), where the scope of the negation is ambiguous.

\eal 
\ex John lost his keys.
\label{ex:cleft-presupposition-test-comp-base}
\ex It was his keys that John lost.
\label{ex:cleft-presupposition-test-base}
\ex John didn't lose his keys.
\label{ex:cleft-presupposition-test-comp}
\ex \citep[884]{Prince.1978}\\
It wasn't his keys that John lost. 
\label{ex:cleft-presupposition-test}
\zl 

The proposition expressed by the \textit{that}-clause in (\ref{ex:cleft-presupposition-test-base}), namely that \textit{John lost x}, is still true in (\ref{ex:cleft-presupposition-test}). What is negated is identitity x with the entity \textit{his keys}. This seems to indicate that the content of the \textit{that}-clause is backgrounded.

There are, however, alternative views. According to \citet[96]{Gussenhoven.2007}, the \emph{that}-clause in \emph{it}-clefts can contain reactivated information, i.e.\ old information that bears focus. This would be incompatible with the idea that everything in the \textit{that}-clause is backgrounded.  \citet{Song.2017} concludes that there is not enough evidence for considering \emph{that}-clauses in \emph{it}-clefts as backgrounded and decides not to constrain their information structure in his own analysis.

\subsection{A problem for the FBC constraint?}

If, as assumed by \citet{Prince.1978}, the whole \textit{that}-clause is backgrounded, then \textit{it}-clefts are problematic to the FBC. By definition, \textit{it}-clefts would indeed involve  focalization (by means of extraction) of a backgrounded element.\footnote{I thank Daniel Büring for drawing my attention to this issue.} Typically, subextractions out of the NP object like (\ref{ex:cleft-contradiction}) should violate the FBC constraint. And yet, the empirical evidence in \REF{ch:exp14} show that they are acceptable.

\ea It is [of my car]$_{F}$ [that you hate [the color~\trace{}]]$_{B?}$. \label{ex:cleft-contradiction}
\z 

One solution is offered by \citet{Bresnan.1987}. They propose that \emph{it}-clefts are semantically biclausal and assume the information structure in (\ref{ex:is-cleft-bresnan}). According to this analysis, the pivot is the focus of the main clause, and the topic of the embedded clause. 

\ea Information structure of \emph{it}-clefts \citep[adapted from][758]{Bresnan.1987}:\nopagebreak\\
\begin{forest}
[
[It, no edge] [is$_{e1}$, no edge, name = is] [my, no edge] [car, no edge, name = car] [that, no edge, name = that] [you, no edge] [don't, no edge] [want$_{e2}$., no edge, name = want]
]
\draw[->] ([xshift= -3pt]car.north) .. controls +(up:6mm)  and +(up:6mm)  .. node[above] {Focus}  ([xshift= 0pt]is.north);
\draw[->] ([xshift= 3pt]car.north) .. controls +(up:6mm)  and +(up:6mm)  .. node[above] {Topic}  ([xshift= 0pt]want.north);
\end{forest}
\label{ex:is-cleft-bresnan}
\z 

If it-clauses involve two clauses, elements of the \emph{that}-clause are backgrounded with respect to the embedded clause meaning (e2), while the pivot is focus with respect to the main clause meaning (e1). This solves the conflict with the FBC constraint, since the same element is not focus and backgrounded with respect to the same clause.

But the analysis in (\ref{ex:is-cleft-bresnan}) seems problematic to me. There is no meaningful event e1 associated with the copula. What does it mean then for the focused element to be focus with respect to e1? It seems to me that the set of alternatives opened by the focalization in (\ref{ex:is-cleft-bresnan}) is more likely something like: \{\emph{You don't want my house.}, \emph{You don't want Christine's dog.}, \dots\}. For this reason, in the following I will argue for an analysis of \emph{it}-clefts as semantically monoclausal structures.
%\footnote{Furthermore, the analysis in (\ref{ex:is-cleft-bresnan}) presupposes that the \emph{that}-clause in \emph{it}-clefts is a relative. I will argue in the sections below that this cannot be the case.}

\ea Information structure of \emph{it}-clefts in this work:\nopagebreak\\
\begin{forest}
[
[It, no edge] [is, no edge, name = is] [my, no edge] [car, no edge, name = car] [that, no edge, name = that] [you, no edge] [don't, no edge] [want., no edge, name = want]
]
\draw[->] ([xshift= -3pt]car.north) .. controls +(up:6mm)  and +(up:6mm)  .. node[above] {Focus}  ([xshift= 3pt]want.north);
\end{forest}
\label{ex:is-cleft-mythesis}
\z 



\subsection{Other extractions out of the \textit{that}-clause}

We may add that \emph{wh}-questions with extraction out of the \emph{que}-clause in French seem relatively acceptable:

\ea[]{\gll [De qui]$_i$ est - ce que [c' est toi$_j$ qui~\trace{}$_j$ dois tenir [la main~\trace{}$_i$]]?\\
\sbar{}of who is {} it that \sbar{}it is you who must hold \sbar{}the hand\\
\glt `Of whom are you the one who must hold the hand?'}
\label{ex:introspective-cleft-extraction-que-clause}
\z 

In example (\ref{ex:introspective-cleft-extraction-que-clause}), the object complement \textit{de qui} is necessarily focus, since it is the \textit{wh}-element of the \textit{wh}-question. It is also part of the \textit{que}-clause of the embedded \textit{c'est}-cleft, and therefore backgrounded according to \citet{Prince.1978}.

Yet, it would be necessary to confirm this intuition with empirical data, because it has been claimed that such examples are ungrammatical. For example, \citet{Godard.1988} says that extraction out of the \emph{que}-clause in general~-- e.g.\ (\ref{ex:godard-extraction-que-clause})~-- is only possible with a resumptive pronoun. 

\ea \citep[44]{Godard.1988}\\
* \gll Les enfants, [qu$_i$' il était convenu que [c' était le père de Paul qui$_j$~\trace{}$_j$ devait raccompagner~\trace{}$_i$]], ont décidé de rentrer seuls.\\
the children \sbar{}that it was agreed that \sbar{}it was the father of Paul who must\textsc{.past} take.back have decided of come.back alone\\
\glt `The children, that it was agreed that it was Paul's father who was supposed to bring (them) back, have decided to come back on their own.'
\label{ex:godard-extraction-que-clause}
\z\largerpage[2.25]

If \citeauthor{Godard.1988}'s intuition concerning (\ref{ex:godard-extraction-que-clause}) was true, then this case would challenge the FBC constraint, because relativization should be acceptable regardless of the discourse status of the \emph{que}-clause. But examples with a similar structure and unquestionable focalization of the pivot such as (\ref{ex:online-cleft-extraction-que-clause}) can be found online.\footnote{Example (\ref{ex:online-cleft-extraction-que-clause}) from \url{https://www.lorientlejour.com/article/699098/Un\_systeme\_tampon,\_en\_attendant\_un\_nouveau\_pacte\_constitutionnel.html}, last access 19/06/2023}
Hence more work is needed in order to resolve this issue.

\ea[]{\gll des périodes de crise prolongées[,] [[d]ont$_i$ [c' est toujours l' économie et le social qui$_j$~\trace{}$_j$ pâtissent~\trace{}$_i$]]\\
\textsc{det} times of crisis protracted \ssbar{}of.which \sbar{}it is always the economy and the social who suffer\\
\glt `times of long-term crisis, from which it's always the economy and the social affairs that suffer' (i.e.\ the economy and the social affairs suffer from times of long-term crisis)}
\label{ex:online-cleft-extraction-que-clause}
\z 

Examples like (\ref{ex:introspective-cleft-extraction-que-clause}) and (\ref{ex:online-cleft-extraction-que-clause}) seem to indicate either that the FBC constraint is incorrect, or that we have to abandon the assumption that the \emph{que}-clause is backgrounded in French.

\subsection{Assumptions in this book}

In an experiment to be published elsewhere \citep{Winckel.cleft.prep}, we have tested sentences like (\ref{ex:cleft-presupposition-test}), which serves as the basis for the assumption that the content of the \emph{that}-clause is presupposed. The empirical evidence suggests that the elements in the \textit{that}-clause are not backgrounded to the same degree. I will therefore follow \citet{Song.2017} and assume in my analysis that the elements in the \emph{que}-clause may have any discourse status. This also allows me to not take into account the distinction between all-focus and narrow-focus \emph{it}-clefts or \emph{c'est}-clefts. However, an empirically grounded investigation of the information structure of \emph{it}-clefts would be very beneficial to our understanding of focalization. 

% seems to contradict the FBC
% [Traces only of resin , gum and extractive matter] can be separated from the mass . . . 		(Philosophical Magazine: 17)
% reported by Heageman et al 2014, p 88

% Not sure what to do with that :
% Cognitive domains / attributes (Langacker 1987): you may use a part to refer to a whole but not the other way round (that is how metonymy works); you may say the engine to refer to the car, but not a car to refer to the engine. Deane 1991,15-16 says that it has to do with subextraction from NPs: the attribute of a domain can be extracted.



\chapter{A French HPSG fragment}
\label{ch:hpsg-fragment}
In this work I adopt the HPSG framework, based mostly on the general principles laid out in \citet{Pollard.1994}, and updated more recently in \citet*{HPSG.2020}. This framework allows for a precise formal representation of all aspects of linguistic utterances (morphology, syntax, semantics and information structure) and provides a construction-based approach to extraction constructions \citep{Sag.1997,Sag.2010,Ginzburg.2000}. There is already considerable work on French in HPSG, which allowed me to build on a large body of detailed analyses for French syntax and semantics.\footnote{I.e.\ \citet{Miller.1997} and \citet{Abeille.2002} on clitics and auxiliaries, \citet{Abeille.2004} and \citet{Abeille.2006.AandDe} on \emph{de} and \emph{à} prepositions, \citet{Sag.1994.Godard} and \citet{Godard.1996} on dependents of nouns, \citet{Abeille.2003.Flexibility} and \citet{Abeille.2006.Correlatives} for extraction out of specifiers (with degree adverbs and comparative correlatives, respectively), \citet{Abeille.1997} and \citet{Kim.2002} on negation, as well as \citet{Abeille.2008.NP-preposing} and \citet{Marandin.2011} on information structure in French.} 
\citet{Abeille.2007.Grammaires} gives an overview of an HPSG grammar for French which I partially adopt here, though with a binary branching approach.\footnote{HPSG implementations, such as the CoreGram Project \citep{Mueller.S.2015} and the DELPH-IN (DEep Linguistic Processing with HPSG) consortium (\url{http://www.delph-in.net}) use binary branching. However, many HPSG accounts do not, including several I use as the basis of my own analysis here \citep[a.o.][]{Sag.1997,Ginzburg.2000}. 
Traditionally, researchers working on French in HPSG assume a flat structure (a.o., \citealt{Abeille.2002} for tense auxiliaries), but this has recently faced some criticism by \citet{Aguila-Multner.2020} who argue in favor of a binary analysis of VPs in French in order to better account for modification and coordination. I consider that the debate about the appropriate branching in French is still an open one, and choose binary branching as a default that would allow straightforward implementation (e.g.\ the French fragment of the CoreGram project). On the other hand, this choice has, to my knowledge, no major implication for my analysis, especially because I only consider headed structures. A conversion of the present analysis into flat structures is possible: Instead of applying to the object in the non-head daughters, the rules would merely have to apply to all objects in the list of non-head daughters.} The formalism that I adopt for semantics is the Minimal Recursion Semantics (MRS, \citealt{Copestake.2005}) and the formalism for the information structure is based on \citegen{Song.2017} proposal.

\section{General principles of an HPSG grammar}
\label{ch:hpsg-basis}

In HPSG, linguistic entities are modeled by feature structures that are described by feature descriptions (or attribute-value matrices, AVMs). Feature structures are of a certain type, and this type defines which features the sign has and what type of values its features have. Types are ordered in a type hierarchy and every type inherits the properties of its supertype. Values are themselves of a certain type, which is also contained in the type hierarchy of the grammar. A value is either a feature-value description if it is complex, or an atomic value.

A simplified hierarchy of French signs is given in \figref{fig:hrch-sign}, from \citet[19]{Ginzburg.2000}. For example, the subtype \textit{word} inherits the characteristics from the type \emph{lexical-sign} (e.g.\ having a feature \textsc{arg-st}, see below).

\begin{figure}
\begin{forest}
[\textit{sign}
    [\textit{lexical-sign}
        [\textit{lexeme}]
        [\textit{word}]
    ]
    [\textit{clause} [\dots]]
    [\textit{phrase}
        [\textit{non-headed-structure}]
        [\textit{headed-structure} [\dots]]
    ]
]
\end{forest}
    \caption{Type hierarchy of \emph{sign}}
    \label{fig:hrch-sign}
\end{figure}


The lexicon of a given language consists of lexical items. Lexical entries are signs of type \emph{lexeme}. Words are derived from these lexemes through lexical rules (unitary branching structures). Phrases are derived through schemata. 
Not only lexical entries, but also constraints and lexical rules are formalized using AVMs. Constraints define the characteristics of the different types and the conditions the linguistic objects have to satisfy in order to be valid in the given language. Lexical rules define how a lexical item licenses another lexical item.\largerpage[2]

A sign has at least two main features: \textsc{phonology (phon)} and \textsc{syntax-se\-man\-tics (synsem)}. The value of \textsc{phon} is a list of phonemes.{\interfootnotelinepenalty=10000\footnote{For the sake of simplicity, I will follow common practice in HPSG and give as value of the feature \textsc{phon} the orthographic representation of the sign, not its phonetic transcription. 
Some researchers encode more information under \textsc{phon} than its mere phonetic transcription. \textsc{phon} may for example contain information about accents and prosody (\citealt[11]{Engdahl.1996}, \citealt[166]{DeKuthy.2002}, \citealt{Bonami.2006}, \citealt[Chapter~3]{Bildhauer.2008}). Even though prosody plays an important role in information structure, I have little to say about it with respect to the topic of this work. For this reason, I assume a minimal structure for \textsc{phon}.}} The value of \textsc{synsem} is a feature structure of type \textit{synsem}. 

\ea Definition of \emph{sign}\nopagebreak:

\avm{[\type*{sign}\\
phon & list of phonemes\\
synsem & synsem]}
\z 

From the syntactic point of view, in the present work I will mostly consider headed structures, i.e.\ structures in which the mother node structure-shares its \emph{pos} value with one of its daughters. An example of a non-headed structure is coordination (at least in most of the recent HPSG analyses, see \citealt{Abeille.2020.HPSG.Coord}). Signs of the type \emph{headed-structure} have as additional features \textsc{head-daughter (head-dtr)} and \textsc{non-head-daughters (nhead-dtrs)}.\footnote{I adopt the terminology from \citet{Sag.1997}. Other HPSG accounts may use another terminology \citep[e.g., \textsc{daughters} in][Section~2.5]{Ginzburg.2000}.}
The feature \textsc{head-dtr} contains the AVM of type \emph{sign} of the head. The feature \textsc{nhead-dtrs} contains a list of feature structures of the type \emph{sign}. For example, in a headed structure, and assuming a binary analysis, this list constrains only one element, i.e.\ the non-head element (specifier, complement, modifier or filler).\largerpage[2]

\ea Head Feature Principle from \citet[34]{Pollard.1994}: \nopagebreak

\textit{headed-structure} \avm{$\Rightarrow$
[synsem|loc|cat|head & \1 \\ 
head-dtr [synsem|loc|cat|head & \1]]
}
\label{avm:head-feature-principle}
\z 

\begin{figure}
\avm{[\type*{synsem}\\
loc & [\type*{loc}\\
       cat & cat\\
       cont & cont]\\
nonloc & nonloc]}
    \caption{Definition of \emph{synsem} and \emph{loc}}
    \label{fig:synsem-loc-def}
\end{figure}

The feature \textsc{synsem} encodes every piece of information concerning syntax and semantics, see \figref{fig:synsem-loc-def}. Features that concern local dependencies are encoded under the feature \textsc{local (loc)}, in an AVM of type \textit{loc}. Features that concern non-local dependencies are encoded under \textsc{nonlocal (nonloc)}, in an AVM of type \textit{nonloc}. I describe the feature \textsc{nonloc} in Section~\ref{ch:hpsg-extraction}, where I present the details of an HPSG analysis of extractions. Under \textsc{loc}, the features related to semantics (but also pragmatics, as we will see) are encoded under the feature \textsc{content (cont)}. The feature \textsc{category (cat)} contains the features related to (local) syntax.\footnote{Most HPSG versions assume a third local feature \textsc{context} \citep[a.o.][16--21]{Pollard.1994}, where information related to the pragmatics of the sign is encoded. The features of \emph{context} objects are for example \textsc{c-indices} (e.g.\ who is the addressor, who is the addressee) and \textsc{background} (related to what is usually called Common Ground, see Chapter~\ref{ch:discourse}). In my fragment of French, pragmatics is part of \textsc{content}, as assumed by \citet{Song.2016,Song.2017}.}

I will now define how syntax and semantics work on the local level. This includes the encoding of information structure, which is part of \textsc{content} in my formalization. In Section~\ref{ch:hpsg-extraction}, I will then turn to non-local dependencies.

\subsection{Syntax}
\label{ch:hpsg-syntax}

The value of \textsc{cat} is an AVM of the type \emph{cat}. It contains a feature \textsc{head}, whose value is an AVM of the type \textit{part-of-speech} (\textit{pos}). The fragment developed in this book uses seven parts of speech: \textit{noun}, \textit{verb}, \textit{determiner} (\textit{det}), \textit{adjective} (\textit{adj}), \textit{adverb} (\textit{adv}), \textit{preposition} (\textit{prep}) and \textit{complementizer} (\textit{comp}). The hierarchy of \emph{pos} is given in \figref{fig:hrch-pos}. The structure of \textsc{head} varies depending on the part of speech. For example, verbs have a verb form (\textsc{vform}), while other parts of speech do not.

\begin{figure}[ht]
\centering
    \begin{forest}
where n children=0{tier=word}{}
[\textit{pos}
    [\textit{verbal}
        [\textit{comp}]
        [\textit{verb}, name = verb]
    ]
    [\textit{non-comp}
        [\textit{noun-or-verb}, name = nv [{}, no edge]]
        [\textit{non-verbal}
        [\textit{noun}, name = noun]
            [\textit{det}]
            [\textit{adj}]
            [\textit{adv}]
            [\textit{prep}]
        ]
    ]
]
\draw[thin] (noun.north)--(nv.south);
\draw[thin] (verb.north)--(nv.south);
\end{forest}
    \caption{Type hierarchy of \emph{pos}}
    \label{fig:hrch-pos}
\end{figure}

Arguments can be realized or non-realized (e.g.\ \textit{I cooked.}\slash\textit{I cooked lasagne.}). In French, realized arguments or adjuncts can be realized as XPs (e.g.\ NPs, PPs) or as clitics. Following \citet{Miller.1997}, \citet{Abeille.2002} and \citet{Aguila-Multner.2020}, I assume that pronominal clitics in a non-subject function are affixes that attach to the verb. Lastly, arguments and adjuncts may be realized non-locally: this is what I have called ``extraction'' throughout this book. All these possibilities give rise to different subtypes of \emph{synsem}, i.e. \emph{canonical}, \emph{pro} (non-realized), \emph{affix} (realized as clitic) and \emph{gap} (non-locally realized). I adopt the hierarchy in \figref{fig:hrch-syssem}, adapted from \citet{Abeille.1998} and already presented in \citet[104]{Winckel.2020}.
% so-called pronominal clitics sont considérés comme affixes (sauf pronoms sujet), clitic climbing, clitic doubling (sujet)

\begin{figure}[ht]
\centering
\begin{forest}
sn edges,
[\textit{synsem}
    [\textit{non-gap}, name = nong
        [\textit{canonical}, name = canon]
        [{}, no edge]
        [{}, no edge]
    ]
    [\textit{non-canonical}, name=nonc
        [\textit{pro}, name = pro] 
        [\textit{gap}]
        [\textit{aff}]
    ]
]
\draw[thin] (pro.north)--(nong.south);
\end{forest}
    \caption{Type hierarchy of \emph{synsem}}
    \label{fig:hrch-syssem}
\end{figure}
% aff=clitic ≠ pro = petit pro (j'ai mangé pro etc.)

\subsubsection{Valence}\largerpage

Signs of the type \emph{word} have a main feature \textsc{arg-st} that has as value a list of \emph{synsem}. This list is traditionally the list of the different arguments dependent on the lexeme.  
In addition, the word selects its canonically realized complements (\avm{[synsem & canonical]}) via the feature \textsc{comps}, its subject via the feature \textsc{subj} and its specifier via the feature \textsc{spr}. This is captured in the Argument Realization Principle, see \figref{avm:arp}. Non-realized arguments (\avm{[synsem & pro]}) are present in \textsc{arg-st}, and can for example serve as semantic arguments, while no syntactic information is needed about them on the phrasal level. Clitic arguments (\avm{[synsem & aff]}) are realized morphologically through lexical rules, as just stated. Gaps (\avm{[synsem & gap]}) are realized through lexical rules and non-local dependencies; this will be explained in Section~\ref{ch:hpsg-extraction}.\largerpage

\begin{figure}[ht]
\avm{[\type*{cat}\\
       head & pos\\
       subj & list of one or less synsem\\
       spr & list of one or less canonical\\
       comps & list of canonical]}
    \caption{Definition of \emph{cat}}
    \label{fig:cat-definition}
\end{figure}

\begin{figure}[ht]
\textit{word} \avm{$\Rightarrow$
[cat & [subj & \1 \\
        spr & \2 \\
        comps & \3 ]\\
arg-st & \1 $\oplus$ \2 $\oplus$ \3 $\bigcirc$ list of non-canonical]
}
    \caption{Argument Realization Principle (adapted from \citealt[171]{Ginzburg.2000})}
    \label{avm:arp}
\end{figure}

The value of \textsc{comps} is a list of \emph{synsem} objects (only canonical ones), which is a sublist of the \textsc{arg-st} list. The value of \textsc{spr} is a list of \emph{canonical} objects as well, but this list contains at most one element. % Extraction of the specifier: Abeillé & Godard 2003 (combien), kakoi in Russian. Subextraction: "was für ein" in German. 
Subjects need to be in \textsc{subj} even if they are extracted, in order to account for the so-called \emph{que-qui} rule in French, see below. The value of \textsc{subj} is a list of \emph{synsem} objects with one or no element. \figref{fig:cat-definition} summarizes how \textit{cat} is defined.

The example in \figref{fig:ex:avm-enthousiasmer} illustrates how verbs state their arguments using the transitive verb \emph{enthousiasme} (`inspires') from example (\ref{ex:basis-sentence}), where both subject and direct object are realized canonically. 

\ea \gll L' originalité de l' innovation enthousiasme mes collègues.\\
the uniqueness of the innovation excites my colleagues\\
\glt `The uniqueness of the innovation excites my colleagues.'
\label{ex:basis-sentence}
\z 

\begin{figure}
\avm{[\type*{word}\\
      phon & < \type{enthousiasme} >\\
      synsem|loc|cat & [head & verb\\
                        subj & <\1 [loc|cat|head & noun]>\\
                        spr & <>\\
                        comps & <\2 [loc|cat|head & noun]>]\\
     arg-st & < \1, \2 >]}
\caption{\emph{enthousiasme} (`inspires') in (\ref{ex:basis-sentence}) \label{fig:ex:avm-enthousiasmer}}
\end{figure}



\subsubsection{Syntactic composition}

After the selection mechanism, I will now present the mechanism in charge of the syntactic composition. For this, I assume the type hierarchy for \emph{headed-structure} presented in \figref{fig:hrch-headed-structure}, which enhances \figref{fig:hrch-sign}.

\begin{figure}
\begin{forest}
[\textit{headed-structure}
    [head-comps-\\structure, align=center, font=\itshape]
    [head-subj-\\structure, align=center, font=\itshape]
    [head-spr-\\structure, align=center, font=\itshape]
    [head-mod-\\structure, align=center, font=\itshape]
    [\dots]
]
\end{forest}
    \caption{Type hierarchy of \emph{headed-structure} (incomplete)}
    \label{fig:hrch-headed-structure}
\end{figure}

\citet[33]{Ginzburg.2000} a.o.\ define a Generalized Head Feature Principle (GHFP): by default, a headed-phrase's \textsc{synsem} features are the \textsc{synsem} features of its head daughter.\footnote{The Head Feature Principle (\ref{avm:head-feature-principle}) already specifies that a headed phrase's \textsc{head} features are the \textsc{head} features of its head daughter. Unlike the GHFP however, the Head Feature Principle is not a default principle.} This alternative analysis accounts, for example, for the fact that by default valence features remain unchanged from head daughter to mother.

\ea Generalized Head Feature Principle (GHFP, adapted from \citealt[33]{Ginzburg.2000}) \nopagebreak

\textit{headed-structure}
\avm{$\Rightarrow$ [synsem & / \1\\
              head-dtr & [synsem & / \1]]
}
\label{avm:ghfp}
\z 

We define a default value (represented with the sign / ) as follows: the default value (as well as all values that are subsumed by it) can only be overwritten by more specific subtypes \citep[85]{Lascarides.1999}. For the GHFP in (\ref{avm:ghfp}), this means that the values of all \textsc{synsem} features are by default the values of the \textsc{synsem} features of the head daughter, except if they are overwritten by some definition of subtypes of \textit{headed-structure}. This is the case for the definitions of \textit{head-comps-structure}, \textit{head-subj-structure}, \textit{head-spr-structure} and \textit{head-mod-structure}. I will now define each of these subtypes.

In structures that are not a combination of a head with one of its complements, the GHFP (\ref{avm:ghfp}) guarantees that the value of the feature \textsc{comps} remains constant from daughter to mother. Structures that combine a head with one of its complements are of the type \emph{head-comps-structure}, defined in (\ref{avm:head-comps-str}). The value of the complement is ``subtracted'' from the \textsc{comps} list of the daughter.\footnote{Analyses not assuming binary branching do not need to use \textit{head-comps-structure} recursively until the \textsc{comps} list is empty, since all complements can be listed in \textsc{nhead-dtrs}. As mentioned above, a non-binary analysis would not affect the central aspects of my analysis.} Because French complements appear in free order, the ``subtracted'' element can be situated anywhere in the list (hence the shuffle $\bigcirc$ symbol). The AVM in \figref{fig:ex:avm-comps-verb} illustrates the case where \emph{enthousiasme} combines with the NP \emph{mes collègues} in (\ref{ex:basis-sentence}).

\ea \textit{head-comps-structure}
\avm{$\Rightarrow$ [comps & \1\\
              head-dtr & [comps & \1 $\bigcirc$ <\2>]\\
              nhead-dtrs & <[synsem & \2]>]
}
\label{avm:head-comps-str}
\z 

\begin{figure}[ht]
\avm{[\type*{head-comps-structure}\\
phon & <\type{enthousiasme mes collègues}>\\
head & \1 \\
subj & \2 \\
comps & \3\\
head-dtr & [\type*{word}\\
            phon & <\type{enthousiasme}>\\
            head & \1 verb\\
            subj & \2 < NP >\\
            comps & < \4 NP > $\oplus$ \3 <>\\
            arg-st & < \2, \4 >]\\
nhead-dtrs & <[\type*{head-spr-structure}\\
             phon & <\type{mes collègues}>\\
             synsem & \4]>]}
\caption{Illustration of a \textit{head-comps-structure}, following example~(\ref{ex:basis-sentence})}
\label{fig:ex:avm-comps-verb}
\end{figure}

Notice that in (\ref{avm:head-comps-str}) and \figref{fig:ex:avm-comps-verb} I am using shortcuts in my nomenclature: e.g.\ \textsc{head} stands for \textsc{synsem|loc|cat|head}, NP stands for a linguistics object with all the characteristics of an NP (\textsc{head} value is \emph{noun}, empty \textsc{comps} and \textsc{spr} lists). I will use this kind of shortcut throughout Part~\ref{part:3}.

The mechanism is the same when the head combines with its subject. The sign is then of the type \emph{head-subj-structure} (\ref{avm:head-subj-str}) and the value of the subject is ``subtracted'' from the \textsc{subj} list of the head, leaving an empty \textsc{subj} list for the mother. For all other headed structures the value of \textsc{subj} remains constant from head-daughter to mother, as stated by the GHFP (\ref{avm:ghfp}). This is illustrated in \figref{fig:ex:avm-spr-verb} where the verb of  (\ref{ex:basis-sentence}) combines with its subject. 

\ea \textit{head-subj-structure}
\avm{$\Rightarrow$ [subj & <>\\
              head-dtr & [subj & <\1>]\\
              nhead-dtrs & <[synsem & \1]>]
}
\label{avm:head-subj-str}
\z 

\begin{figure}[ht]
\avm{[\type*{head-subj-structure}\\
phon & < \type{l'originalité de l'innovation enthousiasme mes collègues} >\\
head & \1 \\
subj & < > \\
comps & \2\\
head-dtr & [\type*{head-comps-structure}\\
            phon & <\type{enthousiasme mes collègues}>\\
            head & \1 verb\\
            subj & < \3 \type{NP} >\\
            comps & \2 <>]\\
nhead-dtrs & <[\type*{head-spr-structure}\\
             phon & <\type{l'originalité de l'innovation}>\\
             synsem & \3]>]}
\caption{Illustration of a \textit{head-subj-structure}, following example~(\ref{ex:basis-sentence})}
\label{fig:ex:avm-spr-verb}
\end{figure}

\emph{head-spr-structure} (\ref{avm:head-spr-str}) defines the combination of the head with its specifier and follows the same mechanism, while the definition of the GHFP (\ref{avm:ghfp}) states that the value of \textsc{spr} is identical from head daughters to mother for the other headed structures. 

\ea \textit{head-spr-structure}
\avm{$\Rightarrow$ [spr & <>\\
              head-dtr & [spr & <\1>]\\
              nhead-dtrs & <[synsem & \1]>]
}
\label{avm:head-spr-str}
\z 

\noindent A complete syntactic analysis for (\ref{ex:basis-sentence}) can be seen in \figref{fig:avm--basis-sentence-syntax}.\largerpage



\begin{figure}
\resizebox{12cm}{!}{%
\begin{forest}
where n children=0{tier=word}{}
[S \\
\avm{[\type*{head-subj-str}\\
      head & \1\\
      subj & < >\\
      comps & \2]}
    [NP\\
    \avm{[\type*{head-spr-str}\\
      synsem & \3 [head & \4\\
      spr & < >\\
      comps & \5]]}
        [Det \\ \avm{[\type*{word}\\
      synsem \6]}
        [l'\\the]]
        [N$'$\\
        \avm{[\type*{head-comps-str}\\
      head & \4\\
      spr & < \6 >\\
      comps & \5]}
        [N\\
        \avm{[\type*{word}\\
      head & \4 noun\\
      spr & < \6 >\\
      comps & < \7 > $\oplus$ \5 <> ]}
        [originalité\\uniqueness]]
        [PP\\
        \avm{[\type*{head-comps-str}\\
      synsem & \7 [head & \8\\
      spr & < >\\
      comps & \9]]}
            [P\\
            \avm{[\type*{word}\\
                head & \8 prep\\
                    spr & < >\\
                comps & < \tag{10} > $\oplus$ \9 <>]}
                [de\\of]
            ]
            [NP\\
            \avm{[\type*{head-spr-str}\\
                synsem & \tag{10} [head & \tag{11}\\
                spr & < >\\
                comps & \tag{12}]]}
                [Det\\
                \avm{[\type*{word}\\
                    synsem & \tag{13}]} [l'\\the]]
                [N\\
                \avm{[\type*{word}\\
                    head & \tag{11} noun\\
                    spr & < \tag{13} >\\
                    comps & \tag{12} <>]}
                    [innovation\\innovation]
                    ]
            ]
        ]
    ]]
    [VP\\
    \avm{[\type*{head-comps-str}\\
      head & \1\\
      subj & < \3 >\\
      comps & \2]}
        [V\\
        \avm{[\type*{word}\\
      head & \1 verb\\
      subj & < \3 >\\
      comps & <\tag{14}> $\oplus$ \2 <> ]}
            [enthousiasme\\excites]
        ]
        [NP\\
        \avm{[\type*{head-spr-str}\\
      synsem & \tag{14} [head & \tag{15}\\
      spr & < >\\
      comps & \tag{16}]]}
            [Det\\
            \avm{[\type*{word}\\
      synsem & \tag{17}]} [mes\\my]]
            [N\\
            \avm{[\type*{word}\\
      head & \tag{15} noun\\
      spr & < \tag{17} >\\
      comps & \tag{16} <>]} [collègues\\colleagues]]
        ]
    ]
]
\end{forest}}
\caption{Syntactic representation for ``L'originalité de l'innovation enthousiasme mes collégues.'' (`The uniqueness of the innovation excites my colleagues.')}
\label{fig:avm--basis-sentence-syntax}
\end{figure}

Adjuncts select the element they modify through a head feature \textsc{mod} defined for adjectives, adverbs, and verbs. The value of \textsc{mod} is an AVM of type \emph{none-or-synsem}, a supertype of \emph{synsem} (see \figref{fig:hrch-syssem}) and of \emph{none} (whenever the linguistic object is not an adjunct). The head and its adjunct combine via the \emph{head-mod-structure}. The element selected by the modifier can have an empty or a non-empty \textsc{comps} list (it is underspecified in this respect). Consequently, modifiers may appear either after (empty \textsc{comps}) or before (non-empty \textsc{comps} list) the complements of the element they modify. This is illustrated by (\ref{ex:comps-underspecified-adjective}) for adjectives and (\ref{ex:comps-underspecified-adverb}) for adverbs.

\eal\label{ex:comps-underspecified-adjective} 
\ex[]{\gll un livre sur les indiens intéressant\\
a book on the Indians interesting\\
\glt `a book about Indians interesting'}
\ex[]{\gll un livre intéressant sur les indiens\\
a book interesting on the Indians\\
\glt `an interesting book about Indians'}
\zl 

\eal\label{ex:comps-underspecified-adverb} 
\ex[]{\gll Ils sont enthousiastes souvent.\\
they are thrilled often\\
\glt `They are often thrilled.'}
\ex[]{\gll Ils sont souvent enthousiastes.\\
they are often thrilled\\
\glt `They are often thrilled.'}
\zl 

\figref{fig:new:ex:avm-mod-noun-b} shows the combination of a noun and an adjective.

\ea \label{ex:avm-mod-noun}
\gll [l' [innovation formidable]$_{\text{N'}}$]$_{\text{NP}}$\\
\sbar{}the \sbar{}innovation amazing\\
\glt `the amazing innovation'
\z

\begin{figure}[h]
N$'$ \avm{[\type*{head-mod-structure}\\
phon & <\type{innovation formidable}>\\
head & \1 \\
head-dtr & [\type*{word}\\
            phon & <\type{innovation}>\\
            synsem & \2 [head & \1 {noun}]]\\
nhead-dtrs & <[\type*{word}\\
             phon & <\type{formidable}>\\
             head & [\type*{adjective}\\
                     mod & \2]]>]}
\caption{Illustration of a \textit{head-mod-structure}, following example~(\ref{ex:avm-mod-noun})}
\label{fig:new:ex:avm-mod-noun-b}
\end{figure}

\subsubsection{Word order}

In French, specifiers precede the head and complements follow it. A linearization rule states that in \emph{head-comps-structure} objects the head element must precede the non-head element (the complement). Another linearization rule states that in \emph{head-spr-structure} objects the non-head element (the specifier) must precede the head element.

Subject-verb inversion in French is a very complex phenomenon, and its proper treatment would require a long discussion. A complete HPSG analysis of subject-verb inversion can be found in \citet{Bonami.1998} and \citet{Bonami.2001}. For the present work, I assume that the linearization for \emph{head-subj-structure} is underspecified. Verbs bear a feature \textsc{inv} with a boolean value (+/$-$). In \emph{head-subj-structure}, the head precedes the non-head if it has the value \avm{[inv & \normalfont{+}]} and follows it if it has the value \avm{[inv & \normalfont{$-$}]}. 

Modifiers can also precede or follow the modified element. I will illustrate this with adjectives: adjectives in French can be prenominal or postnominal, with some adjectives constrained to one or the other position. Example (\ref{ex:avm-mod-noun}) shows a postnominal adjective. I adopt the analysis of \citet{Abeille.1999.Weight,Abeille.1999.Adjectif} in which a feature \textsc{weight} with a value of type \emph{weight} (subtypes: \emph{lite} and \emph{nonlite}) accounts for the syntactic position of adjectives as follows: \emph{lite} adjectives are prenominal and \emph{nonlite} adjectives are postnominal. Modification and complementation of the adjective may cause a \emph{lite} adjective to become \emph{nonlite}. Coordination of two or more \emph{lite} adjectives may also turn the coordination into a \emph{nonlite} modifier. I refer the reader for more details to \citet{MyP.2015}, where we proposed an account of the adjective placement in French and Spanish based on semantic factors. The analysis of adverbs follows roughly the same principles. 
Relative clauses are always \avm{[weight & nonlite]}, i.e.\ postnominal  \citep[343]{Abeille.1999.Weight}. I will come back to the analysis of relative clauses in Section~\ref{ch:hpsg-extraction}.

The different linearization rules are summarized in (\ref{rule:linearization}).

\begin{exe}\ex Linearizations rules:\label{rule:linearization}
\begin{xlist}
\ex \emph{head-spr-structure} $\Rightarrow$ \textsc{nhead-dtr} $<$ \textsc{head-dtr}
\ex \emph{head-comps-structure} $\Rightarrow$ \textsc{head-dtr} $<$ \textsc{nhead-dtr}
\ex \emph{head-subj-structure} $\Rightarrow$ \textsc{head-dtr} \avm{[inv & \normalfont{+}]} $<$ \textsc{nhead-dtrs}
\ex \emph{head-subj-structure} $\Rightarrow$ \textsc{nhead-dtr} $<$ \textsc{head-dtr} \avm{[inv & \normalfont{$-$}]}
\ex \emph{head-mod-structure} $\Rightarrow$ \textsc{nhead-dtr} \avm{[weight & lite]} $<$ \textsc{head-dtr}
\ex \emph{head-mod-structure} $\Rightarrow$ \textsc{head-dtr} $<$ \textsc{nhead-dtr} \avm{[weight & nonlite]}
\end{xlist}
\end{exe}


\subsection{Semantics}

I adopt the Minimal Recursion Semantics (MRS) semantic representation as developed by \citet{Copestake.2005}. This representation is often used in HPSG implementations, e.g.\ the CoreGram Project \citep{Mueller.S.2015}. %cite LKB grammars, ERG for English
Semantics is represented in an AVM of type \emph{mrs}. \citet{Copestake.2005} define three features for the \emph{mrs} objects: \textsc{hook, relations (rels)} and \textsc{handle-constraints (hcons)}.\footnote{\citet{Copestake.2005} define the value of \textsc{rels} and \textsc{hcons} as a ``bag'' rather than a list, but they represent them as lists.%We are going to ignore the difference between the two concepts here.
} \citet{Song.2017} adds an additional feature to encode information about the discourse status of the different parts in an utterance: \textsc{icons}.

\begin{figure}[h]
\begin{floatrow}
\captionsetup{margin=.05\linewidth}
\ffigbox
{\avm{[\type*{mrs}\\
  hook & [\type* {hook}\\
           gtop & handle\\
           ltop & handle\\
           clause-key & event\\
           icons-key & info-str\\
           index & index] \\
  rels & list of relations\\ % list of EPs ?
  hcons & list of qeq constraints\\
  icons & list of icons]
}}
{\caption{Definition of \emph{mrs}}\label{avm:def-mrs}}

\ffigbox{%
\begin{forest}
[\textit{index}
    [\textit{individual}]
    [\textit{event}]
]
\end{forest}}
{\caption{Type hierarchy of \emph{index}}\label{fig:hrch-index}}
\end{floatrow}
\end{figure}

In this section, I describe how MRS works, leaving the \textsc{icons} features aside. In section~\ref{ch:hpsg-is} dedicated to information structure in HPSG, I will come back to \textsc{icons} and the way information structure is encoded in my fragment of French.

The reference marking feature \textsc{index}, embedded under \textsc{cont|hook} in MRS, can also be found in other semantic representations in HPSG \citep[see][24--26]{Pollard.1994}. The value of \textsc{index} is an AVM of type \emph{index} that has different subtypes in accordance with the type of referent concerned (individual or event). 

Each lexeme with a semantic content introduces an ``Elementary Predication'' (EP) in the discourse. In the MRS representation, this is reflected by the fact that every lexeme has (at least) one object of type \emph{relation} in its \textsc{rels} list. Conventionally, the nomenclature for the different types of relations is \emph{lexeme\_rel}. For example, the lexical entry for \emph{collègue} (`colleague') contains in its \textsc{rels} list an object of type \emph{collegue\_rel}. The handle is the label of the EP. It is encoded under a feature \textsc{lbl} with a value of type \emph{handle}  (conventionally labeled \emph{h1}, \emph{h2}, \emph{h3}, etc.). The first argument of an EP is a variable (conventionally labeled \emph{i}, \emph{j}, \emph{k}, etc.\ for individuals and \emph{e1}, \emph{e2}, \emph{e3}, etc.\ for events), and is encoded under a feature \textsc{arg0} with a value of type \emph{index}. On the level of the lexeme, the value of \textsc{arg0} is coindexed with the value of \textsc{index}. The other arguments, if any, are handles, which are encoded under features \textsc{arg1}, \textsc{arg2}, etc.\ with values of type \emph{index}.\footnote{The order of the arguments follows the obliqueness of the arguments \citep{Keenan.1977}: less oblique $<$ more oblique \citep[287]{Copestake.2005}.} \citet{Copestake.2005} also define other possible features (e.g.\ \textsc{restr}, \textsc{body}) for scopal relations, which we do not need in this fragment. 

The (simplified) semantic representation for \emph{collègue} (`colleague') is given in \figref{avm:collegue-semantics}. Being a noun, \emph{collègue} has an \textsc{index} of type \textit{individual}. Because \emph{collègue} is a relational noun, \mbox{\emph{collegue\_rel}} has a feature \textsc{arg1}. In \textsc{arg-st}, the value of \textsc{arg1} is identified with the index of the complement.

\begin{figure}[h]
\avm{[phon & < \type{collègue} >\\
cont &
[\type*{mrs}\\
  hook & [ltop & \tag{h1}\\
          index & \tag{i} individual] \\ % index plus détaillé
  rels & < [\type*{collegue\_rel}\\
           lbl & \tag{h1}\\
           arg0 & \tag{i}\\
           arg1 & \tag{j}] >]\\
arg-st & < [index & \tag{j}] >  ]
}
\caption{Lexical entry for \emph{collègue} (`colleague') -- semantics}\label{avm:collegue-semantics}
\end{figure}


The value of the feature \textsc{handle-constraints (hcons)} is a list of AVMs of the type \emph{equality modulo quantifiers (qeq)}. They link the arguments with each other, especially for scope resolution. It is not absolutely necessary to take it into account for my analysis, so I will leave aside the feature \textsc{hcons} in this work.

The feature \textsc{hook} has a value of type \emph{hook} that contains five features: \textsc{gtop}, \textsc{ltop}, \textsc{clause-key}, \textsc{icons-key} and \textsc{index}. \textsc{clause-key} is related to information structure and I will describe it in the next section. \textsc{gtop} states the global top handle: this is the EP in the sentence whose \textsc{arg0} value is not bound by any other EP. %The \textsc{gtop} -- or as the case may be \textsc{ltop} -- is linked with the label of the highest EP in an AVM of the type \emph{qeq}. 
\textsc{ltop} states the local top handle, i.e.\ the EP of the head in headed structures. On the level of the lexeme, the value of \textsc{ltop} is identified with the value of the \textsc{lbl}, see \figref{avm:collegue-semantics}.

In headed structures, the value of \textsc{gtop} is the same for the mother and its daughters. The mother inherits the value of the whole \textsc{hook} of the head daughter. 

In structures with simple semantic composition, the value of the \textsc{rels} list of the mother is a concatenation of the lists of the daughters. In some structures however, the semantic contribution of the structure is more than the sum of the contributions of the daughters. Therefore, phrases have an additional \emph{loc} feature \textsc{c-cont} in which the contribution of the structure can be encoded. Like \textsc{cont}, \textsc{c-cont} takes as value an \emph{mrs} object. The value of the \textsc{cont|rels} (and \textsc{hcons}) list of the mother is an amalgamation of the lists of both daughters and of its own \textsc{c-cont|rels} (and \textsc{hcons}) list in structures with simple semantic composition.
Headed structures are thus defined as in \figref{fig:new:semantic-composition}.

\vfill
\begin{figure}[H]
\parbox[c]{\widthof{structure}}{\emph{headed-\\structure}} 
\avm{$\Rightarrow$} % or head-non-filler-str ?
\avm{[synsem|loc & [cont & [hook & \1 [gtop & \2] \\
                       rels & \3 $\oplus$ \4 $\oplus$ \5 \\
                       hcons & \6 $\oplus$ \7 $\oplus$ \8]\\
                    c-cont & [rels & \5 \\
                              hcons & \8]] \\
  head-dtr & [synsem|loc|cont & [hook & \1\\
                                   rels & \3 \\
                                   hcons & \6 ] ]\\
  nhead-dtrs & <[synsem|loc|cont & [hook & [gtop & \2 ]\\
                                      rels & \4 \\
                                      hcons & \7 ] ]> ]}
\caption{Semantic composition}\label{fig:new:semantic-composition}
\end{figure}
\vfill
\pagebreak

A sentence is well-formed if all \emph{index} variables are bound, except for one. For the sake of simplicity, I assume that the \emph{handle} corresponding to this unbound variable is then unified with the value of \textsc{gtop}. At the sentence level, the value of \textsc{gtop} must be equal to the value of \textsc{ltop}.

\textsc{clause-key} is a feature introduced by \citet{Song.2017}. Its value is structure-shared with the \textsc{index} value of the semantic head of the clause (usually the main verb). Only finite clauses are considered ``clauses'' for \textsc{clause-key}. For example, in (\ref{ex:many-clauses}), the value of \textsc{clause-key} is \emph{e2} for \emph{Sherry wants Minnie to bring her dog} and all its subtrees, because \emph{bring} is non-finite.

\ea[]{[Maria wonders$_{e1}$ [why Sherry wants$_{e2}$ Minnie to bring$_{e3}$ her dog], [whereas Erica is$_{e4}$ allergic to them]].}
\label{ex:many-clauses}
\z 

The lexical entry for a verb constrains the NPs and non-finite VPs in its \textsc{arg-st} list to structure-share the value of their \textsc{clause-key} feature with the value of its own \textsc{clause-key}. This ensures that all elements in a clause share the same \textsc{clause-key} value~-- except the finite sentential arguments, as I will explain in Section~\ref{ch:hpsg-basics-ldd}.\footnote{The mechanism is somewhat more sophisticated in \citet{Song.2017}, but this should not affect the analysis.} 
On the clausal level, a constraint ensures that the \textsc{clause-key} value  of the clause  is structure-shared with the \textsc{index} value of the semantic head of the clause. Below I will describe the different clause types and the way the constraint is formalized.

\figref{fig:avm--basis-sentence-semantics} gives a concrete example of how MRS's representation of semantics works for a whole sentence. I do not elaborate on the semantics of quantifiers, developed at length by \citet{Copestake.2005}, since they are not central to the topic of this book. Suffice it to say that I assume the distinction between definite articles (with a relation \emph{def\_rel}), indefinite articles (with a relation \emph{indef\_rel}) and possessives (with a relation \emph{poss\_rel}).

\begin{sidewaysfigure}
\oneline{%
\begin{forest}
where n children=0{tier=word}{}
[S \\
\avm{[hook & \1\\
      rels & < \2, \3, \4, \5, \6, \7, \8 >]}
    [NP\\
    \avm{[hook & \9 \\
          rels & < \2, \3, \4, \5 > ]}
        [Det \\ \avm{[rels & < \2 \type{def\_rel} >]}
        [l'\\the]]
        [N$'$
        \avm{[hook & \9\\
              rels & < \3, \4, \5 >]}
        [N\\
        \avm{[hook & \9 [gtop & \tag{h1}\\
                         ltop & \tag{h2}\\
                         clause-key & \tag{e}\\
                         index & \tag{i}]\\
              rels & < \3 [\type*{originalite\_rel}\\
                           lbl & \tag{h2}\\
                           arg0 & \tag{i}\\
                           arg1 & \tag{j}] >]}
        [originalité\\uniqueness]]
        [PP\\
        \avm{[hook & [gtop & \tag{h1}\\
                      ltop & \tag{h3}\\
                      clause-key & \tag{e}\\
                      index & \tag{j}]\\
      rels & < \4 \type{def\_rel} , \5 [\type*{innovation\_rel}\\
                           lbl & \tag{h3}\\
                           arg0 & \tag{j}] >]}
        [de l'innovation\\ of the innovation, roof]]
    ]]
    [VP\\
    \avm{[hook & \1\\
      rels & < \6 , \7 , \8 >]}
        [V\\
        \avm{[hook & \1 [gtop & \tag{h1}\\
                      ltop & \tag{h1}\\
                      clause-key & \tag{e}\\
                      index & \tag{e}]\\
      rels & < \6 [\type*{enthousiasmer\_rel}\\
                           lbl & \tag{h1}\\
                           arg0 & \tag{e}\\
                           arg1 & \tag{i}\\
                           arg2 & \tag{k}] >]}
            [enthousiasme\\excites]
        ]
        [NP\\
        \avm{[hook & [gtop & \tag{h1}\\
                      ltop & \tag{h4}\\
                      clause-key & \tag{e}\\
                      index & \tag{k}]\\
      rels & < \7 \type{poss\_rel}, 
            \8 [\type*{originalite\_rel}\\
                           lbl & \tag{h4}\\
                           arg0 & \tag{k}\\
                           arg1 & index] >]}
            [mes collègues\\my colleagues, roof]
        ]
    ]
]
\end{forest}}
\caption{Semantic representation for ``L'originalité de l'innovation enthousiasme mes collégues.'' (`The uniqueness of the innovation excites my colleagues.')}
\label{fig:avm--basis-sentence-semantics}
\end{sidewaysfigure}

\subsection{Information structure}
\label{ch:hpsg-is}

The issue of an adequate representation of information structure in HPSG has not been settled yet, and before I present the representation I adopted in this work, I briefly discuss the other proposals in the literature. \citet{Song.2017} seems best suited for my aims, and I chose his proposal despite some minor problems and open questions that I leave for future work to resolve. 

\subsubsection{Different representations of information structure in HPSG}

As yet, there is no broad consensus on the way information structure should be represented and implemented in HPSG. The different proposals make different choices relative to: (i) how many and what discourse statuses they assume, (ii) on which level they encode information structure (e.g.\ main feature of \emph{sign}, \textsc{content} feature), (iii) what type of object is taken as value by the feature expressing discourse status (e.g.\ \emph{sign}, \emph{mrs}) and (iv) whether or not they allow for embedded clauses to have their own internal information structure. 
See \citet[113--122]{Bildhauer.2008} and \citet{DeKuthy.2020} for an overview of the HPSG literature on information structure.

The first solid attempt to develop an information structure representation was made by \citet{Engdahl.1996}, based on \citegen{Vallduvi.1992} theory of information structure. Instead of the double binary distinction Topic/Comment and Focus/Background that was presented in Section~\ref{ch:is}, \citegen{Vallduvi.1992} analysis relies on a binary distinction Focus\slash Background, in which the latter entails a binary distinction Link\slash Tail. The concept of Link is roughly equivalent to what I defined as Topic, and the concept of Tail applies to the elements in the utterance that are neither Link (Topic) nor Focus. \citet{Engdahl.1996} represent information structure under \textsc{context}. They propose a \textsc{context} feature \textsc{info-struc} that has as value an AVM
with three features: \textsc{link}, \textsc{focus} and \textsc{tail} \citep[11]{Engdahl.1996}. They all take as value an object of type \emph{sign} even though \citeauthor{Engdahl.1999} assumes a value of type \emph{content} in her later accounts \citep{Engdahl.1999}.  
In addition, \citeauthor{Engdahl.1996} account for the interface between prosody and information structure, but I leave this aspect of their analysis aside.

\begin{figure}[h] 
\avm{[\type*{sign}\\
synsem|loc|context & [c-indices & cindices\\
                 background & set\\
                 info-struct & [focus & sign\\
                                ground & [link & sign\\
                                           tail & sign]]]]}
\caption{Encoding of information structure in \citet[11]{Engdahl.1996}, adapted to a modern representation}
\end{figure}

\citet{DeKuthy.2002} makes a different proposal: she encodes information structure under a main feature of \emph{sign} \textsc{info-struc}, which takes as value a feature structure of type \emph{info-struc} with only two features: \textsc{focus} and \textsc{topic} \citep[161--165]{DeKuthy.2002}.
Each of these features has as value a list of \emph{meaningful expression} objects (a semantic representation proposed by \citet{Richter.2000} that is closer to Montague Semantics \citep{Dowty.1981} than MRS). There is therefore no encoding of backgroundedness, even though it would be relatively easy to introduce it in her model if needed. Having a list as value enables her analysis to distinguish utterances with multiple foci or topics from utterances with one focus or topic.
\citeauthor{DeKuthy.2002} accounts for the interface between prosody and information structure along the same lines as \citet{Engdahl.1996}. 

\begin{figure}[h]
\avm{[\type*{sign}\\
phon & list\\
synsem & synsem\\
info-struc & [\type*{info-struc}\\
              focus & list of meaningful expressions\\
              topic & list of meaningful expressions]]}
\caption{Encoding of information structure in \citet[161--165]{DeKuthy.2002} (summary)}
\end{figure}

% Wilcock 2005? list of relations 

\citegen{Bildhauer.2008} proposal is relatively similar to \citet{DeKuthy.2002}. Information structure is encoded under the feature \textsc{is}, a main feature of \emph{sign} that takes as value an AVM object with  the two features \textsc{focus} and \textsc{topic}. The value of these two features is a list of lists of EPs (i.e.\ of objects of the type \emph{relation} as defined in MRS, see above). Here again, backgroundedness is not explicitly encoded, but it remains relatively easy to identify.\footnote{As pointed out by \citeauthor{Bildhauer.2008}, ``it will correspond to those EPs that are present on the \textsc{rels} list, but absent from the \textsc{foc} and \textsc{topic} list'' \citep[147]{Bildhauer.2008}. My definition of background would rather correspond to the EPs that are present on the \textsc{rels} list, but absent from the \textsc{foc} list.} Having as value a list of lists enables the analysis to identify multiple foci or topics, similarly to \citeauthor{DeKuthy.2002}.  \citeauthor{Bildhauer.2008} also develops a very elaborate representation of accents and tonality that can be mapped onto the information structure. 

\begin{figure}[h]
\captionsetup{margin=.05\linewidth}
\begin{floatrow}
\ffigbox{%
\avm{[\type*{sign}\\
phon & list\\
synsem & synsem\\
is & [foc & list\\
      topic & list]]}}
{\caption{Encoding of information structure in \citet[147]{Bildhauer.2008}}}
\ffigbox
{\avm{[\type*{sign}\\
synsem & [loc & local\\
nonloc & nonloc\\
is & [\type*{is}\\
       topic & list\\
       focus & list]]]}}
{\caption{Encoding of information structure in \citet[74]{Bildhauer.2010}}}
\end{floatrow}
\end{figure}

\citet{Bildhauer.2010} and \citet*{Mueller.S.2020?.chapter5} use a sightly different variant of \citegen{Bildhauer.2008} proposal, where the feature \textsc{is} is a feature in \emph{synsem}.

\citegen{Webelhuth.2007} proposal differs in many respects from other proposals in the literature. He represents information structure under \textsc{content}. The value of \textsc{content} is an AVM of type \emph{content} with two features, \textsc{background (bg)} and \textsc{focus (foc)}. The value of \textsc{focus} is a list of \emph{meaningful expression} objects \citep{Richter.2000}. The value of \textsc{bg} is an AVM with two features: \textsc{focus variables (fvars)} which takes a list of variables (based on \citet{Krifka.1992} in order to account for multiple foci or topics); and \textsc{core}, which takes an object of the type \emph{meaningful expression} and encodes the meaning of the sign.
Furthermore, \citet[310]{Webelhuth.2007} makes a distinction between Focus\slash Topic\slash Background, which are mapped to prosody; and Theme, which is mapped to syntax. His concept of Theme is similar to what I defined as ``aboutness topic'', with an important aspect of gradience (one constituent being more or less thematic than another one). Thematicity is encoded in a meaningful expression.

\begin{figure}[h] 
\avm{[\type*{sign}\\
synsem|loc|cont & [\type*{cont}\\
                   bg & [\type*{bg}\\
                         fvars & list of variables\\
                         core & meaningful expression]\\
                   foc & list of meaningful expressions]]}
\caption{Encoding of information structure in \citet[312]{Webelhuth.2007}}
\end{figure}

\citet{Song.2017} encodes information structure inside \emph{mrs} objects, thus under \textsc{content}, with a feature called \textsc{icons}. The mechanism of \textsc{icons} mimics the mechanism of \textsc{rels}. The value is a list of objects of type \emph{info-struc}, and each word that introduces an EP also introduces such an object. The type hierarchy of \emph{info-struc} is fine-grained and contains subtypes like \emph{focus}, \emph{topic} and \emph{bg} (background). \emph{info-struc} objects have two features, \textsc{target} and \textsc{clause}, that make it possible to map the semantic variable bearing the discourse function to the clause with respect to which it has this discourse function. Embedded clauses can then have their own internal information structure independently of the main clause (e.g.\ a relative clause may be backgrounded with respect to the meaning of the main clause, but have its own topic or focus domain). I present \citegen{Song.2017} proposal in more detail below.

\begin{figure}[h] 
\avm{[\type*{sign}\\
synsem|loc|cont & [\type*{mrs}\\
                   hook & hook\\
                   rels & list of relations\\
                   hcons & list of qeq constraints\\
                   icons & list([\type*{info-str}\\
                                clause & individual\\
                                target & individual]) $\bigcirc$ list of icons]]}
\caption{Encoding of information structure in \citet[116]{Song.2017} (simplified)}
\end{figure}

\subsubsection{Desiderata for a representation of information structure}

As shown by the variety of proposals, there is disagreement on many points, but especially on where the features for information structure should be embedded. Some define them at the level of the \emph{sign} \citep{DeKuthy.2002,Bildhauer.2008}, however, \citet*[145]{Mueller.S.2020?.chapter5} point out that discourse status must be accessible in \emph{synsem} objects because some elements, like focus particles (e.g.\ \emph{only}),  are sensitive to information structure, and must be able to select via the valence features an element with the appropriate information structure. \citet[160--161]{DeKuthy.2002} argues that \textsc{info-struc} should not be part of a \emph{local} object, otherwise this has undesirable consequences for focus projection in her analysis. Focus projection is a property of the interface between prosody and information structure: the word(s) bearing the main stress in the sentence may project this focus status to domains wider than where the stress falls. Focus projection is therefore the reason why the German sentence in (\ref{ex:focus-proj-insitu}) can have an all-focus reading (e.g.\ as an answer to the question \emph{What happened?}), while the main stress is on \emph{Auto}.

\ea all-focus reading \citep[160]{DeKuthy.2002}\\
\gll Hans hat ein AUTO gewonnen.\\
Hans has a car won\\
\glt `Hans won a car.'
\label{ex:focus-proj-insitu}
\z 

If the word that receives the main stress appears in the prefield (before the finite verb), focus projection is blocked: (\ref{ex:focus-proj-VF}) cannot have an all-focus reading. 

\ea all-focus reading \citep[160]{DeKuthy.2002}\\
\# \gll Ein AUTO hat Hans \emph{t} gewonnen.\\
a car has Hans {} won\\
\glt `Hans won a car.'
\label{ex:focus-proj-VF}
\z 

But \citeauthor{DeKuthy.2002} analyses the prefield position as a filler-gap dependency with a trace (\emph{t}) at the canonical position in the middlefield, and traces as empty categories that structure-share their \emph{loc} value with their filler. This is the reason why \citeauthor{DeKuthy.2002} argues against encoding information structure in \emph{loc} objects: otherwise, the information structure of (\ref{ex:focus-proj-insitu}) is indistinguishable from  the information structure of (\ref{ex:focus-proj-VF}), and the difference in available readings between the two sentences remains unexplained.

For the formalization of the FBC constraint, however, it is crucial that the focus interpretation received by the extracted element be structure-shared with the noun that subcategorizes for this element. Hence, contra \citet{DeKuthy.2002}, I argue that information structure has to be encoded in \emph{loc} objects. I avoid the problem pointed out by \citet{DeKuthy.2002} because I adopt a traceless account of extraction along the lines of \citet{Bouma.2001}.\footnote{\citet{DeKuthy.2002} refers to it as a possible way to avoid focus projection in (\ref{ex:focus-proj-VF}), but says that it is otherwise incompatible with her own analysis of German.} Traceless analyses have been supported by \citet{Sag.1994.Fodor,Sag.1996} on grounds, among others, that they are more compatible with an incremental processing model \citep{Pickering.1991,Tanenhaus.2000}.%
\footnote{On the other hand, traceless analyses may also have some drawbacks: as explained by \citet[570--574]{Mueller.S.2016}, more rules have to be added to the grammar, and especially the German V2 position seems to be hard to analyze without the use of a trace.} % Yiddish ?

% Kuhn 1995 : the value of FOCUS should not be a syntactic object
% Kuhn, Jonas. 1995a. Information Pa kaging in German - some motivation from HPSG-based translation. Theoreti al Linguisti s 2 (Syntax), Term Paper.

Different accounts use different terminologies with respect to information structure, and this is especially true for ``background'', which may or may not include the topic, depending on the model. In \citegen{Engdahl.1996} analysis, \textsc{link} (Topic) is included in \textsc{(back)ground}, whereas for \citet{DeKuthy.2002}, \citet{Bildhauer.2008} and \citet{Song.2017}, Background, Topic and Focus are in complementary distribution, i.e.\ Background is what is neither Topic nor Focus. The FBC constraint in (\ref{rule:FBC-bis}) presupposes direct access to a list of the focused elements on the one hand, and to the ``backgrounded'' (i.e.\ non-focus) ones on the other hand. This is possible in all accounts that I have mentioned, though more straightforwardly so in \citegen{Engdahl.1996} model and in \citegen{Song.2017} model (which defines a supertype \emph{non-focus} for \emph{background} and \emph{topic}). 

% no analysis seems to account for partial focus

However, \citegen{Song.2017} proposal is the only one that offers the possibility to model different layers of information structure for each clause in an utterance, thanks to the pair of features \textsc{clause/target}. This means that any element in an embedded clause can be presupposed with respect to the main clause (e.g.\ if the whole clause is backgrounded, as in restrictive relative clauses) and still be the topic with respect to the embedded clause (e.g.\ the extracted element in a relative clause\footnote{``On the basis of the pervasive parallelism between topicalization and relativization, I proposed that in Japanese what is relativized is the theme of the relative clause.'' \citep[15]{Kuno.1987}}). This advantage is fundamental for my purposes, and the main reason for choosing his model.

For consistency, I follow \citegen{Song.2017} representation whenever possible, even though it has some weaknesses. For example, this model does not distinguish between a clause with multiple foci or topics and one with a single focus or topic (an aspect addressed by \citet{DeKuthy.2002}, \citet{Webelhuth.2007}, \citet{Bildhauer.2008}, and \citet*{Mueller.S.2020?.chapter5}). It is beyond the scope of this work to define a new and more optimal representation of information structure in HPSG that incorporates all relevant aspects. In general, I am also confident that my analysis is compatible with any well thought-out representation of information structure. 

\subsubsection{The representation of information structure adopted in this work}

In this work, information structure is encoded under the \emph{mrs} feature \textsc{individual constraints (icons)}. The definition of \emph{mrs} presented in \figref{avm:def-mrs} is reproduced in \figref{avm:def-mrs-bis}.

\begin{figure}[h]
\avm{[\type*{mrs}\\
  hook & [\type* {hook}\\
           gtop & handle\\
           ltop & handle\\
           clause-key & event\\
           icons-key & info-str-or-none\\
           index & index] \\
  rels & list of relations\\ % list of EPs ?
  hcons & list of qeq constraints\\
  icons & list of icons]
}
\caption{Definition of \emph{mrs}}
\label{avm:def-mrs-bis}
\end{figure} 

The feature \textsc{icons} ``incorporate[s] discourse-related phenomena into semantic representations of sentences'' \citep[31]{Song.2016}.\footnote{\citet{Song.2012} credit Dan Flickinger and Ann Copestake with first suggesting the \textsc{icons} feature.} The feature \textsc{cont} thus encodes not only the semantics but also the pragmatics of the sign. This is the reason why \citet{Song.2016,Song.2017} does not define any \textsc{context} feature for \emph{local} objects. 

The value of \textsc{icons} is a list of \emph{icons} objects. All \textsc{icons} objects express a binary relation between \emph{index} variables. For example, they express anaphora resolution either as an identity relation (``eq'') or a non-identity relation (``neq'') between two variables.

\begin{exe}
\ex \citep[31]{Song.2016}
\begin{xlist}
\ex John$_i$ likes himself$_j$. [\textit{i} eq \textit{j}]
\ex John$_i$ likes him$_j$. [\textit{i} neq \textit{j}] 
\end{xlist}
\end{exe}

Honorific relations are also encoded by means of \emph{icons} object, namely \emph{rank} objects, as a binary relation between the addressor and the addressee. Information structure is encoded in \emph{info-str} objects, in accordance with the hierarchy of \emph{icons} objects in \figref{fig:hrch-icons}, adapted from \textcites[36]{Song.2016}[114]{Song.2017}.

\begin{figure}[ht]
\centering
\oneline{
\begin{forest}
sn edges,
[\textit{icons}
[\textit{non-is}
    %[\textit{dialogue}
    %    [\textit{addressor}]
    %    [\textit{addressee}]
    %]
    [\dots]
    [\textit{rank} [\dots]]
]
[\textit{is-status}
[\textit{info-str}
    [\textit{non-topic}, name = nontopic
        [\textit{focus}, name = focus
            [\textit{semantic-focus}]
        ]
    ]
    [\textit{contrast-or-focus}, name = contrastorfocus]
    [\textit{focus-or-topic}, name = focusortopic
        [\textit{contrast}, name = contrast
            [\textit{contrast-focus}, name = contrastfocus]
            [\textit{bg},no edge, name = bg]
            [\textit{contrast-topic}, name = contrasttopic]
        ]
    ]
    [\textit{contrast-or-topic}, name = contrastortopic]
    [\textit{non-focus}, name = nonfocus
        [\textit{topic}, name = topic
            [\textit{aboutness-topic}]
        ]
    ]
]
[\textit{i-empty}]
]]
\draw[thin] (nontopic.south)--(bg.north);
\draw[thin] (contrastorfocus.south)--(focus.north);
\draw[thin] (contrastorfocus.south)--(contrast.north);
\draw[thin] (focusortopic.south)--(focus.north);
\draw[thin] (focusortopic.south)--(topic.north);
\draw[thin] (contrastortopic.south)--(contrast.north);
\draw[thin] (contrastortopic.south)--(topic.north);
\draw[thin] (nonfocus.south)--(bg.north);
\draw[thin] (focus.south)--(contrastfocus.north);
\draw[thin] (topic.south)--(contrasttopic.north);
\end{forest}
}
\caption{Type hierarchy of \textit{icons}}
\label{fig:hrch-icons}
\end{figure}

Objects of the type \emph{info-str} express a binary relation between a clause (an \emph{event} variable) and some element in the sentence (through its \emph{index} variable). Accordingly, they have two features, \textsc{clause} and \textsc{target}. Every word that introduces an EP in the semantic representation also introduces an \emph{info-str} object.\footnote{The corollary of this rule is that semantically empty words like expletives or copulas do not introduce any \emph{info-str} object. \citet[112]{Song.2017} also assumes that syncategorematic items, e.g.\ relative pronouns, are ``informatively empty'' and do not introduce any \emph{info-str} objects. He does not explain in detail how such words are analyzed in his model, and especially what the value of \textsc{icons-key} is. 
I add the possibility of a value \emph{i-empty} to his hierarchy of \emph{icons}, and assume that semantically and informatively empty words are \avm{[icons-key & i-empty]}. This is not very elegant because all \emph{icons} objects are supposed to introduce a binary relation. Furthermore, an unfortunate consequence is that the \textsc{icons} list of a sign may contain  \emph{i-empty} elements. This could probably be avoided, but I leave the task of resolving this challenge for future work.
%I assume that semantically and informatively empty words do not have a \textsc{icons-key} feature. This means that the definition in \figref{avm:def-mrs-bis} is not universal and that some \emph{mrs} objects are missing (at least) this feature. -> this does not work because icons-key is supposed to be inherited by the mother node
} 
The value of \textsc{target} is structure-shared with the value of the \textsc{index} of the word. The value of \textsc{clause} is structure-shared with the value of \textsc{clause-key}, which itself is the current clause's event variable.

\ea Definition of \emph{info-str}: \nopagebreak

\avm{[\type*{info-str}\\
  clause & event\\
  target & index]
}
\label{avm:def-infostr}
\z 

\figref{avm:collegue-info-str} illustrates how words in the lexicon introduce an underspecified object of type \emph{info-str}. \figref{avm:collegue-focus} is the AVM for \emph{collègues} (`colleagues') in (\ref{ex:basis-sentence-with-is}) where the direct object bears the informational focus of the utterance.

\ea
\gll [L' originalité de l' innovation]$_T$ enthousiasme [mes collègues]$_F$.\\
\sbar{}the uniqueness of the innovation excites \sbar{}my colleagues\\
\glt `The uniqueness of the innovation excites my colleagues.'
\label{ex:basis-sentence-with-is}
\z

\begin{figure}[h] 
\avm{[phon & < \type{collègue} >\\
cont|hook & [index & \tag{i} individual\\
             clause-key & \tag{e} event \\
             icons-key & [\type*{info-str}\\
                 clause & \tag{e}\\
                 target & \tag{i}]]]
}
\caption{Lexical entry for \emph{collègue} (`colleague') -- information structure}\label{avm:collegue-info-str}
\end{figure}

\begin{figure}[h]
\avm{[phon & < \type{collègues} >\\
cont|hook & [index & \tag{i} individual\\
             clause-key & \tag{e} event \\
             icons-key & [\type*{semantic-focus}\\
                 clause & \tag{e}\\
                 target & \tag{i}]]]
}
\caption{Lexical entry for \emph{collègue} (`colleague') in example~(\ref{ex:basis-sentence-with-is}) -- focussed}\label{avm:collegue-focus}
\end{figure}

The feature \textsc{icons-key} encodes the ``main'' information structure of the head. The other \emph{icons} objects (the information structure for the rest of the phrase, or other pragmatic relations) are part of the \textsc{icons} list. There can be only one object for each \textsc{target-clause} pair in the \textsc{icons} list, i.e.\ an element may have different discourse statuses for different clauses but not different discourse statuses for one single clause.

\ea Discourse-clash Avoidance Principle:\\
The \textsc{icons} list can contain only one \emph{info-str} element for each \textsc{target-clause} pair. 
\label{rule:discourse-clash-avoid}
\z 

As long as the word is the head of the structure, the values of \textsc{icons-key} and \textsc{clause-key} remain the same, as they are part of \textsc{hook}. In a headed structure, the \textsc{icons-key} of the non-head daughter is incorporated into the \textsc{icons} list of the mother node. The mother node also inherits the \textsc{icons} lists of its daughters and its own \textsc{c-cont}.\footnote{\citet{Song.2017} uses some mechanisms of the LKB grammars (e.g.\ difference lists) to formulate his constraint. I am using a more traditional approach and thus translate \citeauthor{Song.2017}'s idea into the constraint in (\ref{ex:avm-icons-accumulation}).}

\begin{figure} 
\textit{headed-structure} \avm{$\to$} 
\avm{[cont|icons & \1 $\oplus$ <\2> $\oplus$ \3 $\oplus$ \4\\
        c-cont & [icons & \4]\\
        head-dtr  & [cont|icons & \1]\\
        nhead-dtrs & <[cont & [hook|icons-key & \2\\
                             icons & \3]]>]}
\caption{\textsc{icons} Accumulation}
\label{ex:avm-icons-accumulation}
\end{figure}

\section{Extraction in HPSG}
\label{ch:hpsg-extraction}

The \emph{synsem} feature \textsc{nonloc} encodes features that are used for non-local dependencies. I take as a starting point the definition of \textsc{nonloc} adopted in \citet{Borsley.2020.HPSG.UDC}. They follow a lexicalist approach to extractions and long-distance dependencies~-- strongly inspired by \citet{Bouma.2001} and \citet{Ginzburg.2000}~-- in which (i) no empty categories are needed; and (ii) constructions involving non-local dependencies receive different analyses via different non-local features. \citet{Sag.2010} provides a detailed overview of the different extractions in English and shows that they are heterogeneous. With that in mind, I propose a way to account for the different extraction constructions discussed in this work: relative clauses (based on the analysis of \citet{Abeille.2007.Relatives}), interrogatives and \emph{c'est}-clefts (based on the analysis of \citet{Winckel.2020}). My goal is to show how syntax and information structure interact in these constructions.

\subsection{A traceless analysis}

% discussion about empty elements in Müller 2016, chapter 19.2
% Müller, Stefan (2016), Grammatical Theory: From Transformational Grammar to Constraint-Based Approaches (Textbooks in Language Sciences 1). Berlin: Language Science Press.

Objects of the type \emph{nonloc} have three features, \textsc{slash}, \textsc{rel} and \textsc{que}. They all take as value a set: a set of \emph{loc} objects for \textsc{slash}, a set of \emph{index} objects for \textsc{rel} and a set of \emph{relations} (EPs) objects for \textsc{que}.
A non-empty \textsc{slash} set denotes the presence of a ``missing'' element, i.e.\ an object of type \emph{gap}.
HPSG analyses generally assume a set value for the \textsc{slash} feature (instead of a list), because this allows two gaps to combine into a single gap \citep{Pollard.1990}. This property accounts for cases like (\ref{ex:slash-set-double-gap}), where two gaps are linked to the same filler.%
\footnote{\citet{Chaves.2020.UDC} propose that the value of \textsc{slash} (\textsc{gap} in their terminology) is a list. Instead of the \textsc{slash} Amalgamation Principle (\ref{avm:slash-amalgamation}) below, they assume a function of list joining \citep[248]{Chaves.2020.UDC}. Their proposal has the advantage of licensing not only cases like (\ref{ex:slash-set-double-gap}), but also cases in which two gaps combine into a single gap while keeping distinct indices, as in (i): while in (\ref{ex:slash-set-double-gap}) the document signed is the same as the document read, in (i) what is eaten can hardly be the same as what is drunk.
\begin{itemize}
    \item[(i)] What$_{i+j}$ do you think [Ed ate~\trace{}$_i$ and drank~\trace{}$_j$ at the party]? \citep[246]{Chaves.2020.UDC} 
\end{itemize}
For the sake of simplicity, I adopt the analysis of extractions in \citet{Borsley.2020.HPSG.UDC}, which is sophisticated enough to formalize the FBC constraint. But my analysis is also compatible with \citeauthor{Chaves.2020.UDC}'s proposal.}

\ea[]{[Which document]$_i$ did you sign~\trace{}$_i$ [without reading~\trace{}$_i$]?} 
\label{ex:slash-set-double-gap}
\z 

Cases like (\ref{ex:slash-set-double-gap}) are challenging under a movement-based approach, which is why these accounts treat one of the gaps as ``parasitic'' (see Section~\ref{ch:parasitic-gaps} for a discussion of parasitic gaps and arguments against them). 

\ea Definition of \emph{nonloc}:\nopagebreak

\avm{[\type*{nonloc}\\
slash & set of loc\\
rel & set of index \\
que & set of relations]}
\z 

In Section~\ref{ch:hpsg-syntax} I described the way arguments in the \textsc{arg-st} list are selected for canonical or non-canonical realization, based on the type hierarchy of \emph{synsem} in \figref{fig:hrch-syssem}. Dependents of the type \emph{gap} are defined as follows:\footnote{The original definition is in \citet[160]{Pollard.1994}.} 

\ea Definition of \emph{gap}:\nopagebreak

\avm{[\type*{gap}\\
loc & \1\\
nonloc|slash & \{\1\}]}
\z 

A word ``gathers'' the elements of its arguments' \textsc{slash} set into its own \textsc{slash} set. Because \emph{gap} objects have a non-empty \textsc{slash} set, their \textsc{loc} value is also stored in the \textsc{slash} set of the word that subcategorized for them. If the NP argument of a verb is realized canonically but has an element in its \textsc{slash} set, as is the case in subextractions from subjects or objects, this gap element recursively ends up in the \textsc{slash} set of the verb. 

\ea \textsc{slash} Amalgamation Principle adapted from \citet[169]{Ginzburg.2000}\nopagebreak

\textit{word} \avm{$\to$ 
 [slash & / \1 $\cup$ \dots $\cup$ \tag{n}\\
arg-st & <[slash & / \1] \dots [slash & / \tag{n}]>]}
\label{avm:slash-amalgamation}
\z 

This \textsc{slash} Amalgamation Principle is a default constraint. Some kinds of words, like the copula in \emph{c'est}-clefts (see below) or adjectives that allow so-called ``\emph{tough} constructions'' \citep{Pollard.1994,Ginzburg.2000}, select a complement with a non-empty \textsc{slash} set and do not contain the slashed element in their own \textsc{slash} list. 
%We can avoid the use of a default rule if we define two subtypes for \emph{word}, one with a canonical \textsc{slash} amalgamation like (\ref{avm:slash-amalgamation}), one (or more) with specific ways to build their \textsc{slash} set. 

The same mechanism applies to the other \textsc{nonloc} features \textsc{rel} and \textsc{que}. The \textsc{slash} Amalgamation Principle (\ref{avm:slash-amalgamation}) can therefore have as a corollary a (default) \textsc{que} Amalgamation Principle and a (default) \textsc{rel} Amalgamation Principle. Alternatively, we may adopt the general \textsc{nonloc} Amalgamation Principle in (\ref{avm:nonloc-amalgamation}).

\ea \textsc{nonloc} Amalgamation Principle\nopagebreak

For every \textsc{nonloc} feature \textsc{f}: \nopagebreak

\textit{word} \avm{$\to$ 
 [f & / \1 $\cup$ \dots $\cup$ \tag{n}\\
arg-st & <[f & / \1], \dots, [f & / \tag{n}]>]}
\label{avm:nonloc-amalgamation}
\z 

The Argument Realization Principle (\ref{avm:arp}) and the \textsc{nonloc} Amalgamation Principle (\ref{avm:nonloc-amalgamation}) are sufficient to account for non-local dependencies in this work, which only deals with the extraction of arguments in headed structures. Specific requirements would be necessary to account for extractions of adjuncts, out of adjuncts and out of non-headed structures.\footnote{For adjuncts, some HPSG analyses posit a supplementary valence feature in \textsc{cat} which appends the elements of \textsc{arg-st} and the modifiers (e.g.\ the feature \textsc{deps} in \citealt{Bouma.2001}). Others assume that extracted modifiers are also part of the \textsc{arg-st} list (similar to the proposal in \citealt[17]{Abeille.1997}), and extraction of or out of adjuncts is treated using the same rules as in extraction of and out of arguments. Yet others such as \citet{Przepiorkowski.2016} cast doubt on the relevance of the argument/adjunct distinction and see ``argumenthood'' as a continuum.} I leave this question open.

Let us go through the argument realization in a prototypical sentence in a bottom-up fashion: The \textsc{comps} list of the verb is first worked off as the verb combines with its complement(s) through \emph{head-comps-structure}; then the \textsc{subj} list is worked off as the verb combines with its subject through   \emph{head-subj-structure}; and finally, the \textsc{slash} set of the verb is worked off as the verb combines with the filler(s) through \emph{head-filler-structure}. \figref{fig:hrch-headed-structure-with-filler} shows the type hierarchy in \figref{fig:hrch-headed-structure} completed with \emph{head-filler-structure}. %AVMs of the type \emph{head-filler-structure} take as their head daughter an AVM with an empty \textsc{comps} list and an empty \textsc{spr} list. 

\begin{figure}[ht]
\begin{forest}
[\textit{headed-structure}
    [head-filler-\\structure, font=\itshape, align=center, name = filler]
    [head-comps-\\structure, font=\itshape, align=center, name = comps]
    [head-subj-\\structure, font=\itshape, align=center, name = subj]
    [head-spr-\\structure, font=\itshape, align=center, name = spr]
    [head-mod-\\structure, font=\itshape, align=center, name = mod]
]
\end{forest}
    \caption{Type hierarchy of \emph{headed-structure}}
    \label{fig:hrch-headed-structure-with-filler}
\end{figure}

The formal definition of linguistic objects of the type \emph{head-filler-structure} is given in \figref{avm:head-filler-structure}. They have a non-head daughter with a \textsc{loc} value that is structure-shared with one element of the head daughter's \textsc{slash} set. This same element is missing from the mother node's \textsc{slash} set. The type \emph{head-filler-structure} is used to account for interrogatives and relative clauses with a relative pronoun, but \emph{c'est}-clefts and relative clauses with a complementizer are analyzed differently, as I will show below.

\begin{figure}[ht]
\textit{head-filler-structure} \avm{$\to$} 
\avm{[slash & \1\\
head-dtr & [\type*{phrase}\\
              slash & \{\2\} $\cup$ \1]\\
nhead-dtrs & <[local & \2]>]}
\caption{Definition of \emph{head-filler-structure}}
\label{avm:head-filler-structure}
\end{figure}

The GHFP (\ref{avm:ghfp}) guarantees that the mother inherits the \textsc{slash}-set from the head daughter in the other headed structures. Consequently, all heads along an extraction path have the property \avm{[slash & non-empty set]}. Extraction path effects are indeed attested cross-linguistically: there are phonological or morphosyntactic alternations depending on this factor, one form being used outside the extraction path, another one along the extraction path (\citealt{Zaenen.1983,Hukari.1995}; \citealt[Section 3.2]{Bouma.2001}).
% French qui/que alternation

\figref{fig:avm-extraction-simple} is the representation of the \emph{wh}-question (\ref{ex:extraction-simple}), in which the direct object undergoes simple extraction.

\ea[]{\gll Qui l' innovation enthousiasme-t-elle?\\
who the innovation[\textsc{f}] excites-0-\textsc{3sg.f.sbj}\\
\glt `Who does the innovation excite?'} 
\label{ex:extraction-simple}
\z 

\begin{figure}[ht]
\centering
\begin{forest}
where n children=0{tier=word}{}
[S\\
\avm{[\type*{head-filler-structure}\\
slash & <>]}
    [NP\\
    \avm{[\type*{word}\\
          loc & \1]}
          [qui]
    ]
    [S\\
    \avm{[\type*{head-subj-structure}\\
          slash & \2]}
          [NP\\
          \avm{[synsem & \3]}
            [l'innovation, roof]
          ]
          [V\\
          \avm{[\type*{word}\\
          slash & \2 \{\1\}\\
          arg-st & <\3 [\type*{canonical}\\
                        slash & <>], 
                        [\type*{gap}\\
                         loc & \1\\
                         slash & \{\1\}]>]}
            [enthousiasme-t-elle]
          ]
    ]
]
\end{forest}
\caption{Simple extraction}
\label{fig:avm-extraction-simple}
\end{figure}

\citet[7]{Ginzburg.2000} define a variety of different constructions to account for interrogatives and relative clauses in English. For example, \emph{wh-int-cl} (\emph{wh}-questions with extraction), \emph{polar-int-cl} (polar questions) and \emph{in-situ-int-cl} (question without extraction) all have certain semantic properties in common (they denote a question) and are therefore all subtypes of \emph{inter-cl}. But only \emph{wh-int-cl} is also a subtype of \emph{head-filler-structure}.\footnote{\emph{hd-fill-ph} in their terminology.} 

I adopt a similar analysis, but with the type hierarchy of \emph{clause} from \citet{Winckel.2020}, adapted from \citet[48]{Abeille.2007.Relatives}. The hierarchy is given in \figref{fig:hrch-clause} on page \pageref{fig:hrch-clause}. This hierarchy differs from \citet[7]{Ginzburg.2000}, because in their proposal relative clauses are all headed by a filler, and hence a subtype of \emph{head-filler-structure}.\footnote{This is the case for \emph{that} relative clauses, even though in many analyses \emph{that} is not treated as a filler.} I have argued above that \emph{que}, \emph{qui} and \emph{dont} in French are complementizers and not fillers (Section~\ref{ch:intro-disscussion-French}). As a consequence, only some relative clauses are a subtype of \emph{head-filler-structure}. Furthermore, some proposals treat clefts as a special type of clause or construction \citep[a.o.][]{Kim.2012}, but in my analysis \emph{c'est}-clefts result from a special lexical entry of \emph{être} (`be').

\begin{figure}
\begin{forest}
[\textit{sign}
    [\textit{clause}
        [\textit{core-cl}
            [\textit{decl-cl}] 
            [\textit{inter-cl}, name = intercl]
        ]
        [\textit{rel-cl}, l sep=2\baselineskip,
            [\textit{comp-rel-cl}, name = obj]
            [\textit{wh-rel-cl}, name = subj
                [\textit{wh-inter-cl}, name = inter, no edge
                    [\textit{standard-wh-inter-cl}]
                    [\textit{rhetorical-wh-inter-cl}]
                ]
                [{}, no edge]
            ]
        ]
    ]
    [\textit{phrase}
        [head-comps-\\structure, align=center, base=bottom, font=\itshape, name = comp]
        [\textit{...}]
        [{}, no edge[{}, no edge
            [noncomp-comps-\\structure, align=center, base=bottom, font=\itshape, no edge, name = noncomp]
            [\textit{sent-comps}, no edge, name = sentcomps]
        ]]
        [head-filler-\\structure, align=center, base=bottom, font=\itshape, name = filler]
    ]
]
\draw[thin] (inter.north)--(filler.south);
\draw[thin] (inter.north)--(intercl.south);
\draw[thin] (subj.north)--(filler.south);
\draw[thin] (obj.north)--(comp.south);
\draw[thin] (noncomp.north)--(comp.south);
\draw[thin] (sentcomps.north)--(comp.south);
\end{forest}
\caption{Cross-classification of \emph{clause} and \emph{phrase}}
\label{fig:hrch-clause}
\end{figure}


\subsection{\emph{Wh}-questions}

I will first discuss interrogatives before turning to the other extractions. I only present \emph{wh}-questions (with a \emph{wh}-word), but refer the interested reader to \citet[218--222]{Ginzburg.2000} for an analysis of polar questions. 

There are different question types in French, and some of them have special pragmatic properties \citep[see a corpus study in][]{Abeille.2012}. Three of them involve extraction of the \emph{wh}-word: SVO-questions (\ref{ex:question-svo}), questions with suffixed subjects (\ref{ex:question-inversion}), and \emph{est-ce que} questions (\ref{ex:question-esk}).\footnote{Another interrogative form is possible, but it involves \emph{c'est}-clefting and will be addressed in Section~\ref{ch:hpsg-clefts}.} There is a fourth type: in-situ questions, with no extraction of the \emph{wh}-word (\ref{ex:question-insitu}). 

\eal 
\ex[]{\gll Qui$_i$ l' innovation enthousiasme~\trace{}$_i$?\\
who the innovation excites\\
\glt `Who does the innovation excite?'} 
\label{ex:question-svo}
\ex[]{\gll Qui$_i$ (l' innovation) enthousiasme-t-elle~\trace{}$_i$?\\
who the innovation[\textsc{f}] excites-0-\textsc{3sg.f.sbj}\\
\glt `Who does it excite?'} 
\label{ex:question-inversion}
\ex[]{\gll Qui$_i$ est - ce que l' innovation enthousiasme~\trace{}$_i$?\\
who is {} it that the innovation excites\\
\glt `Who does the innovation excite?'} 
\label{ex:question-esk}
\ex[]{\gll L' innovation enthousiasme qui?\\
the innovation excites who\\
\glt `The innovation excites who?'} 
\label{ex:question-insitu}
\zl 

Embedded questions must be formed via extraction, and cannot involve a suffixed subject. 

\eal 
\judgewidth{??}
\ex[]{\gll Je me demande qui$_i$ l' innovation enthousiasme~\trace{}$_i$.\\
I \textsc{refl} ask who the innovation excites\\
\glt `I wonder who the innovation excites.'} 
\ex[??]{\gll Je me demande qui$_i$ (l' innovation) enthousiasme-t-elle~\trace{}$_i$.\\
I \textsc{refl} ask who \sbar{}the innovation[\textsc{f}] excites-0-\textsc{3sg.f.sbj}\\
\glt `I wonder who the innovation excites.'} 
\label{ex:question-inversion-iq}
\ex[]{\gll Je me demande qui$_i$ est - ce que l' innovation enthousiasme~\trace{}$_i$.\\
I \textsc{refl} ask who is {} it that the innovation excites\\
\glt `I wonder who the innovation excites.'} 
\label{ex:indir-question-esk}
\ex[*]{\gll Je me demande l' innovation enthousiasme qui.\\
I \textsc{refl} ask the innovation excites who\\
\glt `I wonder who the innovation excites.'} 
\label{ex:question-insitu-iq}
\zl 

The feature \textsc{main clause (mc)} is a head feature of verbs, and has a value of type \emph{bool} (+/$-$), positive for the main clause and negative for embedded clauses. Direct questions are \avm{[mc & \normalfont{+}]} and embedded questions \avm{[mc & $-$]}.\footnote{The topic of subject-verb inversion in French is not completely orthogonal to our main concern, given that sentence-final subjects seem to bear focus \citep{Lahousse.2011}. In this respect, the FBC constraint certainly makes some prediction for subject-verb inversion, but there are no empirical data bearing on this issue yet so I leave it for future work.}

To account for the different question types, I assume that subjects can be realized as clitics if the verb subcategorizes for an extracted interrogative phrase. Clitic doubling is also allowed, thus the subject may be realized as a clitic and as an NP like in (\ref{ex:question-inversion}). However, subject suffixes are restricted to \avm{[mc & \normalfont{+}]}. 

To my knowledge, there is no complete HPSG analysis of \emph{est-ce que} questions in French. \citet[70]{Abeille.2012} treat \emph{est-ce que} as a complementizer and assume a special subtype of interrogatives, \emph{est-ce-que-cl}. If this is on the right track, the type \emph{est-ce-que-cl} should have two subtypes, one inheriting from polar questions (\emph{polar-int-cl}) and one inheriting from \emph{wh}-questions with an extraction (\emph{wh-inter-cl}). In polar \emph{est-ce que} questions, the complementizer selects an S without subject-verb inversion and contributes the interrogative interpretation of the sentence. In a \emph{wh}-question with \emph{est-ce que}, the head daughter is \avm{[head & comp]} and has a non-empty \textsc{slash} set. 
I leave a more detailed analysis of \emph{est-ce que} questions for future work and, for the sake of simplicity, I do not include \emph{est-ce-que-cl} and its subtypes in my hierarchy of clauses (\figref{fig:hrch-clause}).

All interrogatives are instances of \emph{inter-cl}. \citet[42]{Ginzburg.2000} assume that objects of the type \emph{inter-cl} have some common semantic properties. They may also have some common pragmatic properties. In my fragment, the semantic content of questions is contributed on a lexical level by the \emph{wh}-word with an appropriate EP. For example, the interrogative \emph{qui} can either have an empty \textsc{que} set when it is in-situ or a non-empty \textsc{que} set when it is extracted. This is shown by the lexical entry in \figref{ex:avm-qui}, adapted from \citet[185]{Ginzburg.2000}, with a simplified semantic representation like the one adopted in the LinGO English Resource Grammar for \emph{wh}-words.
%\footnote{I assume in general that all interrogative words have an EP of the type \emph{which\_q}. It is a quantifier, and as such can have scope over the whole clause even if the \emph{wh}-word is in situ. For the same reason, \citet{Ginzburg.2000} also assume that the semantic contribution of \emph{wh}-words are quantifiers, but adopt a different representation. I do not elaborate on the mechanism for quantification, but see \citet{Copestake.2005} for a detailed explanation of MRS and underspecified quantification scopes.}.

\begin{figure}[h]
\avm{[phon & < \type{qui} >\\
loc & [cat|head & noun\\
                 cont|rels & <[\type*{person}\\
                               arg0 & \tag{i}],
                               \1 [\type*{which\_q}\\
                               arg0 & \tag{i}]>]\\
        nonloc & [que & \{\1\} $ \vee$ \{\}\\
        rel & \{\}]\\
arg-st <>]}
\caption{Lexical entry for interrogative \emph{qui} (`who')}\label{ex:avm-qui}
\end{figure}


As far as information structure is concerned, I have also argued in Section~\ref{ch:exp11} that in-situ questions may not always involve focalization. Therefore, I do not propose any constrain on the type \emph{inter-cl}, but acknowledge that other semantic or pragmatic aspects may be involved.

\subsubsection{Extracted \emph{wh}-phrase}

Objects of the type \emph{wh-inter-cl} inherit the properties of \emph{head-filler-structure}. The filler in \emph{wh-inter-cl} must have a non-empty \textsc{que} value \citep[see the Filler Inclusion Constraint in][228]{Ginzburg.2000}. As a consequence, in-situ interrogative words are never used in \emph{wh-inter-cl} structures. The element in the \textsc{que} set is saturated when the filler combines with the S on the clausal level. This is reflected by the definition of \emph{wh-inter-cl} in (\ref{ex:avm-wh-inter-cl}). 

\ea \textit{wh-inter-cl} \avm{$\Rightarrow$}\nopagebreak

\avm{[que & \1\\
head-dtr & [que & \{\2\} $\cup$ \1]\\
nhead-dtrs & <[que & \{\2\}]>]}
\label{ex:avm-wh-inter-cl}
\z 

\figref{fig:wh-question-simple} demonstrates the analysis of the \emph{wh}-question in (\ref{ex:question-svo}). Through the \textsc{nonloc} Amalgamation Principle (\ref{avm:nonloc-amalgamation}), the value of \textsc{que} is struc\-ture-shared with the whole filler phrase. This is how pied-piping structures like (\ref{ex:wh-question-complex}) are accounted for as well, as illustrated by \figref{fig:wh-question-complex} on page \pageref{fig:wh-question-complex}.

\ea[]{\gll [De l' anniversaire de qui]$_i$ tu parles~\trace{}$_i$?\\
\sbar{}of the birthday of who you talk\\
\glt `Whose birthday are you talking about?'} 
\label{ex:wh-question-complex}
\z

\begin{figure}[ht]
\centering
\begin{forest}
where n children=0{tier=word}{}
[S\\
\avm{[\type*{wh-inter-cl}\\
      slash & \{\}\\
      que & \{\}]}
    [NP\\
    \avm{[loc & \1\\
          que & \{\2\}]}
          [qui\\who]
    ]
    [S\\
    \avm{[\type*{head-subj-structure}\\
          slash & \{\1\}\\
          que & \{\2\}]}
          [NP
            [l'innovation\\the innovation, roof]
          ]
          [VP\\
          \avm{[slash & \{\1\}\\
          	que & \{\2\}]}
            [enthousiasme\\excites]
          ]
    ]
]
\end{forest}
\caption{Simplified tree for ``Qui$_i$ l'innovation enthousiasme~\trace{}$_i$?'' (`Who does the innovation excite?')}
\label{fig:wh-question-simple}
\end{figure}

\begin{figure}[htbp]
\centering
\begin{forest}
where n children=0{tier=word}{}
[S\\
\avm{[\type*{wh-inter-cl}\\
      subj & <>\\
      comps & <>\\
      slash & \{\}\\
      que & \{\}]}
    [PP\\
    \avm{[loc & \1\\
          que & \{\2\}]}
          [P \\ \avm{[que & \{\2\}]} [de\\of]]
          [NP \\ \avm{[que & \{\2\}]}
            [Det [l'\\the]]
            [N$'$ \\ \avm{[que & \{\2\}]}
                [N \\ \avm{[que & \{\2\}]}[anniversaire\\birthday]]
                [P \\ \avm{[que & \{\2\}]}
                    [P \\ \avm{[que & \{\2\}]} [de\\of]]
                    [NP \\ \avm{[que & \{\2\}]} [qui\\who]]
                ]
            ]
          ]
    ]
    [S\\
    \avm{[\type*{head-subj-structure}\\
          subj & <>\\
          comps & <>\\
          slash & \{\1\}]}
          [\avm{\3} NP
            [tu\\you]
          ]
          [VP\\
          \avm{[subj & <\3>\\
                comps & <>\\
                slash & \{\1\}]}
            [parles\\talk]
          ]
    ]
]
\end{forest}
\caption{Simplified tree for [\textit{De l'anniversaire de qui}]$_i$ \textit{tu parles}~\trace{}$_i$\textit{?} (`Whose birthday are you talking about?')}
\label{fig:wh-question-complex}
\end{figure}

Furthermore, I assume cross-classifications of \emph{clause} and \emph{speech-act}. The type hierarchy of \emph{speech-act} is given in \figref{fig:hrch-speech-act} on page~\pageref{fig:hrch-speech-act}. Speech acts refer to the act performed when expressing an utterance. Typically, we think of interrogatives as requests for information, and in our analysis this speech act is defined as \emph{standard-question}. Interrogatives can, however, be non-standard, as is the case in rhetorical questions. Thus all \emph{wh}-questions are not necessarily focalizations. I presented in Section~\ref{ch:attested-questions-subextraction} some examples with felicitous extraction out of the subject, such as (\ref{ex:internet-ahmadinejad}) reproduced in (\ref{ex:internet-ahmadinejad-bis}). I argued that it is probably best analyzed as a rhetorical question, where the filler is a continuation topic in the context of the utterance. This explains why extraction out of the subject is felicitous. A rhetorical question is a ``biased question whose answer is Common Ground and whose dialogue impact requires the activation of such a content'' \citep[441]{Marandin.2008}. 

\begin{figure}[ht]
\centering
\oneline{
\begin{forest}
sn edges,
[\textit{speech act}
    [\textit{assertion}]
    [\textit{question}
        [\textit{standard-question}]
        [\textit{non-standard-question}]
    ]
    [\dots]
]
\end{forest}}
    \caption{Type hierarchy of \emph{speech-act}}
    \label{fig:hrch-speech-act}
\end{figure}

\ea[]{\gll [De quel pays]$_i$ [la dépense militaire~\trace{}$_i$] dépasse annuellement mille milliards de dollars~[\dots]~?\\
\sbar{}of which country \sbar{}the budget military exceeds yearly thousand billion of dollars\\
\glt `Of which country does the military budget exceed yearly 1000 B.\ dollars?'}
\label{ex:internet-ahmadinejad-bis}
\z 

I assume that all subtypes of \emph{inter-cl} can inherit from either \textit{standard-question} or \textit{non-standard-question}. A linguistic object can therefore be \emph{wh-inter-cl} and \emph{standard-question}: these objects are \emph{standard-wh-inter-cl}. A linguistic object can also be \emph{wh-inter-cl} and \emph{non-standard-question}: these objects are \emph{non-standard-wh-inter-cl}. 

I propose the constraint \REF{avm:standard-wh-inter-cl} on information structure for standard \emph{wh}-questions, borrowed from \citet[117]{Winckel.2020}\footnote{The main difference is that  \citeauthor{Winckel.2020} define this constraint on \emph{wh-inter-cl}. The FBC constraint then excludes extraction of the complement of topic subjects, allowing only extraction out of non-topic subjects. I doubt that it is what is at stake in example (\ref{ex:internet-ahmadinejad-bis}) and therefore think that it is necessary to make a distinction between the speech acts involved.}:\largerpage[2.25]

\ea 
\textit{standard-wh-inter-cl} \avm{$\to$}\nopagebreak

\avm{[head-dtr & [index & \1]\\
nhead-dtrs & <[ index & \2 \\
                      icons-key & [\type*{semantic-focus}\\
                                 target & \2\\
                                 clause & \1]]>]
}
\label{avm:standard-wh-inter-cl}
\z 

The exact information structure for non-standard questions would need further investigation. I assume for the time being that non-standard \emph{wh}-questions imply a \emph{non-focus} object in the \textsc{icons} list, though this is most probably an oversimplification.

\ea 
\textit{non-standard-wh-inter-cl} \avm{$\to$}\nopagebreak

\avm{[head-dtr & [index & \1]\\
filler-dtr & [index & \2 \\
            icons-key [\type*{non-focus}\\
                                 target & \2\\
                                 clause & \1]]]}
\z

\subsubsection{\emph{Wh}-phrase in situ}

As stated previously, in-situ \emph{wh}-words have an empty \textsc{que} set. The type \textit{in-situ-int-cl} is constrained to be \avm{[mc & \normalfont{+}]}, so that it cannot apply to embedded questions \citep[271]{Ginzburg.2000}. However, if there are several \emph{wh}-words in an embedded question, only one has to be extracted, the other one(s) can remain in situ. Additional rules are necessary to account for superiority effects in questions \citep[247--254]{Ginzburg.2000}.

The discussion around the pragmatic status of in-situ questions was briefly touched upon in Section~\ref{ch:exp14-discussion}. At present there is not enough evidence concerning the real status of the \emph{wh}-word in situ, but it would be possible, for example, to constrain \textit{in-situ-int-cl} to be backgrounded or to be discourse-given. 

\subsection{Relative clauses}\largerpage[2]

Just as extracted interrogative words have a non-empty \textsc{que} set, relative words have a non-empty \textsc{rel} set. A \emph{wh}-word like \emph{où} (`where') can be used as an interrogative word or as a relative word. Hence, it has the lexical entry in \figref{ex:avm-ou}.\footnote{French \emph{où} can also have a temporal interpretation, but I disregard this detail as it is unrelated to my analysis.}

\begin{figure}[h]
\small
\avm{[phon & < \type{où} >\\
loc & [cat|head & prep\\
                 cont|rels & <[\type*{place}\\
                               arg0 & \tag{i}]>]\\
        nonloc & [que & \{\}\\
        rel & \{\tag{i}\}]  \\
arg-st < >]} $ \vee$ \avm{[phon & < \type{où} >\\
loc & [cat|head & prep\\
                 cont|rels & <[\type*{place}\\
                               arg0 & \tag{i}],\\
                               \1 [\type*{which\_q}\\
                               arg0 & \tag{i}]>]\\
        nonloc & [que & \{\1\}\\
        rel & \{\}]\\
arg-st < >]}
\caption{Lexical entry for the \emph{wh}-word \emph{où} (`where')}\label{ex:avm-ou}
\end{figure}

According to \citet{Godard.1988}, relative words in French are either relative pronouns (e.g.\ \emph{où} `where', \emph{lequel} `which') or complementizers (e.g.\ \emph{dont} `of which').\footnote{Contrary to \citet{Sag.1997}, who analyses the relativizer \emph{that} as a relative pronoun, homonymous with the complementizer \emph{that}.} Details of her arguments are presented in Section~\ref{ch:intro-disscussion-French}. I adopt \citegen{Abeille.2007.Relatives} cross-classification of relative clauses.
I assume with them that only relative pronouns or PPs comprising relative pronouns can serve as fillers, and that complementizers are heads \citep[see also][]{Borsley.2020.HPSG.UDC}. Following the type hierarchy of \emph{clause} presented in \figref{fig:hrch-clause}, relative clauses (\textit{rel-cl}) either inherit from \textit{head-comps-structure} and are \textit{comp-rel-cl} or they inherit from \textit{head-filler-structure} and are \textit{wh-rel-cl}. 

Relative clauses are headed by a verbal category, i.e.\ either a \emph{verb} or a \textit{comp}(\textit{lementizer}), see \figref{fig:hrch-pos}.\footnote{I do not discuss here gapless and verbless relatives in French. See \citet{Bilbiie.2010} for verbless relative clauses and \citet{Abeille.2007.Relatives} for HPSG analysis of gapless relative clauses.} Relative clauses have an empty \textsc{slash} set, because extraction out of relative clauses is not allowed in French, and an empty \textsc{rel} set for the same reason.

\ea \textit{rel-cl} \avm{$\to$
       [ head & [\type*{verbal} \\
                 mod & [head & noun\\
                        spr & <>] ]\\
                slash & \{\}\\
                 rel & \{\}] 
    \label{avm:rel-cl}
}
\z\largerpage[-1]\pagebreak

Relative clauses modify NPs.\footnote{\citet{Jackendoff.1977} suggests that restrictive relative clauses attach to N$'$ while non-restrictive relative clauses attach to NP. My analysis is compatible with this proposal, if the value of \textsc{spr} is left unconstrained in (\ref{avm:rel-cl}). However, it would then be necessary to explain how restrictive relative clauses can attach to coordinated nouns \citep[293]{Kiss.2005}: 
\begin{itemize}
\item[(i)] \gll la femme et l' enfant [dont Nicole a parlé hier~\trace{}]\\
the woman and the child \sbar{}of.which Nicole has talked yesterday\\
\glt `the woman and the child that Nicole talked about yesterday'
\end{itemize}}
We also follow \citegen{Kuno.1976} claim that extraction in relative clauses is topicalization. In \citegen{Song.2017} model, relative pronouns and complementizers do not have an information structure status, thus the relative phrase is not the topic of the relative clause (contra \citealt{Bresnan.1987} a.o.). Furthermore, not all languages have relative phrases to introduce relative clauses and bear the topic function. Rather, the antecedent serves as the topic of the relative clause. 
That the antecedent is a topic with respect to the relative clause is a semantic contribution of the relative clause (via \textsc{c-cont}). The following constraint can then be added to the French fragment:{\interfootnotelinepenalty=10000\footnote{\label{fn:non-standard-gapless-rc}The constraint in (\ref{avm:rel-cl-is}) differs from a similar constraint proposed by \citet[182]{Song.2017} because he assumes that all relatives are \emph{filler-head-str}. In (\ref{avm:rel-cl-is}), the antecedent, and not the filler, is the topic of the relative clause. (\ref{avm:rel-cl-is}) is also more compatible with the analysis of relative clauses without a gap in non-standard French, in which the antecedent is the topic \citep{Abeille.2007.Relatives}:
\begin{itemize}
    \item[(i)] (\citealt{Deulofeu.1981}; cited by \citealt[38]{Abeille.2007.Relatives})
    \item[] \gll Vous avez des feux [qu' il faut appeler les pompiers tout de suite].\\
    you have some fires \sbar{}that it must call.\textsc{inf} the firemen all at now\\
    \glt `There are some fires that one needs to call the firemen immediately.'
\end{itemize}
}}

\ea \textit{rel-cl} \avm{$\to$
       [head|mod|index & \tag{i}\\
       cont|clause-key & \tag{e}\\
       c-cont|icons &  <[\type*{topic}\\
                          target & \tag{i}\\
                          clause & \tag{e}]>] 
    \label{avm:rel-cl-is}
}
\z 

% aboutness topic would be better, but contrastive topic is needed for clefts. Anyway, Song has focus-or-topic.

\subsubsection{\textit{comp-rel-cl}}\largerpage

\emph{Qui} (`who') relative clauses are extractions of the subject (regardless of animacy); \emph{que} (`that') relative clauses are extractions of the object; and both \emph{qui} and \emph{que} are complementizers. Complementizers select an S complement. The complementizer \emph{qui} is a variant of \emph{que} used whenever the S complement has a gapped subject. It follows that, in long distance dependencies, \emph{qui} is used to introduce the S containing the missing subject, and not as the head of the relative clause, see example (\ref{ex:qui-que-rule}). This alternation is known as the \emph{que-qui} rule \citep[see a.o.][]{Pesetsky.1982,Koopman.2014}. 

\eal\label{ex:qui-que-rule}
\ex \gll Je veux [que Daniel vienne]$_{\text{comp}}$.\\
I want \sbar{}that\textsubscript{que} Daniel comes\\
\glt `I want Daniel to come.'
\ex \citep[194]{Melis.1988}\\
\gll l' homme [que je veux [qui~\trace{} vienne]$_{\text{comp}}$]$_{\text{RC}}$\\
the man \sbar{}that\textsubscript{que} I want \sbar{}that\textsubscript{qui} comes\\
\glt `the man who I want to come' (i.e., I want that man to come)
\zl 

The two lexical entries for \emph{que} and complementizer \emph{qui} in \figref{avm:lexical-entry-que-qui} are directly borrowed from \citet[50]{Abeille.2007.Relatives}. The complement S must be finite (see Section~\ref{ch:intro-disscussion-French}).\largerpage

\begin{figure}[h]
\caption{Lexical entries for the French complementizers \emph{que} and \emph{qui}}\label{avm:lexical-entry-que-qui}

\begin{subfigure}[b]{\linewidth}\centering
\avm{[phon <\type{que}>\\
synsem|loc|cat & [head & comp\\
marking & que\\
comps & <[vform & finite\\
          subj & <>]>]]}

\caption{\emph{que}}
\end{subfigure}\medskip\\
\begin{subfigure}[b]{\linewidth}\centering

\avm{[phon <\type{qui}>\\
synsem|loc|cat & [head & comp\\
marking & que\\
comps & <[vform & finite\\
          subj & < [\type{gap}] >] > ]]}
          
\caption{Complementizer \emph{qui}}
\end{subfigure}
\end{figure}

Notice the presence of a syntactic feature \textsc{marking}, defined for complementizers and prepositions (also possibly nouns, see a.o.\ \citealt[159]{Sportiche.1998}; I will come back to this issue in Section~\ref{ch:hpsg-prep}). The value of \textsc{marking} is an object of type \emph{marking}, whose hierarchy is given in \figref{fig:hrch-marking}. Complementizers may have a \textsc{marking} \emph{que} or \emph{dont}. Prepositions have a \textsc{marking} value matching their form (e.g.\ \emph{de}, \emph{sur}). 

\begin{figure}[ht]
\centering
\scalebox{1}{
\begin{forest}
sn edges,
[\textit{marking}
    [\textit{comp-marking}
        [\textit{que}]
        [\textit{dont}]
    ]
    [\textit{prep-marking}
        [\textit{de}]
        [\textit{a}
            [\textit{a-dat}]
            [\textit{a-loc}]
        ]
        [\textit{sur}]
        [\textit{par}]
        [\textit{avec}]
        [\dots]
    ]
]
\end{forest}}
    \caption{Type hierarchy of \emph{marking}}
    \label{fig:hrch-marking}
\end{figure}

The \textsc{marking} for PPs ensures that the right preposition is selected if the PP is a complement; ensures the use of the right clitic for \emph{à}-PP (\emph{lui} for dative, \emph{y} for locative); and also ensures that \emph{dont} is used to only relativize a \emph{de}-PP. Hence, the lexical entry for \emph{dont} is (\ref{avm:dont-lex}).


\ea Lexical entry for \emph{dont}:\nopagebreak

\avm{[phon & < \type{dont} >\\
synsem|loc|cat & [head & comp\\
marking & dont \\
comps & <S[slash & \{[marking & de\\
                      subj & < >]\}]>]]}
\label{avm:dont-lex}
\z 

In the standard relative clauses discussed so far, the complementizers take as complement an S with a non-empty \textsc{slash} set.\footnote{\label{fn:que-non-standard}But notice that the lexical entries for \emph{que/qui} in \figref{avm:lexical-entry-que-qui} can take complements with empty \textsc{slash} sets. This is necessary for two reasons: (i) they can introduce a gapless clause as complement of a verb (e.g.\ \emph{dire} `say'); and (ii) in non-standard French, \emph{que/qui} relative clauses can be gapless (see fn.\ \ref{fn:non-standard-gapless-rc}). Furthermore, in non-standard French, the gap in the \emph{que/qui} relative clause sometimes does not correspond to an NP:
\begin{itemize}
    \item[(i)] \gll J' ai besoin du livre.\\
    I have need of.the book\\
    \glt `I need the book.'
    \item[(ii)] \gll le livre dont j' ai besoin\\
    the book of.which I have need\\
    (standard French)
    \glt `the book I need'
    \item[(iii)] \gll le livre que j' ai besoin\\
    the book that I have need\\
    (non-standard French)
    \glt `the book I need'
\end{itemize}
} The gap is coindexed with the antecedent of the relative clause. The whole relative clause is monoclausal, the main event being the event of the main verb (hence of the non-head daughter).
%Even though the complementizer is the syntactic head of the structure, the semantic head is the S, and the relative clause has the same \textsc{icons-key} than the S. --> we don't need that?
This leads to the definition of \textit{comp-rel-cl} in (\ref{avm:comp-rel-cl}). Notice that this constraint overwrites the default \textsc{slash} Amalgamation Principle (\ref{avm:slash-amalgamation}), because the mother node does not inherit from the \textsc{slash} values of its daughters, even though \textit{comp-rel-cl} is an instance of \emph{head-comps-str}.\footnote{Other kinds of \emph{head-comps-str} are \emph{noncomp-comps-structure} and \emph{sent-comps}. The type \emph{noncomp-comps-structure} is defined as \avm{[head & non-comp]}. The type \emph{sent-comps}, for sentential complements introduced by a complementizer, is defined as \avm{[mod & none]}.} Furthermore, because \textit{rel-cl} may not have an empty \textsc{slash} set in French, the constraint means that the S complement of the complementizer necessarily has only one element in its \textsc{slash} set in French.\footnote{\citet{Abeille.2007.Relatives} assume that \emph{dont} in standard French takes a finite complement, and that \emph{comp-rel-cl} hence always implies a finite complement. But infinite \emph{dont}-CPs seem at least marginally acceptable (see Section~\ref{ch:intro-disscussion-French}), so I see no need to rule them out.

The rule in (\ref{avm:comp-rel-cl}) implies that the non-head daughter has a non-empty \textsc{slash} set. To account for non-standard relative clauses, a disjunction is probably necessary: either the non-head daughter has an empty \textsc{slash} set and has then \avm{[marking & que]} (not compatible with \emph{dont}), or  (\ref{avm:comp-rel-cl}) applies.}

\ea \textit{comp-rel-cl} \avm{$\to$
[clause-key & \tag{e}\\
slash & \1\\
head-dtr & [head & [\type*{comp}\\
                    mod & [index & \tag{i}]]]\\
nhead-dtrs & <[loc|clause-key & \tag{e}\\ 
             non-loc|slash & \{[index & \tag{i}]\} $\cup$ \1]>]
}
\label{avm:comp-rel-cl}
\z

\figref{fig:rc-simple} shows the relative clause introduced by a complementizer in (\ref{ex:rc-que}).

\ea[]{\gll mes collègues$_i$ [que l' innovation enthousiasme~\trace{}$_i$]\\
my colleagues \sbar{}that the innovation excites\\
\label{ex:rc-que}
\glt `my colleagues that the innovation excites'}
\z 

\begin{figure}[htbp]
\centering
\begin{forest}
where n children=0{tier=word}{}
[NP\\
\avm{[\type*{head-mod-str}]}
[NP\\
\avm{[synsem & \1 [index & \2]]}
[mes collègues\\my colleagues, roof]]
[S\\
\avm{[\type*{comp-rel-cl}\\
      head & comp\\
      mod & \1\\
      subj & <>\\
      comps & <>\\
      slash & \{\}\\
      rel & \{\}]}
    [COMP
          [que\\that]
    ]
    [S\\
    \avm{[\type*{head-subj-structure}\\
          subj & <>\\
          comps & <>\\
          slash & \{\3\}]}
          [\avm{\4} NP
            [l'innovation\\the innovation, roof]
          ]
          [VP\\
          \avm{[subj & <\4>\\
                comps & <>\\
                slash & \{\3 [index & \2]\}]}
            [enthousiasme\\excites]
          ]
    ]
]]
\end{forest}
\caption{Simplified tree for \textit{mes collègues} [\textit{que}$_i$ \textit{l'innovation enthousiasme}~\trace{}$_i$] (`my colleagues that the innovation excites')}
\label{fig:rc-simple}
\end{figure}

\subsubsection{\textit{wh-rel-cl}}

I will assume that fillers in relative clauses must be PPs: subjects and objects are extracted with a complementizer, and pied-piping of an NP is not allowed in French, as illustrated in (\ref{ex:prep-filler}).\footnote{But see a discussion of some exceptions in Section~\ref{ch:deq-corpus}.}
\largerpage[-1]\pagebreak

\eal  \label{ex:prep-filler}\judgewidth{??}
\ex[]{the people who live in Purus, [the majority of whom] are poor\footnote{\url{https://www.theguardian.com/environment/andes-to-the-amazon/2013/may/24/peru-amazon-rainforest}, last access 25/07/2020}}
\ex[??]{\gll les habitants de Purús, [la majorité desquels] sont pauvres\\
the inhabitants of Purus \sbar{}the majority of.the.which are poor\\}
\zl 

The definition of \textit{wh-rel-cl} is given in \figref{avm:wh-rel-cl}. The non-local feature \textsc{rel} of the filler is coindexed with the antecedent of the relative clause. The relative clause is headed by the verb. 

\begin{figure}[ht]
\textit{wh-rel-cl} \avm{$\to$
[head-dtr & [head & [\type*{verb}\\
mod & [index & \tag{i}]]]\\
filler-dtr & [head & prep\\
rel & \{\tag{i}\}]]
}
\caption{Definition of \textit{wh-rel-cl}}
\label{avm:wh-rel-cl}
\end{figure}

\figref{fig:rc-complex} shows the relative clause introduced by a filler in (\ref{ex:wh-question-complex}). Notice that \emph{qui} (`who') here is not the complementizer but a relative pronoun that is used only as a complement to prepositions.\footnote{There are hence four versions of \emph{qui} (`who'): the interrogative \emph{qui} extracted, the interrogative \emph{qui} in situ (\figref{ex:avm-qui}), the complementizer \emph{qui} (\figref{avm:lexical-entry-que-qui}b) and the relative pronoun \emph{qui}.} In accordance with the \textsc{nonloc} Amalgamation Principle (\ref{avm:nonloc-amalgamation}), the value of \textsc{rel} of the relative pronoun is percolated to the maximal projection of the filler.

\ea[]{\gll Gaetan, [[de l' anniversaire de qui]$_i$ tu parles~\trace{}$_i$]\\
Gaetan \ssbar{}of the birthday of who you talk\\
\glt `Gaetan, whose birthday you are talking about'} 
\label{ex:rc-complex}
\z

\begin{figure}
\oneline{%
\begin{forest}
where n children=0{tier=word}{}
[NP\\
\avm{[\type*{head-mod-str}]}
[NP\\
\avm{[synsem & \1 [index & \3]]}
[Gaetan\\Gaetan]]
[S\\
\avm{[\type*{wh-rel-cl}\\
      head & verb\\
      mod & \1\\
      slash & \{\}\\
      rel & \{\}]}
    [PP\\
    \avm{[loc & \2\\
          rel & \{\3\}]}
          [P \\ \avm{[rel & \{\3\}]} [de\\of]]
          [NP \\ \avm{[rel & \{\3\}]}
            [Det [l'\\the]]
            [N$'$ \\ \avm{[rel & \{\3\}]}
                [N \\ \avm{[rel & \{\3\}]}[anniversaire\\birthday]]
                [P \\ \avm{[rel & \{\3\}]}
                    [P \\ \avm{[rel & \{\3\}]} [de\\of]]
                    [NP \\ \avm{[rel & \{\3\}]} [qui\\who]]
                ]
            ]
          ]
    ]
    [S\\
    \avm{[\type*{head-subj-structure}\\
          slash & \{\2\}]}
          [NP
            [tu\\you]
          ]
          [VP\\
          \avm{[slash & \{\2\}]}
            [parles\\talk]
          ]
    ]
]]
\end{forest}}
\caption{Simplified tree for \textit{Gaetan,} [[\textit{de l'anniversaire de qui}]$_i$ \textit{tu parles}~\trace{}$_i$] (`Gaetan, whose birthday you are talking about')}
\label{fig:rc-complex}
\end{figure}

% Ev. have an semantic analysis of restrictive/non-restrictive

\subsection{\emph{C'est}-clefts}
\label{ch:hpsg-clefts}

% prosodic contrast
%C'est avec des fleurs] que Pierre a reçu Marie
%C'est avec plaisir que je vous recevrai ] 
%Doetjes et al. 2004

In my analysis, I will distinguish two kinds of \emph{c'est}-clefts, which both involve focalization of the pivot. I leave aside presentationals introduced by \emph{c'est} or \emph{il y a} \citep[see][]{Karssenberg.2018}.  The first kind of \emph{c'est}-cleft is the one usually discussed in the French literature \citep{Doetjes.2004}. It has a \emph{que}-clause, similar to the \emph{that}-clause in English \emph{it}-clefts. An analysis of these \emph{c'est}-clefts was already published in \citet{Winckel.2020}.

I assume the entry for \emph{être} in \figref{ex:etre-cleft-entry}, which takes (expletive) \emph{ce} as a subject and two complements: the pivot, which can be of any category; and the \emph{que}-clause with a gap coindexed with the pivot.%
\footnote{It follows from \figref{ex:etre-cleft-entry} that colloquial French should allow \emph{que}-clauses in \emph{c'est}-clefts like (ii) in which the gap does not correspond to an NP (see fn.~\ref{fn:que-non-standard} on page~\pageref{fn:que-non-standard}):

\begin{itemize}
    \item[(i)] \gll C' est ce livre dont j' ai besoin. (standard French)\\
it is the book of.which I have need\\
\glt `It's the book that I need.'
    \item[(ii)] \gll C' est ce livre que j' ai besoin. (non-standard French)\\
it is the book that I have need\\
\glt `It's the book that I need.'
\label{ex:cleft-colloquial}
\end{itemize}

% "C'est un feu qu'il faut appeler les pompier tout de suite" is not possible (no gap), but it's probably a presentational rather than a cleft with focus. 
}
The pivot is interpreted as focus, and the whole \emph{c'est}-cleft is treated as a single semantic clause (the main event is the event denoted by the finite verb of the \emph{que}-clause).

\begin{figure}
\caption{\emph{être}$^1$ in \textit{c'est}-cleft}\label{ex:etre-cleft-entry}
\resizebox{\textwidth}{!}{\avm{[loc|clause-key & \1\\
nonloc|slash & \2 $\cup$ \3 \\
arg-st <NP[\type{ce}], 
        [loc & \4 [index & \5\\
                  icons-key & [\type*{focus}\\
                                      target & \5 \\
                                      clause & \1]]\\ 
          slash & \2], 
        S[marking & que\\
          clause-key & \1\\
          slash & \{\4\} $\cup$ \3] >]}}
\end{figure} 

The lexical entry in \figref{ex:etre-cleft-entry} overrides the default Slash Amalgamation Principle (\ref{avm:slash-amalgamation}): the verb inherits the \textsc{slash} information of the pivot and the \textsc{slash} information of the \emph{que}-clause that is not coindexed with the pivot. This enables extraction out of the \emph{que}-clause (see the discussion in Section~\ref{ch:analysis-clefts}, and (\ref{ex:online-cleft-extraction-que-clause}) for an attested example).
Extraction out of the pivot is allowed as well:\footnote{Example (\ref{ex:internet-cleft-extraction-pivot}) from \url{http://mysticlolly.eklablog.com/la-maitresse-a-une-vie-a45200461}, last access 03/08/2020}

\eal 
\ex \citep{Winckel.2020}\\
\gll un élève [dont$_i$ [c' est toujours [le père~\trace{}$_i$]$_j$ que je vois~\trace{}$_j$ aux réunions]]\\
a pupil \sbar{}of.which \sbar{}it is always \sbar{}the father that I see at.the meetings\\
\glt `a pupil of which it is always the father that I see at the meetings'
\ex \gll les enfants [dont$_i$ [c' est [les parents~\trace{}$_i$]$_j$ qui~\trace{}$_j$ vous ont repérée]]\\
the children \sbar{}of.which \sbar{}it is \sbar{}the parents who you\textsc{.acc} have spotted\\
\glt `the children of whom it is the parents who spotted you' (i.e.\ the parents of this child spotted you)
\label{ex:internet-cleft-extraction-pivot}
\zl 

Notice that the \emph{que}-clause can be elided. Furthermore, at least in colloquial French, the whole pivot can be extracted, as in (\ref{ex:Renaud}). \citet{Bresnan.1987} also provide example (\ref{ex:bresnan}) for English that they judge acceptable.

\eal \label{ex:cleft-extraction-pivot-interr}
\ex (title of a song by Renaud, 1980)\\
\gll Où$_i$ [c' est~\trace{}$_i$ qu' j' ai mis mon flingue~\trace{}$_i$]?\\
where \sbar{}it is that I have put my gun\\
\glt `Where did I leave my gun?'
\label{ex:Renaud}
\ex \citep[759]{Bresnan.1987}\\
Who$_i$ [it was~\trace{}$_i$ that Marilyn suspected~\trace{}$_i$]?
\label{ex:bresnan}
\zl 

Extraction in (\ref{ex:cleft-extraction-pivot-interr}) is possible because the pivot and the interrogative filler are both focus. There is thus no discourse clash. However, relativization is not felicitous in this configuration. This contrast with interrogatives is also noted by \citet[759]{Bresnan.1987} for English.

\eal 
\ex \citep[759]{Bresnan.1987}\\
* the person who$_i$ [it was~\trace{}$_i$ that Marilyn suspected~\trace{}$_i$]
\ex[*]{\gll Marine, dont [c' est~\trace{}$_i$ que je me méfiais]\\
Marine of.which \sbar{}it was that I \textsc{refl} distrusted\\
\glt `Marine, who it was that I did not trust' (i.e. Marine, it was her that I did not trust)}
\zl 

The contrast can be straightforwardly accounted for in the analysis of relative clauses and \emph{c'est}-clefts sketched above. The clefted pivot should be focus (constrained by the element in the \textsc{slash} list) with respect to the semantic head, or \textsc{clause-key}, of the clause (\emph{suspected/méfiait}), but the antecedent of the relative clause is constrained to be topic with respect to the semantic head of the relative (which is again \emph{suspected/méfiait}). This results in a discourse clash and the sentence is unacceptable.\footnote{This case is a further argument in favor of encoding information structure inside \textsc{loc}, see Section~\ref{ch:hpsg-is}.}

%Pollard and Sag discuss binding theory: It is himself that John likes most
%slash could also inherit from que-cl if we want to allow extraction out of cleft: un endroit où c'est toujours moi qui vais; otherwise no other extraction is allowed even though it is a complement

The \emph{que}-clause in \figref{ex:etre-cleft-entry} is not a relative clause, but a sentential complement with a gap. Relative clauses have an empty \textsc{slash} set and therefore cannot match the description of the selected complement. The fact that the \emph{que}-clause is a sentential complement could explain it allows extraction, in contrast to extraction out of relative clauses. Cross-linguistically, the sentential complement in \emph{it}-clefts does not always have the same syntactic properties as relative clauses. For example, in Martinique Creole, relative clauses have an optional post-clausal article \emph{a}, while sentential complements in \emph{it}-clefts do not (Stéphane Térosier, p.c.):

\begin{exe}
\ex (Stéphane Térosier, p.c.)
\begin{xlist}
\ex[]{\gll jardinié  a      man wè  (a)\\
gardner \textsc{det} \textsc{1sg} see \textsc{det}\\
\glt `the gardner that I saw'}
\ex[]{\gll Sé     jardinié    a      man wè   (*a)\\
\textsc{foc} gardner \textsc{det} \textsc{1sg} see \textsc{det}\\
\glt `It is the gardner that I saw.'}
\end{xlist}
\end{exe}

The pivot in \figref{ex:etre-cleft-entry} can be of any category, for example a PP as in (\ref{ex:cleft-basic}). It may also be an interrogative phrase, at least in colloquial French, like in (\ref{ex:cleft-pivot-question}). \figref{fig:cleft-basic} shows the structure of (\ref{ex:cleft-basic}).

\eal
\ex[]{\label{ex:cleft-basic}
\gll C' est de Gaetan que je parle.\\
it is of Gaetan that I talk\\
\glt `It's Gaetan that I'm talking about.'}
\ex[]{\label{ex:cleft-pivot-question}
\gll C' est [avec qui]$_i$ que tu parles~\trace{}$_i$~?\\
it is \sbar{}with who that you talk\\
\glt `Who are you talking to?'}
\zl 

\begin{figure}[h]
\begin{forest}
sn edges,where n children=0{tier=word}{}
[S \\
 \avm{[\type*{head-subject-structure}\\
 slash & \{\}]}
    [NP [C'\\it]]
    [VP\\
\avm{[\type*{noncomp-comps-structure}\\
slash & \{\}]}
        [VP\\
        \avm{[\type*{noncomp-comps-structure}\\
        slash & \{\}]}
        [V \\
        \avm{[slash & \{\}]}
        [est\\is]]
       [\avm{\1} PP [de Gaetan\\of Gaetan, roof]]
       ]
        [S\\
\avm{[\type*{sent-comps}\\
slash & \{\1\}]}
        [COMP [que\\that]]
        [S \\
        \avm{[\type*{head-subject-structure}\\
            slash & \{\1\}]}\\
         [NP\\
       [je\\I]]
       [V \\
       \avm{[slash & \{\1\}]}
       [parle\\talk]]]
    ]
]
]
\end{forest}
\caption{Simplified tree for \textit{C'est de Gaetan que je parle.} (`It's Gaetan that I'm talking about.')}
\label{fig:cleft-basic}
\end{figure}

The second kind of \emph{c'est}-cleft always has an NP pivot and the second complement closely ressembles a relative clause. Compare (\ref{ex:cleft-basic-2}) with (\ref{ex:cleft-basic}).

\ea[]{\gll C' est Gaetan de qui je parle.\\
it is Gaetan of who I talk\\
\glt `It's Gaetan that I'm talking about.'}
\label{ex:cleft-basic-2}
\z 

These \emph{c'est}-clefts have not received much attention, thus I have to make several assumptions, mostly relying on consistency within the analysis. Undoubtedly, more work needs to be done on this kind of \emph{c'est}-clefts. First, I will assume that the second complement of such clefts is indeed a relative.

In the corpus studies presented in this work, the only example of extraction out of a subject in a cleft that is non-presentational was (\ref{ex:d1900-clefts-subj-1}), reproduced in (\ref{ex:d1900-clefts-subj-1-bis}).

\ea (Jean-Christophe : Le Buisson ardent, Romain Rolland, 1911)\\
\gll C' était lui maintenant, dont [les yeux~\trace{}] évitaient les yeux de l' autre.\\
it was him now of.which \sbar{}the eyes avoided the eyes of the other\\
\glt `Now it was him whose eyes avoided the other's eyes.'
\label{ex:d1900-clefts-subj-1-bis}
\z 

The pivot in this sentence seems to be contrastive.\footnote{In the book, this scene, in which Jean-Christophe avoids Ana's gaze, is echoing a previous scene, in which Ana was avoiding Jean-Christophe's gaze.} For this reason, I assume that it is a contrastive topic. Because \emph{contrast-topic} is a subtype of \emph{topic}, this analysis is compatible with the information structure of relative clauses. And because contrastive topics are non-focus, the subextraction in (\ref{ex:d1900-clefts-subj-1-bis}) does not violate the FBC constraint. 

Consequently, I assume the second entry for \emph{être} in \figref{ex:etre-cleft-entry-2}, which takes (expletive) \emph{ce} as a subject and two complements: an NP pivot and a relative clause that modifies an NP coindexed with the pivot.
The pivot is interpreted as contrastive topic, and the whole \emph{c'est}-cleft is considered a single clause, as in \figref{ex:etre-cleft-entry}.

\begin{figure}[h]
\caption{\emph{être}$^2$ in \textit{c'est}-cleft}\label{ex:etre-cleft-entry-2}
\avm{[loc|clause-key & \1\\
nonloc|slash & \2 \\
arg-st <NP[\type{ce}], 
        NP \3[icons-key & [\type*{contrast-topic}\\
                                      target & \3 \\
                                      clause & \1]\\ 
          slash & \2], 
        S[\type{rel-cl}\\
          clause-key & \1\\
          mod & \3] >]}
\end{figure} 

The lexical entry in \figref{ex:etre-cleft-entry-2} predicts that extraction out of the pivot is allowed, while extraction out of the relative clause is ruled out (the \textsc{slash} set of relative clauses is empty). These predictions need to be corroborated with empirical evidence, which I leave for future work.
\figref{fig:cleft-basic-2} shows the structure of (\ref{ex:cleft-basic-2}).

\begin{figure}[h]
\begin{forest}
where n children=0{tier=word}{}
[S \\
 \avm{[\type*{head-subj-structure}\\
 slash & \{\}]}
    [NP [C'\\it]]
    [VP\\
\avm{[\type*{noncomp-comps-structure}\\
slash & \{\}]}
        [VP\\
\avm{[\type*{noncomp-comps-structure}\\
slash & \{\}]}
        [V \\
        \avm{[slash & \{\}]}
        [est\\is]]
       [\avm{\1} NP [Gaetan\\Gaetan]]
       ]
        [S\\
\avm{[\type*{wh-rel-cl}\\
mod & \1 [index & \2]\\
slash & \{\}]}
        [PP\\
        \avm{[loc & \3\\
              rel & \{\2\}]}
            [de qui\\of who, roof]]
        [S \\
        \avm{[\type*{head-subj-structure}\\
            slash & \{\3\}]}\\
         [NP\\
       [je\\I]]
       [V \\
       \avm{[slash & \{\3\}]}
       [parle\\talk]]]
    ]
]
]
\end{forest}
\caption{Simplified tree for \textit{C'est Gaetan de qui je parle.} (`It's Gaetan that I'm talking about.')}
\label{fig:cleft-basic-2}
\end{figure}

Notice that the \emph{c'est}-cleft in (\ref{ex:cleft-both}) can be an instance of both kinds of clefts, and the \emph{que}-clause can be either a sentential complement or a relative clause. This should not be a problem, because in the present analysis each possibility leads to a different information structure for the pivot.

\ea[]{\gll C' est mes collègues que l' innovation enthousiasme.\\
it is my colleagues that the innovation excites\\
\label{ex:cleft-both}
\glt `It's my colleagues that the innovation excites.'}
\z 

% LFG analysis with information structure in Bresnan & Mchombo 1987
% Bresnan, Joan & Sam A Mchombo. 1987. Topic, pronoun, and agreement in Chicheŵa. Language 63(4). 741–782.

% Song (2017:129):
% - focus constituent is head
% - that-clause is realized as a relative  clause

\subsection{Long-distance dependencies}
\label{ch:hpsg-basics-ldd}

On the clausal level, the head's \textsc{clause-key} is identified with its \textsc{index} \citep[120]{Song.2017}. This ensures that the whole clause shares the same \textsc{clause-key} down the tree.

\ea
\textit{clause} \avm{$\to$}\nopagebreak

\avm{[head-dtr & [hook & [index & \tag{e}\\
clause-key & \tag{e}]]]}
\label{avm:clause-clause-key}
\z 

To simplify, I will assume that sentential and infinitival subjects and complements are licensed for some verbs, and that this is part of their lexical entry.\footnote{This is subject to further restrictions that are not relevant to my analysis, see \citet[151--156]{Pollard.1994} and \citet{Webelhuth.2012} a.o.\ for more details.}

For example, the lexical entry for bridge verbs like \emph{suppose} (`to suppose') subcategorizes for a sentential complement, and the lexical entry for experiencer object verbs like \emph{agacer} (`annoy') subcategorizes for a sentential subject. 

\eal
\ex[]{I suppose [that you agree with me].}
\ex[]{\gll Je suppose [que tu es d' accord avec moi].\\
I suppose \sbar{}that you are of agreement with me\\
\glt `I suppose that you agree with me.'}
\label{ex:supposer-sentential-complement}
\zl 

\eal 
\ex[]{[That Kim was late] annoyed Lee.}
\ex[]{\gll [Que Kim soit en retard] aga\c{c}ait terriblement Lee.\\
\sbar{}that Kim be\textsc{.subj} in late annoyed 
awfully Lee\\
\glt `That Kim was late annoyed Lee awfully.'}
\zl 

When sentential complements and sentential subjects are finite, the lexical entry additionally specifies the discourse relation between the embedded clause and the embedding clause via an \emph{info-str} object in \textsc{icons}. The information structure may be underspecified.
The \textsc{clause-key} of the embedded clause and the \textsc{clause-key} of the embedding verb are not structure-shared: this results in two different clauses.\footnote{Except for raising and control verbs, which co-index their \textsc{clause-key} with the \textsc{index} (or \textsc{clause-key}) of the embedded clause, see \citet[141]{Song.2017}. The \emph{être} in \emph{c'est}-clefts does the same, as discussed previously.} 

See as an example the lexical entry for \emph{supposer} (`to suppose') in \figref{ex:avm-supposer} that licenses (\ref{ex:supposer-sentential-complement}). From now on, I use the shortcut \emph{lnis} for \emph{list of non-is} (cf.\ \figref{fig:hrch-icons}).

\begin{figure}

\avm{[phon & < \type{suppos-} >\\
cont & [hook & [clause-key & \tag{e1}\\
                icons-key & [\type*{info-str}\\ 
                             target & \tag{e2}\\
                             clause & \tag{e1}]\\
                index & \tag{e2}]\\
        icons & < [\type*{info-str}\\
                  target & \tag{e3}\\
                  clause & \tag{e2}] > $\oplus$ \type{lnis}]\\
arg-st & < NP[clause-key & \tag{e1}], S[marking & que\\
                                        mod & none\\
                                        index & \tag{e3}] >]}
\caption{Lexical entry for \emph{supposer} (`to suppose')}
\label{ex:avm-supposer}
\end{figure}

Notice that the verb in \figref{ex:avm-supposer} selects for a \emph{que}-clause (therefore finite) that is not a modifier (and can thus not be a relative clause). 

In general, any NP, PP, infinitival complement or infinitival subject is defined by the lexical entry of the verb that selects for it as sharing its \textsc{clause-key} value with its own.

Extractions out of sentential and infinitival complements are handled straightforwardly by the same mechanisms that account for extraction in general. The content of the \textsc{slash} set of the sentential or infinitival complement is inherited through \textsc{arg-st} by the embedding verb. It can then lead to interrogatives, relative clauses or \emph{it}-clefts.


\chapter{Extraction out of subject NPs}
\label{ch:hpsg-fbc}
\section{Noun dependents in French}
\label{ch:hpsg-prep}

So far, we have treated all \emph{de}-dependents as \emph{de}-PPs, but there is a long tradition in French linguistics of distinguishing between a preposition \emph{de} that heads \emph{de}-PPs and a weak head \emph{de} in genitive \emph{de}-NPs. This mostly stems from the distinction between extractable \emph{de}-dependents and non-extractable \emph{de}-dependents of nouns. \citet{Sportiche.1981} shows that \emph{de}-PPs that denote a local origin cannot be extracted out of an NP, see (\ref{ex:depp-local-origin}). Furthermore, the presence of a second \emph{de}-dependent may block the extraction of a \emph{de}-dependent that is otherwise acceptable, see (\ref{ex:depp-double-de}).\largerpage 

\eal 
\ex \citep[adapted from][225]{Sportiche.1981}\\
* \gll la prison, de laquelle le transfert s' effectua avec du retard\\
the jail of which the transportation \textsc{refl} performed with some delay\\
\glt `the jail, from which the transportation has been performed with some delay'
\label{ex:depp-local-origin}
\ex \citep[63]{Godard.1996}\\
\gll La jeune femme dont le portrait (*de Corot) est à la fondation Barnes est
inconnue.\\
the young woman of.which the portrait \hspace{10pt}of Corot is at the foundation Barnes is unknown\\
\glt `The young woman whose portrait (by Corot) is at the Barnes foundation is unknown.'
\label{ex:depp-double-de}
\zl 

This latter problem with multiple \emph{de}-dependents has been analyzed either in terms of a hierarchy of semantic roles \citep{Sag.1994.Godard,Godard.1996}, or as a contrast between individual and property denoting interpretations \citep{Kolliakou.1999,Mensching.2018}. It has also been explained syntactically as a distinction between extractable argument \emph{de}-NPs and non-extractable adjunct \emph{de}-PPs \citep{Kolliakou.1999}. I have previously argued that the problem of multiple \emph{de}-dependents of nouns should be analyzed in semantic rather than syntactic terms \citep{MyP.2015}.
%Both accounts, however, agree on the fact that only the first \emph{de}-argument in the \textsc{arg-st} of the N can be extracted, iff this \emph{de}-phrase is indeed a NP[de] and not a PP[de]. 

Consequently, I continue to consider all \emph{de} as prepositions and all \emph{de}-de\-pen\-dents of nouns as PPs with \avm{[marking & de]}, as also suggested by \citet[246--251]{Milner.1978.syntaxe} and \citet{Abeille.2006.AandDe}.  
I also assume that these \emph{de}-dependents are all complements (or at least elements of the \textsc{arg-st} list of nouns), even though the distinction between arguments and adjuncts is even more blurry for dependents of nouns than for dependents of verbs.\largerpage

Prepositions in French (and all Romance languages) cannot be stranded, see (\ref{ex:prep-str}). Extraction out of the \emph{de}-PP is possible, see (\ref{ex:extraction-out-of-de}), but extraction out of other PPs seems very marginal (but compare example (\ref{ex:extraction-pour})).
The NP complement of some prepositions can be left out (\ref{ex:null-pro-avec}). The NP complement of other prepositions cannot: this is the case for \emph{de}, as illustrated by (\ref{ex:null-pro-de}). 

\eal 
\ex[*]{\gll Qui$_i$ as - tu parlé avec~/ de~\trace{}$_i$?\\
who have {} you spoken with of\\
\glt `Who did you speak with / about?' 
\label{ex:prep-str}}
\ex[]{(Chateaubriand, Mémoires d'outre-tombe, 1ère partie, livre 4, 1848)\\
\gll cette déclaration, dont$_i$ je me suis assuré [de la vérité~\trace{}$_i$]\\
this statement of.which I \textsc{refl} have ensured \sbar{}of the truth\\
\glt `this statement, whose truth I verified'
\label{ex:extraction-out-of-de}}
\ex[?]{	\gll l' eau d' irrigation dont$_i$ il plaide [pour la rationalisation [de l' usage~\trace{}$_i$]]\footnotemark\\
the water of irrigation of.which he argues \sbar{}for the rationalization \sbar{}of the use\\
\glt `the irrigation water, whose usage he argues for the rationalization of'
\label{ex:extraction-pour}}
\ex[]{\gll L' ampli de guitare fait des grésillements quand je joue avec.\\
the amp of guitar makes some crackles when I play with\\
\glt `The guitar amp crackles when I play with (it).'
\label{ex:null-pro-avec}}
\ex[*]{\gll J' éteinds la musique quand j' ai pas envie de.\\
I switch.off the music when I have not desire of\\
\glt `I switch off the music when I don't fancy (it).'
\label{ex:null-pro-de}}
\zl
\footnotetext{Source: \url{https://www.djazairess.com/fr/lesoirdalgerie/208772}, last access 05/08/2020.}

Therefore, the \textsc{arg-st} list of prepositions is constrained to only contain objects of type \textit{non-gap} (see \figref{fig:hrch-syssem}). But \emph{de} itself has only \emph{canonical} objects in its \textsc{arg-st} and its \textsc{slash} set is not constrained (it may be non-empty). The other prepositions are defined as having an empty \textsc{slash} set.\footnote{In order for the grammar to account for marginal cases like (\ref{ex:extraction-pour}), the \textsc{slash} set of all prepositions could be left unconstrained.}

To account for the semantics of prepositions, we distinguish meaningful prepositions like (\ref{ex:sur-lexical}) from meaningless prepositions like (\ref{ex:sur-functional}). We assume that many prepositions can be either meaningful or meaningless and have then two different lexical entries (or equivalently, have a lexical entry with a disjunction). 

\eal 
\ex[]{\gll Susanne travaille sur son balcon.\\
Susanne works on her balcony\\
\glt `Susanne is working on her balcony.'}
\label{ex:sur-lexical}
\ex[]{\gll Susanne travaille sur un nouveau projet.\\
Susanne works on a new project\\
\glt `Susanne is working on a new project.'}
\label{ex:sur-functional}
\zl 

The lexical entry of meaningful prepositions has a non-empty \textsc{rels} list, while the lexical entry of meaningless prepositions does not introduce any EP in the \textsc{rels} list. 

Meaningful prepositions 
%\footnote{Some prepositions like English \emph{beforehand} or French \emph{dedans} (`inside') are intransitive. They have then their own \textsc{index} value %, or structure-share its value with the \textsc{index} value of some other element, like \emph{down} in [\emph{turn the volume down}]
%\citep[97--98]{Tseng.2000}. I ignore these special cases here.}
have their own \textsc{index} value (probably an event) and their own \textsc{icons-key}, like any word that introduces an EP.

Meaningless prepositions structure-share their \textsc{index} value with the \textsc{index} value of the NP they subcategorize for. 
I also assume that all meaningless prepositions that take a complement structure-share their \textsc{icons-key} value with the \textsc{icons-key} value of their complement. This latter point conflicts somewhat with the underlying idea that semantically empty elements do not introduce an object of type \emph{info-str} into the \textsc{icons} list of the utterance. But, because prepositions have the same \textsc{index} value as their complement, and by virtue of the Discourse-clash Avoidance Principle (\ref{rule:discourse-clash-avoid}) on page~\pageref{rule:discourse-clash-avoid}, they must have the same \textsc{icons-key} value. Consequently, prepositions do not introduce a new element into the \textsc{icons} list of the utterance, but merely treat the NP as the semantic head of the PP.\largerpage

I assume that \emph{de} is a meaningless preposition with the lexical entry in \figref{avm:de}.\footnote{It could be useful to posit a separate lexical entry for \emph{de} expressing possessive or origin, and consider it as meaningful. But this has no impact on the formalization of the FBC constraint I propose. For the sake of simplicity, I assume only one lexical entry for \emph{de}. 

Furthermore, \citet{Abeille.2006.AandDe} define two usages of the preposition \emph{de}, one in which it is the head, and one in which it is a marker (i.e.\ in \emph{beaucoup de}, `many'). Here, I only consider the former, which is the use I investigated in the empirical parts.} 

\begin{figure}
\avm{[phon & < \type{de} >\\
cat & [head & prep\\
marking & de]\\
cont & [index & \tag{i}\\
        icons-key & \1\\
        rels & <>]\\
arg-st & <[\type*{canonical}\\
           index & \tag{i}\\
           icons-key & \1]>]}
\caption{Lexical entry for \emph{de} (`of')}
\label{avm:de}
\end{figure}

For a preposition like \emph{sur} (`on') that can be used either as a meaningful preposition like in (\ref{ex:sur-lexical}) or as a meaningless preposition like in (\ref{ex:sur-functional}), we can define the lexical entries in Figures~\ref{avm:sur-mf} and~\ref{avm:sur-ml}, which summarizes the assumptions made so far.

\begin{figure}
\small\captionsetup{margin=.05\linewidth}%
\begin{floatrow}
\ffigbox
{\avm{[phon & < \type{sur} >\\
cat & [head & prep]\\
cont & [hook & [index & index\\
        icons-key & info-str]\\
        rels & <[\emph{on\_rel}\\ 
        arg0 & \tag{i}]> ]\\
slash & \{\}\\
arg-st & <[\type*{non-gap}\\
           index & \tag{i}\\
           icons-key & info-str]>]}}
{\caption{Lexical entry for meaningful \emph{sur} (`on')}\label{avm:sur-mf}}
\ffigbox
{\avm{[phon & < \type{sur} >\\
cat & [head & prep]\\
cont & [hook & [index & \tag{i}\\
        icons-key & \1]\\
        rels & <> ]\\
slash & \{\}\\
arg-st & <[\type*{non-gap}\\
           index & \tag{i}\\
           icons-key & \1]>]}}
{\caption{Lexical entry for meaningless \emph{sur} (`on')}\label{avm:sur-ml}}
\end{floatrow}
\end{figure}

The extraction of the PP-dependent of a noun out of NPs takes place in a straightforward manner via the mechanisms of extraction explained earlier. Under the \textsc{nonloc} Amalgamation Principle (\ref{avm:nonloc-amalgamation}), the verb that selects an NP with a \textsc{slash} element inherits this element. There is no difference between extraction of a PP-dependent of a verb and subextraction of a PP-dependent of a noun out of an NP.

\section{The subject as Designated Topic}

One of the central claims in this book is that the phenomenon usually called ``subject island'' is actually the result of a discourse clash: Typically, the subject is the topic of the clause, and focalizing part of the subject leads to a contradiction in that the subject NP is simultaneously treated as part of the Common Ground and as the main information of the sentence (contra Grice's maxim that a sentence should be informative) or even as unknown information (internal contradiction). 

The fact that the subject is the preferred topic of the clause has been captured in several HPSG proposals. \citet{Webelhuth.2007} has a function ``more thematic than'' that yields a hierarchy of thematicity (subject $>$ direct object $>$ oblique object). I adopt here \citegen{Bildhauer.2010} notion ``Designated Topic'', which is based on the verb's preference for a certain argument to be the topic \citep*[see also][Section~5.3.3.2]{Mueller.S.2020?.chapter5}. For example, German \emph{herrschen} (`to reign') in its existential meaning preferably has the locative as its topic. 

\ea \citep[72]{Bildhauer.2010}\nopagebreak\\
\gll Weiterhin Hochbetrieb herrscht am Innsbrucker Eisoval.\\
further high.traffic reigns at.the Innsbruck icerink\\
\glt `It's still all go at the Innsbruck icerink.'
\z 

In \citeauthor{Bildhauer.2010}'s proposal, \textsc{d(esignated)t(opic)} is a head feature of verbs with a list as value, either empty or containing at most one \emph{synsem} object. This object is structure-shared with one element of the \textsc{arg-st} list. For verbs with the subject as default topic, the lexical entry is then the following:\largerpage

\ea Lexical entry for a verb with a subject default Topic: \nopagebreak

\avm{[synsem|loc|cat & [head|dt & <\1>\\
subj & <\1>]\\
arg-st & < \1 > $\oplus$ list]}
\z 

The Designated Topic is realized as topic in sentences that ``involv[e] a Topic-Comment structure plus an assessment of the extent to which the Comment holds of the Topic'' \citep[73]{Bildhauer.2010}. In their model, these sentences are labeled as \emph{a(ssessment)-topic-comment}, which is itself a subtype of \emph{topic-comment}. \citegen{Song.2017} model also provides a hierarchy of the information structure form of sentences. Its supertype is called \emph{sform} (\emph{sentential form}) and the hierarchy is reproduced in \figref{fig:hrch-sform}. Phrases inherit from both the appropriate \emph{headed-structure} and the appropriate \emph{sform}.{\interfootnotelinepenalty=10000\footnote{Separate features ensure that the whole clause keeps the same \emph{sform}. These features are not relevant for my analysis, but see \citet[Chapter~7]{Song.2017} for a detailed explanation.}}

\begin{figure}[ht]
\centering
\begin{forest}
sn edges,
[\textit{sform}
    [\textit{focality}
        [\textit{narrow-focus}
            [\textit{focus-bg}, name = focusbg]
        ]
        [\textit{wide-focus}
            [\textit{all-focus}, name = allfocus]
        ]
    ]
    [\textit{topicality}
        [\textit{topicless}, name = topicless]
        [\textit{topic-comment}
            [\textit{frame-setting}]
            [\textit{non-frame-setting}]
        ]
    ]
]
\draw[thin] (focusbg.north)--(topicless.south);
\draw[thin] (allfocus.north)--(topicless.south);
\end{forest}
    \caption{Type hierarchy of \emph{sform} \citep[125]{Song.2017}}
    \label{fig:hrch-sform}
\end{figure}

I assume that \citegen{Bildhauer.2010} \emph{a-topic-comment} can be directly translated into \citegen{Song.2017} \emph{non-frame-setting}, and propose the following constraint:

\ea
\oneline{
\avm{[\type*{non-frame-setting}\\
      cat|head|dt & < [index & \tag{i}] >\\
      cont|hook & [clause-key & \tag{e}]]}
      \impl
\avm{[c-cont|icons & <[\type*{aboutness-topic}\\
                       target & \tag{i}\\
                       clause & \tag{e}]>]}      
}
\label{avm:rule-dt-is-topic}              
\z 

The implication in (\ref{avm:rule-dt-is-topic}) is that if an AVM has the type \emph{non-frame-setting}, then the Designated Topic is the topic of the clause (i.e.\ an appropriate \textsc{target-clause} pair with the status \emph{aboutness-topic} is introduced in the construction).\footnote{Recall that the Discourse-clash Avoidance Principle (\ref{rule:discourse-clash-avoid}) ensures that an element can have only one discourse status with respect to a clause. It follows that \emph{info-str} element introduced by the Designated Topic must match the one introduced by the construction \emph{non-frame-setting}, except if the element has a discourse status with respect to two or more different clauses.}

\section{Formalization of the Focus-Background Conflict constraint in HPSG}

Recall that in \citeauthor{Song.2017}'s terminology, ``background'' applies to the elements in the utterance that are neither topic nor focus (\figref{fig:hrch-icons}). In my terminology so far, and in the formulation of the FBC constraint in particular, I assumed that the topic belongs to the background (Section~\ref{ch:is}). In order to match \citeauthor{Song.2017}'s terminology, the constraint (\ref{rule:FBC-bis}) can be reformulated as: ``A focused element should not be part of a non-focus constituent.'' 

Another way to formulate this is to say that all dependents of a non-focus word should be non-focus as well, which is exactly the meaning of the  (simplified) formalization in \figref{avm:rule-FBC}. 

\begin{figure}
\caption{Focus-Background Conflict constraint (simplified)}
\label{avm:rule-FBC}
    \begin{tabular}{@{}ll@{}}
	& \avm{
		[\type*{word}\\
		synsem|loc & [cat|head & non-verbal\\
		cont|hook & [icons-key & non-focus\\
		           clause-key & \tag{e}]]]} \\\addlinespace
	\impl & \avm{[arg-str & < ... [\type*{non-focus}\\
		               clause key & \tag{e}] ... >]
                }
    \end{tabular}
\end{figure}

The implication in \figref{avm:rule-FBC} should be understood as follows: a non-verbal word that has non-focus status with respect to a certain clause can only subcategorize for elements that are also non-focus with respect to the same clause. 


% Webelhuth, p310 :
% Preferably, themes are unfocussed.
% Preferably, themes are discourse-familiar.

This applies to all parts of speech except \emph{verbal}: As discussed in Chapter~\ref{ch:fbc-generals}, a verbal element can have non-focus status without constraining its arguments to have non-focus status as well. Complementizers, on the other hand, are semantically empty, and are therefore not affected by the FBC constraint. 

A non-focus noun, for example a topic subject NP as in (\ref{ex:FBC-topic-subject}), can only have non-focus complements, otherwise it would violate the Focus-Background Conflict constraint (\figref{avm:rule-FBC}). This is illustrated by \figref{fig:FBC-topic-subject}.

{\judgewidth{\#}%
\ea[\#]{\gll [l' originalité [de cette innovation]$_F$]$_T$\\
\sbar{}the uniqueness \sbar{}of this innovation\\
\glt `the uniqueness of this innovation'
\label{ex:FBC-topic-subject}}
\z}

\begin{figure}
\oneline{%
\begin{forest}
if n children=0{tier=word}{}
[NP [\avm{\1} Det [l'\\the]]
[N$'$
    [N\\
\avm{[head & noun\\
hook & [icons-key & [\type*{topic}\\
                  clause & \tag{e}\\
                  target & \tag{i}]\\
    clause-key & \tag{e}]\\
arg-str & < \1, \2 > ]}
    [originalité\\uniqueness]]
    [PP \\
\avm{\2 [index & \tag{j}\\
         icons-key & [\type*{focus}\\
         clause & \tag{e}\\
         target & \tag{j}]]}
    [de cette innovation\\of this innovation, roof]]
]]
\end{forest}}
\caption{Simplified tree for the infelicitous NP ``l' originalité [de cette innovation]$_F$]$_T$'' (`the uniqueness of this innovation')}
    \label{fig:FBC-topic-subject}
\end{figure} 

\subsection{Implementing the FBC constraint}

The formalization in \figref{avm:rule-FBC} is sufficient for my demonstration, but it would be insufficient for a direct implementation. Technically, we need to make sure that the \textsc{arg-str} list only contains \textit{info-str} elements that are non-focus with respect to the clause \avm{\tag{e}}, while allowing any other \textit{icons} element that is not of the type \textit{info-str}, and also potentially allowing \textit{info-str} elements that are non-focus with respect to another clause. This point is crudely represented by the dots in \figref{avm:rule-FBC}.

For the reader interested in the technical details of implementing the constraint, here is a method in two steps. First, we can define a function \texttt{non-focus()} that takes as arguments an event and a list of objects of the type \emph{info-str}:

\ea \texttt{non-focus(\avm{\1},\avm{\2[\type*{info-str}\\clause & \3]}|Rest):-}\\ 
if \texttt{\avm{\1} == \avm{\3}}, then \texttt{\avm{\2[\type*{non-focus}]}} and \texttt{non-focus(\avm{\1},Rest).}

\texttt{non-focus(\avm{\1},\avm{<>}).}
\z 

The function \texttt{non-focus()} checks whether the \textsc{clause} value of the first \emph{info-str} object of its second argument is identical with its first argument, and if so, constrains the \emph{info-str} object to be \emph{non-focus}. It then recursively checks each element of the list until the end in the same way.

Second, we may reformulate \figref{avm:rule-FBC} as \figref{avm:rule-FBC-more-exact}.

\begin{figure}[h]
\avm{
      [\type*{word}\\
      synsem|loc & [cat|head & non-verbal\\
      cont|hook & [icons-key & non-focus\\
      clause-key & \tag{e}]]]}
\quad\impl\quad
\parbox[c]{\widthof{non-focus(m, n)}}{\raggedright\avm{[arg-str & \1]} $\wedge$ \\ non-focus(\avm{\tag{e},\1})}
\caption{Focus-Background Conflict constraint}\label{avm:rule-FBC-more-exact}
\end{figure}

I will now consider how the FBC constraint interacts with each of the following constructions: interrogatives, \emph{c'est}-clefts and relative clauses.

\subsection{The FBC constraint in interrogatives}

In standard interrogatives with extraction of the \emph{wh}-word (\emph{standard-wh-inter-cl}), the filler receives a \emph{semantic-focus} interpretation, see (\ref{avm:standard-wh-inter-cl}). For a sentence like (\ref{ex:FBC-topic-subject-interr}), with extraction out of the subject, the consequence is that the Designated Topic cannot be the topic of the utterance as defined in (\ref{avm:rule-dt-is-topic}), because then it would violate the rule in \figref{avm:rule-FBC}. 

\ea[]{\gll [De quelle innovation]$_i$ [l' originalité~\trace{}$_i$] enthousiasme-t-elle mes collègues?\\
\sbar{}of which innovation \sbar{}the uniqueness excites-0-\textsc{3sg.sbj.fem} my colleagues\\
\glt `Of which innovation does the uniqueness excite my colleagues?'}
\label{ex:FBC-topic-subject-interr}
\z 

\begin{sidewaysfigure}[hp]
\resizebox{!}{12cm}{%
\begin{forest}
where n children=0{tier=word}{}
[S\\
\avm{[\type*{standard-wh-inter-cl}\\
      clause-key & \tag{e}\\
      slash & \{\}]}
    [PP\\
    \avm{[loc & \1\\
          index & \tag{i}\\
          icons-key & [\type*{focus}\\
                       target & \tag{i}\\
                       clause & \tag{e}]]}
        [de quelle innovation\\of which innovation, roof]]
    [S\\
    \avm{[\type*{head-subj-structure \& all-focus}\\
          index & \tag{e}\\
          icons & < ... \4 ...>\\
          slash & \{\1\}]}
          [\avm{\2} NP
            [\avm{\3} Det [l'\\the]]
            [N\\
            \avm{[head & noun\\
                hook & [icons-key & \4 [\type*{semantic-focus}\\
                                     clause & \tag{e}\\
                                     target & \tag{j}]\\
                        index & \tag{j}]\\
                slash & \{ \1 \}\\
                arg-str & < \3, [loc & \1] >]}
            [originalité\\uniqueness]]
          ]
          [VP\\
          \avm{[slash & \{\1\}]}
            [V\\
            \avm{[dt & <>\\
                  subj & <\2>\\
                  slash & \{\1\}]} [enthousiasme\\excites]]
            [NP [mes collègues\\my colleagues, roof]]
          ]
    ]
]
\end{forest}}
\caption{Simplified tree for [\textit{De quelle innovation}]$_i$ [\textit{l' originalité}~\trace{}$_i$] \textit{enthousiasme-t-elle mes collègues?} (`Of which inovation does the uniqueness excite my colleagues?')}
\label{fig:FBC-topic-subject-interr}
\end{sidewaysfigure} 

In extraction out of the subject in an interrogative, the subject cannot be topic, i.e.\ either another element is topic (\emph{frame-setting-topic}) or the clause is \emph{topicless}. For example, \citeauthor{Chaves.2013} shows that example (\ref{ex:Chaves2013}) is acceptable in English. In \citet{Abeille.2020.Cognition}, we use a test for topicality in order to show that the clause has an all-focus interpretation, see (\ref{ex:Chaves2013-demo}). 

\ea \citep[313]{Chaves.2013}\\
Which problem will [the solution to~\trace{}] never be found? 
\label{ex:Chaves2013} 
\z 

\eal \label{ex:Chaves2013-demo}
\ex[]{A solution to this problem will never be found.}
\ex[\#]{Speaking of a solution to this problem, it will never be found.}
\zl 

\figref{fig:FBC-topic-subject-interr} shows the analysis of such a case. In this example, the sentence in (\ref{ex:FBC-topic-subject-interr}) is all-focus, as in (\ref{ex:Chaves2013}).

On the other hand, if the filler is not focused, as in rhetorical questions, then the extraction is felicitous, as in example (\ref{ex:internet-ahmadinejad}) reproduced in (\ref{ex:internet-ahmadinejad-ter}).

\ea[]{\gll [De quel pays]$_i$ [la dépense militaire~\trace{}$_i$] dépasse annuellement mille milliards de dollars~[\dots]~?\\
\sbar{}of which country \sbar{}the budget military exceeds yearly thousand billion of dollars\\
\glt `Of which country does the military budget exceed yearly 1000 B.\ dollars?'}
\label{ex:internet-ahmadinejad-ter}
\z 

Notice that the constraint predicts that in languages in which postverbal subjects are focused, interrogatives with extraction out of postverbal subjects should be more felicitous than extraction out of preverbal subjects. Spanish, for example, is such a language.
%amelioration in French too?
%\# De quel fruit (est ce que) le gout te plaît ?\\
% De quel fruit (est-ce que) te plaît le gout ?

As such, the constraint in \figref{avm:rule-FBC} predicts that an interrogative with extraction out of the subject in a long-distance dependency is felicitous. A \emph{standard-wh-cl} constrains the filler from bearing focus with respect to the main clause (the \textsc{head-dtr}), see (\ref{avm:standard-wh-inter-cl}). In this case, the value of the embedded subject's \textsc{icons-key|clause} does not match the value of the filler's \textsc{icons-key|clause}, and \figref{avm:rule-FBC} is not violated.
And indeed, I argued in Section~\ref{ch:analysis-ldd} that I do not have evidence that focalization involving a long-distance dependency violates the FBC constraint.

\figref{fig:FBC-topic-subject-interr-ldd} illustrates the HPSG analysis for an interrogative with long-distance dependency in which the filler is focus and the subject of the embedded clause topic. There is no violation of \figref{avm:rule-FBC}.

\ea[]{\gll [De quelle innovation]$_i$ suppose-t-il [que [l' originalité~\trace{}$_i$] enthousiasme mes collègues]~?\\
\sbar{}of which innovation supposes-0-\textsc{3sg.sbj.masc} \sbar{}that \sbar{}the uniqueness excites my colleagues\\
\glt `Of which innovation does he suppose that the uniqueness excites my colleagues?'}
\label{ex:FBC-topic-subject-interr-ldd}
\z 

\begin{sidewaysfigure}[hp]
\resizebox{!}{12cm}{%
\begin{forest}
where n children=0{tier=word}{}
[S\\
\avm{[\type*{standard-wh-inter-cl}\\
      clause-key & \tag{e1}\\
      slash & \{\}]}
    [PP\\
    \avm{[loc & \1\\
          index & \tag{i}\\
          icons-key & [\type*{focus}\\
                       target & \tag{i}\\
                       clause & \tag{e1}]]}
        [de quelle innovation\\of which innovation, roof]]
    [S\\
    \avm{[\type*{noncomp-comps-structure}\\
          slash & \{\1\}]}
    [V\\
    \avm{[arg-st & < NP[\type{aff}], \2 >]}
    [suppose-t-il\\supposes]]
    [\avm{\2} S\\
    \avm{[\type*{sent-comp}\\
          hook & [index & \tag{e2}\\
                  clause-key & \tag{e2}]\\
          slash & \{\1\}]}
    [COMP [que\\that]]
    [S\\
    \avm{[\type*{head-subj-structure \& non-frame-setting}\\
          index & \tag{e2}\\
          icons & < ... \5 ...>\\
          slash & \{\1\}]}
          [\avm{\3} NP
            [\avm{\4} Det [l'\\the]]
            [N\\
            \avm{[head & noun\\
                hook & [icons-key & \5 [\type*{aboutness-topic}\\
                                     clause & \tag{e2}\\
                                     target & \tag{j}]\\
                        index & \tag{j}]\\
                slash & \{ \1 \}\\
                arg-str & < \4, [loc & \1] >]}
            [originalité\\uniqueness]]
          ]
          [VP\\
          \avm{[slash & \{\1\}]}
            [V\\
            \avm{[dt & <\3>\\
                  spr & <\3>\\
                  slash & \{\1\}]} [enthousiasme\\excites]]
            [NP [mes collègues\\my colleagues, roof]]
          ]
    ]
    ]
    ]
]
\end{forest}}
\caption{Simplified tree for [\textit{De quelle innovation}]$_i$ \textit{suppose-t-il} [\textit{que} [\textit{l'originalité}~\trace{}$_i$] \textit{enthousiasme mes collègues}]\textit{?} (`Of which innovation does he suppose that the uniqueness excites my colleagues?')}
    \label{fig:FBC-topic-subject-interr-ldd}
\end{sidewaysfigure}

\subsection{The FBC constraint in \emph{c'est}-clefts}

The FBC constraint makes the same predictions for \emph{it}-clefts as for interrogatives involving focalization. \figref{fig:FBC-topic-subject-cleft} shows a case of \emph{c'est}-clefts with extraction out of a subject. Even though it is extraction out of a sentential complement, the structure is monoclausal and the subject cannot be topic, otherwise \figref{avm:rule-FBC} would be violated. 

\ea[]{\gll C' est [de cette innovation]$_i$ [que [l' originalité~\trace{}$_i$] enthousiasme mes collègues].\\
it is \sbar{}of this innovation \sbar{}that the uniqueness excites my colleagues\\
\glt `It's of this innovation that the uniqueness excites my colleagues.'}
\label{ex:FBC-topic-subject-cleft}
\z 

\begin{sidewaysfigure}[htbp]
\scalebox{0.7}{
\begin{forest}
where n children=0{tier=word}{}
[S\\
\avm{[\type*{head-subj-structure}\\
      clause-key & \tag{e}\\
      slash & \{\}]}
    [NP [c'\\it]]
    [VP\\
    \avm{[\type*{noncomp-comps-structure}\\
          clause-key & \tag{e}\\
          slash & \{\}]}
    [VP\\
    \avm{[\type*{noncomp-comps-structure}\\
          clause-key & \tag{e}\\
          slash & \{\}]}
    [V\\
    \avm{[clause-key & \tag{e}\\
    slash & \{\}]}
    [est\\is]]
    [PP\\
    \avm{[loc & \1\\
          index & \tag{i}\\
          icons-key & [\type*{focus}\\
                       target & \tag{i}\\
                       clause & \tag{e}]]}
        [de cette innovation\\of this innovation, roof]]
    ]
    [\avm{\2} S\\
    \avm{[\type*{sent-comps}\\
          hook & [index & \tag{e}\\
                  clause-key & \tag{e}]\\
          slash & \{\1\}]}
    [COMP [que\\that]]
    [S\\
    \avm{[\type*{head-subj-structure \& all-focus}\\
          index & \tag{e}\\
          icons & < ... \3 ...>\\
          slash & \{\1\}]}
          [\avm{\4} NP
            [\avm{\5} Det [l'\\the]]
            [N\\
            \avm{[head & noun\\
                hook & [icons-key & \3 [\type*{semantic-focus}\\
                                     clause & \tag{e}\\
                                     target & \tag{j}]\\
                        index & \tag{j}]\\
                slash & \{ \1 \}\\
                arg-str & < \5, [loc & \1] >]}
            [originalité\\uniqueness]]
          ]
          [VP\\
          \avm{[slash & \{\1\}]}
            [V\\
            \avm{[dt & <>\\
                  subj & <\4>\\
                  slash & \{\1\}]} [enthousiasme\\excites]]
            [NP [mes collègues\\my colleagues, roof]]
          ]
    ]
    ]
    ]
]
\end{forest}}
\caption{Simplified tree for \textit{C'est} [\textit{de cette innovation}]$_i$ [\textit{que} [\textit{l'originalité}~\trace{}$_i$] \textit{enthousiasme mes collégues}]. (`It's of this innovation that the uniqueness excites my colleagues.')}
    \label{fig:FBC-topic-subject-cleft}
\end{sidewaysfigure}



%contradiction with a previously salient focal alternative (but contrariness and not contradiction: doxatic dimension) -- but the contradiction is not required ``the presence of a clearly identifiable alternative (set) is not required''
%3 points in contrast: `` the size of the alternative set, the identifiability of its elements, and the exclusion requirement of the alternatives'' \citep[12]{Destruel.2019}
%\citet{Repp.2016}: not only alternative, but type of discourse relation plays a role in contrast (gradation : non-contrastive, oppose, correction). In non-contrastive, we have alternatives which can be compatible. In an oppose relation, the contribution is different. In a correction relation, they are not mutualy compatible.  
            % Repp, S. (2016). “Contrast: dissecting an elusive information-structural notion and its role in grammar,” in Handbook of Information Structure, eds C. Féry and S. Ishihara (Oxford: Oxford University Press), 270–289.
% mirative focus (Cruschina 2012): new information that is particularly surprising or unexpected to the hearer
            % Cruschina, S. (2012). Discourse-Related Features and Functional Projections. Oxford: Oxford University Press.
            
Notice that an interrogative with extraction out of the pivot is expected to be felicitous under the FBC constraint: the pivot is focused and so is the \emph{wh}-phrase, therefore there is no discourse clash.

\eal 
\ex[]{[Which car]$_i$ is it [the color of~\trace{}$_i$]$_j$ that you loved~\trace{}$_j$ most?}
\ex[]{\gll [De quelle innovation]$_i$ est - ce [l' originalité~\trace{}$_i$]$_j$ que mes collègues apprécient~\trace{}$_j$~?\\
\sbar{}of which innovation is {} it \sbar{}the uniqueness that my colleagues appreciate\\
\glt `Of which innovation is it the uniqueness that my colleagues appreciate?'}
\zl 

Extraction out of the \emph{que}-clause falls under the same constraints as any extraction in a long-distance dependency. An interrogative involving extraction out of the subject would not violate \figref{avm:rule-FBC}. Relativization out of the \emph{que}-clause is also expected to be felicitous.

\subsection{The FBC constraint in relatives}

Finally, the FBC constraint has no impact on  relative clauses. \figref{fig:FBC-topic-subject-rc} on page~\pageref{fig:FBC-topic-subject-rc} illustrates relativization out of a topic subject in a \emph{comp-rel-cl} (with a complementizer). In this case, the slashed element is never realized; its \textsc{index} value is only structure-shared with the \textsc{index} value of the noun modified by the relative clause. The implication in \figref{avm:rule-FBC} constrains the \textsc{icons-key} value of the slashed element to be \emph{non-focus}, but it is completely underspecified in other respects. 

\ea
\gll une innovation [dont$_i$ [l' originalité~\trace{}$_i$] enthousiasme mes collègues]\\
an innovation \sbar{}of.which \sbar{}the uniqueness excites my colleagues\\
\glt `an innovation of which the uniqueness excites my colleagues'
\label{ex:FBC-topic-subject-rc}
\z 



\begin{sidewaysfigure}[hp]
\resizebox{!}{12cm}{%
\begin{forest}
where n children=0{tier=word}{}
[NP\\
\avm{[\type*{head-mod-structure}\\
      clause-key & \tag{e1}]}
[NP\\
\avm{[synsem & \1 [index & \tag{i}\\
                clause-key & \tag{e1}]]}
[une innovation\\an innovation, roof]]
[S\\
\avm{[\type*{comp-rel-cl}\\
      mod & \1\\
      clause-key & \tag{e}\\
      c-cont|icons & <[\type*{topic}\\
                        target & \tag{i}\\
                        clause & \tag{e2}]>\\
      slash & \{\}]}
    [COMP [dont\\of.which]]
    [S\\
    \avm{[\type*{head-subj-structure \& non-frame-setting}\\
          index & \tag{e}\\
          icons & < ... \3 ...>\\
          slash & \{\2 [index & \tag{i}]\}]}
          [\avm{\4} NP
            [\avm{\5} Det [l'\\the]]
            [N\\
            \avm{[head & noun\\
                hook & [icons-key & \3 [\type*{aboutness-topic}\\
                                     clause & \tag{e2}\\
                                     target & \tag{j}]\\
                        index & \tag{j}]\\
                slash & \{ \2 \}\\
                arg-str & < \5, [loc & \2] >]}
            [originalité\\uniqueness]]
          ]
          [VP\\
          \avm{[slash & \{\2\}]}
            [V\\
            \avm{[dt & <\4>\\
                  spr & <\4>\\
                  slash & \{\2\}]} [enthousiasme\\excites]]
            [NP [mes collègues\\my colleagues, roof]]
          ]
    ]
]
]
\end{forest}}
\caption{Simplified tree for \textit{une innovation} [\textit{dont}$_i$ [\textit{l'originalité}~\trace{}$_i$] \textit{enthousiasme mes collégues}] (`an innovation of which the uniqueness excites my colleagues')}
    \label{fig:FBC-topic-subject-rc}
\end{sidewaysfigure} 

\figref{fig:FBC-topic-subject-rc-dequi} on page \pageref{fig:FBC-topic-subject-rc-dequi} illustrates extraction out of the topic subject of a \emph{wh-rel-cl} (with pronominal filler). In this particular example, given in (\ref{ex:FBC-topic-subject-rc-dequi}), the filler is [P + pronoun]. The FBC constraint does not constrain the elements of the \textsc{arg-st} list, because relative pronouns are informatively empty \citep[112]{Song.2017} and therefore their \textsc{icons-key} value is \emph{i-empty}.

\ea
\gll un avocat [[de qui]$_i$ [l' associé~\trace{}$_i$] aide mon cousin]\\
a lawyer \ssbar{}of who \sbar{}the associate helps my cousin\\
\glt `a lawyer of whom the associate helps my cousin'
\label{ex:FBC-topic-subject-rc-dequi}
\z 

\begin{sidewaysfigure}[hp]
\resizebox{!}{12cm}{%
\begin{forest}
where n children=0{tier=word}{}
[NP\\
\avm{[\type*{head-mod-structure}\\
      clause-key & \tag{e1}]}
[NP\\
\avm{[synsem & \1 [index & \tag{i}\\
                clause-key & \tag{e1}]]}
[un avocat\\a lawyer, roof]]
[S\\
\avm{[\type*{wh-rel-cl}\\
      mod & \1\\
      clause-key & \tag{e}\\
      c-cont|icons & <[\type*{topic}\\
                        target & \tag{i}\\
                        clause & \tag{e2}]>\\
      slash & \{\}]}
    [PP \\
    \avm{[loc & \2 [icons-key & \3]]}
    [P\\
    \avm{[icons-key & \3]}
    [de\\of]]
    [NP\\
    \avm{[icons-key & \3 i-empty]}
    [qui\\who]]
    ]
    [S\\
    \avm{[\type*{head-subj-structure \& non-frame-setting}\\
          index & \tag{e}\\
          icons & < ... \4 ...>\\
          slash & \{\2 [index & \tag{i}]\}]}
          [\avm{\5} NP
            [\avm{\6} Det [l'\\the]]
            [N\\
            \avm{[head & noun\\
                hook & [icons-key & \4 [\type*{aboutness-topic}\\
                                     clause & \tag{e2}\\
                                     target & \tag{j}]\\
                        index & \tag{j}]\\
                slash & \{ \2 \}\\
                arg-str & < \6, [loc & \2] >]}
            [associé\\associate]]
          ]
          [VP\\
          \avm{[slash & \{\2\}]}
            [V\\
            \avm{[dt & <\5>\\
                  spr & <\5>\\
                  slash & \{\2\}]} [aide\\helps]]
            [NP [mon cousin\\my cousin, roof]]
          ]
    ]
]
]
\end{forest}}
\caption{Simplified tree for \textit{un avocat} [[\textit{de qui}]$_i$ [\textit{l'associé}~\trace{}$_i$] \textit{aide mon cousin}] (`a lawyer of who the associate helps my cousin')}
    \label{fig:FBC-topic-subject-rc-dequi}
\end{sidewaysfigure} 

Another case of extraction out of the topic subject of a \emph{wh-rel-cl} is illustrated by \figref{fig:FBC-topic-subject-rc-PPdequi} on page \pageref{fig:FBC-topic-subject-rc-PPdequi}. Here, the relative pronoun is embedded in an NP, and the \textsc{icons-key} value of the filler is constrained by the implication in \figref{avm:rule-FBC} to be \emph{non-focus}. It is otherwise underspecified. 

\ea
\gll Christelle, [[de la soeur de qui]$_i$ [l' arrogance~\trace{}$_i$] rebute mes collègues]\\
Christelle \ssbar{}of the sister of who \sbar{}the arrogance repels my colleagues\\
\glt `Christelle, whose sister's arrogance repels my colleagues'
\label{ex:FBC-topic-subject-rc-PPdequi}
\z 



\begin{sidewaysfigure}[hp]
\resizebox{!}{12cm}{%
\begin{forest}
where n children=0{tier=word}{}
[NP\\
\avm{[\type*{head-mod-structure}\\
      clause-key & \tag{e1}]}
[NP\\
\avm{[synsem & \1 [index & \tag{i}\\
                clause-key & \tag{e1}]]}
[Christelle\\Christelle]]
[S\\
\avm{[\type*{wh-rel-cl}\\
      mod & \1\\
      clause-key & \tag{e}\\
      c-cont|icons & <[\type*{topic}\\
                        target & \tag{i}\\
                        clause & \tag{e2}]>\\
      slash & \{\}]}
    [PP \\
    \avm{[loc & \2 [icons-key & \3]]}
    [P\\
    \avm{[icons-key & \3]}
    [de\\of]]
    [NP\\
    \avm{[icons-key & \3 non-focus]}
    [la soeur de qui\\the sister of who, roof]]
    ]
    [S\\
    \avm{[\type*{head-subj-structure \& non-frame-setting}\\
          index & \tag{e}\\
          icons & < ... \4 ...>\\
          slash & \{\2 [index & \tag{i}]\}]}
          [\avm{\5} NP
            [\avm{\6} Det [l'\\the]]
            [N\\
            \avm{[head & noun\\
                hook & [icons-key & \4 [\type*{aboutness-topic}\\
                                     clause & \tag{e2}\\
                                     target & \tag{j}]\\
                        index & \tag{j}]\\
                slash & \{ \2 \}\\
                arg-str & < \6, [loc & \2] >]}
            [arrogance\\arrogance]]
          ]
          [VP\\
          \avm{[slash & \{\2\}]}
            [V\\
            \avm{[dt & <\5>\\
                  subj & <\5>\\
                  slash & \{\2\}]} [rebute\\repels]]
            [NP [mes collègues\\my colleagues, roof]]
          ]
    ]
]
]
\end{forest}}
\caption{Simplified tree for \textit{Christelle,} [[\textit{de la soeur de qui}]$_i$ [\textit{l'arrogance}~\trace{}$_i$] \textit{rebute mes collègues}] (`Christelle, whose sister's arrogance repels my colleagues')}
    \label{fig:FBC-topic-subject-rc-PPdequi}
\end{sidewaysfigure} 


\chapter{Extraction out of infinitival and sentential subjects}
\label{ch:hpsg-sent-subj}
Infinitival and sentential subjects receive a simple analysis in HPSG, and there is not much HPSG literature about them. Their impersonal variants like (\ref{ex:extraposed-sentential-subject}) have attracted more attention.

\ea \citep[example from COCA cited by][72]{Lee.2018}\\
It  was  assumed  [that  the  teachers  answered  all  written  and  oral questions honestly]. 
\label{ex:extraposed-sentential-subject}
\z 

As previously said in Section~\ref{ch:hpsg-basics-ldd}, I assume that sentential and infinitival subjects are licensed for some verbs, similarly to sentential and infinitival complements, as is commonly assumed in HPSG. A lexical rule allows NP elements in \textsc{arg-st} to be an S or a VP when the element may refer to a situation or an event. The subject of \emph{bark}, for example, cannot be sentential nor infinitival, but the subject of \emph{annoy} can. Sentential subjects are finite (S [\textsc{marking} \textit{que}]) and infinitival subjects are non-finite (VP [\textsc{vform} \textit{infinitive}]).

In Section~\ref{ch:exp-conclusion-CP-subject}, I have argued~-- based on the results of Experiments~15 and 16 and following \citegen{Kluender.2004} proposal~-- that the effects observed in relativizing out of infinitival subjects may be best explained by processing factors. Overall, the experiments show that native speakers do not strongly reject these relativizations. The specificity of the filler, the complexity of the subject and probably many other intervening factors play a role in the finding that some extractions out of verbal subjects are very unnatural, and therefore received degraded acceptability judgments. Sentential subjects and infinitival subjects are not common: in French, they are very rare in the corpora (\citet{Abeille.2019.FTB} found only 24 sentential subjects and 99 infinitival subjects in the French Treebank; \citet[153]{Berard.2012} found in her corpora only complex NPs and no sentential or infinitival subject). Sentential subjects are harder to process than sentential complements \citep{Frazier.1988} and seem to require pragmatic licensing (according to \citet[685]{Miller.2001}, English sentential subjects are only felicitous if their content is ``discourse-old or inferrable'').
% Eye-tracking experiments conducted by Frazier and Rayner (1988) indicate that sentential subjects were harder to process than their extraposed sentential subject analog.
My HPSG proposal does not account for these effects and licenses extraction out of sentential subjects. Hence, there is no constraint on the \textsc{slash} set of sentential and infinitival subjects. 

%Smolka 2005

% [That Kim was late annoyed Lee.] -> [It annoyed Lee that Kim was late.] : lexical rule in Pollard & Sag 1994 page 150

%Koster jan 1978 Why subject sentences don't exist
%davies william d & stanley dubinsky 2000 why sentential subjects do so exist 

\section{Sentential subjects}

\citet{Erteschik-Shir.1973} has shown that sentential subjects are backgrounded -- i.e.\ \emph{non-focus} -- with respect to the main clause \citep[see also later][]{Goldberg.2006}. Similarly, \citet{Lee.2018} claims that the way to focalize the sentential subject in English is to turn it into a sentential complement in an impersonal construction like (\ref{ex:extraposed-sentential-subject}).

Hence, the lexical rule which allows NP elements in \textsc{arg-st} to be an S [\textsc{marking} \textit{que}] also assigns a \emph{non-focus} value to the embedded clause with respect to the embedding clause whenever it is the subject. This rule may lead to \figref{ex:avm-agacer} for the verb \emph{agacer} (`to annoy').

\begin{figure}[h]

\avm{[phon & < \type{agac-} >\\
cont & [hook & [clause-key & \tag{e1}\\
                icons-key & [target & \tag{e2}\\
                             clause & \tag{e1}]\\
                index & \tag{e2}]\\
        icons & < [\type*{non-focus}\\
                  target & \tag{e3}\\
                  clause & \tag{e2}] > $\oplus$ \type{lnis}]\\
arg-st & < S[marking & que\\
                                        mod & none\\
                                        index & \tag{e3}], NP[clause-key & \tag{e1}] >]}
\caption{Lexical item for \emph{agacer} (`to annoy') with  sentential subject}
\label{ex:avm-agacer}
\end{figure}

The FBC constraint in \figref{avm:rule-FBC} makes no particular prediction with respect to sentential subjects, because it does not apply to \emph{verbal} heads.
Still, extraction out of the sentential subject seems less felicitous in interrogatives and \emph{it}-clefts than in relative clauses. Example (\ref{ex:CP-subject-finite-interr-extr}) is an interrogative with extraction out of the sentential subject of (\ref{ex:CP-subject-finite-interr}). 

\eal
\ex[]{\gll [Que Kim parle à Frank] aga\c{c}ait Lee.\\
\sbar{}that Kim talked\textsc{.subj} at Frank annoyed Lee\\
\glt `That Kim talked to Frank annoyed Lee.'}
 \label{ex:CP-subject-finite-interr}
\ex[*]{\gll À qui [que Kim parle~\trace{}] aga\c{c}ait Lee?\\
at who that Kim talked\textsc{.subj} annoyed Lee\\
\glt `Who did that Kim talked to annoy Lee?' (intended: Who did it annoy Lee that Kim talked to?)}
\label{ex:CP-subject-finite-interr-extr}
\zl 

The reason why (\ref{ex:CP-subject-finite-interr-extr}) is degraded has probably nothing to do with the FBC constraint. \citet{Davies.2009} argue that such extractions may be difficult because they are a case of center-embedding. Center-embedded structures, albeit grammatical, are very difficult to parse. \citet{Fodor.2017} observe that typical center-embedded sentences become more acceptable with the right prosody, as in (\ref{ex:Fodor}) where || indicates that a pause is inserted. It would be interesting to see if similar results can be achieved on sentences like (\ref{ex:CP-subject-finite-interr-extr}). \citet{Chaves.2012} shows that it is at least the case for extractions out of subject NPs. 

\eal 
\ex \citep[ex.\ 11]{Fodor.2017}\\
The elegant woman || that the man I love met || moved to Barcelona.
\label{ex:Fodor}
\ex \citep[ex.\ 56a]{Chaves.2012}\\
Which book || did a review of || appear in the Times?
\zl 

Example (\ref{ex:CP-subject-finite-cleft-extr}) is a \emph{c'est}-cleft with extraction out of the sentential subject of (\ref{ex:CP-subject-finite-interr}). It is also unacceptable.

\ea[*]{\gll C' est à Frank [que [que Kim parle~\trace{}] aga\c{c}ait Lee].\\
it is at Frank \sbar{}that \sbar{}that Kim talked\textsc{.subj} annoyed Lee\\
\glt `It's Frank that that Kim talked to annoyed Lee.' (intended: It's Frank that it annoyed Lee that Kim talked to.)}
\label{ex:CP-subject-finite-cleft-extr}
\z  

In this case, the reason is probably a general ban on repeating the complementizer, which is probably ruled out for processing reasons as well. For example, a sentential subject cannot contain a sentential subject, as in (\ref{ex:recursive-sentential-subject}). 

\eal \label{ex:recursive-sentential-subject}
\ex[*]{[That [that Kim was late] annoyed Lee] is not a secret.}
\ex[*]{\gll [Que [Que Kim parle à Frank] aga\c{c}ait Lee] était connu de tous.\\
\sbar{}that \sbar{}that Kim talked\textsc{.subj} at Frank annoyed Lee was known of all\\
\glt `That that Kim talked to Frank annoyed Lee was known by all.' (intended: It was know by all that it annoyed Lee that Kim talked to Frank)}
\zl 

Therefore, this problem has nothing to do with extraction. It is also not restricted to sentential subjects, as shown by (\ref{ex:that-that}). 

\ea[??]{I understood [that [that Kim was late] annoyed Lee].}
\label{ex:that-that}
\z  

\section{Infinitival subject}

Infinitival arguments are defined as VP, hence they are not independent clauses as far as information structure is concerned. As explained earlier, they share the value of their \textsc{clause-key} feature with the value of the \textsc{clause-key} feature of the embedding verb, like NP arguments (in contrast to sentential complements, which form a clause and whose \textsc{clause-key} value is not structure-shared with the one of the embedding verb). Compare the lexical entry in \figref{ex:avm-agacer} with the lexical entry in \figref{ex:avm-agacer-inf}.

\begin{figure}[h]
\avm{[phon & < \type{agac-} >\\
cont & [hook & [clause-key & \tag{e1}\\
                icons-key & [\type*{info-str}\\
                             target & \tag{e2}\\
                             clause & \tag{e1}]\\
                index & \tag{e2}]\\
        icons & < [\type*{info-str}\\
                  target & \tag{e3}\\
                  clause & \tag{e2}] > $\oplus$ \type{lnis}]\\
arg-st & < VP[vform & non-finite\\
            index & \tag{e3}\\
            clause-key & \tag{e1}], NP[clause-key & \tag{e1}] >]}
\caption{Lexical entry for \emph{agacer} (`to annoy') with infinitival subject}
\label{ex:avm-agacer-inf}
\end{figure}

\figref{fig:CP-inf-ou} illustrates the HPSG analysis of a relative clause with a relative pronoun that contains extraction out of the infinitival subject. An interrogative would be similar, except for the information structure of the filler, which would be \emph{focus}. Focalizing part of a \emph{non-focus} infinitival subject is not restricted by the FBC constraint, because the infinitival subject has the feature \avm{[head & verb]}. \figref{fig:CP-inf-que} on page~\pageref{fig:CP-inf-que} shows the HPSG analysis of a relative clause with a complementizer that contains extraction out of an infinitival subject. 

\eal 
\ex[]{\gll Amsterdam, [où$_i$ [flâner~\trace{}$_i$] est charmant]\\
Amsterdam \sbar{}where \sbar{}wander\textsc{.inf} is charming\\
\glt `Amsterdam, where wandering is charming'} 
\label{ex:CP-inf-ou}
\ex[]{\gll Amsterdam, [qu$_i$' [observer~\trace{}$_i$] est charmant]\\
Amsterdam \sbar{}that \sbar{}observe\textsc{.inf} is charming\\
\glt `Amsterdam, observing which is charming'}
\label{ex:CP-inf-que}
\zl

\begin{figure}[h]
\oneline{%
\begin{forest}
where n children=0{tier=word}{}
[NP\\
\avm{[\type*{head-mod-structure}\\
      clause-key & \tag{e1}]}
[NP\\
\avm{[synsem & \1 [index & \tag{i}\\
                clause-key & \tag{e1}]]}
[Amsterdam\\Amsterdam]]
[S\\
\avm{[\type*{wh-rel-cl}\\
      mod & \1\\
      clause-key & \tag{e}\\
      c-cont|icons & <[\type*{topic}\\
                        target & \tag{i}\\
                        clause & \tag{e2}]>\\
      slash & \{\}]}
    [PP \\
    \avm{[loc & \2 [icons-key & \3]]}
    [où\\where]
    ]
    [S\\
    \avm{[\type*{head-subj-structure \& non-frame-setting}\\
          index & \tag{e}\\
          icons & < ... \4 ...>\\
          slash & \{\2 [index & \tag{i}]\}]}
          [\avm{\5} VP\\
            \avm{[head & verb\\
                hook & [icons-key & \4 [\type*{aboutness-topic}\\
                                     clause & \tag{e2}\\
                                     target & \tag{e3}]\\
                        index & \tag{e3}]\\
                slash & \{ \2 \}\\
                arg-str & < \6, [loc & \2] >]}
            [flâner\\wander\textsc{.inf}]]
          [VP\\
          \avm{[slash & \{\2\}]}
            [V\\
            \avm{[dt & <\5>\\
                  spr & <\5>\\
                  slash & \{\2\}]} [est\\is]]
            [AdjP [charmant\\charming]]
          ]
    ]
]
]
\end{forest}}
\caption{Simplified tree for \textit{Amsterdam,} [\textit{où}$_i$ [\textit{flâner}~\trace{}$_i$] \textit{est charmant}] (`Amsterdam, where wandering is charming')}
\label{fig:CP-inf-ou}
\end{figure}

\begin{figure}[hp]
\oneline{%
\begin{forest}
where n children=0{tier=word}{}
[NP\\
\avm{[\type*{head-mod-structure}\\
      clause-key & \tag{e1}]}
[NP\\
\avm{[synsem & \1 [index & \tag{i}\\
                clause-key & \tag{e1}]]}
[Amsterdam\\Amsterdam]]
[S\\
\avm{[\type*{comp-rel-cl}\\
      mod & \1\\
      clause-key & \tag{e}\\
      c-cont|icons & <[\type*{topic}\\
                        target & \tag{i}\\
                        clause & \tag{e2}]>\\
      slash & \{\}]}
    [COMP [qu'\\that]]
    [S\\
    \avm{[\type*{head-subj-structure \& non-frame-setting}\\
          index & \tag{e}\\
          icons & < ... \3 ...>\\
          slash & \{\2 [index & \tag{i}]\}]}
          [\avm{\4} VP\\
            \avm{[head & verb\\
                hook & [icons-key & \3 [\type*{aboutness-topic}\\
                                     clause & \tag{e2}\\
                                     target & \tag{e3}]\\
                        index & \tag{e3}]\\
                slash & \{ \2 \}\\
                arg-str & < \5, [loc & \2] >]}
            [observer\\observe\textsc{.inf}]]
          [VP\\
          \avm{[slash & \{\2\}]}
            [V\\
            \avm{[dt & <\4>\\
                  spr & <\4>\\
                  slash & \{\2\}]} [est\\is]]
            [NP [charmant\\charming]]
          ]
    ]
]
]
\end{forest}}
\caption{Simplified tree for \textit{Amsterdam,} [\textit{qu}$_i$'[\textit{observer}~\trace{}$_i$] \textit{est charmant}] (`Amsterdam, that observing is charming')}
    \label{fig:CP-inf-que}
\end{figure} 




\chapter{Conclusion}

In this last part, I discussed in more detail the FBC constraint and its implications, and proposed a formal analysis in HPSG. In the process, I sketched a substantial fragment of a French grammar in HPSG. I showed a cross-classification of interrogatives \citep[inspired by][]{Ginzburg.2000} and of relative clauses \citep[based on][]{Abeille.2007.Relatives}, while \emph{c'est}-clefts are formed by specific lexical entries of \emph{être} (`be') \citep{Winckel.2020}. I explained how information structure interacts with syntax in these short- and long-distance dependencies. The impression of a ``subject island'' in focalizing constructions arises from the fact that subjects are preferred topics for the majority of verbs. I modeled this preference by treating subjects of such verbs as the \textsc{designated topic}, following a proposal by \citet{Bildhauer.2010}. This allowed me to then propose a formalization of the FBC constraint (\ref{avm:rule-FBC}).
% This bears some resemblance with Webelhuth \citet{Webelhuth.2007}'s Preference Principle: A topic should not contain any focused element.
% Ressemblance with Bresnan & McHomb too

When reflecting on all this evidence showing the non-existence of subject islands, one cannot help but ask why so many linguists have postulated a subject island constraint. I suppose it is possible to see this as a consequence of the fact that most of the examples on which the discussion was based were interrogatives, as were most of the previous experiments.

One major element is missing in my HPSG analysis. In Section~\ref{ch:general-analysis-gradability}, I emphasized that the FBC constraint presupposes gradient information structure. Besides \citegen{Webelhuth.2007} ``more thematic than'' function (which, I think, is not compatible with \citegen{Song.2017} representation of information structure), there has been no attempt in HPSG to capture this gradience. As the goal of the present work was not to determine how and why an element is more or less focus, topic or backgrounded, it can therefore not contribute much to constructing such a model. However, presumably several factors contribute to making an element more or less focused, topicalized or backgrounded. For example, the topic of an embedded clause may be less topical than the topic of the embedding clause. It would then follow that the FBC constraint applies less strongly to elements that are less backgrounded, and this could explain why we do not see superadditive effects in the interrogatives with long-distance dependencies. For the time being, this is unfortunately only speculation. Once the factors have been identified, it should be possible to treat them as weighted constraints and attempt to model the gradience of information structure. Some proposals have been made to introduce weighted constraints in HPSG (in the \emph{Verbmobil} project \citep{Mueller.S.2000}, in \citet{Brew.1995}, in \citet{Guzman.2015}, or in \citet{An.2020}, see \citet[164]{Mueller.S.2017}), but their aim was to compute probabilistic models of sentence processing (production and comprehension)~-- for example the probability of using one or the other dative construction for ditransitive verbs in English (\emph{give Sandra the book / give the book to Sandra}). Any effort to adapt these methods and propose a new way to treat information structure as a gradient feature instead of a categorical feature on lexical signs would contribute greatly to the scientific debate. However, such an attempt goes far beyond the scope of this work.
% Optimality Theory (see especially Keller 2000, 2003)
% Harmonic Grammar, d'après ce que je comprends, est une version de Optimality Theory qui utilise des weighted constraints

%``The challenge is to develop a grammatical framework that is permissive enough to account for  gradient  data  without  idealizing  it,  but  restrictive  enough  to  allow  us  to  formulate precise, testable linguistic analyses.'' \citep[1499]{Sorace.2005}
