Infinitival and sentential subjects receive a simple analysis in HPSG, and there is not much HPSG literature about them. Their impersonal variants like (\ref{ex:extraposed-sentential-subject}) have attracted more attention.

\ea \citep[example from COCA cited by][72]{Lee.2018}\\
It  was  assumed  [that  the  teachers  answered  all  written  and  oral questions honestly]. 
\label{ex:extraposed-sentential-subject}
\z 

As previously said in Section~\ref{ch:hpsg-basics-ldd}, I assume that sentential and infinitival subjects are licensed for some verbs, similarly to sentential and infinitival complements, as is commonly assumed in HPSG. A lexical rule allows NP elements in \textsc{arg-st} to be an S or a VP when the element may refer to a situation or an event. The subject of \emph{bark}, for example, cannot be sentential nor infinitival, but the subject of \emph{annoy} can. Sentential subjects are finite (S [\textsc{marking} \textit{que}]) and infinitival subjects are non-finite (VP [\textsc{vform} \textit{infinitive}]).

In Section~\ref{ch:exp-conclusion-CP-subject}, I have argued~-- based on the results of Experiments~15 and 16 and following \citegen{Kluender.2004} proposal~-- that the effects observed in relativizing out of infinitival subjects may be best explained by processing factors. Overall, the experiments show that native speakers do not strongly reject these relativizations. The specificity of the filler, the complexity of the subject and probably many other intervening factors play a role in the finding that some extractions out of verbal subjects are very unnatural, and therefore received degraded acceptability judgments. Sentential subjects and infinitival subjects are not common: in French, they are very rare in the corpora (\citet{Abeille.2019.FTB} found only 24 sentential subjects and 99 infinitival subjects in the French Treebank; \citet[153]{Berard.2012} found in her corpora only complex NPs and no sentential or infinitival subject). Sentential subjects are harder to process than sentential complements \citep{Frazier.1988} and seem to require pragmatic licensing (according to \citet[685]{Miller.2001}, English sentential subjects are only felicitous if their content is ``discourse-old or inferrable'').
% Eye-tracking experiments conducted by Frazier and Rayner (1988) indicate that sentential subjects were harder to process than their extraposed sentential subject analog.
My HPSG proposal does not account for these effects and licenses extraction out of sentential subjects. Hence, there is no constraint on the \textsc{slash} set of sentential and infinitival subjects. 

%Smolka 2005

% [That Kim was late annoyed Lee.] -> [It annoyed Lee that Kim was late.] : lexical rule in Pollard & Sag 1994 page 150

%Koster jan 1978 Why subject sentences don't exist
%davies william d & stanley dubinsky 2000 why sentential subjects do so exist 

\section{Sentential subjects}

\citet{Erteschik-Shir.1973} has shown that sentential subjects are backgrounded -- i.e.\ \emph{non-focus} -- with respect to the main clause \citep[see also later][]{Goldberg.2006}. Similarly, \citet{Lee.2018} claims that the way to focalize the sentential subject in English is to turn it into a sentential complement in an impersonal construction like (\ref{ex:extraposed-sentential-subject}).

Hence, the lexical rule which allows NP elements in \textsc{arg-st} to be an S [\textsc{marking} \textit{que}] also assigns a \emph{non-focus} value to the embedded clause with respect to the embedding clause whenever it is the subject. This rule may lead to \figref{ex:avm-agacer} for the verb \emph{agacer} (`to annoy').

\begin{figure}[h]

\avm{[phon & < \type{agac-} >\\
cont & [hook & [clause-key & \tag{e1}\\
                icons-key & [target & \tag{e2}\\
                             clause & \tag{e1}]\\
                index & \tag{e2}]\\
        icons & < [\type*{non-focus}\\
                  target & \tag{e3}\\
                  clause & \tag{e2}] > $\oplus$ \type{lnis}]\\
arg-st & < S[marking & que\\
                                        mod & none\\
                                        index & \tag{e3}], NP[clause-key & \tag{e1}] >]}
\caption{Lexical item for \emph{agacer} (`to annoy') with  sentential subject}
\label{ex:avm-agacer}
\end{figure}

The FBC constraint in \figref{avm:rule-FBC} makes no particular prediction with respect to sentential subjects, because it does not apply to \emph{verbal} heads.
Still, extraction out of the sentential subject seems less felicitous in interrogatives and \emph{it}-clefts than in relative clauses. Example (\ref{ex:CP-subject-finite-interr-extr}) is an interrogative with extraction out of the sentential subject of (\ref{ex:CP-subject-finite-interr}). 

\eal
\ex[]{\gll [Que Kim parle à Frank] aga\c{c}ait Lee.\\
\sbar{}that Kim talked\textsc{.subj} at Frank annoyed Lee\\
\glt `That Kim talked to Frank annoyed Lee.'}
 \label{ex:CP-subject-finite-interr}
\ex[*]{\gll À qui [que Kim parle~\trace{}] aga\c{c}ait Lee?\\
at who that Kim talked\textsc{.subj} annoyed Lee\\
\glt `Who did that Kim talked to annoy Lee?' (intended: Who did it annoy Lee that Kim talked to?)}
\label{ex:CP-subject-finite-interr-extr}
\zl 

The reason why (\ref{ex:CP-subject-finite-interr-extr}) is degraded has probably nothing to do with the FBC constraint. \citet{Davies.2009} argue that such extractions may be difficult because they are a case of center-embedding. Center-embedded structures, albeit grammatical, are very difficult to parse. \citet{Fodor.2017} observe that typical center-embedded sentences become more acceptable with the right prosody, as in (\ref{ex:Fodor}) where || indicates that a pause is inserted. It would be interesting to see if similar results can be achieved on sentences like (\ref{ex:CP-subject-finite-interr-extr}). \citet{Chaves.2012} shows that it is at least the case for extractions out of subject NPs. 

\eal 
\ex \citep[ex.\ 11]{Fodor.2017}\\
The elegant woman || that the man I love met || moved to Barcelona.
\label{ex:Fodor}
\ex \citep[ex.\ 56a]{Chaves.2012}\\
Which book || did a review of || appear in the Times?
\zl 

Example (\ref{ex:CP-subject-finite-cleft-extr}) is a \emph{c'est}-cleft with extraction out of the sentential subject of (\ref{ex:CP-subject-finite-interr}). It is also unacceptable.

\ea[*]{\gll C' est à Frank [que [que Kim parle~\trace{}] aga\c{c}ait Lee].\\
it is at Frank \sbar{}that \sbar{}that Kim talked\textsc{.subj} annoyed Lee\\
\glt `It's Frank that that Kim talked to annoyed Lee.' (intended: It's Frank that it annoyed Lee that Kim talked to.)}
\label{ex:CP-subject-finite-cleft-extr}
\z  

In this case, the reason is probably a general ban on repeating the complementizer, which is probably ruled out for processing reasons as well. For example, a sentential subject cannot contain a sentential subject, as in (\ref{ex:recursive-sentential-subject}). 

\eal \label{ex:recursive-sentential-subject}
\ex[*]{[That [that Kim was late] annoyed Lee] is not a secret.}
\ex[*]{\gll [Que [Que Kim parle à Frank] aga\c{c}ait Lee] était connu de tous.\\
\sbar{}that \sbar{}that Kim talked\textsc{.subj} at Frank annoyed Lee was known of all\\
\glt `That that Kim talked to Frank annoyed Lee was known by all.' (intended: It was know by all that it annoyed Lee that Kim talked to Frank)}
\zl 

Therefore, this problem has nothing to do with extraction. It is also not restricted to sentential subjects, as shown by (\ref{ex:that-that}). 

\ea[??]{I understood [that [that Kim was late] annoyed Lee].}
\label{ex:that-that}
\z  

\section{Infinitival subject}

Infinitival arguments are defined as VP, hence they are not independent clauses as far as information structure is concerned. As explained earlier, they share the value of their \textsc{clause-key} feature with the value of the \textsc{clause-key} feature of the embedding verb, like NP arguments (in contrast to sentential complements, which form a clause and whose \textsc{clause-key} value is not structure-shared with the one of the embedding verb). Compare the lexical entry in \figref{ex:avm-agacer} with the lexical entry in \figref{ex:avm-agacer-inf}.

\begin{figure}[h]
\avm{[phon & < \type{agac-} >\\
cont & [hook & [clause-key & \tag{e1}\\
                icons-key & [\type*{info-str}\\
                             target & \tag{e2}\\
                             clause & \tag{e1}]\\
                index & \tag{e2}]\\
        icons & < [\type*{info-str}\\
                  target & \tag{e3}\\
                  clause & \tag{e2}] > $\oplus$ \type{lnis}]\\
arg-st & < VP[vform & non-finite\\
            index & \tag{e3}\\
            clause-key & \tag{e1}], NP[clause-key & \tag{e1}] >]}
\caption{Lexical entry for \emph{agacer} (`to annoy') with infinitival subject}
\label{ex:avm-agacer-inf}
\end{figure}

\figref{fig:CP-inf-ou} illustrates the HPSG analysis of a relative clause with a relative pronoun that contains extraction out of the infinitival subject. An interrogative would be similar, except for the information structure of the filler, which would be \emph{focus}. Focalizing part of a \emph{non-focus} infinitival subject is not restricted by the FBC constraint, because the infinitival subject has the feature \avm{[head & verb]}. \figref{fig:CP-inf-que} on page~\pageref{fig:CP-inf-que} shows the HPSG analysis of a relative clause with a complementizer that contains extraction out of an infinitival subject. 

\eal 
\ex[]{\gll Amsterdam, [où$_i$ [flâner~\trace{}$_i$] est charmant]\\
Amsterdam \sbar{}where \sbar{}wander\textsc{.inf} is charming\\
\glt `Amsterdam, where wandering is charming'} 
\label{ex:CP-inf-ou}
\ex[]{\gll Amsterdam, [qu$_i$' [observer~\trace{}$_i$] est charmant]\\
Amsterdam \sbar{}that \sbar{}observe\textsc{.inf} is charming\\
\glt `Amsterdam, observing which is charming'}
\label{ex:CP-inf-que}
\zl

\begin{figure}[h]
\oneline{%
\begin{forest}
where n children=0{tier=word}{}
[NP\\
\avm{[\type*{head-mod-structure}\\
      clause-key & \tag{e1}]}
[NP\\
\avm{[synsem & \1 [index & \tag{i}\\
                clause-key & \tag{e1}]]}
[Amsterdam\\Amsterdam]]
[S\\
\avm{[\type*{wh-rel-cl}\\
      mod & \1\\
      clause-key & \tag{e}\\
      c-cont|icons & <[\type*{topic}\\
                        target & \tag{i}\\
                        clause & \tag{e2}]>\\
      slash & \{\}]}
    [PP \\
    \avm{[loc & \2 [icons-key & \3]]}
    [où\\where]
    ]
    [S\\
    \avm{[\type*{head-subj-structure \& non-frame-setting}\\
          index & \tag{e}\\
          icons & < ... \4 ...>\\
          slash & \{\2 [index & \tag{i}]\}]}
          [\avm{\5} VP\\
            \avm{[head & verb\\
                hook & [icons-key & \4 [\type*{aboutness-topic}\\
                                     clause & \tag{e2}\\
                                     target & \tag{e3}]\\
                        index & \tag{e3}]\\
                slash & \{ \2 \}\\
                arg-str & < \6, [loc & \2] >]}
            [flâner\\wander\textsc{.inf}]]
          [VP\\
          \avm{[slash & \{\2\}]}
            [V\\
            \avm{[dt & <\5>\\
                  spr & <\5>\\
                  slash & \{\2\}]} [est\\is]]
            [AdjP [charmant\\charming]]
          ]
    ]
]
]
\end{forest}}
\caption{Simplified tree for \textit{Amsterdam,} [\textit{où}$_i$ [\textit{flâner}~\trace{}$_i$] \textit{est charmant}] (`Amsterdam, where wandering is charming')}
\label{fig:CP-inf-ou}
\end{figure}

\begin{figure}[hp]
\oneline{%
\begin{forest}
where n children=0{tier=word}{}
[NP\\
\avm{[\type*{head-mod-structure}\\
      clause-key & \tag{e1}]}
[NP\\
\avm{[synsem & \1 [index & \tag{i}\\
                clause-key & \tag{e1}]]}
[Amsterdam\\Amsterdam]]
[S\\
\avm{[\type*{comp-rel-cl}\\
      mod & \1\\
      clause-key & \tag{e}\\
      c-cont|icons & <[\type*{topic}\\
                        target & \tag{i}\\
                        clause & \tag{e2}]>\\
      slash & \{\}]}
    [COMP [qu'\\that]]
    [S\\
    \avm{[\type*{head-subj-structure \& non-frame-setting}\\
          index & \tag{e}\\
          icons & < ... \3 ...>\\
          slash & \{\2 [index & \tag{i}]\}]}
          [\avm{\4} VP\\
            \avm{[head & verb\\
                hook & [icons-key & \3 [\type*{aboutness-topic}\\
                                     clause & \tag{e2}\\
                                     target & \tag{e3}]\\
                        index & \tag{e3}]\\
                slash & \{ \2 \}\\
                arg-str & < \5, [loc & \2] >]}
            [observer\\observe\textsc{.inf}]]
          [VP\\
          \avm{[slash & \{\2\}]}
            [V\\
            \avm{[dt & <\4>\\
                  spr & <\4>\\
                  slash & \{\2\}]} [est\\is]]
            [NP [charmant\\charming]]
          ]
    ]
]
]
\end{forest}}
\caption{Simplified tree for \textit{Amsterdam,} [\textit{qu}$_i$'[\textit{observer}~\trace{}$_i$] \textit{est charmant}] (`Amsterdam, that observing is charming')}
    \label{fig:CP-inf-que}
\end{figure} 
