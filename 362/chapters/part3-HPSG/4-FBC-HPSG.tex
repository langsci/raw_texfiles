\section{Noun dependents in French}
\label{ch:hpsg-prep}

So far, we have treated all \emph{de}-dependents as \emph{de}-PPs, but there is a long tradition in French linguistics of distinguishing between a preposition \emph{de} that heads \emph{de}-PPs and a weak head \emph{de} in genitive \emph{de}-NPs. This mostly stems from the distinction between extractable \emph{de}-dependents and non-extractable \emph{de}-dependents of nouns. \citet{Sportiche.1981} shows that \emph{de}-PPs that denote a local origin cannot be extracted out of an NP, see (\ref{ex:depp-local-origin}). Furthermore, the presence of a second \emph{de}-dependent may block the extraction of a \emph{de}-dependent that is otherwise acceptable, see (\ref{ex:depp-double-de}).\largerpage 

\eal 
\ex \citep[adapted from][225]{Sportiche.1981}\\
* \gll la prison, de laquelle le transfert s' effectua avec du retard\\
the jail of which the transportation \textsc{refl} performed with some delay\\
\glt `the jail, from which the transportation has been performed with some delay'
\label{ex:depp-local-origin}
\ex \citep[63]{Godard.1996}\\
\gll La jeune femme dont le portrait (*de Corot) est à la fondation Barnes est
inconnue.\\
the young woman of.which the portrait \hspace{10pt}of Corot is at the foundation Barnes is unknown\\
\glt `The young woman whose portrait (by Corot) is at the Barnes foundation is unknown.'
\label{ex:depp-double-de}
\zl 

This latter problem with multiple \emph{de}-dependents has been analyzed either in terms of a hierarchy of semantic roles \citep{Sag.1994.Godard,Godard.1996}, or as a contrast between individual and property denoting interpretations \citep{Kolliakou.1999,Mensching.2018}. It has also been explained syntactically as a distinction between extractable argument \emph{de}-NPs and non-extractable adjunct \emph{de}-PPs \citep{Kolliakou.1999}. I have previously argued that the problem of multiple \emph{de}-dependents of nouns should be analyzed in semantic rather than syntactic terms \citep{MyP.2015}.
%Both accounts, however, agree on the fact that only the first \emph{de}-argument in the \textsc{arg-st} of the N can be extracted, iff this \emph{de}-phrase is indeed a NP[de] and not a PP[de]. 

Consequently, I continue to consider all \emph{de} as prepositions and all \emph{de}-de\-pen\-dents of nouns as PPs with \avm{[marking & de]}, as also suggested by \citet[246--251]{Milner.1978.syntaxe} and \citet{Abeille.2006.AandDe}.  
I also assume that these \emph{de}-dependents are all complements (or at least elements of the \textsc{arg-st} list of nouns), even though the distinction between arguments and adjuncts is even more blurry for dependents of nouns than for dependents of verbs.\largerpage

Prepositions in French (and all Romance languages) cannot be stranded, see (\ref{ex:prep-str}). Extraction out of the \emph{de}-PP is possible, see (\ref{ex:extraction-out-of-de}), but extraction out of other PPs seems very marginal (but compare example (\ref{ex:extraction-pour})).
The NP complement of some prepositions can be left out (\ref{ex:null-pro-avec}). The NP complement of other prepositions cannot: this is the case for \emph{de}, as illustrated by (\ref{ex:null-pro-de}). 

\eal 
\ex[*]{\gll Qui$_i$ as - tu parlé avec~/ de~\trace{}$_i$?\\
who have {} you spoken with of\\
\glt `Who did you speak with / about?' 
\label{ex:prep-str}}
\ex[]{(Chateaubriand, Mémoires d'outre-tombe, 1ère partie, livre 4, 1848)\\
\gll cette déclaration, dont$_i$ je me suis assuré [de la vérité~\trace{}$_i$]\\
this statement of.which I \textsc{refl} have ensured \sbar{}of the truth\\
\glt `this statement, whose truth I verified'
\label{ex:extraction-out-of-de}}
\ex[?]{	\gll l' eau d' irrigation dont$_i$ il plaide [pour la rationalisation [de l' usage~\trace{}$_i$]]\footnotemark\\
the water of irrigation of.which he argues \sbar{}for the rationalization \sbar{}of the use\\
\glt `the irrigation water, whose usage he argues for the rationalization of'
\label{ex:extraction-pour}}
\ex[]{\gll L' ampli de guitare fait des grésillements quand je joue avec.\\
the amp of guitar makes some crackles when I play with\\
\glt `The guitar amp crackles when I play with (it).'
\label{ex:null-pro-avec}}
\ex[*]{\gll J' éteinds la musique quand j' ai pas envie de.\\
I switch.off the music when I have not desire of\\
\glt `I switch off the music when I don't fancy (it).'
\label{ex:null-pro-de}}
\zl
\footnotetext{Source: \url{https://www.djazairess.com/fr/lesoirdalgerie/208772}, last access 05/08/2020.}

Therefore, the \textsc{arg-st} list of prepositions is constrained to only contain objects of type \textit{non-gap} (see \figref{fig:hrch-syssem}). But \emph{de} itself has only \emph{canonical} objects in its \textsc{arg-st} and its \textsc{slash} set is not constrained (it may be non-empty). The other prepositions are defined as having an empty \textsc{slash} set.\footnote{In order for the grammar to account for marginal cases like (\ref{ex:extraction-pour}), the \textsc{slash} set of all prepositions could be left unconstrained.}

To account for the semantics of prepositions, we distinguish meaningful prepositions like (\ref{ex:sur-lexical}) from meaningless prepositions like (\ref{ex:sur-functional}). We assume that many prepositions can be either meaningful or meaningless and have then two different lexical entries (or equivalently, have a lexical entry with a disjunction). 

\eal 
\ex[]{\gll Susanne travaille sur son balcon.\\
Susanne works on her balcony\\
\glt `Susanne is working on her balcony.'}
\label{ex:sur-lexical}
\ex[]{\gll Susanne travaille sur un nouveau projet.\\
Susanne works on a new project\\
\glt `Susanne is working on a new project.'}
\label{ex:sur-functional}
\zl 

The lexical entry of meaningful prepositions has a non-empty \textsc{rels} list, while the lexical entry of meaningless prepositions does not introduce any EP in the \textsc{rels} list. 

Meaningful prepositions 
%\footnote{Some prepositions like English \emph{beforehand} or French \emph{dedans} (`inside') are intransitive. They have then their own \textsc{index} value %, or structure-share its value with the \textsc{index} value of some other element, like \emph{down} in [\emph{turn the volume down}]
%\citep[97--98]{Tseng.2000}. I ignore these special cases here.}
have their own \textsc{index} value (probably an event) and their own \textsc{icons-key}, like any word that introduces an EP.

Meaningless prepositions structure-share their \textsc{index} value with the \textsc{index} value of the NP they subcategorize for. 
I also assume that all meaningless prepositions that take a complement structure-share their \textsc{icons-key} value with the \textsc{icons-key} value of their complement. This latter point conflicts somewhat with the underlying idea that semantically empty elements do not introduce an object of type \emph{info-str} into the \textsc{icons} list of the utterance. But, because prepositions have the same \textsc{index} value as their complement, and by virtue of the Discourse-clash Avoidance Principle (\ref{rule:discourse-clash-avoid}) on page~\pageref{rule:discourse-clash-avoid}, they must have the same \textsc{icons-key} value. Consequently, prepositions do not introduce a new element into the \textsc{icons} list of the utterance, but merely treat the NP as the semantic head of the PP.\largerpage

I assume that \emph{de} is a meaningless preposition with the lexical entry in \figref{avm:de}.\footnote{It could be useful to posit a separate lexical entry for \emph{de} expressing possessive or origin, and consider it as meaningful. But this has no impact on the formalization of the FBC constraint I propose. For the sake of simplicity, I assume only one lexical entry for \emph{de}. 

Furthermore, \citet{Abeille.2006.AandDe} define two usages of the preposition \emph{de}, one in which it is the head, and one in which it is a marker (i.e.\ in \emph{beaucoup de}, `many'). Here, I only consider the former, which is the use I investigated in the empirical parts.} 

\begin{figure}
\avm{[phon & < \type{de} >\\
cat & [head & prep\\
marking & de]\\
cont & [index & \tag{i}\\
        icons-key & \1\\
        rels & <>]\\
arg-st & <[\type*{canonical}\\
           index & \tag{i}\\
           icons-key & \1]>]}
\caption{Lexical entry for \emph{de} (`of')}
\label{avm:de}
\end{figure}

For a preposition like \emph{sur} (`on') that can be used either as a meaningful preposition like in (\ref{ex:sur-lexical}) or as a meaningless preposition like in (\ref{ex:sur-functional}), we can define the lexical entries in Figures~\ref{avm:sur-mf} and~\ref{avm:sur-ml}, which summarizes the assumptions made so far.

\begin{figure}
\small\captionsetup{margin=.05\linewidth}%
\begin{floatrow}
\ffigbox
{\avm{[phon & < \type{sur} >\\
cat & [head & prep]\\
cont & [hook & [index & index\\
        icons-key & info-str]\\
        rels & <[\emph{on\_rel}\\ 
        arg0 & \tag{i}]> ]\\
slash & \{\}\\
arg-st & <[\type*{non-gap}\\
           index & \tag{i}\\
           icons-key & info-str]>]}}
{\caption{Lexical entry for meaningful \emph{sur} (`on')}\label{avm:sur-mf}}
\ffigbox
{\avm{[phon & < \type{sur} >\\
cat & [head & prep]\\
cont & [hook & [index & \tag{i}\\
        icons-key & \1]\\
        rels & <> ]\\
slash & \{\}\\
arg-st & <[\type*{non-gap}\\
           index & \tag{i}\\
           icons-key & \1]>]}}
{\caption{Lexical entry for meaningless \emph{sur} (`on')}\label{avm:sur-ml}}
\end{floatrow}
\end{figure}

The extraction of the PP-dependent of a noun out of NPs takes place in a straightforward manner via the mechanisms of extraction explained earlier. Under the \textsc{nonloc} Amalgamation Principle (\ref{avm:nonloc-amalgamation}), the verb that selects an NP with a \textsc{slash} element inherits this element. There is no difference between extraction of a PP-dependent of a verb and subextraction of a PP-dependent of a noun out of an NP.

\section{The subject as Designated Topic}

One of the central claims in this book is that the phenomenon usually called ``subject island'' is actually the result of a discourse clash: Typically, the subject is the topic of the clause, and focalizing part of the subject leads to a contradiction in that the subject NP is simultaneously treated as part of the Common Ground and as the main information of the sentence (contra Grice's maxim that a sentence should be informative) or even as unknown information (internal contradiction). 

The fact that the subject is the preferred topic of the clause has been captured in several HPSG proposals. \citet{Webelhuth.2007} has a function ``more thematic than'' that yields a hierarchy of thematicity (subject $>$ direct object $>$ oblique object). I adopt here \citegen{Bildhauer.2010} notion ``Designated Topic'', which is based on the verb's preference for a certain argument to be the topic \citep*[see also][Section~5.3.3.2]{Mueller.S.2020?.chapter5}. For example, German \emph{herrschen} (`to reign') in its existential meaning preferably has the locative as its topic. 

\ea \citep[72]{Bildhauer.2010}\nopagebreak\\
\gll Weiterhin Hochbetrieb herrscht am Innsbrucker Eisoval.\\
further high.traffic reigns at.the Innsbruck icerink\\
\glt `It's still all go at the Innsbruck icerink.'
\z 

In \citeauthor{Bildhauer.2010}'s proposal, \textsc{d(esignated)t(opic)} is a head feature of verbs with a list as value, either empty or containing at most one \emph{synsem} object. This object is structure-shared with one element of the \textsc{arg-st} list. For verbs with the subject as default topic, the lexical entry is then the following:\largerpage

\ea Lexical entry for a verb with a subject default Topic: \nopagebreak

\avm{[synsem|loc|cat & [head|dt & <\1>\\
subj & <\1>]\\
arg-st & < \1 > $\oplus$ list]}
\z 

The Designated Topic is realized as topic in sentences that ``involv[e] a Topic-Comment structure plus an assessment of the extent to which the Comment holds of the Topic'' \citep[73]{Bildhauer.2010}. In their model, these sentences are labeled as \emph{a(ssessment)-topic-comment}, which is itself a subtype of \emph{topic-comment}. \citegen{Song.2017} model also provides a hierarchy of the information structure form of sentences. Its supertype is called \emph{sform} (\emph{sentential form}) and the hierarchy is reproduced in \figref{fig:hrch-sform}. Phrases inherit from both the appropriate \emph{headed-structure} and the appropriate \emph{sform}.{\interfootnotelinepenalty=10000\footnote{Separate features ensure that the whole clause keeps the same \emph{sform}. These features are not relevant for my analysis, but see \citet[Chapter~7]{Song.2017} for a detailed explanation.}}

\begin{figure}[ht]
\centering
\begin{forest}
sn edges,
[\textit{sform}
    [\textit{focality}
        [\textit{narrow-focus}
            [\textit{focus-bg}, name = focusbg]
        ]
        [\textit{wide-focus}
            [\textit{all-focus}, name = allfocus]
        ]
    ]
    [\textit{topicality}
        [\textit{topicless}, name = topicless]
        [\textit{topic-comment}
            [\textit{frame-setting}]
            [\textit{non-frame-setting}]
        ]
    ]
]
\draw[thin] (focusbg.north)--(topicless.south);
\draw[thin] (allfocus.north)--(topicless.south);
\end{forest}
    \caption{Type hierarchy of \emph{sform} \citep[125]{Song.2017}}
    \label{fig:hrch-sform}
\end{figure}

I assume that \citegen{Bildhauer.2010} \emph{a-topic-comment} can be directly translated into \citegen{Song.2017} \emph{non-frame-setting}, and propose the following constraint:

\ea
\oneline{
\avm{[\type*{non-frame-setting}\\
      cat|head|dt & < [index & \tag{i}] >\\
      cont|hook & [clause-key & \tag{e}]]}
      \impl
\avm{[c-cont|icons & <[\type*{aboutness-topic}\\
                       target & \tag{i}\\
                       clause & \tag{e}]>]}      
}
\label{avm:rule-dt-is-topic}              
\z 

The implication in (\ref{avm:rule-dt-is-topic}) is that if an AVM has the type \emph{non-frame-setting}, then the Designated Topic is the topic of the clause (i.e.\ an appropriate \textsc{target-clause} pair with the status \emph{aboutness-topic} is introduced in the construction).\footnote{Recall that the Discourse-clash Avoidance Principle (\ref{rule:discourse-clash-avoid}) ensures that an element can have only one discourse status with respect to a clause. It follows that \emph{info-str} element introduced by the Designated Topic must match the one introduced by the construction \emph{non-frame-setting}, except if the element has a discourse status with respect to two or more different clauses.}

\section{Formalization of the Focus-Background Conflict constraint in HPSG}

Recall that in \citeauthor{Song.2017}'s terminology, ``background'' applies to the elements in the utterance that are neither topic nor focus (\figref{fig:hrch-icons}). In my terminology so far, and in the formulation of the FBC constraint in particular, I assumed that the topic belongs to the background (Section~\ref{ch:is}). In order to match \citeauthor{Song.2017}'s terminology, the constraint (\ref{rule:FBC-bis}) can be reformulated as: ``A focused element should not be part of a non-focus constituent.'' 

Another way to formulate this is to say that all dependents of a non-focus word should be non-focus as well, which is exactly the meaning of the  (simplified) formalization in \figref{avm:rule-FBC}. 

\begin{figure}
\caption{Focus-Background Conflict constraint (simplified)}
\label{avm:rule-FBC}
    \begin{tabular}{@{}ll@{}}
	& \avm{
		[\type*{word}\\
		synsem|loc & [cat|head & non-verbal\\
		cont|hook & [icons-key & non-focus\\
		           clause-key & \tag{e}]]]} \\\addlinespace
	\impl & \avm{[arg-str & < ... [\type*{non-focus}\\
		               clause key & \tag{e}] ... >]
                }
    \end{tabular}
\end{figure}

The implication in \figref{avm:rule-FBC} should be understood as follows: a non-verbal word that has non-focus status with respect to a certain clause can only subcategorize for elements that are also non-focus with respect to the same clause. 


% Webelhuth, p310 :
% Preferably, themes are unfocussed.
% Preferably, themes are discourse-familiar.

This applies to all parts of speech except \emph{verbal}: As discussed in Chapter~\ref{ch:fbc-generals}, a verbal element can have non-focus status without constraining its arguments to have non-focus status as well. Complementizers, on the other hand, are semantically empty, and are therefore not affected by the FBC constraint. 

A non-focus noun, for example a topic subject NP as in (\ref{ex:FBC-topic-subject}), can only have non-focus complements, otherwise it would violate the Focus-Background Conflict constraint (\figref{avm:rule-FBC}). This is illustrated by \figref{fig:FBC-topic-subject}.

{\judgewidth{\#}%
\ea[\#]{\gll [l' originalité [de cette innovation]$_F$]$_T$\\
\sbar{}the uniqueness \sbar{}of this innovation\\
\glt `the uniqueness of this innovation'
\label{ex:FBC-topic-subject}}
\z}

\begin{figure}
\oneline{%
\begin{forest}
if n children=0{tier=word}{}
[NP [\avm{\1} Det [l'\\the]]
[N$'$
    [N\\
\avm{[head & noun\\
hook & [icons-key & [\type*{topic}\\
                  clause & \tag{e}\\
                  target & \tag{i}]\\
    clause-key & \tag{e}]\\
arg-str & < \1, \2 > ]}
    [originalité\\uniqueness]]
    [PP \\
\avm{\2 [index & \tag{j}\\
         icons-key & [\type*{focus}\\
         clause & \tag{e}\\
         target & \tag{j}]]}
    [de cette innovation\\of this innovation, roof]]
]]
\end{forest}}
\caption{Simplified tree for the infelicitous NP ``l' originalité [de cette innovation]$_F$]$_T$'' (`the uniqueness of this innovation')}
    \label{fig:FBC-topic-subject}
\end{figure} 

\subsection{Implementing the FBC constraint}

The formalization in \figref{avm:rule-FBC} is sufficient for my demonstration, but it would be insufficient for a direct implementation. Technically, we need to make sure that the \textsc{arg-str} list only contains \textit{info-str} elements that are non-focus with respect to the clause \avm{\tag{e}}, while allowing any other \textit{icons} element that is not of the type \textit{info-str}, and also potentially allowing \textit{info-str} elements that are non-focus with respect to another clause. This point is crudely represented by the dots in \figref{avm:rule-FBC}.

For the reader interested in the technical details of implementing the constraint, here is a method in two steps. First, we can define a function \texttt{non-focus()} that takes as arguments an event and a list of objects of the type \emph{info-str}:

\ea \texttt{non-focus(\avm{\1},\avm{\2[\type*{info-str}\\clause & \3]}|Rest):-}\\ 
if \texttt{\avm{\1} == \avm{\3}}, then \texttt{\avm{\2[\type*{non-focus}]}} and \texttt{non-focus(\avm{\1},Rest).}

\texttt{non-focus(\avm{\1},\avm{<>}).}
\z 

The function \texttt{non-focus()} checks whether the \textsc{clause} value of the first \emph{info-str} object of its second argument is identical with its first argument, and if so, constrains the \emph{info-str} object to be \emph{non-focus}. It then recursively checks each element of the list until the end in the same way.

Second, we may reformulate \figref{avm:rule-FBC} as \figref{avm:rule-FBC-more-exact}.

\begin{figure}[h]
\avm{
      [\type*{word}\\
      synsem|loc & [cat|head & non-verbal\\
      cont|hook & [icons-key & non-focus\\
      clause-key & \tag{e}]]]}
\quad\impl\quad
\parbox[c]{\widthof{non-focus(m, n)}}{\raggedright\avm{[arg-str & \1]} $\wedge$ \\ non-focus(\avm{\tag{e},\1})}
\caption{Focus-Background Conflict constraint}\label{avm:rule-FBC-more-exact}
\end{figure}

I will now consider how the FBC constraint interacts with each of the following constructions: interrogatives, \emph{c'est}-clefts and relative clauses.

\subsection{The FBC constraint in interrogatives}

In standard interrogatives with extraction of the \emph{wh}-word (\emph{standard-wh-inter-cl}), the filler receives a \emph{semantic-focus} interpretation, see (\ref{avm:standard-wh-inter-cl}). For a sentence like (\ref{ex:FBC-topic-subject-interr}), with extraction out of the subject, the consequence is that the Designated Topic cannot be the topic of the utterance as defined in (\ref{avm:rule-dt-is-topic}), because then it would violate the rule in \figref{avm:rule-FBC}. 

\ea[]{\gll [De quelle innovation]$_i$ [l' originalité~\trace{}$_i$] enthousiasme-t-elle mes collègues?\\
\sbar{}of which innovation \sbar{}the uniqueness excites-0-\textsc{3sg.sbj.fem} my colleagues\\
\glt `Of which innovation does the uniqueness excite my colleagues?'}
\label{ex:FBC-topic-subject-interr}
\z 

\begin{sidewaysfigure}[hp]
\resizebox{!}{12cm}{%
\begin{forest}
where n children=0{tier=word}{}
[S\\
\avm{[\type*{standard-wh-inter-cl}\\
      clause-key & \tag{e}\\
      slash & \{\}]}
    [PP\\
    \avm{[loc & \1\\
          index & \tag{i}\\
          icons-key & [\type*{focus}\\
                       target & \tag{i}\\
                       clause & \tag{e}]]}
        [de quelle innovation\\of which innovation, roof]]
    [S\\
    \avm{[\type*{head-subj-structure \& all-focus}\\
          index & \tag{e}\\
          icons & < ... \4 ...>\\
          slash & \{\1\}]}
          [\avm{\2} NP
            [\avm{\3} Det [l'\\the]]
            [N\\
            \avm{[head & noun\\
                hook & [icons-key & \4 [\type*{semantic-focus}\\
                                     clause & \tag{e}\\
                                     target & \tag{j}]\\
                        index & \tag{j}]\\
                slash & \{ \1 \}\\
                arg-str & < \3, [loc & \1] >]}
            [originalité\\uniqueness]]
          ]
          [VP\\
          \avm{[slash & \{\1\}]}
            [V\\
            \avm{[dt & <>\\
                  subj & <\2>\\
                  slash & \{\1\}]} [enthousiasme\\excites]]
            [NP [mes collègues\\my colleagues, roof]]
          ]
    ]
]
\end{forest}}
\caption{Simplified tree for [\textit{De quelle innovation}]$_i$ [\textit{l' originalité}~\trace{}$_i$] \textit{enthousiasme-t-elle mes collègues?} (`Of which inovation does the uniqueness excite my colleagues?')}
\label{fig:FBC-topic-subject-interr}
\end{sidewaysfigure} 

In extraction out of the subject in an interrogative, the subject cannot be topic, i.e.\ either another element is topic (\emph{frame-setting-topic}) or the clause is \emph{topicless}. For example, \citeauthor{Chaves.2013} shows that example (\ref{ex:Chaves2013}) is acceptable in English. In \citet{Abeille.2020.Cognition}, we use a test for topicality in order to show that the clause has an all-focus interpretation, see (\ref{ex:Chaves2013-demo}). 

\ea \citep[313]{Chaves.2013}\\
Which problem will [the solution to~\trace{}] never be found? 
\label{ex:Chaves2013} 
\z 

\eal \label{ex:Chaves2013-demo}
\ex[]{A solution to this problem will never be found.}
\ex[\#]{Speaking of a solution to this problem, it will never be found.}
\zl 

\figref{fig:FBC-topic-subject-interr} shows the analysis of such a case. In this example, the sentence in (\ref{ex:FBC-topic-subject-interr}) is all-focus, as in (\ref{ex:Chaves2013}).

On the other hand, if the filler is not focused, as in rhetorical questions, then the extraction is felicitous, as in example (\ref{ex:internet-ahmadinejad}) reproduced in (\ref{ex:internet-ahmadinejad-ter}).

\ea[]{\gll [De quel pays]$_i$ [la dépense militaire~\trace{}$_i$] dépasse annuellement mille milliards de dollars~[\dots]~?\\
\sbar{}of which country \sbar{}the budget military exceeds yearly thousand billion of dollars\\
\glt `Of which country does the military budget exceed yearly 1000 B.\ dollars?'}
\label{ex:internet-ahmadinejad-ter}
\z 

Notice that the constraint predicts that in languages in which postverbal subjects are focused, interrogatives with extraction out of postverbal subjects should be more felicitous than extraction out of preverbal subjects. Spanish, for example, is such a language.
%amelioration in French too?
%\# De quel fruit (est ce que) le gout te plaît ?\\
% De quel fruit (est-ce que) te plaît le gout ?

As such, the constraint in \figref{avm:rule-FBC} predicts that an interrogative with extraction out of the subject in a long-distance dependency is felicitous. A \emph{standard-wh-cl} constrains the filler from bearing focus with respect to the main clause (the \textsc{head-dtr}), see (\ref{avm:standard-wh-inter-cl}). In this case, the value of the embedded subject's \textsc{icons-key|clause} does not match the value of the filler's \textsc{icons-key|clause}, and \figref{avm:rule-FBC} is not violated.
And indeed, I argued in Section~\ref{ch:analysis-ldd} that I do not have evidence that focalization involving a long-distance dependency violates the FBC constraint.

\figref{fig:FBC-topic-subject-interr-ldd} illustrates the HPSG analysis for an interrogative with long-distance dependency in which the filler is focus and the subject of the embedded clause topic. There is no violation of \figref{avm:rule-FBC}.

\ea[]{\gll [De quelle innovation]$_i$ suppose-t-il [que [l' originalité~\trace{}$_i$] enthousiasme mes collègues]~?\\
\sbar{}of which innovation supposes-0-\textsc{3sg.sbj.masc} \sbar{}that \sbar{}the uniqueness excites my colleagues\\
\glt `Of which innovation does he suppose that the uniqueness excites my colleagues?'}
\label{ex:FBC-topic-subject-interr-ldd}
\z 

\begin{sidewaysfigure}[hp]
\resizebox{!}{12cm}{%
\begin{forest}
where n children=0{tier=word}{}
[S\\
\avm{[\type*{standard-wh-inter-cl}\\
      clause-key & \tag{e1}\\
      slash & \{\}]}
    [PP\\
    \avm{[loc & \1\\
          index & \tag{i}\\
          icons-key & [\type*{focus}\\
                       target & \tag{i}\\
                       clause & \tag{e1}]]}
        [de quelle innovation\\of which innovation, roof]]
    [S\\
    \avm{[\type*{noncomp-comps-structure}\\
          slash & \{\1\}]}
    [V\\
    \avm{[arg-st & < NP[\type{aff}], \2 >]}
    [suppose-t-il\\supposes]]
    [\avm{\2} S\\
    \avm{[\type*{sent-comp}\\
          hook & [index & \tag{e2}\\
                  clause-key & \tag{e2}]\\
          slash & \{\1\}]}
    [COMP [que\\that]]
    [S\\
    \avm{[\type*{head-subj-structure \& non-frame-setting}\\
          index & \tag{e2}\\
          icons & < ... \5 ...>\\
          slash & \{\1\}]}
          [\avm{\3} NP
            [\avm{\4} Det [l'\\the]]
            [N\\
            \avm{[head & noun\\
                hook & [icons-key & \5 [\type*{aboutness-topic}\\
                                     clause & \tag{e2}\\
                                     target & \tag{j}]\\
                        index & \tag{j}]\\
                slash & \{ \1 \}\\
                arg-str & < \4, [loc & \1] >]}
            [originalité\\uniqueness]]
          ]
          [VP\\
          \avm{[slash & \{\1\}]}
            [V\\
            \avm{[dt & <\3>\\
                  spr & <\3>\\
                  slash & \{\1\}]} [enthousiasme\\excites]]
            [NP [mes collègues\\my colleagues, roof]]
          ]
    ]
    ]
    ]
]
\end{forest}}
\caption{Simplified tree for [\textit{De quelle innovation}]$_i$ \textit{suppose-t-il} [\textit{que} [\textit{l'originalité}~\trace{}$_i$] \textit{enthousiasme mes collègues}]\textit{?} (`Of which innovation does he suppose that the uniqueness excites my colleagues?')}
    \label{fig:FBC-topic-subject-interr-ldd}
\end{sidewaysfigure}

\subsection{The FBC constraint in \emph{c'est}-clefts}

The FBC constraint makes the same predictions for \emph{it}-clefts as for interrogatives involving focalization. \figref{fig:FBC-topic-subject-cleft} shows a case of \emph{c'est}-clefts with extraction out of a subject. Even though it is extraction out of a sentential complement, the structure is monoclausal and the subject cannot be topic, otherwise \figref{avm:rule-FBC} would be violated. 

\ea[]{\gll C' est [de cette innovation]$_i$ [que [l' originalité~\trace{}$_i$] enthousiasme mes collègues].\\
it is \sbar{}of this innovation \sbar{}that the uniqueness excites my colleagues\\
\glt `It's of this innovation that the uniqueness excites my colleagues.'}
\label{ex:FBC-topic-subject-cleft}
\z 

\begin{sidewaysfigure}[htbp]
\scalebox{0.7}{
\begin{forest}
where n children=0{tier=word}{}
[S\\
\avm{[\type*{head-subj-structure}\\
      clause-key & \tag{e}\\
      slash & \{\}]}
    [NP [c'\\it]]
    [VP\\
    \avm{[\type*{noncomp-comps-structure}\\
          clause-key & \tag{e}\\
          slash & \{\}]}
    [VP\\
    \avm{[\type*{noncomp-comps-structure}\\
          clause-key & \tag{e}\\
          slash & \{\}]}
    [V\\
    \avm{[clause-key & \tag{e}\\
    slash & \{\}]}
    [est\\is]]
    [PP\\
    \avm{[loc & \1\\
          index & \tag{i}\\
          icons-key & [\type*{focus}\\
                       target & \tag{i}\\
                       clause & \tag{e}]]}
        [de cette innovation\\of this innovation, roof]]
    ]
    [\avm{\2} S\\
    \avm{[\type*{sent-comps}\\
          hook & [index & \tag{e}\\
                  clause-key & \tag{e}]\\
          slash & \{\1\}]}
    [COMP [que\\that]]
    [S\\
    \avm{[\type*{head-subj-structure \& all-focus}\\
          index & \tag{e}\\
          icons & < ... \3 ...>\\
          slash & \{\1\}]}
          [\avm{\4} NP
            [\avm{\5} Det [l'\\the]]
            [N\\
            \avm{[head & noun\\
                hook & [icons-key & \3 [\type*{semantic-focus}\\
                                     clause & \tag{e}\\
                                     target & \tag{j}]\\
                        index & \tag{j}]\\
                slash & \{ \1 \}\\
                arg-str & < \5, [loc & \1] >]}
            [originalité\\uniqueness]]
          ]
          [VP\\
          \avm{[slash & \{\1\}]}
            [V\\
            \avm{[dt & <>\\
                  subj & <\4>\\
                  slash & \{\1\}]} [enthousiasme\\excites]]
            [NP [mes collègues\\my colleagues, roof]]
          ]
    ]
    ]
    ]
]
\end{forest}}
\caption{Simplified tree for \textit{C'est} [\textit{de cette innovation}]$_i$ [\textit{que} [\textit{l'originalité}~\trace{}$_i$] \textit{enthousiasme mes collégues}]. (`It's of this innovation that the uniqueness excites my colleagues.')}
    \label{fig:FBC-topic-subject-cleft}
\end{sidewaysfigure}



%contradiction with a previously salient focal alternative (but contrariness and not contradiction: doxatic dimension) -- but the contradiction is not required ``the presence of a clearly identifiable alternative (set) is not required''
%3 points in contrast: `` the size of the alternative set, the identifiability of its elements, and the exclusion requirement of the alternatives'' \citep[12]{Destruel.2019}
%\citet{Repp.2016}: not only alternative, but type of discourse relation plays a role in contrast (gradation : non-contrastive, oppose, correction). In non-contrastive, we have alternatives which can be compatible. In an oppose relation, the contribution is different. In a correction relation, they are not mutualy compatible.  
            % Repp, S. (2016). “Contrast: dissecting an elusive information-structural notion and its role in grammar,” in Handbook of Information Structure, eds C. Féry and S. Ishihara (Oxford: Oxford University Press), 270–289.
% mirative focus (Cruschina 2012): new information that is particularly surprising or unexpected to the hearer
            % Cruschina, S. (2012). Discourse-Related Features and Functional Projections. Oxford: Oxford University Press.
            
Notice that an interrogative with extraction out of the pivot is expected to be felicitous under the FBC constraint: the pivot is focused and so is the \emph{wh}-phrase, therefore there is no discourse clash.

\eal 
\ex[]{[Which car]$_i$ is it [the color of~\trace{}$_i$]$_j$ that you loved~\trace{}$_j$ most?}
\ex[]{\gll [De quelle innovation]$_i$ est - ce [l' originalité~\trace{}$_i$]$_j$ que mes collègues apprécient~\trace{}$_j$~?\\
\sbar{}of which innovation is {} it \sbar{}the uniqueness that my colleagues appreciate\\
\glt `Of which innovation is it the uniqueness that my colleagues appreciate?'}
\zl 

Extraction out of the \emph{que}-clause falls under the same constraints as any extraction in a long-distance dependency. An interrogative involving extraction out of the subject would not violate \figref{avm:rule-FBC}. Relativization out of the \emph{que}-clause is also expected to be felicitous.

\subsection{The FBC constraint in relatives}

Finally, the FBC constraint has no impact on  relative clauses. \figref{fig:FBC-topic-subject-rc} on page~\pageref{fig:FBC-topic-subject-rc} illustrates relativization out of a topic subject in a \emph{comp-rel-cl} (with a complementizer). In this case, the slashed element is never realized; its \textsc{index} value is only structure-shared with the \textsc{index} value of the noun modified by the relative clause. The implication in \figref{avm:rule-FBC} constrains the \textsc{icons-key} value of the slashed element to be \emph{non-focus}, but it is completely underspecified in other respects. 

\ea
\gll une innovation [dont$_i$ [l' originalité~\trace{}$_i$] enthousiasme mes collègues]\\
an innovation \sbar{}of.which \sbar{}the uniqueness excites my colleagues\\
\glt `an innovation of which the uniqueness excites my colleagues'
\label{ex:FBC-topic-subject-rc}
\z 



\begin{sidewaysfigure}[hp]
\resizebox{!}{12cm}{%
\begin{forest}
where n children=0{tier=word}{}
[NP\\
\avm{[\type*{head-mod-structure}\\
      clause-key & \tag{e1}]}
[NP\\
\avm{[synsem & \1 [index & \tag{i}\\
                clause-key & \tag{e1}]]}
[une innovation\\an innovation, roof]]
[S\\
\avm{[\type*{comp-rel-cl}\\
      mod & \1\\
      clause-key & \tag{e}\\
      c-cont|icons & <[\type*{topic}\\
                        target & \tag{i}\\
                        clause & \tag{e2}]>\\
      slash & \{\}]}
    [COMP [dont\\of.which]]
    [S\\
    \avm{[\type*{head-subj-structure \& non-frame-setting}\\
          index & \tag{e}\\
          icons & < ... \3 ...>\\
          slash & \{\2 [index & \tag{i}]\}]}
          [\avm{\4} NP
            [\avm{\5} Det [l'\\the]]
            [N\\
            \avm{[head & noun\\
                hook & [icons-key & \3 [\type*{aboutness-topic}\\
                                     clause & \tag{e2}\\
                                     target & \tag{j}]\\
                        index & \tag{j}]\\
                slash & \{ \2 \}\\
                arg-str & < \5, [loc & \2] >]}
            [originalité\\uniqueness]]
          ]
          [VP\\
          \avm{[slash & \{\2\}]}
            [V\\
            \avm{[dt & <\4>\\
                  spr & <\4>\\
                  slash & \{\2\}]} [enthousiasme\\excites]]
            [NP [mes collègues\\my colleagues, roof]]
          ]
    ]
]
]
\end{forest}}
\caption{Simplified tree for \textit{une innovation} [\textit{dont}$_i$ [\textit{l'originalité}~\trace{}$_i$] \textit{enthousiasme mes collégues}] (`an innovation of which the uniqueness excites my colleagues')}
    \label{fig:FBC-topic-subject-rc}
\end{sidewaysfigure} 

\figref{fig:FBC-topic-subject-rc-dequi} on page \pageref{fig:FBC-topic-subject-rc-dequi} illustrates extraction out of the topic subject of a \emph{wh-rel-cl} (with pronominal filler). In this particular example, given in (\ref{ex:FBC-topic-subject-rc-dequi}), the filler is [P + pronoun]. The FBC constraint does not constrain the elements of the \textsc{arg-st} list, because relative pronouns are informatively empty \citep[112]{Song.2017} and therefore their \textsc{icons-key} value is \emph{i-empty}.

\ea
\gll un avocat [[de qui]$_i$ [l' associé~\trace{}$_i$] aide mon cousin]\\
a lawyer \ssbar{}of who \sbar{}the associate helps my cousin\\
\glt `a lawyer of whom the associate helps my cousin'
\label{ex:FBC-topic-subject-rc-dequi}
\z 

\begin{sidewaysfigure}[hp]
\resizebox{!}{12cm}{%
\begin{forest}
where n children=0{tier=word}{}
[NP\\
\avm{[\type*{head-mod-structure}\\
      clause-key & \tag{e1}]}
[NP\\
\avm{[synsem & \1 [index & \tag{i}\\
                clause-key & \tag{e1}]]}
[un avocat\\a lawyer, roof]]
[S\\
\avm{[\type*{wh-rel-cl}\\
      mod & \1\\
      clause-key & \tag{e}\\
      c-cont|icons & <[\type*{topic}\\
                        target & \tag{i}\\
                        clause & \tag{e2}]>\\
      slash & \{\}]}
    [PP \\
    \avm{[loc & \2 [icons-key & \3]]}
    [P\\
    \avm{[icons-key & \3]}
    [de\\of]]
    [NP\\
    \avm{[icons-key & \3 i-empty]}
    [qui\\who]]
    ]
    [S\\
    \avm{[\type*{head-subj-structure \& non-frame-setting}\\
          index & \tag{e}\\
          icons & < ... \4 ...>\\
          slash & \{\2 [index & \tag{i}]\}]}
          [\avm{\5} NP
            [\avm{\6} Det [l'\\the]]
            [N\\
            \avm{[head & noun\\
                hook & [icons-key & \4 [\type*{aboutness-topic}\\
                                     clause & \tag{e2}\\
                                     target & \tag{j}]\\
                        index & \tag{j}]\\
                slash & \{ \2 \}\\
                arg-str & < \6, [loc & \2] >]}
            [associé\\associate]]
          ]
          [VP\\
          \avm{[slash & \{\2\}]}
            [V\\
            \avm{[dt & <\5>\\
                  spr & <\5>\\
                  slash & \{\2\}]} [aide\\helps]]
            [NP [mon cousin\\my cousin, roof]]
          ]
    ]
]
]
\end{forest}}
\caption{Simplified tree for \textit{un avocat} [[\textit{de qui}]$_i$ [\textit{l'associé}~\trace{}$_i$] \textit{aide mon cousin}] (`a lawyer of who the associate helps my cousin')}
    \label{fig:FBC-topic-subject-rc-dequi}
\end{sidewaysfigure} 

Another case of extraction out of the topic subject of a \emph{wh-rel-cl} is illustrated by \figref{fig:FBC-topic-subject-rc-PPdequi} on page \pageref{fig:FBC-topic-subject-rc-PPdequi}. Here, the relative pronoun is embedded in an NP, and the \textsc{icons-key} value of the filler is constrained by the implication in \figref{avm:rule-FBC} to be \emph{non-focus}. It is otherwise underspecified. 

\ea
\gll Christelle, [[de la soeur de qui]$_i$ [l' arrogance~\trace{}$_i$] rebute mes collègues]\\
Christelle \ssbar{}of the sister of who \sbar{}the arrogance repels my colleagues\\
\glt `Christelle, whose sister's arrogance repels my colleagues'
\label{ex:FBC-topic-subject-rc-PPdequi}
\z 



\begin{sidewaysfigure}[hp]
\resizebox{!}{12cm}{%
\begin{forest}
where n children=0{tier=word}{}
[NP\\
\avm{[\type*{head-mod-structure}\\
      clause-key & \tag{e1}]}
[NP\\
\avm{[synsem & \1 [index & \tag{i}\\
                clause-key & \tag{e1}]]}
[Christelle\\Christelle]]
[S\\
\avm{[\type*{wh-rel-cl}\\
      mod & \1\\
      clause-key & \tag{e}\\
      c-cont|icons & <[\type*{topic}\\
                        target & \tag{i}\\
                        clause & \tag{e2}]>\\
      slash & \{\}]}
    [PP \\
    \avm{[loc & \2 [icons-key & \3]]}
    [P\\
    \avm{[icons-key & \3]}
    [de\\of]]
    [NP\\
    \avm{[icons-key & \3 non-focus]}
    [la soeur de qui\\the sister of who, roof]]
    ]
    [S\\
    \avm{[\type*{head-subj-structure \& non-frame-setting}\\
          index & \tag{e}\\
          icons & < ... \4 ...>\\
          slash & \{\2 [index & \tag{i}]\}]}
          [\avm{\5} NP
            [\avm{\6} Det [l'\\the]]
            [N\\
            \avm{[head & noun\\
                hook & [icons-key & \4 [\type*{aboutness-topic}\\
                                     clause & \tag{e2}\\
                                     target & \tag{j}]\\
                        index & \tag{j}]\\
                slash & \{ \2 \}\\
                arg-str & < \6, [loc & \2] >]}
            [arrogance\\arrogance]]
          ]
          [VP\\
          \avm{[slash & \{\2\}]}
            [V\\
            \avm{[dt & <\5>\\
                  subj & <\5>\\
                  slash & \{\2\}]} [rebute\\repels]]
            [NP [mes collègues\\my colleagues, roof]]
          ]
    ]
]
]
\end{forest}}
\caption{Simplified tree for \textit{Christelle,} [[\textit{de la soeur de qui}]$_i$ [\textit{l'arrogance}~\trace{}$_i$] \textit{rebute mes collègues}] (`Christelle, whose sister's arrogance repels my colleagues')}
    \label{fig:FBC-topic-subject-rc-PPdequi}
\end{sidewaysfigure} 
