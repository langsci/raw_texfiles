\chapter{Concluding remarks}
The present work aimed to contribute to the current debate on subextraction from subjects in French. Subjects have often been considered to be ``island'' environments, so in the first part of the book I discussed ways to define islands and the classification of sentential subjects and subjects in general as islands. I also presented a state of the art identifying three main directions in accounting for the phenomenon: syntax-based approaches (often defended by Minimalists, but also by researchers working in other frameworks, as well as in the early HPSG analyses), processing-based approaches and discourse-based approaches. The focus of this book is on a discourse-based hypothesis that I presented with some colleagues in \citet{Abeille.2020.Cognition} and called the Focus-Background Conflict (FBC) constraint. The FBC constraint states that part of a background cannot be focused. 

I have shown that there is disagreement on the data where subextraction out of the subject is concerned: Is it ungrammatical or only degraded? Are the constraints cross-linguistically valid? Do they apply in all constructions? Are all subjects affected by the same constraints? Working on French, where the data have been particularly debated \citep[a.o.][]{Godard.1988,Tellier.1991,Heck.2009}, I gathered empirical evidence that enabled me to answer these questions at least partially.

In the second part of the this work, I presented the results of eight corpus studies and 16 experiments. 

The corpus studies show that extraction out of the subject is very common in relative clauses. In fact it is the most common usage with the complementizer \emph{dont} (`of which'). Long-distance dependencies are attested as well, which shows that \emph{dont} relative clauses involve extraction, contrary to a proposal made by \citet{Heck.2009}. Relativization out of the subject with \emph{de qui} (`of who') is common, contrary to \citegen{Tellier.1991} claims, and so is relativization out of the subject with \emph{duquel} (`of the which'). Relativization out of the subject with \emph{avec qu-} (`with \textit{wh-}') is not very common, but attested. The important finding of my corpus studies is that there is no attested extraction out of the subject in interrogatives. Because interrogatives, unlike relative clauses, are focusing constructions, I argue that these findings are in line with the FBC constraint, while the cross-construction difference is unexpected under the other hypotheses.

The experiments tested extraction out of the subject on different constructions (relative clauses, interrogatives, \emph{it-}clefts), and compared it with extraction out of the direct object. All experiments on relative clauses show that extraction out of the subject is not degraded compared to extraction out of the object: it receives similar or even higher acceptability ratings and does not cause a slow-down during reading. In interrogatives with short-distance dependencies, ratings for extraction out of the subject are lower than for extraction out of the object, but only in \emph{wh}-questions and not in questions with \emph{wh} in situ. Furthermore, extraction out of the subject in long-distance dependencies does not show a significant decrease of acceptability. In \emph{it}-clefts, there is a tendency such that extraction out of the subject receives lower ratings than extraction out of the object. Since interrogatives and \emph{it}-clefts are focalizing constructions, unlike relative clauses, the FBC constraint seems to capture the results quite well.

Finally, I presented a formalization of the FBC constraint. I clarified some details about the FBC constraint, especially the way it applies to long-distance dependencies and \emph{it}-clefts. Then I proposed a formalization of the FBC constraint in HPSG. For this, I sketched the basis for a French HPSG fragment in which I adopt \citegen{Song.2017} representation of information structure in MRS objects \citep{Copestake.2005}. The FBC constraint states that parts of a non-focus element (elements in its \textsc{arg-str} list) must be non-focus as well. Extraction out of the subject in interrogatives and clefts is therefore not ruled out, but must have a specific interpretation. Extraction out of the subject in relative clauses is not constrained by the FBC, because the extracted element is non-focus.

Throughout the present work, while emphasizing the impact of the FBC constraint, I also said that many factors are at play in extraction, and that non-discourse factors such as the complexity of the subject, the length of the dependency, the number of gaps or the familiarity with the structure (habituation) are all important. Therefore, I am not claiming that the FBC constraint by itself completely accounts for what is called the ``subject island''. On the other hand, the formulation of the constraint clearly implies that it is not restricted to extraction (let alone to subextraction from the subject). It may, therefore, offer explanatory potential far beyond the scope I identified in this book.

% Erteschik-Shir (in press) points out that all modes of perception are organized'into foreground and background constituents; in this sense, focusing, a task-specific mechanism which identifies the foregrounded constituent, is a universal property of perceptual systems and hence not unique to language 
% Erteschik-Shir, The Dynamics of Focus Structure

I leave for future research the task to see in which respect the FBC constraint is able to account for other structures identified as ``islands'' in the literature. For example, restrictive relative clauses are presupposed and as such fall under by the FBC constraint. Relative clauses are claimed to be islands to extraction, but there has been no report of a difference between relative clauses and interrogatives. Similarly, complements of factive verbs are claimed to be islands to extraction. \citet{Ambridge.2008} explain the contrast between bridge verbs and factive verbs through backgroundedness. Example (\ref{ex:bridge-factive}) shows that extraction out of the complement of bridge verbs (\ref{ex:bridge-factive-bridge}) is more acceptable than extraction out of the complement of factive verbs (\ref{ex:bridge-factive-factive}). 

\eal \label{ex:bridge-factive}
\ex \citep[371]{Ambridge.2008}\\
What did Jess think that Dan liked? \label{ex:bridge-factive-bridge}
\ex[?]{What did Jess know that Dan liked? \label{ex:bridge-factive-factive}}
\zl 

According to \citet{Liu.Y.2019}, the acceptability of subextraction from a sentential complement is better explained by the frequency of the introducing verb than by the propositional status of the sentential complement, but they tested only interrogatives. In these two cases alone, the FBC constraint could be shown by future research to explain previously ill-understood problems in linguistics.

Even though my focus in this book was on French, there are strong reasons to expect that the FBC constraint applies cross-linguistically. If we assume that it is a reflection in the language of a more general principle of cognitive attention, then it should have an impact in any language. However, languages seem differ in their sensitivity to the constraint as far as extraction out of the subject is concerned. In French interrogatives, subextraction from the subject is rated clearly higher than ungrammatical controls. In parallel experiments on English, however, extraction out of the subject receives ratings that are not significantly higher than ungrammatical controls \citep[third experiment]{Abeille.2020.Cognition}. By contrast, subextraction from the subject in Japanese is claimed to be acceptable in all constructions \citep[a.o.][]{Ross.1967,Kuno.1972,Kayne.1983,Stepanov.2007}.%
\footnote{Notice that extraction out of relatives is supposed to be acceptable in Japanese as well.
\begin{itemize}
    \item[(i)] \citep[15]{Kuno.1987}
    \item[] \gll Kore~wa~[[\trace{} kawaigatte~ita] neko~ga sinde~simatta] kawaisoo~na kodomo desu.\\
this loved cat died poor child is\\
\glt `This is a poor child who the cat that (he) loved died.'
\end{itemize}}
More research is needed in order to see what may cause these cross-linguistic differences, and why languages seem to be more or less constrained by the FBC. This may be related to how information structure manifests itself in these languages. 

From a methodological point of view, the present work has shown that linguistic research can greatly profit from carefully conducted corpus studies and experiments. This is not to say that intuitive judgments are necessarily unworthy. In fact, on many occasions I was able to confirm or explain linguists' intuitions. The problem comes from a misinterpretation of these intuitions. Some contrasts may not receive the attention they deserve, as for example the contrast between extraction out of the subject in interrogatives and in relative clauses, which was mentioned in early works \citep[32]{Chomsky.1973} but hardly ever addressed afterwards. On the other hand, some contrasts may be considered by scholars to be more important than they actually are, and we can often observe a tendency to overgenerate an observation. For example, the contrast between extraction out of the subject with \emph{dont} vs.\ \emph{de qui} in French asserted by \citet{Tellier.1991} has some grounds, but it seems to come from a general preference for subextraction out of NPs with \emph{dont} and not from a problem specific to subextraction from subjects. A preference for extraction out of subjects of passives over extraction out of subjects of transitives can also be explained by the fact that transitive verbs tend to have less complex subjects, but this does not mean that subjects of passives are special as far as extraction is concerned. Looking carefully at the corpora and testing different extraction types with the appropriate controls allowed me to clarify some inadequate intuitions while partially explaining where they come from. 

%\begin{quote}
%    In order for there to be a crisis, however, it would need to be the case that (i) Intuitive  judgments  have  led  to  generalizations  that  are  widely  accepted  yet bogus. (ii) Misleading judgments form the basis of important theoretical claims or  debates.   (iii)   Carefully  controlled   judgment   studies  would   solve  these problems. \citep[3]{Phillips.2009}
%\end{quote} --> I did no find the book and could not check the quote

%\begin{quote}
%    In opening the investigation of core linguistic phenomena to processing approaches, the possibility arises that the explanatory mechanisms involved may not be specific to language. This then raises the interesting question of how such processes might be utilized across cognitive domains. This possibility does not even arise in a competence-based approach, because linguistic competence is domain-specific by definition. \citep[245]{Kluender.1998}
%\end{quote}
