\author{Elodie Winckel} %use this field for editors as well
\title{French subject islands}
\subtitle{Empirical and formal approaches}
\renewcommand{\lsSeries}{eotms}
\renewcommand{\lsSeriesNumber}{14}
\ISBNdigital{978-3-96110-477-2}
\ISBNhardcover{978-3-98554-106-5}
\BookDOI{10.5281/zenodo.11635096}
% \typesetter{}
\proofreader{%
Amir Ghorbanpour,
Annie Zaenen,
Brett Reynolds,
Elliott Pearl,
Hannah Schleupner,
Janina Rado,
Mary Ann Walter,
Wilson Lui
}
% \lsCoverTitleSizes{51.5pt}{17pt}

\BackBody{This book examines extractions out of the subject, which is traditionally considered to be an island for extraction. There is a debate among linguists regarding whether the “subject island constraint” is a syntactic phenomenon or an illusion caused by cognitive or pragmatic factors. The book focusses on French, that provides an interesting case study because it allows certain extractions out of the subject despite not being a typical null-subject language. The book takes a discourse-based approach and introduces the “Focus-Background Conflict” constraint, which posits that a focused element cannot be part of a backgrounded constituent due to a pragmatic contradiction. The major novelty of this proposal is that it predicts a distinction between extractions out of the subject in focalizing and non-focalizing constructions.

The central contribution of this book is to offer the detailed results of a series of empirical studies (corpus studies and experiments) on extractions out of the subject is French. These studies offer evidence for the possibility of extraction out of the subject in French. But they also reveal a clear distinction between constructions. While extractions out of the subject are common and highly acceptable in relative clauses, this is not the case for interrogatives and clefts.

Finally, the book proposes a Head-Driven Phrase Structure Grammar (HPSG) analysis of subject islands. It demonstrates the interaction between information structure and syntax using a representation of information structure based on Minimal Recursion Semantics (MRS).}
\renewcommand{\lsID}{362}
