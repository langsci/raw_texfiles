\author{Jean Nitzke and Silvia Hansen-Schirra}
\title{A short guide to post-editing}

\renewcommand{\lsSeries}{tmnlp}
\renewcommand{\lsSeriesNumber}{16}

\BackBody{Artificial intelligence is changing and will continue to change the world we live in. These changes are also influencing the translation market. Machine translation (MT) systems automatically transfer one language to another within seconds. However, MT systems are very often still not capable of producing perfect translations. To achieve high quality translations, the MT output first has to be corrected by a professional translator. This procedure is called post-editing (PE). PE has become an established task on the professional translation market. The aim of this text book is to provide basic knowledge about the most relevant topics in professional PE. The text book comprises ten chapters on both theoretical and practical aspects including topics like MT approaches and development, guidelines, integration into CAT tools, risks in PE, data security, practical decisions in the PE process, competences for PE, and new job profiles.}

\typesetter{Jean Nitzke, Silvia Hansen-Schirra, Sebastian Nordhoff}
\proofreader{%
Alena Witzlack,
Annika Schiefner,
Bev Erasmus,
Jeroen van de Weijer,
Lachlan Mackenzie,
Marten Stelling,
Oliver Czulo,
Prisca Jerono,
Tihomir Ragenlov
}

\renewcommand{\lsID}{319}
\renewcommand{\lsISBNdigital}{978-3-96110-333-1}
\renewcommand{\lsISBNhardcover}{978-3-98554-029-7}
\BookDOI{10.5281/zenodo.5646896}


