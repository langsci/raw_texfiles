\chapter{Food for thought and wrap-up}\label{sec:10}

PE has changed the field of professional translation and the market has become more diverse with more possibilities to transfer a source text into a target text. Hence, professional translators have to adapt to these changes and have to decide whether they want to broaden their range of services and offer PE. If so, many practical aspects have to be considered - many of which we could not address in this short textbook, amongst others because these aspects are often very individual. 

One example is price calculation. As in translation, there are possibly many different ways to calculate prices, e.g. per source/target text character/word/line, per hour, according to the editing distance, calculation of MT segments equally to fuzzy match segments, etc.\footnote{If you want to learn more about how to generate time or editing distance reports in memoQ, read this article \url{https://blog.memoq.com/time-tracking-and-editing-distance-reporting}, last accessed on 21 April 2021. The article is well written and might give you interesting insights even if you do not use memoQ.} Reasons for choosing one or the other are similarly manifold. Hence, it might also be reasonable to decide on  each project individually depending on the given constraints and characteristics. In the end, both translators and clients should profit from the PE process. Translators should save time (and hence gain money, not lose money) and clients should save money, as well. You can also find some more information in the \href{https://www.taus.net/academy/best-practices/postedit-best-practices/pricing-machine-translation-post-editing-guidelines}{TAUS pricing guidelines}\footnote{last accessed 11 June 2021}.


As we already briefly discussed in \sectref{sec:2}, PE has also brought interesting new aspects into the field of research, which we cannot discuss in more detail here. However, we would like to mention that both practice and research are in close dialogue with each other. \citet{arenas2014correlations}, for example, presents a study on the productivity and quality of PE in a translation memory tool compared to fuzzy and no matches. After presenting the results, she also discusses how those variables influence the pricing and that the potential benefits of PE jobs vary for each translator individually. \begin{quote}``[...] and it might be difficult to find a satisfactory solution to determine a “fair” price. In most cases, the translators should really analyze if the compensation scheme applied for a particular project is beneficial for them according to the productivity they experience during this job or series of similar jobs. Moreover, they should also consider if the use of MT and TM segments might benefit the quality they deliver, as we have seen in this study." \citet[183]{arenas2014correlations}\end{quote}
Also, different publications on PE by professional associations, e.g. \citet{ottmann_best_2017} or \citet{porsiel_machine_2017}, combine contributions of researchers and professionals.

Another aspect we only partly -- or indirectly -- focused on is machine translation ethics. Ethics is a large and important topic in artificial intelligence \citep{liao2020ethics} as with advancing AI, more and more decisions have to be made. One of the most famous ethical questions in AI is concerning self-driving cars. Although it is assumed that fewer accidents will happen when the human error is eliminated in traffic, the question remains whom to harm in an emergency situation with different actors involved \citep{bonnefon2016social}. The ethical dilemmas do not seem to be that extreme for MT systems. Nonetheless, it seems appropriate to discuss this aspect. 

So far, there has been little research on MT ethics. The main focus of the existing studies has been mainly on translating literature (e.g. \citealt{taivalkoski2019ethical} or \citealt{kenny2020machine}), which is of course not the only area that should be concerned with ethics. Some important issues are raised in \cite{moorkens2020ethics}. They discuss, for example, that the ownership of data and translations is a matter of ethical considerations as large amounts of high-quality human translations are needed to train MT systems. The use of the data is often not transparent, and it is often not clear who was asked for permission to use the data for MT training. Further, MT ethics should also be concerned with reporting the use of MT and PE giving the reader the knowledge which MT system, which PE style, and which post-editor were involved in creating the texts. This should also be true for domain-specific texts, where usually not even the translators are listed. Although most readers might not be interested in the nature of the translation, it might force the clients to fairer processes as a long list of different MT systems and post-editors, who only light post-edited the MT output might reflect badly on the client/company.

Additionally, when we talk about highlighting MT use, we might want to talk about the use of MT on websites, where the MT output is not post-edited. Let's take as an example the website of Tripadvisor\footnote{\url{https://www.tripadvisor.de/}, last accessed 10 February 2021}, where, among others, users can rate hotels, restaurants, etc. The website implements Google Translate and according to the locale settings, a translation of the users' comments is presented automatically. Under the comments there is a note in an unremarkable, gray font that the translation was created by Google and that the reader has the possibility to rate the translation. It is quite likely that many users do not see this information when they read the comment. From our perspective (taking aside all the benefits an MT systems provides on such websites), it should be open to discussion whether this report of MT use is sufficient and whether it is ethical to present it automatically.

We wanted to finish the book with this little bit of food for thought. Post-editing is still a new area and thanks to the ongoing technological developments and innovations in artificial intelligence, many changes and new challenges are still to come. Now that we are at the end of the book, think about what has changed in your perception? How do you feel about MT and PE now? You might want to go back to \sectref{sec:1} and \sectref{sec:2} and look at the answers you gave at the very beginning of this short introduction to PE. Do you still agree with your assessment or has something changed?

\newpage

\section*{Solutions to crossword puzzles}

\textbf{Section 2}

\begin{enumerate}
  \item TRANSLATOR
  \item EFFORT
  \item CATTOOLS
  \item PREEDITING
  \item RELEVANCE
  \item EMPIRICAL
  \item EYETRACKING
  \item CRITT
\end{enumerate}

\textbf{Section 3}

\begin{enumerate}
  \item WEAVER
  \item GEORGETOWN
  \item RUSSIAN
  \item ALPAC
  \item SYSTRAN
  \item WEATHERFORECASTS
  \item BABELFISH
  \item STATISTICAL
  \item INTERLINGUA
  \item NEURAL
\end{enumerate}

\textbf{Section 4}

\begin{enumerate}
    \item ENOUGH
    \item SYNTAX
    \item OUTPUT
    \item TERMINOLOGY
    \item MONOLINGUAL
    \item FULL
\end{enumerate}

\textbf{Section 5}

\begin{enumerate}
    \item CONTROLLED
    \item CREATIVITY
    \item RESTRICTIVE
    \item SUBTITLES
\end{enumerate}

\textbf{Section 6}

\begin{enumerate}
    \item SEGMENT
    \item MANAGEMENT
    \item TERMINOLOGY
    \item FUZZYMATCH
    \item EMPTY
    \item INTERACTIVE
    \item ADAPTIVE
\end{enumerate}

\textbf{Section 7}

\begin{enumerate}
    \item OPERATIVE
    \item STRATEGIC
    \item CONFIDENTIAL
    \item LIABILITY
    \item RISKS
\end{enumerate}

\textbf{Section 8}

\begin{enumerate}
    \item PREPARATION
    \item REVISION
    \item SENSITIVITY
    \item DISTANT
    \item DEFECTIVE
    \item READABILITY
    \item SPELLING
    \item TURNAROUND
\end{enumerate}

\textbf{Section 9}

\begin{enumerate}
    \item TRANSLATION
    \item EXTRALINGUISTIC
    \item CORRECTION
    \item CONSULTING
    \item ENGINEERING
    \item PSYCHOPHYSIOLOGICAL
    \item ERRORHANDLING
\end{enumerate}
