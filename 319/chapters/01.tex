\chapter{Let's get started...}\label{sec:1}
 \largerpage[-1]
Artificial intelligence is changing and will continue to change the world we live in. Many industries and jobs are also changing, with some jobs even vanishing. Since the industrial revolution, the human work force has been increasingly replaced by machines. Some people are scared and fear for their jobs. Others are happy that mediocre jobs can finally be carried out by machines and technologies, and that humans can concentrate on more meaningful work. 

These changes are also influencing the translation market. Machine translation (MT) systems automatically transfer one language to another within seconds and are coming close to achieving a dream that humans have had for centuries: the ability to overcome language barriers. However, MT systems have existed for over 70 years now and are still not capable of producing perfect translations. So, how do these technologies influence the market? Are translators or language service providers (LSPs) on the verge of extinction?

The general translation market is continuing to grow. And the demand is huge. Common Sense Advisory (CSA) research, founded in 2002, conducts what they consider “independent, objective” research on “the global content and language services markets” (csa-research.com\footnote{ \href{https://csa-research.com/More/Media/Press-Releases/ArticleID/546/Global-Market-for-Outsourced-Translation-and-Interpreting-Services-and-Technology-to-Reach-US-49-60-Billion-in-2019}{“Global market for outsourced translation and interpreting services and technology to reach US\$49.60 billion in 2019.”}, last accessed 07/10/2020}). Let us look at the following two statements. First, CSA research “found that the market for language services and supporting technologies will grow 6.62\% from 2018 to 2019 […]. The industry’s compound annual growth rate over the last 11 years was 7.76\%” (csa-research.com). And the results were very similar in the years before. DePalma, who is the Chief Research Officer of CSA Research, comments on these developments: ``People worldwide prefer consuming information in their own language. Meeting this expectation […] fuels an indispensable global industry that continues growing due to global digital transformation (GDX)." (csa-research.com) Even during the COVID-19 pandemic, when steady reporting, market evaluations, and forecasts were not possible, CSA outlined that ``preliminary revenue reports from LSPs [...] have produced better than expected returns for calendar year 2020" and predicted ``that the language services industry will grow faster than the overall economy in 2021" (csa-research.com\footnote{\href{https://insights.csa-research.com/reportaction/305013272/Marketing}{Sizing the language industry: March 2021 update}, last accessed 25/05/2021}) in March 2021.

So, the market itself is growing, but how do the MT systems influence these developments? There are different opinions on this topic. If we listen to the statement of a German politician, the prospects are rather bleak. Lars Klingbeil (SPD) stated in the political talk show \textit{Anne Will} in November 2018, where they discussed changing work environments due to AI in general, that 
\begin{quote}Soon, whole industries will be gone […] Industries that still exist at the moment, that we still need now, but they will be gone in a few years, because of artificial intelligence, because of technical developments. So, the question is, how will we, as the government, take care of the people working in these industries? Let’s discuss translators, interpreters for example […] in a couple of years, we will not need their services anymore, because technological developments will render them useless.\footnote{translated by authors. Original quote: „Es werden bald ganze Branchen verschwinden. […] die noch da sind, die gebraucht werden, aber die in den nächsten Jahren verschwinden werden, durch künstliche Intelligenz, durch technologische Entwicklung. Und da ist die Frage, wie stellt der Staat sich eigentlich gegenüber den Menschen auf, die da arbeiten. Ich nehme mal nur das Beispiel der Übersetzer, der Dolmetscher […] die wird es in ein paar Jahren als Dienstleister nicht mehr geben, weil technologische Entwicklung das überflüssig macht.“}\end{quote}

After the talk show was aired, many professional translators were outraged and many voices were raised. The BDÜ\footnote{https://bdue.de/der-bdue, last accessed 15/12/2020} – a German professional association for interpreters and translators – responded to Klingbeil’s statement: 
\begin{quote}Digitalisation and developments in the area of artificial intelligence (AI) are changing the working environment […] However, these developments have always influenced our industry in particular, and we have always known not only how to adapt to the circumstances but also how to use them to our advantage by harnessing the technology instead of fighting it. […] Additionally, the translation industry is growing due to globalisation and digitalisation.\footnote{translated by authors. Original quote: „Mit der Digitalisierung und den Fortschritten im Bereich Künstlicher Intelligenz (KI) verändern sich die Arbeitsbedingungen […] Derartige Entwicklungen haben aber gerade diesen Berufsstand schon von jeher begleitet und er hat es immer wieder verstanden, sich den neuen Bedingungen nicht nur anzupassen, sondern diese sinnvoll zu nutzen. Und zwar unter Zuhilfenahme der technischen Werkzeuge und nicht im Wettlauf gegen sie. […] Im Zuge der Globalisierung und Digitalisierung wächst zudem seit Jahren der Bedarf an Übersetzungen.“}\end{quote}

As mentioned above, the influence of technology and AI is not only noticeable in the translation industry, but in the majority of industries. The economist \citet[8-9]{autor2014polanyi} argues that many everyday tasks cannot be automated, because ``we don't know 'the rules'" - something that is challenged by machine learning - and he continues that
\begin{quote}
[t]he fact that a task cannot be computerised does \textit{not} imply that computerisation has no effect on that task. On the contrary: tasks that cannot be substituted by computerisation are generally complemented by it. This point is as fundamental as it is overlooked. Most work processes draw upon a multifaceted set of inputs: labor and capital; brains and brawn; creativity and rote repetition; technical mastery and intuitive judgment; perspiration and inspiration; adherence to rules and judicious application of discretion. (emphasis in original quote)
\end{quote}

\citet[267-268]{bowker2020translation} paints a much more pessimistic picture in her article about translation technologies and ethics:
\begin{quote}
    Several authors […] highlight a major risk associated with using CAT\footnote{short for ``computer-assisted translation"} tools: the concealing, overshadowing or downgrading of the translator’s contribution. Rather than seeing a translator who interprets a source text’s meaning and intention and renders these in an appropriate target text, clients may perceive the language professional as a copy editor who simply makes minor revisions to the “real” work that has been largely done by a machine, which has retrieved the correct solutions from its database or corpus.
\end{quote}

On the other hand, there are also more optimistic voices. \citet{depalma_augmented_2017} and \citet{lommel_augmented_2020} describe a model for augmented translation in the CSA Research blog. In this approach, they argue that translators are at the centre of the translation process, surrounded by different technologies that support their work. The augmented translators are provided
\begin{quote}
    with more context and guidance for their projects. They work in a technol\-o\-gy-rich environment that automatically processes many of the low-value tasks that consume an inordinate amount of their time and energy. It brings relevant information to their attention when needed. This computing power will help language professionals be more consistent, more responsive, and more productive, all the while allowing them to focus on the interesting parts of their jobs rather than on “translating like machines.” \citep{depalma_augmented_2017}
\end{quote}

So where do we stand at the moment? Will one of the oldest professions become extinct? Will the work environment of translators 'simply' change? Or has the profession taken a step forward and started to release the professionals from redundant and boring work? As we already mentioned at the very beginning, MT output is still not perfect, neither linguistically nor in terms of content. This is true for all language combinations, although different quality in the output can be observed for different language pairs. For some language directions and domains, the quality might even be exceptionally good.\footnote{See also the discussion concerning human parity, e.g. \citet{laubli2020jair}.} The quality depends on various factors, among them text type, amount and quality of training data, but also similarity of the languages. To achieve high quality translations, the MT output has to be post-edited, which will be the topic of this textbook.

\bigskip

Post-editing (PE) has become a well-established task for professional translators. The raw machine-translated output can help the post-editor to accelerate the translation process and to make the translation process more profitable and less expensive for the client. However, the professional post-editor needs basic knowledge of MT and post-editing to assess PE tasks and to make informed decisions. This textbook will give you an introduction to the most relevant topics in professional PE. We assume, of course, that you as a user of this textbook already have translation experience either as a professional translator or as a translation student. Similar to the professionalisation in translation-from-scratch, we can only provide you with a starting point for PE. To make your assignment truly effective and profitable, you will need to practise the task with real PE jobs.
 
We will provide you with examples and practical scenarios that you can apply to your specific PE tasks and that will guide you through your first steps as a professional post-editor.
 
The textbook will be structured as follows. First, we will give you a brief general introduction to post-editing in \sectref{sec:2} and introduce machine translation basics in \sectref{sec:3}. In \sectref{sec:4}, we will discuss different guidelines for the PE task. Then, we will talk about PE in general and crucial concepts like different text types in MT (\sectref{sec:5}), how PE can be integrated into CAT tools (\sectref{sec:6}), risks in PE and data security (\sectref{sec:7}), practical decisions for PE jobs (\sectref{sec:8}), required competences for PE, new job profiles and training opportunities (\sectref{sec:9}). And finally, we will wrap things up in \sectref{sec:10}.

At the end of \sectref{sec:2} to \sectref{sec:9} you will find a little crossword puzzle on the contents of the preceding chapter so you can review some of the buzzwords and main concepts (or every now and again some details) of the chapter.\footnote{We used the package \textit{cwpuzzle} to create the crossword puzzles in LATEX. \citep{neugebauer_latex_2020}} Feel free to go back if you can't remember the answer. Maybe you want to reflect a little on the concepts behind the answers.

Finally, we want to point out that you can find many PE exercises on the \href{https://learn.digiling.eu}{website of the Digiling project} that will complement the contents of this book. Please do not hesitate to create a free account and use the materials to get more insights into practical PE and to strengthen your knowledge.

\bigskip

Enjoy!
