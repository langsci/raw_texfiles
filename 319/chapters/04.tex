\chapter{Post-editing guidelines -- how to post-edit?}\label{sec:4}


\objectives{
        You will learn...
        \begin{itemize}
            \item about different guidelines, especially light and full PE,
            \item about monolingual PE and the associated problems,
            \item what is included in the PE Norm,
            \item how to adhere to different guidelines.
        \end{itemize}
        }

\vspace{\baselineskip}

Every translation and thus also every PE project has different requirements. Especially in PE, the required target text quality might vary a lot, depending on various factors like reader, distribution, or duration of use. Hence, the effort invested in the PE process might differ and how much effort should be invested while post-editing a text has to be defined in advance. 

\section{Considerations on PE guidelines}\label{sec:4:1}

You might wonder why we need guidelines for PE. PE is not per se intended to generate perfect, high quality texts. The main goal is to save time and money. Therefore, it is very important to define the final quality criteria and editing needs in advance. You have to ask yourself different questions to judge the required target text quality and thus the PE effort that needs to be invested. As these decisions are usually not made by the post-editors themselves, we will discuss the decision making process later in \sectref{sec:8}. For now, we have to keep in mind that we need PE guidelines to achieve the defined target text quality, which should ideally be communicated with the post-editing job. In addition, guidelines help us to make the target text as consistent as possible even if different post-editors work on the same text -- similar to the processes in technical documentation or other domain-specific translations. 

In general, we differentiate between light vs. full PE. This differentiation is quite superficial for real PE projects as discussed in \citet{nunziatini2020machine}. The guidelines might differ a lot from project to project and every PE project might focus on different quality aspects, e.g. using the correct, pre-defined terminology might be far more important in technical documentation than in post-editing a newspaper article. However, the differentiation between light and full PE is well-established and will give you an impression of the continuum in which we are working.

The guidelines we want to introduce to you were established by TAUS (Translation Automation User Society).\footnote{Find the guidelines and further recommendations on their \href{https://www.taus.net/academy/best-practices/postedit-best-practices/machine-translation-post-editing-guidelines}{webpage}, last accessed on 28 April 2021.} The society was established in 2005. It is an independent organisation that is concerned with automation and innovation in translation. 

\subsection{Light PE}\label{sec:4:1:1}

The first set of guidelines we want to talk about are what TAUS calls the guidelines for achieving “good enough” quality.\footnote{See also the discussions under the buzzword "fit-for-purpose translation", e.g. \citet{bowker2019}.} This equals what is generally considered light PE.

The criteria for “good enough” are defined by TAUS as follows: 

\begin{itemize}
    \item comprehensible: the contents of the text should be comprehensible
    \item accurate: the meaning of the source text should be preserved
    \item but stylistic quality plays a minor role
\end{itemize} 

This means that the text may appear unidiomatic and unnatural as it is generated by a computer. The grammar and syntax can be incorrect as long as the meaning is comprehensible. Concerning the guidelines for light post-editing, TAUS puts it this way: 

\begin{itemize}
    \item Aim for semantically correct translation. 
    \item Ensure that no information has been accidentally added or omitted.
    \item Edit any offensive, inappropriate or culturally unacceptable content.
    \item Use as much of the raw MT output as possible.
    \item Basic rules regarding spelling apply.
    \item No need to implement corrections that are of a stylistic nature only.
    \item No need to restructure sentences solely to improve the natural flow of the text.
\end{itemize}

The greatest challenge in light PE for most professional translators is leaving incorrect grammar and syntax unedited as we are usually used to creating high-quality translations. Keep in mind that you will not be paid for those corrections and try to remind yourself that for this job high quality is not needed. Light PE requires training and it will become easier to adhere to the guidelines after a while.

\subsection{Full PE}\label{sec:4:1:2}

Another set of guidelines are the guidelines for achieving quality that is similar or equal to human translation, often called full post-editing.
Besides being comprehensible and accurate (see above), stylistic quality is also important for full PE. However, it may still not be as good as it would be when translated from scratch. Syntax, grammar and punctuation need to be correct.
Concerning the guidelines for full post-editing, TAUS puts it this way: 

\begin{itemize}
    \item Aim for grammatically, syntactically and semantically correct translation.
    \item Ensure that key terminology is correctly translated and that untranslated terms belong to the client’s list of “Do Not Translate” terms.
    \item Ensure that no information has been accidentally added or omitted.
    \item Edit any offensive, inappropriate or culturally unacceptable content.
    \item Use as much of the raw MT output as possible.
    \item Basic rules regarding spelling, punctuation and hyphenation apply.
    \item Ensure that formatting is correct.
\end{itemize}

Full PE is more in line with the quality standards most professional translators are used to. However, it is still important to remember to use as much of the raw MT output as possible and to not get too lost in the fine-tuning.

\subsection{Monolingual PE}\label{sec:4:1:3}

Especially when talking to laypersons, the argument might come up that a post-editor only needs to be fluent in the target language, because the translation was created by the machine and the fine-tuning of the MT output takes place in the target text. However, think about whether you as a professional translator would be willing to revise a translation for which you do not know the source language or for which you do not have access to the source text. You would probably be reluctant to do so, because you know that many mistakes cannot be identified without the source text, especially content mistakes. And, as you can probably guess, this is also true for post-editing MT output -- maybe even more so.

Bilingual PE involves the comparison of source and target text. This means that the post-editor has to check the quality of the translation but he/she also has to assess whether the adequate meaning of the source text has been transferred by the MT system. In contrast, monolingual PE (MPE) suggests that the quality control of the translation can be carried out without taking the source text into account. The professional translation market and research on PE in translation studies hardly discuss MPE, but you should be aware that the option of MPE might come up when negotiating with clients. To have some arguments prepared, we want to discuss the topic briefly and present some findings of research on monolingual PE.

\citet{mitchell2013community} showed in their study that monolingually post-edited sentences are most often rated as an improvement to the pure MT output. However, there are also a number of incidents where the monolingual PE processes even negatively affected the final quality. This shows that monolingual PE mostly has a positive or neutral effect on the MT output, but it does not say whether the final product was good or even acceptable -- it was merely better than the pure MT output.

In her study, \citet{nitzke2016monolingual} compared the translation and PE products of 24 participants (twelve professionals, twelve students) for six texts that were translated/ post-edited from English to German. She found that ``superficial" mistakes like grammar, spelling, punctuation, etc. occurred similarly often in the final target texts for all three tasks (translation, bilingual PE, monolingual PE). However, content mistakes could be found much more frequently in the monolingually post-edited final texts than when the texts were translated from scratch or bilingually post-edited. However, we have to keep in mind that both studies used MT output of statistical MT systems.

When we talk about NMT, the picture is the same but different. Studies (e.g. \citealt{burchardt2017linguistic} or \citealt{macketanz2017machine}) showed that NMT generated translations still contain many errors, although it might seem that the quality of the MT output has improved for some language pairs as the output can be read more fluently. However, we assume that it might even become more difficult to find certain error types, especially content errors, even in bilingual PE, because the MT output is fluent and seems to be correct. Hence, the content mistakes are, in a way, more hidden.

Even though there might be some scenarios where monolingual PE may be sufficient or better than no PE at all (similar to how proof-reading a target text instead of revising it against its source text might be better than no quality assessment at all), we discourage you from engaging in monolingual PE in professional PE settings as some mistakes in the MT output cannot be found without consultation of the source text. Since monolingual PE neglects the assessment of the equivalence relations between source and target text, the adequacy of the meaning of the MT output cannot be evaluated at all. This therefore means that it is not possible to evaluate whether the output is accurate, i.e. whether the meaning of source and target text is identical. This is why we strongly recommend refusing jobs involving solely monolingual PE. The adherence to given standards (\sectref{sec:4}) is not possible at all for this kind of task and questions concerning risk assessment and liability (\sectref{sec:7}) cannot be addressed at all. 

\section{ISO 18587 -- the post editing standard}\label{sec:4:2}

In addition to the general guidelines, there is also an ISO standard addressing professional PE workflows. The PE standard ISO 18587 is called “Translation services — Post-editing of machine translation output — Requirements.” After having introduced the differences between light and full PE, we will now focus on this international standard and its requirements. We refer to the first edition which was published in April 2017.

The standard first discusses the reasons and advantages of post-editing machine translation: Translation costs can be decreased, the launch of products or the flow of information can be accelerated, translation productivity as well as the turn-around times can be improved, and translation service providers can remain competitive in a globalised world. Additionally, MT gives clients the possibility to translate material that could otherwise not be translated at all.

\begin{quote}
“However, there is no MT system with an output which can be qualified as equal to the output of human translation and, therefore, the final quality of the translation output still depends on human translators and, for this purpose, their competence in post-editing.” (DIN ISO \citealt[Introduction]{din_iso_18587_ubersetzungsdienstleistungen_2018})  \end{quote}

This is an important statement that can help to regulate expectations of what MT and PE can and should deliver. The ISO standard does not apply to general MT developments or general translation processes, but only to PE processes. Further, the scope of the standard is that it refers to ``the process of full, human post-editing of machine translation output and post-editors’ competences". This means that it does not apply to light PE jobs, to automatic PE, to interactive PE, or to monolingual PE. 

The standard coins its own definition of PE: ``Post-editing is performed on MT output for the purpose of checking its accuracy and comprehensibility, improving the text, making the text more readable, and correcting errors." PE differs from a regular translation because it comprises three texts that need to be processed and not only two, namely the source text, the MT output, and the final target text. The two main varieties of PE, which we have already mentioned, are light and full PE. The standard also coins definitions for full and light PE: Full PE is defined as the “process of post-editing to obtain a product comparable to a product obtained by human translation.” In comparison, light PE is defined as the “process of post-editing to obtain a merely comprehensible text without any attempt to produce a product comparable to a product obtained by human translation.” The decision which type of PE is needed depends on the purpose and requirements of the final text, the PE brief and the client.

Three actors are assumed in the PE process according to the standard: the client, the translation service provider (TSP), and the post-editor. The focus of the standard is on the role of the translation service provider. However, the other roles are defined indirectly as well. According to the standard, the translation service provider has to determine whether the source text content is suitable for MT and accordingly for PE.\footnote{We will also discuss this topic in \sectref{sec:8}.} The efficiency depends on the kind of MT system, the languages, the domain and the text type. The TSP has to decide whether pre-editing is reasonable before the text is machine-translated. Further, relevant specifications have to be communicated to the post-editor such as who the target text readers are or what quality level is aimed for in the target text. The TSP has to assure that the source content is in an appropriate format so that the post-editor can access it as well as any reference material or other resources. The TSP shall inform the post-editor about how useful the MT output is expected to be. Similar to the TAUS guidelines, the standard defines the following aims for the post-editing process: 
\begin{itemize}
    \item the post-edited MT output must be comprehensible
    \item the content in the source text must correspond to the content in the target language
    \item the post-editor must comply with the agreed requirements and specifications
\end{itemize}

The final target text should meet the following requirements (remember that the standard focuses on full PE!): The terminology must be consistent and comply with domain-specific requirements. Syntax, spelling, punctuation and other orthographic characteristics as well as formatting must be correct. If applicable, specification according to relevant standards must be satisfied. The post-editor must consider the target audience and the target purpose of the final text. Typically, the client and the translation service provider have agreed upon all requirements in advance.

The post-editor’s tasks should be to first evaluate whether the MT output needs any editing at all referring to the source text and then to provide a target text either by using the existing machine-translated elements or by creating a new translation. Finally, the translation service provider should check the final text and deliver it to the client. The post-editor should be able to give feedback about the performance of the MT system, so that weaknesses are known and the system can be improved.

The ISO standard describes six competences, which are mandatory for post-editors: The post-editor needs to have translation competences as well as linguistic and textual competences in the source and target languages. The post-editor must also be able to conduct efficient research to find and process information. Cultural competences are necessary to ensure that the target text audience understands the final text. Further, the post-editor needs technical competences to be able to process the text using the appropriate tools. Finally, the post-editor must have knowledge of the domain that the text deals with.\footnote{We will present our competence model in \sectref{sec:9:1}. As you will see, there are many similarities and also some differences and enhancements.} The qualifications of a post-editor must be similar to those of a translator and should either include a degree in translation, another university degree and two years of full-time professional experience in translation or post-editing, or five years of full-time professional experience in translation or post-editing.

Last but not least, the standard proposes instructions for full PE tasks: The final text should be accurate, comprehensible, and stylistically acceptable. Grammar, syntax and punctuation should be correct. The aim is to create a final text that cannot be distinguished from a human translation. Nonetheless, the post-editor should use as much of the MT output as possible. The post-editor should concentrate on the following aspects while post-editing: He or she has to make sure that no information has been added or omitted. Any inappropriate content must be edited. Sentences should be restructured if the syntax is incorrect or if the meaning is not clear. As mentioned before, grammar, syntax and punctuation should be correct. The same applies for spelling, punctuation and hyphenation.

In conclusion, we would like to emphasise that the aim of this chapter was to present an overview of the ISO standard for PE. For further details or if you are considering becoming certified or working according to the standard, we recommend buying and studying the standard.


\newpage

\section*{Crossword puzzle -- chapter 4}

\begin{Puzzle}{7}{14}
|[5]M	|{}	|[2]S	|{}	|{}	|{}	|{}	|.
|O	|{}	|Y	|{}	|{}	|{}	|{}	|.
|N	|{}	|N	|{}	|{}	|{}	|{}	|.
|[3]O	|U	|T	|P	|U	|[4]T	|{}	|.
|L	|{}	|A	|{}	|{}	|E	|{}	|.
|I	|{}	|X	|{}	|{}	|R	|{}	|.
|N	|{}	|{}	|{}	|{}	|M	|{}	|.
|G	|{}	|{}	|{}	|{}	|I	|{}	|.
|U	|{}	|{}	|{}	|{}	|N	|{}	|.
|A	|{}	|{}	|{}	|{}	|O	|{}	|.
|L	|{}	|{}	|[6]F	|U	|L	|L	|.
|{}	|{}	|{}	|{}	|{}	|O	|{}	|.
|{}	|[1]E	|N	|O	|U	|G	|H	|.
|{}	|{}	|{}	|{}	|{}	|Y	|{}	|.
\end{Puzzle}

\begin{PuzzleClues}{\textbf{Across}}
\Clue{1}{ENOUGH}{The quality of light post-edited texts should be ``good ..."}
\Clue{3}{OUTPUT}{In both full and light PE, we should use as much of the raw MT ... as possible.}
\Clue{6}{FULL}{What kind of PE does the PE standard ISO 18587 focus on?}
\end{PuzzleClues}

\begin{PuzzleClues}{\textbf{Down}}
\Clue{2}{SYNTAX}{In light PE, grammar and ... can be incorrect as long as the meaning is comprehensible.}
\Clue{4}{TERMINOLOGY}{In full PE, key ... needs to be translated correctly.}
\Clue{5}{MONOLINGUAL}{What kind of PE only uses the MT output for PE?}
\end{PuzzleClues}
