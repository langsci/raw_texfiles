\chapter{Connectives on the text level}\label{clink-rest}


This chapter deals with connectives on the text level. These are invariable particles that introduce grammatically independent sentences, but refer back to the content of the previous sentence (or sentences).  These connectives are always sentence-initial. Some of them are also deictic temporal adverbs. They indicate relationships of temporal sequence, cotemporality, causality, or negativity. Many of these particles look as if they have been calqued upon Nepali connectives such as  \emph{kinabhane} \rede{because} or \emph{tyaspachi} \rede{afterwards, and then}. Another possibility is that both languages follow a more general areal pattern. Other  clause linkage markers occur clause-finally in Yakkha, as described in Chapters  \ref{ch-nmlz}, \ref{adv-cl} and \ref{compl}.

\section{Sequential connectives}

Sequential connectives have developed from demonstratives (\emph{na, kha, ŋkha}) to which  the sequential clause linkage marker \emph{=hoŋ} 
has been added, exactly like the Nepali connective \emph{tyas-pachi}. The following forms have been found, and are translatable with \rede{after this}, \rede{after that}, or more generally with \rede{and} or \rede{and then}: \emph{nhaŋ} (\emph{na-hoŋ}), \emph{khoŋ} (\emph{kha-hoŋ}) and \emph{ŋkhoŋ} (\emph{ŋkha-hoŋ}). It is likely that the form \emph{nnhaŋ} exists, too, but as these forms are utterend sentence-initially and usually in very fast speech, such a distinction could not be established reliably. Examples are shown in  \Next and \NNext. The connective \emph{nhaŋ} is by far the most frequent one, featuring 230 occurrences in the corpus (of 3012 clauses), while \emph{ŋkhoŋ} is found only 17 times and \emph{khoŋ} is found no more than 38 times. Occasionally, \emph{nhaŋ} occurs as \emph{nhaŋŋa}, marked by the instrumental marker in his function of indicating the time of an event.

\ex. \ag.uŋ=ŋa pheri, lu, maiti=ci=be, ka haku bagdata=ca nakt-wa-ŋ-ci-ŋ=ha bhoŋ=cen eko ceʔya cekt-wa. khoŋ bagdata  nak-se khe-meʔ=na.\\
{\sc 3sg=erg} again {\sc init} natal\_home{\sc =nsg=loc} {\sc 1sg} now marriage\_finalization{\sc =add} ask{\sc -npst-1sg.A-3nsg.P-1sg.A=nmlz.nsg} {\sc compl=top} one matter speak{\sc -npst[3A;3.P]} and\_then marriage\_finalization ask{\sc -sup} go{\sc [3]-npst=nmlz.sg}\\
\rede{She says: Well, now I will ask my parents for my \emph{bagdata}, too. And she goes to ask for the bagdata.}\footnote{The \emph{bagdata} ritual belongs to the  marriage custom, see \sectref{social}.} \source{26\_tra\_02.011--2}
\bg.ah.    ŋkhoŋ    i    n-jog-u=ha?\\
yes and\_then what {\sc 3pl-}do{\sc -3.P[pst]=nmlz.nc}\\
\rede{Yes. And what did they do then?} \source{20\_pea\_02.011}
\bg. luŋkhwak hoŋ-ma                pʌrne, nhaŋ     khibak=ca        ip-ni-ma                       pʌrne    sa=bu.\\
stone hole\_out{\sc inf[deont]} having\_to and rope{\sc =add} twist{\sc -compl-inf[deont]} having\_to {\sc cop.pst[3]=hsy}\\
\rede{He had to hole out a grinding stone, and he had to complete making a rope, too, it is said.} \source{11\_nrr\_01.007--8}


The  temporal ablative \emph{nhaŋto} derived from the most frequent sequential connective \emph{nhaŋ}. Its literal meaning is \rede{and then up}. This connective is also found with complements (see \sectref{postpos}), referring to a point in time when an ongoing event has started. The event or activity does not necessarily have to go on at the time of speaking. The connective only signifies the initial boundary, as in English \rede{from then on}. An example is provided in \Next.
 
\exg. nhaŋto, garo    n-cheŋd-et-wa=na,                             tokhaʔla\\
from\_then\_on terace {\sc 3pl.A-}mason{\sc -V2.carry.off-npst=nmlz.sg} upwards\\
\rede{From then on, they mason the terrace, upwards.} \source{31\_mat\_01.093}

\section{Cotemporal connectives} 

The cotemporal connectives \emph{khaʔniŋ} and \emph{ŋkhaʔniŋ \ti nnakhaʔniŋ} are constructed of deictic adverbs based on demonstratives and the cotemporal adverbial clause linkage marker, \emph{=niŋ} (see \Next).

\exg.ŋkhaʔniŋ eko paŋ=ca         m-ma-ya-n=niŋa tumhaŋ=ŋa   paŋ  cog-uks-u.\\
that\_time one house{\sc =add} {\sc neg-cop-pst-neg=ctmp} Tumhang{\sc =erg} house make{\sc -prf-3.P}\\
\rede{At that time, when there was not even a single house, Tumhang built a house.} \source{27\_nrr\_06.038}


\section{Adversative connectives}

Cotemporal connectives have developed into an adversative connective, indicating that the propositional content of the clause stands in contradiction or in contrast to some previous content, or that it restricts the previous information in some way. Their structure is transparent; they are cotemporal connectives marked by the standard topic marker \emph{=ko}. In \Next[a] the speaker ponders about arranged marriages. The sentence, about  a hypothetical groom, stands in contrast to an earlier (hypothetical) statement that he might leave a good impression or talk nicely. Example  \Next[b] is self-explanatory.

\ex.\ag.khaʔniŋgo imin=na,       imin=na,       i=na           tha?\\
but how{\sc =nmlz.sg} how{\sc =nmlz.sg} what{\sc =nmlz.sg} information\\
\rede{But how, how will he (a hypothetical husband) be, how to know?}  \source{36\_cvs\_06.336}
\bg.kha liŋkha=ci=ŋa camyoŋba i=ya=go m-bi-ci, khaʔniŋ=go phophop=na sumphak, sumphak phophop n-jog-uks-u=hoŋ=se camyoŋba m-by-uks-u-ci\\
these Linkha{\sc =nsg=erg} food some{\sc =nmlz.nc=top} {\sc 3pl.A-}give{\sc -3ngs.P} but upside\_down{\sc =nmlz.sg} leaf leaf upside\_down  {\sc 3pl.A-}do{\sc -prf-3.P=seq=restr} food {\sc 3pl.A-}give{\sc -prf-3ngs.P}\\
\rede{These Linkhas  gave them some food, but they gave it to them only after turning the leaf plates upside-down.} \source{22\_nrr\_05.047--8}

\section{Causal connectives}

The connective \emph{ijaŋbaŋniŋ} \rede{because} is used for causal clause linkage. It is constructed in parallel to the Nepali \emph{kina bhʌne}, out of the interrogative word \emph{ijaŋ} \rede{why} and the textual topic marker \emph{baŋniŋ} (see Chapter \ref{particles}). An example is provided in \Next.

\exg.ŋkhaʔniŋgo ka  con-si-saŋ,   ijaŋbaŋniŋ   nna  len=bhaŋto=maŋ      ka  heʔniŋ=ca        chocholaplap n-jog-a-ŋa-n.\\
	but {\sc 1sg}   be\_happy{\sc -mddl-sim} because that day{\sc =temp.abl=emph} {\sc 1sg} when{\sc =add} naughty {\sc neg-}do{\sc -pst-1sg-neg} \\
	\rede{But I was happy, because from that day on I never did mischievous things again.} \source{42\_leg\_10.053}

\section{The connective of negative effect}

The connective \emph{manhoŋ} (also \emph{manoŋ}, \emph{manuŋ}) stands at the beginning of clauses that contain information about what happens when a previously mentioned condition has not been fulfilled, like English \rede{if not} or \rede{otherwise} (see \Next). It can be decomposed into the  negated existential copula \emph{man} and the sequential clause linkage marker \emph{=hoŋ}.

\ex. \ag.nhaŋ,    henca-khuba=cen               yakkha          om, man=hoŋ, me-henca-khuba               men=na.\\
and\_then win{\sc -nmlz=top} Yakkha\_person {\sc cop} if\_not {\sc neg-}win{\sc -nmlz=top} {\sc neg.cop[3]=nmlz.sg}\\
\rede{Then, the winner is Yakkha; otherwise, the loser, he is not.} \source{39\_nrr\_08.17}
\bg.ah,    manhoŋ,  adhi barkha khiŋ     khoʔ-ni-me.\\
yes if\_not half year this\_much be\_enough{\sc -compl-npst[3sg]}\\
\rede{Otherwise (if the harvest is not enough), we can survive for as much as half a year.} \source{28\_cvs\_04.042}



