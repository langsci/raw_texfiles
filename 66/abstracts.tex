\documentclass[a4paper, 12pt, bibliography=totoc, headsepline]{scrartcl} 
% bibtotocnumbered: bib ins inhaltsverzeichnis, und mitnumeriert
% headsepline: Linie unter der Kopfzeile

\usepackage{fontspec} % unicode, xetex schriften einbinden
%\usepackage{english}
\usepackage{lscape}
\usepackage{xunicode} % tex-kuerzel in unicode-zeichen umwandeln
\usepackage{natbib}%***REDO, just taken out to generate paradigm chapter!!!! + bibpunct, too!******
\usepackage{linguex} % sternefeld-packung für glossen  
\usepackage{vowel}
\usepackage{qtree}
\usepackage{graphicx}
\usepackage{longtable}
\usepackage{colortbl}
\usepackage[colorlinks=true,allcolors=black]{hyperref}
\usepackage{booktabs}
\usepackage{tabularx}
\usepackage{pdfpages}

\urlstyle{same}

%\usepackage{fancyhdr}
% \pagestyle{fancy}
%\lhead{}
% \rhead{}
\usepackage{scrpage2}
\usepackage{multirow}
\usepackage{tikz}
\usepackage{tablefootnote}% command has the same name

\pagestyle{scrheadings} % seitenstil von scrpage2
\bibpunct[: ]{(}{)}{,}{a}{}{;} %in Präambel, für referenz-stil im text
%\bibpunct[:]{(}{)}{,}{a}{}{,} %Doppelpunkt bei Seitenangaben, z.B. Chomsky2025:34

\KOMAoptions{BCOR=12mm}
\KOMAoptions{DIV=10}

\hyphenation{va-rious crea-ting Satz-ver-bin-dung Yak-kha So-weit ver-glei-chen-de ty-po-lo-gi-sche zahl-rei-chen}

\textheight21cm

\glossglue = 8pt plus 2pt minus 2pt


% Seiten zentrieren
%\oddsidemargin1mm % linker Rand der ungeradzahligen Seiten (= rechte Seite) + x
%\evensidemargin1mm % linker Rand der geradzahligen Seiten (= linke Seite) + x

% oberen Rand einstellen
%\voffset-5mm % passender Wert für headsepline


\setromanfont[Mapping=tex-text]{Linux Libertine O}
\setsansfont[Mapping=tex-text,Scale=MatchLowercase]{DejaVu Sans}
\setmonofont[Mapping=tex-text,Scale=MatchLowercase]{DejaVu Sans Mono}
\newfontfamily\Deva[Script=Devanagari]{Kalimati}
\setkomafont{sectioning}{\rmfamily \bfseries}
% \sloppy (nur für latex)
\newcommand{\rede}[1]{‘#1’}
\newcommand{\ti}{\textasciitilde\ }
\newcommand{\pr}{\char"002A} % stern
\newcommand{\leer}{\char"2205} % nullmorphem/leere menge
%\newcommand{\source}[1]{\hfill {[\small #1]}} %source-formatierung
\newcommand{\source}[1]{\hspace{\fill}\mbox{}\linebreak[0]\hspace*{\fill}\mbox{[\small #1]}}
\renewcommand{\firstrefdash}{} %kein '-' in beispielref. in text

\newcommand{\kommentar}[1]{}

% geringerer abstand nach punkten (satzende)
\frenchspacing
	
% glossing:
 	\let\eachwordone=\it % italic data line (default is \rm)
	\let\eachwordtwo=\small % for small print glosses.
	\let\eachwordthree=\small % for small print glosses.
	
%	\def\gltoffset{0pt} % distance between gloss and translation; with most fonts 0 is OK, but some fonts require 2 or 3 pt (e.g Linux Libertine)
%	\newcommand{\source}[1]{\hfill {[\small #1]}} % source placed at the right margin of data line if \source follows '\\' of the gloss line; or of the translation line if \source follows the translation line.
	% NOTE: if lines spill over, use \ex \raggedright \gll....

%   \frontmatter (füer vorspann)
%	\fancypagestyle{fancyfrontmatter}{%
%	\addtolength{\headheight}{18pt}
%	\fancyhf{}  
%	\fancyfoot[RO,LE]{\sffamily \small \thepage}  
%	\renewcommand{\headrulewidth}{0pt}
%	\renewcommand{\footrulewidth}{0pt}
%	}

 \renewcommand{\textfraction}{0}
 \renewcommand{\topfraction}{1}
 \renewcommand{\bottomfraction}{1}


\clubpenalty = 10000 %10000 wird als 'unendlich' interpretiert..
\widowpenalty = 10000
\raggedbottom
\setcounter{tocdepth}{3}
\setcounter{secnumdepth}{3}




\begin{document}


\section*{Zusammenfassung}

Die vorliegende Arbeit ist die erste ausführliche grammatische Beschreibung des Yakkha, einer sino-tibetischen Kiranti-Sprache, die im Osten Nepals in den Distrikten Sankhuwa Sabha und Dhankuta von noch ca. 14.000 Sprechern gesprochen wird. Die Arbeit basiert auf einem Korpus von ca. 13.000 Sätzen sowie zahlreichen Elizitationen, die auf Forschungsreisen zwischen 2009 und 2012 von der Autorin zusammengetragen wurden. Als Nebenprodukt zur Grammatik und zum Audiokorpus entstand  desweiteren ein Yakkha-Nepali-Englisch-Wörterbuch mit 2400 Einträgen (\url{http://dianaschackow.de/?nav=dictionary}).

Den theoretischen Rahmen der Grammatik bilden funktionale und typologische Fragen. Der thematische Schwerpunkt liegt auf Morphosyntax und Semantik, welche in Kiranti-Sprachen ein hochkomplexes und vergleichsweise wenig erforschtes Feld sind. Die Grammatik folgt der traditionellen Reihenfolge von phonologischen, morphologischen, syntaktischen und informationsstrukturellen Beschreibungen. Ebenso bietet sie eine historische und soziolinguistische Einführung und eine Analyse der komplexen Verwandtschaftsterminologie. Besonderes Augenmerk gilt Themen wie verbaler Personenmarkierung, Argumentstruktur, Transitivität, komplexen Prädikaten, grammatischen Relationen, Satzverbindung, Nominalisierung und dem topographie-basierten Orientierungssystem. Soweit dies möglich war, wurden die gefundenen Strukturen in einer historisch-vergleichenden Perspektive erklärt, um die Entstehung ihrer Eigenheiten zu erhellen. 


\section*{Abstract}


This dissertation provides the first comprehensive grammatical  description of Yakkha, a Sino-Tibetan language of the Kiranti branch. Yakkha is spoken by about 14,000 speakers in eastern Nepal, in the
Sankhuwa Sabha and Dhankuta districts. The grammar is based on original fieldwork in the Yakkha community. Its primary source of data is a corpus of 13,000 clauses from narratives and naturally-occurring social interaction which the author recorded and transcribed between 2009 and 2012. Corpus analyses were complemented by targeted elicitation. In addition, a Yakkha-Nepali-English dictionary with 2400 entries has been compiled (\url{http://dianaschackow.de/?nav=dictionary}).


The grammar is written in a functional-typological framework. It focusses on morphosyntactic and semantic issues, as these present highly complex and comparatively under-researched fields in Kiranti languages. The sequence of the chapters follows the well-established order of phonological, morphological, syntactic and discourse-structural descriptions. These are supplemented by a historical and sociolinguistic introduction as well as an analysis of the complex kinship terminology. Topics such as verbal person marking, argument structure, transitivity, complex predication, grammatical relations, clause linkage, nominalization, and the topography-based orientation system have received in-depth treatment. Wherever possible, the structures found  were explained in a historical-comparative perspective, in order to shed more light on how their particular properties have emerged.




\end{document}