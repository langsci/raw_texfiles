%%%%%%%%%%%%%%%%%%%%%%%%%%%%%%%%%%%%%%%%%%%%%%%%%%%%
%%%                                              %%%
%%%     Language Science Press Master File       %%%
%%%         follow the instructions below        %%%
%%%                                              %%%
%%%%%%%%%%%%%%%%%%%%%%%%%%%%%%%%%%%%%%%%%%%%%%%%%%%%
 
% Everything following a % is ignored
% Some lines start with %. Remove the % to include them

\documentclass[output=coverbod             % long|short|inprep       
		,biblatex
		  ]{LSP/langsci}    
  
%%%%%%%%%%%%%%%%%%%%%%%%%%%%%%%%%%%%%%%%%%%%%%%%%%%%
%%%                                              %%%
%%%          additional packages                 %%%
%%%                                              %%%
%%%%%%%%%%%%%%%%%%%%%%%%%%%%%%%%%%%%%%%%%%%%%%%%%%%%

% put all additional commands you need in the 
% following files. If you do not know what this might 
% mean, you can safely ignore this section

\author{Hossep Dolatian}
\title{Adjarian's {\itshape Armenian dialectology} (1911)}
\subtitle{Translation and commentary}

\renewcommand{\lsSeriesNumber}{4}
\renewcommand{\lsSeries}{loc}

\BackBody{Armenian is an Indo-European language that boasts a rich linguistic landscape comprising Classical Armenian (CA), Standard Western Armenian (SWA or WA), Standard Eastern Armenian (SEA or EA), and numerous non-standard dialects, many of which were tragically lost due to the Armenian Genocide. This book is an English translation and commentary of Hratchia Adjarian's seminal work \armenian{Հայ բարբառագիտութիւն} \textit{Armenian dialectology}, originally written in Armenian in 1911. Adjarian describes 31 non-standard Armenian varieties, offering insights into their linguistic structures and historical roots. To enhance accessibility and understanding, this translation unpacks implicit knowledge embedded in Adjarian's text, providing morpheme segmentation, glossing, and translations. This translation is tailored for three distinct audiences: linguists of non-Armenian, traditional Armenian dialectologists, and linguists of Armenian who were trained outside Armenia. This translation aims to bridge linguistic methodologies and facilitate deeper comprehension of Armenian dialectology. The translator supplements Adjarian's prose with commentary, ensuring clarity and accessibility across diverse readerships. This translation provides access to a linguistic landscape of Armenian before the genocide, with the hope of fostering broader scholarly engagement on Armenian dialects.}

\renewcommand{\lsID}{385}
\renewcommand{\lsISBNdigital}{978-3-96110-489-5}
\renewcommand{\lsISBNhardcover}{978-3-98554-118-8}
\BookDOI{10.5281/zenodo.14008766}
\typesetter{Sebastian Nordhoff}
\proofreader{Yasna}

% add all extra packages you need to load to this file

\usepackage{tabularx,multicol}
\usepackage{url}
\urlstyle{same}

\usepackage{listings}
\lstset{basicstyle=\ttfamily,tabsize=2,breaklines=true}

\usepackage{langsci-basic}
\usepackage{langsci-optional}
\usepackage{langsci-lgr}
\usepackage{langsci-osl}
% \usepackage{./langsci/styles/langsci-lgr}
% \usepackage{./langsci/styles/langsci-osl}
% \usepackage{langsci-gb4e}

\usepackage{tikz}
\usetikzlibrary{patterns,calc}
\pgfdeclarepatternformonly{south east lines}{\pgfqpoint{-0pt}{-0pt}}{\pgfqpoint{3pt}{3pt}}{\pgfqpoint{3pt}{3pt}}{
    \pgfsetlinewidth{0.6pt}
    \pgfpathmoveto{\pgfqpoint{0pt}{3pt}}
    \pgfpathlineto{\pgfqpoint{3pt}{0pt}}
    \pgfpathmoveto{\pgfqpoint{.2pt}{-.2pt}}
    \pgfpathlineto{\pgfqpoint{-.2pt}{.2pt}}
    \pgfpathmoveto{\pgfqpoint{3.2pt}{2.8pt}}
    \pgfpathlineto{\pgfqpoint{2.8pt}{3.2pt}}
    \pgfusepath{stroke}}
    
\usepackage{stmaryrd}
\usepackage{wasysym}
\usepackage{multirow}
\usepackage{caption}
\usepackage{subcaption}
\usepackage{mathrsfs}
\usepackage{qtree}

\usepackage{linguex}


%% hyphenation points for line breaks
%% Normally, automatic hyphenation in LaTeX is very good
%% If a word is mis-hyphenated, add it to this file
%%
%% add information to TeX file before \begin{document} with:
%% %% hyphenation points for line breaks
%% Normally, automatic hyphenation in LaTeX is very good
%% If a word is mis-hyphenated, add it to this file
%%
%% add information to TeX file before \begin{document} with:
%% %% hyphenation points for line breaks
%% Normally, automatic hyphenation in LaTeX is very good
%% If a word is mis-hyphenated, add it to this file
%%
%% add information to TeX file before \begin{document} with:
%% \include{localhyphenation}
\hyphenation{
    Beck-man
    Ngu-yen
    back-chan-nel
    back-chan-nels
    mo-not-o-nous
    ste-reo-typ-i-cal
}

\hyphenation{
    Beck-man
    Ngu-yen
    back-chan-nel
    back-chan-nels
    mo-not-o-nous
    ste-reo-typ-i-cal
}

\hyphenation{
    Beck-man
    Ngu-yen
    back-chan-nel
    back-chan-nels
    mo-not-o-nous
    ste-reo-typ-i-cal
}

%pminos do not split footnotes
% \interfootnotelinepenalty=10000 %Footnote in Laporte chapters has to be split SN


%\DeclareIndexNameFormat{default}{%
%\nameparts{#1}%
%\usebibmacro{index:name}%
%{\index[names]}%
%{\namepartfamily}%
%{\namepartgiveni}%
% {}% L1
% {}% L2
%{\namepartprefix}% generates spurious space L3
%{\namepartsuffix}% generates spurious space L4
%}

%  {\DeclareIndexNameFormat{default}{%
%     \usebibmacro{index:name}{\index[names]}{#1}{#3}{#5}{#7}}}

%\DeclareIndexNameFormat{default}{%
%  \usebibmacro{index:name}{\sindex[nom]}{#1}{#3}{#5}{#7}}

%\DeclareIndexNameFormat{default}{%
%  \usebibmacro{index:name}{\sindex[person]}{#1}{#3}{#5}{#7}}
%\DeclareIndexNameFormat{default}{%
%\nameparts{#1} \usebibmacro{index:name}{\sindex[person]]}{\namepartfamily}{‌​\namepartgiven}{\nam‌​epartprefix}{\namepa‌​rtsuffix}}

%\newcommand{\smiley}{:)}

%\renewbibmacro*{index:name}[5]{%
%\usebibmacro{index:entry}{#1}%
%{\iffieldundef{usera}{}{\thefield{usera}\actualoperator}\mkbibindexname{#2}{#3}{#4}{#5}}}

% \newcommand{\noop}[1]{}

%remove for final
%\overfullrule=1mm

\newcommand{\tobi}[2]}}
\renewcommand{\S}[1]{\tobi{#1}{\textsc{*}}}

% this volume references
% puts: [this volume]
% already defined: \citetv
%\newcommand{\citepv}[1]{(\citeauthor{#1} \citeyear*{#1} [this volume])}
\newcommand{\citealtv}[1]{\citeauthor{#1} \citeyear*{#1} [this volume]}

%parentheses around example number
\newcommand{\pref}[1]{(\ref{#1})}

% in-text examples

\newcommand{\lnex}[1]{\textit{#1}} %target lang word
\newcommand{\lnlit}[1]{(lit.: `#1')} %literal reading
\newcommand{\lnlat}[1]{(#1)} % latinization
\newcommand{\lntrans}[1]{`#1'} %translation
\newcommand{\lnexl}[2]%
{\lnex{#1}{} \lnlat{#2}} % ex with latinization
\newcommand{\lnexlat}[3]{\lnex{#1}{} \lnlat{#2}{} \lntrans{#3}} % ex with latinization and tranl.

%ch01
\newcommand{\co}[1]{\mbox{\textbf{#1}}}

%ch09

\newcommand{\cyrbulg}[1]{\begin{otherlanguage*}{bulgarian}#1\end{otherlanguage*}}


%ch10
\newcommand{\nlp}{{\small NLP}}
\newcommand{\mwe}{{\small MWE}}
\newcommand{\rae}{{\small RAE}}
\newcommand{\lvc}{{\small LVC}}
\newcommand{\pos}{{\small P}o{\small S}}
%\newcommand{\todo}[1]{ \textcolor{red}{#1} }

%\renewcommand{\labelenumi}{\theenumi}
%\ainamefmt{{vv}{ll}{, ff}{, jj}} % fullname

\newcommand{\biberror}[1]{{\color{red}#1}}

\newcommand{\osenovaitem}{--~} 
\bibliography{localbibliography}

%%%%%%%%%%%%%%%%%%%%%%%%%%%%%%%%%%%%%%%%%%%%%%%%%%%%
%%%                                              %%%
%%%             Frontmatter                      %%%
%%%                                              %%%
%%%%%%%%%%%%%%%%%%%%%%%%%%%%%%%%%%%%%%%%%%%%%%%%%%%%
\begin{document}              
\maketitle                
\frontmatter 
\tableofcontents      
\addchap{\lsAcknowledgementTitle} 

\textit{Núu'etaa numá'kaaki wáakapusmak wakápusa wakų́'ro'sh}. 

This book would not have been possible without the generosity and foresight of various Mandan speakers who gave their time and energy into recording their language and their knowledge. Some of the key Mandan individuals who have provided to this corpus, which extends back over a century and a half, include: Mrs. James Kipp (a.k.a., Earth Woman), Mrs. Holding Eagle (a.k.a., Scatter Corn), Mr. Ben Benson (a.k.a., Buffalo Bull Head), Mr. Flat Bear (a.k.a., Bear on the Flat), Mr. Walter Face (a.k.a., Wounded Face), Mrs. Little Crow (a.k.a., Otter Woman), Mr. Paul Crows Heart (a.k.a., Crows Heart), Mr. Sitting Rabbit, Mr. Little Crow, Mr. Wolfs Head, Mr. Wolf Ghost, Mr. White Calf, Mr. Foolish Woman, Mr. Little Owl, Mrs. Good Bear, Mrs. Calf Woman, Mrs. Owen Baker, Mrs. Front Woman, Mr. Wolf Chief, Mrs. White Duck, Mrs. Edna Face, Mrs. Nora Baker, Mr. Stephen Bird, Mr. Mark Mato, Mrs. Bessie Medicine Stone, Mrs. Mary Atkins, Mrs. Blanche Benson, Mr. Burr Crows Breast, Mrs. Annie Eagle, Mrs. Mattie Grinnell, Mr. Albert Little Owl, Mr. Ralph Little Owl, Mrs. Otter Sage, Mr. John Stone, Mr. Clyde Baker, Mr. Jacob, Bird, Mr. William Bell, Mrs. Louella Benson Young Bear, Mr. Ernest Medicine Stone, Mr. Carl Whitman, Ms. Ann Solano, Mr. Leon Little Owl, Mr. Corey Spotted Bear, and Mr. Edwin Benson.

I also gratefully acknowledge that I was able to assemble the materials and information found in this book over the years thanks to the generous financial support of the American Philosophical Society's Phillips Fund for Native American Research, Yale University's A. Richard Diebold Jr. Graduate Fellowship and the Frederick W. Hilles Memorial Scholarship Fellowship, Northeastern Illinois University's Dr. Bernard J. Brommel Doctoral Scholarship, and freelance employment by the Language Conservancy through its contract with the Mandan-Hidatsa-Arikara Nation's Department of Education. I also could never have completed this book without the support provided to me by my employer, the University of Oklahoma, where I had been afforded a semester of teaching release after my third year to ensure that this book could be completed.

My long-time advisor at Yale, Prof. Stephen Anderson, saw many early versions of papers and projects that led to what is now this book. His guidance was instrumental to the theoretical conclusions I made in my dissertation, and that dissertation formed the backbone of the present monograph. Prof. Claire Bowern, my other dissertation co-chair and professor was likewise key to my ability to complete my dissertation, especially after I left New Haven and worked on my dissertation \textit{in absentia} for several years. She held me accountable to making progress, and I deeply appreciate the time and energy she spent in getting me across the finish line. Prof. Natalie Weber, a dissertation committee member, looked over many versions of the analyses that became the chapter on Mandan phonetics and phonology in this book. I am grateful for their time in walking me through more effective ways of presenting the data there. Prof. Marianne Mithun, as a member of my dissertation committee, provided valuable insight into how different processes in Mandan worked and how they fit into the typology of North American languages. Her own experience in studying Siouan languages and her breadth of knowledge on language change improved how I presented the morphological data in what has become the verbal morphology chapter of this book.

On a more personal level, my parents Raymond and Kathleen Kasak have contributed to much to the evolution of this book by asking me repeatedly over the years, ``is it done yet?'' Since the spring of 2015, many phone conversations ended with a question as to the status of this work. First, they asked when I was working on my dissertation, which this book uses as a launchpad for a more comprehensive grammar. Later, as I was preparing this manuscript for submission to Language Science Press, they would ask me whether I was done with this chapter or that chapter yet. Any academic surely understands how incredibly welcome and not at all stressful such questions were.

The friends I made at my M.A in Linguistics at Northeastern Illinois University helped get me to Yale and were, therefore, important to getting this book finished. Dr. John Boyle, first as my professor and then as my friend, introduced me to Siouan linguistics, so I could not have even begun to build the foundations of this book without him. I am thankful for him taking me to my first Siouan and Caddoan Languages Conference in Kansas and him introducing me to the late, great Bob Rankin, one of the monumental figures of Siouan linguistics. My former professors Drs. Shahrzad Mahootian and Lewis Gebhart were great moral support in convincing me to apply to grad school and checked in with me often to see how things were going. My former NEIU classmates Dr. Binh Ngo and Galini Gkartzonika have likewise been great friends and supporters throughout my time at Yale and beyond. I also could not have written this book without the editorial assistance of my wonderful friend Hadley Austin and my former student Lizz Evalen, whose proofreading of an earlier version of my dissertation has made the writing of this book much easier.

My final and most important acknowledgment goes to my partner and wife Dr. Colbi Beam, who has been a singular source of inspiration to complete this work. I am forever grateful for her love and support. I have no doubt that this book would have taken far longer to complete without her encouragement and her understanding of what it takes to write academically. Having written a dissertation herself, she understood the nature of writing expansive pieces of work and she was helpful in putting aspects of the writing process in perspective for me during times when I struggled with how to present some data or was unsure how to best resolve some impasse in writing. She has been insightful and inspirational throughout this process. Colbi, you are simply the best.
\addchap{\lsAbbreviationsTitle}
% \addchap{Abbreviations and symbols}
The category labels for abbreviations follow the Leipzig Glossing Rules.\footnote{\url{http://www.eva.mpg.de/lingua/resources/glossing-rules.php}}
\vspace{.5cm}

\begin{tabularx}{.45\textwidth}{lQ}
    ∅ & zero form \\
    \textbackslash.../ & verb stem\\
    . & multi-item gloss (3\textsc{sg.m})\\
    \_ & multi-item lexemes (like\_that)\\
    - & for affixes (\textit{-thé})\\
    = & for clitics (\textit{=en})\\
    | & syncretism (2|3 person)\\
    > & argument structure (1>3 `first person acting on third person')\\
    1 & first person\\
    2 & second person\\
    3 & third person\\
    \textsc{abs} & absolutive\\    
    \textsc{absc} & absconditive (attention getter, `look here')\\
	\textsc{adjz} & adjectivizer\\
    \textsc{all} & allative\\
    \textsc{and} & andative (`away')\\
    \textsc{anim} & animate\\
    \textsc{appr} & apprehensive\\
    \textsc{assoc} & associative case\\
    \textsc{char} & characteristic case\\
    \textsc{dat} & dative case\\
    \textsc{dem} & anaphoric demonstrative\\
\end{tabularx}
\begin{tabularx}{.45\textwidth}{lQ}
    \textsc{dia} & diathetic prefix\\
    \textsc{dim} & diminutive\\
    \textsc{dist} & distal deictic\\
    \textsc{distr} & ditributive\\
    \textsc{du} & dual number\\
    \textsc{dur} & durative\\
    \textsc{emph} & emphatic\\
    \textsc{erg} & ergative case\\
    \textsc{etc} & et cetera (`and all')\\
    \textsc{f} & feminine\\
    \textsc{fut} & future\\
    \textsc{futimp} & future imperative\\
    \textsc{iam} & iamitive (`already')\\
    \textsc{ic} & inclusory case\\
    \textsc{imn} & imminent (`about to')\\
    \textsc{imp} & imperative\\
    \textsc{indf} & indefinite\\
    \textsc{ins} & instrumental case\\
    \textsc{io} & indirect object\\
    \textsc{ipfv} & imperfective\\
    \textsc{ipst} & immediate past\\
    \textsc{irr} & irrealis\\
    \textsc{iter} & iterative\\
    \textsc{loc} & locative case\\
    \textsc{lpl} & large plural\\
    \textsc{m} & masculine\\
    \textsc{med} & medial deictic\\
\end{tabularx}
\newpage
\begin{tabularx}{.45\textwidth}{lQ}
    \textsc{nd} & non-dual\\
    \textsc{neg} & negator\\
    \textsc{nmlz} & nominalizer\\
    \textsc{npl} & non-plural\\
	\textsc{npst} & non-past\\
	\textsc{nsg} & non-singular\\
    \textsc{only} & exclusive marker (`only', `just')\\
    \textsc{pfv} & perfective\\
    \textsc{ph} & placeholder (`thingamajig')\\
    \textsc{pl} & plural\\
    \textsc{pn} & proper name\\
    \textsc{pln} & place name\\
    \textsc{poss} & possessive\\
\end{tabularx}
\begin{tabularx}{.45\textwidth}{lQ}
    \textsc{pot} & potential\\
    \textsc{priv} & privative case\\
	\textsc{prop} & proprietive case\\
    \textsc{prox} & proximal deictic\\
    \textsc{pst} & past\\
    \textsc{purp} & purposive case\\
    \textsc{q} & question\\    
    \textsc{quot} & quotative\\
    \textsc{redup} & reduplication\\
	\textsc{rpst} & recent past\\
    \textsc{sg} & singular\\
    \textsc{simil} & similative case\\
    \textsc{stat} & stative\\
    \textsc{temp} & temporal case\\
    \textsc{vent} & venitive (`towards')\\
\end{tabularx}
 \mainmatter         
 
% 
% %%%%%%%%%%%%%%%%%%%%%%%%%%%%%%%%%%%%%%%%%%%%%%%%%%%%
% %%%                                              %%%
% %%%             Chapters                         %%%
% %%%                                              %%%
% %%%%%%%%%%%%%%%%%%%%%%%%%%%%%%%%%%%%%%%%%%%%%%%%%%%%
% 
 \input{indexed/02_Frontmatter.tex}
 \input{indexed/03_How-to-use.tex}
 \input{indexed/03_Intro_Yakkha.tex}
 \input{indexed/04_Phonology.tex}
 \input{indexed/06_Pronouns_etc.tex}
 \input{indexed/06_NounPhrase.tex}
 \input{indexed/09_AdjectivesAdverbs.tex}
 \input{indexed/09_Geomorphic.tex}
 \input{indexed/07_VerbalMorphology.tex}
 \input{indexed/07_VerbalMorphology_b.tex}
 \input{indexed/08_NounVerb.tex}
 \input{indexed/08_ComplexPredicates.tex}
 \input{indexed/10a_VerbalValency.tex}
 \input{indexed/10a_ValClasses.tex}
 \input{indexed/10b_TransitivityOperations.tex}
 \input{indexed/10b_TransitivityOperations_b.tex}
 \input{indexed/11_SimpleClause.tex}
 \input{indexed/12_Nomlz-Relativization.tex}
 \input{indexed/13_Converbs-AdverbialClauses.tex}
 \input{indexed/15_Complement.tex}
 \input{indexed/14_Coordination.tex}
 \input{indexed/18_Particles.tex}
\backmatter
\part*{Appendix}
\addchap{Appendix A: Texts}
 \input{indexed/20_app_phopciba.tex}
 \input{indexed/20_app_namthalungma.tex}
 \input{indexed/20_app_linkhedhunga.tex}  

\addchap{Appendix B: Yakkha kinship terms}

The two charts on the following pages show the kinship system for the own family and in-laws. 
\medskip

\noindent
The following further terms are used:
\begin{itemize}
 \item  great-grandparents: \textit{cottu} % {\deva चोत्तु}
 \item  great-great-grandparents: \textit{kektu} %{\deva केक्तु
 \item  great-grandchildren: \textit{sapsik}, % {\deva साप्सिक}, 
\textit{khopsik} %{\deva खोप्सिक}
 \item  great-great-grandchildren: \textit{poʔloŋ} %{\deva पोʔलोङ}
 \item  great-great-great-grandchildren: \textit{joʔloŋ} %{\deva जोʔलोङ}
\end{itemize}
 
% Charts made by Diana Schackow, Magman Jimi and Kamala Linkha. Charts typeset by Lennart Bierkandt
 
 

\newpage 

\section*{Own family}
\newcommand{\lat}[1]{{\it #1}\par}  
\newcommand{\dev}[1]{}  


\begin{sideways}
\resizebox{\textheight}{!}{
  \input{./figures/kinship/kinA.tex}
}

\end{sideways}
\newpage

\section*{In-laws}
\begin{sideways}
\resizebox{\textheight}{!}{ 
  \input{./figures/kinship/kinB.tex} 
}
\end{sideways}

\addchap{Appendix C: Index of Yakkha formatives}
  % \addchap{Index of Yakkha formatives} 

%\addcontentsline{toc}{chapter}{Index of Yakkha formatives}\markboth{Index of Yakkha formatives}{Index of Yakkha formatives}
\begin{center}
\begin{longtable}{lll}
\lsptoprule
{\scshape marker}&{\scshape function} & {\scshape section}\\
\midrule\endfirsthead \midrule {\scshape marker}&{\scshape function} & {\scshape section}\\\midrule\endhead
\emph{a-}&possessive prefix, {\scshape 1sg}&\ref{poss-pron}\\
\emph{-a}&past&\ref{tense}\\
\emph{-a}&\isi{imperative}, subjunctive&\ref{mood}\\
\emph{-a}&nativizer on loans&\ref{furtherverbal}\\
\emph{-a \ti -na}&function verb, \rede{leave}&\ref{V2-leave}\\
\emph{-ap}&function verb, \rede{come}&\ref{V2-come}\\
\emph{-apt}&function verb, \rede{bring}&\ref{V2-bring}\\
\emph{anciŋ-}&possessive prefix, {\scshape 1du.excl}&\ref{poss-pron}\\
\emph{aniŋ-}&possessive prefix, {\scshape 1pl.excl}&\ref{poss-pron}\\
\emph{au}&initiative particle&\ref{ptcl-excla}\\
\emph{baŋna}&complementizer&\ref{noun-compl}\\
\emph{baŋha}&complementizer&\ref{noun-compl}\\
\emph{baŋniŋ}&\isi{textual topic}, \isi{quotative}&\ref{ptcl-top}\\
\emph{-bhes}&function verb, \rede{deliver}&\ref{V2-bhes}\\
\emph{-bhoks \ti -bhoŋ}&function verb, \rede{split}&\ref{V2-split}\\
\emph{bhoŋ}&conditional, complementizer, \isi{quotative}&\ref{fin-comp}, \ref{adv-cl-cond}, \ref{adv-cl-fin-purp}\\
\emph{-ca}&function verb, middle, reflexive&\ref{V2-eat}, \ref{refl}\\
\emph{ca}&auxiliary, reciprocal&\ref{refl}\\
\emph{=ca}&additive focus&\ref{ptcl-foc}, \ref{adv-cl-conc}\\
\emph{=chen}&topic&\ref{ptcl-top}\\
\emph{-ci \ti -cin}&dual (verbal)&\ref{verb-infl}\\
\emph{-ci}&3 nonsingular P (verbal)&\ref{verb-infl}\\
\emph{=ci}&nonsingular (nominal)&\ref{number}\\
\emph{-eba}&polite \isi{imperative}&\ref{mood}\\
\emph{=em}&alternation particle&\ref{ptcl-alt}\\
\emph{eN-}&possessive prefix, {\scshape 1pl.incl}&\ref{poss-pron}\\
\emph{-end}&function verb, \rede{insert}&\ref{V2-insert}\\
\emph{enciŋ-}&possessive prefix, {\scshape 1du.incl}&\ref{poss-pron}\\
\emph{=ge \ti =ghe}&\isi{locative}&\ref{case}\\
\emph{-get}&function verb, \rede{bring up}&\ref{V2-bringup}\\
\emph{=gaŋ \ti =ghaŋ}&\isi{ablative}&\ref{case}\\
\emph{-ghet \ti -het}&function verb, \rede{carry off}&\ref{V2-carryoff}\\
\emph{-ghond}&function verb, \rede{roam}&\ref{V2-roam}\\
\emph{=ha \ti =ya}&\isi{nominalizer}, {\scshape nsg/nc}&\ref{nmlz-uni}\\
\emph{-haks \ti -nhaŋ}&function verb, \rede{send}&\ref{V2-send}\\
\emph{haksaŋ}&comparative&\ref{postpos}\\
\emph{haʔniŋ}&comparative&\ref{postpos}\\
\emph{-heks}&function verb, \rede{cut}&\ref{V2-cut}\\
\emph{=hoŋ}&\isi{sequential clause linkage}&\ref{adv-cl-seq}, \ref{adv-cl-conc}\\
\emph{=hoŋca}& \isi{concessive clause} linkage&\ref{adv-cl-conc}\\
\emph{hau}&\isi{exclamative}&\ref{ptcl-excla}\\
\emph{=i}&sentential focus&\ref{ptcl-foc}\\
\emph{i}&question marker&\ref{ptcl-q}\\
\emph{-i \ti -ni}&\isi{completive}&\ref{completive}, \ref{V2-compl}\\
\emph{-i \ti -in}&{\scshape 1pl, 2pl} (verbal)&\ref{verb-infl}\\
\emph{-ka}&2nd \isi{person }(verbal)&\ref{verb-infl}\\
\emph{=ka}&\isi{genitive}&\ref{case}, \ref{maga}\\
\emph{=khaʔla}&directional, manner&\ref{postpos}\\
\emph{-kheʔ}&function verb, \rede{go}&\ref{V2-go}\\
\emph{-khuba}&\isi{nominalizer}&\ref{nmlz-khuba}\\
\emph{-khusa}&reciprocal marker&\ref{refl}\\
\emph{=ko}&topic&\ref{ptcl-top}\\
\emph{=lai}&\isi{exclamative}&\ref{ptcl-excla}\\
\emph{=le}&\isi{contrastive focus}&\ref{ptcl-foc}\\
\emph{-les}&suffix of knowledge or ability&\ref{furtherverbal}\\
\emph{-lo}&\isi{interruptive clause} linkage&\ref{adv-cl-int}\\
\emph{loppi}&probability&\ref{ptcl-evid}\\
\emph{-loʔa}&equative&\ref{postpos}\\
%\emph{lu}&initiative particle&\ref{}\\
\emph{-m}&{\scshape 1pl.A>3, 2pl.A>3 }&\ref{verb-infl}\\
\emph{-ma}&\isi{infinitive}&\ref{nonfiniteforms}, \ref{nonfin-comp}, \ref{maga}, \ref{manga}\\
\emph{-ma}&event numeral, \rede{times}&\ref{sec-num}\\
\emph{-ma}&\isi{nominalizer} &\ref{nmlz-pa}\\
\emph{-ma \ti -mi}&perfect&\ref{tense}\\
\emph{=maŋ}&\isi{emphatic particle}&\ref{ptcl-foc}\\
\emph{-masa \ti -misi}&past perfect&\ref{tense}\\
\emph{maʔniŋ}&privative&\ref{postpos}\\
\emph{meN-}&\isi{negation} &\ref{neg}\\
\emph{meN-...-le}&\isi{negative converb}&\ref{menle}\\
\emph{-met}&causative&\ref{caus}\\
\emph{-meʔ}&nonpast&\ref{tense}\\
\emph{N-}&\isi{negation} (verbal) &\ref{neg}\\
\emph{N-}&{\scshape 3pl} &\ref{verb-infl}\\
\emph{N-}&possessive prefix, {\scshape 2sg}&\ref{poss-pron}\\
\emph{-n}&\isi{negation}&\ref{neg}\\
\emph{=na}&\isi{nominalizer}, {\scshape sg}&\ref{nmlz-uni}\\
\emph{-nen}&1>2 (verbal)&\ref{verb-infl}\\
\emph{-nes}&function verb, \rede{lay}&\ref{V2-lay}\\
\emph{-nhaŋto}&temporal \isi{ablative}&\ref{postpos}\\
\emph{-ni}&\isi{optative}&\ref{mood}\\
\emph{-nin}&plural and \isi{negation} (verbal)&\ref{neg}, \ref{verb-infl}\\
\emph{njiŋ-}&possessive prefix, {\scshape 2du}&\ref{poss-pron}\\
\emph{=niŋ \ti =niŋa}&cotemporal clause linkage&\ref{sim-finite}\\
\emph{=niŋgobi}&\isi{counterfactual clause} linkage&\ref{adv-cl-count}\\
\emph{nniŋ-}&possessive prefix, {\scshape 2pl}&\ref{poss-pron}\\
\emph{=nuŋ}&\isi{comitative} \isi{case} and clause linkage&\ref{case}, \ref{com-cl}\\
\emph{-ŋ \ti -ŋa}&{\scshape 1sg, excl}&\ref{verb-infl}\\
\emph{=ŋa}&\isi{ergative} \isi{case} and clause linkage&\ref{case}, \ref{manga}\\
\emph{-pa}&\isi{nominalizer}&\ref{nmlz-pa}\\
\emph{=pa}&sentential focus&\ref{ptcl-foc}\\
\emph{-paŋ}&numeral classifier &\ref{sec-num}\\
\emph{=pe}&\isi{locative}&\ref{case}\\
\emph{=phaŋ}&\isi{ablative}&\ref{case}\\
\emph{=pi}&irrealis&\ref{ptcl-evid}\\
\emph{-piʔ}&function verb, \rede{give}&\ref{V2-give}, \ref{benefactive}\\
\emph{=pu}&\isi{reportative} marker&\ref{ptcl-evid}\\
\emph{rahecha}&\isi{mirative}&\ref{ptcl-evid}\\
\emph{-raʔ}&function verb, \rede{come}&\ref{V2-comeneut}\\
\emph{-raʔ}&function verb, \rede{bring}&\ref{V2-bringneut}\\
\emph{-ris}&function verb, \rede{place}&\ref{V2-place}\\
\emph{-sa}&\isi{infinitive}&\ref{nonfiniteforms}, \ref{nonfin-comp}\\
\emph{-saŋ}&\isi{simultaneous converb}&\ref{sim}\\
\emph{-se}&\isi{supine converb}&\ref{sup}\\
\emph{=se}&\isi{restrictive focus}&\ref{ptcl-foc}\\
\emph{-siʔ}&\isi{progressive}&\ref{progressive}\\
\emph{-siʔ}&middle&\ref{V2-mddl}, \ref{middle}\\
\emph{-siʔ}&function verb, \rede{avoid}&\ref{V2-avoid}\\
\emph{-soʔ}&function verb, \rede{look}&\ref{V2-look}\\
\emph{-t}&\isi{benefactive}&\ref{benefactive}\\
\emph{u-}&possessive prefix, {\scshape 3sg}&\ref{poss-pron}\\
\emph{-u}&3.P (verbal)&\ref{verb-infl}\\
\emph{=u}&vocative&\ref{ptcl-further}\\
\emph{-uks}& function verb, \rede{come down}&\ref{V2-comedown}\\
\emph{-uks \ti -nuŋ}& perfect&\ref{tense}\\
\emph{-uks \ti -nuŋ}& function verb, \isi{continuative}&\ref{V2-nung}\\
\emph{-uks \ti -uksa}&past perfect&\ref{tense}\\
\emph{-ukt}& function verb, \rede{bring down}&\ref{V2-bringdown}\\
\emph{uŋci-}&possessive prefix, {\scshape 3nsg}&\ref{poss-pron}\\
\emph{-wa}&nonpast&\ref{tense}\\
%\emph{-yukt}&function verb, \rede{keep for}&\ref{}\\ only 1 example, not treated, maybe lexical
\emph{=ʔlo}&\isi{exclamative}&\ref{ptcl-excla}\\
\lspbottomrule
\end{longtable}
\end{center}

\input{indexed/19_Nocite.tex}

 

%%%%%%%%%%%%%%%%%%%%%%%%%%%%%%%%%%%%%%%%%%%%%%%%%%%%
%%%                                              %%%
%%%             Backmatter                       %%%
%%%                                              %%%
%%%%%%%%%%%%%%%%%%%%%%%%%%%%%%%%%%%%%%%%%%%%%%%%%%%%

% There is normally no need to change the backmatter section
\backmatter
 
\phantomsection 
\addcontentsline{toc}{chapter}{Index} 

\addcontentsline{toc}{section}{Name index}
\ohead{Name index} 
\printindex 
  
\phantomsection 
\addcontentsline{toc}{section}{Language index}
\ohead{Language index} 
\printindex[lan] 
  
\phantomsection 
\addcontentsline{toc}{section}{Subject index}
\ohead{Subject index} 
\printindex[sbj]
\end{document} 


% you can create your book by running
% xelatex lsp-skeleton.tex
%
% you can also try a simple 
% make
% on the commandline
