
\chapter{Nominalization and relativization}\label{ch-nmlz}

In Sino-Tibetan languages one commonly finds a pattern of syntactic nominalizations that participate in various constructions, creating noun phrases, nominal modifiers and relative clauses, but also nominalized embedded clauses,  and independent main clauses. This convergence of functions has been referred to as the \rede{Standard Sino-Tibetan Nominalization} (\textsc{sstn}) pattern \citep[271]{Bickel1999Nominalization} and has been widely studied \citep{Matisoff1972Lahu, DeLancey1989Relativization, Genetti1992Semantic,  Genettietal2008_Nominalization, Saxena1992_Finite, Ebert1994The-structure, DeLancey1999Relativization, Bickel1999Nominalization, Watters2002A-grammar, Noonan2008_Nominalization, Doornenbal2008_Nominalization}. \citet{DeLancey2011_Finite} has even proposed nominalization as the major driving force for syntactic change in \isi{Tibeto-Burman}. 
 
The nominalization processes found in Yakkha fit well into the broader Sino-Tibetan pattern, extending far beyond the derivation of \isi{lexical nouns}. Their functions cover (a) the derivation of nouns, (b) the construction of relative clauses, \isi{adjectives} and other adnominal modifiers, and (c) a function beyond  reference: they may occur on finite embedded complement clauses, in auxiliary constructions, and  in independent main clauses. The nominalization of  main clauses serves discourse-structural purposes, as will be shown in \sectref{nmlz-uni-3} below.

Yakkha has three sets of nominalizers, one set for lexical nominalizations (marked by \emph{-pa} and \emph{-ma}, treated in \sectref{nmlz-pa}), and two for mainly syntactic nominalizations, namely the subject nominalizers \emph{-khuba} and \emph{-khuma} (treated in \sectref{nmlz-khuba}) and the universal nominalizers \emph{=na} and \emph{=ha} (see \sectref{nmlz-uni}). \sectref{correlative} briefly  deals with correlative clause constructions.


\section{Lexical nominalization: \emph{-pa} and \emph{-ma}}\label{nmlz-pa}
\largerpage
The first set contains the  suffixes \emph{-pa \ti -ba \ti -wa} and \emph{-ma}, of which \emph{-pa} (and its allomorphs) can be traced back at least to Proto-Bodic; it is found with nominalizing and related functions in Tibetan,  Sherpa,  Tamangic languages and other Kiranti languages \citep{DeLancey2002_Relativization, DeLancey2011_Finite, Genetti1992Semantic}. In Yakkha, this set is used solely for lexical nominalizations, with \emph{-pa} and its allomorphs for generic and male reference, while \emph{-ma} is generally reserved for female reference (related to an old system of gender marking). Some exceptions where both \emph{-pa} and \emph{-ma} have generic reference were found, too. 

These nominalizers are mostly found in occupational titles, in names of mythical beings and gods, in zoological terms, in names for kinds of food and in \isi{kinship} terminology (cf. appendix). Some examples are provided in \tabref{table-pa}. As this table shows, many of the forms are not transparent; their base or part of it does not occur independently. A few \isi{adjectives} and adverbs were found with \emph{-pa}, too, such as \emph{ulippa} \rede{old} and \emph{tamba} \rede{slowly}, but this is not the typical derivational pattern for \isi{adjectives} or adverbs in Yakkha; the suffixes \emph{-pa} and \emph{-ma} are not productive in Yakkha. 

\noindent

\begin{table}[h]
\begin{center}

\begin{tabular}{lll}
\lsptoprule
{\sc lexeme}&{\sc gloss}&{\sc bases}\\
\midrule
\emph{kucuma}&\rede{dog}& –\\
\emph{kiba} &\rede{tiger}&  –\\
 \emph{hibumba}&\rede{dung beetle}&dung-roll-\textsc{nmlz}\\
\emph{cikciŋwa}&\rede{wasp}& –\\
\emph{caleppa}&\rede{bread}&eat-fry-\textsc{nmlz}\\
\emph{miksrumba}&\rede{blind person}&eye-[{\sc stem}]-\textsc{nmlz}\\
\emph{cagaŋba}&\rede{grain dish} (Nep. \emph{ɖheɖo})&eat–[{\sc stem}]-\textsc{nmlz}\\
\emph{maŋgaŋba}&\rede{ritual specialist}&god-[{\sc stem}]-\textsc{nmlz}\\
\emph{camyoŋba}&\rede{food}&eat-[{\sc stem}]-\textsc{nmlz}\\
%\emph{toŋba}&\rede{beer served in barrel}&drink-\\ - must be fossilized! otherwise, ungpa would be expected
\emph{wariŋba}&\rede{tomato}&  –\\
\emph{khibrumba}&\rede{fog, cloud}& –\\
\lspbottomrule
\end{tabular}
\caption{Lexical nominalizations with \emph{-pa} and \emph{-ma}}\label{table-pa}
\end{center}
\end{table}

 

\section{Participant nominalization (S/A arguments): \emph{-khuba}}\label{nmlz-khuba}

\subsection{Formal properties} 
The default form of this \isi{nominalizer} is \emph{-khuba}, but \emph{-khuma} is found occasionally with female reference. This marker derives nominals that may either modify a head noun (see \Next[a]) or function as noun phrases themselves (see \Next[b] and (c)). Morphologically, it is an affix; it always attaches directly to the verbal stem. Syntactically, it has the whole phrase  in its scope.
\newpage 

\ex. \ag.   heko=ha=ci mok-khuba babu\\
			other{\sc =nmlz.nsg=nsg} beat{\sc -nmlz} boy\\
			\rede{the boy who beats the others} 
			\bg. heko=ha=ci mok-khuba \\
			other{\sc =nmlz.nsg=nsg} beat{\sc -nmlz} \\
			\rede{someone who beats others} 
			\bg. mok-khuba\\
			beat{\sc -nmlz}\\
			\rede{beater}
	
These nominals may be long and internally complex, as shown by the examples in \Next. This makes \emph{-khuba} different, for instance, from the English nomina agentis in \emph{-er}.
	
\ex. \ag.	eŋ=ga yakkhaba=ga kha  ceʔya   yok-khuba Helihaŋ\\
		{\sc 1pl.incl.poss=gen} Yakkha\_person{\sc =gen} this language  search{\sc -nmlz}	Helihang	\\ 
		\rede{Helihang, who searches for this language of us Yakkha} \source{18\_nrr\_03.032}
		\bg. samundra=ga    u-yum=be            inca-khuba\\
		ocean{\sc =gen} {\sc 3sg.poss}-side{\sc =loc} play{\sc -nmlz}\\
		\rede{someone (who is) playing on the shores of the ocean} \source{13\_cvs\_02.057}
		
In a relative clause structure, the \isi{constituent order} is usually head-final, but postposed relative clauses are possible as well. Impressionistically, restrictive relative clauses (those that narrow down the reference of a head noun out of a set of possible referents) tend to occur  preposed, while appositional relative clauses (those that add descriptive information about a noun) tend to occur postposed. 
	
	\ex.\ag.eko yapmi kisa si-khuba\\
	one person deer kill-\sc{nmlz}\\
	\rede{a man killing/having killed a deer}
	\bg. babu sem-khuba\\
	boy pluck-\sc{nmlz}\\
	\rede{the boy who is plucking}
	
\newpage	
If the nominalization results in a noun, it may head NPs and host all nominal morphology: \isi{case}, \isi{number} and \isi{possessive prefixes}. The examples in \Next serve to illustrate that the resulting nominals can be modified or quantified just as simple nouns can. 

\ex. \ag.kha cyabruŋ-lak-khuba=ci\\
	these drum-dance{\sc -nmlz=nsg}	\\
	\rede{these drum-dancers} \source{25\_tra\_01.071}
 	\bg.jammai kam cok-khuba=ci\\
	all work do{\sc -nmlz=nsg}\\
	\rede{all the workers} \source{25\_tra\_01.098}	

	
As the \isi{nominalizer} attaches directly to the verbal stem, there is no \textsc{tam} marking on the verb, and the \textsc{tam} interpretation is retrieved from the context. The verb may host the \isi{negation} marker \emph{men-}, which is also found on other nonfinite verbal forms such as converbs and infinitives (see \Next[a]). Another verbal property is illustrated in \Next: the noun in \emph{-khuba} may still be modified by adverbs. However, it is not clear yet whether nominal and verbal properties could occur simultaneously, e.g., in a phrase like \emph{jammai ʈhwaŋ namkhubaci} \rede{all the awfully stinking ones}.
	
\ex.\ag. kaniŋ=nuŋ   men-doŋ-khuba     nwak\\
	{\sc 1pl=com} {\sc neg}-agree-{\sc nmlz} bird		\\ 
	\rede{the bird that does not belong to us/that is different from us} \source{21\_nrr\_04.004}
\bg. makhurna waghui loʔa ʈhwaŋ nam-khuba\\
black chicken\_droppings like {[smelling]}awfully smell-{\sc nmlz}\\
\rede{something black, smelling awfully, like chicken droppings}\source{42\_leg\_10.017}


Complement-taking verbs with embedded infinitives can be in the scope of the \isi{nominalizer} as well, as \Next shows. The \isi{infinitive} \emph{hiŋma} \rede{support} is embedded into \emph{yama} \rede{be able to X}.
	
	\exg.hiŋ-ma   ya-khuba babu=be    kheʔ-ma.\\
	support{\sc -inf} be\_able{\sc -nmlz} boy{\sc =loc} go{\sc -inf[deont]}\\
	\rede{You have to marry a man who can support you.}\source{28\_cvs\_04.112}
	

The constituents of the relativized clause can also be focussed on by means of the \isi{additive focus} marker (see \Next[a]), and  they can be emphasized, as in \Next[b], with \emph{ŋkhiŋ} \rede{that much}.

   \ex. \ag. sa=maʔniŋ=ca leŋ-khuba\\
		meat=without{\sc =add} be\_alright{\sc -nmlz}\\
	\rede{someone who is fine even without (eating) meat}     
	\bg.yapmi=ci=nuŋ=ca  ŋkhiŋ   sala   me-leŋ-khuba=ci.\\
		people{\sc =nsg=com=add} that\_much matter {\sc neg}-exchange-{\sc nmlz=nsg}\\
	\rede{They were of the kind that does not talk that much with people, too.} \source{22\_nrr\_05.046}     

	
A few lexicalized nominalizations with \emph{-khuba} and \emph{-khuma} can be found, too: \emph{khuncakhuba} \rede{thief} (steal-eat-\textsc{nmlz}), \emph{thukkhuba} \rede{tailor} (sew-\textsc{nmlz}), \emph{hiŋ\-khuba} \rede{husband}, \emph{hiŋ\-khuma} \rede{wife} (support-\textsc{nmlz}), \emph{yaben-pekkhuba} \rede{diviner} (sign-div\-ine-\textsc{nmlz}). Note the parallelism of \emph{-khuba} referring to generic/male nouns and \emph{-khuma} referring to female nouns, as in the lexical nominalizations discussed in \sectref{nmlz-pa}. This suggests that \emph{-khuba} and\emph{-khuma} are historically complex.\footnote{The \isi{syllable} \emph{khu} is also found in the reciprocal marker \emph{-khusa}.}

\subsection{Grammatical relations}\label{khuba-gr}

Descriptions of other Kiranti languages call equivalent constructions \rede{nomen agentis} or \rede{active/agentive participle} \citep{Tolsma1999A-grammar, Rutgers1998Yamphu, Ebert1997A-grammar, Ebert1999Nonfinite, Doornenbal2009A-grammar}. One should note, however, that the nominalization may apply to verbs of any semantics, and the resulting nouns do not just refer to typical agents (see \citet[180]{Bickel2004Hidden} for the same point on closely related Belhare). Subjects of stative verbs like \emph{namma} \rede{smell, emit odour}, \emph{haŋma} \rede{have spicy sensation} or \emph{tukma} \rede{be ill} can also be the targets of this nominalization. 
 
The resulting nominal always refers to the S in intransitive verbs, and to the A for all transitive verbs (two-argument and three-argument), but never to any lower argument. This is illustrated by \Next: S and A arguments are possible results of the nominalization, while P arguments are not (see \Next[c]). Nominalizing morphology that indicates a grammatical relation is common in Kiranti languages, and it is also known, e.g., from  Dolakha \ili{Newari} and from Kham (see \citet[409]{Genetti1992Semantic}, \citealt[376]{Ebert1999Nonfinite}). 
	
\ex. \ag.paip pek-khuba babu\\
	pipe break-{\sc nmlz} boy\\
	\rede{the boy who broke/breaks the water pipe} (A)
	\bg. leŋ-khuba tabhaŋ\\
	become{\sc -nmlz} son-in-law\\
	\rede{the prospective son-in-law} (S)
	 \bg.*babu=ŋa pek-khuba paip\\ 
	boy{\sc =erg} break-{\sc nmlz} pipe		\\ 
	Intended: \rede{the pipe that was/will be broken by the boy} (*P)
	
 
Non-canonically marked S and A arguments, e.g., possessive experiencers or \isi{locative} marked possessors undergoing the nominalization have the same status as standard S and A arguments (i.e., in the \isi{ergative} or \isi{nominative} \isi{case} and being indexed on the verb). In  \Next[a], the \isi{experiencer} S argument is coded as possessor of the sensation, literally translatable as \rede{someone whose laziness comes up}. In \Next[b], the semantic relation expressed is possession, but it is coded with an existential construction and a combination of \isi{genitive} and \isi{locative} marking on the A argument.


\ex. \ag. o-pomma kek-khuba yapmi\\
	{\sc 3sg.poss}-lazyness come\_up-{\sc nmlz} person	\\
	\rede{a lazy guy} 
\bg.kai=ga=be  wa-ya, wa-khuba=ŋa   me-wa-khuba    m-bi-n-ci-nin\\
		some{\sc =gen=loc} exist{\sc [3sg]-pst}  exist-{\sc nmlz=erg} {\sc neg}-exist-{\sc nmlz} {\sc 3pl.A}-give{\sc [pst]-c-3nsg.P-neg}	\\
	\rede{Some (people) had (food), and those who had (food) did not give it to those who did not have it.} \source{14\_nrr\_02.012}

	
\subsection{Predicative use of the nominalized forms}
	
\largerpage %long distance
Some examples even point towards a predicative use of the nominalized forms, as opposed to the expected referential use shown in the examples above. As we will see in \sectref{nmlz-uni} below, this function of \emph{-khuba} is similar to what is found for the nominalizers \emph{=na} and \emph{=ha}.

Shown below is a prime example to illustrate the function of  nominalization as a discourse-structural device in Yakkha (see \Next). It  is from a written narrative. The narrator remembers a fight with his brother. Both boys want to let out the chicks from the cage, and in the course of the fight, they accidentally kill them by squeezing them with the cage door.  The nominalization is employed to yield a vivid narrative style, at a point where the event line approaches its climax, i.e., the accidental killing of the chicks. The verb of saying has to be set back against the content of the embedded direct speech, which contains the crucial information for what happens next. The fact that the nominalized form \emph{kakhuba} \rede{the  one who says} occurs before the embedded direct speech supports this explanation. In the typical structure, the finite verb of saying would be the last element in the sentence. But as focal elements tend to come sentence-finally, and as the embedded speech here contains the focal information, the constituent structure is reversed.  On a further note, the embedded speech itself contains a \isi{nominalizer}  \emph{=ha}. This \isi{nominalizer} has quite the opposite function here: it serves to put emphasis on the claim uttered by the boys (cf. \sectref{nmlz-uni-3} below).  The example nicely illustrates how nominalizers are employed to carve out a text by means of backgrounding and foregrounding chunks of information.
	
	
	\exg.ka  ka-khuba ka  hon-wa-ŋ-ci-ŋ=ha!, a-phu ka-khuba ka   hon-wa-ŋ-ci-ŋ=ha!\\
	\sc{1sg}  say\sc{-nmlz} \sc{1sg} open\sc{-npst-c-3nsg.P-1sg.A=nmlz.nsg}, {\sc 1sg.poss-}eB  say\sc{-nmlz} \sc{1sg} open\sc{-npst-c-3nsg.P-1sg.A=nmlz.nsg}\\
	\rede{I said: “I will let them out!”, and my elder brother said: “I will let them out!”} \source{40\_leg\_08.065}
	
	
Example \Next is from a narrative account of what happened at a festival. The sentence describes some young men who feel ashamed of what they had done the night before when they were drunk. As such, the sentence which is nominalized by  \emph{-khuba} is rather like a comment, set apart from the main event line. 

	\exg. nhaŋ  koi-koi sulemwalem leŋ-khe-khuba.\\
	and\_then some-some with\_hanging\_heads become\sc{-V2.go-nmlz}\\
	\rede{And some of them hung their heads.} \source{37\_nrr\_07.074}


 
\section{The nominalizers \emph{=na} and \emph{=ha \ti =ya} }\label{nmlz-uni}

The nominalizers \emph{=na} and \emph{=ha \ti =ya}  have a wide range of functions. They are clitics, attaching to the rightmost element of a phrase, whether this is an inflected verb, a stem (of any word class), a case-marked phrase or a clause. The resulting nominal may  fill the structural position of a nominal head, an adnominal modifier (\isi{adjectives}, participles, relative clauses), a complement clause, or a finite,  independent  main clause. Nominalized main clauses have several and, at first sight, contradictory discourse functions.

\sectref{nmlz-uni-1} is concerned with uses of this nominalization in adnominal modification, while \sectref{nmlz-uni-2} deals with \isi{complementation}. \sectref{nmlz-uni-3} discusses main clause nominalization.


The markers \emph{=na} and \emph{=ha \ti =ya} indicate singular and nonsingular \isi{number}, respectively, shown in \Next. The marker \emph{=ha} also refers to non-countables, substances and more abstract concepts (see \Next[c]). Example \NNext illustrates the same point with the interrogative root \emph{i}.\footnote{The interrogative root \emph{i} is the base for many interrogative words. In isolation, it may be used to ask about states-of-affairs, as in  \emph{i leksa?} \rede{What happened?}.} The nominalizers turn the interrogative root into a pronoun, in order to inquire about a particular referent. This \isi{number} distinction in nominalization is also found in Athpare and in Belhare, both also from the \rede{Greater Yakkha} branch of Eastern Kiranti (\citealt[130]{Ebert1997A-grammar}, \citealt[278]{Bickel1999Nominalization}).


\ex.\ag. pham=na wariŋba\\
red{\sc =nmlz.sg} tomato\\
\rede{red tomato} 
\bg. pham=ha wariŋba=ci\\
red{\sc =nmlz.nsg} tomato{\sc =nsg}\\
\rede{red tomatoes} 
\bg. onek=ha ceʔya\\
joking{\sc =nmlz.nc} matter\\
\rede{jokes} 

\ex.\ag. i=na\\
what{\sc =nmlz.sg}\\
\rede{what} (presupposing one item)
\bg. i=ha\\
what{\sc =nmlz.nsg/nc}\\
\rede{what} (presupposing many items or mass reference)

\largerpage
It should be emphasized for the following discussion that the traditional sense of the term \emph{nominalizer} is too narrow with regard to Yakkha and in Sino-Tibetan nominalization in general (hence Bickel's (1999) term \emph{Standard Sino-Tibetan Nominalization}).  The markers \emph{=na} and \emph{=ha} do not regularly derive nouns, although one occasionally finds lexicalized expressions, such as \emph{chemha} \rede{liquor} (be trans\-parent-\textsc{nmlz}),  \emph{tumna} \rede{senior} (ripen-\textsc{nmlz}), \emph{pakna} \rede{junior} (be raw-\textsc{nmlz}), \emph{haŋha} \rede{hot spice} (taste hot-\textsc{nmlz}), \emph{bhenikna} \rede{morning ritual} (morning-\textsc{nmlz}). Rather, they  turn any material into referential expressions, behaving identically or very much like noun phrases, either with or without a head noun. 


Etymologically, the nominalizers are related to a set of \isi{demonstratives}: \emph{na} \rede{this}, \emph{kha} \rede{these} (see \sectref{dem-pron}). These \isi{demonstratives} have exactly the same distribution with regard to \isi{number} and mass/abstract reference as the nominalizers (compare examples \LLast and \Last with \Next).


\ex.\ag.na sambakhi\\
this potato\\
\rede{this potato}
\bg. kha sambakhi=ci\\
these potatoes\sc{=nsg}\\
\rede{these potatoes}
\bg. kha kham\\
this soil\\
\rede{this soil}
\bg. kha ceʔya\\
this matter\\
\rede{this matter/language}


		 
So far, not many restrictions on the inflectional properties of the nominalized or relativized verb phrases could be detected. The only restriction is that certain clausal moods which are expressed by verbal inflection (\isi{optative}, \isi{imperative}) cannot be fed into the nominalization process. As far as person and \isi{tense}/\isi{aspect} marking is concerned, anything is possible. Example \Next[a] shows a verb inflected for the \isi{progressive}. The nominalization may also apply recursively (see \Next[b]).


 \ex.\ag.lop pok-ma-si-me=ha yaŋli\\
 now rise-{\sc inf-aux.prog-npst[3sg]=nmlz.nsg} sprout\\
	\rede{the sprouts that are shooting now}   
\bg.  	heʔ=na=beʔ=ya=ci?\\
		which{\sc =nmlz.sg=loc=nmlz.nsg=nsg}\\
		\rede{[The people] from which place?} 

\subsection{Relativization}\label{nmlz-uni-1}

\subsubsection{Adnominal modification and relativization}

The nominalizers are frequently found as relativizers. In contrast to the \isi{nominalizer} \emph{-khuba}, \emph{=na} and \emph{=ha} are almost unresticted with respect to grammatical relations. Core participants as well as non-core participants can serve as a \isi{relativization} site for  \emph{=na} and \emph{=ha}. The only thematic relation that has not been found is A, as A arguments get nominalized by \emph{-khuba}.\footnote{There is no direct negative evidence, unfortunately, but in the whole corpus of recorded language data (roughly 13.000 words), not a single instance of A arguments nominalized by \emph{=na} or \emph{=ha} was detected, neither did I hear it in conversations or elicitations. Thus, even if nominalization or \isi{relativization} over A was possible, it would  be a rather marked structure.} In this respect, Yakkha is radically different from its neighbors, where the corresponding markers are unconstrained with respect to grammatical relations \citep{Ebert1997A-grammar, Bickel1999Nominalization}. Relativizations on S arguments do occur, but they are much rarer than relativizations on objects or other kinds of participants, since this grammatical relation is also covered by the marker \emph{-khuba}. The examples in  \Next show relativizations on core arguments.  

\ex. \ag.ci=ha maŋcwa\\
be\_cold{\sc =nmlz.nsg} water\\
\rede{cold water} (S)
\bg.nda nis-u-ga=na chem\\
{\sc 2sg[erg]} know-\sc{3.P[pst]-2=nmlz.sg} song\\
\rede{a song that you know} (P)
	\bg. chemha yukt-u=na mamu\\
liquor  put\_for-\sc{3.P[pst]=nmlz.sg} girl\\
\rede{the girl that was served liquor} (G)
\bg. beula=ŋa khut-u=ha tephen\\ 
groom\sc{=erg} bring\sc{-3.P[pst]=nmlz.nsg} clothes\\
\rede{the clothes brought by the groom} (T) \source{25\_tra\_01.054}

Some  relativizations of objects have lexicalized into \isi{adjectives}. Example \Next[a] is from the canonical transitive class. Example \Next[b] is originally from the class of transimpersonal verbs (cf. Chapter \ref{verb-val}). In this class, the sole argument is expressed as the object of a morphologically transitive verb, and the verb shows default third person A marking, i.e., zero. An A cannot be expressed with transimpersonal predicates.
\largerpage

\ex.\ag.a-na mi cend-u=na sa-ya=na.\\
{\sc 1sg.poss-}eZ  a\_little wake\_up\sc{-3.P=nmlz.sg} \sc{cop.pst-pst[3sg]=nmlz.sg}\\
\rede{My elder sister was rather witty.}\source{40\_leg\_08.057}
\bg. ikhiŋ yeŋd-u=na yapmi lai!\\
how\_much be\_tough\sc{-3.P=nmlz.sg} \isi{person} \sc{excla}\\
\rede{What a tough person!}


The \isi{relativization} of non-core participants such as locations, temporal expressions or comitatives is illustrated by \Next.\footnote{Regarding (c), the question with whom one may eat is fundamental in the highly stratified Hindu society. This example from a narrative thus also illustrates the impact that Hindu rule has had on Kiranti society in the past centuries.} 	
 
 
\ex. \ag.nna  o-hop wa-ya=na siŋ, nna=ca  et-haks-u!\\
		that {\sc 3sg.poss}-nest exist-{\sc pst[3sg]=nmlz.sg} tree, that{\sc =add} strike-{\sc V2.send-3.P[imp]}\\
	\rede{That tree \textbf{where he has his nest}, destroy that too!}  \source{21\_nrr\_04.020}
 	\bg. la   mem-phem-meʔ=na   seʔni=ŋa\\
	moon {\sc neg}-shine{\sc -npst=nmlz.sg} night{\sc =ins} 		\\
	\rede{in a \textbf{moonless} night} \source{14\_nrr\_02.21}
	\bg. {ca-m}=ha  yapmi  men-ja-m=ha yapmi,  kha   imin=ha=ci?\\
	eat-{\sc inf[deont]=nmlz.nsg} people {\sc neg}-eat-{\sc inf[deont]=nmlz.nsg} people these how{\sc =nmlz.nsg=nsg} 		\\
	\rede{What kind (of people) are they? (Are they) people \textbf{with whom we should eat}, or \textbf{with whom we should not eat}, of what kind (are they)?}\\ \source{22\_nrr\_05.040}

	
Not only inflected verbs but also case-marked phrases \Next and simple nouns \NNext[a] can be turned into adnominal modifiers by means of the nominalizers. Example \NNext[b] shows that even converbs can undergo this nominalization, though it has to be mentioned that this possibility was only found for the \isi{negation} \isi{converb} \emph{men-...-le}.
\largerpage

\ex. \ag.jarman=beʔ=na mamu\\ 
	germany{\sc =loc=nmlz.sg} girl		\\ 
	\rede{the girl from Germany}
	\bg. nasa=ci, u-ʈiŋ=nuŋ=ha=ci\\
		fish{\sc =nsg}	{\sc 3sg.poss-}spike{\sc =com=nmlz.nsg=nsg}\\
			\rede{the fish, those with spikes} \source{13\_cvs\_02.046}
			
		
			
 	\ex.\ag.bhenik=na cama\\
	morning{\sc =nmlz.sg} rice\\
	\rede{the (portion of) rice from the morning} 
	 \bg. u-laŋ men-da-le=na picha\\ 
	{\sc 3sg.\textsc{poss}}-leg {\sc neg-}come-{\sc cvb=nmlz.sg} child\\ 
	\rede{the child that cannot walk yet}

	
Many roots in Yakkha may be used either as adverbs or as \isi{adjectives}. Adjectives in adnominal use are again derived by means of the nominalizers. Compare the adverbs and predicative \isi{adjectives} in \Next[a, c, e, g]  with the adnominal \isi{adjectives} in \Next[b, d, f, h].
	
 
%\item compare adverbial and adnominal use of interrogatives (\Next[a] and \Next[b]), deictic roots (\Next[c] and \Next[d]), and adverbs/\isi{adjectives} (\Next[e] and \Next[f]):

\ex.\ag.haku imin coŋ-me-ci-g=ha?\\
	now how do{\sc -npst-du-2=nmlz.nsg}{\sc }{\sc }	\\
	\rede{Now how will you do it?} \source{22\_nrr\_05.109}
 	\bg. na  imin=na,        kaniŋ=nuŋ   men-doŋ-khuba     nwak.\\
	this how{\sc =nmlz.sg} {\sc 1pl=com} {\sc neg}-agree-{\sc nmlz} bird\\
	\rede{What kind of bird is this, not belonging to us.} \source{21\_nrr\_04.004}
	\bg. ka to taŋkhyaŋ=be   pe-nem-me-ŋ=na.\\ 
	{\sc 1sg} up sky{\sc =loc} fly-{\sc V2.lay-npst-1sg=nmlz.sg}		\\ 
	\rede{I will keep flying up into the sky.} \source{21\_nrr\_04.033}
	\bg.to=na paŋ\\
	up{\sc =nmlz.sg} house\\
	\rede{the upper house} 
	 \bg. khem nis-u-ŋ=na.\\ 
	before see-{\sc 3.P[pst]-1sg=nmlz.sg}		\\ 
	\rede{I saw it before.} 
	\bg.khem=na  kamniwak\\
	before{\sc =nmlz.sg}  friend\\
	\rede{the friend from before} 
	\bg.  luŋkhwak sahro cancan sa-ma=na.\\
	stone very high {\sc cop.pst-prf=nmlz.sg}	\\
	\rede{The rock was really high.} \source{38\_nrr\_07.039}
 	\bg.cancan=na luŋkhwak\\
	high{\sc =nmlz.sg} stone\\
	\rede{a high rock}


Adjectives may also be derived from verbal roots with (ingressive-)stative semantics \Next (cf. also Chapter \ref{adj-adv}).

%\footnote{Note the classifying function of \emph{=na} and \emph{=ha} for nouns that are ambiguous w.r.t. countable or uncountable reference.}
% ingressive-stative, e.g., yungma: pst means sit right now, but ipfv means sit as well, pst can mean sat down or sat (the whole day)

	  \ex. \ag.  chem=ha maŋcwa\\
	be\_clear{\sc =nmlz.nsg}  water\\
	\rede{clear water} 	
	\bg.haŋ=ha macchi\\
	be\_spicy{\sc =nmlz.nsg}	pickles\\
	\rede{spicy pickles} 

	
 

\subsubsection{Headless and internally headed relative clauses}\label{internally-headed-rc}
	
Besides  adnominal modification, one often encounters headless relative clauses, i.e., noun phrases that lack a head noun (see \Next). The relative clause takes the structural position that would otherwise be filled by the head noun.

\exg.  nhaŋ   sapthakt-wa-c-u=na                ibilag-ibilag       khus-het-i-ya-ma-c-u=na.\\
 and\_then like{\sc -npst-du-3.P=nmlz.sg} secretly-{\sc redup} steal{\sc -V2.carry.off-compl-pst-prf-du-3.P=nmlz.sg}\\
\rede{And they (dual) secretly stole one (girl) whom they liked.} \source{22\_nrr\_05.064}


The reference of the head noun is retrieved from the context, but there is a syntactic constraint, too.
As A arguments may not undergo this kind of \isi{relativization}, headless relative clauses are always interpreted as referring to the object of a transitive verb (see \Next[a]) or as the sole argument of an intransitive verb (see \Next[b]). As the S and the P arguments are treated identically, this is a \isi{case} of \isi{ergative} \isi{alignment} in syntax.

\ex.\ag.nna  tas-wa=na=be  yog-a-ca-n-u-m, nna=be.\\
	that reach{\sc [3sg.A]-npst[3.P]=nmlz.sg=loc} search{\sc -imp-du.A[imp]-3pl.P-3.P-2pl.A}	that{\sc =loc}	\\
	\rede{Look for it \textbf{where it lands}, in that (place).}\source{22\_nrr\_05.090}
	\bg. nda  cekt-a-ga=na ŋ-kheps-u-ŋa-n=na.\\
{\sc 2sg[nom ]} speak-\sc{pst-2=nmlz.sg} {\sc neg}-hear-\sc{3.P[pst]-1sg.A-neg=nmlz.sg}\\
\rede{I did not hear \textbf{what you said}.}  (presupposing one word was said)


In contrast to the almost unconstrained adnominal modification, headless relative clauses referring to non-core participants were not found. The absence of the head noun would make their interpretation rather difficult. For instance, leaving out \emph{din} \rede{day} from a clause like in \Next is not possible. 

\exg.na   mamu=ŋa   nna  luŋkhwak khet-u=na  *(din) i leks-a-ma=na  baŋniŋ, ...\\
this girl\sc{=erg} that stone  carry\_off\sc{-3.P[pst]=nmlz.sg} *(day) what happen\sc{-pst{[3sg]}-prf=nmlz.sg} about ...\\
\rede{As for what happened on the day when the girl carried off that stone, ...} \source{38\_nrr\_07.042}


While headless relative clauses show  properties of noun phrases, such as \isi{number} and \isi{case} marking and the possibility of being referred to anaphorically by \isi{demonstratives} (see \emph{nna} in \LLast[a]), they do not have noun properties: there is  no evidence for \isi{possessive prefixes} attaching to the headless relative clause. Furthermore, the argument marking inside the headless relative clause remains as in simple clauses. There are, for instance, no genitives on core arguments, as, e.g., in the English clause \emph{His talking annoyed me}. 

A marginally occurring type of relative clause are internally headed relative clauses (called “circumnominal” in \citealt{Lehmann1984Der-Relativsatz}). Internally headed relative clauses are relative clauses whose head noun is not extracted but remains in the same structural position as it would be in a main clause. This type has been reported for other \isi{Tibeto-Burman} languages, too; see, e.g., \citet[3]{Bickel2005On-the-typological} and \cite{Bickel1999Nominalization} for closely related Belhare, \citet[245]{DeLancey1999Relativization} for Tibetan, and \citet[255]{Coupe2007_Mongsen} for Mongsen Ao. In Yakkha, this type is rather marginal. All examples are elicited, and natural data would be necessary for a better understanding of this structure. An example is shown in \Next. The main verb \emph{tumma} \rede{find} can only take nominal, but not clausal complements. Thus one cannot, for instance, add the complementizer \emph{bhoŋ} to the embedded verb, or interpret it as \rede{I found out that a man was killed by a tiger}. The object can only be the noun \emph{yapmi} \rede{person}, so that the surrounding material must be a relative clause.


\exg.kiba=ŋa  eko \textbf{yapmi} sis-u=na tups-u-ŋ=na.\\
tiger\sc{=erg} one person kill\sc{-3.P[pst]=nmlz.sg} find\sc{-3.P[pst]-1sg.A=nmlz.sg}\\ 
\rede{I found a man who was killed by a tiger.} 


The \isi{ergative} \isi{alignment} found for headless relative clauses is also found for internally headed relative clauses. Relativizing over an A argument is ungrammatical and instead, a relative clause marked by \emph{-khuba} was offered in the elicitation (see \Next[a]). Example \Next[b] also resulted from the attempt to elicit a transitive clause relativizing over an A argument. The transitive verb was changed to an imperfective structure which, by means of an intransitively inflected auxiliary, is also  (morphologically) intransitive. This suggests that the ergativity is the result of a  morphosyntactic, not a semantic constraint. An  A, at least in the third person, would carry an \isi{ergative} marker, which would clash with the object properties that the noun has with respect to the main clause. On the other hand, the ergativity is not surprising anyway, as \isi{relativization} by \emph{=na} and \emph{=ha} generally does not allow A arguments as head nouns. The difference to the more common head-final structure lies only in the exclusion  of non-core participants such as locations or comitatives.


\ex. \ag. eko  yapmi kiba si-khuba tups-u-ŋ=na.\\
one person tiger kill\sc{-nmlz} find\sc{-3.P[pst]-1sg.A=nmlz.sg}\\
\rede{I found a man who killed a tiger.} (A: \emph{-khuba})
\bg. eko yapmi syau sem-ma-sy-a=na tups-u-o-ŋ=na.\\
one person apple pluck\sc{-inf-aux.prog-pst[3sg]=nmlz.sg} find\sc{-3.P[pst]-V2.leave-1sg.A=nmlz.sg}\\
\rede{I found (and passed) a man who was plucking apples.} (S)


An ambiguity with finite complement clauses further complicates the analysis of headless relative clauses. All of the potential instances of internally headed \isi{relativization} found in my corpus could also be complements of  verbs of perception or cognition. As nothing is extracted in internally headed \isi{relativization}, the constituent structure of the relative clause is identical to simple clauses, and one cannot distinguish structurally between \rede{I heard the one who was talking} and \rede{I heard that someone talked}. Both clauses refer to identical situations in the real world; one cannot hear that someone talks without actually hearing the person talking. One structural criterion to find out whether the embedded clause is a complement or a relative clause could be agreement. Example \Next[a] is from a narrative, while \Next[b] is made up in analogy, but with different \isi{number} features. As the argument \emph{tori} (\rede{mustard}, a mass noun) triggers \emph{=ha}  on the main verb \emph{oʔma} \rede{be visible} and \emph{eko phuŋ} \rede{one flower} triggers \emph{=na}, one could infer that they are arguments of the main verb, rather than taking the whole clause to be the argument. However, one could as well interpret this behavior as long distance agreement out of a finite complement clause, which would not be surprising in Yakkha \isi{complementation} (see Chapter \ref{compl}). Hence, the question of how to distinguish internally headed relative clauses and complement clauses cannot be answered satisfactorily, at least not with perception verbs that allow both clausal and nominal complements. 

\ex.\ag. saptakosi=ga u-lap=pe tori phet-a=ha=ca ot-a=ha=bu, nna=bhaŋ.\\
		a\_river\_confluence{\sc =gen} {\sc 3sg.poss}-side{\sc =loc} mustard bloom{\sc [3sg]-pst=nmlz.nsg=add}  be\_visible{\sc [3sg]-pst=nmlz.nsg=rep} that{\sc =abl}	\\
	\rede{Even the mustard blooming at the shores of Saptakosi was visible, from that (rock).} \source{38\_nrr\_07.041} 
 \bg.	saptakosi=ga u-lap=pe eko phuŋ phet-a=na=ca ot-a=na=bu, nna=bhaŋ.\\
		a\_river\_confluence{\sc =gen} {\sc 3sg.poss}-side{\sc =loc} one flower bloom{\sc [3sg]-pst=nmlz.sg=add}  be\_visible{\sc [3sg]-pst=nmlz.sg=rep} that{\sc =abl}	\\
	\rede{Even a flower blooming at the shores of Saptakosi was visible, from that (rock).} 

Ambiguities between relative clauses and complement clauses are common (see, e.g., \citet[272]{Bickel1999Nominalization}, \citet[120, 143]{Noonan2007Complementation}. In Yakkha, the complemental structure probably gave rise to internally headed relative clauses. All instances  displaying this ambiguity in the Yakkha corpus involve verbs of perception or cognition (e.g., \rede{see}, \rede{hear}, \rede{remember}, \rede{forget}), which leads to the conclusion that the complemental structure must have been the original structure. It must have been gradually expanded  to other types of main verbs, namely those which rule out a complemental reading, as the elicited example in \LLast above.




\subsection{Complementation}\label{nmlz-uni-2}

Complementation is the topic of Chapter \ref{compl}. For now it suffices to say here that the finite clausal complements of verbs of saying, perception or cognition are always marked by one of the nominalizers, except for quoted direct speech. Optionally, a complementizer \emph{bhoŋ}  can be added to the nominalized complement clause (see \Next[b]). It is worth noting that all complement-taking verbs of this class also take nominal objects. The verb \emph{miʔma} has very unspecific semantics; it translates as \rede{think, remember}, and as a complement-taking verb, it translates as \rede{hope, think, want}, depending on whether the embedded clause is finite or infinitival. 

Finite \isi{complementation} in Yakkha exhibits double agreement. The embedded subject (S or A) simultaneously triggers agreement in the matrix verb and in the embedded verb, shown by both clauses in \Next (also known as \rede{copy-raising}). Rather than seeking for a purely structural explanation, a semantic motivation seems more likely to me. Perceiving or thinking about an event always involves perceiving or thinking about the participants, and the agreement marking on the matrix verb reflects this semantic property.


\ex. \ag. nda cama ca-ya-ga=na mi-nuŋ-nen=na.\\
{\sc 2sg} rice eat-{\sc pst-2=nmlz.sg} think-{\sc prf-1>2=nmlz.sg}\\
\rede{I have thought you ate the (portion of) rice.} 
\bg. yag-a-sy-a-ŋ=na bhoŋ n-nis-a-ma-ŋ-ga-n=na?\\
be\_exhausted{\sc -pst-middle-pst-1sg=nmlz.sg} comp {\sc neg}-see-{\sc pst-prf-1sg.P-2.A-neg=nmlz.sg}\\
\rede{Don't you see that I am exhausted?}

		
The nominalized clause can also be the complement of \isi{postpositions} that are otherwise found to embed nouns (see \Next). The postposition \emph{anusar} is borrowed from \ili{Nepali}.

\ex.\ag. ka-ya=na anusar\\
	say{\sc -pst[3sg]=nmlz.sg} according\_to\\
	\rede{according to what he promised}\source{11\_nrr\_01.008}
\bg. ka       luʔ-meʔ-nen-in=ha      loʔa cog-a-ni.\\
{\sc 1sg[erg]} tell{\sc -npst-1>2-pl=nmlz.nsg} like  do{\sc -imp-pl.imp}\\
\rede{Do as I tell you.}\source{14\_nrr\_02.19}


\subsection{Stand-alone nominalizations}\label{nmlz-uni-3}
\subsubsection{A versatile discourse strategy}

The extension of nominalizations to main clauses is a common feature of Sino-Tibetan languages. Finite nominalizations, or \rede{stand-alone nominalizations} were first noted by \citet{Matisoff1972Lahu} for Lahu (Loloish). Despite the wealth of syntactic studies on nominalization in \isi{Tibeto-Burman} (see, e.g., \citealt{Matisoff1972Lahu, Noonan1997Versatile, Noonan2008_Nominalization, DeLancey1999Relativization, DeLancey2002_Relativization, Genetti1992Semantic, Doornenbal2008_Nominalization, Genettietal2008_Nominalization, Watters2008_Nominalization, DeLancey2011_Finite}), the main clause function was said to be poorly understood until recently \citep[101]{Genettietal2008_Nominalization}. \citet[110]{Ebert1994The-structure} mentions nominalized sentences in several Kiranti languages and associates them with lively speech and with  focus, as they frequently occur  in questions and in negated sentences.

 The most detailed discussions of this phenomenon can be found for Belhare and some other Kiranti languages \citep{Bickel1999Nominalization} and for Kham \citep{Watters2002A-grammar}. \citet{Bickel1999Nominalization} identifies focus marking as the functional core of main clause nominalizations, i.e., highlighting controversial information in discourse and (re-)instantiating information with strong assertive force and authoritative power. For Kham, \citet[369]{Watters2002A-grammar} concludes that main clause nominalization serves to mark thematic discontinuity with the surrounding context, which is employed for narrative stage-setting and for highlighting pivotal events in narratives.  In Chantyal, nominalized main clauses may also have a \isi{mirative} reading \citep{Noonan2008_Nominalization}. \citet[89]{Doornenbal2008_Nominalization} made similar findings for \ili{Bantawa}. Main clause nominalization is not restricted to Sino-Tibetan languages, though. \citet{Yapetal2010_Non-referential} discuss the non-referential uses of nominalization with respect to languages spoken in Asia in general. This phenomenon is not restricted to Asia either, as studies by \citet{Woodbury1985Noun} and by \citet{Wegener2012_Savosavo} show (the list is not meant to be exhaustive). 
 
 The crosslinguistic occurrence of finite nominalizations suggests a deeper func\-tional-prag\-matic motivation for this process. As \citet[246]{Matisoff1972Lahu} has already observed in his study of Lahu, nominalizations “objectify and reify a proposition”. By applying a linguistic strategy, namely turning a proposition into a noun-like entity, inherently ephemeral events are identified with inherently time-stable objects, for the purpose of giving them more “reality”. 
 
As I will show, this effect gave rise to various functions in Yakkha, which seem to be  contradictory at first sight. The functions are very similar to what has been found for Belhare and Kham. Main clauses are nominalized to set them apart from the surrounding discourse, which may result both in backgrounding and in foregrounding information, depending on the genre and the given discourse context.

%In Yakkha, especially in conversational data, nominalized main clauses are very frequent in assertions and questions. They are obligatory in \isi{polar questions}. By using a nominalized indicative main clause, the speaker either questions or re-assures the certainty of the propositional content. Nominalized main clauses are so frequent that they even might develop into regular person markers. \citet{DeLancey2011_Finite}, who observed related developments in other \isi{Tibeto-Burman} languages, has proposed nominalization as the major driving force in syntactic change in \isi{Tibeto-Burman}.
 




Plenty of the examples above have already shown that the clitics \emph{=na} and \emph{=ha} may attach to finite, independent clauses.  They may also attach to nonverbal predicates in \isi{copula} constructions, as \Next shows. The use of the  \isi{copula} \emph{om} is possible, but not obligatory here.
	
	\exg. m-muk a-laŋ hiŋ=na (om).\\
 {\sc 2sg.poss-}hand {\sc 1sg.poss-}foot 	as\_big\_as{\sc =nmlz.sg} ({\sc cop})	\\
\rede{Your hand is as big as my foot.}
	
	
As already noted by \citet{Matisoff1972Lahu}, nominalized clauses are often paraphrasable with  “It is the \isi{case} that [proposition]”. By nominalizing a clause, the speaker emphasizes some state-of-affairs as an independent fact. This is often necessary when some controversial or contrastive information is involved, for instance in negotiations, as in \Next and \NNext from narratives (cf. \citet{Bickel1999Nominalization} for the same point). In \Next, there are two parties, namely a bride against her natal home, arguing about  a megalith that she wants to take to her new home as dowry. 

%**check also (cf. Kölver 1977, van Driem 1993). - found in BB1999**

\ex.\ag.ŋkhatniŋgo na  mamu=ŋa   ka, eko=chen ka mit-u-ŋ=na.\\ 
	 but  this girl{\sc =erg}  {\sc 1sg}, one{\sc =top} {\sc 1sg[erg]} want{\sc -3.P[pst]-1sg.A=nmlz.sg}\\
	\rede{But this girl (said): I, I want one thing.}  
\bg. saman   py-haks-a=na,  n-lu-ks-u-ci.\\
property give-V2.send{\sc -pst[3sg]=nmlz.sg} {\sc 3pl.A}-tell{\sc -prf-3.P.pst-nsg}\\
\rede{The property was already transferred, they told them.}  \source{38\_nrr\_07.004}\\


In \Next, the speakers, inhabitants of a village, assure the protagonists of the narrative that they can have anything they ask for. They say this out of gratitude and possibly fear, because the protagonists are believed to be sourcerers, and they had just miraculously \rede{found} a girl (whom they had in fact kidnapped themselves some days earlier). 


\exg.  i=ya  njiŋda yoŋ-me-c-u-ga, ŋkha kaniŋ  pi-meʔ-nen-in=ha.\\
what{\sc =nmlz.nsg} {\sc 2.dual} search{\sc -npst-du.A-3.P-2} that {\sc 1pl[erg]}	give-{\sc npst-1>2-pl=nmlz.nc}\\
	\rede{Whatever you look for, we will give it to you.} \source{22\_nrr\_05.079}\\
	

Example \Next is another instance of a nominalized clause, which is uttered in order to convince the hearer about the truth of the propositional content (cf. \citet{Ebert1997A-grammar} for the same point on Athpare). 


\exg. nda n-si-me-ka-n=na,  ka ucun=na  ʈhoŋ=be khem-meʔ-nen=na.\\
		{\sc 2sg} {\sc neg}-die-{\sc npst-2-neg=nmlz.sg} {\sc 1sg}  nice{\sc =nmlz.sg} place{\sc =loc} carry\_off-{\sc npst-1>2=nmlz.sg}\\
	\rede{You will not die, I will take you to a nice place.} \source{27\_nrr\_06.010}
\newpage


The nominalized clauses are very frequent in assertions (both affirmative and negated, in fact, negated clauses without nominalizers are very rare), and also in  questions, particularly in \isi{polar questions} (see \Next[a]). Nominalization is absent from  \isi{mood}  paradigms (\isi{imperative}, subjunctive and \isi{optative}), from irrealis and counterfactual clauses, and from adverbial subordination (exemplified by a sequential clause in \Next[b]), i.e., from any non-assertive clause type, except for interrogatives. Given this distribution, the main clause nominalization could best be characterized as assertive force marker. The proposition in question is not part of the background that is shared by all discourse participants, which is why it needs to be reinstantiated, a process which is in line with the focus analysis found in \citet{Bickel1999Nominalization}. 


\ex. \ag. ta-ya=na=i?\\
		come{\sc [3sg]-pst=nmlz.sg=q}	\\
	\rede{Did she come?} 
 	\bg.khus-het-a-ma-c-u-(*=na)=hoŋ, ...\\
	steal-{\sc V2.carry.off-pst-prf-du.A-3.P(*=nmlz.sg)=seq} ...\\
	\rede{After they have stolen her, ....} \source{22\_nrr\_05.060}


The use of main clause nominalization differs greatly across text genres. In narratives, nominalized main clauses are rarer than in conversational data. For Kham, a \isi{Tibeto-Burman} language spoken in western \isi{Nepal}, \citet{Watters2002A-grammar} observed three uses of nominalization, which are (i) stage-setting, (ii) marking pivotal events or turning points in a story, and (iii) marking comments or some information that is set apart from the main event line. 

\largerpage
The situation in Yakkha is  similar to Watters's observations on Kham. Example \Next[a] shows a nominalized clause used in stage-setting, as it is often found in the beginning of narratives. Example  \Next[b] is the first nominalized sentence after a long stretch of non-nom\-i\-nal\-ized sentences in a narrative containing the autobiographical story of a girl who was attacked by an owl. It is self-explaining that the sentence in \Next[b] is pivotal for the story. The example also illustrates the use of nominalization in \isi{mirative} contexts. The sentences in \Next[c] and \Next[d] illustrate the third function of main clause nominalization: both represent comments that are set apart from the main event line. They are from the same owl story, uttered as comments, after the whole event has been told. The  pragmatic function of nominalization –  establishing facts by reifying propositions – also has an effect on the interpretation of non-nominalized clauses.  As they are in complementary distribution to the nominalized clauses, the hearer knows that more information is to come after a non-nominalized clause and that the speaker has not finished unfolding the main event line yet.

%nice ex. c for center-embedding (the full sentence) 


\ex.\ag.  eko Selele-Phelele baŋna nwak wa-ya=na=bu.\\
one Selele-Phelele so-called bird exist{\sc -pst[3sg]=nmlz.sg=rep}\\
\rede{(Once) there was a bird called Selele-Phelele.} \source{21\_nrr\_04.01}
\bg. siŋ-choŋ=be        so-ŋ=niŋa=go              phopciba=le          weʔ=na!\\
tree-top{\sc =loc} look{\sc -1sg.A[pst]=ctmp=top} owl{\sc =mir} exist{\sc [3sg.npst]=nmlz.sg}\\
\rede{When I looked up into the tree, there is an owl!} \source{42\_leg\_10.018}
\bg. jeppa nna  len ka ... ollobak paro=be tas-u-ŋ=na.\\
really that day {\sc 1sg} ... almost heaven{\sc =loc} arrive{\sc -3.P[pst]-1sg.A=nmlz.sg}\\
\rede{Really, that day, ... I had almost gone to heaven. } \source{42\_leg\_10.051}
\bg. a-ʈukhuruk=pe  og-a-ŋ=na loʔwa en-si-me-ŋ=na.\\
{\sc 1sg.poss-}head{\sc =loc} peck{\sc -pst-1sg.P=nmlz.sg} like perceive{\sc -middle-npst-1sg=nmlz.sg}\\
\rede{It feels as if it was still pecking me on my head.} \source{42\_leg\_10.056}


The already mentioned \isi{mirative} use is exemplified by example \Next. Yakkha has a \isi{contrastive focus} particle \emph{=le} which is found in \isi{mirative} contexts (for  examples see \Last[b] and \NNext[b], cf. Chapter \ref{particles}). Yakkha has also borrowed the \ili{Nepali} \isi{mirative} marker \emph{rahecha \ti raicha}, and the  majority of sentences marked by \emph{rahecha} occur in the nominalized form \Next.


\exg.khus-het-u=ha   raicha!\\
steal\sc{-V2.carry.off-3.P[pst]=nmlz.nsg}  \sc{mir}\\
\rede{He stole it!} \source{20\_pea\_02.016}

\largerpage
In conversations, nominalized main clauses tend to be  the norm, and non-nom\-i\-nal\-ized clauses are the exception, since conversations can be perceived as constant negotiations about the status of some propositional content. Speakers express their opinions, try to convince the hearers about something, or they ask about facts. In \Next[a], one can see a deontic clause, containing an assessment of the speaker about the necessity of the propositional content. Example \Next[b], as mentioned above, shows another instance of a \isi{mirative} clause. The speaker, having lost her way in the dark, talks to herself, first asking herself and then correcting her own assumptions on which way leads to her home. 
%As suggested by \citet[293]{Bickel1999Nominalization}, main clause nominalization started out as propositional operator focus. This also makes sense for Yakkha, as questions, for instance, occur nominalized in almost all instances. 

 
\ex.\ag.u-milak          meŋ-khok-ma=na=bu,                       kucuma=ga,   kahile=ca.\\
{\sc 3sg.poss-}tail  {\sc neg-}cut{\sc -inf[deont]=nmlz.sg=rep}  dog{\sc =gen} when{\sc =add}\\
\rede{Their tail should not be cut off, the dog's, never.}\source{28\_cvs\_04.225}
\bg. are,   heʔne khy-a-ŋ=na lai,        ka? lambu=go     naʔmo=le             sa=na.\\
ohǃ?  where  go{\sc -pst-1sg=nmlz.sg} {\sc excla} {\sc 1sg} way{\sc =top} down\_here{\sc =ctr} {\sc cop.pst[3sg]=nmlz.sg}\\
\rede{Holy crackers, where did I go? The way was down here!}\source{28\_cvs\_04.027}



Nominalized main clauses could also be perceived as instances of insubordination – the  recruitment of formally subordinate clauses to provide material for new main-clause types \citep{Evans2007_Insubordination}. The semantic range of insubordination is typically associated with interpersonal control, \isi{modals} (such as \isi{hortative} and deontic meanings), and presupposed material, \isi{contrastive focus}, or reiteration. Nominalized main clauses in Yakkha fit the characterization of “formally subordinate”, as they are are formally identical to embedded clauses (complement clauses and relative clauses). 
Historically,   \emph{=na} and \emph{=ha} can be related to a set of \isi{demonstratives}, which renders both their development to relativizers, complementizers, and also their use in  non-embedded  clauses plausible (assuming embedding to a zero \isi{copula}). This reasoning is supported by the fact that the copular structure can also be found synchronically, especially in negated forms as in \Next.\footnote{Generally,  Kiranti languages do not need a \isi{copula} for the expression of \isi{equation} or identification (\citet[276]{Bickel1999Nominalization}, \citealt[105]{Ebert1994The-structure}).}  

 \exg. kanciŋ  mokt-a-ŋ-c-u-ŋ=na  men=na.\\
{\sc 1du[erg]} beat{\sc -pst-excl-du.A-3.P-excl=nmlz.sg} {\sc neg.cop=nmlz.sg}\\
 \rede{It is not the \isi{case} that the two of us have beaten him.} \source{36\_cvs\_06.154}
 

However, the high frequency of nominalized main clauses makes it unlikely that there is synchronically an underlying embedded structure in each instance with a zero verb meaning \rede{be the \isi{case} that}.\footnote{\citet{Matisoff1972Lahu} made the same conclusion for Lahu.} Rather, the main clause function is the result of reanalyzing a subordinate structure. The same path is described in \citet{DeLancey2011_Finite}, who proposed nominalization as the major driving force for syntactic change in \isi{Tibeto-Burman}: 
 
\begin{quote} 
[...] in many \isi{Tibeto-Burman} languages the finite construction of the verb reflects an earlier construction in which the sentence or verb phrase is nominalized. The construction  often  includes  a  \isi{copula},  of  which  the  nominalized  sentence  is  then  an argument, but the \isi{copula} may be dropped over time [...]. Frequently  such  constructions  lose  their marked  status  and  become  the  ordinary  finite  construction,  resulting  in  the creation  of  new  verbal  categories  and  systems. \citep[343]{DeLancey2011_Finite}
\end{quote}
 
The “drift from referent identification to event predication and the expression of speaker’s stance” in \isi{Tibeto-Burman} languages apparently even belongs to a broader Asian typological picture, as suggested in \citet{Yapetal2010_Non-referential}. The Yakkha main clause nominalization might be on its way to lose its marked status and to become an integral part of the verbal \isi{person marking}, moving on from the domain of pragmatics to syntax. In one finite verb form, this has already happened: in the third person plural intransitive forms, \emph{=ha} is optionally followed by the nominal nonsingular \isi{clitic} \emph{=ci}, which is not found in any other inflected verbal forms. Other Kiranti languages, e.g., \ili{Limbu} and \ili{Bantawa}, employ nominalized forms in \isi{tense} and \isi{aspect} marking \citep{Driem1993Einige, Doornenbal2008_Nominalization}. 



\subsubsection{The alignment of \emph{=na} and \emph{=ha} in main clauses}

In relative clauses, the choice of  \emph{=na} and \emph{=ha} is naturally determined by the \isi{number} features of the head noun, i.e., \emph{=na} when the head noun has singular \isi{number}, and \emph{=ha} when it has nonsingular or non-countable reference. In the nominalization of main clauses, since the nominalizers have become part of the predicate, there must be a different strategy.  What we find is a combination of role-based and reference-based \isi{alignment} (see \tabref{nmlz-uni-table}). The examples in  \Next illustrate this for intransitive sentences. Here, the sole argument of the clause determines the choice of either  \emph{=na} or \emph{=ha}. In transitive verbs, in scenarios with third person objects (3.P) and with third person acting on second \isi{person} (3→2), one finds \isi{ergative} \isi{alignment}. The choice of the markers is determined by the \isi{number} of the P argument, shown in  \NNext. 


\ex. \ag. khy-a=na.\\
go-\sc{pst[3sg]=nmlz.sg}\\
\rede{He/she went.}
\bg. khy-a-ci=ha.\\
go-\sc{pst-du=nmlz.nsg}\\
\rede{They (dual) went.}


\ex.  \ag. 	na toŋba imin et-u-ga=na?\\
			this beer how like\sc{-3.P-2=nmlz.sg}\\
			\rede{How do you like this \emph{tongba}?}\footnote{\emph{Tongba} is  beer served in a small barrel, to be drunken through a pipe.}
	\bg.	kha=ci imin et-u-g=ha?\\
			these\sc{=nsg} how like\sc{-3.P-2=nmlz.nsg}\\ 
			\rede{How do you like these?}



Reference-based \isi{alignment} can be found in scenarios with speech-act participant objects, except for 3→2. Nonsingular \isi{number} outranks singular, which means that as soon as one participant has nonsingular \isi{number}, the marker \emph{=ha} has to be used, regardless of the \isi{syntactic role}s. The marker \emph{=na} is only found when both A and P have singular \isi{number}. The combination of role-based and reference-based \isi{alignment} can also be found for other person markers (see \sectref{verb-infl}).

\begin{table}[htp]
\resizebox{\textwidth}{!}{
%\begin{tabular}{|l||l|p{1.4cm}|p{1.4cm}|l|p{1.4cm}|}
\begin{tabular}{l|l|l|l|l|l|l|l}
\lsptoprule
		% & {\bf INTRANS.}&	\multicolumn{6}{c|}{ {\bf TRANS.}} \\
		& \multirow{2}{*}{{\bf \textsc{intransitive}}}	&	\multicolumn{6}{c}{ {\bf \textsc{transitive}}} \\
		\cline{3-8}
		&		&{\sc 1sg.P}& {\sc 1nsg.P}&{\sc 2sg.P}&{\sc 2nsg.P}&{\sc 3sg.P}&{\sc 3nsg.P}\\
\hline % The padding of \midrule is not desired here. Therefore \hline.
 {\sc 1sg.A} 		& \emph{=na}& \multicolumn{2}{c|}{\cellcolor[gray]{.8}}&\emph{=na}& & &\\
 % \cline{1-2} 
 \cline{5-5} 		
{\sc 1nsg.A} 		&\emph{=ha} & \multicolumn{2}{c|}{\cellcolor[gray]{.8}}&\multicolumn{2}{r|}{\emph{=ha}}&&\\
 \cline{3-6}
{\sc 2sg.A} 		&\emph{=na}&\emph{=na}	&&\multicolumn{2}{c|}{\cellcolor[gray]{.8}} &\emph{=na}&\emph{=ha}\\
\cline{3-3}
{\sc 2nsg.A}		&\emph{=ha}&	\multicolumn{2}{r|}{\emph{=ha}} &\multicolumn{2}{c|}{\cellcolor[gray]{.8}} &&\\ 
 \cline{3-6}
{\sc 3sg.A}		&\emph{=na}&\emph{=na}&& & &&\\
\cline{3-3}
{\sc 3nsg.A}	& \emph{=ha}&\multicolumn{2}{r|}{\emph{=ha}}&\emph{=na}  &\emph{=ha}&&\\
\lspbottomrule
\end{tabular}
}
\caption{Alignment of the nominalizers \emph{=na} and \emph{=ha}}\label{nmlz-uni-table}
\end{table}     


Participants that have non-countable reference trigger \emph{=ha} (see \Next). When \isi{mass nouns}, e.g., \emph{maŋcwa} \rede{water} or \emph{cabhak} \rede{uncooked rice} trigger the singular marker \emph{=na}, this means that they have a bounded quantity, like water in a cup or one portion of rice. Thus, the nominalizers have also acquired a classificatory function.


\ex. \ag.n-jek-les-wa-ŋa-n=na.\\
		{\sc neg}-speak-know{\sc -npst-1sg-neg=nmlz.sg} 	\\
	\rede{I do not know how to speak it (one particular word).}  
 	\bg. yakkha ceʔya n-jek-les-wa-ŋa-n=ha.\\
	Yakkha matter {\sc neg}-speak-know{\sc -npst-1sg.A-neg=nmlz.nsg} 		\\
	\rede{I do not know how to speak Yakkha.}  
	
	
If the object is a proposition, \emph{=ha} is found as well, except for complements of verbs of perception and cognition, which show agreement with arguments of the embedded proposition.

%comment L: direct embedded clause would be better as example

\exg. cilleŋ=be     haku nhaŋto camyoŋba py-a;  kanciŋ khiŋ=se   naŋ-me-ŋ-c-u-ŋ=ha.\\
 overturn{\sc =loc} now since   food give{\sc -imp[1.P]} {\sc 1du[erg]} this\_much{\sc =restr} ask{\sc -npst-excl-du-3.P-excl=nmlz.nc}\\
 \rede{From now on, give us the food in a leaf that is turned around (to the right side); we only ask for this much.} \source{22\_nrr\_05.123}	

   
In ditransitive verbs of the \isi{double object frame} (see \sectref{three-arg-frame}), the choice of \emph{=na} or \emph{=ha}  is determined by the T argument (i.e., indirective \isi{alignment}), at least if it has third person reference, which is mostly the \isi{case}. As the regular \isi{person marking} is determined by the G argument (i.e., secundative \isi{alignment}), a double object verb ends up indexing all three arguments (A, G and T). The marking may undergo alternations under certain conditions (see \sectref{three-arg}).


 \ex. \ag. ka nda eko cokleʈ piʔ-nen=na.\\
			\sc{1sg[erg]} \sc{2sg} one sweet give\sc{[pst]-1>2=nmlz.sg}\\
			\rede{I gave you a sweet.} 
			\bg. ka nda pyak cokleʈ(=ci) piʔ-nen=ha.\\
			\sc{1sg[erg]} \sc{2sg} many sweet\sc{(=nsg)} give\sc{[pst]-1>2=nmlz.nsg}\\
			\rede{I gave you many sweets.}



\section{Correlative clauses}\label{correlative}

The correlative construction consists of a relative clause and a main clause. The first clause contains  a deictic element that narrows down the possible reference, and the second clause contains the statement about that referent. This parallelism of question and answer was also called  “correlative diptych” by \citet{Lehmann1988Towards}, because both clauses contain a reference to the participant about whom a statement is made. As Yakkha does not have relative pronouns, it utilizes interrogative pronouns in the relative clause. The main clause contains a noun or a demonstrative. 

\citet[25]{Bickel1999Principles} suggests that the correlative construction has developed from a commonly found strategy of structuring discourse, which he calls “informational diptych”, in parallel to Lehmann's terminology. Here, too, the first part is used to announce the kind of information one would supply, often followed by a \isi{topic} marker, before conveying the information in the second part of the diptych. This strategy is common in Yakkha, too, especially in narratives (see Chapter \ref{particles}). 

All kinds of participants and relations can be expressed with \isi{correlative clauses}: quantities, locations, points in time, manner. The relation in  \Next[a] expresses the amount of an object, \Next[b] shows \isi{relativization} over a possessor, while \Next[c] relativizes over a temporal reference.

\ex.\ag. ka  ikhiŋ  nis-uks-u-ŋ, khiŋ  ka-me-ŋ=na.\\
		{\sc 1sg[erg]} how\_much know{\sc -prf-3.P[pst]-1sg} as\_much say-{\sc npst-1sg=nmlz.sg}\\
		\rede{I will say as much as I (got to) know.}
		\bg.  isa=ga     u-cya=ci  mokt-u-ci=ha  uŋ  bappura=ci!\\
		who\sc{=gen} {\sc 3sg.poss-}child\sc{=nsg}  beat\sc{-3.P[pst]-3nsg.P=nmlz.nsg} \sc{3sg} pitiful\_person\sc{=nsg}\\
		\rede{The poor people whose children got beaten!} (Lit.: \rede{Those whose children got beaten are pitiful.})\source{41\_leg\_09.042}
		\bg.  heʔniŋ  hoŋ-khe-me  ŋkhatniŋ=ŋa     ka=ca         sy-a=na                    mit-a.\\
		when crumble\sc{-V2.go-npst[3sg]} then\sc{=ins} \sc{1sg=add} die\sc{-pst[3sg]=nmlz.sg} think\sc{-imp[1.P]}\\
		\rede{When it (the stele) crumbles down, consider me dead, too.}\source{18\_nrr\_03.017--018}
		
	
Usually, the question pronoun in the relative clause and the demonstrative in the main clause come from the same paradigm, but the next example shows that this is not a rigorous constraint. While the question pronoun in the relative clause refers to a location, the corresponding \isi{noun phrase} in the main clause refers to the food found in this location.	
		
	\exg. nhaŋ   heʔne  camyoŋba wa=ya nakha camyoŋba nak-se khe-i.\\
		and\_then where food exist{\sc [3sg]-sbjv} that  food ask-{\sc sup} go{\sc -1pl[sbjv]}\\
		\rede{Let us go where the food is, to ask for that food, they said.} \source{14\_nrr\_02.009}


Correlative clauses are rather known as a typical Indo-Aryan feature. They are not very common in \isi{Tibeto-Burman} languages further north, which suggests that \isi{correlative clauses} in Yakkha have developed as calques on \ili{Nepali} \isi{correlative clauses}. 
 