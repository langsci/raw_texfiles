\chapter{Simple clauses}\label{simp-cl}

As pointed out in \citet[224]{Dryer2007Clause}, at least four perspectives come to mind when talking about clause types: (i) the  distinction between \isi{declarative}, interrogative, and \isi{imperative} speech acts, (ii) the distinction between main and subordinate clauses, (iii) the properties of clauses and their constituents as building blocks of discourse, and (iv) types of clauses based on different kinds of predicates and their \isi{argument structure}. The purpose of this chapter is to provide a bird's eye view from all four perspectives. An in-depth treatment of the \isi{argument structure} can be found in Chapter \ref{verb-val}, some aspects of \isi{information structure} are dealt with in Chapter \ref{particles} and subordinate clauses are the topic of  Chapters \ref{ch-nmlz},  \ref{adv-cl} and  \ref{compl}.  

The chapter is structured as follows: \sectref{simp-cl1} discusses general structural properties of simple independent clauses, \sectref{simp-cl2} lays out constituent structure. Different types of illocutionary acts and how they affect the shape of a clause are discussed in \sectref{simp-cl3}. Finally, \sectref{flex-agr} introduces the regularities of agreement in Yakkha. In line with what is known about agreement in \isi{Tibeto-Burman} languages in general, the agreement in Yakkha is less restricted than in Indo-European languages and does not require a complete matching between the referential features of the agreement markers and their controllers.

\section{Basic clausal properties}\label{simp-cl1}

Independent clauses present propositional content independently of other syntactic units. They are the unit in which any grammatical category of a language can be expressed, other than in dependent clauses (relative clauses, complement clauses or adverbial clauses), which are often restricted in some way. Some operators generally function on the level of an independent clause, like the \isi{mirative} \emph{rahecha} (a \ili{Nepali} loan) or  \isi{exclamative} particles like  \emph{baʔlo} and \emph{haʔlo} (see \Next[a]). The \isi{reportative} marker \emph{=bu} is also found clause-finally (see \Next[b]), but occasionally it is also found   on embedded clauses containing indirect speech and on constituents inside the clause (see \sectref{hearsay}). 
 
 Most markers, for instance information-structural particles and \isi{case} markers, are found  both on clausal constituents and on clauses, which is not unusual in \isi{Tibeto-Burman}. There are, however, tendencies for certain markers to appear on noun phrases (e.g., the \isi{topic marker} \emph{=ko}, the \isi{contrastive focus} marker \emph{=le}, the emphatic marker \emph{=maŋ}) or rather on clauses or both (\isi{case} markers, several \isi{postpositions}, the \isi{restrictive focus} marker \emph{=se}, the \isi{additive focus} marker \emph{=ca}, the \isi{reportative} marker \emph{=bu}). 
 
 Another feature which distinguishes independent clauses is the clause-final afterthought position which is accompanied by an intonation break as a means to provide additional information about one of the (overt or omitted) referents (see \NNext). All kinds of arguments are possible in the afterthought position. 
 
 Yakkha has verbal clauses, copular clauses and verbless clauses, the latter  reflecting a copular structure without a \isi{copula}, usually with identificational or ascriptive semantics.

 \ex. \ag.n-so-ks-u=na                           rahecha baʔloǃ\\
 {\sc neg-}look{\sc -prf-3.P[pst]=nmlz.sg} {\sc mir} {\sc excla}\\
 \rede{Oh, he did not look into it!} \source{34\_pea\_04.039}
  \bg. yondaŋ=ca           khendaŋ=ca         pik=ci=le        n-du-ma-sim-me=ha=bu,           ibebe=ga.\\
 from\_thither{\sc =add} from\_hither{\sc =add} cow{\sc =nsg=ctr} {\sc 3pl-}step\_on{\sc -inf-aux.prog-npst=nmlz.nsg=rep} everywhere{\sc =gen}\\
 \rede{Here and there, cows are trampling everywhere, it is said, everywhere.}  \source{39\_nrr\_08.25}

 \ex.\ag. liŋkha=ci nam=nuŋ bʌgʌri n-jog-a,  bʌgʌri.\\
 	a\_clan{\sc =nsg} sun{\sc =com}   bet    {\sc 3pl-}do{\sc -pst} bet\\
 	\rede{The Linkhas had a bet with the sun, a bet.} \source{11\_nrr\_01.003}
 \bg.  ŋkhaʔla   lu-saŋ        uŋ=ŋa   nna,  luŋkhwak, cuʔlumphi thukt-uks-u,           nna  kolenluŋ=ga.\\
 like\_that tell{\sc -sim} {\sc 3sg=erg} that stone stele erect{\sc -prf-3.P[pst]} that marble{\sc =gen}\\
 \rede{Telling them like that, he erected that stone, the stele, out of marble.} \source{18\_nrr\_03.021}
\bg. heksaŋ nna  miyaŋ n-thog-haks-uks-u=na     raecha, pik=ci=ŋa,      goru=ci=ŋa.\\
later that a\_little {\sc 3pl.A-}hit{\sc -V2.send-prf-3.P=nmlz.sg} {\sc mir} cow{\sc =nsg=erg} ox{\sc =nsg=erg}\\
\rede{Later, they hit off some of it (the stone), the cows, the  oxen.} \source{39\_nrr\_08.03}

 
\section{Constituent order}\label{simp-cl2}

Yakkha has a flat clause structure; there is no evidence for  a unit like the verb phrase (verb and arguments). The unmarked \isi{constituent order} is head-final in phrases and \textsc{sov} in clauses, with increasingly rhematic status towards the end of the clause. Focal information is often put in pre-verbal position (see \Next). Noun phrases are optional (see \NNext[a]), and clauses where all arguments are represented by noun phrases are rather rare. In complex clauses, the main clause is mostly in final position (see \NNext[b]). Yakkha is both head- and dependent-marking on the clausal level, since it has both \isi{case} marking of arguments and verbal person indexing. 

\ex. \ag. ilen paŋdaŋba=ga    u-niʔma   mas-a-by-a-masa.\\
 one\_day landlord{\sc =gen} {\sc 3sg.poss-}money lose{\sc -pst-V2.give-pst-pst.prf[3]}\\
 \rede{One day, the landlord's money got lost.}
 \bg.  aniŋ=ga         liŋkha=ga      uhile         utpatti mamliŋ=be      leks-a=na=bu.\\
 	{\sc 1pl.excl=gen}  a\_clan{\sc =gen} long\_ago origin  Mamling{\sc =loc} happen{\sc -pst=nmlz.sg=rep}\\
 	\rede{Our Linkha clan's origin, long ago, was in Mamling, they say.} \source{11\_nrr\_01.002}
 \bg. tabaŋ hetne tas-wa-ga=na.\\
 	male\_in-law where arrive-{\sc npst-2=nmlz.sg}\\
 	\rede{Where will (your) husband arrive?}
 
 \ex. \ag.mund-y-uks-u-ga=na=i?\\
 forget{\sc -compl-prf-3.P-2=nmlz.sg=q}\\
 \rede{Did you forget it?}
 \bg.tek whap-se khe-me-ŋ=na.\\
clothes wash{\sc -sup} go{\sc -npst-1sg=nmlz.sg}\\
 \rede{I go to wash clothes.}

 
\textsc{sov} is the default \isi{constituent order}, but a fronted P argument is possible when it is more topical than the A argument, as in  \Next[a]. Such examples are best rendered by English passive constructions. When the object position is filled by embedded direct speech, the preferred order is the one with the object preceding the A argument, as shown in \Next[b]. 

\ex. \ag.   hoʈʌl=beʔ=ya          cameŋwa  cica=ci=ŋa      m-bupt-wa=ha.\\
hotel{\sc =loc=nmlz.nsg} food fly{\sc =nsg=erg} {\sc 3pl.A-}surround{\sc -npst[3.P]=nmlz.nsg}\\
\rede{The food in the hotel is surrounded by flies.} \source{01\_leg\_07.053}
\bg.ka,  ka  khaʔla   ŋana=ba                uŋ=ŋa   lu-ks-u-ci.\\
{\sc 1sg} {\sc 1sg} like\_this {\sc cop.1sg=emph} {\sc 3sg=erg} tell{\sc -prf-3.P-3nsgP}\\
\rede{I, I am just like this; it (the bird) told them (the other birds).} \source{21\_nrr\_04.006}
 
 Initial constituents often carry \isi{topic marker}s  like \emph{=chen} (a  loan, from \ili{Nepali} \emph{cāhĩ}) or \emph{=ko}. In \Next, these particles mark  contrastive topics.
 
 \ex. \ag.eŋ=ga=chen                nna  man=na.\\
 {\sc 1pl.incl=gen=top} that  {\sc cop.neg=nmlz.sg}\\
 \rede{We do not have that (custom).}  \source{29\_cvs\_05.165}
 \bg.kham=go     m-bi-me=ha.\\
 soil{\sc =top} {\sc 3pl.A-}give{\sc -npst=nmlz.nsg}\\
 \rede{As for soil, they offer it (throwing it into the grave).}\footnote{In contrast to water, as one interlocutor had claimed before.}\source{29\_cvs\_05.159}
 
 In copular clauses the \isi{topic}  (referential or locational) naturally precedes the complement, which can be adjectival, nominal or \isi{locative} (see \Next). Copulas in affirmative clauses are not obligatory. There are different copulas for equational/ascriptive  and existential/\isi{locative} complements. The details are discussed in \sectref{cop}. 
 
 \ex. \ag.nhaŋ,    henca-khuba=chen  yakkha          om.\\
 and\_then win{\sc -nmlz=top} of\_Yakkha\_affiliation {\sc cop}\\
 \rede{The winner is/will be Yakkha.} (from a story where two groups fight about whether they are Yakkha or not) \source{ 39\_nrr\_08.17}
   	\bg.  haku        camyoŋba  nak-ma=na               ʈhaun ma-n.\\
		now food ask{\sc -inf=nmlz.sg} place {\sc cop.neg-neg}\\
	\rede{Now there is nowhere to ask for food.} \source{14\_nrr\_02.10}
	\bg.u-yum=be waiʔ=na.\\
	{\sc 3sg.poss-}side{\sc =loc} exist{\sc [3]=nmlz.sg}\\
	\rede{It is next to it.}
	
\section{Illocutionary functions}\label{simp-cl3}
 

\subsection{Declarative clauses }
 Yakkha distinguishes \isi{declarative}, \isi{imperative}, \isi{hortative}, and various types of interrogative and \isi{exclamative} clauses.
 Independent \isi{declarative} clauses either have a verbal or a copular predicate, as many examples in  \sectref{simp-cl1} and \sectref{simp-cl2} have shown. Among the clauses with a verbal predicate, the \isi{declarative} and interrogative clauses have to be specified for \isi{tense}. There is one \isi{emphatic particle} \emph{=pa} (originating in a \isi{nominalizer}) which is found in \isi{declarative}, \isi{hortative} and  \isi{imperative} clauses, but never in questions (see \Next). The \isi{exclamative} force can be amplified by attaching the \isi{exclamative} particle \emph{=ʔlo} to \emph{=pa}. The resulting unit expressing both emphasis and \isi{exclamative} force is an independent word as far as stress is concerned, but the \isi{voicing} rule still applies (see \Next[c]). 
 
 \ex. \ag.ɖuŋga=go     ka-i-wa=ba,               kaniŋ=go.\\
 boat{\sc =top} say{\sc -1pl-npst=emph} {\sc 1pl=top}\\
 \rede{We just say \rede{ɖuŋga}, we.} (speaker puzzled about not finding a Yakkha word for \rede{boat} and having to use the \ili{Nepali} word instead) \source{13\_cvs\_02.006 }
 \bg.men=ba!\\
 {\sc cop.neg[3]=emph}\\
 \rede{Of course not!}
 \bg.yo,   nhe  pi-haks-a-masa       baʔlo!\\
 across here give{\sc -V2.send-pst-pst.prf[1.P]} {\sc emph.excla}\\
 \rede{(My father) over there had sent me here (in marriage), man!} \source{06\_cvs\_01.016}
 
Much more frequent on \isi{declarative} as well as on interrogative clauses are the clitics \emph{=na} and \emph{=ha}, originating from clausal nominalizations, a common development in \isi{Tibeto-Burman} languages (and beyond). They are never found on hortatives, optatives or imperatives. Their function is hard to pin down by one neat term, but “assertive force” may give the reader an idea of their function. They occur more frequently than the emphatic \emph{=ba}. They are discussed at length in Chapter \ref{ch-nmlz}. 

Declaratives (and imperatives) can also be emphasized by another marker \emph{=i}, as shown in \Next. 

\newpage 

\ex.\ag.ca-i-ŋ=na=i.\\
eat{\sc -compl-1sg=nmlz.sg=emph}\\
\rede{I finished eating.} \source{36\_cvs\_06.241}
\bg.ab-a-ga=i!\\
come{\sc -imp-2=emph}\\
\rede{Come!}

Other clause-final markers that have already been mentioned are the \isi{mirative} and the \isi{reportative} marker, the latter is frequently found in reported speech and in passed-down narratives, and in all other contexts where the speaker wants to free himself of the responsibility for the content of the utterance.

 
\subsection{Hortative and optative clauses}\label{clstr-opt}
 
Hortative clauses are uttered when the speaker wants to urge or encourage someone to do something together. There is no dedicated \isi{hortative} marker, the verb just appears in the main clause subjunctive\footnote{There are two subjunctives in Yakkha: the main clause subjunctive and the dependent clause subjunctive, the latter being  identical to the past inflection in most person configurations, cf. Chapter \ref{verbalmorph}.}  first person dual or plural inflection and without any \isi{tense}/\isi{aspect} specification. \Next[a] shows an intransitive example, \Next[b] shows a transitive example with an additional particle \emph{au}. It may occur in hortatives and imperatives, lending force to the request or order.  The \isi{constituent order} is like in \isi{declarative} clauses, presuming  overt arguments are there at all.

In general, the subjunctive just expresses vagueness about some future event. If it is used with first person singular or with \isi{exclusive} inflection, it becomes a \isi{permissive} question, with the typical intonation contour of high pitch at the end of the clause (see \NNext[a]). With second or third person,  the subjunctive can be rendered with \rede{you might [...]} or \rede{he might [...]}, and can express warnings (see \NNext[b]), threats \NNext[c] and statements about possibilities in general. 
\largerpage

\ex. \ag.cuŋ-i,    kaks-i.\\
wrestle{\sc -1pl[sbjv]} make\_fall{\sc -1pl[sbjv]}\\
\rede{Let us wrestle, let us fight!} \source{39\_nrr\_08.15}
\bg. a-na,     a-ma=ŋa                py-a=ha   yaŋ=ŋa    limlim inca-c-u,       au?\\
{\sc 1sg.poss-}sister {\sc 1sg.poss-}mother{\sc =erg} give{\sc -pst[1.P]=nmlz.nsg} money{\sc =ins} sweets buy{\sc -du-3.P[sbjv]} {\sc insist}\\
\rede{Sister, let us buy sweets with the money that mother gave us, shall we?} \source{01\_leg\_07.042}

\ex.\ag. ka kheʔ-ŋa, au?\\
  {\sc 1sg} go{\sc -1sg[sbjv]} {\sc insist}\\
  \rede{I am off, allright?}
  \bg.kaks-i-khe-i-gaǃ\\
  fall{\sc -pl-V2.go-pl-2[sbjv]}\\
  \rede{You (plural) might fall downǃ}
  \bg.lem-nhaŋ-nen?\\
  throw{\sc -V2.send-1>2[sbjv]}\\
  \rede{Shall I throw you out?}
  
The above-mentioned \isi{emphatic particle} \emph{=pa} in combination with hortatives is shown in \Next.
  
  \ex.\ag.lamdhaŋ=ca        khond-u-m=ba, aniŋ=ga ya=ca           hond-u-m=ba\\
 field{\sc =add} dig{\sc -3.P-1pl.A=emph} {\sc 1pl.excl=gen} mouth{\sc =add} open{\sc -3.P-1pl.A=emph}\\
 \rede{Let us dig our fields and also open our mouths.}\footnote{The \isi{exclusive} is used here because the researcher was present during the song (a spontaneous song about the endangerment of the Yakkha language, and an encouragement for the hearers to speak the language), even though the people addressed are of the singer's group. It might also be due to the unnatural recording situation that the singer was not sure whom to address with her song. In later recordings, speakers vary as to whether they use the \isi{inclusive} or the \isi{exclusive} forms when the fieldworker is present, which can partly be attributed to including her and partly to ignoring her or forgetting about her presence.}  \source{07\_sng\_01.14}
  \bg.haku khe-ci=ba\\
  now go{\sc -du=emph}\\
  \rede{Now let us (dual) goǃ}
  
  
  The \isi{optative} expresses the wish of the speaker for something to happen that is beyond his immediate control. It is constructed with subjunctive forms to which the marker \emph{-ni}  is added \Next. Optative forms can also be found in purposive adverbial clauses (rendered by \rede{in order to [...]}). Verbal \isi{negation} is possible in these forms as well. Full paradigms can be found in \sectref{mood}.
  
\largerpage

  \ex. \ag.siŋ pu-ni.\\
  tree grow{\sc [3sg]-opt}\\
  \rede{May the tree grow.}
  \bg.miʔ-ŋa-ni.\\
  think{\sc -1sg.P-opt}\\
  \rede{May he remember me.}
 
\subsection{Imperative and prohibitive clauses}

Imperatives are uttered to make the addressee do something. Prohibitives are used to prevent the addressee from doing something. The \isi{imperative} marker is \emph{-a},\footnote{Often deleted as a strategy to avoid vowel hiatus.} identical to the past marker, but the person inflection at least partly differs from declaratives, so that most forms are distinct from their past counterparts. Imperatives have a colloquial register \Next[a] and a polite register \Next[b]. The polite forms have an additional marker \emph{-eba}, which has probably developed from the emphatic particles \emph{=i} and \emph{=pa}. They are used for elders, guests and other respected people. This is the only instance of grammaticalized politeness forms in the Tumok dialect of Yakkha.\footnote{In Dandagaun village, people make politeness distinctions also in declaratives and questions. They use a calqued form from \ili{Nepali}, which is discussed in \sectref{honorific}.} 

\ex. \ag.hani ket-u!\\
quickly bring\_up{\sc -3.P[imp]}\\
\rede{Bring it up quickly!}
\bg.naʔtorok naʔtorok ket-u-eba!\\
a\_bit\_further\_up a\_bit\_further\_up  bring\_up{\sc -3.P[imp]-pol.imp}\\
\rede{Bring it a bit further up, please!} 

Formally, prohibitives are  negated imperatives (see \sectref{mood} for the morphology). The examples in \Next contrast  imperatives with prohibitives in the colloquial register.

\ex. \ag.khy-a!\\
go{\sc -imp}\\
\rede{Go!}
\bg.ŋ-khy-a-n!\\
{\sc neg-}go{\sc -imp-neg}\\
\rede{Do not go!}
\bg.kisa ab-a-n-u-m!\\
deer shoot{\sc -imp-pl-3.P-2pl.A}\\
\rede{Shoot (plural) the deer!}
\bg.maksa ŋ-ab-a-n-u-m-nin!\\
bear {\sc neg-}shoot{\sc -imp-pl-3.P-2pl.A-neg}\\
\rede{Do not (plural) shoot the bear!}

\newpage %long distance effects
Arguments can be expressed overtly in imperatives, including S and A arguments. They are not calls or vocatives, since they are not set apart by an intonational break, and since vocatives may precede them, as in \Next[a]. If the person addressed by the \isi{imperative} of a transitive verb is referred to by a noun and not by a pronoun, it will be marked by the \isi{ergative}.

\ex. \ag.lu,         lu,         mamu, nda=ca        phat-a-ŋ.\\
{\sc init} {\sc init} girl {\sc 2sg=add} help{\sc -imp-1sg.P} \\
\rede{Come on, come on girl, help me too!} \source{07\_sng\_01.01}
\bg.ka  um-me-ŋ,            nniŋda lakt-a-ni,           nhaŋ    kaniŋ ikhiŋ   lakt-iʔ-wa.\\
{\sc 1sg} enter{\sc -npst-1sg}  {\sc 2pl} dance{\sc -imp-pl} and\_then {\sc 1pl} how\_much dance{\sc -1pl-npst}\\
\rede{I will enter (the basket), you (plural) dance. And how much we will dance!} \source{14\_nrr\_02.29}

Occasionally, one finds the second person marker combined with the emphatic marker \emph{=i} attached to the \isi{imperative}, which increases the insistence  of the request or command. This marker is also found on \isi{declarative} clauses.

\ex. \ag.mendhwak ghororo sa-ga=i!\\
goat forcefully pull\_with\_rope{\sc [imp]-2=emph}\\
\rede{Pull the goat forcefully!}
\bg.ab-a-ga=i!\\
come\_across{\sc -imp-2=emph}\\
\rede{Come here!}

\subsection{Interrogative clauses}
  
  
  
\subsubsection{Polar questions}
 
 Polar questions obligatorily host the clause-final clitics \emph{=na} and \emph{=ha} if they contain a verb,\footnote{It is misleading, however, to perceive these clitics as markers of \isi{polar questions}, because they occur in other clause types as well.} and  a particle \emph{i}  that is found both phonologically bound and unbound. When it occurs unbound, it carries its own stress and has an initial glottal stop prothesized (not written in the \isi{orthography}), as all  vowel-initial words have (see \Next). The conditions for the alternation between bound and unbound are not clear yet. The word order is the same as in declaratives. Polar questions are typically answered by repeating the (verbal or nonverbal)  predicate (see \NNext), or by one of the \isi{interjections} \emph{om} \rede{yes}, and \emph{menna} or \emph{manna} for \rede{no} (identificational and existential/locational, respectively).\footnote{On a sociolinguistic side note, when meeting someone familiar (or calling on the phone, nowadays), one often asks whether the interlocutor has already eaten rice or drunken tea, depending on the time of day.} 
 
 \ex. \ag.nda yakkhama              mekan=na=i? \\
 {\sc 2sg} Yakkha\_woman {\sc cop.2sg.neg=nmlz.sg=q}\\
 \rede{Aren't you a Yakkha woman?} \source{36\_cvs\_06.547}
 \bg.nda lak-ma miʔ-me-ka=na=i?\\
 {\sc 2sg} dance{\sc -inf} think{\sc -npst-2=nmlz.sg=q}\\
 \rede{Do you want to dance?}
 \bg.ka i?\\
 {\sc 1sg} {\sc q}\\
 \rede{You mean me?}
 \bg.men=na=i,                khaʔla   so-nhaŋ-se               i?\\
 {\sc neg.cop=nmlz.sg=emph} like\_this watch{\sc -V2.send-sup} {\sc q}\\
 \rede{In order to watch them (as they leave)?} \source{36\_cvs\_06.489}
 
 \ex. \ag.yakthu=i?\\
 enough{\sc =q}\\
 \rede{Did you have enough?}
 \bg.yakthu.\\
 enough.\\
 \rede{I had enough.}
 


In conversations, one often hears tag questions like \emph{mennai?} \rede{isn't it?}. They request a confirmation from the interlocutor that the propositional content of the preceding utterance is true (see examples in \Next). Sometimes they may just convey uncertainty on behalf of the speaker, since in \Next[c], from a pear story, the interlocutor is not supposed to know about the content. 
 
\ex. \ag.  to  thaŋ-a-ŋ-ci-ŋ,   men=na=i?\\
up climb{\sc -pst-excl-du-excl}  {\sc cop.neg=nmlz.sg=q}\\
\rede{We climbed up, didn't we?} \source{36\_cvs\_06.267}
\bg. pi-uks-u,            men=na=i?\\
give{\sc -prf-3.P[pst]}  {\sc cop.neg=nmlz.sg=q}\\
\rede{She gave it to her, didn't she?} \source{36\_cvs\_06.381}
  \bg.    nhaŋ=ŋa      khaʔla   lukt-a-sy-a-ci,    men=na=i?\\
  and\_then{\sc =ins} like\_this collide{\sc -pst-mddl-pst-du} {\sc cop.neg=nmlz.sg=q} \\
  \rede{And then they collided like this, didn't they?} \source{34\_pea\_04.025  }
  
  
\subsubsection{Disjunctive questions}
 
 Disjunctive questions consist of two juxtapposed alternative scenarios, both marked by the alternation marker \emph{=em} \Next. If it attaches to a word that ends in /a/ or /e/, the first vowel gets deleted, e.g., /nhaŋ=le=em/ \rede{or afterwards} is pronounced	[nhaŋlem]. This marker is not only found in interrogative clauses. Occasionally, it also attaches to hypothetical clauses, not following the template of two juxtaposed alternatives, but rather expressing uncertainty (see \Next[c]). Although the verb in this clause is marked for nonpast, and hence is in realis \isi{mood}, the marker \emph{=em} weakens the realis interpretation of this utterance, and thus it is best rendered with a subjunctive in the English translation.
 
 \ex. \ag.nniŋda yakkha          om=em,    men=em?\\
 {\sc 2pl} of\_Yakkha\_affiliation {\sc cop=alt} {\sc cop.neg=alt}\\
 \rede{Are you (plural) Yakkha or not?} \source{39\_nrr\_08.10}
 \bg.khumdu=em ŋ-khumd-in=em?\\
 tasty{\sc =alt} {\sc neg-}tasty{\sc -neg=alt}\\
 \rede{Is it tasty or not?}
 \bg.nda i=ya=ca men-gap-khuba luʔ-ni-me-ŋ=n=em.\\
 {\sc 2sg} what{\sc =nmlz.nsg=add} {\sc neg-}carry{\sc -nmlz} tell{\sc -compl-npst-1sg.P=nmlz.sg=alt}\\
 \rede{He might tell me: You have nothing.} \source{36\_cvs\_06.349}
 
\subsubsection{Content questions}

Content questions contain one of the interrogative pro-forms introduced in \sectref{interr}. Question words remain in situ (see \Next[a]), even in adverbial clauses such as \Next[b] and embedded clauses such as \Next[c]. The embedded clause in \Next[c] reflects direct speech (see the \isi{imperative} and the \isi{person marking}). Question words or phrases are often marked by the focus marker \emph{=le}. In declaratives, it marks contrastive information,  and it is also often found in \isi{mirative} contexts. In questions, it implies that the speaker is particularly clueless and very eager to get the answer.
\newpage 

\ex. \ag. ŋ=ga          paŋ  heʔne om?\\
{\sc 2sg.poss=gen} house where {\sc cop}\\
\rede{Where do you live?} \source{01\_leg\_07.160}
\bg.heʔna ceʔya yok-se ta-ya=na.\\
which  matter search{\sc -sup} come{\sc [3sg]-pst=nmlz.sg}\\
\rede{Which language did she come to search for?}
\bg. ka ina=le khut-a-ŋ ly-a-ŋ-ga=na?\\
{\sc 1sg} what{\sc .sg=ctr} bring{\sc -imp-1sg.P} tell{\sc -pst-1sg.P-2.A=nmlz.sg}\\
\rede{What did you tell me to bring (you)?}\footnote{The somewhat unusual verb form \emph{khutaŋ} \rede{bring me!} is found here because structurally, this is embedded direct speech.}


There is an interjection \emph{issaŋ} that can be provided as an answer when the person asked does not know the answer either. It has a rising intonation contour just like questions, and means as much as \rede{I do not know} or \rede{no idea}, but often contains the subtext \rede{How am I supposed to know?}, and is thus very similar in usage to the \ili{Nepali} interjection \emph{khoi}. It can be used as an answer to any question type.

More information on the discourse-structural particles touched upon in this chapter can be found in Chapter \ref{particles}.

\subsection{Exclamative clauses}

Strictly speaking, \isi{exclamative} clauses are not a distinct clause type because their formal structure is identical to interrogative clauses. They always contain the interrogative quantifier \emph{ikhiŋ} (or its nominalized forms \emph{ikhiŋna/ikhiŋha}) which is used to inquire about the size, amount or \isi{degree} of nominal, verbal and adjectival concepts. Functionally, exclamations can be defined as declaratives with high expressive value, containing some extreme and especially remarkable information \citep[316]{Koenig2007_Speech}. Although they are questions syntactically, their falling intonation differs from that of questions, which have a rising intonation towards the end of the clause.

\ex. \ag. ikhiŋ   ucun, hen=na din=be    tub-i=ha!\\
how\_much nice today{\sc =nmlz.sg} day{\sc =loc} meet{\sc -1pl=nmlz.nsg}\\
\rede{How nice (that) we met today!} \source{10\_sng\_03.003}
\bg.ka  ikhiŋ   pe-me-ŋ=na!\\
{\sc 1sg} how\_much fly{\sc -npst-1sg=nmlz.sg}\\
\rede{How much (how high) I will flyǃ} \source{21\_nrr\_04.032}
\bg.ikhiŋ chippakekeʔ=na takabaŋǃ\\
how\_much disgusting{\sc =nmlz.sg} spider\\
\rede{What a disgusting spiderǃ}

\section{Flexible agreement}\label{flex-agr}

The syntax of the agreement in Yakkha and \isi{Tibeto-Burman} in general is much more flexible than in Indo-European languages. As noted earlier by \cite{Bickel2000On-the-syntax}, the purely identificational agreement that is known from Indo-Aryan is accompanied by associative (appositional and partitive) agreement types. 

In appositional agreement, the \isi{noun phrase} that corresponds to the agreement marker is semantically an apposition to the antecedent of that marker, but syntactically it is the argument. In the clauses in \Next the person value of the arguments is only revealed by the verbal \isi{person marking}. The corresponding nouns provide additional information about the referent, while  they are the arguments syntactically. We know that the \isi{ergative} \isi{case} is not overtly marked on arguments that are represented by first and second person pronouns. It is overtly marked on nouns with first or second person reference (see \Next[c]), which shows that the differential agent marking is determined by a combination of person features and word class.

\ex.  \ag. mamu heʔne khe-i-g=ha.\label{verb-infl-example}\\
		girl where go{\sc [pst]-2pl-2=nmlz.nsg}\\
		\rede{Where did you girls go?}
	\bg.  kamnibak sori yuŋ-i=hoŋ uŋ-u-m.\\
		friend together sit{\sc -1pl[pst]=seq} drink{\sc -3.P-1pl.A[sbjv]}\\
		\rede{Having sat down together, let us friends drink.}
	\bg. a-koŋma=ŋa=le   ta-ga=na           raecha.\\
		{\sc 1sg.poss-}aunt{\sc =erg=ctr} bring{\sc [pst]-2.A[3.P]=nmlz.sg} {\sc mir} \\
		\rede{You, auntie, really brought her (the second wife)!} \source{06\_cvs\_01.042}
		\bg.phu=na yapmi pham=na yapmi leks-a-ŋ=na.\\
		white{\sc =nmlz.sg} person red{\sc =nmlz.sg} person become{\sc -pst-1sg=nmlz.sg}\\
		\rede{I, the white person, turned red.}
 
 
 The second type is partitive agreement, shown in \Next. Here, the verbal \isi{person marking} refers to a group of potential referents, while the \isi{noun phrase} refers to the subset of actual referents. The verb here shows nonsingular \isi{number} marking, although the referent of the A argument has singular \isi{number}.\footnote{With pronouns like \rede{who} and \rede{none} it is not always easy to determine the semantic \isi{number}. In Yakkha, however, if one inquires about the identity of more than one person, one generally uses the nonsingular marker \emph{isa=ci}.}  In \Next[b] this referential mismatch extends from an embedded clause into the main clause, as it is found on both the matrix verb \emph{kama} \rede{say} and the verb in the embedded speech \emph{khuma} \rede{steal}. The sentence is paraphrasable with \rede{But none of them said: {\bf we} stole the money}, although it is clear from the question pronoun (and from how the story proceeds) \emph{isa} \rede{who} that the purpose of the clause is to single out one referent.
  
 \ex. \ag. nniŋda  sum-baŋ=be          isa=ŋa   yaŋ  khus-uks-u-m-ga?\\
 	{\sc 2pl} three{\sc -clf.hum=loc} who{\sc =erg} money steal{\sc [pst]-prf-3.P-2pl.A-2}\\
 	\rede{Who of you three stole the money?}
 	\bg. khaʔniŋgo isa=ŋa=ca      khus-u-m-ŋa=ha ŋ-ga-ya-ma-nin.\\
 	but who{\sc =erg=add} steal{\sc [pst]-3.P-1pl.A-excl=nmlz.nc} {\sc 3pl.A-}say{\sc -pst-prf-pl.neg}\\
 	\rede{But none of them said: I stole the money.} \source{04\_leg\_03.018}
\bg.isa=ja n-nis-u-nin=na.\\
who{\sc =add} {\sc neg-}see{\sc -3.P[pst]-neg.pl=nmlz.sg}\\
\rede{None of them had seen her.} \source{22\_nrr\_05.071}







