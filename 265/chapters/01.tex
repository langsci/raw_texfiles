\chapter{Introduction: Rationale and research questions}\label{sec:1}

The purpose of this work is to analyse the strategies of input processing employed by beginner learners of Polish in the earliest stages of second language acquisition (SLA), with a particular focus on morphosyntax. It targets a minimal fragment of grammar, i.e. a subset of the singular number of the paradigm of feminine nouns in -\textit{a} (e.g. \textit{siostra} ‘sister’). The morphosyntactic opposition of interest contrasts the nominative case (NOM), encoded by the ending -[a] <a>, and the accusative case (ACC), encoded by the ending -[e] <ę>, which respectively correspond to the subject (SUBJ) and object (OBJ) syntactic functions \REF{ex:01:1}.

\ea%1
    \label{ex:01:1}
    \gll    siostr-a 	lubi 	Warszaw-ę\\
            sister-\textsc{nom} likes   Warsaw-\textsc{acc}\\
    \glt    ‘(my) sister likes Warsaw’
    \z

Case morphology makes it possible to vary word order as required for pragmatic purposes, as in \REF{ex:01:2}, in which word order is manipulated to topicalise the OBJ. 

\ea%2
    \label{ex:01:2}
    \gll    Warszaw-ę           lubi   siostr-a,            nie  brat-${\emptyset}$\\
            Warsaw-\textsc{acc} likes  sister-\textsc{nom}  not  brother-\textsc{nom}\\
    \glt    ‘it is (my) sister who likes Warsaw, not (my) brother’
    \z

This is the target system which the learners mentioned in the present book will be called to master. 

The data are a subset of the results of the VILLA project \citep{DimrothEtAl2013}, a large SLA experiment devoted to the earliest stages of the acquisition of L2 Polish. 188 participants divided into five L1 groups (English, German, Dutch, French, Italian) were selected based on their lack of experience in Slavic languages, so as to make sure that the acquisition process started approximately from the same baseline for all of them, i.e. from scratch. Each L1 group took part in a two-week, 14-hour L2 Polish course taught by a specially trained native speaker of Polish, who delivered the same material in the various editions of the project. The course was designed in such a way as to expose the participants to input carefully planned in terms of both range of lexical and grammatical structures and frequency; moreover, learners were often asked to engage in simple question-and-answer exchanges with the teacher as well as among themselves. The development of their interlanguage was observed through several tasks targeting various layers of language, such as phonology, morphosyntax and pragmatics.

The teacher’s speech and the PowerPoint slides she used during classes represented the only sources of Polish input available to the learners. This was entirely recorded and transcribed, so that it is possible to search for regular correspondences between learner performance and the relevant features of teacher input.

The processing of the target structure is investigated in three different tasks: a comprehension task, an elicited imitation task, and a semi-spontaneous production task. This approach makes it possible to obtain a wider picture as to learner skills in the interlanguage than is typically possible in a psycho-linguistic experiment, while at the same time maintaining the same amount of control over a wide range of variables. In addition to providing a comprehensive picture of learner skills in the manipulation of target morphosyntax, this work also attempts to identify a scale of task difficulty, defined as the interaction of the skill required to perform the exercise (comprehension, production), the target structure (subject-object (SO) vs. object-subject (OS) sentences) and the context in which the elicitation of the data takes place (structured test vs. semi-spontaneous production). The analysis is pursued on the basis of a fairly large dataset collected in a methodologically thorough manner, which makes it possible to exclude an uncontrolled effect of such variables as the amount and quality of the input received and the existing skills in the target language. Indeed, it has long been argued that acquisition success (and the related notion of acquisition difficulty) results from the complex interactions of a wide range of factors (\citealt{HousenSimoens2016}). The present work aims to experimentally exclude or control as many variables as possible in order to focus on the three main factors mentioned above.

This introductory chapter aims to contextualise this work against the wider picture of SLA studies, with particular respect to the initial stages of acquisition. The main research questions may be summarised as follows:

\begin{itemize}
    \item What principles of utterance organisation do initial learners of L2 Polish adopt in order to identify and express syntactic functions?
    \item Do principles of utterance organisation vary depending on the communicative situation in which the target language is processed? Can a scale of task difficulty be identified?
    \item Does the learner’s L1 influence the speed or accuracy of the acquisition process?
    \item Is there a relation between selected input parameters and the observed L2 output? Three frequency-based descriptors of the controlled VILLA input are considered, i.e. the strength of the association between inflectional endings and the corresponding grammatical meaning, the type frequency of SUBJ and OBJ and the frequency of transitive structures described in terms of word order and referent animacy and gender.
\end{itemize}

\section{Word order vs. inflectional morphology}\label{sec:01:1}

The general research question pursued in this work is whether and to what extent learners employ inflectional morphology to decode and encode utterance meaning in comprehension and production. To exemplify, a simple SO utterance like \REF{ex:01:3} may be successfully interpreted by relying on at least two processing strategies, lying at the opposite poles of a continuum. 

\ea%3
    \label{ex:01:3}
    \gll    dziewczynk-a ciągnie portugalk-ę\\
            little.girl-\textsc{nom} pulls Portuguese.woman-\textsc{acc}\\
    \glt    ‘the little girl pulls the Portugues woman’
    \z

First, learners may adopt a morphosyntactic principle, i.e. derive syntactic functions from inflectional morphology. This requires that case endings are categorised into paradigms depending on several features of the individual item, such as inflectional class, gender, animacy and number, as the same morph — understood here as a sound or string of sounds — may vehiculate different meanings in different inflectional paradigms: for instance, even within the limited set of VILLA lexical items considered in the present work, the ending \textit{{}-a} may encode meanings such as NOM of feminine nouns or ACC of masculine animate nouns. Thus, in order to derive meaning from inflectional endings, the learner should know what inflectional class a given lexical item belongs to.

Alternatively, the utterance may be interpreted based on a positional principle, whereby words are assigned syntactic functions depending on their relative position in the utterance, i.e. the order in which they appear in the string. In this respect, various sources suggest that the SO word order should be considered the unmarked option. Firstly, SO is the dominant constituent order in all L1s involved in the VILLA project, although with various degrees of rigidity. Secondly, typological research indicates that this constituent order is by far more widespread among the languages of the world than OS orders \citep{Dryer2013b}. The reasons for such biased distribution, in turn, are generally believed to be of a cognitive nature (\citealt{SiewierskaBakker2008}).

The original research question may thus be reformulated as whether or not learners will be able to process an utterance’s syntactic structure based on inflectional morphology, instead of relying on a default constituent order. 

The skill referred to is sometimes labelled in the literature as the ability to process grammatical form, in addition to lexical and pragmatic meaning. While most studies point to the fact that initial learners tend to focus on lexical morphemes and ignore grammatical ones (e.g. the primacy of meaning principle defined by \citet{VanPatten1996} or \citeauthor{Klein1986}'s \citeyear{Klein1986} and \citeauthor{Rast2008}'s \citeyear{Rast2008} experimental evidence from Elicited Imitation tasks), cases in which learners attend to grammatical meaning (i.e. form) first are also documented. \citet{Park2013} and \citet{HanPeverly2007} reported that their beginner learners of Korean and Norwegian, respectively, employed a form-based approach, supposedly contrary to the primacy of meaning principle. In both cases, it could be supposed that this was the case because the target language was lexically so different from any languages known to the participants that no processing for meaning seemed possible. As a result, learners turned to the analysis of formal regularities in the text.

As far as the extraction of grammatical meaning is concerned, word order lies at the core of another of VanPatten’s principles, namely the \textsc{first} \textsc{noun} \textsc{principle}, according to which learners tend to process the first noun or pronoun they encounter in a sentence as the subject or agent, provided that common sense does not suggest an alternative interpretation based on the learner’s world knowledge, the situation in question or the nature of the referents involved (e.g. an inanimate noun is unlikely to be the subject of a sentence, even though it does appear in utterance-initial position). 

In the absence of any context, as is the case in the VILLA experiment described in this book, grammatical meaning can only be identified based on word order or inflectional morphology; in the case of syntactically marked structures which depart from the basic SO word order, only the latter principle will lead to the correct interpretation. For this reason, the manipulation of word order is a key diagnostic tool to detect the learner's processing strategy.

As explained in the previous section, even in a morphologically rich language such as Polish, inflectional morphology is normally not the only cue to grammatical meaning, which may be suggested — although not with full certainty — by other hints such as word order and the semantics of the lexical items involved. The structured tasks employed in the present work aim to eliminate all these ancillary resources, in order to make inflectional morphology the only source of information as to grammatical meaning. The manipulation of word order then becomes a crucial diagnostic tool of learner morphosyntactic skills: if in SO targets meaning can be identified independently of inflectional morphology by relying on the default relative order of SUBJ and OBJ, this is not possible in the case of OS targets, in which the same approach would lead to an incorrect interpretation of the utterance. This rationale is applied to the structured tests described in \chapref{sec:4} (the Elicited Imitation task) and \chapref{sec:5} (the Comprehension task).

In the case of the latter task, the research question is fairly straightforward: if learners manage to correctly identify the syntactic structure of OS targets, one can hypothesise that they are able to associate case endings to the corresponding syntactic meaning.

The situation is more complex as far as the production task is concerned, in which learners are asked to listen to a stimulus question and repeat it as accurately as possible following a distracting pause. The rationale of the test is that it does not require learners to repeat the target as a string of sounds, but rather to decode it and then re-produce it based on the present state of the interlanguage grammar (\chapref{sec:4}). In addition to a comprehension stage, then, this task also implies production, which raises an additional question: provided that learners correctly identify the syntactic structure of the target, will they rely on inflectional morphology to express it, or will they fall back on default word order? In addition, learners are required not only to understand the target, but also to repeat it as accurately as possible, which naturally includes inflectional morphology. One may therefore investigate whether the word order configuration of the stimulus question has an impact on the ability of learners to correctly inflect nouns for case.

Regarding the development of inflectional paradigms, some authors signal a phase of non-basic marking, in which a sort of mini-paradigm (\citealt{BittnerEtAl2000}) develops with only two forms: a basic one, typically modelled on the nominative case, and a marked, or non-basic one, as shown once again in examples from Slavic languages, specifically L2 Russian (4a: \citealt[188]{ArtoniMagnani2015}) and Serbian as a heritage language (4b: \citealt[209]{Di-BiaseBettoni2015}). 

\ea%4
    \label{ex:01:4}
    \ea
    \label{ex:01:4a}
    \gll    videla volk-e\\
            saw  wolf-\textsc{nonnom}\\
    \glt    ‘(she) saw a wolf’
    \ex
    \label{ex:01:4b}
    \gll    onak su videli krevet-a\\
            then  \textsc{aux.3sg}  see.\textsc{aux.3sg}  bed-\textsc{nonnom}\\
    \glt    ‘then (they) saw a bed’
    \z
\z

This non-nominative form may be modelled on various target case endings, and is not necessarily produced consistently or systematically. It does show, however, that learners have at least noticed the morphological variability of the target and are trying to make sense of it.

Similar observations were made concerning the acquisition of L1 Polish. Łuczyński (\citeyear{Łuczyński2002, Łuczyński2004, Łuczyński2010}) shows that paradigms start off with three forms marking three grammatical functions, namely nominative, accusative and vocative. However, because of frequent instances of syncretism, some of these functions are performed by the same form, e.g. \textit{dom}, ‘home’\textsc{[nom/acc]} as opposed to \textit{chłopak}, ‘boy’\textsc{[nom]} vs. \textit{chłopak-a}, ‘boy’-\textsc{acc}. Smoczyńska (\citeyear{Smoczyńska1972, Smoczyńska1985, Smoczyńska1997} observes that the first recognisable noun forms produced by young children are modelled on the nominative case. Later on, a new phase begins, in which words appear in two forms, one of which is modelled on the nominative and the other simply contrasts with it. Further, \citet{Dziubalska-Kołaczyk1997}, following \citet{DresslerKarpf1994}, applies the terms \textit{pre-} and \textit{proto-morphology} to the acquisition of Polish L1. During the first stage, basic morphological operations, such as reduplication, are experimented with by the young learner. Proto-morphology marks the beginning of the morphological system of the language according to the principles of Natural Morphology (\citealt{Dressler1985, Dressler1987, Dressler2011, Wurzel1989, Crocco-Galeas1998}). The phase of morphology proper, finally, entails the full development and completion of the inflectional and derivation systems of the target language.

Concerning the role of word order in the acquisition of inflectional morphology, studies conducted within the framework of Processability Theory (\citealt{Pienemann1998, Pienemann2015, Di-BiaseBettoni2015}) show quite clearly that accusative case marking first emerges in syntactically unmarked SVO sentences, in which the marked, non-nominative marking appears in post-verbal position, as shown by  \citet[190]{ArtoniMagnani2015} \REF{ex:01:5a}. Indeed, this tendency is so strong as to generalise to contexts which do not require accusative marking, like the post-verbal SUBJ in the German L2 examples in \REF{ex:01:5b} \citep[490]{Baten2011} and \REF{ex:01:5c} (\citealt[235]{DiehlEtAl2000}).

\ea%5
    \label{ex:01:5}
    \ea\label{ex:01:5a}
    \gll    vilk-a prinës balerin-u\\
            fork-\textsc{nom}  brought  dancer-\textsc{acc}\\
    \glt    ‘the dancer brought (the) fork’
    \ex\label{ex:01:5b}
    \gll    nicht weit von hier befindet sich den Bahnhof \\
            Not far from here  find    itself  the.\textsc{acc}  station\\
    \glt    ‘the station is not far from here’
    \ex\label{ex:01:5c}
    \gll    es ist ein-en Aprilfisch\\
            it   is   an-\textsc{acc}   April's.fool\\
    \glt    ‘it is an April fool's joke’
    \z
\z

Only at more advanced developmental stages do learners acquire the ability to correctly case-mark the object constituent in syntactically marked structures like OVS, so as to manipulate word order for pragmatic purposes, while at the same time clearly marking syntactic functions. 

These observations highlight a further reason why the OS word order may be considered as the marked, more demanding option compared to its SO counterpart. Building on \citegen{Levelt1989} speech model and lexico-functional grammar \citep{Bresnan2001}, Processability theory postulates that a disalignment between the semantic level (argument structure), the syntactic level (syntactic functions) and the actual order in which the arguments appear (constituent structure) is associated with a higher processing cost. To exemplify, placing in sentence-initial position an argument other than the semantically prominent agent argument, in turn associated to the subject syntactic function, requires a disruption of the default alignment which requires time and practice to be mastered (see \citealt{BettoniDi-Biase2015} for a comprehensive description of this theoretical approach).

\section{Task effects: structured tests vs. semi-spontaneous production}\label{sec:01:2}

While structured tasks undoubtedly provide the researcher with a fully controlled environment to test linguistic hypotheses, it can be argued that they hardly resemble any realistic communicative situation, so that making general claims as to the learner’s linguistic skills on the basis of structured tests alone may not be an unproblematic operation (\citealt[289—290]{Ellis1985}). This criticism is not new, from \citegen{Krashen1981} distinction between acquisition and learning to research conducted within the Learner Variety approach (\citealt{Perdue1993, Perdue1996, Starren2001, Bernini2003, Giacalone-Ramat2003}), which is almost entirely based on spontaneous production data, to Processability Theory, whose claims are programmatically founded on the quantitative analysis of spontaneous speech. 

After studying the learners’ morphosyntactic skills through structured tasks, questions of a more applied nature emerge, such as to what extent input really becomes intake, i.e. is sufficiently acquired and automatized to be ready for use when needed for communication. Is it possible that input is assimilated to a degree sufficient to use the target structure in a given context (such as a structured task), but not others? These questions are pursued in the present work by comparing the learners’ performance on the same target structure in two different contexts, i.e. two structured tasks as opposed to semi-spontaneous production, in which participants are required to talk to each other in Polish in order to solve a practical extra-linguistic task. In addition to lexical and grammatical accuracy, the learner here has to pay attention to discourse structure and to the development of the interaction with the interlocutor.

Such semi-spontaneous production is seen here as a concrete test of morphosyntactic skills previously observed in the controlled, yet artificial, environment of the structured tests. The question one asks at this stage is “given what learners can do in a laboratory context, what will they prove able to do when using language not just to perform an exercise, but to actually communicate a message?”. To this end, the results of the structured tests will be compared to performance in semi-spontaneous production, so as to highlight any systematic discrepancies in morphosyntactic accuracy and perhaps even a threshold in the structured test score which learners have to meet in order to be able to produce inflectional morphology in semi-spontaneous production. 

Comparisons between task types are often encountered in the debate on task effects (\citealt{RévészMichelEtAl2016, PlonskyKim2016, Sasayama2016}) and linguistic-cognitive complexity, defined as the mental resources allocated and cognitive mechanisms deployed in processing and using a given structure (\citealt{HousenSimoens2016}). Although a detailed discussion of these topics lies beyond the scope of this work, it is worthwhile to briefly sketch why some tasks may seem harder than others. 

Skehan and Foster’s (\citealt{Skehan2009}; \citealt{SkehanFoster2001}) Limited Attentional Capacity model advocates that the amount of information (in terms of data and goals) one can keep track of is limited. When performing a task, various components compete with each other for attention. This leads to trade-off effects, as only those processes which are allocated sufficient attention will be performed at the optimal level; performance in all others will inevitably decline. Crucially, if there is an extra-linguistic communicative objective, this receives priority. In other words, learners first aim to express their message in an effective, though not necessarily accurate manner. If enough attentional resources are left, they can be allocated to objectives such as complexity, accuracy and fluency (\citealt{SkehanFoster2007}). 

This view is not shared by another influential approach, namely Robinson’s (\citeyear{Robinson2001, Robinson2005, Robinson2015}) Cognition Hypothesis, whereby trade-off effects do not necessarily occur because different processes may draw from different attentional pools. Decreases in performance only take place when task complexity increases in terms of resource-dispersing factors (as opposed to resource-directing), such as reduced planning time. In contrast, increasing complexity in terms of resource-directing factors may actually result in production which is both more accurate and more complex, if required to reach the communicative goal. Indeed, this is partly the case of the tasks considered in this work, in which the production task requires a much wider range of lexical items and grammatical structures than the structured tests, although, as stated before, some particularly complex structures (like OS transitive sentences) may be avoided because they are either too difficult or simply unnecessary. 

By comparing learner performance in two very different contexts, i.e. structured tests and semi-spontaneous production, the analysis presented in this book aims to approximate a comprehensive view of learner skills as far as NOM/ACC case marking is concerned. The two structured tests make it possible to explore what learners are able to do in comprehension and production in the best possible conditions, i.e. in a laboratory setting and with a task of limited complexity. The semi-spontaneous production component shows what the same learners can do in a realistic, complex communicative situation.

\section{L1 influence}\label{sec:01:3}

It may be hypothesised that speakers of specific L1s may be advantaged in the processing of the target structure. The rationale behind this claim is that the processing of the L2 target structure rests on mechanisms which are similar to those of the L1 and are consequently available to the speaker (\citealt{TokowiczMacWhinney2005, Ellis2006a}). Effectively, it appears that speakers of morphologically complex languages are more at ease when processing target languages characterised by complex morphology. Ellis and Sagarra (\citeyear{EllisSagarra2010a, EllisSagarra2011}) studied the acquisition of temporal reference in Latin after only one hour of input exposure. While they found that focusing learner attention to verbs or adverbs orients their processing strategies towards that category, they also highlighted important L1 effects, such as the fact that speakers of morphologically poor languages such as Chinese and English tended to rely more heavily on lexical cues than did speakers of morphologically more complex L1s, such as Spanish and Russian. When paradigm complexity increased, however, all learners seemed biased towards lexical cues. The same researchers (\citealt{SagarraEllis2013, Sagarra2014}) eye-tracked the processing of Spanish L2 temporal reference by English L1 and Romanian L1 learners, discovering that the intermediate and advanced speakers of the more complex L1 are sensitive to tense incongruencies and tend to rely more heavily on verbs than their English equivalents, who mainly focus on lexical cues such as adverbials.

In the context of the present work, L1 interference is relevant in light of the fact that the speakers who were exposed to the same Polish input and took the same experimental tasks were speakers of five different L1s. Crucially, only German behaves similarly to Polish in terms of the morphological expression of case and word order manipulation for pragmatic purposes: all other languages only encode fragments of case in the pronominal paradigm and tend to adhere to a default SO word order, although other word orders are also possible for pragmatic purposes. If the rationale of the hypothesis is correct, then the German learners should prove faster and more accurate in the processing of Polish morphosyntax. More generally, differences in learner output are expected which can be attributed to an L1 effect.

\section{Input control}\label{sec:01:4}

Studying the output of learners confronted with a completely novel language may be illuminating with regard to the general mechanisms of input processing (\citealt{Perdue2002}), provided that it is possible to have full control over the input and correlate its relevant input parameters with learner output. In the present work, input is defined as any item of the target language that learners are exposed to through any channel.

Everyone roughly agrees that input should play some role in SLA, but opinions start to differ broadly as soon as the debate moves to the way in which input, “what is available to go in” is processed and transformed into intake, “what goes in”, in Corder's \citeyear{Corder1967} words. Following Hulstijn (\citeyear{Hulstijn2015}) and MacWhinney (\citeyear{kail_tale_2010,macwhinney_language_2015}), two main streams of theories may be identified in the prolific literature that has developed around this topic, namely the generative position and a constellation of emergentist approaches.

In the generative framework (see \citealt{RankinUnsworth2016} for a recent review), input is mainly seen as the activator of an innate mechanism \citep{Chomsky1981}. An innate capacity for language must be postulated, since language acquisition cannot be solely based on the input received because the learners’ input is deficient, an argument known as the poverty of the stimulus \citep{Chomsky1980}. The language faculty is thus seen as an innate system, input only providing a few examples which the language processor will take care to shape and systematise through the acquisition process.

In contrast, emergentist approaches maintain that input contains a wealth of information which learners are equipped to analyse using a variety of cognitive processes \citep{Tomasello2005}, including the statistical search for form-function associations.  \citet[1]{Ellis2006b} describes learners as “intuitive statisticians, weighing the likelihoods of interpretations and predicting which constructions are likely in the current context”, while the acquisition process is viewed as “the gathering of information about the relative frequencies of form-function mappings”.

Within the emergentist universe, the Learner Variety approach seems particularly important in the architecture of the present work. Input is seen as a wealth of linguistic material which learners interpret and shape based on their provisional interlanguage grammar. Moreover, the L2 learner is seen as a proficient speaker of at least another language, an expert communicator ready to employ all known strategies to transmit the intended message. In an attempt to do so, linguistic elements may be reinterpreted and assigned meaning which they do not possess in the native variety. To exemplify, \citet{Bernini2018} and \citet[28--33]{Dimroth2018} suggest that in some VILLA data the instrumental word form \textit{strażakiem} contrasts with its nominative equivalent \textit{strażak} in that with some learners it seems to express plurality.

But in the absence of clear, reliable data on the input received, the doubt remains that different acquisition outcomes may simply derive from input that differs in quantity or quality, rather than from the systematic, predictable effects of the various parameters under investigation, whether related to the input or not. Full control, in turn, requires that the target language should be completely unknown to the learner, so that input effects may be teased apart from existing knowledge. But since most of the commonly investigated languages are relatively widespread, it is usually hard to find learners who have never had any exposure to the target language, however minimal. 

One possibility is to employ short samples of “exotic” (\citealt{GullbergEtAl2010, Carroll2012First, Carroll2012Input, Carroll2012Segmentation, CarrollWidjaja2013, Carroll2014}) or artificial languages (\citealt{HulstijnDeKeyser1997, Williams2010}). Both solutions make it possible to perfectly tune the target language to the desired research questions. In addition, since the target language can only be learned during the experiment itself, every learner’s learning experience is necessarily identical. For these reasons, numerous studies have made this choice. However, the language samples considered typically lack the complexity and idiosyncrasies of natural languages, so that the ecological validity of such studies may be questioned (\citealt{Hulstijn1989, Robinson2010}).

The VILLA project was designed to draw a clearer picture of input processing based on the results of the research just referred to. Thanks to its methodology, learner performance can be directly checked against the input received through two fundamental methodological steps. First, the learners were selected in such a way as to make sure that they had no previous knowledge of the target language. The choice of an uncommonly taught language like Polish facilitated their recruitment. This approach ensured that all learners began the acquisition process from the same baseline. Second, input was entirely controlled throughout the experiment. Therefore, it may be argued that one of the general research questions pursued in this book is whether or not a relation may be identified between the features of the input received by learners and their own output. Input control also makes it possible to verify if learners can generalise the patterns contained in the input to target structures which differ in various respects. 

A very natural question concerns the effect of additional exposure to the input. Clearly, such effect is expected to be positive: more precisely, it may also be argued that the closer the learner variety gets to the target language, the more learners will be able to rely on form-based processing, i.e. on grammar, if necessary. Indeed, this expectation seems confirmed in the existing literature. Lower proficiency adults primarily rely on lexical cues and other non-morphological means to express grammatical relations, such as word order in syntax and chronological order and adverbials in temporality (\citealt{Bardovi-Harlig2000, Lee2002, Leeser2004, EllisSagarra2010a}), while higher proficiency learners behave more similarly to native speakers in that they rely on the cues which are most relevant in the particular language learnt (\citealt{Bardovi-Harlig1992b}, \citeyear{Bardovi-Harlig2000}, \citealt{Giacalone-Ramat1992,  SkibaDittmar1992, DietrichEtAl1995, Starren2001, ParodiEtAl2006, BordagPechmann2007}). To verify this claim, most VILLA tasks were repeated several times throughout the course in order to monitor the progress of the interlanguage. This is also the case of the two structured tests considered in the present work, which were repeated twice with a 4:30 hour lag. This makes it possible to verify whether additional, albeit limited, exposure to the input contributed to modify the learners’ strategies of input processing in any way, though presumably towards the target-like morphosyntactic principle.

In addition to experimentally isolating the target variables, input control may directly contribute to explaining learner errors and shedding light on the main issue investigated in this book. The following sections detail two input parameters which appear to be particularly relevant for the study at hand. 

\subsection{Markedness: frequency and form-function association}\label{sec:01:4.1}

The first question is whether or not the statistical distribution in the input of the target endings -[a] NOM and -[e] ACC may favour any of the two endings — said otherwise, with an ambiguous \citep{Haspelmath2006} yet practical terminology, if any of the two terms may be considered as a marked alternative. This topic is pursued in terms of form-function association, with two predictions:

\begin{enumerate}
\item[a)] the form which is most closely associated to the corresponding meaning will be more likely to be over-generalised, if errors indeed occur;
\item[b)] more speculatively, it may be argued that if the function of one of the forms under consideration is not easy to identify, learner preferences may shift towards a principle of utterance organisation in which that form is not required. In the present context, that implies the positional principle, whereby grammatical meaning is independent of inflectional morphology and only relies on the relative position of nouns within the utterance.
\end{enumerate}

Form-function association refers to the strength of the link between a given linguistic meaning and the forms which express it: in other words, how frequently and unambiguously a given form is used to convey a given function, and vice versa. The rationale has been developed in different theoretical frameworks, such as Competition model (\citealt{MacWhinneyBates1987}) and Natural Morphology \citep{Dressler1987}. Research applied to several L1s has shown that the degree to which a given form suggests the corresponding function varies across languages: for instance, the agent function is signalled with the greatest reliability by ut\-ter\-ance-initial position in English, but by subject-verb agreement in Italian \citep{MacWhinneyEtAl1984}. Further, the strength of the form-function association is a powerful predictor of acquisition success and rapidity. \citet{KempeMacWhinney1998} demonstrated that the Russian case system, although much more complex than its German equivalent, is more rapidly acquired because of the more systematic relationship between case endings and grammatical meaning.

The analysis presented in this book attempts to calculate the strength of the association between the two target endings -[a] and -[e] and the meaning they express within the paradigm considered (SUBJ and OBJ, respectively). The relative strength of the form-function association should be a particularly good predictor of what form is selected as the basic word form of the learner variety.

\subsection{Generalisability of input models}\label{sec:01:4.2}

The second question regards the learners’ ability to generalise target structures as they occur in the input to other models of utterances, differing with respect to a few characteristics of the lexical items involved, like animacy, gender etc. This question is of great relevance for the more general debate on the role of input, especially with regard to the generativist and usage-based perspectives. More specifically, if learners only prove able to process the target structure in the same type of utterances encountered in the input, one may consider the acquired construction as a chunk. The analysis of semi-spontaneous production further makes it possible to verify whether or not learners choose to adhere to the input model when given a choice.

For the purposes of a comparison with learner performance, the input is characterised in terms of two factors, namely the type frequency of the SUBJ and OBJ syntactic functions, on the one hand, and the token frequency of transitive structure models defined in terms of word order, gender, animacy and word class.

Type frequency refers to the number of lexical items which occur in a construction. The literature on this topic maintains two positions which may seem mutually exclusive. On the one hand, several researchers argue that learners might benefit from high type frequency, whereby the same construction is instantiated by a greater number of types (\citealt{Bybee1985, Bybee1995, Bybee2000, Bybee2006, GoldbergCasenhiserEtAl2004, OnnisEtAl2008}), especially as far establishing abstract patterns is concerned (\citealt{McdonoughKim2009}). Type frequency ensures productivity, as hearing several different lexical items in a certain context makes it less likely that that construction may become specifically associated with any of them. Further, if a construction is instantiated by many items, it is probably quite general in meaning and easily generalisable to other items. Conversely, \citet{KruschkeBlair2000} argue that learning that a particular stimulus is associated with a particular outcome hinders the association of the same outcome with another stimulus at a later time, a phenomenon knows as \textsc{blocking} \citep{Ellis2006a}. Finally, high type frequency ensures frequent use in speech (\citealt{BybeeThompson2000}).

Other researchers claim that highly skewed distributions may be just as beneficial. In a skewed distribution, the vast majority of the occurrences of a given grammatical constructions is instantiated by a small number of lexical items. The rationale is that construction learning is a process of categorisation (\citealt{GoldbergCasenhiserEtAl2007}), by which the learner — either child or adult — begins to recognise a similarity of meaning from an identical structure, albeit instantiated by different lexical items. Studies on non-linguistic categorisation have shown that learners are indeed facilitated in the construction of categories by low-variance input (\citealt{GentnerEtAl2007, Casasola2005}). The same is true for language (\citealt{CasenhiserGoldberg2005, MaguireEtAl2008}), with the additional difficulty that linguistic constructions are by nature abstract (\citealt{GentnerMedina1998}). 

A typical example of skewed distribution which is commonly encountered in language is \citeauthor{Zipf1935}'s \citeyear{Zipf1935} law, whereby the frequency of a given word is inversely proportional to its rank in a frequency table. As a result, a small number of very common words account for a substantial proportion of all tokens in a text (\citealt{MintzNewportEtAl2002}). There may be various reasons for this, usually linked with the semantics of the words involved (\citealt{KiddEtAl2006, Thompson2002, EllisFerreira—Junior2009}).

The comparison of learner performance under different input skewedness conditions shows that its beneficial effects are not completely clear (\citealt{BorovskyElman2006, CasenhiserGoldberg2005}). In most cases the results do not point to a single, univocal predictor of acquisition success, but rather suggest that all the parameters considered jointly drive acquisition (\citealt{YearGordon2009, WulffEtAl2009}).

In the present book, the analysis of type frequency is designed to quantify the extent to which target structures encountered in the input are associated with particular lexical items. In the case where structures are strongly associated with a limited number of items in the input, one can assume that applying a construction to a different set of lexical items in the structured tests will require a degree of abstraction and generalisation. 

With the same aim, the input analysis presented in the book also consists in searching for the most common models of transitive utterances in terms of word order, noun animacy and noun gender. The purpose of this step is to compute the number of input examples corresponding to the test target structures to which the learners were exposed throughout the course. 

The comparison of a prototypical input transitive sentence \REF{ex:01:6a} with a target sentence of the Elicited Imitation task \REF{ex:01:6b} highlights the fact that although the target structure is arguably the same, at least from a morphosyntactic point of view (the expression of SUBJ and OBJ through -[a] NOM and -[e] ACC), the two utterances differ in several respects. In addition to word order (SVO vs OVS), which is a variable controlled for experimentally, notable differences exist in terms of semantics, whereby the OBJ function is instantiated by an inanimate noun in \REF{ex:01:6a} but by a common nationality noun in \REF{ex:01:6b}; by the same token, the SUBJ is a person name in \REF{ex:01:6a} and again a common noun in \REF{ex:01:6b}. The verbs also differ in terms of argument structure, so that \textit{ciągnie} ‘pulls’, but not \textit{lubi} ‘likes’, may be considered a prototypical transitive verb. This in turn is defined here as a verb in which the syntactic subject performs the semantic role of agent. Thus, \textit{lubi} ‘likes’ is clearly transitive from a syntactic point of view, because it requires a subject (marked as NOM) and a direct object (marked as ACC); however, it is not prototypically transitive because the syntactic functions SUBJ and OBJ do not correspond to the semantic roles \textsc{agent} and \textsc{patient}.

\ea%6
    \label{ex:01:6}
    \ea\label{ex:01:6a}
    \gll    Mari-a lubi kaw-ę.\\
            Maria-\textsc{nom}  likes  coffee-\textsc{acc}\\
    \glt    ‘Maria likes coffee.’
    \ex\label{ex:01:6b}
    \gll    Portugalk-ę ciągnie dziewczynk-a.\\
            Portuguese.woman-\textsc{acc}   pulls     little.girl-\textsc{nom}\\
    \glt    ‘The little girl pulls the Portuguese woman.’
    \z
\z

The research question thus asks whether or not learners can identify the morphosyntactic structure of interest in the input and apply it to somewhat different sentence models, in which semantics is of no help to the expression of grammatical meaning.

\section{A note on labels and notation}\label{sec:01:5}

Polish examples in this book are normally transcribed in standard Polish orthography if they were uttered by a native speaker or if they are used within a theoretical argumentation. A guide to reading Polish orthography is provided in the Appendix. 

A broad IPA transcription (\citealt{LandauEtAl1999, Jassem2003}) is used to transcribe utterances produced by learners, in order to avoid any undue morphosyntactic interpretation of the raw data (\citealt{Saturno2015La}). The rationale for this decision is as follows. Because of the rich inflectional system of Polish, grammatical meaning is often indicated by a single word-final sound \REF{ex:01:7a}, potentially in addition to a stress shift \REF{ex:01:7b} due to the substitution of a zero morph with a vocalic ending (the lexical stress of virtually all Polish words falls on the penultimate syllable).

\ea%7
    \label{ex:01:7}
    \ea\label{ex:01:7a}
    \textit{studentk-a} ‘female student-\textsc{nom.sg’} vs. \textit{studentk-ę} ‘female student-\textsc{acc.sg’}
    \ex\label{ex:01:7b}
    \textit{strażak-${\emptyset}$} ‘fireman-\textsc{nom.sg’} vs. \textit{strażak-a} ‘fireman-\textsc{gen/acc.sg’}
    \z
\z

On the one hand, learner varieties typically exhibit very conspicuous phonological deviations from their native target. On the other hand, phonological variability may occur in the earlies stages, albeit with no functional value: “There is no inflection in the B[asic] V[ariety] […]. Thus, lexical items typically occur in one invariant form. […] Occasionally, a word shows up in more than one form, but this (rare) variation does not seem to have any functional value: the learners simply try different phonological variants” (\citealt[311]{KleinPerdue1997}). \citet[160—161]{BroederEtAl1993} hypothesise that “random variation on the phonetic and phonological level at the first stages of second language acquisition is gradually replaced by variation produced by the acquisition of proper morphological rules”, especially with regard to verbs, the most highly inflected word class in the target languages of the ESF project \citep{Perdue1993}. Indeed, the VILLA production data too contain examples which suggest a productive, systematic use of phonological variation to express grammatical meaning. Because of their fluid state, though, it is often problematic to distinguish phonological variability from contrasts reflecting an opposition in meaning (\citealt{Bernini2018Fonetica, Bernini2018Pattern, Dimroth2018}). It is argued that a phonetic transcription of learner output leaves the question open for analysis, without imposing an \textit{a} \textit{priori} interpretation which may later condition the discussion of the results.

A few words should be spent to clarify what labels will be used in order to refer to the functions that nouns may perform in an utterance. Within this study, a prototypical transitive sentence is composed of two noun phrases (NP) in utterance-initial (NP1) and utterance-final (NP2) position, as well as a bivalent verb in utterance-medial position. It can be argued that the two NPs tend to concentrate three functions belonging to the layers of information structure, syntax and semantics.

First, in unmarked transitive structures the NP1 is part of the topic, i.e. what is talked about in the utterance, while NP2 is part of the comment, i.e. what is said about it. Further, Since Polish is a predominantly SO language, NP1 is most often the grammatical subject of the sentence, identified by noun-verb agreement. It follows that NP2 must be the grammatical object.

From the semantic point of view, NP1 is usually characterised by a higher degree of agency, which in terms of semantic roles corresponds to agent or experiencer (\tabref{tab:01:1}). It is no coincidence that more often than not the referent of NP1 is animate; for the same reason, NP2 is typically inanimate (\chapref{sec:3}).

\begin{table}
    \begin{tabular}{lll}
    \lsptoprule
        Mari-a & pije & herbat-ę\\
        Maria-\textsc{nom} & drink.\textsc{3sg} & tea-\textsc{acc}\\
        \cmidrule(lr){1-1}\cmidrule(lr){2-3}
        TOP & \multicolumn{2}{c}{ COMMENT}\\
        \cmidrule(lr){1-1}\cmidrule(lr){3-3}
        SUBJ &  &  OBJ\\
        \cmidrule(lr){1-1}\cmidrule(lr){3-3}
        A &  &  P\\
    \lspbottomrule
    \end{tabular}
    \caption{alignment of information structure, syntax and semantics}
    \label{tab:01:1}
\end{table}

\newpage
If the whole VILLA input were composed of prototypical transitive sentences, the three labels would appear to be interchangeable. That is not the case, however: although N1 and NP2 tend to be characterised by a coincidence of functions in terms of information structure, syntax and semantics, there may be occasions in which that arrangement is disrupted. It is therefore desirable to identify the label which — independently of the position of the NPs relative to each other — illustrates best the role of the corresponding referents in the situation described by the sentence.

Clearly TOP does not suit this purpose because it is rigidly linked to the ut\-ter\-ance-initial position. In marked word orders such as OS, NP1 is still the TOPIC of the sentence but it encodes a syntactic function other than SUBJ, such as OBJ (\tabref{tab:01:2}).

\begin{table}
    \begin{tabular}{lll}
    \lsptoprule
        herbat-ę & pije & Mari-a \\
        tea-\textsc{acc} & drink.\textsc{3sg} & Maria-\textsc{nom}\\
        \cmidrule(lr){1-1}\cmidrule(lr){2-3}
        TOP & \multicolumn{2}{c}{COMMENT}\\
        \cmidrule(lr){1-1}\cmidrule(lr){3-3}
        OBJ &  & SUBJ\\
        \cmidrule(lr){1-1}\cmidrule(lr){3-3}
        P &  & A\\
    \lspbottomrule
    \end{tabular}
    \caption{Disalignment of information structure, syntax and semantics}
    \label{tab:01:2}
\end{table}

Semantic roles may appear to be more intuitive to the linguistically untrained learner. The notion of subject after all rests on noun-verb agreement, which is a meta-linguistic concept. Semantic roles, on the other hand, may be thought to more faithfully reflect the role of arguments (i.e. referents) in a given situation, which does not seem to imply any meta-linguistic reasoning. However, this is only true with respect to prototypically transitive verbs, i.e. verbs whose first argument can be identified as the agent. In other cases, it may be difficult to clearly identify a true agent and a true patient. Indeed, this case is fairly common in the VILLA input, some representative examples of which are presented in \REF{ex:01:8}. The first argument of these verbs is better described as experiencer (\ref{ex:01:8a} and \ref{ex:01:8b}) or possessor, although syntactically (i.e. based on noun-verb agreement) it is clearly the subject of the utterance.

\ea%8
    \label{ex:01:8}
    \ea\label{ex:01:8a}
    \gll    student-${\emptyset}$   zna     język-${\emptyset}$   polsk-i\\
            student-\textsc{nom}  know.\textsc{3sg}  language-\textsc{acc}  Polish-\textsc{acc}\\
    \glt    ‘the student speaks Polish’
    \ex\label{ex:01:8b}
    \gll    siostr-a lubi brat-a\\
            sister-\textsc{nom}  like.\textsc{3sg}  brother-\textsc{acc}\\
    \glt    ‘the sister likes the brother’
    \ex\label{ex:01:8c}
    \gll    dziewczynk-a ma balonik-${\emptyset}$\\
            little.girl-\textsc{nom}  have.\textsc{3sg}  balloon-\textsc{acc}\\
    \glt    ‘the little girl has a baloon’
    \z
\z

Similar considerations apply to the labels controller and controllee, often employed in the literature related to the Learner Variety approach (\citealt{KleinPerdue1992, Perdue1993}) to refer to the “argument of a verb by the greater or lesser degree of control that its referent exerts, or intends to exert, over the referents of the other argument(s)” (\citealt[314]{KleinPerdue1997}).

To summarise, a VILLA transitive structure may always be described with reference to the syntactic functions SUBJ and OBJ, whereas A and P are not always appropriate because the two syntactic functions (especially SUBJ) may correspond to more than one semantic role. Moreover, SUBJ and OBJ suggest the role that the corresponding constituent would presumably fulfil in the target language, whose principles of utterance organisation are of a syntactic nature. On the one hand, this argument recalls \citegen{Bley-Vroman1983} comparative fallacy, whereby learner output is interpreted in light of the target model, rather than in its own right: in this respect, the comparative fallacy is frowned upon in SLA studies, because it obscures the internal structure of the interlanguage and introduces a clearly evaluative (rather than descriptive or interpretative) approach to L2 data. On the other hand, at times the use of the labels SUBJ and OBJ in the description of interlanguage output may represent a useful terminological shortcut to indicate the meaning which the learners would presumably express if they mastered the target grammar sufficiently. Although learner output may be organised around functions which do not play the same role in native varieties (such as controller or topic, as argued above), for practical purposes it may be useful to refer to their intended function in the target language, which (like the learners’ L1s) is based on syntactic categories.

It may be anticipated that the target sentences of the structured tasks described in chapters 4 and 5 do contain prototypically transitive verbs such as 'push', 'pull', 'call', 'cheer', so that the argument stated above may appear not to be particularly influential. This is not the case in the semi-spontaneous interaction described in \chapref{sec:6}, in which participants were free to use the whole range of known lexical items, which indeed includes non-prototypical transitive verbs like 'love', 'have', 'know' etc.

For these reasons the labels SUBJ and OBJ will be used throughout this study. 
