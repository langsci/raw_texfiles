\chapter{Introduction} \label{chInt}

\ea 
\ea You need a permit to climb Mount Everest. \label{ex11a}
\ex They don't permit sailing on this lake. \label{ex11b}
\z
\z

Consider the word \textit{permit} when reading out loud \ref{ex11a} and \ref{ex11b}. It becomes clear that the two versions of that same word sound different. The sole distinction between the two is generally considered as a difference in word stress (henceforth also `stress'). Word stress is commonly understood as a single, most prominent syllable in a word that may be acoustically marked (e.g. \citealt{hyman_word-prosodic_2006}; \citealt{ladd_intonational_2008}; \citealt{gordon_acoustic_2017}). In the example above either the first syllable /pɜr/ (\ref{ex11a}) or the second syllable /mɪt/ (\ref{ex11b}) is the most prominent one, signalling a noun or a verb respectively.

For illustrative purposes, the example above is taken from English, simplified to show the most essential aspects of word stress. These are important observations to understand the motivation for the research conducted in this book, as they reflect a number of issues regarding our current understanding of word stress.

It is safe to assume that the majority of stress research to date has been conducted on English and other (related) well-studied languages. This is problematic because it has shaped the linguistic definition and understanding of word stress in ways that are not representative of a vast number of other languages. The problem of generalizing scientific theory based on predominantly Western data, is a problem found in other scientific disciplines as well (\citealt{henrich_weirdest_2010}). Indeed, some aspects of word stress in English appear exceptional when compared to other languages (\citealt{cutler_native_2012}).

Scientific work on stress research can be divided according to a number of subtopics, each representing a different aspect of word stress. Phonetic research has mainly focused on revealing which acoustic features make the stressed syllable prominent (e.g. \citealt{gordon_acoustic_2017}). Language description, as often found in grammars, generally analyses word stress on the basis of its distribution; i.e. the possible positions in a word to which the stressed syllable can be assigned. These descriptions aim to reveal which rules predict the location of the stressed syllable, i.e. its phonology (e.g. \citealt{hyman_word-prosodic_2006}) and its role in prosodic structure (e.g. \citealt{gordon_disentangling_2014}). For a small number of languages the communicative functions of word stress were investigated, i.e. how listeners perceive stress (e.g. \citealt{peperkamp_typological_2002}) and how it might help them to process the incoming speech signal (e.g. \citealt{cutler_lexical_2005}).

Thus, languages for which stress has been studied on all of the above mentioned aspects are still few and generally biased towards West-Germanic languages such as English, Dutch or German. The aim of this book is to contribute to stress research by including a typologically more distinct and less researched language; Papuan Malay. In addition, this book provides experimental investigations of most word stress aspects that have previously only been studied in well-documented languages. In this way, this book provides a solid basis for new typological comparisons that refine our understanding of word stress.

The remainder of this introduction provides an overview of stress research and is categorised according to the core aspects of word stress (Section \ref{secCoreaspects}). A discussion of why Papuan Malay is particularly relevant to study in this context is given in Section \ref{secPM}. Thereafter, the research questions addressed in the remaining four chapters are provided (Section \ref{secRQ}).

\section{Core aspects in research on word stress} \label{secCoreaspects}

This section discusses several aspects that have all been reported as relevant for the manifestation of word stress cross-linguistically. These aspects are often studied in different research (sub-)disciplines. They are categorised depending on whether they are relevant for the acoustic realization of word stress (Section \ref{sec111}), determining the position of the stressed syllable (Section \ref{sec112}), the role of word stress in the prosodic structure (Section \ref{sec113}) or the communicative function of word stress (Section \ref{sec114}).

\subsection{Acoustics} \label{sec111}
When a language has word stress it often has one or more acoustically measurable correlate(s) that make stressed syllables more prominent than unstressed syllables. In a survey of 75 languages, \citet{gordon_acoustic_2017} investigated which acoustic correlates turned out to be most revealing for word stress. The results showed that duration is the most often measured as well as the aspect most often reported to correlate with word stress. That is, stressed syllables were often reported to be longer than unstressed syllables. The lengthening due to word stress mostly affected syllable rhymes or entire syllables, depending on the language. Measures of F0 were reported as a second major correlate, followed by intensity, formants and spectral tilt. It was noted that F0 is a particularly confounding correlate as it has been shown to (also) reflect phrase level prosody in a number of languages (\citealt{gordon_acoustic_2017}). Interestingly, while in most of the investigated languages F0 was higher in the stressed syllable, some languages showed lower F0 in stressed syllables than in unstressed syllables (e.g. Lahore Urdu: \citealt{hussain_phonetic_2007}, and Italian: \citealt{eriksson_acoustics_2016}). As for intensity, few studies have shown overall intensity to correlate with word stress. Measures that take the intensity of low versus high frequencies in the speech spectrum into account (e.g. spectral tilt), however, were more accurate. That is, the natural reduction of intensity towards higher frequencies is steeper in unstressed syllables than in stressed syllables (e.g. Dutch; \citealt{sluijter_spectral_1996}). In other stress languages, spectral tilt did not correlate with word stress (e.g. Peninsular Spanish: \citealt{ortega-llebaria_acoustic_2011}). Formants, in particular F1 and F2, correlate with word stress as well. Commonly observed is the effect that vowels occupy a more peripheral position in the acoustic space when stressed (\citealt{crosswhite_vowel_2004}). This form of decentralization has been explained from the perspective of enhancing prominence and can be observed in languages where corner vowels (/i/, /u/ and /a/) are likely candidates for word stress (e.g. in Bulgarian). An opposing, vowel centralizing effect of word stress has been observed in some languages and has been explained in terms of contrast enhancement, where the articulatory goal is not to increase prominence, but rather to be maximally different from other vowels (e.g. in Belarusian). In the latter case, corner vowels are less likely candidates for stress (\citealt{crosswhite_vowel_2004}). Crucially, none of the above mentioned acoustic correlates can serve as an exclusive diagnostic for word stress. In particular, correlates of phrase level prosody can overlap with word stress cues and many studies have not explicitly separated the word level from the phrase level in their description of stress (\citealt{roettger_methodological_2017}). This problem exists in particular for languages in which phrase level accents align with stressed syllables at the word level. In these languages, e.g. West-Germanic ones, it is particularly challenging to disentangle the two prosodic levels, although research has shown that their acoustic cues are distinct (e.g. \citealt{vanheuven_acoustic_2018} on `sentence stress'). It can therefore be difficult to group languages based on which acoustic cues signal word stress. Nevertheless, some generalizations can be made.

Several studies have observed that in tone languages, F0 is not or only weakly available as a correlate of word stress (\citealt{gordon_acoustic_2017}). Thus, the set of acoustic correlates in a particular language depends on other functions these correlates might have in the prosodic structure. The role of word stress in prosodic structure is further discussed in Section \ref{sec113}. In addition, it has been shown that stress is weakly realised acoustically in languages with a highly predictable stress pattern (\citealt{dogil_phonetic_1999}). In other words, the less the position of word stress varies, the less the communicative need for large acoustic differences between stressed and unstressed syllables. The next section provides a more detailed discussion of various factors affecting the position of word stress across languages.

\subsection{Distribution} \label{sec112}
When it comes to the position of the stressed syllable in a word, a general distinction has been made between languages in which the position of the stressed syllable is fixed, e.g. on the final syllable of a word, and languages in which the position of the stressed syllable can vary. In the latter type of language, stress is labeled ``variable'' or ``free'' as its position is not defined relative to word boundaries, but rather depends on the (morpho-)phonological properties of the syllable. It is more useful to interpret the fixed versus free/variable stress languages as a continuum, rather than two distinct groups. It has been shown, for example, that languages can shift over time between one and the other (\citealt{baerman_evolution_1999}). Where a language is placed on the fixed-variable continuum depends on a number of factors that determine the position of stress. These factors, as discussed in the remainder of this subsection, define the relationship between stress as a suprasegmental notion on the one hand, and the actual consonantal and vocalic makeup of the segments in the syllable structure on the other hand.

For example, the quantity of a syllable (i.e. weight: \citealt{hyman_theory_1985}) can play a role such that heavy syllables are more likely to be stressed than light syllables. Which syllable structures count as heavy or light differs per language, but the distinction always depends on their segmental makeup. To capture the fine-grained weight distinctions, phonological theories assume the mora on an intermediate level between the segments and the syllable (\citealt{mccawley_phonological_1968}). Furthermore, much of the stress positioning across languages can be explained by the sonority of the segments in the syllable. Vowels are assumed to form the most sonorous class of speech sounds with potentially relevant sonority differences among them as well (\citealt{parker_quantifying_2002}). Vowel quality as such can determine the position of stress, as has been shown for Kobon (\citealt{kenstowicz_quality-sensitive_1997}). Although it is challenging to find accurate acoustic correlates of sonority (\citealt{parker_quantifying_2002}; \citealt{albert_modeling_2022}), it should be noted that sonorous segments are particularly rich in acoustic properties that could correlate with stress. For example, F0 or spectral tilt are more meaningful correlates for vowels than for consonants. Articulatorily, it is easier to increase or decrease the duration or intensity of sonorous speech sounds than of obstruents. This makes sonorous segments the preferred loci for suprasegmental contrasts such as the difference between stressed and unstressed syllables. The close relationship between segments and the position of word stress can be a challenge to a strict suprasegmental interpretation of word stress, which is discussed in more detail in the next section with examples from English and Dutch.

In addition, directionality influences the way stress is assigned to syllables. That is, the location of the stressed syllable is either counted from the left or the right edge of the word (\citealt{howard_directional_1972}). An example of the former type can be found in Latvian (\citealt{halle_essay_1987}), whereas the latter, cross-linguistically more common type can be found in Armenian (\citealt{vaux_phonology_1998}). It has also been shown that the position of the stressed syllable can be ``bounded'' in that it falls within a certain number of syllables in the word (e.g. in Kobon: \citealt{kenstowicz_quality-sensitive_1997}. Other languages, however, are ``unbounded'' in that there is no such restriction (e.g. Hindi: \citealt{bailey_nonmetrical_1995}).
Note that the predictability of the stress position differs per language and affects the perception of word stress (\citealt{peperkamp_perception_2010}). That is, for some languages (morpho-) phonological rules are more successful in predicting the position of stressed syllables (e.g. Finnish), than for others (e.g. Spanish). Predictability in this sense refers to how advantageous the ruleset is for listeners to process stress patterns. This notion is crucially different from the fixed-free continuum, in that high predictability does not necessarily mean that a language has a small number of exceptions to the standard stress pattern. It has been shown that in a language with more unpredictable exceptions to the default stress placement, listeners perform better when asked to recall sequences of nonsense syllables with a given pattern of stressed and unstressed syllables (\citealt{peperkamp_perception_2010}). Thus, listeners appear to memorise stress patterns better when these patterns are more challenging to capture by rules.


\subsection{Prosodic structure} \label{sec113}
Another aspect according to which stress languages differ is the role of word stress within larger linguistic structures. We can distinguish work on the role of stress in several stages of speech production (e.g. \citealt{levelt_speaking_1989}) from work on the role of stress on different phonological levels (e.g. \citealt{nespor_prosodic_2007}). As for the former, the aim of investigation is where in the speech production process word stress information is available. A distinction has been made between languages in which stress is an inherent property of words in the lexicon and languages in which stress information is not stored in the lexicon. In the former language type, word stress needs to be learned by heart and plays a more important role in distinguishing between word meanings. Word stress in this type of language is termed ``phonemic'' when it can be the only distinction between two otherwise identical words. This is the case in English, where word class can be indicated by stress on the first syllable (noun) or on the second syllable (verb) in words such as \textit{permit}. \par 

Note that the primary acoustic difference between stressed and unstressed syllables in English concerns vowel quality, whereas other cues are present in a rather marginal way (\citealt{fear_strong_1995}). This indicates that word stress is not strictly suprasegmental, as it can be a property of the segments. Studies have therefore referred to ``strong'' versus ``weak'' syllables instead of ``stressed'' versus ``unstressed'' syllables (see \citealt{cutler_prosody_1997} for an overview). However, in a closely related language such as Dutch, vowel quality is, among other cues, a suprasegmental property of word stress (\citealt{sluijter_spectral_1996}). To illustrate the difference between Dutch and English stress; Dutch listeners are better at recognizing stressed syllables in English than native listeners of English, because the former are sensitive to more acoustic cues than the latter (\citealt{cutler_dutch_2007}). In Dutch, stress has also been described as ``lexical'', thus referring to the fact that there are minimal stress pairs (e.g. /ˈka:n\symbol{"0254}n/ and /ka:ˈn\symbol{"0254}n/, translating as `canon' in the musical and military sense, respectively). Given the differences between Dutch and English just described, languages thus differ in the extent to which word stress is phonemic and lexically stored. Although this aspect relates to the mental representation of stress, it clearly parallels the mobility distinction in terms of ``free'' and ``fixed'' (see Section \ref{sec112}). That is, in languages where the position of word stress is generally fixed (e.g. Polish; \citealt{jassem_akcent_1962}), the need to store stress information lexically is limited. \par
When it comes to other levels in the prosodic structure, stress has been shown to interact with phrase-level prosody in a number of languages. This has been a recurring problem in the search for an accurate set of acoustic correlates of word stress (\citealt{roettger_methodological_2017}) and for the phonological definitions of ``stress'' and ``accent'' (\citealt{beckman_stress_1986}; \citealt{hyman_word-prosodic_2006}). Research on Germanic languages has been dominant in both revealing and dealing with this problem. This is most likely the result of a prosodic feature found in these languages: phrase-level prominences in terms of pitch accents are structurally aligned with stressed syllables at the word level (e.g. \citealt{ladd_intonational_2008}). In other languages, however, the prosody of words and phrases are related in a different way or not related at all. For example, acoustic analyses of Kuot (\citealt{lindstrom_aspects_2005}) have shown a strict separation of the word-and phrase-prosodic level, both in terms of acoustic cues and in terms of functionality. It appeared that in Kuot, duration is exclusively correlated with word stress, whereas pitch is used to signal phrase boundaries. Crucially, there is no overlap between these cues or functions in this language. Furthermore, research has not been able to find reliable acoustic correlates of word stress in Seoul Korean, let alone a predictable position of stressed syllables. While early work analyzed Seoul Korean as a stress language, later work showed that the phrase level rather than the word level is relevant for predicting prominent syllables (\citealt{jun_phonetics_1996}). Recent accounts, therefore, separate word prosody explicitly from phrase prosody in order to draw more accurate distinctions between languages (e.g. \citealt{gordon_disentangling_2014}). It has been proposed, for example, that word-level prominences in many languages might actually reflect phrase-level prominences. That is, languages sometimes avoid the co-occurrence of phrasal boundary tones and phrase accents on the same single syllable (i.e. tonal crowding) by means of edge repulsion. Note that other languages add vocalic material to achieve the same goal (e.g. \citealt{grice_word_2018}). In this case, pre-final instead of final syllables in the phrase are marked by F0 movements that constitute the phrase accent. \citet{gordon_disentangling_2014} observes a gap in the typology of word and phrase prosody in that there seem to be no languages with edge repulsion at the word level and not at the phrase level. Languages have edge repulsion either at both levels (e.g. Egyptian Arabic; \citealt{hellmuth_relationship_2007}), only at the phrase level (e.g. Chickasaw; \citealt{gordon_intonational_2005}), or at neither level regardless of whether the language makes use of pitch accents (e.g. Hebrew; \citealt{becker_hebrew_2003}) or not (e.g. Wolof; \citealt{rialland_intonational_2001}). Thus, there is no typological evidence for languages where stress shows edge repulsion and phrase level accents do not, as would be the case for languages with penultimate stress where phrase level accents are realised on the final syllable. This observation is taken as an indication that word stress is actually a reflection of phrase prosody in languages of the world. Note that this analysis is centered around the finding that F0 is a correlate of phrase prosody rather than word prosody (e.g. in Dutch: \citealt{sluijter_spectral_1996} and in Chickasaw: \citealt{gordon_intonational_2005}).

\subsection{Communicative functions} \label{sec114}
The final aspects of word stress discussed here are its communicate functions. Most of these functions relate to the processing and recognition of words, while others concern the contribution of word stress to the rhythm of a language. This distinction thus reflects the extent to which word stress is functional only for the word domain or (also) beyond. One function that has often been attributed to word stress is its ability to discriminate between two otherwise identical words, as referred to in the discussion of phonemic stress above. However, it appears that the number of minimal stress pairs in languages is marginal (\citealt{cutler_lexical_2005}). Thus, minimal stress pairs mainly provide interesting opportunities for experimentation, instead of reflecting a core function of word stress.

Psycholinguistic studies make a distinction between word stress as a cue to word segmentation and word stress as a cue to word identification (\citealt{cutler_prosody_1997}). In the former, listeners use stress cues to detect word boundaries, whereas in the latter listeners use stress cues to detect the target word among possible competitors. As for word boundaries, studies on induced misperceptions have shown that listeners expect a word boundary before a stressed syllable, a perceptual strategy that explains much of the stress distributions in Dutch (\citealt{vroomen_cues_1996}) and English (\citealt{cutler_rhythmic_1992}). Note that languages in which the position of the stressed syllable varies less than in Dutch or English, stress cues could be assumed to be more helpful to listeners for word segmentation (i.e. fixed stress languages). However, the acoustic differences between stressed and unstressed syllables are small in languages where stress is rather fixed (see \citealt[p.73]{cutler_lexical_2005} for a discussion). It therefore remains to be seen to what extent word stress cues are functional in this type of languages. The latter question is a challenge, given that listeners are less sensitive to acoustic cues, when stress patterns in their language are more predictable (\citealt{peperkamp_perception_2010}). For example, ERP data showed that Turkish listeners were hardly sensitive to the default (fixed) ultimate stress, whereas deviations were readily detected (\citealt{domahs_processing_2013}). A study on Polish revealed similar results in that listeners mainly showed an ERP effect associated with anomaly detection (P300) for the non-default stress pattern rather than for the default one (\citealt{domahs_stress_2012}). These results corroborate the idea that the number of exceptions to the default position of word stress is of crucial importance to determine the functionality of stress cues for listeners (\citealt{peperkamp_perception_2010}; \citealt{cutler_lexical_2005}). When it comes to actual word identification, studies on Dutch and English have shown that word stress has a facilitating effect in tasks without time-pressure (see \citealt{cutler_prosody_1997} for an overview). However, in online (time-pressured) processing, listeners have more difficulty selecting target words among competitors that only differ in the position of word stress. Strong-weak patterns were recognised consistently faster than weak-strong patterns, which seemed to suggest a stress related facilitation effect (\citealt{cutler_use_1984}). This could have been interpreted as an indication that word recognition starts as soon as the stressed syllable is heard; hence shorter reaction times for strong-weak patterns than for weak-strong patterns. However, the reaction time differences turned out to be artifacts of the stimulus duration. When stimulus duration was corrected for, both patterns led to similar reaction times (\citealt{cutler_use_1984}).

Word stress also contributes to the perceived rhythm of a language. Studies have attempted to find acoustic correlates of isochrony, the idea that speech rhythm is achieved by intervals of equal length between either stresses or syllables (\citealt{pike_intonation_1945}; \citealt{abercrombie_elements_1967}). No acoustic support has been found for the dichotomous distinction between stress-timed and syllable-timed languages (e.g. \citealt{dauer_stress-timing_1983}; \citealt{arvaniti_usefulness_2012}), although languages appear to differ on a continuum when other rhythm metrics are taken into account (e.g. \citealt{grabe_durational_2002}; \citealt{ramus_correlates_2000}; \citealt{prieto_phonotactic_2012}). As the search for strict isochrony in speech production has been largely abandoned, the view that speech \textit{perception} relies on equal intervals of some sort has become dominant (e.g. \citealt{lehiste_isochrony_1977}; \citealt{ramus_correlates_2000}). It is furthermore clear that syllable structure, vowel reduction and stress all contribute to speech rhythm (\citealt{dauer_stress-timing_1983}). It has also been shown that phrase-level phenomena such as final lengthening and accentual lengthening have consequences for speech rhythm metrics (\citealt{prieto_phonotactic_2012}). How the contribution of word stress to perceived rhythm can best be quantified, remains an open question (see \citealt{jun_prosodic_2014} for a proposal on \textit{macro-rhythm}).
In sum, most of the functions of word stress have been demonstrated for languages with variable positions of word stress (e.g. English and Dutch). Although there is evidence that the predictability of the stress pattern influences its cognitive processing (\citealt{peperkamp_perception_2010}; \citealt{domahs_stress_2012}; \citealt{domahs_processing_2013}) and helps to segment incoming speech into words (\citealt{cutler_rhythmic_1992}; \citealt{vroomen_cues_1996}), more research on other languages with predictable stress patterns is needed (\citealt{cutler_native_2012}).

\section{Papuan Malay} \label{secPM}

Particularly interesting to our understanding of word stress are Indonesian languages. It has been a matter of debate whether Indonesian is a stress language (e.g. \citealt{ode_perception_1994}) and studies have pointed out that this question is too generic. That is, the specific language variety needs to be taken into account to answer this question (e.g. \citealt{goedemans_stress_2007}). Although more recent research has addressed this, there are still many aspects uncovered. Before the studies presented in this book were undertaken, no Indonesian variety had been studied systematically and comprehensively for its existence of word stress. Javanese, Toba Batak, Ambonese Malay, Manado Malay, Betawi Malay, Besemah and some other varieties all received some attention in stress studies. However, none of them have been studied for acoustic, perceptual and functional aspects together.

Before the studies in this book, Papuan Malay had been studied extensively in a grammar (\citealt{kluge_grammar_2017}) and preliminary work on phrase-level prosody had been carried out (\citealt{riesberg_perception_2018}). Nevertheless, word prosody had not been studied empirically, despite the claim that this language has regular penultimate stress (\citealt{kluge_grammar_2017}). The language is spoken in the Indonesian provinces Papua and West Papua by approximately one million speakers. More relevant background information on this language is given in the respective studies. The reader is referred to the most recent grammar (\citealt{kluge_grammar_2017}) for an extensive historical and sociolinguistic overview of Papuan Malay. The availability of native speakers makes this language suitable for extensive fieldwork and experimentation, which elicited (semi-)spontaneous speech (e.g. \citealt{riesberg_dobes_2012}, \citealt{kluge_papuan_2014}). The work conducted for this book would not have been possible without the existing documentation just mentioned. This applies in particular to the clearly defined and testable stress hypothesis, which led to the studies outlined here (\citealt{kluge_grammar_2017}). Thus, Papuan Malay provides an opportunity to study word stress from different angles, combining new as well as (modern versions of) existing methodologies. The reader is referred to \citet{kaland_demarcating_2020} and \citet{kaland_red_2023} for studies on phrase-level aspects, and to \citet{kaland_repetition_2018} and to \citet{kaland_role_2022} for other word-level aspects of Papuan Malay prosody.

\section{Summary and research aims} \label{secRQ}
The literature overview above is far from complete. However, the previous sections have provided a coarse categorisation of aspects that should be covered in a comprehensive study of word stress: acoustic cues (in production and perception), distributional aspects, its role in the prosodic structure and its communicative functions. While in theory all stress aspects could be covered using this categorisation, it is likely that when research advances, different perspectives ask for a different categorisation or shed a different light on existing categories. It should also be noted that there is natural overlap between these aspects. For example, the investigation of stress perception cannot go without a study of its acoustic or distributional aspects. Furthermore, the current book maintains an empirical approach to word stress, using acoustic analyses, experiments and corpus analyses. This does not entail that the approach is purely phonetic. As becomes clear from the studies discussed above, the understanding of word stress crucially relies on our understanding of the phonetics-phonology interface. This book is no exception and the (phonological) distribution of word stress patterns has therefore a central place in this work. At the same time, this book does not cover each of the aforementioned aspects in equal detail. The studies presented in the following chapters cover the acoustic realisation, the perception of stress patterns, the distribution and function of word stress in the lexicon, and whether listeners benefit from stress cues for word recognition. Thus, the interaction between word stress and other levels in the prosodic structure are covered only to smaller extent.\\

\noindent Research question: \textit{Does Papuan Malay have word stress?}\\

\noindent Ch.\ref{chAc}) \textit{To what extent does the acoustic signal provide evidence for word stress?} \\
Ch.\ref{chPerc}) \textit{To what extent are listeners able to perceive the stress patterns?}\\
Ch.\ref{chLex}) \textit{To what extent do lexical analyses provide evidence for word stress?}\\
Ch.\ref{chGat}) \textit{Are listeners able to use stress cues to identify words?}\\

The central research question (RQ) investigated in this book is formulated above. Four sub-questions are formulated that each cover a specific aspect, following the order of chapters. RQ Ch.\ref{chAc} is investigated by an acoustic analysis of spontaneous monologues, covering more than ten different acoustic measurements of stressed versus unstressed syllables. RQ Ch.\ref{chPerc} is investigated by a series of three perception experiments on the relevance of acoustic cues to listeners in online and offline processing of words. RQ Ch.\ref{chLex} is investigated using two word-corpus-based studies on the word-disambiguation function of stress patterns and the phonological properties that underlie these patterns respectively. RQ Ch.\ref{chGat} is investigated using a gating experiment. \par
