\chapter{Discussion and conclusion} \label{chDisc}

This book investigated whether Papuan Malay has word stress in four subsequent studies, focusing on different aspects of word stress in each of them. The short version of the answer to the research question at this stage is `yes'. However, to understand what it means for a language to `have stress', a brief reconsideration of the results is needed.\par

The first study showed that stressed syllables in Papuan Malay are acoustically more prominent than unstressed ones, in particular due to their increased duration, shallower spectral tilt and decentralised vowel quality. Furthermore, ultimate stress was found to be acoustically more marked than penultimate stress. In the second study, listeners were shown to be sensitive to this asymmetry and able to benefit from assuming the default penultimate pattern, unless otherwise signaled acoustically. The stress patterns were shown to play a word disambiguating role in the lexicon in the third study. The distinction between the stress patterns was ascribed to the avoidance of mid-vowels, rather than just /\symbol{"025B}/ rejecting penultimate stress. Finally, the fourth study showed that listeners can successfully use their sensitivity to the acoustic cues to stress for word disambiguation.\par

While most of the evidence provided in this book speaks undoubtedly in favor of an analysis of Papuan Malay as a stress language, some issues remain unanswered. The most pregnant ones are discussed in the following and could shed a crucially different light on the results of this book when further investigated.\par

\section{Acoustic signal}

Chapter \ref{chAc} selected disyllabic words that did not appear in phrase-final position, in order to avoid any acoustic effects related to phrase-finality in Papuan Malay, i.e., final lengthening and effects of phrase-final tonal movements such as a rise-fall. In this way, no acoustic evidence for word stress was obtained across phrase positions that differed to a larger extent. As the results stand, they mainly provide acoustic evidence for prominence differences within disyllables. As a control, antepenultimate and pre-antepenultimate syllables from longer words could have been measured. Note that no paradigmatic investigation of Papuan Malay word stress was possible due to the lack of minimal pairs (\citealt{vanheuven_acoustic_2018}). In addition, ultimate word stress is confounded by final lengthening, which could have given speakers more time to articulate more extreme vowel positions. In other words, the word-boundary marking and potential stress marking cannot be fully disentangled. With regard to vowel quality, the acoustic measures were not corrected for intrinsic properties of vowels, which may have caused variance in the results that was not accounted for in the modeling. As for F0, the measurement did not capture fine-grained trajectories of the movement, as would a time-series measurement have done. Note that those aspects are studied in depth for phrase-final positions in \citet{kaland_red_2023}.

\section{Perception}

The perception of the acoustic patterns reflecting word stress was tested in Chapter \ref{chPerc}. Listeners in Experiment I and II appeared to have a bias for hearing ultimate word stress, rather than for penultimate word stress. It cannot be concluded from the results whether this bias was a consequence of the acoustic manipulation in the stimuli or the listeners' perceptual system. A control condition in which the two syllables were identical would have been needed to draw conclusions the nature of the bias. In Chapter \ref{chGat}, a gating task testing word disambiguation on the basis of word stress cues was carried out. The results appeared to be affected by co-articulation cues in the stimuli. One way of avoiding these effects would be to use a version of the stimuli without segmental cues (i.e., filtered or hummed speech). Although this would further reduce the naturalness of the stimuli, co-articulation would have been likely to be accounted for.

\section{Schwa}
A puzzling result in the acoustic analysis (Chapter \ref{chAc}) pertains to the role of schwa in the vowel inventory of Papuan Malay. Although the reports in \citet{kluge_grammar_2017} were followed in that /\symbol{"025B}/ might reduce to schwa and therefore causes a stress shift, an opposite effect was observed in the vowel quality measures: stressed syllables had a more central /\symbol{"025B}/ than unstressed syllables. The number of items analysed and the resulting (in)stability of the vowel quality measures should be kept in mind. Nevertheless, schwa has played a central role in phonological theories on stress. The remaining theoretical issue here concerns the question to what extent schwa should always remain unstressed, or whether it is in fact possible to stress it as other vowels. The last possibility has been supported with acoustic data from Indonesian (\citealt{laksman_location_1994}), but remains a difficult finding to reconcile with other stress research. The core problem lies at the central nature of schwa. Assuming that it is truly a central vowel, in what direction in the acoustic space should it go when it is stressed? For Papuan Malay, when assuming an influence of Standard Indonesian, the unexpected factors underlying the vowel quality results could be speculated on (Chapter \ref{chAc}). More research is needed, in particular on the extent to which schwa can or should be seen part of the vowel inventory of Papuan Malay. The importance of the status of schwa in Trade Malay has been illustrated in \citet{paauw_malay_2009} and one way forward would be to further investigate Papuan Malay's (historical) origin of stress loss and/or stress acquisition.\par

\section{Lexical status}
Traditionally, the lexical status of stress has been taken as an important diagnostic to analyse a language as a stress language. The outcomes of this book cast doubt on whether this criterion should be taken as a decisive diagnostic. It is not so much a question of whether the lexical storage of suprasegmental stress information should not be taken as an indication of word stress, but rather whether the absence of it is sufficient to conclude that a language is stressless. On the basis of the results reported in this book the answer to the latter question would be `no'. \par

Chapter \ref{chLex} and \ref{chGat} showed that the Papuan Malay stress patterns provide cues that can be used by listeners to disambiguate words. The cues of both stressed and unstressed syllables were, however, presented to listeners (Chapter \ref{chGat}), eliminating the need to use lexically stored suprasegmental information. That information would have been needed in a task in which only one syllable would have been presented and in which the listeners' word choice would depend on the representation in their mental lexicon. Another way to tap into lexical storage would be by memory tasks using nonce words. The latter setup is essentially used in stress `deafness' tasks (e.g. \citealt{dupoux_destressing_1997}). This was done in \citet{kaland_when_2024}, which compares Papuan Malay listeners' stress recall performance to those of German listeners. The results showed that Papuan Malay listeners perform worse than German ones, somewhat comparable to the French listeners in \citet{dupoux_destressing_1997}. How should this result be reconciled with the results of this book, in particular with the ones in Chapter \ref{chGat}?\par

It is crucial to understand why suprasegmental information should be stored lexically. It has been shown that the need for storage largely depends on the predictability of stress patterns. Thus, it is not so much an issue of how many different stress patterns are found in a language. The crucial aspect is the extent to which listeners benefit from applying phonological rules to predict the stress patterns. For highly predictable patterns, phonological rules have a larger role to play than for unpredictable patterns. Thus, more stress information needs to be stored lexically when there is no rule that helps to predict its pattern. This is crucially a \textit{degree}, as shown in previous research (\citealt{peperkamp_typological_2002}) and not a matter of finding minimal stress pairs or not (as acknowledged in \citealt{cutler_native_2012}).\par

Turning back to the Papuan Malay results, it remains to be seen how a stressless analysis of this language would account for the results presented in this book. The studies so far ask for a wider definition of word stress as sometimes maintained in previous work. The proposal would be to `diagnose' stress on a scale, similar to the `stress deafness' one, allowing for intermediate functionality between the classic extremes of lexical stress (minimal pairs) and complete `stress deafness'. The work conducted in Chapter \ref{chGat} has shown that near-minimal pairs can be used to show listeners' ability to use suprasegmental cues to recognise words, even though the stress pattern distribution of their language is highly predictable and they are shown to be stress `deaf' in the classic sense. The lexical analyses in Chapter \ref{chLex} indicate that the lexicon is built in such a way that there is indeed a role to play for the stress patterns to disambiguate words. Given the acoustic presence (Chapter \ref{chAc}) and the listeners' auditory sensitivity (Chapter \ref{chPerc}), the results of this book taken together are consistent. That is, Papuan Malay has a type of word stress that facilitates word recognition, although it does not need to be stored lexically (\citealt{kaland_when_2024}).\par

\section{Prosodic structure}
Another remaining issue pertains to how word stress in Papuan Malay relates to other levels in the prosodic structure. So far, studies have mainly focused on the word level (this book and \citealt{kaland_repetition_2018}; \citealt{kaland_role_2022}) or phrase level (\citealt{riesberg_perception_2018}; \citealt{riesberg_using_2020}, \citealt{kaland_demarcating_2020}; \citealt{kaland_red_2023}). From these results it seems that Papuan Malay phrase prosody mainly functions as a phrase boundary marker, with little to no use of pitch accents, as described for other languages (e.g. \citealt{jun_prosodic_2005}; \citealt{jun_prosodic_2014}). The typological model by \citet{jun_prosodic_2014} relies on stressed syllables as docking sites for phrasal accents, as is reflected in the rhythmicity of head-languages. There are languages in which the word and phrase level are, however, more independent (e.g. \citealt{lindstrom_aspects_2005}). Papuan Malay seems to be such a language, as stressed syllables do not appear to have a function at the phrase level in medial positions. It should be noted, though, that at phrase-final positions, stressed syllables tend to coincide with a rising F0 movement, as observed in two studies (\citealt{kaland_demarcating_2020}; \citealt{kaland_red_2023}). Current research is undertaken to investigate the variation in F0 movements phrase-finally. The open question remains whether F0 movements are strictly boundary tones or whether they have a highlighting function as well, and what the difference between final and pre-final syllables is in this respect. As the phrase-final F0 movements often span (at least) two syllables, it might be a challenge for Papuan Malay to tease apart the two core prosodic functions assumed in typological models of prosody so far (\citealt{jun_prosodic_2014}; see \citealt{kaland_demarcating_2020}).

\section{Vowel quality and stress in English}
Before wrapping up the discussion of this book, a final comment should be made on the way stress in English has been studied. English has played a central role in (shaping) our understanding of word stress, at least since \citet{fry_duration_1955}. As shown in psycholinguistic studies, English listeners mainly attend to vowel quality differences rather than to other possible stress cues to distinguish words that are in lexical competition (e.g. \citealt{cooper_constraints_2002}; and discussion in \citealt{cutler_native_2012}). Vowel quality differences in English stress research have often been referred to as `segmental cues to stress' (e.g. \citealt{tremblay_dutch_2021}). As argued in the following, such a description of word stress cues is theoretically problematic.\par

Traditionally, word stress has been analysed as a suprasegmental property of words, i.e. concerning the prosodic cues that `overlay' the segments. This aspect lies at the core of a widely accepted definition of word stress in prosodic theory (Chapter \ref{chInt}). However, allowing segmental properties to cue word stress troubles the diagnosis of stress. That is, how do we know whether we are dealing with a suprasegmental phenomenon if its main cue is categorised as segmental?\par

The problem lies in the use of terminology. Strictly speaking, the `segmental cue' to English word stress, vowel quality, is not more or less segmental than other possible cues to word stress (e.g. duration). That is, these acoustic properties are physical aspects of the speech signal, regardless of how we categorise them in theories. The ambiguity with vowel quality is that it can be interpreted as the acoustic property of (phonemic) vowel identity as well as vowel reduction due to stress, where the former could be called `segmental' and the latter `suprasegmental'. Thus, vowel quality in English should be called a suprasegmental cue if we accept the analysis of English as a stress language. Otherwise, a re-analysis of English word stress is needed. This would mean that alleged minimal stress pairs of nouns and verbs such as (\textit{permit} in \ref{ex11a} and \ref{ex11b}, Chapter \ref{chInt}) are actually segmentally different and the vowel inventory of English should be re-analysed accordingly (see \citealt{maskikit-essed_no_2016} for a re-analysis along such lines for Ambonese Malay). While this re-analysis would undoubtedly be controversial for English, lexical analyses have shown that the disambiguating function of word stress in English is minimal (\citealt{cutler_phonemic_2004}). Whether these findings indeed give rise to a rejection of word stress as a suprasegmental feature in English, therefore, remains to be a crucial topic for further investigation and discussion. For the time being, this book maintained and advocated a definition that interprets word stress as a prosodic phenomenon that is therefore signaled by acoustic cues \textit{supra}segmentally.

\section{Conclusion}
Although future research still needs to contribute to the stress analysis in Papuan Malay, the current book has shown that, with the coverage of multiple aspects of word stress, a comprehensive understanding can be achieved. This has implications for the many languages that have only been described in grammars, often lacking empirical investigations of prosody. On the theoretical level, this book has shown that a definition of word stress as purely lexical might not apply to Papuan Malay. Thus, the results presented here ask for a view on the presence or absence of word stress that does not solely depend on its lexical status in the narrow sense. That is, this book has argued for the possibility to have predictable and functional stress patterns that do not need to be stored in the mental lexicon. Although future work should investigate this in more detail, it is possible that Papuan Malay stress is assigned post-lexically, however still at a lower level than the phonological phrase (\citealt{nespor_prosodic_2007}). In traditional lexical stress languages, the assignment of stress is `earlier', in the sense that it is stored with the phonemic information in the mental lexicon, whereas in the Papuan Malay type assignment is `later'. It is unclear whether existing phonological hierarchies are able to fully account for the potential variation among the prosodic levels at which different types of word stress could be assigned. There might be multiple underlying levels that all have a potentially similar sounding effect at the word level surface, thereby challenging the phonological analyses. A crucial question remains about which degree of memorization is needed for the Papuan Malay stress patterns, whether it is indeed possible to distinguish lexical from `post-lexical' word stress, and how this can be shown in experimental (word-recognition) tasks.