\chapter{Lexical analyses of the function and phonology of Papuan Malay word stress} \label{chLex}

\section{Introduction} \label{sec41}
A number of recent studies found support for the existence of word stress in Papuan Malay, a language spoken in the easternmost provinces of Indonesia (Chapter \ref{chAc}, Chapter \ref{chPerc}, \citealt{kluge_grammar_2017}). The findings are a challenge to integrate with the different outcomes in studies on closely related languages such as Ambonese Malay (\citealt{maskikit-essed_no_2016}). In addition, the word stress controversy in Trade Malay languages fits into an existing debate on word stress in Indonesian languages (e.g., \citealt{vanzanten_stress_2010}).\par

The current support for word stress in Papuan Malay comes from a grammar based on auditory impressions (\citealt{kluge_grammar_2017}), acoustic analyses (Chapter \ref{chAc}) and perception experiments (Chapter \ref{chPerc}). These studies agree that the penultimate syllable in a word is stressed by default, except when that syllable contains /\symbol{"025B}/, in which case the ultimate syllable is stressed. This stress distribution has been reported (and disputed) for numerous Austronesian languages (\citealt{vanzanten_stress_2010}). However, detailed phonetic or phonological analyses of stress in these languages are few. Unlike other Malay variants that have at least one mid centralised vowel (schwa), Papuan Malay has no schwa; its five vowels are /i, \symbol{"025B}, a, \symbol{"0254}, u/. Papuan Malay is in this respect similar to Toba Batak, both in the lack of schwa and in stress placement (\citealt{goedemans_stress_2007}; \citealt{vanheuven_effects_1997}; \citealt{vanzanten_indonesian_1984}). Recent analyses have advanced the study of stress in Papuan Malay. Duration, vowel quality and spectral tilt were found to be the strongest acoustic correlates of word stress (Chapter \ref{chAc}). Regarding perception, Papuan Malay listeners appeared to assume the penultimate stress pattern as a default and were sensitive mainly to deviant (ultimate) stress patterns in lexical decision tasks (Chapter \ref{chPerc}). These perception effects could be ascribed to acoustic differences in duration, (spectrally weighed) intensity and fundamental frequency (F0). However, none of these cues alone was sufficient for listeners to discriminate between penultimate and ultimate stress. Although vowel quality was not tested perceptually, it was found to be a strong acoustic correlate (Chapter \ref{chAc}).\par

The work just discussed focused on the acoustic nature of stress in production and perception. Less is known about whether and how Papuan Malay stress patterns play a functional role in the lexicon. Although word stress in Papuan Malay is unlikely to be lexically contrastive (e.g., no minimal stress pairs were reported in \citealt{kluge_grammar_2017}), the distinction between penultimate and ultimate stress could still facilitate word recognition (e.g., \citealt{cutler_lexical_2005}; \citealt{vanzanten_word_2004}). The applicability of such a function could make the case of word stress in Papuan Malay less controversial. In addition, acoustic patterns of penultimate word stress in Austronesian languages have often been explained as reflexes of phrase prosody (\citealt{vanheuven_betawi_2008} on Betawi Malay; \citealt{vanzanten_stress_2010}). To substantiate the claim that the acoustic results obtained for Papuan Malay are indeed bound to the word level, additional research is needed on the morphological and phonological factors underlying these stress patterns. This advances the existing sketch of Papuan Malay phonology (\citealt{kluge_grammar_2017}). Therefore, the current study extends the existing work in two ways. First, the role of word stress in word disambiguation is investigated, comparing Papuan Malay with other languages for which similar analyses were carried out. Second, the phonological nature of Papuan Malay word stress is further explored comparing multiple (morpho-)phonological factors and their effects on stress placement. To this end, a lexical analysis of word embeddings and a random forest analysis were carried out respectively, using a list of 1127 words obtained from extensive fieldwork (\citealt{kluge_grammar_2017}; \citealt{kluge_papuan_2014}).

\subsection{Stress controversy in Indonesian languages}
The existence of word stress patterns in Indonesian languages has been the topic of several studies, sometimes leading to contradictory findings (see \citealt{goedemans_stress_2007} for an overview). Papuan Malay, the focus of this chapter, is spoken by more than one million people and serves as the common language for many different communities in the Indonesian provinces of Papua and West Papua (\citealt{kluge_grammar_2017}). For most speakers, Papuan Malay is therefore their second native language, alongside their inherited language. Before illustrating the controversy in a discussion of stress in Trade Malay languages, to which Papuan Malay belongs, some general remarks should be made that are also relevant for other Indonesian languages. Part of the controversy resulted from the lack of distinguishing word stress from phrase prosody in early work (see \citealt{goedemans_stress_2007} for a discussion). In addition, attempts to generalise over language families in Indonesia have been prone to overlook language variation. Regional variation is a crucial variable in explaining the distribution of stress patterns (\citealt{goedemans_stress_2007}; \citealt{himmelmann_austronesia_2020}). Although generalizations over language families could provide important typological insights, they are at best based on a limited number of studies, as most languages of Indonesia are still under-researched.\par

Contradictory claims on word stress can be observed for at least two Trade Malay languages: Ambonese Malay and Papuan Malay. As for Ambonese Malay, word pairs with penultimate/ultimate stress as their minimal difference were reported (\citealt{vanminde_malayu_1997}). However, an acoustic analysis of correlates of word stress and pitch accents in this language provided a different explanation for these minimal pairs. Vowel quality differences were no longer seen as the main correlate of stress in the minimal pairs, but were rather interpreted as acoustic evidence for two different vowels. Thus, Ambonese Malay was re-analyzed as a language without stress, having six instead of five vowels (\citealt{maskikit-essed_no_2016}).\par

For Papuan Malay, a rating task on phrase-level prominences and boundaries indicated that native listeners mainly agreed on where prosodic phrase-final boundaries occurred (\citealt{riesberg_perception_2018}). It was concluded that prosodic prominence is unlikely to be a relevant concept in this language and that Papuan Malay therefore does not have word stress or pitch accents, following the most recent claim for Ambonese Malay. However, the rating task concerned phrase prosody only and did not provide direct evidence about word stress patterns. An analysis of word stress correlates (Chapter \ref{chAc}), crucially excluding phrase-final words, showed that the distinction between penultimate and ultimate stress is signaled acoustically, confirming earlier reports (\citealt{kluge_grammar_2017}). In perception studies, Papuan Malay listeners were found to be sensitive to acoustically prominent syllables at the word and phrase level. It should be noted that the perceived prominence resulted from either sequences of manipulated Papuan Malay syllables (Chapter \ref{chPerc}) or German phrases with salient pitch accents (\citealt{riesberg_using_2020}).\par

\subsection{Types of evidence for lexical stress} \label{sec412}
Given these contradictory findings, it is important to consider the type of evidence used in the studies. For Ambonese Malay, the argument in favor of word stress appeared to be mainly stem from the author's impressions based on a small number of minimal pairs (\citealt{vanminde_malayu_1997}). The acoustic analysis (\citealt{maskikit-essed_no_2016}) was carried out on a small number of speakers ($N$ = 4) and stimuli ($N$ = 9), the latter being selected to test minimal pair contrasts only. In this respect, it should be noted that the literature has cast doubt on diagnosing word stress based only on the existence of minimal pairs. Psycholinguistic studies have shown that even in Germanic languages that are uncontroversially analyzed as having stress, minimal stress pairs (except for ones involving affixation) are highly infrequent in their lexicons (e.g., \citealt{cutler_native_2012}). Thus, the lexically contrastive function does not necessarily provide a useful stress diagnostic. More evident and insightful functions of stress were revealed in studies on listener expectations and lexicon structure. For example, it has been shown that rhythmical expectations based on the (regular) placement of the stressed syllable facilitate word recognition processes (see \citealt{cutler_lexical_2005} for an overview). Furthermore, studies have shown that these processes are not identical across languages. Much depends on whether stress cues that are available in the speech signal are indeed used by listeners. It has been shown that in early stages of word recognition, stress cues do not have a word discriminatory function in English (\citealt{cutler_forbear_1986}), although they do in Dutch (\citealt{vanheuven_effects_1988}). How sensitive listeners are to stress cues depends to some extent on the distribution of regular versus irregular stress patterns in the lexicon (\citealt{domahs_stress_2012}; \citealt{peperkamp_perception_2010}). In particular, it has been shown that stress may help listeners to disambiguate words (\citealt{cutler_explaining_2006}; \citealt{cutler_phonemic_2004}).\par

\subsection{Word disambiguation}
The disambiguating role of stress can be illustrated when considering the English embedded word \textit{bee} in either the carrier word \textit{beanie} or the carrier word \textit{belay}. In this example, suprasegmental stress cues could disambiguate bee and belay, as the first consonant-vowel sequence is enough to recognise it as a stressed (\textit{bee}) or unstressed (\textit{belay}) syllable. However, stress is not a disambiguating cue between \textit{bee} and \textit{beanie} as the matching first syllable is stressed in both words. In the latter case, word disambiguation is rather based on segmental differences between the embedded word and its carrier word. Studies have compared the total number of embedded words and the number of embedded words when stress was taken into account as a disambiguating cue for a small number of languages (\citealt{cutler_phonemic_2004}). These counts revealed differences between English on the one hand and Dutch, German and Spanish on the other (Table \ref{tab41}). That is, the relative decrease in mean embeddings per word due to mismatching stress was the largest in Spanish and the smallest in English. Dutch and German occupied middle positions. For all of these languages the mean value of stress-matched embeddings was below the limit of one per carrier word, showing that stress information successfully reduces the competition between the candidates. The limit of one embedding per carrier word is crucial, as the disambiguation problem that listeners face can be successfully reduced by stress cues only if there are sufficient embeddings (\citealt{cutler_explaining_2006}). For English, the statistics showed that even when stress information was ignored, the mean number of embeddings per carrier was below one. This would indicate that in English there is little need for word disambiguation, and therefore limited room for stress cues to play a role, even when they are present in the signal (\citealt{cooper_constraints_2002}). It has been argued that the differences among the lexical statistics of each language could indeed be explained by the type of stress cues listeners exploit in word recognition. In Spanish, listeners use mainly suprasegmental cues (see also \citealt{peperkamp_perception_2010}). In Dutch and German, both suprasegmental and segmental cues are used, whereas in English, segmental cues are most important (\citealt{cooper_constraints_2002}; \citealt{yu_vocabulary_2020}). In English, vowel reduction is the most important segmental cue to stress and serves to distinguish, for example, word class in disyllabic words, e.g., \textit{subject} is a verb when the vowel in the first syllable is reduced (ultimate stress) and a noun when the vowel in the first syllable is full (penultimate stress). The limited role of suprasegmental cues was also hypothesised to be the consequence of phoneme inventory size (\citealt{cutler_phonemic_2004}), such that languages with large inventories (e.g., 44 in English) have more options to disambiguate words by segmental means than languages with small inventories (e.g., 25 in Spanish). This would explain the limited degree of disambiguation by means of suprasegmental stress cues found for English (Table \ref{tab41}). Although the hypothesis suggested by \citet{cutler_phonemic_2004} is intuitively quite appealing, there is considerable counter-evidence against it from more recent typological quantification. \citet[and references therein]{maddieson_phonological_2011} demonstrated that size of the consonant and vowel inventories, syllable complexity measures, size and complexity of tone systems and the distinguishing role of stress are all positively correlated in the UCLA Phonological Segment Inventory Database (UPSID), a genealogically representative sample of between 450 and 650 languages (depending on the linguistic property at issue).

\begin{table}
\caption{Mean number of embedded words per complex word when ignoring stress (left) and when considering stress (right). Data from \citet{cutler_phonemic_2004} and \citet{cutler_explaining_2006}.}
\label{tab41}
\begin{tabularx}{0.8\textwidth}{Xrrr}
\lsptoprule
& All embeddings & Stress-matched & Proportion\\
\midrule
 Dutch & 1.52 & 0.74 & 0.49\\
 English & 0.94 & 0.59 & 0.62\\
 German & 1.62 & 0.80 & 0.49\\
 Spanish & 2.32 & 0.73 & 0.31\\
\lspbottomrule
\end{tabularx}
\end{table}

\subsection{Vowel quality} \label{sec414}
It remains to be seen whether the disambiguating role of stress can be used to shed light on the stress controversy in Austronesian languages. As for Papuan Malay, field elicitations reported in \citet{kluge_grammar_2017} do not contain any minimal stress pairs. The acoustic evidence (Chapter \ref{chAc}) was based on unscripted story re-tellings by 19 speakers and showed structural support for penultimate word stress as the default pattern and ultimate stress as the exceptional pattern (i.e., when the penultimate syllable contains /\symbol{"025B}/). Vowel quality appeared as one of the stronger acoustic correlates, showing that vowel formants in stressed syllables are further displaced from the center of the acoustic space compared to those in unstressed syllables. However, this result was found for the Papuan Malay vowels /i/, /a/, /\symbol{"0254}/ and /u/, but not for /\symbol{"025B}/, which showed the opposite effect (i.e., further away from the center when unstressed). A speculative explanation was provided, in which /\symbol{"025B}/ was produced as stressed schwa, a possibility reported for Jakarta Indonesian (\citealt{laksman_location_1994}; see Chapter \ref{chAc}). However, schwa as the only cause of a stress shift has been reported as a doubtful indicator of the existence of stress, and the avoidance of schwa could actually be an effect of phrase prosody (\citealt[88]{goedemans_no_2014}). It thus remains to be seen how the role of /\symbol{"025B}/ should be interpreted in the light of potential other phonological factors that could cause a stress shift. The interpretation is particularly challenged by the fact that /\symbol{"025B}/ appears to be the main reason for stress to shift from penultimate to ultimate (e.g., \textit{lama}, /ˈla.ma/, `to be long in duration' vs. \textit{lema}, /l\symbol{"025B}.ˈma/, `to be weak'), and at the same time shows exceptional formant displacement when stressed (Chapter \ref{chAc}). The latter finding could be related to the extreme sparsity of stressed /\symbol{"025B}/, a pattern mainly observed in penultimate position when the ultimate syllable also contains /\symbol{"025B}/. It remains unclear, however, to what extent the vowel quality difference between centralised /\symbol{"025B}/ (stressed) and decentralised /\symbol{"025B}/ (unstressed) should indeed be interpreted as a stress difference.\par

One option would be to assume that the alleged stress differences found for /\symbol{"025B}/ are actually segmental differences, i.e., two distinct but acoustically similar vowels. This type of explanation, as provided for Ambonese Malay (\citealt{maskikit-essed_no_2016}), could in theory also be applied to the other vowels, assuming that the Papuan Malay inventory includes twice as many vowels as reported by \citet{kluge_grammar_2017}. Such an inventory would then consist of five vowels each in a full/long version and a reduced/short version, which would imply that Papuan Malay does not have word stress. Vowel quality or length differences are indeed commonly reported as distinguishing between two subsets of vowels in an inventory (e.g., \citealt{maddieson_patterns_2009}). However, given that the acoustic distinctions between the vowel categories concern several prosodic cues at the same time (Chapter \ref{chAc}), the stance taken in this chapter is that the acoustic results found so far need to be explained as being suprasegmental (i.e., word stress) rather than segmental properties. More research is needed to complement the existing acoustic findings. The research questions addressed in this chapter are outlined in what follows.

\subsection{Research questions} 
Given the above discussion of the literature, this chapter addresses two different aspects of the lexical status of word stress in Papuan Malay using non-acoustic analyses. The first is a confirmatory analysis investigating the disambiguating function of word stress (RQ1). We predict that if stress is indeed a suprasegmental property of Papuan Malay words, listeners might benefit from how the patterns are distributed. If this holds true, stress could facilitate word disambiguation. This hypothesis is investigated in a lexical analysis similar to the ones carried out by \citet{cutler_phonemic_2004} and \citet{cutler_explaining_2006}. The null hypothesis would be that Papuan Malay stress patterns do not facilitate word disambiguation, which could be an indication that stress needs to be interpreted as a segmental property, similar to English (\citealt{cooper_constraints_2002}).\\ \par

\noindent RQ1: Do Papuan Malay stress patterns reduce the number of alternative word candidates?\\

Second, more research is needed on the phonology of the stress placement in Papuan Malay, in particular to understand the unique role of /\symbol{"025B}/. This is investigated in the current chapter by means of an exploratory analysis using the random forests classification technique (\citealt{breiman_random_2001}). This type of analysis ranks the relative importance of multiple interrelated variables. In this way, the effect of /\symbol{"025B}/ can be compared to factors that could also be relevant for stress placement, such as other vowels, syllable structure or word class.\\ \par

\noindent RQ2: What phonological factors determine stress placement in Papuan Malay? \\

These issues are investigated in this chapter by means of a lexical analysis of word embeddings (Section \ref{sec42}) and a random forest analysis of morpho-phonological predictors of stress (Section \ref{sec43}) using a corpus of Papuan Malay words.

\section{Lexical analysis of word embeddings} \label{sec42}
The corpus consisted of Papuan Malay words, as provided by \citet{kluge_grammar_2017} and \citet{kluge_papuan_2014}. The words selected for the analyses in this study concerned native roots only (\citealt{kluge_grammar_2017}: Appendix A.1). Loanwords, which occur frequently in this language, were discarded. In this way, potential influences from stress patterns that originate from other languages were avoided. The word lists consisted of a written lexeme, phonetic transcription with syllable boundaries and stress indicated, word class label, and an English gloss.\par

Before obtaining the number of embedded words, duplicates (e.g., homonyms such as \textit{pasang} for `pair' or `to install') were purged from the word list such that only single instances of sound shapes were left. Given the paucity of four-syllable words in the list ($N$ = 3), they were excluded from the counting procedure. Based on the syllable boundary indications in the phonetic transcriptions, the number of syllables was counted per word. In addition, the stressed syllable was indicated as a number referring to the position of that syllable in that word, based on the stress marks in the phonetic transcriptions. The final word list used for analysis contained 1106 words, 1062 of which were polysyllabic and potential candidates for carrier words. Table \ref{tab42} provides the word counts for the analyzed word list.\par

In the existing studies on word embeddings (\citealt{cutler_explaining_2006}; \citealt{cutler_phonemic_2004}) the mean embeddings were weighted by word frequency. For Papuan Malay no corpus data is available to provide word frequencies. In the current analysis, the embedding statistics are therefore unweighted (see Section \ref{sec422} for further discussion). The analysis carried out in this study is based on the word list underlying Table \ref{tab42}. Although a fair number of words might not frequently occur in spontaneous speech, the word list still provides a subset that is representative for the language. That is, the words in the list were elicited in spontaneous conversations and formed the basis for the phonological analysis (\citealt{kluge_grammar_2017}).\par

In the absence of frequency data, the current analysis established the number of polysyllabic words which contained one or more embeddings and counted the subset of these embeddings for which stress matched between carrier word and embedded word. Following previous lexical analysis studies, syllable boundaries were taken into account. For example, \textit{ke} `to' was counted as an embedding in \textit{kewa} (/ˈk\symbol{"025B}.wa/ `dance party'), but not in \textit{kembang} (/ˈk\symbol{"025B}m.ba\symbol{"014B}/ `flower'). As for stress-matching embeddings, ka (/ˈka/, `or') would count in \textit{kali} (/ˈka.li/ `river') or in \textit{sikakar} (/si.ˈka.k\symbol{"0250}r/ `to hold onto tightfisted') but not in \textit{muka} (/ˈmu.ka/ `front'). The counts were done automatically using syllable-level string matching based on the phonetic transcriptions in the word list. String matching was applied after diacritics were removed from the transcriptions, as these marks indicate variation in segmental surface realizations. This was done in order to find phonemically identical (i.e., matching) syllables, following the methods of \citet{cutler_phonemic_2004} and \citet{mcqueen_models_1995}.\par

\begin{table}
\caption{Word list counts by number of syllables ($\sigma$) and word stress.}
\label{tab42}
\begin{tabularx}{\textwidth}{lrYY}
\lsptoprule
Number of $\sigma$ & Penultimate stress & Ultimate stress & All stresses\\
 \midrule
 1 & - & - & 44\\
 2 & 892 & 103 & 995\\
 3 & 63 & 4 & 67\\
\midrule
 Total & 955 & 107 & 1106\\
\lspbottomrule
\end{tabularx}
\end{table}

\subsection{Results} 

\begin{table}[b]
\caption{Word counts and embeddings for each word-length in syllables in the Papuan Malay word list.}
\label{tab43}
\begin{tabularx}{\textwidth}{YYp{0.4cm}YY}
\lsptoprule
Word length ($\sigma$) & Carrier words & & \multicolumn{2}{c}{Embeddings}\\
\cmidrule{4-5}
& & & All & Stress-matched\\
\midrule
 2 & 136 & & 139 & 84\\
 3 & 23 & & 31 & 18\\
\midrule
 All lengths & 159 & & 170 & 102\\
\lspbottomrule
\end{tabularx}
\end{table}

As reported in Table \ref{tab43}, the list consisted of a total of 159 polysyllabic carrier words, i.e., words for which embedded words could be found (column ``carrier words''). The number of embedded words was 170 (column ``embeddings – all''). The embedded words had a length of either one syllable (in carrier words of two or three syllables) or two syllables (in carrier words of three syllables). The total number of embeddings was overall slightly higher than the number of carrier words (170 versus 159 respectively). This result indicated that carrier words had more than one embedding on average ($\mu$ = 1.07). The counts decreased when considering only the embedded words that matched for stress with the carrier word (column ``embeddings – stress-matched''). The latter observation is an indication that when stress is taken into account, the mean number of embeddings per carrier word drops below one ($\mu$ = 0.64). Note that from the 84 stress-matched embeddings in disyllabic carrier words, 82 matched with penultimate stress and 2 with ultimate stress. From the 18 stress-matched embeddings in trisyllabic words all matched with penultimate stress.\par

\begin{table}
\caption{All embeddings: length and location of embedded word (E) in carrier word (C) for each carrier word length in syllables ($\sigma$).}
\label{tab44a}
\begin{tabularx}{\textwidth}{YYp{0.5cm}YYY}
\lsptoprule
\multicolumn{2}{c}{Length ($\sigma$)} & & \multicolumn{3}{c}{Location of onset of E in C}\\
\cmidrule{1-2} \cmidrule{4-6}
Carrier & Embedding & & $\sigma$1 & $\sigma$2 & $\sigma$3\\
 \midrule
 2 & 1 & & 95 & 44 & \\
 \multirow{2}{*}{3} & 1 & & 8 & 8 & 3\\
 & 2 & & 4 & 8 & 0\\
\lspbottomrule
\end{tabularx}
\end{table}

\begin{table}
\caption{Stress-matched embeddings: length and location of embedded word (E) in carrier word (C) for each carrier word length in syllables ($\sigma$).}
\label{tab44b}
\begin{tabularx}{\textwidth}{YYp{0.5cm}YYY}
\lsptoprule
\multicolumn{2}{c}{Length ($\sigma$)} & & \multicolumn{3}{c}{Location of onset of E in C}\\
\cmidrule{1-2} \cmidrule{4-6}
Carrier & Embedding & & $\sigma$1 & $\sigma$2 & $\sigma$3\\
\midrule
 2 & 1 & & 82 & 5 & \\
 \multirow{2}{*}{3} & 1 & & 0 & 8 & 0\\
 & 2 & & 2 & 8 & 0\\
\lspbottomrule
\end{tabularx}
\end{table}

In addition, the location of the embedded word in the carrier word was counted, providing an insight into where in the carrier word the embedding started (see \citealt{cutler_phonemic_2004} for comparable tables with ratios). Tables \ref{tab44a} and \ref{tab44b} report the locations for all embeddings and for stress-matched embeddings respectively. Both tables show that the decrease in embeddings due to the consideration of word stress was particularly large for disyllabic carrier words with embeddings starting in the second syllable (e.g., \textit{yang}, /ˈj\symbol{"0250}\symbol{"014B}/, relativiser, in \textit{goyang}, /ˈg\symbol{"0254}.j\symbol{"0250}\symbol{"014B}/, `to shake'), and for trisyllabic carrier words with embeddings starting in the first syllable (e.g., \textit{sa}, /ˈsa/, 1SG, in \textit{sarana}, /sa.ˈ\symbol{"027E}a.na/, `facility'). In the latter case, all embeddings ($N$ = 8) could be disambiguated on the basis of stress information.\par

In order to assess the potential influence of syllable structure on the degree of disambiguation by stress, the counts of embeddings were split by structure of the first syllable in the embedding (Table \ref{tab45}). These counts show that CV syllable structure is the most frequent. The degree of disambiguation does not appear to differ due to syllable structure. It should be noted, however, that the counts challenge the comparison of syllable structures due to the overall frequency differences. The largest degree of disambiguation by stress is found in monosyllabic embeddings with an onset in the second syllable of the carrier word, both for CV and for CVC syllables.

\begin{table}
\caption{Number of all/stress-matching mono- and polysyllabic embeddings with location in the carrier ($\sigma$1, $\sigma$2, $\sigma$3) as a function of syllable structure of the initial syllable of the embedding.}
\label{tab45}
\begin{tabularx}{\textwidth}{l rrr r rr r r}
\lsptoprule
Structure & \multicolumn{3}{c}{Monosyllabic emb.} & & \multicolumn{2}{c}{Polysyllabic emb.} & & $N$ total\\
\cmidrule{2-4} \cmidrule{6-7}
& $\sigma$1 & $\sigma$2 & $\sigma$3 & & $\sigma$1 & $\sigma$2 & \\
\midrule
CV & 95/74 & 38/8 & 2/0 & & 1/0 & 6/6 & & 142/88\\
CVC & 6/6 & 14/2 & 1/0 & & 3/2 & 0/0 & & 24/10\\
CCV & 2/2 & 0/0 & 0/0 & & 0/0 & 0/0 & & 2/2\\
V & 0/0 & 0/0 & 0/0 & & 0/0 & 2/2 & & 2/2\\
\midrule
$N$ total & 103/82 & 52/10 & 3/0 & & 4/2 & 8/8 & & 170/102\\
\lspbottomrule
\end{tabularx}
\end{table}

\subsection{Conclusion} \label{sec422}
The lexical analysis shows that many word embeddings can be successfully eliminated during word recognition on the basis of mismatching stress patterns in Papuan Malay. The reduction is most clearly found for embeddings with an onset in an unstressed syllable of the carrier word (Tables \ref{tab44a} and \ref{tab44b}). This can be explained when considering that stress is highly regular in Papuan Malay (penultimate) and that most of the embeddings concern monosyllabic words. This result resembles the one reported for Spanish, a language with predominant penultimate stress, for which stress-matched embeddings were found mainly in penultimate syllables (\citealt{cutler_phonemic_2004}: Table \ref{tab45}). It should also be noted that for both Papuan Malay and Spanish, monosyllabic embeddings were counted as stressed and were the most frequent length of embedding. A crucial difference concerns the fact that the words under analysis were shorter in Papuan Malay than in Spanish, despite their similar phoneme inventory sizes (23 and 25 respectively). It is unlikely that the small Papuan Malay corpus did not reflect word lengths in a natural way. For instance, the predominance of disyllabic words has also been observed in spontaneous speech in Chapter \ref{chAc}.\par

In this respect it is interesting to reconsider the hypothesis that languages with small phoneme inventories resort to suprasegmental cues to signal stress to a larger extent than languages with large phoneme inventories (\citealt{cutler_phonemic_2004}). While suprasegmental cues do indeed play a role in stress perception in Papuan Malay (Chapter \ref{chPerc}), it is unclear how these compare to segmental cues such as vowel quality (further discussion in Section \ref{sec44}). Given the higher degrees of disambiguation in Spanish than in Papuan Malay, it seems that phoneme inventory is not the only factor predicting the role of suprasegmental cues (\citealt{maddieson_phonological_2011}). The morphological composition appears equally important, as Spanish has longer words than Papuan Malay. This caused stress-matched embeddings to occur mainly in word-initial syllables in Papuan Malay (predominantly disyllabic words), whereas in Spanish stress-matched embeddings were found in penultimate syllables of longer words as well. From a processing point of view, it has been argued that both the size and the location of the embedding are crucial for successful speech processing (\citealt{mcqueen_models_1995}). Given the word length differences between Papuan Malay and Spanish, however, the disambiguating function simply has less opportunity to facilitate speech processing in the former language. This is plausibly reflected in the relative amount of disambiguation found in both languages.\par

Although word frequency could not be taken into account in the current study, the mean number of embeddings per word can now be provisionally compared with the data from the other languages (Dutch, English, German and Spanish). For a more direct comparison between the Papuan Malay values and those in Table \ref{tab41}, the proportions of stress-matched embeddings can be computed by subtracting the mean value of stress-matched embeddings per carrier word from the mean value of all embeddings per carrier word. In this way, low proportions predict large facilitation, whereas high proportions predict small facilitation. These proportions thus give an insight into the relative magnitude of the facilitatory effect of stress on word recognition and abstracts over language-specific numbers of embeddings (Table \ref{tab46}, right column). The highest proportions are found for English (0.62) and Papuan Malay (0.60), followed by German and Dutch (each 0.49), whereas Spanish shows the lowest proportion of stress-matched embeddings (0.31).\par

\begin{table}
\caption{Mean number of embedded words per word when ignoring stress (left), when considering stress (mid), and the proportion of the latter (right). Data from \citet{cutler_phonemic_2004}, \citet{cutler_explaining_2006} and the current study.}
\label{tab46}
\begin{tabularx}{0.9\textwidth}{p{2.5cm}rrr}
\lsptoprule
& All embeddings & Stress-matched & Proportion\\
\midrule
 Dutch & 1.52 & 0.74 & 0.49\\
 English & 0.94 & 0.59 & 0.62\\
 German & 1.62 & 0.80 & 0.49\\
 Spanish & 2.32 & 0.73 & 0.31\\
 Papuan Malay & 1.07 & 0.64 & 0.60\\
\lspbottomrule
\end{tabularx}
\end{table}

The observations of this analysis lead to the conclusion that word stress in Papuan Malay has a potential function in word recognition in that it may aid the process of rejecting alternative word candidates. Given the similar proportions of stress-matched embeddings in the current study compared to the results of \citet{cutler_phonemic_2004} and \citet{cutler_explaining_2006} for English, it needs to be further discussed to what extent the facilitatory effect of word stress in Papuan Malay can be found in suprasegmental or segmental cues. This question is particularly relevant given that the relatively low degree of disambiguation in English (i.e. high proportions of stress-matched embeddings) can be ascribed to the fact that stress differences in this language are mainly signaled by vowel quality differences and to a much lesser extent by suprasegmental differences. Although it was found that vowel reduction is an acoustic cue to stress in Papuan Malay (Chapter \ref{chAc}), it remains to be seen to what extent listeners make use of this cue, in particular given that /\symbol{"025B}/ did not show the type of formant displacement found in other vowels (Section \ref{sec41}). If suprasegmental stress information is indeed less important for word recognition in Papuan Malay compared to languages such as Dutch, German or Spanish, a larger role could be reserved for vowel reduction in this language.\par

An important difference between Papuan Malay and English concerns the mean embeddings per carrier word when all embeddings are counted. This can be illustrated when recalling that one is the crucial limit for the (mean) number of embeddings per carrier word (Section \ref{sec412}). In English, the mean value is just below one (0.94), whereas in Papuan Malay this number is just above one (1.07). In English, therefore, disambiguation is less of a challenge for listeners to begin with. In Papuan Malay, however, there is more need to disambiguate than in English, predicting that the relative importance of suprasegmental cues is larger in the former language. It has to be noted that taking into account frequency data could still somewhat change this number for Papuan Malay. Despite the fact that the corpus in the current study consisted of commonly used words, no weight differences between high and low frequency words were taken into account. This could have resulted in a more coarse-grained analysis provided here, compared to the analyses in the literature (\citealt{cutler_explaining_2006}; \citealt{cutler_phonemic_2004}; \citealt{mcqueen_models_1995}). In this respect it is important to consider that the small corpus of common words in Papuan Malay (compared to the large corpora used in previous studies) could have reduced the chance that many low-frequency words were taken into account and as a consequence were given too much weight.\par Given that vowel quality plays a crucial role in stress distinctions (Section \ref{sec414}), as established for English and as suggested for Papuan Malay, it is important to further explore the phonological factors underlying stress placement in the latter language, as further outlined in Section \ref{sec43}.



\section{Random forest analysis of stress placement factors} \label{sec43}
Random forest analysis is a classification method based on the construction of a large number of decision trees (\citealt{breiman_random_2001}). In order to assess which variable splits (classifies) the data best, trees are constructed on the basis of random data and variable subsets. Random forests are particularly useful to determine the predictive value of a large set of variables and a small number of observations. Compared with other statistical methods, random forests are more resistant to overfitting and collinearity between predictors. Random forest analyses have only recently been introduced into linguistics (\citealt{tagliamonte_models_2012}), and more specifically into phonetics and phonology (e.g., \citealt{arnold_using_2013}; \citealt{baumann_what_2018}; \citealt{grafmiller_new_2011}; \citealt{grice_tune_2015}). The method is promising as notions such as prominence, stress or phonological weight typically correlate with a large number of acoustic and/or linguistic variables. Random forests can help reveal underlying mechanisms of linguistic structure, by providing powerful generalizations based on a relatively small set of field data. In addition, predictors are allowed to be derivatives of each other, even if there is a considerable degree of correlation among the predictors. For example, a variable with five levels corresponding to all vowels in an inventory and a variable with two levels only distinguishing low and high vowels from the same inventory can be both included in the model without losing predictive accuracy (\citealt{strobl_conditional_2008}). The predictive value of a certain variable in a random forest is expressed by means of its variable importance. Although the variable importance values are affected by correlation among predictors, their ranking relative to each other remains unaffected. The absolute variable importance values are irrelevant, as they are randomly generated (hence ``random forest''). Thus, the interpretation of variable importance generally relies on the relative differences between the respective values (\citealt{shih_random_2013}).\par

The selected subset of the Papuan Malay corpus for the random forest analysis consisted of two-syllable words only, in order to obtain a homogeneous set; words with one syllable ($N$ = 46) or more than two syllables ($N$ = 73) were relatively infrequent. Thus, the representativeness of the corpus was compromised to a minimal extent. An overview of the number of words per lexical category is given in Table \ref{tab47}. Note that words which translate to adjectives in English are expressed by stative verbs in Papuan Malay. For example, /b\symbol{"025B}.ˈsar/ (`big' - lit. `be big') is labeled as a verb in the corpus.

\begin{table}
\caption{Distribution of word classes in the corpus.}
\label{tab47}
\begin{tabularx}{0.5\textwidth}{p{4cm}r}
 \lsptoprule
 Word class & Count\\
 \midrule
 V bi(valent) & 341\\
 V mono stative & 205\\
 V mono dynamic & 51\\
 V other & 7\\
 Adverb & 32\\
 Noun & 355\\
 Function word (all) & 49\\
 \midrule
 Total & 1040\\
\lspbottomrule
\end{tabularx}
\end{table}


\subsection{Predictors}
Of interest to the current analysis are phonological factors that make a syllable likely to be stressed. In this respect it is important to consider the abstract phonological notions of sonority and weight. Sonority has been reported to be an underlying principle of syllable structure, reaching its peak in the vocalic nucleus and its valley at the edge consonant(s). Perhaps the simplest phonetic correlate of sonority is intensity, although the search for such a correlate shows that the relation between the phonetics and the phonology of sonority is far more complex than single correlates can explain (e.g., \citealt{parker_quantifying_2002}; \citealt{albert_model_2023}). Based on sonority scales, speech sounds can be classified following a particular hierarchy, generally with vowels at the top and obstruents at the bottom (e.g., \citealt{clements_role_1990}). More sonorous sounds often play a more decisive role in stress placement than less sonorous sounds, with a sub-hierarchy assumed within sound classes. For example, vowels have been reported to be more sonorous when they are peripheral and when their height is low (\citealt[27]{parker_quantifying_2002}). While sonority is a phonological property generally attributed to individual phonemes, it affects the status of the syllable in terms of weight (\citealt{gordon_syllable_2013}). In many stress languages the weight of the syllable is decisive for stress placement, since heavier syllables generally attract stress more than light ones (weight-to-stress principle; \citealt{chomsky_sound_1968}). One criterion distinguishes light and heavy syllables according to whether the syllable ends with a short vowel (light) or not (heavy), i.e., the ``Latin-criterion'' in \citet{ryan_phonological_2016}. There are different criteria to distinguish light from heavy syllables and it is open to discussion to what extent a strictly binary division or more gradient scales of weight are needed to account for cross-linguistic observations (\citealt{ryan_phonological_2016}). Vowel nuclei (and therefore sonority) are often most determining for weight-sensitive stress placement, although in some languages the coda (and sometimes even the onset) of the syllable co-determines its weight (e.g., \citealt{goedemans_weightless_1998}; \citealt{gordon_perceptually-driven_2005}; \citealt{hayes_metrical_1995}; \citealt{ryan_onsets_2014}). In the current study of the phonological factors underlying Papuan Malay stress, it is important, therefore, to consider the precise segmental makeup of the syllable. In particular, it is crucial to note that /\symbol{"025B}/ can be realised as schwa in Papuan Malay (\citealt{kluge_grammar_2017}), similar to other Trade Malay varieties (\citealt{paauw_malay_2009}). Schwa, the most central vowel, generally has lower sonority compared to the other vowels in the vowel inventory. The frequently reported role of /\symbol{"025B}/ (or schwa) in Indonesian languages as a reliable indicator of the presence of stress has been questioned (\citealt{goedemans_no_2014}).\par

For the current analysis a set of predictors was chosen that potentially affect syllable weight. Some of the predictors were derived from others. This was done to test whether individual sounds (particular vowels or consonants) or rather phonological classes of sounds predict stress placement better, as explained above. The included predictors (in italics, IPA notations following \citealt{kluge_grammar_2017}) concerned \textit{structure} in terms of consonantal and vowel segments, from which the \textit{openness} of the syllable and the actual segments in the \textit{onset}, \textit{nucleus} or \textit{coda} were derived. Papuan Malay has five vowels (/i, \symbol{"025B}, a, \symbol{"0254}, u/) and 17 consonants (stops: /p, b, t, d, k, g/; affricates: /\symbol{"02A7}, \symbol{"02A4}/; nasals: /m, n, \symbol{"014B}/; fricatives: /s, h/; rhotic: /r/; approximants: /l, j, w/). The predictors \textit{manner} of articulation and \textit{height} were derived from the actual consonants and vowels respectively. Voicing of the onset and constriction type of the coda were added as these have been reported as possible contributors to syllable weight (e.g., \citealt{ryan_onsets_2014}). Furthermore, recent work showed a difference in F0 excursion size between function words and content words, in that the former were smaller than the latter in phrase-final positions (\citealt{kaland_demarcating_2020}). Although this difference applied to the phrase level, potential implications for word prosody cannot be excluded based on these results. In fact, in some languages word class correlates with word stress placement (e.g., in English Latinate pairs: ˈ\textit{subject} (noun) and \textit{to sub}ˈ\textit{ject} (verb) form a minimal stress pair; in Dutch and German a difference in stress correlates with verbs with and without separable prefixes, e.g., in German ˈ\textit{übersetzen – setzte über – } ˈ\textit{übergesetzt} `ferry someone across, inf., past, participle' when stressed on ˈ\textit{über}; \textit{über}ˈ\textit{setzen – über}ˈ\textit{setzte – über}ˈ\textit{setzt} `translate(d) inf, past, participle' when stressed on the second part of the compound). Therefore, \textit{word class} was included as a predictor.

\subsection{Statistical analysis} 
The analysis was done in R (\citealt{rcoreteam_project_2019}) using the package ``ranger'' (\citealt{wright_ranger_2017}), which offers a computationally less intensive way to perform random forests compared with packages such as ``party'' (\citealt{strobl_party_2009}) or ``randomForest'' (\citealt{liaw_classification_2002}). The response variable in the random forest was stress location (2 levels: penultimate, ultimate). The predictors (in italics) were \textit{syllable structure} (6 levels: CCV, CCVC, CV, CVC, V, VC), \textit{onset} (18 levels: /b, \symbol{"02A7}, d, g, h, \symbol{"02A4}, k, l, m, n, \symbol{"014B}, p, r, s, t, w, j/, no onset), \textit{nucleus} (5 levels: /i, \symbol{"025B}, a, \symbol{"0254}, u/), \textit{coda} (12 levels: /j, k, l, m, n, \symbol{"014B}, p, r, s, t, w/, no coda), \textit{openness} (2 levels: open, closed), \textit{manner of articulation in onset and coda} (each 6 levels: plosive, fricative, nasal, rhotic, approximant, none), \textit{height of the vowel nucleus} (3 levels: open, mid, close), \textit{voicing of the onset} (3 levels: voiced, voiceless, no onset), \textit{constriction type of the coda} (3 levels: sonorant, obstruent, no coda), and \textit{word class} (7 levels: see Table \ref{tab47}). In addition, a control-predictor \textit{gloss} (the English translation of each word) was added. Gloss is not expected to be of any predictive value and should therefore have a low variable importance. Therefore, variable importance values of other predictors that lie around or below the one of the control-predictor can be used as an additional indication of which predictors do not affect the response variable at all. Except for word class and gloss all predictors were included for both the first and second syllable in the word (total: 22 predictors, see Table \ref{tab48}).\par

\begin{table}[t]
\caption{Overview of all 22 predictors ($\sigma$ for respective syllables) in the random forest analysis and their values for the example word /ˈbam.bu/ `bamboo'.}
\label{tab48}
\begin{tabularx}{0.6\textwidth}{p{0.8cm}p{4cm}X} 
\lsptoprule
& Predictor & Value\\
\midrule
1 & Syllable structure $\sigma$1 & CVC\\
2 & Syllable structure $\sigma$2 & CV\\
3 & Onset $\sigma$1 & b\\
4 & Onset $\sigma$2 & b\\
5 & Nucleus $\sigma$1 & a\\
6 & Nucleus $\sigma$2 & u\\
7 & Coda $\sigma$1 & m\\
8 & Coda $\sigma$2 & \textit{no coda}\\
9 & Openness $\sigma$1 & closed\\
10 & Openness $\sigma$2 & open\\
11 & Manner onset $\sigma$1 & plosive\\
12 & Manner onset $\sigma$2 & plosive\\
13 & Manner coda $\sigma$1 & nasal\\
14 & Manner coda $\sigma$2 & \textit{none}\\
15 & Height nucleus $\sigma$1 & open\\
16 & Height nucleus $\sigma$2 & close\\
17 & Voicing onset $\sigma$1 & voiced\\
18 & Voicing onset $\sigma$2 & voiced\\
19 & Constriction coda $\sigma$1 & sonorant\\
20 & Constriction coda $\sigma$2 & \textit{no coda}\\
21 & Word class & noun\\
22 & Gloss & bamboo\\
\lspbottomrule
\end{tabularx}
\end{table}

The number of trees in the analysis was increased in steps of 1000, starting from 1000 trees. The variable importance of the predictors reached a stable ranking around 5000 trees. To obtain a robust result, the final number of trees was set to 10000 (\citealt{shih_random_2013}). The number of randomly preselected predictors was set to the square root of the total number of predictors in the analysis ($\sqrt{22}$), and variable importance mode was set to ``permutation''. These settings are recommended for analyses with correlating predictors, following \citet{strobl_conditional_2008} and \citet{strobl_introduction_2009}.\par

The distribution analysis (Table \ref{tab42}) consisted of counts; 0 for each word with penultimate stress and 1 for each word with ultimate stress. The proportion of ultimate/penultimate stresses was then calculated by taking the mean of all counts. The two analyses combined appeared particularly helpful to interpret the variable importance values, as their absolute values are not indicative (Section \ref{sec433}).

\subsection{Results} \label{sec433}

\begin{figure}
\includegraphics[width=\textwidth]{401}
\caption{Variable importance plot with the predictors ranked from high (top) to low (bottom).}
\label{fig401}
\end{figure}

Two factors stand out as predictors for the location of stress in Papuan Malay (Figure \ref{fig401}): the nucleus in the first syllable and the height of the vowel nucleus in the first syllable. Other predictors showed considerably lower variable importance values, although the height of the vowel nucleus in the second syllable as well as the nucleus of the second syllable appeared more predictive than the lowest ranked ones. Given the hypothesised irrelevance of the control predictor \textit{gloss} (ranked 11/22), predictors with similar or lower ranking have little to no predictive value. Indeed, from the fifth ranked predictor onwards the variable importance values hardly vary (and yield 0) compared to higher ranked ones. \par

\begin{table}
\caption{Ultimate/penultimate stress ratio for the four most important predictors (italics) in the random forest analysis ($N$ = 1040).}
\label{tab49}
\begin{tabularx}{\textwidth}{p{1.3cm}rrrrr}
\lsptoprule
Nucleus & Height & \textit{Nucleus $\sigma$1} & \textit{Ht. nucl. $\sigma$1} & \textit{Ht. nucl. $\sigma$2} & \textit{Nucleus $\sigma$2}\\
\midrule
/a/ & open & 0 & 0 & 0.14 & 0.14\\
/\symbol{"025B}/ & mid & 0.63 & 0.43 & 0.05 & 0.06\\
/\symbol{"0254}/ & mid & 0 & & & 0.05\\
/i/ & close & 0.01 & 0.01 & 0.08 & 007\\
/u/ & close & 0.01 & & & 0.10\\
\lspbottomrule
\end{tabularx}
\end{table}

The highest ratio of ultimate stresses (0.63) was predicted by \textit{nucleus $\sigma$1} (Table \ref{tab49}). This appeared to be the result of /\symbol{"025B}/ causing stress to move from default penultimate to ultimate syllable (e.g., \textit{lemba} /l\symbol{"025B}m.ˈba/ `valley'). \textit{Height nucleus $\sigma$1} showed the highest ratio for mid vowels. This was again caused by only /\symbol{"025B}/, as /\symbol{"0254}/ was always stressed when it occurred in the first syllable (e.g., \textit{otak} /ˈ\symbol{"0254}.t\symbol{"0250}k/ `brain'). Lower predictive importance was found for \textit{height nucleus $\sigma$2} and \textit{nucleus $\sigma$2} (Table \ref{tab49}), indicating that only a small number of stress patterns could be predicted based on the nucleus in the second syllable. The largest ratio of ultimate stress cases was found when /a/ occurred in the second syllable, indicating that this vowel attracts stress in a small number of cases (see discussion below on the predictive strength of this factor). Note that /a/ is the only open vowel in Papuan Malay, explaining why the predictors \textit{height nucleus $\sigma$2} and \textit{nucleus $\sigma$2} had similar effects (Figure \ref{fig401}). The exact influence of vowel nuclei is further schematised in Table \ref{tab410}. This table shows that /\symbol{"025B}/ in the first syllable causes stress to shift to the ultimate syllable. A consistent exception to this rule, however, appears to be when the ultimate syllable already consists of either /\symbol{"025B}/ or /\symbol{"0254}/ (e.g., \textit{pendek} /ˈp\symbol{"025B}n.d\symbol{"025B}k/ `to be short'; \textit{belok} /ˈb\symbol{"025B}.l\symbol{"0254}k/ `to turn'). In these cases, the ultimate stress ratios are particularly low (Table \ref{tab410}). This result indicates that mid vowels in the second syllable generally prevent a stress shift.\par

\begin{table}
\caption{Proportion of ultimate stress cases ($N$ = 108) as a function of the identity of the vowel in the first (V1) and second (V2) syllable.}
\label{tab410}
\begin{tabularx}{0.9\textwidth}{Xrrrrrp{0.3cm}r}
\lsptoprule
V1 & \multicolumn{7}{c}{V2}\\
 \cmidrule{2-6}
& /i/ & /\symbol{"025B}/ & /a/ & /\symbol{"0254}/ & /u/ & & Total\\
\midrule
/i/ & 0 & 0 & 0 & 0.01 & 0 & & 0.01\\
/\symbol{"025B}/ & 0.12 & 0.04 & 0.60 & 0.02 & 0.19 & & 0.97\\
/a/ & 0 & 0 & 0 & 0 & 0 & & 0\\
/\symbol{"0254}/ & 0 & 0 & 0 & 0 & 0 & & 0\\
/u/ & 0 & 0 & 0 & 0.02 & 0 & & 0.02\\
\midrule
Total & 0.12 & 0.04 & 0.60 & 0.05 & 0.19 & & 1.00\\
\lspbottomrule
\end{tabularx}
\end{table}

However, none of the predictors found by the random forest predicted the stress distribution entirely. To analyze precisely how the different predictors work phonologically, three criteria were formulated (Table \ref{tab411}). First, ultimate stress is mainly found when /\symbol{"025B}/ occurred in the first syllable, confirming \citet{kluge_grammar_2017}. The three exceptions to this criterion are \textit{kitong} /ki.ˈt\symbol{"0254}\symbol{"014B}/ `we/us, including adressee', \textit{kumur} /ku.ˈmur/ `rinse mouth' and \textit{kuskus} /kus.ˈkus/ `cuscus', see also \citet[96]{kluge_grammar_2017}. Note that /ki.ˈt\symbol{"0254}\symbol{"014B}/ is short for /ki.ˈt\symbol{"0254}.ra\symbol{"014B}/ (\citealt[326]{kluge_grammar_2017}), which has penultimate stress (from /ˈki.ta/ and /ˈ\symbol{"0254}.ra\symbol{"014B}/, lit. `us humans'). Re-evaluation of /kus.ˈkus/ showed that it could be analyzed as a Malay loanword (\citealt{scott_malayan_1896}), indicating that its inclusion in the corpus might not have been justified.\par

\begin{table}
\caption{Word counts after applying the criterion that decreased the penultimate stress ratio/increased the ultimate stress ratio (Table \ref{tab48}).}
\label{tab411}
%\small
\begin{tabularx}{\textwidth}{p{4cm}XXX} 
\lsptoprule
Criterion & $N$ ˈpenult. & $N$ ˈult. & Exceptions\\
\midrule
Total & 932 & 108 & -\\
\midrule
\multirow{3}{*}{Nucleus $\sigma$1 = /\symbol{"025B}/} & \multirow{3}{*}{61} & \multirow{3}{*}{105} & /ki.ˈt\symbol{"0254}\symbol{"014B}/\\
& & & /ku.ˈmur/\\
& & & /kus.ˈkus/\\
\midrule
\multirow{5}{*}{Height nucleus $\sigma$2 $\neq$ mid} & \multirow{5}{*}{25} & \multirow{5}{*}{100} & /\symbol{"02A7}\symbol{"025B}.ˈr\symbol{"025B}j/\\
& & & /s\symbol{"025B}.ˈr\symbol{"025B}j/\\
& & & /\symbol{"02A4}\symbol{"025B}.ˈl\symbol{"025B}k/\\
& & & /\symbol{"02A4}\symbol{"025B}m.ˈp\symbol{"0254}l/\\
& & & /s\symbol{"025B}.ˈd\symbol{"0254}t/\\
\midrule
Nucleus $\sigma$2 = /a/ & 16 & 65 & ...\\
\lspbottomrule
\end{tabularx}
\end{table}

Second, 61 words had /\symbol{"025B}/ in the first syllable and penultimate stress. From these words, 36 had a mid vowel (/\symbol{"025B}/ or /\symbol{"0254}/) in the second syllable. Five exceptions to this criterion had ultimate stress, with /\symbol{"025B}/ in the first syllable and a mid vowel in the second syllable; /\symbol{"02A7}\symbol{"025B}.ˈr\symbol{"025B}j/ `to divorce', /s\symbol{"025B}.ˈr\symbol{"025B}j/ `lemongrass', /\symbol{"02A4}\symbol{"025B}.ˈl\symbol{"025B}k/ `be bad', /\symbol{"02A4}\symbol{"025B}m.ˈp\symbol{"0254}l/ `thumb', /s\symbol{"025B}.ˈd\symbol{"0254}t/ `to suck'. Note that [\symbol{"025B}j] in /\symbol{"02A7}\symbol{"025B}.ˈr\symbol{"025B}j/ and /s\symbol{"025B}.ˈr\symbol{"025B}j/ is analyzed as the realisation of underlying /aj/ due to the liquid in the onset of the second syllable (\citealt[84]{kluge_grammar_2017}). With /\symbol{"025B}/ in the first syllable, underlying /a/ could make the second syllable the preferred location for stress. The status as native root of two more words is doubtful. That is, /\symbol{"02A4}\symbol{"025B}m.ˈp\symbol{"0254}l/ is reported as a loanword from Javanese/Sundanese (\citealt{haspelmath_world_2009}) and /s\symbol{"025B}.ˈd\symbol{"0254}t/ is reported as a loanword from Javanese (\citealt{stevens_comprehensive_2010}).\par

Third, the presence of an open vowel (/a/) in the second syllable increases the likelihood of ultimate stress. However, from the words with /\symbol{"025B}/ in the first syllable and /a/ in the second syllable, only 65 had ultimate stress. Given that there were 108 ultimate stress cases in total (Table \ref{tab411}), the open vowel in the second syllable did not predict stress placement as strongly as the first two criteria. In other words, the open vowel in the second syllable was of minor importance and could only explain a small additional number of stress cases after the main criteria were applied. This result is reflected in the large variable importance difference between the first two predictors and the lower ranked ones (Figure \ref{fig401}). For this reason, no more criteria were formulated.


\subsection{Conclusion}
The results are best summarised by assuming that the default position of word stress in Papuan Malay is the penultimate syllable and that mid vowels generally reject stress. It is crucial, however, to note the difference in stress rejection between /\symbol{"025B}/ and /\symbol{"0254}/. When the first syllable contains /\symbol{"025B}/, stress shifts to the ultimate syllable when the ultimate syllable does not contain a mid vowel. This result indicates that /\symbol{"025B}/ rejects stress regardless of syllable position, although in 25 words the first syllable was stressed and contained /\symbol{"025B}/ while no mid vowel was found in the second syllable. The occurrence of /\symbol{"0254}/ in the first syllable did not cause a stress shift; all words with /\symbol{"0254}/ in the first syllable and no mid vowel in the second syllable ($N$ = 29) had penultimate stress. However, with /\symbol{"0254}/ in the second syllable only three out of 66 words had ultimate stress (/ki.ˈt\symbol{"0254}\symbol{"014B}/, /\symbol{"02A4}\symbol{"025B}m.ˈp\symbol{"0254}l/, /s\symbol{"025B}.ˈd\symbol{"0254}t/), which could all be explained as phonological exceptions or loanwords. Thus, /\symbol{"0254}/ only rejects stress in the second syllable. This difference could be an indication of a stress hierarchy within the vowel inventory of Papuan Malay. In this hierarchy, corner vowels /i/, /a/ and /u/ can be stressed regardless of position, /\symbol{"0254}/ can only be regularly stressed (i.e., penultimate), whereas /\symbol{"025B}/ should be avoided as stressed regardless of position. In addition to the mid vowels, /a/ was found to attract stress to a limited extent, although it did not predict a stress shift.\par

The results are in line with the literature on Trade Malay with respect to the role of /\symbol{"025B}/ (schwa) in stress placement (\citealt{kluge_grammar_2017}; \citealt{paauw_malay_2009}). Furthermore, the role of /a/ as stress attractor is compatible with phonological accounts that distinguish open and close vowels as more and less sonorous respectively (\citealt{kenstowicz_quality-sensitive_1997}; \citealt{selkirk_major_1984}). Note, however, that the infrequently stressed mid vowels in Papuan Malay cannot be explained on the basis of openness as main correlate of vowel sonority. This would mean that /i/ and /u/ are poorer candidates for stress than /\symbol{"025B}/ and /\symbol{"0254}/, which is incompatible with the observations of this study. The results rather indicate a difference in phonological status between corner vowels and mid vowels in the Papuan Malay inventory.\par

The analysis has shown that random forests provide an insightful analysis of which phonological factors play a role in stress placement. It is worth emphasizing that without the complementary distribution analysis (Tables \ref{tab49}, \ref{tab410}, and \ref{tab411}), the role of the predictors would have been difficult to interpret. Moreover, the direction of the effect of the most powerful predictors in the random forest analysis could be understood when interpreting the stress ratios. The predictive power of the random forest model is particularly clear from the relatively small number of exceptions with ultimate stress ($N$ = 8) after applying the first two criteria in Table \ref{tab411}. In fact, the analysis revealed that three of these words were loanwords, which should not have been included in the corpus. For another three words alternative explanations could be found, indicating that their stress pattern did not violate the phonological criteria (Section \ref{sec433}). As for the exceptions with penultimate stress after applying the first two criteria listed in Table \ref{tab411} ($N$ = 25), we cannot provide alternative explanations or additional criteria that explain why stress did not move to the ultimate syllable in these cases. Nevertheless, 25 of the 932 penultimate stress cases and 8 of the 108 ultimate stress cases constitute less than four percent of all words in the corpus. Additional phonological criteria could theoretically be derived from the remaining highest ranked predictors in the random forest analysis, although these have the risk of generating more exceptions than explained cases (Table \ref{tab411}, bottom row).

\section{Discussion and conclusion} \label{sec44}
This study has shown that non-acoustic distribution analyses complement existing acoustic research on stress in important ways. The primary aim for Papuan Malay was to investigate whether stress patterns could facilitate word disambiguation. Results showed support for this hypothesis. It has to be noted, however, that the total number of embeddings found in this study make up less than 15\% of the word list. This means that for maximally twice that percentage (roughly all embeddings plus their carrier) there is a need for stress-based disambiguation. Thus, the segmental information is sufficient to recognise the majority ($>$70\%) of the Papuan Malay words. The relative frequencies of embeddings in the other languages discussed here are unavailable. However, for these languages too it is expected that prosodic cues are of secondary importance for word recognition compared to segmental cues.\par

Although it remains to be tested perceptually how vowel quality affects stress judgments in Papuan Malay (Section \ref{sec41}), the random forest analysis was able to shed new light on the role of /\symbol{"025B}/ (Section \ref{sec43}). This analysis confirmed that /\symbol{"025B}/ in Papuan Malay indeed determines most of the patterns by rejecting stress. A crucial additional insight concerned /\symbol{"0254}/, the other mid vowel in the inventory. It was shown that this vowel rejects stress in ultimate position, which indicates that the role of /\symbol{"025B}/ in Papuan Malay stress might not be so unique as acoustic results have suggested (see Section \ref{sec41}). Rather, stress placement seems to follow phonological principles such as weight sensitivity, based on the sonority hierarchy, with an important distinction between corner vowels and mid vowels. In this respect it is worth noting that the mid vowels were the least frequent vowels in the Papuan Malay syllables analyzed in Chapter \ref{chAc}. In addition, shifted stress (to ultimate position) only occurred for the three corner vowels (Chapter \ref{chAc}: Table 2a). It seems likely, therefore, that the phonological class of mid vowels are avoided as stressed nuclei, if possible. Stressed mid vowels would appear only when the segmental structure offers no alternative, i.e., when there are only mid vowels in the word. These infrequent cases might therefore explain why the acoustic realization of stress is particularly weak or atypical (Section \ref{sec412}). The acoustic effects of stress on mid vowels thus seem to mainly reflect their weak position in the sound inventory of Papuan Malay, with differences among them in the way they reject stress.\par

Concerning the controversy of word stress claims in Indonesian languages, this study has provided new data in support of the stress claim for Papuan Malay. Given the acoustic and perceptual support (Chapter \ref{chAc}, Chapter \ref{chPerc}) and the current results, it seems that, with respect to the disambiguation function, word stress in Papuan Malay is at least as functional as in English, which is uncontroversially a stress-language. In addition, the outcomes strongly suggest that Papuan Malay stress placement follows phonological principles that apply to the syllable structure of words (i.e., weight). This limits the options to maintain the claim that Papuan Malay is a language without word stress, or to interpret word prosodic patterns as reflexes of phrase prosody (e.g., \citealt{vanzanten_stress_2010}), although it is likely that both levels interact (\citealt{kaland_demarcating_2020}). In this respect, Papuan Malay appears different from Ambonese Malay, which has been claimed not to have word stress at all (\citealt{maskikit-essed_no_2016}). The present study therefore reconfirms that regional variation, even among closely related languages, is an important factor (\citealt{goedemans_stress_2007}). Note, however, that the type of analysis of Papuan Malay and Ambonese Malay fundamentally differs, plausibly affecting the conclusions drawn so far (see Section \ref{sec412} for a discussion). A major difference concerns the availability of several acoustic, perceptual and (in this chapter) lexical analyses for Papuan Malay, whereas stress in Ambonese Malay has only been investigated in a small number of studies and with an acoustic analysis of data from a small number of speakers. For a more thorough comparison, stress in other Trade Malay varieties, including Ambonese Malay, should be studied using a variety of empirical methods that crucially go beyond acoustic investigations and the presence of minimal pairs, as aimed for in the current chapter.\par

The outcomes of this study alone do not allow an answer regarding whether Papuan Malay has word stress or not, although the results are predominantly compatible with an affirmative answer. Rather, this study primarily provided complementary insights into the lexical status of Papuan Malay word prosody. More research is needed and a plausible outcome is that the nature of word stress found in this language crucially differs from the one found in other languages, as hinted at in the comparisons between Papuan Malay, Spanish and English discussed in the previous sections. For example, it should be investigated perceptually whether native listeners can indeed use acoustic cues to make lexical decisions. Answering these types of questions is left for future work.
