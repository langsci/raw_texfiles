\title{One-to-many relations in morphology, syntax, and semantics}
\BackBody{The standard view of the form-meaning interfaces, as embraced by the great majority of contemporary grammatical frameworks, consists in the assumption that meaning can be associated with grammatical form in a one-to-one correspondence. Under this view, composition is quite straightforward, involving concatenation of form, paired with functional application in meaning. 
In this book, we will discuss linguistic phenomena across several grammatical sub-modules (morphology, syntax, semantics) that apparently pose a problem to the standard view, mapping out the potential for deviation from the ideal of one-to-one correspondences, and develop formal accounts of the range of phenomena. We will argue that a constraint-based perspective is particularly apt to accommodate deviations from one-to-many correspondences, as it allows us to impose constraints on full structures (such as a complete word or the interpretation of a full sentence) instead of deriving such structures step by step.}
%\dedication{Change dedication in localmetadata.tex}
\typesetter{Berthold Crysmann, Felix Kopecky}
\proofreader{Alexander Rice,
Alexandr Rosen,
Amir Ghorbanpour,
Brett Reynolds,
Christian Döhler,
Conor Pyle,
James Tauber,
Janina Radó,
Jeroen van de Weijer,
Katja Politt,
Lachlan Mackenzie,
Sauvane Agnès,
Steven Kaye,
Tom Bossuyt}
\author{Berthold Crysmann and Manfred Sailer}
\BookDOI{10.5281/zenodo.4638824}%ask coordinator for DOI
\renewcommand{\lsISBNdigital}{978-3-96110-307-2}
\renewcommand{\lsISBNhardcover}{978-3-98554-003-7}
\renewcommand{\lsSeries}{eotms} % use lowercase acronym, e.g. sidl, eotms, tgdi
\renewcommand{\lsSeriesNumber}{7} %will be assigned when the book enters the proofreading stage
\renewcommand{\lsID}{262}

\SpineAuthor{Crysmann \& Sailer}

\lsCoverTitleSizes{40pt}{14.75mm}
