\chapter{Experiment}\label{ch:2}

This chapter presents an experimental study of Korean-English and Japanese-English \ac{CS} with three inter-related subparts, eliciting judgment data that bear on the two research questions outlined in Chapter 1. Assuming that parametric variation, such as word order, is determined by feature specifications of a functional category as assumed in the Minimalist Program, the study asked how different functional categories in typologically different languages play a role in deriving various word orders in \ac{CS}. More specifically, the role of light or functional verbs was investigated in comparison with the role of heavy or lexical verbs in different types of code-switched phrases, especially with respect to their contribution to \ac{OV} and \ac{VO} order variation in Korean-English and Japanese-English \ac{CS}. 

The study also investigated whether the syntactic flexibility of a code-switched phrase plays a role in word order in \ac{CS}. Specifically, the syntactic flexibility of an idiomatic phrase in English was tested against the hypothesis that syntactically flexible and less flexible phrases would exhibit different patterns in \ac{CS}: a flexible phrase is subject to \ac{CS} whereas less flexible or inflexible phrases may not undergo \ac{CS} and maintain the internal order of the phrase throughout the derivation. 

Data for the quantitative analyses were obtained via (i) a \ac{CS} judgement task (\sectref{ch2:sect:2.1}), (ii) a syntactic flexibility judgment task (\sectref{ch2:sect:2.2}), and (iii) an idiom familiarity task (\sectref{ch2:sect:2.3}). The study aimed to elicit evidence to shed light on the role of light verbs and syntactic flexibility in determining \ac{OV}-\ac{VO} variation in \ac{CS} where an English \ac{VP} is incorporated into utterances in Korean or Japanese. The evidence comprises acceptability judgments elicited using contextually appropriate materials from Korean-English and Japanese-English bilingual speakers, whose competence and experience in their two languages made them familiar with \ac{CS} behavior. The participant populations of interest are exemplified by the bilingual communities of the New York City area, where daily use of each of the speakers’ languages is common, as is switching between languages within a conversation (cf the work of \citealt{Chung2012}). The study, therefore, made crucial use of a language-history questionnaire and an exit interview probing experience in \ac{CS} in order to screen participants recruited from these communities. All participants gave their informed consent for inclusion before they participated in the study. 

\section{Code-switching judgment task}\label{ch2:sect:2.1}

To elicit judgments on \ac{OV} vs \ac{VO} ordered code-switched sentences between Korean and English and between Japanese and English, a two-alternative forced choice task was used in this experiment. For each of a series of items, participants were asked to select between two utterances that were considered as a (near-)minimal pair.\textsuperscript{} Both utterances included an English-sourced \ac{VP} followed by the Korean light verb \textit{ha} or the Japanese light verb \textit{su}, and the \ac{VP} that is code-switched into English was presented in \ac{OV} order in one utterance and \ac{VO} order in the other. The participant’s task was to select the utterance that “sounded (more) natural” between the two sentences.

The rationale to use a 2-alternative forced choice method over a Likert scale method, which is more commonly used to elicit judgment/acceptability of test items, was based on the results from a pilot study, which suggested that the acceptance rate of a code-switched sentence may be influenced by other factors (e.g. lexical choice) than the \ac{OV}-\ac{VO} order contrast within a code-switched constituent, which the present study aims to investigate. Therefore, a 2-alternative forced choice task was considered more appropriate than a Likert scale method in order to elicit a bilingual speaker’s judgment on \ac{OV}-\ac{VO} order variation in \ac{CS} while minimizing the potential influence of other factors in his/her judgment.\footnote{\textrm{While an anonymous reviewer disapproves of the use of a two-alternative forced choice task to elicit acceptability considering both choices could be acceptable or unacceptable, a two-alternative forced choice task is proven to be suitable to investigate \ac{CS} competence by means of acceptability judgments, which provides granular details that remain invisible in a Likert scale experiment (\citealt{Stadthagenetal2018}).}} 

Due to the fact that many of the critical items included English \ac{VP} idioms and light verb constructions, the protocol was designed to provide strong contextual support of the intended interpretation. Each item presentation, therefore, had three parts:

\begin{exe}
\exi{(a)} A short scenario introduction, mentioning two standard characters (Kibo and Donna) to establish a discourse context. This introduction material was always presented, in written form, in English, and always closed by asking what Donna would say in the situation sketched.

\exi{(b)} A cartoon depicting the content of Donna’s statement. This was presented in an advance of the statement and remained visible while two versions of that statement were heard.

\exi{(c)}The code-switched pair of utterances, presented in spoken form.
\end{exe}

By presenting each code-switched sentence not only in an appropriate context but also with a matching cartoon, the intended meaning of the code-switched phrase in a sentence, whether literal or non-literal, was successfully delivered without ambiguity. Instructions emphasized that the participants should attend to the cartoon while they were listening to the sentences. As an illustration of this protocol, \REF{ex:35} below offers an example scenario introduction, followed by a cartoon describing the content of Donna’s statement and a Korean-English \ac{CS} pair between which the participant was asked to choose. 

\ea\label{ex:35} 
Kibo told Donna that his roommate had an extra iPod to give away, and later asked Donna whether she called and got it. What does Donna say?
\sn \begin{center} \includegraphics[height = 5cm]{figures/Donnamissedtheboat.png} \end{center}
\ea \gll nwu-ka mence cenhwahayse, \textbf{\textit{boat-lul} \textit{miss}} hayss-e \\
someone-\textsc{nom} before call.because {~~~~~~~~}-\textsc{acc} \textsc{do.past-decl} \\
\ex  \gll nwu-ka mence cenhwahayse \textbf{\textit{miss the boat}} hayss-e \\
someone-\textsc{nom} before call.because  {} \textsc{do.past-decl} \\
\glt `Someone else called first, so (I) missed the boat.'
\z
\z

\subsection{Materials and methods}\label{ch2:sect:2.1.1}

To assess the role of heavy vs light verb status in \ac{CS} in both literal and non-literal phrases, materials were constructed in accordance with a 2 × 2 factorial design, combining Verb Type (heavy vs light) and Interpretation (literal vs non-literal). For light verbs, those listed in \REF{ex:36}, which are \textit{have}, \textit{give}, \textit{get}, \textit{take}, \textit{make}, \textit{keep}, \textit{hold} and \textit{raise}, were included.

\ea\label{ex:36}
    \ea Heavy verb, literal interpretation   \hfill  e.g. \textit{miss} \textit{the} \textit{bus}
    \ex Heavy verb, non-literal interpretation \hfill e.g. \textit{miss} \textit{the} \textit{boat}
    \ex Light verb, literal interpretation  \hfill e.g. \textit{have} \textit{a} \textit{small} \textit{head}
    \ex  Light verb, non-literal interpretation \hfill e.g. \textit{have} \textit{a} \textit{big} \textit{mouth}
    \z
\z

For each verb within a given type, items instantiating literal vs non-literal interpretations were constructed as a closely matched pair, with only those changes necessary made to the introductory scenario, the interpretation-supporting cartoon, and the code-switched sentences between which a choice was to be made. There were 16 such matched items constructed for each verb type, which were distributed to form an experiment in two versions, each with 8 items per condition. The counterbalancing of items across versions meant that any participant saw examples of each of the experimental conditions, without repetition of lexical or discourse context. \REF{ex:37} exemplifies the closely matched pair to \REF{ex:35}, both of which include the same heavy verb \textit{miss} in a non-literal or idiomatic phrase (e.g. \textit{miss} \textit{the} \textit{boat}) and a literal phrase (e.g. \textit{miss} \textit{the} \textit{bus}), respectively.

\ea\label{ex:37}
Kibo was disappointed that Donna didn’t show up at the party he had told her about, and asked her what happened. What does Donna say?

\sn \begin{center} \includegraphics[height = 5cm]{figures/Donnamissedthebus.png} \end{center}

    \ea \gll cengmal ka-ko siph-ess-nuntey, \textbf{\textit{bus-lul}} \textbf{\textit{miss}} hayss-e \\
    really go-\textsc{comp} wish-\textsc{past}-but {~~~~~}-\textsc{acc} {} \textsc{do.past-decl} \\
    \ex \gll cengmal ka-ko siph-ess-nuntey, \textbf{\textit{miss the bus}} hayss-e\\
    really go-\textsc{comp} wish-\textsc{past}-but {} \textsc{do.past-decl} \\
    \glt  `(I) really wanted to go, but (I) missed the bus.'
    \z
\z
       

The function of heavy vs light verbs in \ac{CS} was also evaluated in light verb constructions, and materials were constructed in accordance with a 2 × 1 factorial design, combining Verb Type (heavy vs light) and Construction (\acl{LVC}), as illustrated in \REF{ex:38}.

\begin{exe}\ex\label{ex:38}
    \begin{xlist}\ex  Heavy verb, light verb construction   \hfill  e.g. \textit{pay} \textit{a} \textit{visit/a} \textit{compliment}
    \ex  Light verb, light verb construction \hfill   e.g. \textit{have} \textit{a} \textit{look/have} \textit{a} \textit{try}
    \end{xlist}
\end{exe}

For each construction, 16 items were constructed with 8 verbs. For each verb, two similar items were constructed as a closely matched pair and distributed to form an experiment in two versions, each with 8 items per condition. This allowed the participant taking either version of the experiment to see the same set of verbs for each of the experimental conditions, in closely matched contents. For both the Korean and Japanese versions of the experiment, materials were constructed so as to produce two subsets, each of which included 48 critical items and 24 filler items. An additional set of 10 items, ranging across item types, was constructed to provide practice without the task procedure. 

\largerpage[2]
\subsection{Item implementation}\label{ch2:sect:2.1.2}

As described in \REF{ex:36} and \REF{ex:38}, the object within the code-switched phrase included various types of nouns, such as an indefinite noun (e.g. \textit{some} \textit{money,} \textit{an} \textit{ear}), a definite noun (e.g. \textit{the} \textit{boat}, \textit{the} \textit{stairs}), and a bare noun (e.g. \textit{basketball}, \textit{cold} \textit{feet}). The definiteness and length of the objects were controlled for in all the closely matched items throughout the conditions to prevent various linguistic and non-linguistic factors, such as definiteness, from influencing the selection of a particular order, either \ac{OV} or \ac{VO}. All code-switched \ac{VP}s of English were constructed in two orders, \ac{OV} and \ac{VO}, and followed by either the Korean light verb \textit{ha} ‘do’ or the Japanese light verb \textit{su} ‘do’. The final form of the code-switched sentence in \ac{OV} order was slightly different from that in \ac{VO} order. 


First, various function words preceding the object, such as determiners (\textit{a}(\textit{n}), \textit{the}), possessors (\textit{my}, \textit{your}, \textit{his}, \textit{her}, etc), were omitted from the \ac{OV} order sentences, based on the results of a pilot study that revealed that many Korean-English and Japanese-English bilingual speakers judged the code-switched sentences as sounding more natural without the determiner in the \ac{OV} order but not in the \ac{VO} order. Though the exact reason still needs to be investigated, it is speculated that since both Korean and Japanese lack determiners, determiner omission also seems to be preferred, following the grammar of Korean or Japanese, when the code-switched sentence preserves the structure of Korean or Japanese, exhibiting their canonical \ac{OV} order.\footnote{One exception was the phrase \textit{spill the beans}, \textrm{where the determiner is indispensable in order for the phrase to deliver its idiomatic reading, ‘to reveal a secret’: there is no grammatical constraint imposed on the presence of the determiner in} \textrm{\textit{spill} \textit{the} \textit{beans}}\textrm{; a plural noun can occur without an article in English. Thus, while the phrase} \textrm{\textit{spill} \textit{beans}} \textrm{is grammatical, it does not denote the idiomatic interpretation that is obtained from} \textrm{\textit{spill} \textit{the} \textit{beans}}. Thus, the definite article} \textrm{\textit{The}} \textrm{in} \textrm{\textit{spill}} \textrm{\textit{the} \textit{beans}} \textrm{was kept both in \ac{OV} and \ac{OV} orders.}
\clearpage

Besides determiners, English-type pronominal possessors also do not exist in either Korean or Japanese. Instead, a possessive pronoun is expressed as a phrasal form in which a personal pronoun or a noun takes a genitive suffix realized as -\textit{uy} in Korean and -\textit{no} in Japanese, respectively. Thus, English-type pronominal possessors were also omitted when the code-switched phrase was constructed in \ac{OV} order. On the other hand, all other noun-modifying elements such as lexical adjectives (\textit{i.e.}, \textit{big} in \textit{a} \textit{big} \textit{present}) and quantifiers (i.e. \textit{a} \textit{few} in \textit{a} \textit{few} \textit{brows}) were kept intact in both \ac{OV} and \ac{VO} orders of the code-switched sentences.

In Korean and Japanese, the subject is marked with the subject marker (the normative Case), \textit{i} or \textit{ka} in Korean and \textit{ga} in Japanese, and the object is indicated with the object marker (the accusative Case), \textit{{}-(l)ul} in Korean and \textit{{}-o} in Japanese.\footnote{\textrm{These case markers may be dropped in colloquial speech.}} Following the grammars of Korean and Japanese, the accusative Case marker, \textit{{}-(l)ul} in Korean and \textit{{}-o} in Japanese, was inserted after the object when a code-switched phase was constructed in \ac{OV} order, but not in \ac{VO} order, as exemplified in \REF{ex:35} and \REF{ex:37}.\footnote{Korean-English and Japanese-English \ac{CS} data in the literature show that the accusative case marker appears only in the \ac{OV} order, but not in \ac{VO} order. Case and case markers will be discussed further in Chapter \ref{ch:6}}. It has been noted that overt morphological accusative markers, such as Korean \textit{{}-(l)ul} and Japanese \textit{{}-o}, may be used as focus particles, hence the presence of the overt accusative marker on the object in \ac{OV} order, but not in \ac{VO} order, may cause the participants to choose the focused element in \ac{OV} order. However, the results showed that \ac{VO} order was strongly preferred over \ac{OV} order in certain contexts, despite the systematic presence of the accusative marker. This suggests that the presence of the overt accusative marker in the code-switched phrase in \ac{OV} order does not seem to lead to a word order bias towards \ac{OV} in Korean-English and Japanese-English \ac{CS}. 

All the materials were voice-recorded by a fluent Korean-English bilingual speaker (for the Korean-English \ac{CS} task) and a Japanese-English bilingual speaker (for the Japanese-English \ac{CS} task). 

\subsection{Procedure}\label{ch2:sect:2.1.3}

The experiment was conducted via a Power Point slide show. At the beginning of the experiment, participants received instructions for the task before the trial began. The trial session included 6 practice questions (out of 10 in total) in order for the participants to familiarize themselves with the task. After half of the test items were presented, the participants were encouraged, but not forced, to take a five-minute break. The experiment either continued or resumed, depending on the needs of each participant. After the break, 4 additional practice items were presented prior to the rest of test materials, allowing the participants to recover their own performance pace after taking a short break. However, participants were not told which were practice items and which were test items.

As described earlier, each question consisted of three subparts framed in two frames; the short scenario introduction (Frame 1) and the interpretation-sup\-por\-ting cartoon and the code-switched pair of utterances (Frame 2). The participants heard each utterance only once, and the pace of the experiment was pre-program\-med by the investigator. The duration of each and all frames was preset, and the transition from Frame 1 to 2 was automatic. However, the transition from one question to another (Frame 2 to next Frame 1) was fully controlled by the participants, allowing them to take as much time as needed for answering each question. The next question began only when the participant clicked the mouse.

A separate answer sheet was provided, consisting of an abbreviated version of the scenario, a reduced size of the cartoon, and two checkboxes numbered 1 and 2, corresponding to each code-switched sentence the participant heard. No time limit was imposed on the task, and the participants, on average, took 30 minutes to complete the experiment.


\subsection{Participants}\label{ch2:sect:2.1.4} 

A total of 28 Korean-English bilingual speakers (age range 18--27; mean age 21.1; 19 female) and 8 Japanese-English bilingual speakers (age range 25--38; mean age 31.9; 7 female) successfully completed the experiment. The onset of acquisition of Korean and  English were 0.9 and 4.6 years old, respectively. The onset of learning Japanese and that of English were 0 and 5.3 years old, respectively.

Initially 34 Korean-English and 12 Japanese-English bilingual speakers in total participated in the experiment, but six Korean-English and four Japanese-English participants were excluded from the data analysis after three screening procedures. First, four Korean-English and two Japanese-English bilingual participants were excluded because the information provided in the language history questionnaire revealed that they had begun learning one of their two languages, either English or Korean/Japanese, after the age of 12, suggesting they were not early bilinguals, the target population that is the focus of this study. The rationale for including only early bilingual speakers was based on the fact that code-switching is a property of highly proficient bilinguals, and the delayed acquisition of one of their languages indicates that the speaker may have not reached native or near-native proficiency for one of the languages they speak. 

Second, one Korean-English bilingual speaker was excluded due to the high error rate in her performance (25\%) in filler type I, where she chose \ac{OV} rather than \ac{VO} as the natural order of a full English sentence, in 4 out of 16 trials. It turned out that this participant also failed to meet the criteria for inclusion based on the information given in the language history questionnaire. In addition, two Korean-English and two Japanese-English bilingual participants were excluded because they exhibited a strongly biased word order preference for the code-switched sentences, either \ac{OV} or \ac{VO}, regardless of item types. Except for one Korean-English bilingual who chose \ac{OV} order in 77\% of his trials, the other three participants who were excluded selected \ac{VO} order predominantly, in 79\% to 95\% of their responses. Excluding these four participants, the average distribution of \ac{VO} vs \ac{OV} word order preference of the 28 Korean-English and 8 Japanese-English bilingual speakers in the study was 47.4\% vs 52.6\% and 56.7\% vs 43.3\%, respectively.

In sum, three Korean-English bilinguals and two Japanese-English bilinguals who took Version A of the experiment, and three Korean-English bilinguals and two Japanese-English bilinguals who took Version B of the experiment were excluded from subsequent data analysis.

\subsection{Results} \label{ch2:sect:2.1.5}

\subsubsection{Korean-English code-switching}\label{ch2:sect:2.1.5.1} 

\figref{fig:2.1} graphically shows the percentage of \ac{VO} order preference by Verb Type (heavy vs light) and Interpretation (literal vs non-literal) in Korean-English \ac{CS}.

\begin{figure}
% \includegraphics[width=\textwidth]{figures/Figure 2.1.png}
\includegraphics[width=\textwidth]{figures/2-1.pdf}
 \caption{\%\textsc{vo} preference by Verb Type and Interpretation} 
\label{fig:2.1}
\end{figure}

In the overall analysis, a main effect of verb type (heavy vs light verbs) was found. The preference of \ac{VO} order of the code-switched phrase involving a heavy vs a light verb was 44.0\% and 57.4\%, respectively (F\textsubscript{1}(1,26) = 25.49, p<.001; F\textsubscript{2}(1,28) = 3.07, .05<p<.10). A main effect of interpretation (literal vs non-literal) was also found. The preference of \ac{VO} order of the code-switched phrase in literal vs non-literal interpretations was 33.9\% and 67.4\% (F\textsubscript{1}(1,26) = 25.49, p<.001; F\textsubscript{2}(1,28) = 52.35, p<.001). These results replicate the previous findings in \citet{Shim2011} that both verb types and interpretations play a role in \ac{OV}-\ac{VO} variation in \ac{CS}. In addition, an interaction between Verb Type and Interpretation was found (F\textsubscript{1}(1,26) = 8.95, p<.01; F\textsubscript{2}(1,28) = 5.36, p<.05).\footnote{The source of the interaction between Verb Type and Interpretation is not clear. It could be a ceiling effect of the test, revealing that the percentages of \ac{VO} preference have reached maximum points in non-literal interpretations for both heavy and light verbs. Alternatively, the interaction could be explained by the fact that not all verbs in Chapter \ref{ch:1} \REF{ex:28}, which were initially categorized as English light verbs in this study, behave the same way as the analyses of individual items reveal that a subset of these verbs (e.g. \textit{hold}) or their uses follow the pattern of heavy verbs with respect to word order preference in Korean-English and Japanese-English \ac{CS} data, showing that \ac{OV} was strongly preferred. Thus, an inclusion of heavy verbs or heavy uses of light verbs in the group of light verbs may have resulted in a lower percentage of \ac{VO} preference, as observed in the interaction between heavy and light verbs in non-literal interpretations. A detailed discussion of individual light verbs in English will follow in \sectref{ch5:sect:5.7}.}

The sub-analyses revealed that with heavy verbs, the percentage of \ac{VO} order preference was significantly lower in literal interpretation than in non-literal interpretation, 21.9\% vs 66.1\% (F\textsubscript{1}(1,26) = 86.73, p<.001; F\textsubscript{2}(1,14) = 41.88, p<.001). A similar pattern emerged with light verbs; the percentage of \ac{VO} order preference was significantly lower in literal interpretation than in non-literal interpretation, 46.0\% vs 68.8\% (F\textsubscript{1}(1,26)=25.21, p<.001; F\textsubscript{2}(1,14)=13.29, p<.005). The difference between heavy and light verbs was also found in light verb constructions, presented in \figref{fig:2.2}.

\begin{figure}
% \includegraphics[width=\textwidth]{figures/Figure-2.2.png}
\includegraphics[width=\textwidth]{figures/2-2.pdf}
\caption{\%\ac{VO} preference by Verb Type and Construction}
\label{fig:2.2}
\end{figure}

A one-way \textsc{anova} analysis of heavy vs light verbs in light verb constructions showed that while the preference of \ac{VO} order was 40.6\% with heavy verbs, it was 63.8\% with light verbs (F\textsubscript{1}(1,26)=29.10, p<.001; F\textsubscript{2}(1,14)=6.51, p<.025).

With the filler materials including heavy vs light uses of verbs, the preferred word order seemed to differ by the heavy use of the verb and the light use of the verb (54.5\% \ac{VO} vs 41.1\% \ac{VO}). However, this result was not statistically confirmed due to low item power (F\textsubscript{1}(1,26)=5.70, p<.025; F\textsubscript{2}(1,6)=1.43, p>.25). With the filler materials with direct object vs indirect objects, the difference of preferred word order was smaller between the verb with the direct object (e.g. \textit{pay} \textit{the} \textit{gas} \textit{bill}) and the verb with the indirect object (e.g. \textit{pay} \textit{the} \textit{gardener}), exhibiting 20.5\% \ac{VO} and 27.7\% \ac{VO} occurrence, respectively. Thus, the prediction that the case mismatch found in the triadic verb with the indirect object between English (accusative case) and Korean (dative case) would result in \ac{VO} order was not borne out. However, it was not confirmed at a statistically significant level either, again due to the low number of items included in the experiment (F\textsubscript{1}(1,26)=1.30, p>.25; F\textsubscript{2}<1). 


\subsubsection{Japanese-English code-switching}\label{ch2:sect:2.1.5.2}

In general, the results from the Japanese-English \ac{CS} data were not supported at a statistically significant level due to the small number of subjects. Nonetheless, a clearly emerging pattern was found between Korean-English and Japanese-English \ac{CS}, which is illustrated in \figref{fig:2.3}.

\begin{figure}
\includegraphics[width=\textwidth]{figures/2-3.pdf}
\caption{\%\textsc{vo} preference, overall, as a function of Phrase Type and Speaker Group} 
\label{fig:2.3}
\end{figure}

Similar to the results obtained from Korean-English \ac{CS}, the overall results from the \ac{CS} judgment experiment with Japanese-English bilinguals revealed that the preference of \ac{VO} order was higher with light verbs than heavy verbs both in literal interpretations and light verb constructions (e.g. \ac{HV}, Lit vs. \ac{LV}, Lit and \ac{HV}, \ac{LVC} vs. \ac{LV}, \ac{LVC}), while this difference disappeared in non-literal interpretations, in which \ac{VO} order was strongly preferred regardless of verb type (e.g. \ac{HV}, Non-Lit and \ac{LV}, Non-Lit).

In sum, the overall pattern of results found in the \ac{CS} judgment task provides evidence to support the hypothesis that the selection between English light verbs and heavy verbs within a code-switched phrase would lead to word order variation in \ac{CS}: while light verbs lead to \ac{VO} order, heavy verbs derive \ac{OV} order. Yet, this difference was only observed in the \acp{VP} with literal interpretations and in light verb constructions, not in non-literal or idiomatic interpretations, to which I will return in Chapter \ref{ch:5}. 

The present findings are in accordance with the results reported in \citet{Shim2011} that both the selection of heavy vs light verbs within a code-switched constituent and the idiomaticity of the code-switched phrase play a role in deriving word order in \ac{CS}. 

\section{Syntactic flexibility judgment task}\label{ch2:sect:2.2}

The results from the \ac{CS} judgment task showed that the distinction between heavy and light verbs did not make a difference in non-literal interpretations, both of which were strongly preferred in \ac{VO} order in Korean-English and Japanese-English \ac{CS}. Yet, a microscopic analysis of each code-switched phrase in non-literal or idiomatic interpretations revealed that variation still exists among them, suggesting that not all idioms behave the same way. Under the assumption that the internal argument of the syntactically flexible phrase is subject to \ac{CS}, while the internal argument of the less flexible or inflexible phrase may not undergo \ac{CS}, a syntactic flexibility judgment task was designed to see whether word order variation in \ac{CS} is related to the syntactic flexibility of the code-switched phrase, especially in non-literal interpretations. 

\subsection{Materials and methods}\label{ch2:sect:2.2.1}

To see whether different degrees of syntactic flexibility of idiomatic expressions would play a role in deriving different word orders in \ac{CS}, \ac{VP} idioms (16 \acsp{HV}, non-literal and 16 \acsp{LV}, non-literal) included in the \ac{CS} judgment task were selected as critical materials. The items were inserted in an appropriate sentential context and syntactically manipulated with three different operations: (a) passivization, (b) relative clause formation, and (c) \textit{wh}-question formation, as displayed in \REF{ex:39}.

\ea\label{ex:39} At a conference, participants can rub shoulders with many leading figures in the field.
    \ea At a conference, shoulders can be rubbed with many leading figures in the field.
    \ex Naïve participants are only interested in the shoulders that they rub with famous people at a conference.
    \ex How many shoulders did you rub with famous people at the conference?
    \z 
\z


In addition, 32 filler items were added, consisting of a light verb construction with 16 heavy verbs and 16 light verbs from the \ac{CS} judgment task. The filler items were also inserted in an appropriate sentential context and syntactically manipulated with passivization, relativization, and \textit{wh}{}-movement, similar to critical materials. 

\subsection{Procedure}\label{ch2:sect:2.2.2.} 

The experiment was a self-paced pencil-and-paper task. On the first page of the questionnaire, given in \REF{ex:40}, participants were instructed to read each sentence and judge to what extent the meaning associated with the underscored phrase was available in the following two or three  sentences, using a 4-point Likert scale. 

\ea\label{ex:40} 
{\small In this task, sentences are presented in groups.  Within each group, the first sentence is the “standard”, and it contains an underscored expression.  Two or three further sentences in the group use something similar to that expression, but in slightly varied forms.  Your task is to decide whether \textbf{\textit{the} \textit{meaning} \textit{of} \textit{the} \textit{underscored} \textit{expression}} in the standard sentence remains available in the sentences that follow. } \\
\glt {\small To give your opinion, please choose the best-fitting value on the scale below.  Value 1 is used to say that the expression’s meaning is no longer available at all, and Value 4 to say that exactly the original meaning remains available.  Values 2 and 3 are used for intermediate judgments. } \bigskip
\\
 \includegraphics[width=.9\textwidth]{figures/Section 2.2.2.jpg}
% \sn  \begin{adjustbox}{max width=\textwidth} 
% \begin{tabular}{lllllllllll} % add l for every additional column or remove as necessary
%   \lsptoprule
% \multicolumn{3}{l}{\textbf{Meaning not available}}  & \multicolumn{4}{r}{\textbf{Same meaning, exactly}} \\ %table header
% & 1 & ~~~~~~~~~~~~~~~~~~~~~~~~~~~~~~~~~~~~~~~~~~~~~2~~~~~~~~~~~~~~~~~~~~~~~~~~~~~~~~~~~~~~~~~~~~~~~~~~~~~~~~3 & & & & 4 \\  \midrule
%   & & Let's practice a few questions first.\\
%   \\
%   1.  &  & Can you \textbf{get a suntan} through a glass window?         & \textbf{1} & \textbf{2} & \textbf{3} & \textbf{4}  \\
%       & a. & That’s a great suntan that you’ve got through a glass window!
%       & $\Box$ & $\Box$ & $\Box$ & $\Box$ \\      
%       & b. & How much of a suntan can you get through a glass window?  
%       & $\Box$ & $\Box$ & $\Box$ & $\Box$ \\
%   2.  &  &   The company \textbf{made a request} for the workers to return to their jobs immediately.   & \textbf{1} & \textbf{2} & \textbf{3} & \textbf{4}  \\
%       & a. & A request was made for the workers to return to their jobs immediately. 
%       & $\Box$ & $\Box$ & $\Box$ & $\Box$ \\
%       & b. & The request the company made to the workers is their immediate return to their jobs. 
%       & $\Box$ & $\Box$ & $\Box$ & $\Box$ \\
%       & c. & What request did the company make of the workers? 
%       & $\Box$ & $\Box$ & $\Box$ & $\Box$ \\
%   \lspbottomrule
%  \end{tabular}\end{adjustbox}
\z

After participants read the written instructions and tried the first two practice items, they were offered a verbal clarification of the instructions from the investigator before the experiment began. The participants were allowed to take a break whenever needed, but no one took a break. 

\subsection{Participants} \label{ch2:sect:2.2.3}  

The same 28 Korean-English and 8 Japanese-English bilinguals whose data were analyzed for the \ac{CS} judgment task were also included for the data analysis of this task. In addition, 7 monolingual English speakers (age range 23--57; mean age 32.6; 3 female) participated as a control group. The results from the bilingual speakers were compared to those obtained from the monolingual native speakers of English. On average, participants finished the task in 60 minutes.

\subsection{Results and discussion} \label{ch2:sect:2.2.4}

\figref{fig:2.4} shows the mean syntactic flexibility scores for different types of code-switched phrases by Speaker group.


\begin{figure}
\caption{Mean syntactic flexibility rating, overall, as a function of Phrase Type and Speaker Group. ‘Overall’ rating collapses individually scored passivization, relativization, and \textit{wh}-question formation tests.} \label{fig:2.4}
% \includegraphics[width=\textwidth]{figures/Figure-2.4.png}
\includegraphics[width=\textwidth]{figures/2-4.pdf}
\end{figure}

The results showed that the idioms (heavy verb, non-literal interpretation and light verb, non-literal interpretation) were judged less syntactically flexible than the non-idiomatic expressions (heavy verb, light verb constructions and light verb, light verb constructions) by all three speaker groups, as predicted. The minor difference between the English monolingual group and the Korean-English and Japanese-English bilingual groups was that the scores for flexibility assigned by the two bilingual groups were slightly higher than those assigned by the monolingual group, regardless of phrase type.

To see whether syntactic flexibility plays a role in deriving word order in \ac{CS}, the results obtained from the Korean-English and Japanese-English bilingual speakers in the syntactic flexibility judgment task was compared to the results from the \ac{CS} judgment task, illustrated in \figref{fig:2.5} below. 

\begin{figure}
\subfigure[Korean-English bilinguals]{
% \includegraphics[width=0.7\textwidth]{figures/Figure-2.5_up.png}
\includegraphics[width=0.7\textwidth]{figures/2-5korean.pdf}
}
\vspace*{2cm}

\subfigure[Japanese-English bilinguals]{
% \includegraphics[width=0.7\textwidth]{figures/Figure-2.5_down.png}
\includegraphics[width=0.7\textwidth]{figures/2-5japan.pdf}
}
\caption{Percentages of \ac{VO} order preference predicted by syntactic flexibility scores assigned by Korean-English and Japanese-English bilingual speakers} \label{fig:2.5}
\end{figure}

\figref{fig:2.5} shows that the more flexible the phrase was judged, the less it was favored in \ac{VO} order in \ac{CS} in both Korean-English and Japanese-English bilingual groups. For instance, the syntactic flexibility score for the idiom \textit{take} \textit{a} \textit{hike}, meaning ‘leave’, was 1.44, which was much lower than the mean syntactic flexibility score 2.75, thus showing that the phrase was judged much less flexible than most phrases included. And it was preferred in \ac{VO} order 100\% in the \ac{CS} judgment task by Korean-English bilinguals. On the other hand, when the same phrase \textit{take} \textit{a} \textit{hike} was interpreted literally, it was judged much more flexible and scored 3.32 in the syntactic flexibility judgment task. And the \ac{VO} preference of the literal phrase \textit{take} \textit{a} \textit{hike} was 64\% by the same group of bilinguals. In other words, the more syntactically flexible code-switched phrase was preferred in \ac{OV} order in \ac{CS}. 

\newpage
The overall pattern of results found in the task supports the hypothesis that syntactically flexible and inflexible phrases behave differently with respect to word order variation in \ac{CS}, leading to \ac{OV} and \ac{VO}, respectively. This can be further corroborated by the argument that while the internal argument of a syntactically flexible phrase is subject to \ac{CS}, a syntactically inflexible phrase is frozen and undergoes \ac{CS} as a unit. Hence, the internal order of the phrase is maintained throughout the derivation.

  However, the correlation between the preferred word order and the syntactic flexibility of a code-switched phrase was found to be rather weak in both groups (\textit{r} = -.033 for Korean-English bilinguals and \textit{r} = -.0.38 for Japanese-English bilinguals), revealing that there is a variation among idiomatic phrases.\footnote{\textrm{An item-based analysis is provided in Chapter \ref{ch:5}.} } The weak correlation between the syntactic flexibility of the code-switched phrase and word order variation found in Korean-English and Japanese-English \ac{CS} may be accounted for by the fact that the three syntactic operations that were involved in the present task, passivization, relative clause formation, and \textit{wh}{}-question formation, may not be directly related to the syntactic phenomenon that derives \ac{OV}-\ac{VO} variation under the assumption that \ac{OV} is derived from \ac{VO} via object shift. Although it is true that the results from passivization, relative clause formation, and \textit{wh}{}-question formation revealed different degrees of syntactic flexibility of the code-switched phrases, the nature of these three syntactic operations is different from that of object shift leading to \ac{OV}-\ac{VO} variation in \ac{CS}, summarized in \REF{ex:41}.
 
 \ea\label{ex:41}
    \ea Object shift is object movement caused by the \ac{EPP} property on \textit{v}\footnotemark \\
    ~[\textit{\textsubscript{v}}\textsubscript{P} OBJ\textit{\textsubscript{i} } \textit{v} [\textsubscript{VP} V t\textit{\textsubscript{i}}]]
    \ex  Passivization is due to Case: the underlying object is assigned the nominative Case from T, which is specified for \ac{EPP} \\
    ~[\textsubscript{TP} OBJ\textit{\textsubscript{i} } T [\textit{\textsubscript{v}}\textsubscript{P} \textit{v} [\textsubscript{VP} V t\textit{\textsubscript{i}}]]]
    \ex (Object) Relativization is a syntactic dependency between the head noun in the matrix clause and the gap in the embedded clause (no movement involved) \\
    \ldots ~head noun\textit{\textsubscript{i}}~ \ldots ~ [\textsubscript{CP} C [\textsubscript{TP} T ~[\textit{\textsubscript{v}}\textsubscript{P} \textit{v} [\textsubscript{VP} V gap\textit{\textsubscript{i}}]]]]
    \ex  (Object) \textit{Wh}{}-question is movement caused by a  \textit{Wh}{}-feature on C, which is specified for \ac{EPP} \\
    ~[\textsubscript{CP} OBJ\textit{\textsubscript{i} } C [\textsubscript{TP} T [\textit{\textsubscript{v}}\textsubscript{P} \textit{v} [\textsubscript{VP} V t\textit{\textsubscript{i}}]]]]
    \z
 \z
 
\footnotetext{ \textrm{In Chapter \ref{ch:3}, it will be argued that the object moves to Spec, A}\textrm{\textsc{sp}}\textrm{P, not Spec,} \textrm{\textit{v}}\textrm{P, which does not concern us here. The assumption that the \ac{EPP} property on} \textrm{\textit{v}} \textrm{derives object movement, resulting in \ac{OV}, will remain constant regardless of the object landing site.}} 

We see in \REF{ex:41} that neither relativization nor \textit{wh}{}-movement have the same driving force as object shift, whereas syntactic procedures for object shift and passivization appear to be similar; the object raises to a specifier of a functional head, such as \textit{v} and T respectively, due to the \ac{EPP} specification on the functional head.\footnote{ \textrm{As will be discussed in Chapter \ref{ch:3}, the \ac{EPP} property T is not intrinsic to T but is inherited from C via feature inheritance.}} Yet, there are additional properties present in passive constructions cross-linguistically, which is distinguished from object shift. In an active sentence, the external argument of the verb serves as the grammatical subject of the sentence and gets the nominative Case, whereas the internal argument appears in the object position of the verb and gets the accusative Case. In a passive sentence, on the other hand, the internal argument of the verb becomes the grammatical subject of the sentence, which is assigned the nominative Case, and the external argument of the verb is not projected as an argument but may be realized as an adjunct phrase, such as the \textit{by}-phrase in English and the dative phrase in Korean and Japanese. Most importantly, the demotion of the external argument, coupled with the accusative case absorption, brings about valency decrease, which is reflected in a morphological change in the verb. Examples are provided in \REF{ex:42}.\footnote{\textrm{Both Korean and Japanese passive constructions are analyzed as a causative construction with an experiencer-reading in \citet{Shim2008} and \citet{ShimNakajima2012}.}}

\ea\label{ex:42}
    \ea\label{ex:42a} Bibi kicked the dog.
    \exi{a.\parbox{0mm}{$'$}} The dog was kicked by Bibi.
    \ex\label{ex:42b} \gll Bibi-ka     kay-lul   cha-ss-ta    \\
   Bibi-\textsc{nom} dog-\textsc{acc} kick-\textsc{past-decl}  \\ \hfill Korean
    \glt  ‘Bibi kicked the dog.’
    \exi{b.\parbox{0mm}{$'$}} \gll kay-ka    Bibi-eykey chai-i-ess-ta \\
      dog-\textsc{nom} Bibi-\textsc{dat}   kick-\textsc{pass-past-decl} \\
    \glt  ‘The dog was kicked by Bibi.’
    \ex\label{ex:42c} \gll Bibi-ga     inu-o      ker-ta   \\
    Bibi-\textsc{nom} dog-\textsc{acc} kick-\textsc{past} \\ \hfill         Japanese
    \glt  ‘Bibi kicked the dog.’
    \exi{c.\parbox{0mm}{$'$}} \gll inu-ga       Bibi-ni   ker-are-ta \\
     dog-\textsc{nom} Bibi-\textsc{dat}   kick-\textsc{pass-past} \\
     \glt  ‘The dog was kicked by Bibi.’
     \z
\z

In this regard, the passive construction differs from object shift: while both constructions involve object raising, the passive construction involves a valency-reducing operation on the verb in addition. It is likely that the valency-changing morphological operation involved in passivization is subject to restrictions that constrain the applicability of passive \textit{beyond} the restrictions imposed by DP/object raising \textit{per} \textit{se}. For that reason, it is to be expected that object shift and passives will not have identical distributions.

Hence, from the fact that none of the three syntactic procedures included in the syntactic flexibility task of the present study, passivization, relativization and \textit{wh}{}-questions, involve the exactly same syntactic process as object shift, which is the source of \ac{OV}-\ac{VO} variation in \ac{CS}, it is perhaps not surprising that there is no strong correlation between the syntactic flexibility of the code-switched phrase based on the results from these operations and word order patterns obtained in the \ac{CS} judgment task.  

\section{Idiom familiarity task}\label{ch2:sect:2.3}

The test items of the \ac{CS} judgment task included a number of idiomatic expressions, thus an idiom familiarity task was designed to measure the bilingual participants’ familiarity with English idioms included in the \ac{CS} judgment task and the syntactic flexibility judgment task. 

\subsection{Materials and methods}\label{ch2:sect:2.3.1} 

A total of 32 \ac{VP} idioms that were included in in the \ac{CS} judgment task and the syntactic flexibility judgment task were used in this task. In the syntactic flexibility task, each idiom was inserted in an appropriate sentential context and used as an input sentence. These sentences continued to serve as the test items of the present task. Since the idiom familiarity task was originally designed to be used as a screening tool, no additional filler materials were added. 

\subsection{Procedure}\label{ch2:sect:2.3.2}

The experiment was a self-paced pencil-and-paper task. Each idiom was incorporated in a sentence and underscored. Participants were asked to read each sentence and write down the meaning of the underscored phrase of the sentence in their own words. To prevent the two judgment tasks (i.e. \ac{CS} and syntactic flexibility) from being affected by lexical and contextual redundancy, the idiom familiarity task was administered \textit{after} the participants completed the two judgment tasks. 

\subsection{Participants}\label{ch2:sect:2.3.3}

The same group of Korean-English and Japanese-English bilinguals, and native speakers of English that participated in the syntactic flexibility task continued to participate in the idiomatic familiarity task. No time limit was imposed on the task, and the participants, on average, took 30 minutes or less to finish it.

\subsection{Results and discussion}\label{ch2:sect:2.3.4}

All responses provided by the participants were sorted into five categories: (a) a correct description or interpretation of the phrase; (b) no response; (c) a wrong interpretation of the phrase; (d) a literal interpretation of the phrase; and (e) an approximate description of the phrase, yet not correct. For each speaker group, a correct answer response rate was calculated based only on (a) a correct description/interpretation of the phrase, among the answer types. 

Overall, the percentages of correct answers provided by the English monolingual, Korean-English bilingual and Japanese-English bilingual speakers were 95\%, 84\%, and 85\% respectively, indicating that the two bilingual groups were slightly \emph{less} familiar with English idioms than the monolingual native speakers of English. However, such a difference between the monolingual and the bilingual speakers was limited to the idioms involving heavy verbs, especially only those in which the usual meaning of a lexical verb is not available at all in the idiomatic phrase. \figref{fig:2.6} shows the correct response rate by Phrase Type and Speaker Group. 

\begin{figure} 
    \centering
%     \includegraphics[width=\textwidth]{figures/Figure-2.6.png}
    \includegraphics[width=\textwidth]{figures/2-6.pdf}
    \caption{Percentages of correct answers by Phrase Type and Speaker Group}
    \label{fig:2.6}
\end{figure}

\begin{figure}
    \centering
%     \includegraphics[width=\textwidth]{figures/Figure-2.7.png}
    \includegraphics[width=\textwidth]{figures/2-7.pdf}
    \caption{Idioms with the highest error rates obtained from the bilingual speakers}
    \label{fig:2.7}
\end{figure}

\newpage
 \figref{fig:2.7} lists five idiomatic phrases that were most frequently interpreted incorrectly by the Korean-English and Japanese-English bilingual speakers. Among them, the idiom \textit{pound} \textit{the} \textit{pavement} ‘to look for a job’ was interpreted literally in its aspectual sense by the majority of Korean-English and Japanese-English bilingual speakers, such as ‘to do something repeatedly’, which preserved the aspectual meaning of the verb, iterativeness. Interestingly, \textit{pound} \textit{the} \textit{pavement} was strongly preferred in \ac{OV} order in the \ac{CS} judgment task, suggesting the possibility that the emergence of an aspectually literal meaning leads to \ac{OV} order, parallel to the predominant \ac{OV} order with a heavy verb in literal interpretations.

  However, other idioms interpreted close to their literal meaning do not exhibit a similar pattern as \textit{pound} \textit{the} \textit{pavement}. Two idioms \textit{shoot} \textit{the} \textit{breeze} ‘to talk aimlessly’ and \textit{climb} \textit{the} \textit{walls} ‘to be anxious or frantic’ were interpreted close to their literal sense by some speakers, such as ‘to feel the breeze’ and ‘to promote’, respectively. Nonetheless, they were both favored in \ac{VO} order in both Korean-English and Japanese-English \ac{CS}. Thus, it seems that the unexpected order of the idiom \textit{pound} \textit{the} \textit{pavement} was not related to the failure of its idiomatic reading. 

\section{Chapter summary and conclusion}\label{ch2:sect:2.4}  

This chapter presented an experimental study to investigate how \ac{OV} and \ac{VO} orders are systematically distributed in Korean-English and Japanese-English \ac{CS}. Based on the findings of a previous pilot study \citep{Shim2011}, (i) the status of the verb, heavy vs light, within a code-switched constituent and (ii) the syntactic flexibility of a code-switched phrase were identified as two factors that seemed to be related to the distribution of \ac{OV} and \ac{VO} orders in Korean-English and Japanese-English \ac{CS}, which were further investigated in the present study. 

  The pattern of the results found in the \ac{CS} judgment task and the syntactic flexibility judgment task from 28 Korean-English and 8 Japanese-English bilingual speakers provided supporting evidence that \ac{OV}-\ac{VO} variation in Korean-English and Japanese-English \ac{CS} is related to the above-mentioned two factors. Overall, \ac{VO} order was preferred when the verb is light and also with idioms, which were judged syntactically less flexible. On the other hand, \ac{OV} order was favored with a heavy/lexical verb and when the code-switched phrase was non-idiomatic and syntactically flexible. 

  The results obtained from the \ac{CS} judgment task will be further explained in Chapter \ref{ch:5} where \ac{OV}-\ac{VO} variation in Korean-English and Japanese-English \ac{CS} will be accounted for against the minimalist framework, especially the feature inheritance mechanism, which will be discussed in the next two chapters. 

