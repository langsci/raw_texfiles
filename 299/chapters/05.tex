\chapter{Deriving OV and VO in code-switching}\label{ch:5}

Based on the theoretical model of feature inheritance developed in the previous chapter, this chapter offers a \ac{FI}-based account of \ac{OV}-\ac{VO} variation in Korean-English and Japanese-English \ac{CS}, which was tested against 28 Korean-English and 8 Japanese-English bilingual speakers’ introspective judgments of the \ac{CS} patterns presented to them in an experimental setting (Chapter \ref{ch:2}). Overall, the results confirmed two research hypotheses of the study, which are repeated below. 

\begin{exe}
\sn \textbf{Research} \textbf{Hypothesis} \textbf{1} \\
 Assuming that linguistic variation is determined by the way features are parameterized in functional categories and how these features are valued in syntactic derivations, \ac{OV}-\ac{VO} variation in Korean-English and Japanese-English \ac{CS} will be determined by feature specifications on functional categories represented by light verbs in Korean, Japanese and English and how these features are valued in syntactic derivations.
\sn \textbf{Research} \textbf{Hypothesis} \textbf{2} \\
 Syntactically flexible phrases and inflexible phrases will behave differently with respect to word order derivation in \ac{CS}. More specifically, while the internal argument of a syntactically flexible phrase is subject to \ac{CS}, a syntactically inflexible phrase is frozen and undergoes \ac{CS} as a unit. Hence, the internal order of the phrase will be maintained throughout the derivation.
\end{exe}


The results of the \ac{CS} judgment task revealed that the selection between an English heavy verb and an English light verb within a code-switched phrase led to \ac{OV} and \ac{VO} orders respectively in Korean-English and Japanese-English \ac{CS}, which confirmed the research hypothesis 1. \figref{fig:2.1again}, repeated from Chapter \ref{ex:2} (\figref{fig:2.1}), shows the percentage of \ac{VO} order preference by Verb Type (heavy vs light) and Interpretation (literal vs non-literal) in Korean-English \ac{CS}.

\begin{figure}
% \includegraphics[width=.9\textwidth]{figures/Figure 2.1.png}
\includegraphics[height=.3\textheight]{figures/2-1.pdf}
\caption{\%\textsc{vo} preference by Verb Type and Interpretation} 
\label{fig:2.1again}
\end{figure}

While a main effect of the verb type (heavy vs light) was found with respect to the distribution of \ac{OV} and \ac{VO} orders, this was rather more evident in the contexts where the code-switched phrase was interpreted literally or involved a light verb construction. When a code-switched phrase was a non-literal/idiomatic phrase, on the other hand, the status of the verb did not seem to play a role, by showing that the \ac{VO} was preferred with both heavy and light verbs. \figref{fig:2.2again}, a repetition of \figref{fig:2.2}, shows the percentage of \ac{VO} order preference for different types of code-switched constituents in Korean-English and Japanese-English \ac{CS}. 

\begin{figure}
% \includegraphics[width=.9\textwidth]{figures/Figure-2.2.png}
\includegraphics[height=.3\textheight]{figures/2-2.pdf}
\caption{\%\ac{VO} preference, overall, as a function of Phrase Type and Speaker Group} 
\label{fig:2.2again}
\end{figure}

\largerpage
It should be noted that the results from Japanese-English \ac{CS} were not confirmed at a statiscally significant level due to its small sample size (N=8). Nonetheless, we can clearly see that the overall outcome of Japanese-English \ac{CS} is similar to that of Korean-English \ac{CS} in \figref{fig:2.2again}: the distributional pattern of \ac{VO} order preference is similar across the different types of phrases in both Korean-English and Japanese-English \ac{CS}. Thus, the following discussion is primarily based on the results obtained from the Korean-English \ac{CS} judgment task, whose significance was statistically confirmed. Based on this, I propose a \ac{FI}-based account of the distribution of \ac{OV} and \ac{VO} orders in Korean-English \ac{CS}. It is expected that the same account should also hold for \ac{OV}-\ac{VO} variation in Japanese-English \ac{CS}.

We notice that \ac{VO} order is preferred less than 50\% of the time in Korean-English and Japanese-English \ac{CS} alike in three conditions (i.e. (i) heavy verb, literal interpretation, (ii) light verb, literal interpretation, and (iii) heavy verb, light verb construction). Assuming that \ac{VO} is the underlying order for \ac{VO} languages (e.g. English) as well as \ac{OV} languages (e.g. Korean \& Japanese), one may raise a question of why \ac{VO} preference is so low (or even less than 50\%) in some of the Korean-English and Japanese-English \ac{CS} examples. However, it should be clarified that the assumption that \ac{VO} is the underlying order does not state that \ac{VO} is the default order. On the contrary, in monolingual grammars of Korean and Japanese and also bilingual grammars of Korean-English and Japanese-English \ac{CS}, the default order is \ac{OV}, which is necessarily derived from \ac{VO} via \ac{FI}, as explained in Chapter 4. As will be shown in the following sections, the underlying \ac{VO} order surfaces if and only if \ac{FI} from \textit{v} to \ac{ASP} fails to occur in Korean-English and Japanese-English \ac{CS} and one instance was mentioned already in Chapter 4: when a code-switched phrase includes an English light verb, which represents \ac{ASP}, \ac{FI} does not take place, for \ac{ASP} is not a valid head to inherit \textit{v}’s features. We will see other cases when \ac{FI} from \textit{v} to \ac{ASP} does not happen, which violates the principles of \ac{FI}.

Another important issue that needs to be addressed: despite the fact that several individual code-switched phrases were unanimously favored in either \ac{OV} or \ac{VO} order by all speakers (e.g. ‘break the glass’, ‘spill the soup’ were preferred in \ac{OV}order and ‘have a look (at)’, and ‘make waves’ were favored in \ac{VO} order by all Korean-English bilingual speakers), the percentage of either \ac{OV} or \ac{VO} preference did not reach 100\% in any of the six phrase types, as we see in Fig. 2.2. This is in fact expected from any experimental work involving human subjects, among whom there is a great level of variation in their performance, which may stem from non-linguistic factors such as participants’ lack of attention, fatigue, memory span, etc. Also there is a dialectal variation among individuals and such variance and flexibility in linguistic competence becomes greater when the study involves bilingual participants whose two language grammars are very different and even in conflict, as in the Korean-English and Japanese-English bilinguals in this study. Although we should account for such variation (inter-subject variation) as well, here I limit myself to providing an analysis based on the more preferred/dominant word order, either \ac{OV} or \ac{VO}, in each phrase type. 

\section{Type 1: English heavy verbs and literal interpretation}\label{ch5:sect:5.1}

The results of the \ac{CS} judgment task revealed that \ac{OV} order was strongly preferred by Korean-English bilingual speakers (78\%) and Japanese-English bilinguals (81\%) when the code-switched verb phrase included an English heavy verb in a literal interpretation (e.g. \textit{miss} \textit{the} \textit{bus}). The structure in \REF{ex:112} represents the underlying structure for the phrase \textit{miss the bus}, for instance, which was favored in \ac{OV} order as in [(the) bus-\textsc{acc} miss]-\textit{ha}/\textit{su} in Korean-English and Japanese-English \ac{CS}.

\ea\label{ex:112}
\adjustbox{width=0.9\textwidth}{\begin{forest}
[\textit{v}P   [\textit{v} \\   \textit{ha}\textsuperscript{KR}/\textit{su}\textsuperscript{JP} \\ { [\colorbox{lightgray}{\textit{u}$\upphi$}, \textit{u}Asp\textsuperscript{EPP}, \colorbox{lightgray}{\textit{u}D\textsuperscript{EPP}}]}, name = src ]
[\textsc{Asp}P [\textsc{Asp} \\ $\varnothing$ \\ {[\textit{u}$\upphi$, \textit{u}D\textsuperscript{EPP}]}, name = tgt ]
[VP [V\textsubscript{[Asp,T]} \\ \textit{miss} ]
[OBJ\textsubscript{[$\upphi$, D]} \\ \textit{the bus}]]]]
\draw[->] (src) to[in=south, out=south] (tgt) ;
\end{forest}}
\z

In \REF{ex:112}, the null-headed \ac{ASP} does not bear any formal features of its own and is selected by \textit{v} = \textit{ha}\textsuperscript{KR}\textit{/su}\textsuperscript{JP}, which is a Case-checking light verb with \ac{EPP} specifications. Via \ac{FI}, \ac{ASP} inherits the features [\textit{u}$\upphi$, \textit{u}D\textsuperscript{EPP}] from \textit{v} = \textit{ha}\textsuperscript{KR}\textit{/su}\textsuperscript{JP}. Recall from Chapter 4 that \textit{v} = \textit{ha}\textsuperscript{KR}\textit{/su}\textsuperscript{JP} tries to transfer all of its features according to the rule of \textit{Earliness}, yet \ac{ASP} only inherits [\textit{u}$\upphi$, \textit{u}D\textsuperscript{EPP}] from \textit{v}, which obeys the other operational rules of \ac{FI}, namely \textit{Economy} and \textit{Multiple} \textit{Agree} \textit{under} \textit{Antisymmetry}. After \ac{FI} from \textit{v} to \ac{ASP}, the uninterpretable $\upphi$ and D-features on \ac{ASP} are valued against the interpretable $\upphi$ and D-features of the object. The \ac{EPP} property of [\textit{u}D] on \ac{ASP} triggers its goal to raise as well, and as a result the object moves to Spec, \ac{ASP}P where the object gets accusative Case; the D-feature is a Case feature.\footnote{This will be further discussed in Chapter \ref{ch:6}  (\sectref{ch6:sect:case}).} The tree in \REF{ex:113} illustrates this. 

\ea\label{ex:113}
\adjustbox{max width = 0.9\textwidth}{
\begin{forest}
[\textit{v}P, s sep = 1mm [ \textit{v}{=}\textit{ha}\textsuperscript{KR}/\textit{su}\textsuperscript{JP} \\   {  [\colorbox{lightgray}{\textit{u}$\upphi$}, \textit{u}Asp\textsuperscript{EPP}, \colorbox{lightgray}{\textit{u}D\textsuperscript{EPP}}]}    ]
[\textsc{Asp}P [OBJ\textsubscript{\textit{i} [$\upphi$, D] }  \\ \textit{the bus}, name = tgt]
[\textsc{Asp}P [\textsc{Asp}\textsubscript{[\sout{\textit{u}$\upphi$}, \sout{\textit{u}D}\textsuperscript{\sout{EPP}}]} \\ $\varnothing$]
[VP [V\textsubscript{[Asp, T]} \\ \textit{miss} ]  [t$i$, name = src]]]]]
\draw[->,dashed] (src) to[out=south west, in = south east] (tgt) ;
\end{forest}}
\z

While [\textit{u}$\upphi$, \textit{u}D\textsuperscript{EPP}] transferred from \textit{v} to \ac{ASP} are no longer active as probing features on \textit{v}, the remaining feature [\textit{u}\ac{ASP}\textsuperscript{EPP}] on \textit{v} still needs to be valued. To do so, \textit{v} = \textit{ha}\textsuperscript{KR}\textit{/su}\textsuperscript{JP} agrees with V, and movement of \acs{VP} is triggered by the \ac{EPP} specification of the Asp-feature on \textit{v} = \textit{ha}\textsuperscript{KR}\textit{/su}\textsuperscript{JP}. However, movement of \acs{VP}, the projection of a lexical root, is not possible and \ac{ASP}P is pied-piped by the \acs{VP} and raised to Spec, \textit{v}P, which correctly delivers the surface order \ac{OV}-\textit{ha} in Korean-English \ac{CS} and \ac{OV}-\textit{su} in Japanese-English \ac{CS}, as shown in \REF{ex:114}.

\ea\label{ex:114}
\adjustbox{max width = \textwidth}{
\begin{forest}
[\textit{v}P 
[\textsc{Asp}P$_k$ [OBJ$_i$ \\ \textit{the bus}]
[\textsc{Asp}$'$ [\textsc{Asp}]
[VP [V \\ \textit{miss}][t$i$]]]]
[\textit{v}$'$ [\textit{v} \\\textit{ha}\textsuperscript{KR}/\textit{su}\textsuperscript{JP}][t$_k$]]
]
\end{forest}}
\z

\section{Type 2: English heavy verbs and non-literal interpretations}\label{ch5:sect:5.2}

The second type of code-switched phrases included an English heavy verb and an object in a non-literal/idiomatic interpretation (e.g. \textit{miss the boat}) and both Korean-English and Japanese-English bilingual speakers generally favored \ac{VO} order, 66\% and 63\%, respectively. The same set of English (heavy) verbs were included in literal interpretations (Type 1) and non-literal interpretations (Type 2), yet the overall preferred word order was \ac{OV} and \ac{VO}, respectively. This contrast is highlighted with several examples in \tabref{tab:5.1} below.

\begin{table}
\caption{Cross-item analysis between Type 1 and Type 2 conditions in Korean-English \textsc{CS}}
\label{tab:5.1}\small
 \begin{tabularx}{.8\textwidth}{rrYr} % add l for every additional column or remove as necessary
  \lsptoprule
    Type 1        & \% \textsc{VO} & Type 2 & \% \textsc{VO} \\ %table header
  \midrule
   shoot an alien &    14 & shoot the breeze &    86\\
\tablevspace
spill the soup &    0 & spill the beans &    86\\
\tablevspace
ring the bell &    21 & ring a bell &    79\\
\tablevspace
pull weeds &    36 & pull strings &    79\\
\tablevspace
lose your job &    7 & lose your marbles &    71\\
\tablevspace
feel the pain &    0 & feel the pinch &    64\\
\tablevspace
break the glass &    0 & break the bank &    57\\
  \lspbottomrule
 \end{tabularx}
\end{table}

The \ac{OV}-\ac{VO} contrast between Type 1 and Type 2 items can be explained by the availability of object shift: object shift takes place in the former, but not in the latter. Then a question arises as to what prevents the object from raising in a non-literal interpretation (Type 2). I argue that this is due to the idiomatic status of the \acs{VP}.\footnote{Or we can think of this differently. The derivation \textit{per se} is possible, similar to the case of Type 1, but the idiomatic reading may be lost if the object moves out. Since the \ac{CS} judgment task ensured that the participants interpret the code-switched phrase as an idiom, not as a literal meaning, I focus on providing an account of the preferred \ac{VO} order with English \ac{VP} idioms.} 

\ea\label{ex:115}

\adjustbox{max width = 0.9\textwidth}{
\begin{forest}
[\textit{v}P   [\textit{v} \\   \textit{ha}\textsuperscript{KR}/\textit{su}\textsuperscript{JP} \\ { [\colorbox{lightgray}{\textit{u}$\upphi$}, \textit{u}Asp\textsuperscript{EPP}, \colorbox{lightgray}{\textit{u}D\textsuperscript{EPP}}]}, name = src ]
[\textsc{Asp}P [\textsc{Asp} \\ $\varnothing$ \\ {[\textit{u}$\upphi$, \textit{u}D\textsuperscript{EPP}]}, name = tgt ]
[VP, tikz={\node [draw,ellipse,inner sep=-1pt,fit to=tree, label=below:inflexible!] {};}
[V\textsubscript{[Asp,T]} \\ \textit{miss} ]
[OBJ\textsubscript{[$\upphi$, D]} \\ \textit{the boat}]]]]
\draw[->] (src) to[in=south, out=south] (tgt) ;
\end{forest}}
\z


\hspace*{-1.5mm}Similar to \REF{ex:112}, the null \ac{ASP} head in \REF{ex:115} may inherit [\textit{u}$\upphi$, \textit{u}D\textsuperscript{EPP}] from \textit{v} = \textit{ha}\textsuperscript{KR}\textit{/su}\textsuperscript{JP} via \ac{FI} and trigger object shift. Yet, the object resists being extracted from the   \acs{VP} due to the idiomatic/non-literal status of the   \acs{VP} and thefore judged less flexible in the syntactic flexibility judgment task in Chapter \ref{ch:2}. The syntactic flexilbity task included three syntactic operations in which the object was extracted from the   \acs{VP}: (a) passivization, (b) object relative clause formation, and (c) \textit{wh}-object question formation. As described in \figref{fig:2.4again}, which is reproduced below from \figref{fig:2.4}, the   \acs{VP} idioms (\acs{HV}, Non-Lit and   \acs{LV}, Non-Lit) were judged syntactically less flexible than the non-idiomatic expressions (\acs{HV},   \acs{LVC} and   \acs{LV},   \acs{LVC}) by all speaker groups.

\begin{figure}
\caption{Mean syntactic flexibility rating, overall, as a function of Phrase Type and Speaker Group. ‘Overall’ rating collapses individually scored passivization, relativization, and \textit{wh}-question formation tests.} \label{fig:2.4again}
% \includegraphics[width=\textwidth]{figures/Figure-2.4.png}
\includegraphics[width=\textwidth]{figures/2-4.pdf}
\end{figure}

\newpage
This suggests that it is difficult for the object to be extracted from the \acs{VP} in \REF{ex:115}, but the \ac{EPP} property of the D-feature on \ac{ASP} still needs to be satisfied via object raising to Spec, \ac{ASP}P. If the object does not raise, the derivation crashes. One way to save the derivation from crashing is that the object pied-pipes the   \acs{VP} and the entire \acs{VP} raises to Spec, \ac{ASP}P, as shown in \REF{ex:116}.

\ea\label{ex:116}
\adjustbox{max width = 0.8\textwidth}{\begin{forest}
[\textit{v}P, s sep = 1mm [\textit{v}{=}\textit{ha}\textsuperscript{KR}/\textit{su}\textsuperscript{JP} \\ {[\colorbox{lightgray}{\textit{u}$\upphi$}, \textit{u}Asp\textsuperscript{EPP}, \colorbox{lightgray}{\textit{u}D\textsuperscript{EPP}}] }]
[\textsc{Asp}P, s sep = 25mm 
[VP$_i$, name = tgt
[V\textsubscript{[Asp, T]} \\ \textit{miss}][OBJ\textsubscript{[$\upphi$, D]} \\ \textit{the boat}]]
[\textsc{Asp}P
[\textsc{Asp} \\ $\varnothing$][t$_i$, name =src]]]]
\draw[->,dashed] (src) to[out=south west, in=east] (tgt) ;
\end{forest}}\vspace*{-10mm}
\z

\largerpage
However, the \acs{VP} that is severed from its selecting functional category \ac{ASP} cannot undergo phrasal movement and must pied-pipe \ac{ASP}P, as discussed in \sectref{ch4:sect:4.4}.  Thus, \acs{VP} moment in \REF{ex:116} is an illicit movement and the smallest unit that is pied-piped by object movement is \ac{ASP}P. Since \ac{ASP}P raising to Spec, \ac{ASP}P is not a possible derivation, the derivation crashes again. Then how can we derive \ac{VO}-\textit{ha/su} order here? 

\ac{FI} is designed to value uninterpretable features on a phase head in an efficient and economical way, and it happens automatically as long as a derivation converges. This was stated as the first principle of \ac{FI}, \textit{Obligation}: \ac{FI} is obligatory as long as it does not lead to a syntactic crash. To put it differently, while \ac{FI} from \textit{v} to \ac{ASP} is otherwise spontaneous, it is blocked in \REF{ex:115}, for it leads to a derivational crash. Thus, \textit{v} \textit{=} \textit{ha}\textsuperscript{KR}\textit{/su}\textsuperscript{JP} may not transmit any of its features to \ac{ASP}, and all of \textit{v}’s features remain on the \textit{v} head, as depicted in \REF{ex:117}.

\ea\label{ex:117}
\adjustbox{max width = 0.8\textwidth}{
\begin{forest}
[\textit{v}P   [\textit{v} \\   \textit{ha}\textsuperscript{KR}/\textit{su}\textsuperscript{JP} \\ { [\textit{u}$\upphi$, \textit{u}Asp\textsuperscript{EPP}, {\textit{u}D\textsuperscript{EPP}}]}, name = src ]
[\textsc{Asp}P [\textsc{Asp} \\ $\varnothing$, name = tgt ]
[VP, tikz={\node [draw,ellipse,inner sep=-1pt,fit to=tree, label=below: inflexible!] {};}
[V\textsubscript{[Asp,T]} \\ \textit{miss} ]
[OBJ\textsubscript{[$\upphi$, D]} \\ \textit{the boat}]]]]
\draw[->] (src) to[in=south, out=south] 
node[pos=0.5]{\ding{54}} (tgt) ;
\end{forest}
}\vspace*{-10mm}
\z

\clearpage
Notice that two of \textit{v}’s features [\textit{u}Asp, \textit{u}D] are \ac{EPP}-specified, which attracts a goal to Spec, \textit{v}P in \REF{ex:117}. \textit{v} = \textit{ha}\textsuperscript{\MakeUppercase{kr}}/\textit{su}\textsuperscript{\MakeUppercase{jp}} agrees with both V and \acs{OBJ}, each of which satisfies the \ac{EPP} property on [\textit{u}Asp] and [\textit{u}D] of \textit{v}, respectively. However, following the operational rule of \textit{Multiple Agree under Antisymmetry}, only one goal can be raised and spelled out at Spec, \textit{v}P. Here, I assume that Spec, \textit{v}P is the final landing site for both object movement and the phrasal movement headed by V, which violates the rule of  \textit{Multiple Agree under Antisymmetry}. If only one goal raises, not all \ac{EPP} properties on \textit{v} will be satisfied, which in turn leads to a derivational crash. Now we seem to have a dilemma: If \ac{FI} takes place as in \REF{ex:115}, the derivation crashes. If \ac{FI} does not occur, it also leads to a crash as well, as in \REF{ex:117}. So, are we all doomed here? 

Fortunately, there is a way to save the derivation. In \REF{ex:117} the \acs{VP} is inflexible and the object cannot be extracted out of the \acs{VP}. Instead, movement of (the maximal projection of) two goals V and OBJ together target \ac{ASP}P, which raises to Spec, \textit{v}P. The raised \ac{ASP}P can satisfy the \ac{EPP} properties on the Asp-feature and the D-feature on \textit{v} = \textit{ha}\textsuperscript{\MakeUppercase{kr}}/\textit{su}\textsuperscript{\MakeUppercase{jp}}\MakeUppercase{.} In other words, \ac{ASP}P raising at one fell swoop values all the features of \textit{v}, including \ac{EPP} properties, which can be explained as the effect of ``killing two birds with one stone.'' This correctly delivers \ac{VO}-\textit{ha} and \ac{VO}-\textit{su} order \REF{ex:118}.

\ea\label{ex:118}
\adjustbox{max width = 0.8\textwidth}{\begin{forest}
[\textit{v}P, s sep = 15mm 
[\textsc{Asp}P$_i$, name = tgt [\textsc{Asp} \\ $\varnothing$]
[VP [V \\ \textit{miss}][OBJ \\ \textit{the boat}]]]
[\textit{v}P [\textit{v} \\ \textit{ha}\textsuperscript{KR}/\textit{su}\textsuperscript{JP}  ][t$i$, name = src]]]
\draw[->, dashed] (src) to[out=south west, in = east] (tgt) ;
\end{forest}}\vspace*{-5mm}
\z

Some comments are in order: in \REF{ex:118}, \ac{ASP}P raising satisfies both \ac{EPP} specifications of [\textit{u}Asp] and [\textit{u}D] on \textit{v} = \textit{ha}\textsuperscript{\MakeUppercase{kr}}/\textit{su}\textsuperscript{\MakeUppercase{jp}}\MakeUppercase{.} Yet, the \ac{ASP} head itself is not a goal: it shares no matching features with \textit{v}. Instead, the \ac{EPP} specifications on \textit{v} induce movement of the \acs{VP} and the object, which are in the complement domain of \ac{ASP}. The \acs{VP} and the object together pied-pipe \ac{ASP}P, which saves the derivation. Although it may sound economical that the \ac{EPP} requirement of [\textit{u}Asp] and [\textit{u}D] on \textit{v} = \textit{ha}\textsuperscript{\MakeUppercase{kr}}/\textit{su}\textsuperscript{\MakeUppercase{jp}} can be met by \ac{ASP}P raising, \ac{ASP}P pied-piping by object shift is the last resort strategy to save the derivation, which otherwise crashes as a result of failure of object shift.  
In sum, when an English \acs{VP} idiom with an English heavy or lexical verb is code-switched into Korean or Japanese, \ac{FI} from \textit{v} = \textit{ha}\textsuperscript{\MakeUppercase{kr}}/\textit{su}\textsuperscript{\MakeUppercase{jp}}\MakeUppercase{} to \ac{ASP} does not occur and all of \textit{v}’s features including its \ac{EPP} specifications are valued via \ac{ASP}P raising, which delivers \ac{VO}-\textit{ha} order in Korean-English \ac{CS} and \ac{VO}-\textit{su} order in Japanese-English \ac{CS}.

\section{Type 3: English light verbs and literal interpretations}\label{ch5:sect:5.3}

The results from the \ac{CS} judgment task (\figref{fig:2.2again}) shows that the occurrence of \ac{VO} order with an English light verb in a literal interpretation (e.g. \textit{have small head}) is 46\% in Korean-English \ac{CS} (and 36\% for Japanese-English \ac{CS}). At first glance, this seems to suggest that \ac{OV} and \ac{VO} orders are more or less equally distributed in Korean-English \ac{CS} (and even \ac{OV} order is preferred in Japaese-English \ac{CS}). However, as discussed earlier, a main effect of verb types, heavy vs light verbs, was found, revealing a higher preference of \ac{VO} order with light verbs than heavy verbs in literal interpretations, which should be accounted for. Nonetheless, there was a great variation found among the light verbs as well as among the different test items.

\begin{table}[b]
\caption{Item-based analysis for light verbs in literal interpretations in Korean-English \textsc{cs}}
\label{tab:5.2}
 \begin{tabularx}{\textwidth}{XrXr}
  \lsptoprule
   code-switched phrase & \% \textsc{vo} & code-switched phrase & \% \textsc{vo} \\
  \midrule
  have a small head & 86&               hold water & 43\\
have an upset stomach & 86&             give a big present & 36\\
make friends & 79&                      hold the bowl & 29\\
keep a respectful manner & 71&          raise the fee & 29\\
get a cold sore & 64&                   give the job & 14\\
get a new girlfriend & 64&              make a million bucks & 7\\
take a hike & 64&                       take a window seat & 7\\
raise their hands & 64&                 keep your receipt & 0\\
   \lspbottomrule
 \end{tabularx}
\end{table}

While the average percentage of \ac{VO} order preference is 46\% with an English light verb in a non-idiomatic context, there was a great variation among the eight light verbs as well as among the individual items within the same verb, as shown in \tabref{tab:5.2}. While a detailed item-based analysis of each light verb will be provided in \sectref{ch5:sect:5.7}, here the focus will be on the contrast found between heavy verbs and light verbs in the same condition, namely, heavy verbs in literal interpretations (Type 1) and light verbs in literal interpretations (Type 3) where a main effect of verb types was found: the preferred order was \ac{OV} and \ac{VO}, respectively. 

The \ac{VO} order of a code-switched constituent with an English light verb in a literal context can be explained by the failure of object raising, which results from the proposal that English light verbs represent the head \ac{ASP} (Recall (\ref{ex:67}) in Chapter \ref{ch:2}). The \ac{FI} principle of \textit{Validation} states that \ac{FI} occurs if and only if the recipient head is a valid head, with a valid head being a featureless nonphase head. In \REF{ex:119}, \textit{have} qua light verb lexicalizes the head A\textsc{sp,}  which bears [Asp, T] features. Thus, it is not a valid head to receive \textit{v}’s features. As a consequence, none of \textit{v}’s features are discharged to \ac{ASP}, and \textit{v} enters into a probe-goal relationship.

\ea\label{ex:119}
\adjustbox{max width = 0.9\textwidth}{\begin{forest}
[\textit{v}P, [\textit{v} \\ \textit{ha}\textsuperscript{KR}/\textit{su}\textsuperscript{JP} \\ {[\textit{u}$\upphi$, \textit{u}Asp\textsuperscript{EPP}, \textit{u}D\textsuperscript{EPP}]} ,name = src]
[\textsc{Asp}P [\textsc{Asp} \\ \textit{have}, name = tgt][OBJ \\ \textit{a small head}]]]
\draw[->] (src) to[in=south west,out=south] node[pos=0.5]{\ding{54}} (tgt);
\end{forest}}
\z
              
Notice that two of \textit{v}’s features, [\textit{u}D, \textit{u}Asp], are \ac{EPP}-specified, triggering a goal to raise to the specifier of \textit{v}P. There are two matching goals here, the object, which has the matching D-feature, and the \ac{ASP} head, which has the matching Asp-feature. However, the operational rule of \textit{Multiple} \textit{Agree} \textit{under} \textit{Antisymmetry} states that only one goal can be spelled out at the specifier of \textit{v}P. Thus, only one goal should move to the specifier of \textit{v}P while this movement should be able to satisfy the \ac{EPP} property on both [\textit{u}D] and [\textit{u}Asp] on \textit{v}. We have seen that this is in fact possible by \ac{ASP}P raising to Spec, \textit{v}P, the derivation depicted in \REF{ex:118} where the \acs{VP} and the object together pied-pipe \ac{ASP}P, which satifies the \ac{EPP}-specifications on [\textit{u}D] and [\textit{u}Asp] on \textit{v} = \textit{ha}\textsuperscript{KR}\textit{/su}\textsuperscript{JP}. In fact, it is the only possible derivation since the \acs{VP} is inflexible. However, in \REF{ex:119} where an English light verb lexicalizes A\textsc{sp,} \acs{VP} is not projected. As explained earlier in Chapter \ref{ex:3}, this follows from the idea that structure is built via Merge between two lexical items, here in this case the light verb \textit{have} and the object, which is translated into the X-bar/tree structure. And the absence of \acs{VP} provides an answer to why the \ac{EPP} properties on \textit{v} are satisfied by \ac{ASP}P raising in \REF{ex:119}.

Normally when V is lexicalized by a verb and fully featurally specified, the features of the object do not manifest themselves on \acs{VP} due to the fact that there is a featural clash between the V head and the object. Here I appeal to the old dichotomous featural distinction between nominal elements/Ns and verbal elements/Vs proposed in the X-bar theory: N has [+N, -V] features and V has [-N, +V] features. In other words, when V is saturated by verbal features, the nominal features on the object cannot be present on the higher \acs{VP} due to the presence of contradictory features, [-N, +V] vs [+N, -V]. On the other hand, when V is empty/null, it contains only Asp and T features but is arguably lack of [-N, +V] features, which makes a verb a verb. I argue that this is the case with a light verb. Thus, the object’s features can be represented on \ac{ASP}P together with the light verb’s Asp and T features without a crash. Thus, all the features on the \ac{ASP} head and the object percolate onto the \ac{ASP}P, as in \REF{ex:120}.

\ea\label{ex:120}
\adjustbox{max width = 0.9\textwidth}{\begin{forest}
[\textit{v}P, [\textit{v} \\ \textit{ha}\textsuperscript{KR}/\textit{su}\textsuperscript{JP} \\ {[\textit{u}$\upphi$, \textit{u}Asp\textsuperscript{EPP}, \textit{u}D\textsuperscript{EPP}]} ,name = src]
[\textsc{Asp}P\textsubscript{[Asp, T, $\upphi$, D]} 
[\textsc{Asp} \\ \textit{have}, name = tgt][OBJ \\ \textit{a small head}] { \draw (.east) node[right]{~~~~~~~$\Uparrow$}; }
] { \draw (.east) node[right]{\textsc{feature percolation\footnotemark}}; }
]
\end{forest}}
\z

\footnotetext{The idea of feature percolation was first proposed by \citet{Lieber1980} and \citet{Williams1981}.}

The \ac{ASP}P has an Asp-feature thanks to percolation from its head \ac{ASP} and also $\upphi$ and D-features thanks to percolation from the object. Now all the uninterpretable features on \textit{v} and their \ac{EPP} properties can be valued and satisfied via feature matching between \textit{v} and \ac{ASP}P. The \ac{EPP} properties on \textit{v} trigger \ac{ASP}P movement to Spec, \textit{v}P. Consequently, the final surface order \ac{VO}-\textit{ha}/\textit{su} is derived after \ac{ASP}P raises to the left of \textit{v} = \textit{ha}\textsuperscript{KR}\textit{/su}\textsuperscript{JP}, as shown in \REF{ex:121}. 

\ea\label{ex:121}
\adjustbox{max width = 0.9\textwidth}{\begin{forest}
[\textit{v}P, s sep = 25mm 
[\textsc{Asp}P$_i$, name = tgt [\textsc{Asp} \\ \textit{have}] 
[OBJ \\ \textit{a small head}]]
[\textit{v}P [\textit{v} \\ \textit{ha}\textsuperscript{KR}/\textit{su}\textsuperscript{JP}  ][t$i$, name = src]]]
\draw[->, dashed] (src) to[out=south west, in =east] (tgt) ;
\end{forest}}
\z

\section{English light verbs and non-literal interpretations}\label{ch5:sect:5.4}

When an English light verb takes an object in a non-literal or idiomatic interpretation (e.g. \textit{have a big mouth}), \ac{VO} order was preferred both in Korean-English and Japanese-English \ac{CS} (69\% and 72\%, respectively), similar to non-literal phrases with an English heavy verb (Type 2). 

As shown in \figref{fig:2.4again}, \acs{VP} idioms with both heavy verbs and light verbs were judged less flexible than non-idiomatic phrases, thus suggesting the \ac{ASP}P headed by the light verb \textit{have} in \REF{ex:122}, for instance, is syntactically inflexible, which makes object extraction difficult, similar to the example in \REF{ex:117}.

\ea\label{ex:122}
\adjustbox{max width = 0.9\textwidth}{\begin{forest}
[\textit{v}P, [\textit{v} \\ \textit{ha}\textsuperscript{KR}/\textit{su}\textsuperscript{JP} \\ {[\textit{u}$\upphi$, \textit{u}Asp\textsuperscript{EPP}, \textit{u}D\textsuperscript{EPP}]} ,name = src]
[\textsc{Asp}P\textsubscript{[Asp, T, $\upphi$, D]},
tikz={\node [draw,ellipse,inner sep=-1pt,fit to=tree,label=below:inflexible] {};}
[\textsc{Asp} \\ \textit{have}, name = tgt][OBJ \\ \textit{a small head}] 
]
]
\end{forest}}
\z

Also, the \ac{ASP} head filled by an English light verb does not inherit probing features from \textit{v} = \textit{ha}\textsuperscript{\MakeUppercase{kr}}/\textit{su}\textsuperscript{\MakeUppercase{jp}}\MakeUppercase{,} and all of \textit{v}’s features need to be valued against a goal with corresponding features. Similar to \REF{ex:120}, the features of \ac{ASP} and the object percolate onto \ac{ASP}P without a crash. Thus, all of \textit{v}’s features can be valued via \ac{ASP}P raising, as a result of which the surface \ac{VO}-\textit{ha/su} order is derived in \REF{ex:123}.

\ea\label{ex:123}
\adjustbox{max width = 0.9\textwidth}{\begin{forest}
[\textit{v}P, s sep = 25mm 
[\textsc{Asp}P$_i$, name = tgt [\textsc{Asp} \\ \textit{have}] 
[OBJ \\ \textit{a big mouth}]]
[\textit{v}P [\textit{v} \\ \textit{ha}\textsuperscript{KR}/\textit{su}\textsuperscript{JP}  ][t$i$, name = src]]]
\draw[->, dashed] (src) to[out=south west, in =east] (tgt) ;
\end{forest}}
\z

English idioms with a heavy verb (Type 2) and with a light verb (Type 4) were preferred in \ac{VO} order alike in Korean-English and Japanese-English \ac{CS}. Yet, their syntactic derivations to deliver \ac{VO} order slightly differ from each other under the current proposal that English lexical verbs are V and English light verbs correspond to \ac{ASP}. When an English \acs{VP} idiom includes a heavy verb, the head \ac{ASP} is empty and can in principle inherit \textit{v}’s features. However, \ac{FI} is blocked because of the failure of object shift, which is caused by the inflexibility of the \acs{VP} idiom. The inflexibility of the \acs{VP} results in \ac{ASP}P pied-piping by object shift, and the \ac{EPP}-specifications of [\textit{u}D] and [\textit{u}Asp] on \textit{v} = \textit{ha}\textsuperscript{\MakeUppercase{kr}}/\textit{su}\textsuperscript{\MakeUppercase{jp}}\MakeUppercase{} are satisfied by \ac{ASP}P raising; movement of both \acs{VP} and the object pied-pipe \ac{ASP}P, whose effect is ``killing two birds with one stone.''

With an idiom with an English light verb, on the other hand, \ac{FI} cannot happen because \ac{ASP} is lexically filled by the English light verb. Thanks to the fact that V is not projected in the structure, all the features of \ac{ASP} and the object percolate onto \ac{ASP}P, and A\textsc{spP} agrees with \textit{v} = \textit{ha}\textsuperscript{\MakeUppercase{kr}}/\textit{su}\textsuperscript{\MakeUppercase{jp}}\MakeUppercase{.} The \ac{EPP}-specifications of [\textit{u}D] and [\textit{u}Asp] on \textit{v} = \textit{ha}\textsuperscript{\MakeUppercase{kr}}/\textit{su}\textsuperscript{\MakeUppercase{jp}}\MakeUppercase{} are satisfied by A\textsc{spP} raising. What is common between Type 2 (English \acs{VP} idioms with a heavy verb) and Type 4 (English \acs{VP} idioms with a light verb) is that \ac{FI} from \textit{v} to \ac{ASP} does not take place and \ac{VO} order is derived after \ac{ASP}P raising without object shift in Korean-English and Japanese-English \ac{CS}. Yet, it is not an idiom \textit{per} \textit{se} that prevents object shift. What matters is the degree of syntactic flexibility of the idiom. 

\section{Type 5: English heavy verbs in light verb constructions}\label{ch5:sect:5.5}

In a light verb construction where an English heavy verb takes an object (e.g. \textit{play} \textit{a} \textit{trick}), the average percentage of \ac{VO} order preference was 41\% in Korean-English \ac{CS} (and 47\% in Japanese-English \ac{CS}). While \ac{OV} order was slightly preferred with several items (e.g. \textit{deliver a talk}, \textit{deliver a speech}, \textit{reach an agreement}, \textit{pass sentence,} \textit{play joke}, and \textit{pay a compliment}), the majority of items were not biased towards either order in Korean-English \ac{CS}, ranging from 36\% to 64\% of \ac{VO} preference: to put it differently, when a light verb construction with an English heavy verb was code-switched, both \ac{OV} and \ac{VO} orders were possible.\footnote{The only exception was \textit{pay a visit}, which was strongly preferred in \ac{VO} order (86\%), in contrast with its closely matched phrase, \textit{pay a compliment}, which was preferred in \ac{OV} order (81\%).  To see if the preferred word order contrast between them is related to different degrees of their syntactic flexibility, the results from the syntactic flexibility judgment task were compared. In summary, \textit{pay a visit} was judged to be less flexible than \textit{pay a compliment} only in relative clause formation (2.78 vs 3.52), but neither in passivization (3.07 vs 2.93) nor in \textit{wh}-question formation (2.52 vs 2.74), which seems to suggest that object shift and relativization are closely related in terms of information structure. Nonetheless, this pattern was not consistent across items and more research is needed.}

The alternation between \ac{OV} and \ac{VO} orders, I will argue, is related to the categorial status of the verb: in a light verb construction with a heavy verb, the verb may still represent V, following its lexical root, or \ac{ASP}, based on its “light” use in a light verb construction. And depending on the category the verb represents, either V or A\textsc{sp,} the code-switched phrase will alternate between \ac{OV} and \ac{VO} orders.

When the verb is V as in (\ref{ex:124}a), \ac{OV} order is derived; the null-headed \ac{ASP} inherits [\textit{u}$\upphi$, \textit{u}D\textsuperscript{EPP}] from \textit{v} = \textit{ha}\textsuperscript{\MakeUppercase{kr}}/\textit{su}\textsuperscript{\MakeUppercase{jp}}\MakeUppercase{} and triggers the object to move to its specifier position (\ref{ex:124}b), after which \ac{ASP}P is pied-piped and raises to Spec, \textit{v}P via agree between \textit{v} = \textit{ha}\textsuperscript{\MakeUppercase{kr}}/\textit{su}\textsuperscript{\MakeUppercase{jp}} and V. As a consequence, \ac{OV} order is derived (\ref{ex:124}c).

\ea\label{ex:124}
    \ea\adjustbox{max width = 0.9\textwidth}{\begin{forest}
    [\textit{v}P   [\textit{v} \\   \textit{ha}\textsuperscript{KR}/\textit{su}\textsuperscript{JP} \\ { [\colorbox{lightgray}{\textit{u}$\upphi$}, \textit{u}Asp\textsuperscript{EPP}, \colorbox{lightgray}{\textit{u}D\textsuperscript{EPP}}]}, name = src ]
    [\textsc{Asp}P [\textsc{Asp} \\ $\varnothing$ \\ {[\textit{u}$\upphi$, \textit{u}D\textsuperscript{EPP}]}, name = tgt ]
    [VP [V\textsubscript{[Asp,T]} \\ \textit{play} ]
    [OBJ\textsubscript{[$\upphi$, D]} \\ \textit{a trick}]]]]
    \draw[->] (src) to[in=south, out=south] (tgt) ;
    \end{forest}}
    \ex\adjustbox{max width = 0.9\textwidth}{\begin{forest}
    [\textit{v}P   [\textit{v} \\   \textit{ha}\textsuperscript{KR}/\textit{su}\textsuperscript{JP} \\ { [\colorbox{lightgray}{\textit{u}$\upphi$}, \textit{u}Asp\textsuperscript{EPP}, \colorbox{lightgray}{\textit{u}D\textsuperscript{EPP}}]}, name = src ]
    [\textsc{Asp}P [OBJ\textsubscript{\textit{i}[$\upphi$, D]} \\ \textit{a trick}] [\textsc{Asp}$'$ [\textsc{Asp} \\ $\varnothing$ \\ {[\sout{\textit{u}$\upphi$}, \sout{\textit{u}D}\textsuperscript{\sout{EPP}}]}, name = tgt ]
    [VP [V\textsubscript{[Asp,T]} \\ \textit{play} ]
    [t$i$]]]]]
    \end{forest}}
    \ex\adjustbox{max width = 0.9\textwidth}{\begin{forest}
    [\textit{v}P
    [\textsc{Asp}P$_k$ [OBJ\textsubscript{\textit{i}[$\upphi$, D]} \\ \textit{a trick}] [\textsc{Asp}$'$ [\textsc{Asp}]
    [VP [V\textsubscript{[Asp,T]} \\ \textit{play} ]
    [t$i$]]]]
    [\textit{v}$'$ [\textit{v} \\   \textit{ha}\textsuperscript{KR}/\textit{su}\textsuperscript{JP}][t$_k$]]]
    \end{forest}}
    \z
\z

Alternatively, the verb may lexicalize \ac{ASP}, following its function similar to the light verb in a light verb construction, despite the fact that it is originally a lexical verb. When the verb merges as \ac{ASP}, \ac{FI} from \textit{v} = \textit{ha}\textsuperscript{\MakeUppercase{kr}}/\textit{su}\textsuperscript{\MakeUppercase{jp}}\MakeUppercase{} to the lexically filled \ac{ASP} does not take place. Also the object’s features can be present on the \ac{ASP}P, together with \ac{ASP}’s features, as shown in (\ref{ex:125}a). After that, feature matching between \textit{v} and \ac{ASP}P takes place, and the \ac{EPP} properties on \textit{v} triggers \ac{ASP}P-raising, which derives \ac{VO}-\textit{ha} and \ac{VO}-\textit{su} orders, as illustrated in (\ref{ex:125}b). 

\ea\label{ex:125}
    \ea \adjustbox{max width = 0.9\textwidth}{\begin{forest}
[\textit{v}P, [\textit{v} \\ \textit{ha}\textsuperscript{KR}/\textit{su}\textsuperscript{JP} \\ {[\textit{u}$\upphi$, \textit{u}Asp\textsuperscript{EPP}, \textit{u}D\textsuperscript{EPP}]} ,name = src]
[\textsc{Asp}P\textsubscript{[Asp, T, $\upphi$, D]} 
[\textsc{Asp} \\ \textit{play}, name = tgt][OBJ \\ \textit{a trick}] { \draw (.east) node[right]{~~~~~~~$\Uparrow$}; }
] { \draw (.east) node[right]{\textsc{feature percolation}}; }
]
\draw[->] (src) to[in=south west,out=south] node[]{\ding{54}} (tgt);
\end{forest}}
    \ex\adjustbox{max width = 0.9\textwidth}{\begin{forest}
[\textit{v}P, s sep = 35mm 
[\textsc{Asp}P$_i$, name = tgt [\textsc{Asp} \\ \textit{play}] 
[OBJ \\ \textit{a trick}]]
[\textit{v}P [\textit{v} \\ \textit{ha}\textsuperscript{KR}/\textit{su}\textsuperscript{JP}  ][t$i$, name = src]]]
\draw[->, dashed] (src) to[out=south west, in =east] (tgt) ;
\end{forest}}
    \z
\z

An anonymous reviewer mentions that a categorial item does not have a choice where it merges, thus questioning the current proposal that the heavy verb in a light verb construction may merge as V or \ac{ASP}. Yet, the idea is not so strange. In fact, a similar idea was proposed by \citet{Haider2013}, who argues that in addition to \ac{OV} and \ac{VO} languages, there is the third type of language in the world, which does not belong to either \ac{OV} or \ac{VO} languages in a strict sense. For instance, while the basic sentence structure of modern Germanic languages exhibits either \ac{OV} (e.g. West Germanic languages such as Afrikaans, Dutch, Frisian, German, Letzeburgish, Swiss German and all other more local varieties) or \ac{VO} (e.g. North Germanic languages such as Danish, Faroese, Icelandic, Norwegian, Swedish and all regional varieties), Yiddish exemplifies the third type of language, according to Haider. In order to account for the third type languages, Haider entertains the idea that the verb may end up in different positions, and this is possible since the directionality is unspecified: after all, the third type of language does not belong to either \ac{OV} or \ac{VO} languages, and the verb may remain in its base position as in \ac{OV} languages (\ref{ex:126}a = \ref{ex:65}a) or moves up to a higher position as in \ac{VO} languages (\ref{ex:126}b = \ref{ex:65}b). Recall that Haider assumes that \ac{VO} order is derived from \ac{OV} via verb movement, as explained in Chapter \ref{ch:3}. 

\ea\label{ex:126}
    \ea  ~[ IO [ DO [ PP V ]]] \hfill \textsc{ov} languages
    \ex ~[ V\textit{\textsubscript{i}} [ IO t\textit{\textsubscript{i''}} [ DO t\textit{\textsubscript{i'}} [ PP t\textit{\textsubscript{i}} ]]]] \hfill \textsc{vo} languages
    \z
\z

The present analysis does not share the exactly the same view offered in Hai\-der's: instead of V raising to \ac{ASP}, an English heavy verb in a light verb construction corresponds to either V or A\textsc{sp.} Nonetheless, there is a common view shared between the present analysis and Haider’s: when order is unspecified between \ac{OV} and \ac{VO}, the verb may be located in different positions.

Alternatively, we can modify the present proposal into the view that the verb raises to \ac{ASP} instead of merging as \ac{ASP} as in \REF{ex:127}. However, this may result in an undesirable result.

\ea\label{ex:127}\adjustbox{max width = 0.9\textwidth}{\begin{forest}
    [\textit{v}P   [\textit{v} \\   \textit{ha}\textsuperscript{KR}/\textit{su}\textsuperscript{JP} \\ { [\textit{u}$\upphi$, \textit{u}Asp\textsuperscript{EPP},{\textit{u}D\textsuperscript{EPP}}]}, name = src ]
    [\textsc{Asp}P [\textsc{Asp} {+} V\textsubscript{\textit{i}[Asp,T]} \\ \textit{play}, name = tgt ]
    [VP [ t$_i$]
    [OBJ\textsubscript{[$\upphi$, D]} \\ \textit{a trick}]]]]
    \draw[->] (src) to[in=south, out=south] node{\ding{54}} (tgt) ;
    \end{forest}}
\z
After V raises to \ac{ASP}, \textit{v} cannot pass down its features to the complex \ac{ASP} + V head, which has not only a phonological value but also has V’s features: only a featureless non-phase head can be a beneficiary of \ac{FI}. Thus, \textit{v} agrees with both the complex \ac{ASP} + V head, which has the Asp-feature, and the object, which has [${\upphi}$, D] features, and the \ac{EPP} properties on the Asp-feature and the D-feature trigger both \ac{ASP}P raising and object movement to Spec, \textit{v}P. And this is forbidden by the rule of \textit{Multiple} \textit{Agree} \textit{under} \textit{Antisymmetry}, which restricts the number of goals that can raise and be spelled out to only one.\footnote{\textrm{The nominal features on the object cannot percolate onto A}\textrm{\textsc{sp}}\textrm{P via \acs{VP} since the complex [A}\textrm{\textsc{sp}} \textrm{+ V] head contains the [+V] feature originated on the V head.}} Thus, the V to \ac{ASP} raising approach does not seem to return a desirable result, which leads to a derivational crash. For this reason, I will not assume that V raises to \ac{ASP} when an English heavy verb occurs in a light verb construction, although this is certainly parallel with Haider’s insight. More research should be done here.

In summary, the proposal that the heavy verb in a light verb construction may merge either as V or as \ac{ASP} accounts for the results from the Korean-English and Japanese-English \ac{CS} judgment task, which showed the preferred word order of most examples was not biased towards either \ac{OV} or \ac{VO}.

\section{Type 6: English light verbs in light verb constructions}\label{ch5:sect:5.6}

In a light verb construction where an English light verb takes an object (e.g. \textit{have} \textit{a} \textit{look}), the average percentage of \ac{VO} order preference was 69\% in Korean-English \ac{CS} and 71\% in Japanese-English \ac{CS}. The proposal that an English light verb is base-generated as \textsc{Asp} correctly accounts for the surface \ac{VO}-\textit{ha} and \ac{VO}-\textit{su} orders; \ac{FI} from \textit{v} = \textit{ha}\textsuperscript{KR}\textit{/su}\textsuperscript{JP} to the lexically filled \textsc{Asp} is blocked, and the object remains in situ. Due to the absence of the V head and its lack of verbal features, the \ac{ASP}P bears all the features from \ac{ASP} and the object, and \textit{v} and A\textsc{spP} enters a probe-goal relationship. The \ac{EPP} property on \textit{v} = \textit{ha}\textsuperscript{KR}\textit{/su}\textsuperscript{JP} induces movement of \ac{ASP}P to the left of \textit{v} = \textit{ha}\textsuperscript{KR}\textit{/su}\textsuperscript{JP}, resulting in \ac{VO}-\textit{ha}/\textit{su} order. An example is provided below \REF{ex:128}.

\ea\label{ex:128}
    \ea \adjustbox{max width = 0.9\textwidth}{\begin{forest}
[\textit{v}P, [\textit{v} \\ \textit{ha}\textsuperscript{KR}/\textit{su}\textsuperscript{JP} \\ {[\textit{u}$\upphi$, \textit{u}Asp\textsuperscript{EPP}, \textit{u}D\textsuperscript{EPP}]} ,name = src]
[\textsc{Asp}P\textsubscript{[Asp, T, $\upphi$, D]} 
[\textsc{Asp}\textsubscript{[Asp, T]} \\ \textit{have}, name = tgt][OBJ\textsubscript{[$\upphi$, D]} \\ \textit{a look}] { \draw (.east) node[right]{~~~~~~~$\Uparrow$}; }
] { \draw (.east) node[right]{\textsc{feature percolation}}; }
]
\draw[->] (src) to[in=south west,out=south] node[]{\ding{54}} (tgt);
\end{forest}}
    \ex\adjustbox{max width = 0.9\textwidth}{\begin{forest}
[\textit{v}P, s sep = 35mm 
[\textsc{Asp}P$_i$, name = tgt [\textsc{Asp} \\ \textit{have}] 
[OBJ \\ \textit{a look}]]
[\textit{v}P [\textit{v} \\ \textit{ha}\textsuperscript{KR}/\textit{su}\textsuperscript{JP}  ][t$i$, name = src]]]
\draw[->, dashed] (src) to[out=south west, in =east] (tgt) ;
\end{forest}}
    \z
\z

To summarize, \sectref{ch5:sect:5.1} to \sectref{ch5:sect:5.6} provided a \ac{FI}-based account of \ac{OV}-\ac{VO} variation in Korean-English and Japanese-En\-glish \ac{CS}. What crucially determines \ac{OV}-\ac{VO} variation in Korean-English and Japanese-English \ac{CS} is feature specification on \textit{v} in Korean and Japanese and \ac{FI} from \textit{v} to \ac{ASP}. When \textsc{Asp} inherits [\textit{u}$\upphi$, \textit{u}D \textsuperscript{EPP}] from \textit{v}= \textit{ha}\textsuperscript{\MakeUppercase{kr}}/\textit{su}\textsuperscript{\MakeUppercase{jp}}, object shift occurs within \ac{ASP}P, delivering \ac{OV} order within \ac{ASP}P. \ac{ASP}P further raises after feature matching between \textit{v} and \acs{VP}, and surface \ac{OV}-\textit{ha} and \ac{OV}-\textit{su} orders are derived. When \textsc{Asp} is overtly realized by an English light verb, on the other hand, \ac{FI} from \textit{v} to \ac{ASP} is blocked and all of \textit{v}’s features remain on \textit{v}. Subsequently, \ac{ASP}P raises to Spec, \textit{v}P without object shift, as a result of which \ac{VO}-\textit{ha} and \ac{VO}-\textit{su} orders surface in Korean-English and Japanese-English \ac{CS}.

When \ac{ASP} is null, nothing prevents \ac{FI}. However, \ac{FI} from \textit{v} to \textsc{Asp} is not possible if it leads to a derivational crash (the principle of \textit{Obligation}). We have seen this in the case of an idiomatic phrase with a heavy verb. Also, \ac{FI} is blocked when \textsc{Asp} is lexicalized by an English light verb, which is no longer featureless, therefore it is not a valid recipient head of \ac{FI} (the principle of \textit{Validation}). 

The \ac{FI}-based account of \ac{OV}-\ac{VO} variation in Korean-English and Japanese-En\-glish \ac{CS} predicts that \ac{OV} order and \ac{VO} order would be in a complementary distribution. Yet, the results from the \ac{CS} judgment task revealed that it may not be the case: in a given context, neither \ac{OV} nor \ac{VO} order was preferred 100\% except for a few examples, as mentioned earlier. One may ask whether the word order pattern predicted by the \ac{FI} account is clear at the level of individual speakers. A microscopic analysis of individual speakers shows that in a given condition (e.g. verb type and interpretation), a dominant word order pattern, either \ac{OV} or \ac{VO}, was preferred by many individual speakers, suggesting that \ac{FI} is at play at the individual level. Yet, there exists inter-subject variation: while the predicted word order is more robustly confirmed by some speakers, it may not be the case with others. I have no immediate answer for such subject variation and will leave this for future research.

\section{Reanalyzing English light verbs }\label{ch5:sect:5.7} 

Overall, the study found that functional/light verbs behave differently from lexical/heavy verbs in English. Except for idiomatic phrases, a code-switched phrase with an English heavy verb was preferred in \ac{OV} order in Korean-English and Japanese-English \ac{CS}. On the other hand, a code-switched phrase including an English light verb was generally favored in \ac{VO} order in all phrase types, including a light verb construction and literal interpretations. Under the proposal that English light verbs are \textsc{Asp} while heavy verbs are V, this is a desirable result. Yet, there were instances where the preferred order was \ac{OV}, not \ac{VO}, with some light verbs, which cannot be explained by the current proposal and needs further explanation. 

In the following, we look at each of the eight English light verbs in \REF{ex:28} that were included in the \ac{CS} experiment and their occurrences in different phrase types in relation to the preferred order. The item-based analyses are obtained from the results of the Korean-English \ac{CS} judgment task. Unless it is mentioned separately, the patterns of Japanese-English \ac{CS} data were parallel to those of Korean-English \ac{CS} data.

\subsection{\textit{Have}}\label{ch5:sect:5.7.1}

\tabref{tab:5.3} shows the percentages of \ac{VO} order preference of the code-switched phrase with the verb \textit{have} in Korean-English \ac{CS}.

\begin{table}
\caption{Item-based analysis for \textit{have} in Korean-English code-switching
(n$^0$ of occurrence 0–4: \textsc{OV} biased; 5–9: not biased; 10–14: \textsc{VO} biased)
}
\label{tab:5.3}\adjustbox{max width = \textwidth}{
 \begin{tabular}{lllrr} % add l for every additional column or remove as necessary
  \lsptoprule
 code-switched phrase & phrase type & preferred word order & \%V & n$^0$ of occurrence \\ %table header
  \midrule
have a look & LVC & \ac{VO} & 100 & 14\\
have a try & LVC & \ac{VO} & 93 & 13\\
have a small head & Literal & \ac{VO} & 86 & 12\\
have an upset stomach & Literal & \ac{VO} & 86 & 12\\
have a big mouth & Non-Lit & \ac{VO} & 79 & 11\\
have a total blast & Non-Lit & neither \ac{OV} nor \ac{VO} & 64 & 9\\
  \lspbottomrule
 \end{tabular}}
\end{table}

Except for the idiom \textit{have a total blast}, which was not favored in any particular order, \ac{VO} order was robustly preferred with the verb \textit{have} in all of its occurrences, revealing a stark contrast with the code-switched phrase with a heavy verb, with which most \ac{VO} occurrences were limited to idiomatic phrases. This contrast with heavy verbs is further highlighted in light verb constructions and literal interpretations: while \ac{VO} order was robustly favored with \textit{have} both in light verb constructions (e.g. \textit{have} \textit{a} \textit{look} and \textit{have} \textit{a} \textit{try}) and literal interpretations (e.g. \textit{have} \textit{a} \textit{small} \textit{head} and \textit{have} \textit{an} \textit{upset} \textit{stomach}), only one heavy verb in a light verb construction was favored in \ac{VO} order (e.g. \textit{pay} \textit{a} \textit{visit}), and 12 out of 16 heavy verbs in literal interpretations were very strongly preferred in \ac{OV} order. 


\largerpage
Such a remarkably consistent \ac{VO} order with the verb \textit{have} is distinct from the \ac{OV} order with heavy verbs and reveals that the nature of the verb \textit{have} is very different from heavy verbs. \ac{VO} order was consistent with the verb \textit{have} not only in a light verb construction where the verb does not contribute any lexical-semantic information but also in other phrases in which the meaning of possession arises with \textit{have} (e.g. \textit{have} \textit{a} \textit{small} \textit{head}, \textit{have} \textit{a} \textit{big} \textit{mouth}). This further tells us that the possessive meaning of \textit{have} is not intrinsic to the verb, but is derived from the syntactic structure of the light verb \textit{have}. For instance, researchers have proposed that \textit{have} is underlyingly decomposed into two abstract elements such as \textsc{be} \textsc{+} \textsc{to} (\citealt{Benveniste1966,DenDikken1995,Kim2012}), from which the possessive interpretation of \textit{have} is derived. Cross-linguistically, we find possessive constructions are expressed by copular(-like) constructions with a dative preposition, shown in \REF{ex:129}. Based on this, I conclude that \textit{have} is a light verb in all its uses and lexicalizes a functional category, which I claimed to be \ac{ASP}.

\ea\label{ex:129}
    \ea \gll Bibi-ka    chayk-ul   kaci-ess-ta  \\ 
    Bibi-\textsc{nom} book-\textsc{acc} have-\textsc{past-decl} \\ \hfill  Korean
    \ex \gll Bibi-eykey chayk-i     iss-ta \\
    Bibi-\textsc{dat}     book-\textsc{nom} exist-\textsc{decl} \\
    \ex \gll Bibinique a  le livre.   \\
     Bibinique has the book  \\  \hfill  French 
    \ex \gll Le  livre  est à Bibinique. \\
     the book is   to Bibinique \\
     \glt `Bibi has a book.'
     \ex \gll habeo libram  \\ 
        I-have book.\textsc{acc}    \\    \hfill Latin
     \ex \gll mihi    est liber          \\
        me.\textsc{dat} is   book.\textsc{nom}  \\
        \glt `I have a book.' (\citealt[130, (46)]{DenDikken1995})
     \z
\z

\subsection{\textit{Get}}\label{ch5:sect:5.7.2}
\largerpage
\ac{VO} order was favored with the verb \textit{get} in 3 out of 6 examples, including an idiom (e.g. \textit{get} \textit{a} \textit{grip}) and a light verb construction (e.g. \textit{get} \textit{a} \textit{sense}, \textit{get} \textit{a} \textit{suntan}), while the other 3 examples were not biased toward either \ac{OV} or \ac{VO} order, as summarized in \tabref{tab:5.4}.

\begin{table}
\caption{Item-based analysis for \textit{get} in Korean-English code-switching
(n$^0$ of occurrence 0–4: \textsc{OV} biased; 5–9: not biased; 10–14: \textsc{VO} biased)
}
\label{tab:5.4}\adjustbox{max width = \textwidth}{
 \begin{tabular}{lllrr} % add l for every additional column or remove as necessary
  \lsptoprule
 code-switched phrase & phrase type & preferred word order & \%V & n$^0$ of occurrence \\ %table header
  \midrule
get a grip & Non-lit & \ac{VO} & 100 & 14\\
get a sense & LVC & \ac{VO} & 71 & 10\\
get a suntan & LVC & \ac{VO} & 71 & 10\\
get a cold sore & Lit & neither \ac{OV} nor \ac{VO} & 64 & 9\\
get a new girlfriend & Lit & neither \ac{OV} nor \ac{VO} & 64 & 9\\
get cold feet & Non-lit & neither \ac{OV} nor \ac{VO} & 64 & 9\\
  \lspbottomrule
 \end{tabular}}
\end{table}

\tabref{tab:5.4} shows that \ac{VO} order was strongly preferred with half of the examples with the verb \textit{get} whereas the other phrases were not biased towards either \ac{OV} or \ac{VO} order. But we can see that \ac{OV} order was never preferred more than \ac{VO} order in any of the examples, which suggests that the verb \textit{get} behaves differently from heavy verbs. In addition, it is noticeable that the \acs{VP} idiom \textit{get} \textit{a} \textit{grip} was unanimously preferred in \ac{VO} order by all of the Korean-English and Japanese-English bilingual speakers, which exhibits a huge contrast with idioms with heavy verbs:  no idioms with a heavy verb was 100\% preferred in \ac{VO} order. Besides, both \textit{get}{}-light verb constructions (e.g. \textit{get} \textit{a} \textit{sense}, \textit{get} \textit{a} \textit{suntan}) were also favored in \ac{VO} order, which is again in contrast with light verb constructions with heavy verbs: only one item \textit{pay} \textit{a} \textit{visit} was preferred in \ac{VO} order. Thus, it is reasonable to conclude that \textit{get}, along with \textit{have}, is a light verb, and it does not merge as V but lexicalizes a functional category.


The present finding supports the existing analyses of \textit{get} in the literature in which \textit{get} is analyzed as an abstract syntactic head representing inchoativity, such as \textsc{become} \citep{McIntyre2005a} or \textsc{inch} \citep{Shim2006}. As exemplified in \REF{ex:130}, the English verb \textit{get} has a variety of uses/meanings, ranging from onset of possession (\ref{ex:130}a), ingressive (\ref{ex:130}b--c), \textit{get}-passive (\ref{ex:130}d), and experiencer-\textit{get} (\ref{ex:130}e). With the exception of the experiencer-\textit{get} construction, each use of \textit{get} has inchoative (\ref{ex:130}a--d) and causative varieties (\ref{ex:130}$'$a--d)

\exewidth{(235)}
\ea\label{ex:130}
    \ea Joa got a present (from a friend).	
    \ex Joa got tired.
    \ex Joa got to the airport on time. 
    \ex Joa got fired/hired. 
    \ex Bibi got people telling lies about him.
    \z
\exp{ex:130}
    \ea Bibi got Joa a present.
    \ex Bibi got Joa tired.
    \ex Bibi got Joa to the airport on time.
    \ex Bibi got Joa fired/hired.
    \z
\z

\exewidth{(23)}

\citet{McIntyre2005a} broadly divides the examples in \REF{ex:130} into unaccusative and transitive uses of \textit{get}, as shown in \REF{ex:131} and \REF{ex:132}, and only a subset of the latter involves the causative component. 

\ea\label{ex:131} Unaccusative \textit{get}          
    \ea Joa got to the airport on time. \hfill      \textit{get} = \textsc{become}
    \ex Joa got tired.    \hfill \textit{get} = \textsc{become}
    \ex Joa got fired/hired. \hfill  \textit{get} = \textsc{become}
    \z
\clearpage
\ex\label{ex:132}
    \ea Joa got a present.  \hfill  \textit{get} = \textsc{become} \textsc{+} \textsc{have}
    \ex Bibi got Joa a present. \hfill   \textit{get} = \textsc{become} \textsc{+} \textsc{have}
    \ex Bibi got Joa to the airport on time. \hfill  \textit{get} = \textsc{become} \textsc{+} \textsc{have}\textsuperscript{resp}
    \ex Bibi got Joa tired. \hfill \textit{get} = \textsc{cause} \textsc{+} \textsc{become}
    \ex Bibi got Joa fired/hired. \hfill  \textit{get} = \textsc{cause} \textsc{+} \textsc{become}
    \ex Joa got people telling lies about her. \hfill    \textit{get} = \textsc{become} \textsc{+} \textsc{have}\textsuperscript{unint}  
    \z
\z

What is common to all uses of \textit{get} in \REF{ex:131} and \REF{ex:132} is \textsc{become}, which McIntyre adopts from \citegen{Dowty1979}, which is arguably the underlying representation of get in most uses of \textit{get}. To put it differently, \textit{get} is underlyingly an unaccusative verb and is analyzed as a spell-out of \textsc{become}. Similar to McIntyre, \citet{Shim2006} also proposes a unified analysis of various \textit{get}-constructions in which \textit{get} spells-out an abstract head \textsc{inch}, which denotes pure inchoativity. For further discussion of \textit{get}, readers are advised to refer to \citet{McIntyre2005a} and \citet{Shim2006}.


\subsection{\textit{Keep}}\label{ch5:sect:5.7.3}

Similar to \textit{have} and \textit{get}, most code-switched phrases with the verb \textit{keep} were strongly preferred in \ac{VO} order in Korean-English \ac{CS}, which suggests that \textit{keep} is also a light verb. Yet, there was one instance of \textit{keep} which displayed a noticeable difference from the other examples. \tabref{tab:5.5} below summarizes the result. 

\begin{table}
\caption{Item-based analysis for \textit{keep} in Korean-English code-switching
(n$^0$ of occurrence 0–4: \textsc{OV} biased; 5–9: not biased; 10–14: \textsc{VO} biased)
}
\label{tab:5.5}\adjustbox{max width = \textwidth}{
 \begin{tabular}{lllrr} % add l for every additional column or remove as necessary
  \lsptoprule
 code-switched phrase & phrase type & preferred word order & \%V & n$^0$ of occurrence \\ %table header
  \midrule
keep close watch & LVC & \ac{VO} & 93 & 13\\
keep track & LVC & \ac{VO} & 93 & 13\\
keep your cool & Non-Lit & \ac{VO} & 79 & 11\\
keep a respectful manner & Lit & \ac{VO} & 71 & 10\\
keep a civil tongue & Non-Lit & \ac{VO} & 71 & 10\\
keep your receipt & Lit & \ac{OV} & 0 & 0\\  \lspbottomrule
 \end{tabular}}
\end{table}

\largerpage
Among 6 phrases with the verb \textit{keep}, \textit{keep} \textit{the} \textit{receipt} was the only phrase that was universally chosen in \ac{OV} order by both Korean-English and Japanese-English bilingual speakers. What is striking is that the choice of \ac{OV} order was unanimous in both bilingual groups. It should be also noted that there were only 5 switching items which were unanimously preferred in \ac{OV} order, among which \textit{keep} \textit{the} \textit{receipt} was the only phrase with a light verb and the other 4 examples included heavy verbs in literal interpretations (e.g. \textit{break} \textit{the} \textit{glass}, \textit{spill} \textit{the} \textit{soup}, \textit{throw} \textit{his} \textit{cell} \textit{phone}, \textit{feel} \textit{the} \textit{pain}). Thus, the unexpected \ac{OV} order of the phrase \textit{keep} \textit{the} \textit{receipt} calls for a further analysis. To do so, we compare two phrases \textit{keep} \textit{your} \textit{receipt} and \textit{keep} \textit{a} \textit{respectful} \textit{manner}, both of which are interpreted literally. Yet, the former is preferred in \ac{OV} order and the latter in \ac{VO}. So, what makes the verb \textit{keep} in \textit{keep} \textit{your} \textit{receipt} different from \textit{keep} \textit{a} \textit{respectful} \textit{manner} and other uses or meanings of \textit{keep}? 

In Chapter \ref{ch:2}, the verb \textit{keep} was proposed as a light verb, representing \textit{keep} = \textsc{caus} \textsc{+} \textsc{be}. While this analysis can represent the meaning of the verb \textit{keep} in most phrases, including \textit{keep} \textit{a} \textit{respectful} \textit{manner,} it does not seem to convey the meaning of the verb in \textit{keep} \textit{your} \textit{receipt} where the verb is interpreted as ‘to retain or to save’. In other words, \textit{keep} in \textit{keep} \textit{your} \textit{receipt} is not as light as \textit{keep} in other examples. And this contrast seems to be reflected in different word orders in \ac{CS}. The difference of \textit{keep} in \textit{keep} \textit{your} \textit{receipt} from other uses of the verb is not just limited to its semantics. The verb \textit{keep} in \textit{keep} \textit{your} \textit{receipt} arguably selects a secondary predicate, as in \textit{keep} \textit{your} \textit{receipt} \textit{with} \textit{you}, and I propose that the verb \textit{keep} in \textit{keep} \textit{your} \textit{receipt} takes a small clause headed by a null preposition as its complement, shown in
\REF{ex:133}.

\ea\label{ex:133}
\begin{forest} fairly nice empty nodes,
[VP[V \\ \textit{keep}]
[SC [DP\textsubscript{SUB} \\ \textit{your receipt}]
[[P\textsubscript{$\varnothing$} \\ (\textit{with}) ][pred \\ (\textit{you})]]]]
\end{forest}
\z

In \REF{ex:133} the verb \textit{keep} is V rather than \ac{ASP}, and takes a null-headed small clause complement in which \textit{your} \textit{receipt} is the subject of the null predicate. What is crucial is that \textit{keep} is V, not \ac{ASP}, based on the fact that a small clause is selected by V only, but nothing else. The fact that only verbs can select a small clause as their complement is further supported by the contrast shown in \REF{ex:134} and \REF{ex:135} (cf \citealt{Kayne1984}).

\ea\label{ex:134}
    \ea Bibi considers [\textsubscript{SC} Joa smart]
    \ex The old man robbed [\textsubscript{SC} Joa of her wallet]
    \ex The bank credited [\textsubscript{SC} Joa with the money]
    \z
\ex\label{ex:135}
    \ea[*]{Bibi’s consideration of [\textsubscript{SC} Joa smart]}
    \ex[*]{The old man’s robbery of [\textsubscript{SC} Joa of her wallet]}
    \ex[*]{The bank’s credit of [\textsubscript{SC} Joa with the money]}
    \z
\z

\largerpage
In \REF{ex:134} the verb (e.g. \textit{consider}, \textit{rob}, \textit{credit}) takes a small clause as its complement. But their nominal counterparts, including derived nominals (e.g. \textit{consideration}, \textit{robbery}, \textit{credit}) do not accept secondary predicates in \REF{ex:135}. The same is true for deverbal adjectives: the subject of the small clause in \REF{ex:134} can be passivized, as exemplified in \REF{ex:136}. However, their adjectival counterparts in \REF{ex:137} are ungrammatical, which indicates that adjectives cannot co-occur with a secondary predicate.

\ea\label{ex:136}
    \ea Joa is considered smart.
    \ex Joa was robbed of her wallet.
    \ex Joa was credited with her money.
    \z
\ex\label{ex:137}
    \ea[*]{Joa is considerable smart.} 
    \ex[*]{Joa was robbable of her wallet .}  
    \ex[*]{Joa was creditable with her money.} 
    \z
\z

The examples provided above demonstrate that it is indeed only verbs that can select a secondary predicate. To put it differently, whenever there is a small clause complement, there is V. Thus, from the structure in \REF{ex:133}, \ac{OV} order of the phrase \textit{keep} \textit{the} \textit{receipt} is derived, which is depicted in \REF{ex:138}.

\ea\label{ex:138}\adjustbox{max width = 0.9\textwidth}{
\begin{forest} fairly nice empty nodes,
[\textit{v}P [\textit{v} \\ \textit{ha}\textsuperscript{KR}/\textit{su}\textsuperscript{JP} \\ {[\colorbox{lightgray}{\textit{u}$\upphi$}, \textit{u}Asp\textsuperscript{EPP}, \colorbox{lightgray}{\textit{u}D\textsuperscript{EPP}}]}, name = src]
[\textsc{Asp}P [\textsc{Asp} \\ $\varnothing$ \\ {[\textit{u}$\upphi$, \textit{u}D\textsuperscript{EPP}]}, name = tgt]
[VP[V \\ \textit{keep}]
[SC [DP\textsubscript{SUB} \\ \textit{your receipt}]
[[P\textsubscript{$\varnothing$} \\ (\textit{with}) ][pred \\ (\textit{you})]]]]]
]
\draw[->] (src) to[out=south,in=south west] (tgt);
\end{forest}}
\z

In \REF{ex:138} the null-headed \ac{ASP} inherits [\textit{u}$\upphi$, \textit{u}D\textsuperscript{EPP}] from \textit{v} and \ac{ASP} agrees with the subject of the small clause. The \ac{EPP} property on the D feature on \textit{v}, which is inherited by A\textsc{sp,} triggers DP\textsubscript{\MakeUppercase{sub}} (\textit{your} \textit{receipt}) move to Spec, \ac{ASP}P, which delivers the \ac{OV} ordered \textit{your} \textit{receipt} \textit{keep} within \ac{ASP}P.  \textit{v} = \textit{ha}\textsuperscript{\MakeUppercase{kr}}/\textit{su}\textsuperscript{\MakeUppercase{jp}}and V agree and the \ac{EPP}-specification on the Asp-feature on \textit{v}, the entire \ac{ASP}P further moves left of \textit{v} = \textit{ha}\textsuperscript{\MakeUppercase{kr}}/\textit{su}\textsuperscript{\MakeUppercase{jp}}, yielding to surface \ac{OV}-\textit{ha/su} order. So, the proposal that \textit{keep} in \textit{keep} \textit{your} \textit{receipt} is not a light verb but a lexical verb which takes a secondary predicate explains why \ac{OV} order is derived in Korean-English and Japanese-English \ac{CS}. 

The analysis that the English verb \textit{keep} meaning `to retain or to save' is a lexical verb taking a small-clause complement, which is otherwise an aspectual light verb, is corroborated by cross-linguistic examples. For example, in the sense of ‘to retain or to save’, the verb \textit{keep} in Dutch is translated as a particle verb, containing an additional secondary predicate, which is the prefixal particle \textit{be}- as in (\ref{ex:139}a) (\citealt{HoesktraEtAl1987}).\footnote{\textrm{The term} \textrm{\textit{particle}} \textrm{refers to the class of non-Case-assigning, argument-taking prepositional elements (\citealt[33,~fn 31]{DenDikken1995})}} On the other hand, \textit{keep} is not compatible with the prefixal particle in its aspectual interpretation. This is illustrated in (\ref{ex:139}b).

\ea\label{ex:139}
    \ea \gll je    moet het bonnetje be-houden/be-waren \\
    you must the receipt   \textsc{be}-keep/\textsc{be}-keep \\
    \glt `You must keep your receipt.'
    \ex \gll je    moet je   mond (*be)-houden \\
    you must your mouth \textsc{be}-keep \\
    \glt `You must hold your tongue.'     \hfill (Den Dikken, p.c.)
    \z
\z


If we follow \citegen{DenDikken1995} proposal that a particle merges as a small clause head, the verb in (\ref{ex:139}a) should be V rather than \textsc{Asp}, which takes a small clause headed by the particle \textit{be-} as its complement, similar to its counterpart in English in \REF{ex:133}. To summarize, cross-linguistic evidence obtained from English and Dutch provides a uniform analysis of \textit{keep} as a light verb except for its sense of ‘to retain or save’, for which \textit{keep} is a lexical verb taking a secondary predicate. 

\subsection{\textit{Hold}}\label{ch5:sect:5.7.4}

In the present \ac{CS} study, the verb \textit{hold} was treated as a light verb, which is further decomposed into \textsc{caus} \textsc{+} \textsc{be}, the structure that was also proposed for its synonymous verb \textit{keep}. However, the results from the \ac{CS} judgment task revealed a huge disparity between these two verbs. As we saw, the verb \textit{keep} behaved as a light verb in most code-switched phrases, most of which were preferred in \ac{VO} order in Korean-English and Japanese-English \ac{CS}. The only exception was when \textit{keep} was used as a heavy verb meaning ‘to retain’, exhibiting \ac{OV} order. 

By contrast, the majority of the \acs{VP}s including \textit{hold} were preferred in \ac{OV} order in Korean-English and Japanese-English \ac{CS}, with an exception of the \acs{VP} idiom \textit{hold} \textit{water} ‘to be sound and valid’, whose order was favored in \ac{VO} by both Korean-English and Japanese-English bilingual speakers. \tabref{tab:5.6} summarizes this.

\begin{table}
\caption{Item-based analysis for \textit{hold} in Korean-English code-switching
(n$^0$ of occurrence 0–4: \textsc{OV} biased; 5–9: not biased; 10–14: \textsc{VO} biased)
}
\label{tab:5.6}\adjustbox{max width = \textwidth}{
 \begin{tabular}{lllrr} % add l for every additional column or remove as necessary
  \lsptoprule
 code-switched phrase & phrase type & preferred word order & \%V & n$^0$ of occurrence \\ %table header
  \midrule
hold water & Non-Lit & \ac{VO} & 71 & 10\\
hold water & Lit & neither \ac{OV}  or \ac{VO} & 43 & 6\\
hold the bowl & Lit & \ac{OV} & 29 & 4\\
hold the fort & Non-Lit & \ac{OV} & 29 & 4\\
hold a conversation & LVC & \ac{OV} & 29 & 4\\
hold a debate & LVC & \ac{OV} & 21 & 3\\ \lspbottomrule
 \end{tabular}}
\end{table}

The contrast between \textit{hold} and other verbs that were included as light verbs in this study was also observed in light verb constructions: while \ac{OV} order was dominant with two light verb constructions with \textit{hold} (e.g. \textit{hold} \textit{a} \textit{conversation}, \textit{hold} \textit{a} \textit{debate}), none of light verb construction with other light verbs was chosen in \ac{OV} order. In fact, the word order pattern of the code-switched phrase including \textit{hold} seemed to be similar to that of the code-switched phrase with a heavy verb: a number of light verb constructions with various heavy verbs were favored in \ac{OV} order, which further suggests that \textit{hold} is not a light verb but perhaps a heavy or a lexical verb.

While \textit{hold} and \textit{keep} are usually considered to be synonyms, \citet{Levin1993} makes a subtle distinction between \textit{hold} verbs (e.g. \textit{clasp}, \textit{clutch}, \textit{grasp}, \textit{handle}, \textit{hold}, \textit{wield}) and \textit{keep} verbs (e.g. \textit{hoard}, \textit{keep}, \textit{leave}, \textit{store}): while the former describes “prolonged contact with an entity”, the latter is related to “maintaining something at some location” (pp 145--146). Along this line of thought, the decomposed structure of \textsc{caus} \textsc{+} \textsc{be} can represent the meaning of the light verb \textit{keep} ‘to maintain something’, but it may not reflect the extra semantic information contributed by \textit{hold} as explained by Levin. Although I have no further insight to explain how such subtle differences among diverse near-synonymous verbs are encoded in the argument structure, the heavy vs light distinction of verbs is not only a matter of lexical semantics, but is also reflected in the syntactic structure, as evidenced by word order contrast in this \ac{CS} study, which surfaced \ac{OV} and \ac{VO} in Korean-English and Japanese-English \ac{CS}, respectively. This is certainly a very interesting finding presented by the present \ac{CS} research, which might have not emerged from a study of monolingual speech.

\subsection{\textit{Make} and \textit{Take}}\label{ch5:sect:5.7.5}

\begin{table}[b]
\caption{Item-based analysis for \textit{make} in Korean-English code-switching
(n$^0$ of occurrence 0–4: OV biased; 5–9: not biased; 10–14: VO biased)
}
\label{tab:5.7}\adjustbox{max width = \textwidth}{
 \begin{tabular}{lllrr} % add l for every additional column or remove as necessary
  \lsptoprule
 code-switched phrase & phrase type & preferred word order & \%V & n$^0$ of occurrence \\ %table header
  \midrule
make waves & Non-Lit & \ac{VO} & 100 & 14\\
make a request & LVC & \ac{VO} & 86 & 12\\
make friends & Lit & \ac{VO} & 79 & 11\\
make a suggestion & LVC & neither \ac{OV}  or \ac{VO} & 57 & 8\\
make a bundle & Non-Lit & \ac{OV} & 29 & 4\\
make a million bucks & Lit & \ac{OV} & 7 & 1\\
\lspbottomrule
 \end{tabular}}
\end{table}

\begin{table}[b]
\caption{Item-based analysis for \textit{take} in Korean-English code-switching
(n$^0$ of occurrence 0–4: \textsc{ov} biased; 5–9: not biased; 10–14: \textsc{vo} biased)
}
\label{tab:5.8}\adjustbox{max width = \textwidth}{
 \begin{tabular}{lllrr} % add l for every additional column or remove as necessary
  \lsptoprule
 code-switched phrase & phrase type & preferred word order & \%V & n$^0$ of occurrence \\ %table header
  \midrule
take a hike & Non-Lit & \ac{VO} & 100 & 14\\
take a walk & LVC & \ac{VO} & 71 & 10\\
take a hike & Lit & neither \ac{OV}  or \ac{VO} & 64 & 9\\
take a vacation & LVC & neither \ac{OV}  or \ac{VO} & 57 & 8\\
take a back seat & Non-Lit & \ac{OV} & 29 & 4\\
take a window seat & Lit & \ac{OV} & 7 & 1\\
 \lspbottomrule
 \end{tabular}}
\end{table}


\tabref{tab:5.7} and \tabref{tab:5.8} show the percentages of \ac{VO} order occurrence with the verbs \textit{make} and \textit{take} in Korean-English \ac{CS}, respectively. They show that the word order patterns of a code-switched phrase with \textit{make} and \textit{take} resemble each other. For instance, three phrases with \textit{make}, including an idiom (e.g. \textit{make} \textit{waves}), a light verb construction (e.g. \textit{make} \textit{a} \textit{request}) and a literal expression (e.g. \textit{make} \textit{friends}), were preferred in \ac{VO} order. Analogously, two examples with \textit{take}, an idiom (e.g. \textit{take} \textit{a} \textit{hike}) and a light verb construction (e.g. \textit{take} \textit{a} \textit{walk}), were preferred in \ac{VO} order. On the other hand, two examples with \textit{make} and \textit{take}, including an idiom (e.g. \textit{make} \textit{a} \textit{bundle}, \textit{take} \textit{a} \textit{back} \textit{seat}) and a literal expression (e.g. \textit{make} \textit{a} \textit{million} \textit{bucks}, \textit{take} \textit{a} \textit{window} \textit{seat}), were favored in \ac{OV} order.




It should be noted that most idioms, regardless of the status of a verb, either heavy or light, were preferred in \ac{VO} order by both Korean-English and Japanese-English code-switchers, but the idioms \textit{make} \textit{a} \textit{bundle} ‘make a lot of money’ and \textit{take} \textit{a} \textit{back} \textit{seat} ‘take a less prominent in some situation or a lower priority’ were chosen in \ac{OV} order, along with their closely matched non-idiomatic expressions such as \textit{make} \textit{a} \textit{million} \textit{bucks} and \textit{take} \textit{a} \textit{window} \textit{seat.} Thus, it seems that idiomaticity is not a factor determining word order for these \textit{make}{}- and \textit{take}{}-idioms, which needs to be explained. Assuming that it is not the idiom \textit{per} \textit{se} that prevents object shift but the degree of its syntactic flexibility influences the extraction of the object of the \acs{VP}, an immediate question arose as to whether these two idioms were judged more flexible than other idioms by the Korean-English and Japanese-English bilingual speakers of the study. The results from the syntactic flexibility judgment task confirmed this:  the two idioms that were preferred in \ac{OV} order, \textit{make} \textit{a} \textit{bundle} and \textit{take} \textit{a} \textit{back} \textit{seat}, were judged more syntactically flexible than the other two \textit{make}{}- and \textit{take}{}- idioms (e.g. \textit{make} \textit{waves}, \textit{take} \textit{a} \textit{hike}), which were unanimously preferred in \ac{VO} order in Korean-English \ac{CS}, as summarized in \tabref{tab:5.9}. This suggests that the object may move out of the two idiomatic expressions, \textit{make} \textit{a} \textit{bundle} and \textit{take} \textit{a} \textit{back} \textit{seat}, and \ac{OV} order is derived in \ac{CS}. 

\begin{table}
\caption{Word order predicted by syntactic flexibility scores for \textit{make}- and \textit{take}-idioms in Korean-English \textsc{cs}}
\label{tab:5.9}
\fittable{
 \begin{tabular}{lrl} % add l for every additional column or remove as necessary
  \lsptoprule
 code-switched phrase & syntactic flexibility mean score	& preferred word order\\ %table header
  \midrule
  make a bundle & 2.95 & \ac{OV}\\
make waves & 2.11 & \ac{VO}\\
&  & \\
take a back seat & 2.19 & \ac{OV}\\
take a hike (non-lit) & 1.44 & \ac{VO}\\
  \lspbottomrule
 \end{tabular}
 }
\end{table}

However, as discussed in Chapter \ref{ch:2}, the correlation between the syntactic flexibility of a code-switched phrase and word order in \ac{CS} did not hold strongly across all items, and therefore, it needs further investigation in future research. 

Returning to \tabref{tab:5.7} and \tabref{tab:5.8}, the fact that light verb constructions such as \textit{make} \textit{a} \textit{request} and \textit{take} \textit{a} \textit{walk} were preferred in \ac{VO} order suggests that these verbs belong to the class of light verbs, exhibiting a similar word order pattern obtained from \textit{have}, \textit{get}, and \textit{keep}: light verb constructions with these verbs were predominantly selected in \ac{VO} order whereas light verb constructions with heavy verbs were generally favored in \ac{OV}. Yet, in a few examples with \textit{make} and \textit{take}, \ac{OV} order was favored (e.g. \textit{make} \textit{a} \textit{million} \textit{bucks}, \textit{make} \textit{a} \textit{bundle}, \textit{take} \textit{a} \textit{window} \textit{seat}, \textit{take} \textit{a} \textit{back} \textit{seat}). 

In Chapter \ref{ch:2}, I proposed that \textit{make} represents ‘\textsc{caus} + exist’ in \REF{ex:28}. While this analysis can capture the meaning of most phrases with the light verb \textit{make}, it seems to fail to express the meaning of the verb in \textit{make} \textit{a} \textit{million} \textit{bucks} and \textit{make} \textit{a} \textit{bundle} in which \textit{make} means ‘to earn’. Similarly, the proposal that \textit{take} is an aspectual light verb lexicalizing \textsc{inch} (or \textsc{become}) does not reflect the meaning of the verb in \textit{take} \textit{a} \textit{window} \textit{seat} and \textit{take} \textit{a} \textit{back} \textit{seat} where \textit{take} is interpreted ‘to accept’ or ‘to be prepared to get’. Thus, it seems that the verbs \textit{make} and \textit{take} in those particular examples denote idiosyncratic lexical meanings beyond the meaning of the light verbs, and they seem to be used as a heavy verb in these examples, which might have resulted in delivering \ac{OV} order in Korean-English and Japanese-English \ac{CS}.

\subsection{\textit{Give} and \textit{Raise}}\label{ch5:sect:5.7.6}

The word order patterns of code-switched phrases with the verb \textit{give} and \textit{raise} were similar to each other, which are presented in \tabref{tab:5.10} and \tabref{tab:5.11}, respectively.

\begin{table}
\caption{Item-based analysis for \textit{give} in Korean-English code-switching
(n$^0$ of occurrence 0–4: OV biased; 5–9: not biased; 10–14: VO biased)
}
\label{tab:5.10}\adjustbox{max width = \textwidth}{
 \begin{tabular}{lllrr} % add l for every additional column or remove as necessary
  \lsptoprule
 code-switched phrase & phrase type & preferred word order & \%V & n$^0$ of occurrence \\ %table header
  \midrule
give a big hand & Non-Lit & \ac{VO} & 93 & 13\\
give a presentation & LVC & neither \ac{OV}  or \ac{VO} & 57 & 8\\
give the axe & Non-Lit & neither \ac{OV}  or \ac{VO} & 50 & 7\\
give a big present & Lit & neither \ac{OV}  or \ac{VO} & 36 & 5\\
give a speech & LVC & neither \ac{OV}  or \ac{VO} & 36 & 5\\
give the job & Lit & \ac{OV} & 14 & 2\\
\lspbottomrule
 \end{tabular}}
\end{table}

\begin{table}
\caption{Item-based analysis for \textit{raise} in Korean-English code-switching
(n$^0$ of occurrence 0–4: OV biased; 5–9: not biased; 10--14: VO biased)
}
\label{tab:5.11}\adjustbox{max width = \textwidth}{
 \begin{tabular}{lllrr} % add l for every additional column or remove as necessary
  \lsptoprule
 code-switched phrase & phrase type & preferred word order & \%V & n$^0$ of occurrence \\ %table header
  \midrule
raise the bar & Non-Lit & \ac{VO} & 79 & 11\\
raise their hands & Lit & neither \ac{OV}  or \ac{VO} & 64 & 9\\
raise a few eyebrows & Non-Lit & neither \ac{OV}  or \ac{VO} & 64 & 9\\
raise some objection & LVC & neither \ac{OV}  or \ac{VO} & 43 & 6\\
raise some suspicions & LVC & neither \ac{OV}  or \ac{VO} & 36 & 5\\
raise the fee & Lit & \ac{OV} & 29 & 4\\
\lspbottomrule
 \end{tabular}}
\end{table}

The fact that most code-switched phrases with \textit{give} and \textit{raise} are not biased towards either \ac{OV} or \ac{VO} order in Korean-English \ac{CS} suggests that these verbs are not as light as \textit{have}, \textit{get}, or \textit{keep}, with which \ac{VO} order is strongly preferred in most of their uses. However, it is not clear whether they belong to heavy verbs based on the results of the \ac{CS} judgment task. While the occurrence of \ac{VO} order was limited to an idiomatic expression (e.g. \textit{give} \textit{a} \textit{big} \textit{hand}, \textit{raise} \textit{the} \textit{bar}) similar to heavy verbs, the preference of \ac{OV} order was also limited to only one example in literal interpretations with each verb (e.g. \textit{give} \textit{the} \textit{job}, \textit{raise} \textit{the} \textit{fee}), which differs from most lexical verbs included in the study, with which \ac{OV} order was predominant. For instance, with the verb \textit{hold}, which was initially proposed as a light verb, \ac{OV} order was favored 4 out of 6 examples, including its occurrence in a light verb construction, which mimic the word order pattern of heavy verbs (\tabref{tab:5.6}). But this is not the case with \textit{give} and \textit{raise}.


Due to the fact that the present findings do not reveal either \ac{OV} or \ac{VO} is a choice of word order with the verbs \textit{give} and \textit{raise}, I do not attempt to provide an analysis of their lexical status, and will leave it for future research. 

\section{Revisiting the contrast between light verbs and light verb constructions}\label{ch5:sect:5.8}

In this monograph, light verbs and light verb constructions were distinguished from each other and their definitions were provided in Chapter \ref{ch:1} \REF{ex:27}, which is repeated in \REF{ex:140}.

\ea\label{ex:140}
    \ea A \textbf{\textit{light verb}} never has idiosyncratic lexical meaning of its own, but only lexicalizes an abstract functional head. 
    \ex  In a \textbf{\textit{light} \textit{verb} \textit{construction}}, the verb does not contribute any lexical-semantic information, but only its complement does. Both heavy and light verbs may participate in light verb constructions.
    \z\z
    
\ea \label{ex:141}
    \ea Korean and Japanese \\
    \begin{forest}
    [\textit{v}P [SUB]
    [\textit{v}$'$, s sep = 1mm [\textit{v}\textsuperscript{[+EPP]} \\ \textit{ha}\textsuperscript{KR}/\textit{su}\textsuperscript{JP} ]
    [\textsc{Asp}P [\textsc{Asp} \\ $\varnothing$]
    [VP[V][OBJ]]]]]
    \end{forest}
    \ex English \\
    \begin{forest}
    [\textit{v}P [SUB]
    [\textit{v}$'$, s sep = 1mm [\textit{v}\textsuperscript{[-EPP]} \\ $\varnothing$ ]
    [\textsc{Asp}P [\textsc{Asp} \\ LV\textsubscript{ENG}]
    [OBJ]]]]
    \end{forest}
    \z
\z

The syntactic structures in \REF{ex:141} show that Korean/Japanese-type light verbs are \textit{v} and English-type light verbs are \ac{ASP}. In addition, they can represent the underlying structures of light verb constructions in these languages where the light verb itself does not contribute any lexical semantic information. While the structure in (\ref{ex:141}a) is the only possible syntactic configuration for light verb constructions in Korean and Japanese,\footnote{The verbal noun appears as V in Korean and Japanese light verb construction in (\ref{ex:134}a).} the structure in (\ref{ex:141}b) is one of two possible syntactic structures for light verb constructions in English. When a light verb participates in a light verb construction (e.g. \textit{have} \textit{a} \textit{look}), the light verb is \ac{ASP} as in (\ref{ex:141}b). On the other hand, if a heavy or lexical verb occurs in a light verb construction (e.g. \textit{play} \textit{a} \textit{trick}), the verb may be \ac{ASP} or V, as discussed in \sectref{ch5:sect:5.5}. Although both English heavy verbs and light verbs may participate in a light verb construction, their respectively different syntactic status as a lexical category and a functional category will play a role in syntactic derivations and be reflected in \ac{OV}-\ac{VO} in Korean-English and Japanese-English \ac{CS}. 

\section{Chapter summary and conclusion}\label{ch5:sect:5.9} 

This chapter offered a \ac{FI}-based account of \ac{OV}-\ac{VO} variation in Korean-English and Japanese-English \ac{CS}, which was tested against 28 Korean-English and 8 Japanese-English bilingual speakers’ introspective judgments of the \ac{CS} patterns presented to them in an experimental setting. Based on the proposal that morphosyntactic features on a phase head, such as C and \textit{v}, lead to linguistic parameterization and \ac{FI} occurs from a phase head to the head of its complement, I have shown that \ac{OV} order in Korean-English and Japanese-English \ac{CS} is a result of \ac{FI} from \textit{v} to \ac{ASP}, which results in object raising. If \ac{FI} does not happen, the object stays in situ and the underlying \ac{VO} order surfaces. In conclusion, the overall results from the experiments and grammatical accounts of word order patterns in Korean-English and Japanese-English \ac{CS} confirmed the two research hypotheses made in Chapter \ref{ch:1}, repeated below. 

\begin{exe}
\sn \textbf{Research Hypothesis 1} \\
 Assuming that linguistic variation is determined by the way features are parameterized in functional categories and how these features are valued in syntactic derivations, \ac{OV}-\ac{VO} variation in Korean-English and Japanese-English \ac{CS} will be determined by feature specifications on functional categories represented by light verbs in Korean, Japanese and English and how these features are valued in syntactic derivations.
\end{exe}

\begin{exe} 
\sn \textbf{Research} \textbf{Hypothesis} \textbf{2} \\
 Syntactically flexible phrases and inflexible phrases will behave differently with respect to word order derivation in \ac{CS}. More specifically, while the internal argument of a syntactically flexible phrase is subject to \ac{CS}, a syntactically inflexible phrase is frozen and undergoes \ac{CS} as a unit. Hence, the internal order of the phrase will be maintained throughout the derivation.
\end{exe}
