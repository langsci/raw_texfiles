\chapter{Word order and feature inheritance}\label{ch:4}

In Chapter \ref{ch:3}, \ac{OV} vs \ac{VO} order in Korean/Japanese and English was accounted for due to different feature specifications on the functional category \textit{v}, based on the view that parametric variation is attributed to differences in the features of functional categories. Then we may ask a question such as why is it \textit{v} that determines linguistic variation, not other functional heads for example \ac{ASP}? What is so special about \textit{v}? Can we explain this in a principled account? In fact, this is possible under the notion of \textit{phases} in minimalist syntax, which limits the locus of linguistic variation to certain functional categories, namely C and \textit{v}, which are defined as phase heads. In other words, feature specifications on phase heads and how these features are valued in syntactic derivations will lead to linguistic variation such as word order. The process of valuing features on phase heads will be explained in the \textit{feature} \textit{inheritance} mechanism, which will be developed in this chapter. 

\section{Feature Inheritance}\label{ch4:section:4.1}

In the past few decades, Chomsky (\citeyear{Chomsky2000,Chomsky2001,Chomsky2006,Chomsky2008}) developed fundamental ideas of the Minimalist Program: syntactic derivations are strongly cyclic and proceed phase by phase, a phase being defined as a syntactic object, which is in some sense complete, like \ac{CP} and \textit{v}P. A phase is a cyclic domain of syntactic computation where all \ac{LF}-uninterpretable features on a probe/target P (or a functional category) must be checked or deleted or valued against the matching interpretable features on a goal via \textit{Agree}. Agree removes uninterpretable features on the probe, which allows a derivation to converge at \ac{LF} in accordance with the principle of Full Interpretation (\citealt{Chomsky1986,Chomsky1995}). Not only is a phase a domain where uninterpretable features on a probe are valued and transferred to \ac{LF} (for convergence of meaning), but it is also a domain where the computed lexical items are transferred to \ac{PF} (for convergence of sound) when they are spelled-out/pronounced.


It is generally agreed in the literature that a probe must be a functional category such as C(omplementizer), T(ense), \textit{v} or \ac{ASP}. While any functional category can serve as a probe, it is not the case that all functional categories have ‘probing’ features (or \textit{Agree} \textit{features} or \textit{edge} \textit{features} in Chomsky’s term). Chomsky (\citeyear{Chomsky2000,Chomsky2001,Chomsky2006,Chomsky2008}) proposes that C and \textit{v}, which he calls \textit{phase} \textit{heads}, are core functional categories and feature specifications on phase heads determine linguistic variation. T is also a core functional category according to \citet[102]{Chomsky2000}, but it is not a phase head and inherently lacks any probing features in itself, including \ac{EPP}. In order for T to serve as a probe, it must be selected by the phase head C, from which T must inherit probing features via \textit{\acf{FI}}. This implies that T completely lacks any features and all features on T, such as $\upphi$-features and Tense, are not inherent on T but come from C.\footnote{It is generallyed agreed that \citet{DenBesten1983} is the first linguist who suggested that the tense feature is located on C. Later \citet{PesetskyTorrego2001} also proposed that C bears an uninterpretable T feature (with the \ac{EPP} property).}  

The hypothesis of \ac{FI} reflects the very idea of Minimalism by strengthening and simplifying the theory of phases, which are arguably a necessary part of any well-designed language system \citep{Richards2007}. A phase is a domain where uninterpretable features on a probe are valued and transferred to \ac{LF} (for meaning) and \ac{PF} (for sound). And by assuming that uninterpretable features can only be a property of phase heads, not all functional heads, we can simplify the computational design in an elegant way: phases are the locus of linguistic variation and also a domain where linguistic features on phase heads are valued and transferred to \ac{LF} and \ac{PF}. 

While the notion of phases provides us with a simplified computational tool to account for linguistic variation, it also leaves us a question; If we can assign all work on phase heads, C and \textit{v}, why do we need other functional categories in a syntactic derivation, such as T and \ac{ASP}, which are non-phase heads and featureless? Having superfluous non-phase heads seems to be non-minimal. \citet{Richards2007} provides an answer to this question on two grounds, (a) a simultaneous value and transfer of uninterpretable features on a phase head and (b) the \ac{PIC} \REF{ex:PIC} (\citealt[108]{Chomsky2000}).

\ea\label{ex:PIC} \textbf{\acl{PIC}} \\
In Phase ${\alpha}$ with head H, the domain of H is not accessible to operations outside ${\alpha}$, only H and its edge are accessible to such operations.
\z

\largerpage
If we interpret the \ac{PIC} in the X-bar theory, it means that in the \textsc{xp}, where X is C or \textit{v}, the complement of \textsc{xp} is not accessible for further syntactic computations, which must be transferred to \ac{LF} and \ac{PF}. Only the phase head and the material in its specifier position, which is the edge of X, will continue to participate in a further syntactic derivation up to the next phase level. This has a consequence: while the phase head X itself will not be spelled out until the next phase level, its uninterpretable features must be valued and transferred within the \textsc{xp}. Assuming that value and transfer must occur simultaneously, the phase head must discharge all of its features down to the head that will be spelled out for \ac{LF} and \ac{PF} convergence. This is demonstrated in \REF{ex:71}.

\ea\label{ex:71} Feature Inheritance \\
\begin{forest}
[XP [X \\ \textsc{phase}, name=src]
[YP [Y,name=tgt][ZP]]]
\draw[->] (src) to[in=west,out=south] (tgt)
node[pos=0.5,xshift=-5mm,yshift=-30mm]{FI from X to Y} (tgt) ;
\end{forest}
\z
             
\citet{Chomsky2001} suggests that \ac{FI} is a general property of all phase heads and should be at play in the domain of \textit{v}-V, analogous to that of C-T. However, \ac{FI} from C to T and \ac{FI} from \textit{v} to V do not seem to be parallel; T is a functional category and V is a lexical (or \textit{substantive} in Chomsky’s term) category. Chomsky is aware of this non-parallelism and notes that ``T should be construed as a substantive rather than a functional category, falling together with N and V. \ldots We can regard T as the locus of tense/event structure. The C-T relationship is therefore analogous to the \textit{v}*-V relation'' (p 9).\footnote{As mentioned in Chapter \ref{ch:3}, Footnote \ref{fn:30}, the distinction between \textit{v} and \textit{v}* is not made in this monograph, since it does not play a role in providing an account of \ac{OV}-\ac{VO} variation in Korean-English and Japanese-English \ac{CS}.} Thus, it seems that Chomsky offers a contradictory view on the status of T: on one hand, he says T is a core functional category along with C and \textit{v}, but he also argues that T should be regarded as a non-functional, lexical category.

\largerpage
A central premise of research exploring the lexical vs functional distinction is that it is only functional categories can serve as a probe and have parameterized features, which reins in syntactic variation (the Borer-Chomsky Conjecture as stated in (\ref{ex:BorChomCon}) in Chapter \ref{ch:3}. If we abide strictly by this hypothesis, a lexical category V cannot inherit the features from \textit{v} and become a probe, unless V is considered as a functional category. However, this problem disappears if we assume that the complement of \textit{v} is not \ac{VP} but \ac{ASP}P, as proposed in this monograph, and that the functional head \ac{ASP} is the beneficiary of \ac{FI} from \textit{v}, parallel to \ac{FI} from C to T \REF{ex:72}.

\ea\label{ex:72}\adjustbox{width=0.9\textwidth}{
\begin{tabular}[t]{llll}
a.  & C-T domain  & b. & \textit{v}-\textsc{Asp} domain \\
    & \begin{forest}
[CP [C \\ \textsc{phase}, name=src]
[TP [T,name=tgt][\textit{v}P]]]
\draw[->] (src) to[in=west,out=south] (tgt)
node[pos=0.5,xshift=-5mm,yshift=-30mm]{FI from C to T} (tgt) ;
\end{forest}
& &
\begin{forest}
[\textit{v}P [\textit{v} \\ \textsc{phase}, name=src]
[\textsc{Asp}P [\textsc{Asp},name=tgt][VP]]]
\draw[->] (src) to[in=west,out=south] (tgt)
node[pos=0.5,xshift=-5mm,yshift=-30mm]{FI from \textit{v} to \textsc{Asp}} (tgt) ;
\end{forest}
\end{tabular}}
\z

Analogous to the C-T relation, \textit{v} selects the \ac{ASP}P and transmits its probing features to \ac{ASP} via \ac{FI} in (\ref{ex:72}b). All features on \ac{ASP}, such as $\upphi$-features and Aspect, are inherited from its selecting phase head \textit{v}. We have now established \ac{FI} in the \textit{v}-\ac{ASP} domain, which is perfectly parallel to \ac{FI} in the C-T domain. Based on this, we proceed to explore which features are inherited from C to T and from \textit{v} to \ac{ASP}. 

\subsection{Features on C and \textit{v}}\label{ch4:sect:4.1.1} 

Chomsky proposes that the formal features on T (e.g. $\upphi$-features and Tense) belong to C and T inherits them from C via \ac{FI}. Yet, T seems to have more properties than just $\upphi$-features and Tense; T assigns the nominative Case (to the subject). Chomsky considers Case to be an uninterpretable feature – the ``uninterpretable feature par excellence'' (\citeyear{Chomsky1995}: 278-279) and suggests that it should also belong to the phase heads only, C (for the nominative Case) and \textit{v} (for the accusative Case). On this assumption, T’s $\upphi$-features, Tense, and (nominative) Case features all come from C via \ac{FI}.\footnote{Researchers who work on \ac{FI} differ in their views on which features are generated on C. For instance, \citet{Gallego2010} assumes that $\upphi$-features are generated on C, but T and Case features are intrinsic to T.} Likewise, I propose that \ac{ASP}'s $\upphi$-features, Aspect, and (accusative) Case features are all innate to \textit{v}, and they are passed down to \ac{ASP} via \ac{FI}.\footnote{The Case Filter \citep{Chomsky1981} states that every overt noun phrase must be assigned (structural) case: the subject is assigned the nominative Case and the object is assigned the accusative Case. In some languages, such as Korean and Japanese, structural case (nominative and accusative) may be marked by a specific morphological case marker. To distinguish between structural case and morphological case, Case, with the capital letter C, refers to structural case, which may or may not be marked by an overt morphological case marker.

\ea\label{ex:fn50}
    \ea \gll Bibi-ka      chayk-ul  sass-ta    \\
    Bibi-\textsc{nom} book-\textsc{acc}  buy.\textsc{past-decl} \\
  \hfill Korean
    \ex \gll Bibi-ga      hon-o        kat-ta  \\
    Bibi-\textsc{nom} book-\textsc{acc} buy-\textsc{past} \\ \hfill Japanese
    \glt `Bibi bought a book.'
    \zlast
\zlast}

\ea\label{ex:73}
\begin{tabular}[t]{ll}
a.  & C-T domain \\
    & \begin{forest}
[CP [C \\ {[\textit{u}$\upphi$,\textit{u}T,\textit{u}Case]}, name=src]
[TP [T \\ {[\textit{u}$\upphi$,\textit{u}T,\textit{u}Case]},name=tgt][\textit{v}P]]]
\draw[->] (src) to[in=west,out=south] (tgt)
node[pos=0.5,xshift=-15mm,yshift=-29mm]{FI} (tgt) ;
\end{forest}
\\ \\ b. & \textit{v}-\textsc{Asp} domain\footnotemark \\ &
\begin{forest}
[\textit{v}P [\textit{v} \\ {[\textit{u}$\upphi$,\textit{u}Asp,\textit{u}Case]}, name=src]
[\textsc{Asp}P [\textsc{Asp} \\ {[\textit{u}$\upphi$,\textit{u}Asp,\textit{u}Case]},name=tgt][VP]]]
\draw[->] (src) to[in=west,out=south] (tgt)
node[pos=0.5,xshift=-15mm,yshift=-29mm]{FI} (tgt) ;
\end{forest}
\end{tabular}
\z

\footnotetext{``\textit{u}F'' stands for ``uninterpretable features''; for instance, ``\textit{u}$\upphi$'' means an uninterpretable $\upphi$ feature. }

Chomsky does not discuss what happens when C’s features are transferred to T: it is not clear whether all of C’s features are inherited by T or features are selectively transmitted to T.\footnote{\citet{Richards2007}, on the other hand, argues that all uninterpretable features on phase heads must be discharged via \ac{FI}, which does not seem to be the case as we will discuss shortly.} Also it is not well discussed whether these features disappear from C after they are discharged to T or they remain active on C. To make the \ac{FI} mechanism transparent, I propose the following four principles that govern \ac{FI} \REF{ex:74}.

\ea\label{ex:74} \textbf{Principles of \acl{FI}} \\
    \ea \textit{Obligation} \\
     \ac{FI} is obligatory whenever possible.
    \ex \textit{Validation} \\
    \ac{FI} occurs if and only if the recipient head is a valid head.
    \ex \textit{Selection} \\
    Features may be selectively inherited.
    \ex  \textit{Expiration} \\ 
    Inherited features are only active on the heir (T, \ac{ASP}) and lose their probing capability on the donor (C, \textit{v}). 
    \z
\z

The first principle of \ac{FI}, \textit{Obligation}, is based on the assumption that \ac{FI} is designed to facilitate a syntactic derivation to proceed economically and efficiently and as long as it does not lead to a derivational crash, \ac{FI} will automatically happen. But if \ac{FI} leads to a syntactic crash rather than aiding a syntactic derivation, it will not occur.\footnote{A similar view is advocated for in \citet{Ouali2008}.} 

The second principle of \textit{Validation} states that \ac{FI} will take place successfully if and only if the recipient head of \ac{FI} is a valid head, which must be a featureless non-phase head, following the original idea of Chomsky and the further development of \ac{FI} made in \citet{Richards2007}. In other words, when the recipient head, such as T and \ac{ASP}, is not a featureless head, it is no longer eligible to inherit features from C and \textit{v}, respectively, as a result of which \ac{FI} does not take place. This is precisely the case when the \ac{ASP} head lexicalized by an English light verb, which has verbal features in it. As we will see in Chapter \ref{ch:5}, this will explain why \ac{VO} order is derived instead of \ac{OV} when a code-switched constituent includes an English light verb in Korean-English and Japanese-English \ac{CS}.

\ea\label{ex:75}
\begin{forest}
[\textit{v}P [\textit{v} \\ {[\textit{u}$\upphi$,\textit{u}Asp,\textit{u}Case]}, name=src]
[\textsc{Asp}P [\textsc{Asp} \\ LV\textsubscript{ENG} ,name=tgt][VP]]]
\draw[->] (src) to[in=west,out=south] node[pos=0.5]{\ding{54}} (tgt)
node[pos=0.5,xshift=-15mm,yshift=-29mm]{FI} (tgt) ;
\end{forest}
\z           

\newpage % due to footnote 8
The empirical evidence to support the third principle of \textit{Selection} comes from the fact that not all features of C are inherited by T (\textit{contra} \citealt{Richards2007}); for instance, C’s \textit{wh}-feature is never acquired by T but remains on C.\footnote{\citet{Ouali2008} proposes that there are three logical possibilities of \ac{FI} from C to T, which is compatible with the principle of \textit{selection}: (a) all of C’s $\upphi$-features are transferred to T (Donate), (b) C does not transfer its $\upphi$-features (Keep), and (c) C does not transfer but share its $\upphi$-features with T (Share). These three options are ranked in the order of Donate > Keep > Share, and if Donate leads to a syntactic crash, Keep is applied. Likewise, if Keep results in a crash, Share will happen. On the other hand, \citet[111]{Gallego2010} advocates the option of Share as the only mechanism of \ac{FI} by saying that features on C are downloaded to non-phase heads in its complement domain with no subsequent deletion on it.} 

The principle of \textit{Expiration}, on the other hand, may not be easy to prove on the basis of empirical evidence at this point. However, it is conceptually required for syntactic computation, and I will explain why this is the case. To do so, I take the view that syntactic structure building is a bottom-up process in a familiar fashion and consists of \textit{Merge} and \textit{Move} as proposed in the minimalist framework. 

When features are passed down from C to T, all of C’s features may be inherited by T, including the T(ense) feature, after which none of the features remain active as probing features on C. As a result, all of C’s features, [\textit{u}$\upphi$, \textit{u}T, \textit{u}Case], are valued via a probe-goal relationship between T and a goal or multiple goals via \textit{Multiple Agree}.\footnote{Multiple Agree will be discussed in \sectref{ch4:section:4.2}.} Despite their inactivity, these features are not depleted in C. After all, C is itself specified for tense and gets a vocabulary item inserted under it depending on its specification for tense -- for instance, in English finite C is lexicalized as \textit{that} and infinitival C is spelled out as \textit{for}, as in \REF{ex:76}.\footnote{I assume that finite C has [\textit{u} $\upphi$, \textit{u}T, \textit{u}Case] and non-finite C has [\textit{u}$\upphi$, \textit{u}Case] only.}

\ea\label{ex:76}
    \ea Bibi expected that Joa would go.
    \ex Bibi expected for Joa to go
    \z
\z

Moreover, the matrix verb can also select the tense (finiteness) of its \ac{CP} complement, which strongly suggests that the T feature is present on C even after \ac{FI} from C to T. While the features may be still present on C after being transferred to T, they are no longer need to be valued. If they were, valuation would be reflexive (the features on C that are inherited by T would be valued against the features on T, which are T’s own features). In other words, once probing features are transferred from a phase head to the head of its complement (from C to T and from \textit{v} to \ac{ASP}), they no longer act as probing features on the phase head; their probing capability expires on the donor. By contrast, features that are not transmitted retain their probing ability on the phase head. \REF{ex:77} schematically depicts this.

\newpage

\ea\label{ex:77}\ea
\hspace*{-5mm}\begin{adjustbox}{max width=0.9\textwidth}
    \begin{forest}
[CP, s sep = 1mm 
[C \\ {[\colorbox{lightgray}{\textit{u}$\upphi$,\textit{u}T,\textit{u}Case}]}, name=src]
[TP [T \\ {[\textit{u}$\upphi$,\textit{u}T,\textit{u}Case]},name=tgt][\textit{v}P]
{ \draw (.east) node[right]{\small {[\textit{u}$\upphi$,\textit{u}T, \textit{u}Case]} probe on T only}; }
]
{ \draw (.east) node[right]{\small {[\textit{u}$\upphi$,\textit{u}T, \textit{u}Case]} probing expires on C}; }
]
\draw[->] (src) to[in=west,out=south] (tgt)
node[pos=0.5,xshift=-15mm,yshift=-29mm]{FI} (tgt) ;
\end{forest} \end{adjustbox}

\ex
\hspace*{-5mm}\begin{adjustbox}{max width=0.9\textwidth}
 \begin{forest}
[CP, s sep = 1mm
[C \\ {[\colorbox{lightgray}{\textit{u}$\upphi$},\textit{u}T,\colorbox{lightgray}{\textit{u}Case}]}, name=src]
[TP [T \\ {[\textit{u}$\upphi$, \textit{u}Case]},name=tgt][\textit{v}P]
{ \draw (.east) node[right]{\small {[\textit{u}$\upphi$, \textit{u}Case]} probe on T}; }
]
{ \draw (.east) node[right]{\small {[\textit{u}$\upphi$, \textit{u}Case]} probing expires on C,}; }
]
\draw[->] (src) to[in=west,out=south] (tgt)
node[pos=0.5,xshift=-15mm,yshift=-29mm]{FI} 
node[pos=0.5,xshift=60mm,yshift=-14.1mm]{\small but {[\textit{u}T]} probes on C} (tgt);
\end{forest}
\end{adjustbox}
\z\z 

In (\ref{ex:77}a), all of C’s features are inherited by T, and C no longer acts as a probe. It is T that enters into a probe-goal relation in the derivation. In comparison, only a subset of C’s features may be inherited by T in (\ref{ex:77}b) and both C and T enter into probe-goal relationships with matching goals: while C looks for a goal with the T feature, the T head is engaged in feature matching with a goal or goals with $\upphi$ and Case features. 

One question might arise as to whether the principle of \textit{Expiration} is equally applied to the domain of \textit{v}-\ac{ASP}. Unlike the C-T relation where T is generally assumed to value the (nominative) Case feature on the subject, \textit{v} is generally presumed to value the (accusative) Case feature on the object in the structure of \textit{v}{}-V in Chomsky’s original proposal and subsequent work. Thus, it seems that the Case feature on \textit{v} is not discharged down to \ac{ASP} on this assumption. However, it has been proposed in the literature that \ac{ASP} is responsible for accusative Case checking in the \textit{v}-\ac{ASP} structure \citep{Richardson2003}, and it will be shown in Chapters \ref{ch:5} and \ref{ch:6} that the Case feature on the object may be valued either against \ac{ASP} or \textit{v} in Korean-English and Japanese-English \ac{CS}, which provides further empirical evidence for the principle of \textit{Expiration} in the \textit{v}{}-\ac{ASP} domain as well. 

One note should be in order: in Richardson’s proposal, the accusative Case is not the property of \textit{v} but the uninterpretable Aspect feature [\textit{u}Asp] on \ac{ASP} head.\footnote{Richardson does not adopt the \ac{FI} mechanism.} Under the mechanism of \ac{FI}, however, all features on \ac{ASP} come from \textit{v} and there is no feature intrinsic to the \ac{ASP} head per se. Thus, I claim that the Aspect feature on \ac{ASP}, [Asp], is innate to \textit{v}, not to \ac{ASP}, contra Richardson. I also depart from her viewpoint that [Asp] is the accusative Case feature: the presence of the Asp feature does not entail the presence of the accusative Case. For instance, unaccusative accomplishment verbs are specified for [Asp] but they do not assign the accusative Case. So, the accusative Case feature is not the same as the Asp-feature, but is an additional feature. Instead, I take the stance that Case features are D-features based on the fact that only nominal elements are assigned Case. Thus, I propose Case features are uninterpretable D-features: the D-feature on C-T represents nominative and the D-feature on \textit{v}-\ac{ASP} values accusative Case.\footnote{By extension I also depart from \citeauthor{PesetskyTorrego2001}'s (2001;2004) view that nominative is an [\textit{u}T].} Accordingly, \REF{ex:73} is modified into \REF{ex:78}.

 \ea\label{ex:78}
\begin{tabular}[t]{ll}
a.  & C-T domain \\
    & \begin{forest}
[CP [C \\ {[\textit{u}$\upphi$,\textit{u}T,\textit{u}D]}, name=src]
[TP [T \\ {[\textit{u}$\upphi$,\textit{u}T,\textit{u}D]},name=tgt][\textit{v}P]]]
\draw[->] (src) to[in=west,out=south] (tgt)
node[pos=0.5,xshift=-15mm,yshift=-29mm]{FI} (tgt) ;
\end{forest}
\\ \\ b. & \textit{v}-\textsc{Asp} domain\footnotemark \\ &
\begin{forest}
[\textit{v}P [\textit{v} \\ {[\textit{u}$\upphi$,\textit{u}Asp,\textit{u}D]}, name=src]
[\textsc{Asp}P [\textsc{Asp} \\ {[\textit{u}$\upphi$,\textit{u}Asp,\textit{u}D]},name=tgt][VP]]]
\draw[->] (src) to[in=west,out=south] (tgt)
node[pos=0.5,xshift=-15mm,yshift=-29mm]{FI} (tgt) ;
\end{forest}
\end{tabular}
\z

\citet{Pesetsky2012} proposed an analysis of case morphology in Russian, arguing that the nominative Case is a result of the D-feature on a noun phrase assigned by its selecting head, and nominative case morphology is an affixal realization of the D-feature on a noun phrase. The present proposal that the nominative Case is D-feature matching between T and a noun phrase/the subject converges with Pesetsky’s analysis. Yet, there are differences: according to Pesetsky, the accusative Case is the V-feature rather than the D-feature on a noun, which is assigned from the verb. This differs from the proposal made in this monograph, which views Case as a D-feature for both nominative and accusative. Although more research is needed to investigate this, I maintain the unified point of view of analyzing Case as a D-feature both on the subject and the object and will show in Chapter \ref{ch:6} that it seems to be a plausible approach to analyze the accusative Case as a D-feature based on various cross-linguistic data. 

\section{Agree}\label{ch4:section:4.2} 

In Chomsky’s original proposal, the probe-goal relation is one-to-one, limited to a single probe and a single goal and it occurs step by step until all the uninterpretable features on the probe are valued within a phase level. \REF{ex:79} below schematically shows a probe-goal relationship where a probe F enters into an Agree relation with a goal in its c-command domain. 

\ea\label{ex:79}
\begin{forest}
[FP[F\textsubscript{probe}]
[XP [X\textsubscript{goal}]
[YP [Y\textsubscript{goal}]
[ZP [Z\textsubscript{goal}]]]
]]
\end{forest}
\z

In \REF{ex:79} the probe F may have more than one goal, such as X, Y, and Z, which have (a subset of) matching features of F. According to Chomsky, the operation \textit{Agree} (and \textit{Move}) requires a goal that is both local and active, with locality being limited to the probe’s closest c-command domain (\citealt[122--123]{Chomsky2000}). On this assumption, it is X that enters into an Agree relation with the probe F. If X is an inactive goal, however, feature deletion under matching between F and X is blocked, and the probe searches a goal in its next closest c-command domain. As a result, Y, the goal in the next closest c-command domain of F, enters into an Agree relation with F. 

However, based on Japanese multiple nominative constructions, \citet{Ura1996,Ura2000} and \citet{Hiraiwa2001,Hiraiwa2005} argue that a single probe can agree with more than one goal simultaneously, which is called \textit{Multiple Agree} \REF{ex:80}. 

\ea\label{ex:80} \textbf{Multiple Agree} (\citealt{Hiraiwa2001,Hiraiwa2005}) \\
A probe can agree with more than one goal derivationally simultaneously.
\z

Hiraiwa assumes that the number of goals and the number of specifiers of a probe are both unlimited within a phase level. Many researchers have adopted Hiraiwa’s proposals, and both \textit{Multiple Agree} and \textit{Multiple Move} have gained more empirical support from various researchers (\citealt{Boskovic1999,Boeckx2004,Henderson2006}; to name a few). Yet, it is still a point of controversy in the literature whether a single probe can agree with multiple goals, and multiple specifiers of a head can be projected at the same time.

In this monograph, I adopt the notion of \textit{Multiple Agree} with some reservation. I assume that a single probe may agree with multiple goals within a given domain/phase. However, I depart from Hiraiwa's idea that the number of specifiers is unrestricted. As said in the precious chapter, I adopt \citegen{Kayne1994} approach to word order, viewing that the sequence of Specifier-Head-Complement is the universal order in all languages imposed by Universal Grammar, which is part of his proposal of the \acf{LCA} stating that asymmetric c-command invariably maps into linear precedence. Accordingly, the number of specifiers and complements are limited: multiple specifiers and complements do not yield to asymmetric c-command relations among the elements. While the number of specifiers (and complements) of a head is limited to only one in Kayne’s theory, it has been broadly accepted that Spec, \textit{v}P is a position where the subject is base-generated (the \ac{VP} internal subject hypothesis) and the moved object also lands in the derivation, as shown in (\ref{ex:51}a). Thus, the structure in \REF{ex:81} is not a possible structure under the \ac{LCA} where there is no asymmetric relation established between the subject and the moved object. 

\ea\label{ex:81}
\begin{forest}
[\textit{v}P [OBJ$_i$]
[\textit{v}P [SUB]
[\textit{v}$'$ [\textit{v}]
[VP [V] [t$_i$]]]]]
\end{forest}
\z

If we consider \textit{v}-\ac{ASP} structure proposed for Korean/Japanese light verb constructions in (\ref{ex:53}a), it is not the object but the \ac{ASP}P that raises to Spec, \textit{v}P, as depicted in \REF{ex:82}
, which is still not allowed in Kayne’s model for the same reason: the subject and the moved \ac{ASP}P show a symmetric relation.

\ea\label{ex:82}\begin{forest}
[\textit{v}P [\textsc{Asp}P$_k$ [OBJ$_i$]
[\textsc{Asp}$'$ [\textsc{Asp}]
[VP [V][t$_i$]]]]
[\textit{v}P [SUB]
[\textit{v}$'$ [\textit{v} = \textit{ha}\textsuperscript{KR}/\textit{su}\textsuperscript{JP}][\ldots~t$_k$]]]]
\end{forest}
\z

As explained in Chapter \ref{ch:3} (\sectref{ch3:sect:3.3}), the \ac{LCA} has been re-interpreted as an operation applied to \ac{PF} (\citealt{Chomsky1995,Moro2000}), therefore it is possible for multiple specifiers to be projected in the narrow syntax as long as the violation of the \ac{LCA} is limited before the structure is spelled out at PF, which is the view that I take and also taken by others (e.g. \citealt{Nagai2010,Ochi2009}). Supporting this view, the symmetric relation between the object and the subject in \REF{ex:81} and the symmetric relation between the raised \ac{ASP}P and the subject in \REF{ex:82} will not pose a problem. In \REF{ex:81} and \REF{ex:82}, the subject at Spec, \textit{v}P further moves up to Spec, \ac{TP} in the course of derivation and leaves a trace at Spec, \textit{v}P, and the trace of the subject is not subject to the \ac{LCA} at \ac{PF}: the element without any phonological feature is not conditioned by the \ac{LCA}. In other words, neither \REF{ex:81} and \REF{ex:82} violate the \ac{LCA}: only the object and the \ac{ASP}P, not the trace of the subject, are spelled out at Spec, \textit{v}P at \ac{PF} in\REF{ex:81} and \REF{ex:82}, respectively. Based on this, I propose a revised version of Multiple Agree, which I will call \textit{Multiple Agree under Antisymmetry} \REF{ex:83} as the first operational rule of \ac{FI}, which allows multiple agree relations but restricts the number of goals to only one that can be spelled out at a specifier position.

\ea\label{ex:83} \textbf{Multiple Agree under Antisymmetry} (\textit{First Operational Rule of \ac{FI}}) \\
Only one goal can be spelled out at the specifier of a probe in multiple agree relations.
\z

\newpage
It is generally agreed in the literature that a probe must be a functional category such as C, T, \textit{v}, or \ac{ASP}. However, Chomsky does not specify the nature of a goal except that in order for \textit{Agree} to occur, the goal must bear a feature that is matched to an uninterpretable feature [\textit{u}F] on a probe. Here, I assume that the morphosyntactic features are the properties of a head and either a functional category or a lexical category can be a goal. Thus, a probe-goal relation is defined to be feature matching between functional categories or between a functional category and a lexical category.\footnote{We will see in Chapters \ref{ch:5} and \ref{ch:6} that the functional category \ac{ASP} may serve as a goal for the probe \textit{v}.} Also, following the assumptions made in minimalist syntax, I assume that morphological features on phase heads are uninterpretable. On this assumption, all the features on C and \textit{v} are uninterpretable, including T(ense) and Asp(ect) features. Conversely, researchers have different views on whether (the matching) features on a goal are interpretable or not. As will be discussed in \sectref{ch4:sect:4.4.2}, I assume that verbs have both T and Asp features and enter into a probe-goal relation with C and \textit{v}. Although it is possible to assume that T feature on the verb is interpretable, Asp feature on the verb cannot be interpretable due to the fact that it is not the verb per se that determines the aspect of the \ac{VP}: the verb and the object together contribute to the aspectual information (recall \sectref{ch3:sect:3.2.1}). Instead, the Asp feature is uninterpretable everywhere, both on the probe and the goal, and aspectual interpretation is not determined by a morphological feature but by something else.\footnote{This means that verbs, whether heavy or light, have Aspect and T features and the presence of [uAsp] on the verb does not imply that the verb itself has aspectual properties.} Similar to Aspect, I also assume that the T feature is uninterpretable both on the probe and the goal, and temporal semantics may be brought in by a temporal operator. I leave further investigation of the interpretation of morphological features on a goal for future research. 

\section{EPP}\label{ch4:sectio:4.3}  

\ac{EPP} is generally claimed to be a feature on a functional category (or a probe) that induces syntactic movement. In this respect, the \ac{EPP} feature is distinguished from other features on a probe. Unlike other probing features, \ac{EPP} is not stated either as an interpretable or an uninterpretable feature, and only a probe, never a goal, has the \ac{EPP} feature. In an Agree relation, uninterpretable features on a probe are valued via feature matching between a probe and a goal, and valuation does not require a Spec-Head configuration and only involves feature matching between a probe and a goal.

\newpage
On the other hand, \ac{EPP} never involves feature matching (the goal does not have the \ac{EPP} feature). \ac{EPP} is not a feature that is valued. It is a feature on a probe that disappears via a Spec-Head relation with a goal: to satisfy the \ac{EPP} on the probe, the goal must raise to the specifier of the probe. Thus, the nature of \ac{EPP} seems to be very different from that of other features.

In \REF{ex:84} the probe F enters into multiple Agree relations with X and Y: X has a subset of matching features of F, ${\alpha}$ and ${\beta}$, and Y has ${\gamma}$ (and another feature). Now suppose that F has the \ac{EPP} feature triggering a goal to move to the specifier of FP. Which goal, X or Y, moves?

\ea\label{ex:84}
\begin{forest}
[FP [F\textsubscript{[\textit{u}$\alpha$,\textit{u}$\beta$,\textit{u}$\gamma$]}]
[XP [X\textsubscript{[\textit{u}$\alpha$,\textit{u}$\beta$]}]
[YP [Y\textsubscript{[\textit{u}$\gamma$,\textit{u}$\delta$]}][ZP]]]]
\end{forest}
\z

This problem is resolved once we abandon treating the \ac{EPP} as a feature per se. \ac{EPP} is a property of a feature rather than a feature on a probe, and a probing feature can be specified for the \ac{EPP} property (cf. \citealt{Chomsky2000}). So, when \textit{Multiple Agree} happens, the probe may trigger movement of a goal with a matching feature if and only if that feature on the probe has the \ac{EPP} property. 

Suppose that the feature $\gamma$ on F is specified for \ac{EPP} in \REF{ex:84}. While both X and Y agree with F, X remains in situ and only Y, which has the matching feature $\gamma$, is induced by the \ac{EPP} property of the feature $\gamma$ on the probe F. However, Y itself cannot move; a specifier cannot be occupied by a head.\footnote{This is the case in the X-bar theory, which this monograph adopts, not in the bare phrase structure.} The \ac{EPP}-specified feature $\gamma$ on F forces syntactic movement of the maximal projection of Y, which is YP, to its specifier position. 

\ea\label{ex:85}\begin{forest}
[FP, s sep = 19mm
[YP$_i$, name = tgt [Y\textsubscript{[\textit{u}$\gamma$,\textit{u}$\delta$]}][ZP]]
[FP [F\textsubscript{[\sout{\textit{u}$\alpha$},\sout{\textit{u}$\beta$},\sout{\textit{u}$\gamma$}\textsuperscript{\sout{EPP}}]}]
[XP [X\textsubscript{[\textit{u}$\alpha$,\textit{u}$\beta$]}][t$_i$, name = src]]]]
\draw[->,dashed] (src) to[in=east,out=south west] (tgt);
\end{forest}
\z

Suppose instead that it is $\alpha$, not $\gamma$, which is specified for \ac{EPP} on F. Then it is XP, not YP raises to Spec, FP, as shown in \REF{ex:86}.

\ea\label{ex:86}
\begin{forest}
[FP, s sep = 25mm [XP$_i$,name = tgt [X\textsubscript{[$\alpha$,$\beta$]}]
[YP [Y\textsubscript{[$\gamma$,$\delta$]}][ZP]]]
[FP[F\textsubscript{[\sout{\textit{u}$\alpha$}\textsuperscript{\sout{EPP}},\sout{\textit{u}$\beta$},\sout{\textit{u}$\gamma$}]}
][t$_i$, name = src]]]
\draw[->,dashed] (src) to[out=south west,in= east] (tgt);
\end{forest}
\z

What happens if both $\alpha$ and $\gamma$ have \ac{EPP} properties as in \REF{ex:87}? Can both XP and YP raise to Spec, FP simultaneously? 

\ea\label{ex:87}
\begin{forest}
[FP [F\textsubscript{[\textit{u}$\alpha$\textsuperscript{EPP},\textit{u}$\beta$,\textit{u}$\gamma$\textsuperscript{EPP}]} ]
[XP [X\textsubscript{[$\alpha$,$\beta$]}]
[YP [Y\textsubscript{[$\gamma$,$\delta$]}][ZP]]]]
\end{forest}
\z

The rule of \textit{Multiple Agree under Antisymmetry} in \REF{ex:83}, which is repeated below, may prevent the raising of XP and YP simultaneously, unless one of them raises further up during the course of derivation.

\ea\label{ex:88} \textbf{Multiple Agree under Antisymmetry} (\textit{First Operational Rule of \ac{FI}}) \\
Only one goal can be spelled out at the specifier of a probe in multiple agree relations.
\z

\section{Feature inheritance in Korean and Japanese vs English}\label{ch4:sect:4.4} 

I have identified probing features on C and \textit{v} and how \ac{EPP} is set onto these features and operates under a probe-goal relationship. In this section, I will show that languages differ from each other with respect to \ac{EPP}-specifications on the features on C and \textit{v}, and propose different ways of valuing these features via \ac{FI}. I will explain how \ac{FI} takes place in the C-T domain and the \textit{v}{}-\ac{ASP} structure in Korean and Japanese in comparison with English. The detailed mechanism of \ac{FI} the \textit{v}-\ac{ASP} structure will be applied to account for \ac{OV}-\ac{VO} variation in Korean-English and Japanese-English \ac{CS} in Chapter \ref{ch:5}.


\subsection{Features on C}\label{ch4:sect:4.4.1}

We have seen in Chapter \ref{ch:1} that Korean and Japanese display head-final structure where all heads uniformly follow all their complements. By contrast, English exhibits head-initial structure, where all complements are followed by their heads. This contrast is also observed in the C-T domain where the C head follows its complement \ac{TP} in Korean and Japanese, while it precedes \ac{TP} in English, as exemplified in \REF{ex:89}.



\ea\label{ex:89}\ea Korean: \\
\gll Bibi-ka     [\textsubscript{CP} [\textsubscript{TP} Joa-ka    chayk-ul    sass-ta]-\textbf{ko}]     malhayss-ta \\
Bibi-\textsc{nom}  {}   {}    Joa-\textsc{nom} book-\textsc{acc} buy.\textsc{past-decl-comp} say.\textsc{past-decl} \\ 
\ex Japanese: \\ \gll Bibi-ga [\textsubscript{CP} [\textsubscript{TP} Joa-ga    hon-o    kat-ta]-\textbf{to}]     it-ta \\ 
Bibi-\textsc{nom} {} {} Joa-\textsc{nom} book-\textsc{acc} buy.\textsc{past-comp} say.\textsc{past} \\
\ex Bibi said [\textsubscript{CP} \textbf{that} [\textsubscript{TP} Joa bought a book]]
\z\z

Despite the relative position of C head, before \ac{TP} in English but after \ac{TP} in Korean and Japanese, the subject is located at the beginning of a sentence in all of these languages; both the matrix subject \textit{Bibi} and the embedded subject \textit{Joa} are positioned at Spec, \ac{TP}. All of these empirical facts can be explained if we assume that the subject that is base-generated at Spec, \textit{v}P moves to Spec, \ac{TP} in all three languages and the \ac{TP} further moves up to Spec, \ac{CP} in Korean and Japanese, whereas it remains in situ in English \REF{ex:90}.

\ea\label{ex:90}\ea Korean and Japanese \\\small
      \begin{forest}
      [CP[TP$_k$]
      [C$'$, s sep = 1mm
      [C \\ \textit{ko}\textsuperscript{KR}/\textit{to}\textsuperscript{JP}]
      [(TP)$_k$[SUB$_i$]
      [T$'$ [T]
      [\textit{v}P [(SUB)$_i$][~~]]]]]]
      \end{forest} 
      

      
      \ex  English \\\small
      \begin{forest}
      [CP[~~]
      [C$'$, s sep = 1mm  [C \\ \textit{that}]
      [TP [SUB$_i$]
      [T$'$ [T]
      [\textit{v}P [(SUB)$_i$][~~]]]]]]
      \end{forest} 
      \z
\z

Both subject movement and \ac{TP} raising are induced by the \ac{EPP} specification on a feature on the probe C. What the structures in \REF{ex:90} reveal is that C in Korean and Japanese and C in English have different \ac{EPP} specifications. In Korean and Japanese, two of the features on C have the \ac{EPP} property, and each \ac{EPP} property triggers movement of the subject and movement of the \ac{TP}, respectively. By contrast, only one feature on C is specified for \ac{EPP} in English, as a result of which the subject raises. Under this scenario, we ask the following questions: which features on C are specified for \ac{EPP} in Korean, Japanese, and English and how do they trigger the raising of C’s goals? 

I assume that C in all these languages share the same features, [\textit{u}$\upphi$], [\textit{u}T], and [\textit{u}D].\footnote{\citet{Sigurdsson2004} proposes that all languages have the same set of features in narrow syntax.}  I also propose that [\textit{u}D] on C in Korean, Japanese and English is all specified for \ac{EPP}, which is responsible for subject raising to \ac{TP}. The evidence to support this comes from the fact that in all three languages the only element that can raise to Spec, \ac{TP} is a noun phrase, which checks the nominative Case/D-feature against T. Also, I claim that [\textit{u}T] on C in Korean and Japanese is \ac{EPP}-specified, triggering \ac{TP} raising to Spec, \ac{CP}. \REF{ex:91} shows feature specifications on C in Korean, Japanese, and English. 

\ea\label{ex:91}
    \ea C [\textit{u}$\upphi$, \textit{u}T\textsuperscript{\ac{EPP}}, \textit{u}D\textsuperscript{\ac{EPP}}]    \hfill  Korean, Japanese
    \ex  C [\textit{u}$\upphi$, \textit{u}T, \textit{u}D\textsuperscript{\ac{EPP}}]   \hfill    English
    \z
\z

It is legitimate to ask why it is [\textit{u}T], not [\textit{u}$\upphi$], is specified for \ac{EPP} on C in Korean and Japanese. Although I assume that Korean and Japanese have $\upphi$-features, parallel to English, the presence of $\upphi$-features in these languages is subject to debate in the literature. Due to the fact that Korean and Japanese do not show any morphological indication of $\upphi$-features on their nominals (none of the $\upphi$-features, such as person, number and gender, are morphologically marked in these languages), some researchers argue that $\upphi$-features may be lacking altogether in Korean and Japanese (\citealt{Kuroda1988,Namai2000,Saito2007,Saito2011,SellsKim2007,SenerTakahashi2009}). On the other hand, other researchers contend that subject (and object) honorification in Korean and Japanese is an instance of subject (or object)-verb agreement. Based on this, they argue for the presence of $\upphi$-features in these languages (\citealt{Ahn2002LVC,BoeckxNiinuma2004,Choe2004,Harada1976,Hasegawa2005,Koopman2005,Takita2006}). 

\ea\label{ex:92}
    \ea \gll sensayng-nim-i    haksayng-ul kitali-si-n-ta        \\
    teacher-\textsc{hon-nom} student-\textsc{acc} wait-\textsc{hon-pres-decl} \\ \hfill Korean
    \glt `The teacher is waiting for a/the student.'
    \ex \gll haksayng-i   sensayng-nim-ul  kitali-(*si)-n-ta \\
    student-\textsc{nom} teacher-\textsc{hon-acc} wait-\textsc{hon-pres-decl} \\
    \glt `A/The student is waiting for the teacher.'
    \ex \gll  sensei-ga      gakusei-o     o-mati      ni naru      \\
      teacher-\textsc{nom} student-\textsc{acc} \textsc{hon}{}-wait \textsc{hon} \\ \hfill Japanese
      \glt  `The teacher waits for a/the student.'
    \ex \gll gakusei-ga    sensei-o        matu/*o-mati ni-naru \\
    student-\textsc{nom} teacher-\textsc{acc} wait/\textsc{hon}{}-wait \textsc{hon} \\ 
    \glt `A/The student waits for the teacher.'
    \z
\z

Skirting the debate about the absence or presence of $\upphi$-features in Korean and Japanese, I will assume that [\textit{u}T] on C is \ac{EPP}-specified in Korean and Japanese and responsible for \ac{TP} raising to Spec, \ac{CP}, rather than proposing that the \ac{EPP} property on [\textit{u}$\upphi$] on C yields to parametric variation, in view of the fact that the presence of $\upphi$-features is controversial in Korean and Japanese.

\subsection{Features on \textit{v}}\label{ch4:sect:4.4.2}

The contrast between head-final structure in Korean and Japanese and head-initial structure in English is also observed in the verbal domain: the verb follows the complement in Korean and Japanese whereas the verb precedes its complements in English, thus showing \ac{OV} and \ac{VO} order, respectively. Thus, the \ac{OV}-\ac{VO} contrast between Korean/Japanese and English can be captured in the \textit{v}{}-\ac{ASP}P structure in \REF{ex:93}, which is parallel to the \ac{CP} structure of Korean/Japanese and English in \REF{ex:90}.

\ea\label{ex:93}\ea Korean and Japanese \\
      \begin{forest}
      [\textit{v}P [\textsc{Asp}P$_k$]
      [\textit{v}P, s sep = 1mm
      [\textit{v} \\ \textit{ha}\textsuperscript{KR}/\textit{su}\textsuperscript{JP}]
      [(\textsc{Asp}P)$_k$ [OBJ$_i$]
      [\textsc{Asp}P [\textsc{Asp} \\ $\varnothing$]
      [VP [V][(OBJ)$_i$]]]]]]
      \end{forest} 
      
      \ex  English \\
      \begin{forest}
      [\textit{v}P, s sep = 1mm [\textit{v} \\ $\varnothing$ ]
      [\textsc{Asp}P  [\textsc{Asp}]
      [VP [V] [OBJ]]]]
      \end{forest} 
      \z
\z                

In Korean and Japanese, the object first moves to Spec, \ac{ASP}P and \ac{OV} order is derived within \ac{ASP}P. Then the entire \ac{ASP}P raises to Spec, \textit{v}P, resulting in \ac{OV}-\textit{ha} order in Korean and \ac{OV}-\textit{su} order in Japanese. By contrast, the object remains in situ and \ac{ASP}P may or may not raise in English; since \textit{v} is null in English, there is no obvious way to tell if \ac{ASP}P moves to the left of \textit{v}. Absent any indication to the contrary, I assume that \ac{ASP}P remains in situ in English. The Korean and Japanese \textit{v}-\ac{ASP} structure in (\ref{ex:93}a) is entirely parallel to their C-T structure in (\ref{ex:90}a), and based on this, I propose the following feature specifications on \textit{v} in Korean/Japanese and English.

\ea\label{ex:94}
    \ea  \textit{v} [\textit{u}$\upphi$, \textit{u}Asp\textsuperscript{\ac{EPP}}, \textit{u}D\textsuperscript{\ac{EPP}}]      \hfill Korean, Japanese
    \ex \textit{v} [\textit{u}$\upphi$, \textit{u}Asp, \textit{u}D]   \hfill    English
    \z
\z

Notice that none of the features on \textit{v} is \ac{EPP}-specified in (\ref{ex:94}b). This means that after \ac{ASP} inherits features from \textit{v}, no overt movement occurs and the underlying \ac{VO} order maintains in English.\footnote{This is limited to the case of lexical verbs. With a light verb lexicalizing the \ac{ASP} head, \ac{FI} from \textit{v} to \ac{ASP} does not take place. However, whether the verb is heavy or light, \ac{VO} order remains due to the absence of \ac{EPP} on any of \textit{v}'s features in English.} However, several researchers have argued that the (direct) object does not stay in situ but moves out of \ac{VP} in English (e.g. \citealt{Johnson1991,Kawakami2017,Runner1995}).\footnote{Working in the pre-minimalist framework, \citet{Johnson1991} proposes that the object raises to Spec, \ac{VP}. } If we adopt this view, [\textit{u}D] on \textit{v} is \ac{EPP}-specified, triggering object raising in English. Then we may modify the feature specifications on \textit{v} in English by adding an \ac{EPP} property on the D-feature, as shown in (\ref{ex:95}b), where feature specifications on \textit{v} are perfectly parallel to feature specifications on C in English.\footnote{\textrm{V further raises to} \textrm{\textit{v}} \textrm{\citep{Chomsky1995}, therefore, the underlying \ac{VO} order maintains.} } In this monograph, I will continue to assume (\ref{ex:94}b) for features on \textit{v} in English unless there is a compelling reason to take (\ref{ex:95}b). In fact, whether we take (\ref{ex:94}b) or (\ref{ex:95}b), it will not make much of a difference in the discussion of \ac{FI} in English.

\ea\label{ex:95}
    \ea C [\textit{u}$\upphi$, \textit{u}T, \textit{u}D\textsuperscript{\ac{EPP}}]   \hfill     English
    \ex \textit{v} [\textit{u}$\upphi$, \textit{u}Asp, \textit{u}D\textsuperscript{\ac{EPP}}]
    \z
\z

On the other hand, in Korean and Japanese, two of \textit{v}’s features are \ac{EPP}-specified (\ref{ex:94}a) and trigger object movement and \ac{ASP}P raising. The \ac{ASP} head itself does not bear any features and inherits features from \textit{v}. \REF{ex:96} shows that the \ac{ASP} head inherits a subset of features of \textit{v}, [\textit{u}$\upphi$, \textit{u}D\textsuperscript{\ac{EPP}}], following the principle of \textit{Selection}, which triggers object shift to Spec, \ac{ASP}P. Due to the principle of \textit{Expiration}, [\textit{u}$\upphi$, \textit{u}D\textsuperscript{\ac{EPP}}] on \textit{v}, which are inherited by \ac{ASP}, no longer function as probing features and remain inactive. On the other hand, [\textit{u}Asp\textsuperscript{\ac{EPP}}] on \textit{v}, which has not been transmitted to \ac{ASP}, probes for a goal with the matching feature and triggers \ac{ASP}P raising to Spec, \textit{v}P.\footnote{\textrm{A}\textrm{\textsc{sp}}\textrm{P raising will be discussed in \sectref{ch4:sect:4.4.3}}} 



\ea\label{ex:96}
\hspace*{-1cm}\begin{adjustbox}{max width=1\textwidth}
\begin{forest}
[\textit{v}P, s sep = 1mm
[\textit{v} \\ {[\colorbox{lightgray}{\textit{u}$\upphi$},\textit{u}Asp\textsuperscript{EPP},\colorbox{lightgray}{\textit{u}D\textsuperscript{EPP}}]}, name=src]
[TP [\textsc{Asp} \\ {[\textit{u}$\upphi$, \textit{u}D\textsuperscript{EPP}]},name=tgt][\textit{v}P]
{ \draw (.east) node[right]{\small {[\textit{u}D\textsuperscript{EPP}]} on \textsc{Asp} triggers OBJ raising}; }
]
{ \draw (.east) node[right]{\small {[\textit{u}Asp\textsuperscript{EPP}]} on \textit{v} triggers \textsc{Asp}P raising}; }
]
\draw[->] (src) to[in=west,out=south] (tgt)
node[pos=0.5,xshift=-15mm,yshift=-29mm]{FI} (tgt) ;
\end{forest}\end{adjustbox}
\z

But why does \ac{ASP} inherit [\textit{u}D\textsuperscript{\ac{EPP}}] from \textit{v}, not [\textit{u}Asp\textsuperscript{\ac{EPP}}] in Korean and Japanese? Which features are selectively inherited by \ac{ASP} from \textit{v}? The principle of \textit{Selection} in \REF{ex:74} states that features may be selectively inherited, but it does not specify which features are selected from a phase head. Yet, features just cannot be randomly selected and inherited, as we will see that random selection and the inheritance of features may result in a crash in a derivation, which violates the principle of \textit{Obligation}. Thus, I propose that \ac{FI} must operate for a syntactic derivation to converge and it is regulated by the following three operational rules stated in \REF{ex:98} in addition to the four principles in \REF{ex:74}, which are repeated in \REF{ex:97}.

\ea\label{ex:97} \textbf{Principles of \acl{FI}} \\
    \ea \textit{Obligation} \\
     \ac{FI} is obligatory whenever possible.
    \ex \textit{Validation} \\
    \ac{FI} occurs if and only if the recipient head is a valid head.
    \ex \textit{Selection} \\
    Features may be selectively inherited.
    \ex  \textit{Expiration} \\ 
    Inherited features are only active on the heir (T, \ac{ASP}) and lose their probing capability on the donor (C, \textit{v}). 
    \z
    
\ex\label{ex:98} \textbf{Operational rules of \acl{FI}}
    \ea \textit{Multiple Agree under Antisymmetry} \\
Only one goal can be spelled out at the specifier of a probe in multiple agree relations.
    \ex \textit{Earliness} (cf. \citealt{Pesetsky1989}) \\
    Value features and satisfy \ac{EPP} as early as possible. 
    \ex \textit{Economy} (cf. \citealt{PesetskyTorrego2001})\footnotemark \\
     Value features via the minimum number of Agree relations.
    \z
\z

\footnotetext{\citet{PesetskyTorrego2001} proposes the Economy Principle which states that ``A head H triggers the minimum number of operations necessary to satisfy the properties (including \ac{EPP}) of its uninterpretable features'' (p359).}

The three operational rules in \ac{FI} in \REF{ex:98} provide an answer to the question ``why does \ac{ASP} inherit [\textit{u}$\upphi$, \textit{u}D\textsuperscript{\ac{EPP}}], from \textit{v}, not [\textit{u}$\upphi$, \textit{u}Asp\textsuperscript{\ac{EPP}}], in Korean and Japanese in \REF{ex:96}?'' In order to see this, we examine how features are inherited from \textit{v} to \ac{ASP} following the operational rules of \ac{FI} in Korean and Japanese, in comparison with \ac{FI} from \textit{v} to \ac{ASP} in English. 

In English, \textit{v} has [\textit{u}$\upphi$, \textit{u}Asp, \textit{u}D], none of which are specified for \ac{EPP}, as proposed in (\ref{ex:94}b). Following the rule of \textit{Earliness} in (\ref{ex:98}b), \textit{v} may transmit all of its features to \ac{ASP} at once, which provides an opportunity for all of \textit{v}’s features to be valued within \ac{ASP}P. With a heavy/lexical verb, which corresponds to V, \ac{ASP} is null and featureless, thus it can inherit [\textit{u}$\upphi$, \textit{u}Asp, \textit{u}D] from \textit{v}, after which \ac{ASP} enters into Multiple Agree relations with V, which has [Asp, T], and also with the D head of the object, which has [$\upphi$, D]. Since none of the features on \textit{v} are \ac{EPP}-specified, no goal raises to Spec, \ac{ASP}P after \ac{FI} and the derivation converges. As a result, the underlying \ac{VO} order maintains on the surface.\footnote{I assume that V has both Aspect and Tense features and enters into a probe-goal relationship with \textit{v}-\ac{ASP} and C-T. How the T feature on V plays a role in \ac{FI} in the C-T domain will be discussed in Chapter \ref{ch:5}.}

\newpage
\ea\label{ex:99} English heavy verbs \\
\begin{adjustbox}{max width = 0.92\textwidth}\begin{forest}
[\textit{v}P, s sep = 1mm
[\textit{v} \\ {[\colorbox{lightgray}{\textit{u}$\upphi$},\textit{u}Asp\textsuperscript{EPP},\colorbox{lightgray}{\textit{u}D\textsuperscript{EPP}}]}, name=src]
[\textsc{Asp}P [\textsc{Asp} \\ {[\textit{u}$\upphi$, \textit{u}D\textsuperscript{EPP}]},name=tgt]
[VP [V\textsubscript{[Asp,T]}][OBJ\textsubscript{[$\upphi$,D]}]]]]
\draw[->] (src) to[in=west,out=south] (tgt)
node[pos=0.5,xshift=-15mm,yshift=-29mm]{FI} (tgt) ;
\end{forest}\end{adjustbox}
\z

When the verb is light, on the other hand, \ac{FI} does not occur, for the lexically filled \ac{ASP} is no longer an eligible head to inherit \textit{v}’s features (à la the principle of validation) as in \REF{ex:100}. Instead, \textit{v}’s features are valued against \ac{ASP} and the object, and \ac{VO} order remains. Thus, regardless of the status of the verb, whether it is heavy or light, \ac{VO} order is derived in English.


\ea\label{ex:100} English light verbs \\
\begin{adjustbox}{max width = 0.95\textwidth}\begin{forest}
[\textit{v}P, s sep = 1mm
[\textit{v} \\ {[\textit{u}$\upphi$, \textit{u}Asp\textsuperscript{EPP}, \textit{u}D\textsuperscript{EPP}]}, name=src]
[\textsc{Asp}P [\textsc{Asp} \\ V\textsubscript{[Asp, T]} ,name=tgt]
[OBJ\textsubscript{[$\upphi$, D]} ]]]
\draw[->] (src) to[in= west,out=south]
node[pos=0.5]{\ding{54}} (tgt) ;
\end{forest}\end{adjustbox}
\z

In Korean and Japanese, \textit{v} has [\textit{u}$\upphi$, \textit{u}Asp\textsuperscript{\ac{EPP}}, \textit{u}D\textsuperscript{\ac{EPP}}], two of which are specified for \ac{EPP} as identified in (\ref{ex:94}a). Similar to English, \textit{v} tries to transmit all of its features to \ac{ASP} all at once, in accordance with the \textit{Earliness} rule of \ac{FI}, which allows not only all of \textit{v}’s features to be valued within \ac{ASP}P but also the \ac{EPP} specifications on [\textit{u}Asp] and [\textit{u}D] to be satisfied as early as possible. If \ac{ASP} inherits all of \textit{v}’s features, it enters into Multiple Agree relations with V and the D head of the object. However, both [\textit{u}Asp]\textsuperscript{} and [\textit{u}D] on \ac{ASP} are specified for \ac{EPP}, which triggers movement of the maximal projection of a goal with the matching features. Thus, both the DP-object and the \ac{VP} are forced to move to Spec, \ac{ASP}P. Although the derivation in \REF{ex:101} obeys the rule of \textit{Earliness}, it violates \textit{Multiple Agree under Antisymmetry}; both the \ac{VP} and the object cannot be spelled out at Spec, \ac{ASP}P in the \textit{v}P phase. As a consequence, the derivation crashes.

\ea\label{ex:101} Korean, Japanese \\
\begin{adjustbox}{max width = 0.9\textwidth}\begin{forest}
[*\textit{v}P, s sep = 1mm
[\textit{v} \\ {[\colorbox{lightgray}{\textit{u}$\upphi$, \textit{u}Asp\textsuperscript{EPP}, \textit{u}D\textsuperscript{EPP}}]}, name=src]
[\textsc{Asp}P [\textsc{Asp} \\ {[\textit{u}$\upphi$, \textit{u}Asp\textsuperscript{EPP},  \textit{u}D\textsuperscript{EPP}]},name=tgt]
[VP [V\textsubscript{[Asp, T]}][OBJ\textsubscript{[$\upphi$, D]}]]]]
\draw[->] (src) to[in=west,out=south] (tgt)
%node[pos=0.5,xshift=-19mm,yshift=-29mm]{FI} (tgt) 
;
\end{forest}\end{adjustbox}
\z

This means that \ac{ASP} cannot inherit all of \textit{v}’s features but can be endowed with either [\textit{u}Asp\textsuperscript{\ac{EPP}}] or [\textit{u}D\textsuperscript{\ac{EPP}}] from \textit{v}. Now this leads us to the question raised earlier: which features are selectively inherited by \ac{ASP} from \textit{v}? Why does \ac{ASP} inherit [\textit{u}D\textsuperscript{\ac{EPP}}] from \textit{v}, not [\textit{u}Asp\textsuperscript{\ac{EPP}}] in Korean and Japanese? It is the rule of \textit{Economy} that plays a role. 

Suppose that \ac{ASP} inherits [\textit{u}Asp\textsuperscript{\ac{EPP}}] from \textit{v}. \ac{ASP} also inherits [\textit{u}$\upphi$]. After the \ac{ASP} head inherits [\textit{u}$\upphi$, \textit{u}Asp\textsuperscript{\ac{EPP}}] from \textit{v}, it enters into two feature checking relationships, one with the object, which has the matching $\upphi$-feature and the other with the V, which has the Asp-feature. After [\textit{u}$\upphi$, \textit{u}Asp\textsuperscript{\ac{EPP}}] are transferred from \textit{v} to \ac{ASP}, these features remain inactive on \textit{v} and only [\textit{u}D\textsuperscript{\ac{EPP}}] will probe a goal with the matching feature (the principle of \textit{Expiration}). Thus, \textit{v} agrees with the object, which is headed by D and has the matching D-feature. All in all, \textit{v}’s features are valued via three rounds of feature matching, as illustrated in \REF{ex:102}.

\ea\label{ex:102}
\begin{adjustbox}{max width = 0.9\textwidth}
\begin{forest}
[\textit{v}P, s sep = 1mm
[\textit{v} \\ {[\colorbox{lightgray}{\textit{u}$\upphi$, \textit{u}Asp\textsuperscript{EPP}}, \textit{u}D\textsuperscript{EPP}]}, name=src]
[\textsc{Asp}P [\textsc{Asp} \\ {[\textit{u}$\upphi$, \textit{u}Asp\textsuperscript{EPP}]},name=tgt]
[VP [V\textsubscript{[Asp, T]}][OBJ\textsubscript{[$\upphi$, D]}]]]]
\draw[->] (src) to[in=west,out=south] (tgt) ;
\end{forest}
\end{adjustbox}
    \begin{exe}
    \exi{i.}  feature matching between \ac{ASP} and OBJ: [\textit{u}$\upphi$] is valued
    \exi{ii.} feature matching between \ac{ASP} and V: [\textit{u}Asp\textsuperscript{EPP}] is valued
    \exi{iii.} feature matching between \textit{v} and OBJ: [\textit{u}D\textsuperscript{EPP}] is valued
    \end{exe}
\z

Instead \ac{ASP} may inherit [\textit{u}$\upphi$, \textit{u}D\textsuperscript{\ac{EPP}}] from \textit{v}, as in \REF{ex:103}. Then, both $\upphi$-features and D-feature on \ac{ASP} can be valued via a single probe-goal relationship with the object, which has both $\upphi$ and D features. The remaining Asp-feature on \textit{v} is valued against the verb. 

\ea\label{ex:103}\begin{adjustbox}{max width = 0.92\textwidth}
\begin{forest}
[\textit{v}P, s sep = 1mm
[\textit{v} \\ {[\colorbox{lightgray}{\textit{u}$\upphi$}, \textit{u}Asp\textsuperscript{EPP}, \colorbox{lightgray}{\textit{u}D\textsuperscript{EPP}}]}, name=src]
[\textsc{Asp}P [\textsc{Asp} \\ {[\textit{u}$\upphi$, \textit{u}D\textsuperscript{EPP}]},name=tgt]
[VP [V\textsubscript{[Asp, T]}][OBJ\textsubscript{[$\upphi$, D]}]]]]
\draw[->] (src) to[in=west,out=south] (tgt) ;
\end{forest}
\end{adjustbox}
    \begin{exe}
    \exi{i.} feature matching between \ac{ASP} and OBJ: [\textit{u}$\upphi$, \textit{u}D\textsuperscript{EPP}] are valued
    \exi{ii.} feature matching between \textit{v} and V: [\textit{u}Asp\textsuperscript{EPP}] is valued
    \end{exe}
\z

Although both derivations in \REF{ex:102} and \REF{ex:103} obey the rules of \textit{Earliness} and \textit{Multiple Agree under Antisymmetry}, the latter involves a smaller number of feature matching operations than the former: \textit{v}’s features are valued in two steps in \REF{ex:103} and three in \REF{ex:102}. In other words, the derivation in \REF{ex:102} wins over the derivation in \REF{ex:102} according to the operational rule of \textit{Economy} in (\ref{ex:98}c), which states that ``Value features via the minimum number of Agree relations''. Hence, it provides an answer to the question why \ac{ASP} inherits [\textit{u}D\textsuperscript{\ac{EPP}}] from \textit{v}, not [\textit{u}Asp\textsuperscript{\ac{EPP}}], in Korean and Japanese. 

Based on this, I will proceed to explain how \ac{OV} order is derived in light verb \textit{ha} and \textit{su} constructions in Korean and Japanese under the \ac{FI} mechanism developed here. Before doing so, I would like to add a few words to the general operation of \ac{FI}. I assume that features are clustered and valued en-bloc, and the operational rules of \ac{FI} certainly support this view: the rule of \textit{Economy} demands the number of feature checking operations be minimized, which in turn applies feature valuation to as many features at once as possible (cf. the principle of maximized matching in \citealt{Chomsky2001}). Although en-bloc feature checking may not seem to play a crucial role in \REF{ex:103} where the \ac{EPP}-specifications on [\textit{u}D] and on [\textit{u}Asp] of \textit{v} are valued separately by two different goals, the object and the verb, respectively, the notion of feature clusters will become more prominent when \textit{v} = \textit{ha}\textsuperscript{KR}\textit{/su}\textsuperscript{JP} enters into a feature checking relationship with a single goal, which is \ac{ASP}, and all the features of \textit{v} = \textit{ha}\textsuperscript{KR}\textit{/su}\textsuperscript{JP}, including the \ac{EPP}-specifications on [\textit{u}D] and [\textit{u}Asp], are clustered and valued against a single goal. We will see this in Chapter \ref{ch:5}. 

\subsection{Deriving OV in Korean and Japanese}\label{ch4:sect:4.4.3} 
\largerpage
The structure in \REF{ex:103} above, where \ac{ASP} inherits [\textit{u}$\upphi$, \textit{u}D\textsuperscript{\ac{EPP}}] from \textit{v}, accounts for the sequence of \ac{OV}-\textit{ha} and \ac{OV}-\textit{su} in light verb constructions in Korean and Japanese in \REF{ex:104}.

\ea\label{ex:104}  
\gll Bibi-ka     yenge-lul    kongpwu hayss-ta  \\
Bibi-\textsc{nom} English-\textsc{acc} study       \textsc{do.past-decl} \\ \hfill Korean
\ex \gll Bibi-ga     eigo-o     benkyoo sita    \\
Bibi-\textsc{nom} English-\textsc{acc} study      \textsc{do.past-decl} \\ \hfill Japanese
\glt `Bibi studied English.'
\z

After \ac{ASP} inherits [\textit{u}$\upphi$, \textit{u}D\textsuperscript{\ac{EPP}}] from \textit{v}, \ac{ASP} agrees with the D head of the object, which bears the matching features. While [\textit{u}$\upphi$] on \ac{ASP} can be valued against the $\upphi$-feature on the object in-situ via Agree, the \ac{EPP} property on the D-feature on \ac{ASP} triggers movement of the maximal projection of a goal with the corresponding feature. Consequently, the object (the maximal projection of D head) raises to Spec, \ac{ASP}P, delivering \ac{OV} order within \ac{ASP}P, as shown in \REF{ex:105}. For the sake of simplicity, \ac{DP} and \ac{NP} are not distinguished here.

\ea\label{ex:105}\begin{adjustbox}{max width = 0.92\textwidth}
\begin{forest}
[\textit{v}P, s sep = 1mm
[\textit{v}{=}\textit{ha}\textsuperscript{KR}/\textit{su}\textsuperscript{JP} \\ {[\colorbox{lightgray}{\textit{u}$\upphi$}, \textit{u}Asp\textsuperscript{EPP}, \colorbox{lightgray}{\textit{u}D\textsuperscript{EPP}}]}]
[\textsc{Asp}P, s sep = 15mm [OBJ\textsubscript{\textit{i}[$\upphi$, D]},name=tgt]
[\textsc{Asp}P [\textsc{Asp}\textsubscript{[\sout{\textit{u}$\upphi$}, \sout{\textit{u}D}\textsuperscript{\sout{EPP}}]}]
[VP [V\textsubscript{[Asp, T]}][t$_i$, name = src]]]]]
\draw[->,dashed] (src) to[in=south,out=south west] (tgt) ;
\end{forest}
\end{adjustbox}\vspace*{-5mm}
    \z

After \ac{FI} from \textit{v} to \ac{ASP}, [\textit{u}$\upphi$, \textit{u}D\textsuperscript{\ac{EPP}}] on \textit{v} no longer function as probing features (the principle of \textit{Expiration}). But [\textit{u}Asp\textsuperscript{\ac{EPP}}] on \textit{v} still needs to be valued and the \ac{EPP} property on [\textit{u}Asp] on \textit{v} also triggers movement of the maximal projection of the goal with the matching feature, which is the \ac{VP}: V bears the matching Asp-feature. However, if the \ac{VP} moves to Spec, \textit{v}P, \ac{OV} order is not derived. Instead, the surface order would be V-\textit{ha}-O in Korean or V-\textit{su}{}-O in Japanese, as shown in \REF{ex:106}, which is an unattested order in Korean and Japanese: the linear order must be O-V-\textit{ha} in Korean and O-V\textit{-su} in Japanese, as evidenced in \REF{ex:104}.

\ea\label{ex:106}
\begin{adjustbox}{max width = 0.9\textwidth}
\begin{forest}
[*\textit{v}P, s sep = 10mm
[VP$_k$, name = tgt [\textbf{V}\textsubscript{[Asp, T]}][t$_i$]]
[\textit{v}P, s sep  = 1mm [\textit{v}{=}\textit{\textbf{ha}}\textsuperscript{KR}/\textit{\textbf{su}}\textsuperscript{JP} \\ {[\colorbox{lightgray}{\textit{u}$\upphi$}, \textit{u}Asp\textsuperscript{EPP}, \colorbox{lightgray}{\textit{u}D\textsuperscript{EPP}}]}]
[\textsc{Asp}P[\textbf{OBJ}\textsubscript{\textit{i}[$\upphi$, D]}]
[\textsc{Asp}$'$ [\textsc{Asp}\textsubscript{[\sout{\textit{u}$\upphi$}, \sout{\textit{u}D}\textsuperscript{\sout{EPP}}]}] [t$_k$, name = src]]]]
]
\draw[->,dashed] (src) to[in=south east, out=south west] (tgt) ;
\end{forest}\end{adjustbox}\vspace*{-5mm}
\z

\largerpage
How do we get O-V-\textit{ha/su} order from \REF{ex:105} then? To put it differently, how do we rule out the derivation in \REF{ex:106}, which yields an impossible word order in Korean and Japanese? In \REF{ex:105}, \textit{v} enters into a probe-goal relation with V and [\textit{u}Asp\textsuperscript{\ac{EPP}}] on \textit{v} triggers \ac{VP} movement, which does not derive \ac{OV}-\textit{ha/su} order as we saw in \REF{ex:106}. If \ac{ASP}P is pied-piped by \ac{VP}, on the other hand, the entire \ac{ASP}P raises to Spec, \textit{v}P, and we do get the correct order, O-V-\textit{ha} and O-V-\textit{su} as illustrated in \REF{ex:107}.

\ea\label{ex:107}
\begin{adjustbox}{max width = 0.9\textwidth}
\begin{forest}
[\textit{v}P, s sep = 10mm
[\textsc{Asp}P$_k$ [\textbf{OBJ}\textsubscript{\textit{i}[$\upphi$, D]}]
[\textsc{Asp}$'$ [\textsc{Asp}\textsubscript{[\sout{\textit{u}$\upphi$, \sout{\textit{u}D}\textsuperscript{\sout{EPP}}}]}]
[VP [\textbf{V}\textsubscript{[Asp, T]}][t$_i$]]]]
[\textit{v}$'$, s sep  = 1mm [\textit{v}{=}\textit{\textbf{ha}}\textsuperscript{KR}/\textit{\textbf{su}}\textsuperscript{JP} \\ {[\colorbox{lightgray}{\sout{\textit{u}$\upphi$}}, \sout{\textit{u}Asp}\textsuperscript{\sout{EPP}}, \colorbox{lightgray}{\sout{\textit{u}D}\textsuperscript{\sout{EPP}}}]}] [t$_k$]
]
]
\end{forest}\end{adjustbox}\vspace*{-1cm}
\z
\clearpage

So, how can we rule out the derivation in \REF{ex:106} on a principled ground and have the derivation in \REF{ex:107} as the only legitimate derivation in light \textit{ha/su} constructions in Korean and Japanese? Here, I appeal to a perspective from Distributed Morphology towards functional and lexical categories (\citealt{HalleMarantz1993,HarleyNoyer1999}). \ac{DM} offers a syntactic approach to word formation, in which a word is syntactically derived via merging a category-neutral root with a category-defining functional head (\citealt{Marantz1997}). On this view, a lexical category (or \textit{l}-morpheme in \ac{DM}’s terminology) such as V, N and A, is a root, whose lexical status is underspecified, and requires selection by a functional category (or \textit{f}-morpheme), such as \textit{v}, \textit{n}, and \textit{a}, in order for its lexical status to be determined and spelled out via Vocabulary Insertion at \ac{PF}.

\largerpage
Under this view, the fact that the \ac{VP} cannot move in \REF{ex:106} can be explained by the claim that the projection of lexical roots is incapable of undergoing syntactic movement arguably because the root would be severed from the functional category which determines its category and with the aid of which it can be subjected to Vocabulary Insertion at \ac{PF} (den Dikken, p.c.). Assuming that the determination of a lexical category is not done derivationally in the syntax but representationally in the \ac{PF} component, Vocabulary Insertion for any lexical roots requires the local presence of a functional category in the \ac{PF} representation, which can determine the lexical root’s categorial status. As a consequence, movement of \ac{VP} severing it from its selecting functional head \ac{ASP} cannot occur, but the entire \ac{ASP}P pied-piped by \ac{VP} must be raised to Spec, \textit{v}P to satisfy the \ac{EPP} property on [\textit{u}Asp] of \textit{v}, as in \REF{ex:107}\footnote{On this view, object raising involves \textit{n}P rather than \ac{NP}, in which a functional category \textit{n} selects \ac{NP}.}

One may question whether \textsc{Asp}P raising is licit in \REF{ex:107}, assuming that movement of \textsc{Asp}P, which is the complement of \textit{v}, to the specifier of the same head \textit{v} is too short or local and would violate the so-called \textit{anti-locality} constraint that disallows local movement of the complement of a head to the specifier of the very same head (e.g. \citealt{Abels2003,Cinque2005,Grohmann2003}).

\ea\label{ex:108} Anti-locality constraint \\
\begin{forest}
[XP[YP, name = tgt]
[X$'$ [X$^0$][YP, name = src]]]
\draw[->, dashed] (src) to[in=south, out = south west] node[pos=0.5]{\ding{54}} (tgt);
\end{forest}\vspace*{-.5cm}
\z
\clearpage

Working in the minimalist framework, \citet{Abels2003} claims that movement is allowed only if a new feature checking relation is established and all features can be satisfied or checked in the head-complement relation, which is the closest relation in syntax. Thus, there is no good reason to move a phrase from the complement to the specifier position of the same head for such movement does not establish a new feature satisfaction/checking relation. 

Assuming that all feature checking occurs uniformly in a checking domain, which corresponds to the c-command domain of the probe (\citealt{Chomsky1995,Chomsky2000}), it is true that all features could be checked, in principle, via Agree; uninterpretable features on the head/probe can be checked against the matching features of a goal without movement of the goal. However, given that the \ac{EPP} on the probe can be only satisfied in a Spec-Head relation, as assumed in the minimalist syntax, and if the closest match is the probe’s complement, the complement must raise to satisfy the \ac{EPP} property on the probe. Unless this movement is extrinsically constrained, considerations of anti-locality are moot. Thus, I conclude that \ac{ASP}P movement is legitimate in \REF{ex:109}.

\ea\label{ex:109}
\begin{forest}
[XP, s sep = 17mm [YP, name = tgt]
[X$'$ [X$^0$\textsubscript{[+EPP]}][YP, name = src]]]
\draw[->, dashed] (src) to[in=south, out = south west] (tgt)
node[pos=0.6,xshift=-5mm, yshift=-30mm]{OK} (tgt);
\end{forest}
\z

In fact, comp-to-spec raising has been justified by various researchers. \citet{Aboh2004}, for instance, proposes that the whole \ac{ASP}P raises to the specifier position of its dominating functional/F head in Gungbe.\footnote{When this happens, \ac{OV} order is derived, which may co-vary with \ac{VO} order in this language. } Outside the \textit{v}P domain, \citet{RichardsBiberauer2005} also account for expletive elements distribution in Germanic languages and argue that that \textit{v}P raising to T satisfies the \ac{EPP} of T. Such cross-linguistic evidence supports the current proposal of \ac{ASP}P raising to Spec, \textit{v}P, which delivers correct linear order of \ac{OV}-\textit{ha} and \ac{OV}-\textit{su} in Korean and Japanese light verb constructions. In Chapter \ref{ch:6}, it will be shown that \ac{TP} is pied-piped and raises to Spec, \ac{CP} in the C-T domain in Korean and Japanese, which is entirely analogous to the \textit{v}-\ac{ASP} domain where \ac{ASP}P is pied-piped and raised, as explained in this section. 

\section{Chapter summary and conclusion}\label{ch4:section:4.5} 

In this chapter, I reviewed several issues of recent developments in syntactic theories in the Minimalist Program. Adopting the proposal that morphosyntactic features of functional categories lead to cross-linguistic variation and C and \textit{v} are core functional categories, word order variation is attributed to feature contents of these functional categories. This idea was furthermore taken in connection with \ac{FI}, which was primarily proposed for the domain of C-T by Chomsky where T inherits its probing features from C. Parallel to \ac{FI} from C to T, I proposed features are inherited by \ac{ASP} from \textit{v}. To promote \ac{FI} as a fully-fledged mechanism to derive syntactic derivations more efficiently and economically, I proposed four principles of \ac{FI}, \textit{Obligation}, \textit{Validation}, \textit{Selection} and \textit{Expiration}. In addition, I claimed that \ac{FI} is operated by three rules, \textit{Multiple} \textit{Agree} \textit{under} \textit{Antisymmetry}, \textit{Earliness}, and \textit{Economy} and have shown how head-final structure in Korean and Japanese and head-initial structure in English is derived via \ac{FI}.

To account for \ac{OV}-\textit{ha} order in Korean and \ac{OV}-\textit{su} order in Japanese in contrast with \ac{VO} order in English, I proposed the underlying \textit{v}-\ac{ASP} structure for Korean and Japanese \REF{ex:110} (repeated from \ref{ex:103}), where two of \textit{v}’s features have \ac{EPP} properties. 

\ea\label{ex:110}\begin{adjustbox}{max width = 0.92\textwidth}
\begin{forest}
[\textit{v}P, s sep = 1mm
[\textit{v} \\ {[\colorbox{lightgray}{\textit{u}$\upphi$}, \textit{u}Asp\textsuperscript{EPP}, \colorbox{lightgray}{\textit{u}D\textsuperscript{EPP}}]}, name=src]
[\textsc{Asp}P [\textsc{Asp} \\ {[\textit{u}$\upphi$, \textit{u}D\textsuperscript{EPP}]},name=tgt]
[VP [V\textsubscript{[Asp, T]}][OBJ\textsubscript{[$\upphi$, D]}]]]]
\draw[->] (src) to[in=west,out=south] (tgt) ;
\end{forest}
\end{adjustbox}
    \begin{exe}
    \ex feature matching between \ac{ASP} and OBJ: [\textit{u}$\upphi$, \textit{u}D\textsuperscript{EPP}] are valued
    \ex feature matching between \textit{v} and V: [\textit{u}Asp\textsuperscript{EPP}] is valued
    \end{exe} 
\z

In \REF{ex:110}  null-headed \ac{ASP} inherits [\textit{u}$\upphi$, \textit{u}D\textsuperscript{\ac{EPP}}] from \textit{v} = \textit{ha}\textsuperscript{KR}\textit{/su}\textsuperscript{JP}\textsuperscript{\MakeUppercase{} }and triggers object shift, delivering \ac{OV} order within \ac{ASP}P. Then entire \ac{ASP}P is pied-piped by the \ac{VP} and raises to Spec, \textit{v}P due to feature matching between \textit{v} and V, as a result which the linear order of \ac{OV}-\textit{ha}/\textit{su} in Korean/Japanese is derived. The structure in \REF{ex:110} will be used to account for \ac{OV}-\textit{ha} and \ac{OV}-\textit{su} order in Korean-English and Japanese-English \ac{CS} in Chapter \ref{ch:5}. Yet, we will see that there are cases when \ac{FI} from \textit{v} = \textit{ha}\textsuperscript{KR}\textit{/su}\textsuperscript{JP} to \ac{ASP} is blocked and the object fails to move to Spec, \ac{ASP}P. As exemplified in \REF{ex:111}, when \ac{ASP} is lexicalized by an English light verb, it bears the verbal features of the light verb and cannot be a beneficiary of \ac{FI} from \textit{v}. \ac{FI} occurs only when the recipient functional head is empty (the principle of validation). As a result, object raising does not occur whenever an English light verb lexicalizes \ac{ASP}, which in turn derives \ac{VO}-\textit{ha}/\textit{su} order in Korean-English and Japanese-English \ac{CS}. 

\ea\label{ex:111}
\begin{adjustbox}{max width = 0.95\textwidth}\begin{forest}
[\textit{v}P, s sep = 1mm
[\textit{v}{=}\textit{ha}\textsuperscript{KR}/\textit{su}\textsuperscript{JP} \\ {[\colorbox{lightgray}{\textit{u}$\upphi$}, \textit{u}Asp\textsuperscript{EPP}, \colorbox{lightgray}{\textit{u}D\textsuperscript{EPP}}]}, name=src]
[\textsc{Asp}P [\textsc{Asp}\textsubscript{[Asp, T]} \\ LV\textsubscript{ENG} ,name=tgt]
[OBJ\textsubscript{[$\upphi$, D]} ]]]
\draw[->] (src) to[in= west,out=south]
node[pos=0.5]{\ding{54}} (tgt) ;
\end{forest}\end{adjustbox}
\z
                         

While I reserve an explanation of how \ac{OV} and \ac{VO} orders are derived in Korean-English and Japanese-English \ac{CS} in next chapter, the main proposal can be summarized as the following: \ac{OV}-\ac{VO} variation in Korean-English and Japanese-En\-glish \ac{CS} is a result of object raising to Spec, \ac{ASP}P; when object shift occurs, \ac{OV} order is derived. If the object stays in situ, the underlying \ac{VO} order maintains. Regardless of object shift, the entire \ac{ASP}P always raises to Spec, \textit{v}P whenever \textit{v} comes from Korean or Japanese, and the linear order would be \ac{OV}-\textit{ha/su} or \ac{VO}-\textit{ha/su} in Korean-English and Japanese-English \ac{CS}. All these movements are a consequence of feature checking and \ac{EPP} specifications on the phase head \textit{v}, as proposed in the minimalist framework.
