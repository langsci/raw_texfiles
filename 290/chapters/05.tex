\documentclass[output=paper]{../langscibook}
\ChapterDOI{10.5281/zenodo.4449776}
\author{Pierre-Luc Paquet\affiliation{University of Texas at Tyler}\lastand Nina Woll\affiliation{University of Quebec at Trois-Rivieres}}
\title{Debunking student teachers’ beliefs regarding the target-language-only rule}
\abstract{\sloppy Despite the recognized usefulness of crosslinguistic interaction, additional language (LN) teachers seem to be resistant to opening this Pandora’s box to more multilingual classroom practice. While the ultimate goal of the broader research project is to promote student teachers’ reflective stance concerning the potential benefits of crosslinguistic pedagogy, this study examines their beliefs regarding the TL-only rule established in their respective programs. Forty student teachers from Quebec and Mexico completed a vignette-based questionnaire where they were asked to reflect on different situations focused on the use of other languages, including L1. Results suggest that while a monolingual bias is prevalent in the Quebec context, participants from Mexico appear to be more open to resorting to other languages in the classroom. However, responses from both populations suggest that this perceived usefulness is restricted to situations where L1 is used as a vehicle rather than a resource. Suggestions for addressing these issues in teacher education are explored and avenues for further research are provided.}

\IfFileExists{../localcommands.tex}{
  \addbibresource{../localbibliography.bib}
  \usepackage{langsci-optional}
\usepackage{langsci-gb4e}
\usepackage{langsci-lgr}

\usepackage{listings}
\lstset{basicstyle=\ttfamily,tabsize=2,breaklines=true}

%added by author
% \usepackage{tipa}
\usepackage{multirow}
\graphicspath{{figures/}}
\usepackage{langsci-branding}

  
\newcommand{\sent}{\enumsentence}
\newcommand{\sents}{\eenumsentence}
\let\citeasnoun\citet

\renewcommand{\lsCoverTitleFont}[1]{\sffamily\addfontfeatures{Scale=MatchUppercase}\fontsize{44pt}{16mm}\selectfont #1}
   
  %% hyphenation points for line breaks
%% Normally, automatic hyphenation in LaTeX is very good
%% If a word is mis-hyphenated, add it to this file
%%
%% add information to TeX file before \begin{document} with:
%% %% hyphenation points for line breaks
%% Normally, automatic hyphenation in LaTeX is very good
%% If a word is mis-hyphenated, add it to this file
%%
%% add information to TeX file before \begin{document} with:
%% %% hyphenation points for line breaks
%% Normally, automatic hyphenation in LaTeX is very good
%% If a word is mis-hyphenated, add it to this file
%%
%% add information to TeX file before \begin{document} with:
%% \include{localhyphenation}
\hyphenation{
affri-ca-te
affri-ca-tes
an-no-tated
com-ple-ments
com-po-si-tio-na-li-ty
non-com-po-si-tio-na-li-ty
Gon-zá-lez
out-side
Ri-chárd
se-man-tics
STREU-SLE
Tie-de-mann
}
\hyphenation{
affri-ca-te
affri-ca-tes
an-no-tated
com-ple-ments
com-po-si-tio-na-li-ty
non-com-po-si-tio-na-li-ty
Gon-zá-lez
out-side
Ri-chárd
se-man-tics
STREU-SLE
Tie-de-mann
}
\hyphenation{
affri-ca-te
affri-ca-tes
an-no-tated
com-ple-ments
com-po-si-tio-na-li-ty
non-com-po-si-tio-na-li-ty
Gon-zá-lez
out-side
Ri-chárd
se-man-tics
STREU-SLE
Tie-de-mann
} 
  \togglepaper[5]%%chapternumber
}{}

\begin{document}
\maketitle
\shorttitlerunninghead{Debunking beliefs regarding the target-language-only rule}


\section{Introduction}


Despite the unprecedented rise of crosslinguistic pedagogy around the globe, many teacher-training programs are still constrained by a monolingual bias. Beyond any doubt, learners need maximal exposure to the target language (TL), especially when access is limited. Meanwhile, this so-called “direct method assumption” is largely responsible for an ideology of language separation that has been prevalent in additional language (LN) pedagogy since the rise of the communicative era \citep{Cummins2007}. However, to ban other languages from the classroom might reduce the benefits of active reflection on crosslinguistic correspondences. Throughout the last two decades, the quest for European intercomprehension has given rise to whole curricula aimed at cultivating crosslinguistic awareness (\citealt{CouncilOfEurope2001}). In North America, there have been numerous attempts to challenge these deep-rooted assumptions, namely to acknowledge the dynamics of multilingual societies in LN classrooms. For instance, \citet{Duff2007} refers to concrete examples of classroom projects intended to foster multilingual diversity, namely to counter what she calls “assimilationist policies in Canadian schools [that] lead to subtractive bi- or multilingualism, where French or English are privileged exclusively at the expense of students’ other languages” (p.~153). Accordingly, investigations into teacher beliefs regarding multilingual practices have also become more widespread. Studies conducted in different parts of the world revealed that despite the perceived usefulness of crosslinguistic classroom activities known or shown to teachers, the dominant discourse favoring exclusive TL use was not easily overridden \citep{ArocenaEgañaEtAl2015,Haukås2016,MartinezEtAl2015}. In contexts where teacher training programs that are not conceived with a view to fostering multilingual proficiency, it may thus be expected that monolingual ideology is prevalent among LN teachers, especially in contexts where the TL-rule is invoked by policy makers and educators. The goal of this study is to examine student teachers’ beliefs regarding strictly monolingual vs. multilingual practice in LN classrooms in Canada and Mexico. More specifically, it addresses their perceptions of resorting to L1 or other languages and the presumed reasons for this use.

To accomplish this goal, we have taken a different approach to the narrative by weaving the trajectories of two researchers’ experiences with studies in multilingualism and LN instruction. We use these personal experiences to help the readers, especially those working and learning in different teacher training programs, reflect on their own beliefs about multilingual teaching and learning, and the beneficial effects of incorporating these pedagogical practices.


\section{The story}


To situate the theoretical foundations from which the present study has emerged, it was deemed relevant to refer to the personal and professional trajectories of two researchers whose stories might reflect that of other scholars and teachers in the field of LN pedagogy. It’s the story of two young professionals who have started academic careers in LN didactics in different parts of the world far away from their respective homes. One of the many things they had in common was that they were leading lives with multiple languages. One grew up in the French-speaking province of Quebec and the other in Germany and both had learned several non-native languages at school. After years of training in higher education, both were now stranded in other countries where they used two languages at work none of which was their native language. Another common trait was that they were both trained language teachers, who had acquired experience in teaching LNs in their respective countries of birth, as well as their native languages abroad. 

Throughout their teaching career, their ever-growing curiosity led them to pursue graduate studies. In the end, they both acquired PhDs in psycholinguistics, with a focus on crosslinguistic influence. More specifically, they were interested in the positive influence of previously acquired languages on the acquisition of an additional language. When they discovered the field of third language acquisition (TLA), they were both mesmerized. As they plunged into the work of scholars such as Jasone Cenoz and Britta Hufeisen, it was as if someone had explained and theorized their own trajectories as language learners.



\subsection{Findings from third language acquisition}



Specifically, while Cenoz’ work on “additive” multilingualism \citep{Cenoz2003} was a milestone that inspired researchers to examine the beneficial effects of learning additional languages, \citegen{Hufeisen2000} \emph{factor model} gives an extensive overview of the distinctive features of TLA. In particular, third language learners can draw on their knowledge of previously acquired LN, while related language learning experiences and strategies are likely to enhance cognitive factors such as metalinguistic awareness. 

Although social/affective and contextual factors are also taken into consideration by this model, its main components are psycholinguistic in nature and have inspired researchers to investigate language development with a focus on the positive effects of learning and using more than two languages (e.g., \citealt{PeyerEtAl2010}).

What follows from these studies is that crosslinguistic interaction occurs naturally between all the languages of a multilingual. Especially when languages are typologically related, learners are led to make assumptions about underlying correspondences in function as well \citep{Ringbom2007}. Moreover, findings reflect the premises made by the dynamic model of multilingualism (\citealt{HerdinaJessner2002}), which views multiple language learning as a dynamic process depending on numerous factors related to each individual learner’s history. Specifically, the constant interaction between the different languages of a multilingual generates new structures that are specific to each speaker and different from monolingual systems. In this sense, the dynamic model is consistent with \citegen{Cook1992} notion of multicompetence. What is specific to  \citegen{HerdinaJessner2002} conceptualization of multilingual proficiency is that when lacking some relevant linguistic knowledge, the multilingual’s metalinguistic abilities can make up for it. Not only were these predictions confirmed in subsequent research, but they also reflected the personal experiences of the two young scholars mentioned earlier. Namely, throughout their own trajectories as language learners, they became aware of their growing knowledge base in several languages, thus increasing their adaptability to complex communicative situations. Essentially, the conscious manipulation of the whole repertoire is likely to lead to increased levels of metalinguistic awareness, which in turn facilitates further learning \citep{Jessner2017}. On the whole, this line of research emphasizes the benefits of learning and using multiple languages instead of focusing on the obstacles.



\subsection{Crosslinguistic pedagogy}



When it comes to pedagogical approaches that build on these findings, the field has also been flourishing. Again, much of what the two young professionals had experienced as language teachers was reinforced by the literature on the practical implementation of crosslinguistic pedagogy in the LN classroom. Especially in Europe, changes at the policy level have generated models of language education aimed at fostering intercultural and plurilingual competence \citep{Candelier2007}. The main focus of projects such as Eurocom \citep{HufeisenMarx2007} is to foster awareness of correspondences across languages. A number of comparative approaches such as \emph{focus on multilingualism} \citep{CenozGorter2014} have been developed to tap into the repertoire of learners from various linguistic and cultural backgrounds, namely to help them discover the rich knowledge base that they can build on when learning additional languages (i.e., \citealt{LeonetEtAl2020}). In addition, recent developments in Canada, especially crosslinguistic awareness pedagogy \citep{HorstEtAl2010} and plurilingual and pluricultural tasks (\citealt{GalanteEtAl2019,GalanteEtAl2020}), have also been a major inspiration to the two young professionals who had both been hired in teacher training programs.

In sum, years of personal and professional engagement in the learning and teaching of multiple languages, paired with the growing awareness of a community of practice in which multilingual practices are not only considered acceptable but encouraged, have grown into a feeling of empowerment for two educators whose individual careers in language didactics and teacher training has yet to be built.


\section{The problem}
Picture these same two young professionals, who are obsessed with language and equipped with a toolbox full of pedagogical material ready to be adapted to various linguistic and cultural contexts. Full of inspiration for practical applications that draw on the rich theoretical basis for crosslinguistic pedagogy, they are ready to share their obsession with future teachers. 

And that is when they hit a wall. In fact, the student teachers were not ready.

Based on our current knowledge, only a handful of studies have explored LN student teachers’ and teachers’ beliefs about multilingual pedagogy (\citealt{ArocenaEgañaEtAl2015}; \citealt{CreeseBlackledge2010}; \citealt{DeAngelis2011,Haukås2016,Woll2020}). It has already been established that beliefs influence teachers’ pedagogical decisions, and that such beliefs are ingrained and resistant to change (\citealt{PhippsBorg2009}). Be that as it may, results of these studies are quite similar in two ways: (1) these participants think multilingualism should be promoted, but not necessarily in their own classrooms, (2) they do not feel competent enough at doing so, which according to them may be detrimental to their students’ language learning (\citealt{DeAngelis2011}). These results indicate that experienced teachers do not feel comfortable using multilingual practices, and if that is the case, how can we expect student teachers to be open to this discussion?

Whether it is called translanguaging (\citealt{GarciaLi2014}) or crosslinguistic interaction \citep{Jessner2008}, multilingual practices in LN classrooms are at variance with the “monolingual principle” \citep{Cummins2007} that appears to be prevalent in both contexts of study. For example, anecdotal evidence from interactions with student teachers in Quebec revealed that using the TL exclusively basically meant doing their job properly. To resort to the learners’ native language (L1), however, was perceived as a failure. Even if no generalizations can be drawn from such scattered statements, they reflect the “anti-L1-attitude” identified by \citet{Cook2001} as a “mainstream element in twentieth-century teaching methodology”, thus leading teachers to “feeling guilty for straying from the L2 path” (p. 405). Furthermore, Cummins pointed out that

\begin{quote}
Despite the continuing academic debate on these issues, policy and practice operate as though the “monolingual principle” had been established as axiomatic and essentially “common sense” \citep[224]{Cummins2007}. 
\end{quote}

A closer look at the Quebec education program provides further evidence for this monolingual orientation:

\begin{itemize}
\item  As models and guides, teachers speak English at all times and require students to use English as well (\citealt[7]{MEQ2007}).
\item Students and the teacher use English as the language of communication in the classroom for all personal, social and task-related purposes \citep{MEQ2007}.
\end{itemize}



Even if there is no explicit policy of language separation in Mexico, evidence from teachers and students suggests that the monolingual principle may also reflect the dominant discourse in higher education in this part of the world. For example,  \citet{MoraPabloEtAl2011} speak of an “implicit policy” whereby teachers are “told not to use L1 and this practice continues even when the prescribed methodology is no longer in use” (p. 121).

Despite this limited amount of evidence from both educational contexts, the two professionals, who had only recently started working as teacher educators in Quebec and Mexico respectively, found themselves confronted to a similar problem: Their enthusiasm regarding crosslinguistic classroom practices and their willingness to make student teachers take a critical look at these written and unwritten norms were generally met with hostility. 

Students indicated that they had trouble believing the opposite of what they had always thought to be best practice. This perceived resistance may be related to what they had been told by other professors or practicum supervisors at the university, to personal experiences during their practicum or to their own experience as language learners. In particular, research on the so-called “apprenticeship of observation” \citep{Lortie1975} suggests that student teachers often “fail to realize that the aspects of teaching which they perceived as students represented only a partial view of the teacher’s job” \citep[274]{Borg2004}.

\subsection{Research questions and hypotheses}
While the ultimate goal of the broader research project is to promote student teachers’ reflective stance regarding the potential benefits of crosslinguistic classroom practice, the first step to be undertaken was to examine their beliefs regarding the monolingual bias stipulated in their program of study. In particular, with a view to confronting the apprenticeship of observation, to question deeply-rooted beliefs and move beyond their limits, this study sets out to examine the student teachers’ personal and professional opinions underlying those beliefs. 

More specifically, to examine student teacher beliefs regarding the target-lan\-guage-only rule, the present study aims at answering the following two research questions: 

\begin{itemize}
\item [RQ1:] In what classroom situations is the use of L1 or other languages perceived as acceptable or even useful?
\item [RQ2:] To what extent do future teachers in Quebec and in Mexico appear to be open to using other languages in their own projected practice?
\end{itemize}



While the first question aims at providing an overview of factors which, according to the participants, might justify resorting to L1 or other languages, the second question aims at describing personal belief profiles. As for RQ1, certain acceptable conditions for using other languages may be prevalent within or across samples, thus pointing to possible anchor points to pursue pedagogical reflection. In other words, the most common factors could be systematically addressed and critically discussed within university programs. Regarding RQ2, a monolingual bias is expected in both student populations, yet teachers from Quebec might be more fervent promoters of the TL-only-rule given the explicit guidelines in their program of study. Overall, the anticipated results are expected to reflect concrete situations on which student teachers’ beliefs are based. Finally, the reported events that appear to motivate student teachers’ beliefs have the potential of informing teacher trainers on avenues for challenging the apprenticeship of observation and creating new learning opportunities.


\section{Methodology}
\subsection{Participants}
Twenty student teachers in Quebec aged between 22 and 52 and twenty student teachers in Mexico aged between 22 and 40 participated in this study. All forty participants were enrolled in a LN teacher-training program, more specifically in English ($n=27$) and Spanish ($n=13$). Subjects were recruited from four different universities, two of which were located in Quebec and two in Mexico. While the large majority of the participants were enrolled in undergraduate programs ($n=33$), some participants were from different graduate programs ($n=7$). All participants had at least 80 hours of classroom-based face-to-face experience teaching LNs. Moreover, all participants were bi- or multilingual learners and teachers.

\subsection{Vignette-based technique}
To elicit participants’ perceptions of the use of other languages than the TL in the classroom, this study used a vignette-based technique. Jeffries and Maeder defined vignettes as 

\begin{quote}
Incomplete short stories that are written to reflect, in a less complex way, real-life situations in order to encourage discussions and potential solutions to problems where multiple solutions are possible. (\citealt[18]{JeffriesMaeder2005}). 
\end{quote}

According to \citet{SimonTierney2011}, using vignettes reduces defensiveness of responses which helps capture complex thought processes and stimulate critical thinking, even of sensitive information. In the present study, vignettes were used to tap into not only the participants' learning but also their teaching experience. 

To select the focus of the vignettes, \citet{MoraPabloEtAl2011} was considered, which explored the presumed reasons for resorting to L1 in an LN classroom in Mexico. This way, the extraction of linguistic situations where using other languages was qualified as acceptable for LN teachers and teacher trainers in Mexico. As for the Quebec context, the study relied on the researcher’s classroom experience as teacher trainers and practicum supervisors and on the monolingual stance from the Quebec education program. Three different settings were selected, all of which were salient in both Mexico and Quebec, namely using other languages for pedagogical purposes, for establishing rapport and discipline, and for clarification. Afterwards, different real-life situations were created that are relevant to participants’ experience as language learners and student teachers, and that allow them to reflect on this sensitive topic without feeling threatened or judged. Half of the vignettes presented the situations from a learner’s point of view and the other half from a teacher’s. More specifically, when creating the vignettes, the goal was to describe real-life situations that LN teachers and learners face during their learning and/or their professional career and in which they have to come to a decision about a potential conflict of beliefs regarding strictly monolingual vs. multilingual practice.

To design the vignettes, the general guidelines indicated by \citet{SimonTierney2011} were followed. First, the vignettes were developed considering the participants’ profile and experience to make them relevant and to maintain their interest. For this reason, the present study included some challenging situations, which most participants could relate to and even had previously experienced in their practice. Second, the vignettes did not exceed 200 words. As \citet[1189]{StravakouLozgka2018} stated “[…] the provision of less information in hypothetical scenarios favors the personal elements of participants to come to the surface”. That is, vignettes needed to be purposefully incomplete in order to allow for multiple solutions and to elicit participants’ critical thinking. Finally, as recommended by \citet{SimonTierney2011}, the current study contained a reasonable number of vignettes ($n=4$) and made sure that participants would take no more than 30 minutes to complete the online questionnaire \citep{Dornyei2010} -- see \REF{ex:paquet:1} and \REF{ex:paquet:2} for examples.

  
\ea Vignette from a teacher's point of view\\\label{ex:paquet:1}
Sarah is a French as a second language teacher in an English-speaking high school. When it comes to presenting aspects of the French grammar, she sometimes has difficulty to make herself understood when giving explanations on those elements exclusively in the target language (French). So, when she notices that her students are not following her explanations, she instinctively changes from French to English in order to assure a better understanding. This way, she moves forward to other aspects more quickly and students seem to better understand.
\ex Vignette from a learner’s point of view\\\label{ex:paquet:2}
In his German as a third language class, Raphael could not understand anything during the first few weeks. Truthfully, he did not know what he was doing there. Everyone else seemed to follow what the teacher said, except Raphael who did not have any point of reference. After a few weeks, the teacher wrote a sentence in English on the board (Raphael's second language) and invited the group to figure out how this would be said in German. Raphael gained confidence as he tried to analyze the structure with his classmates, speaking in French (Raphael's first language). Since then, Raphael has the impression that there might be more parallels between French, English and German than he thought.
\z

The vignettes were followed by two different prompts. The first prompt requested participants to reflect on the situation described and to offer advice to the main character of the scenarios, the learner or teacher respectively. In this manner, participants adopted the role of consultants, which was central since the use of other languages is a sensitive topic to discuss with student teachers in both contexts. This way, it made the participants feel more comfortable revealing something closer to their true opinion without the feeling of being judged. The second prompt invited participants to further explain their thoughts specifying the knowledge, experiences and beliefs that may have influenced their responses.



\subsection{Data analysis}
To tackle the first research question, the content of all responses was analyzed thematically. Two research assistants were hired as independent raters. First, the entire data set was analyzed to identify instances in which participants referred to situations perceived as acceptable for resorting to other languages, including L1. The first coding phase constituted a clustering of recurrent patterns. Then, in a second coding phase, these were regrouped into six factors related to conditions that would justify resorting to other languages. In the last coding phase, the authors regrouped these factors into three main categories: learner factors, pedagogical strategies and practical constraints. Finally, the general trends within each category were interpreted. The frequency of each response type was listed for each context (see \tabref{tab:paquet:1}).

With a view to answering the second research question, responses were analyzed for each participant separately. First, the remaining responses, in which no justification for resorting to other languages was implied, were coded for instances where exclusive TL-only use was explicitly promoted. Depending on the recurrence of statements reflecting TL-only promotion or openness to other languages respectively, participants were classified into three belief profiles: (1)~hard\-line-TL-only, (2) open-to-other-languages and (3) multivoicedness. The first profile corresponded with participants who showed resistance towards the use of other languages across the board, even when responding to vignettes presenting cases where using other languages seemed beneficial. Second, participants who manifested openness to using other languages in all three categories analyzed above were classified open-to-other-languages. The third classification referred to profiles where participants expressed two apparently contradicting positions within or across responses. Namely, while asserting the need for maximal TL exposure, participants in this belief profile would concurrently point to the benefit of using other languages. Instead of viewing these perspectives as plain contradictions, \citegen{Bakhtin1981} notion of \emph{multivoicedness} seemed more appropriate to refer to the competing views that were shown to coexist within teachers’ accounts of their beliefs on various topics (\citealt{BallWarshauerFreedman2004}). Namely, multivoiced discourse was tangible in the present study, when student teachers drew on apparently opposing ideologies to express opinions on language choices. Finally, having identified belief profiles across the sample, the frequency of profile type was also listed for each context of study (see \tabref{tab:paquet:2}).


\section{Results and interpretation}


\subsection{Conditions for using other languages}



The main purpose of the current study, where vignettes were implemented, was to explore student teachers’ beliefs regarding strictly monolingual vs. multilingual practice in LN classrooms. This issue was particularly interesting for research since student teachers, from both contexts, seem to feel constrained by a monolingual bias. The aim of the first research question was to examine in what classroom situations the use of L1 or other languages is perceived as acceptable or even useful for student teachers. Looking at all responses, a total of 123 comments related to the present issue were obtained. All comments were then classified into one of three broad categories: Learner factors ($n=42$), pedagogical strategies ($n=50$) and practical constraints ($n=31$). \tabref{tab:paquet:1} provides the distribution of student teachers’ responses listed in the respective categories.

\begin{table}
% \footnotesize
% \begin{tabularx}{\textwidth}{XYYYYYY}
% \lsptoprule
% Categories & \multicolumn{2}{c}{Learner factors} & \multicolumn{2}{c}{Pedagogical strategies} & \multicolumn{2}{c}{Practical constrains}\\\midrule
% & Proficiency & Anxiety & Instructions /clarifications & Metalinguistic description & Time & Discipline\\
% \midrule
% Quebec & 4 & 9 & 5 & 12 & 4 & 17\\
% Mexico & 15 & 14 & 11 & 22 & 4 & 6\\
% \lspbottomrule
% \end{tabularx}

\begin{tabular}{lrr}
\lsptoprule
Categories & Quebec & Mexico\\\midrule
Learner factors \\
Proficiency & 4 & 15\\
Anxiety & 9 & 14\\\midrule
Pedagogical strategies \\
Instructions\slash clarifications & 5 & 11\\
Metalinguistic description & 12 & 22\\\midrule
Practical constrains\\
Time & 4 & 4\\
Discipline & 17 & 6\\
\lspbottomrule
\end{tabular}
\caption{Conditions for using other languages\label{tab:paquet:1}}
\end{table}


\subsubsection{Learner factors}


As for the first category, two factors were highlighted from our participants’ responses: proficiency level and anxiety. Regarding proficiency level, participants seemed to be more inclined to resorting to other languages when teaching beginner levels. This can be seen in the following excerpt:

\begin{quote}
\emph{Considero que solo se debe de hacer uso del L1 como un recurso en niveles principiantes para garantizar la comprensión y también para optimizar tiempos.}\smallskip\\
`I consider that it is acceptable to use the learners’ L1 as a resource only in beginner levels in order to guarantee their comprehension and to maximize time' (Participant 925 (\textsc{mx}), our translation)
\end{quote}

This comment suggests that it is acceptable to resort to other languages with lower proficiency learners since they have neither the knowledge nor the strategies to understand the teacher’s explanations in the TL. With reference to the learners’ stress and anxiety, student teachers seem to perceive the use of other languages, mainly the use of the learners’ L1, as beneficial and a powerful tool to lower their students’ stress and anxiety. This can be seen in the following excerpt: 

\begin{quote}
\emph{Lorsque le ou la professeur(e) fait référence aux deux autres langues assez bien maîtrisées des élèves du groupe pour initier de la nouvelle matière en espagnol, les élèves ressentent moins de stress, font des liens et ont des références sur lesquelles se fier pour évoluer dans leur apprentissage.}\smallskip\\
`When the teacher refers to other languages known by the students, they feel less stress, make links and have some references to rely on in order to progress in their learning' (Participant 842 (\textsc{qc}), our translation)
\end{quote}

In this excerpt, the participant states that using other languages not only helps to lower anxiety, but also provides learners with the opportunity to make links between the various languages they know. L1 is described as a tool that helps with understanding language structures but more importantly as a means to reduce anxiety. Regarding this first category, learner factors, it is important to mention that there is a difference between the participants from Quebec (13 comments) and those from Mexico (29 comments). Where few comments were made by the Quebec student teachers in relation to proficiency level, participants from Mexico seem to perceive the use of other languages for beginner learners as acceptable and even necessary. According to them, beginners need more guidance in L1 in order to grasp the new system. As stated in  \citet{MoraPabloEtAl2011}, this observation could suggest that in the Mexican context, the learning environment is mostly controlled by the teacher and little or no control is given to the learners.


\subsubsection{Pedagogical strategies}


Concerning the second category, the participants’ responses referred mostly to instructions and clarification, and metalinguistic description. Student teachers indicated that, when they were in the learners’ shoes, they would make connections between the TL and any other languages (usually their L1). On the other hand, from a teacher’s perspective, the main concern appears to be the students understanding justifying L1 use. The following quotes illustrate the experience of two student teachers as both LN learners and teachers. 

\begin{quote}
\emph{Obviously, whenever a learner has a few languages bouncing in his head, it is favorable to use prior knowledge of given languages to make sense of the target language currently being learned}. (Participant 358 (\textsc{qc}))
\end{quote}

\begin{quote}
\emph{Cuando uso el inglés (L1 u otro idioma) para explicar cierto concepto, ellos reaccionan con expresiones de mayor entendimiento y efectivamente, entienden mejor lo explicado.}\smallskip\\
`When I use English (L1 or other language) to explain a certain concept, they react with better understanding and indeed, understand better what was explained' (Participant 001 (\textsc{mx}), our translation)
\end{quote}

The above excerpts suggest that the student teachers are aware that other languages can actually be beneficial when it comes to understanding the structure of the TL. However, as for the first category, there is an apparent difference between the participants from Quebec (17 comments) and from Mexico (33 comments). While most participants from Quebec seem to perceive using other languages as a last resort, the Mexican student teachers appear to see other languages as facilitators for the development of metalinguistic knowledge in the TL. It could be hypothesized that these observations are in line with the communicative classroom practice implemented in the Quebec education program versus the more traditional focus on forms used in Mexico.


\subsubsection{Practical constrains}


As for the third category, participants’ responses were classified in two sub-categories: Time constraint and disciplinary issues. Regarding the first sub-cat\-e\-go\-ry, participants made very few comments invoking the use of other languages for time constraint purposes. However, some of them mentioned that it all depends on the institutions’ curriculum and the learners’ needs. There is a certain openness to the use of other languages in the classroom due to time constraint, as the following excerpt shows:

\begin{quote}
\emph{[…] por cuestiones de tiempo muchas veces un maestro debe de ser práctico. Considero válido el hacer uso de la lengua materna […] para lograr el objetivo y optimizar tiempo.}\smallskip\\
`[…] because of time constraint, in many cases the teacher needs to be practical. I consider valid the use of L1 […] in order to reach the objective and optimize time.' (Participant 925 (\textsc{mx}), our translation)
\end{quote}

The above excerpt suggests that student teachers sometimes feel pressured and make decisions based on factors such as time constraint without considering their impact on the learners’ language development. As for the last sub-category, student teachers seem to be inclined to resorting to the learners’ L1 when it comes to disciplinary issues. This can be seen in the following excerpt:

\begin{quote}
\emph{Speaking in the target language as much as possible, using L1 for important or meaningful events (intense need for clarification, discipline, etc.) […]} (Participant 957 (\textsc{qc}))
\end{quote}

This quote is quite powerful since it actually demonstrates that when the language is used in a meaningful context, such as the one presented above, student teachers favor the use of the learners’ L1. It is important to mention that most comments related to disciplinary issues were written by participants from Quebec (17 comments). In other words, it seems that the Quebec student teachers believe that this factor could be the most relevant one when it comes to resorting to other languages in the classroom.

To summarize results for the first research question, as previously mentioned, there is a definite difference between the two populations participating in the study. On the one hand, the Quebec student teachers seem to be resistant to resorting to other languages than the TL in the classroom. In fact, the only situation where it appears to be acceptable for them to use other languages is for disciplinary issues. This being said, the “anti-L1-attitude” identified by \citet{Cook2001} regarding teaching methodologies in the twentieth century still seems to be existent nowadays. On the other hand, the Mexican student teachers demonstrated that they were open to the use of other languages in most situations. More specifically, their responses suggest that the use of the learners’ L1 in beginner levels is perceived as essential for both teaching and learning purposes. Moreover, they seem to be inclined to use other languages as pedagogical strategies, for instance, for metalinguistic descriptions, instructions and clarification, which could further support the idea of a more teacher-centered approach used in this context.



\subsection{Student teacher’s belief profiles}
\largerpage


With reference to the second research question, this study explored to what extent student teachers in Quebec and in Mexico appear to be open to using L1 or other languages in their personal projected practice. As previously mentioned, all participants were classified in one of the resulting belief profiles: Hardline-TL-only, open-to-other-languages and multivoicedness. 

Considering the forty participants that completed the vignette-based questionnaire, a total of eight who were classified as hardline-TL-only. As for the open-to-other-languages, sixteen participants showed a tendency towards this belief profile. Finally, a total of sixteen participants were identified as demonstrating multivoicedness. Before looking further at the student teachers’ responses, it is important to mention that there were great discrepancies between contexts in terms of their belief profiles (see \tabref{tab:paquet:2}).

\begin{table}
\begin{tabular}{lrr}
\lsptoprule
Context & Quebec & Mexico\\\midrule
Hardline-TL-only & 8 & 0\\
Open-to-other-languages & 3 & 13\\
Multivoicedness & 9 & 7\\
\lspbottomrule
\end{tabular}
\caption{\label{tab:paquet:2}Distribution of student teachers’ belief profiles}
\end{table}

As shown in \tabref{tab:paquet:2}, the student teachers from Quebec are the only ones who manifest a hardline-TL-only belief profile ($n=8$). The following excerpt reflects this observation.

\begin{quote}
\emph{Non. On doit apprendre le français en français et l'anglais en anglais.}\smallskip\\
`No. We need to learn French in French and English in English' (Participant 816 (\textsc{qc}), our translation)
\end{quote}

This excerpt suggests that the participant does not recognize the potential benefit of using other languages when it comes to learning and/or teaching a new one. In other words, participants from this belief profile do not perceive language learning as a dynamic process, as \citet{HerdinaJessner2002} suggested in their dynamic model of multilingualism. 

With reference to the second belief profile, open-to-other-languages, most participants classified in this category were from the Mexican context ($n=13$). These student teachers’ responses were mentioning exclusively the use of the students’ L1, excluding other languages than the TL from the equation, and tended to focus on explicit grammar instruction. This can be seen in the next quote:

\begin{quote}
\emph{Una explicación de gramática (por ejemplo) en primera lengua ayuda a los estudiantes entender la lengua meta y las diferencias entre la lengua meta y la lengua materna mejor.}
\newpage\noindent
`A grammar explanation (for example) in the L1 helps learners better understand the target language and the differences between the target language and their L1' (Participant 103 (\textsc{mx}), our translation)
\end{quote}

In this quote, the participant’s response is focusing on the teacher providing a metalinguistic description of a grammar concept using the learners’ L1. As mentioned earlier in this section, it seems that the Mexican student teachers tend to teach in a more traditional way, which may lead them to resort to other languages (mostly the learners’ L1) as they go from one grammar element to the other. Finally, the two following excerpts reflect what was identified as multivoiced discourse:

\begin{quote}
\emph{Ella debería seguir usando la lengua meta, según lo que se me ha enseñado y según el objetivo.}\smallskip\\
`She should keep on using the target language, according to what I was taught and in accordance with the objectives' (Participant 349 (\textsc{mx}), our translation)
\end{quote}

\begin{quote}
\emph{El uso de la lengua de los estudiantes me parece una buena estrategia cuando se puede tener dificultad para que entiendan en la lengua meta […]}\smallskip\\
`The use of the students’ L1 seems to me as a good strategy when they have difficulties to make sure they understand the target language […]' (Participant 349 (\textsc{mx}), our translation)
\end{quote}

\begin{quote}
\emph{Depending on the degree of difficulty of the grammar point she is currently explaining, it could be more efficient (time-wise) to switch to the mother tongue of the students, however it has been proven that the teacher is doing a disservice to the students if he/she uses the mother tongue extensively. It lets the learners be intellectually lazy because they know that the teacher will use their tongue, therefore they do not have to struggle to understand the target language.} (Participant 358 (\textsc{qc}))
\end{quote}

As these excerpts illustrate, participants in the multivoicedness profile recognize both the importance of maximal exposure and the potential benefit of using other languages in the classroom, thus reflecting two apparently contradicting perspectives. The Mexican student teacher indicates that using exclusively the TL was enforced by teacher educators, which may not reflect the participant’s professional choices. Moreover, it seems that using other languages consists in the teacher resorting to the learners’ L1 to explain new TL structures. As for the Quebec participant, using other languages appears to be a ``last resort'' strategy to cope with structural difficulties. In other words, when problem-solving is needed, this participant tends to resort in the learners’ L1, which appears to reflect a spontaneous decision rather than a theoretically informed judgment.  

To summarize results for the second research question, as previously mentioned, a line can be drawn between the two populations participating in the study. The only participants that demonstrated a hardline-TL{}-only position are from Quebec. More specifically, these student teachers indicated that it was essen\-tial to use the TL in the LN classroom at all time, and that resorting to other languages has a negative effect on learners’ language development. As for the Mexican participants, the majority were classified in the open-to-other-languages category. The results suggest that their decisions were influenced by the presumed focus of their courses, that is to say where language is the object of study. Finally, the multivoicedness profile was distributed across contexts and will be further addressed in the discussion section. 


\section{Discussion and conclusion}


In this section, the overall findings will be reviewed and discussed with a view to debunking student teachers’ beliefs regarding the TL-only rule and to making concrete suggestions to promote reflective teaching and to initiate changes in teacher training programs. As for conditions that were perceived as acceptable for resorting to other languages, the distribution of factors was different in the two student populations. Specifically, while student teachers from Mexico mostly referred to situations related to pedagogical strategies and learner factors, the main factor listed among Quebec participants was discipline. 

On the whole, Mexican student teachers seem to be more inclined to accepting the use of other languages, especially when this deviation from the TL-only rule had a pedagogical purpose, or to palliate individual learners’ needs. At first sight, this tendency would seem encouraging to teacher trainers since it appears to reflect professionally motived decisions rather than spontaneous reactions to overwhelming situations. However, when answering vignettes from a teacher perspective, Mexican participants never mentioned that other languages, including the L1, could be used as a reference to better understand TL structures. Rather, the comments subsumed under “pedagogical strategies” were mostly related to explaining grammar in the L1 to ensure understanding. This apparent focus on explicit grammar instruction may stem from their own experience as learners, supposedly based on a more textbook-based, deductive approach where “the much-discredited presentation--practice--production procedure still prevails regardless of the pedagogical label on the coursebook” (\citealt{TomlinsonMasuhara2018}). To summarize findings regarding the Mexican participants’ justifications for resorting to other languages, the latter do not reflect the kind of pedagogical reflection targeted in crosslinguistic awareness pedagogy, that is to draw on correspondences across the whole repertoire. Instead, L1 is perceived as a vehicle rather than a resource. This observation could serve as an anchor point for teacher educators.

Even though less of the Quebec participants’ comments were related to “pedagogical strategies”, their main focus was also on ensuring understanding when explaining grammatical structures in L1. Meanwhile, situations that would justify L1 use in this context were mostly related to discipline. Despite the fact that classroom management is not related to pedagogical strategies, the overall considerations still seem to reflect a similar concern: For really important things, teachers should use the L1. This is consistent with previous research indicating that “using the L1 for discipline signals to the students that when ‘real’ communication needs are at stake there is no need to use the L2” (\citealt[234]{EllisShintani2014}). In other words, TL should be used at all times, except when serious things are being discussed, those that require understanding. The reported experiences also reflect their apprenticeship of observation. That is, the LN instruction many Quebeckers seem to have received would essentially reflect TL-only, except for certain grammar points and discipline. While the program prescribes exclusive TL use, which would include classroom management and form-focused instruction, it may be inferred that the student teachers participating in the present study had no models available for efficient TL use in those specific situations. In sum, these findings parallel those from the Mexican context in that the comments from the Quebec population was also at variance with pedagogical reflections in which resorting to other languages is perceived as a steppingstone for crosslinguistic awareness. Despite this comparable perception of L1 as a vehicle to face serious or problematic classroom situations, the observations related to the Quebec participants’ beliefs might serve as a different anchor point for teacher educators. Namely, instead of directly addressing the potential benefits of using other languages as a resource, teacher trainers might also want to debunk the idea of failure when resorting to L1. Specifically, they could examine whether exclusive TL use is favored because student teachers believe that this is what their cooperative teachers, their practicum supervisors and professors want them to do, or because they were shown how expert teachers “fail”.

As for the belief profiles identified across the sample, the predictions were confirmed in that more hardline TL-only promoters were listed among the Quebec participants, while Mexican student teachers were generally more open to using other languages. However, if openness merely translates into L1-is-ok, this does not necessarily entail pedagogical reflection. In other words, the presumed openness to using other languages may not reflect the potential for those participants to integrate crosslinguistic activities in their own classrooms. As for the profiles reflecting multivoiced discourse, which represented roughly a third of the participants in each context, they point to a compromise position, where student teachers find themselves navigating the theory-practice divide. While initially interpreted as contradictions, these multiple voices may rather be considered a reflection of student teachers’ complex realities. These insights are highly valuable since they reveal apprenticeships of observation, thus providing anchor points for teacher education.

To conclude, the two researchers, who have started exploring their role as teacher educators, are hopeful. Even though barely tangible, their goal remains to implement findings from TLA research, whereby language learners make the most of their multilingual repertoire, by drawing on their linguistic resources effectively and with teacher guidance. A first step to be undertaken is to steer away from the assumption that using the TL exclusively makes for efficient language learning and teaching. To achieve this perceptual shift, there is a need to raise student-teachers, teachers as well as teacher-trainers’ awareness of the importance of treating students’ as multilingual learners (\citealt{ArocenaEgañaEtAl2015}), to introduce crosslinguistic pedagogies as a regular feature of teacher-training programs (\citealt{DeAngelis2011}), and to create more researcher-teacher collaboration to facilitate the implementation of multilingual tasks \citep{GalanteEtAl2020}. These three moves would lead people to question their own beliefs critically, to challenge those beliefs by experiencing new ways of learning and trying out new ways of teaching, and by adding a layer of pedagogical intention to the existing layers of apprenticeship of observation.

\section*{Abbreviations}

\begin{tabular}{@{}llll}
  LN & Additional language & TLA & Third language acquisition
\end{tabular}




\sloppy\printbibliography[heading=subbibliography,notkeyword=this]
\end{document}
