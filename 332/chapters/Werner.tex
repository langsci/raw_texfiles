\documentclass[output=paper,hidelinks]{langscibook}
\ChapterDOI{10.5281/zenodo.7525463}
\author{Eduard Werner}
\title{Evaluating linguistic variation in light of sparse data in the case of Sorbian}
\abstract{The severely endangered Sorbian languages (ISO hsb, dsb), endemic to the Eastern part of Germany, are dramatically under-researched. This lack of research extends from basic knowledge about numbers of speakers, competence, and language transmission, but includes also core aspects of linguistics, like phonology, morphology and syntax. 

Having experienced centuries of marginalisation, Sorbian texts are (sparsely) attested only from the 16\textsuperscript{th} century. This makes evaluation of variation especially difficult, since the variation might be caused by a certain register of the language, for example, a special dialect (our default assumption), but it might also be caused by other factors such as traditions of verbal art, notably in folksongs which unfortunately have not been preserved in their original form either and are therefore hard to evaluate, but which contain very old layers of language.

In this chapter, one of the oldest Sorbian monuments will be compared to folksongs, applying knowledge about neighbouring Germanic and Celtic literatures. From the linguistic side, the results of the comparison lend greater insights into historical sound changes in Sorbian; from the cultural side, we learn about aesthetic concerns of verbal art in this language, which, in turn shed light on a range of linguistic phenomena beyond sound patterns.
}
\IfFileExists{../localcommands.tex}{
  \addbibresource{../localbibliography.bib}
  % add all extra packages you need to load to this file

\usepackage{tabularx,multicol}
\usepackage{url}
\urlstyle{same}

\usepackage{listings}
\lstset{basicstyle=\ttfamily,tabsize=2,breaklines=true}

\usepackage{langsci-basic}
\usepackage{langsci-optional}
\usepackage{langsci-lgr}
\usepackage{langsci-osl}
% \usepackage{./langsci/styles/langsci-lgr}
% \usepackage{./langsci/styles/langsci-osl}
% \usepackage{langsci-gb4e}

\usepackage{tikz}
\usetikzlibrary{patterns,calc}
\pgfdeclarepatternformonly{south east lines}{\pgfqpoint{-0pt}{-0pt}}{\pgfqpoint{3pt}{3pt}}{\pgfqpoint{3pt}{3pt}}{
    \pgfsetlinewidth{0.6pt}
    \pgfpathmoveto{\pgfqpoint{0pt}{3pt}}
    \pgfpathlineto{\pgfqpoint{3pt}{0pt}}
    \pgfpathmoveto{\pgfqpoint{.2pt}{-.2pt}}
    \pgfpathlineto{\pgfqpoint{-.2pt}{.2pt}}
    \pgfpathmoveto{\pgfqpoint{3.2pt}{2.8pt}}
    \pgfpathlineto{\pgfqpoint{2.8pt}{3.2pt}}
    \pgfusepath{stroke}}
    
\usepackage{stmaryrd}
\usepackage{wasysym}
\usepackage{multirow}
\usepackage{caption}
\usepackage{subcaption}
\usepackage{mathrsfs}
\usepackage{qtree}

\usepackage{linguex}


  %pminos do not split footnotes
% \interfootnotelinepenalty=10000 %Footnote in Laporte chapters has to be split SN


%\DeclareIndexNameFormat{default}{%
%\nameparts{#1}%
%\usebibmacro{index:name}%
%{\index[names]}%
%{\namepartfamily}%
%{\namepartgiveni}%
% {}% L1
% {}% L2
%{\namepartprefix}% generates spurious space L3
%{\namepartsuffix}% generates spurious space L4
%}

%  {\DeclareIndexNameFormat{default}{%
%     \usebibmacro{index:name}{\index[names]}{#1}{#3}{#5}{#7}}}

%\DeclareIndexNameFormat{default}{%
%  \usebibmacro{index:name}{\sindex[nom]}{#1}{#3}{#5}{#7}}

%\DeclareIndexNameFormat{default}{%
%  \usebibmacro{index:name}{\sindex[person]}{#1}{#3}{#5}{#7}}
%\DeclareIndexNameFormat{default}{%
%\nameparts{#1} \usebibmacro{index:name}{\sindex[person]]}{\namepartfamily}{‌​\namepartgiven}{\nam‌​epartprefix}{\namepa‌​rtsuffix}}

%\newcommand{\smiley}{:)}

%\renewbibmacro*{index:name}[5]{%
%\usebibmacro{index:entry}{#1}%
%{\iffieldundef{usera}{}{\thefield{usera}\actualoperator}\mkbibindexname{#2}{#3}{#4}{#5}}}

% \newcommand{\noop}[1]{}

%remove for final
%\overfullrule=1mm

\newcommand{\tobi}[2]}}
\renewcommand{\S}[1]{\tobi{#1}{\textsc{*}}}

% this volume references
% puts: [this volume]
% already defined: \citetv
%\newcommand{\citepv}[1]{(\citeauthor{#1} \citeyear*{#1} [this volume])}
\newcommand{\citealtv}[1]{\citeauthor{#1} \citeyear*{#1} [this volume]}

%parentheses around example number
\newcommand{\pref}[1]{(\ref{#1})}

% in-text examples

\newcommand{\lnex}[1]{\textit{#1}} %target lang word
\newcommand{\lnlit}[1]{(lit.: `#1')} %literal reading
\newcommand{\lnlat}[1]{(#1)} % latinization
\newcommand{\lntrans}[1]{`#1'} %translation
\newcommand{\lnexl}[2]%
{\lnex{#1}{} \lnlat{#2}} % ex with latinization
\newcommand{\lnexlat}[3]{\lnex{#1}{} \lnlat{#2}{} \lntrans{#3}} % ex with latinization and tranl.

%ch01
\newcommand{\co}[1]{\mbox{\textbf{#1}}}

%ch09

\newcommand{\cyrbulg}[1]{\begin{otherlanguage*}{bulgarian}#1\end{otherlanguage*}}


%ch10
\newcommand{\nlp}{{\small NLP}}
\newcommand{\mwe}{{\small MWE}}
\newcommand{\rae}{{\small RAE}}
\newcommand{\lvc}{{\small LVC}}
\newcommand{\pos}{{\small P}o{\small S}}
%\newcommand{\todo}[1]{ \textcolor{red}{#1} }

%\renewcommand{\labelenumi}{\theenumi}
%\ainamefmt{{vv}{ll}{, ff}{, jj}} % fullname

\newcommand{\biberror}[1]{{\color{red}#1}}

\newcommand{\osenovaitem}{--~} 
  %% hyphenation points for line breaks
%% Normally, automatic hyphenation in LaTeX is very good
%% If a word is mis-hyphenated, add it to this file
%%
%% add information to TeX file before \begin{document} with:
%% %% hyphenation points for line breaks
%% Normally, automatic hyphenation in LaTeX is very good
%% If a word is mis-hyphenated, add it to this file
%%
%% add information to TeX file before \begin{document} with:
%% %% hyphenation points for line breaks
%% Normally, automatic hyphenation in LaTeX is very good
%% If a word is mis-hyphenated, add it to this file
%%
%% add information to TeX file before \begin{document} with:
%% \include{localhyphenation}
\hyphenation{
    Beck-man
    Ngu-yen
    back-chan-nel
    back-chan-nels
    mo-not-o-nous
    ste-reo-typ-i-cal
}

\hyphenation{
    Beck-man
    Ngu-yen
    back-chan-nel
    back-chan-nels
    mo-not-o-nous
    ste-reo-typ-i-cal
}

\hyphenation{
    Beck-man
    Ngu-yen
    back-chan-nel
    back-chan-nels
    mo-not-o-nous
    ste-reo-typ-i-cal
}
 
  \togglepaper[10]%%chapternumber
}{}

\shorttitlerunninghead{Evaluating linguistic variation in light of sparse data}
\begin{document}

\maketitle



\section{Goal of this paper}

 
The goal of the investigation is to get a more reliable reconstruction for older layers of Sorbian (Upper Sorbian, ISO hsb and Lower Sorbian, ISO dsb) on the one hand and more concrete ideas about Sorbian verbal art on the other hand. As a starting point, elements of verbal art from other European cultures (restricting ourselves to examples from Old High German and Welsh in this paper in order to demonstrate such elements) will be applied to Sorbian folk song texts and monuments. It will be shown that these elements make a lot of sense especially if applied to a reconstructed text with a phonologically older layer.
 

\section{Introduction}
 
The Sorbs, a Slavic people indigenous to the Eastern part of Saxony and Brandenburg (see \figref{fig:werner:1}), are one of the four acknowledged autochthonous minorities of Germany, the others being the Frisians, the Sinti/Roma and the Danes.
 

\begin{figure}
  \includegraphics[width=\textwidth]{figures/Sorbian.pdf}
  \caption{Area of settlement of the Sorbs in Germany}
 \label{fig:werner:1}
\end{figure}
%%please move the includegraphics inside the {figure} environment
%%\includegraphics[width=\textwidth]{figures/iclavearticlenewformatted2-img002.png}
 
They are first mentioned in the year 631 of Fredegar’s chronicle from the early Middle Ages and they are the last remnants of the Slavic-speaking population which once reached the Baltic sea in the North and Frankfurt/ Main and Hamburg in the West \citep[13]{Stone2015}. During the 8\textsuperscript{th} century a stable border between German (the area without red dots) and Slavic (the area dominated by red dots) population emerged as can be seen from the following map (\figref{fig:werner:2}), which shows the distribution of place names ending in -itz\footnote{These place names trace back to Slavic patronymic names ending in *-ici with only rare exceptions like \textit{Urmitz} in Rheinland-Pfalz which come from Latin.}. However, German slowly expanded, and after the assimilation of the Polabians (in the region of Lüchow-Danneberg at end of the 18\textsuperscript{th} century) the Sorbians were the only autochthonic Slavs left.\footnote{For a comprehensive study see \citealt{Stone2015}.}
 



\begin{figure}
%%please move the includegraphics inside the {figure} environment
\includegraphics[width=\textwidth]{figures/itzcities.png}
\caption{Names ending in -itz. The region outlined in black is roughly the contemporary region of the Sorbian languages. Source: http://deutschlandkarten.nationalatlas.de/wp-content/namensatlas/}\label{fig:werner:2}
\end{figure}

 
For Sorbian, the first monuments maintained appear at the beginning at the 16\textsuperscript{th} century, roughly 900 years after their first mentioning in the chronicle of Fredegar.\footnote{An exception is the Kayna stamp presumably from the beginning of the 10\textsuperscript{th} century which is an Old Polabian monument (and the only one) found on Old Sorbian territory \citep{Werner2004}. It shows that the Slavic languages were used in official contexts as well. Unfortunately, no other monument from this time seems to have been preserved.} Moreover, the first monuments are usually translations of Christian clerical texts for missionary purposes so they do not reflect the normal context of language usage of that time.\footnote{The oldest Sorbian sentence from 1510, however is a declaration of love \citep{Wornar2014}.} For most of the originally Slavic territory which has been germanised long ago, no information of the language and culture other than place names has survived. 
 

 
The Slavic gentry was germanised early on and Christianity extinguished or significantly diminished old domains of Sorbian language and culture like pagan religion and connected fields such as sorcery and medicine. Apart from the monuments mentioned, only songs have survived. However, they have also changed due to the lack of professional bards; songs got passed on by peasant workers coming from other villages for seasonal work. The oldest Sorbian songs we know have been collected during the 19\textsuperscript{th} century. Details regarding they were performed, the context in which they were performed, and even the melodies are sketchy \citep[199]{Nedo1966} for technological and methodological reasons\footnote{It was obviously not possible to record performances and intervals and rhythm were written through the filter of musical notation for classical European sheet music.}. Moreover, \citet{HauptSmoler1843} were no folk musicians. They often relied on songs that had been collected by others such as Jordan or Zejler. Altogether we have 1,500 Sorbian song lyrics documented \citep[176]{Nedo1966}.
 
Nowadays, the Sorbian languages are highly endangered; for Lower Sorbian the number of native speakers is down to a few hundred \citep{Walde2004, Lewaszkiewicz2014}. For Upper Sorbian, there is still a territory where Sorbian is being passed on as a family language, but their number is declining fast as well and might be no more than 5,000 (ibid.). On the one hand, the Sorbian languages display strong German influence in every subsystem of the language, on the other hand we find archaic phenomena like a complete dual\footnote{The dual does not only occur with nouns and adjectives, but \textsc{pron}ouns, verbs etc. as well, and it does not need a trigger (like \textit{two} or \textit{both}) to be used, e.g. ‘those (two) children are playing’ would be USo \textit{wonej dźěsći sej hrajkatej \textsc{pron}.\textsc{nom}.\textsc{du}} \textit{N.\textsc{nom}.\textsc{du}} \textit{\textsc{pron}.\textsc{dat}} \textit{V\_2/3du}as opposed to ‘those children are playing’ \textit{wone dźěći sej hrajkaja. \textsc{pron}.\textsc{nom}.\textsc{pl}} \textit{N.\textsc{nom}.\textsc{pl}} \textit{\textsc{pron}.\textsc{dat}} \textit{V.\textsc{3pl}} (cf. \citealt[429ff, fn. 29]{Faßke1980}).
} or supine\footnote{In Lower Sorbian, the supine is a form of the infinitive required when expressing a movement involved in order to act, e.g. \textit{I go to sleep} vs. \textit{I want to sleep.} The first sentence would be expressed with a supine (LSo \textit{du spat [V.\textsc{1sg}} \textit{V\_\textsc{sup}]}), the second with an infinitive (\textit{cu spaś [V.\textsc{1sg} V.\textsc{inf}]}). Cf. \citet[354]{Janaš1976}}.
 

\section{Verbal art from Proto-Indo-European to Slavic}
\begin{quote}
We are inclined to view the oral tradition of a culture as an early stage of progress, succeeded by the more stable and permanent stage of recorded language. \citep[340]{Berleant1973}%(Berleant~1973:~340)
\end{quote}

 
PIE seems to have had two distinct types of metrics, a syllable-based one with a quantitative rhythm and a fixed number of syllables as well as a so-called strophic style with relatively short lines and no syllable count. According to \citet[35]{Fortson2010}, this style is especially characteristic of archaic liturgical and legal texts.
 

 
\hspace*{-1.5pt}The \textit{syllabic} type is commonly known from antique epics like the Vedic Rigveda (presumably from the second millenium BCE\footnote{Cf. \citet[208]{Fortson2010}.}), the Sanskrit Mahābhārata (ca. 400 BCE\footnote{No exact date can be given, but the compilation must have been undertaken after Pā
ṇini’s work \citep[208]{Fortson2010}.}), the Old Greek Illiad and Odyssey (presumably 8\textsuperscript{th} century BCE\footnote{\citet[249]{Fortson2010}.}) as well as all the Latin classics. The system is based on vowel length and consonant clusters. Vowels can feature \textit{natural} or \textit{positional} length whereby \textit{positional length} basically means a short vowel and a consonant cluster.
 

 
In this type, end rhymes, alliterations and so on can occur. However end-rhymes do not play a central role \citep[17f, 211f]{Coulson2017}.
 

 
The strophic type in Hittite, Avestan, Umbrian, Classical Armenian and Old Irish shows that both forms have co-existed with the same geographical scope, since Umbrian coexisted with Latin, and so did Avestian and Sanscrit. This type is characterised by grammatical and phonetic parallelism \citep[35]{Fortson2010}.
 

 
As has been argued that the strophic style might be the older one \citep[35]{Fortson2010}, but this discussion is outside the scope of this article. Let it suffice to agree that both types were present at a late PIE period. As it seems, it was perfectly possible to combine both types and end up with a syllable-counting system containing alliterations, as can be seen from the example of a South Picene epitaph given by \citet[301]{Fortson2010}, the inscription Sp TE 2, found in Bellante near Teramo:
 
\ea\gll postin viam videtas tetis tokam alies esmen vepses vepeten\\
     along road-\textsc{acc.sg} see-\textsc{2.pl} Titus-\textsc{gen.sg} toga-\textsc{acc.sg} Alius-\textsc{gen.sg} this-\textsc{loc} buried grave-\textsc{loc.sg}\\
\glt “Along the road you see the toga of Titus Alius buried in this grave”\z

While much of the interpretation is still open to guesswork, the artistic part is much clearer: We have three alliterating phrases consisting of seven syllables each which can again be divided into two two-syllable units and a final trisyllable. In each of the phrases we have alliterations: \textit{viam – videtas}, \textit{tetis – tokam}, \textit{vepses – vepeten}.
 

Repetition of sounds (including alliteration, assonance, and, less frequently, end-rhyme) is characteristic of IE poetry even outside the strophic style. A line like the following, from the Roman comic playwright Plautus (\textit{Miles Gloriosus} 603), is quite typical of the technique:

\ea
sī minus cum cūrā aut cautēlā locus loquendī lēctus est\\
“If your place of conference is chosen with insufficient care or caution ...(trans. P. Nixon)\z


We have the alliterating \textit{k} sounds (spelled \textit{c}) of \textit{cum cura aut cautela} followed by \textit{l}’s in \textit{locus loquendi lectus}, all of which also have \textit{k} sounds in their interior [...]. In Plautus, the repetition of these sounds is partly for comic effect [...] \citep[37]{Fortson2010}


 
Of course, the sound effects can vary and all sound patterns as well as the rhythmical patterns can also serve to provide a mnemonic aid and, therefore, ensure that the text is passed on unchanged. That would be especially important in liturgical and legal texts of all sorts as well as folk medicine, sorcery, etc.
 

 
The sound changes which led to the rise of Proto-Slavic\footnote{According to \citet[419]{Fortson2010} this occurred before the 5\textsuperscript{th} century.} (PSl) had a very significant impact on both of these systems. First, the PIE opposition of long and short vowels was abandoned. As a result there was no phonological vowel length (although the PSl yers\footnote{Yers were ultrashort vowels reflecting mostly PIE *u and *i; they got lost in many positions later in the individual Slavic languages.} must have been much shorter than the other vowels, cf. \citealt[48ff]{Schaarschmidt1997}). Secondly, the principle of open syllables greatly reduced the possibility of having consonant clusters.\footnote{For an extensive introduction of the sound changes of PSl see the introductory chapters of \citet{Leskien1969} and \citet{Trunte1990}.} This would have impacted both the syllabic system and the strophic system as well because many consonants would disappear reducing the possibilities of consonant alliterations. So while it is still possible to count syllables, the aesthetic system of versification must have undergone significant changes in PSl times.
 

\section{Verbal art and Sorbian}

In Old Sorbian, closed syllables were possible again after the fall of the weak yers.\footnote{Since the yers were vowels, the loss of them in certain positions (e.g. at the end of the word) led to loss of syllables and to closed syllables preceding the syllable containing the yer, e.g. PSl *on{{ъ}} > USo wón, cf. \citet[57f]{Schaarschmidt1997}} This reduced the number of syllables which would have been the only preserved feature of verbal art from PIE times then. Other changes also occurred affecting consonants or eliminating certain consonant groups. (Due to the lack of written documents it is not possible to date these changes absolutely.)
 
 
The following facts are noteworthy:
 
\begin{itemize}
\item While there is a well-known bardic tradition in both South and East Slavic cultures (as well as Celtic and Germanic), no such traditions are mentioned for Sorbian. This likely owes to the fact that the Sorbian gentry (as well as other Slavic tribes of the Slavia Germanica) was germanised early on. Accordingly, Sorbian bards would not find a society with good conditions to support or sustain Sorbian bardic traditions.\footnote{In the 13th century, there was a known Slavic ministrel Wizlav III from the Isle of Rügen (Cf. \citealt[12f]{Rawp1978}). While the melodies are said to contain Slavic elements (ibid.), the words of all 17 songs which have been preserved are in German, cf. \url{https://archive.thulb.uni-jena.de/collections/receive/HisBest_cbu_00008218} (accessed 18-09-2019).} This would mean that Sorbian lacked professional bards who could ensure that songs are being passed on properly\footnote{\citet[197]{Nedo1966} remarks that in many cases the lyrics of the songs have been corrupted.} and that the hidden meanings and metaphors will be taught to the people.
\item Traditional South Slavonic and Eastern Slavonic poems are dekasyllabic. \citet{Rawp1957} showed that not even the oldest surviving Sorbian songs are dekasyllabic. So we do not find a cultural connection between Sorbian and East/South Slavonic bardic tradition here.
\item Virtually all Sorbian cultural traditions were living and being passed on orally in the so-called \textit{přaza/pśěza}\footnote{While the translation would simply be \textit{spinning room,} it was more an institution of village life where not only work would be done (like typically spinning and shearing of feathers), but stories were told, songs were sung, people of all generations met and traditions were passed on.} (until well after WWII) by normal village people who were mostly unable to read or write their Sorbian native tongue.
\item Songs were passed on by ordinary people coming over from other villages to work. Usually people were keen to be taught new songs and Sorbians travelling to other Sorbian villages could be sure of being asked for all the songs they knew.
\end{itemize}
 
Sorbian Studies have so far focused on written texts (e.g. \citealt{Jenč1956}) depicting Jurij Mjeń (1727–1785) as the founder of profane Sorbian literature\footnote{Cf. \citet[130]{Jenč1956}, and \citet[76]{ČermákMaiello2011}.} with his translation of a part of Klopstock’s Messias and his \textit{Ryćerski kěrliš} in clean hexametres \citep{Stone2012}, Handrij Zejler as the father of Sorbian poetry and songs and so on.
 

 
Pawoł Nedo (1908–1984) declares Sorbian folksongs\footnote{The songs at issue are the traditional songs found mostly in \citet{HauptSmoler1843}, not the romantic songs composed by Kocor and Zejler during the 19\textsuperscript{th} century.} as primitive, non-elab\-o\-rated, imperfect, lacking rhyme, neglecting the fact that in German (which he tacitly keeps in mind) end-rhymes only became predominant in the High MA in the courts of Europe:
 

\begin{quote}
Das wichtigste Kennzeichen der gebundenen Volkssprache ist der Rhythmus, mit dem das Volk [...] recht großzügig verfuhr, entweder, weil es ihm keine entscheidende Bedeutung beimaß oder weil es eine klare rhythmische Durcharbeitung nicht bewältigte. \citep[197]{Nedo1966}\footnote{“The principal feature of folk poetry is rhythm, which has been treated rather sloppily by the ordinary people, either because it did not seem relevant to them or because they were unable to \textsc{cop}e with it properly.” [EW]}
\end{quote}

 
Nedo does notice alliterations, but does not investigate them, and calls them a substitute for the missing or imperfect rhyme\footnote{``[...] ein Stilmittel [...], das offenbar den fehlenden oder unsauberen Reim ersetzen soll” \citep[199]{Nedo1966} The term \textit{imperfectness} is also being used by \citet[98]{Kayser1954} if an assonance is used at the end of a verse.}:
 

\begin{quote}
Die sorbischen Lieder kennen auch keinen bewußt und systematisch angewandten Endreim. Wenn der Endreim erscheint, so hat man oft den Eindruck der Zufälligkeit, und er ähnelt oft mehr einer Assonanz. \citep[197]{Nedo1966}\footnote{“Sorbian songs do not even know a systematical and intentional end-rhyme. Where an end-rhyme occurs, it seems to be accidental and it looks more like an assonance.” [EW]}
\end{quote}

 
Nedo does not give examples for rhymes he considers to be imperfect, so it can only be guessed what he had in mind. He never considers the possibility that traditional Sorbian songs and music might have had its own artistic criteria so that a rhyme he perceives as imperfect or accidental might have been completely fine for the artistically educated Sorb of that time because perception of artistic means is influenced by education to a large degree.\footnote{The Welsh verses we will discuss later in this paper are such an example:~Welsh end-rhymes usually consist of a stressed and an unstressed verse while e.g. in German poetry both \textrm{normally} display the same stress pattern \citep[40f]{Kayser1954}. Of course, there are other types of end-rhyme, notably \textit{proest} which requires that the vowel of the rhyming syllables be of different quality \citep[149ff]{Llwyd2007}.} For music, something similar has been attested: Christian missionaries described Sorbian singing as cacophonic \citep[11]{Rawp1978}.\footnote{\citet[22]{Slaviunas1958} attests that in traditional Lithuanian songs parallel seconds are being perceived as aethetically pleasing.} Thus, it would have been logical and even compelling to consider such a possibility for poetry as well.
 

 
After the crushing verdict of one of the most outstanding and well-known protagonists of Sorbian culture and Sorbian public life (Nedo had been e.g. head of the Domowina, the chief organisation of the Sorbs, for many years), Sorbian folksongs have not been researched further.\footnote{A few years before, Jan Rawp (Raupp) had written on Sorbian songs: “Die formale Seite der [...] Liedtexte erweist sich in manchem als eigenartig und reizvoll. Sprachliche Gestaltungsmerkmale wie Epitheta, Alliterationen, Interjektionen u. a. vertiefen Ausdruck und Sinngebung.” [Formally, the lyrics of the songs are in many ways strange and appealing. Linguistic means such as epitheta, alliterations, interjections etc. emphasise expression and meaning.] \citep[10]{Raupp1966}. Rawp was not only a scientist, but also a musician and composer, so he had a different approach than Nedo, but unfortunately he was unable to continue his work due to health reasons.}
 

 
On general grounds, Nedo’s view which implies that Sorbian folk songs lack artistic elaboration or structures should be rejected. But his view on his own culture shows that such structures are either very much different from what he expected or hoped to find or veiled by language change (or both). Such different aesthetic systems can be found in Germanic \textit{Stabreimdichtung} or Welsh \textit{cynghanedd} (as seen in the following chapter). In both cases, verses are structured by means of alliterations.
 

\section{Verbal art in Old German poetry}
 
While Old Germanic poetry is usually dominated by alliteration rhymes \citep[435ff]{Jankuhn2005}, this is not true for Old German poetry. In German literature these structures have mostly disappeared: 
 

\begin{quote}
Die Menge überlieferter Stabreimdichtung ist in den einzelnen germanischen Sprachen recht verschieden. Im Ahd. sind es nicht mehr als 200 Zeilen. Hier scheint die Tradition im 9. Jh abgerissen zu sein [...]” \citep[1f]{vonSee1967}\footnote{The amount of traditional alliterational poetry varies according to the language at issue. In Old High German there are not more than 200 verses. Here, the tradition was discontinued during the 9\textsuperscript{th} century.} 
\end{quote}

 
Here is a well-known example from Hildebrand’s song \citep[864]{vonEckhart1729}:
 
\ea 
\textup{hiltibraht enti haðubrant, untar heriun tuem}\\
“Hildebrand and Hadubrand, between hosts two”\z

 
This is the so-called \textit{long verse} according to Snorri Sturluson (1179–1241) which is supposed to be divided into two parts, the (underlined) \textit{staff} occurs twice in the first and once in the second part (where it is supposed to be in the only stressed word). It should be pointed out that this means that there had been at least about 200 years of contact and cultural exchange between Sorbs and Germans at that time.
 

\section{Verbal art in Celtic poetry (Welsh)}
 \largerpage
There are no Celtic monuments documenting verbal art from the region which is now Germany and from the Celtic cultures that must have been in contact with the Slavic and German cultures. Therefore, some Welsh verses will be taken as an example for Celtic since Welsh versification is very well documented \citep{Morris-Jones1930, Llwyd2007}. The following verses from the 15\textsuperscript{th} century by Dafydd ap Edmwnd will illustrate the intricate sound patterns as well as how easy it is to miss them if one is not familiar with them:\footnote{The following analysis is by no means complete. It mainly serves as an example of an intricate system of alliteration which is still alive and well documented.}
 
\ea
Oeri y bûm ar y barth er cyn cof, a’r ci’n cyfarth.\\
“Freezing I was on the ground longer than memory, and the dog was barking.”\z

 
People mainly familiar with modern Sorbian, Lithuanian or German poetry might only spot the end-rhyme and some individual alliterations (\textit{bûm – barth, ci – cyfarth}), but only very few will be aware of the full complexity of the verse if they have never been introduced to the system of \textit{cynghanedd:}\footnote{For a full description of \textit{cynghanedd} as well as the \textit{cynghaneddion} given here, see \citet{Morris-Jones1930} and \citet{Llwyd2007}.}
 


\ea Oeri y bûm ar y barth \z


 
features a repeating sound pattern \textit{r-b} repeating in the first and the second part of the verse
 


\ea er cyn cof, a’r ci’n cyfarth.\z


 
shows an even more intricate pattern \textit{r-c-n-c-f}.
\clearpage

 
As mentioned above, there is no surviving poetry from the continental Celtic tribes that were originally in contact with the German and Slavic tribes, but it can be safely assumed that they will also have possessed a system of sound alliterations for verbal art, because, as aforementioned, these alliteration systems can be traced back to PIE times. For the same reason, it can be assumed that the Slavic tribes that were in contact with Germanic and Celtic tribes must have had an at least remotely similar system due to both heritage and ongoing cultural contact.
 

\section{Verbal art in Sorbian folk songs}
 \largerpage
Nedo dedicates only three pages to the language of the Sorbian folk songs \citep[196--199]{Nedo1966} and describes figures of speech only superficially. He notices the existence of repetitions, epitheta and assonances without investigating them further. \figref{fig:werner:3} provides a sample verse from an Upper Sorbian folk song from \citet[86]{HauptSmoler1843}, but comprehensive research is necessary.
 

 
The theme of the song is the widespread (in folk songs throughout Europe) \textit{knight takes girl} which might have some grounds in Slavic exogamy, but interesting is the lexeme \textit{šelma} since the word originally means ‘carrion, rotting carcass’ and later ‘villain’ and ‘executioner’ \citep[798]{Kluge2001}.
 



\begin{figure}
\caption{Text fragment from Upper Sorbian Folk Song}
\label{fig:werner:3}
\includegraphics[width=\textwidth]{figures/iclavearticlenewformatted2-img0004.png}
\end{figure}


\ea
\ea\glll Přišoł je šelma mi šelmowski,\\
V\_ł.\textsc{sg}.\textsc{m} \textsc{cop}.\textsc{3sg} N.\textsc{nom}.\textsc{sg}.\textsc{m} \textsc{pron}.\textsc{dat}.\textsc{1sg} A.\textsc{nom}.\textsc{sg}.\textsc{m}\\
Come has villain me villainous\\
\glt{}
\ex\glll Přišoł je šelma a wzał je ju preč,\\
V\_ł-\textsc{sg}-\textsc{m} \textsc{cop}.\textsc{3sg} N.\textsc{nom}.\textsc{sg}.\textsc{m} \textsc{con} V\_ł-\textsc{sg}-\textsc{m} \textsc{cop}.\textsc{3sg} \textsc{pron}.\textsc{acc}.\textsc{sg}.\textsc{f} \textsc{adv}\\
Come has {[a] villain} and taken has her away\\
\glt{}
\ex\glll Hišće mje njeje na kwas prosył.\\
\textsc{adv} \textsc{pron}.\textsc{acc}.\textsc{1sg} \textsc{cop}.\textsc{3sg}.\textsc{neg} \textsc{prep} N.\textsc{sg}.\textsc{acc}.\textsc{m} V\_ł.\textsc{sg}.\textsc{m}\\
    even me hasn’t to wedding asked.\\
\glt “Came a villainous villain,\\
Came a villain and took her away\\
Did not even invite me to the wedding.”\z\z

 
Following Nedo’s statements there is a typical repetition (\textit{přišoł je šelma}) and a tautological adjective follwing the noun (\textit{šelma} … \textit{šelmowski}) “for greater poetical expressiveness” \citep[198]{Nedo1966}. However, if an older version of the text\footnote{This has so far only been tried for a Sorbian song by \citet{Rawp1957} in order to establish the original number of syllables of the song verses, but not in order to identify other linguistic means of verbal art.} is assumed, a different picture emerges. Therefore we assume a state after the falling of the weak yers and denasalisation of nasal vowels, but before assibilation of *ŕ, vowel changes (caused by palatalisations and labialisations) and the establishing of word-initial stress in prefixes:\footnote{The following (reconstructed) verses are otherwise identical to the ones already annotated. Discussing the individual sound changes in Sorbian is beyond the scope of this article, and we must point the reader to \citet{Schaarschmidt1997}.}
 

\ea 
Prišel je šelma mi šelmowski,\\

Prišel je šelma a wzäl je ju proč,\\

Hišće mje njeje na kwas prosyl.\\
\z

 
When looking at the first two verses in example~\ref{ex:ver1-2}, note the complex alliteration rhyme with a staff \textit{šel/zäl}. The \textit{w} belongs phonetically to the preceding syllable:
 

\ea
Prišel je šelma mi šelmowski,\\
Prišel je šelma a wzäl je ju proč,
\label{ex:ver1-2}
\z

 
The first two verses start with the same pattern \textit{pr}, which is repeated in the last stressed syllables of the last two verses creating a frame:
 

\ea
Prišel je šelma mi šelmowski,\\
Prišel je šelma a wzäl je ju proč,\\
Hišće mje njeje na kwas prosyl.
\z

 
The vowel scheme of the first four syllables is identical in all three verses which suggests that the song was originally sung in a canon-like way\footnote{Slaviūnas states that many of the Lithuanian threefold \textit{sutartin\.es} were sung in as strict canons \citep[14]{Slaviunas1958} and that assonances are a necessary part of them (op. cit.:18).} :
 

\ea
Prišel je šelma mi šelmowski, \\
Prišel je šelma a wzäl je ju proč,\\
Hišće mje njeje na kwas prosyl.
\z

 
Sound structures such as the ones outlined here can be found in many songs. The function of connecting verses can also be taken over by a cynghanedd-like assonance (they also have the same stress and are accented) which can be found e.g. in the first verse of \textit{Rubježnicy} \citep[29]{HauptSmoler1843}:
 

\ea\glll Jědlenki su rubali a rěbliki su dźěłali \\
N.\textsc{acc}-pl \textsc{cop}.\textsc{3pl} V\_ł.\textsc{pl} \textsc{conj} N.\textsc{acc}.\textsc{pl} \textsc{cop}.\textsc{3pl} V\_ł.\textsc{pl}\\
pine-trees they-have chopped and ladders they-have wrought\\

 \glt{}
\z
 
\hspace*{-4pt}Here a structure \textit{r-b-l – r-b-l} occurs between the verses (\textit{rubali – rěbliki}). Around these, there is twice \textit{ki-su,} adding another element of symmetry as well as an onomatopoëtic feature (the chopping of the axes). Furthermore, the \textit{d-l} of \textit{jědlenki} finds an equivalent in \textit{dź-ł} of \textit{dźěłali} with palatalisations inversed. Finally, there are also inner rhymes \textit{ki-li-ki-li} (all these syllables are accentuated in the song).
 

 
As can be seen, there seems to have been a very rich, dense, and intricate alliterational system which needs further investigation in order to try and reconstruct individual elements of this system.
 

\section{Verbal art in Sorbian monuments}
 
In this part, the results of the analysis of the folk song will be applied to one of the oldest Lower Sorbian monuments, i.e. Richter’s baptizing agenda from 1543. It sports several unique features, but because of the sparseness of documents from this period, some of them are hard to evaluate (see \figref{fig:werner:4}):
 


\begin{figure}
\includegraphics[width=\textwidth]{figures/baptisingagenda.png}
\caption{Lower Sorbian baptising agenda from 1543 (Source: https://sachsen.digital/werkansicht/dlf/172704/3/0/\#)}
\label{fig:werner:4}
\end{figure}
 
%%please move the includegraphics inside the {figure} environment
%%\includegraphics[width=\textwidth]{figures/iclavearticlenewformatted2-img005.png}

\begin{itemize}
\item The monument shows a change *aj > ej throughout \textbf{(*}dajśo 2pl \textsc{imp} ´to give´ > dejśo, *pytajśo 2pl \textsc{imp} ´to look for´ > pytejśo, *pukajśo 2pl \textsc{imp} ´to make burst´ > pukejśo\footnote{Since orthographical issues are not being discussed here, all examples are given in a modernised orthography.}), which is phonetically very plausible, but not known from other monuments.
\item There is \textit{kśignuś} (or maybe \textit{kšygnuś}) instead of the expected *\textit{krygnuś} ‘to get’ (loanword from German \textit{kriegen}). \citet[690]{Schuster1989} interprets this \textit{kśignuś} as a hypercorrect form, but to all that is known it might as well be a dialectal variation. Probably the sound change at issue had not occurred long ago (people were still aware of it in the late 18\textsuperscript{th} century, cf. \citealt[31]{Schlegel2019}).
\end{itemize}
 
Even more striking is the term \textit{blogoslowjenje ‘}blessing’ which is obviously a Church Slavonic (ChSl) term (\textit{blagoslavljen{ь}je}) where we would strongly expect something like \textit{žognowanje} (from German \textit{segnen}) because that is the only attested term for ‘blessing’ in any other Sorbian monuments and because there are many other German loanwords in this monument as well. The phonetic adaptation of the word, however, makes it very plausible that it is part of a very old (and at that time long severed) ChSl connection and not merely an ad-hoc-loanword introduced by an educated writer.
 

 
The most interesting part, however, is the rendering of Mt 7,7:
 

\begin{quote}
Ask, and it shall be given you; seek, and ye shall find; knock, and it shall be opened unto you.
\end{quote}

 
From the phonological system of the monument and other Sorbian (or Polish or Czech) sources, we would expect the following translation:
 

\ea
\gll Pšosćo, ga buźo wam dano; pytejśo, ga buźośo namakaś; klapejśo, ga buźo wam wotworjono.\\
V.2\textsc{pl}.\textsc{imp} \textsc{conj} \textsc{cop}.\textsc{3sg}.\textsc{fut} \textsc{pron}.\textsc{d}.2\textsc{pl} \textsc{\textsc{part}}.\textsc{nom}.\textsc{sg}.\textsc{n}; V.2\textsc{pl}.\textsc{imp} \textsc{conj} \textsc{cop}.2\textsc{pl}.\textsc{fut} V.\textsc{inf} V.2\textsc{pl}.\textsc{imp} \textsc{conj} \textsc{cop}.\textsc{3sg}.\textsc{fut} \textsc{pron}.\textsc{dat}.2\textsc{pl} \textsc{part}.\textsc{n}.\textsc{sg}.\textsc{n}\\
\glt{}
\z
However, the words in the monument read\footnote{Lines 6-8 of the manuscript, rendered in contemporary orthography for convenience´s sake.}:

\ea 
\gll Pšosćo, ga buźo braś;\\ 
V.2\textsc{pl}.\textsc{imp} \textsc{conj} \textsc{cop}.\textsc{3sg}.\textsc{fut} V.\textsc{inf} \\

\gll pytejśo, ga buźośo spotkaś; pukejśo, ga buźo wam wotworjono.\\
V.2\textsc{pl}.\textsc{imp} \textsc{conj} \textsc{cop}.2\textsc{pl}.\textsc{fut} V.\textsc{inf} V.2\textsc{pl}.\textsc{imp} \textsc{conj} \textsc{cop}.\textsc{3sg}.\textsc{fut} \textsc{pron}.\textsc{dat}.2\textsc{pl} \textsc{part}.\textsc{n}.\textsc{sg}.\textsc{n}\\
\glt{}
\z


 
According to any other source of Lower Sorbian, this sentence should be translated as:
 

\begin{quote}
Ask, and he will take; seek, and ye shall stumble; make it burst, and it shall be opened for you.
\end{quote}

\largerpage
 
Schuster does not discuss this passage at all \citep[293]{Schuster1967}, but simply states in the Sorbian etymological dictionary \citep{Schuster1989} that \textit{spotkaś} also means ‘to find’, and that \textit{pukaś} also means ‘to knock at a door’ (as in Polish \textit{pukać}) although this document is the only source for these meanings. But even then, the passage remains unclear (\textit{ask, and he will take}). In his edition of Sorbian language monuments (\citeyear{Schuster1967}) Schuster therefore tacitly conjectures \textit{pšosćo, ga buźo braś} to \textit{pšosćo, ga buźośo braś,} which means \textit{ask, and you will take} \citep[293]{Schuster1967}. But in spite of the conjecture, the translation is still not an acceptable translation of Mt 7,7.\footnote{It should be mentioned that there is a translation in the oldest Upper Sorbian catechism of Warichius from 1595 \citep[126]{Schuster2001} which is in line with Schuster´s conjecture (not with the manuscript discussed here) which could be from a different part of the Bible (John 16:24) and which we will not discuss here as it is significantly more recent and from a different region.} Keeping in mind that the document does not contain any obvious errors, I would be unwilling to accept the conjecture, especially since it fails to provide a full explanation of the deviations from other documents.
 

 
One interpretation is that this passage is an old parody\footnote{I
    would like to thank Patrick McCafferty from the UL for pointing out that such parodies exist in Irish. In a Slavic context, one could compare the oldest Western Slavonic (presumably Old Sorbian) sentence ukriwolsa.
    }
which is found in the manuscripts of Thietmar von Merseburg, where the Greek $\kappa\acute{\upsilon}\rho \iota \varepsilon \hspace*{0.1cm} \acute{\varepsilon} \lambda \acute{\varepsilon}\eta \sigma o\nu $ had been turned into a sentence meaning `there is an alder-tree at the bush’ \citep[27]{Stone2015}. The first part (\textit{ask and he shall take}) could then refer to taxes and duties. But it is also possible that the changes were introduced solely in order to produce an aesthetically more pleasing text. Sorbian culture at that time was an oral culture, not a written one, and Christian contents were in any case unintelligible to the Sorbian peasants. Therefore, while a priest could not excel by conveying content, he could still earn the respect of his parish demonstrating oratory skills. Accordingly, the sentence in example~\ref{ex:priest} will be considered from an artistic standpoint, starting with the expected (reconstructed) wording (differences between this reconstruction and the monument are underlined):\footnote{Newer phonology can be applied here. Only *r is reintroduced instead of š in \textit{pšosćo} so the reader without knowledge in historical Sorbian phonology can follow more easily. Speakers were still aware of the change *r > š at the end of the 18\textsuperscript{th} century \citep[31]{Schlegel2019}, and the aforementioned example of \textit{kšygnuś} vs. \textit{krygnuś} shows that this sound change was still active for the time of the monument examined here.}

\ea
\begin{quote}
    Prosćo, ga buźo \uline{Wam dano}.\\
    Pytejśo, ga buźośo \uline{namakaś}.\\
\uline{Klapejśo}, ga buźo Wam wotworjono.\\
\end{quote}
\label{ex:priest}
\z
 
Observe how all the verses consist of two parts. There is an alliteration p-p in the first two lines, while the second part of all three verses start with \textit{ga buźo.} There is also a sort of climax in the second parts of the verses in so far as there are six syllables in the first verse (\textit{ga buźo Wam dano}), seven in the second (\textit{ga buźośo namakaś}) and eight in the last verse (\textit{ga buźo Wam wotworjono}). This might have been perceived as an aesthetically pleasing starting point, but from an artistic point of view, it could definitely be improved.
 

 
Looking at the first verse, \textit{Wam dano} displays no alliteration or assonances and is not linked to anything. However, substituting \textit{Wam dano} with \textit{braś} does two things:
 

\begin{enumerate}
\item It creates a sound chain \textit{pr – b – br.}
\item It changes the meaning from \textit{it will be given to you} to \textit{(he) will take}, which could be a message along the lines of: \textit{Now it sounds correctly and we can understand it properly: so that is what the Christians really mean – they are not giving, but taking.} This might reflect the experience of the Sorbians with the Christian church as they were forced to attend church service \citep[150]{Knauthe1767}.
\end{enumerate}
 
Assuming a change of the text here from \textit{Wam dano} to \textit{braś} would also explain the fact that we have \textit{buźo} \textsc{3.sg} and not \textit{buźośo} \textsc{2.pl} and would render Schuster’s tacit conjecture unnecessary.
 

 
In the second verse, we have the same artistic problem – \textit{namakaś} is not connected. The substitution of \textit{namakaś} with \textit{spotkaś} again has two effects:
 

\begin{itemize}
\item It creates an alliteration \textit{p-t – p-t} with \textit{pytejśo – spotkaś}.
\item It creates a parody \textit{seek and you shall – no, not find, but stumble.} \citet[120ff]{Knauthe1767} mentions the Sorbians did not like to go into the cold and dark Christian churches.
\end{itemize}
 
And finally, when substituting \textit{klapejśo} with \textit{pukejśo} (which is not so far away semantically), there is a much better alliteration, connecting all three verses with \textit{prosćo – pytejśo – pukejśo} (not unlike \textit{veni – vidi – vici}). Furthermore, the second and the third verse are connected with \textit{pytejśo – spotkaś – pukejśo} through the voiceless stops \textit{p-t – p-tk – p-k.} The substitution again adds to the parody as well, stating that you will \textit{not} find open doors by knocking on them, but that you have to force them open.
 

 
So there is a case to be made that the citation of Mt 7,7 found here is an old parody rather than a translation. The fact that this passage is found in a baptising agenda and therefore in a context where a parody should not appear would require that this parody is much older than the monument and that the parody was not perceived as such by the person who incorporated it into the agenda.\footnote{Perhaps the parody was at that time so old that it had faded away, or the priest who originally adopted them (who might not have had fluent Sorbian) had been “taught” these words by the community. Cf. again the oldest Western Slavonic sentence cited in the chronicle of Thietmar von Merseburg, where the people claim that their corrupted version is what they had been taught by Boso, the first bishop of Merseburg. \citep[27]{Stone2015}} 
 

 
The alternative explanation is much less convincing: It requires the dialect of Zossen to have evolved lexically rather differently to everything else we know about Lower Sorbian of that time; it would not explain the artistic sound structures (or explain them as coindicences), but even then, it would require a conjecture and leaves the part \textit{pšosćo, ga buźo[śo] braś} partly unexplained.
 

\section{Summary}
 
In spite of the assertions made by \citet{Nedo1966}, the analyzed samples of Sorbian traditional folk songs feature interesting artistic means which are, however, significantly different from what we would expect from a German perspective. They seem to be based on various types of rhymes including alliteration, not unlike Germanic \textit{Stabreimdichtung} or Welsh \textit{cynghanedd.} Especially alliterating verse may be the oldest surviving attestation of Sorbian verbal art.
 

 
As shown, original rhymes were lost to corruption during the oral transmission of texts or obscured by later Sorbian sound changes. We therefore have to assume that it will only be possible to recover parts of the alliteration schemes. However, with \textit{Stabreimdichtung} disappearing in Old High German as early as the 9\textsuperscript{th} century, the Sorbian songs would be the oldest (and only) source of poetry of that kind that survived to this day from the former Slavic and now germanised region. As such patterns can also be found in translations of liturgical texts, they for sure go beyond aesthetic purposes. Indeed, apart from their former functions, they help reconstruct texts. Investigating them in detail would require a multidisciplinary philological project since all areas at issue are sparsely documented and have to be further explored. Historical linguistics needs the input from literature, cultural studies, and musical studies as well. For example, on the one hand, historical sound changes help to unearth elements of verbal art; on the other hand, it is possible to date the historical sound changes more exactly because of their effects on alliterations.
 
\section*{Abbreviations}
\begin{tabularx}{.45\textwidth}{lQ}
V & verb\\
 \textsc{cop} & copula\\
 N & noun\\
 A & adjective\\
 \textsc{adv} & adverb\\
 \textsc{con} & conjunction\\
 \textsc{pron} & pronoun\\
 \textsc{prep} & preposition\\
 \textsc{part} & participle\\
 ł & ł-form used for most compound tenses and moods\\
 \textsc{imp} & imperative\\
 \end{tabularx}
\begin{tabularx}{.45\textwidth}{lQ}
 \textsc{sg} & singular\\
 \textsc{du} & dual \\
 \textsc{pl} & plural\\
 \textsc{m} & masculine\\
 \textsc{f} & feminine\\
 \textsc{n} & neuter\\
 \textsc{nom} & nominative\\
 \textsc{gen} & genitive\\
 \textsc{dat} & dative\\
 \textsc{acc} & accusative\\
 %\textsc{ins} & instrumental\\
 %loc & locative\\
 \textsc{inf} & infinitive\\
 \textsc{sup} & supine\\
 \\
\end{tabularx}

\sloppy\printbibliography[heading=subbibliography,notkeyword=this]
\end{document} 
