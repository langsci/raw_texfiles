\documentclass[output=paper,hidelinks,draftmode]{langscibook}
\ChapterDOI{10.5281/zenodo.7446983}
% \usepackage{tipa}
% \usepackage{textcomp}
% \usepackage{float}

\title{Modelling accommodation and dialect convergence formally: Loss of the infinitival prefix \textit{tau} `to' in Brazilian Pomeranian}
\author{Gertjan Postma\affiliation{Meertens Institute Amsterdam}}

%When various dialects enter in intense and prolonged mutual contact in a new sociological setting, they may converge in a process of koineisation. This situation occurs with enhanced intensity in newly-colonised areas, i.e. in so-called language islands, where a conglomerate of mutually intelligible dialects converges towards a new \textit{koinè}.
\abstract{Various pathways with their respective outcomes of multi-dialect interaction have been described in the literature: levelling in the sense of the erasure of linguistic communal differentiation, interdialect formation with compromise forms or fudging, and reallocation of doubles to distinct functions. In this paper I re-evaluate a well-known, but often ignored mechanism and outcome: revert to default settings, the rise of the unmarked, i.e. whenever the result of the change is not a sum or subset of the input forms, but an innovative pattern. Two related models are developed, one for koineisation and one for accommodation, that can serve as an evaluation scheme for a language change. The case study pursued is the loss of the infinitival prefix \textit{tau} `to' in Pomeranian, a West Germanic language, extinct in Europe, but still spoken in isolated communities in Brazil. While the original Pomeranian dialects in Europe had a considerable variation in this particular domain, Pomeranian in Brazil has converged to a remarkably uniform new construction, which was not present in Pomerania in the days of emigration. I show that underlying structures remain constant in all Pomeranian dialects, European as well as Brazilian Pomeranian, but the spellout pattern in Brazil is the cross-linguistic default.}


\IfFileExists{../localcommands.tex}{
 \addbibresource{../localbibliography.bib}
 \usepackage{langsci-optional}
\usepackage{langsci-gb4e}
\usepackage{langsci-lgr}

\usepackage{listings}
\lstset{basicstyle=\ttfamily,tabsize=2,breaklines=true}

%added by author
% \usepackage{tipa}
\usepackage{multirow}
\graphicspath{{figures/}}
\usepackage{langsci-branding}

 
\newcommand{\sent}{\enumsentence}
\newcommand{\sents}{\eenumsentence}
\let\citeasnoun\citet

\renewcommand{\lsCoverTitleFont}[1]{\sffamily\addfontfeatures{Scale=MatchUppercase}\fontsize{44pt}{16mm}\selectfont #1}
  
 %% hyphenation points for line breaks
%% Normally, automatic hyphenation in LaTeX is very good
%% If a word is mis-hyphenated, add it to this file
%%
%% add information to TeX file before \begin{document} with:
%% %% hyphenation points for line breaks
%% Normally, automatic hyphenation in LaTeX is very good
%% If a word is mis-hyphenated, add it to this file
%%
%% add information to TeX file before \begin{document} with:
%% %% hyphenation points for line breaks
%% Normally, automatic hyphenation in LaTeX is very good
%% If a word is mis-hyphenated, add it to this file
%%
%% add information to TeX file before \begin{document} with:
%% \include{localhyphenation}
\hyphenation{
affri-ca-te
affri-ca-tes
an-no-tated
com-ple-ments
com-po-si-tio-na-li-ty
non-com-po-si-tio-na-li-ty
Gon-zá-lez
out-side
Ri-chárd
se-man-tics
STREU-SLE
Tie-de-mann
}
\hyphenation{
affri-ca-te
affri-ca-tes
an-no-tated
com-ple-ments
com-po-si-tio-na-li-ty
non-com-po-si-tio-na-li-ty
Gon-zá-lez
out-side
Ri-chárd
se-man-tics
STREU-SLE
Tie-de-mann
}
\hyphenation{
affri-ca-te
affri-ca-tes
an-no-tated
com-ple-ments
com-po-si-tio-na-li-ty
non-com-po-si-tio-na-li-ty
Gon-zá-lez
out-side
Ri-chárd
se-man-tics
STREU-SLE
Tie-de-mann
}
 \togglepaper[11]
}{}


\shorttitlerunninghead{Accommodation and Convergence}
\begin{document}
\shorttitlerunninghead{Accommodation and Convergence}
\maketitle

\section{Introduction}


Dialectology and sociolinguistics do not only have a value in themselves, they also offer a window to the formal aspects of language and may function as a method to reveal underlying structures of natural language. Language contact is an especially valuable tool for formal research. In language contact, the result transcends the input variants and where the final state is no obvious function (addition, selection, split, superposition, etc.) of the initial state. In this study I report on dialect convergence of a set of mutually intelligible dialects and its outcome. I discuss a grammatical change in a language island in Brazil: the loss of the infinitival prefix \textit{tau} `to' in Pomeranian, a West Germanic language. I will argue that the dialectology and sociolinguistics of this minority language provide evidence for the T-to-C movement in infinitival constructions, as was argued for in \citet{Pesetsky2007}. First I provide a brief overview of the various mechanisms of convergence that have been discussed in the literature, as well as other mechanisms of language change, especially convergence and accommodation. Then, as a background, I give a description of the nature of the complementiser and the infinitival prefix in Pomeranian. In Section \ref{sec:postma:3} I discuss a possible source of the change: the original Pomeranian dialects in Europe had considerable variation in this particular domain. The pattern of this variation is investigated as well as the underlying syntactic pattern. I list two mechanisms of resolving this variation: convergence of the various dialects to a new koine and accommodation to Portuguese. I then repeat the arguments from my \citeyear{Postma2016} study, which show that Portuguese is not the likely source of change. The arguments in my previous publication that lead to the conclusion that accommodation to Portuguese is not likely to have given direction and impetus to the change, but rather dialect-internal convergence within the Pomeranian diasystem, still hold. But these must be balanced by new considerations of occurrence frequency.



\section{Contact-induced language change}\label{sec:postma:2}


While traditional diachronic linguistics has focused on language change by inherent processes, such as (phonological) erosion and inherent instabilities of linguistic cycles (e.g. Jespersen's cycle), modern sociolinguistics has made contact-induced language change a major object of investigation. For instance, the arrival of considerable numbers of immigrants usually changes the dynamics of a community thoroughly and its language with it. Colonisation, e.g. the settlement of various dialect speakers in a foreign country, usually gives rise to new social dynamics, a new society, and a language with new properties. Two extreme cases are noteworthy: the circumstance of huge immigration of mutually unintelligible speakers, outside the immediate realm of a roof language, may initiate a creolisation process: the emergence of a completely new structure, albeit with words of various source languages \citep{bickerton2015roots}. The other end of the spectrum is the circumstance of a linguistically inhomogeneous but mutually intelligible group of immigrants, which by some social factor is isolated from the environment. This creates a so called \textit{language} \textit{island}, where the various source dialects \textit{converge} to a new \textit{koine} \citep{Frings1936, Rosenberg2005}. Finally, there is the more moderate circumstance when an (immigrant) group has moderate contact with the dominant group ``outside'', the superstrate. In such interactions two processes can be observed: the influence of the minority language on the dominant language, usually by the switch of immigrants to the dominant language (substrate effect, Van Coetsem's source-language agentivity), and the influence of the dominant language on the minority language (accommodation, prestige, Van Coetsem's recipient-language agentivity, \citealt{Coetsem1988}). It may be clear that an actual situation never realises one of these processes in pure form. Accommodation goes together with convergence, creolisation is not always clearly separable from convergence.

\subsection{Accommodation}


Accommodation is omnipresent in linguistic interactions. When an American hears a speaker who pronounces /o/ in \textit{socks} lower, i.e. identical to \textit{sacks}, he nevertheless perceives it as /o/ if it is embedded in a broader context \citep[68--70]{Labov2010}. The process is automatic and usually unconscious. This is accommodation \textit{in} \textit{perception}. Accommodation \textit{in} \textit{production} is a speaker's adaptation to a hearer in a specific situation. This can be in lexis when one speaks to young children. It can be changes in phonology if one talks with friends in a bar, etc. It is also possible to accommodate in syntactic structures. When accommodation becomes systematic, and conventionalised, it is a source of language change, for instance if it occurs in a linguistic group in interaction with another linguistic group. 

Though ``accommodation'' is used in the literature in various senses, I will reserve it in this paper to the situation where a group of speakers changes its language in order to become acceptable or intelligible to another group, usually the dominant, more prestigious group, i.e. it is \textit{asymmetric}. It is also possible to accommodate the superstrate language to some minority group, i.e. to a substrate. For instance, if Turkish immigrants in the Netherlands use periphrastic constructions to realise the V2 constraint in Dutch more often, it might be seen as an accommodation strategy to retain the basic SOV structure in accommodation to the more rigid SOV order in their Turkish mother tongue \citep{Craats2009}.

\subsection{Koineisation}\label{sec:postma:2.2}


While accommodation is conceptually an asymmetric process – one language accommodates to another – \textit{koineisation} is, at least conceptually, a process whereby language variants influence each other. The process is, conceptually at least, symmetric \citep{GumperzWilson1971}. The mechanism involved is \textit{convergence}. Koineisation may give rise to a \textit{Sprachbund}, but it most typically occurs in \textit{Sprachinseln}, language islands: settlements with colonist of various dialect regions. Such (German) language islands were studied ``as relics from the past'' \citep[222]{Rosenberg2005} from the 19\textsuperscript{th} century onwards, though the explicit \textit{mechanisms} of the changes only received attention in the 20\textsuperscript{th} century. Rosenberg notices that the language islands are not homogeneous, neither linguistically nor socially. ``(...) they were often inhabited by settlers of different origins, i.e. by speakers of different dialects'' \citep[223]{Rosenberg2005}. Below I mention four mechanisms by which the process of koineisation can come about: levelling, interdialect formation, reallocation, and revert to the default settings. The first and the last mechanism can be considered simplification (L1-L2 language contact), the other two mechanisms are complexification in the sense of \citet{Trudgill2011}: they typically occur with 2L1 language contact (bilingualism).


\subsubsection{Levelling}\label{sec:postma:2.2.1}


Most researchers mention levelling as the major process of new dialect formation in closed immigrant groups. It is the process of eliminating prominent stereotypable features of the input dialects \citep{Dillard1972}. Notice that all stereotypable features and locally specific features are typically the first to be eliminated \citep{Thelander1980, Hinskens1996}. The process is symmetrical, despite the fact that the result is eliminating a certain feature from one of the two dialects in interaction. In many cases, it leads to reduction of inflectional paradigms and morphology in general. 

\subsubsection{Interdialect formation}\label{sec:postma:2.2.2}


Interdialect formation is the rise of compromise forms. In the case of two dialects, this can be by simple optionality of two forms, by neutralisation of the feature that defines the two forms, or by superposing the two forms. \citet{Chambers1998} mention the case of [ʌ] and [ʊ] in \textit{strut} in East Anglia, which merge around the isogloss to [ɤ]. A clear example of the superposing process, mentioned in \citet[366]{Hinskens1996} is the emergence of superheavy syllables on the borderline of Limburgian dialects. The eastern dialect has [x] drop in \textit{nacht} [naxt] `night' under compensatory lengthening of the vowel ([na:t]). The dialect west of the isogloss has [naxt]. On the borderline, new forms such as [na:xt], i.e. with both [x] and the long vowel can be observed. In the latter case, the superposed form is clearly a transitional phenomenon, under the assumption that superheavy syllables are marked. A more complex syntactic example is given in \citet{Postma2014}, where on the borderline of two Limburgian dialects with two types of Verb-second (the German type with uniform C-V2, and the Dutch type with C-V2 and T-V2), complex V-AGR-T forms emerge, such as \textit{klöp-s-de} `knock.\textsc{2sg}{}-\textsc{ed'}. This can be explained if the interdialect complies with both types of V2: the V complex moving to C skipping T, where AGR is the so-called \textsc{comp}-inflection. This mechanism clearly works on underlying rules, rather than on surface forms. Once again, these superposed forms are marked and often transitional (cf. \citealt{Cornips2006}).


\subsubsection{Reallocation}\label{sec:postma:2.2.3}


Just like interdialect formation, reallocation gives rise to complexification. Reallocation takes two or more inputs from source dialects and redistributes these over two or three sub-contexts. As an illustration, in Jundiai, an Italian immigrant city in Brazil, people use both the Portuguese word \textit{pavor} [pa'vor] `fear' and the Italian word \textit{paura} [pa'ura] `fear' in their \textit{Caipira} version of Portuguese, but limit \textit{paura} for the meaning `strong fear'. \citet{Britain1997} and \citet{Taeldeman1989} provide more complex phonological cases where two alternates from source dialects distribute in a contact dialect. The distribution is \textit{rule-governed}. These are, of course, the more interesting cases linguistically, because they potentially shed light on underlying linguistic processes.


 \subsubsection{Revert to the default}\label{revert}


The final mechanism that I would like to mention, is revert to the default. If two dialects, one with a marked setting, the other with an unmarked setting in some feature, come into contact, the result tends to lean towards the unmarked setting. For instance, if there are two features involved, say, F\textsubscript{1} and F\textsubscript{2}, and if we call + the marked and ∅ the unmarked value, contact of a dialect with [+F\textsubscript{1}, ∅F\textsubscript{2}] and a dialect with [∅F\textsubscript{1}, +F\textsubscript{2}] might give rise to the new variant [∅F\textsubscript{1}, ∅F\textsubscript{2}]. Dependent on the nature and abstractness of F\textsubscript{1} and F\textsubscript{2}, the contact variant might have a rather different appearance without obvious connection to the properties of the source dialects. In \citet{Postma2004, Postma2011}, I give a case of two variants of late Middle Dutch (MD), that lack a reflexive pronoun, i.e. these dialects circumvent the Binding Theory, albeit for different (marked) mechanisms. Cross-linguistically, the reflexivity of pronouns is dependent of feature underspecification, typically number, but also person, or case \citep{Reuland1995}, while referential pronouns are (fully) specified. By a marked parameter setting, however, the referential pronoun, MD \textit{hem} `him' had number underspecification in the Southern Dutch dialects (meaning either `him' or `them') and could be used as a reflexive, while it had \textsc{acc}/\textsc{oblique} underspecification in the Northern dialects (cf. \citealt{Hoekstra1994} for modern Frisian). Both settings are marked settings, cf. \tabref{tab:postma:1}.\footnote{It is slightly more complicated. In the case of oblique, it is the feature inventory that is marked, not the setting.}


\begin{table}
\small
	\begin{tabularx}{\textwidth}{l ll CC CC}
		\lsptoprule
 & Variety & Pattern & \multicolumn{2}{c}{Feature Setting} & \multicolumn{2}{c}{Markedness}\\\cmidrule{4-7}
		 &  &  & F\textsubscript{1}& F\textsubscript{2} & F1 & F2\\\midrule
		a. & Southern MD & NP\textsubscript{i} \dots { }hem\textsubscript{i} & yes & no & + & 0\\
		b. & Northern MD & NP\textsubscript{i} \dots { }hem\textsubscript{i} & no & yes & 0 & +\\
		c. & \textit{koine} & *NP\textsubscript{i} \dots { }hem\textsubscript{i}
		~/~NP\textsubscript{i} \dots { }sick\textsubscript{i} & no & no & 0 & 0\\
		\lspbottomrule
		\multicolumn{7}{l}{\begin{footnotesize} ~F\textsubscript{1} = Number neutralisation in pronouns; F\textsubscript{2} = \textsc{acc/obl} neutralisation in pronouns. \end{footnotesize}}
\end{tabularx}
\caption{Feature analysis of a koineisation process in Dutch reflexive constructions.}
\label{tab:postma:1}
\end{table}


What we observe then is that both marked strategies are lost in the contact-induced variant. The contact dialect then comes in need of a special, underspecified, reflexive pronoun. It then \textit{actively} borrows it from neighboring German dialects, first \textit{sick}, later \textit{sich}. It was in need of the borrowed form, rather than accommodating to it. The result with a reflexive is a result of contact between two variants without reflexive. It may be clear that the grammatical system is a creative force which transcends the dialectal input. We might call this tendency towards the default variant in contact ``micro-creolisation''. This shows that convergence to the default can not only be the result in cases of a set of unrelated source languages without mutually intelligibility, but also in closely related mutually intelligible dialects. 

In the next sections, I present a case of contact of many minimally distinct Pomeranian dialects, which merge in a language island in Brazil. I will investigate if revert to the default is active in this case.


\section{European Pomeranian (EP)}\label{sec:postma:3}

\subsection{Background}

Pomeranian is the dialect (or set of dialects) of Coastal Germanic roughly between the Oder river and the Vistula river, an area which is called Hinterpommern. Until 1945 it was first part of Prussia, later Germany, but lays in present-day Poland. The dialect of Mecklenburg-Vorpommern in present-day Germany is rather different (henceforth Mecklenburgian) and should be discussed separately from Hinterpommersch, henceforth simply Pomeranian. The map in \figref{fig:postma:1} below, slightly adapted from \citet[128]{Brockhaus2012}, gives an impression of the Pomeranian area, indicated with ``Ostpommersch''.

%%please move the includegraphics inside the {figure} environment
%%\includegraphics[width=\textwidth]{figures/PostmaModelingAccommodationFormattedLSP-img001.emf}


\begin{figure}
% \includegraphics[width=\textwidth]{figures/Postma-2.pdf}
\includegraphics[width=\textwidth]{figures/Postma-1.png}
  \caption{
Coastal Germanic dialectal areas in the first decades of the 20\textsuperscript{th} century (after \citealt{Brockhaus2012}). 
}\label{fig:postma:1}
\end{figure}

Pomerania was Germanised in a geographically scattered way during the so-called \textit{Ostsiedlung}, the ``going east'' of settlers, land developers, and merchants coming from Flandres, Holland, and Frisia and later from the core Saxon areas. The newly emerged variant of Low Saxon, Pomeranian, has been in close contact to High-German and Slavonic, especially Slovincian/Kashubian.\footnote{Slavonic influence on Pomeranian can be ignored from the 13\textsuperscript{th} century onward, except for Slovincian. In the 20\textsuperscript{th} century, Slovincians were, like the Pomeranians, predominantly Lutheran, and were expelled with them from the new Polish areas in 1945.} The origin from the North Sea area might explain the consistent Ingwaeonisms in the language, characteristics of the North Sea Germanic area, such as loss of /n/ before spirants, development of a \textit{-s} plural in nouns. The linguistic roof of High German through religion and education explains the many German loans, e.g. in the ordinals (\textit{fünft} instead of the expected \textit{fi:wd} {}`fifth'), in kinship terms (\textit{grosmuter} instead of the expected \textit{groutmuder} `grandmother', etc.). Virtually all Pomeranians in Europe were Lutherans.\footnote{Data for the entire Pommern Province in the year 1932: Lutherans (90.7 \%), other Protestants (1.3\%), Catholics (6.7\%), Jews (0.5\%). For the region of emigration (see the map in \figref{fig:postma:2}), the ratio of Lutherans ranges from 97-98.9\%. Cf. \citet{GLFP1932}.}

A distinguishing feature of the Pomeranian vis à vis Mecklenburgian in the west and Low Prusian in the east is the existence of two infinitival forms: an infinitive in \textit{{}-a} ([ə] or [ɐ]), and one in \textit{{}-en} ([ən] or [ṇ], \citealt[295]{Wrede1895}).\footnote{Neither Vor-Pommersch (to the west) nor Low Prusian (to the east) participates in this characteristic feature.} Two types of infinitives are further encountered in Frisian and North Frisian \citep[4--5]{Hoekstra1997}.\footnote{Alemannic dialects also have two infinitival forms, one in \textit{{}-a/e} and one in \textit{{}-i(n)t} \citep{BayerBrandner2004}. The syntactic distribution is rather different from the \textit{{}-ə/ɐ} vs -\textit{en} infinitive in Coastal Germanic. See also \citet{Höhle2006}.} In Pomeranian, the infinitive in \textit{{}-a}, which we call infinitive-1 (\textsc{inf1}), is used in clauses under modals, under causatives (\textit{låta} `let', \textit{daua} `do'), verbs of motion (\textit{gåa} `go'), and control predicates, as exemplified by the Wenker-sentence\footnote{The Wenker-sentences are a set of 40 sentences that Georg Wenker used in a questionnaire for dialect research in 1880 in 40,000 locations in Germany. The sentences have also been elicitated in The Netherlands, Belgium, Luxemburg, Austria, and Switzerland.} 16b in example \REF{ex:postma:1}. The example is taken from location 20, the village of Schloenwitz (present-day Słonowice) in the municipality Schivelbein (see map). This schwa-infinitive (\textsc{inf1})\footnote{Please see the Abbreviations section.} is used without complementiser and without infinitival prefix.


\ea\label{ex:postma:1}
\langinfo{European Pomeranian}{}{19\textsuperscript{th} century (Schloenwitz)} \\
\gll Du must eista no {'a} inn wass-a\\
     you must first still a bit grow.\textsc{inf1}\\
\glt `you must first still grow a bit'\z

The infinitive in \textit{{}-en}, which we will call infinitive-2 (\textsc{inf2}), is used in embedded infinitivals with a leading complementiser, as exemplified in the Wenker-sentence 16a in \REF{ex:postma:2}, again taken from the village of Schloenwitz.

\ea\label{ex:postma:2}

\langinfo{European Pomeranian}{}{19\textsuperscript{th} century (Schloenwitz)} \\
\gll Du bust nog ni groot naug um {'n} Flasch Wiin ut-tau-drink-en\\
     you are yet not big enough \textsc{comp} a bottle wine \textsc{prt}{}-to-drink.\textsc{inf2}\\
\glt {}`you are not big enough to drink out a bottle of wine' \z

\hspace*{-2pt}In this paper I study the changes in infinitival syntax of such rationale clauses.

\section{Variation in the Infinitival Syntax of European Pomeranian}

Rationale clauses in European Pomeranian can be studied using the Wenker sentences,\footnote{Cf. \citet{Demske2011}. The Margburg digitalisation project, led by Jürg Fleischer, made Wenker sentence 16 available through a grid of 1250 datapoints (of the 40,000 data points).} that were elicitated around 1880.\footnote{The Wenker sentences are not available in digital format, but scans of the questionnaires can be inspected at \url{www.regionalsprache.de}.} Using the online database, I checked more than 300 locations in coastal Pomerania i.e. in municipalities Schivelbein, Regenwalde, Belgard, Colberg-Cörlin, Cöslin, Greifenberg, and Schlawe, as the emigration into Espirito Santo was mainly fed from this coastal area (cf. \citealt[167]{Granzow2009}). The various municipalities are indicated in \figref{fig:postma:2}.


\begin{figure}
\includegraphics[width=.7\textwidth]{figures/Postma-2.png}
  \caption{
  Municipalities (Kreise) covered in the search on infinitival constructions.\\ 
% {\tiny Based on: \url{https://commons.wikimedia.org/wiki/File:Pomerania_counties_1939_map.svg}, CC-BY-4.0 \url{https://commons.wikimedia.org/wiki/User:Hellerick}}
}
\label{fig:postma:2}
\end{figure}

It turns out that there is some variation in the realisation of this construction in European Pomeranian with respect to the infinitival prefix \textit{tau} {}`to'. Apart from \REF{ex:postma:3a} where, as in Standard German, Dutch and Frisian, both \textit{um} and \textit{tau} are realised, (e.g. \textit{um} and \textit{zu} in German, \textit{om} and \textit{te} in Dutch/Frisian), we observe two alternative patterns in Pomeranian. In one of these, the `\textit{to'}{}-prefix \textit{tau} remains unrealised \REF{ex:postma:3b}, and in another variant, \textit{um,} the `for' complementiser, remains unrealised \REF{ex:postma:3c}.\footnote{These are not necessarily different dialects, as optionality might be involved.}
\newpage

\ea\label{ex:postma:3}\langinfo{European Pomeranian}{}{1880, Schloenwitz, Lankow, and Schlenzig, resp.} \\
\ea\label{ex:postma:3a} du bust nog nich grot naug üm an Flasch Wiin ut-tau-drinken
\ex\label{ex:postma:3b} du bust nog nich grot naug üm an Flasch Wiin ut-∅-drinken
\ex\label{ex:postma:3c} \gll du büst no ni groot naug ∅ ain Flasch Winn ut-tau-drinken\\
     you are yet not big enough \textsc{comp} a bottle wine \textsc{prt}{}-to-drink.\textsc{inf2}\\
\glt `you are not big enough to drink out a bottle of wine' \z\z

The fourth conceivable option with both \textit{üm} `for' and \textit{tau} `to' unrealised, is not found. I summarise the patterns in \tabref{tab:postma:2} for the entire coastal area. From now on I will gloss \textit{üm} as `for' and \textit{tau} as `to'.


\begin{table}
	\begin{tabularx}{.8\textwidth}{lCCrr}
		\lsptoprule
& Pattern & Occurrence & Frequency & N\\ \midrule
a.& { }\textsc{for} ... \textsc{to} & general & 83\% & 258\\
b. & ~~∅ ... \textsc{to} & rare & 11\% & 34\\
c. & ~~\textsc{for} ... ∅ & rare & 6\% & 20\\
d. & *∅ ... ∅ & absent & 0\% & 0\\\lspbottomrule
	\end{tabularx}
\caption{Occurrences of infinitive constructions in European Pomeranian}
\label{tab:postma:2}
\end{table}


\largerpage[2]
The complementiser \textit{üm} `for' can remain empty only if the verbal prefix \textit{tau} `to' is not empty; conversely, the verbal prefix \textit{tau} can be empty only if the complementiser \textit{um} is not. This is cast in a cross table in \tabref{tab:postma:3} on the basis of the Wenker sentences of 312 locations in Pomerania.\footnote{The six places where the Wenker sentence 16 has been translated by a finite embedded clause (\textit{du} \textit{bist} \textit{noch} \textit{nicht} \textit{groß} \textit{genug} \textit{daß} \textit{du} \textit{eine} \textit{Flasche} \textit{Wein} \textit{austrinken} \textit{kannst}) were ignored. They occur scattered over the area and it does not seem a structural effect.}


\begin{table}
	\begin{tabularx}{.4\textwidth}{XrY}
		\lsptoprule
		   & $+$\textsc{for}~ & {}$-$\textsc{for}\\
		   \midrule
		  {}$+$\textsc{to} & 258 & 34\\
		  {}$-$\textsc{to} & 20 & 0\\
		\lspbottomrule
	\end{tabularx}
\caption{Cross table of occurrences of infinitival constructions in European Pomeranian}
\label{tab:postma:3}
\end{table}


This shows a structural absence of the [∅ ... ∅] pattern with $p$-value of 0.09 in Fisher's test. To be more precise: The hypothesis H\textsubscript{0} that the absence of the [∅ ... ∅] pattern is a mere result of the (low) probability (\textsc{for}=∅) and the (low) probability (\textsc{to}=∅), is rejected with a p-value of 0.09.

I therefore conclude that both positions T and C must ``see'' each other at some level of representation \citep[55]{Bennis1984}. This suggests that the \textit{tau}{}-marker in Pomeranian, at least in these rationale clauses, concerns the syntactic type of the infinitival marker as described in \citet{Brandner2006}. Following standard assumptions on these markers, I assume that \textsc{for} (\textit{um,} \textit{om,} \textit{üm} ...) sits in C (\citealt[133]{Koster1982}, \citealt[108]{VandenWyngaerd1987}) while \textsc{to} (\textit{zu}, \textit{te}, \textit{to}, \textit{tau}, ...) sits in T \citep{Evers1990, Sabel1996}.\footnote{\citet{Bennis1987} argues that so-called prepositional adjunct clauses have P in the C position.}\textsuperscript{,}\footnote{Arguments have been raised against treating ZU in German as a functional head (I or T), see e.g. \citet[273--274]{haider2010syntax}. \citet{Brandner2006} argues that one should distinguish morphological ZU from syntactic ZU. If this is correct, dialects with only syntactic ZU cannot be excluded. There is no evidence in Pomeranian that a morphological \textit{tau} should be distinguished. On the contrary, most of the evidence forwarded in \citet{Postma2014} only follows under the assumption of an exclusively syntactic \textit{tau} in Pomeranian.} Since we are dealing with constructions that have a lexicalised complementiser in continental Germanic, I assume that there is T-to-C movement at some level of representation and that the complementiser C must be lexical at that level. I, therefore, make the assumptions in \REF{ex:postma:4}, taken from \citet[106, 116]{Hoekstra1997} developed for (Fering) Frisian. The lexicalisation requirement of C already holds in West Germanic for main and embedded \textit{finite} clauses and \REF{ex:postma:4b} is a natural extension to non-finite sentential constructions.

\ea\label{ex:postma:4}
\ea\label{ex:postma:4a} C ...... T => \textsubscript{C°}[C+T] ...... \sout{T}
\ex\label{ex:postma:4b} {[C°]} {is} overt in all types of clauses in Pomeranian\footnote{It would be attractive to extend this to West Germanic infinitivals without \textit{um} in general, as in \citet{Bayer1984}. I only defend the claim for Pomeranian here. \citet{Kayne1999} argues that all Romance complementisers are complex: a W head that have attracted the infinitival prefix \textit{di}.} \z\z

Notice that T-to-C movement in infinitival constructions is independently motivated from a theoretical perspective, cf. \citet{Pesetsky2007}, who derive the T-movement chain from basic syntactic principles. In the next section I will provide evidence that these assumptions also hold in Brazilian Pomeranian.

\section{Brazilian Pomeranian (BP)}
\largerpage[2]
\subsection{Background}


While Pomeranian is not used anymore in cohesive communities in Europe since 1945, it is still in full use in various parts of Brazil, with many children not learning Portuguese at all until schooling at age six or so. These communities derive from immigration as early as 1850, and have been rather isolated until recently. In this article I will use the variant spoken in the state of Espirito Santo, in the municipality of Santa Maria de Jetibá and surroundings.\footnote{Santa Maria de Jetibá, Caramuru, Garrafão, Melgaço, and Domingo Martins.} I simply call it Brazilian Pomeranian, though there might be differences with the variants in the South (in the states of Santa Catarina and Rio Grande do Sul) or in the Amazone region (Rondônia), which left the Northern parts of ES in the 1970s. This community is rather big\footnote{\citeauthor{tressmann1998bilinguismo} (\citeyear{tressmann1998bilinguismo}) estimates the population to be 300,000.}. Virtually all Brazilian Pomeranians are Lutherans \citep{Droogers2008}. Although Pomeranian was never used in the liturgy until quite recently (first in High German, since 1942 in Portuguese), the religion is an important factor of social cohesion that safeguards the language in Brazil \citep{Schaffel2014}. Within the various groups of Germanic immigrants, the Pomeranians have become the dominant group, both economically, religiously, and sociologically. For instance, virtually all Dutch immigrants that arrived at the same time and who were Calvinists, have converted to Lutheranism and speak Pomeranian now. 

Recently, a collection of Brazilian Pomeranian tales was published under the title \textit{Upm} \textit{Land} (\citealt{Tressmann2006a}, henceforth \textit{UmL}), as well as a dictionary of Brazilian Pomeranian \citep{Tressmann2006b}. The data used in this paper are mainly from this corpus of tales, provided by a variety of authors and registered by Anivaldo Kuhn and Ismael Tressmann. The orthography that is used is the one developed in \citet{Tressmann2006b}. Apart from this corpus\footnote{Cf. \citet{Postma2014} for the details.} I completed my data with two interviews in March 2013 (Elizana Schaffel) and September 2013 (Elizana Schaffel and Tereza Gröner). 

\subsection{The infinitival syntax of Brazilian Pomeranian}
\largerpage[2]
As said above, the distinction between the two infinitives has been fully retained in ES.\footnote{Under influence of High German (in older speakers) or Hunsrückisch (in some areas), deviations from the Pomeranian pattern occur: overgeneralised \textit{n}-forms and overgeneralised \textit{e}-forms, respectively. These are not present in the corpus used in this study.} The complementiser in infinitive-2 constructions, however, is never realised as \textit{üm}, but as \textit{taum} ([tɑ\textsuperscript{u}m]/[tɑm]). Interestingly, the verbal prefix is always null, indicated with ∅. So while the \textit{tau} prefix position is systematically zero, the complementiser position has changed from \textit{üm} to \textit{taum}. I give some examples in \REF{ex:postma:5}, rationale clauses, taken from UmL (78, 114, 115).


\ea\label{ex:postma:5} Brazilian Pomeranian
\ea\label{ex:postma:5a} \gll Dai lüür sin arm un häwa kair gild [taum sich air huus ∅ buugen].\\
     The people are poor and have no money \textsc{for.to} \textsc{refl} a house ∅ build.\textsc{inf2}\\
\glt {}`The people are poor and have no money to build themselves a house'
\ex\label{ex:postma:5b} \gll Dai blaumasuuger is air seir hübsch tijr. Hai hät aina langa snåbel [taum dai saft uuta blauma ∅ suugen].\\
     The flowersucker is a very elegant animal. He has a long beak \textsc{for.to} the juice out-the flowers ∅ suck.\textsc{inf2}\\
\glt `The hummingbird is a very elegant animal. It has a long beak to suck the juice out of the flowers'
\ex\label{ex:postma:5c} \gll Dai ima maga seir geirn dai maluulabüsch eer blauma [taum sich eera hoinig ∅ måken].\\
     the bees like.\textsc{pl} very much the {malula bushes} their flowers \textsc{for.to} \textsc{refl} their honey ∅ make.\textsc{inf2}\\
\glt `The bees like the malula-bushes's flowers very much to make honey'\z\z

In contrast to the situation in European Pomeranian, there is virtually no variation left in Brazilian Pomeranian. There is no variation in the complementiser position, which is always \textit{taum}. Only in 3 of the 127 cases (2\%) in the corpus does the original \textit{tau} show up, but it is not adjacent to the verb, i.e. we may assume that it has always moved up to the C-position.\footnote{For further reference, I give these three cases. Only in one case (i) is \textit{tau} a true complementiser. In the other cases (ii--iii), \textit{tau} assigns a deviant dative case to the embedded object under surface adjacency, similar to English \textit{For me to go...}. Apparently, the intervening subject PRO does not block case assignment to the object in BP.\\
\vspace{-\baselineskip}
\ea\gll[Tau-∅ dai köirn afstampen] gewt dat aina stampküül. \\
\textsc{to} the.\textsc{acc} grains crush.\textsc{inf2} is there a pounder \\
\glt `There is a pounder to crush the grains'\\
\ex\gll Suurdaig dörwt ni feigla [tau dem daig anmåken ] < taum de daig anmåken \\
sourdough may not fail \textsc{to} the.\textsc{dat} dough produce.\textsc{inf2} {} < \textsc{to} the.\textsc{acc} dough produce \\
\glt `Sourdough may not be absent upon making dough'\\
\ex\gll[Tau dem rijs weglegen] mud man em mita slusa forwåra < taum de rijs weglegen \\
 \textsc{to} the.\textsc{dat} rice store.\textsc{inf2} must one him with {the peel} stock.\textsc{inf1} < \textsc{to} the.\textsc{acc} rice store.\textsc{inf2} \\ 
\glt `In order to store the rice, one needs to store it with the chaff'. \\
\z
These are not performance errors, as informants accept both variants. I leave these sentences for further research.} There is no variability in the lexicalisation of the lower infinitival prefix position, either: it is without exception without spellout. Hence we may conclude that Brazilian adjunct infinitivals have obligatorily lexicalisation of C and no low spellout of T in these infinitival constructions.

\hspace*{-2.2pt}One way of understanding this innovation is to postulate that Brazilian Pomeranian has reanalyzed \textit{taum}, which was originally a P+CASE complex [tau+m], into [tau+um], i.e. as a C+T-complex of \textit{tau} and \textit{um}. It is then an overt realisation of the rule in \REF{ex:postma:4} that I inferred from European Pomeranian dialect set. So, the surface \textit{variability} of lexicalizing C and T in European Pomeranian has been replaced by a surface \textit{rigidity} in Brazilian Pomeranian. The underlying formal rigidity of spelling out the C-T chain in European Pomeranian has been retained and recaptured by an overt marking of the head of the C-T chain.

\ea
\ea\label{ex:postma:6a} in [C\textsubscript{i} ....... T\textsubscript{i} ..], the \textit{chain} must be lexicalised in EP
\ex\label{ex:postma:6b} in [C+T]\textsubscript{i}.....∅\textsubscript{i}, the C+T complex must be lexicalised in BP\z\z

The scheme in \REF{ex:postma:6a} shows that the variability in spellout in EP has been replaced by one spellout form, under retention of the more abstract underlying syntax.

The \textit{taum}+\textsc{inf2} construction had a precursor in European Pomeranian, illustrated in \REF{ex:postma:7}. It is the nominalised use of the -\textit{en} form, illustrated by Wenker sentence 20, given for Schloenwitz.

\ea\label{ex:postma:7}
\langinfo{European Pomeranian}{}{19\textsuperscript{th} century (Schloenwitz)} \\
\gll Hai deer so, as hann-e in taum dörsch-en bistellt \\
     He did so, {as if} {had he} him for-the.\textsc{dat} threshing invited\\
\glt `...as if he had invited him for the threshing'\z

In this construction, \textit{tau} is a preposition enriched with a dative marker (\textit{taum} < \textit{tau}+\textit{(de)m}). This construction allows modification but it must be done adjectivally, by PPs, or under incorporation: no direct object arguments between \textit{taum} and the nominalised verb are possible, because the deverbal noun cannot assign case.\footnote{Incorporated objects are possible even when no accusative is available. Incorporated objects do not need Accusative Case cross-linguistically \citep{Baker1988}. The dimension of case assignment is often ignored in the literature (cf. for instance \citealt{Demske2011}).} The infinitival construction in \textit{{}-en} has been reinterpreted in Brazilian Pomeranian as a verbal construction\footnote{Cf. \citet{Haspelmath1989} for the grammaticalisation pathway of infinitives.} in which the verb in the -\textit{en} infinitive does assign Accusative case, e.g. \textit{air} \textit{huus} `a house' in \REF{ex:postma:5a}, \textit{dai} \textit{saft} `the juice' in \REF{ex:postma:5b} and \textit{eera} \textit{hoinig} `their honey' in \REF{ex:postma:5c}. Since syntactic categories that \textit{receive} case cannot \textit{assign} case, cf. the Case Resistance Principle \citep{Stowell1981} or the Unlike Category Constraint \citep{Hoekstra1984}, the case assigning preposition \textit{tau(m)} was obligatorily reanalyzed into a non case assigning tense head.

The question is now what caused this change, which is minimal with respect to the surface string but with considerable structural consequences. Why does only the complementiser position receive lexicalisation in Brazilian Pomeranian? Is it an accident that the superstrate language Portuguese does not have an infinitival prefix and systematically lexicalises C in this context (\textit{para} `for')? 

\section{Other contact varieties}


In the previous section, I showed that the BP verbal \textit{taum} construction is a Brazilian innovation. It does not occur in the Wenker material of the Pomeranian areas in Europe. But I also showed that the C-T link also had deep structural parallels in the dialect continuum of European Pomeranian. Therefore, it does not come as a surprise that we encounter similar constructions in other West Germanic dialects. In this section I review some of these. 

\subsection{Middle English}\label{ME}

The oldest West Germanic counterpart to the \textit{taum} construction of Brazilian Pomeranian is found in Middle English. We can compare this construction with the Middle English split infinitive (where the verbal prefix \textit{to} has undergone T-to-C in forming a complex \textit{for-to} complementiser \REF{ex:postma:8}, taken from \citet[par. 982]{Visser1963}; see also \citet{Mustanoia1960}).


\ea \label{ex:postma:8}
\langinfo{Middle English}{}{Pecock, Repr. 219}\\
\ea A nurish or a modir is not bounde forto alwey and for euere ∅ fede her children.\\
`a nurse or a mother is not bound to always and for ever feed her children.'
\ex \gll He eoden (...) forto fully that folk and godes lawe {} techen\\
     he went (…) \textsc{for.to} fuly that people and god's law ∅ teach.\textsc{inf}\\
\glt {}`he went in order to fully teach God's law to that people'
\ex\gll if it schulde plese god forto bi miracle {} make a fier and a watir togidere\\
     if it should please god \textsc{for.to} by miracle ∅ make a fire and a water together\\
\glt {}`if it would please God to combine fire and water'\z\z

If we identify Eng. \textit{for} with Pom. \textit{um} and Eng. \textit{to} with \textit{tau}, the parallel is striking. Admittedly, it is not certain that \textit{for} actually resides in C. It might sit in a lower position \citep{Gelderen1998}. Nevertheless, the processes share the raising of the infinitival prefix away from the verb and clustering with a higher functional morpheme. The question is: ``What triggered this change? English went through a process of dramatic changes in the Middle English period with respect to word order and morphology. But it is also tempting to tie it to external influence. Did these changes emerge under French influence from the south? Was it accommodation to a dominant language like French without infinitival prefix?

\subsection{Pella (Wisconsin)}

The \textit{taum}-construction is also found in a Low-German recording from Pella (Wisconsin), available from the \textit{Databank für Gesprochenes Deutsch}. Though without metadata documentation, the recording seems Pomeranian to my ear, and my transcription of the same Wenker sentence 16 in \REF{ex:postma:3} and \REF{ex:postma:5} in this variety is presented in \REF{ex:postma:9}.\footnote{IDS database, \url{http://dgd.ids-mannheim.de}.\\
\vspace{-\baselineskip}
\ea File: MV-{}-\_E\_00138\_SE\_01\_A\_01\_DF\_01.WAV, time 00:02:01.0. \\ 
\ex du büs no ni groot nauch \textbf{to} ... ain flash wiin ut-\textbf{∅}{}-drinke: \\File: MV-{}-\_E\_00136\_SE\_01\_A\_01\_DF\_01.WAV, time 00:01: 54.0.\\
\z
\citet[175]{Louden2009} reports the more traditional [um ... ∅]-pattern in Hamburg (Marathon county (Wisconsin)): 
\\ \vspace{-\baselineskip} \ea Du bist noh nit groot genaug, \textbf{um} et Glas Wien ut-\textbf{∅}{}-drinken.
\z}

\ea\label{ex:postma:9}\langinfo{Pella Pomeranian}{}{DGD-IDS, MV-E138}\\
\gll Du büst no nit groot nauch to {'n} bottel ut-∅-drinken. \\
       you are yet not big enough \textsc{to} a bottle \textsc{prt}-∅-drink.\textsc{inf2}\\
       \glt {}\z
Notice that C is lexicalised with simple \textit{tau} rather than \textit{taum.} This is evidence for the movement of T to C. These data might feed the idea that the split infinitive originates from Europe. However, as T-to-C is, by hypothesis, a formal option of UG that can arise at various moments, we should not exclude the possibility that the split \textit{tau} + V-\textit{en} construction has arisen as a consequence of language contact between Germanic with a prefix (European Pomeranian) and a language without such prefix (Modern English).\footnote{Modern English lost \textit{to} as a prefix, as \textit{to} can be separated from the verb by adverbs (``split infinitives''). \\
\vspace{-\baselineskip}
\ea My mother asked me \textbf{to} quickly \textbf{∅}{}-go to the market.\z
It is unclear to which functional projection it has raised. It has not raised as far as C.}

\subsection{Altschlage}

In the Pomeranian area that was checked (the regions Schivelbein, Regenwald, Belgard, Colberg-Cörlin, Cöslin, Greifenberg, Schlawe), I found one case with a raising of the \textit{tau} prefix, though without deletion of the lower copy, given in \REF{ex:postma:10}.


\ea\label{ex:postma:10} \langinfo{Altschlage}{}{W 00148}\\
\gll Du büst no nie grot nouch to ne Flasch Wiin ut tau drinken. \\
     you are yet not big enough \textsc{for} a bottle wine empty \textsc{to} drink.\textsc{inf2}\\
     \glt {}\z
We might see this as a precursor of a high spellout of \textit{tau} in the chain. Altschlage is present-day Sława (Świdwin). It used to be a Wendish settlement. Slavic languages lack an infinitival prefix, and it lexicalises the complementiser. In this case, accommodation to a language with infinitival prefix is not very plausible as the prefix is retained. Only the chain as such is lexicalised, which is a universal structure. It maximally shows a kind of agreement between the C position and the T position. It might be used as evidence for the abstract movement of T to C, but not for accommodation.

\subsection{Alemannic}

The \textit{taum} construction also occurs in the Wenker material in the Alemannic dialects of Switzerland and Austria (Vorarlberg) \citep{Seiler2005}, as illustrated in \REF{ex:postma:11a} and \REF{ex:postma:11b}, respectively.


\ea \langinfo{Alemannic}{}{Fläsch (Graubünden) and Krumbach, resp}\\
\ea\label{ex:postma:11a} du bisch noh z Klii zum a Fläscha Wi us-∅-trinka\\
\ex\label{ex:postma:11b} du binscht no nit gros gnug, zum a flöscha wing us-∅-trinken\\
\z\z
Direct influence of Alemannic on Brazilian Pomeranian is improbable. Though there is a Swiss community in the Pomeranian area in Espirito Santo, its earliest immigration to the Santa Leopoldina area consisted of 30 Catholic families \citep[155]{Franceschetto2014}. The Pomeranian community and the \textit{Suiça} community were segregated by religion: Lutheran versus Catholic.\footnote{In 1860, there was a big Catholic church in the center of the area, and a small Protestant chapel at the edge, which were in conflict to a point that the governor of the state had to intervene \citep[139]{Tschudi1860}.} As to the origin in Europe, it must be noticed that Alemannic is in close contact with Rhaeto-Romance and vice versa. For instance the V2 properties in Rhaeto-Romance are probably due to language contact with Germanic. If so, the \textit{taum} construction could be a sign of language contact in reverse direction. Notice that this contact has happened before the split in religion during the reformation. There is evidence that there is T-to-C as early as in Middle Alemannic of around 1470.\footnote{Examples from MHG bible of {\textasciitilde}1470 in an Alemanic/Schwabian dialect: 
% \vspace{-\baselineskip}
\ea vnd er gabe in [zewerden offen]\\ 
\glt `and he gave him to become open' \\
\ex Du gibst nit deinen heiligen [ze sehen die zerbrochenkeit]\\ 
\glt `you give not your saints to see the broken-ness'
\z 
The infinitival prefix \textit{ze} `to', being a bound morpheme, pied-piped the verb, creating VO contexts.} Such contacts with Romance in the Vorarlberg are also reported, as it has been germanised from the 9\textsuperscript{th}–16\textsuperscript{th} century \citep[4]{Klausmann1995}.\footnote{I thank one of the reviewers for drawing my attention to this.}

\subsection{Schwabian}

A similar construction is reported in Schwabian (cf. \citealt[23]{Hoekstra1997}), who analyzes the floating `to' as head movement to C or to Asp. I give three examples in \REF{ex:postma:12}.


\ea\label{ex:postma:12} \langinfo{Schwabian}{}{\citealt{Müller1996}} \\
\ea\label{ex:postma:12a} \gll dass'r extra hoimkomma isch [zom schnell des Päckle auf-∅-macha]\\
     {that he} expressly {home come} is \textsc{for.to} quickly the parcel {open make}\\
\glt `that he came specially home to open the parcel quickly'

\ex\label{ex:postma:12b} \gll dass'r mir a weng Zoet brauchat [zom des neie Haus en dr Garteschtross en Tiebenga zom/z' baua]\\
     that-\textsc{prt} we a little time needed [\textsc{for.to} the new house in the Gartenstraße in Tübingen \textsc{for.to/to} build\\
\glt `that we needed little time to build a new house in Garden Street in Tübingen'
\ex\label{ex:postma:12c} \gll dass'r [kurz den Zättl zom aus zom/z' fülla] ogfanga hat.\\
     {that he} quickly the form \textsc{for.to} out \textsc{for.to}/\textsc{to} fill begun has\\
\glt `that he began to quickly fill out the form.'\z\z

Schwabian is not in direct contact with Romance, though it, of course, participates in the wider Alemannic linguistic space which is in contact with Italian and French. The lack of direct contact, however, makes the accommodation scenario improbable. 

\subsection{Tyrolese}

The taum-construction can be found in the Wenker sentence 16 in some villages in Tyrol, as given in \REF{ex:postma:13}.


\ea\label{ex:postma:13} \langinfo{Tyrolese}{}{Reith b. Brixlegg}\\
 Du bischt no nit grousz gnuag zum a flosch win aus-∅-drink'n\\
`you are not big enough yet to drink a bottle of wine'\z

Direct influence of Tyrolese on Brazilian Pomeranian is improbable. There is a community \textit{Tirol} in Espirito Santo not far from the Pomeranian area, but the inhabitants are separated by religion (Catholic versus the Lutheran Pomeranians). Segregation on the basis of religion has always been strong \citep{Schabus2009}, even until the present day. As to the origin of the construction in Europe, the construction might have emerged in Tyrol in Europe by contact with Romance, in this case Rhaeto-Romance.

\subsection{Twentieth century European Pomeranian}

There is also the possibility that the \textit{taum} construction is native from Pomerania. In \citet[69]{Stritzel1974}, a Pomeranian grammar from the 1930's, a similar construction for the village of Grossendorf\footnote{Present-day Wielka Wieś (Pomeranian Voivodeship).} is reported, given in \REF{ex:postma:14a}. Furthermore, there is at least one example (an idiomatic expression) in a Pomeranian dictionary (cf. \ref{ex:postma:14b}, taken from \citealt{Laude1995}).


\ea \langinfo{20\textsuperscript{th} c European Pomeranian}{}{Großendorf and Kowalk, resp.}\\
\ea\label{ex:postma:14a} \gll {dɑn} is {də} {\`{s}\={e}\textsuperscript{i}nstə} {t\={i}d} [tum {drɑxən} {st\={i}jən} {} l\={o}\textsuperscript{u}tən] \\
     then is the nicest time \textsc{for.to} drake rise ∅ let.\textsc{inf2}\\
\glt {}`then it is the best time to let climb the dragon/kite'
\ex\label{ex:postma:14b} \gll dat is jå tam up d' boim {} kleppre \\
     that is \textsc{prt} \textsc{for.to} upto the tree ∅ climb.\textsc{inf}\\
\glt {}`that is to become desperate.'\footnote{Notice the form in \textit{{}-e}, where one would expect \textsc{inf2}. Kowalk (Kowalki, no Wenker location) patterns with the villages Zeblin (Cybulino, W00453), Groß Leistikow (Lestkowo, W50506), Barfussdorf (Zolwia Bloc,W51121), Köpik (Kopice, W50482), Drammin (Dramino, W50731), Liepnitz (Lipnica, W00374) in two Pomeranian properties: they have no \textsc{inf2} form and display strong adjectival endings. Kowalk's neighboring village Groß-Tychow (Tychowo, W00346) displays \textit{n}-infinitive and weak endings.}\footnote{A reviewer draws attention to the fact that this expression also exists in Standard German: ``Das ist ja zum auf die Bäume klettern''. The Pomeranian example may be a translation of the Standard German saying.}\z\z

This might be a sign of an older presence of the \textit{taum} construction, but it may also be a later, parallel development. It is certainly not evidence that the construction was already in Pomerania in the days of the Pomeranian emigration to Brazil.

\subsection{Flemish in Brazil}\label{flemish}

An extremely interesting case is the mixed speech of the Dutch immigrants from the Flemish part of the province of Zeeland (Zeeuws-Flemish). These settlers were Calvinist but converted to Lutheranism and are now part of the Pomeranian community. In multilingual speakers (Flemish, Pomeranian, Portuguese), one can observe constructions, like the ones in \REF{ex:postma:15}, where the infinitival prefix \textit{te} has been attached to the complementiser \textit{om}. These are obvious constructions under a strong Pomeranian influence (calques), as might be derived from the typical \textit{do}-support, and from the participial without prefix \textit{ge}{}- in \textit{kommen} `come' instead of the expected form \textit{gekommen}. The fact that Flemish \textit{te} has overly moved to \textit{om} in C is a welcome confirmation of my analysis of Pomeranian \textit{taum} as \textit{um}+\textit{tau}. What makes this construction special is that the order of the lexical ingredients in \textit{om-te} is reversed with respect to the \textit{taum} construction, where the prefix is initial.


\newpage
\ea\label{ex:postma:15} (Zeeuws) Flemish in Brazil\\
\ea \gll dat es dan vier dagen om-te naar Santa Leopoldina {} kommen en dan vier dagen weer om-te terug {} kommen\\
     that is then four days \textsc{for.to} near Santa Leopoldina ∅ come.\textsc{inf} and then four days again \textsc{for.to} back ∅ come.\textsc{inf}\\
\glt `It is then four days to go to SL and four days to come back'

\ex \gll as jinne krank worden deed, dan was gien auto om-te die weg {} bringen.\\
     if one ill get did, then was no car \textsc{for.to} those away ∅ bring\\
\glt `if somebody got ill, there was no car to bring them away'

\ex \gll om-te dan goeid {} bikieken waar ons folk kommen es\\
     \textsc{for.to} then well ∅ look.\textsc{inf} where our folks come is\\
\glt `to see well in what situation our people has arrived' \z\z

These structures are, therefore, more similar to the Middle English constructions discussed in Section \ref{ME}. The more conservative linear order \textit{om-te} is also what I expect, as Flemish lacks a precursor like Pomeranian \textit{taum} + N, illustrated in \REF{ex:postma:7} above. The Standard Dutch counterpart [\textit{ten} + N] is a high-register structure, and absent in Dutch dialects. Without doubt, language contact with Pomeranian is responsible for the emergence of this overt T-to-C movement. Important to note is that the overt movement of T-to-C can be observed in the Garrafão area (with a high density Pomeranian speakers), not in the Holandinha area (with a low number of Pomeranians), cf. \REF{ex:postma:16}.

\ea\label{ex:postma:16}
 Dutch and Pomeranian Varieties in ES
\ea om frunne te maken \hfill Flemish in Holandinha 
\ex om-te frenne ∅ maken \hfill Flemish in Garrafão 
\ex taum farijn ∅ måken \hfill Brazilian Pomeranian\\
\glt `to make manioc flour' \z\z

Apparently, the presence of Portuguese is not a sufficient trigger for the change I am discussing in this paper, while the presence of Pomeranian did cause such a change in this variant of Flemish.\footnote{Their Flemish is a rather uncertain heritage Flemish, while their Pomeranian is robust, just as the Pomeranian of the Pomeranians. These people are Pomeranians with an additional heritage Flemish.} Hence, accommodation is not a sufficiently explanatory factor. We rather must think in terms of variation: while the internal variation of these Flemish variants was not rich enough to cause a change to overt T-to-C, the Flemish-Pomeranian melting pot was sufficiently rich for dialect convergence towards both the \textit{taum}{}-construction and the \textit{om-te} construction. 

\section{Dialect convergence or contact-induced accommodation?}

In the previous sections, I discussed a range of West Germanic varieties that have lost the infinitival prefix and realised it, or rather its functional head, higher up in the syntactic hierarchy. We are now in the position to evaluate the various scenarios that might have led to the innovation, shared by Middle English and modern Alemannic. These dialects behave parallel to Pomeranian in Brazil in that they lexicalise the T-chain high. The null hypothesis is that all these parallel cases receive a parallel explanation. There is the accommodation scenario, which hypothesises that the \textit{taum} construction emerged in Brazilian Pomeranian in contact with Portuguese, which lacks the infinitival prefix, just like French, Slavic, and modern English. Alternatively, we have the koineisation scenario in a newly created melting pot community. This scenario fundamentally reduces the number of variants furnished by the source dialects. This explanation has the variability in the source dialects as a fundamental ingredient. It is obviously an advantage of the latter scenario that it puts the variability discussed in Section \ref{sec:postma:3} on a fundamental footing. Long-term, structural accommodation is only possible upon intensive contact. If we now see to what extent there has been actual contact in all these cases, the balance is not completely positive, as can be seen in \tabref{tab:postma:4}.


\begin{table}
\small
	\begin{tabularx}{\textwidth}{lp{1.8cm}Ql}
		\lsptoprule
		Language & Loss of\newline prefixal \textsc{to} & Contact with prefix-less language & Sufficient contact\\
		\midrule
		Middle English & + & Anglo-Norman & yes\\
		Pella(Wisconsin) & + & English & yes\\
		Altschlage & – & Slavic & doubtful\\
		Alemannic & + & Rhaeto-Romance/ Franco-Provençal/ French & yes\\
		Schwabian & +/– & no & no\\
		Tyrolese & + & Ladin & \mbox{yes (Western part)}\\
		20\textsuperscript{th} c EPom & +/– & Slavic & unknown\\
		BPom & + & Portuguese & yes\\
		Flemish in Hollandinha & – & Portuguese & yes\\
		Flemish in Garraf\~{a}o & + & BrPomeranian & yes\\
		\lspbottomrule
	\end{tabularx}
	
\caption{Various Germanic contact varieties with complex \textsc{for-to} complementisers.}
\label{tab:postma:4}
\end{table}

%\tablefirsthead{}
%
%\tabletail{}
%\tablelasttail{}
%\begin{tabularx}{\textwidth}{XXXXX}
%\lsptoprule
%\hhline{~----} & Language & { Loss of}
%
% prefixal TO & Contact with 
%
%prefix-less language & Sufficient
%
% contact\\
%\hhline{~----} & Middle English & + & Anglo-Norman & yes\\
%\hhline{~----} & Pella(Wisconsin) & + & English & yes\\
%\hhline{~----} & Altschlage & – & Slavic & doubtful\\
%\hhline{~----} & Alemannic & + & Rhaeto-Romance/
%
%FrancoProvençal/French & yes\\
%\hhline{~----} & Schwabian & +/– & no & no\\
%\hhline{~----} & Tyrolese & + & Ladin & yes, in Western part\\
%\hhline{~----} & 20\textsuperscript{th} c EPom & +/– & Slavic & unknown\\
%\hhline{~----} & BPom & + & Portuguese & yes\\
%\hhline{~----} & Flemish1 in Br & – & Portuguese & yes\\
%\hhline{~----} & Flemish2 in Br & + & BrPomeranian & yes\\
%\hhline{~----}
%\lspbottomrule
%\end{tabularx}


Let us discuss the table briefly. Language contact between Middle English and Anglo-Norman is uncontroversial in both directions \citep{Mustanoia1960, Dalton1996, Ingham2012, Rothwell2001, Steiner2010}.\footnote{The influence of French on Middle English in the domain of the lexicon is better studied than for syntax and morphology. For some curious reason, the influence of (Anglo)French on Middle English is not as well studied as the influence of Middle-English on (Anglo)French. It is often downplayed as in \citet[306ff]{Thomason1988}, but see \citet{Ingham2009} for noteworthy remarks on this issue.} In the case of Altschlage, there is no positive evidence of the contact with Slavic, but it cannot be excluded, as it was a Wendic settlement. This might have triggered a C+T complex, as the high \textit{to} in \REF{ex:postma:10} indicates. However, the lower copy \textit{tau} is not silent. If language contact was involved, it apparently did not occur on surface level. We leave this case open. For Pella (Wisconsin), language contact may have been present beyond doubt, but it is not clear if there has been a Pomeranian cohesive community. There are too few speakers to evaluate this single fact\footnote{The Wenker-sentences given in \citet[175]{Louden2009} make a distinction between two infinitives 1 and 2, as in EP and BP. The infinitival \textit{tau} is silent and the complementiser is \textit{um}, as in Lankow \REF{ex:postma:4b} above.\\(i)Du bist noh nit groot genaug, \textbf{um} et Glas Wien ut-\textbf{∅}{}-drinken}, but contact with English has been strong. For Schwabian, direct contact with a prefix-less language is absent, though it can have happened indirectly through Swiss sister dialects. For Brazilian Pomeranian, contact with Portuguese is present in modern times, as has been shown by \citet[177, graph 4]{Schaffel2014}, though 50\% of the older present-day speakers are still monolingual. If accommodation were the causing factor, we would expect that the \textit{taum}{}-construction would be less used by older speakers. There is no evidence of this kind.\footnote{In 4 interviews by Anivaldo Kuhn in 2003 of a {\textasciitilde}75 years old Pomeranian, the \textit{taum}{}-construction already occurs abundantly: 30 times (on {\textasciitilde}4000 words) of which 15 with an actual lexical split [\textit{taum} xxx V-\textit{en}] (of which 5 bare nouns/actjectives might have been incorporated into the verb). The interviews are in \citet[507--556]{Seibel2010}.} Taking all these doubts into account, I conclude that there is too little evidence to either support or to reject the accommodation hypothesis.

This brings us to evaluating the koineisation scenario with its four mechanisms as discussed in Section \ref{sec:postma:2}. The options in Sections \ref{sec:postma:2.2.1}--\ref{sec:postma:2.2.3} only fit with some artificiality on the facts under scrutiny. One could argue that instead of lexicalizing a chain optionally in a scattered way, as the European Pomeranian dialects do, Brazilian Pomeranian opts for lexicalizing T higher up jointly with C (\textit{taum}=\textit{tau}+\textit{um}). This strategy can be seen as a very particular variant of levelling (i.e. loss of most source variants): in this case loss of \textit{all} variants. However, Brazilian Pomeranian did not just lose the three input variants of the scheme in \REF{ex:postma:5}, it also created a new one \REF{ex:postma:6b} on the basis of the underlying syntactic skeleton. So, levelling is an insufficient mechanism to capture what happened. One could also argue that it is a very particular kind of interdialect formation: the emergence of new forms that are intermediate of the input dialects. To what extent lexicalizing two positions higher up in the syntactic hierarchy instead of scattered lexicalisation of a coindexed chain is a case of ``intermediate'', is of course open to debate. Finally, one could argue that it must be interpreted as a very particular version of \textit{fudging}: the combination or superposition of two ingredients taken from distinct dialects: lexicalisation of the higher member of the C-T chain (dialects with \textit{um}) and silence of the lower member of the C-T chain (dialects with \textit{tau}{}-drop) is reanalyzed as \textit{movement}: the lower copy is spelled out high as C+T: \textit{taum} emerges. This is what comes closer to what has happened. But probably the most apt interpretation of the facts is that it should be explained as \textit{revert} \textit{to} \textit{the} \textit{default} \textit{setting}. Most of the world's languages lack an infinitival prefix comparable to \textit{tau/to/zu}. Absence of it seems to be the default.\footnote{The claim that the infinitive is without prefix does not only hold for rationale clauses, but for infinitival clauses in general. In the perspective of \textit{revert} \textit{to} \textit{the} \textit{default}, this does not come as a surprise, cf. (i):\\ \vspace{-\baselineskip}
\ea\gll ik fersuik ais [aira nå hus gåa] \\ 
I try \textsc{prt} early to house go.\textsc{inf1}\\ \glt `I finally try to go home early'
\z 
In most of the cases, the German/Dutch construction corresponds to a bare infinitive1 or a finite clause in BP.} And Brazilian Pomeranian complies with it. Moreover, the majority of the world's languages do lexicalise complementisers in purpose infinitivals, and Brazilian Pomeranian patterns with it as well.\footnote{It is often difficult to separate prepositions and complementisers in this context. For a discusssion and tests, cf. \citet{Bennis1987}.} Finally, as \citet{Pesetsky2007} have argued on formal grounds, there is always an overt or covert T-to-C movement in infinitivals. And this is precisely what \textit{taum} is, the lexicalisation of T+C. So, on all points does Brazilian Pomeranian pattern with the default setting, while this default setting was not present in the source variants. So, theoretically, the dialect convergence scenario seems to have strong cards. Is there then any empirical evidence that can be decisive? In the next section I make a feature analysis of the constructions and design two models on its basis.

\section{Modelling accommodation and dialect convergence formally}

In this section I make a formal implementation of the two scenarios by which Brazilian Pomeranian infinitival construction [taum ... ∅] can be explained: accommodation to Portuguese and/or dialect convergence to the default settings. I will take the mechanism of \textit{revert} \textit{to} \textit{the} \textit{default}, discussed in Section \ref{revert}, as starting point.

\subsection{Modelling dialect convergence}\label{conv}

As we have seen in \tabref{tab:postma:2}, European Pomeranian shows at least 3 variants of this infinitival construction, while one is structurally absent. These variants were the input for the newly created \textit{lingua} \textit{franca} in Brazil. In the first columns of \tabref{tab:postma:5}, I characterise these 3 + 1 variants in terms of their spellout patterns of functional morphemes in the secound column. Logically, there are 2\textsuperscript{3}=8 possible patterns in total. For completeness, I have added the remaining possibilities below the separator.\footnote{The extra patterns include those of Altschlage (cf.\REF{ex:postma:10}), the BP pattern (i) in note 18 and Pella Pomeranian (cf. \REF{ex:postma:9}), and the Schwabian variant mentioned by \citet{Müller1996} in \REF{ex:postma:12b}}

%\footnote{The zero frequency value in column 3, row-d is not predicted by the markedness settings (+, 0, +) themselves. It is caused by an additional and independent requirement that \textsc{comp} is lexicalised \citep{Hoekstra1997}, by the \textsc{for} chain and/or \textsc{to} chain, see the discussion in section 4. This is a major argument for the \textsc{for-to} interaction.}


\begin{table}
	\begin{tabularx}{\textwidth} {l C C r c}
		\lsptoprule
      & Surface Pattern & Underlying Pattern & {Frequency} (\%) & Variant \\\midrule
a.& um ... tau & um-\sout{tau} ... tau & 83 & EP \\
b.& um ... ∅ & um-\sout{tau} ... \sout{tau} & 6 & EP \\
c.&∅ ... tau & \sout{um}-\sout{tau} ... tau & 11 & EP \\
d.&∅ ... ∅ & \sout{um}-\sout{tau} ... \sout{tau} & 0 & (EP) \\
\midrule
e.&taum ... ∅ & um-tau ... \sout{tau} & 98 & BP \\
f.& tau ... tau & \sout{um}{}-tau ... tau & {}--- & Alt-Slage \\
g.& tau ... ∅ & \sout{um}-tau ... \sout{tau} & {}2 & BP/Pella \\
h.&taum ... tau & um-tau ... tau & {}--- & Schwab \\
		\lspbottomrule
\end{tabularx}

\caption{Chain analysis of infinitival constructions}
\label{tab:postma:5}
\end{table}


The classification in terms of its surface appearance does not display the underlying grammatical factors, however. There are three grammatical features involved, which all concern the spellout of \textit{chains}. First, there is a ±lexicalisation of the \textsc{for}-chain, which is a singleton chain with or without spellout. Secondly, there is a ±lexicalisation of the \textsc{to}-chain, which is a binary chain (``movement''). It does or does not have a chain spellout. Thirdly, the \textsc{to}-chain, which is a movement chain, can have a high spellout (overt movement) or a low spellout (covert movement). This is ruled by the \textsc{delete}{}-process of chain reduction, as described in \citet{Nunes1995}. This is captured by ±low-\textsc{delete}. In columns 3-5 of \tabref{tab:postma:6}, I describe the parameter settings of these input variants with values yes/no. Finally, these features must be projected in a consistent way on markedness of the settings: marked (+) or default (0). Let us assume that lexicalizing a chain is the default (applied to \textsc{for} and \textsc{to}-chain equally). Let us furthermore assume that overt movement is the default i.e. \textsc{delete} of the lower copy is the default. I indicate the corresponding markedness values of the input in gray-shade. These are the EP input varieties upon entering Brazil. The BP parameter output is in the fifth row (dashed) in (row e).

\begin{table}
\fittable{
\begin{tabular}{clccccccc}
\lsptoprule
\multicolumn{1}{l}{} & \multicolumn{2}{l}{Infinitival construction} & \multicolumn{3}{c}{Parameter Settings} & \multicolumn{3}{l}{Markedness} \\\midrule
\multicolumn{1}{l}{} & Pattern & \multicolumn{1}{l}{Variety} & \begin{tabular}[c]{@{}c@{}}P1=\textsc{for}\\ chain\end{tabular} & \begin{tabular}[c]{@{}c@{}}P2=\textsc{to}\\ chain\end{tabular} & \begin{tabular}[c]{@{}c@{}}P3=low \\ \textsc{delete}\end{tabular} & P1 & P2 & P3 \\
\midrule
a. & um-\sout{tau}... tau & EP & yes & yes & no & \cellcolor[HTML]{A6A6A6}\textbf{0} & \cellcolor[HTML]{A6A6A6}\textbf{0} & \cellcolor[HTML]{A6A6A6}\textbf{+} \\
b. & um-\sout{tau}... \sout{tau} & EP & yes & no & yes & \cellcolor[HTML]{A6A6A6}\textbf{0} & \cellcolor[HTML]{A6A6A6}\textbf{+} & \cellcolor[HTML]{A6A6A6}\textbf{0} \\
c. & \sout{um}-\sout{tau}... tau & EP & no & yes & no & \cellcolor[HTML]{A6A6A6}\textbf{+} & \cellcolor[HTML]{A6A6A6}\textbf{0} & \cellcolor[HTML]{A6A6A6}\textbf{+} \\
d. & \sout{um}-\sout{tau} ... \sout{tau} & – & no & no & yes & + & + & 0 \\
\midrule
e. & um-tau ... \sout{tau} & BP & yes & yes & yes & \cellcolor[HTML]{E1E1E1}0 & \cellcolor[HTML]{E1E1E1}0 & \cellcolor[HTML]{E1E1E1}0 \\
f. & \sout{um}-tau ... tau & Alt-Sl & no & yes & no & + & 0 & + \\
g. & \sout{um}-tau ... \sout{tau} & BP/Pella & no & yes & yes & + & 0 & 0 \\
h. & um-tau ... tau & Schwab & yes & yes & no & 0 & 0 & + \\
\lspbottomrule
\end{tabular}}

\caption{Convergence Model - Feature analysis and markedness}
\label{tab:postma:6}
\end{table}


\largerpage
Let me now show the convergence mechanism in progress. As to the \textsc{for}-chain lexicalisation (shaded P1 column), the input dialect set contains two dialect types with a default setting (row a and b) and one dialect type with a marked setting (row c). Upon interaction, the outcome in (row e) is the default value. As to the lexicalisation of the \textsc{to}-chain (shaded P2 column), the input set contains two dialect types with default setting (row a and c) and one dialect type with marked setting (row b). The outcome in (row c) opts for the default setting. Finally, as to the chain spellout (low of high) in the last column, we observe interaction of one dialect type with default setting (row b) and two dialect types with marked settings (row a and c). Once again, the outcome is the default setting. In sum, for the three relevant parameters, revert to the default setting describes the dominant outcome in Brazil adequately. This default setting of the three features as well as the marked settings were already present in one of the input dialects. Therefore, the process can be described as a purely Pomeranian-internal effect: dialect mixing can produce Brazilian Pomeranian under revert to the default if present in the linguistic input. This might be taken as evidence that the Pomeranian \textit{lingua} \textit{franca} in Brazil has resorted to the default setting in all three relevant parameters upon language contact with conflicting input in the three parameters.

\largerpage
One might wonder why the majority choice of European Pomeranian [\textsc{for} …. \textsc{to}] did not impose itself in Brazil. Under the assumption that the figure of 83\% in \tabref{tab:postma:5} is valid for the immigrants as well, it might come as a surprise that the emigrants followed a completely different path, especially considering the fact that the European [\textsc{for} ... \textsc{to}]-variant is identical to the Standard German variant, a prestige variety that was taught in the parochial schools to some of the community members. There are three points to consider here. In the first place, the interaction (convergence) of two closely related dialects takes place on parameter level, not on surface level. This is precisely the point I want to make: the default setting approach can produce something new, which cannot be explained by considerations of dialect dominance. So the outcome in BP converging to the new [\textsc{for-to} …. ∅] is a strong argument in favor of the parameter approach. Secondly, neither the dominant EP variety nor HG with [\textsc{for ... to}] realise the default setting according to the analysis in \tabref{tab:postma:5}. Hence, even these varieties might decline if they were sufficiently shuffled upon social changes. Third, in the case of, say, two or three caretakers with slightly different dialects, we have the situation of 2L1 or 3L1 and the interaction takes place according to the scheme in \tabref{tab:postma:5}, not on the level of societal statistics. That being said, I do think that societal statistics are relevant: They play a role in the case of accommodation, as we will see in the next section.

\subsection{Modelling accommodation}\label{accom}



By a simple modification of the model of Section \ref{conv}, I can turn it into a model of accommodation, as we will see in an instance. Since we use universal claims of what is default and what is marked, the only locus for a different implementation is the parameter describing covert and overt movement. I captured this dimension in Section \ref{conv} by checking if the lower link of the chain was deleted or not, which was default or not default, respectively. However, it can also be checked if the upper link is deleted or not, of course with the reverse markedness assignments. Let us, therefore, consider a parameter P4 that describes the lexicalisation of the higher copy. For this P4, upper link deletion is marked (instead of lower-link deletion being the default). The core cases of overt and covert movement then still project on the same markedness as they did in the convergence model of the previous section. Only in the case of double spellout or non-spellout does the new parameter give distinct results. The two models are compared in \tabref{tab:postma:7}.


\begin{table}[h]
\fittable{
	\begin{tabular}{lccccc}
\lsptoprule
&  & \multicolumn{2}{c}{ Convergende Model using P3 } & \multicolumn{2}{c}{Accomodation Model using P4}\\\cmidrule{2-6}
& Pattern & low \textsc{delete} & markedness & high \textsc{delete} & markedness\\\cmidrule{2-6}
a. & tau ...tau & no & + & no & 0\\
b. & tau ...\sout{tau} & yes & 0 & no & 0\\
c. & \sout{tau} ...tau & no & + & yes & +\\
d. & \sout{tau} ...\sout{tau} & yes & 0 & yes & +\\
\lspbottomrule
	\end{tabular}
	}
\caption{Chain reduction: low/high \textsc{delete} as ruling parameters and their respective markedness.}
P3 is low \textsc{delete}; P4 is high \textsc{delete}.
\label{tab:postma:7}
\end{table}





With the Model-2 implementation, I arrive at the evaluation \tabref{tab:postma:8}. To see how it works, let us inspect Row-a and Row-b in \tabref{tab:postma:8} with respect to P4 (the features P1 and P2 remain unchanged). Row-a has [um ... tau], which is, as to the \textit{tau}-string: [\sout{tau} ... tau], which is the case of \tabref{tab:postma:6}c with markedness value +. The next case in Row-b is [um … ∅], which is, as to the \textit{tau}-string, [\sout{tau} ... \sout{tau}], which is the case of \tabref{tab:postma:6}d with markedness +, etc. Only the P4 column differs from the Convergence Model of \tabref{tab:postma:5}. Once again, the Brazilian Pomeranian [\textit{taum} ... ∅]-pattern realises the default setting (000), which BP now shares with the Schwabian [\textit{taum ... tau}]-pattern. This model has two absolute default settings: the BP [\textit{taum} ... ∅] in \tabref{tab:postma:8}e and the Schwabian [\textit{taum ... tau}] in \tabref{tab:postma:8}h.



\begin{table}
	\fittable{
\begin{tabular}{clccccccc}
\lsptoprule
\multicolumn{1}{l}{} & \multicolumn{2}{l}{Infinitival construction} & \multicolumn{3}{c}{Parameter Settings} & \multicolumn{3}{l}{Markedness} \\\midrule
\multicolumn{1}{l}{} & Pattern & \multicolumn{1}{l}{Variety} & \begin{tabular}[c]{@{}c@{}}P1=\textsc{for}\\ chain\end{tabular} & \begin{tabular}[c]{@{}c@{}}P2=\textsc{to}\\ chain\end{tabular} & \begin{tabular}[c]{@{}c@{}}P4=high \\ \textsc{de}lete\end{tabular} & P1 & P2 & P4 \\
\midrule
a. & um-\sout{tau}... tau & EP & yes & yes & yes & \cellcolor[HTML]{A6A6A6}\textbf{0} & \cellcolor[HTML]{A6A6A6}\textbf{0} & \cellcolor[HTML]{A6A6A6}\textbf{+} \\
b. & um-\sout{tau}... \sout{tau} & EP & yes & no & yes & \cellcolor[HTML]{A6A6A6}\textbf{0} & \cellcolor[HTML]{A6A6A6}\textbf{+} & \cellcolor[HTML]{A6A6A6}\textbf{+} \\
c. & \sout{um}-\sout{tau}... tau & EP & no & yes & yes & \cellcolor[HTML]{A6A6A6}\textbf{+} & \cellcolor[HTML]{A6A6A6}\textbf{0} & \cellcolor[HTML]{A6A6A6}\textbf{+} \\
d. & \sout{um}-\sout{tau} ... \sout{tau} & – & no & no & yes & + & + & + \\
\midrule
e. & um-tau ... \sout{tau} & BP & yes & yes & no & \cellcolor[HTML]{E1E1E1}0 & \cellcolor[HTML]{E1E1E1}0 & \cellcolor[HTML]{E1E1E1}0 \\
f. & \sout{um}-tau ... tau & Alt-Sl & no & yes & no & + & 0 & 0 \\
g. & \sout{um}-tau ... \sout{tau} & BP/Pella & no & yes & no & + & 0 & 0 \\
h. & um-tau ... tau & Schwab & yes & yes & no & 0 & 0 & 0 \\
\lspbottomrule
\end{tabular}}

\caption{Accommodation Model - Feature analysis and markedness}
\label{tab:postma:8}
\end{table}


The most important consequence is that the four source dialects, \tabref{tab:postma:8}a-d, are homogenous in P4 (with a marked setting), while all high-contact varieties in Brazil, Pella(Wisconsin), and Alt-Schlawe, are homogenous with an unmarked setting. In this model, the flip in the P4-value cannot be produced by internal dialect convergence (there is no variation in the P4 parameter) and must be due to an external trigger of the P4-flip. If one can prove that the Portuguese pattern [\textit{para} ... ∅] does not realise the case of \tabref{tab:postma:8}b, but either \tabref{tab:postma:8}e or \tabref{tab:postma:8}g, then the flip in P4 might have been caused by language contact and accommodation to Portuguese. Let us assume that there is such evidence.

The question is then if we can find independent evidence to choose between the two models in \tabref{tab:postma:6} and \tabref{tab:postma:8}, i.e. we must choose between the features P3 and P4. I will now show that frequency values of the dialects provides us with such independent evidence. To see how, one should realise that it is plausible that a higher level of markedness corresponds to a lower occurrence of the variant and vice versa. So let us define the total markedness, µ, of a language variant as the sum of its markedness values. In \tabref{tab:postma:9} I have represented the Dialect Convergence Model (Model 1) with P1, P2, P3 and the Accommodation Model (Model 2) with the features P1, P2, P4. In the columns headed by µ, I added the respective sums of the marked settings.

\begin{table}
\fittable{
\begin{tabular}{rlccccccccccc} 
\lsptoprule
& \multicolumn{2}{c}{\textbf{Infinitival}} & \textbf{} & \multicolumn{4}{c}{\textbf{Convergence}} & \textbf{} & \multicolumn{4}{c}{\textbf{Accommodation}} \\ & \textit{\begin{tabular}[c]{@{}l@{}}Surface \\ Pattern\end{tabular}} & \textit{\begin{tabular}[c]{@{}c@{}}Freq.\\ in \%\end{tabular}} & & P1 & P2 & P3 & µ & & P1 & P2 & P4 & µ \\
\midrule
a. & um ... tau & 83 & \textbf{} & \cellcolor[HTML]{A6A6A6}\textbf{0} & \cellcolor[HTML]{A6A6A6}\textbf{0} & \cellcolor[HTML]{A6A6A6}\textbf{+} & \cellcolor[HTML]{A6A6A6}\textbf{1} & \textbf{} & \cellcolor[HTML]{A6A6A6}\textbf{0} & \cellcolor[HTML]{A6A6A6}\textbf{0} & \cellcolor[HTML]{A6A6A6}\textbf{+} & \cellcolor[HTML]{A6A6A6}\textbf{1} \\
b. & um ... ∅ & 6 & \textbf{} & \cellcolor[HTML]{A6A6A6}\textbf{0} & \cellcolor[HTML]{A6A6A6}\textbf{+} & \cellcolor[HTML]{A6A6A6}\textbf{0} & \cellcolor[HTML]{A6A6A6}\textbf{1} & \textbf{} & \cellcolor[HTML]{A6A6A6}\textbf{0} & \cellcolor[HTML]{A6A6A6}\textbf{+} & \cellcolor[HTML]{A6A6A6}\textbf{+} & \cellcolor[HTML]{A6A6A6}\textbf{2} \\
c. & ∅ ... tau & 11 & \textbf{} & \cellcolor[HTML]{A6A6A6}\textbf{+} & \cellcolor[HTML]{A6A6A6}\textbf{0} & \cellcolor[HTML]{A6A6A6}\textbf{+} & \cellcolor[HTML]{A6A6A6}\textbf{2} & \textbf{} & \cellcolor[HTML]{A6A6A6}\textbf{+} & \cellcolor[HTML]{A6A6A6}\textbf{0} & \cellcolor[HTML]{A6A6A6}\textbf{+} & \cellcolor[HTML]{A6A6A6}\textbf{2} \\
d. & ∅ ... ∅ & 0 & & + & + & 0 & 2 & & + & + & + & 3 \\
\midrule
e. & taum ... ∅ & 100 &  & \cellcolor[HTML]{E1E1E1}0 & \cellcolor[HTML]{E1E1E1}0 & \cellcolor[HTML]{E1E1E1}0 & \cellcolor[HTML]{E1E1E1}0 &  & \cellcolor[HTML]{E1E1E1}0 & \cellcolor[HTML]{E1E1E1}0 & \cellcolor[HTML]{E1E1E1}0 & \cellcolor[HTML]{E1E1E1}0 \\
f. & tau ... tau & 0.3 & & + & 0 & + & 2 & & + & 0 & 0 & 1 \\
g. & tau ... ∅ & - & & + & 0 & 0 & 1 & & + & 0 & 0 & 1 \\
h. & taum ... tau & - & & 0 & 0 & + & 1 & & \cellcolor[HTML]{E1E1E1}\cellcolor[HTML]{E1E1E1}0 & \cellcolor[HTML]{E1E1E1}0 & \cellcolor[HTML]{E1E1E1}0 & \cellcolor[HTML]{E1E1E1}0 \\
\lspbottomrule
\end{tabular}}
P1--4 are the features involved (see the text); µ is the total markedness
\caption{Comparison of the Convergence Model and the Accommodation Model}
\label{tab:postma:9}
\end{table}



\begin{figure}
\includegraphics[width=\textwidth]{figures/PostmaMarkedness.pdf}
  \caption{
Markedness graphs belonging to the Convergence Model with occurrence rates. 
}\label{fig:postma:3}\end{figure}


In order to evaluate the two models with more ease, I displayed the values of the total markedness µ and the occurrence rates of the varieties into the \textit{markedness} \textit{graphs} under \figref{fig:postma:3} and \figref{fig:postma:4}. These graphs have the total markedness µ on the vertical axis. The horizonatal axis is the time axis with before and after the language contac, convergence in \figref{fig:postma:3} and accomodation in \figref{fig:postma:4}. In both graphs we observe a local minimum before and after the interaction. Moreover, the local minimum before the interaction is higher than the local minimum after the interaction. So, what happens in both models is a decrease in markedness. However, the models differ in what feature(s) cause(s) this decrease. In the model in \figref{fig:postma:3}, all three features are involved and choose the value of the lowest markedness. Hence, this can be interpreted as a convergence model. However, if I put the occurrence rates in the graph (as a \% subscript), we must conclude that the occurrence rates do not correlate in any way with the level of markedness.


In the markedness graph in \figref{fig:postma:4}, on the other hand, we observe two sets of dialects with respect to their value of P4. The flip in P4 coincides with their identification as low and high contact varieties. Interestingly, the value of their markedness neatly correlates with their relative frequencies. The absent [∅ ... ∅]-pattern has the highest markedness of µ = 3. The most general [\textsc{for} ... \textsc{to}] pattern is a local minimum of 1. The general [\textsc{for.to} ... ∅] in BP has markedness 0.


\begin{figure}[t]
\includegraphics[width=\textwidth]{figures/PostmaMarkedness2.pdf}
  \caption{
Markedness graph of the Accommodation Model with occurrence rates.
}\label{fig:postma:4}\end{figure}

We may, therefore, use the occurrence rates of the varieties and their relation to markedness as independent evidence that the P4-feature provides a better model of the change that Pomeranian underwent upon its settlement in Brazil, than the convergence model with the P3 parameter. It might also be taken as evidence that P4 is a better measure of the difference in markedness of covert-overt movement in general. 

If we take the occurrence rates into account, I come to a different conclusion than my 2016 study: what has happened in the emergence of BP, is not dialect convergence within Pomeranian itself, triggered by the high level of \textit{variation} present in the input dialects, but accommodation to an external language, Portuguese.

\newpage
\section{Conclusions} %9. /

The sociolinguistic observations on Pomeranian, with language variation in Europe and convergence to a uniform construction in Brazil, provides evidence for an underlying syntactic C-T chain in natural languages, as was argued for in \citet{Pesetsky2007} on formal grounds. While European Pomeranian shows \textit{variation} in the lexicalisation of this
[\textsc{for ... to}] chain with a three-fold optionality, Brazilian Pomeranian displays obligatory lexicalisation of the higher link of the chain and obligatory silence of the lower link. This configuration is reanalyzed as an overt movement relation of T to C, which is the default option in natural language. There are language-internal arguments that the new construction is a result of dialect-convergence to the default setting of the parameters involved. However, when we take the external occurrence rates into account, the data indicate that the similarity in this respect between Brazilian Pomeranian and (Brazilian) Portuguese might be analyzed as accommodation of Brazilian Pomeranian to the dominant language Portuguese.

\section*{Abbreviations}
\begin{tabularx}{.45\textwidth}{lQ}
    \textsc{acc} & accusative\\
    \textsc{comp} & complementiser\\
    \textsc{dat} & dative\\
    \textsc{inf1} & infinitive in -e\\
    \textsc{inf2} & infinitive in -en\\
\end{tabularx}
\begin{tabularx}{.45\textwidth}{lQ}
    \textsc{obl} & oblique\\
    \textsc{prt} & particle\\
    \textsc{pl} & plural\\
    \textsc{refl} & reflexive\\
    \textsc{sg} & singular\\
\end{tabularx}

\section*{Acknowledgements}

This is an extended version of \citet{Postma2016}, written in German, that models dialect convergence. This English version models both accommodation and dialect convergence. Moreover, it adds a graphical tool to render the level of markedness of the various variants. These insightful diagrams turned out so powerful that they partly changed the conclusions with respect to \citet{Postma2016}. I thank my colleagues at the Meertens Institute, the audiences of the Workshop ``German Abroad'', Vienna 2014, the audience of a talk at FFLCH at USP, Nov 11, 2013, of the Colóquios de Sintaxe, Aquisição e Mudança, University of Campinas, Nov 12th, 2013, of the workshop on Heritage Languages, Amsterdam, August 18, 2014, and of ICLaVe10, 26 June 2019, Leeuwarden for their comments and suggestions. I am grateful to two anonymous reviewers their helpful comments. A word of gratitude to my informants Elizana Schaffel, Teresa Gröner, and Hilda Braun for their help and patience. A special thanks to Andrew Nevins with whom I did part of the fieldwork. All errors are mine.

\sloppy
\printbibliography[heading=subbibliography,notkeyword=this]
\end{document} 
