\documentclass[output=paper,hidelinks]{langscibook}
\ChapterDOI{10.5281/zenodo.7446963}
\author{Manuela Pellegrino\affiliation{CHS, Harvard University} and Maria Olimpia Squillaci\affiliation{University of Naples “L’ Orientale”}}

\title{What is the role of the addressee in speakers’ production? Examples from the Griko- and Greko-speaking communities }


\abstract{In the literature on minority languages, language use and variation are commonly analyzed with reference to the speaker. In this contribution we instead focus on the addressees and how they can impact the speaker’s language use and influence speech production. We will discuss these issues in relation to Griko and Greko, two endangered Italo-Greek varieties spoken in the south of Italy, Salento (Puglia) and Calabria respectively.}

\IfFileExists{../localcommands.tex}{
 \addbibresource{../localbibliography.bib}
 % add all extra packages you need to load to this file

\usepackage{tabularx,multicol}
\usepackage{url}
\urlstyle{same}

\usepackage{listings}
\lstset{basicstyle=\ttfamily,tabsize=2,breaklines=true}

\usepackage{langsci-basic}
\usepackage{langsci-optional}
\usepackage{langsci-lgr}
\usepackage{langsci-osl}
% \usepackage{./langsci/styles/langsci-lgr}
% \usepackage{./langsci/styles/langsci-osl}
% \usepackage{langsci-gb4e}

\usepackage{tikz}
\usetikzlibrary{patterns,calc}
\pgfdeclarepatternformonly{south east lines}{\pgfqpoint{-0pt}{-0pt}}{\pgfqpoint{3pt}{3pt}}{\pgfqpoint{3pt}{3pt}}{
    \pgfsetlinewidth{0.6pt}
    \pgfpathmoveto{\pgfqpoint{0pt}{3pt}}
    \pgfpathlineto{\pgfqpoint{3pt}{0pt}}
    \pgfpathmoveto{\pgfqpoint{.2pt}{-.2pt}}
    \pgfpathlineto{\pgfqpoint{-.2pt}{.2pt}}
    \pgfpathmoveto{\pgfqpoint{3.2pt}{2.8pt}}
    \pgfpathlineto{\pgfqpoint{2.8pt}{3.2pt}}
    \pgfusepath{stroke}}
    
\usepackage{stmaryrd}
\usepackage{wasysym}
\usepackage{multirow}
\usepackage{caption}
\usepackage{subcaption}
\usepackage{mathrsfs}
\usepackage{qtree}

\usepackage{linguex}


 %pminos do not split footnotes
% \interfootnotelinepenalty=10000 %Footnote in Laporte chapters has to be split SN


%\DeclareIndexNameFormat{default}{%
%\nameparts{#1}%
%\usebibmacro{index:name}%
%{\index[names]}%
%{\namepartfamily}%
%{\namepartgiveni}%
% {}% L1
% {}% L2
%{\namepartprefix}% generates spurious space L3
%{\namepartsuffix}% generates spurious space L4
%}

%  {\DeclareIndexNameFormat{default}{%
%     \usebibmacro{index:name}{\index[names]}{#1}{#3}{#5}{#7}}}

%\DeclareIndexNameFormat{default}{%
%  \usebibmacro{index:name}{\sindex[nom]}{#1}{#3}{#5}{#7}}

%\DeclareIndexNameFormat{default}{%
%  \usebibmacro{index:name}{\sindex[person]}{#1}{#3}{#5}{#7}}
%\DeclareIndexNameFormat{default}{%
%\nameparts{#1} \usebibmacro{index:name}{\sindex[person]]}{\namepartfamily}{‌​\namepartgiven}{\nam‌​epartprefix}{\namepa‌​rtsuffix}}

%\newcommand{\smiley}{:)}

%\renewbibmacro*{index:name}[5]{%
%\usebibmacro{index:entry}{#1}%
%{\iffieldundef{usera}{}{\thefield{usera}\actualoperator}\mkbibindexname{#2}{#3}{#4}{#5}}}

% \newcommand{\noop}[1]{}

%remove for final
%\overfullrule=1mm

\newcommand{\tobi}[2]}}
\renewcommand{\S}[1]{\tobi{#1}{\textsc{*}}}

% this volume references
% puts: [this volume]
% already defined: \citetv
%\newcommand{\citepv}[1]{(\citeauthor{#1} \citeyear*{#1} [this volume])}
\newcommand{\citealtv}[1]{\citeauthor{#1} \citeyear*{#1} [this volume]}

%parentheses around example number
\newcommand{\pref}[1]{(\ref{#1})}

% in-text examples

\newcommand{\lnex}[1]{\textit{#1}} %target lang word
\newcommand{\lnlit}[1]{(lit.: `#1')} %literal reading
\newcommand{\lnlat}[1]{(#1)} % latinization
\newcommand{\lntrans}[1]{`#1'} %translation
\newcommand{\lnexl}[2]%
{\lnex{#1}{} \lnlat{#2}} % ex with latinization
\newcommand{\lnexlat}[3]{\lnex{#1}{} \lnlat{#2}{} \lntrans{#3}} % ex with latinization and tranl.

%ch01
\newcommand{\co}[1]{\mbox{\textbf{#1}}}

%ch09

\newcommand{\cyrbulg}[1]{\begin{otherlanguage*}{bulgarian}#1\end{otherlanguage*}}


%ch10
\newcommand{\nlp}{{\small NLP}}
\newcommand{\mwe}{{\small MWE}}
\newcommand{\rae}{{\small RAE}}
\newcommand{\lvc}{{\small LVC}}
\newcommand{\pos}{{\small P}o{\small S}}
%\newcommand{\todo}[1]{ \textcolor{red}{#1} }

%\renewcommand{\labelenumi}{\theenumi}
%\ainamefmt{{vv}{ll}{, ff}{, jj}} % fullname

\newcommand{\biberror}[1]{{\color{red}#1}}

\newcommand{\osenovaitem}{--~} 
 %% hyphenation points for line breaks
%% Normally, automatic hyphenation in LaTeX is very good
%% If a word is mis-hyphenated, add it to this file
%%
%% add information to TeX file before \begin{document} with:
%% %% hyphenation points for line breaks
%% Normally, automatic hyphenation in LaTeX is very good
%% If a word is mis-hyphenated, add it to this file
%%
%% add information to TeX file before \begin{document} with:
%% %% hyphenation points for line breaks
%% Normally, automatic hyphenation in LaTeX is very good
%% If a word is mis-hyphenated, add it to this file
%%
%% add information to TeX file before \begin{document} with:
%% \include{localhyphenation}
\hyphenation{
    Beck-man
    Ngu-yen
    back-chan-nel
    back-chan-nels
    mo-not-o-nous
    ste-reo-typ-i-cal
}

\hyphenation{
    Beck-man
    Ngu-yen
    back-chan-nel
    back-chan-nels
    mo-not-o-nous
    ste-reo-typ-i-cal
}

\hyphenation{
    Beck-man
    Ngu-yen
    back-chan-nel
    back-chan-nels
    mo-not-o-nous
    ste-reo-typ-i-cal
}
 
 \togglepaper[5]%%chapternumber
}{}

\shorttitlerunninghead{What is the role of the addressee in speakers’ production? }
\begin{document}
\shorttitlerunninghead{What is the role of the addressee in speakers’ production? }
\lehead{Pellegrino and Olimpia Squillaci}
\maketitle

\section{{Introduction}}

Over the past decades, there has been an increase in Greko and Griko written production, involving both elderly mother-tongue speakers and especially so-called ``semi-speakers'', and local language experts (cf. \citealt{Martino2009} and \citealt{Pellegrino2016a}). By contrast, the spoken use of both varieties has long been decreasing, progressively losing domains. At present, on an everyday basis, locals mostly communicate in the local Romance varieties -- Salentine (Puglia) and Southern Calabrian (Calabria) -- or in Italian, and they use Griko and Greko in limited contexts and with an increasingly smaller number of people. This applies in particular to Calabria (Area Grecanica), where at the community level Greko is used significantly less than Griko in Salento (Grecìa Salentina). The dynamics leading to this decrease are multiple and complex, and concern the broader processes of language shift and abandonment.



In this paper, we provide a preliminary analysis of data in relation to speaker-addressee dynamics and how these affect language use and may potentially lead to what we refer to as ``temporary variation''. In particular, we highlight the role of the addressee in such dynamics and demonstrate how the addressee’s linguistic competence, age, and shared linguistic repertoire with the speaker may lead to style-shift in speakers’ production; this, in turn, contributes to the emergence of puristic attitudes and even inhibits the use of the varieties themselves. This will also shed light on key differences between the two communities with regard to reception of language maintenance and/or revitalisation programs. 



This chapter builds on the authors’ joint project, “Investigating the future of the Greek linguistic minorities of Southern Italy”. This project provides a comparative examination of the responses to language maintenance and revitalisation initiatives among the Griko- and Greko-speaking communities. It was part of Sustaining Minoritized Languages in Europe (SMiLE), an interdisciplinary research program developed by the Center for Folklife and Cultural Heritage (Smithsonian Institution). One goal was to produce ethnographic studies of six communities in Europe and analyze how language-related initiatives build on motivational responses to social, cultural, political, and economic factors.



Our research rested at the intersection between social and linguistic domains, using qualitative methods such as participant observation and ethnography of speaking. Between January 2018 and June 2019, we conducted semi-structured interviews with leaders of cultural associations, elderly speakers, new and non-speakers, young people, academics, and representatives of institutions, totaling over 70 individuals.\footnote{All interviewees’ names in this contribution are pseudonyms.} We also participated in and observed 40 local cultural activities, speaking with those involved whenever possible. These cultural activities included music festivals, seminars, poetry competitions, and school projects. This allowed us to compare the current maintenance and revitalisation activities being implemented in both areas, their reception by and their impact on the communities, the target audience, and the degree of involvement of the young people in such activities. Our analysis is also enriched by the previous anthropological and linguistic research that we independently carried out during our doctoral work, as well as by our personal connection to and long-standing engagement with the Griko- and Greko-speaking communities from which we authors hail, respectively.



The article is structured as follows. In Section \ref{sec:background} we provide some historical background on the Griko and Greko varieties. In Section \ref{sec:key:3} we move on to discuss language use; using some key examples we will argue that the age and minority-language competence of the addressee play a crucial role in inhibiting or favoring a speaker’s use of Griko and Greko in conversational settings. This analysis will bring to light some of the main differences between the two communities, which are the result of past and current dynamics unique to each. In Section \ref{sec:key:4} we focus specifically on the language competence of the addressee in (standard and regional) Greek,\footnote{In this work we distinguish between the word \textit{Greek}, (which we use to refer to the language as it is spoken throughout Greece and Cyprus, including its regional and local varieties) and \textit{Standard Modern Greek} (SMG) to refer to the official language as it is taught by state institutions.} and show how this may bring about instances of ``temporary variation'' in the speech production of some Griko and Greko speakers. As we will demonstrate, such an influence, although limited to specific contexts, has significant repercussions for inter- and intra-community dynamics which transcend speech production itself. In this respect, we also draw attention to researchers’ implicit or explicit attitudes towards interference and variation, which may promote prescriptivist and puristic values when dealing with minority languages.



\section{Background information}
\label{sec:background}
There has been extensive debate surrounding the origin of Griko and Greko, as scholars have not reached agreement on whether the Greek of southern Italy originates in the Magna Graecia period, as claimed by Gerhard Rohlfs (\citeyear{Rohlfs1924, Rohlfs1974}) or the Byzantine period (\citealt{Falcone1973}, among others). Indeed, southern Italy experienced two waves of Greek influence: one in the 8\textsuperscript{th} century B.C. when the first Greek colonies were founded, forming what we know as Magna Graecia or Greater Greece, and the other in the 6\textsuperscript{th} century A.D., when the Eastern Roman Empire, also known as the Byzantine Empire, reconquered the southern regions of the Italian Peninsula after the fall of the Western Roman Empire. Between these two periods, southern Italy was under Roman control. The question is hence whether Greek has been spoken continuously in southern Italy since the Magna Graecia period or if Griko and Greko descend from Byzantine Greek. The Italian linguist \citet[69]{Fanciullo2001a} highlighted the ideological nature of this controversy and defined it as a ``false problem'' since it was based on the assumption that Greek and Latin could not co-exist. Indeed, while Italian philologists have tended to support the Byzantine theory -- as arguing the contrary would have jeopardised the ``Italianness'' of these people -- Greek scholars have tended to favor the theory of continuity since ancient times. As discussed by Pellegrino (\citeyear{Pellegrino2015, Pellegrino2021}) this represents a ``language ideological debate'' \citep{Blommaert1999} highlighting how contested language ideologies are appropriated differently in different historical periods, and by people with diverging aims. 



What is undeniable is that the end of Byzantine rule marked the beginning of a language shift towards Romance, a slow process that was initially characterised by a long period of intense bilingualism between the local Greek and Romance varieties, which led to a progressive decrease in the number of speakers of the former. Nonetheless, the most abrupt decrease in the number of speakers occurred after the unification of Italy. In particular, from the beginning of the 20\textsuperscript{th} century the newly formed Italian state fostered a wholesale Italianisation project of the peninsula, coupled with significant discrimination against people speaking local and in particular non-Romance varieties such as Greko and Griko. Furthermore, compulsory monolingual education in Italian, promoted particularly under the Fascist regime, the flow of emigration to the north of Italy and abroad, and later, the influence of mass media significantly contributed to the loss of the minority languages. In addition to this, in the case of Greko, the area’s geographical isolation, along with natural disasters in the 1950s and 1970s, played a major role in the depopulation of the Greko-speaking mountain villages. This displacement had major repercussions for the community, contributing to its disaggregation and favoring language abandonment \citep{Stamuli2008, Martino2009, Squillaci2018}. 



However, the profound socio-economic changes that took place, particularly following WWII, did not mechanically determine language shift. As noted by Pellegrino (2016a:, 2021), there was instead “an existential shift” from a traditional to a modern worldview, causing communities to go through a difficult negotiation process of redefining their perceptions of themselves and of their group, along with their values and goals. These were then encoded through language. Although painful, abandoning Griko and Greko was considered a passe-partout for social enhancement (cf. \citealt{Martino1980, Stamuli2008}, and \cite{Squillaci-inprep}). 


\largerpage
Currently, the Griko-speaking community is composed primarily of middle-aged and elderly people with various degrees of language competence. Griko is mainly spoken in seven villages in the province of Lecce: Calimera, Castrignano dei Greci, Corigliano d’Otranto, Zollino, Sternatia, Martano, and Martignano (there were additionally Griko speakers in Melpignano and Soleto until the beginning of the 20\textsuperscript{th} century). In Calabria, Greko is spoken today, after the aforementioned displacements, mainly in Condofuri, particularly in the hamlet of Gallicianò, in Roghudi Nuovo; in Bova; and in a few neighbourhoods in Reggio Calabria, Bova Marina as well as Melito P.S. The Greko-speaking community is smaller than its Griko counterpart, and the most concerning aspect is the age of the majority of speakers: most are over 80, with only a few in their 60s and 70s. Thus, unlike Griko, Greko has very few speakers with limited competence aged between 40 and 50.


\section{Griko and Greko language activism}
\largerpage
\citet{Pellegrino2016a, Pellegrino2021} provides a diachronic account of Griko activism and refers to the first, middle, and current revivals of Griko. The first revival stretches from the end of the 19\textsuperscript{th} century to the mid-1970s and is centred around the activity of the Philhellenic circle of Calimera. This was constituted by local intellectuals who were influenced by the contacts they had established with Greek folklorists. Because of the intellectualist nature of their efforts, however, language practice was not affected, so the revival did not prevent the shift from Griko to the local Romance variety and then to Italian. By the mid-1960s, the number of Griko mother-tongue speakers had dramatically decreased. The middle revival occurred in the late 1970s and 1980s. It was promoted by local cultural activists who funded cultural associations in the various Griko-speaking villages. This revival was not restricted to Griko and not limited to activities in support of the language, but included the local culture as a whole. It was a response to the break in cultural practices caused by the abrupt modernisation process in the years following WWII; it therefore presented itself as a form of redemption for a long-stigmatised South. 



The current revival started in the 1990s and continues until the present. It is the outcome of interaction between the language policies and ideologies promoted by the EU, Italy, and Greece. Among the effects of the most recent revival of Griko, there is a sense of empowerment and pride in the rediscovered value given to the local cultural heritage (music and language included). However, while the revival has become a springboard for expressing a range of local claims, it does not include efforts specifically linked to the use of language as a tool of daily communication or to the training of new speakers. Crucially, over the years, the performative and artistic use of the language has increased while the use of Griko as a vehicle to convey information has progressively diminished. The more the language dies, the more it is resurrected performatively, as it were. The semiotic approach adopted by Pellegrino shows how Griko has now become a cultural and social resource, a form of performative post-linguistic capital, where the intentional, albeit limited, use of Griko becomes more important than ``speaking'' it as a means of exchanging information.\footnote{ \textrm{This also applies to the younger generation (starting from the 1970s), who typically are not speakers of Griko; however, they may use Griko by citing and re-appropriating words or entire expressions from memory, re-contextualised in the present, a practice which Pellegrino – building on Rampton (\citeyear{Rampton1995}) – calls ``generational crossing'' \citep{Pellegrino2021}.}} 



With the exception of the first revival, which is only relevant to Griko, the chronological/analytical framework proposed by \citet{Pellegrino2013} can be partially applied to the Greko community. In Calabria too, language activism developed in the late 1960s and 1970s generated a broader valorisation of the area as a whole which prompted the beginning of a decisive change in the use of and attitudes towards Greko at a community level. \citet{Martino1980} defines this first phase as the \textit{awakening}. During the second phase (from the late 1980s and 1990s) the activities dedicated to the Greko cultural and linguistic heritage have become more and more fragmented and often dependent on national and international program for minority languages, without long-term planning for revitalisation \citep{Martino2009}. As noted by Pipyrou (\citeyear{Pipyrou2016}, among others), in the long run, the potential of minority language discourse and struggle turned into an appealing tool for many to achieve socio-economic benefits and prestige, often leading to tension and conflict within the community. At present, Greko has largely lost its communicative value in favor of a new symbolic function; however, rather than being used performatively and for performative purposes, as in the case of Griko reported above, what is mostly attested for Greko is its folklorisation: folklorised use of the language in specific contexts, such as official salutations and celebrations, to assert belonging to the Calabrian Greek heritage (\cite{Squillaci-inprep}, drawing from \citealt{fishman1991}; see also \citealt{Martino2009} and \citealt{Pipyrou2016}). 



Simultaneously however, we are witnessing a new awakening, which is mostly reflected in the recent coming together of a group of fifteen young people who, through the active involvement of Squillaci, have been carrying out language revitalisation activities. The group is part of the local association Jalò tu Vua, which has been working to promote the language since the early 1970s. Like the first movement of activism in the 1960s, this too stems from a sense of awareness of the cultural and linguistic heritage of the area, and is based in particular on a shared sense of responsibility towards language loss.



\section{The addressee’s age and competence in Griko/Greko}
\label{sec:key:3}

To begin, we discuss the cases in which the addressee’s age and competence in the minority language may influence the speaker’s use/non-use of Griko and Greko. As argued by \citet{Pellegrino2021}, Griko is largely considered to belong to the older generation; this age-related factor seems to effectively exclude younger people from the world -- the past -- usually associated with Griko, since they did not live it. In a recursive way, perceptions of who can claim authority over Griko also define the authenticity of language and language practices. 



This also became clear during our joint fieldwork. As one of our middle-aged informants said, “The real speakers are the elderly, only what they speak is true Griko.” Consequently, anyone who is not old is not perceived as an ``authentic'' speaker, irrespective of competence or fluency. When Squillaci remarked that Pellegrino had indeed been speaking Griko throughout the evening, our informant commented, “The Griko that Manuela speaks is not the same language of the elderly”. Similarly, older speakers question the competence of middle-aged and younger speakers regardless of their actual production. We witnessed this attitude again in a conversational setting, when an older mother-tongue speaker, aged 92, commented on the production of a younger speaker in his mid-60s, pointing out that, “This is not Griko really” (\textit{En' ene probbiu griko}); this clearly links the age factor to language authority and competence. Likewise when we met another middle-aged informant in Grecìa Salentina, we noticed that he would speak Griko with his elderly neighbours and also with us in the flow of the same conversation -- but he spontaneously switched to Italian when interacting just with us. Pellegrino initially tried to keep the conversation in Griko, but as he clarified, he was not used to speaking Griko with her because of her age, and doing so would be odd \citep[]{pellegrino2022agency}.



Indeed, older speakers argue that it does not come ``naturally'’ to speak Griko to younger people, and when approached by someone who makes an effort to speak the language, they tend to make fun of their mistakes. These instances reveal the power struggles embedded in the current revival of Griko, whereby the older generation claim authority over the language based on their embodied knowledge, and express skepticism and resistance towards younger speakers, and towards the more recent proliferation of language experts \citep{Pellegrino2016b, Pellegrino2021}.\footnote{For a more in-depth analysis of locals' ideologies with respect to the ``purity'' and ``authenticity'' of Griko, and the resulting power struggles within the community, see \citealt{Pellegrino2016b} and \citeyear{Pellegrino2021}. Here it is argued that the multiple and competing criteria by which locals define the authenticity of language and language practices recursively determine who can claim authority over it, and vice versa.} This constant control over the language and resistance to change by the older generation or by language experts often leads younger speakers to switch and continue the conversation in the local Romance variety and/or Italian, as they feel sanctioned over the incorrect use of the language and thus disempowered.



Moving to southern Calabria we find a different picture. Here, middle-aged speakers are not a priori considered less competent due to their age, or because they are not necessarily native speakers, nor are they considered less authoritative. This is particularly true for long-standing activists -- in our specific case the interviewees have been engaging with Greko since their 20s and are considered part of the speakers' community on a par with older speakers. “Here in Bova Marina there is no one who speaks Greko anymore, just me [native speaker in his 80s], Demetrio [speaker in his 60s], Carmelo [speaker in his 60s], Salvatore [native speaker in his 80s], and Bruno [speaker in his 80s]”. Pasquale thus includes as part of the pool of speakers those who -- like Demetrio and Carmelo -- are today in their 60s and 70s. Over fluency, in this case it is their long-standing engagement with the language which grants middle-aged speakers or semi-speakers legitimacy as full members of the speaker community.



The difference between Calabria and Salento becomes even more evident on consideration of the current intergenerational communication between older native Greko speakers and younger new speakers in their 20s and 30s. Here, fluency rather than age or long-standing language engagement seems to be the main criterion for establishing a conversation with elderly speakers. Fluency is intended by the majority of speakers as the capacity to \textit{speak}, that is to answer back in Greko to the speakers’ input, regardless of grammatical mistakes which are not taken into consideration by the majority, as we shall see. This is clear from the answer of an old Greko woman when a fellow villager commented on her use of the language with Squillaci, given that she usually refuses to speak Greko. The speaker replied, \textit{“Egò to platego manachò me cinu ti to plategu!”} `I speak it [Greko] only with those who do speak it', showing how fluency was crucial in establishing a conversation with her, over the age difference, as in Salento. Speakers’ initial resistance is thus overcome, once they have the confirmation that the person wants to establish a conversation with them, not just to ``hear them speak Greko'', as speakers complain.\footnote{\textrm{For many speakers, this is also because they feel treated as guinea pigs, as \citet{Petropoulou1992} had already noticed back in the 1990s. The situation has been exacerbated over the years given the high numbers of visitors, journalists, and researchers who regularly visit the villages (when compared to the smaller numbers of inhabitants of such villages). In addition to this, many, especially women, refuse to be videoed or photographed and, given that their requests often are not honoured, they leave as soon as someone approaches them.}} This often comes either after they are reassured by another speaker that their addressee does indeed have some competence in the language or after testing the addressee’s competence themselves, usually asking the translation of some basic words. In the case of the recently formed group of new speakers, for instance, Squillaci’s presence -- as a young community member well known to the older speakers -- and her organisation of multiple informal and formal intergenerational encounters facilitated this transition. Yet, it was the new speakers’ ability to ``answer back'' that caught older speakers’ attention and translated into respect for their efforts. At all the events we participated in during our fieldwork, new and older/middle-aged speakers would naturally interact in Greko and the latter would rarely correct or comment on younger speakers’ linguistic production in the way we observed in Grecìa Salentina; when they did, it was clear that their aim was to explain the correct way unpretentiously. Some older speakers would at times clarify the fact that younger speakers ``have not learned the language as we did'' since they are the last generation ``of this type'' of Greko speakers (again including middle-aged speakers in this pool). However, different from Salento, it seems that they do so as a mere recognition of differences, which does not imply a delegitimisation of new speakers' language efforts. These are, instead, publicly recognised and valued by many as ``the only possible future of the language'', as phrased by one of the older speakers and an activist \citep{Squillaci-inprep}. 



Like age, provenance also seems not to influence speakers’ attitudes; many of these young people do not come from the last Greek-speaking area, but are nonetheless well accepted within the community. ``Could you ever imagine we would have a Greko speaker in Campo Calabro [a non–Greko-speaking town]? This is a miracle!'', an older speaker said after a conversation with one of the new Greko speakers. Particularly interesting in this respect was the fact that a very old speaker happily accepted as Greko language teacher a new speaker who does not originally come from the Greko area; he would instead dismiss his family members’ language competence, as it is mainly passive. Indeed, Enzo considers self-evident that ``\textit{ecini en to plategu}'' ``they [his family members] don’t speak it'', as he put it, and therefore do not know it, inasmuch as they can hardly carry a full conversation in Greko. What they value over age and provenance is therefore that the addressee can \textit{speak} Greko, regardless of potential mistakes. Old speakers adopted the same open attitude with Pellegrino regardless of the fact that she originally comes from another region and irrespective of her age; her knowledge of Griko facilitated conversations in Greko and thus her subsequent acceptance into the community of speakers. 



We noticed, however, how the attitude of openness which characterises these older speakers with whom we have been in contact, does not hold true among all the members of the generation of language activists, most of whom are today in their 60s and 70s despite several attempts from the new speakers to also actively include them in the various events and activities.\footnote{During the first months of activities in particular, Squillaci regularly held one-to-one informal meetings with older and middle-aged Greko-speakers who have always been involved in language-related activities. These meetings were crucial for acquiring the support of many former activists for new speakers.} We have indeed seen how, in the name of ``authenticity'', some long-standing activists are critical of the activities that new speakers and activists carry out; this appears to be an attempt to undermine and perhaps delegitimise such language revitalisation efforts. As in the Salentine case, this situation reveals how such dynamics are in fact an expression of internal struggles over authority, attested in numerous minoritised contexts (\citealt{RourkeRamallo2013, Costa2015, Sallabank2017, SallabankMarquis2018}, among others). Unlike in Salento, however, in Calabria middle-aged activists’ engagement with the language has actually decreased considerably over time, and their presence in language-related activities is limited nowadays, allowing younger activists the space to carry out their projects.



With respect to the factors behind the difference in attitudes towards using the language with new/younger speakers, we identified the different degree of language endangerment, community size, and different paths of language activism among the main contributing factors. In Grecìa Salentina, locals engage in multiple ways with Griko, and this translates into various forms of activism. These reveal how claims to language authority are diffused, and often translate into discussions around and about the language which generate tensions among activists, experts, and speakers. Such dynamics may lead not only to interpersonal but also intra-group estrangements based on locals’ divergent positions on the form and the future envisioned for Griko \citep[157]{Pellegrino2021}. In turn, this climate may discourage non-fluent and/or younger speakers from using the language, and from embarking on activities in its support. Indeed, as is often observed in minority language contexts, their efforts are to varying degrees delegitimied by the older generations, who consider themselves and are considered to be the gatekeepers of the language. 



Interpersonal and intra-group estrangements were also commonly reported in the case of Greko up until the past decade, including extensive language monitoring and much negative judgment on the language activities that each association promoted; on the other hand, there also used to be substantially more activities, program, and events related to the Greko cultural and linguistic heritage (cf. \citealt{Martino2009} and \citealt{Pipyrou2016}). “If we had talked less about the language and more in the language we might not have been in this situation today,” an activist told Squillaci. Today, instead, the even smaller size of the speaker community, its less active participation in language-related activities, the overall decrease in the number of activities and of active associations dedicated to the language, as well as the growing level of language ``endangerment'' lead to little intra-community discussion of language issues, and crucially to less conflict over language authority. As we have seen, this emerges mostly in relation to the former activists’ position within the community, and it seems to be nonetheless limited. As is typical among dying language communities, “self-appointed monitors of grammatical norms may become increasingly rare” \citep[154]{Dorian1981}, favoring a “relaxation of internal grammatical monitoring”; this attitude creates a more favorable environment for language learners, who are not discouraged from continuing their efforts. “\textit{Echome tunda pedìa ti to sceru to greko, echome ta pilastria, tuto spiti den petti pleo}”, ``We have these young people who know Greko, we have the basement, our house won’t fall anymore'', an older activist proudly commented during a public event, publicly endorsing new speakers’ language revitalisation efforts. 



In conclusion, in addition to the generally limited use of the language discussed at the beginning of this section, language use in the Griko and Greko communities might be further inhibited by the addressee’s younger age regardless of fluency, as seen in Salento, or favored by the addressee’s language fluency, as shown in Calabria, regardless of his/her age. Such a difference between the two communities mirrors their different reception of language maintenance and/or revitalisation efforts. While in Salento multiple claims over language authority and widespread internal monitoring might discourage such attempts (cf. \citealt{RourkeRamallo2013, Costa2015, Sallabank2017, SallabankMarquis2018}, among others.), in Calabria the more open and supportive attitude towards new speakers/learners favors the language revitalisation activities they promote. 



In the next section, we focus on the temporary variation produced by the speaker under the influence of the addressee's linguistic competence. In particular, we analyze this with reference to Greek, since this is an aspect which has not yet been investigated. We leave for further research the analysis of how Italian and the local Romance varieties (Salentine and Southern Calabrese) may trigger temporary variation.



\section{The addressee's language competence in Greek}
\label{sec:key:4}

As is well known, multiple linguistic and extralinguistic factors influence speech production in minority language contexts, leading to temporary and/or permanent variation. Within the Griko and Greko communities at least three different types of variation can be identified. First, historical variation: both Griko and Greko display a great deal of internal variation, which has historically led to the emergence of Griko and Greko varieties specific to individual villages. This variation mostly amounts to phonological differences -- the Greek consonant cluster \textit{ξ} /ks/ has become /ʃ/ or /ts/ in Greko and /ʃ/, /ts/, /s:/, and /fs/ in Griko, for instance -- as well as lexical differences among villages. Speakers are well aware of these distinctions, and may use them to claim the authenticity of one variant over another. Second, ``emergent variation'': both communities have long experienced increasing variation due to the endangered status of the varieties, which has affected speakers’ production and has led to the creation of idiolects. Third is ``temporary variation'', which is linked to various factors, including speech production on demand (i.e., when speakers are asked to speak the language by researchers or tourists), speakers’ personal preferences, setting (e.g., interviews, public events), and temporary interference due to the spectrum of speakers’ linguistic resources (i.e., competence in Italian, local Romance varieties, and Standard Modern Greek). 



In addition to this, we observed that speakers’ production may also be affected -- albeit temporarily -- by the addressee’s shared multiple linguistic competences. Interestingly, if this temporary interference involves Romance, it does not raise any questions at the community and academic levels about the competence of the speaker.\footnote{Similarly, no questions at the community level are raised in the case of elderly speakers, who tend to code-shift if the addressee has limited or no knowledge of the minority language.} Different, instead, is the response to temporary variation, which involves Greek, and which regards, in particular, the speech of middle-aged and younger Griko and Greko speakers with a certain degree of knowledge of SMG. Instances of temporary variation are often perceived as a lack of competence in the minority language, and therefore may ultimately influence speakers’ use of the varieties. 



The potential influence of SMG on Griko and Greko has indeed been a disputed topic both at the community and academic level over the years, and this dispute has intensified with the progressive increase in relations with Greece/Greek speakers, together with the introduction of SMG courses in both regions funded by the Greek Ministry of Education. Yet the language courses at schools have not proven successful in spreading SMG, so to speak, given the overall lack of interest in the subject and the fragmentation of the courses (these are also common issues regarding Griko/Greko language courses in schools). Similarly, those organised in collaboration with associations have attracted the curiosity of some locals; in particular in Salento, they tend to be attended by retired/elderly Griko speakers. Such courses, however, have not influenced the overall frequency of Griko and Greko usage. 



Instead, what has played a bigger role in affecting the speech of some speakers -- although to different degrees and in specific contexts -- is the increase in contact with speakers of Greek (visitors, researchers, tourists, journalists) who regularly visit the Griko- and Greko-speaking communities: this applies, in particular, to the teachers sent by the Greek Ministry of education, who establish strong bonds with locals and take active part in the life of these communities. Moreover, the numerous exchange trips to Greece (and more rarely Cyprus) and the language scholarships it provided have also played a role. In Calabria, such contacts and exchanges have interested a larger part of the community than in Salento, and have left a mark in particular on the speech of younger (today, middle-age) generations who have been involved in language activism over the past 30 to 50 years, as well as of the older generations who have taken active part in these exchanges, albeit to a different degree. In Salento, instead, this dynamic has not played a significant role, particularly with regard to elderly speakers, as they have comparatively been less actively involved in these activities. This difference is in part linked to the actual size of the two communities: as the Greko-speaking community is comparatively smaller, more people had the opportunity to visit Greece or to come into direct contact with Greek visitors in Calabria, thereby acquiring varying degrees of knowledge of Greek. Interestingly, however, we also found that Greko speakers and activists have and had a more open attitude than their Griko counterparts with respect to the use of SMG loanwords; this is also reflected in less judgment from within the community over such use; we will come back to this topic in Section \ref{sec:key:4.1}. In this section, we discuss the observed increase in their use of SMG loanwords when addressing people who do not originally come from the community.



In Salento, speakers who have taken courses in SMG and have achieved competence in the language also tend to be more directly involved in welcoming Greek visitors. During such encounters, therefore, they may use for instance \textit{dromo} instead of the Salentine borrowing \textit{stra/strata}, \textit{oxi} instead of the Griko \textit{de/nde/degghe}, or pronounce Griko words according to SMG phonological rules, as \textit{ekino} over Griko \textit{ecino} (with palatalisation of velar /k/) with the aim of facilitating comprehension -- but sometimes also to demonstrate their competence in SMG. We noticed how they may equally reproduce such dynamics when interacting with anyone outside the community. For example, Michele used SMG words when speaking to Squillaci, since she was equally perceived as a visitor. In the first minutes of the conversation he used SMG words such as \textit{pollà, kai, oikogenia}, before abandoning his role as a ``tourist entertainer'' and turning to Griko, also in response to Squillaci’s linguistic input. 



Similar issues are attested in the case of Greko. For instance, we witnessed an increase in SMG words in the speech of a Greko native speaker and activist (in his 70s) when talking to a journalist of Greek origin. The speaker used \textit{mikrò} over Greko \textit{cceḍḍi} ``small'', \textit{vunì} over Greko \textit{oscìa} ``mountain'', \textit{katalavennise} over Greko \textit{kapegghise} -- ``you understand'', \textit{dimarko} ``mayor'', \textit{taxidi} ``trip'', and other borrowings which do not have a Greko equivalent. We must reckon that these words are attested in the speech and in the writings of the speaker in question, regardless of the context, inasmuch as they belong to the class of SMG borrowings which are more widely employed in Greko (see Section \ref{sec:key:4.1}). However, we noticed a significant increase in their use when the speaker addressed the journalist, and a considerable decrease and, in some recordings, a total absence of these borrowings, when the speaker addressed older speakers. This is in line with studies of audience design which demonstrate the active role of the addressee -- or hearer, in Bell’s terminology \citep[144]{bell1984language, bell2001back} -- in shaping the stylistic production of the speaker. This factor must be taken into consideration when performing a linguistic analysis\footnote{This is different from the case attested in Calabria, which concerns non-fluent speakers who have a passive knowledge of Greko and have been in contact with Greek people. Taking into account the generally limited use of Greko within the community, these speakers end up putting the language into practice mostly when interacting with Greeks. Consequently, they tend to employ Greek words or expressions in conversation, regardless of the addressee. This type of interaction reveals that opportunities to speak SMG, through visits by Greek tourists for instance, are becoming more frequent than opportunities to speak Greko.}. 



This happens all the more so with Griko and Greko speakers who have acquired a certain degree of knowledge of SMG, or who visit Greece regularly. In this case, in addition to lexical borrowings, we also observe temporary confusion between the two codes, which may partially and temporarily affect the morphosyntax of Greko and Griko. For instance, after returning from Greece, Giuseppe would consistently code-mix Greko with SMG for the first few days we spoke to him during our fieldwork. Similarly Giuseppe code-mixed and at times code-shifted to SMG when asked to speak Greko with Greek speakers, thus showing the effort to keep Greko and SMG apart in these specific interactions. In these cases, we mostly attested: (i) the use of the imperfective stem in finite complements introduced by the subordinator \textit{na}, as shown in \REF{ex:key:1}, rather than the perfective stem, which would be obligatory in Greko, shown in \REF{ex:key:2}:



\ea 
\label{ex:key:1}

\gll thelo na arot\textbf{ào} \\
I.want NA I.ask.IMP\\
\z 

\ea 
\label{ex:key:2}

\gll thelo na arot\textbf{ìo} \\
I.want NA I.ask.PERF \\
\z 



Notably, this phenomenon may be due not only to the influence of SMG, which permits both perfective and imperfective here, but also to language decay (\citealt{Sasse1992}, among others). Indeed, speakers in general -- not only those potentially influenced by SMG -- often use the imperfective stem of the verb over the perfective one (cf. \cite{Squillaci-inprep}). 



(ii) The use of the -\textit{ate} ending rather than the Greko -\textit{ete} for the second person plural of the imperfect and aorist. 



\ea 
\label{ex:key:3}

\gll eplatègg\textbf{ate} / eplatèggh\textbf{ete} \\
speak.\textsc{ipfv}.2\textsc{pl} \\
\z 



\ea 
\label{ex:key:4}

\gll epiàs\textbf{ate} / epià\textbf{ete} \\
take.\textsc{aor}.2\textsc{pl}\\
\z 



Interestingly, however, \citet[288--290]{Katsoyannou1995} reports the use of the -\textit{ate} ending as part of the Greko morphological system. Given that this is not attested in other descriptions of the language, nor does it appear in our corpus unless under influence of Greek, it becomes difficult to establish whether Katsoyannou reported a previously undocumented case of an additional morphological ending in Greko or whether this is one of the first attestations of the use of this new ending under Greek influence.



(iii) We also observed, albeit extremely infrequently, the use of the SMG pluperfect form of the type {είχα πει} `I had said’ -- with the auxiliary HAVE + the invariable form of the lexical verb, rather than the Greko \textit{immon iponda}, composed of the imperfect of BE + the invariable participial form of the lexical verb.



Similarly, in Griko we find examples of SMG endings, such as the use of -\textit{a} instead of Griko -\textit{i} for the 2sg of the present indicative of -\textit{ao} verbs, or the -\textit{ate} desinence for the second person plural imperfect instead of Griko -\textit{ato}:

\ea 
\label{ex:key:5}
 mil\textbf{à} [SMG pattern \textit{milàs}] 
\z 

\ea 
\label{ex:key:6}

\gll mil\textbf{ì} [Griko] \\
you.speak \\
\z 


\ea 
\label{ex:key:7}
 irt\textbf{ate} [SMG pattern \textit{irthate}]
\z 


\ea 
\label{ex:key:8}
irt\textbf{ato}	[Griko]
\z 

\largerpage
However, all these changes (as well as any others that might occur) are restricted to specific contexts. Most typically, contexts in which Griko and Greko speakers who also have some competence in SMG converse with Greek individuals, with people outside of the community, or when they have recently returned from Greece, sometimes also emulating SMG intonation. SMG influence seems to be otherwise attested in the lexis. 


\section{Speakers' responses to temporary language variation}
\label{sec:key:4.1}
In the case of language use in intergenerational settings, discussions about language variation also take on an ideological dimension, which in turn impinges on the perceived authenticity of the language. Griko speakers and activists remain concerned about the role that should be attributed to SMG. Some among them consider it ``an agent of renewal'' of Griko, helping to enrich and update its vocabulary through borrowings and adaptation. In this respect, contact with Greek visitors and/or the availability of SMG language courses has partially affected linguistic ``taste'' and choices. In some instances, such interplay and influence have generated or strengthened the ``drive to ‘purity'' on behalf of some locals. This leads to practices of ``verbal hygiene'' \citep{Cameron1995} wherein speakers avoid using old or new borrowings from Salentine/Italian, or sanctioning such use. Salentine has indeed long been perceived as an agent of contamination of the perceived authenticity of Griko in a pre-contact past (hence the longstanding label of Griko as a ``bastard language''; see \citealt{Pellegrino2016b}). Such an unfortunate definition as an ``agent of corruption'', of ``contamination'' of Griko is now increasingly extended to SMG, since the majority of local activists and speakers alike tend to condemn its use as an ``artificial intervention'' which may ``kill Griko'' by erasing its historical specificities. Indeed, conscious attempts to integrate Griko with SMG occurred in the past but were criticised for creating an ``abstract'' language. The linguistic boundaries between SMG and Griko are therefore under constant surveillance. Examples of even momentary confusion between the two, and of interference from Greek, are promptly noticed, commented on, and subject to negative judgement, thus casting doubt on the competence and hence the authority of the speaker, and in turn highlighting the moral dimension embedded in the perceptions of authenticity \citep{Pellegrino2016b, Pellegrino2021}. This phenomenon can be taken to extremes. Indeed, as we attested, a Griko speaker argued that he would not even use the greeting \textit{kalimera}, since Griko speakers would instead greet each other in Italian. 



In Calabria, as mentioned previously, there seems to be a greater tolerance towards loanwords in general, including those from SMG. In fact, some of the most widespread SMG borrowings are today also known to a wider number of locals who have not been in direct contact with Greek speakers. Most of these words first entered the language via contact and are attested in interviews, documents, and writings from the late 1970s. See for instance \textit{charistò}, a Greko adaptation from SMG {ευχαριστώ} to say ‘thank you’ (instead of Greko \textit{tosso obbligato} or \textit{grazzi} from Romance), or SMG {λουλούδι} for ‘flower’ (instead of the Greko \textit{attho} or the more productive Romance loanword \textit{chiuri}), which has been used in poems and songs since the late 1970s and 1980s \citep{Squillaci2021}. Additionally, since the early years of the Greko language movement, several activists have proposed that SMG should be used as the source language for Greko neologisms and borrowings as Greko is a variety of Greek. On 21 November 2004, during a conference of local associations at the Regional Institute for the study of the Greko language, the decision was officially taken to use SMG as a source language for loanwords although, in written texts, people always had to include the Italian translation in parentheses so that older speakers would be able to understand \citep[10]{Condemi2006}. Despite this decision, however, the question has remained open over the years, being revisited and discussed from time to time in local meetings. Interference and temporary variation from SMG is thus overall not sanctioned within the community in the way we see in Salento and in fact, many of those who officially do not embrace the use of SMG loanwords often show cases of interference, in specific contexts and according to their addressee, as shown in the previous section. Recently, however, speakers have increasingly begun to pay attention to such temporary changes as they feel under external pressure regarding authenticity.



Indeed, in addition to intra-community discussion, the academic community has also been actively involved in discussing the role of SMG. Whereas the impact/influence of local Romance varieties on Griko and Greko is regarded by scholars as the outcome of centuries-long contact, the potential insertion of SMG loanwords has been considered to dehistoricise both Griko and Greko (see Rohlfs in \citealt{Petropoulou1992, Karanastasis1984}), and would prove detrimental to local maintenance and revitalisation efforts. Particularly in Calabria, the attempt to use SMG loanwords has been defined as “fanciful and contaminant” \citep[263]{Martino2009}, and as a threat to the authentic Greko language spoken by the older generations (\citealt{Karanastasis1984, Katsoyannou2017}, among others). As shown above, in most cases, the influence of SMG seems to be only a perceived threat rather than an actual abrupt and substantial linguistic change.\footnote{What contributes to a degree of confusion is that some locals with limited competence in the minority language may instead be fluent in SMG. They may equally be involved in activities to valorise/promote Griko/Greko, access funding, and enjoy respect as minority language experts, locally as well as abroad. We wish to emphasise that these are specific and isolated cases which should be analyzed separately, as they highlight some of the controversial dynamics embedded in minority language contexts.} 



The academic debate has nevertheless reached the communities and, in particular in Calabria, it has had significant repercussions on language use. In recent years, this has led to the emergence of linguistic practices of verbal hygiene (cf. \citealt{Pellegrino2021}), with the aim of achieving what is assumed to be authenticity in the language, i.e. avoiding any SMG loanwords. Greko speakers have thus begun to ``clean up'' their speech or their older poems, removing the SMG words they had inserted back in the 1990s and replacing them with their Romance counterparts to avoid criticism. For instance, they may replace \textit{daskali} with \textit{maistri} ‘teachers’, \textit{elpizo} with \textit{speregguo} ‘to hope’, \textit{vivlio} with \textit{libbro} ‘book’. Similarly, interviews and public events often arouse anxieties, as people feel the need to declare that what they speak is ``real Greko'' and often excuse themselves if some Greek word or expression ``escapes from their mouth'', as they say. On the other hand, people who have been engaged with the language for years but who are not fluent have started to justify their limited use of the language in public by saying that they prefer to avoid speaking the language altogether rather than inserting SMG words. These people are then viewed externally as an expression of authenticity, thereby leading to significant power imbalances among community members. More crucially, these discussions have significant repercussions for language use, as they cast doubt on the authenticity of the language as well as of individual speakers, and favor a sort of monitoring and self-monitoring mechanism which in turn might increasingly discourage use of the language by those who actually speak it/can speak it.


\section{Conclusions}
In this chapter, we have highlighted a number of extra-linguistic factors that influence speech production, and that might inhibit or encourage the use of the minority languages. By analysing the role of the addressee we showed how these factors are not necessarily related to the linguistic competence of the speakers, and are instead the result of the wider dynamics at play within the communities. In the first half of the paper, we provided examples of the different responses of the older speakers in interacting with younger and/or non-fluent addressees. This has highlighted how in Grecìa Salentina, elderly speakers tend to be more reluctant to use the language with younger or new speakers. Moreover, the larger size of the community compared to the Calabrian case, along with the more widespread knowledge of the language -- albeit with different levels of competence -- leads to multiple claims of language authority, which results in various degrees of resistance to younger and new speakers, and to language change. This also affects speakers’ production, as internal monitoring and metalinguistic discussions tend to favor puristic attitudes and to delegitimise attempts to use the language. 



In contrast, older speakers in Calabria favor the use of the language and related activities regardless of the age or provenance of the addressee, provided that they are fluent. This more inclusive attitude is also reflected in people’s positive response to ongoing language revitalisation programs. Here, the smaller size of the community and the very old age of the speakers have led to less conflict over language authority, to a significant decrease in language-internal monitoring, and less resistance to language change. 



In the second half, we discussed cases of temporary variation from SMG. Interestingly, in both communities we found an increase in the number of borrowings from SMG which are not necessarily related to speakers’ competence in the minority language; instead, they are linked to the specific multilingual contexts in which the language is used, and crucially to specific characteristics of the addressee, among which language competence in Greek and provenance. This further proves the active role of the addressee in influencing speakers’ production \citep{bell2001back} and it highlights, we argue, the need to include such dynamics in linguistic analyses, as these too might lead to predictable cases of temporary variation.



To conclude, this paper has also shown how the long tradition of prescriptivism has had direct repercussions on language use and activism in parallel, albeit opposite, ways in the two communities. In the case of Greko, speakers feel judged -- by themselves and by others -- to ``fail'' at speaking Greko if they insert SMG borrowings; Griko speakers instead may avoid inserting borrowings from Salentine/Italian, as they increasingly perceive doing so as failing to speak ``Griko''. Such widespread resistance to and monitoring of language variation and change tend to undermine efforts to maintain or revitalise Griko and Greko. In this respect, both communities seem to have reached a stalemate. 


\sloppy\printbibliography[heading=subbibliography,notkeyword=this]
\end{document} 
