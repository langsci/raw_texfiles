\documentclass[output=paper]{langscibook} 
\ChapterDOI{10.5281/zenodo.7446971}
\author{Tomasz Wicherkiewicz\affiliation{Adam Mickiewicz University Poznań}}
\title{Language variation, language myths, and language ideology as constructive elements of the Wymysiöeryś  ethnolinguistic identity}
\abstract{Wymysiöeryś is a highly endangered language spoken today the last few speakers and a growing number of new speakers in the small town of Wilamowice
% in Southern Poland.
After two generations of language decay and attrition,
% , caused by the sociopolitical consequences of World War II,
the language (community) has recently undergone intensive revitalisation processes.

In the past, the town Wilamowice formed a part of the so-called Bielsko-Biała linguistic enclave (\textit{Bielitz-Bialaer Sprachinsel}), which had its roots in the massive German colonisation in the 12/13\textsuperscript{th} century and created and/or populated several villages and towns in the border areas of Silesia and Galicia. In modern times, Wilamowice has constituted either a peripheral or an entirely distinct exclave (out) of the \textit{Sprachinsel}.

% Frequently,
The Wymysiöeryś language has been classified as a colonial variety of East Central German. Both sociolinguistic research and historical records indicate, however, that at various periods, different ideas on the origin and identity of the community have been shared and uttered by and on the Wilamowiceans and Wymysiöeryś. Such ethnotheories of provenance, including some folk linguistic evidence and myths, referred to various Germanic countries as places of origin of the first settlers.
% and were presented in various forms and instances by local men of letters or amateur investigators, who provided numerous facts of intra-, extra-, and folk linguistic character.

From a contact linguistic perspective, the microlect of Wilamowice has certainly undergone interactions of various types and intensities with Polish (and its varieties) and standard High German. The evidence of such contacts, shift or even hybridisation can be found in all subsystems of the microlanguage; however, it is a rather perceptual dialectology and ethnoscience perspective of language variation which has been adopted in the present outline.
}

\IfFileExists{../localcommands.tex}{
 \addbibresource{../localbibliography.bib}
 % add all extra packages you need to load to this file

\usepackage{tabularx,multicol}
\usepackage{url}
\urlstyle{same}

\usepackage{listings}
\lstset{basicstyle=\ttfamily,tabsize=2,breaklines=true}

\usepackage{langsci-basic}
\usepackage{langsci-optional}
\usepackage{langsci-lgr}
\usepackage{langsci-osl}
% \usepackage{./langsci/styles/langsci-lgr}
% \usepackage{./langsci/styles/langsci-osl}
% \usepackage{langsci-gb4e}

\usepackage{tikz}
\usetikzlibrary{patterns,calc}
\pgfdeclarepatternformonly{south east lines}{\pgfqpoint{-0pt}{-0pt}}{\pgfqpoint{3pt}{3pt}}{\pgfqpoint{3pt}{3pt}}{
    \pgfsetlinewidth{0.6pt}
    \pgfpathmoveto{\pgfqpoint{0pt}{3pt}}
    \pgfpathlineto{\pgfqpoint{3pt}{0pt}}
    \pgfpathmoveto{\pgfqpoint{.2pt}{-.2pt}}
    \pgfpathlineto{\pgfqpoint{-.2pt}{.2pt}}
    \pgfpathmoveto{\pgfqpoint{3.2pt}{2.8pt}}
    \pgfpathlineto{\pgfqpoint{2.8pt}{3.2pt}}
    \pgfusepath{stroke}}
    
\usepackage{stmaryrd}
\usepackage{wasysym}
\usepackage{multirow}
\usepackage{caption}
\usepackage{subcaption}
\usepackage{mathrsfs}
\usepackage{qtree}

\usepackage{linguex}


 %pminos do not split footnotes
% \interfootnotelinepenalty=10000 %Footnote in Laporte chapters has to be split SN


%\DeclareIndexNameFormat{default}{%
%\nameparts{#1}%
%\usebibmacro{index:name}%
%{\index[names]}%
%{\namepartfamily}%
%{\namepartgiveni}%
% {}% L1
% {}% L2
%{\namepartprefix}% generates spurious space L3
%{\namepartsuffix}% generates spurious space L4
%}

%  {\DeclareIndexNameFormat{default}{%
%     \usebibmacro{index:name}{\index[names]}{#1}{#3}{#5}{#7}}}

%\DeclareIndexNameFormat{default}{%
%  \usebibmacro{index:name}{\sindex[nom]}{#1}{#3}{#5}{#7}}

%\DeclareIndexNameFormat{default}{%
%  \usebibmacro{index:name}{\sindex[person]}{#1}{#3}{#5}{#7}}
%\DeclareIndexNameFormat{default}{%
%\nameparts{#1} \usebibmacro{index:name}{\sindex[person]]}{\namepartfamily}{‌​\namepartgiven}{\nam‌​epartprefix}{\namepa‌​rtsuffix}}

%\newcommand{\smiley}{:)}

%\renewbibmacro*{index:name}[5]{%
%\usebibmacro{index:entry}{#1}%
%{\iffieldundef{usera}{}{\thefield{usera}\actualoperator}\mkbibindexname{#2}{#3}{#4}{#5}}}

% \newcommand{\noop}[1]{}

%remove for final
%\overfullrule=1mm

\newcommand{\tobi}[2]}}
\renewcommand{\S}[1]{\tobi{#1}{\textsc{*}}}

% this volume references
% puts: [this volume]
% already defined: \citetv
%\newcommand{\citepv}[1]{(\citeauthor{#1} \citeyear*{#1} [this volume])}
\newcommand{\citealtv}[1]{\citeauthor{#1} \citeyear*{#1} [this volume]}

%parentheses around example number
\newcommand{\pref}[1]{(\ref{#1})}

% in-text examples

\newcommand{\lnex}[1]{\textit{#1}} %target lang word
\newcommand{\lnlit}[1]{(lit.: `#1')} %literal reading
\newcommand{\lnlat}[1]{(#1)} % latinization
\newcommand{\lntrans}[1]{`#1'} %translation
\newcommand{\lnexl}[2]%
{\lnex{#1}{} \lnlat{#2}} % ex with latinization
\newcommand{\lnexlat}[3]{\lnex{#1}{} \lnlat{#2}{} \lntrans{#3}} % ex with latinization and tranl.

%ch01
\newcommand{\co}[1]{\mbox{\textbf{#1}}}

%ch09

\newcommand{\cyrbulg}[1]{\begin{otherlanguage*}{bulgarian}#1\end{otherlanguage*}}


%ch10
\newcommand{\nlp}{{\small NLP}}
\newcommand{\mwe}{{\small MWE}}
\newcommand{\rae}{{\small RAE}}
\newcommand{\lvc}{{\small LVC}}
\newcommand{\pos}{{\small P}o{\small S}}
%\newcommand{\todo}[1]{ \textcolor{red}{#1} }

%\renewcommand{\labelenumi}{\theenumi}
%\ainamefmt{{vv}{ll}{, ff}{, jj}} % fullname

\newcommand{\biberror}[1]{{\color{red}#1}}

\newcommand{\osenovaitem}{--~}
 %% hyphenation points for line breaks
%% Normally, automatic hyphenation in LaTeX is very good
%% If a word is mis-hyphenated, add it to this file
%%
%% add information to TeX file before \begin{document} with:
%% %% hyphenation points for line breaks
%% Normally, automatic hyphenation in LaTeX is very good
%% If a word is mis-hyphenated, add it to this file
%%
%% add information to TeX file before \begin{document} with:
%% %% hyphenation points for line breaks
%% Normally, automatic hyphenation in LaTeX is very good
%% If a word is mis-hyphenated, add it to this file
%%
%% add information to TeX file before \begin{document} with:
%% \include{localhyphenation}
\hyphenation{
    Beck-man
    Ngu-yen
    back-chan-nel
    back-chan-nels
    mo-not-o-nous
    ste-reo-typ-i-cal
}

\hyphenation{
    Beck-man
    Ngu-yen
    back-chan-nel
    back-chan-nels
    mo-not-o-nous
    ste-reo-typ-i-cal
}

\hyphenation{
    Beck-man
    Ngu-yen
    back-chan-nel
    back-chan-nels
    mo-not-o-nous
    ste-reo-typ-i-cal
}

 \togglepaper[11]
}{}

\shorttitlerunninghead{Variation, myths, and ideology in Wymysiöeryś  ethnolinguistic identity}
\begin{document}
\shorttitlerunninghead{Variation, myths, and ideology in Wymysiöeryś  ethnolinguistic identity}
\maketitle
\section{Introduction} 
\subsection{Concepts}
Wilamowice is the unique home to the speech community of Wymysiöeryś (ISO \textbf{wym}),\footnote{At the request of Tymoteusz Król, a young language activist from Wilamowice, the US Library of Congress added Wymysiöeryś to the register of languages in July 2007. Later, the International Organisation for Standardisation registered it under the ISO 639-3 code wym. 
At the turn of the 20\textsuperscript{th} century, efforts to obtain ISO codes became one of the procedures aimed at recognizing the linguistic status of language varieties.} a severely endangered Germanic microlanguage, now spoken by several tens of users, both the local last speakers and a growing group of new speakers (in a 2009 UNESCO report, Wymysiöeryś was referred to as "seriously endangered" and almost extinct).

Usually, the term \textit{microlanguage} describes a language variety used by a numerically small and geographically or culturally isolated speech community. The geographical isolation may result from, e.g., peripheral, transborder, or insular location, while the cultural one from religious, ethnic, or social factors. In the case of many \textit{speech microcommunities}, the criteria of territorial and social separation have been of decisive importance for both the internal and external identity of the group. 

The term \textit{microlanguage} is used here in its most neutral possible sense, although relating to two other terms used in the Slavic and Germanic sociolinguistic tradition: \textit{literary microlanguage} (Russian: \textit{микроязык}) and \textit{language enclave} (German: \textit{Sprachinsel}) respectively.

The former, introduced by Aleksandr Dulichenko (e.g. \citeyear[27]{dulichenko_sovremennoe_2006}), refers to “such a form of an existing language (or dialect), which has been used in written texts and is characterised by normalizing trends emerging as a result of the functioning of the literary writing form within the framework of a more or less organised literary and linguistic process”. Slavic linguistics uses the term mainly regarding those literary and linguistic forms that exist alongside the major Slavic languages of historically prominent nations and often possess a written form, with a certain degree of standardisation, and are used in a limited variety of circumstances and always alongside a national literary language. Genetically, Slavic literary microlanguage refers genetically to one of the major Slavic languages. According to \citet[43]{balowska_mikrojkezyki_1999}, among the key criteria for distinguishing micro- from macrolanguages are the number of users, the degree of normalisation achieved by the language code, as well as the extent of its polyvalence. Some Slavic linguists, however, question the sense of distinguishing such a category from among minority languages \citep{stern_languages_2018}.

The category of literary microlanguages has become a significant (research) domain in Slavic linguistics, and recently also in other areas of minority studies. That growing attention and applicability is probably a direct result of the growing role of written language standardisation for minority language preservation, maintenance, and revitalisation strategies. Quite underresearched remains, however, the development and course of language contacts between such \textit{literary microlanguages} and their roofing \textit{national} literary (standard) languages. Yet, in the case of close genetic linguistic relationship between the microlanguage and its \textit{Dachsprache},\footnote{\textit{Dachsprache} (German for ‘roofing language’, or ‘umbrella language’) is a term proposed by Heinz Kloss in reference to a language form that serves as a standard language in a country (or another polity with the same standard literary language) above other lects/idioms, regardless of their genetic affiliation or position in the actual dialect continuum (cf. e.g. \citealt{kloss_abstand_1967}).} language contacts have often been neglected by researchers. On the other hand, contacts between a microlanguage and its \textit{roofing} literary national standard might be of a completely different range when the two lects are not close genetic cognates, as is the case between the substantially Germanic Wymysiöeryś and its modern \textit{Dachsprache} Polish, especially when considering other language contacts of the microlanguage in the past (with Silesian German, standard German, Polish dialects, etc.)

Admittedly, Wymysiöeryś has not been extensively studied as a \textit{literary microlanguage sensu stricto}, yet the role of Wilamowicean literature and literacy in its modern history (and historical sociolinguistics) has been quite crucial. It is the growing corpora of Wymysiöeryś literary texts and on Wilamowice speech community that keep developing into the main source of language material for comparative or contact linguistics, as well as sociolinguistics of the speech community, including the historical analysis of language ideologies and their impact upon the actual language system of Wymysiöeryś.\footnote{For an outline of literary output in Wymysioryś, cf. \citealt{wicherkiewicz_mikroliteratura_2019}.}

Another keyword essential for researching and understanding the role of language variation in the development of Wilamowice’s language ideology is \textit{Sprachinsel} (German for ‘language enclave’). However, the term and concept of \textit{Sprachinsel} is still problematic, mainly due to its strongly ideological contents and methodological context in the past (cf. below). Numerous attempts which have been made to devise a definition are closely connected with certain research traditions on the one hand, and with the history of individual \textit{Sprachinseln} on the other. A widely accepted definition among German linguists was suggested by \citet[901]{wiesinger1983deutsche}, who asserted that language islands were either discrete or areal, relatively small, enclosed linguistic settlements situated within a relatively large area where another language is used. Other definitions have stressed the aspects of isolation and \textit{roofing} by a majority language, as, e.g., \citet[55]{protze_erforschung_1995} “language islands are mainly characterised by minimal contact with not only the motherland but also the surrounding speech community. The minority language is roofed by the standard variety of the majority language and the communities are linguistically and culturally isolated)”. 

As summed up in Claudia M. Riehl's (\citeyear[333]{riehl_discontinous_2010}) outline on \textit{Sprachinseln} (interpreted as \textit{discontinuous language spaces}), “difficulties in defining a Sprachinsel are primarily due to the fact that the term cannot be readily detached from the historical-political constellations and social changes connected with the emergence of language islands (i.e., the colonisation of Eastern Europe and the New World)”. Nonetheless, the concept of \textit{Sprachinsel} might be indispensable for a better discernment of both the non-existent \textit{Bielitz-Bialer Sprachinsel} macro-enclave and its erstwhile exclave of Wilamowice.

A combined, thus interdisciplinary, definition by Klaus \citet{mattheier_methoden_1996} may be quoted here: 

\begin{quote}
    “A language island is a linguistic community which develops as a result of disrupted or delayed linguistic cultural assimilation. Surrounded or enclosed by a linguistic and/or ethnic dominant culture, such an island is made up of a linguistic minority that has become separated from its original roots; and, as a function of its unique socio-psychological disposition (i.e., its ``island mentality'') is held separate and apart from the majority culture with which it maintains tangential contact (\ldots)” 
\end{quote}

The main characteristic of the \textit{delayed assimilation} is based, among others, on the speakers’ attitude toward the linguistic majority. Therefore, language islands exist “rather in the minds of speech islanders than on the landscape” \citep[106]{mattheier_theorie_1994}. 

\subsection{Location}
The town of Wilamowice (\textit{Wymysoü} in Wymysiöeryś) is currently situated in the Southern Poland province of Silesia, county of Bielsko-Biała, and inhabited by ca. 3 thousand people. The town is also the administrative center for the eponymous commune with the total population of ca. 18 thousand. Worth stressing, however, is the actual lack of cultural and linguistic continuity with the other settlements in the municipality, with one exception the village of Stara Wieś (\textit{Wymysdiöf}), referred to as a medieval sister settlement of Wilamowice.

The community was founded most probably around 1300 as a result (of multiple waves) of the German eastward expansion, which could have also included migrants from Germanic-speaking lands of Flanders, Friesland, or even Romance Wallonia. The colonists established a cluster of settlements circumjacent to what was to become the towns of Bielitz/Bielsko and Biała/Biala, on both sides of the river Biała/Bialka, which constituted a centuries-old bunch of boundaries between various polities or macroregional districts, as duchies, dioceses, kingdoms, crown-lands, or state borders between partitioners of Poland.\footnote{For a detailed outline of socio-cultural and glottopolitical history see: 

\url{http://inne-jezyki.amu.edu.pl/Frontend/Language/Details/10} and 

\url{http://inne-jezyki.amu.edu.pl/Frontend/Language/Details/11} (30 December, 2020).}

The colonies developed into what used to be called the \textit{Bielitz-Bialaer Sprachinsel} in the German historical dialectology, i.e., a mixed urban-rural complex with its own cultural profile and a dialect-cluster consisting of several subvarieties as markers of both extra-and intra-group identities. The enclave broke up into the \textit{Bielitz-Biala Sprachinsel} proper and Wilamowice (as a secondary sub-exclave) as a consequence of Polonisation of some villages such as Pisarzowice, which has since separated Wilamowice from the rest of the Sprachinsel until the 1940s. The majority populations of those two towns were German-speaking, as were the inhabitants of all villages which formed the proper \textit{Sprachinsel}. However, most of those who had inhabited the Bielsko-Biała language enclave were aware that the language varieties they spoke were different from the standard \textit{Hochdeutsch}. The varieties shared the endolinguonym \textit{Pauerisch/Päuersch}, referring to the rural, \textit{peasant} character of the ring of villages which surrounded the urban multilingual core of Bielsko and Biała (those two towns were merged twice in their history: 1941-1945 by the Nazi German occupants, and in 1951 now by the Polish administration). Always a microcosm in itself, either as an autonomous duchy or located at the boundaries of Prussia and Austria, who had partitioned Poland in the 18\textsuperscript{th} century, Bielsko-Biała \textit{sensu largo} was long considered an archetypic paradigm of a German \textit{Sprachinsel}. In fact, the varieties under concern formed a local dialectal continuum, with the northeasternmost extreme in Wilamowice. When the continuum was broken, and Wilamowice got separated territorially and culturally (probably in the 18\textsuperscript{th} century) from the rest of the enclave, the village of Hałcnów/Alzen/Alca\footnote{Polish/German/Hałcnoian name of the village; nowadays a quarter of the Bielsko-Biała conurbation.}
became the geographically and culturally nearest extreme of the German \textit{Sprachinsel}. The relations with the inhabitants of its “sister” community of Hałcnów became the closest ‘counter-reference’ of Wilmowicean identity until the end of World War II. The two speech communities, two settlements, and two language varieties shared quite a lot of linguistics and extralinguistic characteristics, but it is the German identity of Hałcnowians that became a decisive factor of their doom after 1945. 

As described by \citet[10]{metrak_wymysorys_2019} in his contrastive-comparative study of those two communities:

\begin{quote}
    ``(\ldots) modern national conflicts arrived in the region when Poland regained independence in 1918. The whole Bielsko-Biała enclave became a part of the Polish state, but as one of the centers of German minority it posed a problem for the state policies. Both Polish and German activists tried to convert the local population to their cause. In the interwar period, German nationalist-oriented scholars and ideologists treated the linguistic enclave inhabitants as exemplary ``Arch-Germans'', who have been struggling for centuries to preserve their identity surrounded by Slavic ``barbarians''.
    
    Most of the German population of Bielsko-Biała fell for the Nazi propaganda (as seen in the voting results of the Young-German Party),\footnote{\textit{Jungdeutsche Partei in Polen} -- a National Socialist political party founded in 1931 by members of the German ethnic minority in Poland.} and the German army arriving to Bielsko-Biała on September 3\textsuperscript{rd}, 1939 was greeted as liberators. The inhabitants of Hałcnów, however, traditionally supported the Christian Democratic Party. When the war was lost for Germany, some of the people evacuated with the fleeing Wehrmacht, others awaited the advancing Red Army. When in February 1945 Soviet troops marched into the city, its German population was subjected to harsh persecution: arrests, murders, rapes, and looting. Out of almost 50 thousand Germans of Bielsko-Biała, only a few «indispensable» specialists or pro-Polish activists were allowed to stay as the rest was forcibly resettled to Germany.
    
    The only settlement of the former enclave whose inhabitants were not officially persecuted was Wilamowice, due to their non-German identity. (\ldots) The lack of state-organised repression has not stopped discrimination on the local level, often performed by Poles from neighboring villages, envying the Vilamovicean wealth.''
\end{quote}

The very economic wealth of Wilamowicea(ns) has been a decisive factor in the modern history of the development of a separate Sprachinsel-identity and its microlanguage ideology.

\section{Classifications and Myths of Origin}
Genetic classifications and scientific taxonomies rarely coincide with the folk linguistic theories of language origin and language variation, or with perceptual classification of language varieties (lects). Furthermore, microlanguage communities have been both objects and subjects of quite repetitive discourses on \textit{language-or-dialect} status, which actually followed and resulted from the unending debates and agendas concerning the national declarations of communities without their (kin) nation-states. These issues have actually been most visible and influenced the modern history of the entire Central-Eastern Europe, as well as abundant individual cases within.

Wilamowice, however, thanks to an exceptional tangle of historical, cultural, political, linguistic, economic, and even individual developments, happened to be the only survivor on that Central-European battlefield of languages, nation-states and \textit{Sprachinseln}, macro- and microideologies, into the 21\textsuperscript{st} century, one of the most salient factors being the overt and covert, documented and mythical history of the microlanguage.

The actual constellations of languages, patterns of individual and community multilingualism and polyglossia in Wilamowice, as well as in (other parts of) \textit{Bielitz-Bialer Sprachinsel} still require thorough research.\footnote{Actually, these questions constitute one of the main topics of the research project \textbf{Multilingual worlds – neglected histories. Uncovering their emergence, continuity and loss in past and present societies} (MULTILING-HIST), to be started in 2021 by Justyna Olko and her team (including the present author), and financed by a European Research Center grant.}
There have been, however, some earlier contributions to the historical sociolinguistics of Wymysioryś that focused on the ethnotheory of origin, shared by and imposed upon the community. To investigate and understand the current position and self-identification of Wilamowiceans, one should sort and analyze the myths and theories that have referred to and tried to categorise Wymysiöeryś as a language (variety).

Some of them have already been researched earlier by \citet{morciniec_flamische_1984, ryckeboer_flamen_1984, morciniec_stellung_1995, wicherkiewicz_making_2003, ritchie_considerations_2012, wicherkiewicz_researching_2016} or \citet{wicherkiewicz_awakening_2018}.

A cutting-edge, comprehensive study of linguistic micro- and macroideologies of and in Wilamowice was compiled by \citet{chromik_mikro_2019}.\footnote{In Polish, although an English-language edition is expected.}
Using the approach based on Michael Silverstein’s methodology, Chromik set down a monograph of linguistic ideologies understood as a bridge between linguistic and social structures, proposing a twofold typology of linguistic macroideologies (which attempt to hegemonise the way languages are perceived), and microideologies (as an alternative to the former):

\begin{quote}
    “Due to specific conditions, they are easier to observe in communities using minority languages or nonstandard forms of dominant languages. Unlike macroideologies, they are predominantly transmitted orally within families. As a consequence, they often remain invisible to sociologists, linguists, or historians. They are easier to observe when an ethnographic method, based on direct interaction with people, is used.”\footnote{Quoting the English summary of unpublished \citealt{chromik_mikro_2019}.}
\end{quote}

During the long 19\textsuperscript{th} and 20\textsuperscript{th} centuries, these language microideologies in Wilamowice have frequently been directly interwoven with (mythical) ethnotheories which pertained to the origin of the settlers in Wilamowice. Moreover, during the 20\textsuperscript{th} century, they also got tangled with and into the terminological and taxonomical disputes on the linguistic status of Wymysiöeryś (recently also in the never-ending and unproductive debate on \textit{language-vs.-dialect} status). 

Therefore, it is language ideology that became an indispensable factor and link to the understanding of the sociolinguistic processes \textit{sensu largo} within the Wilamowicean community, as indeed they significantly influenced the role, prestige, status and attitudes to the native language variety. Even the systemic language variation might be considered not only a consequence but, to some extent, even a product of language ideologies, as most paradigms of historical sociolinguistics do not ignore such variables as language ideologies, language attitudes or social networks as markers or factors of language variation.

\section{Language ideologies as variables of “Wilamowiceanness”}
In the last two centuries, the community of Wilamowice has constantly been subject to constellations of internal and external language ideologies that resulted in the ethnolinguistic choices and declarations, especially during the long 20\textsuperscript{th} century. Many linguists (as well as dialectologists and historians) often associated the Bielitz-Bialaer Sprachinsel with Silesian German. It must be remembered, however, that the emergence of those varieties is to be ascribed to the medieval settlers who came from various regions of today's Germany, not to speak about adjacent areas. Silesian, thus, developed an admixture of dialects which consisted of elements from various German varieties. Initially, it probably had been much more diversified than it was in the 20\textsuperscript{th} century. Throughout the centuries of co-existence of various groups, their lects converged. The scholars of language and history did some attempts to reach into the depths of history and determine what particular German dialects influenced the \textit{Bielitz-Bialaer Mundart(en)}. Some of them (e.g., \citealt[192]{wagner_beeler_1935}) referred to the apparent influence of Thuringian and Upper Saxon as the base of Silesian German varieties in the cluster of Bielsko-Biała.

The situation of Wymysiöeryś was different. Actually, every linguist (including philologists, dialectologists and amateur \textit{Heimatkundige}) interested in the (linguistic) history of Wilamowice attempted to explain or determine the origins of Wymysiöeryś. Frequently, those explanations were either based on folk theories or created new myths themselves. 
One of the proposed hypotheses indicated Schaumburg in north-western Germany (Lower Saxony) as a possible place of origin of the first Wilamowiceans. The theory found a quite a wide distribution at the beginning of the 20\textsuperscript{th} century. Józef Latosiński, a teacher and schoolmaster in Wilamowice, in his massively popular \textit{Monografia Miasteczka Wilamowic} (based on authentic sources; \citeyear{latosinski_monografia_1909}), argued that: 

\begin{quote}
    “ (\ldots) the first Wilamowiceans did, indeed come from there; firstly, because of the local dialect which had both the native Low German element and Dutch and Anglo-Saxon features as well, and secondly, because to this day the textile industry has been thriving in the Principality of Schaumburg-Lippe (\ldots)”
\end{quote}

J. Latosiński’s monograph has effectively influenced and to some extent constructed the collective identity of Wilamowice in the face of nationalisms breaching through Central Europe and restructuring the identities of lands and ethnic groups (more in: \citealt[23ff]{wicherkiewicz_researching_2016}, and \citealt{chromik_mikro_2019}). The importance of Latosiński’s thesis has been crucial for the Wilamowice’s group identity within the 20\textsuperscript{th} century, emphasised by the fact that the \textit{Monografia} was reprinted in 1990.

What Latosiński’s input meant for constructing the Polishness of Wilamowice(ans) was Walter Kuhn’s mission-work in respect to their Germanness. Born in Bielitz, W. Kuhn was a prominent and productive ethnohistorian of the German \textit{Ostgebiete} (Eastern territories), which included the areas colonised (also discretely, as scattered \textit{Sprachinseln}) by Germans since the Middle Ages (cf. \citealt[23ff]{wicherkiewicz_researching_2016}). His \citeyear{kuhn_deutsche_1934} monograph on \textit{Deutsche Sprachinselforschung: Geschichte, Aufgaben, Verfahren} became an actual foundation of the entire discipline and research paradigm of \textit{Sprachinselforschung}, an interdisciplinary study of German language exclaves in Central and Eastern Europe, with Bielitz-Biala and vicinities as the actual archetype of a \textit{Sprachinsel}. Himself an involved and convinced Nazi, professor of Universities of Breslau and Hamburg, Kuhn did his best to prepare and later support the ethnic cleansings of local non-German populations and the settlement of German colonists to Germanise Central and Eastern Europe. His texts on Wilamowice endeavored to explicitly display the Germanness of Wilamowice, considering any other theories and ethnic myths as inventions of nationalist Polish propaganda.\footnote{Nevertheless, Kuhn’s articles on Wilamowice may still be considered an inexhaustible source of ethnographic and folkloristic information, as may works by several other Nazi-engaged German fieldworkers (e.g. Alfred Karasek or Hertha Strzygowski) – cf. \citet{wicherkiewicz_researching_2016} and \citet{chromik_mikro_2019}.}

Many sources and records have revealed that since the beginning of modern European state nationalism, Wilamowiceans rarely overtly considered themselves Germans. During modern history, i.e., as Habsburg Austrian subjects, they hardly had to declare their ethnicity; instead, the identity of belonging to the Austrian political nation has been expressed for generations: “Usually they said ‘wir sein Esterreichyn’, \textit{we are Austrians}, because we belong to Austria. All inhabitants of the Monarchy were perceived here as Austrians, and it did not depend on the language used by them" (quotation from \citealt[162]{filip_flamandowie_2005}, after \citealt[96]{chromik_wilamowice_2016}). The linguistic/dialectal facet of Austrianness was not a stable and independent marker either, as the dialect continuum, referred to as Silesian German (\textit{Schlesisch, Schläsisch/ Schlässch, Deutschschlesisch}), covered the territory administered by various polities in modern times. The \textit{Bielitz-Bialaer Sprachinsel} and the exclave of Wilamowice constituted the easternmost confines of that Silesian-German dialectal continuum, where intense language contact with the Polish language (continuum) left a remarkable imprint. The German(ic)-Polish language contact was much stronger in Wilamowice than, e.g., in Hałcnów/Alzen/Alza, not to mention much less “Polonised” varieties spoken by the much more nationally German-oriented urban population of the nearby Bielitz/Bielsko. The religious factors were also not without significance; worth reckoning is the prevailing Catholic profile of Wilamowice and Hałcnów, while Bielsko and Biała also had considerable shares of Protestant and Jewish population.

Subsequently, Wymysiöeryś has not been considered by its speakers’ community just a language variety of German, as standard German was regarded as a separate language, and as such was hardly present in their everyday lives, with exceptions of the Austrian administration or Nazi occupation times during World War II. Some of the testimonies might have referred to the German substrate of Wymysiöeryś; its “mixed” nature was, however, apparent to external observers, as, e.g., the Moravian-German minister Karl Franz Augustin, who commented in his 1842 \textit{Jahrbuch oder Zusammenstellung geschichtlicher Thatsachen, welche die Gegend con Oswieczyn und Saybusch angehen}: 

\begin{quote}
    “The commune of Wilamowice has a peculiarity. The German language has remained here until the present day. From the immemorial times, despite the fact that all round Polish was spoken in the town, there are remains of the old Wandals and Burgunds, who saved their German language, which naturally is now mixed with Polish” (quoted after \citealt[96]{chromik_wilamowice_2016}).
\end{quote}

What appeared \textit{proto}-Wandal or Burgund to Augustin sounded “like Yiddish or English” to Jan \citet{lepkowski_notatki_1853}, professor of the University in Cracow, who reported:

\begin{quote}
    “A peculiar vernacular! (\ldots) In spite of centuries-old language contacts between Wilamowice and Hałcnów from one side and the Polish and Silesian Slavdom from the other, the former have totally preserved their Gothic asperity; this is a Germanic dialect, caught and ossified in its medieval form. Some people consider the settlers Dutch, who had arrived here during the earliest religious unrests (\ldots). Apart from the settlers themselves, nobody understands that language. They speak Polish or German to strangers; they pray in Polish; nevertheless, they have extraordinarily preserved their embalmed dialect for ages.”
\end{quote}

Another intriguing message concerning the (linguistic) origin of Wilamowiceans was expressed in a collection of folk (or rather folklorised) poems from the region by an amateur ethnographer from Biala, Dr. Jacob Bukowski. In his \citeyear{bukowski_poems_1860} volume, the poem \textit{A Welmeßajer ai Berlin} tells the story of a merchant from Wilamowice in Berlin, who tries to sell woven fabrics, shouting:

\begin{quote}
    “Buy (\ldots) beautiful coutils from Wilamowice!”; “he spent a week there” (\ldots) // but he was hardly understood by the foreigners // as he was thought to be from England. // Nothing strange: as, indeed, that is where // the Wilamowiceans came from.”
\end{quote}

It is this “embalmed dialect”, “mixed with Polish” and exotic “from-England”, that started functioning as the main factor of the ethnolinguistic identity of Wilamowice in the era of growing nationalism brought to Central Europe in the second half of the 19\textsuperscript{th} century.


According to \citet[96]{chromik_wilamowice_2016}:

\begin{quote}
    “(\ldots) in the nineteenth century and at the beginning of the twentieth century, Wilamowice drew attention of nationalistically oriented scholars from the nearby big and rich town of Bielsko (Bielitz). They saw the inhabitants of Wilamowice as \textit{Supergermans}, who, although "devoid of any contact with «Germanness», retained their archaic language in the period of German national weakness, [when] the whole German foreground at the edge of the Carpathian Mountains started to yield to Polonisation." (quotations from \citealt[]{filip_flamandowie_2005}) However, if we listen to the voices of Wilamowiceans of the same period, we would see a portrait of a phenomenon that is very seldom nowadays, a prenationalist community. Wilamowiceans, with their distinct language and culture, were fully aware of their separateness.”
\end{quote}

The previously mentioned amateur historian and community leader, Józef Latosiński, was probably the most influential and remembered propagator of an “embedded” national character of Wilamowiceans – according to his worldview, they could maintain and cherish their German(ic) language, enclave and ancestry, only if overtly and massively “superstructured” by Polishness at the national level. The overt national plan by Latosiński was preceded by a covert artistic form by the father of literary Wymysiöeryś, Fliöra-Fliöra.\footnote{Flora-Flora (or Fliöra-Fliöra, according to the current standardised orthography of Wymysiöeryś) the nickname and later penname of Florian Biesik. His literary output and impact have already been documented and analyzed quite thoroughly, by e.g. \citet{wicherkiewicz_making_2003, wicherkiewicz_researching_2016} or \citet{wicherkiewicz_mikroliteratura_2019}. Polish versions (and a selection of English translations) of F. Biesik’s texts can be found in \citet{wicherkiewicz_making_2003} and \citet{wicherkiewicz_mikroliteratura_2019}.}

Florian Biesik (1850-1926), the first and most famous Wymysiöeryś-language writer and patron of the modern Wymysiöeryś cultural and linguistic identity, ceaselessly asserted non-German origins of the first Wilamowiceans. The major part of the foreword to his most important work, \textit{Uf jer wełt} (‘In the other world’, \citeyear{biesik_welt_1924}), described the odyssey of the predecessors of today's Wilamowiceans. It also depicted the folk linguistic myths promoted by Biesik in order to replace the German plots in the local ethnolinguistic history by new nation-like links and soft language ideologies, connecting Wilamowice(an) with Friesland, Low Countries, or even with the Anglo-Saxons:

\begin{quote}
    “(\ldots) with that language they conquered England and then came back to the mainland, to Friesland and to Wilamowice. Therefore, this language, being the oldest one, is the father of other living Germanic languages of today, i.e., English, Dutch, German, Swedish, and although it served for the transition among them, it has been left isolated (\ldots) The Wilamowicean language has not had much luck; as the oldest one, it served as the basis for other languages, like earlier Latin, which had served as the basis for the Romance peoples (\ldots)”
\end{quote}

Florian Biesik saw himself as a carrier of language ideology, whose mission was both to instruct his compatriots on the “correct” version of their history, and to strengthen the history by means of a Wilamowicean literary microlanguage, following the steps of the \textit{founding fathers} of Europe’s major languages:

\begin{quote}
    “(\ldots) Everything in this world depends on chance; and so the speech of the Tuscan peasants accidentally found its great poet Dante, who (\ldots) became the father of the Italian language, or as Luther, with his translation of the Bible (\ldots) became the father of modern German. (\ldots) and so it came to my mind to find all those dear friends (\ldots) in the other world, and at least there, to talk to them in the language of my childhood, i.e. the thousand-year-old Anglo-Frisian-Saxon (or Gothic?) language (\ldots)”
\end{quote}

The structure of native bilingualism in the community did and was to exclude German, as Biesik stated when referring to his own family members: “For half a century (\ldots) I have been reading my mother's letters in Polish (around 50, and none in German, as if it had been implied by my brother Dr. Mojmir)”.

Consequently, the ethn(ograph)ic and linguistic archaicity of German Wilamowice was to be counterbalanced by their exotic \textit{Occidentalism} (understood as a “West-looking” counterpart of E. Said’s \textit{Orientalism}) stressed by the supporters of the Polish national ideology, such as F. Biesik or J. Latosiński. Through the mythical \textit{Occidentalisation}, they would have eagerly seen Wilamowiceans as \textit{Gente Germanici, natione Poloni} (“Germanic by origin, Polish by nationality”), regarding ethnic constructs enforced during the long-lasting cultural and political colonisation of its Eastern territories by the Polish state, where Slavic Ruthenians were encouraged to accept the covertly assimilating status of \textit{Gente Ruthenus, natione Polonus}, while the Balto-Slavic Lithuanians – that of \textit{Gente Lithuanus, natione Polonus}. The embedded ethnic-national identification is still lacking a proper corresponding label as far as (folk) linguistic classification is concerned. Therefore, a mixed Germanic-Polish \textit{variation} would be, and was, the most desirable categorisation from the Polish perspective, while the German linguistics and, primarily, language ideology required the paradigm of \textit{Sprachinsel(forschung)}.

Some linguists, both academic and amateur, however, tried to avoid an ideological context in their studies of Wymysioryś. In that respect, one of the most intriguing actors on the stage of Wilamowice’s ethnolinguistic history was the above-mentioned Dr. Mojmir. Hermann Biesik (1874-1919),\footnote{H. (Biesik) Mojmir’s most complete biography is available in Polish by \citet{krol_hermann_2020}.} F. Biesik’s much younger brother, a devoted physician and philologist, eventually changed his family name to a more Slavic-sounding \textit{Mojmir}, as a result of a fraternal conflict in the Biesik family, and/or the protest against Austria’s Germanisation policy. Hermann Mojmir’s most monumental work of enduring value (also in linguistics, ethnolinguistics, and revitalisation of Wymysiöeryś) became his two-volume \textit{Wörterbuch der deutschen Mundart von Wilamowice} (\citeyear{mojmir_worterbuch}). That straightforward classification of the titular \textit{German dialect of Wilamowice} was probably suggested by (either of) the academic supervisors and coeditors: Prof. Adam Kleczkowski (Vol. I) or Heinrich Anders (Vol. II). The former, professor of German linguistics at the University of Poznań and Jagiellonian University in Cracow, published a two-volume grammatical reference handbook of Wilamowicean – with the title (\citealt{kleczkowski_dialekt_1920} and \citeyear{kleczkowski_dialekt_1921}) referring neutrally to the \textit{dialect of Wilamowice in western Galicia}. The former, H. Anders (otherwise a researcher of the medieval German varieties in Poznań/Posen region, and the first editor of selected poem texts by F. Biesik), referred to Mojmir’s dictionary as:

\begin{quote}
    “compiled by a native-speaker and completely edited by A. Kleczkowski. (\ldots) Despite the eventful history, the inhabitants of Wilamowice and the surrounding villages have preserved their Silesian dialect with an East-Central German character, very little influenced by the High German written language and by Austrian German, a dialect containing numerous words and word-forms from Old Polish and from Polish.” 
\end{quote}

Thanks to the works by H. Mojmir, A. Kleczkowski, and H. Anders, Wymysiöeryś became quite famous and one of the best-documented \textit{Sprachinsel}-microlan\-guages in the world of Germanic linguistics. In \citeyear{weinreich_yiddish_1958}, Uriel Weinreich presented his study on “the differential impact of Slavic upon Yiddish and \textit{Colonial German} in Europe”, where Wilamowice and Wymysiöeryś served as one of the main case studies of the latter, and were presented from a perspective of contact linguistics and \textit{bilingual dialectology}. Weinreich tried to draw also some “folkloristic and ethnographic parallels” between the Yiddish-speaking communities and German \textit{Sprachinseln} in Central-Eastern Europe. Nevertheless, \citet{weinreich_yiddish_1958} also criticised the \textit{Sprachinselforschung} because of its ideological background, too:

\begin{quote}
    “Germanistic research, even at its best, has been preoccupied with the origins and chronology of German eastward migration and with the patterns of dialect mixture and leveling, and it has therefore sought out the archaic, Germanic elements of language and culture, while new acquisitions, seemingly irrelevant to migration history, were deemphasised or overlooked (\ldots)
    
    The physical destruction or dislocation of the societies under discussion [including Wilamowice] rules out the possibility of fresh field work and forces the student to rely to a considerable degree on materials published before the Second World War. For the problem at hand, this turns out to be a major handicap for the ideological framework;”
\end{quote}

Weinreich's \citeyear{weinreich_yiddish_1958} work, however, seems to remain unknown to the community of Wilamowice and their researchers, most probably because of the lack of academic communication between the West and East and/or the official \textit{désintéressement} in any German(ic) heritage in post-War Poland (cf. \citealt{wicherkiewicz_impact_1993}).

For the long years of prevailing communist-nationalist ideologies, between the end of World War II and the political changes of the 1980s, both Wymysiöeryś and its research must have fallen into oblivion by the Polish scholarship and academia. Except for occasional studies (as \citealt{weinreich_yiddish_1958}), the only perspective Wilamowice was researched from was that of the German linguists and ethnographers (as Walter Kuhn), who acted mainly as members and representatives of the \textit{Landsmannschaften der Heimatvertriebenen} (associations of expellees). They actually continued research on “the Supergermans” from Wilamowice/Wilmesau, \textit{Bielitz-Bialer Sprachinsel}, and other \textit{altschlesische Sprachinseln}.

The post-War history of Wilamowice community was then studied either from German exile perspective, or subject to the official communist-nationalist propaganda in Poland. Only recently, several projects have addressed the problems of acute language abandonment, community trauma, and the resulting sociolinguistic developments, including the language micro- and macroideologies at stake (e.g. \citealt{chromik_mikro_2019}).

The critical developments that influenced (also mythically and ideologically) the contemporary history of Wymysiöeryś were: signing of the \textit{Volksliste} by Wilamowiceans during the Nazi occupation, massive appropriation of their properties by the neighbors based on an old-established economic grudge, direct ethnic cleansings in the former \textit{Bielitz-Bialer Sprachinsel}, or the official ban issued on Easter 1945: 

\begin{quote}
    “(\ldots) from now on, we ban any use of the local dialect – also in family and private situations, the forgoing concerns also wearing the distinct folk costumes. Those who do not comply with the present ban will be brought to severe punishment; since it is the high time to put stop to any distinctness and its lamentable results” \citep[183--184]{filip_flamandowie_2005}.
\end{quote}

A steady and systematic language loss eradicated Wilamowicean from most, and eventually from all, its traditional domains, from family circles to community life. Wymysiöeryś lost its communicative and integrative functions, with a few exceptions including individual curricula, as, e.g., grandchildren brought up solely by their grandparents, or displaced families (cf. \citealt{wicherkiewicz_impact_1993} and \citealt{wicherkiewicz_researching_2016}).

Any return to any noticeable token of Wilamowicean identity became possible in the late 1970s, when a Belgian television station made a documentary on and in Wilamowice \textit{Een dorp van Vlamingen?} [‘A village of Flemings?’, 1977].\footnote{Available: \url{https://youtu.be/ibUn82Odjpo} (1 February, 2021).} It does constitute a priceless record of the Wilamowicean language and culture of the stage; the movie has also become a constitutive factor for the common perception of Wilamowice as a forgotten and remote colony established by settlers from Flanders (or Holland). The fame of a Flemish village attracted also researchers, such as Hugo Ryckeboer (of Ghent) and Norbert Morciniec (of Wrocław), who published two extensively documented articles \citep{ryckeboer_flamen_1984, morciniec_flamische_1984}, discussing possible origins of the Wymysiöeryś language; the latter evolved his debate as \citet{morciniec_stellung_1995}.

The turn of the 1980s marked essential changes in Poland’s social, political, and economic system. The initially top-down changes were to significantly influence the position, perception and role of minority communities in the new democratic order. The country and society were to restructure their self-view of a monolingual, monoethnic, and homogenous state-nation, and the community of Wilamowice was to start an entirely new period in their history: reclamation of the regional “microethnic” identity and revitalisation of the “microlanguage”, not as a (part of any) \textit{Sprachinsel}, but as a newly re-established speech community \textit{in statu renascendi} (cf. \citealt{wicherkiewicz_awakening_2018}).

This article is not discussing any later, recent, and current developments in Wilamowice, as the actual language ideologies that are currently at stake do not focus anymore on language myths or endo- and exo-ethnotheories of origin. The new generations, who have been assuming their roles of leaders and players in the current movements (civic, local, regional, linguistic, language documentation, institutionalisation and legal recognition, etc.), are forming new paradigms. Research of Wymysiöeryś as a language system (e.g. \citealt{andrason_polish_2014, andrason_slavic_2015, andrason_vilamovicean_2015, zak_influence_2016}) and as a sociolinguistic community \citep{krol_czym_2016, neels_language_2016, wicherkiewicz_researching_2016} continues on an unprecedented scale.

This sketch, thus, may serve as actually a starting point for further research on the correlates between language variation and language ideology from the perspectives of historical sociolinguistics or perceptual dialectology.
\sloppy
\printbibliography[heading=subbibliography,notkeyword=this]

\end{document}
