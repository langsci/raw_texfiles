\documentclass[output=paper,hidelinks]{langscibook}
\ChapterDOI{10.5281/zenodo.7446959}
\author{Justyna Olko\affiliation{University of Warsaw} and Szymon Gruda\affiliation{University of Warsaw} and Joanna Maryniak\affiliation{University of Warsaw} and Elwira Dexter-Sobkowiak\affiliation{University of Warsaw} and Humberto Iglesias Tepec\affiliation{Instituto de Educación Media Superior de la Ciudad de México} and Eduardo de la Cruz\affiliation{University of Warsaw; Instituto de Docencia e Investigación Etnológica de Zacatecas} and Beatriz Cuahutle Bautista}
\title{Spanish-Nahuatl bilingualism in Indigenous communities in Mexico: Variation in language proficiency and use} 
\abstract{The focus of this paper is bilingualism in Spanish and Nahuatl from the sixteenth century until the present day, with an exploration of its scope, functions and stability. We include a historical perspective to provide the necessary background for the contemporary context, which is approached with both qualitative and quantitative data acquired during fieldwork carried out in four different regions where Nahuatl and Spanish bilingualism is present today. Of special importance for the present study is the analysis of the results of proficiency assessment in both languages, performed with the participation of members of selected Nahua communities, which represent different degrees of assimilation to Mexican identity and shift to Spanish. We conclude that due to power differentials, economic, sociopolitical and cultural pressures and discriminatory language policies, contemporary Spanish-Indigenous bilingualism at the community level is unstable and transitional.}

\IfFileExists{../localcommands.tex}{
 \addbibresource{../localbibliography.bib}
 % add all extra packages you need to load to this file

\usepackage{tabularx,multicol}
\usepackage{url}
\urlstyle{same}

\usepackage{listings}
\lstset{basicstyle=\ttfamily,tabsize=2,breaklines=true}

\usepackage{langsci-basic}
\usepackage{langsci-optional}
\usepackage{langsci-lgr}
\usepackage{langsci-osl}
% \usepackage{./langsci/styles/langsci-lgr}
% \usepackage{./langsci/styles/langsci-osl}
% \usepackage{langsci-gb4e}

\usepackage{tikz}
\usetikzlibrary{patterns,calc}
\pgfdeclarepatternformonly{south east lines}{\pgfqpoint{-0pt}{-0pt}}{\pgfqpoint{3pt}{3pt}}{\pgfqpoint{3pt}{3pt}}{
    \pgfsetlinewidth{0.6pt}
    \pgfpathmoveto{\pgfqpoint{0pt}{3pt}}
    \pgfpathlineto{\pgfqpoint{3pt}{0pt}}
    \pgfpathmoveto{\pgfqpoint{.2pt}{-.2pt}}
    \pgfpathlineto{\pgfqpoint{-.2pt}{.2pt}}
    \pgfpathmoveto{\pgfqpoint{3.2pt}{2.8pt}}
    \pgfpathlineto{\pgfqpoint{2.8pt}{3.2pt}}
    \pgfusepath{stroke}}
    
\usepackage{stmaryrd}
\usepackage{wasysym}
\usepackage{multirow}
\usepackage{caption}
\usepackage{subcaption}
\usepackage{mathrsfs}
\usepackage{qtree}

\usepackage{linguex}


 %pminos do not split footnotes
% \interfootnotelinepenalty=10000 %Footnote in Laporte chapters has to be split SN


%\DeclareIndexNameFormat{default}{%
%\nameparts{#1}%
%\usebibmacro{index:name}%
%{\index[names]}%
%{\namepartfamily}%
%{\namepartgiveni}%
% {}% L1
% {}% L2
%{\namepartprefix}% generates spurious space L3
%{\namepartsuffix}% generates spurious space L4
%}

%  {\DeclareIndexNameFormat{default}{%
%     \usebibmacro{index:name}{\index[names]}{#1}{#3}{#5}{#7}}}

%\DeclareIndexNameFormat{default}{%
%  \usebibmacro{index:name}{\sindex[nom]}{#1}{#3}{#5}{#7}}

%\DeclareIndexNameFormat{default}{%
%  \usebibmacro{index:name}{\sindex[person]}{#1}{#3}{#5}{#7}}
%\DeclareIndexNameFormat{default}{%
%\nameparts{#1} \usebibmacro{index:name}{\sindex[person]]}{\namepartfamily}{‌​\namepartgiven}{\nam‌​epartprefix}{\namepa‌​rtsuffix}}

%\newcommand{\smiley}{:)}

%\renewbibmacro*{index:name}[5]{%
%\usebibmacro{index:entry}{#1}%
%{\iffieldundef{usera}{}{\thefield{usera}\actualoperator}\mkbibindexname{#2}{#3}{#4}{#5}}}

% \newcommand{\noop}[1]{}

%remove for final
%\overfullrule=1mm

\newcommand{\tobi}[2]}}
\renewcommand{\S}[1]{\tobi{#1}{\textsc{*}}}

% this volume references
% puts: [this volume]
% already defined: \citetv
%\newcommand{\citepv}[1]{(\citeauthor{#1} \citeyear*{#1} [this volume])}
\newcommand{\citealtv}[1]{\citeauthor{#1} \citeyear*{#1} [this volume]}

%parentheses around example number
\newcommand{\pref}[1]{(\ref{#1})}

% in-text examples

\newcommand{\lnex}[1]{\textit{#1}} %target lang word
\newcommand{\lnlit}[1]{(lit.: `#1')} %literal reading
\newcommand{\lnlat}[1]{(#1)} % latinization
\newcommand{\lntrans}[1]{`#1'} %translation
\newcommand{\lnexl}[2]%
{\lnex{#1}{} \lnlat{#2}} % ex with latinization
\newcommand{\lnexlat}[3]{\lnex{#1}{} \lnlat{#2}{} \lntrans{#3}} % ex with latinization and tranl.

%ch01
\newcommand{\co}[1]{\mbox{\textbf{#1}}}

%ch09

\newcommand{\cyrbulg}[1]{\begin{otherlanguage*}{bulgarian}#1\end{otherlanguage*}}


%ch10
\newcommand{\nlp}{{\small NLP}}
\newcommand{\mwe}{{\small MWE}}
\newcommand{\rae}{{\small RAE}}
\newcommand{\lvc}{{\small LVC}}
\newcommand{\pos}{{\small P}o{\small S}}
%\newcommand{\todo}[1]{ \textcolor{red}{#1} }

%\renewcommand{\labelenumi}{\theenumi}
%\ainamefmt{{vv}{ll}{, ff}{, jj}} % fullname

\newcommand{\biberror}[1]{{\color{red}#1}}

\newcommand{\osenovaitem}{--~} 
 %% hyphenation points for line breaks
%% Normally, automatic hyphenation in LaTeX is very good
%% If a word is mis-hyphenated, add it to this file
%%
%% add information to TeX file before \begin{document} with:
%% %% hyphenation points for line breaks
%% Normally, automatic hyphenation in LaTeX is very good
%% If a word is mis-hyphenated, add it to this file
%%
%% add information to TeX file before \begin{document} with:
%% %% hyphenation points for line breaks
%% Normally, automatic hyphenation in LaTeX is very good
%% If a word is mis-hyphenated, add it to this file
%%
%% add information to TeX file before \begin{document} with:
%% \include{localhyphenation}
\hyphenation{
    Beck-man
    Ngu-yen
    back-chan-nel
    back-chan-nels
    mo-not-o-nous
    ste-reo-typ-i-cal
}

\hyphenation{
    Beck-man
    Ngu-yen
    back-chan-nel
    back-chan-nels
    mo-not-o-nous
    ste-reo-typ-i-cal
}

\hyphenation{
    Beck-man
    Ngu-yen
    back-chan-nel
    back-chan-nels
    mo-not-o-nous
    ste-reo-typ-i-cal
}
 
 \togglepaper[3]%%chapternumber
}{}

\shorttitlerunninghead{Spanish-Nahuatl bilingualism in Indigenous communities in Mexico}
\begin{document}
\shorttitlerunninghead{Spanish-Nahuatl bilingualism in Indigenous communities in Mexico}
\lehead{Justyna Olko et al}
\maketitle

\section{Introduction: Goals of the present study}

The origins of Nahuatl-Spanish bilingualism go back to first encounters between Europeans and Indigenous people living in the area controlled by the Aztec Empire in 1519. At first largely limited to individual bilingual specialised skills, the contact between the two languages and the growing pressures within the colonial system and then under the independent Mexican state (1821) gradually led to the appearance of societal bilingualism. This was particularly the case in multiethnic urban contexts during this time. In the twentieth and twenty-first century, it was also attested in rural areas. While we give importance to the historical perspective underlying current developments, the main focus of this paper is the bilingual situation of four different regions where unstable Nahuatl and Spanish bilingualism is present today. The study is based on the analysis of qualitative and quantitative data acquired during fieldwork, supplemented by historical sources.

By unstable bilingualism we mean the situation of parallel acquisition and use of the heritage language and Spanish, not exceeding two-three generations and leading, eventually, to language shift. This can be contrasted with the notion of stable bilingualism, when two languages are used – perhaps in a complementary manner and not necessarily at the same level of proficiency – for an extended period of time without either of the languages displacing the other one; such a situation has been observed e.g. in Quebec (French and English), Belgium (Dutch and French, especially in Flanders), Paraguay (Spanish and Guaraní) or in many states of India (regional languages and Hindi, and, to a lesser extent, English). Furthermore, we use the term ``individual bilingualism'' to refer to the capacity of particular individuals, such as translators, friars, Indigenous notaries and other members of local communities, to speak two languages proficiently. Consequently, we use the term ``societal bilingualism'' to refer to a widespread use of two languages by significant parts of speech communities for whom this kind of linguistic practice is part of everyday interaction and not a specialised skill.

\section{Historical context: The Colonial Period}

The Spanish colonisation of Mesoamerica, initiated by the landing of Hernán Cortés and his people on the shore of the Gulf of Mexico in the spring of 1519, created the urgent need for the development of bilingual and multilingual skills in Spanish and local languages. During the first stage of contact, the availability of Spanish-Indigenous translators was the greatest need, but with the ongoing colonisation and organisation of the European rule, the demand for individual Indigenous-Spanish bilingualism grew on both sides. Among all of the Indigenous languages spoken in sixteenth-century Mesoamerica, it was Nahuatl, belonging to the Uto-Aztecan family, that was most frequently used between local populations and Spaniards and in bilingual arrangements with Spanish speakers, e.g. in administration, courts or in efforts at Christianisation. It was also the language most often used by other non-Nahua ethnic groups, no doubt due to its central role in the Aztec Empire and other powerful states of earlier Pre-Hispanic Mesoamerica.

The Aztec imperial infrastructure collapsed and disintegrated rapidly after the arrival of the Spaniards, but the local organisation of Nahua states, called \textit{altepetl}, proved resistant to conquest and colonisation. During initial attempts to introduce their rule, the Spanish had to rely on local Indigenous organisation structures, which meant dealing directly with particular \textit{altepetl}. Such interactions contributed to the survival of preexisting entities and political-territorial units, ensuring their continued importance in the early colonial period (\citealt[63--74]{gibson1964}, \citealt[28--29]{lockhart1992}). In large urbanised zones, such as the capital city of México-Tenochtitlán, an organisational duality was introduced, with parallel Indigenous and Spanish municipal structures and organisations. Newly founded centers for European populations, such as the town of Puebla de los Angeles, tended to replicate Spanish structures more closely, despite housing significant portions of the Indigenous population.

This sociopolitical and sociolinguistic situation contributed to the Spaniards’ reliance on Nahuatl as the language of administration and religious instruction. Moreover, beyond the core area of New Spain, Nahuatl was widely used as a vehicular language by other Indigenous groups, including in southern and northern Mesoamerica. The linguistic landscape and associated power relationships were complex and varied between regions, depending on the numbers and kinds of languages spoken in the areas and the extent of local multilingualism. From the sixteenth century on, different forms of polyglossia existed in New Spain: some were brought over from Spain – including Latin, high Spanish and low Spanish – and some originated in the Indigenous world. Thus, Latin and Spanish were high varieties in New Spain, while Nahuatl occupied a higher position with regard to other Indigenous languages \citep[308--310]{parodi2010}. Nahuatl therefore served different purposes, including that of an intermediary language in translation between Spanish and other local languages, as well as being, to a certain degree, the language of direct communication between Spaniards and local Indigenous populations \citep{nesvig2012, schwaller2012}. As a result, the uses of Nahuatl in colonial New Spain were by no means limited to members of Nahua communities (\citealt[669--670]{yannanakis2012}, \citealt[739--758]{nesvig2012}). For example, a large number of mestizos and creoles learned the language due to continuous daily contact with the speakers in households where Indigenous servants and workers were employed \citep[334]{parodi2010}.
\newpage

While it seems unquestionable that extensive multilingualism in local languages existed in the early and mid-colonial period, the scale of societal bilingualism with Spanish and its geographical extent is a much more contentious issue. The combined evidence from Spanish and Indigenous sources points to what could have been, in the first phase of the colonial period, a partial and elitist bilingualism present among Native nobles and Spanish friars and clerics, accompanied by incipient and growing general bilingualism in the second part of the colonial period between Indigenous populations in specific regions and higher social groups \citep[945, Tab. 15]{zimmermann2010}. More debatable are other phenomena proposed for the colonial period (see \citealt[945]{zimmermann2010}), such as a notable increase of Indigenous people monolingual in Spanish. However, the available evidence suggests that this situation varied greatly, depending on the particular setting. Most notably it would have applied to large urban contexts, where societal bilingualism does not seem to have been particularly stable, thus giving way to Spanish monolingualism across several generations. This scenario seems to be supported by the decreasing number of documents in Nahuatl in the late seventeenth and eighteenth centuries in cities such as México-Tenochtitlán, Toluca or the even more ethnically homogeneous Tlaxcala, where Indigenous legal matters were increasingly conducted in Spanish. However, monolingual speakers of Nahuatl were still present in numerous communities in Tlaxcala until the second half of the twentieth century, including in some of the communities located close to highly urbanised areas, so widespread bilingualism toward the end of the colonial period in this area seems rather improbable.

\hspace*{-1mm}The areas that most favored bilingualism and multilingualism in Indigenous languages and Spanish were large towns with already existing local populations, or those attracting immigrant labor. Although the legal divisions of ``mixed'' towns into Spanish and Indigenous municipalities created a jurisdictional and administrative separation between these ethnic groups, these divisions were not impermeable, and the process of mixing and contact between Natives and Spaniards contributed to the development of different levels of bilingualism and/or multilingualism. Interesting pieces of evidence on inter-ethnic contact and relationships come from the capital city of México-Tenochtitlán. By 1612, approximately 80,000 Indigenous persons reportedly lived in this city, as well as some 50,000 persons of African and mixed African-Indigenous origin, and about 15,000 ``Spaniards'' (including creoles) \citep[34]{nutiniisaac2009}. In Zacatecas in the north and Puebla de los Ángeles to the south of the Valley of Mexico, both of which were formally established as ``Spanish'' towns serving the purposes of colonial settlers and businesses, the cities’ migrant and locally born populations of Indigenous people and Africans outnumbered their Spanish counterparts. Already in the 1570s, some 3,000 Indigenous people, 500 Africans and 500 Spaniards, as well as many mulattos, are reported for the entire city of Puebla \citep[34--36]{nutiniisaac2009}. The situation changed dramatically by the time of the 1777 census, however, when the town’s population reached over 56,000 inhabitants, with ca. 31.8\% of Spaniards, 21.4\% of Indigenous people, 16.1\% of mestizos, 4.6\% of mulattos, and the rest constituting “other castes” \citep[48]{nutiniisaac2009}. In Zacatecas, the center of the silver mining industry, by 1572 the approximately 1,500 Natives and 500 slaves of African descent outnumbered the resident population of 300 Spaniards \citep[109--117]{velascomurillo2012}. The situation in Native towns, and especially in rural areas, was distinct because in many cases a very limited, mainly individual bilingualism survived until the second half of the twentieth century, and even more recently for those in more secluded, peripheral locations. Nonetheless, while varying between regions, depending on their degree of accessibility, the influx of Spanish-speaking settlers into Indigenous areas grew steadily during the colonial period, eventually causing language pressure, along with a growing pressure on Native land.

\section{Modern and contemporary Mexico}

Fostering transitional Spanish-Indigenous societal bilingualism was an important goal of the Mexican state right from its creation. It abolished the legal category of \textit{indios} at the expense of ``citizens'' and, in the associated rhetoric of ``progress'', Indigenous tongues were deprived of their importance as essential components of ethnic identity. Instead they were reduced to symbols of backwardness and obstacles to modernisation, as well as to successful integration into society (\citealt[62--64]{heath1972}, \citealt[582--583]{estradafernandez2010}). In terms of diglossia, Spanish, as an official and national language, was the only prestigious and high ``variety'', while all Indigenous languages became low varieties \citep[911]{zimmermann2010}. As a consequence, Spanish gained more and more presence in different social domains of daily life and in communicative interaction between Indigenous and non-Indigenous people, contributing to the growth of societal bilingualism and the shift to the dominant language (\citealt[723--730]{villavicenciozarza2010}, \citealt[92--93]{gonzalezluna2012}).

While the level of implementation of the state’s linguistic policy varied, significant changes came about in the second half of the twentieth century. In 1948 the National Indigenist Institute (INI) of Mexico was established by the President, Miguel Alemán, with the aim of exploring problems affecting the Indigenous population and seeking ways to improve their living conditions, e.g. by sending special educators (\textit{promotores culturales}) to regional centers. Envisioned as ``agents of change'' within local communities, they were Natives from the same region and knew both Spanish and the Indigenous tongue of the area \citep[135--138]{heath1972}. In the following decades, the Hispanisation of Native children, achieved via the direct method of using Spanish as the only language of school instruction, became the most widely applied educational model, even though ``bilingual education'' was one of the state’s official goals \citep[162--163]{heath1972}. In view of its failure, this program gave way in 1981 to another initiative called \textit{educación bilingüe-bicultural} – the most recent myth serving as a disguise for the imposition of Spanish. The aim of this approach is officially to develop literacy in a Native language before teaching Spanish, yet, ultimately, the role of local languages is reduced to a medium of instruction of the target language \citep[40--41]{floresfarfan1999}.

The most widespread assimilationist education model, based exclusively on Spanish, has remained the most common practice in local communities until the present day. It has often been combined with a more or less official prohibition of the use of Indigenous tongues at school and the consequent stigmatisation of children who do not speak Spanish. As language shift has deepened, attitudes of internal racism have surfaced in Indigenous communities. They have been directed toward those community members, including children, who have been less successful in achieving Hispanisation. The Mexican school system and its teachers have been instrumental in such cases. The widespread shift is reflected in the census data, even if this is treated with extreme caution. These data show a steady and rapid decrease in the numbers of Indigenous monolinguals and a subsequent increase in bilinguals and monolingual Spanish speakers. This is confirmed by ethnographic and linguistic surveys: for example, in the Tlaxcala Pueblan Valley at the end of the nineteenth century, more than 70\% of the population was Nahua, living in traditional, monolingual communities almost untouched by secularisation. However, it is estimated that only about 2\% of the valley’s population could still be considered ``Indigenous'' in the year 2000, with rapidly fading Indigenous-mestizo differences \citep[194]{nutiniisaac2009}. Transitional societal bilingualism and an accelerating shift to Spanish has come to be the dominant situation for Nahua communities, albeit occurring on different timescales for different communities.

In terms of currently dominant language ideologies and associated power relationships, members of Native communities usually situate Nahuatl (and other local languages) at the very bottom of the language hierarchy. Spanish is in the middle as a national language and that of the dominant ``modern'' society, and most recently, English has claimed its place at the very top as a symbol of upward social mobility and opportunities, associated with technology, business, youth and popular culture. For communities with high rates of migration to the US, it is also the language of remote opportunities and a symbol of a better life. Spanish remains linked to all basic dimensions of social life as the unique language of education, politics, work, and legal and public services. In comparison, Nahuatl’s typical (and often only) domains include household, family and agriculture. It is regarded as a lower-status tongue of \textit{campesinos} (peasants), who are situated in a much less advantageous societal position than Spanish-speaking professionals \citep{sandoval2017}. Decisions to favor the unmarked choice of Spanish are often behind a community-level shift to this national language, in accordance with the strong discourse of \textit{salir adelante}, ``forging ahead'' and improving one’s socioeconomic position \citep[569--572]{messing2007}.

\section{Research in four Nahua-speaking regions: contexts, study and participants}

While language ideologies and attitudes shed important light on the nature of contemporary Spanish-Nahuatl bilingualism, important insights also come from quantitative research. The research we report on here is part of a team project that included four Nahuatl-speaking regions of Mexico: the town of Atliaca in the municipality of Tixtla in the state of Guerrero; rural communities in the municipality of Chicontepec (Huasteca Veracruzana, in the state of Veracruz); Xilitla and other municipalities in Huasteca Potosina (the state of San Luis Potosí); and the municipality of Contla de Juan Cuamatzi in the state of Tlaxcala. These regions represent complex cultural traditions dating back to pre-Hispanic Mesoamerica. While they share a general cultural background and history that is typical of the broad Mesoamerican cultural area, the members of these communities speak distinct variants of Nahuatl, which are nevertheless mutually intelligible to a high degree. They also differ in terms of Indigenous language retention and strength of Indigenous identity.

The most traditional are rural communities located in the municipality of Chicontepec, where, according to the 2010 census, 67\% of the population spoke Nahuatl, including 51\% of children \citep{inegi2010}. Community members continue many core elements of traditional religion and corn-based agriculture, sharing the strong identity of \textit{macehualmeh} or Indigenous people. In Atliaca, a small town in the municipality of Tixtla in Guerrero, some 80\% of inhabitants knew Nahuatl, according to the 2010 census. However, Spanish is becoming increasingly dominant, especially in central sectors of the town and among the younger generations. Inhabitants live on traditional agriculture, brick production and other specialised professions. In the third locality, Xilitla, some 40\% of residents were reported to be speakers of Nahuatl in 2010. Cultural assimilation (\textit{mestizaje}) is quite strong here, with traditional agriculture increasingly being eroded; many inhabitants of the region rely on state support and small-scale wage work. Even more culturally assimilated, and most urbanised, are communities in the municipality of Contla in Tlaxcala, where the shift to Spanish is the most advanced. According to the 2010 census data, only ca. 15.5\% inhabitants identified themselves as speakers of Nahuatl; among children under 14, less than 3\% were reported to speak the language. While local communities continue some forms of traditional religious organisation and corporate government, the economy is mainly based on wage labor, local industries (such as textile production) and other small businesses. All four regions share a history of discrimination and stigmatisation of Nahuatl-speaking children at school. Almost all speakers of Nahuatl also speak Spanish (with the oldest generations displaying differing levels of proficiency), while the youngest often exhibit reduced or passive skills in Nahuatl.

The survey for this project, carried out in 2018 and 2019, was based on an extensive panel questionnaire in Spanish, which was conducted mainly in person by local Nahuatl-speaking project members and collaborators; in the case of respondents whose preferred language of communication was Nahuatl (in Chicontepec and Atliaca), the interviews were conducted in this language and questions were translated into Nahuatl. Some of the younger participants, mainly in the region of Huasteca Potosina and in Tlaxcala, completed the questionnaire online. In total, the survey reached 552 respondents, whose mean age (M\textsubscript{age}) was 37.9 (\textit{SD} = 18.3). 55.4\% of the sample were women (n =306). Samples in the four regions varied from 108 to 156: Atliaca (n = 152; M\textsubscript{age} = 33.3, \textit{SD} = 18.26; 75 women), Chicontepec (n = 108; M\textsubscript{age} = 59.17, \textit{SD} = 17.91; 62 women), Xilitla and neighboring municipalities\footnote{Matlapa, Axtla de Terrazas, Tampacán, Tamazunchale and Coxcatlán.} (n = 136; M\textsubscript{age} = 30.79, \textit{SD} = 17.77; 65 women), and Contla (n = 156; M\textsubscript{age} = 40.25, \textit{SD} = 14.62; 104 women). Language use was assessed with a set of 14 items relating to narrow subdomains of everyday communicative situations, including different functional and social network-related domains. This scale aimed to reflect a broad range of domains of language use, taking into account the use of both the minority and dominant languages within family circles, immediate social networks, with friends, in schools, institutions, services, public events, and on social media (\tabref{tab:olko:3}). Previous scales of this kind include those by \citet{landry1994, ehala2014}, and in the EuLaViBar Project \citep{akermark2013}. In contrast with the previous tools, we addressed frequent patterns of interrupted intergenerational language transmission, in which language teaching skips the parents’ generation, with the oldest family members transmitting the language to the youngest generations.

\section{Results}

Preliminary qualitative analysis of the frequency of Nahuatl and Spanish language use across ethnic groups was assessed in the 14 subdomains of everyday communication with: parents, grandparents, children, friends, neighbors, doctors, attendees of cultural activities, people on social media, municipal authorities, community authorities and healers, as well as during participation in family meetings, ceremonies, and church services. A Likert scale of 1-7 was used to rate language use across different domains, where steps 1-3 indicated prevalent use of the Spanish language (over the Nahuatl language), step 4 represented an equal use of Spanish and Nahuatl, and steps 5-7 indicated prevalent use of the Nahuatl language (over Spanish). These frequencies are reported in \tabref{tab:olko:1} and Figures \figref{fig:olko:1} and \figref{fig:olko:2}. The results of the survey confirm and further reveal significant differences between the four regions. The highest retention of Nahuatl was found in Chicontepec in Veracruz, followed by Atliaca in Guerrero and Xilitla region in San Luis Potosi; the lowest use of Nahuatl and the most widespread expansion of Spanish to all domains of life were found in the region of Contla in Tlaxcala. These results confirm preliminary observations and conclusions drawn from qualitative data acquired in fieldwork, but at the same time they show measurable differences across regions and domains, revealing aspects of life where Spanish has almost completely taken over spaces previously reserved for Indigenous languages. The outcomes of the quantitative survey also illustrate the strong dominance of Spanish in new spheres of usage, such as the Internet, social media and health services. Nahuatl’s strongest bastion is the family domain and, in particular, communication with grandparents and parents. However, that drops abruptly, even in Chicontepec and Atliaca, in the case of communication with children. This pattern bespeaks widespread ruptures in the intergenerational transmission of the heritage language, and an ongoing and rapid shift to Spanish. This accelerated process can be described as a generational turn from transitional Spanish-Nahuatl bilingualism to monolingualism in the national language. In Chicontepec and Atliaca, communication in Nahuatl outside of the family domain is the strongest with healers, and remains strong with neighbors, friends, community authorities, and during traditional ceremonies, with the averages showing an equal use of Spanish and Nahuatl, or a slightly more prevalent usage of Nahuatl. Diagram 2 illustrates the expansion of Spanish into different domains of life. In the case of the Contla region it is almost exclusively the only language used in every context of life, except for communication with grandparents and parents, where there is still some retention of Nahuatl. An ANOVA test was run to compare mean level differences in Nahuatl language use across the four groups. The results, presented in \tabref{tab:olko:1}, revealed statistically significant differences between communities in mean levels of all variables regarding the relative use of Nahuatl and Spanish in various domains of life. The highest use of Nahuatl was in Chicontepec with grandparents (6.14), parents (5.74) and healers (5.72), followed by the communication with grandparents in Atliaca (5.52). The lowest values across all 14 domains are found invariably in Contla.

\begin{sidewaystable}
\fittable{
\begin{tabular}{@{}llllllllllll@{}}
\lsptoprule
                                                                              & \begin{tabular}[c]{@{}l@{}}Merged\\ sample\\ (n=235)\end{tabular} & \textit{SD}    & \begin{tabular}[c]{@{}l@{}}Atliaca\\ (n=63)\end{tabular} & \textit{SD}    & \begin{tabular}[c]{@{}l@{}}Chicon- \\ tepec\\ (n=30)\end{tabular} & \textit{SD}    & \begin{tabular}[c]{@{}l@{}}Contla\\ (n=99)\end{tabular} & \textit{SD}    & \begin{tabular}[c]{@{}l@{}}Xilitla\\ (n=43)\end{tabular} & \textit{SD}    & F (df)          \\ \midrule
Parents                                                                       & 4.23                                                            & 2.328 & 4.82                                                     & 2.044 & 5.74                                                         & 1.787 & 2.73                                                    & 2.019 & 4.12                                                     & 2.33  & 48.531*** (519) \\
Grandparents                                                                  & 4.85                                                            & 2.296 & 5.52                                                     & 1.769 & 6.14                                                         & 1.634 & 3.24                                                    & 2.231 & 4.96                                                     & 2.333 & 46.48*** (491)  \\
Children                                                                      & 3.17                                                            & 1.92  & 4.15                                                     & 1.691 & 3.94                                                         & 1.91  & 1.96                                                    & 1.344 & 3.36                                                     & 1.938 & 38.893*** (355) \\
Friends                                                                       & 3.42                                                            & 2.013 & 4.27                                                     & 1.94  & 4.76                                                         & 1.776 & 2.04                                                    & 1.455 & 3                                                        & 1.65  & 68.16*** (523)  \\
Neighbors                                                                     & 3.6                                                             & 2.121 & 4.36                                                     & 2.057 & 4.98                                                         & 1.75  & 2                                                       & 1.456 & 3.52                                                     & 1.889 & 68.399*** (521) \\
\begin{tabular}[c]{@{}l@{}}Doctor's \\ office\end{tabular}                                                               & 1.71                                                            & 1.299 & 1.78                                                     & 1.368 & 2.06                                                         & 1.866 & 1.32                                                    & 0.713 & 1.81                                                     & 1.116 & 7.663*** (520)  \\
\begin{tabular}[c]{@{}l@{}}Cultural \\ events\end{tabular}                                                               & 2.93                                                            & 2.004 & 3.41                                                     & 2.147 & 4.55                                                         & 1.899 & 1.78                                                    & 1.287 & 2.51                                                     & 1.571 & 55.569*** (518) \\
\begin{tabular}[c]{@{}l@{}}Internet/\\ social media\end{tabular}              & 1.66                                                            & 1.031 & 1.72                                                     & 1.066 & 1.8                                                          & 1.181 & 1.35                                                    & 0.77  & 1.92                                                     & 1.125 & 7.279*** (427)  \\
\begin{tabular}[c]{@{}l@{}}Municipal\\ authorities\end{tabular}               & 2.46                                                            & 1.826 & 2.16                                                     & 1.466 & 4.26                                                         & 2.114 & 1.49                                                    & 0.876 & 2.6                                                      & 1.803 & 64.486*** (514) \\
\begin{tabular}[c]{@{}l@{}}Community\\ authorities\end{tabular}               & 3.29                                                            & 2.245 & 3.9                                                      & 2.438 & 4.64                                                         & 1.916 & 1.5                                                     & 0.947 & 3.75                                                     & 2.038 & 66.722*** (510) \\
\begin{tabular}[c]{@{}l@{}}Local healers\end{tabular} & 4.09                                                            & 2.39  & 5.04                                                     & 2.157 & 5.72                                                         & 1.631 & 2.03                                                    & 1.65  & 4.09                                                     & 2.186 & 89.661*** (509) \\
Family events                                                                 & 3.5                                                             & 2.018 & 4.4                                                      & 1.839 & 4.59                                                         & 1.75  & 2.01                                                    & 1.402 & 3.37                                                     & 1.916 & 64.853*** (522) \\
\begin{tabular}[c]{@{}l@{}}Traditional\\ ceremonies\end{tabular}              & 3.88                                                            & 2.22  & 4.78                                                     & 2.052 & 5.22                                                         & 1.784 & 2.01                                                    & 1.426 & 4.02                                                     & 2.012 & 82.958*** (517) \\
Church                                                                        & 3.13                                                            & 2.084 & 4.28                                                     & 2.099 & 3.8                                                          & 2.138 & 1.63                                                    & 1.089 & 3.14                                                     & 1.855 & 57.471*** (499) \\ \lspbottomrule
\end{tabular}
}
\caption{\label{tab:olko:1}Averages regarding the use of Nahuatl and Spanish in different domains of life and social networks. *** \textit{p} <0 .001}
\end{sidewaystable}

%The table in question - version 1

% \begin{sidewaystable}
% \begin{tabularx}{\textwidth}{@{}llllllllllllll@{}}
% \lsptoprule
%                                                                           & \multicolumn{12}{l}{Language type across ethnic groups}                                                                                                                                                                                                                                                                                                                                                                                                                                                                                                                                                                                           &            \\ \midrule
%                                                                           & \multicolumn{3}{l}{Atliaca (n=152)}                                                                                                                         & \multicolumn{3}{l}{Chicontepec (n=108)}                                                                                                                   & \multicolumn{3}{l}{Contla (n=156)}                                                                                                                         & \multicolumn{3}{l}{Xilitla (n=136)}                                                                                                                        &            \\ \midrule
% Subdomain                                                                  & S                                                       & \begin{tabular}[c]{@{}l@{}}Equal\\ N\&S\end{tabular} & N                                                      & S                                                      & \begin{tabular}[c]{@{}l@{}}Equal\\ N\&S\end{tabular} & N                                                     & S                                                       & \begin{tabular}[c]{@{}l@{}}Equal\\ N\&S\end{tabular} & N                                                     & S                                                       & \begin{tabular}[c]{@{}l@{}}Equal\\ N\&S\end{tabular} & N                                                     & χ2         \\
% Parents                                                                    & \begin{tabular}[c]{@{}l@{}}39\\ (26)\end{tabular}  & \begin{tabular}[c]{@{}l@{}}28\\ (18)\end{tabular} & \begin{tabular}[c]{@{}l@{}}80\\ (53)\end{tabular}  & \begin{tabular}[c]{@{}l@{}}11\\ (10)\end{tabular} & \begin{tabular}[c]{@{}l@{}}21\\ (20)\end{tabular} & \begin{tabular}[c]{@{}l@{}}69\\ (64)\end{tabular} & \begin{tabular}[c]{@{}l@{}}105\\ (67)\end{tabular} & \begin{tabular}[c]{@{}l@{}}17\\ (11)\end{tabular} & \begin{tabular}[c]{@{}l@{}}28\\ (18)\end{tabular} & \begin{tabular}[c]{@{}l@{}}48\\ (35)\end{tabular}  & \begin{tabular}[c]{@{}l@{}}23\\ (17)\end{tabular} & \begin{tabular}[c]{@{}l@{}}51\\ (38)\end{tabular} & 116.071*** \\
% Grandparents                                                               & \begin{tabular}[c]{@{}l@{}}20\\ (13)\end{tabular}  & \begin{tabular}[c]{@{}l@{}}23\\ (15)\end{tabular} & \begin{tabular}[c]{@{}l@{}}104\\ (68)\end{tabular} & \begin{tabular}[c]{@{}l@{}}8\\ (7)\end{tabular}   & \begin{tabular}[c]{@{}l@{}}8\\ (7)\end{tabular}   & \begin{tabular}[c]{@{}l@{}}72\\ (67)\end{tabular} & \begin{tabular}[c]{@{}l@{}}81\\ (52)\end{tabular}  & \begin{tabular}[c]{@{}l@{}}17\\ (11)\end{tabular} & \begin{tabular}[c]{@{}l@{}}41\\ (26)\end{tabular} & \begin{tabular}[c]{@{}l@{}}33\\ (24)\end{tabular}  & \begin{tabular}[c]{@{}l@{}}12\\ (9)\end{tabular}  & \begin{tabular}[c]{@{}l@{}}73\\ (54)\end{tabular} & 115.862*** \\
% Children                                                                   & \begin{tabular}[c]{@{}l@{}}24\\ (16)\end{tabular}  & \begin{tabular}[c]{@{}l@{}}30\\ (20)\end{tabular} & \begin{tabular}[c]{@{}l@{}}32\\ (21)\end{tabular}  & \begin{tabular}[c]{@{}l@{}}29\\ (27)\end{tabular} & \begin{tabular}[c]{@{}l@{}}33\\ (31)\end{tabular} & \begin{tabular}[c]{@{}l@{}}22\\ (20)\end{tabular} & \begin{tabular}[c]{@{}l@{}}114\\ (73)\end{tabular} & \begin{tabular}[c]{@{}l@{}}12\\ (8)\end{tabular}  & \begin{tabular}[c]{@{}l@{}}5\\ (3)\end{tabular}   & \begin{tabular}[c]{@{}l@{}}31\\ (23)\end{tabular}  & \begin{tabular}[c]{@{}l@{}}12\\ (9)\end{tabular}  & \begin{tabular}[c]{@{}l@{}}12\\ (9)\end{tabular}  & 185.922*** \\
% Friends                                                                    & \begin{tabular}[c]{@{}l@{}}46\\ (30)\end{tabular}  & \begin{tabular}[c]{@{}l@{}}41\\ (27)\end{tabular} & \begin{tabular}[c]{@{}l@{}}62\\ (41)\end{tabular}  & \begin{tabular}[c]{@{}l@{}}19\\ (18)\end{tabular} & \begin{tabular}[c]{@{}l@{}}37\\ (34)\end{tabular} & \begin{tabular}[c]{@{}l@{}}44\\ (41)\end{tabular} & \begin{tabular}[c]{@{}l@{}}122\\ (78)\end{tabular} & \begin{tabular}[c]{@{}l@{}}17\\ (11)\end{tabular} & \begin{tabular}[c]{@{}l@{}}13\\ (8)\end{tabular}  & \begin{tabular}[c]{@{}l@{}}73\\ (54)\end{tabular}  & \begin{tabular}[c]{@{}l@{}}37\\ (27)\end{tabular} & \begin{tabular}[c]{@{}l@{}}13\\ (10)\end{tabular} & 149.319*** \\
% Neighbors                                                                  & \begin{tabular}[c]{@{}l@{}}48\\ (32)\end{tabular}  & \begin{tabular}[c]{@{}l@{}}31\\ (20)\end{tabular} & \begin{tabular}[c]{@{}l@{}}71\\ (47)\end{tabular}  & \begin{tabular}[c]{@{}l@{}}14\\ (13)\end{tabular} & \begin{tabular}[c]{@{}l@{}}32\\ (30)\end{tabular} & \begin{tabular}[c]{@{}l@{}}53\\ (49)\end{tabular} & \begin{tabular}[c]{@{}l@{}}124\\ (80)\end{tabular} & \begin{tabular}[c]{@{}l@{}}18\\ (12)\end{tabular} & \begin{tabular}[c]{@{}l@{}}10\\ (6)\end{tabular}  & \begin{tabular}[c]{@{}l@{}}53\\ (39)\end{tabular}  & \begin{tabular}[c]{@{}l@{}}35\\ (26)\end{tabular} & \begin{tabular}[c]{@{}l@{}}33\\ (24)\end{tabular} & 161.292*** \\
% Doctor’s office                                                            & \begin{tabular}[c]{@{}l@{}}138\\ (91)\end{tabular} & \begin{tabular}[c]{@{}l@{}}2\\ (1)\end{tabular}   & \begin{tabular}[c]{@{}l@{}}8\\ (5)\end{tabular}    & \begin{tabular}[c]{@{}l@{}}83\\ (77)\end{tabular} & \begin{tabular}[c]{@{}l@{}}2\\ (2)\end{tabular}   & \begin{tabular}[c]{@{}l@{}}13\\ (12)\end{tabular} & \begin{tabular}[c]{@{}l@{}}152\\ (97)\end{tabular} & \begin{tabular}[c]{@{}l@{}}0\\ (0)\end{tabular}   & \begin{tabular}[c]{@{}l@{}}1\\ (1)\end{tabular}   & \begin{tabular}[c]{@{}l@{}}115\\ (85)\end{tabular} & \begin{tabular}[c]{@{}l@{}}5\\ (4)\end{tabular}   & \begin{tabular}[c]{@{}l@{}}2\\ (2)\end{tabular}   & 46.105***  \\
% Cultural events                                                            & \begin{tabular}[c]{@{}l@{}}79\\ (52)\end{tabular}  & \begin{tabular}[c]{@{}l@{}}29\\ (19)\end{tabular} & \begin{tabular}[c]{@{}l@{}}40\\ (26)\end{tabular}  & \begin{tabular}[c]{@{}l@{}}24\\ (22)\end{tabular} & \begin{tabular}[c]{@{}l@{}}39\\ (36)\end{tabular} & \begin{tabular}[c]{@{}l@{}}34\\ (32)\end{tabular} & \begin{tabular}[c]{@{}l@{}}135\\ (87)\end{tabular} & \begin{tabular}[c]{@{}l@{}}9\\ (6)\end{tabular}   & \begin{tabular}[c]{@{}l@{}}9\\ (6)\end{tabular}   & \begin{tabular}[c]{@{}l@{}}91\\ (67)\end{tabular}  & \begin{tabular}[c]{@{}l@{}}22\\ (16)\end{tabular} & \begin{tabular}[c]{@{}l@{}}8\\ (6)\end{tabular}   & 140.051*** \\
% \begin{tabular}[c]{@{}l@{}}Internet/\\ social media\end{tabular}           & \begin{tabular}[c]{@{}l@{}}121\\ (80)\end{tabular} & \begin{tabular}[c]{@{}l@{}}10\\ (7)\end{tabular}  & \begin{tabular}[c]{@{}l@{}}3\\ (2)\end{tabular}    & \begin{tabular}[c]{@{}l@{}}35\\ (32)\end{tabular} & \begin{tabular}[c]{@{}l@{}}3\\ (3)\end{tabular}   & \begin{tabular}[c]{@{}l@{}}2\\ (2)\end{tabular}   & \begin{tabular}[c]{@{}l@{}}135\\ (87)\end{tabular} & \begin{tabular}[c]{@{}l@{}}3\\ (2)\end{tabular}   & \begin{tabular}[c]{@{}l@{}}1\\ (1)\end{tabular}   & \begin{tabular}[c]{@{}l@{}}104\\ (76)\end{tabular} & \begin{tabular}[c]{@{}l@{}}10\\ (7)\end{tabular}  & \begin{tabular}[c]{@{}l@{}}1\\ (1)\end{tabular}   & 136.874*** \\
% \begin{tabular}[c]{@{}l@{}}Municipal\\ authorities\end{tabular}            & \begin{tabular}[c]{@{}l@{}}123\\ (81)\end{tabular} & \begin{tabular}[c]{@{}l@{}}12\\ (8)\end{tabular}  & \begin{tabular}[c]{@{}l@{}}10\\ (7)\end{tabular}   & \begin{tabular}[c]{@{}l@{}}35\\ (32)\end{tabular} & \begin{tabular}[c]{@{}l@{}}22\\ (20)\end{tabular} & \begin{tabular}[c]{@{}l@{}}39\\ (36)\end{tabular} & \begin{tabular}[c]{@{}l@{}}148\\ (95)\end{tabular} & \begin{tabular}[c]{@{}l@{}}3\\ (2)\end{tabular}   & \begin{tabular}[c]{@{}l@{}}1\\ (1)\end{tabular}   & \begin{tabular}[c]{@{}l@{}}91\\ (67)\end{tabular}  & \begin{tabular}[c]{@{}l@{}}14\\ (10)\end{tabular} & \begin{tabular}[c]{@{}l@{}}17\\ (13)\end{tabular} & 143.783*** \\
% \begin{tabular}[c]{@{}l@{}}Community\\ authorities\end{tabular}            & \begin{tabular}[c]{@{}l@{}}71\\ (47)\end{tabular}  & \begin{tabular}[c]{@{}l@{}}13\\ (9)\end{tabular}  & \begin{tabular}[c]{@{}l@{}}65\\ (43)\end{tabular}  & \begin{tabular}[c]{@{}l@{}}21\\ (19)\end{tabular} & \begin{tabular}[c]{@{}l@{}}27\\ (25)\end{tabular} & \begin{tabular}[c]{@{}l@{}}42\\ (39)\end{tabular} & \begin{tabular}[c]{@{}l@{}}141\\ (90)\end{tabular} & \begin{tabular}[c]{@{}l@{}}7\\ (5)\end{tabular}   & \begin{tabular}[c]{@{}l@{}}2\\ (1)\end{tabular}   & \begin{tabular}[c]{@{}l@{}}54\\ (40)\end{tabular}  & \begin{tabular}[c]{@{}l@{}}30\\ (22)\end{tabular} & \begin{tabular}[c]{@{}l@{}}38\\ (28)\end{tabular} & 181.422*** \\
% \begin{tabular}[c]{@{}l@{}}Local healers\end{tabular} & \begin{tabular}[c]{@{}l@{}}34\\ (22)\end{tabular}  & \begin{tabular}[c]{@{}l@{}}19\\ (13)\end{tabular} & \begin{tabular}[c]{@{}l@{}}95\\ (63)\end{tabular}  & \begin{tabular}[c]{@{}l@{}}8\\ (7)\end{tabular}   & \begin{tabular}[c]{@{}l@{}}18\\ (17)\end{tabular} & \begin{tabular}[c]{@{}l@{}}71\\ (66)\end{tabular} & \begin{tabular}[c]{@{}l@{}}121\\ (78)\end{tabular} & \begin{tabular}[c]{@{}l@{}}9\\ (6)\end{tabular}   & \begin{tabular}[c]{@{}l@{}}16\\ (10)\end{tabular} & \begin{tabular}[c]{@{}l@{}}41\\ (30)\end{tabular}  & \begin{tabular}[c]{@{}l@{}}27\\ (20)\end{tabular} & \begin{tabular}[c]{@{}l@{}}51\\ (38)\end{tabular} & 195.588*** \\
% Family events                                                              & \begin{tabular}[c]{@{}l@{}}34\\ (22)\end{tabular}  & \begin{tabular}[c]{@{}l@{}}58\\ (38)\end{tabular} & \begin{tabular}[c]{@{}l@{}}58\\ (38)\end{tabular}  & \begin{tabular}[c]{@{}l@{}}15\\ (14)\end{tabular} & \begin{tabular}[c]{@{}l@{}}50\\ (46)\end{tabular} & \begin{tabular}[c]{@{}l@{}}34\\ (32)\end{tabular} & \begin{tabular}[c]{@{}l@{}}124\\ (80)\end{tabular} & \begin{tabular}[c]{@{}l@{}}22\\ (14)\end{tabular} & \begin{tabular}[c]{@{}l@{}}6\\ (4)\end{tabular}   & \begin{tabular}[c]{@{}l@{}}58\\ (43)\end{tabular}  & \begin{tabular}[c]{@{}l@{}}38\\ (28)\end{tabular} & \begin{tabular}[c]{@{}l@{}}26\\ (19)\end{tabular} & 173.132*** \\
% \begin{tabular}[c]{@{}l@{}}Traditional\\ ceremonies\end{tabular}           & \begin{tabular}[c]{@{}l@{}}34\\ (22)\end{tabular}  & \begin{tabular}[c]{@{}l@{}}36\\ (24)\end{tabular} & \begin{tabular}[c]{@{}l@{}}77\\ (51)\end{tabular}  & \begin{tabular}[c]{@{}l@{}}13\\ (12)\end{tabular} & \begin{tabular}[c]{@{}l@{}}30\\ (28)\end{tabular} & \begin{tabular}[c]{@{}l@{}}57\\ (53)\end{tabular} & \begin{tabular}[c]{@{}l@{}}122\\ (78)\end{tabular} & \begin{tabular}[c]{@{}l@{}}19\\ (12)\end{tabular} & \begin{tabular}[c]{@{}l@{}}10\\ (6)\end{tabular}  & \begin{tabular}[c]{@{}l@{}}41\\ (30)\end{tabular}  & \begin{tabular}[c]{@{}l@{}}35\\ (26)\end{tabular} & \begin{tabular}[c]{@{}l@{}}44\\ (32)\end{tabular} & 177.013*** \\
% Church                                                                     & \begin{tabular}[c]{@{}l@{}}46\\ (30)\end{tabular}  & \begin{tabular}[c]{@{}l@{}}35\\ (23)\end{tabular} & \begin{tabular}[c]{@{}l@{}}57\\ (38)\end{tabular}  & \begin{tabular}[c]{@{}l@{}}36\\ (33)\end{tabular} & \begin{tabular}[c]{@{}l@{}}31\\ (29)\end{tabular} & \begin{tabular}[c]{@{}l@{}}27\\ (25)\end{tabular} & \begin{tabular}[c]{@{}l@{}}138\\ (89)\end{tabular} & \begin{tabular}[c]{@{}l@{}}9\\ (6)\end{tabular}   & \begin{tabular}[c]{@{}l@{}}3\\ (2)\end{tabular}   & \begin{tabular}[c]{@{}l@{}}63\\ (46)\end{tabular}  & \begin{tabular}[c]{@{}l@{}}35\\ (26)\end{tabular} & \begin{tabular}[c]{@{}l@{}}20\\ (15)\end{tabular} & 147.756*** \\ \lspbottomrule
% \end{tabularx}
% \caption{Frequencies (n and percentage in brackets) of individuals using Spanish, Nahuatl or both languages equally in 14 everyday domains in four ethnic groups. * \textit{p} < 0,05; ** \textit{p} < 0,010; *** \textit{p} < 0,001.}
% \end{sidewaystable}

% The table in question - version 2

\begin{sidewaystable}
\fittable{
\begin{tabular}{@{}llllllllllllll@{}}
\lsptoprule
                                                                           & \multicolumn{12}{l}{Language type across ethnic groups}                                                                                                                                                                                                                                                                                &            \\ \midrule
                                                                           & \multicolumn{3}{l}{\begin{tabular}[c]{@{}l@{}}Atliaca\\ (n=152)\end{tabular}} & \multicolumn{3}{l}{\begin{tabular}[c]{@{}l@{}}Chicontepec\\ (n = 108)\end{tabular}} & \multicolumn{3}{l}{\begin{tabular}[c]{@{}l@{}}Contla\\ (n = 156)\end{tabular}} & \multicolumn{3}{l}{\begin{tabular}[c]{@{}l@{}}Xilitla\\ (n = 136)\end{tabular}} &            \\ \midrule
Subdomains                                                                 & S       & \begin{tabular}[c]{@{}l@{}}Equal\\ N\&S\end{tabular} & N      & S         & \begin{tabular}[c]{@{}l@{}}Equal\\ N\&S\end{tabular}    & N       & S        & \begin{tabular}[c]{@{}l@{}}Equal\\ N\&S\end{tabular}  & N     & S        & \begin{tabular}[c]{@{}l@{}}Equal\\ N\&S\end{tabular}  & N      & χ2         \\ \midrule
Parents                                                                    & 39 (26)  & 28 (18)                                                 & 80 (53)  & 11 (10)    & 21 (20)                                                    & 69 (64)   & 105 (67)  & 17 (11)                                                  & 28 (18) & 48 (35)   & 23 (17)                                                  & 51 (38)  & 116.071*** \\
Grandparents                                                               & 20 (13)  & 23 (15)                                                 & 104 (68) & 8 (7)      & 8 (7)                                                      & 72 (67)   & 81 (52)   & 17 (11)                                                  & 41 (26) & 33 (24)   & 12 (9)                                                   & 73 (54)  & 115.862*** \\
Children                                                                   & 24 (16)  & 30 (20)                                                 & 32 (21)  & 29 (27)    & 33 (31)                                                    & 22 (20)   & 114 (73)  & 12 (8)                                                   & 5 (3)   & 31 (23)   & 12 (9)                                                   & 12 (9)   & 185.922*** \\
Friends                                                                    & 46 (30)  & 41 (27)                                                 & 62 (41)  & 19 (18)    & 37 (34)                                                    & 44 (41)   & 122 (78)  & 17 (11)                                                  & 13 (8)  & 73 (54)   & 37 (27)                                                  & 13 (10)  & 149.319*** \\
Neighbors                                                                  & 48 (32)  & 31 (20)                                                 & 71 (47)  & 14 (13)    & 32 (30)                                                    & 53 (49)   & 124 (80)  & 18 (12)                                                  & 10 (6)  & 53 (39)   & 35 (26)                                                  & 33 (24)  & 161.292*** \\
Doctor’s office                                                            & 138 (91) & 2 (1)                                                   & 8 (5)    & 83 (77)    & 2 (2)                                                      & 13 (12)   & 152 (97)  & 0 (0)                                                    & 1 (1)   & 115 (85)  & 5 (4)                                                    & 2 (2)    & 46.105***  \\
Cultural events                                                            & 79 (52)  & 29 (19)                                                 & 40 (26)  & 24 (22)    & 39 (36)                                                    & 34 (32)   & 135 (87)  & 9 (6)                                                    & 9 (6)   & 91 (67)   & 22 (16)                                                  & 8 (6)    & 140.051*** \\
\begin{tabular}[c]{@{}l@{}}Internet/\\ social media\end{tabular}           & 121 (80) & 10 (7)                                                  & 3 (2)    & 35 (32)    & 3 (3)                                                      & 2 (2)     & 135 (87)  & 3 (2)                                                    & 1 (1)   & 104 (76)  & 10 (7)                                                   & 1 (1)    & 136.874*** \\
\begin{tabular}[c]{@{}l@{}}Municipal\\ authorities\end{tabular}            & 123 (81) & 12 (8)                                                  & 10 (7)   & 35 (32)    & 22 (20)                                                    & 39 (36)   & 148 (95)  & 3 (2)                                                    & 1 (1)   & 91 (67)   & 14 (10)                                                  & 17 (13)  & 143.783*** \\
\begin{tabular}[c]{@{}l@{}}Community\\ authorities\end{tabular}            & 71 (47)  & 13 (9)                                                  & 65 (43)  & 21 (19)    & 27 (25)                                                    & 42 (39)   & 141 (90)  & 7 (5)                                                    & 2 (1)   & 54 (40)   & 30 (22)                                                  & 38 (28)  & 181.422*** \\
\begin{tabular}[c]{@{}l@{}}Local healers\end{tabular} & 34 (22)  & 19 (13)                                                 & 95 (63)  & 8 (7)      & 18 (17)                                                    & 71 (66)   & 121 (78)  & 9 (6)                                                    & 16 (10) & 41 (30)   & 27 (20)                                                  & 51 (38)  & 195.588*** \\
Family events                                                              & 34 (22)  & 58 (38)                                                 & 58 (38)  & 15 (14)    & 50 (46)                                                    & 34 (32)   & 124 (80)  & 22 (14)                                                  & 6 (4)   & 58 (43)   & 38 (28)                                                  & 26 (19)  & 173.132*** \\
\begin{tabular}[c]{@{}l@{}}Traditional\\ ceremonies\end{tabular}           & 34 (22)  & 36 (24)                                                 & 77 (51)  & 13 (12)    & 30 (28)                                                    & 57 (53)   & 122 (78)  & 19 (12)                                                  & 10 (6)  & 41 (30)   & 35 (26)                                                  & 44 (32)  & 177.013*** \\
Church                                                                     & 46 (30)  & 35 (23)                                                 & 57 (38)  & 36 (33)    & 31 (29)                                                    & 27 (25)   & 138 (89)  & 9 (6)                                                    & 3 (2)   & 63 (46)   & 35 (26)                                                  & 20 (15)  & 147.756*** \\ \lspbottomrule
\end{tabular}
}
\caption{\label{tab:olko:2}Frequencies (n and percentage in brackets) of individuals using Spanish, Nahuatl or both languages equally in 14 everyday domains in four ethnic groups. * \textit{p} < 0,05; ** \textit{p} < 0,010; *** \textit{p} < 0,001.}
\end{sidewaystable}

\begin{figure}
\includegraphics[width=\textwidth]{figures/olkonah.png}
\caption{\label{fig:olko:1}The use of Nahuatl across different domains in the four \mbox{regions}}
\end{figure}

\begin{figure}
\includegraphics[width=\textwidth]{figures/olkoes.png}
\caption{\label{fig:olko:2}The use of Spanish across different domains in the four \mbox{regions}}
\end{figure}

During the survey, participants were also asked to self-assess their oral and writing skills in Nahuatl and Spanish, indicating how well they speak and write according to the following 6-item Likert scale: 1 not at all (I can’t speak it or I can’t write it), 2 hardly any, 3 a little bit, 4 moderately (neither good nor bad), 5 well, 6 very well. They were also asked to assess the degree of difficulty or ease with which they speak both languages, using a 6-item Likert scale where 1 indicated a lack of knowledge of the language in question, 2 very difficult, 3 difficult, 4 moderate (neither difficult nor easy), 5 easy, 6 very easy. The results are presented in \tabref{tab:olko:2}. In general, it is clear that in all the regions except Chicontepec, respondents declared a higher spoken proficiency in Spanish than in Nahuatl. In Chicontepec the average oral skills in Nahuatl are slightly higher than in Spanish (4.80 to 4.63); in Atliaca and Xilitla the average proficiency in Spanish is only slightly higher than self-assessed proficiency in Nahuatl (4.65 to 4.22 and 5.24 to 4.22 respectively). The difference is most striking in Tlaxcala (4.99 to 2.78).

The same pattern is seen in responses to the question regarding the difficulty of expression in both languages, with only respondents from Chicontepec self-declaring more difficulty speaking Spanish than Nahuatl, although the difference is relatively small (5.04 to 4.69). With regard to writing skills, participants in all regions declared a much higher writing proficiency in Spanish (4.80 to 2.44 in the overall sample). The highest Nahuatl writing skills were recorded in the region of Xilitla, which is explained by the participation of students at a local university where some courses are given in Nahuatl (this also accounts for the highest average score for Spanish literacy in this sample). In Chicontepec and Atliaca, despite a generally high oral proficiency in Nahuatl, written competence was low, which attests to the role of Nahuatl as a predominantly oral language, absent from written spaces and school education. This very limited presence of the Indigenous language in written media and the limited literary culture among its speakers makes it more difficult to expand its use to less traditional domains of life associated with technology, education or administration. As a matter of fact, this is a significant difference with regard to the colonial period, when written Nahuatl was widely present and used in administrative, legal, religious, economic, educational and even private or personal spheres of life. An ANOVA test was run to compare mean level differences in Nahuatl and Spanish self-assessed proficiency across the four groups. The results, presented in \tabref{tab:olko:3}, revealed statistically significant differences between communities in mean levels of all variables regarding language proficiency. Thus, summing up, the highest self-assessed oral skills in Nahuatl were found among respondents from Chicontepec, followed by Atliaca and Xilitla with the same average value. The same pattern is confirmed in the self-assessment of the feeling of ease while speaking Nahuatl. The highest self-reported writing skills in Nahuatl were found among Xilitla respondents, whereas Spanish skills were ranked highest in Xilitla and lowest in Chicontepec.

\begin{sidewaystable}
\begin{tabularx}{\textwidth}{@{}llllllllllllll@{}}
\lsptoprule
                                                                             & \multicolumn{2}{l}{\begin{tabular}[c]{@{}l@{}}Merged sample\\    (n=516)\end{tabular}} & \multicolumn{2}{l}{\begin{tabular}[c]{@{}l@{}}Atliaca \\ (n=150)\end{tabular}} & \multicolumn{2}{l}{\begin{tabular}[c]{@{}l@{}}Chicontepec\\     (n=100)\end{tabular}} & \multicolumn{2}{l}{\begin{tabular}[c]{@{}l@{}}Contla\\     (n=132)\end{tabular}} & \multicolumn{2}{l}{\begin{tabular}[c]{@{}l@{}}Xilitla\\    (n=114)\end{tabular}} &                 \\ \midrule
                                                                             & M                                           & \textit{SD}                                          & M                & \textit{SD}               & M                                          & \textit{SD}                                         & M                                       & \textit{SD}                                       & M                                        & \textit{SD}                                       & F (df)          \\ \midrule
I speak Nahuatl.                  & 3.91                                        & 1.478                                       & 4.22             & .874             & 4.80                                       & 1.123                                      & 2.78                                    & 1.529                                    & 4.22                                     & 1.423                                    & 63.083*** (526) \\
I speak Spanish.                  & 4.88                                        & .990                                        & 4.65             & .803             & 4.63                                       & 1.084                                      & 4.99                                    & 1.003                                    & 5.24                                     & .983                                     & 11.327*** (526) \\
I write Nahuatl.              & 2.44                                        & 1.486                                       & 2.21             & 1.166            & 2.13                                       & 1.376                                      & 1.92                                    & 1.277                                    & 3.68                                     & 1.512                                    & 46.583*** (525) \\
I write Spanish.                   & 4.80                                        & 1.178                                       & 4.53             & 1.079            & 4.50                                       & 1.324                                      & 4.88                                    & 1.170                                    & 5.28                                     & 1.006                                    & 12.550*** (522) \\
\begin{tabular}[c]{@{}l@{}}When I speak \\ Nahuatl, it is...\end{tabular} & 4.37                                        & 1.317                                       & 4.64             & .861             & 5.04                                       & 1.044                                      & 3.46                                    & 1.465                                    & 4.65                                     & 1.187                                    & 46.620*** (522) \\
\begin{tabular}[c]{@{}l@{}}When I speak \\ Spanish, it is…\end{tabular}   & 5.04                                        & .958                                        & 4.94             & .771             & 4.69                                       & 1.134                                      & 5.11                                    & .987                                     & 5.37                                     & .855                                     & 10.587*** (516) \\ \lspbottomrule
\end{tabularx}
\caption{\label{tab:olko:3}Averages regarding language competences. *** \textit{p} <0.001}
\end{sidewaystable}

The results discussed above are fully congruent with Pearson’s correlations (all assumptions of Pearson’s correlations hold) between analyzed variables based on the overall sample from the four regions. \tabref{tab:olko:4} presents statistically significant correlations between Nahuatl use in the family domain (including with family members, neighbors and during family gatherings), the use of Nahuatl across different domains, and self-assessed oral skills in Nahuatl and Spanish. It is not surprising that a high proficiency in spoken Nahuatl is strongly and positively correlated with its use in the household and immediate neighborhood, as well as with its usage in different aspects of life. However, proficiency in Spanish is negatively correlated with the usage of Nahuatl. Moreover, it is negatively correlated with oral proficiency in Indigenous languages, which confirms that Spanish-Nahuatl bilingualism is highly unstable, competitive and transitional toward the national language.

\begin{table}
\begin{tabularx}{\textwidth}{llll}
\lsptoprule
                                                                   & \begin{tabular}[c]{@{}l@{}}Nahuatl\\ general use\end{tabular} & \begin{tabular}[c]{@{}l@{}}Nahuatl\\ oral proficiency\end{tabular} & \begin{tabular}[c]{@{}l@{}}Spanish\\ oral proficiency\end{tabular} \\ \hline
Nahuatl use in family                                              & .920**                                                          & .762**                                                               & -.388**                                                              \\
Nahuatl general use                                                &                                                                 & .738**                                                               & -.422**                                                              \\
\begin{tabular}[c]{@{}l@{}}Nahuatl oral proficiency\end{tabular} &                                                                 &                                                                      & -.190**                                                              \\
\lspbottomrule
\end{tabularx}
\caption{\label{tab:olko:4}Pearson’s correlations between Nahuatl use in family, Nahuatl use across different domains and self-assessed proficiencies in Nahuatl and Spanish; **\textit{p} < .01 (n=525)}
\end{table}

These data are additionally explained by outcomes of a complementary survey of proficiencies in Spanish and Nahuatl using visual elicitation tools. 74 participants from the four regions (mean age: 46.02; 37 men, 37 women) were interviewed and recorded using purely visual elicitation tools: a series of pictures embracing both traditional and non-traditional objects (including some unusual ones, e.g. hybrid animals, included in order to assess language skills and lexical creativity, such as the ad-hoc creation of neologisms), and two movies, one showing some traditional daily activities of an Indigenous family and another presenting a short story featuring agricultural work and children’s activities. The participants were asked to name each object and describe the movies as they watched them, using either Nahuatl and Spanish in a randomised order (changing the language after each whole sequence of elicitation).

\tabref{tab:olko:5} shows the vocabulary density and the frequency of the usage of loanwords in both languages for the recorded elicitation sample. The vocabulary density, or the ratio of the number of unique words (types) to the number of all words (tokens) in the utterance of a specific person, serves as a proxy for the complexity of the utterance. Where an utterance includes, on average, many tokens of the same type, the density is lower, whereas a higher ratio indicates a richer repertoire of word types used. In all the regions studied, Spanish elicitations had a lower density than Nahuatl ones; however, the smallest difference between the two languages was observed in the two regions with the most advanced shift to Spanish: Contla and Xilitla. The differences between the mean Nahuatl and Spanish results obtained via an ANOVA test are presented in \tabref{tab:olko:6}. They were found to be statistically significant for all three measures discussed (i.e. ratio of borrowed words to all words, ratio of borrowed types to all types and vocabulary density) in all four communities, with the exception of the difference in vocabulary density in Atliaca. The lack of statistical significance might be explained by a smaller number of elicitations in that community.

While it is hard to draw any far-reaching conclusions from such a comparison between utterances in two languages with profound differences in morphosyntax, it is clear that Nahuatl spoken in communities with the highest vitality of this language – Chicontepec and Atliaca – reveals a higher density, i.e. richer vocabulary in utterances, than among speakers living in the more assimilated and linguistically endangered regions, Xilitla and Contla. This is also fully consistent with the data previously discussed, in that the shift to Spanish is ongoing and widespread especially in these two latter regions. In addition, the shrinking proficiency and reduced semantic functions of the Indigenous language are notable.

\begin{table}
\begin{tabularx}{\textwidth}{lllllll}
\lsptoprule
                                                        & \multicolumn{2}{l}{Vocabulary density} & \multicolumn{2}{l}{\begin{tabular}[c]{@{}l@{}}Borrowed words :\\ all words ratio\end{tabular}} & \multicolumn{2}{l}{\begin{tabular}[c]{@{}l@{}}Borrowed types : \\ all types ratio\end{tabular}} \\ \hline
                                                        & Nahuatl            & Spanish           & Nahuatl                                        & Spanish                                       & Nahuatl                                        & Spanish                                        \\
Atliaca                                                 & 0.55               & 0.45              & 0.22                                           & 0.01                                          & 0.25                                           & 0.02                                           \\
Chicontepec                                             & 0.47               & 0.33              & 0.16                                           & 0.01                                          & 0.20                                           & 0.02                                           \\
Contla                                                  & 0.37               & 0.31              & 0.18                                           & 0.01                                          & 0.23                                           & 0.02                                           \\
Xilitla                                                 & 0.39               & 0.33              & 0.11                                           & 0.01                                          & 0.13                                           & 0.01                                           \\ \hline
\begin{tabular}[c]{@{}l@{}}Merged\\ sample\end{tabular} & 0.45               & 0.36              & 0.17                                           & 0.01                                          & 0.20                                           & 0.02                                          
\\ \lspbottomrule
\end{tabularx}
\caption{\label{tab:olko:5}Quantitative results of the visual elicitation assessment of \mbox{proficiency} in Nahuatl and Spanish}
\end{table}

\begin{table}
\begin{tabularx}{\textwidth}{@{}lllll@{}}
\lsptoprule
                                 & Atliaca     & Chicontepec    & Contla         & Xilitla        \\ \midrule
\begin{tabular}[c]{@{}l@{}}Borrowed words :\\ all words ratio\end{tabular} & 6.855* (12) & 72.037*** (45) & 94.312*** (59) & 32.259*** (39) \\
\begin{tabular}[c]{@{}l@{}}Borrowed types :\\ all types ratio\end{tabular} & 7.781* (12) & 79.173*** (45) & 96.209*** (59) & 38.627*** (39) \\
Vocabulary density               & 1.328 (12)  & 39.638*** (45) & 12.831** (59)  & 7.255* (39)    \\ \lspbottomrule
\end{tabularx}
\caption{\label{tab:olko:6}\textit{F}-tests and degrees of freedom of the differences between the quantitative results of proficiency assessment in Spanish and in \mbox{Nahuatl}: * \textit{p} <0 .05, ** \textit{p} <0 .01, *** \textit{p} <0 .001}
\end{table}

The analysis of the frequency of loanwords in Spanish and Nahuatl elicitations is also quite revealing. In the overall sample, the percentage of Spanish loanwords into Nahuatl is relatively high: from 13\% in Xilitla to 25\% in Atliaca. The rate of borrowing in Contla is not much different – 23\% – even though the Nahuatl used in this region was characterised in the past \citep{hillhill1986} as a ``syncretic language'', drawing heavily on the Spanish lexicon. This categorisation is not confirmed in the documented elicitations of older proficient speakers of Nahuatl from the region, whose borrowing rate is lower than in the less assimilated Atliaca region, and not much higher than in Chicontepec, where language transmission still occurs, and where Nahuatl-Spanish bilingualism is more widespread. The average rate of usage of Spanish loanwords in the four regions is 17\%, and this rises to 20\% when overall word types and borrowed word types are compared.

What is even more striking, however, is an almost complete absence of Nahuatl loanwords in the Spanish utterances of the participants of our survey: regardless of the region, the rate is always ≤1\% and ≤ 2\% for all tokens and types respectively. Moreover, the few loanwords from Nahuatl which are attested in the Spanish utterances are essentially limited to those commonly used in Mexican Spanish, such as \textit{aguacate} ‘avocado’, \textit{chiquihuite} ‘basket’, \textit{comal} ‘type of griddle traditional in Mesoamerica’ and \textit{zacate} ‘forage’. The lack or avoidance of Nahuatl loanwords in the speech of persons for whom, in the majority of cases, Nahuatl was the first language\footnote{59\% declared Nahuatl to be their first language, 27\% Nahuatl and Spanish, 9\% Spanish, and 5\% did not specify.} suggests that Spanish was probably learned largely at school or outside the community as a more ``standardised'' language devoid of easily perceptible (i.e. lexical) Indigenous impact. Perhaps Nahuatl loanwords were avoided because of their association with a stigmatised identity. This finding is even more striking when compared to the Spanish language used by non-Indigenous members of the colonial society of New Spain, including Spaniards and creoles (as attested in numerous genres of colonial written documents), where Nahuatl loanwords were quite common, especially for local objects, plants, animals and even concepts that went on to become part of the general culture and lexicon. This strong asymmetry in the results of language contact between Spanish and Nahuatl is yet further salient evidence of very unstable and transitional bilingualism.

\section{Discussion and conclusions}

Widely shared and popularised views about generalised Spanish-Indigenous societal bilingualism and \textit{mestizaje} that developed during the colonial era find little support in the available data, nor in the most recent linguistic trajectories of Native communities. This kind of general bilingualism was not very common among the Indigenous population, even if it was increasing in large urbanised zones, particularly during the seventeenth and eighteenth centuries. Given the widespread presence of Nahuatl in New Spain, its official recognition and strong economic and sociopolitical potential, it was also quite common for non-Indige\-nous members of the colonial society to learn this local language for practical purposes. Undeniably, some communities, due to a number of factors, underwent assimilation and experienced a more or less complete shift to Spanish by the latter part of the colonial period. In the nineteenth and twentieth centuries, the marginalisation and discrimination of Indigenous communities that were largely monolingual in Nahuatl and/or bi/multilingual in local Indigenous languages deepened, and many chose the path of quick assimilation toward a \textit{mestizo} status and the use of Spanish. While most of the Nahua communities were exposed to differing degrees to the Spanish language and culture from the first phase of colonisation, this did not constitute a threat to the heritage language used by Indigenous groups, which at this time was still characterised by high ethnolinguistic vitality. The pressure of Spanish became much stronger after Independence, altering the nature of cultural and linguistic contact, which became more aggressive and displacive \citep{olko2018}. Although Mexican bilingualism has been seen as “as a long-term historical process” \citep[332]{floresfarfan2003}, it was, in fact, limited and ephemeral in local communities during the colonial period. In large urban contexts, the scale of societal bilingualism increased over time, gradually becoming transitional, and eventually unstable and transitory during later (post-colonial) times, triggering accelerated assimilation processes toward the national culture.

In contemporary ``bilingual'' Nahua communities this dynamic process is characterised by differing proficiencies in the two languages. Until recently, many speakers of Nahuatl, especially elderly ones, had limited proficiency in Spanish (this is still attested in regions such as Chicontepec). Now, however, it is more common to see highly varying proficiencies in the heritage language, with many non-fluent and/or non-active speakers among the younger generations (see \citealt{Dorian1981, dorian1986, grinevald1998}). Even in communities where Nahuatl is still spoken by the majority of people alongside Spanish, it is not uncommon to find families in which the grandparent and parent generations are fully proficient in Nahuatl, where the oldest, usually adolescent, children can speak with differing levels of competence, while their younger siblings are passive speakers and the very youngest are non-speakers. Such patterns strongly influence the dynamics and patterns of bilingual communication within specific households and across the whole community. Thus, among Nahua communities today we find a broad continuum of proficiency in the ancestral language, strictly related to the mode and circumstances of its transmission and the degree of socialisation in it \citep{olko2018, floresfarfanolko2020}. The results of the quantitative large-scale survey in four different regions where Nahuatl is still spoken, complemented by the assessment of proficiencies in Nahuatl and Spanish, allow us to draw a data-driven and coherent picture of current bilingual arrangements. The sociolinguistic situation can be described as unstable, asymmetrical Spanish-Nahuatl bilingualism leading to shift to the national language. Depending on the region, this may occur as quickly as within two to four generations due to strong power differentials between the two languages, as well as related economic, sociopolitical, cultural and educational pressures.

\section*{Acknowledgements}

We have no conflicts of interests to disclose. The Project “Language as a cure: Linguistic vitality as a tool for psychological well-being, health and economic sustainability” is carried out within the Team program of the Foundation for Polish Science co-financed by the European Union under the European Regional Development Fund. We are very grateful to the members of the Nahua communities for participating in the survey and in other parts of our research, to the anonymous reviewers for their comments regarding the article, and to Ellen Foote and Mary Chambers for the stylistic revision of the text.

\section*{Author contributions}

Justyna Olko played a lead role in Conceptualisation, Funding acquisition, Investigation, Methodology, Supervision, Field research, Writing – original draft, Writing – review {\&} editing and she played a supporting role in Data curation and Project administration. Szymon Gruda played a supporting role in Writing – review {\&} editing. Joanna Maryniak played a lead role in Data curation and a supporting role in Field research, Writing – review {\&} editing. Justyna Olko, Szymon Gruda and Joanna Maryniak played equal roles in the data analysis. Elwira Dexter-Sobkowiak, Humberto Iglesias Tepec, Eduardo de la Cruz, and Beatriz Cuahutle Bautista played a lead role in Field research and supporting role in Conceptualisation.

\sloppy
\printbibliography[heading=subbibliography,notkeyword=this]
\clearpage
\end{document}
