\documentclass[output=paper]{langscibook} 
\ChapterDOI{10.5281/zenodo.7446957}
\author{Andrew Nevins\affiliation{University College London} and Matt Coler\affiliation{University of Groningen -- Campus Fryslân}
}
\title{Introduction} 
\abstract{The introduction is split into three parts. In the first, we provide some background for the book, provide key definitions and reflections on minoritized languages and diaspora languages, and identify overarching themes in the book. In the second, we provide a summary overview of each of the 12 chapters, mentioning the languages investigated, the methods used, and the main findings. Finally, in the third part we reflect on the scope of the book itself and conclude with remarks on multilingualism and minoritization, positing that monolingualism should not be considered the default or status quo -- quite the opposite, in fact. The research here suggests that fighting againt monolingualism as an increasing trend involves not just scientific work, but also social activism.}

\IfFileExists{../localcommands.tex}{
 \addbibresource{../localbibliography.bib}
 % add all extra packages you need to load to this file

\usepackage{tabularx,multicol}
\usepackage{url}
\urlstyle{same}

\usepackage{listings}
\lstset{basicstyle=\ttfamily,tabsize=2,breaklines=true}

\usepackage{langsci-basic}
\usepackage{langsci-optional}
\usepackage{langsci-lgr}
\usepackage{langsci-osl}
% \usepackage{./langsci/styles/langsci-lgr}
% \usepackage{./langsci/styles/langsci-osl}
% \usepackage{langsci-gb4e}

\usepackage{tikz}
\usetikzlibrary{patterns,calc}
\pgfdeclarepatternformonly{south east lines}{\pgfqpoint{-0pt}{-0pt}}{\pgfqpoint{3pt}{3pt}}{\pgfqpoint{3pt}{3pt}}{
    \pgfsetlinewidth{0.6pt}
    \pgfpathmoveto{\pgfqpoint{0pt}{3pt}}
    \pgfpathlineto{\pgfqpoint{3pt}{0pt}}
    \pgfpathmoveto{\pgfqpoint{.2pt}{-.2pt}}
    \pgfpathlineto{\pgfqpoint{-.2pt}{.2pt}}
    \pgfpathmoveto{\pgfqpoint{3.2pt}{2.8pt}}
    \pgfpathlineto{\pgfqpoint{2.8pt}{3.2pt}}
    \pgfusepath{stroke}}
    
\usepackage{stmaryrd}
\usepackage{wasysym}
\usepackage{multirow}
\usepackage{caption}
\usepackage{subcaption}
\usepackage{mathrsfs}
\usepackage{qtree}

\usepackage{linguex}


 %pminos do not split footnotes
% \interfootnotelinepenalty=10000 %Footnote in Laporte chapters has to be split SN


%\DeclareIndexNameFormat{default}{%
%\nameparts{#1}%
%\usebibmacro{index:name}%
%{\index[names]}%
%{\namepartfamily}%
%{\namepartgiveni}%
% {}% L1
% {}% L2
%{\namepartprefix}% generates spurious space L3
%{\namepartsuffix}% generates spurious space L4
%}

%  {\DeclareIndexNameFormat{default}{%
%     \usebibmacro{index:name}{\index[names]}{#1}{#3}{#5}{#7}}}

%\DeclareIndexNameFormat{default}{%
%  \usebibmacro{index:name}{\sindex[nom]}{#1}{#3}{#5}{#7}}

%\DeclareIndexNameFormat{default}{%
%  \usebibmacro{index:name}{\sindex[person]}{#1}{#3}{#5}{#7}}
%\DeclareIndexNameFormat{default}{%
%\nameparts{#1} \usebibmacro{index:name}{\sindex[person]]}{\namepartfamily}{‌​\namepartgiven}{\nam‌​epartprefix}{\namepa‌​rtsuffix}}

%\newcommand{\smiley}{:)}

%\renewbibmacro*{index:name}[5]{%
%\usebibmacro{index:entry}{#1}%
%{\iffieldundef{usera}{}{\thefield{usera}\actualoperator}\mkbibindexname{#2}{#3}{#4}{#5}}}

% \newcommand{\noop}[1]{}

%remove for final
%\overfullrule=1mm

\newcommand{\tobi}[2]}}
\renewcommand{\S}[1]{\tobi{#1}{\textsc{*}}}

% this volume references
% puts: [this volume]
% already defined: \citetv
%\newcommand{\citepv}[1]{(\citeauthor{#1} \citeyear*{#1} [this volume])}
\newcommand{\citealtv}[1]{\citeauthor{#1} \citeyear*{#1} [this volume]}

%parentheses around example number
\newcommand{\pref}[1]{(\ref{#1})}

% in-text examples

\newcommand{\lnex}[1]{\textit{#1}} %target lang word
\newcommand{\lnlit}[1]{(lit.: `#1')} %literal reading
\newcommand{\lnlat}[1]{(#1)} % latinization
\newcommand{\lntrans}[1]{`#1'} %translation
\newcommand{\lnexl}[2]%
{\lnex{#1}{} \lnlat{#2}} % ex with latinization
\newcommand{\lnexlat}[3]{\lnex{#1}{} \lnlat{#2}{} \lntrans{#3}} % ex with latinization and tranl.

%ch01
\newcommand{\co}[1]{\mbox{\textbf{#1}}}

%ch09

\newcommand{\cyrbulg}[1]{\begin{otherlanguage*}{bulgarian}#1\end{otherlanguage*}}


%ch10
\newcommand{\nlp}{{\small NLP}}
\newcommand{\mwe}{{\small MWE}}
\newcommand{\rae}{{\small RAE}}
\newcommand{\lvc}{{\small LVC}}
\newcommand{\pos}{{\small P}o{\small S}}
%\newcommand{\todo}[1]{ \textcolor{red}{#1} }

%\renewcommand{\labelenumi}{\theenumi}
%\ainamefmt{{vv}{ll}{, ff}{, jj}} % fullname

\newcommand{\biberror}[1]{{\color{red}#1}}

\newcommand{\osenovaitem}{--~} 
 %% hyphenation points for line breaks
%% Normally, automatic hyphenation in LaTeX is very good
%% If a word is mis-hyphenated, add it to this file
%%
%% add information to TeX file before \begin{document} with:
%% %% hyphenation points for line breaks
%% Normally, automatic hyphenation in LaTeX is very good
%% If a word is mis-hyphenated, add it to this file
%%
%% add information to TeX file before \begin{document} with:
%% %% hyphenation points for line breaks
%% Normally, automatic hyphenation in LaTeX is very good
%% If a word is mis-hyphenated, add it to this file
%%
%% add information to TeX file before \begin{document} with:
%% \include{localhyphenation}
\hyphenation{
    Beck-man
    Ngu-yen
    back-chan-nel
    back-chan-nels
    mo-not-o-nous
    ste-reo-typ-i-cal
}

\hyphenation{
    Beck-man
    Ngu-yen
    back-chan-nel
    back-chan-nels
    mo-not-o-nous
    ste-reo-typ-i-cal
}

\hyphenation{
    Beck-man
    Ngu-yen
    back-chan-nel
    back-chan-nels
    mo-not-o-nous
    ste-reo-typ-i-cal
}
 
 \togglepaper[1]%%chapternumber
}{}

\begin{document}
\maketitle

\section{Setting the Stage}
These chapters emerged from presentations given at a panel entitled ``Variation in European minority and diaspora languages'' held at the 10\textsuperscript{th} International Conference on Language Variation in Europe in Leeuwarden, Fryslân on June 26\textsuperscript{th} 2019.

Critical to all these contributions is the concept of a diaspora and minoritized languages. For the purposes of this book, we consider diaspora languages to refer to languages spoken by people who have resettled in an area outside of their original linguistic community. Speakers of diaspora languages may have diverse origins and might come from different communities, social strata, and even nations. 

Comparatively, a minority (or, more properly, \textit{minoritized} language) is a language variety, or a cluster of varieties, that is historically spoken in a particular region where another language— usually an official majority language—is predominantly spoken.

It is worthwhile to dive deeper into the minority language vs. minoritized language dichotomy, insofar as it is a theme which shows up frequently in the contributions of this book. The term ‘minority language’ can be said to
suggest an inherent and static quality of the language itself, and in that sense, it obscures the fact that languages become minoritized as a direct outcome of actions and policies. That is, ``minoritization recognizes that systemic inequalities, oppression, and marginalization place individuals into ‘minority” status rather than their own characteristics''\citep{sotto2019time}. Accordingly, many within the field of linguistics promote the term ``minoritized language'' instead of ``minority language'' (see \citealt[Ch. 1]{nevins2022minoritized}). Although the latter term is somewhat entrenched in the field, the former has a history going back at least three decades -- see \citealt{py1989minorisation} and a full entry in Wikipedia.

This provides an opportune point to reconsider diaspora languages. The common sense definition, that is, ``a language spoken by people who have resettled in an area outside of their linguistic community" seems to be lacking. Let us consider two thought experiments. In the first, we can reflect on whether Spanish speakers in Argentina are speakers of a diaspora language in the same way that those in New York are. Next, consider where English is spoken as a diaspora language. Is it accurate to suggest that English is a diaspora language in India, Singapore and Hong Kong? Upon consideration, there seems to be a correlation between colonialism and diaspora status, at least sometimes. That is, people from the Global South are more clearly diaspora whereas less so for the Global North. On the other hand, Yiddish and Pomeranian, spoken in South America but no longer with thriving linguistic communities in their place of origin are clearly diaspora communities as well. By pointing to such contrasts, we hope to raise the issue of the complexity at hand.

There is clearly some overlap between diaspora and minority languages. Many of the languages in our book can be described as both diaspora and minoritized languages (including Pomeranian, Wymysiöeryś and Yiddish)\footnote{Importantly, in some cases, such as that for  Greko and Griko, speakers do not consider themselves as diaspora speakers; see for example \citet{Pellegrino2021}.}. Some others are only diaspora languages, but not, strictly speaking, minoritized languages (Castellano Andino -- spoken amongst Amerindian communities, Italo-Romance and Dalmatian varieties spoken in the Americas, and so forth); yet others are minoritized, but not diaspora languages (including Sorbian, Aymara, Quechua, and Nahuatl). These issues entail a gamut of political repercussions. Consider how the ``European Charter for minority or regional languages'' of the Council of Europe defines minority and regional languages as those languages which are traditionally used within a given territory of a state by inhabitants of that state who form a group numerically smaller than the rest of the state’s population and which are different from the official language(s) of that state. The charter also protects so-called ``non-territorial languages'' which are understood as languages spoken by nationals of a particular state, but these languages are distinct from the language(s) used by the rest of the population of the state and which cannot be identified with a particular region of a given state -- such languages include Ladino, Romani, and Yiddish, for example. Notably, dialects and migrant languages are not included in the charter; see \citet{woehrling2005european} for further discussion.

Aside from the geographic array, the linguistic variation explored attests to phenomena at many levels. This includes phonetic/phonological variation, relating to historical sound changes in Yiddish and Sorbian morphological and morphosyntactic variation attested in Israeli and US American variants of Yiddish, contact phenomena influencing the expression of grammatical mood in Spanish varieties in contact with Quechua and Aymara, and an array of morphological phenomena in Italo-Romance and -Dalmatian varieties spoken in the Americas. The contribution on Pomeranian is dedicated to developing an analysis of the syntactic structure of this language to account for language change. The contributions further extend beyond specific grammatical phenomena, and additionally touch upon anthropological issues relating to verbal art for Sorbian, sociolinguistic and ethnolinguistic identity (for Wymysiöeryś, and Greko/Griko), and phenomena relating to language contact between minority and majority languages in a new sociological setting at different grammatical levels, and whether it results in complexification (as for Wymysiöeryś in contact with varieties of German and Polish), structural reduction (as for Zeelandic-Flemish in contact with Brazilian Portuguese, or other kinds of change as in Spanish in contact with Nahuatl, Aymara, and Quechua, among the many
other examples in these contributions).

%\begin{comment}\section{Relevant remarks for European Charter for minority or regional languages}
%“European Charter for minority or regional languages” defines minority and regional languages as those languages traditionally used within a given territory of a state by nationals of that state who form a group numerically smaller than the rest of the state’s population and which are different from the official language(s) of that state, including neither dialects of the official language(s) nor migrant languages. The charter protect some diaspora languages, whereas others (including Yiddish, Ladino and Romani) are referred to as “non-territorial languages”; i.e., languages used by nationals of the state which differ from the language(s) used by the rest of the state’s population but which, although traditionally used within the state’s territory, cannot be identified with a particular area thereof.\section{Scope of this book} \end{comment}

\section{Overview of the Contributions}
We provide here an overview of the content and themes covered in Chapters 2 through 12 of this book, highlighting the range of diverse language contact scenarios covered throughout.

In Chapter 2, `Documenting Italo-Romance minority languages in the Americas: Problems and tentative solutions’, Luigi Andriani, Jan Casalicchio, Francesco Ciconte, Roberta D’Alessandro, Alberto Frasson, Brechje van Osch, Luana Sorgini, and Silvia Terenghi look at differential object marking, deixis and demonstratives, and subject clitics and null subjects in seven heritage Italo-Romance varieties (Piedmontese, Venetan, Tuscan, Abruzzese, Neapolitan, Salentino, and Sicilian). They describe the process of preparation and implementation of a data collection enterprise targeting Italo-Romance emigrant languages in North and South America, many of which had never been documented, and which, with the exception of the northern Italian-speaking community, are close to extinction. Their project aims to understand language change in contact. Their article describes the steps they took in assessing the speakers' proficiency, designing and running syntactic questionnaires and picture-sentence matching tasks, and general issues concerning experimental design and statistics.

In Chapter 3, ‘Spanish-Nahuatl bilingualism in Indigenous communities in Mexico: Variation in language proficiency and use’, Justyna Olko, Szymon Gruda, Joanna Maryniak, Elwira Dexter-Sobkowiak, Humberto Iglesias Tepec, Eduardo de la Cruz and Beatriz Cuahutle Bautista take a historical perspective on bilingualism in Spanish and Nahuatl from 1519 until the present day. They discuss the results of proficiency assessment in both languages, performed with the participation of members of selected Nahua communities. Their work reveals different degrees of assimilation to Mexican identity and shift to Spanish, most salient in more urbanized and less peripheral regions. The authors conclude that factors such as power differentials, economic marginalization, sociopolitical pressures, culture change, ethnic prejudice and discriminatory language policies lead to contemporary Spanish-Indigenous bilingualism at the community level being highly unstable. Using the term `unstable bilingualism', they suggest that the situation of parallel acquisition and use of Nahuatl and Spanish as well as diminishing and varying proficiency in the heritage language will lead eventually to language shift. Depending on the region, this may occur as quickly as within two to four generations.

In Chapter 4, ‘Trilingual modality: Towards an analysis of mood and modality in Aymara, Quechua and Castellano Andino as a joint systematic concept’, Philipp Dankel, Mario Soto Rodríguez, Matt Coler and Edwin Banegas-Flores examine how indigenous minoritized languages impact majority European ones. They do this by considering the case of Quechua and Aymara, on the one side, and Castellano Andino (CA) on the other. Their analysis demonstrates that regional varieties of CA reflect Aymara and Quechua mood, even in the speech of those who do not speak either indigenous language. The authors emphasize the complex nature and multiple causality of contact induced change which allows for potentialities of how minoritized languages can indeed sometimes impact majority languages.

In Chapter 5, ‘What is the role of the addressee in speakers’ production? Examples from the Griko- and Greko-speaking communities’, Manuela Pellegrino and Maria Olimpia Squillaci focus on the two endangered Italo-Greek varieties, Griko and Greko, spoken, respectively, in Salento (Puglia) and Calabria communities in Southern Italy. The authors examine speaker-addressee dynamics, how these affect language use and may potentially lead to `temporary variation', and how the addressee’s linguistic competence, age, and shared linguistic repertoire with the speaker may lead to style-shift in speakers’ production. They then consider how these factors contribute to the emergence of puristic attitudes which may even inhibit the use of Griko and Greko. The authors show how widespread resistance to and monitoring of language variation and change tend to undermine efforts to maintain or revitalise Griko and Greko. This highlights multiple, entangled power struggles embedded in their current revival.

In Chapter 6, ‘Innovations in the Contemporary Hasidic Yiddish pronominal system’, Zo\"{e} Belk, Lily Kahn, Kriszta Eszter Szendrői and Sonya Yampolskaya present a study involving 29 native Contemporary Hasidic Yiddish speakers, and demonstrate that significant changes have occurred in the personal pronoun, possessive, and demonstrative systems. While the personal pronoun system has undergone significant levelling in terms of case and gender marking, at the same time a new demonstrative pronoun has emerged which exhibits a novel case distinction. They argue that these innovative features are not determined directly by contact with the dominant co-territorial languages, but rather are internal developments which bear witness to the linguistic vibrancy of Contemporary Hasidic Yiddish.

In Chapter 7, ‘Validity of crowd-sourced minority language data: Observing variation patterns in the Stimmen recordings’, Nanna Hilton considers the usability of crowd sourced minority language data for research. She uses speech recordings and reported dialect knowledge collected with a smartphone application for Frisian, focusing on three phonological variables in Frisian speech. The author considers how minority language communities offer a welcome chance for variationist sociolinguistics to revisit principles of linguistic variation and change. It is often assumed that Frisian is converging towards Dutch on all linguistic levels. However, this assumption is based almost entirely on anecdotal evidence. Very few empirical studies of speech variation in Frisian exist. A way to conduct studies of sound change on a larger scale would be to use crowd-sourced speech data. To this end Hilton considers the usability of data from the Stimmen application. Stimmen has a picture naming task comprised of 87 images of everyday objects, and a gamified task that provides an estimate of where people hail from within the Province of Fryslân. The author concludes that the data is of high quality and that it can be used for investigating sound change. That said, one must consider whether crowd sourced data include contributions from so-called ``new speakers'' of minority languages, and whether one has a representative sample of all age groups in such remote studies.

In Chapter 8, ‘Complexity of endangered minority languages: The sound system of Wymysiöeryś’, Alexander Andrason demonstrates that Wymysiöeryś – a severely endangered minority Germanic language – exhibits remarkable complexity despite its moribund status. By analyzing twelve phonetic/phonological properties,the author concludes that the complexity of Wymysiöeryś is greater, both locally and globally, than that of two control languages: Middle High German and Modern Standard German. In most cases, the surplus of complexity attested is attributed to contact with the dominant and aggressive language, i.e. Polish. This confirms the view of language contact as not only having simplifying effects on languages, but also as contributing to their complexification – even in the situation of seemingly imminent language death.

In Chapter 9, Tomasz Wicherkiewicz considers how the Wymysiöeryś language, spoken in Wilamowice, has been frequently classified as a colonial variety of East Central German. He reflects on how such ethnotheories of provenance, including folk linguistic evidence and myths, referred to various Germanic countries as places of origin of the first settlers. However, the microlect of Wilamowice has certainly undergone interactions of various types and intensities with Polish (and its varieties) and standard High German. There is evidence of such contacts, shift, and hybridization in all subsystems of the microlanguage. Wicherkiewicz analyzes this through an approach based on perceptual dialectology and an ethnoscience perspective of language variation.

In Chapter 10, ‘Evaluating linguistic variation in light of sparse data in the case of Sorbian’, Eduard Werner examines one of the oldest Sorbian monuments by applying knowledge on neighboring Germanic and Celtic literature. From the linguistic side, the results of the comparison lend greater insights into historical sound changes in Sorbian. The author shows how historical sound changes help to unearth elements of verbal art. This, in turn, facilitates the possibility of more accurately dating the historical sound changes through their effects on alliterations.

In Chapter 11, ‘Modeling accommodation and dialect convergence formally: Loss of the infinitival prefix \textit{tau} ‘to’ in Brazilian Pomeranian’, Gertjan Postma re-evaluates a well-known, but often ignored mechanism and outcome: retreat to default settings, the rise of the unmarked, that is whenever the result of the change is not a sum or subset of the input forms, but an innovative pattern. While the original Pomeranian dialects in Europe had a considerable variation in this particular domain, Pomeranian in Brazil has converged to a remarkably uniform new construction, which was not present in Pomerania in the days of emigration. This configuration is reanalyzed as an overt movement relation of T to C, which is the default option in natural language. There are language-internal arguments that the new construction is a result of dialect-convergence to the default setting of the parameters involved. However, when we take the external occurrence rates into account, the data indicate that the similarity in this respect between Pomeranian and Brazilian Portuguese might be analyzed as accommodation of Brazilian Pomeranian to the dominant language.


In Chapter 12, ‘Using data of Zeelandic Flemish in Esp\'{i}rito Santo, Brazil for historical reconstruction’, Kathy Rys and Elizana Schaffel Bremenkamp focus on the case of Zeelandic Flemish in Esp\'{i}rito Santo,an obsolescent language variety spoken by about twenty descendants of Dutch immigrants in the 19\textsuperscript{th} century. These speakers are descendants of Dutch immigrants, who left Zeeland in 1858--1862, but have faced deprivation and difficulties in adaptation and integration into Brazilian society, with their language threatened by the majority language Brazilian Portuguese and by another heritage language; namely that of the Pomeranian immigrants who arrived in Esp\'{i}rito Santo at the same time. The speech of rusty speakers can be used to reconstruct the original immigrant language. The authors perform a historical reconstruction of the old Zeelandic Flemish dialect as spoken in the days of emigration, with respect to three linguistic cases: (1) deletion of /l/ in codas and coda clusters, (2) subject doubling in inversion contexts and (3) inflected polarity markers \textit{yes} and \textit{no}. Their findings demonstrate the historical value of transplanted dialects or speech island varieties. Moreover, a comparison of their findings with historical data demonstrates that reliance on rusty speaker data alone may sometimes lead to incorrect conclusions. Instead, such data patterns can also be considered from the perspective of language contact.
\section{Conclusion}
Contact scenarios and minoritization often involve unstable patterns of bilingualism (recall, for example, Chapter 3, which speaks of power differentials, economic marginalization, sociopolitical pressures, culture change, ethnic prejudice and discriminatory language policies). This leads us to  conclude that on the basis of the research presented here -- through investigations of contact effects on complexity, on grammatical shift, on power struggles in language revival, and on new methodologies for engaging with communities of diaspora and minoritized languages -- it would be fair to accept the claim that monolingualism can be viewed as a destructive force (as Carlos Fuentes once pointed out, "a curable disease") perhaps akin to monoculture farming, which threatens the diversity inherent to a healthy linguistic ecology. 
%disease (or perhaps even one symptom of a larger illness) that weakens societies and identities.
The opportunity to consolidate so many different kinds of research happening around the world on questions around contact scenarios and minoritization is scientifically compelling, It is also a testament to the extent to which contributors care  about these topics, having devoted untold hours of their time and energy to work towards a careful understanding of the past, present, and future of their linguistic ecologies. Thus, in a real sense, the contributors to this volume offer piece by piece new sources of validation and support for language diversity. The speakers of the languages represented here and others in minoritized or diaspora communities, by  continuing linguistic traditions in the face of discrimination are, in a very real sense, social activists.%, and is also a testament to the extent to which contributors show deep concern for these issues, having devoted untold hours of their time and energy to work towards a careful understanding of the past, present, and future of specific linguistic ecologies. Thus, in a real sense, the contributors are more than researchers -- they are social activists.



\sloppy
\printbibliography[heading=subbibliography,notkeyword=this]

\end{document}
