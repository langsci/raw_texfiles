\documentclass[output=paper, hidelinks]{langscibook}
\ChapterDOI{10.5281/zenodo.7446965}
\author{Zoë Belk\affiliation{Department of Linguistics, University College London} and Lily Kahn\affiliation{Department of Hebrew and Jewish Studies, University College London} and Kriszta Eszter Szendr\H{o}i\affiliation{Department of Linguistics, University of Vienna} and Sonya Yampolskaya\affiliation{Department of Hebrew and Jewish Studies, University College London}}
\title{Innovations in the Contemporary Hasidic Yiddish pronominal system}
\abstract{Although under existential threat in the secular world, Yiddish continues to be a native and daily language for Haredi (Hasidic and other strictly Orthodox) communities, with Hasidic speakers comprising the vast majority of these. Historical and demographic shifts, specifically in the post-War period, in the population of speakers have led to rapid changes in the language itself. These developments are so far-reaching and pervasive that we consider the variety spoken by today's Haredi speakers to be distinct, referring to it as Contemporary Hasidic Yiddish. This chapter presents a study involving 29 native Contemporary Hasidic Yiddish speakers, and demonstrates that significant changes have occurred in the personal pronoun, possessive, and demonstrative systems. Specifically, the personal pronoun system has undergone significant levelling in terms of case and gender marking, but a distinct paradigm of weak pronominal forms exists, independent possessives have lost case and grammatical gender distinctions completely, and a new demonstrative pronoun has emerged which exhibits a novel case distinction.}




%\IfFileExists{../localcommands.tex}{%hack to check whether this is being compiled as part of a collection or standalone
%move the following commands to the "local..." files of the master project when integrating this chapter
%\usepackage{tabularx}
%\usepackage{langsci-basic}
%\usepackage{langsci-optional}
%\usepackage{langsci-gb4e}
%\usepackage{multirow}
%\bibliography{localbibliography}
%\newcommand{\orcid}[1]{}
%the following allows columns in tabular to be centerd. 
%\newcolumntype{Y}{>{\centering\arraybackslash}X}
%\newcolumntype{R}{>{\raggedleft\arraybackslash}X}
%the following allows certain comments to appear in the pdf
%\usepackage[colorinlistoftodos]{todonotes}
%the following is for nice linguistics examples
%\usepackage{linguex}
%the following suppresses dashes in linguex example references
%\renewcommand{\firstrefdash}{}
%the following makes the first line of examples italicized
%\renewcommand{\eachwordone}{\itshape}
%the following is for making cells with line breaks in tables 
%\usepackage{array, makecell}
%the following allows us to present short examples in two columns
%\usepackage{multicol}
%the following helps with vertical centering in the tables
%\usepackage{array,booktabs}
%the following allows us to typeset URLs
%\usepackage{url}
%}{}

\IfFileExists{../localcommands.tex}{
 \addbibresource{../localbibliography.bib}
 % add all extra packages you need to load to this file

\usepackage{tabularx,multicol}
\usepackage{url}
\urlstyle{same}

\usepackage{listings}
\lstset{basicstyle=\ttfamily,tabsize=2,breaklines=true}

\usepackage{langsci-basic}
\usepackage{langsci-optional}
\usepackage{langsci-lgr}
\usepackage{langsci-osl}
% \usepackage{./langsci/styles/langsci-lgr}
% \usepackage{./langsci/styles/langsci-osl}
% \usepackage{langsci-gb4e}

\usepackage{tikz}
\usetikzlibrary{patterns,calc}
\pgfdeclarepatternformonly{south east lines}{\pgfqpoint{-0pt}{-0pt}}{\pgfqpoint{3pt}{3pt}}{\pgfqpoint{3pt}{3pt}}{
    \pgfsetlinewidth{0.6pt}
    \pgfpathmoveto{\pgfqpoint{0pt}{3pt}}
    \pgfpathlineto{\pgfqpoint{3pt}{0pt}}
    \pgfpathmoveto{\pgfqpoint{.2pt}{-.2pt}}
    \pgfpathlineto{\pgfqpoint{-.2pt}{.2pt}}
    \pgfpathmoveto{\pgfqpoint{3.2pt}{2.8pt}}
    \pgfpathlineto{\pgfqpoint{2.8pt}{3.2pt}}
    \pgfusepath{stroke}}
    
\usepackage{stmaryrd}
\usepackage{wasysym}
\usepackage{multirow}
\usepackage{caption}
\usepackage{subcaption}
\usepackage{mathrsfs}
\usepackage{qtree}

\usepackage{linguex}


 %pminos do not split footnotes
% \interfootnotelinepenalty=10000 %Footnote in Laporte chapters has to be split SN


%\DeclareIndexNameFormat{default}{%
%\nameparts{#1}%
%\usebibmacro{index:name}%
%{\index[names]}%
%{\namepartfamily}%
%{\namepartgiveni}%
% {}% L1
% {}% L2
%{\namepartprefix}% generates spurious space L3
%{\namepartsuffix}% generates spurious space L4
%}

%  {\DeclareIndexNameFormat{default}{%
%     \usebibmacro{index:name}{\index[names]}{#1}{#3}{#5}{#7}}}

%\DeclareIndexNameFormat{default}{%
%  \usebibmacro{index:name}{\sindex[nom]}{#1}{#3}{#5}{#7}}

%\DeclareIndexNameFormat{default}{%
%  \usebibmacro{index:name}{\sindex[person]}{#1}{#3}{#5}{#7}}
%\DeclareIndexNameFormat{default}{%
%\nameparts{#1} \usebibmacro{index:name}{\sindex[person]]}{\namepartfamily}{‌​\namepartgiven}{\nam‌​epartprefix}{\namepa‌​rtsuffix}}

%\newcommand{\smiley}{:)}

%\renewbibmacro*{index:name}[5]{%
%\usebibmacro{index:entry}{#1}%
%{\iffieldundef{usera}{}{\thefield{usera}\actualoperator}\mkbibindexname{#2}{#3}{#4}{#5}}}

% \newcommand{\noop}[1]{}

%remove for final
%\overfullrule=1mm

\newcommand{\tobi}[2]}}
\renewcommand{\S}[1]{\tobi{#1}{\textsc{*}}}

% this volume references
% puts: [this volume]
% already defined: \citetv
%\newcommand{\citepv}[1]{(\citeauthor{#1} \citeyear*{#1} [this volume])}
\newcommand{\citealtv}[1]{\citeauthor{#1} \citeyear*{#1} [this volume]}

%parentheses around example number
\newcommand{\pref}[1]{(\ref{#1})}

% in-text examples

\newcommand{\lnex}[1]{\textit{#1}} %target lang word
\newcommand{\lnlit}[1]{(lit.: `#1')} %literal reading
\newcommand{\lnlat}[1]{(#1)} % latinization
\newcommand{\lntrans}[1]{`#1'} %translation
\newcommand{\lnexl}[2]%
{\lnex{#1}{} \lnlat{#2}} % ex with latinization
\newcommand{\lnexlat}[3]{\lnex{#1}{} \lnlat{#2}{} \lntrans{#3}} % ex with latinization and tranl.

%ch01
\newcommand{\co}[1]{\mbox{\textbf{#1}}}

%ch09

\newcommand{\cyrbulg}[1]{\begin{otherlanguage*}{bulgarian}#1\end{otherlanguage*}}


%ch10
\newcommand{\nlp}{{\small NLP}}
\newcommand{\mwe}{{\small MWE}}
\newcommand{\rae}{{\small RAE}}
\newcommand{\lvc}{{\small LVC}}
\newcommand{\pos}{{\small P}o{\small S}}
%\newcommand{\todo}[1]{ \textcolor{red}{#1} }

%\renewcommand{\labelenumi}{\theenumi}
%\ainamefmt{{vv}{ll}{, ff}{, jj}} % fullname

\newcommand{\biberror}[1]{{\color{red}#1}}

\newcommand{\osenovaitem}{--~} 
 %% hyphenation points for line breaks
%% Normally, automatic hyphenation in LaTeX is very good
%% If a word is mis-hyphenated, add it to this file
%%
%% add information to TeX file before \begin{document} with:
%% %% hyphenation points for line breaks
%% Normally, automatic hyphenation in LaTeX is very good
%% If a word is mis-hyphenated, add it to this file
%%
%% add information to TeX file before \begin{document} with:
%% %% hyphenation points for line breaks
%% Normally, automatic hyphenation in LaTeX is very good
%% If a word is mis-hyphenated, add it to this file
%%
%% add information to TeX file before \begin{document} with:
%% \include{localhyphenation}
\hyphenation{
    Beck-man
    Ngu-yen
    back-chan-nel
    back-chan-nels
    mo-not-o-nous
    ste-reo-typ-i-cal
}

\hyphenation{
    Beck-man
    Ngu-yen
    back-chan-nel
    back-chan-nels
    mo-not-o-nous
    ste-reo-typ-i-cal
}

\hyphenation{
    Beck-man
    Ngu-yen
    back-chan-nel
    back-chan-nels
    mo-not-o-nous
    ste-reo-typ-i-cal
}
 
 \togglepaper[6]%%chapternumber
}{}

\begin{document}
\maketitle

\section{Introduction}\label{sec:introduction}

In this chapter, we describe the innovative features of the pronominal system of Contemporary Hasidic Yiddish. We begin with a brief historical introduction to the language and to its pronominal system.

Yiddish is the traditional language of the Ashkenazic (Central and Eastern European) Jews. It can be divided into two major varieties, Western and Eastern Yiddish. Western Yiddish was used by Jews living in the Western European regions corresponding to present-day Germany, France, the Netherlands, and Switzerland throughout the medieval and early modern periods, but was largely abandoned over the course of the 18\textsuperscript{th} century in favour of the dominant co-territorial languages and had become largely moribund by the 19\textsuperscript{th} century, though it retained a small spoken presence into the 20\textsuperscript{th} century (see \citealt{Fleischer18}). Eastern Yiddish was used by Jews in Eastern Europe, with the largest populations in regions corresponding to present-day Poland, Hungary, Romania, Ukraine, Lithuania, and Latvia. Henceforth all references to Yiddish in this chapter will denote Eastern Yiddish. Eastern Yiddish is characterised by a core Germanic morphosyntactic structure and lexis with a substantial Semitic (Hebrew and Aramaic) lexical component and a smaller but relatively high-frequency Slavic (chiefly Polish and Ukrainian) lexical component, with some Slavic-influenced grammatical features. It can itself be subdivided into three chief dialect areas, termed Northeastern or Lithuanian Yiddish (traditionally spoken in areas corresponding primarily to present-day Lithuania, Latvia, and Belarus), Mideastern or Central Yiddish (traditionally spoken in areas corresponding primarily to present-day Poland and Hungary), and Southeastern or Ukrainian Yiddish (traditionally spoken in areas corresponding primarily to present-day Ukraine). Yiddish in Eastern Europe served not only as a vernacular but also as a written language; it existed in a diglossic relationship with Hebrew, with the former typically used for more low-prestige types of writing and the latter functioning as the high-prestige written vehicle though not as a vernacular. (See \citealt{Harshav90} for discussion of the traditional Hebrew-Yiddish diglossia in Eastern Europe.) In the 1920s and 1930s, a standardised variety of Yiddish based largely on the Northeastern dialect was developed by the Vilna-based YIVO Institute, an organisation devoted to Yiddish pedagogy and linguistic research as well as other scholarly activities focused on Eastern European Jewry (see \citealt{Kuznitz10}).

Hasidism is a Jewish spiritual movement which originated in late 18\textsuperscript{th}-century Ukraine and grew to become a prominent force in Eastern European Jewish society over the course of the 19\textsuperscript{th} century (see \citealt{Biale18} for a historical overview of Hasidism). Like other Eastern European Jews, most followers of the Hasidic movement spoke Yiddish as their native and main language. During the 19\textsuperscript{th} century the various Hasidic rebbes, or spiritual leaders, established dynasties that were typically named after the locations where they were founded. Most of the Hasidic dynasties (e.g. Belz, Satmar, Skver, Tosh, Vizhnitz) were based in the Mideastern (Polish and Hungarian) and Southeastern (Ukrainian) Yiddish dialect areas, with a smaller number (e.g. Chabad and Karlin) based in the Northeastern (Lithuanian) dialect area. The Yiddish spoken by followers of the Hasidic movement did not differ from that of their non-Hasidic counterparts but rather varied by regional dialect. This traditional situation changed dramatically with World War II and the cataclysmic destruction of the majority of Yiddish speakers during the Holocaust. Surviving members of the various Hasidic dynasties were dispersed from their traditional homelands in Eastern Europe and resettled in new locations around the world, chiefly the New York area in the US, a variety of communities in Israel, the Montreal area in Canada, London’s Stamford Hill neighbourhood in the UK, and Antwerp in Belgium. This rapid geographical shift led to a complete realignment whereby speakers of different varieties of Eastern European Yiddish were now living side by side, which contributed to a high degree of dialect contact and mixing. Moreover, these post-War Hasidic communities had little contact with secular Yiddish speakers in the new locations (in contrast to Eastern Europe, where Hasidic and non-Hasidic Yiddish-speaking Jews had typically lived in the same areas), and as such Hasidic Yiddish began to develop separately. This rapid shift was compounded by the fact that these post-War Hasidic communities absorbed a substantial number of L2 speakers who adopted the language as adults. These post-War Hasidic centers also came to accommodate smaller groups of non-Hasidic Haredi (strictly Orthodox) Jews, primarily from Northeastern Yiddish dialect areas, which became increasingly integrated with their Hasidic counterparts as both communities had much in common due to their shared Haredi cultural tradition. As a result of these interrelated factors, the Yiddish of these new Haredi (primarily Hasidic) communities developed very quickly, to the extent that in the 21\textsuperscript{st} century, it can be regarded as a distinct variety of the Yiddish language. Indeed, our previous research has suggested that a greater amount of dialect mixing and L2 speakers at a community level in the years since World War II is associated with increased use of innovative features; see \citet{Author21} for further discussion. We term this new variety Contemporary Hasidic Yiddish. 

One of the most prominent of these is a complete absence of morphological noun case and gender, which contrasts dramatically from the pre-War tripartite case and gender system (see \citealt{Author20, Author21}), although see \citet{Assouline14} for the claim that animate nouns in Jerusalemite Yiddish reliably retain gender but not case morphology.\footnote{Note that Northeastern Yiddish only had two morphological genders, although it had the same three cases as other pre-War varieties of Yiddish \citep{Jacobs90}.} This rapid loss of morphological case and gender, as well as the above-mentioned dialect mixing which contributed to the emergence of Contemporary Hasidic Yiddish, are linked to a number of innovations in the pronominal system. There has been very little research into Hasidic Yiddish pronouns in general, namely Assouline’s (\citeyear{Assouline10}) study of the first person plural pronoun in Haredi Jerusalemite Yiddish, Nove's (\citeyear{Nove18}) study of case syncretism in the first and second person objective singular pronouns in New York Hasidic Yiddish, and Sadock \& Masor’s (\citeyear[95, 103]{Sadock18}) observation on the use of the demonstratives \textit{dey} and \textit{deye} in the language of Bobover Hasidic speakers in New York; \citet{Assouline14} has also briefly discussed this form, while \citet{Krogh12} mentions the existence of \textit{deys}. We seek to complement these previous studies by examining the major innovative features of the Contemporary Hasidic Yiddish pronominal system based on elicited spoken data provided by native speakers in the main Hasidic centers globally. 

Our investigation focuses on the personal pronoun paradigm, strong vs. weak pronouns, possessives, and demonstratives. Our research found that Contemporary Hasidic Yiddish has undergone a realignment of the personal pronoun paradigm, with an increase in case syncretism in both the singular and plural whereby the singular forms exhibit a two-way nominative / objective distinction and the plural forms tend to exhibit no case distinctions, as well as the introduction of a distinct 2\textsc{pl.hon} pronoun. The pronominal system includes a distinction between strong and weak forms, which have different morphological and syntactic properties. With respect to the possessives, our main findings are that there is a clear distinction between singular and plural dependent forms, and between dependent and independent forms, with the independent forms exhibiting a complete lack of case and gender morphology (in contrast to pre-War and Standard Yiddish). Our main findings with respect to the demonstratives are that there is a novel distinction between nominative and objective independent demonstratives suggesting the innovation of a distinct pronominal form, as well as a complete lack of case, gender, and number distinctions. Furthermore, the distribution of the `proximal' and `distal' forms differs from that of pre-War dialects, English and Modern Hebrew.


\subsection{The pronominal system of pre-War and Standard Yiddish}	
\largerpage
Almost all traditional geographical dialects of Eastern Yiddish, as well as Standard Yiddish, exhibit the same tripartite case system for personal pronouns as for full nominals \citep{Kahn16}. Pronouns decline in the nominative, accusative, and dative, as shown below (based on \citealt[678]{Kahn16}); note that there is a degree of syncretism present in the paradigm, provided in Table \ref{tab:StandardParadigm}.\footnote{Published descriptions of the different dialects of pre-War Yiddish are often incomplete or only cover larger dialect areas. It is therefore possible that some of the forms described in this chapter existed in pre-War dialects, but this claim is difficult to substantiate. In any case, it is clear that our findings represent a much more widespread and generalised phenomenon across the Hasidic Yiddish speech community than existed in pre-War Yiddish.}

%TABLE
\begin{table}
\caption{Case and gender marking on personal pronouns in pre-War/Standard Yiddish}
\label{tab:StandardParadigm}
 \begin{tabularx}{\textwidth}{llYYYYYY}
  \lsptoprule
  	  & & Singular &  &  & Plural &  &  \\
	  & & \textsc{nom} & \textsc{acc} & \textsc{dat} & \textsc{nom} & \textsc{acc} & \textsc{dat} \\
  \midrule
	1\textsuperscript{st} & & \textit{ikh} & \textit{mikh} & \textit{mir} & \textit{mir} & \multicolumn{2}{c}{\textit{undz}}  \\
\midrule
	\multirow{2}{*}{2\textsuperscript{nd}} & \textsc{Fam} & \textit{du} & \textit{dikh} & \textit{dir} & \multirow{2}{*}{\textit{ir}}  & \multicolumn{2}{c}{\multirow{2}{*}{\textit{aykh}}} \\
	& \textsc{Hon} & \textit{ir} & \multicolumn{2}{c}{\textit{aykh}} & &  \\
\midrule
	\multirow{3}{*}{3\textsuperscript{rd}} & \textsc{Masc} & \textit{er} &  \multicolumn{2}{c}{\textit{im}}    &     \multicolumn{3}{c}{\multirow{3}{*}{\textit{zey}}}  \\
	& \textsc{fem} & \multicolumn{2}{c}{\textit{zi}}  &  \textit{ir}  &   \\
	& \textsc{Neut} & \multicolumn{2}{c}{\textit{es}}   &  \textit{im}  &     \\
  \lspbottomrule
 \end{tabularx}
\end{table}

Within this system there is a degree of regional variation; for example, the second person plural form in certain Mideastern Yiddish varieties is \textit{ets} (\textsc{nom}) or \textit{enk} (\textsc{Acc/Dat}) rather than \textit{ir} \citep[70]{Jacobs05}. Similarly, there is syntactic variation whereby particular verbs that take an accusative in most dialects of Eastern Yiddish instead take a dative in certain local varieties; this can be seen in examples involving the first person singular and third person feminine singular pronouns. For example, the verb \textit{kenen} ‘to know [a person]’ typically takes the accusative, i.e. \textit{er ken mikh} `he knows me’, \textit{er ken zi} `he knows her’, but the dative in certain (mostly Northeastern but also some Mideastern) varieties, i.e. \textit{er ken mir}, \textit{er ken ir} \citep[142--147]{Wolf69}. \citet[184]{Jacobs05} goes further than Wolf in claiming that pre-War Northeastern Yiddish lacked the historical three-way distinction in the 1\textsc{sg}, 2\textsc{sg}, and 3\textsc{fs} entirely, employing the historically dative forms in both accusative and dative settings. Conversely, use of the accusative in contexts where the dative would typically be used do not seem to be attested \citep[142--146]{Wolf69}. It is important to note that, as with the noun gender and case variation discussed above, these phenomena seem to be restricted to specific individual verbs, rather than pointing to a breakdown in the pronominal case system as a whole. However, the first person phenomenon may be at least partly based on a lack of phonological distinction between the sounds /r/ (often realised as a uvular trill or fricative) and /χ/ in the dialects in question, and may have constituted the first step in a more widespread hypothetical future merger \citep[149]{Wolf69}.



\subsubsection{Verbal agreement}

Verbal agreement for pre-War and Standard Yiddish is shown in Table \ref{tab:standardverbs}. Note that the Mideastern first person plural pronoun \textit{undz} and the second person plural pronoun \textit{ets} are dialectal, and take the same verbal agreement as the equivalent pronouns \textit{mir} and \textit{ir}, namely \textit{-(e)n} and \textit{-t} respectively.\footnote{The 1\textsc{pl} dialectal variant \textit{undz} also appeared with the verbal suffix \textit{-mer/mir}, e.g. \textit{undz zog(n)mer/undz zog(n)mir}, see \citet[70, 189]{Jacobs05}. For discussion of the verbal suffix \textit{-ts} as agreeing with the pronoun \textit{ets}, see Section \ref{sec:verbalfindings}.} 

\begin{table}
\caption{Subject-verb agreement morphology in pre-War and Standard Yiddish}
\label{tab:standardverbs}
 \begin{tabularx}{\textwidth}{llYlYl}
  \lsptoprule
  	  & & \multicolumn{2}{c}{Singular} & \multicolumn{2}{c}{Plural}  \\
  \midrule
	1\textsuperscript{st} & & \textit{ikh} & \textit{V-\varnothing} & \textit{mir (/undz)} & \textit{V-(e)n}  \\
\midrule
	\multirow{2}{*}{2\textsuperscript{nd}} & \textsc{Fam} & \textit{du} & \textit{V-st} & \multirow{2}{*}{\textit{ir (/ets)}} & \multirow{2}{*}{\textit{V-t}}  \\
	& \textsc{Hon} & \textit{ir} & \textit{V-t} & &  \\
\midrule
	\multirow{3}{*}{3\textsuperscript{rd}} & \textsc{Masc} & \textit{er} & \multirow{3}{*}{\textit{V-t}} &  \multirow{3}{*}{\textit{zey}} & \multirow{3}{*}{\textit{V-(e)n}}  \\
	& \textsc{fem} & \textit{zi} & & & \\
	 & \textsc{Neut} & \textit{es} & & & \\
  \lspbottomrule
 \end{tabularx}
\end{table}

\subsubsection{Reflexives}

The reflexive pronoun in Northeastern and Southeastern Yiddish, as well as in the standardised variety, is the invariant form \textit{zikh} `oneself’, which is used for all persons and numbers without case or gender distinctions. In Mideastern Yiddish, by contrast, the 1\textsc{pl} and 2\textsc{pl} accusative forms of the personal pronouns (\textit{mikh} and \textit{dikh} respectively) are used as reflexive forms, while the 1\textsc{pl} and 2\textsc{pl} accusative/dative forms of the personal pronouns (\textit{undz} and \textit{enk} respectively) are used as reflexive forms, though less consistently than in the singular \citep[184--185]{Jacobs05}.


\subsubsection{Possessive pronouns}

%fix text here

The pre-War and Standard Yiddish possessive pronouns show a distinction between dependent and independent forms. Dependent forms lack any case and gender distinctions, but distinguish between forms used to modify a singular noun and forms used to modify a plural noun. The pre-War/Standard Yiddish dependent possessive pronouns are shown in Table \ref{tab:depstandposs}. 


\begin{table}
\caption{Pre-War/Standard Yiddish dependent possessive pronouns }
\label{tab:depstandposs}
 \begin{tabularx}{\textwidth}{llYYYY}
  \lsptoprule
  	  & & \multicolumn{2}{c}{Sg. possessor}   & \multicolumn{2}{c}{Pl. possessor}  \\
	  & & Sg. possessum & Pl. possessum & Sg. possessum & Pl. possessum  \\
  \midrule
	1\textsuperscript{st} & & \textit{mayn} & \textit{mayne}  & \textit{undzer} & \textit{undzere}  \\
\midrule
	\multirow{2}{*}{2\textsuperscript{nd}} & \textsc{Fam} & \textit{dayn} & \textit{dayne}  & \multirow{2}{*}{\textit{ayer}} & \multirow{2}{*}{\textit{ayere}} \\
	& \textsc{Hon} & \textit{ayer} & \textit{ayere}  & &  \\
\midrule
	\multirow{3}{*}{3\textsuperscript{rd}} & \textsc{Masc} & \textit{zayn} &  \textit{zayne}   &     \multirow{3}{*}{\textit{zeyer}}  &  \multirow{3}{*}{\textit{zeyere}}  \\
	& \textsc{fem} & \textit{ir} &  \textit{ire}  &   & \\
	&  \textsc{Neut} & \textit{zayn}   &  \textit{zayne}  &    & \\
  \lspbottomrule
 \end{tabularx}
\end{table}

The independent possessive pronouns, by contrast, take case and gender markings in accordance with the case and gender of the associated noun. This system is exemplified in Table \ref{tab:indstandposs}, which shows the various forms of the first person singular independent possessive pronoun. There are also postpositive forms of the dependent possessive pronouns, and these take the same case and gender endings shown in Table \ref{tab:indstandposs} (e.g. \textit{mitn khaver maynem} `with-\textsc{def.m.sg.dat} my-\textsc{m.sg.dat} friend.\textsc{m}’). See \citet[108--112]{Katz87} and \citet[183--184]{Jacobs05} for further discussion of the possessive pronouns in pre-War and Standard Yiddish.



\begin{table}
\caption{Pre-War/Standard Yiddish independent 1\textsc{sg} possessive pronoun forms}
\label{tab:indstandposs}
 \begin{tabularx}{\textwidth}{lYYYY}
  \lsptoprule
  	& \textsc{Masc} & \textsc{fem} & \textsc{Neut} & \textsc{Pl} \\
\midrule
	\textsc{nom} & \textit{mayner} & \textit{mayne} & \textit{mayns} & \multirow{3}{*}{\textit{mayne}}\\	\textsc{acc} & \textit{maynem} & \textit{mayne} & \textit{mayns} & \\
	\textsc{Dat} & \textit{maynem} & \textit{mayner} & \textit{maynem} & \\
    \lspbottomrule
 \end{tabularx}
\end{table}



\subsubsection{Demonstratives}

The pre-War and Standard Yiddish demonstrative pronouns exhibit morphological case and gender, with masculine, feminine, and neuter forms (except for Northeast Yiddish, which has only masculine and feminine forms). There was a set of proximal demonstratives and a set of distal demonstratives, but no distinction between dependent and independent forms. The stressed definite article was used for the proximal demonstratives, as shown in Table \ref{tab:standproxdem}. The definite article could additionally be accompanied by prepositive \textit{ot} or by postpositive \textit{dozik-} (plus case and gender suffixes) to reinforce the demonstrative sense. See \citet[112--114]{Katz87} for further discussion of the proximal demonstratives in pre-War and Standard Yiddish. 


\begin{table}
\caption{Pre-War/Standard Yiddish proximal demonstratives}
\label{tab:standproxdem}
 \begin{tabularx}{\textwidth}{lYYYY}
  \lsptoprule
  	& \textsc{Masc} & \textsc{fem} & \textsc{Neut} & \textsc{Pl} \\
\midrule
	\textsc{nom} & \textit{der} & \textit{di} & \textit{dos} & \multirow{3}{*}{\textit{di}}\\
	\textsc{acc} & \textit{dem} & \textit{di} & \textit{dos} & \\
	\textsc{Dat} & \textit{dem} & \textit{der} & \textit{dem} & \\
    \lspbottomrule
 \end{tabularx}
\end{table}

The distal demonstratives are based on the stem \textit{yen-}, which takes case and gender endings, as in Table \ref{tab:standdisdem}. Note that Northeastern Yiddish typically lacked the neuter form \textit{yens}, instead employing the masculine or feminine forms. See \citet[115--116]{Katz87} and \citet[186]{Jacobs05} for further discussion of the distal demonstratives in pre-War and Standard Yiddish. 


\begin{table}
\caption{Pre-War/Standard Yiddish distal demonstratives}
\label{tab:standdisdem}
 \begin{tabularx}{\textwidth}{lYYYY}
  \lsptoprule
  	& \textsc{Masc} & \textsc{fem} & \textsc{Neut} & \textsc{Pl} \\
\midrule
	\textsc{nom} & \textit{yener} & \textit{yene} & \textit{yens} & \multirow{3}{*}{\textit{yene}}\\
	\textsc{acc} & \textit{yenem} & \textit{yene} & \textit{yens} & \\
	\textsc{Dat} & \textit{yenem} & \textit{yene} & \textit{yenem} & \\
    \lspbottomrule
 \end{tabularx}
\end{table}




\subsubsection{Road map}

In the remainder of this chapter, we provide an overview of our participants and study design (section \ref{sec:methods}) and introduce the personal pronoun system of Contemporary Hasidic Yiddish including strong and weak pronoun forms (Section \ref{sec:personalpronouns}). We then discuss innovations in possessives (Section \ref{sec:possessives}) and demonstratives (section \ref{sec:demonstratives}). Section \ref{sec:conclusions} provides some concluding remarks.


\section{Methodology}\label{sec:methods}

\subsection{Participants}

Our analysis of the pronoun system in Contemporary Hasidic Yiddish is based on interviews with 29 
native speakers between the ages of 18 and 72 who were born and raised in Haredi communities worldwide. We worked with 15 participants from the New York area (six female) from a range of Haredi neighbourhoods in Brooklyn, as well as several Haredi communities in upstate New York. Five participants (two female) were from communities in Israel such as Bnei Brak, Ashdod, and in and around Jerusalem. In addition, five participants (four female) were from London's Stamford Hill community, two (both female) were from the Montreal area, and two (both male) were from Antwerp.\footnote{Given these sample sizes, the Montreal and Antwerp data should be viewed as some indication that the changes we describe are generalised, but more data would be necessary to get a more reliable picture.} 

Contemporary Hasidic Yiddish constitutes a distinct variety of the language, although it is most closely related to the historical Mid- and Southeastern dialects. Our participants largely identify as speaking \textit{khsidishe yidish} `Hasidic Yiddish', which is associated with a vowel profile most closely matching that of traditional Mid- and Southeastern dialects. Participants from Chabad and those non-Hasidic Haredi speakers who refer to themselves as `Litvish’ or `yeshivish' typically speak a variety more closely related to the historical Northeastern dialect, with the associated vowel profile.\footnote{Certain other groups use a similar vowel profile, notably the Yerushalayimer or Jerusalem Haredi community. See \citealt{Author21} for further discussion.} These groups are often distinguished by their pronunciation of the word וואס ‘what’, with the former group, who pronounce it as [vʊs], described as speaking `vus' and the latter, who pronounce it [vɔs], described as speaking `vos'. We will use these terms throughout the rest of the chapter. In our sample, ‘vos’ speakers include participants from Chabad and those non-Hasidic Haredi speakers who refer to themselves as ‘Litvish’ Yiddish speakers. Those who speak ‘vus’ in our dataset include a wide range of other Hasidic affiliations, including Belz, Dushinsky, Karlin, Pupa, Satmar, Skver, Toldos Avrohom Yitzchok, Tosh, Tsanz, Vizhnitz, Vizhnitz-Monsey, and so-called ‘klal Hasidish’, i.e. non-specific/general Hasidic. It is important to note also that while some Hasidic sects are associated with one pronunciation or the other (e.g. Satmar and Belz are associated with ‘vus’ while Chabad and Karlin are associated with ‘vos’), individual speakers inside those communities might differ. Our sample includes 24 speakers of ‘vus’ (14 female), four speakers of ‘vos’ (none female), and one (female) who speaks a ‘mixed’ phonological variety skewing mostly towards `vos'. Three of the ‘vos’ speakers are from the New York area and one is from Israel.\footnote{Again, given the relative sizes of the samples of `vus' and `vos' speakers, data on on the `vos' variety should be taken as indicative. However, it should be noted that relatively little variation was observed in the language of the `vos' speakers.}

All of our participants were raised in Yiddish-speaking homes and were largely educated in Yiddish, particularly in their early years.\footnote{The medium of education varies by gender, especially in the later years, with boys largely being educated in Yiddish and \textit{loshn koydesh} (the traditional Yiddish term for pre-modern Hebrew, and girls receiving more education, especially in secular subjects, in a co-territorial language.} Many of our participants use Yiddish on a daily basis with family, friends and business contacts. Others employ it only on occasion (e.g. when talking to particular family members or friends). Some speakers also employ English, Modern Hebrew, and/or French (depending on the location) on a regular basis, while others speak Yiddish almost exclusively. All participants are comfortable reading and writing in Yiddish, though not all of them regularly employ the language in these ways. 

The participant codes we use in this paper take the following form. The first character indicates the geographical community in which the participant grew up (N=New York, I=Israel, S=Stamford Hill, M=Montreal, A=Antwerp), the second character represents the participant speaks `vus' or `vos' (U=`vus', O=`vos', M=mixed), and the third character is serves to provide a unique identifier. 


\subsection{Description of tasks}

Our study of developments in the Contemporary Hasidic Yiddish pronominal system is primarily based on elicited oral data. Participants were presented with a series of short sentences in either English or Modern Hebrew (according to the participant's preference) and asked to translate each into Yiddish. The sentences target personal pronouns in each of the person, gender and case combinations, as well as second person familiar and honorific contexts; independent and dependent possessives for each of the pronominal persons in both singular and plural contexts; and independent and dependent demonstratives (both proximal and distal) in singular and plural nominative, accusative, and dative contexts. 

In addition, some participants completed a written task which targeted the familiar/honorific distinction in more detail. This task presented participants with a variety of scenarios in Yiddish involving direct address to a range of interlocutors, and participants were asked to provide the form of address they would most naturally use. 

Participants also engaged in Yiddish-, English-, and Modern Hebrew-medium metalinguistic discussion with the experimenter which both provided additional unelicited data and insight into their linguistic choices. In certain cases, we also made use of Yiddish-language Hasidic internet resources to verify tendencies observed in our elicited data. 

The interview recordings were transcribed in ELAN by fluent or native Yiddish speakers using a modified version of the YIVO standard transliteration system and a modified IPA. 

\subsection{A note on the transliteration system used in this paper}

In most instances, the data presented in this paper are represented in writing according to the standard transliteration system developed by the YIVO Institute, which is in widespread use for Romanisation of Yiddish. This transliteration system is based on the Hebrew-script orthography designed by the same Institute, which is the standard orthography throughout the non-Haredi Yiddish-speaking world. This system is based primarily on the North- and Southeastern vowel profiles of Eastern Yiddish, and as such the representation employed here obscures some of the pronunciation features characteristic of the Mideastern dialect region, which is more typical of the majority of our participants. In most instances, the differences in vowel patterns between these dialect areas are predictable. For example, the vowel /u/ is regularly pronounced as [i] in the Mideastern dialect area, and consequently also in the speech of our ‘vos’ participants; conversely, the diphthong /ɔɪ/ is regularly pronounced as [ɛɪ] in the Northeastern dialect area, and consequently also in the speech of our ‘vos’ participants. As the YIVO transliteration system serves to represent a variety of actual phonetic realisations, we have supplemented it with IPA representations where required for clarity. 


\section{Personal pronouns}\label{sec:personalpronouns}
\sloppy
The paradigm of personal pronouns in Contemporary Hasidic Yiddish exhibits a number of innovative uses of the forms known from the pre-War and Standard varieties of Yiddish. In this section we discuss the basic personal pronoun paradigm, and describe the distinction between the strong and weak forms of these pronouns.
\fussy

\subsection{The personal pronoun system of Contemporary Hasidic Yiddish}

There is a considerable amount of variation in the Contemporary Hasidic Yiddish pronoun paradigm, primarily between different speakers, but also sometimes within the same speaker. This variation seems to reflect a pronominal paradigm in flux, a situation which is likely ascribable to the increased degree of dialect mixing in the speech of previous generations of speakers who came together in the new Hasidic centers in the immediate post-War period; see also \citet{Nove18} for a discussion of the role that sociolinguistic factors play in the tendency towards accusative/dative syncretism in first and second singular pronouns in New York Hasidic Yiddish.
Five main pronominal paradigms can be distinguished among the speakers. The first two, shown in Tables \ref{tab:vusmikh} and \ref{tab:vusmir}, are the most common and, combined, account for most speakers.\footnote{We transliterate the 3\textsc{ms} objective form as \textit{em} despite the fact that in the standard YIVO transliteration system it is represented as \textit{im} because the vast majority of our participants pronounce it as \textit{em} and it does not represent a phonologically predictable vowel change (in contrast to other forms represented here by the YIVO transliteration system, such as \textit{undz}, which speakers with a `vus' vowel profile typically pronounce as [indz] or [ints] in accordance with a predictable regionally based \textit{u/i} vowel alternation plus a final devoicing process). One Chabad speaker with a ‘vos’ vowel profile produces some tokens of [im], which corresponds to the spelling used in most Hebrew-script orthographic conventions and in the standard YIVO system. The form \textit{eym} was also found in some, especially Mideastern, pre-War dialects \citep{Weinreich64, Jacobs05}.}\footnote{In the third person singular, the masculine and feminine forms are almost exclusively used for animate entities by speakers, replacing a former grammatical gender distinction with a semantic one. The former neuter form, \textit{es}, is now used as an inamimate third person singular pronoun. However, there is quite a lot of variation in our data with respect to the (morpho)phonological form of this morpheme and our data questionnaire was not designed to probe this issue. Therefore, in this paper, we do not provide inanimate third singular pronominal forms and leave this topic for future research.

We also note that while the masculine and feminine singular pronouns are reserved for (usually human) animates for most speakers, Israeli speakers sometimes deviate from this general rule by employing the masculine \textit{er} or feminine \textit{zi} 3\textsc{sg} pronouns in conjunction with inanimate nouns in accordance with the gender of those nouns in Modern (Israeli) Hebrew, as exemplified in \REF{ex:foo9}. 

\ea\label{ex:foo9} \gll neyn, de seyfer iz nisht mayne, er iz dayne (IU1) \\
no the book is not mine {it (lit: he)} is yours\\
\z
}



\begin{table}
\caption{‘Vus’ paradigm with \textit{mikh} and \textit{dikh} in both accusative and dative positions }
\label{tab:vusmikh}
 \begin{tabularx}{\textwidth}{llYYYY}
  \lsptoprule
  	  & & Singular  &  & Plural &  \\
	  & & \textsc{nom} & \textsc{obj} & \textsc{nom} & \textsc{obj}\\
  \midrule
	\multirow{2}{*}{1\textsuperscript{st}} & & \textit{ikh} & \textit{mikh} & \multicolumn{2}{c}{\textit{undz}}  \\
	& & /iχ/ & /miχ/ &  \multicolumn{2}{c}{/inz/}\\
\midrule
	\multirow{4}{*}{2\textsuperscript{nd}} & \multirow{2}{*}{\textsc{Fam}} & \textit{du} & \textit{dikh}   & \multicolumn{2}{c}{\textit{enk}} \\
	& & /di/ & /diχ/ & \multicolumn{2}{c}{/ɛŋk/}\\
	& \multirow{2}{*}{\textsc{Hon}} & \textit{ir} & \textit{aykh} & \textit{ir} & \textit{aykh}  \\
	& & /ir/ & /ɑːχ/ & /ir/ & /ɑːχ/\\
\midrule
	\multirow{4}{*}{3\textsuperscript{rd}} & \textsc{Masc} & \textit{er} &  \textit{em}   &     \multicolumn{2}{c}{\multirow{4}{*}{\makecell{\textit{zey} \\ /zaɪ/}}}  \\
	& & /ɛr/ & /ɛm/, /ɛɪm/ &\\
	& \textsc{fem} & \textit{zi}  &  \textit{zi/ir}    & \\
	& & /zi/ & /zi/, /ir/ & \\
	%&  \textsc{Neut} & \textit{se, es, 's}   &  \textit{es} &    \\
  \lspbottomrule
 \end{tabularx}
\end{table}

\begin{table}
\caption{‘Vus’ paradigm with \textit{mir} and \textit{dir} in both accusative and dative positions }
\label{tab:vusmir}
 \begin{tabularx}{\textwidth}{llYYYY}
  \lsptoprule
  	  & & Singular  &  & Plural &  \\
	  & & \textsc{nom} & \textsc{obj} & \textsc{nom} & \textsc{obj}\\
  \midrule
	1\textsuperscript{st} & & \textit{ikh} & \textit{mir} & \multicolumn{2}{c}{\textit{undz}} \\
	 & & & /mir/ &\\
\midrule
	\multirow{4}{*}{2\textsuperscript{nd}} & \multirow{2}{*}{\textsc{Fam}} & \textit{du} & \textit{dir}   & \multicolumn{2}{c}{\textit{enk}} \\
	 & & & /dir/ &\\
	& \multirow{2}{*}{\textsc{Hon}} & \textit{ir} & \textit{aykh} & \textit{ir} & \textit{aykh}  \\
	 & & & /ɑːχ/, /aɪχ/ & & /ɑːχ/, /aɪχ/ \\
\midrule
	\multirow{2}{*}{3\textsuperscript{rd}} & \textsc{Masc} & \textit{er} &  \textit{em}   &     \multicolumn{2}{c}{\multirow{2}{*}{\textit{zey}}}  \\
	& \textsc{fem} & \textit{zi}  &  \textit{zi/ir}    & \\
	%&  \textsc{Neut} &  \textit{se, es, 's}   &  \textit{es} &    \\
  \lspbottomrule
 \end{tabularx}
\end{table}


Together these two paradigms reflect the speech of many speakers from the `vus’ vowel profile. The only difference between them is that in the paradigm shown in Table \ref{tab:vusmikh}, the 1\textsc{sg} and 2\textsc{sg} forms appear as \textit{mikh} and \textit{dikh} respectively in both the accusative and dative positions, whereas in Table \ref{tab:vusmir}, these same forms appear as \textit{mir} and \textit{dir} respectively in both case positions. These two paradigms reflect different patterns of case syncretism: in the first, the traditionally accusative 1\textsc{sg} and 2\textsc{sg} forms have been extended to the dative as well, while in the second, the traditionally dative 1\textsc{sg} and 2\textsc{sg} forms have been extended to the accusative. This syncretism may have been triggered to some extent by the variation in use of accusative vs. dative forms attested in certain regional pre-War varieties of Yiddish as discussed above (and analyzed in more detail in \citet[142--147]{Wolf69}; see also \citet{Jacobs05} on the lack of \textit{mikh/dikh} forms in Northeastern Yiddish): given the increased mixing of speakers from different Yiddish dialect areas in the post-War period, it is possible that speakers who acquired the language in the late 1940s and later were exposed to a large amount of variation in these forms and this contributed to a paradigm levelling whereby only one form was selected for both the accusative and dative.\footnote{We believe the direction of influence to be from `vus' to `vos' speakers; see \citet{Author21} for details.} However, there do not seem to be any clear patterns governing the use of one pattern (\textit{mikh/dikh} vs. \textit{mir/dir}) for any particular speaker (e.g. age, Hasidic affiliation, gender), except that our Israeli participants all have the \textit{mir/dir} pattern. 

Apart from these two distinct patterns for the 1\textsc{sg} and 2\textsc{sg} accusative/dative forms, both paradigms are identical and contain a number of other innovative features. First, the 1\textsc{pl} is \textit{undz} in the nominative, accusative, and dative positions, and as such has no case distinctions. This phenomenon has precedent in pre-War Polish and Hungarian dialects of Yiddish \citep[70]{Weinreich64, Jacobs05}. However, the use of nominative \textit{undz} has spread to younger ‘vos’ speakers, resulting in the innovative nominative form \textit{undz}, which never existed in Lithuanian Yiddish (where the nominative form of the 1\textsc{pl} was exclusively \textit{mir}, with \textit{undz} reserved for the accusative/dative). For example, IO1, a younger ‘Litvish’ speaker, uses nominative \textit{undz}, in contrast to older ‘Litvish’ speaker NO1, who uses the traditional nominal form \textit{mir}. This points to a trend observed elsewhere in the grammar of contemporary Haredi Yiddish whereby the speech of the larger ‘vus’ speaking population has exerted a noticeable influence over that of the smaller ‘vos’-speaking counterparts (see \citealt{Author21}). 

Second, the 2\textsc{pl} form in both of these paradigms is uniformly \textit{enk}, with no case distinctions (like the 1\textsc{pl}). However, in contrast to the 1\textsc{pl}, the use of \textit{enk} in the nominative position seems to be without precedent in pre-War varieties of Yiddish. In the pre-War Mid- and Southeastern dialects, the 2\textsc{pl} nominative form was \textit{ets} \citep[70]{Jacobs05}, with \textit{enk} reserved for the accusative and dative. Speakers of Contemporary Hasidic Yiddish, particularly the younger generations of ‘vus’ speakers, appear to have lost the traditional nominative \textit{ets} and adopted the accusative/dative form in all positions. This may have evolved on analogy with the earlier use of \textit{undz} in all positions, as well as with the more universal traditional use of the 3\textsc{pl} \textit{zey} in all positions in all forms of Eastern European Yiddish. A small minority of our participants still maintain nominative \textit{ets}, but use it in free variation with nominative \textit{enk}, suggesting that they had acquired the traditional form \textit{ets} from their parents and other older-generation speakers, but had already lost the clear case distinction between \textit{ets} and \textit{enk}, which they can use in all positions. 

The 3\textsc{fs} pronoun is also employed differently than in the pre-War and Standard varieties of Yiddish. As shown in Section \ref{sec:introduction}, the traditional 3\textsc{fs} forms were \textit{zi} in the nominative and accusative, and \textit{ir} in the dative. By contrast, in Contemporary Hasidic Yiddish, some speakers employ \textit{zi} only in the nominative, and \textit{ir} in both the accusative and dative, while a much smaller number of speakers employ \textit{zi} in all positions, and do not have \textit{ir} in their feminine pronominal paradigm. As discussed with respect to the 1\textsc{sg} and 2\textsc{sg} accusative and dative forms, both of these innovations may be ascribable in part to the fact that the distribution of \textit{zi} and \textit{ir} in the accusative varied to some extent by region in pre-War Yiddish dialects. This variation may have led to a) a shift from the use of \textit{ir} in the dative only to a single objective form, \textit{ir}, among most speakers, and b) the loss of \textit{ir} and adoption of \textit{zi} in all positions among a small subset of speakers. Whichever pattern our participants employ, none of them follows the pre-War model of \textit{zi} in the nominative and accusative, and \textit{ir} in the dative. This difference is consistent with the rest of the paradigms shown in Tables \ref{tab:vusmikh} and \ref{tab:vusmir}, in which there are two possibilities with regard to case: a) there is only one form that is used in all case environments, such as \textit{undz}, \textit{enk}, and \textit{zey}, or b) there is one form used for the nominative and a second form used for the objective (accusative/dative), e.g. \textit{ikh} vs. \textit{mikh} and \textit{er} vs. \textit{em}.\footnote{Note however that `vus' speakers appear to distinguish between nominative and objective weak 3\textsc{pl} pronouns, as discussed further in Section \ref{sec:weakvsstrong}.}

Another innovation concerns the emergence of a distinct T/V (familiar vs. honorific) distinction in the plural. To the best of our knowledge, in the pre-War and Standard varieties of Yiddish, there was a T/V distinction in the singular, with the familiar variant \textit{du} in contrast to the honorific variant \textit{ir} (accusative/dative \textit{aykh}). The 2\textsc{sg} honorific variant \textit{ir} was also employed as the generic 2\textsc{pl} form, with no T/V distinction.\footnote{We find no discussion in the literature about a distinct honorific form in varieties that used \textit{ets} in the 2\textsc{pl}. We therefore conclude that, like varieties using \textit{ir} in such contexts, \textit{ets} varieties did not distinguish between the \textsc{2pl} and the \textsc{2hon}.} Our oral questionnaire indicated a reluctance to produce honorific forms in some of the discourse contexts we targeted, either because participants provided a familiar form or because they made use of an alternative honorific strategy such as avoiding direct address. Nevertheless, our data still indicate a shift away from the pre-War Standard system, both in terms of the morphemes currently used as honorific forms, which we give in Tables \ref{tab:vusmikh} to \ref{tab:vusaykh}, and in the realms of usage for the different honorific forms. We discuss the newly emerging T/V system and its use elsewhere (see \citealt{AuthorInPrep}).\footnote{A small number of speakers who use \textit{enk} in the 2\textsc{pl} do not appear to have a nominative/objective distinction in the honorific, and use \textit{aykh} as the honorific pronoun in both the nominative and objective. We are unable to determine a pattern governing this distribution with respect to the rest of the paradigm (e.g. the \textit{mikh/mir} distinction).} 

\begin{table}
\caption{Mixed `vus' paradigm }
\label{tab:vusmixed}
 \begin{tabularx}{\textwidth}{llYYYY}
  \lsptoprule
  	  & & Singular  &  & Plural &  \\
	  & & \textsc{nom} & \textsc{obj} & \textsc{nom} & \textsc{obj}\\
  \midrule
	1\textsuperscript{st} & & \textit{ikh} & \textit{mikh/mir} & \multicolumn{2}{c}{\textit{undz}}  \\
\midrule
	\multirow{2}{*}{2\textsuperscript{nd}} & \textsc{Fam} & \textit{du} & \textit{dikh/dir}   & \multicolumn{2}{c}{\textit{enk}} \\
	& \textsc{Hon} & \textit{ir} & \textit{aykh} & \textit{ir} & \textit{aykh}  \\
\midrule
	\multirow{3}{*}{3\textsuperscript{rd}} & \textsc{Masc} & \textit{er} &  \textit{em}   &     \multicolumn{2}{c}{\multirow{3}{*}{\textit{zey}}}  \\
	& \textsc{fem} & \textit{zi}  &  \textit{ir}    & \\
	%&  \textsc{Neut} &  \textit{se, es, 's}   &  \textit{es} &    \\
  \lspbottomrule
 \end{tabularx}
\end{table}

A third paradigm is provided in Table \ref{tab:vusmixed}. This paradigm resembles those shown in Tables \ref{tab:vusmikh} and \ref{tab:vusmir} except that there is intraspeaker variation in the 1\textsc{sg} and 2\textsc{sg} accusative/dative form. Instead of employing only a) \textit{mir/dir} or b) \textit{mikh/dikh} in both the accusative and dative positions, speakers who pattern according to Table \ref{tab:vusmixed} may employ both \textit{mikh/mir} and \textit{dikh/dir} in free variation in either position. Alternatively, some speakers consistently employ the 1\textsc{sg} form \textit{mikh} in the accusative and dative, but use the 2\textsc{sg} form \textit{dir} in the same positions. The opposite pattern is also attested, whereby a speaker employs the 1\textsc{sg} form \textit{mir} in the accusative and dative, but \textit{dikh} in the same positions. This phenomenon again points to a pronominal system in flux, whereby there is a high degree of flexibility in the selection of particular forms.

\begin{table}
\caption{‘Vus’ paradigm with 2\textsc{pl} \textit{aykh}}
\label{tab:vusaykh}
 \begin{tabularx}{\textwidth}{llYYYY}
  \lsptoprule
  	  & & Singular  &  & Plural &  \\
	  & & \textsc{nom} & \textsc{obj} & \textsc{nom} & \textsc{obj}\\
  \midrule
	1\textsuperscript{st} & & \textit{ikh} & \textit{mir} & \multicolumn{2}{c}{\textit{undz}}  \\
\midrule
	\multirow{2}{*}{2\textsuperscript{nd}} & \textsc{Fam} & \textit{du} & \textit{dir}   & \multicolumn{2}{c}{\multirow{2}{*}{\textit{aykh}}} \\
	& \textsc{Hon} & \multicolumn{2}{c}{\textit{aykh}}\\
\midrule
	\multirow{3}{*}{3\textsuperscript{rd}} & \textsc{Masc} & \textit{er} &  \textit{em}   &     \multicolumn{2}{c}{\multirow{3}{*}{\textit{zey}}}  \\
	& \textsc{fem} & \textit{zi}  &  \textit{ir}    & \\
	%&  \textsc{Neut} &  \textit{se, es, 's}   &  \textit{es} &    \\
  \lspbottomrule
 \end{tabularx}
\end{table}

The next paradigm, given in Table \ref{tab:vusaykh}, resembles that shown in Table \ref{tab:vusmir} except that speakers employ a different 2\textsc{pl} pronoun, \textit{aykh} instead of \textit{enk}. This pattern seems to be attested only in the paradigms of speakers from Israel and from London. As in the use of nominative \textit{enk}, the use of nominative \textit{aykh} is innovative because in pre-War and Standard Yiddish this form was solely objective. As with \textit{enk}, it is possible that the Contemporary Hasidic Yiddish usage is based on analogy with other plural forms, such as nominative \textit{undz}, which was attested in the pre-War period, and with \textit{zey}, which has long been employed in all case environments in Yiddish. Some speakers may employ \textit{enk} alongside \textit{aykh}, which suggests that the two originally dialectal forms have been more widely adopted among ‘vus’ speakers in general.\footnote{These speakers also used \textit{aykh} in the honorific. This is a relatively rare pattern among our participants so we hesitate to draw strong conclusions. We believe that these speakers do not use \textit{ir} in the nominative at all, but they may distinguish the familiar and honorific paradigms through the use of nominative \textit{ir} in the latter.}

 

\begin{table}
\caption{'Vos' paradigm with 1\textsc{pl} \textit{mir} and 2\textsc{pl} \textit{ir} }
\label{tab:vos}
 \begin{tabularx}{\textwidth}{llYYYY}
  \lsptoprule
  	  & & Singular  &  & Plural &  \\
	  & & \textsc{nom} & \textsc{obj} & \textsc{nom} & \textsc{obj}\\
  \midrule
	1\textsuperscript{st} & & \textit{ikh} & \textit{mir} & \textit{mir} & \textit{undz}  \\
	& & & & & /unz/ \\
\midrule
	\multirow{3}{*}{2\textsuperscript{nd}} & \textsc{Fam} & \textit{du} & \textit{dir}   & \multirow{3}{*}{\textit{ir}} & \multirow{3}{*}{\makecell{\textit{aykh} \\ /aɪχ/}}\\
	& \multirow{2}{*}{\textsc{Hon}} & \textit{ir} & \textit{aykh} & & \\
	& & & /aɪχ/ & &\\
\midrule
	\multirow{3}{*}{3\textsuperscript{rd}} & \textsc{Masc} & \textit{er} &  \textit{em, im}   &     \multicolumn{2}{c}{\multirow{3}{*}{\textit{zey}}}  \\
	&&& /ɛm/, /im/ & \\
	& \textsc{fem} & \textit{zi}  &  \textit{ir}    & \\
	%&  \textsc{Neut} &  \textit{se, es, 's}   &  \textit{es} &    \\
  \lspbottomrule
 \end{tabularx}
\end{table}

\newpage
The final paradigm is shown in Table \ref{tab:vos}. This paradigm is attested primarily in the speech of older ‘vos’ speakers (i.e. those over 40). This older ‘vos’ paradigm resembles that of the pre-War Northeastern dialect of Yiddish. It retains the distinction between nominative \textit{mir} and objective \textit{undz} in the 1\textsc{pl}, and the distinction between nominative \textit{ir} and objective \textit{aykh} in the 2\textsc{pl}. In addition, \citep[184]{Jacobs05} suggests that pre-War Northeastern Yiddish already lacked the historical three-way distinction in the 1\textsc{sg}, 2\textsc{sg}, and 3\textsc{fs}, employing the historically dative forms in both accusative and dative settings. If true, the 1\textsc{sg}, 2\textsc{sg}, and 3\textsc{fs} two-way distinction shown in Table \ref{tab:vos} corresponds to this traditional Northeastern pattern (i.e. using \textit{mir, dir, ir} in objective cases) rather than being innovative, in contrast to the ‘vus’ paradigms where the same synchronic pattern is innovative. 

Interestingly, one ‘vos’ speaker [NO1], who was born in the immediate post-War period, partially follows the pre-War Mideastern Yiddish pattern, employing \textit{mikh} for the 1\textsc{sg} accusative but \textit{mir} for the 1\textsc{sg} dative (though he exhibits the traditional Northeastern syncretism in the 2\textsc{sg}, employing dir in both accusative and dative contexts). This \textit{mikh/mir} distinction may actually point to influence from another dialect, such as Contemporary Hasidic Yiddish or a traditional Mid- or Southeastern dialect, but more research would be needed in order to confirm this possibility. The younger ‘vos’ speaker mentioned above (IO1) has a different paradigm, employing \textit{undz} in the nominative instead of the more conservative \textit{mir} and using the innovative nominative \textit{aykh} alongside the older \textit{ir}. Conversely, another younger speaker whose vowel profile corresponds largely to the ‘vos’ model (IM1) provided the 1\textsc{pl} nominative form \textit{mir} alongside the objective form \textit{undz}, but employed the traditionally objective 2\textsc{pl} form \textit{aykh} in nominative contexts.\footnote{There are also a few ‘vus’ speakers, including younger ones, who provide the form \textit{mir} for the 1\textsc{pl}; however, these speakers typically also provide \textit{undz}, rather than employing \textit{mir} exclusively. In such cases, metalinguistic discussion with speakers indicates that speakers consider the \textit{mir} variant to be more literary and higher register (as it is frequently seen in writing but is less common in everyday speech), and as such are more likely to provide it in formal contexts while they may tend to use it less in spontaneous conversation.}


%\todo{Can we talk about this?}In all paradigms, the 3\textsc{sg} inanimate pronoun is \textit{es} (or variants ‘s and se), and this form is used with all inanimate referents. This is another innovative feature, given that in pre-War and Standard Yiddish the dative of the 3\textsc{ns} pronoun \textit{es} was \textit{im} (\citealp[240]{Mark78}, \citealp[185]{Jacobs05}). This is in line with the trend seen elsewhere in the Contemporary Hasidic Yiddish personal pronoun paradigm towards a two-way nominative/objective distinction, in contrast to the previous three-way nominative/accusative/dative distinction that previously obtained in much of the singular paradigm. 

 

%Both of these patterns seem to be the result of contact with the dominant co-territorial language. In pre-War and Standard Yiddish, inanimate referents were referred to by either the 3\textsc{ms} pronoun \textit{er} or the 3\textsc{fs} pronoun \textit{zi} in accordance with the grammatical gender of the noun in question \citep[239]{Mark78}. The shift to using the neuter 3\textsc{sg} pronoun \textit{es} and its variants with all inanimate nouns may be due to contact with English, which exhibits this pattern. Conversely, in Hebrew (including the historical forms of the language as well as the present-day variety), inanimate referents are referred to be either the 3\textsc{ms} or 3\textsc{fs} pronoun. 

\subsubsection{Verbal agreement with novel forms}\label{sec:verbalfindings}

Verbal agreement with the 1\textsc{sg}, 2\textsc{sg}, 3\textsc{sg}, and 3\textsc{pl} personal pronouns is the same in Contemporary Hasidic Yiddish as in other varieties of the language. However, there are some innovative or otherwise noteworthy features relating to verbal agreement in the 1\textsc{pl} and 2\textsc{pl}. First, speakers who employ the nominative 1\textsc{pl} pronoun \textit{undz}, especially those who also use the 1\textsc{sg} objective form \textit{mikh}, may use a verbal form ending in \textit{‑mir}, e.g. \textit{undz geyenmir} [indz gaɪnmir]. This form is historically attested in certain varieties of Mideastern and Southeastern Yiddish \citep[70, 189]{Jacobs05}, such as those spoken in the Polish and Hungarian regions to which many ‘vus’ speakers trace their ancestry. This form can be used interchangeably with the variant \textit{‑n/‑en}, e.g. \textit{undz geyen}, which is also the verbal suffix employed with the 1\textsc{pl} pronoun \textit{mir}. In Contemporary Hasidic Yiddish the verbal form ending in \textit{‑mir} may be a marker of more informal speech, though this needs further research. 

Second, the emergence of the innovative 2\textsc{pl} \textit{enk} and \textit{aykh} in nominative contexts has resulted in the development of a new type of verbal agreement modelled on the 1\textsc{pl} and 3\textsc{pl} agreement, whereby the verbal suffix is \textit{‑(e)n}. Thus, we see forms such as \textit{enk geyen} ‘you.\textsc{pl} go’, \textit{enk kumen} ‘you.\textsc{pl} come’, \textit{aykh hobn} ‘you.\textsc{pl} have’. These findings are in keeping with those of \citet[86]{Assouline07}, who discusses the nominative form \textit{aykh} with verbal agreement in \textit{-(e)n}. These forms are to the best of our knowledge unattested in pre-War Polish and Hungarian varieties of Yiddish, in which the 2\textsc{pl} nominative pronoun was \textit{ets} and this was used in conjunction with verbs ending in the suffix \textit{‑t}, e.g. \textit{ets hot} ‘you.\textsc{pl} have’ \citep[190]{Jacobs05}.\footnote{An anonymous reviewer points out that the form \textit{ets hots} is attested in pre-War Yiddish. We have found only one such example \citep[277]{Okrutny53} in the Yiddish Book Center archive, although it is possible that the form was more common in speech than in writing. A detailed examination of this issue is beyond the scope of this paper.} In our study, participants who employ \textit{ets} in the nominative do so in conjunction with the verbal suffix \textit{‑ts}, e.g. \textit{ets geyts} ‘you.\textsc{pl} go’, which is actually based on the plural imperative suffix \textit{‑ts} that was historically employed with \textit{ets} in Mideastern Yiddish, e.g. \textit{gayts} ‘go-2\textsc{pl}!’ \citep[70]{Jacobs05}. In fact the imperative suffix \textit{-ts} continues to be used by speakers even if they use \textit{enk} as the second person plural personal pronoun and \textit{V-(e)n} as the corresponding verb form. In contrast, speakers who use 2\textsc{pl} nominative \textit{ir}, whether as an honorific or as a general pronoun use verbal agreement in \textit{-t}, which is unchanged from pre-War varieties. 

\subsubsection{Reflexive forms}

Most of our Contemporary Hasidic Yiddish participants have an invariable reflexive pronoun, \textit{zikh}, which is used for all persons. A minority of participants exhibit some variation in the 1\textsc{sg} and 2\textsc{sg}, using the invariable \textit{zikh} seemingly interchangeably with the objective 1\textsc{sg} and 2\textsc{sg} personal pronouns \textit{mikh} and \textit{dikh}. Even fewer participants invariably employ \textit{mikh} and \textit{dikh} in 1\textsc{sg} and 2\textsc{sg} reflexive contexts without ever providing \textit{zikh} in the same contexts. The latter patterns, whereby objective personal pronouns are used as reflexive pronouns, has precedent in a more extensive older Mideastern Yiddish pattern whereby the 1\textsc{sg}, 2\textsc{sg}, 1\textsc{pl}, and 2\textsc{pl} personal accusative or dative pronouns are used as reflexive pronouns, in contrast to other Eastern Yiddish dialects which use \textit{zikh} in all persons and cases, see \citet[184--185]{Jacobs05}.\footnote{Though see \citet[125--126]{Katz87} for the observation that the 1\textsc{pl} and 2\textsc{pl} forms were used less consistently than their 1\textsc{sg} and 2\textsc{sg} counterparts.} The Contemporary Hasidic Yiddish use is more restricted than its Mideastern Yiddish predecessor as it is limited to the singular, and is rarer than the generalised \textit{zikh} pattern even among ‘vus’ speakers. This tendency towards paradigm levelling of the reflexive pronoun is consistent with the general trend towards syncretism seen elsewhere in the personal pronoun paradigm.

\subsubsection{Overall trends in the personal pronoun paradigm}

The above patterns indicate a trend of increased syncretism in the Contemporary Hasidic Yiddish personal pronoun paradigm vis-à-vis the pre-War and Standard varieties of the language. The most common patterns can be summarised in the abstracted paradigm shown in Table \ref{tab:urparadigm}.


\begin{table}
\caption{Abstract Contemporary Hasidic Yiddish personal pronoun paradigm}
\label{tab:urparadigm}
 \begin{tabularx}{\textwidth}{llYYYY}
  \lsptoprule
  	  & & Singular  &  & Plural &  \\
	  & & \textsc{nom} & \textsc{obj} & \textsc{nom} & \textsc{obj}\\
  \midrule
	1\textsuperscript{st} & & \textsc{nom} & \textsc{obj} & \multicolumn{2}{c}{\textsc{Single form}}  \\
\midrule
	\multirow{2}{*}{2\textsuperscript{nd}} & \textsc{Fam} & \textsc{nom} & \textsc{obj} & \multicolumn{2}{c}{\textsc{Single form}}  \\
	& \textsc{Hon} & \textsc{nom} & \textsc{obj}  & \textsc{nom} & \textsc{obj}  \\
\midrule
	\multirow{2}{*}{3\textsuperscript{rd}} & \textsc{Masc} & \textsc{nom} & \textsc{obj}    &     \multicolumn{2}{c}{\multirow{2}{*}{\textsc{Single form}}}  \\
	& \textsc{fem} & \textsc{nom} & \textsc{obj}    & \\
  \lspbottomrule
 \end{tabularx}
\end{table}

This typical Contemporary Hasidic Yiddish paradigm reflects syncretism in both the singular and the plural. In the singular, the paradigm is the result of a merger of the previously distinct accusative and dative cases in the 1\textsc{sg} and 2\textsc{sg}, so that instead of a three-way distinction there is now a two-way distinction between nominative and objective (though the objective form varies from speaker to speaker, with some employing \textit{mikh/dikh}, others employing \textit{mir/dir}, and still others employing a mix). This may be based on analogy with other historical pronoun forms, e.g. the 3\textsc{ms} (nominative \textit{er} vs. objective \textit{em}), and the 2\textsc{pl} (nominative \textit{ir} or \textit{ets} vs. objective \textit{aykh} or \textit{enk}). This pattern seems to have spread to the 1\textsc{sg} and 2\textsc{sg}, possibly due in part to historically varying distribution of the 1\textsc{sg} and 2\textsc{sg} accusative and dative forms with different verbs. With respect to the 3\textsc{fs}, the development seems to have followed a slightly different route: while, like the 3\textsc{ms}, it had only a two-way distinction, this was between the nominative/accusative \textit{zi} and the dative \textit{ir}. This pattern has not been retained by any of our participants, indicating that the emergence of the two-way distinction between the 1\textsc{sg} and 2\textsc{sg} forms may have contributed to a realignment of the 3\textsc{fs} forms to a nominative/objective distinction, resulting in a regularised singular paradigm with this pattern. In the plural, where there was historically a two-way distinction between the nominative and objective cases in the first and second persons, the objective case was extended to use in nominative contexts as well. As in the singular, this shift may have happened by analogy with the third person plural, which historically had only one form in all case environments (though, as in the 1\textsc{sg} and 2\textsc{sg}, the 2\textsc{pl} form employed may differ from speaker to speaker between \textit{enk} and, more rarely, \textit{aykh}). Thus, in both the singular and the plural, the Contemporary Hasidic Yiddish paradigm reflects one degree of syncretism greater than that found in the pre-War and Standard varieties, having shifted from a three-way distinction in the first and second person singular and a two-way distinction in the 1\textsc{pl} and 2\textsc{pl} to a two-way distinction across the board in the singular and a single form across the board in the plural.\footnote{Although such forms can be difficult to elicit, at least some speakers retain a nominative/objective distinction in the 2\textsc{pl.hon}.} 
	
These trends suggest to us that in future, the remaining variations in usage (e.g. the continued optional employment of the pre-War and Standard nominative \textit{mir} and \textit{ets} in the 1\textsc{pl} and 2\textsc{pl} respectively) will decrease (a tendency supported by the fact that our participants under the age of 40 are much less likely to supply them), resulting in a more stable levelled paradigm with a clear two-way nominative/objective distinction in the singular and a single form in the plural. The only likely exception to this is the possible retention of the 1\textsc{pl} form \textit{mir} in nominative contexts, which may continue to persist for longer due to the fact that it is frequently found in writing and as such may be preserved as a higher-register form in speech as well. This is reinforced by the fact that \textit{mir} is the only exclusively nominative option for the 1\textsc{pl}, in contrast to the 2\textsc{pl}, for which two different traditional nominative variants exist (\textit{ir} and \textit{ets}), a situation which may have contributed to the increased instability of the nominative 2\textsc{pl} forms.



\subsection{Weak vs. strong pronouns}\label{sec:weakvsstrong}


Yiddish personal pronouns typically occupy different syntactic positions than corresponding full noun phrases. The phenomenon, which resembles Scandinavian object shift, is known to have existed already in pre-War and Standard Yiddish. As \REF{ex:1} shows, Yiddish personal pronouns typically appear right-adjacent to the auxiliary, which is in the second position.\footnote{Weak pronouns can occur in the neutral clause final position if they are inside a prepositional phrase, as shown in \REF{ex:foo18}. In this case main stress falls on the participle, as indicated. As \citet{Ruys08} observes, the fact that PPs are differently affected by pronoun shift suggests a motivation for pronoun shifting that is related to case assignment, as suggested by \citet{Neeleman98}. We will not be concerned with motivating pronoun shift in Yiddish, but we note that in this respect the Yiddish data seems to pattern with Dutch.
\ea\label{ex:foo18} \gll du host es ge\textsc{shikt} [tsә i].\\
    You have-2\textsc{sg} it sent to her\\
\glt ‘You sent it to her.’
\z

}\textsuperscript{,}\footnote{Grammaticality judgments in this section were provided by a young `vus' speaker from Stamford Hill. He has a typical mixed paradigm using both \textit{mikh/dikh} and \textit{mir/dir}. A nominative-objective contrast is also present in the 3\textsc{fs} (\textit{zi} vs \textit{ir}, pronounced [i:] as this speaker has deletion of /r/ in coda position, which is typical of Stamford Hill Yiddish). He uses \textit{enk} in 2\textsc{pl}.} 

\ea\label{ex:1} \gll 'kh vil <enk> \textsc{rufn} *<enk> ven supper iz \textsc{greyt}.\footnotemark\\
I will <you.\textsc{pl}> call <you.\textsc{pl}> when dinner is ready\\
\glt `I will call you when dinner is ready.'
\z

\footnotetext{In transliterated examples and English glosses and translations, we indicate primary stress/focus with small capitals. In IPA transcriptions we indicate it with the primary stress symbol, ˈ.} 

More than one pronominal can occur in this position; in this case, the pronouns form a phonological unit, or cluster, with the preceding auxiliary which can also include the subject pronoun in the initial position. In fast speech, such clusters can be phonetically substantially non-transparent:

\ea \ea \glll {er hot es ir} nisht gegebn\\
[ɛtәsi] {} {}\\
{he has her} not given\\
\glt `He did not give it to her.'
\ex \glll {er hot dikh} nisht gezen\\
[ɛdәχ] {} {}\\
{he has you} not seen\\
\glt `He did not see you.'
\z
\z

Although this phenomenon is a distinctive characteristic of spoken Yiddish, it has not yet, to the best of our knowledge, been subject to linguistic analysis. We would like to propose that in Contemporary Hasidic Yiddish, and most likely also in pre-War spoken dialects, personal pronouns can be divided into strong and weak categories in the sense of \citet{Cardinaletti99}. Strong pronouns can be used deictically or accompanied by a pointing gesture. As they identify their own referent, they can also be easily contrasted. Weak pronouns are referentially deficient. They are always dependent on the immediate discourse context for reference resolution. They rely on the accessibility of a given referent for reference assignment. A summary table of all the forms used by our participants is given in Table \ref{tab:weak pronouns}. Recall that some of the speakers use \textit{mikh/dikh} forms while others use \textit{mir/dir} forms, and some use both interchangeably. Correspondingly, the speakers would use the weak variant of the forms they use as strong forms, which in Table \ref{tab:weak pronouns} appear in the same line. Strong pronouns are stressed, while weak are unstressed.\footnote{We abstract away from specific length and quality differences in the high front vowels, using /i/ for long and/or tense vowels and /ı/ for short and/or lax vowels. The specific realisation of these vowels depends to a certain extent on the vowel inventory of the co-territorial language(s) of which the speaker has command. Where such languages do not have a tense/lax distinction, the speaker's Yiddish vowel inventory will also lack this distinction.}





\begin{table}
\caption{Strong and weak pronominal forms used by `vus' speakers}
\label{tab:weak pronouns}
 \begin{tabularx}{\textwidth}{lllYYYY}
  \lsptoprule
  	&  & & \multicolumn{2}{c}{\textsc{nom}}  & \multicolumn{2}{c}{\textsc{obj}}  \\
	&  & & Strong & Weak & Strong & Weak\\
  \midrule
	\multirow{5}{*}{\textsc{sg}} & 1\textsuperscript{st} & & [ˈiχ] & [ıx, ɛx, χ] & [ˈmiχ, ˈmir]  & [mıx, mәx, mır, mı] \\
\cline{2-7}
	 & \multirow{2}{*}{2\textsuperscript{nd}} & \textsc{Fam} & [ˈdi] & [dɪ, -i, -ɛ] & [ˈdiχ, ˈdir] & [dɪx, dәx, dɪr, dı]  \\
	 &  &  \textsc{Hon}   & ˈ[ir, i] & [ir, i, ɪ]  & [ˈaıx, ˈɑːχ] & [aıx, ɑːx, ɑχ]  \\
\cline{2-7}
	& \multirow{2}{*}{3\textsuperscript{rd}} & \textsc{Masc} &  [ˈɛr] & [ɛr, ɛ] & [ˈɛım, ˈɛm] & [ɛım, ɛm, әm, m]  \\
	& & \textsc{fem} &  [ˈzi] & [zi, z] & [ˈzi, ˈir, ˈi]  & [zi, ir, i, ɪ] \\
\midrule
	\multirow{4}{*}{\textsc{pl}} & 1\textsuperscript{st} & & [ˈints] & [ɪnts] & [ˈints] & [ɪnts]  \\
\cline{2-7}
	 & \multirow{2}{*}{2\textsuperscript{nd}} & \textsc{Fam} & [ˈɛŋk] & [ɛŋk] & [ˈɛŋk] & [ɛŋk]  \\
	 &  &  \textsc{Hon}   & [ˈir, i] & [ir, i, ɪ] & [ˈaıx, ˈɑːχ]  & [aıx, ɑːx, ɑχ]  \\
\cline{2-7}
	& 3\textsuperscript{rd} &  &  [ˈzaı]  & [zaı, zɑ] & [ˈzaı] & [zaı, zɛ]  \\
  \lspbottomrule
 \end{tabularx}
\end{table}


\begin{table}
\caption{Strong and weak pronominal forms used by `vos' speakers}
\label{tab:weak pronounsvos}
 \begin{tabularx}{\textwidth}{lllYYYY}
  \lsptoprule
  	&  & & \multicolumn{2}{c}{\textsc{nom}}  & \multicolumn{2}{c}{\textsc{obj}}  \\
	&  & & Strong & Weak & Strong & Weak\\
  \midrule
	\multirow{5}{*}{\textsc{sg}} & 1\textsuperscript{st} & & [ˈiχ] & [ıx, χ] & [ˈmir]  & [mır, mı] \\
\cline{2-7}
	 & \multirow{2}{*}{2\textsuperscript{nd}} & \textsc{Fam} & [ˈdu] & [du] & [ˈdir] & [dir, dı]  \\
	 &  &  \textsc{Hon}   & ˈ[ir, i] & [ir, iː, i]  & [ˈaıx] & [aıx]  \\
\cline{2-7}
	& \multirow{2}{*}{3\textsuperscript{rd}} & \textsc{Masc} &  [ˈɛr] & [ɛr, ɛ] & [ˈɛım, ˈɛm, ˈim] & [ɛım, ɛm, әm, m]  \\
	& & \textsc{fem} &  [ˈzi] & [zi, z] & [ˈir]  & [ir] \\
\midrule
	\multirow{3}{*}{\textsc{pl}} & 1\textsuperscript{st} & & [ˈmir] & [mir, mır] & [ˈunts, ˈundz] & [unts, undz]  \\
\cline{2-7}
	 &  2\textsuperscript{nd} &    & [ˈir, i] & [ir, iː, i] & [ˈaıx]  & [aıx]  \\
\cline{2-7}
	& 3\textsuperscript{rd} &  &  [ˈzɛı]  & [zɛı, zɛ] & [ˈzɛı] & [zɛı, zɛ]  \\
  \lspbottomrule
 \end{tabularx}
\end{table}


Two points of interest emerge from these paradigms. The first is that there is a distinction between the weak paradigms of `vos' and `vus' speakers. For the former group of speakers, the weak 3\textsc{pl} is /zɛ/ in both nominative and objective contexts, while the latter group have a case distinction that is not present in the strong form, where the weak 3\textsc{pl} pronoun is /zɑ/ in the nominative and /zɛ/ in the objective.\footnote{The observations as they relate to `vos' speakers are based on a much smaller subset of speakers and so must be considered somewhat provisional.} 


Some examples from our elicited data are given in \REF{ex:MU11}, which were provided by a typical young `vus' speaker, MU1. In all of these examples the context provides an appropriate environment for a weak pronoun, as the referent is established in the previous sentence. MU1 produces clear examples of /i/ as the weak form of the 3\textsc{fs} objective pronoun. In \REF{ex:MU11c} the weak pronominal form [zɛ] is used, while in \REF{ex:MU11a} the participant uses the weak second person objective pronoun [dәχ] which clearly contrasts with the form [diχ].



\ea\label{ex:MU11} \ea \glll rokhl iz a gute meydl. {ikh hob ir} lib.\\
{ } { } { } { } { } [iˌχɔbi ˈlib] \\
Rokhl is a good girl {I have her} love\\
\glt `Rokhl is a nice girl. I like her.'\label{ex:MU11b}
\ex \glll rokhl un sheyni zenen gute meydlekh. {ikh ze zey} {a \textsc{sakh}} in the \textsc{street}.\\
{ } { } { } { } { } { } [χˌzɛızɛ aˈsɑχ] \\
Rokhl and Sheyni are good girls {I see them} {a lot} in the street\\
\glt `Rokhl and Sheyni are nice girls. I see them often in the street.'\label{ex:MU11c}
\ex \glll de lerer {hot dikh} \textsc{lib}. {er hot dikh} zayer shtark lib.\\
{ } { } [hɔtdәχ ˈlib ɛhɔtdχ ˈzajɛr] {} {}\\
the teacher {has you} love {he has you} very strong love\\
\glt `The teacher likes you. He likes you very much.'\label{ex:MU11a}
\z
\z



MU1 uses strong pronouns in other environments. This typically occurs when the pronoun appears at the end of the sentence, as in \REF{ex:MU12a} and \REF{ex:MU12b} and in sentences that she utters with slow tempo and with comma intonation like \REF{ex:MU12c}.


\ea \ea \glll {ikh geb es} {far \textsc{dir}}\\
	    [ɛχˌgɛbɛs fɑˈdir] \\
	    {I give it} {to you}\\\label{ex:MU12a}
	\ex \glll enk {gebn es} {far \textsc{zey}}\\
	    [ˈɛŋk ˈgɛbnɛs fɑˈzaı]\\
	    You.\textsc{pl} {give it} {to them}\\\label{ex:MU12b}
\ex \glll zi get em {de \textsc{paper}}\\
	    [ˈzi ˈgɛt ˈɛım dәˈpʰɛıpәr]\\
	    She, gives, him, {the paper}\\\label{ex:MU12c}
\z
\z

MU1’s data clearly illustrates that the same speaker may use different forms in different discourse situations and in different syntactic environments. This supports our claim that Contemporary Hasidic Yiddish distinguishes between separate weak and strong pronominal forms. 

 \citet{Cardinaletti99} identify a host of syntactic tests to distinguish strong from weak pronouns. One such characteristic difference between the two is that weak pronouns typically have a different syntactic distribution to corresponding full noun phrases, while strong pronouns behave syntactically like full nominals. As the data in \REF{ex:nishtgezen}--\REF{ex:norem} show, the Yiddish personal pronouns we identified as weak pattern with Cardinaletti and Starke's weak pronouns in all relevant respects. In addition to not being allowed in the clause final position, they also cannot occupy the clause initial position, which is restricted to contrastively stressed strong pronouns.



\ea \gll [ˈɛım]/*[(ә)m] hob ikh nisht gezen, ober [ˈiː]/*[i], yo.\\
him have I not seen but her yes\\
\glt `\textsc{Him} I didn't see, but \textsc{her}, yes.’\label{ex:nishtgezen}
\z

In \REF{ex:IPP2a}, we illustrate that only strong pronouns can be coordinated. \REF{ex:IPP2b} shows that only strong pronouns can be fragment answers. 



\ea\label{ex:IPP2} \ea \gll <\textsc{mikh} un \textsc{dikh}> hobn zey shoyn gezen <*\textsc{mikh} un \textsc{dikh}>\\
	     <me and you> have they already seen <me and you>\\
	\glt `They have already seen me and you.’\label{ex:IPP2a}
\ex \gll Q: vemen kenstu hern?\\
     {} who can.2\textsc{sg} hear\\
     \glt `Who do you hear?'\label{ex:IPP2b}
\exi{} \gll A: \textsc{dikh} / \textsc{ir} / \textsc{em} / *dәkh / *dkh / *i / *әm / *m\\
	{} you / her / him / you / you / you / him / him\\
     \glt `\textsc{You/her/him}/*you/*her/*him.'
\z
\z

\REF{ex:norem} illustrates that only strong pronouns can be modified.

%is this really EM aleyn, or em ALEYN?
\ea\label{ex:norem} \ea[] {\gll ikh hob <*nor ’m> gezen <nor \textsc{em}>.\\
	   I have <*only him> seen <only him>\\
	\glt `I only saw him.’}
	\ex[*]{\gll Ikh hob ‘m aleyn gezen.\\
	 I have him alone seen\\}
	\ex[]{\gll \textsc{em} aleyn hob ikh gezen.\\
	him alone have I seen\\
	\glt `I saw only him.’}
\z
\z


The same strong-weak contrast that is prevalent among `vus' speakers can be also be observed in the speech of older `vos' speakers.\footnote{Note that three among our four `vos' speakers displayed a strong-weak distinction. Whether the fourth person does not have a strong-weak distinction or whether perhaps their generally slow speech tempo and formal attitude towards the test situation prevented them from using a faster speech tempo or a more colloquial register is not clear.} In \REF{ex:vos1}, provided by NO1, illustrates this point the 3\textsc{ms} objective pronoun, with \REF{ex:vos1a} exemplifying weak forms, and \REF{ex:vos1b} providing the strong form. Similarly, \REF{ex:vos2a} illustrates the weak 3\textsc{pl} objective pronoun and \REF{ex:vos2b} its strong variant.

\ea \ea \glll dovid iz a fayner mentsh. {ikh glaykh im} {a \textsc{sakh}}. {ikh hob im} {\textsc{lib} a sakh}.\\
		{} {} {} {} {} [χˌglaıχәm aˈsɑχ χˌhɔbәm ˈlibɑsɑχ]	\\
		Dovid is a nice guy {I like him} {a lot} {I have him} {love a lot}\\
		\glt `Dovid is a nice guy. I like him. I like him a lot.'\label{ex:vos1}
	\ex	\glll zi git \textsc{em}... de \textsc{papir}.\\
		[ˌzi ˌgit ˈɛm... dә pɑˈpir]\\
		she gives him the paper \\
		\glt `She gives him... the paper.'\label{ex:vos1a}
	\ex	\glll du, {gib es tsu} \textsc{em}\\
		[ˌdu ˌgibɛstsә ˈɛm]\\
		you {give.\textsc{imp2sg} it to} him\\
		\glt `(Hey) you, give it to him.'\label{ex:vos1b}
\z
\z

\ea \ea \glll rokhl un sheyni zenen fayne meydlekh. {ikh ze zey} zeyer oft {oyf de \textsc{gas}}\\
	{} {} {} {} {} {} [iχˌzɛzɛ ˌzɛjɛr ˌɔft ɔfdәˈgɑs]\\
	Rokhl and Sheyni are nice girls {I see them} very often {on the street} \\
	\glt 'Rokhl and Sheyni are nice girls. I see them very often on the street.'\label{ex:vos2a}
	\ex \glll {gib es} {tsu \textsc{zey}}\\
	    [ˌgibɛs tsәˈzɛı]\\
	    {give.\textsc{imp2sg} it} {to them} \\
	    \glt `Give it to them.'\label{ex:vos2b}
\z
\z


In sum, Contemporary Hasidic Yiddish has a set of corresponding strong and weak pronominal forms with distinct syntactic distributions. Strong pronominal forms can be coordinated or modified and can be used as fragment answers. They can be fronted, and in fact are ideally fronted to a clause-initial position. Weak pronominal forms, in contrast, cannot be coordinated, modified, or used as fragment answers. In subject-initial clauses, weak pronouns follow the finite auxiliary (or verb) in the V2 position. They form a phonological word with the auxiliary, which can also include the subject, if that is also a weak pronoun. One question for future research is whether weak pronominals are actually clitics fitting into a predetermined order or morphological template.\footnote{Another issue that could be considered further concerns empty pronominals, or \textit{pro}-drop. As in many Germanic dialects, the 2\textsc{sg} nominative pronoun is often dropped in Contemporary Hasidic Yiddish, especially in questions.
\ea \gll {(du) host} epes?\\
have.2\textsc{sg}-2\textsc{sg}.\textit{pro} something\\
\glt `Do you have anything?’
\z

} It is not clear whether the distinction between strong and weak pronouns constitutes an innovation in the pronominal system due to the lack of literature on the issue, but the distinction clearly holds in Contemporary Hasidic Yiddish.

\section{Possessives}\label{sec:possessives}


Like the personal pronouns, possessive pronouns in Contemporary Hasidic Yiddish also exhibit various innovations. Parts of the pre-War possessive system are maintained in Contemporary Hasidic Yiddish, such as the distinction between singular and plural forms for dependent possessives. At the same time, certain pre-War and Standard case and gender morphemes have been reanalyzed in this dialect, as is this case in independent possessive pronouns. Compared to the relative variability found in the personal pronoun system, the possessive pronouns are noticeably more stable and exhibit a number of innovations as well as retaining a number of characteristics of the pre-War system.


\subsection{Dependent possessive pronouns}\label{sec:depposs}

Dependent possessive pronouns in Hasidic Yiddish are very similar to those of pre-War and Standard varieties of Yiddish as shown in Table \ref{tab:possessives}. 

\begin{table}
\caption{Possessive pronoun stems in Contemporary Hasidic Yiddish}
\label{tab:possessives}
 \begin{tabularx}{\textwidth}{llYY}
  \lsptoprule
  	  & & Singular  & Plural \\
  \midrule
	\multirow{2}{*}{1\textsuperscript{st}} & & \textit{mayn-} &  \textit{undzer-}  \\
	 & & /mɑːn/, /maɪn/&  /inzɛr/, /unzɛr/  \\
\midrule
	\multirow{4}{*}{2\textsuperscript{nd}} & \textsc{Fam} & \textit{dayn-} &  \textit{enker- (ayer-)}  \\
	& & /dɑːn/, /daɪn/&  /ɛŋkɛr/ (/ajɛr/)  \\
	& \textsc{Hon} & \textit{ayer- } & \textit{ayer- }  \\
	& & /ajɛr/  &  /ajɛr/   \\
\midrule
	\multirow{4}{*}{3\textsuperscript{rd}} & \textsc{Masc} & \textit{zayn-}  &  \multirow{4}{*}{\makecell{\textit{zeyer-}\\/zajɛr/,  /zɛjɛr/ }}  \\
	& & /zɑːn/, /zaɪn/&  \\
	& \textsc{fem} & \textit{ir-}  &    \\
	& & /ir/ &  \\
	%&  \textsc{Inan.} &  \textit{zayn-}  & \\
	%& & /zɑːn/, /zaɪn/ &  \\
  \lspbottomrule
 \end{tabularx}
\end{table}


This paradigm exhibits several noteworthy characteristics. The possessive stem \textit{enker-}, which is attested in the 2\textsc{pl} in some pre-War varieties, is nearly universal among our `vus' speaking participants. This is in line with our findings in the personal pronoun system, which demonstrated that a novel distinction has emerged between 2\textsc{pl} and 2\textsc{hon}: \textit{enker-} is used for the former and \textit{ayer-} for the latter. `Vos' speakers typically retain the more conservative pattern of using a single form for both 2\textsc{pl} and 2\textsc{hon}, \textit{ayer-}.



%Several things are to be clarified in the table. First, for most Hasidic Yiddish speakers there is no gender distinction for inanimate objects \citep[496--504]{Krogh12} and the possessive pronoun \textit{zayn} is used in singular for all of them. However, some Hasidic Yiddish speakers in Israel might distinguish F and M of inanimate objects, which could be an influence of their knowledge of Modern Israeli Hebrew, as discussed above in the section about personal pronouns.

%Second, the possessive \textit{enker} that was in use as a dialectal variant for 2\textsc{pl} in pre-War Central Yiddish, becomes universal for majority of speakers. A new distinction seems to have been emerged for \textit{enk}-speakers: the pronoun \textit{enker} is used for 2\textsc{pl}, and the pronoun \textit{ayer} takes functions of honorific form of address. However, this distinction is not relevant for most Chabad speakers, who keep using ayer in both 2\textsc{pl} and honorifics.

In pre-War and Standard Yiddish, dependent adnominals agree with the noun in number only. This pattern largely survives in Contemporary Hasidic Yiddish. Forms with a \varnothing-ending are used with singular possessa and the \textit{-e} ending is used with plural possessa, as demonstrated in \REF{ex:10}.

\ea\label{ex:10} \ea \gll undzer tish vs. undzere tishn \\
our table vs. our-\textsc{e} tables \\
\ex \gll dayn feder vs. dayne feders\\
your pen vs. your-\textsc{e} pens\\
\ex \gll ir bukh vs. ire bikher \\
her book vs. her-\textsc{e} books\\
\z
\z

This pattern is especially strong with singular possessors; the plural possessors, \textit{undzer, enker, ayer, zeyer}, show somewhat more variation with more of them appearing with a \varnothing-suffix with plural possessa than expected. There may be a phonological explanation for this phenomenon: some participants, especially those from Stamford Hill and certain communities in the New York area, often delete /r/ in syllable codas, and reduce /r/ between two unstressed syllables especially in fast speech. These factors mean that the difference between \textit{undzer} and \textit{undzere} is often difficult to perceive.\footnote{The results of our questionnaire revealed an unexpected difference in the behaviour of singular vs. plural possessors, and we therefore made use of the Haredi Yiddish-medium online forum \citet{KaveShtiebel}. Using Google, we searched the forum for relevant written examples. A small number of examples of the form אונזע \textit{undze} are attested on this forum with both singular and plural posessa, including \textit{undze kehile} `our community', \textit{undze kop} `our head', \textit{undze menahel} `our director' along with \textit{undze nemen} `our names', \textit{undze gvirim} `our rich and influential people'. However, three linguistic consultants all rejected these forms.} Nonetheless, there appears to be a strong overall tendency to use dependent possessive pronouns with a \varnothing-ending for singular possessa and dependent possessive pronouns with a \textit{e}-ending with plural possessa. 

%Hasidic Yiddish has preserved the inflection to agree with the number of the noun it modifies existed in pre-War Yiddish. While \varnothing-ending forms are used for singular nouns as shown in the table above, the \textit{-e} ending denotes agreement with plural nouns as shown in following examples: \textit{undzer tish} vs \textit{undzere tishn} `our table' vs `our tables', \textit{dayn feder} vs \textit{dayne feders} `your pen' vs `your pens', \textit{ir bukh} vs \textit{ire bikher} `her book' vs `her books'.

%Larger degree of confusion is found in plural possessive pronouns \textit{undzer, enker, ayer, zeyer}, which could be ascribed to phonological reasons: the UK and American Hasidic Yiddish speakers tend to reduce /r/ in fast speech in the last syllable of a three syllable word, so that forms \textit{undze(r)} and \textit{undze(re)} are sometimes hardly distinguishable. The existence of this confusion is supported by written data, where forms \textit{undze} are found for both singular and plural nouns: \textit{undze kehile} `our community', \textit{undze kop} `our head', \textit{undze menahel} `our director' along with \textit{undze nemen} `our names', \textit{undze gvirim} `our rich and influential people'. However, in the overall data the convention to use dependent possessive pronouns with \varnothing-ending for singular nouns and with \textit{-e} ending for plural nouns appears to be quite strong.

The fact that the singular/plural distinction survives in Contemporary Hasidic Yiddish is surprising given the fact that the corresponding agreement in attributive adjectives is no longer productive. In pre-War and Standard Yiddish, attributive adjectives agreed with the noun for case, gender, and number, while in Contemporary Hasidic Yiddish the pre-War agreement morpheme \textit{-e} has been reanalyzed as a marker of attribution and is applied to all attributive adjectives regardless of number, case, or gender (\citealt{Author20, Author21}; see also \citealt[489--496]{Krogh12} and \citealt[42]{Assouline14} for a slightly different view). Thus, unlike in attributive adjectives, dependent possessives retain a number distinction.\footnote{We did not expect that our questionnaire would provide these results, and we therefore felt that we did not have enough data for certain informative forms. We therefore searched \citet{KaveShtiebel} for possessive pronouns with all possible endings. We checked approximately 250 of the results, which conformed to our findings.}

%Maintenance of this convention in contemporary Hasidic Yiddish is surprising given the fact that the corresponding agreement in attributive adjective is no longer operative in Hasidic Yiddish, where all adjectives tend to have \textit{-e} ending regardless number of modifying noun (\citealp{Author21}, \citealp[489--496]{Krogh12}. To double-check our findings, we tested this convention using the written texts in Hasidic forum Kave Shtiebel \citep{KaveShtiebel}.\footnote{We used Google search inside this website, looking for possessive pronouns with all possible endings. About 250 entries checked support usage of this convention in Hasidic Yiddish.} The written data fully confirms the existence of this convention.



\subsection{Independent possessive pronouns}

In independent possessive pronouns, we find two competing sets of forms: one with an \textit{-e} suffix, and the other with a \textit{-s} suffix, as demonstrated in Table \ref{tab:possessivessuff}.\footnote{The \textit{-e} suffix is pronounced [ɛ] or [ә], while the \textit{-s} suffix is /s/. After a liquid, /s/ often surfaces as [ts] due to a process of /t/ insertion.} This pattern is in contrast to the situation in pre-War and Standard Yiddish, where independent possessive pronouns inflected according to gender and number: \textit{-er} for \textsc{ms}, \textit{-e} for \textsc{fs}, \textit{-s} for \textsc{ns}; and \textit{-e} for plural \citep[183--184]{Jacobs05}. In contemporary Hasidic Yiddish \textit{-e} forms (historically \textsc{fem} \textsc{sg} or \textsc{pl} forms) and \textit{-s} forms (historically Neut \textsc{sg} forms) used regardless of noun gender and number and represent two competing realisations of independent possessive pronouns. To the best of our knowledge, this constitutes an innovative pattern. 


\begin{table}
\caption{Independent possessive pronouns in Contemporary Hasidic Yiddish}
\label{tab:possessivessuff}
 \begin{tabularx}{\textwidth}{llYYYY}
  \lsptoprule
  	  & &\multicolumn{2}{c}{\textit{\textsc{E}}-forms} &\multicolumn{2}{c}{\textit{\textsc{S}}-forms} \\
  	  & & Singular  & Plural & Singular  & Plural \\
  \midrule
	1\textsuperscript{st} & & \textit{mayne} &  \textit{undzere} & \textit{mayns} &  \textit{undzers}  \\
\midrule
	\multirow{2}{*}{2\textsuperscript{nd}} & \textsc{Fam} & \textit{dayne} &  \textit{enkere (ayere)} & \textit{dayns} &  \textit{enkers (ayers)}  \\
	& \textsc{Hon} & \textit{ayere } & \textit{ayere } & \textit{ayers } & \textit{ayers }  \\
\midrule
	\multirow{2}{*}{3\textsuperscript{rd}} & \textsc{Masc} & \textit{zayne}  &  \multirow{2}{*}{\textit{zeyere}} & \textit{zayns}  &  \multirow{3}{*}{\textit{zeyers}}  \\
	& \textsc{fem} & \textit{ire}  & & \textit{irs}  &    \\
%	&  \textsc{Inan.} &  \textit{zayne}  & &  \textit{zayns}  & \\
  \lspbottomrule
 \end{tabularx}
\end{table}

Both forms are used regardless of case, gender, and number. Some speakers make use of both forms: \REF{ex:IU1a} and \REF{ex:IU1b} were both produced by the same participant, IU1, while the other examples were produced by IU2 \REF{ex:IU2}, NO1 \REF{ex:NO1}, NU1 \REF{ex:NU1}, and MU1 \REF{ex:MU1}.


\begin{multicols}{2}
\ea \gll de pene iz mayne \\
the pen is mine\label{ex:IU2}\\
\z

\ea \gll er iz dayne\\
{it (lit: he)} is yours\label{ex:IU1a}\\
\z

\ea \gll di pens zenen zayne \\
\textsc{dem} pens are his\label{ex:NO1}\\
\z

\ea \gll deye tish iz mayns \\
\textsc{dem} table is mine\label{ex:NU1}\\
\z

\ea \gll yene feders zenen nisht dayns \\
\textsc{dem} pens are not yours\label{ex:IU1b}\\
\z

\ea \gll deye bukh iz nisht zayns \\
\textsc{dem} book is not his\label{ex:MU1}\\
\z
\end{multicols}



`Vos' speakers and Israeli speakers from various groups typically prefer \textit{-e} forms, while speakers in the New York and Montreal areas, especially Satmar Hasidim, tend to prefer \textit{-s} forms. However, two individual exceptions prove instructive. IO1 is a `vos' speaker who now speaks Yiddish mostly with `vus' speakers. She notes that she used to use \textit{-e} forms of independent possessives, but now that she often speaks to `vus' speakers, she is now more likely to use \textit{-s} forms than previously. Contrastingly, NO2, an older `vos' speaker, follows a slightly different pattern, using \textit{-er} forms for singular nouns and \textit{-e} forms for plural nouns. Nevertheless, while tendencies towards one form or another can be observed at a group level, on an individual level most participants produce both forms in seemingly free variation. 

\hspace*{-2.7pt}Independent possessive pronouns with \varnothing-endings are rare in our dataset, which is consistent with the findings of \citep[242--243]{Mark78} for pre-War Yiddish. However, the phonological factors discussed in Section \ref{sec:depposs} play a role here as well: in some cases it is difficult to determine whether a participant produces e.g. \textit{undzer} (which might be pronounced /unzɛ/) or \textit{undzere} (which might be pronounced /unzɛː/) and thus we cannot rule out the existence of \varnothing-forms in stems ending in /r/.


\largerpage
We have also recorded a specific variant for the 3\textsc{\textsc{fs}} possessive pronoun, which has not, to the best of our knowledge, been documented before. The form \textit{zire} (her/hers, parallel to \textit{ire}), can be used as a dependent possessive pronoun, e.g. \textit{zire kinder} `her children'. As an independent possessive pronoun it is only attested in our dataset after a 3\textsc{sg} auxiliary verb, e.g. \textit{dos iz zire hoyz} `it is her house', but not with a plural verb form.~% For example, \textit{de kinder zenen zire} `the children are hers', ``doesn’t sound right'', as SH1 explained. 
 When asked about this form, all three of our consultants recognised it, with one reporting that ``\textit{zire} for `her’ or for `hers’ is very frequent I’d say''. Another consultant notes that the form is used in speech, but not in writing: ``We would sometimes use \textit{zire} and not \textit{ire}. I would definitely write \textit{ire}, but sometimes you'd say \textit{zire}''. While this form is much less widespread than \textit{ire}, we have observed it interactions with native Contemporary Hasidic Yiddish speakers in the New York area, Israel, Stamford Hill, and the Montreal area.


 %The isolated form \textit{zire} when asked specifically is also recognised by some of our participants: ``\textit{Zire} for `her’ or for `hers’ is very frequent I’d say''. Another participant IP1 notes that the form is used in oral speech, but not in writing: ``We would sometimes use \textit{zire} and not \textit{ire}. I would definitely write \textit{ire}, but sometimes you'd say \textit{zire}''. According to our observations this variant is well known in Hasidic communities in Stamford Hill UK, and also to some speakers in the USA. 

\subsection{Overall trends in possessives}

\largerpage
Possessive pronouns in Contemporary Hasidic Yiddish exhibit distinct dependent and independent morphological patterns. Dependent possessive pronouns have two variants: one (with an \varnothing-ending) appears with singular possessa while the other (with an \textit{e}-ending) appears with plural possessa. This pattern, while not innovative as it existed in pre-War and Standard varieties of Yiddish, is nonetheless unexpected as it does not follow trends seen in attributive adjective and definite determiner morphology towards a single uninflected form. Independent possessive pronouns appear with one of two suffixes, \textit{-s} or \textit{-e}. A subset of speakers, particularly those from Israel and speakers of the `vos' variety, uses only one suffix, \textit{-e}, for all independent possessive pronoun stems, regardless of case, gender, or number features. The remaining speakers, who constitute a majority, use stems with the \textit{-s} suffix in free variation with stems with the \textit{-e} suffix. Both patterns represent a departure from pre-War and Standard varieties of Yiddish, in which independent possessives were inflected for case, gender, and number. Thus, the traditional case and gender morphology has been reanalyzed as markers of a distinct syntactic role.


Certain questions remain to be answered. The first concerns the use of a specific possessive-indefinite construction. This construction, where an inflected form of the possessive pronoun is followed by the indefinite article and associated noun, existed in pre-War Yiddish \citep[109]{Katz87}. We have found written evidence (on \citealt{KaveShtiebel}) that this construction is also used in Contemporary Hasidic Yiddish. These preliminary data suggest that, for singular possessives, the \textit{-er} ending is used to mark this construction: \textit{mayner a bakanter} `an acquaintance of mine', \textit{mayner a fraynd} `a friend of mine' regardless of case, gender, and number: \textit{men zingt zayner a nigen} `someone sings a \textit{nigun} (traditional wordless melody) of his', \textit{mit mayner a noenter yedid} `with a close friend of mine'. However, as in pre-War Yiddish \citep[27--28]{Mazin27} \varnothing-ending forms are used for plural possessors: \textit{fun undzer a tayerer khaver} `from a dear friend of ours'.

%This construction is also used in Contemporary Hasidic Yiddish. For singular possessives the \textit{-er} ending is used to mark this construction: \textit{mayner a bakanter} `an acquaintance of mine', \textit{mayner a fraynd} `a friend of mine' regardless of case, gender and number: \textit{men zingt zayner a nigen} `someone sings a \textit{nigun} (traditional wordless melody) of his', \textit{mit mayner a noenter yodid} `with a close friend of mine'. However, as in pre-War Yiddish \citep[27--28]{Mazin27} \varnothing-ending forms are used for plural possessors: \textit{fun undzer a tayerer khaver} `from a dear friend of ours'.

The second question concerns dependent possessive pronouns in postposition, which preliminary written data suggest exhibit a strong tendency to use the \textit{-er} ending regardless of the historical gender of a particular noun: \textit{a khaver mayner} `my friend', \textit{mit a khaverte irer} `with her female friend', \textit{dos harts undzerer} `our heart', \textit{a gantse toyre enkerer} `all your theory', \textit{der mayse enkerer} `your story', \textit{di gefilen mayner} `my feelings'. Morphological case also appears to have no bearing on the morphology of the possessive in this construction, which is consistent with the rest of Contemporary Hasidic grammar. However, on rare occasions some speakers use a distinct \textit{-e} ending for plural nouns: \textit{di zikhroynes mayne} `my memories', \textit{di kinder mayne} `my children'. The tendency suggests the emergence of a new grammatical meaning for the \textit{-er} ending in possessive pronouns, namely a general marker of an attributive possessive pronoun in postposition for any noun regardless case, gender, and number.



\section{Demonstratives}\label{sec:demonstratives}

Standard and pre-War varieties of Yiddish did not have a dedicated proximal determiner: either the stressed definite article or a deictic form such as \textit{ot} or \textit{dozik-} (both roughly meaning `this here'), or a combination of the two, could be used where English uses \textit{this}. The status of distal demonstratives is somewhat less clear. \citet[186]{Jacobs05} claims that the inflected root \textit{yen-} is used. However, \citet{Katz87} claims that in contexts where English would use \textit{that}, the same approaches could be used as with proximal determiners, but that there also existed a distal demonstrative \textit{yener/-e/-em}, which he says can be used neutrally or associated with aggression, `otherness' or derogatory connotations.

Discussing contemporary varieties of Haredi Yiddish, both \citet[58]{Assouline14} and \citet[95]{Sadock18} describe a stressed form of the definite determiner, pronounced \textit{dey}/\textit{dei} or \textit{deye}, which is used as a demonstrative. They link this form to a Hungarian Yiddish pronunciation of the feminine and plural definite determiner \textit{di} as this variety often diphthongises stressed vowels, and they claim that it is absent in Polish Yiddish. Similarly, \citet{Krogh12} reports \textit{des} and \textit{deys} as variants of the neuter pronouns \textit{es} and \textit{dus/dos}. 

In the oral translation task, participants were asked to translate sentences containing a variety of pronominal forms from either English or Modern Hebrew, according to their preference. These sentences included independent and dependent proximal and distal determiners in a range of case and gender contexts.\footnote{The questionnaire was originally composed in English and translated by a native speaker into Modern Hebrew. As the proximal/distal distinction in Modern Hebrew does not map on to that of English, we expect some differences in the choice of demonstrative stem between participants translating the English version of the questionnaire and those using the Modern Hebrew version.This issue is discussed further in Section \ref{sec:dvsy}.} Additionally, speakers produced spontaneous metalinguistic discussion, which was recorded and analyzed, complementing the elicited translation data. Most speakers produced between 40 and 50 demonstrative tokens. Our results indicate a different pattern than that of either Standard and pre-War varieties of Yiddish or previous descriptions of contemporary varieties of Haredi Yiddish. Specifically, we find innovations in demonstrative stems, in the distribution of what \citet{Jacobs05} refers to as proximant and distant demonstratives, and on the inflection associated with them. We discuss these issues in turn.

\subsection{Demonstrative stems}\label{sec:d-stem}

All participants distinguish between two forms of the demonstrative, one beginning with /j/ and one with one beginning with /d/. We will refer to these as \textit{y}-stems and \textit{d}-stems, respectively. The \textit{y}-stem for all participants, regardless of age, geographical origin or Hasidic affiliation, was \textit{yen-}, corresponding to the dedicated distal demonstrative described by \citet{Katz87} and \citet{Jacobs05}. However, much more variation is found in the \textit{d}-stem. 

All `vos' speakers follow the traditional pattern of using the stressed definite determiner as a demonstrative, although younger `vos' speakers may use another \textit{d}-stem in addition to this option. The stressed definite determiner often co-occurs with elements such as \textit{ot} or \textit{dozik-}, as is the case in Standard and historical varieties. While these speakers use a stressed definite determiner in both independent and dependent contexts, they distinguish between these two contexts in the form of the determiner. They overwhelmingly use \textit{dos} as an independent determiner (for some speakers, even in non-nominative contexts) in both the plural and singular, and prefer \textit{di} as a dependent determiner, although \textit{der} and \textit{dem} are also found. This pattern suggests that, for `vos' speakers, \textit{dos} is emerging as a distinct, independent demonstrative, while the older strategy of using a stressed definite determiner persists for dependent demonstratives. 

In dependent contexts, some `vos' speakers use a single invariant definite determiner, while others use a variety of forms (i.e. \textit{der, di, dos, dem}). The form of the determiner is not determined by the case or gender of the DP, as evidenced by the appearance of non-Standard-like usages such as \textit{dem} in the nominative, or \textit{der} in the plural. These findings are consistent with those of \citet{Author21}.

`Vus' speakers show somewhat more variety in their choice of \textit{d}-forms. In dependent contexts, all speakers use the stem \textit{dey-}, which was described by \citet{Krogh12, Assouline14, Sadock18}, although there are a small number of tokens of the more conservative stressed \textit{di} (none use a stressed determiner form other than \textit{di}). In independent contexts, \textit{dey-} is found alongside \textit{dos} (pronounced /dʊs/) and even \textit{diye}. There are no instances of `vus' speakers using a stressed definite determiner other than \textit{dos} as an independent demonstrative. Some participants exclusively use the stem \textit{dey-} in independent contexts while others use a mix of the two (or three) stems, but no participant uses \textit{dos} or \textit{diye} exclusively. For those that mix stems, \textit{dos} is never found outside of nominative contexts, but it is found alongside other \textit{d}-stems in the nominative. The same pattern is found in singular as in plural nouns: \textit{dos} is regularly used and, for some speakers, preferred in nominative contexts and \textit{dey-} is used elsewhere. 

Beyond the nominative/objective distinction and the `vos'/`vus' distinction, there appears to be no pattern to when participants use particular \textit{d}-stems, or to which participants are likely to use particular forms. \textit{Dey-} appears to be the preferred stem for most `vus' speakers regardless of age, gender, Hasidic affiliation, or geographical origin, but these factors do not appear to influence whether a speaker makes use of other \textit{d}-stems and, if so, which they use. `Vos' speakers largely use the traditional strategy of using a stressed definite determiner as a \textit{d}-stem, although younger speakers may be beginning to adopt the \textit{dey-} stem. 




\subsection{Distribution of \textit{d}- and \textit{y}-stems}\label{sec:dvsy}

Regardless of which particular \textit{d}-stems are used, all participants make use of both \textit{d}- and \textit{y}-stems. However, the distribution of these stems does not appear to be entirely determined by the proximal/distal distinction. Before discussing the results of this task, a note on the proximal/distal distinction in English and in Modern Hebrew is in order.

English demonstratives are proximal (\textit{this, these}), i.e. close to the speaker, or distal (\textit{that, those}), i.e. further from the speaker, although the border between the two is obviously subjective (see e.g. \citet{Stirling02} for further discussion). In Modern Hebrew, demonstratives are similarly categorised as either proximal (\textit{(ha)ze, (ha)ele}) or distal (\textit{(ha)hu, (ha)hi, (ha)hem}), but their distribution differs from that of English. The proximal demonstratives are employed much more frequently than their distal counterparts \citep{Halevy13}, with the latter often reserved for contrastive contexts. In such cases, the proximal demonstrative is used to denote a referent which the speaker views as emotionally close or relatable, with the demonstrative serving to denote a referent which the speaker regards as ‘remote or adversative’ \citep{Halevy13}. The Modern Hebrew demonstratives thus differ in function from those of English, as the Hebrew ‘proximal’ demonstratives are used as a default form, often irrespective of spatial deixis, while the ‘distal’ demonstratives are frequently restricted to specifically contrastive settings. Jacobs' (\citeyear{Jacobs05}) categorisation of demonstratives in Yiddish is thus much closer to that of English, whereas Katz's (\citeyear{Katz87}) is perhaps closer to Hebrew, although he does not specifically mention the notion of contrast.

Participants translated from either an English or a Modern Hebrew version of the same questionnaire, with the Modern Hebrew utilising the distal (and less frequent) demonstratives \textit{(ha)hu, (ha)hi, (ha)hem} where the English version used \textit{that, those}. Given the differing distribution of demonstratives in these two languages, we might expect an effect of the language of the questionnaire on participants' responses. 

Regardless of the language they were translating from, participants almost never use \textit{y}-stems to translate English \textit{this} or \textit{these}, or Modern Hebrew \textit{(ha)ze} or \textit{(ha)ele}. This suggests that such stems cannot act as proximal demonstratives, which is consistent with both \citet{Jacobs05} and \citet{Katz87}. However, like their proximal counterparts, English \textit{that} and \textit{those} are also usually translated with \textit{d}-stems, contra \citet{Jacobs05}. Participants using the Modern Hebrew version of the questionnaire were most likely to translate \textit{(ha)hu, (ha)hi, (ha)hem} using \textit{y}-stems, although even they did not do so consistently. This suggests that the Contemporary Hasidic Yiddish proximal/distal distinction does not map directly onto either the English system or the Modern Hebrew system. 

The only factor that is consistently associated with the use of \textit{y}-stems is contrast. All speakers translated at least some distal/proximal contrastive pairs using a \textit{d}-stem and a \textit{y}-stem, and only a very few produced other types of pairs.\footnote{Other contrastive pairs produced by our participants include two \textit{d}-stems (fewer than five tokens overall), two \textit{y}-stems (one token overall) and an unstressed determiner plus a \textit{y}-stem, as well as forms such as \textit{de andere} `the other' and \textit{di other} `the other'.} We therefore suggest that \textit{y}-stems are primarily a marker of contrast, rather than distance from the speaker. This conclusion is supported by metalinguistic discussion with one speaker in particular who, unprompted, proposed after completing the questionnaire that they ``would only use \textit{yene} when I mean \textit{yene} and not \textit{deye}". 




\subsection{Demonstrative morphology}

The final innovation in the demonstrative system we will discuss is the inflection found on independent and dependent demonstratives. In Standard and pre-War varieties, all determiners (i.e. the stressed definite article and the stem \textit{yen-}) were inflected for case, gender and number.%, with the exception of the form \textit{dos}, which could be used as an uninflected proximal demonstrative (REF)\todo{do we actually have a reference for this? I don't think Jacobs or Katz discusses it --> no, reword}.
 However, in Contemporary Hasidic Yiddish, the role of morphology is less straightforward. 

For almost all speakers, dependent \textit{yen-} appears exclusively with the ending \textit{-e}, leading to the existence of a single invariant \textit{y}-form, \textit{yene}. In line with developments in the case and gender system on full nominals discussed in \citet{Author20, Author21}, this form is not inflected for case, gender or number when used as a dependent demonstrative. A small number of the `vos' speakers produce inflected forms of \textit{yen-}, such as \textit{yener} and \textit{yenem} in dependent contexts, although such forms do not appear to be determined by nominal case or gender. Again, this is in line with the findings of \citet{Author21}, who suggest that the loss of case and gender marking on full nominals in Contemporary Hasidic Yiddish happened somewhat later in `vos' speakers than it did in `vus' speakers, and that more vestiges of this system can therefore be found among the former group.

As discussed in Section \ref{sec:d-stem}, `vos' speakers use a stressed definite determiner as the \textit{d}-stem in dependent contexts. However, a significant minority of `vos' speakers' translations of English and Hebrew demonstratives made use of an unstressed definite determiner. In at least some of these cases, a demonstrative reading was clearly intended (i.e. where the participant pronounced the target sentence aloud in English with a demonstrative before translating it to Yiddish), but in other cases this is less clear. We leave this issue aside here. 

For `vus' speakers, \textit{dey-} appears in dependent contexts with either a \textit{-e} suffix or with a $\varnothing$ ending. Due to diphthong smoothing, the distinction between the two can sometimes be difficult to perceive, but clear examples of both \textit{dey} and \textit{deye} in dependent contexts can be identified. Some speakers only use the form \textit{deye}, but no speakers use only \textit{dey}. For those speakers who do use \textit{dey}, this form appears to be in free variation with \textit{deye} as both are produced in all case and gender contexts for both singulars and plurals and for English and Modern Hebrew proximal and distal determiners. 

Despite the existence of the novel form \textit{dey}, it is striking that speakers do not use the unaffixed root *\textit{yen}. One could easily imagine that the dependent demonstrative system could develop by analogy with the possessive system, where a distinction between singular and plural dependent possessives existed in Standard and pre-War varieties and continues to exist in large part in Contemporary Hasidic Yiddish. In the possessive system, singular nominals usually appear with an unaffixed form of the possessive, while plural nouns appear with either an unaffixed possessive or a possessive with the \textit{-e} suffix. Such a system is not evident in the dependent demonstratives, where no bare form of the \textit{y}-stem exists, and the bare and suffixed forms of the \textit{d}-stem appear in free variation. While it is not impossible that these two systems will develop in such a way as to become more similar, we do not see any evidence of such a development in this study. 

\hspace*{-4.5pt}In the independent demonstratives a somewhat different pattern emerges. Some speakers use the \textit{-e}-suffixed forms of both \textit{dey} and \textit{yen} as invariant independent demonstratives \textit{deye} and \textit{yene}. These speakers typically use the same invariant forms in dependent contexts.

A much larger group of speakers distinguish between nominative and objective forms of the independent demonstratives. For these speakers, the objective form appears with a \textit{-e} suffix, while the nominative ends with /s/. Thus, the \textit{y}-stem appears as \textit{yens} in the nominative and \textit{yene} in the objective. However, the \textit{d}-stem is somewhat different: the nominative can use either the stem \textit{dey-}, producing the \textit{deys} observed by \citet{Krogh12}, or what appears to be a distinct (although likely etymologically related) stem, \textit{dos}, pronounced /dʊs/. This pattern obtains for both singular and plural nouns: no number distinction is observed in either dependent or independent demonstratives, except among `vos' speakers who do not use forms other than \textit{di} in the plural. 

A small number of other forms are also found, including \textit{dis} and \textit{diye}, but these fall in to the patterns described above. \textit{Dis} acts as a \textit{d}-stem with an /s/ ending (i.e. it appears in the nominative as an independent demonstrative), while \textit{diye} acts as a \textit{d}-stem with an \textit{-e} ending (i.e. it appears in the objective as an independent demonstrative. None of these innovative forms appear as dependent demonstratives. 


\subsection{Overall trends in demonstratives}

Four patterns or tendencies emerge from our findings. Patterns 1–3 are primarily found in `vus' speakers while Pattern 4 is typical of `vos' speakers. 


\begin{table}
\caption{Patterns of demonstrative use in Contemporary Hasidic Yiddish}
\label{tab:demonstratives}
 \begin{tabularx}{\textwidth}{lXYY}
  \lsptoprule
  	   & Dependent  &Independent   &  \\
	   & & \textsc{nom} & \textsc{obj} \\
  \midrule
	\multirow{2}{*}{Pattern 1}  & \textit{deye} & \multicolumn{2}{c}{\textit{deye}} \\
	 & \textit{yene} & \multicolumn{2}{c}{\textit{yene}}\\
\midrule
	\multirow{2}{*}{Pattern 2}   & \textit{deye/dey}  & \textit{deys} & \textit{deye} \\
	 & \textit{yene}  & \textit{yens} & \textit{yene}\\
\midrule
	\multirow{2}{*}{Pattern 3}   & \textit{deye/dey}  & \textit{dos} & \textit{deye} \\
	 & \textit{yene}  & \textit{yens} & \textit{yene} \\
\midrule
	\multirow{2}{*}{Pattern 4}  & \textit{di/der/dem}   & \textit{dos} & \textit{dos/dem} \\
	 & \textit{yene/yener/yenem}  & \textit{yene, yens} & \textit{yene/yenem} \\
  \lspbottomrule
 \end{tabularx}
\end{table}

However, very few speakers fit into one of these patterns without exceptions. Some speakers produce both \textit{dos} and \textit{deys} in free variation, some speakers from Patterns 2 and 3 occasionally use \textit{deye} or \textit{yene} in the nominative, some speakers from Patterns 1–3 occasionally produce stressed definite determiners as dependent demonstratives, and even some `vos' speakers, who largely follow Pattern 4, produce \textit{deye} on occasion. Pattern 1 appears to be more common among Israelis, while Pattern 2 is more prevalent in the New York area and Montreal. Stamford Hill speakers are split between Patterns 2 and 3, and speakers of Patterns 2 and 3 can be found in all geographical communities.

The restriction of \textit{dos} and \textit{deys} to nominative contexts, as well as the fact that these forms are not found dependently, suggests that independent demonstratives are pronominal forms. As discussed in Section \ref{sec:personalpronouns}, the singular personal pronouns also distinguish between nominal and objective forms. However, perhaps unexpectedly, for demonstratives but not for personal pronouns, the nominative/objective distinction persists in the plural. This appears to be a striking innovation in the demonstrative system compared to Standard and pre-War varieties of Yiddish: while the stems \textit{dey-} and \textit{-yen} existed historically, they do not appear to have been used as pronouns. Rather, as they carried the same case and gender morphology as definite determiners and dependent demonstratives, they appear to have acted as determiners even in dependent contexts. Like other functional categories, pronouns are usually considered a closed class and therefore resistant to new members. It is therefore surprising that the demonstrative pronouns observed in this study, \textit{yens/yene} and \textit{deys (dos)/deye} appear to have been so quickly and pervasively adopted, at least among `vus' speakers.\footnote{As novel pronouns, it is expected that they should carry the same case distinctions as other pronominal forms, such as 1\textsc{sg} \textit{ikh} vs. \textit{mikh/mir}, 2\textsc{sg} \textit{du} vs. \textit{dikh/dir}, 3\textsc{sg} \textit{er} vs. \textit{em}, etc. Indeed, innovative gender neutral pronouns in English, such as \textit{ze/zir} and \textit{sie/hir}, typically have the same case distinctions as other English pronouns (\textit{I/me}, \textit{she/her}, etc.).}

It is similarly striking that speakers, especially `vus' speakers, do not appear to distinguish between singular and plural demonstratives, in either dependent or independent contexts. This innovation can clearly not be a result of contact with English or Modern Hebrew, as demonstratives in both of these languages agree with their nouns for number. It remains to be seen whether such a distinction will emerge. 

\newpage
Demonstratives constitute a distinct pattern to both possessive pronouns and attributive adjectives. The former, like demonstratives, distinguish between dependent and independent forms, although dependent possessives are unlike de\-mon\-stra\-tives in that they consistently distinguish between singular and plural possessa. Attributive adjectives have an invariant suffix \textit{-e} that distinguishes them from predicative adjectives, which appear in the bare form. In demonstratives, there is no bare form of the \textit{y}-stem and the dependent forms \textit{dey} and \textit{deye} appear in free variation. Given that none of these systems has a precedent in pre-War or Standard varieties of Yiddish it is surprising that Contemporary Hasidic Yiddish has innovated three distinct systems in the realms of attributive adjectives, possessive pronouns, and demonstratives.


The development of a single, uninflected form of the dependent demonstrative, as well as `vos' speakers' use of a variety of inflected forms, is in line with developments observed in definite determiners \citep{Author20, Author21}. Additionally, the variety of individual strategies we find in the demonstratives and the overall heterogeneity of the system is consistent with our findings on the personal pronoun and possessive systems in suggesting a system in flux. How this situation will resolve itself is yet to be seen, and a number of questions remain open.

The first regards the apparently free variation between the dependent demonstratives \textit{dey} and \textit{deye}. Further study is required to determine whether their distribution is predictable due to some factor to which our questionnaire was not sensitive. If they are indeed in free variation, this situation may persist; one form might die out in favour of the other (leaving a system more similar to the definite determiners); or the two forms might differentiate into, for example, a singular and a plural form, as appears to be the case with dependent possessives. Similarly, it remains to be seen whether Patterns 2 and 3 will merge, or whether a distinction between \textit{deys} and \textit{dos} will emerge. 

There is also the question of how robust Pattern 4 is or, put more generally, the extent to which the `vos' variety of Contemporary Hasidic Yiddish will remain distinct from the `vus' variety. \citet{Author21} suggests that the loss of morphological case and gender on full nominals happened later among `vos' speakers as these communities historically mixed less with other communities, but that this loss is nonetheless complete. They argue that the change in the `vos' variety may have been driven by contact with `vus' speakers, among whom the innovations in the morphological case and gender system were more established. The same may be true of developments in the demonstrative system, an idea which is supported by the fact that the youngest `vos' speaker in our study produced some tokens of \textit{deye}, a characteristically `vus' form. 

Finally, given the relative rarity of Pattern 1, we might wonder how long it will persist. As the personal pronoun system seems to crystalise into one that distinguishes nominative from objective forms, at least in the singular, it may be that speakers of Pattern 1 adopt one of the other patterns and thereby develop a nominative/objective distinction in demonstrative pronouns as well. 


\section{Conclusions}\label{sec:conclusions}

In this chapter we have shown that the Contemporary Hasidic Yiddish pronominal system has undergone a number of innovations vis-\`a-vis the pre-War and Standard Yiddish varieties. Our results are based on a survey questionnaire which systematically elicited data for the personal pronouns, reflexive pronouns, dependent and independent possessive pronouns, and dependent and independent demonstratives from 29 native speakers of Contemporary Hasidic Yiddish in the main Hasidic centers worldwide (Israel, the New York area, London's Stamford Hill, the Montreal area, and Antwerp). Our findings indicate a system in flux, with a high degree of variation present both between and within speakers regardless of geographical location. This variation applies to all of the pronominal categories that we examined. 

With respect to the independent personal pronominal paradigm, we found a widespread trend towards paradigm levelling, with the traditional three-way nominative/accusative/dative distinction in the singular shifting to a two-way nominative/objec\-tive distinction, and the traditional two-way nominative/objec\-tive distinction in the 1\textsc{pl} and 2\textsc{pl} shifting to a single unchanging form. These changes appear to have been driven by the effects of substantial dialect mixing, with the historical Mideastern Yiddish pronominal paradigm exerting the greatest influence, both in terms of the structure of the paradigm and in terms of actual forms used (e.g. the 2\textsc{pl} form \textit{enk}). The reflexive paradigm has also undergone a degree of levelling in comparison with the pre-War Mideastern variety of Yiddish. Our survey also allowed us to map a system of strong and weak personal pronouns, which are likely to have existed in some form or another in pre-War Yiddish varieties, especially the Mideastern ones, but, perhaps because they are an effect of colloquial speech, have not been clearly documented previously. It is noteworthy that `vus' speakers appear to have a nominative/objective distinction in the weak 3\text{pl}, which does not to the best of our knowledge have historical precedent and stands in contrast to the general trend towards simplification in the personal pronoun paradigm.

The possessive pronouns have dependent and independent variants. The dependent variants exhibit one form (with a \varnothing-ending) when modifying a singular noun and another one (with an \textit{-e} ending) when modifying a plural noun. This is in keeping with the pre-War model and goes against our prediction that the Contemporary Hasidic Yiddish dependent possessive pronouns would have undergone or be undergoing the same streamlining process as attributive adjectives, which we have elsewhere \citep{Author20, Author21} demonstrated to have lost their pre-War case and gender distinctions in favour of an invariant attributive marker \textit{-e}. The independent variants are morphologically distinct from their dependent counterparts, but have lost all of the pre-War case and gender distinctions in favour of a more streamlined innovative model with two paradigms, \textit{-s} and \textit{-e}, which have the same function and can be used interchangeably even within the speech of a single participant. 

The demonstrative pronouns likewise have dependent and independent variants. The proximal/distal distinction exhibited in both the dependent and independent variants differs from those of both English and Modern Hebrew, with the ``distals'' serving primarily to mark contrast. The stem used for the proximal demonstratives varies somewhat and thus seems to be in flux, though `vos’ speakers tend to use the traditional definite determiner forms (\textit{der, di, dos}, and \textit{dem}, though without the pre-War case and gender distinctions), while `vus’ speakers tend to use the novel stem \textit{dey-}. The dependent variants behave morphologically like (Contemporary Hasidic Yiddish) definite determiners, i.e. they show no case or gender distinctions. However, the independent variants show a novel nominative/objective case distinction (in contrast to pre-War Yiddish, which exhibited the same case and gender morphology as the definite article and adjectives). This suggests that the demonstratives have been reanalyzed on analogy with the personal pronouns and follow the same pattern of a two-way nominal/objective distinction that the singular personal pronouns exhibit. As such, we posit that Contemporary Hasidic Yiddish has innovated a novel demonstrative pronoun. 

These innovative features of the pronominal system support our claim that the Yiddish spoken in 21\textsuperscript{st}-century Hasidic communities constitutes a distinct variety of the language, which, though descended from the pre-War Eastern European dialects, has evolved away from them to such an extent that it can no longer be analyzed within this older dialectal framework. While the pronominal system has not lost case and gender in the same way that the nominal system has, the pronominal innovations are of the same magnitude as those affecting the nominal case and gender system. Our analysis shows that Contemporary Hasidic Yiddish has not simply lost forms and functions in comparison with the pre-War varieties, but rather has innovated them. These innovative features are not determined directly by contact with the dominant co-territorial languages, but rather are internal developments which bear witness to the linguistic vibrancy of Contemporary Hasidic Yiddish. 






\section*{Abbreviations}
\begin{tabularx}{.45\textwidth}{lQ}
\textsc{1} & first person \\
\textsc{2} & second person \\
\textsc{3} & third person \\
\textsc{Acc} & accusative \\
\textsc{Dat}& dative\\
\textsc{Def}& definite\\
\textsc{Dem}& demonstrative\\
\textsc{Fam}& familiar\\
\textsc{Fem/F}& feminine\\
\end{tabularx}
\begin{tabularx}{.45\textwidth}{lQ}
\textsc{Hon}& honorific\\
\textsc{Imp}& imperative\\
\textsc{Inan}& inanimate\\
\textsc{Masc/M}& masculine\\
\textsc{Neut} & neuter \\
\textsc{Nom} & nominative \\
\textsc{Pl/P} & plural \\
\textsc{Sg/S} & singular \\
\\
\end{tabularx}


\section*{Acknowledgements}
We gratefully acknowledge Eli Benedict for his help in administering the questionnaire, as well as for his insightful discussions. We are also thankful for the contributions of Shifra Hiley, and for the cooperation of our participants. This research is generously funded by the UK Arts and Humanities Research Council and the Leverhulme Trust.

\sloppy
\printbibliography[heading=subbibliography,notkeyword=this]

\end{document}
