\documentclass[output=paper,hidelinks,draftmode]{langscibook}
\ChapterDOI{10.5281/zenodo.7446985}
\author{Kathy Rys \affiliation{Dutch Language Institute (INT), Leiden; Meertens Institute, Amsterdam} and Elizana Schaffel Bremenkamp\affiliation{{Stricto Sensu} Post-Graduate Linguistics Program of the Federal University of Espírito Santo, Brazil}}
\title{Using data of Zeelandic Flemish in Espírito Santo, Brazil for historical reconstruction}
\abstract{This paper focuses on the case of Zeelandic Flemish in Espírito Santo, an obsolescent language variety spoken by about twenty descendants of Dutch immigrants to Brazil in the nineteenth century. The speech of rusty speakers can be used to reconstruct the original immigrant language. We perform a historical reconstruction of the old Zeelandic Flemish dialect as spoken in the days of emigration, with respect to three linguistic cases: (1) deletion of /l/ in codas and coda clusters, (2) subject doubling in inversion contexts and (3) the inflected polarity markers \textit{yes} and \textit{no}. Our findings demonstrate the historical value of transplanted dialects or speech island varieties (Rosenberg 2005). However, a comparison of our findings with historical data demonstrates that reliance on rusty speaker data alone may sometimes lead to incorrect conclusions and that the data should always be considered from the perspective of language contact as well.}

\IfFileExists{../localcommands.tex}{
  \addbibresource{../localbibliography.bib}
  % add all extra packages you need to load to this file

\usepackage{tabularx,multicol}
\usepackage{url}
\urlstyle{same}

\usepackage{listings}
\lstset{basicstyle=\ttfamily,tabsize=2,breaklines=true}

\usepackage{langsci-basic}
\usepackage{langsci-optional}
\usepackage{langsci-lgr}
\usepackage{langsci-osl}
% \usepackage{./langsci/styles/langsci-lgr}
% \usepackage{./langsci/styles/langsci-osl}
% \usepackage{langsci-gb4e}

\usepackage{tikz}
\usetikzlibrary{patterns,calc}
\pgfdeclarepatternformonly{south east lines}{\pgfqpoint{-0pt}{-0pt}}{\pgfqpoint{3pt}{3pt}}{\pgfqpoint{3pt}{3pt}}{
    \pgfsetlinewidth{0.6pt}
    \pgfpathmoveto{\pgfqpoint{0pt}{3pt}}
    \pgfpathlineto{\pgfqpoint{3pt}{0pt}}
    \pgfpathmoveto{\pgfqpoint{.2pt}{-.2pt}}
    \pgfpathlineto{\pgfqpoint{-.2pt}{.2pt}}
    \pgfpathmoveto{\pgfqpoint{3.2pt}{2.8pt}}
    \pgfpathlineto{\pgfqpoint{2.8pt}{3.2pt}}
    \pgfusepath{stroke}}
    
\usepackage{stmaryrd}
\usepackage{wasysym}
\usepackage{multirow}
\usepackage{caption}
\usepackage{subcaption}
\usepackage{mathrsfs}
\usepackage{qtree}

\usepackage{linguex}


  %pminos do not split footnotes
% \interfootnotelinepenalty=10000 %Footnote in Laporte chapters has to be split SN


%\DeclareIndexNameFormat{default}{%
%\nameparts{#1}%
%\usebibmacro{index:name}%
%{\index[names]}%
%{\namepartfamily}%
%{\namepartgiveni}%
% {}% L1
% {}% L2
%{\namepartprefix}% generates spurious space L3
%{\namepartsuffix}% generates spurious space L4
%}

%  {\DeclareIndexNameFormat{default}{%
%     \usebibmacro{index:name}{\index[names]}{#1}{#3}{#5}{#7}}}

%\DeclareIndexNameFormat{default}{%
%  \usebibmacro{index:name}{\sindex[nom]}{#1}{#3}{#5}{#7}}

%\DeclareIndexNameFormat{default}{%
%  \usebibmacro{index:name}{\sindex[person]}{#1}{#3}{#5}{#7}}
%\DeclareIndexNameFormat{default}{%
%\nameparts{#1} \usebibmacro{index:name}{\sindex[person]]}{\namepartfamily}{‌​\namepartgiven}{\nam‌​epartprefix}{\namepa‌​rtsuffix}}

%\newcommand{\smiley}{:)}

%\renewbibmacro*{index:name}[5]{%
%\usebibmacro{index:entry}{#1}%
%{\iffieldundef{usera}{}{\thefield{usera}\actualoperator}\mkbibindexname{#2}{#3}{#4}{#5}}}

% \newcommand{\noop}[1]{}

%remove for final
%\overfullrule=1mm

\newcommand{\tobi}[2]}}
\renewcommand{\S}[1]{\tobi{#1}{\textsc{*}}}

% this volume references
% puts: [this volume]
% already defined: \citetv
%\newcommand{\citepv}[1]{(\citeauthor{#1} \citeyear*{#1} [this volume])}
\newcommand{\citealtv}[1]{\citeauthor{#1} \citeyear*{#1} [this volume]}

%parentheses around example number
\newcommand{\pref}[1]{(\ref{#1})}

% in-text examples

\newcommand{\lnex}[1]{\textit{#1}} %target lang word
\newcommand{\lnlit}[1]{(lit.: `#1')} %literal reading
\newcommand{\lnlat}[1]{(#1)} % latinization
\newcommand{\lntrans}[1]{`#1'} %translation
\newcommand{\lnexl}[2]%
{\lnex{#1}{} \lnlat{#2}} % ex with latinization
\newcommand{\lnexlat}[3]{\lnex{#1}{} \lnlat{#2}{} \lntrans{#3}} % ex with latinization and tranl.

%ch01
\newcommand{\co}[1]{\mbox{\textbf{#1}}}

%ch09

\newcommand{\cyrbulg}[1]{\begin{otherlanguage*}{bulgarian}#1\end{otherlanguage*}}


%ch10
\newcommand{\nlp}{{\small NLP}}
\newcommand{\mwe}{{\small MWE}}
\newcommand{\rae}{{\small RAE}}
\newcommand{\lvc}{{\small LVC}}
\newcommand{\pos}{{\small P}o{\small S}}
%\newcommand{\todo}[1]{ \textcolor{red}{#1} }

%\renewcommand{\labelenumi}{\theenumi}
%\ainamefmt{{vv}{ll}{, ff}{, jj}} % fullname

\newcommand{\biberror}[1]{{\color{red}#1}}

\newcommand{\osenovaitem}{--~} 
  %% hyphenation points for line breaks
%% Normally, automatic hyphenation in LaTeX is very good
%% If a word is mis-hyphenated, add it to this file
%%
%% add information to TeX file before \begin{document} with:
%% %% hyphenation points for line breaks
%% Normally, automatic hyphenation in LaTeX is very good
%% If a word is mis-hyphenated, add it to this file
%%
%% add information to TeX file before \begin{document} with:
%% %% hyphenation points for line breaks
%% Normally, automatic hyphenation in LaTeX is very good
%% If a word is mis-hyphenated, add it to this file
%%
%% add information to TeX file before \begin{document} with:
%% \include{localhyphenation}
\hyphenation{
    Beck-man
    Ngu-yen
    back-chan-nel
    back-chan-nels
    mo-not-o-nous
    ste-reo-typ-i-cal
}

\hyphenation{
    Beck-man
    Ngu-yen
    back-chan-nel
    back-chan-nels
    mo-not-o-nous
    ste-reo-typ-i-cal
}

\hyphenation{
    Beck-man
    Ngu-yen
    back-chan-nel
    back-chan-nels
    mo-not-o-nous
    ste-reo-typ-i-cal
}
 
  \togglepaper[12]%%chapternumber
}{}

\begin{document}
\maketitle 
\shorttitlerunninghead{Using data of Zeelandic Flemish in Espírito Santo, Brazil}%


\section{Introduction}
\largerpage[-1]
In this paper, we present data from the speech of the last speakers of Zeelandic Flemish in Espírito Santo, Brazil. These speakers are descendants of Dutch immigrants, who left Zeeland in 1858-1862, but faced deprivation and difficulties in adaptation and integration into Brazilian society, with their language threatened by the majority language Brazilian Portuguese and by another heritage language, viz. that of the Pomeranian immigrants who arrived in Espírito Santo at the same time.

Among the last speakers of Brazilian Zeelandic Flemish, there are true semi-speakers, who only acquired the language incompletely and only by listening on an irregular basis to older speakers, and rusty speakers, who came a long way in learning to speak their mother language perfectly, but who stopped using their language on a regular basis and therefore forgot how to use some of its more complex features. In this paper, we discuss three linguistic features that occur in the speech of four rusty speakers of Brazilian Zeelandic Flemish. We discuss indications that the immigrants have been more conservative and less innovative than their counterpart speakers in the Netherlands. Our findings support the view that heritage language research can help in the historical reconstruction of ``protolanguages''.

In this paper, we start out from a reduced data set, i.e. confined to the modern varieties in Brazil and the Netherlands only. This implies that in first instance we ignore the available historical data. In this way we want to investigate the extent to which rusty speaker data alone can contribute to the reconstruction of the old language spoken by the ancestors in the time of emigration, which – under the assumption of lacking historical data – can be called the ``protolanguage''. However, we confront our findings with the available historical data in the end, which forces us to reconsider some of our conclusions.

\section{Zeelandic Flemish in Brazil: a case of a transplanted dialect} 
\largerpage[-1]
\subsection{Historical background}

In the 19\textsuperscript{th} century an association named \textit{Associação Central de Colonização} was established by the Brazilian imperial government. The goal of this association was twofold: on the one hand, they wanted to recruit European immigrants in order to have more manpower for the cultivation of agricultural land after the abolition of slavery, and on the other hand they wanted to attract more ``civilised whites'' to the country \citep[11]{RoosEshuis2008}. To this purpose, leaflets with promises of a better life were distributed in port cities of Europe. Thousands of fortune seekers from different European countries were persuaded in this way to migrate to Brazil.\footnote{\citet{Tschudi2004} gives the following numbers for Santa Leopoldina in 1860: total number of colonists: 1,003 (232 heads of family), of which: Swiss (104), Hannover (4), Luxembourg (70), Prussia (384), Bavaria (10), Baden (27), Hessen (61), Tirol (82), Nassau (13), Holstein (13), Mecklenburg (5), Saxonia (76), Belgium (8), Holland (126), France (1), England (1), and some Brazilians. The Prussian immigrants probably consisted mainly of Pomeranians. Initially, this latter group was only twice as large as the Dutch and Belgian immigrants together.} Among those immigrants, there were also 323 Dutchmen who settled in the state of Espírito Santo. Within Espírito Santo, there were two destinations: the colony of Rio Novo in the South and the colony of Santa Leopoldina in the interior of Vitória, the latter of which is of interest to this paper. The first Dutch immigrants arrived in the Santa Leopoldina colony in 1858, but new immigrants were arriving each year, up until 1862 \citep[50, 121]{RoosEshuis2008}. In total, 243 Dutch immigrants settled in Santa Leopoldina.

Upon arrival, the Dutch immigrants were immediately confronted with many difficulties and deprivations: they had to survive in dense forest, unused to the heat, on infertile land, without the equipment or money to rebuild their lives, short of food and without any assistance. Furthermore, the little that the immigrants were able to cultivate was in the possession of a colonel – also a Zeelandic Flemish immigrant – who controlled the planting and the harvesting. The immigrants sold their products at the colonel’s sale house (the \textit{venda}) for a meager price, but had to buy what they needed (e.g. salt) at the same \textit{venda} for exorbitant prices. Because of a negative report about the colony in 1862, the Brazilian imperial government ceased to offer any support \citep{Tschudi2004}. This resulted in total isolation and abandonment of the Dutch immigrants in Santa Leopoldina. Because of this situation, the Dutch immigrants did not integrate with other groups, which contributed to the maintenance of customs, such as the preservation of their dialect and religion (i.e. Calvinism) \citep{Buysse1984, RoosEshuis2008}. Another factor that contributed to the preservation of the Zeelandic Flemish language throughout the 19\textsuperscript{th} and 20\textsuperscript{th} century was the fact that only few members of the community had attended any school \citep[69]{Schaffel2010}. 

As time went by, the descendants of the Zeelandic Flemish immigrants got intermingled more and more with another group of immigrants, that is, the Pomeranian immigrants, mainly because of two reasons: because the Zeelandic Flemish community never attained their own Calvinist churches in Brazil – apart from a small chapel in Holandinha – they were mainly forced to go to church with the Lutheran Pomeranians, and because the total number of Zeelandic Flemish immigrants was rather small, they sometimes had to marry members from outside the community, which resulted in a growing number of Zeelandic Flemish-Pomeranian marriages.

While the transplanted Zeelandic Flemish dialect in Brazil has been subject to internal changes in the last decades due to the multilingual setting and regular processes of language obsolescence (see Sections \ref{sec:schaffel:1.2} and \ref{sec:schaffel:2}, see \citealt{SchaffelBremenkampPostma2017}), the Zeelandic Flemish dialects as spoken in the motherland were modified by processes of dialect loss and convergence to the Northern Dutch standard language. Consequently, these two varieties increasingly diverged from one other.

\subsection{Sociolinguistic situation}
\label{sec:schaffel:1.2}
Sociolinguistic research into the Zeelandic Flemish dialect spoken in Espírito Santo \citep{Schaffel2010} revealed that the language is currently spoken by just 13 people.\footnote{In 2015 five more speakers were identified.} These people, who are all descendants of the Zeelandic Flemish immigrants of the nineteenth century, speak the language with varying levels of proficiency and are of different ages, though the majority is more than 60 years old.\footnote{ To be precise, only 1 speaker is between 20 and 39 years old, 5 speakers are between 40 and 60 years old, and 7 speakers are older than 60.} Among the younger descendants, there are just a few who can understand the Zeelandic Flemish dialect or who speak a few words.\footnote{See \citet{SchaffelBremenkampPostma2017} for a more elaborate discussion on \citegen{Schaffel2010} sociolinguistic findings.}

Most of these 13 speakers state that they do not use the Zeelandic Flemish language on a regular basis. This is very likely induced by the fact that they live geographically dispersed in the old colony of Santa Leopoldina and have not much contact with each other. Next to this, the fact that the group of immigrants was small,\footnote{According to \citet[50, 121]{RoosEshuis2008} 243 people migrated from West Zeelandic Flanders to Espírito Santo between 1859 and 1862.} that these immigrants were mainly forced to go to church with the Lutheran Pomeranians, that they were abandoned by their motherland rather soon after migration and finally, the highly frequent occurrence of exogamous marriages (especially with Pomeranians), are considered to be the most important factors in the disappearance of Zeelandic Flemish in Espírito Santo \citep[83--85]{Schaffel2010}. Since the Pomeranian immigrants were more numerous,\footnote{ {Of the group of 3933 German immigrants in Espírito Santo, about 2000 were Pomeranian.}} and since they could practise their own Lutheran religion, their cultural values and their language were much better preserved (see also \citealt{Postma2014, Postma2019}).\footnote{This is, among other things, reflected in the fact that there is a Pomeranian language radio program, as well as a dictionary of Brazilian Pomeranian \citep{Tressmann2006b} and a collection of tales \citep{Tressmann2006a}.} As a consequence, the most prevalent home language in exogamous families is Pomeranian, not the Zeelandic Flemish dialect.\footnote{The Zeelandic Flemish language has never been recorded in publications of any kind.} This implies that Zeelandic Flemish is no longer transmitted to the next generation. Cessation in the intergenerational transmission of a language inevitably leads to the gradual loss of that language \citep{Sasse1992}.

Almost none of the contemporary descendants of the Zeelandic Flemish immigrants living in Espírito Santo have Zeelandic Flemish as their mother tongue. \citet[77]{Schaffel2010} found that only eight informants classified it as their only mother tongue, seven of whom were older than 60 years old. A further two informants specified Zeelandic Flemish and Portuguese as their mother tongues, and eight informants mentioned Zeelandic Flemish and Pomeranian as their mother tongues. 

The linguistic situation among the Zeelandic Flemish descendants in Espírito Santo is trilingual. Portuguese is the national language that is used in official organisations and in education. Pomeranian and Zeelandic Flemish are both transplanted languages that were taken to Brazil by immigrants in the 19\textsuperscript{th} century. As stated above, the Pomeranian language was almost always preferred as the home language in the numerous exogamous families. This situation has led to a gradual shift of the Zeelandic Flemish community to the dominant languages Portuguese and Pomeranian.\footnote{Whether the Zeelandic Flemish decendants have shifted to Portuguese or Pomeranian depends on the region they live in: a large majority shifted to Portuguese in (what is today called) Santa Leopoldina, whereas Pomeranian is the dominant language in Santa Maria de Jetibá and Itarana (see \citealt{Schaffel2010, SchaffelBremenkampPostma2017}).} The inevitable outcome of this situation for the Zeelandic Flemish language in Brazil is language death.

\section{Gradual language death and different types of speakers}
\label{sec:schaffel:2}
The last speakers of Zeelandic Flemish in Espírito Santo have not passed on the language to their children, so they can be considered as so-called ``terminal speakers'' \citep{Sasse1992} of a moribund language. As argued by \citet[195]{Dressler1996}, bilingual or multilingual speech communities are the ideal breeding ground for a situation of gradual language death, in which the minority language community shifts to the dominant language(s) of the majority. In our case, Zeelandic Flemish speakers have gradually shifted to Portuguese and/or Pomeranian. The outcome of this ``language shift'' – a notion focusing on the speech community rather than on the language \citep[5]{Rottet1995} – for the receding language is ``language loss'' \citep{AppelMuysken1987, FaseKroon1992} or ``language decay'' \citep{Field1985, Sasse1991} – notions that focus on the internal linguistic changes that the moribund language often undergoes. The final outcome of these internal changes and of the shift to the dominant language is called ``language death'' \citep{Dressler1972, Dorian1977}.

The internal changes affecting a declining language are not always distinguishable from changes affecting languages that are involved in ``normal'' contact situations. However, it is often a combination of various processes that contributes to language decay \citep[188]{CampbellMuntzel1989}. Some characteristic structural changes in dying languages are for example borrowing (resulting in loanwords and loan constructions or calques), reduction in syntagmatic redundancy and inflectional morphology, replacement of synthetic with analytic grammatical structures, reduction of stylistic variation, extreme phonological variation, extensive code switching, and so forth. In \citet[453--465]{SchaffelBremenkampPostma2017} we showed that most of these linguistic characteristics actually occur in the language of the Zeelandic Flemish descendants in Espírito Santo. It is typical of language death situations that these internal changes progress rapidly.\footnote{This effect, in which a transplanted language changes more rapidly than the motherland language, due to fading linguistic norms and the influence of other dominant languages among others, is discussed by \citet{Rosenberg2005} in his work on German language islands in Brazil and Russia.}

In this contribution, however, the focus is not on the internal changes that have affected the Zeelandic Flemish language of the last speakers, but rather on the archaic features that have remained unchanged in the language of so-called ``rusty speakers'', a term coined by \citet{Sasse1992}. \citeauthor{Sasse1992} makes the distinction between two types of imperfect speakers of a dying language, that is, ``rusty speakers'' versus ``semi-speakers''. He defines rusty speakers as “former fluent speakers who were on their way to becoming full speakers, but never reached that degree of competence due to the lack of regular communication in the language” \citep[62]{Sasse1992}. He considers rusty speakers as a special type of L1 learners, who have “a comparably good proficiency in the grammatical system of the language and a perfect passive knowledge”, but who “suffered from severe memory gaps, especially in vocabulary, but also in more complicated areas of the grammatical system” \citep[61]{Sasse1992}. The imperfect language of a rusty speaker is the result of “later loss”. On the other hand, there are also true semi-speakers, whose command of the dying language is, as argued by \citet[61]{Sasse1992}, from the beginning “imperfect to a pathological degree”. Because of the interruption in the transmission of the language, the semi-speaker acquires the language incompletely, and not “by way of normal acquisition processes (i.e. parent-to-child, by means of conscious language transmission strategies on the part of the parents), but rather ``by chance'', by interacting more than usual with elderly members of the community” \citep[36--37]{Rottet1995}. Semi-speakers should therefore be considered as L2 speakers. In practice, however, it is often difficult to categorise strictly between either rusty speakers or semi-speakers, because, as Sasse points out, there is a language proficiency continuum between individuals who learned the dying language by chance and those who learned it in more regular ways.

Regardless of the question of how speakers acquired the moribund language, it is characteristic of situations of language death that linguistic norms are broken down, due to the fact that none of the speakers are regarded by the language community as “local authorities on language questions” \citep[39]{Rottet1995}. As a result, there is a “relaxation of internal monitoring” \citep[154]{Dorian1981}: language community members come to accept and tolerate a gamut of uses of linguistic features that would have been regarded as mistakes in earlier times. This huge amount of variation in the use of a declining language by different speakers is one of the reasons why researchers might wonder “whether data from a dying language can reliably be used by linguists for other purposes, e.g. by historical linguists for purposes of reconstruction of protolanguages” \citep[3]{Rottet1995}. In this paper we argue that linguistic data from rusty speakers of Zeelandic Flemish in Espírito Santo can indeed be used to reconstruct the ``protolanguage'' of the original immigrants.\footnote{The term ``protolanguage'', borrowed from comparative linguistics, may not be completely appropriate for the situation at hand, because there is ample historical data available for the linguistic situation in the days of emigration. However, we use this term in the context of the methodological approach of this paper (see Introduction), in which we start out from a reduced data set, i.e. confined to the modern varieties in Brazil and the Netherlands only (and thus ignoring any historical data), in order to investigate the extent to which rusty speaker data alone can help in reconstructing the old language as spoken in the days of emigration.} Since rusty speakers do no longer have many opportunities to speak their mother language, it does not get mixed up so strongly with the dominant language(s) and remains relatively ``authentic'', i.e. close to the original variety of the ancestors. This variety is not similar to the Zeelandic Flemish dialect as it is spoken in the motherland nowadays, since the latter variety has been subject to processes of dialect loss under the influence of northern Standard Dutch.\footnote{Northern Standard Dutch refers to Standard Dutch as it is spoken in the Netherlands, as opposed to southern Standard Dutch, which is the standard language spoken in the Dutch-speaking part of Belgium, i.e. Flanders.} We discuss three linguistic phenomena that occur in the language of these rusty speakers: (1) deletion of /l/ in codas and coda clusters with concomitant compensatory lengthening of the preceding vowel, (2) subject doubling and (3) inflected polarity markers \textit{yes} and \textit{no}. All three phenomena still occur in some Belgian Flemish dialects, but have disappeared from the contemporary variety as spoken in the Dutch province of Zeelandic Flanders, which has strongly converged to northern Standard Dutch. Our findings demonstrate the potential historical value of transplanted dialects or speech island varieties \citep{Rosenberg2005}, in that such a variety may – as in the case of the rusty speakers of Brazilian Zeelandic Flemish – resemble the original immigrant (proto-)language more closely than is the case for the homeland variety.

\section{Methodology}

The linguistic data that will be discussed in the next section are based on spoken language material that was collected in 2012 and 2013, by Gertjan Postma (Meertens Institute), Andrew Nevins (Federal University of Rio de Janeiro and University College London) and both authors of this paper. Interviews were taken from nine (five male and four female) speakers of Zeelandic Flemish, living in Holandinha (part of Holanda), Garrafão, Alto Jatibocas, Caramuru and Alto Jetibá (see \figref{fig:schaffel:1}).


\begin{figure}

% \includegraphics[width=\textwidth]{figures/schaffel-img001.png}
\includegraphics[width=\textwidth]{figures/Schaffel1.pdf}
\caption{Locations of the speakers of Zeelandic Flemish in Espírito Santo, Brazil\tiny (c) OpenStreetMaps contributors}
	\label{fig:schaffel:1}
\end{figure}


All of these speakers were older than 40. Recordings were made at people’s homes, where two or more speakers were present and talked with each other and with the interviewers. Three of the four interviewers spoke Portuguese with the informants, one also spoke northern Standard Dutch, one of them was a native speaker of Pomeranian, and another interviewer was a native speaker of a Belgian Flemish dialect which closely resembles the Zeelandic Flemish dialects of the places the immigrants originally came from. The interviews were a mixture of these four languages with a lot of code switching in the informants’ speech, but with the general aim of eliciting as much Zeelandic Flemish dialect as possible. Topics that were talked about during the interviews were the informants’ past, their language, their religion, the work on the land, and so forth. The interviews were audio- and video-recorded and cover more than four hours of speech, only part of which is in Zeelandic Flemish. The recordings are preserved at the Meertens Institute (Amsterdam). Next to this, one interview made by Arjan Van Westen and Monique Schoutsen for their movie \textit{Braziliaanse} \textit{Koorts} (``Brazilian Fever'') was also used for the data discussed in this paper. Large parts of the recordings have been transcribed, glossed and annotated, with segmentation in Praat by Gertjan Postma, Kathy Rys and Lea Busweiler. The linguistic data discussed in the next section were selected from these transcripts. These data come from four rusty speakers: they are all speakers who used to speak more Zeelandic Flemish in the past, but who lost the opportunity to speak it on a daily basis from the 1990s onwards \citep{Schaffel2010}. The four speakers concerned have the following profiles:

\begin{itemize}
\item speaker 1, named A., is an 84-year-old man from Holandinha, whose mother tongue is Zeelandic Flemish and who also speaks Portuguese, which he acquired at school, but no sooner than at the age of six; his ancestors came from the West-Zeelandic village of Nieuwvliet;
\item speaker 2, named J., a male informant living in Holandinha, whose mother tongue is Zeelandic Flemish and who also speaks Portuguese, which he acquired at school; his ancestors came from the West-Zeelandic village of Retranchement;
\item speaker 3, named R., is a 65-year-old man from Alto Jatibocas, whose mother tongue is Zeelandic Flemish, who is married to a Pomeranian woman, and who speaks Zeelandic Flemish, Pomeranian, Hochdeutsch and Portuguese, the latter of which he started to acquire at school at the age of six;
\item speaker 4, named B., is R.’s sister, age unknown, a woman from Alto Jatibocas, whose mother tongue is also Zeelandic Flemish,\footnote{R. and B.’s mother only spoke Zeelandic Flemish with her children when they were little.} who is married to a Pomeranian man who is also able to speak the Zeelandic Flemish language because his mother was Zeelandic Flemish; like R., B. speaks four languages: Zeelandic Flemish, Pomeranian, Hochdeutsch and Portuguese;
\item all four speakers have in common that they did not go to school for more than four years.
\end{itemize}

The places where these speakers lived belong to different language areas. Ho\-lan\-dinha belongs to the Portuguese area, i.e. the area where a lot of descendants of the former slaves live. In this area Portuguese is the only dominant language and there is not much interference of Portuguese with Zeelandic Flemish \citep{SchaffelBremenkampPostma2017}. Alto Jatibocas belongs to the Pomeranian area, where the Zeelandic Flemish people have often married Pomeranian partners and where the Pomeranian language is the main language used in families, which has resulted in a lot of interference between Pomeranian and Zeelandic Flemish.

\section{Results}

In this section we argue that the speech of rusty speakers contains certain features that deviate from the current motherland variety, but that cannot be attributed to language contact. Instead, these features are remnants of the protolanguage, that was once also spoken in the motherland, and that was transplanted to Brazil by the immigrants of the 19\textsuperscript{th} century. These protolanguage features allow us to reconstruct the original language of the immigrants and of the motherland. In this way, the speech of so-called ``terminal'' speakers may be of historical value. We give evidence supporting this claim by discussing three linguistic phenomena: (1) the deletion of /l/ in codas and coda clusters with concomitant compensatory lengthening of the preceding vowel, (2) subject doubling, and (3) the inflected polarity markers \textit{yes} and \textit{no}.

\subsection{Deletion of /l/ with concomitant compensatory lengthening of the preceding vowel} 

The speech of all four rusty speakers, contained a phonological feature that is illustrated in the underlined parts of the example sentences \REF{ex:schaffel:1} to \REF{ex:schaffel:6}:\footnote{The phonetic realisation of the underlined words is given with each sentence, as well as the initials and informant number of the speakers involved.}

\ea speaker 1 (A.)
\label{ex:schaffel:1}

\gll Dat is \ul{al} twee of drie keer vermaakt hé \\
{} {} [ɑː] {} {} {} {} {}\\
\glt ‘that has already been repaired two or three times you know’

\z 


\ea speaker 1 (A.)
\label{ex:schaffel:2}

\gll Den eersten a’ j’ ier stoeng, das al \ul{al} kapot \\
{} {} {} {} {} {} {} {} [ɑː] {}\\
\glt ‘the first one that was standing here, that’s already completely broken’

\z 


\ea speaker 3 (R.)
\label{ex:schaffel:3}

\gll Dan is’t zo, \ul{ielk} brieng wat ja \\
{} {} {} [i:k] {} {} {} \\
\glt ‘then it is like this, everybody brings something you know’

\z 

\ea speaker 3 (R.)
\label{ex:schaffel:4}

\gll Koeien zo \ul{kalvers} \\
{} {} [ˈkɑːvərs]\\
\glt ‘cows, you know, calves’

\z 

\ea speaker 4 (B.)
\label{ex:schaffel:5}

\gll Op ‘t land gewèèrkt, ja alles…koffie, koeien, koeien \ul{melken}\\
{} {} {} {} {} {} {} {} [ˈmæːʔ\~{ə̩}]\\
\glt ‘worked on the land, yes everything…coffee, cows, milking cows’

\z 


\ea speaker 4 (B.)
\label{ex:schaffel:6}

\gll Toe ‘s navonds eh{\ldots} achte{\ldots} \uline{half} negene \\
  {}  {} {}      {}         {}              [ɑːf]           {}\\
\glt ‘until in the evening eh… eight o’clock\ldots half past eight’
\z 

\hspace*{-1mm}The underlined words are realised with a deleted /l/ and with concomitant compensatory lengthening of the preceding vowel, resulting in an overlong vowel. The /l/ that is deleted is always a so-called ``dark /l/'' (phonetically transcribed as [l̴]). In Dutch, /l/ is realised differently according to its position in the syllable: clear [l], realised as a lateral approximant, in the onset of a syllable and dark [l̴], realised as a pharyngealised approximant, in the coda before a pause or another consonant. Vocalisation of dark /l/ occurs in some varieties and some speakers of Dutch. A word like \textit{geel} ‘yellow’, with the underlying form /ɣel/, is then realised as [ɣew] \citep[17]{VanReenen1986, BotmaTorre2000}. The complete deletion of prepausal and preconsonantal /l/, however, is not a common feature in varieties of Dutch.\footnote{Cross-linguistically, /l/-vocalisation and /or -deletion occurs in certain dialects of English, Old French, Korean, and Swiss German. In Hungarian, we find complete deletion of prepausal and preconsonantal /l/ and compensatory lengthening of the preceding vowel, comparable to the examples \REF{ex:schaffel:1}-\REF{ex:schaffel:6} \citep{Feyer2012}.} Nevertheless, there is one variety in the Dutch-speaking area which is characterised by the categorical deletion of prepausal and preconsonantal /l/ and by a very strong compensatory lengthening of the preceding vowel. It is a Flemish dialect spoken in the village of Maldegem, which belongs to the Dutch-speaking part of Belgium (represented by the blue dot in \figref{fig:schaffel:2}).


\begin{figure}
\caption{Zeelandic Flemish villages (in the Netherlands) where the immigrants came from and the East Flemish village of Maldegem (in Belgium). Source: \citet[438]{SchaffelBremenkampPostma2017}}
\label{fig:schaffel:2} 
\includegraphics[width=\textwidth]{figures/schaffel-img002.jpg}

\end{figure}

The phonological process of /l/-deletion in the Maldegem dialect occurs in codas and coda clusters of stressed syllables\footnote{In a word like \textit{appel} ‘apple’, which has stress on the first syllable, /l/ is not deleted.} and can be represented as in \REF{ex:schaffel:7} (see \citealt[182--186]{Rys2007}):
%\ea
%	\label{ex:schaffel:7}
%\includegraphics[width=4cm]{figures/schaffel-img0007.png}
%\z

\ea
\begin{exe}
%\exp{ex:schaffel:7}
\phonrule{\featurebox{+stressed\\+lateral}}{\emptyset}{\parbox{2cm}{\longrule$\left\{\begin{array}{l}C\\\#C \end{array}\right\}$}}
\label{ex:schaffel:7}
\end{exe}
\z

The representation in \REF{ex:schaffel:7} indicates that /l/ is deleted before a consonant or a pause, but only if that pause is followed by a consonant. This is illustrated in~\REF{ex:schaffel:8}.

\ea%8
    \label{ex:schaffel:8}
\ea 
de ba\ul{l} pakken [dəmˈbɑːˌpɑʔ\~{ə̩}]\\ 
\glt ‘take the ball’

\ex 
de ba\ul{l} is\ldots [dəmˈbɑlˌɛ̝s]\\
\glt ‘the ball is\ldots’ 

\z
\z

\hspace*{-1mm}The examples \REF{ex:schaffel:1}-\REF{ex:schaffel:6} of our Brazilian informants are instances of exactly the process described in \REF{ex:schaffel:7}, that is, of preconsonantal or prepausal /l/-deletion with compensatory lengthening of the preceding vowel, resulting in an overlong vowel. The occurrence of a phonological feature in the speech of our rusty speakers that is also found in a current Flemish dialect, suggests that this feature might have been brought along with the original immigrants and thus might be a feature that also occurs in the dialects spoken in the villages of the motherland where these immigrants came from (see~\ref{fig:schaffel:2}). However, there are only scant indications of the occurrence of /l/-deletion in the current dialects of West Zeelandic Flanders.

 \Citet{BroeckedeMan1978} mentions /l/ as one of the omitted consonants in her monograph on the dialects of West Zeelandic Flanders. She argues that /l/ is deleted at the end of a word and illustrates this with the examples “à, wè, zà, wì-je?”, which is her own representation of regular Dutch orthography \textit{al} ‘already’, \textit{wel} ‘surely’, \textit{zal} ‘shall’ and \textit{wil je?} ‘do you want? (lit. want-you) \citep[9]{BroeckedeMan1978}.\footnote{ {This author uses the {\textasciigrave}-symbol (e.g. à, è, ì) to represent vowels that she calls “iets gerekt” (‘slightly prolonged’) \citep[9]{BroeckedeMan1978}. It is unclear whether her representation indicates the overlong vowel quality that is typical of the Maldegem dialect.}} It strikes us that the examples that are given are either frequent function words like \textit{al} ‘already’ and \textit{wel} ‘surely’, or frequent verbs like \textit{zal} ‘shall’ and \textit{wil} ‘want’. \citet[20--25]{BroeckedeMan1978} describes some grammatical aspects of the dialect of the village of Eede\footnote{Eede does not appear on \figref{fig:schaffel:2}. It should be located north of the village of Maldegem, just above the border between the Netherlands (which Zeelandic Flanders is part of) and Belgium.} in a separate chapter because of the exceptional character of this dialect, viz. the many grammatical similarities with the Maldegem dialect, which is spoken right across the border. Unlike the other West Zeelandic Flemish dialects, the Eede dialect is characterised not only by prepausal /l/-deletion (i.e. in the coda), but also by preconsonantal /l/-deletion (i.e. in coda clusters, e.g. \textit{melk} $\rightarrow$ [mæːk]) (see \citealt[352]{Rys1999, Rys2000}). Nevertheless, among the 504 Zeelandic Flemish people who migrated to Espírito Santo between 1858 and 1862 there were no inhabitants of Eede \citep[12]{RoosEshuis2008}.

According to \citet[163]{Taeldeman1979}, however, /l/-deletion with compensatory vowel lengthening only occurs in the dialect of Eede, in preconsonantal as well as prepausal contexts,\footnote{This is confirmed in Rys (1999; 2000).} whereas the other Zeelandic Flemish dialects display slight lengthening of vowels preceding /l/ in coda clusters (especially with alveolar consonants), but no deletion of /l/.

\largerpage
In order to check whether /l/-deletion occurs in the dialects of the villages the immigrants departed from, we consulted sound recordings of spoken dialogues that were made in the 1960s and 1970s by researchers of the university of Ghent and the Meertens Institute (Amsterdam). These recordings and the broad transcriptions of it are both part of a database called \textit{Stemmen uit het verleden} (‘Voices from the past’)\footnote{ \url{https://www.variaties.be/portfolio-item/stemmen-uit-het-verleden/}} as well as a database called \textit{Nederlandse Dialectenbank} (‘Dutch Dialect Database’).\footnote{\url{https://www.meertens.knaw.nl/ndb/}} We studied the transcriptions of recordings from several West Zeelandic Flemish places.\footnote{These places were Aardenburg, Biervliet, Breskens, Cadzand, Groede, Hoofdplaat, IJzendijke, Nieuwvliet, Retranchement, Schoondijke, Sint Kruis, Sluis, Waterlandkerkje, and Zuidzande. For all of these places, one transcribed recording was available. Together, these recordings covered more than nine hours of speech.} In these transcriptions, we only came across instances of prepausal /l/-deletion in the following words: \textit{veel} ‘much’ (also \textit{zoveel} ‘this much’, \textit{hoeveel} ‘how much’), \textit{wel} ‘\textsc{part}’, \textit{zal} ‘shall’, \textit{al} ‘already’, and \textit{nogal} ‘quite’. The process of /l/-deletion is thus restricted to a very small set of words, which implies that it is a lexically determined process.\footnote{The transcriptions do not allow us to conclude whether these cases of /l/-deletion are accompanied by a strong lengthening of the preceding vowel. There are indications that the vowels are only slightly lengthened, in agreement with \citet{BroeckedeMan1978}.} This is in contrast to the process of /l/-deletion in the dialects of Maldegem and Eede, where it is applied (nearly)\footnote{In Eede it does not apply in all cases (see below).} categorically \citep{Versieck1989, Rys1999}. In addition, the transcriptions display a lot of realisations of these words without /l/-deletion.\footnote{ {In the transcription of the Nieuwvliet dialect, for example, there are 13 cases of /l/-deletion (in the words} {\textit{wel}}, {\textit{veel}}{, and} {\textit{nogal}}{), but also 13 cases without /l/-deletion (in the words} {\textit{al}}, {\textit{wel}}).} Moreover, all of these instances are examples of prepausal /l/-deletion. Examples of preconsonantal /l/-deletion, as are produced by the Brazilian rusty speakers in the examples \REF{ex:schaffel:3}-\REF{ex:schaffel:6}, do not occur in these recordings of West Zeelandic Flemish dialects, except for one case: one informant from Hoofdplaat is talking about the North Sea flood of 1953, when he produces the following example of preconsonantal /l/-deletion:

\ea%9
    \label{ex:schaffel:9}

\gll en me zien de \ul{golven} zò mà over de diek nar ons toe komm’n\footnotemark{} \\
{} {} {} {} [ɦoːvən] {} {} {} {} {} {} {} {}\\
\glt ‘and we see the waves suddenly come to us over the dike’

\footnotetext{The transcription as an overlong vowel is our own interpretation, since the original (broad) transcription is not explicit about the vowel length.}
\z


Altogether, we can say that the West Zeelandic Flemish dialects of the 1960s -1970s are not characterised by a categorical process of /l/-deletion. Rather, it seems that /l/-deletion is a lexically determined process that is restricted to a small set of words and that is only applied in prepausal contexts. The occurrence of one instance of preconsonantal /l/-deletion might be a remnant of a process that was more common and widespread in the past. 

A more recent source of information on the phonological characteristics of the current West Zeelandic Flemish dialects is the \textit{Fonologische Atlas van de Nederlandse Dialecten}, which is abbreviated as \textit{FAND}, ‘Phonological Atlas of Dutch Dialects’ \citep{DeWulfTaeldeman2005}. This publication is based on a questionnaire that was conducted in 578 places in the Netherlands and the Dutch-speaking part of Belgium. The \textit{FAND} comprises 496 maps showing the most common pronunciations of particular words in different places. Only three Zeelandic Flemish places relevant to our study (i.e. places of origin of the Zeelandic Flemish immigrants) are represented on these maps, viz. Breskens, Zuidzande and IJzendijke\footnote{Eede is another West Zeelandic Flemish place that was included in the} {\textit{FAND}}{{}-research, but it is not represented in Figures~\ref{fig:schaffel:3} and \ref{fig:schaffel:4a}.} (encircled with red in Figures~\ref{fig:schaffel:3} and \ref{fig:schaffel:4a}).

\figref{fig:schaffel:3} shows the phonetic realisations of the word \textit{schuld} ‘guilt’ in different places in the Dutch-speaking areas (viz. the Netherlands and Flanders, Belgium). Thus, this map illustrates the process of preconsonantal /l/-deletion. Prepausal /l/-deletion is illustrated in \figref{fig:schaffel:4a}, which represents the phonetic realisations of the word \textit{vol} ‘full’. The hollow circles in Figures~\ref{fig:schaffel:3} and \ref{fig:schaffel:4a} represent complete deletion of /l/ in front of a consonant (\figref{fig:schaffel:3} ) or a pause (\figref{fig:schaffel:4a}), rendering the phonetic realisations [sxɛ̝ːt] and [vɛ̝ː],\footnote{The dialects of Kleit and Middelburg (both belonging to the municipality of Maldegem) are characterised by unrounding of the rounded vowels /ʏ/ (in \textit{schuld}) and /ɔ/ (in \textit{vol}).} respectively. This only occurs (almost) categorically in two East Flemish (Belgian) places, viz. Kleit and Middelburg (encircled with blue in Figures~\ref{fig:schaffel:3} and \ref{fig:schaffel:4a}), which are both submunicipalities of Maldegem.\footnote{The dialect of the main municipality, which we could call ``Maldegem-centre'', was not included in the data compiled for the \textit{FAND}.}\textsuperscript{,}\footnote{\figref{fig:schaffel:3} also shows complete deletion of /l/ (symbolised by a hollow circle) in the word.} The word \textit{schuld} in the dialect of Nukerke, which is in the southwest of the province of East-Flanders, as well as in some dialects in the deep south of the Dutch province of Limburg; and \figref{fig:schaffel:4a} shows complete deletion of /l/ in the word \textit{vol} in the dialect of Nukerke and the neighbouring dialect of Ronse. Some other maps in the {\textit{FAND}} {show a more widespread deletion of /l/: the map of} \textit{wolf} ‘wolf’, for example, shows deletion of /l/ in some West- and East Flemish places (e.g. in Nieuwpoort, Wervik, Nevele, Gent). However, Kleit is the only place where /l/-deletion applies categorically (in the dialect of Middelburg it applies nearly categorically). This was found by investigating 39 nouns and adjectives containing /l/ in codas or coda clusters that were included in the GTRP-database (\url{https://www.meertens.knaw.nl/projecten/mand/GTRPdataperitem.html}) on which the\textit{Morfologische Atlas van de Nederlandse Dialecten} (abbreviated: \textit{MAND}, ‘Morphological Atlas of Dutch Dialects’; \citealt{DeSchutterJong2005}) was based. In addition, the other West- or East Flemish places where /l/-deletion occasionally occurs do not constitute a concatenated area with the West Zeelandic Flemish places the immigrants originated from, unlike Kleit and Middelburg, which do make up a connected area with these places. None of the Zeelandic Flemish places where the immigrants came from\footnote{Encircled with red in Figures~\ref{fig:schaffel:3} and \ref{fig:schaffel:4a}.} display deletion of /l/. The hollow square in Breskens (Figures~\ref{fig:schaffel:3} and \ref{fig:schaffel:4a}), represents a velar approximant that tends to vocalisation, but without complete deletion. So, in recent sources like the \textit{FAND} (\citeyear{DeWulfTaeldeman2005}) there are no indications of preconsonantal or prepausal /l/-deletion in the current West Zeelandic Flemish dialects.


\begin{figure}[t]
\caption{Phonetic realisations of the word \textit{schuld} ‘guilt’ in dialects of the Dutch-speaking area. (Source: \citealt{DeWulfTaeldeman2005})}
	\label{fig:schaffel:3}
\includegraphics[width=\textwidth]{figures/schaffel-img003.png}
\end{figure}


\begin{figure}[t]
\caption{Phonetic realisations of the word \textit{vol} ‘full’ in dialects of the Dutch-speaking area. (Source: \citealt{DeWulfTaeldeman2005})}
\label{fig:schaffel:4a}
\includegraphics[width=\textwidth]{figures/schaffel-img004.png}
\end{figure}

\largerpage[-1]
Next to the data provided in the \textit{FAND}, the \textit{GTRP}{}-database\footnote{ \url{https://www.meertens.knaw.nl/projecten/mand/GTRPdataperitem.html}}, which constitutes the source material of the \textit{MAND} \citep{DeSchutterJong2005}, also provides information on the pronunciation of lexemes in the dialects of the Dutch-speaking area. In order to investigate whether or not /l/-deletion applies categorically, we looked up the pronunciation of 39 lexemes containing /l/ in a coda or coda cluster\footnote{ {The 39 lexemes under examination were:} {\textit{balk} }{‘beam’,} {\textit{beeld} }{‘statue’,} {\textit{bril} }{‘spectacles’,} {\textit{deel} }{‘part’,} {\textit{dweil} }{‘floor-cloth’,} {\textit{geld} }{‘money’,} {\textit{helft}} {‘half (noun)’,} {\textit{hol} }{‘hole’,} {\textit{kalf} }{‘calf’,} {\textit{kelder} }{‘cellar’,} {\textit{melk} }{‘milk’,} {\textit{naald} }{‘needle’,} {\textit{pols} }{‘wrist’,} {\textit{schelp} }{‘shell’,} {\textit{schuld} }{‘debt, guilt’,} {\textit{spel} }{‘game’,} {\textit{stal} }{‘stable’,} {\textit{steel} }{‘stalk, handle’,} {\textit{stoel} }{‘chair’,} {\textit{uil} }{‘owl’,} {\textit{volk} }{‘people’,} {\textit{wolf} }{‘wolf’,} {\textit{wolk} }{‘cloud’,} {\textit{zalf} }{‘ointment’,} {\textit{zeil} }{‘sail’,} {\textit{dol} }{‘silly’,} {\textit{zolder} }{‘attic’,} {\textit{fel} }{‘fierce’,} {\textit{geel} }{‘yellow’,} {\textit{half} }{‘half (adj.)’,} {\textit{heel} }{‘whole’,} {\textit{kalm} }{‘calm’,} {\textit{scheel} }{‘cross-eyed’,} {\textit{smal}} {‘narrow’,} {\textit{vals} }{‘false’,} {\textit{vol} }{‘full’,} {\textit{vuil} }{‘dirty’,} {\textit{wild} }{‘wild’,} {\textit{zilveren} }{‘silver’.}} in the various dialects of the Dutch-speaking area using this database. As in the \textit{FAND,} Kleit and Middelburg belong to the places that were included in this research, whereas Maldegem-centre does not. Kleit is the only place where complete deletion of /l/ and compensatory lengthening of the preceding vowel takes place in all 39 lexemes. The dialect of Middelburg does not have categorical deletion: for some lexemes it joins the neighbouring West Flemish dialects, in which case /l/ is not deleted (e.g. in \textit{balk} ‘beam’ > [bɑlk]). In other cases /l/ is deleted but there is no compensatory lengthening of the preceding vowel (e.g. in \textit{helft} ‘half’ > [{ӕft}] ), and in several cases /l/ is vocalised (e.g. \textit{kelder} ‘attic’ > [k{ӕ}\textsuperscript{j}dərə]). In addition, the \textit{GTRP-}database offered the opportunity to investigate whether deletion of /l/ occurs in the West Zeelandic Flemish places which the emigrants once departed from.\footnote{ {These are the same places as in the} {\textit{FAND}}{, i.e. Breskens, Zuidzande, IJzendijke and Eede.}} We found one attestation of (preconsonantal) /l/-deletion for the dialect of IJzendijke, more particularly in the word \textit{beeld} ‘statue’ (pronounced as [beːt]). The dialect of Eede, which – as was mentioned above – has some exceptional grammatical features with respect to the other West Zeelandic Flemish dialects (\citealt[20--25]{BroeckedeMan1978}, \citealt{Rys1999}) displays complete deletion of /l/ in 22 out of 39 lexemes,\footnote{In 14 out of 22 lexemes there is deletion of /l/ without compensatory lengthening of the preceding vowel.} in prepausal as well as preconsonantal contexts. To conclude, the \textit{GTRP}-database indicates more or less the same results as the sources discussed above (i.e. \citealt{BroeckedeMan1978, Taeldeman1979, DeWulfTaeldeman2005}, \textit{Nederlandse Dialectenbank}): except for Eede and one instance in IJzendijke, /l/-deletion does not seem to occur in contemporary dialects of West Zeelandic Flanders.

As the examples \REF{ex:schaffel:1}-\REF{ex:schaffel:5} show, however, we find relatively many instances of prepausal as well as preconsonantal /l/-deletion in the speech of rusty speakers of Zeelandic Flemish living in Espírito Santo. In one interview with speaker 1 (A.), /l/ deletes in 8 out of 14 cases, that is, in 57\% of possible cases. In one recording with speaker 3 (R.), we find /l/-deletion in 4 out of 11 cases (36\%). In speaker 4 (B.)'s speech, there are at least 8 clear cases of /l/-deletion.\footnote{Some fragments of the interview were unclear and therefore difficult to transcribe.}

To summarise, rusty speakers of Brazilian Zeelandic Flemish have prepausal as well as preconsonantal /l/-deletion in their speech, whereas this phenomenon is only infrequently observed in Zeelandic Flemish dialects as spoken in the 1960s-1970s (\citealt{BroeckedeMan1978}, \textit{Nederlandse Dialectenbank}) and – except from Eede – hardly observed at all in contemporary Zeelandic Flemish dialects (\textit{FAND}, \textit{GTRP}). These findings suggest that /l/-deletion has disappeared from contemporary varieties, but was a common and more widespread feature of Zeelandic Flemish dialects in the past, and thus, of the protolanguage that was spoken by the original Zeelandic Flemish migrants to Brazil. 

\subsection{Subject doubling}
\label{subsec:52subj}
The rusty speakers’ speech also exhibits a syntactic construction that nowadays still occurs in many (Belgian) Flemish dialects \citep{Haegeman1991, VanCraenenbroeckKoppen2002, DeVogelaerDevos2008, DeVogelaer2008}, but which seems to have disappeared from West Zeelandic Flemish dialects altogether. Again, there are indications that this construction was a feature of the protolanguage of the Zeelandic Flemish immigrants in Espírito Santo. It concerns the construction referred to as pronominal subject doubling, which implies the use of a combination of the full and reduced form of the subject pronoun. As argued by \citet[249]{DeVogelaerDevos2008}, the distribution of this phenomenon in the Flemish dialects depends on a number of parameters, including word order, clause type (main clause vs. subclause), type of subject in the clause (pronoun or not), the number of pronouns, etc. This is illustrated in the examples in \REF{ex:schaffel:11}–\REF{ex:schaffel:13} (examples are based on \citealt[243--244]{DeVogelaerDevos2008}):


\ea Regular word order: clitic preceding the verb and strong pronoun after the verb \\ 
%10
\label{ex:schaffel:10}
\gll {‘}\textbf{k} Zal \textbf{ik} dat wel krijgen\\
I\textsubscript{clitic} shall I\textsubscript{strong} that \textsc{adv} get\\
\glt ‘I will get that’
\z
 
 
In example \REF{ex:schaffel:11} a regular order is attested in which the inflected verb is preceded by the subject. This example contains the 1SG-clitic \textit{‘k} ‘I’ in subject position and it is doubled by an optional strong pronoun \textit{ik ‘}I’.


\ea Inverted word order: clitic and strong pronoun following verbs
\label{ex:schaffel:11}

\gll {Mag=}\textbf{ek=ik} dat wel weten?\\
may=I\textsubscript{clitic}=I\textsubscript{strong} that \textsc{adv} know\\
\glt ‘Am I allowed to know that?’

\z


The example in \REF{ex:schaffel:11} is an ``inverted'' word order in which the inflected verb precedes the subject.\footnote{It is likely that the inverted clitic pronoun duplicated by a strong pronoun is not in the subject position. The inversion can mark discursive effects such as ``focus'', for example.}


\ea Subclause: clitic and strong pronoun following complementiser
%12
\label{ex:schaffel:12}
\gll …da=\textbf{k=ik} dat mag weten\\
that=I\textsubscript{clitic}=I\textsubscript{strong} that may know\\
\glt ‘…that I am allowed to know that’

\z


In example \REF{ex:schaffel:12}, the clitic \textbf{‘k} ‘I’ and the strong pronoun \textbf{ik} ‘I’ follow the complementiser \textit{dat} ‘that’.\footnote{{As observed in example \REF{ex:schaffel:12}, the clitic} {\textit{‘k} }{‘I’ and the strong pronoun} {\textit{ik} }{‘I’ are morphophonologically linked to the complementiser that is on the left periphery of the sentence (and not in subject position).}}

\ea Clitic and strong pronoun following a particle that introduces comparison\footnote{{The combination of reduced and full pronominal form cannot be called ``doubling subject'' in \REF{ex:schaffel:13}. The subject in \REF{ex:schaffel:13} is} {\textit{Hij} }{‘he’ (and it is not doubled).}}\\
%13
\label{ex:schaffel:13}
Hij is groter dan=\textbf{ek=ik}\\
\glt he is taller than=I\textsubscript{clitic}=I\textsubscript{strong}

\z

The construction of subject doubling with inverted word order (following verbs and following complementisers) occurs in the speech of the rusty speakers in Espírito Santo, as is illustrated in the examples \REF{ex:schaffel:14}-\REF{ex:schaffel:19}. Mostly, these speakers use it in the case of the first person singular, but it also occurs with first person plural (as in example \REF{ex:schaffel:15}).

\ea Speaker 1 (A.)\\
%14
\label{ex:schaffel:14}
\gll Dan gaon\textbf{=k=ik} mien tuug uut {de (?)} dan kom {ik (?)}\\
then go=I\textsubscript{clitic}=I\textsubscript{strong} my material out-of {the (?)} then come {I (?)} \\
\glt ‘Then I will (get?) my material from the (?), then I come(?)’

\z



\ea Speaker 1 (A.)
%15
\label{ex:schaffel:15}

\gll Ja die kon goed Hollands praten gelijk a\textbf{=me=wudder} hier praten\\
yes that-one could well Hollandic talk like \textsc{comp}=we\textsubscript{clitic}=we\textsubscript{strong} here talk\\
\glt ‘yes he could speak Dutch well like we talk here’

\z


\ea Speaker 3 (R.)\\
%16
\label{ex:schaffel:16}
\gll Dan doe=\textbf{k=ik} dat in zakken, dan {leg=}\textbf{ek=ik} die weg\\
then do=I\textsubscript{clitic}=I\textsubscript{strong} that in bags then put=I\textsubscript{clitic}=I\textsubscript{strong} that away\\
\glt ‘then I put it in bags, then I put it away’ 

\z


\ea Speaker 3 (R.)
%17
\label{ex:schaffel:17}

\gll En dan doe=\textbf{k=ik} daor weer frisse onderbrengen\\
and then do=I\textsubscript{clitic}=I\textsubscript{strong} there again fresh-one bring-under\\
\glt ‘and then I add some fresh one again’ 

\z


\ea Speaker 4 (B.)\\
%18
\label{ex:schaffel:18}

\gll An=\textbf{k=ik} mee mien zusters topekomen ja\\
\textsc{comp}=I\textsubscript{clitic}=I\textsubscript{strong} with my sisters together-come yes\\
\glt ‘when I come together with my sisters, you know’

\z




\ea Speaker 4 (B.)\\
%19
\label{ex:schaffel:19}
\gll {An=}\textbf{k=ik} ulder eentwa zeggen dan verstaon zulder da\\
\textsc{comp}=I\textsubscript{clitic}=I\textsubscript{strong} them something say\textsubscript{1SG} then understand they that\\
\glt ‘when I say something to them then they understand it’
\z
 

The extent to which subject doubling occurs, depends on the speaker. Speaker 1, for instance, uses the construction in the first person singular following a verb (as in \REF{ex:schaffel:14}) in 1 out of 4 possible cases, whereas speaker 3 uses subject doubling in the first person singular following a verb in 10 out of 13 possible cases.\footnote{This concerns possible contexts for subject doubling, in which doubling may or may not be applied, such as in the utterance in \REF{ex:schaffel:00} in which the subject (\textit{ik} ‘I’) is only expressed once, which means that there is no subject doubling in this case.

\ea 
\label{ex:schaffel:00}
\gll da weet ik zo niet goed\\
that know I \textsc{part} not well\\
\glt `I'm not so sure about that’ (speaker 1, A.) 
\z
}
Although the subject doubling construction occurs in the Zeelandic Flemish language of the speakers in Espírito Santo, it seems to be absent from contemporary Zeelandic Flemish dialects as spoken in the present-day motherland. According to \citet[243]{DeVogelaer2008} and \citet{Barbiers2006} subject doubling in inverted word order for 1\textsuperscript{st} person singular is only observed in one Zeelandic Flemish place (Hulst), but this place is in the eastern part of Zeelandic Flanders and falls outside the region the immigrants came from. Further, the construction is restricted to Belgian (Flemish) dialects, as is illustrated in Figure~\ref{fig:schaffel:4}.


\begin{figure}
\fbox{\includegraphics[width=.95\textwidth]{figures/schaffel-img005.png}}
\caption{Geographical distribution of subject doubling in 1\textsuperscript{st} person singular following a verb, complementiser or comparative in the Dutch-speaking area (Source: \citealt[243]{DeVogelaer2008}, West Zeelandic Flanders is encircled with red).}
	\label{fig:schaffel:4}
\end{figure}


Likewise, only scant evidence of the occurrence of subject doubling in Zeelandic Flemish dialects is found by \citet{Will2004}, who studied the spontaneous speech of respondents from 39 Zeelandic Flemish villages as recorded in the 1960s-1970s by researchers of Ghent University and the Meertens Institute.\footnote{\url{https://www.variaties.be/portfolio-item/stemmen-uit-het-verleden} \url{https://www.meertens.knaw.nl/ndb/}} He observes a kind of remainders of subject doubling in the Zeelandic Flemish region, but only in 3\% of possible cases and mostly in dialects from places that fall outside the region of the immigrants. Will concludes that already in the 1960s, subject doubling was a marginal feature of the Zeelandic Flemish dialects. However, \citet[12]{BroeckedeMan1978} does mention subject doubling as a characteristic of West Zeelandic Flemish dialects and gives examples of the construction in normal and inverted word order.

As in the case of /l/-deletion, the occurrence of a linguistic feature in the speech of the rusty speakers which still occurs in the broader Flemish region, but seems to have disappeared almost entirely from the contemporary Zeelandic Flemish dialects, suggests that this feature is probably a protolanguage feature, which was brought along with the 19\textsuperscript{th} century immigrants. 

\subsection{Inflected polarity markers \emph{yes} and \emph{no}} 
\label{sec:53-inflectedpol}

Two of our rusty speakers, more specifically speakers 1 and 2, both living in Holandinha, use a construction known as inflected \textit{yes-} / \textit{no-}particles, which implies that the polarity markers \textit{yes} and \textit{no} are followed by subject clitics (represented in bold in the examples \REF{ex:schaffel:20}–\REF{ex:schaffel:22}), which refer to the subject of a preceding utterance (underlined in \REF{ex:schaffel:20}–\REF{ex:schaffel:22}). The examples in \REF{ex:schaffel:20} and \REF{ex:schaffel:21} contain an inflected \textit{yes}{}-particle referring to a singular, 3\textsuperscript{rd} person, male antecendent, example \REF{ex:schaffel:22} an inflected \textit{yes}{}-particle referring to a plural, 3\textsuperscript{rd} person antecedent.

\ea Speaker 1 (A.)
%20
\label{ex:schaffel:20}

\gll \ul{Die} \ul{camion} die is kapot gebrook'n. \textbf{Jao=j}, die staot dao nog\\
that truck that-one is {to pieces} broken. yes=he\textsubscript{clitic} that-one stands there yet\\
\glt ‘That truck is broken to pieces. Yes, it is still standing there.’ 

\z

\newpage
\ea Speaker 1 (A.)
%21
\label{ex:schaffel:21}

\gll \ul{Frans} \ul{B.}, \textbf{jao=j} die èè nie meer terug gewist\\
Frans B. yes=he\textsubscript{clitic} that-one has not anymore back been\\
\glt ‘Frans B., yes he did not come back anymore.’
\z


\ea Speaker 1 (A.): 
%22
\label{ex:schaffel:22}
 
\gll \ul{die} \ul{zoeten} \ul{drinken}, da ging nog dansen ne\\
those-ones {sweet liquor} drank, that went still dance\textsubscript{inf} TAG\\
\glt ‘The ones that drank sweet liquor still went dancing.’\\

\medskip

 Speaker 2 (J.): \\
\gll {Dansen} \textbf{jao=s}\\
dance\textsubscript{inf} yes=they\textsubscript{clitic}\\
\glt ‘Dancing, yes they did.’
\z 


Such inflected \textit{yes-} and \textit{no-}particles are nowadays restricted to some (Belgian) Flemish dialects \citep{Paardekooper1993} (especially West and East Flemish dialects), and take the following forms:\footnote{Although other forms are possible (see \citealt{Paardekooper1993} and \citealt[161, 168--170, 179--180, 195--197, 201--202]{DeVogelaerDevos2008}), we restrict ourselves to the forms that are most common in the region surrounding West Zeelandic Flanders.}\textsuperscript{,}\footnote{ {Since we only found instances of the inflected polarity marker} {\textit{yes}} {in the speech of the rusty speakers, we restrict ourselves to the possible forms of the} {\textit{yes}}{{}-particle.}}



\begin{table}
	\caption{Subject clitics following yes (/ no)-particle}
	\label{tab:schaffel:1}
\begin{tabularx}{.5\textwidth}{Xl}
\lsptoprule
1\textsc{sg}   (ja)=k& 1\textsc{pl}  (ja)=m\\
2\textsc{sg}    (ja)=g& 2\textsc{pl}  (ja)=g\\
3\textsc{sg} masc.  (ja)=j& 3\textsc{pl}  (ja)=s\\
3\textsc{sg} fem.  (ja)=s & ~ \\
3\textsc{sg} neuter  (ja)=t & ~ \\
\lspbottomrule
\end{tabularx}
\end{table}

%
\citet{DeVogelaerDevos2008} discuss the distribution of these forms in the Dutch-speaking area (i.e. the Netherlands and Flanders). Whereas the \textit{yes}{}-particle is inflected in many West- and East Flemish dialects (Belgium), De Vogelaer does not find any attestations of an inflected \textit{yes}-particle in West Zeelandic Flemish dialects.\footnote{There
is one attestation of inflected \textit{yes}-particles for 1SG in the municipality of Terneuzen, which is outside the region of West Zeelandic Flanders (see \citealt[161]{DeVogelaer2008}).} The distribution of the inflected \textit{yes-}particle for 3SG masculine (as in the examples \REF{ex:schaffel:20} and \REF{ex:schaffel:21}) is represented in Figure \ref{fig:schaffel:5} (Source: \citealt[196]{DeVogelaerDevos2008}).\footnote{See \citet{DeVogelaer2008} for maps of subject clitics following the} {\textit{yes}}{{}-particle for 1SG (p. 161), 1PL (p. 169), 2SG (p. 179), 3SG feminine (p. 201), and 3SG neuter (p. 202).} As can be seen on this map, there are no clitics (represented by the symbol x) following \textit{ja} (‘yes’) in West Zeelandic Flanders (which is encircled in red on \figref{fig:schaffel:5}), although \textit{ja} is inflected (as \textit{ja=j}, symbol: {\textbackslash}) in the (Belgian) Flemish region to the south of the border with West Zeelandic Flanders.
%

\begin{figure}[t]
\includegraphics[width=\textwidth]{figures/schaffel-img006.jpg}
\caption{Distribution of the inflected \textit{yes-}particle for 3SG masculine in the Dutch-speaking area. (Source: \citealt[161]{DeVogelaer2008})}
\label{fig:schaffel:5}
\end{figure}



The distribution of the inflected \textit{yes-}particle for 3PL (as in example \REF{ex:schaffel:22}) is represented in \figref{fig:schaffel:6} (Source: \citealt{Barbiers2006}) which shows that there is no inflection (symbolised by a blue square) of the \textit{yes-}particle in present-day West Zeelandic Flanders (encircled with green), although inflected forms occur in the surrounding (Belgian) Flemish dialects.


\begin{figure}
\includegraphics[width=\textwidth]{figures/schaffel-img007.png}
\caption{Distribution of the inflected \textit{yes-}particle for 3PL in the Dutch-speaking area. (Source: \citealt{Barbiers2006}). Copyright resides with the Meertens Institute.}
	\label{fig:schaffel:6}
\end{figure}

 

\citet{DeVogelaerDevos2008} do not find any evidence of an inflected \textit{yes}{}-partic\-les in Zeelandic Flemish dialects in whatever context, but according to \citet{Barbiers2006}, one informant from Cadzand (West Zeelandic Flanders) replies in a written questionnaire that the inflected \textit{yes-}particle is possible – although very rare – in his / her dialect in the case of 3SG feminine in the test sentence in \REF{ex:schaffel:24}:\footnote{This corresponds to test sentence 354 in \citet{Barbiers2006}.}

\ea
%23
\label{ex:schaffel:24}
 
\ea Question: 

\gll Gaat ze dansen?\\
goes she dance\textsubscript{inf}\\
\glt ‘Is she going to dance?’

\newpage
\ex Answer: 
{Ja’s(e).}\\
\glt {yes=she\textsubscript{clitic}}

\z 
\z
 

Likewise, one informant from Groede (West Zeelandic Flanders) replies that \textit{ja} ‘yes’ can be followed by a clitic in his /her dialect in the case of 1SG as in the test sentence in \REF{ex:schaffel:24a}\footnote{This corresponds to test sentence 353 in \citet{Barbiers2006}.}, although its occurrence is very rare.

\ea%24
    \label{ex:schaffel:24a}

\ea Question:

\gll Wil je nog koffie, Jan?\\
want you \textsc{part} coffee Jan\\
\glt ‘Do you want more coffee, Jan?’

\ex Answer:

\gll Ja’k.\\
yes=I\textsubscript{clitic}\\
\glt ‘Yes I do.’ 

\z 
\z 

To summarise, the fact that the surrounding Flemish dialects have inflection, and that two West Zeelandic Flemish informants indicate that inflected \textit{yes-}par\-tic\-les can occur, though infrequently, in their dialects, may suggest that inflected particles were more common in the Zeelandic Flemish protolanguage. Their occurrence in the speech of the rusty speakers in Espírito Santo supports this assumption.

\subsection{Conclusion}

In this section we analysed the elicited speech of four rusty speakers of the Zeelandic Flemish dialect living in Espiríto Santo. These speakers are descendants of 19\textsuperscript{th} century Dutch immigrants and belong to a very small group of terminal speakers of Zeelandic Flemish in Brazil. They are considered rusty speakers since they acquired the language as their mother tongue, but they have stopped speaking their language due to a lack of contact with other speakers of Zeelandic Flemish. Because of this absence of regular usage of the language, the language of these rusty speakers is only influenced by the multilingual setting to a relatively small extent. As a consequence, many aspects of their language have remained rather similar to the protolanguage of the original immigrants. At the same time, the present-day motherland language, that is, the current Zeelandic Flemish dialect as spoken in the province of Zeeland (the Netherlands), has been subject to processes of convergence to the northern Dutch standard language and dialect levelling in the past few decades. As a consequence, certain archaic linguistic features can be found in the language of the rusty speakers in Brazil, which have long since disappeared from the motherland language.\footnote{ {Our results about the disappearance of dialect features in the current West Zeelandic Flemish dialects are consistent with findings from the (sociolinguistic) literature that the Zeelandic Flemish regiolect is spoken to a much smaller extent than some other regiolects from the Dutch-speaking area (see \citealt[26]{rys2019onderzoeksrapport}) and that there has been a considerable amount of dialect loss in the province of Zeeland in the last five decades \citep[11]{Versloot2021}.} } In this section we have argued that we can use these linguistic features to reconstruct the original immigrants’ protolanguage. By way of experiment, we started out from a reduced data set, i.e. confined to the modern varieties in Brazil and the Netherlands only.

We have focused on three linguistic features: (1) deletion of /l/ in codas and co\-da clusters with concomitant compensatory lengthening of the preceding vowel, (2) subject doubling and (3) inflected \textit{yes- / no-}particles. A similar pattern was found for all three features: whereas the speech of the rusty speakers contains instances of /l/-deletion, subject doubling, and inflected \textit{yes-}particles, these phenomena hardly occur anymore (except from a few traces) in the contemporary West Zeelandic Flemish dialects, but do occur in the surrounding (Belgian) Flemish dialects. The occurrence of features in the speech of our rusty speakers that are also found in current Flemish dialects, seems to suggest that these features must have been brought along with the original immigrants and thus must once also have occurred in the dialects spoken in the villages of the motherland where these immigrants came from.

On the basis of these results, we can conclude that the speech of rusty speakers of a transplanted language can have historical value for the reconstruction of the original immigrants’ protolanguage. Most research on heritage languages focuses on the internal changes the minority language undergoes in the process of language obsolescence and under influence of language contact. For the historical reconstruction of the protolanguage, however, one has to focus on the unchanged features in heritage language speakers’ speech. The speech of rusty speakers lends itself best to such analysis, because these speakers use their language to such a small extent that some ``original'' language features have remained unchanged. However, when using heritage language data for historical reconstruction, one has to take into account the high degree of inter- and intra-speaker variation in the speech of terminal speakers and the possibility that certain features of their language are idiosyncrasies. Therefore, if historical data are available, one should incorporate them in the research, which we will do in Section \ref{sec:reviewing}. In addition, when studying speech island varieties that are threatened with disappearance because they are dominated by other local or immigrant languages, as is the case for Zeelandic Flemish in Brazil, one should ideally always study the data from the perspective of language contact as well. This will be demonstrated in Section \ref{reviewingourfindings}.

\section{Reviewing our findings from the perspective of historical linguistics}
\label{sec:reviewing}

In order to find out the extent to which rusty speaker data of a transplanted dialect (i.e. a speech island variety) alone suffice to reconstruct the so-called protolanguage of the original immigrants, we confined ourselves to a comparison of modern varieties of Zeelandic Flemish in Brazil and in the present-day motherland. By doing so, we actually ignored available historical data on the three linguistic features discussed. In what follows, we confront our findings with these historical data. This confrontation will demonstrate that reliance on rusty speaker data alone may in some cases (/l/-deletion, to be more specific) lead to the wrong conclusions and that historical data about the motherland variety are therefore essential in the evaluation of the data. 

\subsection{Historical data on /l/-deletion}
\label{sec:61historical}
\subsubsection{Winkler 1874}

As a matter of fact, there are several historical sources that reveal some information on the status of /l/-deletion in older stages of the Flemish dialects (i.e. West- and East Flemish dialects as spoken in Belgium, as well as Zeelandic Flemish dialects as spoken in the Netherlands). The oldest source is the so-called \textit{Dialecticon} of Winkler, published in 1874, in which the parable of the Prodigal Son is translated into various dialects of the Dutch-speaking area by speakers of those dialects.\footnote{ \url{https://www.dbnl.org/tekst/wink007alge02_01/} } Dialects that are included in the \textit{Dialecticon} and which are of interest to our study are those of the East Flemish villages of Maldegem and Kleit and of the West Zeelandic Flemish places Eede, Heille, Aardenburg and Cadzand. In none of these dialects is /l/-deletion found. More specifically, for all fragments together there are zero cases of /l/-deletion out of 36 possible cases. Of course, one can question the accuracy of the broad transcription used in the \textit{Dialecticon}. The transcription used by Winkler is not phonemic, but in normal alphabetic spelling, used in such a way that it represents the pronunciation of lexemes. This spelling is characterised by some limitations, though. A word like \textit{zonen} ‘sons’, for example, is pronounced as [z\~{ø}ːs] in the present-day dialect of Maldegem, but is represented as \textit{zeuns} in \citet{Winkler1874}. Thus, the vocalic phoneme /ø/ is written as \textit{eu}. We do not know for sure, however, whether the /n/ was pronounced or not and – depending on this – whether the vowel was nasalised or not. It might be the case that the alphabetic spelling was not suitable for indicating any degree of nasalisation in this word, although one could imagine that in the case of deletion of /n/ the lexeme would have been represented as \textit{zeu(n)s}, with brackets indicating a certain degree of deletion. Obviously, the same holds for cases of /l/-deletion: we assume that brackets would have been used to indicate (partial) deletion of /l/ or that /l/ would not have been written at all. However, in none of the 36 cases is \textit{l} written between brackets or absent. We thus conclude that /l/-deletion seems to have been absent in the relevant dialects around 1874.

\subsubsection{Corpus Dialectmateriaal Pieter Willems 1885}

Another 19\textsuperscript{th}-century source is the data that was collected by the dialectologist Pieter G. H. Willems (\textit{Corpus Dialectmateriaal Pieter Willems}).\footnote{ \href{https://bouwstoffen.kantl.be/CPWNL/CPWNL.xq?browse=s181 & act=browse\%23browse}{https://bouwstoffen.kantl.be/CPWNL/CPWNL.xq?browse=s181\& act=browse\#browse}} In 1885 and the following years Willems asked speakers of a large number of dialects to translate a list of more than 2000 lexemes into their dialects. Since Willems was particularly interested in phonological and morphological dialect phenomena, he added a document in which he was very explicit about the way the pronunciation of the lexemes had to be represented by the informants. Dialects that were included and that are of interest to this paper are the East Flemish dialect of Maldegem and the West Zeelandic Flemish dialects of Aardenburg, Zuidzande and IJzendijke. The list of lexemes contains 26 contexts for preconsonantal /l/-deletion and 33 contexts for prepausal /l/-deletion. However, we do not find indications that /l/ was deleted in any of these lexemes for any of these dialects around 1885.\footnote{ {Again, we might assume that /l/-deletion would have been indicated by either writing <}{\textit{l>} }{between brackets or not writing it at all.}}

In her attempts to reconstruct the dialect of Maldegem of 100 years earlier, \citet{Versieck1989} uses the \textit{Corpus Dialectmateriaal Pieter Willems}. She evaluates the accuracy of the transcription used by the Maldegem informant in this corpus and concludes that this informant represented certain speech sounds only approximately. She also evaluates the question to which extent phonological processes of the Maldegem dialect (such as /l/-deletion) are manifested in Willems’ material. With respect to /l/-deletion, she reaches the same conclusion as we do: /l/-deletion is not represented in the \textit{Corpus Dialectmateriaal Pieter Willems.} Versieck observes that there are only a few cases in which the lengthening of the vowel preceding /l/ seems to be indicated, for example in the lexeme \textit{volk} ‘people’ (represented as vo̅lk). Versieck also argues that /l/-deletion is not represented in the relevant extract of \citet{Winkler1874} either. \citet[175]{Versieck1989} concludes that “zowel Willems als Winkler suggereren […] dat (volledige) l-deletie in het toenmalige Maldegems niet voorkwam” (``Willems as well as Winkler suggest […] that (complete) l-deletion did not occur in the Maldegem dialect at the time''). Versieck then continues by discussing the data on the Maldegem dialect which can be found in the \textit{Reeks} \textit{Nederlandse Dialectatlassen} (1935), but this evaluation will be discussed below.

\subsubsection{Archief Jacobus van Ginneken 1910-1945} 

At the beginning of the 20\textsuperscript{th} century, a Dutch linguist called Jacobus van Ginneken maintained a correspondence with Piet Meertens, dialectologist and first director of the Meertens Institute (Amsterdam). One of the items from this correspondence is \figref{fig:schaffel:7}, which represents the vocalisation of /l/ in the lexemes \textit{half} ‘half’, \textit{kalk} ‘lime’, \textit{kalf} ‘calf’ and \textit{zalf} ‘ointment’. As can be seen from this map, vocalisation of /l/ in these lexemes occurred in an extensive area in West Flanders, a small area in the southwest of East Flanders and in a vast area in the southeast, which covers a part of present-day Vlaams-Brabant and large parts of Limburg (Belgium as well as the Netherlands). There are no indications on this map that vocalisation (or deletion) of /l/ was observed in the region of West Zeelandic Flanders or the neighbouring East Flemish places. Of course, this map focuses specifically on vocalisation of /l/ in these four lexemes and does not reveal anything about vocalisation of /l/ in other lexemes. This implies that we cannot rule out the possibility that /l/-deletion was present in these dialects at that time.


\begin{figure}[t]
\caption{Vocalisation of /l/ in the lexemes \textit{half} ‘half’, \textit{kalk} ‘lime’, \textit{kalf} ‘calf’ and \textit{zalf} ‘ointment’ (Source: \citeauthor{Ginneken01}. Map: Vocaliseering der L. Archief Meertens Instituut, Kaart 19951). Copyright resides with the Meertens Institute.}
\label{fig:schaffel:7}
\includegraphics[width=\textwidth]{figures/schaffel-img008.jpg}
\end{figure}


\subsubsection{Reeks Nederlandse Dialectatlassen 1935} 
A slightly more recent source is the \textit{Reeks Nederlandse Dialectatlassen} (abbreviated \textit{RND})\footnote{ \url{https://www.dialectzinnen.ugent.be/transcripties/}}, compiled by Edgar Blancquaert, of which the part about North East Flanders and Zeelandic Flanders was published in 1935. The \textit{RND} consists of 141 sentences that are translated into different dialects of the Dutch-speaking area. A great advantage of this source is that the sentences are transcribed in narrow phonetic transcription. We examined the \textit{RND}{}-transcriptions for the East Flemish dialects of Maldegem, Kleit and Middelburg and the West Zeelandic Flemish dialects of Aardenburg, Biervliet, Breskens, Cadzand, Groede, Hoofdplaat and Retranchement. A brief discussion of the results for each of these dialects is necessary to gain more insight into the possible development of /l/-deletion.

The dialect showing most instances of /l/-deletion around 1935 is that of Maldegem. In total, we found 17 cases of full deletion of /l/ (in prepausal and preconsonantal contexts), but without the overlong quality of the preceding vowel that is characteristic of the present-day Maldegem dialect (cf. \citealt{Taeldeman1966, Versieck1989}, and \citealt{Rys2007}). Only 1 case of /l/-deletion and compensatory lengthening of the preceding vowel was observed. This occurred in the lexeme \textit{lijnzaadmeel} ‘linseed meal’ (pronounced as [lyzəmeː]). There are 7 potential cases in which /l/ is not deleted, all of them involving preconsonantal /l/. Finally, the Maldegem dialect displays 2 cases of partial deletion of /l/, indicated by using brackets (i.e. (l)), both of which have preconsonantal contexts. Thus, it seems that /l/-deletion is a phonological process that did not occur in the dialect of Maldegem in the 19\textsuperscript{th} century (cf. \citealt{Winkler1874} and Corpus Materiaal Willems \citeyear{PieterWillems}), but did occur, though not categorically, in 1935. It seems to have been present with the characteristic overlong vowel in prepausal contexts first (cf. \textit{lijnzaadmeel}). Cases of partial deletion likely indicate that /l/-deletion was an ongoing phonological change at that time. This hypothesis is supported by the many cases of /l/-deletion without the compensatory lengthening of the vowel and the cases in which there is no deletion of /l/ at all. 

For Kleit, which nowadays has a dialect which closely resembles that of Maldegem \citep{Taeldeman1966}, we found 9 cases of /l/-deletion lacking the compensatory lengthening of the preceding vowel, and we found 1 case with compensatory lengthening, more specifically in the lexeme \textit{spel} ‘game’, which has a prepausal context. Further, we found 9 potential cases in which /l/ is not deleted, all of them involving preconsonantal /l/. Finally, 10 cases of partial deletion were observed. Thus, Kleit shows more or less the same development of /l/-deletion as Maldegem.

The dialect spoken in  Middelburg, another sub-municipality of Maldegem, barely showed any cases of /l/-deletion in 1935. There is 1 case of deleted /l/ without compensatory lengthening of the vowel (i.e. in \textit{karnemelk} ‘buttermilk’) and 1 case of partial deletion (in \textit{melkboer} ‘milkman’), but 26 potential cases in which /l/ is not deleted.

\largerpage
All of the West Zeelandic Flemish dialects of the places mentioned above\footnote{ {Aardenburg, Biervliet, Breskens, Cadzand, Groede, Hoofdplaat and Retranchement.}} more or less display similar results: potential cases of /l/-deletion in which /l/ is not deleted vary between 30 and 35. Cases of deleted /l/ without compensatory lengthening occur in each of these dialects, but almost always in the lexeme \textit{veel} ‘much, many’, in which /l/ is in a prepausal context.\footnote{ {However, the /l/-deletion in the lexeme} {\textit{veel} }{is not categorical, since we also found cases in which /l/ does not delete.}} Thus, the process of /l/-deletion in the West Zeelandic Flemish dialects seems to be lexically determined.\footnote{ {Recall that in the modern data, more specifically the} {\textit{Nederlandse Dialectenbank,}} {/l/-deletion in the West Zeelandic Flemish dialects was also restricted to prepausal contexts in a limited set of frequently occurring lexemes (viz.} {\textit{veel}} {‘much’ (also} {\textit{zoveel}} {‘this much’,} {\textit{hoeveel}} {‘how much’),} {\textit{wel}} {‘\textsc{part}’,} {\textit{zal}} {‘shall’,} {\textit{al}} {‘already’, and} {\textit{nogal}} {‘quite’).}} Occasionally, we do find some other cases: the dialect of Cadzand has 1 case of /l/-deletion without compensatory lengthening in the word \textit{kelder} ‘cellar’ and the dialect of Biervliet displays partial deletion of /l/ in \textit{kelder} and full deletion (without compensatory lengthening) in \textit{helft} ‘half’ and \textit{wel} ‘\textsc{part}’.

In her reconstruction of the Maldegem dialect of 100 years earlier, \citet{Versieck1989} also discusses the \textit{RND}{}-material. Like us, she observes that /l/-deletion with compensatory lengthening is present in one lexeme only (\textit{lijnzaadmeel}), that there are some cases in which /l/ is deleted but the preceding vowel is short and that there are also cases which do not have /l/-deletion (as opposed to the present-day Maldegem dialect). \citet[176]{Versieck1989} argues that it is unlikely that the process of /l/-deletion, which is so characteristic of the contemporary Maldegem dialect, would have taken place in the relatively short period between the publication of \citet{Winkler1874} and the \textit{Corpus Dialectmateriaal Pieter Willems} (1885) on the one hand and the \textit{RND} (1935) on the other hand.\footnote{As \citet[176]{Versieck1989} points out, the informants of the RND were born in the beginning of the 20\textsuperscript{th} century (more specifically 1904 and 1911), whereas the informants of \citet{Winkler1874} and of the Corpus Dialectmateriaal Pieter Willems in 1811 and 1855, respectively. She further argues that \citet{Taeldeman1966}, who interviewed informants born at the end of the 19}{\textsuperscript{th}} {century, did find /l/-deletion in the speech of his informants.} She therefore assumes that the informants of these latter two sources were not able to represent the phonetic detail necessary to indicate /l/-deletion. She concludes that because of these reasons it may be assumed that the Maldegem dialect of 1885 was already characterised by deletion of /l/. By way of ``evidence'', she mentions the example of the toponym \textit{Eelvelde,} which is pronounced as [eˈv{ӕ}ːdə] and not as [eːˈv{ӕ}ːdə]. \citet[176]{Versieck1989} argues that the fact that the vowel of the first syllable of this toponym is “no longer” lengthened indicates that speakers’ awareness of an underlying /l/ is “already completely blurred”, and that this indicates that the process of /l/-deletion must be “very old”. However, there is a fallacy in this argument, since /l/-deletion with compensatory lengthening of the preceding vowel typically affects stressed syllables. The first syllable of \textit{Eelvelde} is not stressed though, which – according to us – explains the absence of lengthening of the vowel.\footnote{ {Compare with the lexeme} {\textit{soldaat} }{‘soldier’, in which stress is on the second syllable. Although /l/ may be partially deleted in this word, the preceding vowel is not lengthened either.}}

All in all, we believe that \citet{Versieck1989} does not come up with convincing arguments for the presence of /l/-deletion in the Maldegem dialect around 1874/1885. Based on the \textit{RND}, however, we have reason to assume that /l/-deletion was an ongoing phonological process around 1935, which probably originated in the dialects of Maldegem and Kleit and which may have spread to neighbouring places (e.g. Middelburg, West Zeelandic Flanders) to some extent. Around 1935 the process did not apply categorically, as opposed to the present-day situation (cf. \citealt{Taeldeman1966, DeSchutterJong2005}, and \citealt{Rys2007}).

To conclude, historical sources from some years after the migration of Zeelandic farmers to Brazil (more specifically from 1874 and 1885) do not contain indications that /l/-deletion was present in the language of these West Zeelan\-dic Flemish migrants, nor in the surrounding East Flemish dialects of that time. Around 1935, /l/-deletion seems to be an ongoing process in some of these dialects and modern data show that this process is categorical in present-day Maldegem and Kleit and nearly categorical in the West Zeelandic Flemish village of Eede (and to a smaller extent in Middelburg). This may imply that /l/-deletion is a relatively ``new'' feature of these dialects and that its occurrence in the language of the Brazilian rusty speakers cannot be considered as an archaic feature of the so-called protolanguage. So, in this case, the reconstruction of the old Zeelandic Flemish language on the basis of rusty speaker data failed. Obviously, one wonders how the same feature of /l/-deletion and compensatory lengthening of the preceding vowel can be present in this speech island variety as well as in some present-day homeland varieties. An alternative explanation will be discussed in Section \ref{reviewingourfindings}.

\subsection{Historical data on subject doubling} 

For the case of subject doubling, there is historical evidence from only a few years after the time the Zeelandic farmers migrated to Brazil. The \textit{Dialecticon} of \citet{Winkler1874} contains a number of fragments which testify to the occurrence of subject doubling in various West Zeelandic Flemish dialects. We found 16 forms of subject doubling for the dialects of Eede/Heille, Aardenburg and Cadzand, such as example~\REF{ex:schaffel:25}:

\ea%25
    \label{ex:schaffel:25}

oeveel errebeiers van m'n voader ææn alles in de vulte, in \textbf{{}'k vergoane-'k-ik} van oenger\\
\glt ‘how many of my father’s workmen have everything in abundance, and I am starving.’ (Cadzand)\\

\z

Winkler comments that the ``double repetition'' of the personal pronoun of 1SG is typically \textit{“Vlaamsch”} ‘Flemish’, but is used in some Zeelandic Flemish dialects. He argues that it is particularly used to emphasise the personal pronoun and especially in confidential conversations which are characterised by strong emotions like anger or complaint.


Thus, in this case, the rusty speaker data could succesfully be used to reconstruct the West Zeelandic Flemish protolanguage. As we inferred from the language of our rusty speakers, and as confirmed by the historical data, subject doubling is an archaic Flemish feature that was present abundantly in the West Zeelandic Flemish dialects in the days of emigration, but has disappeared almost entirely from the contemporary Zeelandic Flemish dialects (see Section \ref{subsec:52subj}). 


\subsection{Historical data on the inflected polarity markers yes and no}

\citet{Paardekooper1993} discusses various sources that provide information on the occurrence of inflected \textit{yes-} and \textit{no-}particles in the dialects of the Dutch-speaking area. The oldest source is a map based on the \textit{Corpus Materiaal Willems} (1885) (see \figref{fig:schaffel:8}). This map demonstrates (by use of the filled circle) that inflection of \textit{ja} ‘yes’ as well as \textit{nee} ‘no’ and in combination with all person features (i.e. 1SG-3SG and 1PL-3PL) occurs in the West Zeelandic Flemish region the immigrants departed from as well as in the neighbouring West- and East Flemish dialects. Thus, the use of inflected polarity markers was omnipresent in the relevant dialects only a couple of decades after the migration to Espírito Santo. This makes it very likely that inflected polarity markers were present in the protolanguage of the Zeelandic migrants as well.


\begin{figure}[t]
\caption{Occurrence of inflected polarity markers \textit{yes} and \textit{no} in the dialects of the Dutch-speaking area based on the \textit{Corpus Materiaal Willems} 1885. Source: \citet{Paardekooper1993}}
	\label{fig:schaffel:8}
\includegraphics[width=\textwidth]{figures/schaffel-img009.png}
\end{figure}


Another, more recent map of the phenomenon is found in the \textit{Archief Jacobus van Ginneken} (\citeyear{Ginneken01}) (see \figref{fig:schaffel:9}). This map shows that at the beginning of the 20\textsuperscript{th} century, the inflection of the polarity marker \textit{ja} was still ubiquitous in the dialects of West Zeelandic Flanders as well as the surrounding West and East Flemish dialects (all delineated by the pink line in Figure~(\ref{fig:schaffel:9})).



\begin{figure}[t]
\caption{Occurrence of the inflected polarity marker \textit{ja} ‘yes’ in the dialects of the southern part of the Dutch-speaking area (Source: \citeauthor{Ginneken01}. Map: Ja-hij, Kaart20766). Copyright resides with the Meertens Institute.}
\label{fig:schaffel:9}
\includegraphics[width=\textwidth]{figures/schaffel-img010.jpg}
\end{figure}



As we made clear in Section \ref{sec:53-inflectedpol}, according to modern sources (i.e. \citealt{Barbiers2006, DeVogelaerDevos2008}) inflected \textit{yes-} and \textit{no}{}-particles have disappeared entirely from the West Zeelandic Flemish dialects. On the basis of the Brazilian rusty speaker data, we assumed that the phenomenon must have been present in the dialects of West Zeelandic Flanders in the days of migration. The two historical sources discussed in this section indeed testify to the presence of such inflected polarity markers in the protolanguage of the immigrants.

\section{Reviewing our findings from the perspective of language contact}
\label{reviewingourfindings}


In Section \ref{sec:61historical} we discussed the historical data on /l/-deletion and demonstrated that these did not match the conclusions we had drawn on the basis of the rusty speaker data. The historical sources which date back to a couple of decades after the migration of Zeelandic farmers to Brazil (1874 and 1885, to be more precise) do not contain indications that /l/-deletion was present in the dialects of these West Zeelandic Flemish migrants, nor in the surrounding East Flemish dialects of that time. This implies that its occurrence in the language of the rusty speakers in Espírito Santo cannot be considered as an archaic feature of the so-called protolanguage. Therefore, we must look for another explanation of this phenomenon in the language of the rusty speakers. We do this by approaching it from the perspective of language contact, a perspective that is traditionally emphasised within the field of language death studies \citep{Dressler1996, Dressler1972, Dressler1996, Dorian1977, Dorian1981}.

The two contact languages of Zeelandic Flemish in Espírito Santo are Brazilian Portuguese (i.e. the national language) and Pomeranian, which is another transplanted language. In \citet{SchaffelBremenkampPostma2017} it was demonstrated that the Zeelandic Flemish language was heavily influenced by both of these dominant languages (e.g. lexical borrowing, calquing, relative pronoun neutralisation, structural reduction). Therefore, we should also look at Brazilian Portuguese and Pomeranian in order to find out whether the /l/-deletion observed in the speech of the rusty speakers could possibly be related to one of these languages. As a matter of fact, vocalisation of /l/ in the coda is a common feature of Brazilian Portuguese. \citet[229]{BarbosaAlbano2004} observe that “[t]he archiphoneme /L/, which some generations ago used to be a velarised lateral approximant, is changing into a labial-velar approximant throughout the entire Brazilian territory, producing homophones such as \textit{mau} mɑʊ̯ ‘bad’ and \textit{mal} mɑʊ̯ ‘evil’.”
\largerpage[-1]
Thus, in Brazilian Portuguese, historical {[ɫ] (i.e. so-called ‘dark l’, /l/ in the syllable coda) has been vocalised, rendering [ʊ̯].}\footnote{ {Notice that the Brazilian Portuguese step of vocalisation is also present in some Flemish dialects from the relevant region, e.g. the lexeme} {\textit{balk} }{‘beam’ is realised as} {{[b}}{ow.kə}{]} {in Moerkerke (West Flanders) and Eeklo (East Flanders) (see} {\textit{GTRP-}}{database, \citealt{DeSchutterJong2005}).}}{ Assuming that Brazilian Portuguese /l/-vocalisation influences the rusty speakers’ pronunciation of Zeelandic Flemish words with coda /l/, the following alternation is plausible:}


\ea 
\label{ex:schaffel:26}
{\textit{bal-ke} ‘beam’ > /b}ɑl.kə{/ > [b}ɑ{ʊ̯}.kə{]}
\z 

{Brazilian Pomeranian (as opposed to European Pomeranian) is characterised by a productive process of monophthongisation \citep[56]{Postma2019}, in which /ɑu/ is realised as /ɑː/, rendering:}


\ea 
\label{ex:schaffel:27}
{\textit{blaum} $>$ \textit{blaam} ‘flower’ (example from \citealt[56]{Postma2019})}
\z 

If we assume that the Brazilian Portuguese process exemplified in \REF{ex:schaffel:26} is feeding the Brazilian Pomeranian process illustrated in \REF{ex:schaffel:27}, we get the following alternation:

\ea 
\label{ex:schaffel:28}
{\textit{bal-ke} ‘beam’ $>$ /b}ɑl.kə{/ > [b}ɑ{ʊ̯}.kə{] > [b}ɑ{ː}.kə{]}
\z 

This alternation can also be applied to the rusty speaker’s example given in \REF{ex:schaffel:4}, and repeated as \REF{ex:schaffel:29} below,

\ea 
\label{ex:schaffel:29}
Koeien zo kalvers [ˈkɑːvərs] \\
\glt ‘cows, you know, calves’

in which the alternation would be: 

\textit{kal-vers} ‘calves’ > /{k}ɑl.vərs/ > {[k}ɑ{ʊ̯}.vərs{] > [k}ɑ{ː}.vərs{]}

\z 

\hspace*{-5pt}It might then be the case that this combined process of vocalisation and monophthongisation, which in first instance affected lexemes containing the sequence /ɑl/, was by analogy extended to other words as well (e.g. example \REF{ex:schaffel:3} \textit{ielk} ‘each one’, example \REF{ex:schaffel:5} \textit{melken} ‘to milk’).

We can conclude that it is plausible to assume that the occurrences of forms with /l/-deletion in the language of the rusty speakers (see examples \REF{ex:schaffel:1}--\REF{ex:schaffel:6}) are the results of the Brazilian Portuguese process of /l/-vocalisation feeding the Brazilian Pomeranian process of monophthongisation, since the alternative hypothesis, stating that these forms are relics of the Zeelandic Flemish immigrants’ protolanguage, did not ``survive'' the confrontation with the available historical data. This outcome shows the importance of always including the perspective of language contact into the study of linguistic features of declining languages.

\section{Conclusion}

In this paper we have drawn attention to the condition that transplanted (cf. ``diaspora'') languages are not only subject to levelling, koineisation, or other processes of language simplification, but can – in other respects – also be rather conservative in that they can retain archaic features found in the motherland variety. Particularly, when a transplanted language variety is roofed by a language which is structurally very different, archaic features may be expected. We argued that such archaic features are found in the language of a number of so-called rusty speakers of Zeelandic Flemish in Espírito Santo (Brazil). We focused on three linguistic features which we believed to be archaic features of the protolanguage: (1) deletion of /l/ in codas and coda clusters with concomitant compensatory lengthening of the preceding vowel, (2) subject doubling in inversion contexts and (3) inflected polarity markers \textit{yes} and \textit{no}. By way of experiment, we confined ourselves in first instance to a comparison of modern varieties, ignoring any available historical data. In the case of features (2) and (3), our findings demonstrate the potential historical value of transplanted dialects in the reconstruction of the original immigrants’ language: a comparison of modern varieties shows that subject doubling as well as inflected \textit{yes-} and \textit{no-}particles are both phenomena that still occur in some Belgian Flemish dialects, but have disappeared from the contemporary variety as spoken in the Dutch province of Zeelandic Flanders. With respect to these cases, historical data bear witness to the presence of these features in the West Zeelandic Flemish dialect in the days of emigration. However, in the case of /l/-deletion, the available historical data do not support our hypothesis that it concerns a relic feature of the protolanguage. Thus, reliance on rusty speaker data alone leads to wrong conclusions in this case. Alternatively, we find an explanation by approaching the phenomenon from the perspective of language contact. We argued that the cases of /l/-deletion in the rusty speakers’ speech are probably the result of the Brazilian Portuguese process of /l/-vocalisation feeding the Brazilian Pomeranian process of monophthongisation. Summarizing, we may say that a multidisciplinary perspective is the most preferable approach in the study of declining languages.

\section*{Acknowledgements}
 Our special thanks go to Lea Busweiler for her help in the transcription of parts of the recordings. We are also grateful to Andrew Nevins and two anonymous reviewers for their comments on earlier versions of this paper.

\sloppy\printbibliography[heading=subbibliography,notkeyword=this]
\end{document} 
