\documentclass[output=paper,hidelinks]{langscibook}
\ChapterDOI{10.5281/zenodo.7446961}
\author{Philipp Dankel\affiliation{University of Basel} and Mario Soto Rodr\'{i}guez\affiliation{University of Freiburg} and Matt Coler\affiliation{University of Groningen -- Campus Fryslân} and Edwin Banegas-Flores\affiliation{Peruvian Ministry of Education}
}
\title{Trilingual modality: Towards an analysis of mood and modality in Aymara, Quechua and Castellano Andino as a joint systematic concept}
\abstract{This contribution examines how indigenous minority languages impact majority European ones, by considering the case of Quechua and Aymara, on the one side, and Castellano Andino (CA) on the other. We extend particular focus to how Aymara and Quechua have impacted CA's grammatical(ized) modality. We show that regional varieties of CA reflect Aymara and Quechua mood, even in the speech those who do not speak either indigenous language by illustrating the emerging strategies used to express the modal values in Aymara and Quechua	grammar on different structural levels. We also elaborate on how contact induced change arises from multiple impulses.
}

\IfFileExists{../localcommands.tex}{
 \addbibresource{../localbibliography.bib}
 \usepackage{langsci-optional}
\usepackage{langsci-gb4e}
\usepackage{langsci-lgr}

\usepackage{listings}
\lstset{basicstyle=\ttfamily,tabsize=2,breaklines=true}

%added by author
% \usepackage{tipa}
\usepackage{multirow}
\graphicspath{{figures/}}
\usepackage{langsci-branding}

 
\newcommand{\sent}{\enumsentence}
\newcommand{\sents}{\eenumsentence}
\let\citeasnoun\citet

\renewcommand{\lsCoverTitleFont}[1]{\sffamily\addfontfeatures{Scale=MatchUppercase}\fontsize{44pt}{16mm}\selectfont #1}
   
 %% hyphenation points for line breaks
%% Normally, automatic hyphenation in LaTeX is very good
%% If a word is mis-hyphenated, add it to this file
%%
%% add information to TeX file before \begin{document} with:
%% %% hyphenation points for line breaks
%% Normally, automatic hyphenation in LaTeX is very good
%% If a word is mis-hyphenated, add it to this file
%%
%% add information to TeX file before \begin{document} with:
%% %% hyphenation points for line breaks
%% Normally, automatic hyphenation in LaTeX is very good
%% If a word is mis-hyphenated, add it to this file
%%
%% add information to TeX file before \begin{document} with:
%% \include{localhyphenation}
\hyphenation{
affri-ca-te
affri-ca-tes
an-no-tated
com-ple-ments
com-po-si-tio-na-li-ty
non-com-po-si-tio-na-li-ty
Gon-zá-lez
out-side
Ri-chárd
se-man-tics
STREU-SLE
Tie-de-mann
}
\hyphenation{
affri-ca-te
affri-ca-tes
an-no-tated
com-ple-ments
com-po-si-tio-na-li-ty
non-com-po-si-tio-na-li-ty
Gon-zá-lez
out-side
Ri-chárd
se-man-tics
STREU-SLE
Tie-de-mann
}
\hyphenation{
affri-ca-te
affri-ca-tes
an-no-tated
com-ple-ments
com-po-si-tio-na-li-ty
non-com-po-si-tio-na-li-ty
Gon-zá-lez
out-side
Ri-chárd
se-man-tics
STREU-SLE
Tie-de-mann
} 
 \togglepaper[4]%%chapternumber
}{}

\shorttitlerunninghead{Trilingual modality: Towards an analysis of mood and modality}
\begin{document}
\shorttitlerunninghead{Trilingual modality: Towards an analysis of mood and modality}
\maketitle

\section{Introduction}
In this contribution we examine how long-term language contact with the Amerindian languages Aymara and Quechua have impacted mood and modality in the diaspora language Andean Spanish (“Castellano Andino”, hereafter: CA). Historically, all Latin American Spanish varieties developed in diasporic communities. In their recently published volume, \citet{drinka2021spanish} brought up the socio-historical relevance of the fact that all native Spanish speakers outside of the Iberian Peninsula ``\ldots are the linguistic heirs of the Spanish language diaspora that began in 1492 - with the expulsion of Spanish-speaking (Sephardic) Jews, and with the arrival of the Spanish language in the Americas'' \citep[550]{lipski2010spanish}. Currently, Spanish is the majority language in Latin America. However, as we will illustrate, varieties like CA can be considered diaspora languages from another perspective: CA retains the cultural identity of Andean speakers, although standard Spanish, still heavily influenced by the peninsular norm, is clearly the politically dominant code. 


%we may want to motivate why CA is a diaspora language
We show that CA reflects Aymara and Quechua mood and modality, even in the speech of those who do not speak either indigenous language. Examining mood and modality in CA is interesting because modal marking is an integral part of Andean discourse and is part of the grammar of both indigenous languages. We show how minoritized languages impact majority languages and illustrate how contact-induced change arises from multiple impulses, with particular focus on:

\begin{enumerate}
    \item speakers’ need to express semantic notions important for cultural and communicative routines,
    \item the structural and functional potentials of established grammatical constructions in CA, and
    \item decreased standardization.\footnote{This topic falls outside the scope of this chapter. We briefly address it in the conclusion.}
\end{enumerate}

Our use of the terms \textit{mood} and \textit{modality} follows that of Palmer (\citeyear[1]{RN26}), who defines modality as ``the status of the proposition that describes the event''. Our analysis encompasses both of Palmer’s (\citeyear{RN26}) distinctive forms that signal modality: mood, on the one hand, and, on the other, what may be referred to as the \textit{modal systems} of the languages. This seems appropriate, considering that modality is rarely coded in a neat, symmetrical system and that “in fact, it may be impossible to come up with a succinct characterization'' \citep[176]{RN25} of its notional domain. This account is crucial especially when performing an analysis of a dynamic contact situation where incipient grammaticalization processes are taken into account.

It is useful to provide a sketch of the mood and modality phenomena in the languages under investigation. We ask the readers to keep in mind that the descriptions below are, by necessity, brief and so are not meant to provide finer-grained analyses of phenomena in each language. To begin, let us consider Peninsular Spanish (hereafter: PS), broadly speaking, so that we may better appreciate how CA has come to differ as a result of sustained contact with Aymara and Quechua.

PS is typically understood as having four moods: indicative (unmarked form used for making assertions and statements), subjunctive (inflected for tense, occurs mostly in subordinate clauses to express nonfactual states like evaluations, possibilities, necessities, emotional states, and intentions), conditional\footnote{In contemporary PS grammars, conditional verb constructions are part of the past subjunctive paradigm. However, for comparative purposes, and as we focus on mood and modality, we consider them as conditional mood.} (used to express epistemic modality) and the imperative \citep{RN24}. Additionally, some verbal tenses have modal uses (e.g. the future tense can be used to express conjecture as in \textit{Serán las tres de la tarde} ‘It is probably three in the afternoon’) and there is a set of modal periphrasis that are systematically used to express modal distinctions (e.g. \textit{tener que} `holding to’, \textit{deber (de)} `owing to’, \textit{poder} ‘can’, \textit{haber de} ‘having to’ all of which are inflected for person and followed by the infinitive \citep[ch. 25, ch. 28.6]{RN56}.

%Standard Spanish is described as a system with a tripartite mood distinction as a grammatical paradigm: an unmarked indicative, which has a world-describing function and is used for assertive sentences, statements and highly certain states of affairs; a subjunctive, which is inflected for tense, mostly in subordinate clauses and expresses nonfactual states of affair like evaluations, possibilities, necessities, emotions and intention, often triggered by lexical selection in the matrix clause (like \textit{querer} ‘wish’, pedir ‘ask’, lamentar ‘regret’ and others); and an imperative for direct commands, prohibitions, and requests, which is restricted to matrix clauses. (Bosque 2012)

%Furthermore, some verbal tenses have modal uses. The future tense can be used to express conjecture (\textit{Serán las tres de la tarde}. ‘It is probably three o’clock.’) and the conditional can be used to express epistemic modality (\textit{Serían las tres de la tarde}. `It was probably three o’clock.'). Also, there is a set of modal periphrasis like \textit{tener que} `holding to’, \textit{deber (de)} `owing to’, \textit{poder} ‘can’, \textit{haber de} ‘having to’ + \textsc{inf} that are systematically used to express fine grained modal distinctions (RAE-ASALE 2009: ch. 25, ch. 28.6)	
The Aymara suffixes which are classified as part of mood and modality fall into three categories \citep{coler2014grammar}: 
\begin{enumerate}
    \item he past and present counterfactual paradigm (members of which express event modality, i.e. deontic and dynamic modality),
    \item the evidential suffixes which express evidential modality, and
    \item the imperative paradigm.
\end{enumerate}
We do not include those suffixes whose primary grammatical function is not modal. For example, the nominalizer-\textit{ña} can express deontic modality, but since it is not part of the inflectional paradigms for mood and modality, it is not discussed here.

The mood system for Quechua (specifically Southern Quechua, the variety investigated here) is similar to that of Aymara. The three categories of suffixes which are classified as part of mood and modality include: 1) The tense-aspect system with includes suffixes that express experienced and non-experienced events, 2) the evidenital suffixes which express evidential-epistemic modality, and 3) the potential suffix which expresses potential and counterfactual events. 

This text is structured as follows: In the following sections we outline previous research (Section \ref{sec:previousresearch}), our theoretical framework (Section \ref{sec:theoreticalframework}), and describe the corpora we used (Section \ref{sec:corpora}). Then, we provide analyses of tense/evidentiality (Section \ref{analysis1:tense-evid}), hearsay and quotatives (Section \ref{analysis2:hearsay}), inferential/conjecturals (Section \ref{analysis3:inferentials}) and the potential and counterfactual (Section \ref{analysis4:potentialCF}). There is also a final analysis dedicated to outstanding issues (Section \ref{sec:outstandingissues}) in which we outline other issues worthy of deeper reflection. The following section is dedicated to a discussion. In Section \ref{conclusion} we conclude and suggest future research.

\section{Previous research}
\label{sec:previousresearch}
%There are three languages investigated in this work, one diaspora language and two indigenous languages. To begin with the former, Castellano Andino (hereafter: CA) refers to the Spanish varieties that coexist(ed) with local indigenous languages, including Aymara and Quechua. CA is spoken in much of Ecuador, Peru and Bolivia, but also, to a lesser extent, with parts of southern Colombia, northern Argentina and northern Chile (Escobar 2011: 323f., Rivarola 2000: 13f.). %The three main CA varieties are attested in Ecuador, Peru and Bolivia, because these countries are home to most of the speakers of the above mentioned contact languages. 

%Turning our attention now to the indigenous languages, the total number of speakers of Quechua and Aymara in Peru is around 5.5 million (Escobar 2011), many of whom are bilingual in CA.



%The total percentage of indigenous population in Peru, at 35-45\%, is significantly higher than in Ecuador. According to Sälzer (2013) this is the third highest value in Latin America, after Bolivia and Guatemala. For Bolivia, officially, the 2001 census speaks of 2.3 million speakers of Quechua, which corresponds to 20\% of the total population. 
%Of these two, Quechua has the most speakers, most of whom are in Peru, Bolivia and Ecuador. % Ecuador has somewhere between 1.5 million (Escobar 2011: 324) and 2 million speakers, if one includes the variety known as Quichua or Kichwa (Sälzer 2013). Altogether, the percentage of indigenous population in Ecuador is 20-25\% (as of 2001, value taken from Sälzer 2013). In the \textit{resultados censales} of 2017, 13.6\% of the Peruvian population are Quechua speakers (3.8 million)\footnote{Official statistics can be found at \url{http://webinei.inei.gob.pe/anda_inei/index.php/catalog/CENSOS}}. 

% Nearly 3 million speakers in Bolivia speak two or more languages. However, the number of speakers with passive Quechua skills is likely to be significantly higher, as indicated by the surveys of Sichra (1986: 126)\footnote{Official statistics can be found here: \url{https://www.ine.gob.bo/index.php/censos-y-banco-de-datos/censos/}}. 60\% of those over the age of 15 assign themselves to an indigenous group, the majority of them to Quechua. Quechua is the dominant indigenous variety in everyday life, especially in the departamento of Cochabamba (49.9\% of the speakers who use Quechua in everyday life live in Cochabamba; Martínez 1996: 17). At least half of the speakers in Cochabamba are considered bilingual (Pfänder 2004). A significantly smaller number of speakers of the indigenous languages of the Andean region are found in Argentina (about 1 million), southern Colombia (23,000) and in northwestern Chile (9,000) (Escobar 2011: 324). 

%The number of Aymara speakers, and also their areal distribution is a lot smaller. However, they still show a quite significant speech community with 2.2-2.5 million speakers in total (depending on the source) located around the Titicaca lake and south thereof, also being the dominant indigenious language in the urban center La Paz/El Alto (Bolivia). According to the data from Sichra (2009: 517), they are mainly situated in Bolivia (2 million), Peru (434 000) and to a minor degree in Chile (48500 and Argentina (4100). According to the 2017 Peruvian census, 450,010 respondents (1.4% of the Peruvian population) spoke Aymara as a native language (Instituto Nacional de Estadística e Informática 2017). %Map 1 presents census data to show where Aymaran languages are spoken in Peru, Bolivia, and Chile today (Jaqaru, the sister language of Aymara spoken in Central Peru and the sole surviving member of the Central Aymaran branch, is also indicated on the map). %% Ask Nick for more recent data





%\subsection{Spanish in the Andes and its contact with Aymara and Quechua: A short overview of previous research}

Quechuan and Aymaran languages have been studied by many researchers. Key work for Quechua includes 
\citet{adelaar2004, albo1970, cerron1987, RN61, mannheim1991, taylor2006}, and \citet{urton1997social}. 
For Aymara, language descriptions are provided in \citet{briggs1976dialectal, cerron2000origen, coler2014grammar, hardman1973aymara}, and \citet{hardman2001aymara}. 
% See also \citeauthor{cerron2008quechumara}’s (\citeyear{cerron2008quechumara}) overview of similarities between Quechuan and Aymaran.

One theme of research on Quechua and Aymara pertains to the historically-founded structural convergence between the two languages, which were in close relationship over centuries. While a thorough overview is outside the scope of this chapter, the work of \citet{adelaar2012modeling} and \citet{cerron2008quechumara} analyzes the convergence and parallel structures of the two languages. Adelaar with Muysken (\citeyear{adelaar2004}) and \citet{crevels2009lenguas} provide an excellent overview of the indigenous languages in the Andean area. 

CA has also received some scholarly attention, especially in the last decades. Readers are referred to the overview of \citet{RN54} and the studies of \citet{pfander2009gramatica, merma2007contacto, haboud2008mujeres}, and \citet{mendoza1991castellano}. All these authors attest to systematic changes and innovations in CA grammatical structure, attributed to the influence of indigenous languages. Previous work on CA has described the effect of Aymara (e.g. \cite{hardman1982mutual, quartararo2017evidencialidad}) and Quechua (e.g. \cite{RN52, quartararo2017evidencialidad}) on different grammatical levels. However, little work has been dedicated to mood and modality apart from reportative evidentiality and the tense-evidentiality interface.
For Aymara, the tense-evidentiality interface has been described by e.g. \citet{coler2015aymara} and \citet{MartinezVera}. For Quechua, \citet{faller2002semantics, howard2018}, and \citet{manley2015} can be named, among others. The systematic influence on the use of CA tense has, to the best of our knowledge, first been described for Aymara-dominant contact varieties \citep{schumacher1980, hardman1982mutual, stratford1991} and has also been described by various authors for Quechua contact zones (e.g. \cite{andrade2005, andrade2020, dankel2012convergencias, escobar1997, escobarcrespo2021, garcia2015valores, kleeocampo1995, palacios2018introduccion, pfander2013evidencialidad, sanchez2004}). Most of the studies on Aymara and Quechua also analyse reportative evidentiality. For Quechua, \citet{kalt2021} in a recent study specifically focuses on this category. Reportative evidentiality in different CA speaking zones has been studied in \citet{andrade2007, babel2009, chang2018, dankel2015strategien, feke2004quechua, olbertz2005dizque}, among others. The conceptions, categorizations, and associations proposed by these authors slightly diverge and can still be considered open for discussion in certain aspects. For example, a recent extensive overview of the different approaches to the evidential tenses is provided by \citet[79--93]{andrade2020}. Both the research on hearsay/reportative evidentiality as well as the tense-evidentiality interface are as fascinating as they are complex. However, as this topic is outside the scope of our contribution we do not detail it further. Our focus is on a trilingual comparison of the domain of mood and modality based on empirical spoken data, an approach that is unique in this configuration. 

\section{Theoretical framework}
\label{sec:theoreticalframework}
As the convergence of mood and modality in Aymara, Quechua and CA is an ongoing process, we do not measure contact induced change only in terms of direct lexical and structural borrowing. Instead, we understand it as a process of indirect and covert adjustment on different formal and functional levels based on the existing potentialites of the (sub-)systems of the respective languages. This perspective is based on \citeauthor{RN59}’s (\citeyear{RN59}) code-copying framework, and has found support in many studies on language contact over the last decade. For example, \citet{babel2014doing} show in a case study on the use of the past perfect (\textit{había} + past participle) in CA, how “[t]he effects of language contact are the accumulation of communicative routines or habits, which speakers play on as they engage in creative language use” \citep[254]{babel2014doing}. 

\citeauthor{RN59}’s (\citeyear{RN59}) framework models how parts of languages can be combined or copied selectively. He distinguishes four types of copies:

\begin{enumerate}
    \item Combinatorial: Typical cases of combinatorial copies are loan translations or syntactic calques in which a structure or pattern of the target language is partially rearranged to fit into a scheme from the model language. For example, whereas (S)OV word order would be considered highly marked in PS grammar, speakers of Aymara- or Quechua-influenced CA frequently make use of it, particularly in emphatic contexts. This occurs with such frequency that its usage is loosing its marked status, reflecting the unmarked SOV pattern of Aymara and Quechua \citep[102--108]{pfander2009gramatica}.
    \item Material: Material copies include loanwords (which are frequent in Aymara, Quechua and CA) and phonological or morphological copies, though grammatical morphemes are only very rarely copied (see \citealt[242]{pfander2009gramatica}).
    \item Semantic: Semantic copies overlay the notional content of one language with the semantics of the other, as when speakers of CA, adapting the function of Quechua subordination suffixes, use the CA gerund construction mostly for adverbial subordination (\cite[139--147]{pfander2009gramatica}, \citealt{RN53}).
    \item Frequency based: Frequency copies adopt the usage of a feature from the model code in the corresponding feature of the target code: a well-known example is the higher percentage of explicit subject pronouns in the Spanish spoken in the US because of language contact with English \citep{RN58}. \citet[62]{RN59} stresses the fact that copies cannot, by definition, be identical to their models. Most typically, the semantic functions of copies have not reached the same stage of grammaticalization as their models and their use is often pragmatically determined (\citeyear[70]{RN59}).
\end{enumerate}

This framework helps explain how Aymara and Quechua mood and modality find their way into CA in notional transfer as a result of intense cultural and language contact. That is, it explains how languages can influence each other by borrowing conceptual notions (but not necessarily forms) that must be expressed in a speech community. \citeauthor{RN59}’s (\citeyear[62]{RN59}) framework is based in an extensive empirical-observational work on a well-established contact variety. It also receives support by comparable observations from other sub-fields. For example, consider the research of \citeauthor{RN60}’s (\citeyear{RN60}) on L2 acquisition processes. They find that the speakers’ assumptions and perceptions about the L2 (in other words, their beliefs about the congruence between languages) fuel language transfer, also between typologically distant languages. From a more cognitive viewpoint, by focusing first on the dynamics in language processing mechanisms, \citet{slobin2016thinking} explains that speakers who switch between languages frequently conceptualize the world in one language while speaking in another. This leads to contact-induced changes when speakers accommodate their “thinking for speaking” from the source language to the target language as an adjustment of language processing efficiency. 

Accordingly, speakers’ communicative routines, which work on a cognitive level but have been shaped culturally, affect their understanding of how target languages work and give rise to linguistic outcomes in such a way that they are contextually and socially adequate. Speakers creatively operationalize the potential of the available linguistic forms to convey their semantic and pragmatic needs in context-dependent ways. Here we show that mood distinctions and other systemic possibilities to express modality in CA reflect the creative operationalization (that is, reinterpretation) of CA linguistic forms to convey a communicative routine that is fully grammaticalized in Aymara and Quechua.

\section{Corpora}
\label{sec:corpora}
This section describes the source of the data for the analyses in Section \ref{sec:analyses}. Aymara data are from a variety of the language spoken in Southern Peru. All language data come from \citet{coler2014grammar} and directly from Edwin Banegas-Flores, a co-author on this contribution and a native speaker both of this Aymara variety and of CA.

The Quechua data alongside CA spoken in Quechua regions comes from the Corpus Quechua Español (CQE) gathered by Soto Rodr\'{i}guez, a co-author on this publication. The CQE comprises approximately 9 hours of recordings made in different locations in the Cochabamba region of Bolivia in 2009 and 2011. It contains material from daily conversations, interviews, and radio and television broadcasts in both languages. Note that Soto Rodr\'{i}guez himself is a native speaker both of Quechua and CA. 

Additional CA data come from the Corpus Heroina gathered by Dankel and collegues and published in \citet{dankelpagel2012}. For additional information on this data, see also \citet{dankel2015strategien}. Other sources are cited when relevant.

\section{Analyses}
\label{sec:analyses}
This section provides an analysis of mood and modality in Aymara, Quechua and CA from our empirical data. Since mood and modality are complex, and as we want to illustrate phenomena at different grammatical levels across different parts of the language systems, we have divided the analysis as follows: Section \ref{analysis1:tense-evid} tense/evidentiality, Section \ref{analysis2:hearsay} hearsay and quotatives, Section \ref{analysis3:inferentials} inferentials/conjecturals, Section \ref{analysis4:potentialCF} potential and counterfactuals, and Section \ref{sec:outstandingissues} outstanding issues, which deals with other observations that do not fit neatly into the preceding sections.

Each of these sections includes a list of elicited examples that shows which markers in the three languages express the respective modal function. In the Aymara and Quechua examples in this section, we always provide both a CA and PS translation of the indigenous language. The CA translations were provided by the indigenous Peruvian native-speaking authors. The PS translations were provided by asking Spain-born native speakers to translate from the English translations into PS. 

\subsection{Tense/evidentiality}
\label{analysis1:tense-evid}
Contrary to PS, where the tense paradigm has precise temporal functions, grammatical tense in Aymara and Quechua is more modal-evidential than temporal. %Thus, for example, the division between the two past tenses, described below, is evidential, not temporal. 
Both indigenous languages have four mood-tense distinctions. While their scopes differ somewhat, they can be described in roughly the same terms: 

\begin{enumerate}
    \item Unmarked: A non-future (present and immediate past) directly experienced by the speaker. This tense is not glossed in Quechua. In Aymara it is glossed as “simple tense” (\textsc{sim})
    \item Experienced past: A past directly experienced by the speaker. We gloss members of this paradigm as “experienced past” (\textsc{exp.past})
    \item Non-experienced past: A past not experienced by the speaker. Note that even events which occurred in the recent past, but for which the speaker was e.g. intoxicated or sleepwalking are inflected into this paradigm -- thus demonstrating that the tense distinction is evidential. We gloss members of this paradigm as “non-experienced past” (\textsc{nexp.past})
    \item Future: A future not experienced by the speaker, glossed as \textsc{fut}.
\end{enumerate}.

We will show that the three Aymara and Quechua non-future tenses map to CA perfect and pluperfect temporal morphology in a way that indicates contact-induced change.\footnote{Note that Aymara and Quechua non-future tenses map to other CA tenses, not just the perfect and pluperfect, as CA of course makes use of other tenses. This mapping falls outside the scope of this paper.} This change can be considered a selective semantic copy in \citeauthor{RN59}'s (\citeyear{RN59}) framework and therefore is barely noticeable on the structural surface but becomes evident in language use. In examples~\REF{ex:exp.pastAym}--\REF{ex:nexp.CA} we provide constructed examples to illustrate parallel usage.

\paragraph*{Personal knowledge:}

\ea \label{ex:exp.pastAym}
 {Experienced past in Aymara:} \\
\gll uma-na \\
drink-\textsc{3.subj.exp.past} \\ 
\glt ‘he drank’ (the speaker witnessed it)
\z

\ea \label{ex:exp.pastQ}
 {Experienced past in Quechua:} \\
\gll uyja-rqa \\
drink-\textsc{3.subj.exp.past} \\ 
\glt ‘he drank’ (the speaker witnessed it)
\z

\ea \label{ex:exp.CA}
 {Perfect tense in Castellano Andino:} \\
\gll ha tomado \\
have.\textsc{3} drink.\textsc{ptcp} \\ 
\glt ‘he drank’ (the speaker witnessed it)
\z

\paragraph*{Non-personal knowledge:}

\ea \label{ex:nexp.Ay}
 {Non-experienced past in Aymara:} \\
\gll uma-tayna \\
drink-\textsc{3.subj.nexp.past}\\
\glt ‘he drank’ (the speaker did not witness it)
\z

\ea \label{ex:nexp.Qu}
 {Non-experienced past in Quechua:} \\
\gll uyja-sqa\\ 
drink-\textsc{3.subj.nexp.past}\\
\glt ‘he drank’ (the speaker did not witness it)
\z

\ea \label{ex:nexp.CA}
 {Pluperfect in Castellano Andino:} \\
\gll hab\'{i}a tomado\\ have\textsc{.3.past.ipfv} drink.\textsc{ptcp}\\
\glt ‘he drank’ (the speaker did not witness it)
\z



\subsubsection{Aymara tense/evidentiality}
 Beginning with Aymara, personal knowledge can be expressed with the simple tense (as in \REF{ex:simple1}) or the experienced past tense (as in \REF{ex:prox1}). In both of these tenses, the speaker asserts first-hand direct knowledge over the event.

\ea \label{ex:simple1}
\gll kayu-w usu-\textbf{tu}-xa \\
foot-\textsc{acc.decl} hurt-\textsc{3.subj.1.obj.sim-top} \\ 
\glt ‘My foot hurts me.’ \citep[411]{coler2014grammar} \\
CA:~`Me duele mi pie.'\\
PS:~`Me duele el pie.'
\z

\ea \label{ex:prox1}
\gll mama.la-x uñ.ja-\textbf{n} \textit{kartir}-∅ pur-t'a-n-ir-∅-x \\
mom-\textsc{1.poss} see-\textsc{3.subj.3.obj.exp.past} road-\textsc{acc} come-\textsc{m-h-ag-acc-top} \\ 
\glt ‘My mom saw the arrival of the road.’ \citep[413]{coler2014grammar} \\ 
CA: `Mi mamá vio llegar la carretera.'\\
PS: `Mi madre vivió (vio) la llegada de la carretera.'
\z



The non-experienced past, in contrast, not only refers to a remote and mythological past, but also a past which was not witnessed by the speaker. That is, an event can be marked in the non-experienced past even if it occurred during the speaker’s lifetime if they lack personal knowledge of it. This is evident in \REF{ex:distal1} which occurred moments before the speech-act. However, as the speaker did not witness the event, it does not receive either the simple or the experienced personal knowledge tenses. Likewise, events which were witnessed by the speaker, but which they forgot about (owing to amnesia, intoxication, senility, or similar) are in the non-experienced past.

\ea \label{ex:distal1}
\gll jupa-w qullq-∅ chura-\textbf{taytam}-x \\
he-\textsc{decl} money-\textsc{acc} give-\textsc{3.subj.2obj.nexp.past-top} \\ 
\glt ‘He gave you the money.’ \citep[421]{coler2014grammar} \\ 
CA: `Él te \textbf{hab\'{i}a} dado la plata.'\\
PS: `(Él) Te dio el dinero.'
\z

The non-experienced past is also used to express mirative meanings (e.g. \cite[83]{MartinezVera}). Here, the term “mirativity” is taken as the “linguistic marking of an utterance as conveying information which is new or unexpected to the speaker” \citep{delancey2001mirative}. Compare \REF{ex:Aym-mirative} with \REF{ex:Aym-sim}, the former is in the non-experienced past and the latter is in the experienced past.



\ea \label{ex:Aym-mirative}
\gll uñja-\textbf{taysta} \\
see-\textsc{2.subj.1.obj.nexp.past} \\ 
\glt ‘You saw me! (without my knowledge)’ \citep[422]{coler2014grammar}\\
CA: `¡Me \textbf{habías} visto!'\\
PS: `¡(Tú) me viste! (no me di cuenta)' 
\z

\ea \label{ex:Aym-sim}
\gll uñj-\textbf{ista} \\
see-\textsc{2.subj.1.obj.exp.past} \\ 
\glt ‘You saw me.’\\
CA: `Me \textbf{has} visto.'\\
PS: `Me viste.'
\z

\subsubsection{Quechua tense/evidentiality}

This distinction in Quechua is largely parallel to that described for Aymara. Examples \REF{ex:Qnonex} and \REF{ex:Qnonex2} show the distinction between experienced and non-experien\-ced past. The speaker in these two examples is speaking about a break-in that happened in her village which she did not personally witness. Although she has personal knowledge of her own whereabouts the moment the break-in happened, the knowledge of the victim crying for help is indirect in that the speaker did not witness it herself. (In this, and all following examples, CA loanwords are italicized.)

\ea \label{ex:Qnonex}
\gll \textit{miércoles} \textit{tarde} nuqa ma ka-\textbf{rqa}-ni-chu chay-pi \\
Wednesday evening I no be-\textsc{exp.past-1sg-neg} this-\textsc{loc} \\
\glt ‘I was not there on Wednesday evening.’\\
CA: `El miércoles en la tarde yo no estuve ahí.' (CQE)\\
PS: `No estuve allí el miércoles por la tarde.'
\z

\ea \label{ex:Qnonex2}
\gll eh ajina \textit{encapuchado} ka-xti-n auxilio maña-\textbf{sqa} \\
eh like.this hooded be-\textsc{sr-3sg} help ask-\textsc{nexp.past} \\ 
\glt ‘When the hooded guys were there, she asked for help (I did not witness it).’ \citep[96]{dankel2012convergencias} \\
CA: `Cuando ha visto que estaban encapuchados hab\'{i}a pedido auxilio.' (CQE)\\ 
PS: Cuando los hombres encapuchados llegaron, ella pidió ayuda (yo no lo presencié).

\z

As shown in \REF{ex:Q-find}, personal knowledge marking is especially relevant in situations of social or even legal accountability. In these cases its function could be described as providing testimonial specification about a state of affairs, which explains its use with negation markers. The example starts with the telling of a piece of common, undisputed knowledge with the unmarked non-future tense: they found someone dead.

\ea \label{ex:Q-find}
\gll wañu-sqa-lla-ta taripa-nku. Madre Obrera chay chay-s-itu-pi\ldots \\
die-\textsc{part-lim-acc} find-\textsc{3sg} Madre Obrera there there-\textsc{eu-dim-loc} \\ 
\glt ‘They found him already dead. Near the Madre Obrera (hospital)\ldots’ \citep[165]{soto2002interferencia}\\
CA: `Muerto lo han encontrado. Cerca del Madre Obrera\ldots' \\
PS: `(Se) lo encontraron ya muerto. Cerca del Hospital Madre Obrera\ldots'
\z

The speaker, who knows the deceased, then continues with the reporting of the circumstances where she carefully marks the details in a way she cannot be held accountable for the state of affairs.

\ea \label{ex:Q-find2}
\gll qayna \textit{tarde} lluxsi-pu-\textbf{sqa}, ari. Calle-pi-chus toma-\textbf{rqa}, mana nuqa yacha-\textbf{rqa}-ni-chu. \\
yesterday afternoon go.out-\textsc{refl-nexp.past} \textsc{mod.int} street-\textsc{loc-conj} drink-\textsc{exp.past} no I know-\textsc{exp.past-1-neg} \\ 
\glt ‘He went out yesterday afternoon (I did not witness it). Maybe he drank in the street, I didn't know it.’ \citep[165]{soto2002interferencia} \\
CA: `Había salido ayer en la tarde, en la calle ha debido tomar, yo no sabía.'\\
PS: `Salió ayer por la tarde (no lo ví). Bebía, quizás en la calle, yo no lo sabía.'
\z

With the non-experienced past suffix she signals that she did not know of his absence. Interestingly, she marks her supposition that he drank in the street and her assurance that she did not know that he drank in the street with an experienced past suffix. In both cases, this emphasizes her accountability. Such examples demonstrate how these suffixes are more modal than temporal. 

As in Aymara, the Quechua non-experienced past can express the mirative.\footnote{While the Quechua data in this contribution is from Southern Quechua, note that the Central Peruvian variety of Tarma Quechua has a grammatical paradigm that exclusively conveys mirative meaning \citep{RN72}.} 

\ea
\gll Erma qhawa-yku-sa-\textbf{sqa} kay lluqallu \\
Erma.\textsc{acc} look-\textsc{dir-prog-nexp.past} this boy \\ \glt ‘Oh! This guy is interested in Erma!' \\
CA: `A la Erma se había estado mirando este chico.' (CQE)\\
PS: `¡Oh! ¡Este chico está interesado en Erma!'
\z

\subsubsection{Castellano Andino tense/evidentiality}

We show that speakers of CA map the modal-evidential distinction to the Spanish perfect and pluperfect temporal morphology. The speaker in example \REF{ex:CAburgalar1} recounts a break-in at her house. As she did not personally witness the burglars entering into her house, she uses the pluperfect form (\textit{habían entrado}).\footnote{Interestingly, also in this CA example, speaker accountability plays a role for why the non-experienced past marker is chosen, although there might be some directly experienced evidence of how and where the burglars entered. By distancing herself from knowing anything about how the burglars could enter into the house, the speaker cannot be held accountable for any overlooked security measures.} However, as she did experience directly that her audio system was missing when she returned, for this part of her telling she uses the perfect form (\textit{han llevado}).
\largerpage

\ea \label{ex:CAburgalar1}
\gll Los ladrones \textbf{habían} \textbf{entrado} y se \textbf{han} \textbf{llevado} equipos de sonido, ?`no? \\
the burglars had entered and they have carried equipments of sound no \\ 
\glt `The burglars entered (I did not witness it) and they took the audio system with them, right?' \citep{dankelpagel2012}\\
PS: `Los ladrones entraron (no lo presencié) y se llevaron el sistema de música, ¿sabes?'
\z

In \REF{ex:hemos} the use of perfect tense and pluperfect tense forms according to the experience of the speaker can be observed. Furthermore, the pluperfect also gets a mirative reading through its sequential position in this context. The speaker tells about how he learned by arriving at his destination (direct experience marked with the perfect tense form), that the house he was looking for was not where he thought it was (newly revealed knowledge, not formerly experienced, marked with the pluperfect tense form).

\ea \label{ex:hemos}
\gll Y fuimos, hemos llegado. No había sido su casa en el centro, sino había sido más alejadito de Quillacollo, como es, como en aquí, como Achocalla así, un pueblito así alejadito. \\
and we.went we.have arrived no had been his house in the center but had been more far.away from Quillacollo as is as in here as Achocalla thus, a little.town like far.away 
 \\ \glt ‘And we went, we arrived. Her house was not in the town, it was quite a bit further away from Quillacollo, like is, like here, like Achocalla, a small town, quite a bit outside.’ \citep[206]{RN20}\\
 PS: `Y fuimos, llegamos. Su casa no estaba en la ciudad (centro), estaba bastante lejos de Quillacollo, como es, como aquí, como Achocalla, una ciudad pequeña, bastante a las afueras.'
\z

\subsection{Hearsay and quotatives}
\label{analysis2:hearsay}
Both in Aymara and Quechua, hearsay marking and quotatives are a grammaticalized part of a culturally-relevant evidential subsystem. Both languages have no grammatical mechanism to express indirect speech and show certain parallelism in the grammatical marking of evidentiality. In PS, on the other hand, hearsay and quotatives are expressed only selectively by lexical and discourse-pragmatic ways (e.g. \textit{se dice que} ‘it is said that’). There are no comparable systemically used markers. Nevertheless, even with reduced possibilities for the direct transfer of grammatical morphology due to typological incompatibility, CA shows contact induced incipient grammaticalization of hearsay and quotative markers as a result of a step-by-step development based on the potential of constructions with say-verbs that express these two functions in the appropriate contexts: \textit{dice} /\textit{dice que}/\textit{dizque} are used as reportative particles to express hearsay, whereas \textit{diciendo} `saying' is used for as a quotative, often also postponed to the quoted proposition. Again we observe a selective copy of a semantic notion crucial in Aymara and Quechua. There is also evidence of the reverse influence of CA \textit{dice} / \textit{dice que} and \textit{diciendo} on some varieties of Aymara and Quechua, where new reportative strategies emerge modeled on CA -- a combinatorial copy, so to speak, in Johanson's (\citeyear{RN59}) framework. This can be seen as a strong indication that speakers do not distinguish between different systems for each language, but have developed diverse linguistic resources for one system. This is summarized in the constructed examples provided in \REF{ex:hearsay.Ay}--\REF{ex:quotative.CA}.


\paragraph*{Hearsay:}

\ea \label{ex:hearsay.Ay}
 {Verb `to say' (subordinated) in Aymara:} \\
\gll uma-nt-i-w sa.s \\
drink-\textsc{iw-3.subj.top} say.\textsc{subr}\\
\glt ‘She drinks, it is said.’ 
\z

\ea \label{ex:hearsay.Qu}
 {Dedicated suffix or verb `to say' (inflected) in Quechua:} \\
\gll ujya-n ni-n \\
drink-3.\textsc{subj} say-3.\textsc{subj}\\
\glt ‘She drinks, it is said.’ 
\z

\ea \label{ex:hearsay.CA}
 {Dedicated verbs (\textit{dice que}, \textit{dicen que}, \textit{dizque}) in Castellano Andino:} \\
\gll Toma, dice. \\
drink.3.\textsc{subj} say.3.\textsc{subj}\\
\glt ‘She drinks, it is said.’ 
\z

\paragraph*{Quotative:}

\ea \label{ex:quotative.Ay}
 {Verb `to say' (inflected) in Aymara:} \\
\gll Uma-nt-i-x s-i-w \\
drink-\textsc{iw-3.subj-top} say-\textsc{3.subj-decl}\\
\glt `“She drinks'', he says.'
\z



\ea \label{ex:quotative.Qu}
 {Verb `to say' (inflected) in Quechua:} \\
\gll Ujya-n ni-spa n-in\\
drink-\textsc{3.subj} say-\textsc{ger} say-\textsc{3.subj}\\
\glt `“She drinks”, he says.’
\z

\ea \label{ex:quotative.CA}
 {Dedicated verbs (\textit{dice, diciendo}) in Castellano Andino:} \\
\gll “Toma” diciendo dice. \\
drink.\textsc{3.subj}j say.\textsc{prog} say.\textsc{3.subj}\\
\glt `“She drinks”, he says.’
\z

Within the subsystems for hearsay and quotative marking there is a lot of variation regarding how the hearsay and quotative are conveyed across language varieties. This dynamic instability holds, interestingly enough, for Aymara, Quechua, and CA. However, the presence and use of these markers are highly consistent. 

\subsubsection{Aymara hearsay and quotatives}
In the Intermediate Aymara of Southern Peru, there is a distinction between \textit{s-i-w} `s/he says', which became lexicalized and can be seen as a marker meaning contextually, `it is said' or `one says' (referring to common knowledge) and \textit{sa.s} which is used as a quotative. Examples follow in~\REF{ex:siw} and \REF{ex:sas}.


\ea \label{ex:siw}
\gll uka usu-x wali \textit{phiyu}-∅-tayna-w \textbf{s-i-w} \\
that bear-\textsc{top} very bad-\textsc{cop.vbz-3.subj.nexp.past-decl} say-\textsc{3.subj.sim-decl} \\ 
\glt ‘That bear was very bad, they say.’\\
CA: `Muy malo había sido el oso, dice.\\
PS: `El oso era muy malo / malvado, decían.'
\z

\ea \label{ex:sas}
\gll uka-t timpranu-t sara-tan \textbf{sa.s} imilla-nak \\
that-\textsc{abl} early-\textsc{abl} go-\textsc{1.incl.subj.sim} say girl-\textsc{pl} \\ 
\glt `After, “We go early” the girls say.’\\
CA:`Después, temprano nos vamos, dicen las chicas.'\\
PS: `Después, “(nos) vamos temprano”, dicen las chicas.'

\z

\subsubsection{Quechua hearsay and quotatives}

Both the variety of Quechua investigated here and the variety spoken in Cusco have a reportative suffix -\textit{si} which marks hearsay, as in \REF{ex:cusco}.

\ea \label{ex:cusco}
\gll wakin-\textbf{si} maqa-mu-nku \\
some-\textsc{rep} hit-\textsc{cis-3.pl} \\ \glt `Some hit him, it is said.’ \citep[22]{faller2002semantics}\\
CA: `Algunos le pegaron, dice.'\\
PS: `Dicen que alguno (de ellos) / alguien le pegó.'
\z

However, in corpus data from natural conversations in Quechua, the use of lexicalized \textit{nin} (say-\textsc{3}) as a particle with the same hearsay function, replaces the suffix. This new reportative strategy seems to have emerged as a case of reverse influence of CA \textit{dice} \citep{olbertz2005dizque, dankel2012convergencias} on these varieties of Quechua.

\ea \label{ex:Q-nin}
\gll askha \textit{dolares}-ni-n ka-sqa \textbf{ni-n} \\
many dollars-\textsc{eu-3} be-\textsc{nexp.past} say-\textsc{3} \\ \glt `It was a lot of dollars, it is said.' \citep[96]{dankel2012convergencias}\\
CA:`Había tenído muchos dólares, dice.'\\
PS: `Dicen que fueron muchos dólares.'

\z

The Quechua quotative also emerged from a say-verb construction, in this case \textit{diciendo} `saying'. As we can see in \REF{ex:Qquotative}, it usually appears together with a finite say-verb form.


\ea \label{ex:Qquotative}
\gll ni-tax chaya-chi-mu-wa-n-chu ni-spa ni-ri-sa-n mama-yki-qa \\
no-\textsc{cont} arrive-\textsc{caus-cis-1.obj-3-neg} say-\textsc{ger} say-\textsc{inch-prog-3} mother-\textsc{2-top} \\ 
\glt ‘He didn't bring (his girlfriend) home either (that's what) your mother is saying. (CQE)\\
CA: `Ni a la casa no le ha traído está diciendo tu madre.'\\
PS: `Tampoco la trajo [a su novia] a casa, [eso es lo que] está diciendo / dice tu madre'\\
\z



\subsubsection{Castellano Andino hearsay and quotatives}

The situation for CA is very similar to what is attested in Aymara and Quechua. The hearsay marker emerged from the same model. Though for CA it is often lexicalized in combination with the relative pronoun \textit{que}, as in \textit{dice que}: 

\largerpage
\ea \label{ex:CA-diceque}
\gll Villa Pagador \textbf{dice} \textbf{que} es inmenso eso. \\
Villa Pagador say that is enormous that \\ \glt ‘Villa Pagador is said to be enormous.’\\
PS: `Parece que Villa Pagador es enorme.'
\z

Note that \textit{dice que} as an incipient grammaticalization structurally still behaves as a verb + complement construction. Functionally, however, the hearsay marking is unambiguous (see \citealt{dankel2015strategien}).\footnote{For a more complete picture, it is important to emphasize the incipient grammaticalization status of the CA hearsay markers. This means that there is still a highly variable use of forms (\textit{dice}, \textit{dicen}, \textit{dice que}), also in terms of their functional proximity to \textit{decir} `say'. We also find a fully fused form \textit{dizque}. However in the CA spoken in the areas investigated, \textit{dizque} is used with an additional meaning of gossip or doubt, whereas \textit{dice que} can also be used in contexts where hearsay information is used to convey epistemic authority of the distant source.}


The quotative marker is based on the gerund form of \textit{decir}, i.e. \textit{diciendo}. Also in CA, it frequently is constructed with a say-verb that introduces direct speech and \textit{diciendo} to mark the quoted utterance, as in \REF{ex:CA-said}.

\ea \label{ex:CA-said}
\gll Yo le he avisado a doña Simona {pue\ldots} estoy saliendo doña Simona un rato \textbf{diciendo}. \\
I her have told to Mrs. Simona well I.am leaving Mrs. Simona a while saying \\ \glt ‘I told Mrs. Simona ``I am leaving a while'', I said.’ (CQE)\\
PS: `(Yo) le dije a la Sra. Simona: “salgo un ratito”'\\
\z

However, also for the quotative we can speak of a situation of incipient grammaticalization, which means that there is still a lot of paradigmatic and syntagmatic variability regarding the say-verbs that occur with \textit{diciendo} and at best incipient fixation \citep{RN30}, whereas in Quechua and Aymara the say-verb and the quotative tend to occurs phrase-final and in a fixed order. 

The fact that there is not only a clear hearsay marking device as well as a quotative marking device in all three codes, but also that these markers emerge from the grammaticalization of say-verb constructions in all three varieties is a strong indication for the three codes converging into one system. 


\subsection{Inference/Conjecture }
\label{analysis3:inferentials}
Inference/conjecture is integral to Aymara and Quechua morphology. The use of these markers in naturally occurring speech remains under-researched for both languages. The research on inference/conjecture in CA is also minimal. This is likely because the changes in CA caused by a need for inference/conjecture marking are often unnoticed. \citet{hardman1982mutual}, for example, mentions some unusual uses of \textit{seguramente} in the CA of La Paz. In our own work we also found \textit{seguramente} in marked syntactic contexts where speakers refer to inferential knowledge (see \citealt{dankel2012convergencias}). As will be illustrated presently, there are additional techniques in CA, reflecting Aymara and Quechua markers for inference and conjecture. Our data shows that there are differences in the manifestation of the inferencial/conjectural in Aymara and Quechua. This is elaborated in the constructed examples provided in \REF{ex:infer.Aym1}--\REF{ex:infer.Quechua2} %The following examples, taken from our corpus, illustrate this usage. On this note, although more research on these constructions is needed, there is ample evidence to demonstrate a systematic use of inferential and conjectural marking in CA that reflects the patterns found in Aymara and Quechua.

\paragraph*{Inferential:}


\ea
\textit{-spha, -pacha} or counterfactual paradigm (intrinsically deduced) in Aymara:
\ea \label{ex:infer.Aym1}
\gll Uma-nta-\textbf{spha}-wa \\
drink-\textsc{iw-3.subj.3.obj.infr-decl} \\ 
\glt ‘He must have drunk it.’’ 

\ex \label{ex:infer.Aym2}
\gll Uma-nt-\textbf{irki}-wa \\
drink\textsc{-iw-3.subj.3.obj.cf-decl} \\
\glt ‘Maybe he could/should
have drunk it.’
\z
\z


\paragraph*{Conjectural:}


 \ea \label{ex:conjecturalAym}
\textit{-chi} (extrinsically deduced) in Aymara:
\gll Uma-nt-\textbf{ch}-i-xall \\
drink-\textsc{iw-cnj-3.subj.3.obj.sim-cnj} \\ 
\glt `Surely he must have
drunk.' 
\z
 
\newpage
\paragraph*{Inferential/Conjectural:}



\ea \label{ex:infer.Quechua1}
\textit{-cha, -sina, -chus, icha, ichas} in Quechua:
\ea
\gll ujya-n-\textbf{cha} \\
drink-\textsc{3.subj-inf} \\ 
\glt `He must have drunk.'

\ex \label{ex:infer.Quechua2}
\gll \textbf{icha} ujya-n \\
maybe drink-\textsc{3.subj} \\ 
\glt `Maybe he drunk.'
\z
\z

\ea \label{ex:infer.CA1}
\textit{Seguramente, seguro
(que), deber, haiga, creer},
among others in CA.:

\ea
\gll \textbf{Seguramente} habrá
tomado. \\
surely have.3.\textsc{fut} drink.\textsc{ptcp} \\ 
\glt ‘Surely he must have
drunk.’ 

\ex \label{ex:infer.CA2}
\gll \textbf{Debe} haber tomado. \\
must.3 have.\textsc{inf} drink.\textsc{ptcp}\\ 
\glt ‘He must have drunk.’ 


\ex \label{ex:infer.CA3}
\gll \textbf{Haiga} tomado. \\
have.3.\textsc{sbjv} drink.\textsc{ptcp} \\ 
\glt ‘He must have drunk.’ 
\z
\z
 





\subsubsection{Aymara inference/conjecture}
Aymara seems to make a fine-grained distinction between extrinsically (relying on direct evidence) and intrinsically (relying on logical reasoning) deduced knowledge. The former are treated in (\ref{ex:A-spha}--\ref{ex:A-cf2}) and the latter in (\ref{ex:cnj1}--\ref{ex:jalla1}).



\ea \label{ex:A-spha}
\gll kha-n-x trucha-x ut.ja-\textbf{spha}-w \\
yonder-\textsc{loc-top} trout-\textsc{top} exist-\textsc{3.subj.infr-decl} \\ 
\glt ‘There must be trout yonder.’ \citep[291]{coler2014grammar}\\ 
CA: ‘Allá \textbf{debe de} haber trucha.’\\
PS: `Debe haber truchas allí / allá.'
\z

Here, the speaker's claim is based on direct, extrinisically-deduced knowledge of the world. He knows trouts tend to hide in the specific location in that river bend. Types of inference like these, are expressed in Aymara either with a dedicated suffix (-\textit{spha} or -\textit{pacha}, depending on the region), as in \REF{ex:A-spha2}. In this example, the speaker speculates about a relative who she has not seen in a long time.

\ea \label{ex:A-spha2}
\gll lik'i-∅-nta-\textbf{spha}-w \\
fat-\textsc{cop.vbz-iw-3.subj.infr-decl} \\ 
\glt ‘She must have fattened up.’ \citep[439]{coler2014grammar}\\ 
CA: `\textbf{Seguro que} ha engordado.'\\
PS: `Debe haber engordado.'
\z

Next, members of the counterfactual paradigms overlap with inferential{\slash}con\-jec\-tural marking in Aymara and can convey a myriad of meanings. A paradigmatic one appears in \REF{ex:A-cf1} which is premised on the evidence that the accused trickster has revealed himself to be disingenuous and so is liable to deceive again. Consider also \REF{ex:A-cf2} in which a member of the present counterfactual paradigm is used to make a question-like expression without the use of interrogative morphology. Observe that the phrase in the latter is not translated with \textit{seguro}, though the meaning is less inferential than the one in the former. Indeed, there is significant contextual variation in how this morpheme is translated.

\ea \label{ex:A-cf1}
\gll jicha-x wasita-mp \textit{inkaña}-jw-\textbf{itasma}-x \\
now-\textsc{top} again-\textsc{com} deceive-\textsc{bfr-2.subj.1.obj.pres.cf-top}\\ 
\glt ‘Now again you may deceive me.’ \citep[434]{coler2014grammar} \\ 
CA: `Ahora, \textbf{seguro que} de nuevo me puedes engañar.'\\
PS: `Ahora, puedes engañarme de nuevo'
\z

\ea \label{ex:A-cf2}
\gll juma-x thuqu-\textbf{sma}-x \\
you-\textsc{top} dance-\textsc{2.subj.pres.cf-top} \\ 
\glt ‘Might you dance?’ \citep[257]{coler2014grammar}\\ 
CA: ‘¿Tú \textbf{por qué no} bailas?’\\
PS: `¿Bailarías? / ¿Querrías bailar? / ¿Te gustaría bailar?'
\z

All of these notions are extrinsic deductions based on direct experiences. These are differentiated from intrinsically deduced knowledge for which the speaker relies on knowledge arrived at through deductive reasoning based on 
learned experience of the world. This is expressed by the conjectural suffix, as in \REF{ex:cnj1} in which the speaker knows that someone caught some trout in this spot recently, but did not see trout there himself. However, given what he knows about the world, there are probably more .

\ea \label{ex:cnj1}
\gll kha-n \textit{trucha}-x ut.ja-s-\textbf{ch}-i-s \\
yonder-\textsc{loc} trout-\textsc{top} exist-\textsc{refl-cnj-3.subj.sim-ad} \\ 
\glt ‘Perhaps there are trout yonder.’ \\ 
CA: `\textbf{Tal vez} todavía allá hay trucha.' \citep[290]{coler2014grammar}\\
PS: `Tal vez haya truchas allí / allá.'
\z

Finally, in some varieties of Aymara, like those spoken in the Southern Peruvian highlands, there is a phrase-final suffix -\textit{jalla} which can also express the conjectural. This suffix typically co-occurs with the conjectural -\textit{chi}.\footnote{The semantics and distribution of this suffix are described in detail in \citet[560]{coler2014grammar}.} Consider the minimal pair in \REF{ex:jalla1} and \REF{ex:jalla2} and note how the CA (and PS and English) translations are identical, even though the meanings expressed in Aymara are different, again on
the basis of the type of reasoning used.

 \REF{ex:jalla1} may be uttered when a speaker notes that a colleague did not arrive at work (i.e. intrinsic reasoning). \REF{ex:jalla2} may be uttered when a speaker observes a caregiver leaving the referent's house looking distraught (i.e. extrinsic reasoning).



\ea \label{ex:jalla1}
\gll usu-ta-∅-s-ch-i-\textbf{xall} \\
sick-\textsc{re-cop.vbz-prog-cnj-3.subj.sim-cnj} \\ 
\glt ‘He must be sick.’ (Banegas-Flores, p.c.) \\
CA: ‘Seguro que está enfermo.’ \\
PS: `Debe estar enfermo (no ha venido a trabajar).'
\z

\ea \label{ex:jalla2}
\gll usu-ta-∅-\textbf{spha}-w \\
sick-\textsc{re-cop.vbz-3.subj.infr-decl} \\ 
\glt ‘He must be sick.’ \citep[564]{coler2014grammar} \\
CA: ‘Seguro que está enfermo.’ \\
PS: `Debe estar enfermo (la veo muy preocupada por él).'
\z

%In Aymara, the inferential is marked with -\textit{pacha} or -\textit{spha} (depending on the variety of the language). The inferential marks inference through reasoning or common knowledge (\cite[pg. 452]{coler2014grammar}).

\subsubsection{Quechua inference/conjecture}
Inference/conjecture marking is also very common in Quechua, though it does not make a fine-grained distinction between the expression of intrinsically (based on logical reasoning) and extrinsically (based on direct evidence) deduced knowledge on the morphological level. Both are expressed with -\textit{cha}. While Quechua -\textit{cha} isn't identical to the Aymara -\textit{chi}, in some contexts their roles are very similar. Compare example \REF{ex:Aym-Compare} with \REF{ex:Quechua-cha}.

\ea \label{ex:Aym-Compare}
\gll Uta-p-∅-x		alxa-wj-\textbf{chi}-ni-xall \\
house-\textsc{3.poss-acc-top} sell-\textsc{bfr-cnj-3.subj.3.obj.fut-cnj} \\ 
\glt `Probably she will sell her house.' \\
CA: `Seguramente va a vender su casa.'\\
PS: `Probablemente venderá su casa.'
\z

\ea \label{ex:Quechua-cha}
\gll \textit{vende}-nqa-\textbf{cha} wasi-n-ta á \\
sell-\textsc{fut-inf} house-\textsc{3-acc} \textsc{mod.int} \\ 
\glt ‘Probably she will sell her house.’\\
CA: `Seguramente va a vender su casa.'\\
PS: `Probablemente venderá su casa.'
\z

%The suffix -\textit{sina} is also used to express conjecture, but with a greater sense of uncertainty. For example, when asked to speculate about the age of a family member, but lacking exact knowledge, the speaker hazards a guess: 

%\ea \label{ex:dub}
%\gll \textit{cincuenta} \textit{y} \textit{tres}-ni-yux-sina \\
%fifty and three-\textsc{euf-com-dub} \\ 
%\glt ‘I guess she's fifty three.’ \\
%`creo que tiene cincuenta y tres.' (CQE)
%\z


%Finally, there are question-like conjectures formed with the suffix -\textit{chus}. In its ability to express inferential meanings, this suffix resembles the Aymara counterfactual. One difference is that -\textit{chus} can also express a doubt, as in \REF{ex:chus}.


The particle \textit{icha} (or \textit{ichas}) is used in similar contexts as the co-occurrence of the
counterfactual and the conjectural in Aymara (as in~\REF{ex:saludar}). \textit{Icha} expresses conjecture
in the sense of an expected event and can be considered to be at the boundary between
inferentials/conjecturals and potentials/counterfactuals (see the analysis on the potential and counterfactual mood in Section \ref{analysis4:potentialCF})

\ea
\gll \textbf{ichas} chaya-chi-mu-wan-pis ni-n pero á \\
maybe arrive-\textsc{caus-cis-3.subj.1.obj-conc} say-\textsc{3} but \textsc{mod.int} \\ 
\glt `Yet, maybe he brought here, she says, alright.' (CQE)\\
CA: `Puede que tal vez le ha traido, dice pero pues.'\\
PS: `Aunque tal vez la trajo, dice, ¿sabes?'
\z

\subsubsection{Castellano Andino inferentials and conjecturals}
The aforementioned inference/conjecture functions are present in CA, yet are often unnoticed, as they are still at the beginning of being grammaticalized. That is, their individual lexico-syntactic appearance may not be seen as new in comparison to other Spanish varieties. Nevertheless, in their overall systemic-functional configuration and frequency they are unattested in other Spanish varieties.

Constructions with \textit{seguramente} (\textit{que}) or \textit{seguro} (\textit{que}) are used in CA to express knowledge arrived at through intrinsic and extrinsic deduction. \REF{ex:criadelos}, based on the latter, is produced jokingly in response to another speaker's anecdote about looking for gold at an excavation site.

\ea \label{ex:criadelos}
\gll \textbf{Seguro} cría de los españoles eres. \\
surely descendent of the Spanish you.are \\ \glt ‘Then presumably you are a descendent of Spaniards.' (CQE).\\
PS: `Entonces parece que eres descendiente de españoles.'
\z

In \REF{ex:mandolin}, the deduction process is internal logical reasoning. Clearly, the line between direct evidence-based reasoning and reasoning based on already established world knowledge is not always clear cut.

\ea \label{ex:mandolin}
\gll Después, siempre hemos sido aficionados a la música porque mi papá tocaba mandolina, tocaba charango, nunca lo he visto tocar guitarra, pero \textbf{seguramente} que sabía. \\
after always we.have been amateurs to the music because my father played mandolin played charango never it have seen to.play guitar but surely that knew\\ 
\glt `Also, we always have been music enthusiasts, because my father played mandolin, played charango, I never saw him play the guitar, but he surely knew.' \citep[107]{dankelpagel2012}\\
PS: `Además, siempre hemos sido aficionados/as a la música, porque mi padre tocaba la mandolina, tocaba el charango, nunca lo he visto tocar la guitarra, pero seguro que sabía.'
\z

The narrator explains that he and his siblings became music enthusiasts because of his father’s skill with stringed instruments. Interestingly, he signals a difference in his mode of access to this knowledge between ``directly witnessed'' for the mandolin and charango and inferred (introduced by the contrast \textit{nunca lo he visto \ldots pero seguramente que}) for the guitar. For non-Andean speakers, such precise evidential marking would only occur if explicitly accounted for by the interlocutor or the context of their telling, but is produced systematically by CA speakers. This is a clear case for a change in the way of speaking because of a different way of thinking, in other words, Slobins (\citeyear{slobin2016thinking}) “thinking for speaking”.

CA constructions with \textit{deber} can also express inferential/conjectural meanings, as in \ref{12yo}. In PS the last two lines of this example would be \textit{¿Qué edad tenías?} and \textit{Probablemente tendría unos 12 años}.

\ea \label{12yo}
A: Mi hermano estaba tirado en la cama quemado aquí con una ampolla.\\
E: ¿Cuántos años tenías?\\
A: Más o menos unos 12 años \textbf{yo he debido tener}. \\
\medskip
A: My brother was laying on the bed, burned here, with a blister.\\
E: How old were you?\\
A: I \textbf{would probably have been} around 12 years old.\\ 
\citep[195]{dankelpagel2012}
\z



Particularly in the area where Aymara is spoken with CA, speakers use an archaic form of the perfect subjunctive construction, \textit{haiga} + participle, as an inferential construction. This is evident in the following testimony of a confrontation during a social protest, where the speaker got shot. The speaker infers that the military started to shoot because they ran out of tear gas. 

\ea \label{ex:haiga}
\gll A lo así nomás, sus gases se \textbf{haiga} \textbf{acabado}, qué será. Ahí nos han baleado con armas de guerra. \\
to what thus just their gases themselves have run.out what will.be there us have shot with weapons of war \\ \glt ‘Suddenly, \textbf{surely} they ran out of tear gas, I don´t know. They started to shoot with their war guns.’\\
PS: `De repente, seguramente se les acabó el gas lacrimógeno, no lo sé. Empezaron a disparar con sus armas de guerra.'\\
\z

The conjectural reading of the subjunctive in \REF{ex:haiga} can be considered in a state of grammaticalization, as other varieties of Spanish, particularly Peninsular Spanish, do not allow this reading without the co-occurrence of additional conjectural adverbs. This example is even more interesting because the construction is followed by the question-like \textit{qué} \textit{será} (`what will be'), as a dubitative, similar to what we see in Quechua with the dubitative -\textit{chus}. These co-occurrence of
inferentials/conjecturals with certainty markers seem to appear in Aymara, Quechua and especially CA.\footnote{Especially for the dynamic situation in CA, it is not surprising that speakers combine a variety of strategies to get as close as possible to their culturally shaped communicative goals.} We will briefly discuss -\textit{chus} and other certainty markers in Section \ref{sec:outstandingissues}.
Similar to the potential/counterfactual, they are closely intertwined with inference and conjecture and allow for highly nuanced elaborations of knowledge states and
responsibilities. They also deserve more attention in future research.

%We already mentioned the property of some of the markers to co-occur in Aymara, Quechua and especially CA. We therefore might deal with a categorical divide between inferential/conjectural markers on the one hand and certainty markers on the other, however, especially for the dynamic situation in CA it is not surprising that speakers make use of a variety of strategies to get as close as possible to their culturally shaped communicative goals. 



\subsection{Potential and counterfactual mood}
\label{analysis4:potentialCF}
As alluded to earlier, potential/counterfactual mood is closely intertwined with inference and conjecture (treated in Section \ref{analysis3:inferentials}). %This is no surprise, as references to the expression of possible or probable events are naturally related with conjecture.\footenote{There is also a fuzzy boundary between inference/conjecture, the expression of certainty and potential mood. We will refer to this at the beginning of the next section} However, from a structural and grammatical perspective, they deserve a separate treatment. Both languages use resources and mechanisms in accordance with their agglutinating morphological character and therefore, in their conditional constructions work quite different from the complex syntactic schemas typical for European languages.
Aymara uses the counterfactual mood paradigm to refer to possible events and conditional constructions and Quechua has a potential suffix.


%In Quechua and Aymara, the semantics of potentiality-conditionality and counterfactuality are tied to morphosyntactic resources and mechanisms in accordance with the agglutinating morphological character of these languages and therefore work quite differently from the complex syntactic schemas typical for European languages. 

\subsubsection{Aymara counterfactual mood}
Aymara does not have a form that is glossed as `potential', though the conjectural, inferential and counterfactual forms can be employed to express potential meanings. The counterfactual is the only marked mood in Aymara \citep[428]{coler2014grammar}. It is expressed with recourse to a pair of paradigms that inflect for simple and past tense. Usually the counterfactual is translated into CA with the modal verb \textit{deber} `should/must'. 


\ea
\gll jut-\textbf{irki}-w \\
come-3\textsc{.subj.sim.cf-decl} \\ \glt ‘He should/must come.’\\
‘Debe venir.’\\
Evidence: He left something behind, he will certainly return for it.\\
CA: `Debe venir.'\\
PS: `Debería venir. / Tiene que venir.'\\
\z

The following minimal triplet facilitates comparisons of the inferential evidential and counterfactual and conjectural suffixes (both of which were treated in Section \ref{analysis3:inferentials}) and the counterfactual, both in Aymara and CA.


\ea \label{ex:AymaraSPHA}
\gll kha-n-x trucha-x ut.ja-\textbf{spha}-w \\
yonder-\textsc{loc-top} trout-\textsc{top} exist-\textsc{3.subj.infr-decl} \\ \glt ‘There must be trout yonder.’ \citep[456]{coler2014grammar}\\
CA: `Allá debe de haber trucha.' \\
PS: `Debe haber truchas allí / allá.'\\
Evidence: Speaker infers this from his knowledge in what specific location in the river bend trout tend to hide.
\z

\ea
\gll kha-n \textit{trucha}-x ut.ja-s-\textbf{ch}-i-s \\
yonder-\textsc{loc} trout-\textsc{top} exist-\textsc{refl-cnj-3.subj.sim-ad} \\ 
\glt ‘Perhaps there are trout yonder.’ \citep[456]{coler2014grammar} \\
CA: `Tal vez todavía allá hay trucha.' \\
PS: `Tal vez haya truchas allí / allá.' \\
Evidence: Speaker knows that someone caught some trout in this spot recently so there are probably more there.\\
\z

\ea
\gll kha-n-x \textit{trucha}-x ut.ja-s.k-\textbf{irki}-w \\
yonder-\textsc{loc-top} trout-\textsc{top} exist-\textsc{prog-3.subj.pres.cf-decl} \\ \glt ‘There may be trout yonder.’ \citep[456]{coler2014grammar}\\
CA: ‘Truchas podrían estar viviendo allí.’\\
PS: `Puede que haya truchas allí / allá.'

\z

The conjectural suffix -\textit{jalla} can co-occur with a member from the counterfactual paradigm, as in \REF{ex:saludar}.

\ea \label{ex:saludar}
\gll \textit{Salur}-t'a-s-\textbf{irki}-pun-\textbf{jall} \\
salute-\textsc{m-refl-3.subj.pres.cf-em-cnj} \\ 
\glt ‘She would really say hello.’\citep[517]{coler2014grammar} \\
CA: `Me podría saludar.' \\
PS: `Ella realmente saludaría.'
\z 

\subsubsection{Quechua potential mood}
The Quechua potential is expressed through the suffix -\textit{man} and its variant -\textit{wax} added to an inflected verb form thus transforming it into a non-finite verbal constructions that can also convey counterfactuality. The suffix can be used for notions such as ability, possibility, enabling conditions, %%\footnote{Although Quechua lacks a dedicated syntactic construction to express conditionality, speakers can express possible or probable events, as well as counterfactuality using the subordination markers -\textit{xti}- and -\textit{spa}- (switch-reference: same subject and different subject, respectively) to form non-finite constructions to combine temporally competing events. However, these resources and mechanisms are also used to establish other suborational relationships such as concession, simultaneity or cause (Cayetano 2003, Soto Rodríguez 2013).}
and permissions (as in \REF{ex:QPot1}, in which a speaker is calling in to a radio show to request a song). In conversational interaction, these notions also allow for a directive use of the form.

\ea \label{ex:QPot1}
\gll Takiy-s-itu-ta maña-ri-ku-yki-\textbf{man}-chu Doña Rosmery \\
  song-\textsc{euf-dim-acc} ask-\textsc{inch-refl-2-pot-int} Mrs. Rosmery\\ 
  \glt ‘May I request a song, Mrs Rosmery?’ (CQE) \\
CA: `?`Puedo pedir una canción, Doña Rosmery?'\\
PS: `¿Puedo pedir una canción, Sra. Rosmery?'
\z 

Particularly with directive functions, the potential suffix is used in counterfactual constructions. The pragmatic use of -\textit{man} lets speakers refer to events that have occurred which they would have wished to be different. Also, the same suffix in combination with the auxiliary \textit{kay} `be' in the past allows the speaker to express explicit counterfactual events, as in \REF{ex:tiempo}, where the speaker tells of his participation in an armed confrontation and reflects on what may have happened.

\ea \label{ex:tiempo}
\gll ima-ta may \textit{tiempo}-pis wañu-y-\textbf{man}-pis ka-rqa á \\
what-\textsc{acc} how time-\textsc{conc} die-\textsc{1-pot-conc} be-\textsc{exp} \textsc{mod.int} \\ 
\glt ‘You know, I would be dead a long time ago.’ (CQE)\\
CA: `Sabes qué, hace tiempo tal vez ya hubiera muerto, pues.'\\
PS:`"Sabes que yo (ya) estaría muerto/a hace mucho tiempo.' / `Yo estaría muerto/a hace mucho tiempo, ¿sabes?"' \\
\z 

\subsubsection{Castellano Andino counterfactual and potential mood}
 CA mechanisms to refer to potential events reflects Aymara and Quechua indirectly. The most prominent strategies are the use of \textit{poder} `can' (as in \REF{ex:poor}) to express hypothetical events (instead of its canonic dynamic modal function to express ability and willingness or deontic uses),\footnote{
The use of periphrastic constructions with \textit{poder} `can' in the Andean zone, on the other hand, is causing analytical processes in modal constructions of Quechua (see \citealt{Haimovich2016}). It is common to find the use of periphrastic modal constructions with the auxiliary \textit{atiy} `can' such as \textit{imatapis rantikuyta atinki} `you would/could buy anything' (CQE) instead of \textit{imatapis rantikuwax}. In some of those cases, \textit{atiy} `can' converges with the potential Quechua -\textit{man}. For example, the continuity of a dialogue is in danger: \textit{atinman p'akikuyta} `it would break out' (CQE). If we consider uses of \textit{poder} with these notions as an effect of the contact with Quechua, and, conversely, the use of new periphrastic modal constructions with poder in Quechua, we can observe a relationship of round-trip effects in the processes of influx that we consider as a boomerang process.} constructions with modal particles like \textit{tal vez} and \textit{capaz}\footnote{The use of \textit{capaz} with modal value has been reported in other varieties in America, too (\citealt{narrog2012modality, magaly2010, yelin2017capaz}, among others).} (in \REF{ex:colors} and \REF{ex:capaz}, respectively) that clearly differ from Peninsular Spanish patterns, and the use of the pluperfect subjunctive in single utterances with an implicit second component instead of a bipartite subordinate conditional structure\footnote{Again, a detailed account of conditional constructions is outside the scope of this chapter. Nonetheless, note that the patterns in CA do not follow the combinatorial grammatical restrictions that are obligatory in Peninsular Spanish, and, furthermore, orient to patterns which closely resemble Quechua syntax: non-finite subordinate structures (e.g. \textit{sabiendo esto me hubiese ido}) as a conditional or constructions with auxiliaries as matrix clause (parallel to the use of the Quechua auxiliary in example \REF{ex:tiempo}) in exhortative contexts (e.g. \textit{Era que vengas más temprano}) as a contrafactual, and which go beyond a discursive or stylistic phenomenon.} (see \REF{ex:job} in which the speaker complains about the father of her son and ex-partner, and his indifference about her situation when they met after a long time).

 %55
 \ea \label{ex:poor}
\gll Todos eran pobres. Pobres, \textbf{podían} robarnos, pero no, considerados eran. \\
all they.were poor poor they.can rob.us but no considerate they.were\\
\glt `Everybody was poor. Poor, they could have robbed us, but no, they were considerate.' \citep[47]{dankelpagel2012}\\
PS: `Todos eran pobres. Pobres, podrían habernos robado, pero no, tuvieron consideración.'
\z 

%(56)
\ea \label{ex:colors}
\gll O sea el me ha dicho siempre los colores \textbf{tal vez} estoy equivocado. \\
 it be the me has said always the colors maybe I.am wrong \\ 
 \glt ‘I mean, he said it clearly, about colors I could be wrong.’ (CQE)\\
 PS: `Quiero decir, lo dijo claramente, sobre los colores yo podría estar equivocado.'
\z 

%(57)
\ea \label{ex:capaz}
\gll \textbf{Capaz} me dé con el palo así. \\
where you you.go able me give with the stick so \\ 
 \glt ‘She would possibly hit me with the stick or something.' (CQE)\\
 PS: `Probablemente, me iba a pegar con un palo o algo (así).'
\z 


%(58) :
\ea \label{ex:job}
\gll ?`Tú lo mantienes? ?`Estás con trabajo? ?`Tienes plata? Por {lo menos} por ahí hubiera empezado. \\
 you him support you.are with work you.have money for {at.least} for there would.have started\\ \glt ‘Who? Do you support him? Do you have a job? Do you have money? At least so he would have started.' (CQE)\\
 PS: `¿Tú lo mantienes? ¿Tienes un trabajo? ¿Tienes dinero? Al menos así (él) habría empezado.'
\z

 CA speakers reflect the meanings which are possible with dedicated Aymara and Quechua suffixes by using strategies entirely (or nearly) absent in Peninsular Spanish. This is a first step on the linguistic surface for establishing a semantic copy \citep{RN59} or for \citeauthor{slobin2016thinking}'s (\citeyear{slobin2016thinking}) concept of “thinking for speaking” as the cognitive impulse for contact-induced change. 

\subsection{Outstanding issues}
\label{sec:outstandingissues}
We have only scratched the surface of the contact-induced processes lying beneath the parallels we described. In this section we mention some noteworthy observations which future studies may take into account: certainty markers\footnote{We use ``certainty markers'' as a cover term for these suffixes, because they all reflect the speakers stance on the veracity of the state of affairs. However, this is a working definition, as more research is necessary to define the precise functions of these suffixes.}, the expression of the future, the progressive and the habitual. 


Closely intertwined with evidential marking, the languages under investigation also have many ways to express epistemic modality, which is hardly researched. Among these, the following certainty-markers are the most relevant as they overlap with inferentials/conjecturals, especially in Quechua and CA. The Quechua suffix -\textit{sina} expresses uncertain conjecture, for example, when speculating about the age of a family member, as in~\REF{ex:dub}. In CA, as is typical for situations of incipient grammaticalization, this is reflected in the use of \textit{creo que} (`I believe that'). This is illustrated in \REF{ex:shot}, which comes from the testimony of a confrontation during a protest. In this example, \textit{creo que} is neither located at the beginning of enunciative sequences as in its function as a discourse maker, nor is it used with an explicit subject, as in manifestations of points of view.

\ea \label{ex:dub}
\gll \textit{cincuenta} \textit{y} \textit{tres}-ni-yux-sina \\
fifty and three-\textsc{euf-com-dub} \\ 
\glt ‘I guess she's fifty three .’ \\
CA: `Creo que tiene cincuenta y tres.' (CQE)\\
PS: `Creo / supongo que tiene cincuenta y tres años.'
\z


\ea \label{ex:shot}
\gll Pero ya ni bien estábamos defendiendo y todo, regresé {de nuevo} hacia adelante para poder ayudar a mis compañeros y creo que me vieron. Todo eso directo un tiro ¡pum! \\
but already or well we.were defending and all I.returned again toward ahead for be.able to.help to my colleagues and I.believe that me they.saw all this direct a shot bang \\
\glt ‘As soon as we were defending ourselves or so, I went ahead again to help my colleagues and, I guess they saw me -- suddenly a shot: bang!’ \citet[97]{Defensoria2020} \\
PS: `Cuando nos estábamos defendiendo, yo volví de nuevo para ayudar a mis compañeros/as y creo que me vieron – de repente un disparo: ¡pam!'
\z

Furthermore, there are question-like conjectures with the Quechua suffix \mbox{-\textit{chus}}. However, function-wise, speakers use -\textit{chus} to express a doubt, as in~\REF{ex:chus}. The suffix -\textit{chus} is often reflected in CA with the postponed tag question \textit{qué será}, which is frequently attested in our CA data.

\ea \label{ex:chus}
\gll imayna-\textbf{chus} á mana yacha-ni-chu pi-\textbf{chus} ka-n-pis \\
how-\textsc{conj} \textsc{mod.int} no know-\textsc{1-neg} who-\textsc{conj} be-\textsc{3-conc} \\ 
\glt ‘No idea, I don't know, I don't know even who she is.'\\
CA: `Cómo será pues, no sé quién será.' (CQE)\\
PS: `"Ni idea, no lo sé, ni siquiera sé quién es ella.'
\z




Finally, there are clear systematic parallels in the use of the Quechua suffix -\textit{puni}, the Aymara suffix -\textit{puni} and the use of CA adverbial \textit{siempre} `always', which are used as validators \citep{cerron2008quechumara, RN64, coler2014grammar}. 
%%% the coler ref is to pg 519 but since no one else gives pg numbres I removed it 

\ea \label{ex:ukjama}
\gll Uk-jama-\textbf{puni}-tayna-w \\
that-\textsc{cp-emp-3.subj.3.obj.nexp.past-decl} \\ \glt ‘That's how it went.’ \citep[148]{RN17}\\ 
CA: `Así \textbf{siempre} había sido.' \\
PS: `Así fue.'
\z

\ea \label{ex:mmamay}
\gll Anchay, mama-y wasi-n-\textbf{puni} \\
that mother-\textsc{1} house-\textsc{3-cert} \\ \glt ‘That house really belongs to my mother.’ \citep[170]{RN16}\\
CA: `Esa casa de mi mamá siempre es.' \\
PS: `Esa casa realmente es de mi madre.'
\z



Certainty markers in all three varieties often co-occur with evidential inferentials. This leads to epistemically highly complex phrases, especially in CA where one finds sentences like \textit{Alicia \textbf{seguramente} le insultaría, \textbf{qué sería}, \textbf{no sé}} ‘Alicia must have insulted her, how would it be, I don’t know’. Observe the three different epistemic markers (in bold) interacting within one sentence. Speakers recombine a variety of Spanish strategies to get as close as possible to their culturally shaped communicative functions, expressed with suffixes in Aymara and Quechua. A further discussion of these nuances would be interesting, even if it falls outside the scope of this chapter. More research is required.

As regards the future tense, in CA the analytic future is used as a tense, where\-as the synthetic future imparts a modal function (asking for permission/confirma\-tion): \textit{le he dicho iremos a comer jarwi uchu} ‘Let’s go to eat a jarwi uchu’ (where \textit{iremos} `we should go' is used in CA instead of \textit{vamos} `we will go', as in PS). This is also what one finds in the indigenous languages discussed where the distinction between the interpretation of the future tenses may be made contextually and/or by adding temporal adverbs. Observe the synthetic future form in the CA translation of \REF{ex:Elivra} in Quechua.

\ea \label{ex:Elivra}
\gll Elvira t’anta-ta ruwa-pu-\textbf{sqayki}-chu? Ichas qam extraña-yku-chka-nki. \\
Elvira bread-\textsc{acc} make-\textsc{ben-1.subj.2obj.fut-int} perhaps you miss-\textsc{dir-prog-2sg} \\ \glt `Elvira, do you want me to bake bread for you? Perhaps you're missing it (the bread).' \\
`Elvira te lo haré \textbf{pancito}? Tal vez vos estás extrañando (el pan).' \citep[204]{RN16}\\
PS: `Elvira, ¿quieres que te hornee pan? Quizás lo echas / eches de menos (el pan).'
\z

At first glance, the progressive aspect has little to do with mood and modality. According to Quechua grammars, verbal morphology obligatorily marks the immediate as opposed to non-immediate character of an event \citep[231]{adelaar2004, RN48, cerron2008quechumara, RN50}. This is done with the suffix -\textit{sa}. %(Bolivian Quechua; -s̬ka- for Ecuador, -\textit{sya}--sa/-s̬a for Cuzco Quechua and -chka-, for Chanka). In our natural language data, this is at least a strong tendency, reflected clearly in CA. 
Consider the parallel ways in which Quechua and CA use progressive constructions in the following examples. In both, a speaker makes reference to an ongoing activity. This is achieved in \REF{ex:water} with the progressive morpheme and in \REF{ex:todo} with a periphrastic gerund construction.

\ea \label{ex:water}
\gll chara-ta qarpa-x ri-\textbf{sa}-ni \\
crop-\textsc{acc} water-\textsc{sub} go-\textsc{prog-1sg} \\ 
\glt ‘Right now I'm going to water the crop.’ \citep[267]{RN14}\\
CA: `Estoy yendo a regar el terreno.' \\
PS: `Ahora mismo voy a regar la cosecha.'
\z


\ea \label{ex:todo}
\gll y hacen de todo, como está viendo pantalones, camisas, de todo \\
and they.make of all as are seeing pants shirts of all \\ \glt ‘And they make everything, as you see, pants, shirts, everything.’ (CQE)\\
PS: `Y hacen de todo, como ves, pantalones, camisas, de todo.'
\z

Progressives go beyond an aspectual function. They show ``the speaker's attitude and perspective of the situation; and, in so doing, [convey] her epistemic stance at a particular moment in the context of utterance'' \citep[157]{RN15}. The experiential character of the progressive therefore expresses a modal notion. A further modal notion that has been observed is a volitional reading to express deliberate intentions that can also be realized in a non-immediate future, as in a phrase like \textit{Estoy saliendo de la casa. Llego en un rato} `I am [already] leaving the house. I’ll arrive in a bit.' Upon closer examination of the CA and Quechua progressive constructions for current events, it becomes clear that their development is a diachronic path, where both Quechua and Spanish are dynamically intertwined with patterns being exchanged back and forth in both directions at various points in time \citep{Soto2016}. Future research should also take into account the progressive in Aymara.

Habitual actions or states in Aymara and Quechua are expressed with agentive constructions. In Aymara, they are formed with the agentive nominalizer -\textit{iri}. A verb nominalized with the agentive connotes a habitual action, in the sense that \textit{um-iri} ‘drinker’ can refer to one who drinks habitually. Consequently, \textit{jupa-x um-iri-w} ‘she is a drinker’ (of alcohol) may also be translated as ‘she often drinks’ (‘ella sabe tomar’) likewise \textit{jupa-x chur-ir-itu-w} ‘she usually gives it to me’ (‘ella me sabe dar’) can be understood roughly as ‘she was a giver to me’. This mechanism works similarly in Quechua where the agentive suffix -\textit{x} can be used to nominalize verbs such as \textit{ruwa} `do' to become \textit{ruwa-x}, `the one who makes/the one who does habitually'. %This seems to be the origin for the formation of habitual constructions like \textit{runa ruwa-x ka-rqa-nku} (people used to do it). 

Whereas habitual marking is central to Quechua and Aymara the habitual in PS is is not a central systematic category and is formed with a dedicated verb \textit{soler}. CA speakers did not promote the \textit{soler} construction but seemingly chose another mode of marking the habitual based on \textit{saber} ‘to know’+ infinitive: \textit{las toallas saben estar colgadas en allá} ‘the towels usually are hung there’. This strategy already existed rather periferically in old Spanish spoken varieties \citep{pfander2009gramatica}, but is a cross-linguistically frequent grammaticalization path \citep{kouteva2019world} because of its pragmatically and metonymically/metaphorically plausible developmental potential.\footnote{In Aymara and Quechua we also find a development in the opposite direction. The modal verb 'want' (\textit{munay} in Quechua, \textit{muna} in Aymara) is used aspectually for the description of incipient natural events. This use is reflected in CA, where \textit{querer} can be used in a comparable way.

\ea \label{ex:larva}
\gll Iqra-cha-ku-y-ta-ña muna-chka-sqa. \\
wing-\textsc{caus-refl-inf-acc-disc} want-\textsc{prog-nexp.past} \\ \glt ‘Wings were about to grow up [from a larva].’ (Lit. `it wanted to grow up wings') \citep[75]{mamani2018kunanpi}
\z

\ea
\gll Calamina quiere levantar, quiere llevar. \\
metal.roof want to.lift.up want to.take.off \\ 
\glt `It [the strong wind] is about to lift up metal roofs, it is about to take them up'\\
(Lit. It wants to lift up metal roofs, it wants to take off the roofs), (CQE).
\zlast
}

\section{Discussion}

We selected four case studies to illustrate how mood and modality in Aymara, Quechua and CA converge. For more than a century, Aymara and Quechua were minority languages and were hardly integrated. Nonetheless, their conceptual notions highly influenced CA. The three languages developed many similarities and show ongoing exchange of notions and patterns. 

In the analysis of the tense/evidentiality distinction, CA is shown to have developed on the structural and the notional levels. CA speakers do not just map the modal-evidential (non-)experienced past distinction, rooted in Quechua and Aymara morphology, onto the Spanish perfect and pluperfect temporal morphology; they also transfer the mirative interpretation, which is associated with the Aymaran and Quechuan non-experienced past tense, to the newly established CA counterpart (the pluperfect morphology). 

In the discussion of hearsay and quotative marking, we showed how two culturally highly-relevant practices which became grammaticalized in both Aymara and Quechua, lead to the formation of grammatical markers for the same category in CA, formerly not systematically present in other varieties of Spanish. We also pointed to a reverse influence of the new CA marker onto some varieties of Aymara and Quechua, which developed a counterpart modeled on the CA strategy.\footnote{This reverse influence is not exclusive to the reportative and hearsay makers, as we pointed out in our analysis. Accordingly, we want to re-emphasize that contact induced change, and in this vein, convergence, is not a one way street. It works simultaneously in both directions. } This seems counter-intuitive at first sight, however it is an even stronger indication that speakers primarily orient to expressing culturally relevant notions and use diverse linguistic resources to do so. The main aim of speakers is communicational success and processing efficiency (“thinking for speaking” as modelled in \citealt{slobin2016thinking}). This leads to a step-by-step structural convergence, as we can observe here, where the distinction between different systems and their structural symmetries yield to the emergence of one joint system.

The description of inference/conjecture does not show a strict one-to one mapping of grammatical categories between the investigated languages, as Aymara shows a fine grained distinction for the source of the deduced knowledge, which is neither present in Quechua, nor transferred to CA. This might raise questions of comparability regarding the extent to which it is possible to support assertions of parallel structures and convergence. Yet, upon closer examination, the context of use for inferentials/conjecturals in all three varieties shows systematic parallelism within the larger notional category. The dynamic situation in CA, with more than just one strategy to match the Aymara and Quechua morphemes, confirms the premises set out in our theoretical framework. Speakers engage in creative language use to express their communicative needs \citep{babel2014doing} and establish competing patterns to carry over their semantic copy (in this case inference/conjecture) at the stage of incipient grammaticalization. Only in another step do they accommodate their “thinking for speaking” \citep{slobin2016thinking} on the linguistic surface: the accumulation of the most successful communicative routine carves out a new potential grammatical maker. Or in terms of Johanson's (\citeyear{RN59}) observation: the semantic functions of copies typically have not reached the same stage of grammaticalization as their models.

In our description of potential and counterfactual marking, the mapping between languages again is not neat and uniform. Both Aymara and Quechua use dedicated suffixes for this category. However, whereas Aymara uses a paradigm with past and present person-marked suffixes, Quechua has just one suffix that can be attached to the inflected verb form. And CA speakers try to reflect these with newly=formed regular constructional patterns that are hardly (or not at all) present in Peninsular Spanish and may compete with each other at this stage of incipient grammaticalization. This also affects how conditional constructions are formed. This is an interesting topic for future research. 

In comparison, we find a quite homogeneous category mapping for the analysis of tense/evidentiality (Section \ref{analysis1:tense-evid}) and hearsay and quotatives (Section \ref{analysis2:hearsay}) relative to the more dynamic picture for inference/conjecture (Section \ref{analysis3:inferentials}) and potential and counterfactual forms (Section \ref{analysis4:potentialCF}). This dynamicity also holds for certainty markers, addressed in outstanding issues (Section \ref{sec:outstandingissues}). This may be due to an early grammaticalization stage in this dynamic contact situation. But we may add that the fact that these categories are closely intertwined for being used in highly nuanced stance-taking regarding the certainty of a state of affairs. The speakers’ needs for a gradual tuning in communicative interaction therefore is a complex issue where the accumulation of the most successful communicative routines in CA are less straight-forward. This type of graduality is less relevant for the (non-)experienced past distinction from Section \ref{analysis1:tense-evid} and hearsay and quotative from Section \ref{analysis2:hearsay}, as their functional opposition is clearly binary (either one hears/experiences something or not) and they are nuanced in discourse in a different way. The emergence of new grammatical strategies for these categories might therefore be quicker.\footnote{We are aware of the multicausality of grammaticalization processes and do not want to exclude other reasons for these differences in the stage of development. Yet this may be a relevant factor that has not received much attention. } This needs to be confirmed by a broader and more thorough analysis.


Finally, we showed that a more detailed elaboration on the uses of the future tenses, progressive and habitual constructions in future studies, as described in Section \ref{sec:outstandingissues}, could contribute to more complete understanding.

In sum, our approach to extend our perspective beyond a narrow grammatical focus lead us to a complex and highly diverse set of linguistic structures relevant for mood and modality in Aymara, Quechua and CA. This diversity and variation do not invalidate our claim to speak of mood and modality as a joint systematic concept in all three languages. To the contrary, it confirms the crucial role of mood and modality in Aymara and Quechua and underscores how speakers re-purpose CA to fit their needs. 


\section{Conclusion and outlook}
\label{conclusion}
We compared Aymara, Quechua and CA mood and modality. We wanted to show that we find contact induced change in these three languages and that the ongoing long-term contact situation profoundly influences the language systems in the long run and leads, step-by-step and in different speeds for different domains, to a conceptual conflation into a joint system of categories beneath the structural surface.

We compared mood and modality because it plays a key role in Andean discourse. The modal dimension can serve as a good example for how categories that we find in Aymara and Quechua, despite the typological distance and without direct grammatical borrowing, get overwhelmingly reflected in the majority language CA. This also holds in the speech of those who do not speak either indigenous language, the minority status of those two languages notwithstanding. In this regard, we hope to have ilustrated that CA can be seen as a diaspora language, because this variety heavily retains the cultural identity of Andean Spanish speakers although the Standard Spanish norm politically clearly dominates.

We based our analyses on the culturally-shaped communicative routines of the speakers and their ingenuity to operationalize the potential of the available linguistic forms in each of the three codes to convey their semantic and pragmatic needs. For this, we took a procedural perspective on mood and modality, an approach that proved successful and brought to light a level of convergence that goes far beyond superficial contact influence and which could be missed when referring to static grammatical descriptions only. 

There remains much work to be done to better understand these processes. Contact induced change is complex and has multiple causes. One of these aspects, which seems crucial for the success of speakers’ creative language use in operationalizing a different language for their culturally shaped communicative routines, is decreased standardization as a sociopolitical factor that promotes it. The lack of consequent efforts to establish school programs for a systematic second language education in Spanish, especially in the rural areas, where pre-school children grow up monolingual in Quechua or Aymara leaves the learning of Spanish as a practice of assimilation by submersion and improvisation (see \cite{SRodriguez2006} for Bolivia). The same goes for Quechua and Aymara which are still not adequately represented in primary school and beyond. From the perspective of contact research this is an important factor, which does not receive sufficient attention, also in other contact situations, as language education might contribute considerably to the openness of a language system for contact induced change\footnote{The question of language education and standardization are of course also highly relevant from a sociolinguistic perspective, regarding language policy, language preservation, and equal opportunities in access to (higher) education for rural populations.} and therefore might explain how the minoritized languages Aymara and Quechua remain highly influential.

\sloppy
\printbibliography[heading=subbibliography,notkeyword=this]
\end{document}
