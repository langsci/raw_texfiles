\documentclass[output=paper,hidelinks]{langscibook}
\ChapterDOI{10.5281/zenodo.7446969}

\author{Alexander Andrason\affiliation{Stellenbosch University}}

\title{Complexity of endangered minority languages: The sound system of Wymysiöeryś}
\abstract{This paper demonstrates that Wymysiöeryś – a severely endangered Germanic minority language – exhibits remarkable complexity despite its moribund status. By analyzing twelve phonetic/phonological properties, the author concludes that the complexity of Wymysiöeryś is greater, both locally and globally, than that of two control languages: Middle High German and Modern Standard German. In most cases, the surplus of complexity attested is attributed to contact with the dominant language, Polish.}
\IfFileExists{../localcommands.tex}{
 \addbibresource{../localbibliography.bib}
 \usepackage{langsci-optional}
\usepackage{langsci-gb4e}
\usepackage{langsci-lgr}

\usepackage{listings}
\lstset{basicstyle=\ttfamily,tabsize=2,breaklines=true}

%added by author
% \usepackage{tipa}
\usepackage{multirow}
\graphicspath{{figures/}}
\usepackage{langsci-branding}

 
\newcommand{\sent}{\enumsentence}
\newcommand{\sents}{\eenumsentence}
\let\citeasnoun\citet

\renewcommand{\lsCoverTitleFont}[1]{\sffamily\addfontfeatures{Scale=MatchUppercase}\fontsize{44pt}{16mm}\selectfont #1}
   
 %% hyphenation points for line breaks
%% Normally, automatic hyphenation in LaTeX is very good
%% If a word is mis-hyphenated, add it to this file
%%
%% add information to TeX file before \begin{document} with:
%% %% hyphenation points for line breaks
%% Normally, automatic hyphenation in LaTeX is very good
%% If a word is mis-hyphenated, add it to this file
%%
%% add information to TeX file before \begin{document} with:
%% %% hyphenation points for line breaks
%% Normally, automatic hyphenation in LaTeX is very good
%% If a word is mis-hyphenated, add it to this file
%%
%% add information to TeX file before \begin{document} with:
%% \include{localhyphenation}
\hyphenation{
affri-ca-te
affri-ca-tes
an-no-tated
com-ple-ments
com-po-si-tio-na-li-ty
non-com-po-si-tio-na-li-ty
Gon-zá-lez
out-side
Ri-chárd
se-man-tics
STREU-SLE
Tie-de-mann
}
\hyphenation{
affri-ca-te
affri-ca-tes
an-no-tated
com-ple-ments
com-po-si-tio-na-li-ty
non-com-po-si-tio-na-li-ty
Gon-zá-lez
out-side
Ri-chárd
se-man-tics
STREU-SLE
Tie-de-mann
}
\hyphenation{
affri-ca-te
affri-ca-tes
an-no-tated
com-ple-ments
com-po-si-tio-na-li-ty
non-com-po-si-tio-na-li-ty
Gon-zá-lez
out-side
Ri-chárd
se-man-tics
STREU-SLE
Tie-de-mann
} 
 \togglepaper[8]%%chapternumber
}{}

\shorttitlerunninghead{Complexity of endangered minority languages}
\begin{document}
\shorttitlerunninghead{Complexity of endangered minority languages}
\maketitle


\section{Introduction} %1.
\label{AAsect1}
Language endangerment, language shift, language obsolescence, and language death all have “considerable impact” on the \textit{structure} of the languages affected \citep[110]{palosaari_structural_2011}.\footnote{The present article emerged as a result of my PhD dissertation, \textit{Polish borrowings in Wymysiöeryś: A formal linguistic analysis of Germano-Slavonic language contact in Wilamowice} \citep{Andrason2021}.} The most pervasive form of impact is the apparent simplification and impoverishment of the \textit{grammar} of endangered and moribund languages (see \citealt[590--591]{dorian_grammatical_1973}, \citeyear[85]{dorian_language_1980}, \citealt[9]{silva-corvalan_study_1995}, \citealt[256]{mesthrie_introducing_2009}, \citealt[110--117]{palosaari_structural_2011}, \citealt[101, 111, 118]{sallabank_diversity_2012}, \citeyear[126]{sallabank_attitudes_2013}, \citealt[2]{filipovic_introduction_2016}, \citealt[43]{aikhenvald_grammars_2007}, \citealt[294, 297]{meakins_birth_2019}) – whether it is phonetics/phonology, morphology, syntax, or vocabulary (\citealt[591]{dorian_fate_1978}, \citeyear[85]{dorian_language_1980}, \citealt[110, 113--115]{palosaari_structural_2011}).\footnote{Simplification typically implies reduction or loss of marked features (\citealt[113]{palosaari_structural_2011}, \citealt[126]{sallabank_attitudes_2013}) due to regularization and overgeneralization (\citealt[9--10]{silva-corvalan_study_1995}, \citealt[113]{palosaari_structural_2011}, \citealt[126]{sallabank_attitudes_2013}, \citealt[2]{filipovic_introduction_2016}). The features that tend to be reduced or lost involve: phonological contrasts (\citealt[85]{dorian_language_1980}, \citealt[113]{palosaari_structural_2011}), morphological marking and distinctions (\citealt[85]{dorian_language_1980}, \citealt[115]{palosaari_structural_2011}, \citealt[297]{meakins_birth_2019}), synthetic structures (which are replaced by analytic constructions) (\citealt[10]{silva-corvalan_study_1995}, \citealt[115]{palosaari_structural_2011}, \citealt[297]{meakins_birth_2019}), syntactic (\citealt[85]{dorian_language_1980}, \citealt[115]{palosaari_structural_2011}) and stylistic patterns (\citealt[85]{dorian_language_1980}, \citealt[115]{palosaari_structural_2011}), as well as vocabulary \citep[118]{sallabank_diversity_2012}.} Even though these reductive processes are especially patent and the most rampant in the varieties used by semi-speakers or rusty speakers, who do not learn the language fully in an intergenerational transmission and/or do not use it for the greater parts of their lives (see \citealt{palosaari_structural_2011, grinevald_speakers_2011}), simplification and impoverishment also seem to affect speakers whose language acquisition was uninterrupted and/or who have spoken the language relatively continuously. Overall, an endangered or dying language viewed as \textit{a holistic} (though not uniform) \textit{linguistic phenomenon} apparently reduces its complexity generation after generation (\citealt[203]{austin_structural_1986}; see also \citealt{dorian_fate_1978, dorian_language_1980, swiggers_two_2007, sallabank_diversity_2012, sallabank_attitudes_2013, palosaari_structural_2011, filipovic_introduction_2016}).\footnote{It should be noted that many scholars speak about the simplification of endangered and dying languages \textit{in general} (\citealt[85]{dorian_language_1980}, \citealt[203]{austin_structural_1986}, \citealt[24]{swiggers_two_2007}, \citealt[256]{mesthrie_introducing_2009}, \citealt[110, 112]{palosaari_structural_2011}, \citealt[118]{sallabank_attitudes_2013}) rather than referring to the variety that is \textit{only} used by semi-speakers. After all, it would not be surprising that, similar to imperfect L2 speakers, semi-speakers would not make use of the entire linguistic repertoire available in the language.} This especially occurs in destabilized or unbalanced types of language contact in which the endangered language is gradually displaced by the dominant code (\citealt[47]{aikhenvald_grammars_2007}, \citealt{meakins_birth_2019}).\footnote{Certainly, language endangerment and language contact are not the same phenomenon. However, although a language may die “without language shift” \citep[201]{austin_structural_1986}, most cases of language endangerment and language death “involve language \textit{replacement} or shift” \citep[201]{austin_structural_1986}. This presupposes contact between the languages involved and bilingualism of endangered language speakers, at least at the population level.}

Simplification is a common phenomenon in language contact \citep{mcwhorter_identifying_1998, mcwhorter_worlds_2001, matras_grammatical_2007}. 
%\citealt{miestamo_complexity_2006}, \citealt{sinnemaki_complexity_2008}, \citealt{parkvall_simplicity_2008}, \citealt[294]{meakins_birth_2019}) 
It is typical of pidgins (\citealt{muhlhausler_pidginisation_1977, muhlhausler_pidgin_1986, trudgill_dialects_1986, trudgill_dual-source_1996}, \citeyear[306--308]{trudgill_contact_2010}, \citealt{kusters_linguistic_2003}, \citealt[190]{siegel_pidginscreoles_2008}, \citealt[321--322]{juvonen_complexity_2008})
%\citealt[704]{childs_language_2010}, \citealt[624]{joseph_language_2010}, \citealt{parkvall_pidgins_2013}, \citealt[15, 29]{velupillai_pidgins_2015}) 
and, albeit to a lesser extent and not without contest (\citealt{degraff_against_2003, degraff_linguists_2005}, \citealt[12--14]{ansaldo_deconstructing_2007}, \citealt[300]{hammarstrom_complexity_2008}, \citealt{bakker_creoles_2011}) of creoles \citep{mcwhorter_identifying_1998, mcwhorter_worlds_2001, mcwhorter_defining_2005, seuren_semantic_2001, parkvall_simplicity_2008}. 
%\citealt{velupillai_pidgins_2015}). 
The noticeable exceptions are mixed languages, which may maintain or even increase complexity (\citealt[288, 305]{matras_language_2009}, \citealt{meakins_mixed_2013}, \citealt[301--307, 329--330, 402]{velupillai_pidgins_2015}, \citealt[296--297, 326--327]{meakins_birth_2019}) and layered languages, which may exhibit traces of both simplification and complexification \citep[42--43]{aikhenvald_grammars_2007}.\footnote{Although pre\nobreakdash-pidgins, stabilized pidgins, expanded pidgins, creoles, and post-creole varieties are often simpler than the feeding languages \citep{velupillai_pidgins_2015}, they tend to – in the above order – gradually increase their complexity (\citealt[5--11]{muhlhausler_pidgin_1986}, \citealt[281]{parkvall_simplicity_2008}, \citealt[306]{trudgill_contact_2010}, \citealt{velupillai_pidgins_2015}). Thus, both simplification and complexification are important factors in the development of pidgins and creoles (\citealt[258]{heine_language_2005}, \citealt[309]{trudgill_contact_2010}), operating with distinct intensity at different stages of the pidgin-creole life cycle.}

The present paper examines whether the severe endangerment of a language – specifically, a minority language that has nearly been replaced by another code and drifts towards a seemingly imminent death – is correlated with structural simplicity; or, inversely, whether severely endangered languages may exhibit remarkable complexity due to the transfer of elements from the dominant code. The language system under analysis is Wymysiöeryś – a colonial East Central German variety that has interacted with the dominant Polish language for more than seven centuries (\citealt[498]{putschke_ostmitteldeutsch_1980}, \citealt[497--498]{wiesinger_deutsche_1980}, \citealt{wicherkiewicz_making_2003}) and that, due to increasingly aggressive Polonization, has been, more or less, endangered since the end of World War I in 1918 \citep{neels_language_2016}, currently finding itself on the verge of total extinction \citep{andrason_grammar_2016}.\footnote{Wymysiöeryś is classified as “nearly extinct”, in the Extended Graded Intergenerational Disruption Scale and “moribund” or “severely endangered” by \citet{moseley_atlas_2010}. The language is currently used by less than fifty elderly native speakers (\citealt[73]{ritchie_language_2016}, \citealt[91]{chromik_wilamowice_2016}) whose number rapidly decreases every year \citep{andrason_grammar_2016, Andrason2021}.} I will study the linguistic repertoires of fluent speakers – the remaining 65 native Wymysiöeryś speakers born between 1913 and 1993 who have acquired the language in uninterrupted intergenerational transmission and have spoken it relatively continuously.\footnote{The number of these speakers was 65 at the beginning of the 21\textsuperscript{st} century, when I started my research on Wymysiöeryś. Unfortunately, many of them have passed away in the interim (cf. footnote 6 above). For the list of these speakers, their names, and dates of birth see \citet{Andrason2021}.} Inversely, the idiolects of semi-speakers -- who do exhibit radical simplification and impoverishment processes but have no bearing on the transmission of Wymysiöeryś to the younger generations and thus the structure of the language as such -- are not taken into consideration.

This study centers on the idea of absolute complexity \citep{kusters_complexity_2008, dahl_growth_2004, dahl_testing_2009, miestamo_grammatical_2008, miestamo_implicational_2009}. To calculate absolute complexity, I deploy the concept of effective complexity \citep{gell-mann_what_1995, gell-mann_effective_2004} %\citealt{mcwhorter_defining_2005}, \citeyear{mcwhorter_language_2007}, \citeyear{mcwhorter_oh_2009}, \citealt{miestamo_grammatical_2008}, \citeyear{miestamo_implicational_2009}, \citealt{sinnemaki_complexity_2008}, \citeyear{sinnemaki_complexity_2009}, \citeyear{sinnemaki_language_2011} \citealt{parkvall_simplicity_2008})
and take into account two main criteria: distinctiveness and economy \citep{miestamo_complexity_2006, miestamo_grammatical_2008, sinnemaki_complexity_2008, sinnemaki_complexity_2009, sinnemaki_language_2011, parkvall_simplicity_2008}. I focus on local complexities pertaining to twelve distinct phonetic/phonological features, subsequently combining them into a global value representing the complexity of the entire sound-system module \citep{miestamo_complexity_2006, miestamo_feasibility_2006, miestamo_implicational_2009, deutscher_overall_2009, sinnemaki_complexity_2014}. The complexity of Wymysiöeryś, both feature-locally and module-globally, will not be quantified autonomously, but will rather be narratively estimated in relation to two control systems: (a mother-system) Middle High German and (a sister-system) Modern Standard German (cf. \citealt{deutscher_overall_2009} and \citealt{dahl_testing_2009}). In instances where Wymysiöeryś exhibits a surplus of information -- i.e. a positive difference in complexity when compared to the control languages -- I will examine whether this surplus can be attributed to Polish influence, either in whole or in part.

The article will be organized as follows: in Section \ref{sec:wymsorys:2}, I will explain the theoretical framework underlying my study or the manner of complexity measurement adopted. In Section \ref{sec:wymsorys:3}, I will compare the complexities of Wymysiöeryś with those of Middle High German and Modern Standard German – first locally and next globally. In Section \ref{sec:wymsorys:4}, I will verify whether, and how intensively, the surplus of information attested in Wymysiöeryś draws on Polish – again, first locally and next globally. In Section \ref{sec:wymsorys:5}, I will draw conclusions and propose lines of future research.

\section{Framework – measuring language complexity}\label{sec:wymsorys:2}

The approach adopted in this paper is one of the most common and theoretically least problematic manners of analyzing the complexity of natural languages. It draws on the idea of complexity that is:
(a) epistemologically absolute,
(b) computationally effective,
(c) built around the criteria of distinctiveness and economy,
(d) relational, and
(e) primarily local.
Below, I explain these five ideas in detail.

\begin{enumerate}

\item From an epistemological perspective, the type of complexity analyzed in this research is absolute. This complexity type pertains to the system viewed as ``an autonomous entity'' in disconnection from the observer \citep[4]{kusters_complexity_2008}. Absolute complexity is therefore the ``objective property of the [language] system'' (\citealt[23]{miestamo_grammatical_2008}, \citealt{dahl_growth_2004}), regardless of the characteristics of its users, whether speakers, hearers, first-language or second-language learners.\footnote{Inversely, I am not concerned with relative complexity or complexity experienced by users of a given language (see \citealt{kusters_linguistic_2003}, \citealt[23]{miestamo_grammatical_2008}, \citeyear[81--82]{miestamo_implicational_2009}). Some scholars do not regard relative complexity as complexity \textit{sensu stricto} \citep[39--40]{dahl_growth_2004} instead preferring the terms ``difficulty'' and ``cost'' (\citealt[27]{miestamo_grammatical_2008}, \citealt{lindstrom_language_2008}).}

\item To compute absolute complexity, I will deploy the concept of effective complexity otherwise referred to as Gell-Mann complexity \citep{gell-mann_what_1995, gell-mann_effective_2004}. This complexity measure refers to the amount of information necessary to describe non-randomness within a system – the longer the description of regularities is, the more complex a system is \citep[99]{mitchell_complexity_2009}. In linguistics, this corresponds to the amount of information required to describe rules governing a language (\citealt{mcwhorter_defining_2005, mcwhorter_language_2007, mcwhorter_oh_2009}, \citealt[25]{miestamo_grammatical_2008}, \citeyear[81--82]{miestamo_implicational_2009}, \citealt{sinnemaki_complexity_2008, sinnemaki_complexity_2009, sinnemaki_language_2011, parkvall_simplicity_2008}).\footnote{The other common manner of quantifying complexity is algorithmic complexity, also referred to as Kolmogorov complexity. This manner of quantification calculates disorder or randomness, e.g. the ``amount of surprise'' contained in a message \citep[97--98]{mitchell_complexity_2009} or the ``difficulty of description'' \citep[52]{shalizi_methods_2006}.}

\item
In the quantification of effective complexity, I will take into consideration two main criteria: a distinctiveness criterion and an economy criterion \citep{miestamo_complexity_2006, miestamo_feasibility_2006, miestamo_grammatical_2008, sinnemaki_complexity_2008, sinnemaki_complexity_2009, sinnemaki_language_2011, parkvall_simplicity_2008}. According to the distinctiveness criterion, the complexity of a language increases with a greater categorical diversity, i.e. with more distinctions being made and more domains being overtly specified. According to the economy criterion, complexity increases with a greater formal diversity, i.e. with more manners of encoding of a given category or distinction.\footnote{Thus, synonymy, redundancy, allomorphy, and free variations increase complexity \citep{mcwhorter_language_2007, mcwhorter_why_2008}. Similarly, exceptions contribute to the increase in complexity as they constitute additional rules (cf. \citealt[29]{hammarstrom_complexity_2008}; see also \citealt{mcwhorter_language_2007, mcwhorter_why_2008}).} 

\item
The degree of complexity of Wymysiöeryś will be estimated in relation to the complexities of two diachronic and dialectal control-systems (cf. \citealt{deutscher_overall_2009} and \citealt{dahl_testing_2009}): Middle High German and Modern Standard German. Middle High German is regarded as a non-distant ancestor of Wymysiöeryś or its mother-language (\citealt{kleczkowski_dialekt_1920}, \citealt[496, 498]{wiesinger_deutsche_1980}, \citeyear[911]{wiesinger1983deutsche}, \citealt{morciniec_flamische_1984}, \citealt[308]{zieniukowa_-pyjter-jaska_1997}, \citeyear[492--493]{zieniukowa_sutuacja_2001}, \citealt[200]{wicherkiewicz_wir_1998}, \citealt[19]{wicherkiewicz_researching_2016}, \citealt[132]{zak_influence_2016}).\footnote{The selection of Middle High German and Modern Standard German as pre-contact and non-contact ``control'' languages, also stems from the availability of extensive and detailed grammatical studies dedicated to these languages, which render the comparison fully operational.} Modern Standard German is a closely related West Germanic sister-language \citep{chromik_halcnovian_2013}.% \citealt{chromik_Wymysiöeryświlamowicean_2013}).%; see also \citealt{lasatowicz_deutsche_1994}, \citealt{mojmir_worterbuch}, \citealt{kleczkowski_dialekt_1920}, and \citealt{ritchie_considerations_2012}). 
~Neither of the control systems has experienced intense influence from Polish or other Slavonic languages. 

\item
The estimation of effective complexity will mainly be conducted at a local level. I will determine how many of the selected phonetic/phonological features (categories) are instantiated in the tested languages (distinctiveness) and how many expression manners of each feature there are (economy).\footnote{The features analyzed in here are principally phonetic features. Therefore, I will use square bracket notation when writing about the sounds of Wymysiöeryś and the two control languages. However, I will occasionally refer to phonology as well. This phonetic orientation is motivated. First, scholarship still lacks a phonological analysis of the Wymysiöeryś language – all descriptions being virtually phonetic. Second, more generally, phonology is much more theory-dependent than phonetics (see that, in Polish, under certain theoretical premises, [i] and [ɘ̟/ɨ] are treated as a single phoneme despite being clearly distinct from an articulatory perspective and being indeed perceived as two distinct vowels by native speakers). Accordingly, all the inventories from other authors are phonetic (for instance, with respect to Standard Modern German see \citealt{johnson_exploring_2008, fagan_german_2009, obrein_german_2016}; see also \citealt[352]{eisenberg_german_1994, dodd_modern_2003}) unless stated otherwise (for Modern High German see \citealt{russ_german_1994, wiese_phonology_1996}, and \citealt{fox_structure_2005}).} With regard to each feature, the analysis will be captured in a narrative form “drawing on the concepts of (approximate) equality ${\approx}$ (\textit{x} ${\approx}$ \textit{y} means that \textit{x} is approximately equal to \textit{y}), inequality ${\leq}$ (\textit{x} ${\leq}$ \textit{y} means that \textit{x} is less than or equal to \textit{y}), [and] strict inequality < (\textit{x} < \textit{y} means that \textit{x} is less than \textit{y})” \citep{Andrasonforthcoming}.\footnote{Given the narrative approach adopted, the symbols ${\approx}$, ${\leq}$, and < reflect three types of relationships between \textit{x} and \textit{y}: similarity, minimal difference, and substantial difference (cf. \citealt{Andrasonforthcoming}).} However, the analysis of complexity will not be limited to separated local domains. On the contrary, I will combine the local complexities into a global value that indicates the complexity of the entire sound-system module of each of the three languages. 
\end{enumerate}

The approach used in this paper has its limitations, which are inherent to and unavoidable in complexity studies. To begin, contemporary science lacks a single, comprehensive, all-purpose complexity measure. Instead, a variety of measurement methods -- at least 48 as observed by \citet{edmonds_syntactic_1999} -- coexist simultaneously, differing in what is calculated and how the calculation is executed. Crucially, the various methods yield different complexity results (for an overview consult \citealt{peliti_measures_1988, badii_complexity_1997, rescher_complexity_1998, edmonds_syntactic_1999, shalizi_methods_2006}, and \citealt{mitchell_complexity_2009}). Similarly, linguists have not reached an agreement as for how language complexity should be measured and compared \citep[7]{newmeyer_introduction_2014}. Virtually every scholar develops an at least minimally different method of measurement. However, neither this disagreement nor the excessive proliferation of measurement techniques is surprising. They rather reflect the fact that natural languages constitute genuine complex systems, which renders any modeling fragmentary, provisional, and tentative, irrespective of how potent and sophisticated it is (\citealt[3]{cilliers_complexity_2013}, \citealt{andrason_grammar_2016}). With regard to the method adopted in this paper, a number of objections could be raised. First, absolute complexity is theory-oriented and, perhaps, heavily theory-dependent (\citealt[24]{miestamo_grammatical_2008}, \citealt[5, 8]{kusters_complexity_2008}, \citealt[289]{hammarstrom_complexity_2008}). It depends on the theories of language in which the description of grammar and its analysis are conducted (\citealt[27]{miestamo_grammatical_2008}, \citealt[5, 7--8]{kusters_complexity_2008}, \citealt[289]{hammarstrom_complexity_2008}). Second, local complexity allows for various manners of granularity. That is, each local complexity can be fragmentized into more atomic types, which inversely means that each local complexity constitutes global complexity from the perspective of more fragmentary levels of analysis. As a result, local but modular complexities are not free from three further problems typical of global complexity: sampling problems, commensurability problems, and modularity problems. Third, given the quantitative depth of a language, the account of all the details included in a single module and their quantification are unfeasible, both theoretically and practically (\citealt[30]{miestamo_feasibility_2006}, \citeyear{miestamo_language_2008}). Linguists rather determine a more or less restrictive sample of studied phenomena, limiting themselves to analyzing a set of distinctions or categories – rather than all of them. No universal methods of such delimitation exist and often detail-ness is determined by utilitarian aspects such as the aim of the analysis to be undertaken, the (maximal) feasibility of the research, and the scientist's theoretical paradigm \citep[248]{deutscher_overall_2009}. Fourth, even for modular complexity, it is uncertain how to commensurate the different features found in a single module and represent them in identical numerical terms, since each such term refers to qualitatively different phenomena and encapsulates, in principle, incomparable properties (see \citealt[30]{miestamo_complexity_2006, miestamo_feasibility_2006, miestamo_grammatical_2008}, \citeyear[83]{miestamo_implicational_2009}). Fifth, dividing the module into distinct features or categories presupposes the highly problematic division of language into separate parts -- for which local complexities are subsequently calculated. All those limitations -- of which linguists are well aware -- render the measurement of local complexities and that of modular complexity (i.e. global complexity at a module level), as well as their comparison across languages extremely difficult, if not elusive. My method partially responds to these problems. To mitigate theory-dependence, I am eclectic with regard to theory underlying the description of features. By taking into account a number of studies that follow different theoretical principles, I purposefully average the complexities inferred from the available descriptions. To mitigate granularity and sampling problems, the categories selected coincide with categories distinguished in general phonetic/phonological studies and with phonetic/phonological features usually described in comparative works on the Germanic and Slavonic language families \citep{rothstein_polish_1993, jacobs_yiddish_1994, sussex_slavic_2006, harbert_germanic_2007}. Lastly, to mitigate the commensurability problem, I use a narrative method instead of a strictly numerical one. The problem of modularity cannot be mitigated, as the division of the sound-system module into separated units is presupposed by the method used in my study.

Additionally, there are two problems related to the control systems used in estimating the changes in complexity of Wymysiöeryś. First, the data related to the sound systems of Middle High German and Modern Standard German are secondary and draw on the studies presented by other scholars. Although all these studies are generally recognized in Germanic scholarship as authoritative, the information presented in them need not be exhaustive. This especially applies to Middle High German as this language may have contained some features that have not been reported thus far. The results of my comparison of Wymysiöeryś with Middle High German are inevitably contingent on the other linguists' views of Middle High German and their interpretation of direct textual evidence. This type of risk is unavoidable in typological and diachronic studies when one must rely on others' data and analyses. Second, Middle High German itself is an umbrella term that encompasses a number of German varieties that (a) were spoken between the 11\textsuperscript{th}/12\textsuperscript{th} and 14\textsuperscript{th}/15\textsuperscript{th} centuries; (b) were successors of Old High German; and (c) entirely or partially underwent the second consonant shift. This means that the grammatical repertoire of Middle High German is possibly richer than the repertoire of any single German variety that was spoken between the 11\textsuperscript{th}/12\textsuperscript{th} and 14\textsuperscript{th}/15\textsuperscript{th} centuries. To put it simply, while Wymysiöeryś is a single variety of East Middle German, Middle High German is a conglomerate of many varieties. Again, this problem is typical of studies that compare modern languages with old, classical, and extinct languages (see \citealt{Andrasonforthcoming}).

\section{Evidence} %3. /
\label{sec:wymsorys:3}
The Wymysiöeryś language described below is a heterogenous and internally diversified linguistic system. Most sources employed make reference to Wymysiöeryś spoken at the time of its near extinction, i.e. at the end of the 20\textsuperscript{th} and the beginning of the 21\textsuperscript{st} century \citep{lasatowicz_deutsche_1994, wicherkiewicz_wir_1998, wicherkiewicz_making_2003, zieniukowa_sutuacja_2001, ritchie_considerations_2012, weckwerth_polands_2015, zak_influence_2016, zak2019pewnym}. When describing this modern Wymysiöeryś variety, I will widely draw on an original database developed over the course of more than fifteen years of field-work activities conducted by Tymoteusz Król and myself. This primary evidence has been the foundation of several of the papers that I have published alone or in collaboration with Król (in particular \citealt{andrason_polish_2014, andrason_szkic_2014, andrason_vilamovicean_2015, andrason_grammar_2016}). I will refer extensively to the findings of those studies.\footnote{The informants who have participated in the empirical research conducted by Król and myself, and whose language is reflected in previously published articles as well as in this paper, are (or were) fluent native speakers of Wymysiöeryś (see Section \ref{AAsect1}).} Additionally, the data presented will draw on three works dedicated to pre-war Wymysiöeryś – the stage at which true endangerment had already began, even though it was still less pronounced than it is currently (\citealt{kleczkowski_dialekt_1920, kleczkowski_dialekt_1921, mojmir_worterbuch}, and \citealt{wicherkiewicz_making_2003}, who describes the language used by Florian Biesik in his poems, the majority of which were written between 1920 and 1924). The evidence related to Middle High German and Modern Standard German is secondary and draws on canonical studies dedicated to these two languages and the (West) Germanic linguistic family. In particular, regarding Middle High German: \citet{wright_middle_1917, de_boor_mittelhochdeutsche_1973, simmler_phonetik_1985, paul_mittelhochdeutsche_2007, hennings_einfuhrung_2012}, and \citet{hall_underlying_2017}. Regarding Modern Standard German: \citet{hall_syllable_1992, hall_phonologie_2000, russ_german_1994, eisenberg_german_1994, wiese_phonology_1996, dodd_modern_2003, fox_structure_2005, johnson_exploring_2008, fagan_german_2009, caratini_vocalic_2009}, and \citet{obrein_german_2016}, and generally (West) Germanic: \citet{iverson_aspiration_1995, iverson_glottal_1999, iverson_laryngeal_2003, iverson_germanic_2008, goblirsch_voice_1997, goblirsch_lenition_2018, harbert_germanic_2007, van_der_hoek_palatalization_2010}, and \citet{van_oostendorp_germanic_2019}.

In this section, I will first determine the value of the twelve local complexities of Wymysiöeryś in relation to the two control systems (Section \ref{sec:wymsorys:3.1} -- Section \ref{sec:wymsorys:3.12}). Next, I will estimate the relational complexity of the three languages at the global level of the sound-system module (Section \ref{sec:wymsorys:3.13}).

\subsection{Monophthongs}\label{sec:wymsorys:3.1}


Depending on the study, the number of monophthongs in Wymysiöeryś varies. The highest number attested is 15. The lowest number is 9. Most analyses distinguish well above ten vowels. Maximal systems have been proposed by \citet{andrason_grammar_2016, kleczkowski_dialekt_1920} and \citet{mojmir_worterbuch}. \citet[20]{andrason_grammar_2016} identify fifteen core vowels: [i], [ɪ], [e], [ɛ], [a], [ɑ], [o], [ɔ], [u], [y], [ʏ], [ɘ̟], [ø], [œ], and [ə]. \citet[11--12, 171]{kleczkowski_dialekt_1920} and \citet[xii--xiii]{mojmir_worterbuch} recognize thirteen vowels: [i], [ɪ] [e], [ɛ], [a], [ɑ], [o], [ɔ], [u], [ʊ] [y], [ʏ], [ə].\footnote{The phonetic interpretation in IPA terminology is mine. \citet{kleczkowski_dialekt_1920} offers detailed descriptions which allow for such an interpretation. Given the limitations in space, I will not provide examples of phonetic/phonological features in words and/or constructions. These may be found in the works referred to in each section.} Systems of twelve monophthongs are proposed by \citet{lasatowicz_deutsche_1994, zieniukowa_sutuacja_2001, wicherkiewicz_making_2003}, and \citet{andrason_szkic_2014}. \citet[32--41]{lasatowicz_deutsche_1994} distinguishes: [i], [e], [ɛ], [a], [ɑ], [o], [ɔ], [u], [y], [ʏ], [ø], [ə]. \citet[499--500]{zieniukowa_sutuacja_2001} distinguish: [i], [e], [ɛ], [a], [ɑ], [o], [u], [y], [ʏ], [ø], [ə], and [ɨ]. \citet[407]{wicherkiewicz_making_2003} distinguishes the same set of vowels merely replacing [o] with [ɔ]. \citet[126--127]{andrason_szkic_2014} distinguishes: [i], [ɪ], [e], [ɛ], [a], [ɑ], [o], [ɔ], [u], [y], [ʏ], and [ø]. The most reduced systems of monophthongs are postulated by \citet{weckwerth_polands_2015} and \citet{ritchie_considerations_2012} who discern nine vowels: [i], [ɨ], [e], [ʏ], [ø], [a], [ɑ], [ɔ], and [u].\footnote{\citet{lasatowicz_deutsche_1994} limits his set of vowels to eight sounds of an uncertain phonetic interpretation: \textit{i}, \textit{e}, \textit{o}, \textit{ó}, \textit{ö}, \textit{u}, \textit{a}, \textit{y}. Given the lack of linguistic training of its author, this system cannot be regarded as trustful (see a similar observation in \citealt{wicherkiewicz_making_2003}).}



The vocalic inventory of Middle High German consists of at least nine basic short monophthongs: \textit{a} [a], \textit{e} [e], \textit{ë} (transcribed as [ë]), \textit{ä} [ɛ], \textit{i} [i], \textit{o} [o], ö [ø], \textit{u} [u], and \textit{ü} [y] (\citealt[2--5]{wright_middle_1917}, \citealt[1131, 1133]{simmler_phonetik_1985}, \citealt[36--37, 41]{de_boor_mittelhochdeutsche_1973}, \citealt[62--63, 87--97]{paul_mittelhochdeutsche_2007}, \citealt[9]{hall_underlying_2017}, \citealt[69]{schmidt_einfuhrung_2017}).\footnote{On the status of the vowels \textit{e} (closed), \textit{ë} (mid-open), and \textit{ä} (open), consult \citeauthor{simmler_phonetik_1985} (\citeyear[1132, 1134]{simmler_phonetik_1985}; see also \citealt[2--5]{wright_middle_1917}, \citealt[36--37, 41]{de_boor_mittelhochdeutsche_1973}, \citealt[87--91]{paul_mittelhochdeutsche_2007}). In a few studies, the number of vowels is reduced to seven \citep[184--185]{caratini_vocalic_2009}.} Additionally, the reduced vowel \textit{e} [ə] was used in unaccented syllables (\citealt[3]{wright_middle_1917}, \citealt[1133]{simmler_phonetik_1985}, \citealt[9]{hall_underlying_2017}). The language also had long vowels (see Section \ref{sec:wymsorys:3.5} below), of which some may have exhibited slightly different qualities when compared to their short counterparts, apart from distinctive quantity (\citealt[3]{wright_middle_1917}, \citealt[185]{caratini_vocalic_2009}; cf. \citealt[9]{hall_underlying_2017} and \citealt[69]{schmidt_einfuhrung_2017}). This would increase the total number of monophthongs to maximally eighteen vowels. The vocalic system of Modern Standard German includes 15 vowels of different qualities: [iː], [ɪ], [yː], [ʏ], [eː], [øː], [ɛ(ː)], [œ], [uː], [ʊ], [oː], [ɔ], [a(ː)], [ə] and [ɐ] 
(\citealt[7, 17]{fagan_german_2009}, \citealt[109--110]{johnson_exploring_2008}, \citealt[17--19]{obrein_german_2016}; see also \citealt[119]{russ_german_1994}, \citealt[19--21]{wiese_phonology_1996}, and \citealt[35, 41]{fox_structure_2005} who analyze the vocalic phonemes of Modern Standard German). In some studies, additional sounds [ɑ] (\citealt[350]{eisenberg_german_1994}, \citealt[3]{dodd_modern_2003}, \citealt[110]{johnson_exploring_2008}, \citealt[71]{caratini_vocalic_2009}) and [æː], as different in quality from [ɛ], are distinguished \citep[35--36, 38]{fox_structure_2005}.\footnote{Note that \citet{fox_structure_2005} analyzes phonemes. Sporadically, in unassimilated English borrowings, one finds two additional vowels, namely [æ] and [ʌ] (\citealt[53]{fox_structure_2005}, \citealt[73]{caratini_vocalic_2009}).} 


\largerpage
Overall, monophthongs (\textit{M}) are typical features of both Wymysiöeryś and the two control languages. Quantitatively, the number of monophthongs in Wymysiöeryś is (with minor disturbances) similar to that of Modern Standard German and Middle High German. Therefore, the complexity of monophthongs in the three languages can be viewed as approximately equal: \textit{M\textsuperscript{W}} ${\approx}$ \textit{M\textsuperscript{MHG} }and \textit{M\textsuperscript{W}} ${\approx}$ \textit{M\textsuperscript{MSG}}.\footnote{The following abbreviations will be used in this paper: W – Wymysiöeryś; MHG – Middle High German; MSG – Modern Standard German.}


\subsection{Diphthongs}\label{sec:wymsorys:3.2}


The number of Wymysiöeryś diphthongs varies between nine and six.\footnote{The largest set of fourteen diphthongs is posited by \citet[270]{latosinski_monografia_1909}: \textit{ia}, \textit{iu}, \textit{iy}, \textit{ie}, \textit{uö}, \textit{uy}, \textit{oe}, \textit{oi}, \textit{ou}, \textit{ei}, \textit{ae}, \textit{aei}, \textit{au}, \textit{yi}. (cf. \citealt[411]{wicherkiewicz_making_2003}). As in Section \ref{sec:wymsorys:3.1}, I will disregard this system in my discussion.} \citet[12--13]{kleczkowski_dialekt_1920} and \citet[xiii--xiv]{mojmir_worterbuch} distinguish nine diphthongs: [i̯eː], [i̯ɛː], [aj], [oj], [ə(ː)j], [ou], [au], [uøː] / [uːø], and [uːə].\footnote{This system could be extended to ten sounds if [uøː] and [uːø] are viewed as different diphthongs.} Systems of eight diphthongs are proposed by \citet{lasatowicz_deutsche_1994, wicherkiewicz_making_2003}, and \citet{zieniukowa_sutuacja_2001}. Specifically, \citet[33, 40--41]{lasatowicz_deutsche_1994} distinguishes: [i(ː)\textsuperscript{ə}], [eː\textsuperscript{i}], [a͡e], [a͡o], [ɔ͡ø], [ø(ː)\textsuperscript{ə}], [y(ː)\textsuperscript{ə}], and [u(ː)\textsuperscript{ə}]. \citet[407--408]{wicherkiewicz_making_2003} distinguishes: [i(ː)\textsuperscript{ə}], [eː\textsuperscript{i}], [ae], [ao], [ɔø], [ø(ː)\textsuperscript{ə}], [y(ː)\textsuperscript{ə}] and [u\textsuperscript{ə}]. \citet[499--500]{zieniukowa_sutuacja_2001} distinguish: [ai̯], [ye], [ɪ̯ɨ], [ɨj], [au], [øe], [øo], and [ue]. Systems of six diphthongs are formulated by \citet{andrason_grammar_2016} and \citet{weckwerth_polands_2015}. \citeauthor{andrason_grammar_2016}'s system (\citeyear[21]{andrason_grammar_2016}) contains [ai̯], [ei̯̯], [ɔi̯], [œʏ̯], [i̯ø] and [ɪ̯ɘ̟], while \citeauthor{weckwerth_polands_2015}'s system (\citeyear[1]{weckwerth_polands_2015}) includes [aɪ], [eɪ], [ɔʏ], [øə], [yø] and [ɪə].

The system of diphthongs in Middle High German tends to consist of six sounds: \textit{ei}, \textit{ou}, \textit{öu} (\textit{öi}, \textit{öü}), \textit{ie}, \textit{uo}, and \textit{üe} (\citealt[2--3, 5--6, 17]{wright_middle_1917}, \citealt[39--41, 45]{de_boor_mittelhochdeutsche_1973}, \citealt[1133]{simmler_phonetik_1985}, \citealt[62--63, 103--108]{paul_mittelhochdeutsche_2007}, \citealt[9]{hall_underlying_2017}, \citealt[69]{schmidt_einfuhrung_2017}). Their phonetic interpretation is most likely [ei̯], [ou̯], [ø\u{y}], [ie̯], [uo̯], and [ye̯], respectively (\citealt[103--108]{paul_mittelhochdeutsche_2007}, \citealt[9]{hall_underlying_2017}). Some linguists expand this system to nine sounds, adding the diphthongs \textit{au} [aʊ], \textit{eu} [ɔʏ]/[ɔɪ], and \textit{ui} [uɪ] \citep[185]{caratini_vocalic_2009}. In Modern Standard German, the number of genuine diphthongs has decreased to three: [aɪ], [aʊ], and [ɔɪ] (\citealt[354]{eisenberg_german_1994}, \citealt[4]{dodd_modern_2003}, \citealt[112--113]{johnson_exploring_2008}, \citealt[9]{fagan_german_2009}, \citealt[19]{obrein_german_2016}). However, a new wave of diphthongs has also emerged due to the vocalic pronunciation of \textit{r} as [ɐ] and the assimilation of loanwords ending in \textit{{}-ion} and \textit{{}-ation}. As a result, the inventory of diphthongs has been enriched by [i:ɐ], [u:ɐ], [e:ɐ], and [i̯̯o:] (\citealt[354]{eisenberg_german_1994}, \citealt[53--54]{fox_structure_2005}, \citealt[9]{fagan_german_2009}).\footnote{Two additional diphthongs, i.e. [eɪ] and [oʊ], may appear in unadjusted English loanwords \citep[73]{caratini_vocalic_2009}.}



Overall, both Wymysiöeryś and the two control languages contain diphthongs (\textit{D}). Moreover, quantitatively, the number of diphthongs found in Wymysiöeryś, on the one hand, and those present in Middle High German and Modern Standard German, on the other hand, are similar. As a result, the complexity of diphthongs is approximately equal in the three languages: \textit{D\textsuperscript{W}} ${\approx}$ \textit{D\textsuperscript{MHG} }and \textit{D\textsuperscript{W}} ${\approx}$ \textit{D\textsuperscript{MSG}}.


\subsection{Triphthongs}\label{sec:wymsorys:3.3}


The Wymysiöeryś vocalic system contains one true triphthong, i.e. a sequence of three vocalic elements (usually glide, vowel, and glide) used within the same nucleus. This triphthong is [ʏ̯øœ̯], noted alternatively as [\textsuperscript{y}øœ̯] or [\textsuperscript{y}øə] (\citealt[126]{andrason_szkic_2014}, \citealt[23]{andrason_grammar_2016}).


True triphthongs are absent in Middle High German \citep[10]{hall_underlying_2017}. Neither \citet{wright_middle_1917, de_boor_mittelhochdeutsche_1973, simmler_phonetik_1985, paul_mittelhochdeutsche_2007}, nor \citet{schmidt_einfuhrung_2017} mention them in their grammatical analyses. Similarly, there are no true triphthongs in Modern Standard German. Accordingly, they do not feature in works dedicated to German phonetics (see \citealt{eisenberg_german_1994, johnson_exploring_2008, fagan_german_2009}, and \citealt{obrein_german_2016}) and phonology (see \citealt{russ_german_1994, wiese_phonology_1996}, and \citealt{fox_structure_2005}).%\footnote{Sporadically, the sequence [jaʊ] in words such as jauchen % UNDEFINED is analyzed as a marginal example of triphthongs in Modern Standard German (\citealt[243]{abrams_deutsch_2017}).} 

Overall, the category of triphthongs (\textit{T}) is only instantiated in Wymysiöeryś. Hence, by definition, the complexity of triphthongs in the two control languages is strictly lower than in Wymysiöeryś: \textit{T\textsuperscript{W}} > \textit{T\textsuperscript{MHG} }and \textit{T\textsuperscript{W}} > \textit{T\textsuperscript{MSG}}.

\subsection{Vocalic sonorants}\label{sec:wymsorys:3.4}
\largerpage[-1]

A syllable – and in particular its nucleus – can be formed in Wymysiöeryś not only by genuine vowels, but by sonorants as well. Five sonorants can be syllabic or vocalic: [l̩], [r̩], [n̩], [m̩], and [ŋ̬̍] (\citealt[12]{kleczkowski_dialekt_1920}, \citealt[xiii]{mojmir_worterbuch}). By far, the most common are [l̩] and [n̩] \citep{Andrason2021}.


The sound system of Middle High German most likely lacked syllabic or vocalic sonorants, as the Proto-Indo-European [n̩], [m̩], [l̩], and [r̩] developed a full vocalic component, namely \textit{u}, \textit{o}, \textit{ü}, or \textit{ö} (from Old High German \textit{o} and \textit{u}, and an earlier Germanic *\textit{u}) (\citealt[62, 66, 146--150]{paul_mittelhochdeutsche_2007}; see also \citealt[48]{de_boor_mittelhochdeutsche_1973}). A possible case, where the presence of syllabic sonorants has been hypothesized, involves the metathesis of \textit{ər} into \textit{rə} through an `r̩' stage \citep[147]{paul_mittelhochdeutsche_2007}. Even if a few (highly debatable) instances of syllabic sonorants could be hypothesized, the relevance of such sounds was minimal for the vocalic system of Middle High German. In contrast, syllabic liquids ([l̩] and [r̩]) and syllabic nasals ([n̩], [m̩], [ŋ̬̍]) are a common feature of Modern Standard German, where they arose due to the reduction of \textit{shwa} (\citealt[24, 32]{fagan_german_2009}, \citealt[130--131]{johnson_exploring_2008}, \citealt[18]{obrein_german_2016}). %see also a more phonological analysis offered by \citealt[47, 72]{fox_structure_2005}).



Overall, the category of syllabic/vocalic sonorants (\textit{S}) is instantiated in Wymysiöeryś and one of the control languages. It is present in Modern High German but absent in Middle High German. The number of syllabic/vocalic sonorants found in Wymysiöeryś and Modern Standard German are quantitatively similar. Therefore, the complexity of syllabic/vocalic sonorants is approximately equal in these two languages: \textit{S\textsuperscript{W}} ${\approx}$ \textit{S\textsuperscript{MSG}}. In contrast, the complexity relationship between Wymysiöeryś and Middle High German is that of strict inequality: \textit{S\textsuperscript{W}} > \textit{S\textsuperscript{MHG}}. 


\subsection{Vocalic length}\label{sec:wymsorys:3.5}

% \todo{no ibid.}
Most studies recognize the presence of long vowels and the significance of vocalic length in Wymysiöeryś. Their most fervent advocates are \citet{kleczkowski_dialekt_1920} and \citet{mojmir_worterbuch}. \citet[174]{kleczkowski_dialekt_1920} argued that, contrary to contemporary Polish, length was a part of the vocalic system of Wymysiöeryś, and that the opposition between short and long vowels was fundamental from a systemic perspective \citep[11--12, 26--27]{kleczkowski_dialekt_1920}. In fact, \citet[26]{kleczkowski_dialekt_1920} distinguishes four grades of length, splitting long and short vowels into two sub-classes each: extra-long and long, on the one hand; and middle and short, on the other hand. In his model, the following monophthongs can be long or extra-long: [iː], [eː], [ɛː], [aː], [ɑː], [oː], [ɔː], [uː], [yː] \citep[11--12, 26--27]{kleczkowski_dialekt_1920}. More reduced systems of long vowels are discerned by \citet{lasatowicz_deutsche_1994, zieniukowa_sutuacja_2001}, and \citet{wicherkiewicz_making_2003}. \citet[40--41]{lasatowicz_deutsche_1994} distinguishes seven long vowels: [iː], [ɑː], [eː], [oː], [yː], [øː], [uː]. \citet[499--500]{zieniukowa_sutuacja_2001} distinguish six long vowels: [iː], [eː], [ɑː], [uː], [yː], and [øː]. \citet[405--407]{wicherkiewicz_making_2003} distinguishes five long vowels: [iː], [ɑː], [eː], [uː], [yː], [øː]. The relevance of vocalic length is also acknowledged by \citet[27--28]{andrason_grammar_2016}. In contrast, \citet[1--2]{weckwerth_polands_2015} suggests that vocalic length, even though present, has a limited distinctive role, as the main contrast between vowels involves quality rather than quantity.\footnote{My recent studies and fieldwork in Wilamowice show that long vowels are a constant feature of Wymysiöeryś, although contrast between short and long monophthongs often involves at least minimal changes in quality. See words like \textit{fooł} [fo:w] `grey, gray-haired', or the pair \textit{hoon} [ho:n] `rooster' versus \textit{hon} [hɔn]/[hon] `roosters'.}



Length was a relevant and contrastive feature of the vocalic system of Middle High German (\citealt[1133]{simmler_phonetik_1985}, \citealt{Seiler2005}). In addition to short singletons (see Section \ref{sec:wymsorys:3.1}), the language possessed eight long vowels: \textit{â} [aː], \textit{æ} [æː], \textit{ê} (\textit{ie}) [eː], \textit{î} [iː], \textit{ô} (\textit{uo}) [oː], \textit{oe} (\textit{üe}) [øː], \textit{û} [uː], and \textit{iu} [yː] (\citealt[2--7]{wright_middle_1917}, \citealt[37--38, 41]{de_boor_mittelhochdeutsche_1973}, \citealt[1133]{simmler_phonetik_1985}, \citealt[62--63, 97--100]{paul_mittelhochdeutsche_2007}, \citealt[9]{hall_underlying_2017}, \citealt[69]{schmidt_einfuhrung_2017}). As mentioned in Section \ref{sec:wymsorys:3.1}, the short and long vowels likely exhibited at least minimal qualitative differences (\citealt[3]{wright_middle_1917}, \citealt[185]{caratini_vocalic_2009}, \citealt[9]{hall_underlying_2017}, and \citealt[69]{schmidt_einfuhrung_2017}). Length also plays an important role in Modern Standard German, where vowels differ both in quality and quantity (\citealt[7--9]{fagan_german_2009}, \citealt[17, 19]{obrein_german_2016}; see also the short and long phonemes in \citealt[118--119]{russ_german_1994} and \citealt{wiese_phonology_1996}, \citealt[39--41]{fox_structure_2005}).\footnote{That is, both quality and length are used to differentiate sounds and ultimately lexemes (\citealt[352]{eisenberg_german_1994}, \citealt[3]{dodd_modern_2003}, \citealt[109]{johnson_exploring_2008}, \citealt[7--9]{fagan_german_2009}, \citealt[17--19]{obrein_german_2016}).} Length is responsible for the systemic division of all vowels into `lax' and `tense'. Lax vowels are always short, while tense vowels are long in a stressed position. The tense long vowels comprise eight sounds: [iː], [eː], [ɛː], [a/ɑː], [oː], [uː], [yː], and [øː] (\citealt[350--352]{eisenberg_german_1994}, \citealt[109--110]{johnson_exploring_2008}, \citealt[7--9]{fagan_german_2009}, \citealt[17, 19]{obrein_german_2016}; see also \citealt[118--119]{russ_german_1994}, \citealt[19--21]{wiese_phonology_1996}, \citealt[38--39, 41]{fox_structure_2005}, and their discussion of long vowel phonemes) sometimes expanded by an additional vowel [æː] (\citealt[35--36, 38]{fox_structure_2005}).\footnote{As always, \citegen{fox_structure_2005} analysis concerns phonology.}



Overall, the category of vocalic length (\textit{VL}) is instantiated in both Wymysiöeryś and the two control systems. The number of long vowels in the two groups is similar. As a result, the complexity of vocalic length in Wymysiöeryś, Middle High German, and Modern Standard German is approximately equal: \textit{VL\textsuperscript{W}} ${\approx}$ \textit{VL\textsuperscript{MHG} }and \textit{VL\textsuperscript{W}} ${\approx}$ \textit{VL\textsuperscript{MSG}}.


\subsection{Nasality}\label{sec:wymsorys:3.6}


The Wymysiöeryś vocalic system is characterized by the presence of nasal and/or nasalized vowels. The most common nasal sounds are the vowels [ɔ] and \textit{ę} [ɛ] and three nasal approximants [\~{w}], [ɰ], [ȷ] that can accompany oral vowels \citep{Andrason2021}. Sporadically, vowels [ã/ɑ] and [ẽ] are used \citep{Andrason2021}. The actual nasal feature, and thus the nasalization of a vowel, may vary from strong (a genuine nasal vowel) to weak (a slightly nasalized oral vowel with a nasal consonant; see \citealt[12]{kleczkowski_dialekt_1920}, \citealt[xiii]{mojmir_worterbuch}).

Nasal vowels were absent in Middle High German \citep{wright_middle_1917, paul_mittelhochdeutsche_2007, de_boor_mittelhochdeutsche_1973}. Their position in Modern Standard German is similarly weak. In general, ``German vowels are oral [and no] nasal vowel belongs to the core vocalic system'' \citep[71]{caratini_vocalic_2009}. Nasalized vowels – [ɛ(ː)], [ɔ(ː)], [ɑ(ː)], [\~{œ}(ː)] – are only found in loanwords from French (\citealt[9--10]{fagan_german_2009}, \citealt[51, 73--74]{caratini_vocalic_2009}, \citealt[22]{obrein_german_2016}).%; see also \citealt[108]{russ_german_1994} and \citealt[53]{fox_structure_2005}). 
~Even there, however, a pronunciation with an oral vowel and a nasal consonant is grammatical (\citealt[78]{russ_german_1994}, \citealt[53]{fox_structure_2005}, \citealt[9]{fagan_german_2009}, \citealt[22]{obrein_german_2016}). Being ``unstable'' and restricted to a small number of words of foreign origin, the role of nasal vowels in the vowel system of Modern Standard German is marginal (\citealt[53]{fox_structure_2005}, \citealt[10]{fagan_german_2009}, \citealt[90]{johnson_exploring_2008}). 

Overall, the category of nasality (\textit{N}) is instantiated in Wymysiöeryś and one of the control languages, i.e. Modern Standard German. However, the number of nasal vowels and their systemic relevance is greater in Wymysiöeryś than in Modern Standard German.\footnote{That is, fewer sounds can be nasal and the feature of nasality is generally less common.} Hence, the complexity relationship between these two languages is of a strict-inequality type, i.e. \textit{N\textsuperscript{W}} > \textit{N\textsuperscript{MSG}}. This strict inequality is even more evident if Wymysiöeryś is compared to Middle High German: \textit{N\textsuperscript{W}} > \textit{N\textsuperscript{MHG}}.

\subsection{Consonants}\label{sec:wymsorys:3.7}

Both quantitatively and qualitatively, the consonantal system of Wymysiöeryś is remarkable. The size of the fundamental system of consonants oscillates between thirty and thirty-nine sounds. Due to palatalizing processes (see Section \ref{sec:wymsorys:3.9}), this system may be expanded to more than fifty. As far as ``non-palatalized'' models are considered, the maximal system is posited by \citet[17--18]{andrason_grammar_2016}. To be exact, \citet{andrason_grammar_2016} distinguish thirty-nine sounds: (a) plosives: [p], [b], [t], [d], [c], [ɟ], [k], [g], and [ʔ]; (b) fricatives: [f], [v], [s], [z], [ɕ], [ʑ], [s̠], [z̠], [ʃ], [ʒ], [x], and [h]; (c) nasals: [m], [n], [ȵ], and [ŋ]; (d) liquids [l] and [r]; (e) affricates: [ts], [dz], [ʨ], [ʥ], [ṯs̠], [ḏz̠], [tʃ], [dʒ]; and approximants: [w] and [j].\footnote{The set of approximants may be extended to [ɰ] which occurs before nasals (\citealt[17--18]{andrason_grammar_2016}, \citealt{Andrason2021}).} A similarly abundant system of thirty-four consonants was proposed by \citet[13--14]{kleczkowski_dialekt_1920} and \citet[xiv--xv]{mojmir_worterbuch}. The main difference consists in the presence of [ɫ] instead of [w]\footnote{At the beginning of the 20\textsuperscript{th} century, \textit{ł} was still pronounced as a velarized alveolar lateral approximant [ɫ] rather than approximant [w] as it is the rule currently (\citealt[13, 121--126]{kleczkowski_dialekt_1920}, \citealt[xiv]{mojmir_worterbuch}, \citealt[406]{wicherkiewicz_making_2003}, \citealt{zak2019pewnym}, see Section~\ref{sec:wymsorys:4.4}).} and the absence of the distinction between the postalveolars [s̠], [z̠], [ṯs̠], [ḏz̠] and the palatalo-alveolars [ʃ], [ʒ], [tʃ], [dʒ]. More reduced consonantal systems are proposed by \citet{wicherkiewicz_making_2003} and \citet{lasatowicz_deutsche_1994}. \citet[406--409]{wicherkiewicz_making_2003} distinguishes thirty consonants. He expands the set of consonants by [pf] and [ç], on the one hand, but prescinds from [l], [ɟ], [ʔ] and the distinction between the postalveolars [s̠], [z̠], [ṯs̠], [ḏz̠] and the palatalo-alveolars [ʃ], [ʒ], [tʃ], [dʒ], on the one hand.\footnote{Note that [ʥ], [ḏz̠], [dʒ], [ʑ] are unattested in Biesik's poems \citep[408--409]{wicherkiewicz_making_2003}.} \citet[36, 42, 52]{lasatowicz_deutsche_1994} also distinguishes thirty consonants. Similar to \citet{wicherkiewicz_making_2003}, Lasatowicz's system contains the sound [ç] but fails to include consonants [c], [ɟ], [ʔ], [ȵ] and to make the distinction between the postalveolars [s̠], [z̠], [ṯs̠], [ḏz̠] and the palatalo-alveolars [ʃ], [ʒ], [tʃ], [dʒ].\footnote{\citet[270]{latosinski_monografia_1909} distinguishes twenty-six consonants.} Since nearly all consonants (except [ʔ] and [h]) have their palatal(ized) variants (see Section \ref{sec:wymsorys:3.9}; see also \citealt[15]{kleczkowski_dialekt_1920}, \citealt{mojmir_worterbuch, andrason_grammar_2016}), the system of consonants may be extended to fifty-one sounds – or even more – by incorporating, for instance, [p\textsuperscript{j}], [b\textsuperscript{j}], [t\textsuperscript{j}], [d\textsuperscript{j}], [m\textsuperscript{j}], [ŋ\textsuperscript{j}], [f\textsuperscript{j}], [v\textsuperscript{j}], [l\textsuperscript{j}]/[ʎ], [x\textsuperscript{j}]/[ç], [r\textsuperscript{j}] and [w\textsuperscript{j}] \citep{Andrason2021}.

The system of consonants exhibited by Middle High German most likely contained twenty-four sounds. Specifically, (a) plosives: [p], [b], [t], [d], [k], [g]; (b) fricatives: [f], [v], [s], [ʃ], [ʒ] (or [z]), [j], [ç], [x], and [h]; (c) nasals: [m], [n], [ŋ]; (d) liquids: [l], [r] \citep[141, 169--171]{paul_mittelhochdeutsche_2007}; (e) affricates [pf], [ts], [kx] (\citealt[18--19]{de_boor_mittelhochdeutsche_1973}, \citealt[1135]{simmler_phonetik_1985}, \citealt[141]{paul_mittelhochdeutsche_2007}); and possibly an approximant [w] (\citealt[24]{wright_middle_1917}, \citealt[1135]{simmler_phonetik_1985}). Additionally, some studies expand this system by [β] \citep[18]{de_boor_mittelhochdeutsche_1973}. The consonantal inventory of Modern Standard German includes the following twenty-seven sounds: (a) plosives: [p], [b], [t], [d], [k], [g], [ʔ]; (b) fricatives: [f], [v], [s], [z], [ʃ], [ʒ] (the last sound is often treated as peripheral), [ç], [j] (also defined as an approximant), [x], [ʁ], [h]; (c) nasals: [m], [n], [ŋ]; and (d) liquids: [l], [r], [ʀ]; and (e) affricates: [pf], [ts], [tʃ] (\citealt[353--355]{eisenberg_german_1994}, \citealt[92, 99--101, 104]{johnson_exploring_2008}, \citealt[10--14]{fagan_german_2009}, \citealt[13--16]{obrein_german_2016}). If the affricate [dʒ] \citep[16]{obrein_german_2016} and the nasal [ɱ] \citep{hall_phonologie_2000} are included, the number of consonants increases to twenty-nine ``basic'' sounds (see also the more phonologically oriented analyses offered by \citealt[112--115, 121--122]{russ_german_1994}; \citealt[22--26]{wiese_phonology_1996}, \citealt[31]{hall_phonologie_2000}, and \citealt[35--37]{fox_structure_2005}).

Overall, the category of consonants (\textit{C}) is instantiated in both Wymysiöeryś and the two control languages. However, whether maximal (extended) or minimal (basic), the number of Wymysiöeryś consonants is always larger, in certain models radically, than the number of consonants found in Middle High German and Modern Standard German. As a result, the complexity of consonants in Wymysiöeryś is substantially greater than in the control languages, i.e. \textit{C\textsuperscript{W}} > \textit{C\textsuperscript{MHG}} and \textit{C\textsuperscript{W}} > \textit{C\textsuperscript{MSG}}.

\subsection{Consonantal length}\label{sec:wymsorys:3.8}
\largerpage

Consonantal length belongs to the phonetic repertoire of Wymysiöeryś, although different studies ascribe it a distinct systemic relevance. According to \citet[15]{kleczkowski_dialekt_1920} and \citet[xv]{mojmir_worterbuch}, although attested, long consonants are not particularly common. In contrast, \citet[405--407]{wicherkiewicz_making_2003} identifies a number of long consonants in Biesik's poems (namely, [mː], [fː], [pː], [kː], [tː], [t͡sː], and [ɫː]) and notes that they are still pronounced at least ``slightly longer'' than their short counterparts by modern speakers \citet[407]{wicherkiewicz_making_2003}. I have detected a relatively large set of long consonants in my own field work, namely: nasals [nː], [ȵː], [m]; fricatives: [sː], [zː], [fː]; stops: [pː], [tː], [kː]; and affricates [t͡sː] and [d͡z/d͡ʒː]; as well as [rː]
\citep{Andrason2021}.
Although long consonants also allow for a shortened pronunciation as singletons, consonantal length seems to be a regular feature of the Wymysiöeryś sound system.

Long or geminated consonants were a typical component of Middle High German (\citealt[25, 27--28, 30--31]{wright_middle_1917}, \citealt[1134--1135]{simmler_phonetik_1985}, \citealt{goblirsch_voice_1997, goblirsch_lenition_2018}, \citealt[334]{jessen_phonetics_1998}, \citealt[141]{paul_mittelhochdeutsche_2007}), where the length played, to an extent, a phonemic function (\citealt[85--88]{fourquet_einige_1963}, \citealt{moosmuller_phonotactic_2015}). The following long consonantal sounds are usually identified for Middle High German: \textit{pp} [pː], \textit{bb} [bː], \textit{tt} [tː]; \textit{gg} [gː], \textit{ff} [fː], \textit{ss} [sː], \textit{mm} [mː], \textit{nn} [nː], \textit{ll} [lː], and \textit{rr} [rː] (\citealt[25]{wright_middle_1917}, \citealt[141]{paul_mittelhochdeutsche_2007}). The set of long consonants is often extended by [ʃː], [xː], and [kː] (\citealt[1135]{simmler_phonetik_1985}, \citealt[142, 171]{paul_mittelhochdeutsche_2007}, for a discussion consult \citealt{goblirsch_voice_1997, goblirsch_lenition_2018} and \citealt[141--175]{paul_mittelhochdeutsche_2007}).%; see also \citealt[29--30]{de_boor_mittelhochdeutsche_1973}).
 ~Modern Standard German has no geminate or long consonants ``at the phonetic level'' (\citealt[70]{caratini_vocalic_2009}, \citealt{goblirsch_lenition_2018}). The only consonantal sounds present in the language are thus singletons (\citealt[70]{caratini_vocalic_2009}, \citealt{fagan_german_2009}) and spelling them with double consonants generally indicates that a preceding vowel is short \citep[118, 140]{russ_german_1994}.

Overall, the category of consonantal length (\textit{CL}) is instantiated in Wymysiöeryś and one of the control languages, i.e. Middle High German. It is absent in Modern Standard German. The number of long consonants in Wymysiöeryś and Middle High German is relatively similar and, thus, their respective complexities may be viewed as approximately equal, i.e. \textit{CL\textsuperscript{W}} ${\approx}$ \textit{CL\textsuperscript{MHG}}. The comparison between Wymysiöeryś and Modern Standard German reveals the relation of strict inequality, i.e. \textit{CL\textsuperscript{W}} > \textit{CL\textsuperscript{MSG}}.

\subsection{Palatalization}\label{sec:wymsorys:3.9}
A palatalization-based opposition between the so-called hard (non-palatal(ized)) and soft (palatal(ized)) consonants is a pervasive and essential component of the Wymysiöeryś sound system (\citealt{kleczkowski_dialekt_1920, anders_gedichte_1933}, \citealt[402, 405--409]{wicherkiewicz_making_2003}, \citealt{Andrason2021}). Virtually every non-palatal(ized) hard sound possesses its palatal(ized) soft counterpart (\citealt[15]{kleczkowski_dialekt_1920}, \citealt[xv]{mojmir_worterbuch}, \citealt{andrason_grammar_2016}). Apart from the alveolo-palatal (or palatalized postalveolar) sounds (i.e. the fricatives [ɕ] and [ʑ], the affricates [tɕ] and [dʑ], and the nasal [ȵ]), Wymysiöeryś exhibits the following palatal(ized) consonants, each contrastive with a hard equivalent: [p\textsuperscript{j}] – [p], [b\textsuperscript{j}] – [b], [t\textsuperscript{j}] – [t], [d\textsuperscript{j}] – [d], [k\textsuperscript{j}]/[c] – [k], [g\textsuperscript{j}]/[ɟ] – [g], [m\textsuperscript{j}] – [m], [ŋ\textsuperscript{j}] – [ŋ], [f\textsuperscript{j}] – [f], [v\textsuperscript{j}] – [v], [l\textsuperscript{j}]/[ʎ] – [l], [x\textsuperscript{j}]/[ç] – [x], [r\textsuperscript{j}] – [r], and [w\textsuperscript{j}] – [w]. Palatalization also plays a relevant role in the morphology of Wymysiöeryś, for instance, triggering the use of an allomorphic ending -\textit{ja} instead of \textit{{}-a} in inflectional forms of nouns and adjectives, e.g. \textit{ryk} [rɘ̟k] `back' – dat.pl. \textit{rykja} [rɘ̟c(ː)a] \citep{andrason_szkic_2014, andrason_vilamovicean_2015, andrason_grammar_2016}. This large number of palatal(ized) consonants and their extensive use in the lexicon and grammar give the Wymysiöeryś language a soft resonance, fully comparable to Polish, but noticeably distinct from German (\citealt{kleczkowski_dialekt_1920}; see also \cite[271--272]{latosinski_monografia_1909}).\footnote{The perception of Wymysiöeryś as a soft language – and in that regard equal to Polish – is a usual reaction when a non-Wymysiöeryś speaker who is familiar with German (and Polish) is exposed to the Wymysiöeryś language.} 

Palatalization operated in Middle High German residually. The most evident palatalizing process affected the consonant \textit{s} that was softened to [ʃ] before the consonants \textit{k}, \textit{l}, \textit{m}, \textit{n}, \textit{p}, \textit{t}, \textit{w} (\citealt{paul_mittelhochdeutsche_2007}, \citealt[196, 209]{fagan_german_2009}, \citealt[41]{hennings_einfuhrung_2012}). Additionally, \textit{g} was palatalized to \textit{j} \citep[37]{paul_mittelhochdeutsche_2007}. Modern Standard German fails to exploit palatalization and palatal(ized) consonants to an extent that would be comparable to that attested in Wymysiöeryś. The most evident case of palatalization is the softening of [x] to [ç] (\citealt[26--27]{fagan_german_2009}, \citealt[45]{obrein_german_2016}).%; see also \citealt[117, 122]{russ_german_1994}, \citealt[38, 48]{fox_structure_2005}). 
~The language also contains other palatal sounds, namely [j], [ʃ] and [ʒ] (\citealt[92, 95, 104]{johnson_exploring_2008}, \citealt[9, 12--13]{fagan_german_2009}, \citealt{van_der_hoek_palatalization_2010}, \citealt[14, 16]{obrein_german_2016}).

Overall, the category of palatalization (\textit{P}) is instantiated in Wymysiöeryś and the two control languages. However, the number of palatalized consonants and the systemic effects of palatalization are far more significant in Wymysiöeryś than in Middle High German and Modern Standard German. Hence, palatalization is substantially more complex in Wymysiöeryś than in the control languages: \textit{P\textsuperscript{W}} > \textit{P\textsuperscript{MHG}} and \textit{P\textsuperscript{W}} > \textit{P\textsuperscript{MSG}}.

\subsection{Aspiration}\label{sec:wymsorys:3.10}
\largerpage
In Wymysiöeryś, voiceless plosive consonants /p/, /t/, /k/ fail to be aspirated in a word-initial position, and in all other positions (however, see further below), thus appearing as [p], [t], [k] (\citealt[14--15, 28]{kleczkowski_dialekt_1920}, \citealt[xiv--xv]{mojmir_worterbuch}, \citealt[17--19]{andrason_grammar_2016}). This means that they contrast with /b/, /d/, /g/ in terms of voicing as in Polish (\textit{jak w polskiem}) \citep[28]{kleczkowski_dialekt_1920}, the latter series being pronounced voiced in word-initial and word-internal position \citep[17--19]{andrason_grammar_2016}.\footnote{In a final position, /b/, /d/, and /g/ are generally devoiced, and thus their opposition with /p/, /t/, and /k/ is neutralized.} An analogous situation is found in Biesik's poems where /p/, /t/, /k/ were unaspirated, and the opposition with /b/, /d/, /g/ involved the feature of voicing only \citep[399--409]{wicherkiewicz_making_2003}. Aspiration is also absent in the Wymysiöeryś variety described by \citet[42]{lasatowicz_deutsche_1994}.\footnote{\citet[42]{lasatowicz_deutsche_1994} argues for a fortis-lenis distinction between /p/, /t/, /k/ and /b/, /d/, /g/, without explaining how this contrast should be understood in phonetic terms.} Only sporadically, a soft aspiration of /p/, /t/, and /k/ is audible in a word-final position \citep[19]{andrason_grammar_2016}.

\largerpage
In studies on Middle High German, the opposition between /p/, /t/, /k/ and /b/, /d/, /g/ is generally viewed in terms of tenseness \citep{goblirsch_voice_1997, goblirsch_lenition_2018, jessen_phonetics_1998} – alternatively referred to as ``spread glottis'' (\citealt[44]{harbert_germanic_2007}, \citealt[44]{iverson_laryngeal_2003}, \citeyear{iverson_germanic_2008}) – that is, \textit{fortis} versus \textit{lenis} (\citealt[1134]{simmler_phonetik_1985}, \citealt[18]{weddige_mittelhochdeutsch_2007}, \citealt[8--10]{hennings_einfuhrung_2012}, \citealt[131, 141]{paul_mittelhochdeutsche_2007}, \citealt{moosmuller_phonotactic_2015}). However, the determination of the precise phonetic nature of this opposition is elusive. Most likely, the contrast translated onto a set of phenomena, such as force, quantity, voicing, and aspiration, all of them characterized by distinct degrees of relevance \citep[1133--1135]{simmler_phonetik_1985}. The most relevant of them were articulatory force (\citealt[22--23]{wright_middle_1917}, \citealt[18]{weddige_mittelhochdeutsch_2007}) and quantitative augmentation (\citealt{goblirsch_voice_1997, goblirsch_lenition_2018}, \citealt[334]{jessen_phonetics_1998}; see also \citealt[1135]{simmler_phonetik_1985}) – both related to intensity. Often, voicing is considered the third crucial property correlated with tenseness (\citealt[18]{de_boor_mittelhochdeutsche_1973}, \citealt[1133]{simmler_phonetik_1985}, \citealt[19]{weddige_mittelhochdeutsch_2007}, \citealt{seiler_sound_2009}, \citealt[8--10]{hennings_einfuhrung_2012}). In contrast, even though aspiration is apparently ``inherent'' to the Germanic family (\citealt[44]{iverson_laryngeal_2003}, \citealt[2]{iverson_germanic_2008}), its role in Middle High German was secondary and the three \textit{fortes} consonants were most likely not (fully) aspirated (\citealt[1133]{simmler_phonetik_1985}, \citealt[9]{hennings_einfuhrung_2012}; see also \citealt[133]{paul_mittelhochdeutsche_2007}). In Modern Standard German, the system of plosives is also based on the tenseness or spread-glottis contrast, such that the opposition between /p/, /t/, /k/ and /b/, / d/, /g/ is generally explained as \textit{fortis} versus \textit{lenis} (\citealt[115]{russ_german_1994}, \citealt{wiese_phonology_1996}, \citealt[22, 136, 142--143]{jessen_phonetics_1998}, \citealt[42]{fox_structure_2005}, \citealt[3]{iverson_germanic_2008}, \citealt{caratini_vocalic_2009}). The feature of tenseness is correlated primarily with aspiration, with /p/, /t/, /k/ being ``heavily aspirated in prosodically prominent positions'' \citep[3]{iverson_germanic_2008}, e.g. word-initially (\citealt[115, 117, 121]{russ_german_1994}, \citealt{iverson_laryngeal_2003, iverson_germanic_2008}, \citealt[42]{fox_structure_2005}, \citealt{caratini_vocalic_2009}).\footnote{Although most scholars reject voice as a distinctive feature in Modern Standard German, its relevance is also acknowledged \citep[169]{wiese_phonology_1996}, since the series /b/, /d/, /g/ surfaces not only as unaspirated but also as partially voiced \citep[3]{iverson_germanic_2008}. Overall, the plosives contrast in both aspiration (primarily) and voicing (secondarily). The tense consonants /p/, /t/, /k/ are aspirated and/or unvoiced. The lax consonants /b/, /d/, /g/ are unaspirated and/or voiced (\citealt[43--44]{jessen_phonetics_1998}; see also \citealt[70]{caratini_vocalic_2009}).} In this system, voicing is viewed as a secondary feature \citep[334]{jessen_phonetics_1998}.\clearpage

Overall, the category of aspiration (\textit{A}) is not instantiated (or very poorly instantiated) in Wymysiöeryś and Middle High German while it is present and central in Modern Standard German. Therefore, the complexity relationship between Wymysiöeryś and the control languages is as follows: \textit{A\textsuperscript{W}} ${\approx}$ \textit{A\textsuperscript{MHG}} and \textit{A\textsuperscript{W}} < \textit{A\textsuperscript{MSG}}.

\subsection{Onset clusters}\label{sec:wymsorys:3.11}

Wymysiöeryś tolerates elaborated consonant onsets \citep[71]{andrason_vilamovicean_2015}. Mono\-segmental onsets may exhibit all consonants except [ŋ].\footnote{This means that contrary to Modern Standard German (see below), [ʃ] and [x] may form monosegmental onsets.} Bisegmental onsets exhibit a considerable variety, tolerating the following clusters: (a) stop + liquid/na\-sal/fricative/approximant; (b) fricative/liquid/nasal/fricative/stop{\slash}approximant{\slash}affricate; and (c) affricate + liquid/nasal/fricative \citep{Andrason2021}. Contrary to many West Germanic languages (see Section \ref{sec:wymsorys:4.7}), Wymysiöeryś tolerates bisegmental onsets such as [tl] and [dl], onsets with a glide as their second segment [j/w], and onsets with a voiced sibilant as their first element, e.g. [zm], [ʒm], [ʒv]. Additionally, it contains other onsets rare in German and its relatives: [kʃ], [tf], [tx], [ps], [pʃ], [bʒ], [gʒ], [t͡ʃf], and [ʃt͡ʃ]. Three consonant onsets are also common in Wymysiöeryś, and a wide range of combinations are possible: stop + fricative + nasal (e.g. [bʒȵ]); fricative + fricative + stop (e.g. [fsp], [fst], [vzd], [vzg]); fricative + fricative + nasal (e.g. [vzm]); fricative + stop + fricative (e.g. [fkʃ], [stf]); fricative + stop + liquid (e.g. [skr], [spr], [str], [skn]; and [ʃkl], [ʃkr], [ʃpr], [ʃtr]). Although not particularly frequent, a few onsets composed of four consonants are attested, e.g. [vskʃ] and [pstr]. In addition to the quantitative complexity outlined above, Wymysiöeryś attests a significant qualitative variety of onset clusters. This is evident in the fact that the language tolerates not only sequences that conform to the sonority scale, but also those that violate it.\footnote{The sonority scale depicts the increase in the relative sonority of sounds and their ``vowel-likeness'' \citep{foley_foundations_1977, clements_role_1990}. Generally, the sonority increases from obstruents to vowels, via sonorants. A more fine-grained representation of the scale is as follows: voiceless stops > voiced stops > voiceless fricatives > voiced fricatives > nasals > l > r > glides / high vowels > low vowels \citep[65]{harbert_germanic_2007}. This scale imposes sonority restrictions whereby, in onsets, consonants placed higher on the sonority scale may not occur before consonants placed lower on the sonority scale \citep[66, 68]{harbert_germanic_2007}. In other words, the sonority of a syllable may not decrease from the left edge to its nucleus, but rather increases \citep[66, 68]{harbert_germanic_2007}. Inversely, elements in codas must ``decline in sonority toward the right edge of the syllable'' \citep[73]{harbert_germanic_2007}.} Apart from clusters in which the first element is a sibilant [s] or [ʃ] (as is common in Germanic languages; see below), a large number of such ``ill-formed'' sequences is tolerated, especially: (a) fricative + stop + fricative (e.g. [fkʃ]) and (b) fricative + fricative + stop (e.g. [fsp], [fst], [vzd], [vzg]).

Middle High German maximally contains three consonants in onsets, which is typical of the Germanic family in general (\citealt[66]{harbert_germanic_2007}, \citealt[34]{van_oostendorp_germanic_2019}). As elsewhere in the Germanic family \citep[35]{van_oostendorp_germanic_2019}, monosegmental onsets exhibit very few restrictions, e.g. [x], [ç], and [ŋ]. In biconsonantal onsets, sequences composed of an obstruent and a liquid are allowed with the exception of [tl] and [dl] (\citealt[36]{van_oostendorp_germanic_2019}, compare with Wymysiöeryś above). Other types of onsets are more restricted, with a number of combinations being disallowed, e.g. onsets with a glide as second segment [j/ʋ/w] and onsets composed of sibilants and voiced obstruents (cf. \citealt[36--38]{van_oostendorp_germanic_2019}). In general, conforming to the behavior exhibited by West Germanic languages, Middle High German complies with the sonority-based constraints \citep[68, 73]{harbert_germanic_2007}, \citealt[36]{van_oostendorp_germanic_2019}) to a much larger extent than is the case in Wymysiöeryś. In Modern Standard German, monosegmental onsets tolerate most consonants with the exception of [x], [ŋ], and [ʃ] \citep[58]{fox_structure_2005}.\footnote{When describing the consonant clusters in Modern Standard German in Section \ref{sec:wymsorys:3.11} and Section \ref{sec:wymsorys:3.12}, I will often refer to \citet{russ_german_1994, wiese_phonology_1996, hall_phonologie_2000, fox_structure_2005}, and \citet{obrein_german_2016}. It should be noted that, in their discussion of phonotactics, these authors are principally concerned with phonemes.} For complex onsets, only doubles are relatively common. Two basic types can be discerned: obstruent (plosive, fricative, affricate) + liquid ([r/l]) and fricative (mostly, [ʃ]) + C \citep[58]{fox_structure_2005}. Specifically, the following combinations are grammatical: stop + liquid/nasal/fricative; fricative + liquid/nasal; and only for ʃ, fricative + fricative/stop (\citealt[133]{veith_phonetik_1980}, \citealt[356]{eisenberg_german_1994}, \citealt[120]{russ_german_1994}). The only common second segments are sonorants (\citealt[35]{fagan_german_2009}; see also \citealt[66]{obrein_german_2016}, \citealt[231]{hall_phonologie_2000}). Additionally, affricates may combine with a liquid or a fricative \citep[35, 58]{fagan_german_2009}. Triple onsets are scarce and highly restrictive, both qualitatively and quantitively \citep[50]{kucera_comparative_1968}. Only five permutations are grammatical (i.e. [skl], [skr], [ʃpl], [ʃpr], [ʃtr]), all of them of the type [s, ʃ] + stop + liquid (\citealt[58]{fox_structure_2005}, \citealt[36]{fagan_german_2009}, \citealt[69]{hall_syllable_1992}, \citeyear{hall_phonologie_2000}). Quadruple onsets are disallowed (\citealt[50]{kucera_comparative_1968}, \citealt[355]{eisenberg_german_1994}, \citealt[55]{fox_structure_2005}; see also \citealt{wiese_phonology_1996}). Onsets largely comply with the sonority scale principle (\citealt[260]{wiese_phonology_1996}, \citealt[60]{fox_structure_2005}, \citealt[36--37]{van_oostendorp_germanic_2019}). The only common exceptions involve [ʃ] and [s] which may occur before stops (\citealt[60]{fox_structure_2005}, \citealt[39--40]{van_oostendorp_germanic_2019}).

Overall, consonant clusters (\textit{OC}) in onsets are instantiated in Wymysiöeryś and the two control languages. However, onset clusters are significantly longer (i.e. larger sequences are tolerated) and more varied (i.e. a more diverse set of combinations is grammatical) in Wymysiöeryś than in Middle High German and Modern Standard German. This yields the following complexity relationships between Wymysiöeryś and the control languages: \textit{OC\textsuperscript{W}} > \textit{OC\textsuperscript{MHG}} and \textit{OC\textsuperscript{W}} > \textit{OC\textsuperscript{MSG}}.

\subsection{Coda clusters}\label{sec:wymsorys:3.12}

Wymysiöeryś allows for elaborated and qualitatively diversified consonant codas. All consonants are allowed in monosegmental codas with the exception of [ʔ] and [h], as well as voiced plosives, due to devoicing processes operating in a word-final position. Diverse combinations are also tolerated in codas composed of two consonants, including a large set of clusters containing the affricates [t͡s] and [t͡ʃ] and their variants (also resulting from the devoicing of /d͡z/ and /d͡ʒ/).\footnote{For example, one finds [t͡ʃt] and [t͡st] as the first segment, and [ʃt͡ʃ], [lt͡ʃ], [ȵt͡ʃ], [rt͡ʃ], [wt͡s], [lt͡s], [nt͡s], [rt͡s], [nd͡z] as the second segment \citep{Andrason2021}.} Nevertheless, certain constraints operate, of which the most pervasive is the ungrammaticality of the final [j], [h], and [ʔ], as well as the avoidance of final voiced obstruents. Three-segment codas are also widely attested; however, they are only common in morphologically complex forms. Their presence in mono-morphemic (i.e. non-inflected) lexemes is in contrast limited. The most common mono-morphemic three-segment codas are: [nft], [mpt], as well as [mpl] and [ndl] in cases where the final sonorants are not syllabic. The most common combinations found in pluri-morphemic words involve [Cst] and [CCt], which tend to arise in inflected forms of verbs (e.g. in the present tense and the preterite) and adjectives (e.g. superlative), e.g. [tst], [kst], [fst], [hst], [mst], [nst], [ŋst], [wst], [ŋkt], [wtst]. Four-consonant coda are also grammatical being attested exclusively in inflected forms of verbs and adjectives. The first element is always a sonorant, the second is a stop, while the third and fourth segments are usually filled by the cluster [st], e.g. [wdst], [lkst], [ntst], and [nkst]. Five-segment codas are generally avoided in Wymysiöeryś. Codas typically comply with the sonority scale principle. Common exceptions are clusters ending in a plosive and a sibilant, e.g. [ps] or [ks].

As is typical of West Germanic languages \citep[40]{van_oostendorp_germanic_2019}, in Middle High German, [h] and [ʔ] were disallowed in mono-segmental codas in a word-final position \citep[161]{paul_mittelhochdeutsche_2007}, as is, most likely, also true of the voiced plosives due to their final devoicing \citep[19]{paul_mittelhochdeutsche_2007}. Bi-segmental codas were common, and a diversified set of combinations was tolerated, e.g. liquid + obstruent (including [lt]) or nasal + obstruent (cf. \citealt[41--42]{van_oostendorp_germanic_2019}). Codas longer than two consonants – i.e. triplets and quadruplets – usually emerge in forms that exhibit complex morphology, e.g. verbal, nominal, and adjectival inflections. The typical word-final element in complex codas are voiceless coronal obstruents as is the rule in West Germanic languages in general (\citealt[43--44]{van_oostendorp_germanic_2019}; for various examples in Middle High German see \citealt[146--175]{paul_mittelhochdeutsche_2007}). In Modern Standard German, all consonants, except [b], [d], [g], [v], [z], [j], [h], and [ʔ], are grammatical in monosegmental codas (\citealt[59]{fox_structure_2005}, \citealt[41]{van_oostendorp_germanic_2019}). Two-consonant codas also show minimal restrictions and a large variety of combinations are grammatical (\citealt[356]{eisenberg_german_1994}, \citealt[124--125]{russ_german_1994}). Specifically, bisegmental codas cannot end in voiced obstruents, [j], [h], [ʔ], and voiced plosives. The allowed clusters are mostly of the four types: [l/r] + obstruent or nasal; nasal + obstruent; obstruent + [t]; and plosive + fricative \citep[59]{fox_structure_2005}. Codas composed of three segments are common, especially due to the presence of inflectional endings \citep[125]{russ_german_1994}. However, certain important restrictions on their combinatory freedom operate as well (\citealt[59]{fox_structure_2005}, \citealt[37]{fagan_german_2009}). The most pervasive of them is the presence of \textit{t} or \textit{s} in the final position \citep[59]{fox_structure_2005}. Four-segment codas invariably have a sonorant as their first element, and [st\textsuperscript{h}] or [t\textsuperscript{h}s] as their third and fourth elements \citep[37--38]{fagan_german_2009}. Their presence within a single morpheme is extremely rare – only two lexemes are attested (\citealt[59]{fox_structure_2005}, \citealt[3]{fagan_german_2009}; see also \citealt[121]{hall_syllable_1992}). Inversely, they tend to appear in multi-morphemic forms, e.g. in inflected nouns, adjectives, and verbs (\citealt[121]{hall_syllable_1992}, \citealt[125]{russ_german_1994}, \citealt[37--38]{fagan_german_2009}, \citealt[43--44]{van_oostendorp_germanic_2019}). Five-segment codas are highly problematic. There are only three inflectional forms that exhibit such combinations of sounds (\citealt[121]{hall_syllable_1992}, \citealt[125]{russ_german_1994}). For some scholars, these clusters are ``not well formed'' \citep[48]{wiese_phonology_1996} being indeed ``unpronounceable for many Germans'' (\citealt[51]{fagan_german_2009} referring to \citealt[121]{hall_syllable_1992}). As was the case of onsets (see Section \ref{sec:wymsorys:3.11}), codas generally conform to the sonority scale principle, a common exception being [ps] and [t\textsuperscript{h}s] (\citealt[260]{wiese_phonology_1996}, \citealt[60]{fox_structure_2005}, \citealt[37]{fagan_german_2009}).

Overall, complex codas (\textit{CC}) are instantiated both in Wymysiöeryś and the two control systems. In all three languages, coda clusters exhibit similar length and variety. These degrees of complexity of Wymysiöeryś, Middle High German, and Modern Standard German are thus approximately equal: \textit{CC\textsuperscript{W}} ${\approx}$ \textit{CC\textsuperscript{MHG}} and \textit{CC\textsuperscript{W}} ${\approx}$ \textit{CC\textsuperscript{MSG}}.

\subsection{Module-global complexity}\label{sec:wymsorys:3.13}

The relational complexities of Wymysiöeryś, Middle High German, and Modern Standard German determined locally for the twelve phonetic/phonological features are recapitulated in \tabref{tab:Wymysiöeryś:1} below. For each feature studied, relational complexity was determined in terms of equality ${\approx}$ (similar), inequality ${\leq}$ / ${\geq}$ (minimally lower/greater), and strict inequality < / > (substantially lower/greater). The evidence shows that the complexity of Wymysiöeryś is substantially greater than the complexity exhibited by Middle High German and Modern Standard German with regard to the features of triphthongs, nasality, consonants, palatalization, and onset clusters. With regard to the feature of vocalic sonorants and consonantal length, the complexity of Wymysiöeryś is substantially lower than the complexities of Middle High German and Modern Standard German, respectively. Lastly, with regard to aspiration, the complexity of Wymysiöeryś is substantially lower than the complexity exhibited by Modern Standard German. In the remaining cases, the complexity of Wymysiöeryś is equal to the complexities of Middle High German and Modern Standard German.

\begin{table}
\caption{Local complexity of Wymysiöeryś in relation to Middle High German and Modern Standard German}
\label{tab:Wymysiöeryś:1}
\begin{tabularx}{\textwidth}{lQQ}

\lsptoprule

Features & Wymysiöeryś versus Middle High German & Wymysiöeryś versus Modern Standard German\\
\midrule
Monophthongs & W ${\approx}$ MHG & W${\approx}$ MSG\\
Diphthongs & W ${\approx}$ MHG & W ${\approx}$ MSG\\
Triphthongs & W > MHG & W > MSG\\
Vocalic sonorants & W > MHG & W ${\approx}$ MSG\\
Vocalic length & W ${\approx}$ MHG & W ${\approx}$ MSG\\
Nasality & W > MHG & W > MSG\\
Consonants & W > MHG & W > MSG\\
Consonantal length & W ${\approx}$ MHG & W > MSG\\
Palatalization & W > MHG & W > MSG\\
Aspiration & (?) W ${\approx}$ MHG & W < MSG\\
Onset clusters & W > MHG & W > MSG\\
Coda clusters & W ${\approx}$ MHG & W ${\approx}$ MSG\\
\lspbottomrule
\end{tabularx}
\end{table}

When analyzed globally from the perspective of the entire sound-system module, Wymysiöeryś exhibits greater complexity than the two control systems. In comparison to Middle High German, Wymysiöeryś exhibits greater complexity in half of the features – in the other half, the complexity of the two languages is equal. In comparison to Modern Standard German, Wymysiöeryś also exhibits greater complexity in six features; in one feature, the complexity of Modern Standard German surpasses that of Wymysiöeryś; in the remaining five features, the complexities of the two languages are equal. If one allocates 1 for being substantially greater (>), 0 for equality (${\approx}$), and -1 for being substantially lower (<), the relational global sound-system complexity values (\textit{SS-COMPL}) are the following: Wymysiöeryś (+6) versus Middle High German (-6); and Wymysiöeryś (+5) and Modern Standard German (-4). Accordingly, \textit{SS-COMPL}\textsuperscript{W} > \textit{SS-COMPL}\textsuperscript{MHG} and \textit{SS-COMPL}\textsuperscript{W} > \textit{SS-COMPL}\textsuperscript{MSG}. 

It is also possible to relate the complexities of the three languages simultaneously, and thus take into account complexity relationship linking Middle High German and Modern Standard German in addition to relationships involving Wymysiöeryś. The scale in \figref{fig:Wymysiöeryś:1} below represents the relative position of the three languages. The scale is limited by two extremes: maximal score +24 (the complexity of language \textit{x} is substantially greater than the complexity of languages \textit{y} and \textit{z} in all features) and minimal score -24 (the complexity of language \textit{x} is substantially lower than the complexity of languages \textit{y} and \textit{z} in all features). Out of possible 24 points (12 for each comparative analysis with one of the other systems), Wymysiöeryś scores +11 (12 >; 11 ${\approx}$; 1 <). Middle High German scores -7 (1 >; 15 ${\approx}$; 8 <). Modern Standard German scores -4 (3 >; 14 ${\approx}$; 7 <). This again demonstrates the greater module-global complexity of Wymysiöeryś if compared to the two control languages.

\begin{figure}
 \caption{Relative complexity of Wymysiöeryś, Middle High German and Modern Standard German}
 \includegraphics[width=\textwidth]{figures/Wymysiöeryś-img001.png}
 \label{fig:Wymysiöeryś:1}
 \end{figure}

\largerpage
For some features (i.e. triphthongs, vocalic sonorants, nasal vowels, and consonantal length) the contrast between Wymysiöeryś, Middle High German, and Modern Standard German concerns the presence of a category (distinctiveness), not only the number of the category's expression manners (economy). That is, in one language (or two languages) a certain category is instantiated, while in the remaining one(s), it is not instantiated at all. In total, Wymysiöeryś instantiates eleven categories (only the category of aspiration is not expressed); Middle High German instantiates eight categories (the absent categories are triphthongs, vocalic sonorants, nasality, and most likely aspiration); Modern Standard German instantiates ten categories (the absent categories are triphthongs and consonantal length). If, for the instantiation of each category, a language is allocated 1 point, Wymysiöeryś scores 11, Middle High German 8, and Modern Standard German 10 – Wymysiöeryś being thus more complex than the control languages. 

As explained in Section \ref{sec:wymsorys:2}, the estimation of module-global complexity is always problematic due to the issue of commensuration and the availability of various manners of combining local complexities. Therefore, the converging results of the three complexity measurements of the sound-system module used in this section – which all identify Wymysiöeryś as the most complex among the analyzed languages – demonstrate that Wymysiöeryś' greater complexity score is not accidental or theory-driven. It is thus very likely that the sound system of Wymysiöeryś is \textit{objectively} more complex than the two control languages.\footnote{Since the absence of a category has more far-reaching systemic effects than the distinct numbers of encoding manners, one could arguably give even more weight to the complexity of those languages where a category is present. Since the complexity hierarchy of the three languages would remain the same, this would have no important bearings on the result of my study.}

\section{The origin of the surplus} %4. /
\label{sec:wymsorys:4}
Having established that as far as the sound system is concerned, the global complexity of Wymysiöeryś is greater than the complexity of Middle High German and Modern Standard German, I will determine whether this complexity surplus exhibited by Wymysiöeryś is attributable to contact with Polish. First, the source of the complexity surplus found in seven features will be analyzed: triphthongs, vocalic sonorants, nasality, consonants, consonantal length, palatalization, and onset clusters (Section \ref{sec:wymsorys:4.1}--Section \ref{sec:wymsorys:4.7}). This will subsequently allow me to determine the surplus' origin from a global perspective (Section \ref{sec:wymsorys:4.8}).

\subsection{Triphthongs}\label{sec:wymsorys:4.1}
\largerpage
Currently, the vocalic system of Polish contains only short monophthongs.\footnote{These vowels are /i/, /ɨ/ɘ̟/, /u/, /e/, /a/, /o/. Often, [ɨ/ɘ̟] and [i] are considered allophones of a single phoneme \citep[154, 156]{sussex_slavic_2006}.} Inversely, syllables may not exhibit complex nuclei, and thus long vowels, diphthongs, and triphthongs (\citealt[181]{gussmann_phonology_2007}; see also \cite[59--60, 72, 74]{strutynski_gramatyka_1998}, \citealt[105--106]{jassem_polish_2003}, \citealt[154, 156]{sussex_slavic_2006}, see however \citealt{wagiel_fonematyka_2016}). At previous diachronic stages, specifically between the 10\textsuperscript{th} and the 15\textsuperscript{th} century, Polish did exhibit long vowels. Vocalic length was however lost in the 16\textsuperscript{th} century \citep[132, 136--137]{dlugosz-kurczabowa_gramatyka_2006}. Although absent at a phonemic level, diphthongs emerge in Polish due to nasalization processes \citep[53, 62--64, 83]{wagiel_fonematyka_2016}\footnote{Alternatively, the emergent nasal components are analyzed as approximants \citep{gussmann_phonology_2007}.} and also exist in dialects \citep{dejna_dialekty_1973, bak_gramatyka_1997}. Sometimes, sequences composed of a vowel and the approximants [j] or [w] are analyzed as diphthongs \citep[81--83]{jassem_podstawy_1973, demenko_analiza_1999, wagiel_fonematyka_2016}. In any case, neither presently nor at its previous developmental stages, Polish possessed true triphthongs. As a result, the presence of triphthongs in Wymysiöeryś cannot result from contact with Polish.\footnote{One should note that, although absent in Modern Standard German (see Section \ref{sec:wymsorys:3.3}), triphthongs are attested in the West Germanic family, e.g. in Bavarian German and High Alemannic varieties.}

\subsection{Vocalic sonorant}\label{sec:wymsorys:4.2}


Although Polish admits sequences with ``trapped'' sonorants, in which a sonorant is enclosed between two elements of lower sonority, typically two obstruents, e.g. [drg] \textit{drgać} `vibrate' \citep[62, 66]{kijak_polish_2008}, it fails to possess true syllabic sonorants. This contrasts with the situation attested in other (neighboring) Slavonic languages where sonorants used in the above-mentioned sequences tend to exhibit a syllabic status (\citealt[66]{sussex_slavic_2006, kijak_polish_2008}, see the absence of syllabic sonorants in discussions of Polish phonetics and phonology, e.g. \citealt{strutynski_gramatyka_1998, jassem_polish_2003, gussmann_phonology_2007}, and \citealt{wagiel_fonematyka_2016}). As a result, the presence of vocalic sonorants in Wymysiöeryś cannot be attributed to Polish influence.\footnote{On the other hand, syllabic sonorants are also relatively common in various varieties of (Modern Standard) German (see Section \ref{sec:wymsorys:3.4}) and in other Germanic languages.}


\subsection{Nasality}\label{sec:wymsorys:4.3}
Nasality is a prominent feature of the Polish sound system. Polish has two nasal phonemes \textit{ą} /ɔ/ and \textit{ę} /ɛ/ (\cite[297--298]{urbanczyk_encyklopedia_1991}, \citealt[659]{rothstein_polish_1993}, %\citealt{bak_gramatyka_1997}, \cite[72, 74]{strutynski_gramatyka_1998}, 
\citealt{bloch-rozmej_nasal_1997, gussmann_phonology_2007}, \cite[88, 100]{wagiel_fonematyka_2016}). In addition to those two phonemes, which are usually realized as [ɔ] and [ɛ], Polish contains a large number of nasal vowels at a phonetic level, e.g. [ĩ], [ã], [ũ], and [ɨ/ɘ̟] (\citealt[95]{bloch-rozmej_nasal_1997}, \cite[58--59, 61, 72]{strutynski_gramatyka_1998}). Overall, for every oral vowel, there is a nasal equivalent used in certain environments \citep[298]{urbanczyk_encyklopedia_1991}.\footnote{Such environments are: /n/ + /fricative/ and /m/ + /f, v/ \citep[298]{urbanczyk_encyklopedia_1991}.} As a result, nasality is viewed as a key phonological and phonetic category in Polish (\citealt[77]{bak_gramatyka_1997, strutynski_gramatyka_1998}, \citealt[269--287]{gussmann_phonology_2007}, \citealt{wagiel_fonematyka_2016}).\footnote{In a careful Standard Polish speech, the realization of nasality is asynchronous (\cite[297--298]{urbanczyk_encyklopedia_1991}, \citealt{bak_gramatyka_1997}). This gives rise to the emergence of nasal approximants such as [\~{w}], [ɰ], and [ȷ] (\citealt[660]{rothstein_polish_1993}, \citealt[270--271]{gussmann_phonology_2007}). In colloquial speech, nasal vowels often resolve into oral vowels and nasal consonants (\citealt[659]{rothstein_polish_1993}, \citealt{bak_gramatyka_1997, rubach_nasalization_1977, rowicka_nasal_1992}, %\citealt[659]{rothstein_polish_1993}, \citealt[158--159, 162]{sussex_slavic_2006},
\citealt[84--86]{bloch-rozmej_nasal_1997}, \citealt[271]{gussmann_phonology_2007}).}

Contrary to Polish, nasal vowels do not constitute a prominent feature in the phonetics and phonology of continental Germanic languages (see their absence in general works on the Germanic family, e.g. \citealt{harbert_germanic_2007} and \citealt{jacobs_yiddish_1994}).\footnote{The exception is a chapter dedicated to Old Icelandic \citep[147]{rainsson_icelandic_1994}.} In West Germanic and German varieties, nasal vowels are generally restricted to loanwords, often allowing for an alternative oral pronunciation (\citealt[78, 108]{russ_german_1994}, \citealt[9]{fagan_german_2009}, \citealt[51, 73--74]{caratini_vocalic_2009}). German dialects in which nasality is more prominent are: Swabian (an Upper German, Alemannic dialect), Pfaelzisch (\textit{Pfälzisch}) or Palatine German (\citealt[423]{van_ness_pennsylvania_1994}, \citealt[71]{stevenson_german-speaking_1997}, \citealt[197]{niebaum_einfuhrung_1999}), and the dialect of Luzern \citep[179]{bacher_deutsche_1905}. Secondary nasal vowels are also found in Yiddish (\citealt[583--585, Addendum 606]{weinreich_history_2008}, \citealt[19--20, 41]{herzog_language_1992}, \citealt[97--99]{jacobs_yiddish_2005}) and Frisian \citep[508]{hoekstra_frisian_1994}.\footnote{Nasality is more consistently present in peripheral languages: Surinam Dutch \citep[444]{deschutter_dutch_1994}, Afrikaans \citep[481]{donaldson_afrikaans_1994}, and – albeit rather as an archaism used by older speakers -- Pennsylvania German \citep[423]{van_ness_pennsylvania_1994}.} 

Although nasality is present in the Germanic family, being a common phonetic process from a cross-linguistic perspective, its emergence in Wymysiöeryś most likely stems from Polish influence. Indeed, nasal vowels appear most commonly and most consistently in Polish loans. This complies with the origin of nasality in Yiddish where its presence is attributed to Slavonic influence \citep[583--585]{weinreich_history_2008}.\footnote{Note also that continental German varieties where nasality is more visible (e.g. Pfaelzisch/Palatine and Luzern) are usually spoken in areas adjacent to languages containing prominent nasal vowels, in particular French.}

\subsection{Consonants}\label{sec:wymsorys:4.4}

Polish has a large and diversified set of consonants. The basic consonantal inventory consists of 41 sounds: 38 consonants and 3 approximants [j, w, w\textsuperscript{j}] (\citealt{bak_gramatyka_1997}, \citealt[74]{strutynski_gramatyka_1998}, \citealt{jassem_polish_2003}, \citealt[3--8]{gussmann_phonology_2007}). This set is often expanded to nearly fifty due to the inclusion of voiceless sonorants [m̥, n̥, l̥, r̥] and a voiced velar [ɣ] \citep[4]{gussmann_phonology_2007}. With the incorporation of palatalized consonants and the approximant [ɰ], the maximal system ascends to nearly seventy consonants \citep[54, 72--73]{strutynski_gramatyka_1998}. Crucially, the consonants that are absent in Middle High German and Modern German but that currently feature in Wymysiöeryś are all found in Polish too. This includes: (a) laminal alveolo-palatal ([ɕ], [ʑ], [tɕ], and [dʑ]) and postalveolar sibilants and affricates ([s̠], [z̠], [ṯs̠], [ḏz̠]) (\citealt{hamann_phonetics_2003, hamann_retroflex_2004}; cf. \citealt{karas_slownik_1977} and \citealt[75--78]{gussmann_phonology_2007});\footnote{The sounds of the ``hard'' series are defined – especially by Polish scholars – as postalveolars and represented by [ʃ], [ʒ], [tʃ], [dʒ] (cf. \citealt{biedrzycki_abris_1974, spencer_non-linear_1986, dogil_hissing_1990, jassem_polish_2003}, and \citealt{gussmann_phonology_2007}; see also \citealt{stieber_rozwoj_nodate, rospond_gramatyka_1971, wierzchowska_fonetyka_1980}). The same class has also been viewed – mostly by Anglo-Saxon and German researchers – as retroflex, the respective sounds being transcribed as [ʂ], [ʐ], [ʈ], and [ɖ] (cf. \citealt{keating_coronal_1991, ladefoged_sounds_1996, padgett_evolution_2003, hamann_phonetics_2003} and \citeyear{hamann_retroflex_2004}). While the former notation suggests a partially palatalized sound, the latter implies that the tongue shape is concave and apical or subapical. The actual realization of these consonants is, however, neither palatal(ized) nor fully retroflex, but rather laminal and flat – their closest IPA equivalents being [s̠], [z̠], [ṯs̠], and [ḏz̠] (cf. \citealt{hamann_phonetics_2003}).} (b) alveolo-palatal nasal [ȵ] (\citealt[104]{jassem_polish_2003}, alternatively transcribed as a palatal [ɲ]); (c) a series of other palatal(ized) consonants (see Section \ref{sec:wymsorys:4.7}, \citealt[687--690]{rothstein_polish_1993}, \citealt[38, 42--44, 54]{strutynski_gramatyka_1998}, \citealt[165--166]{sussex_slavic_2006}, \citealt[4--7]{gussmann_phonology_2007}); and (d) the labialized velar approximant [w] \citep{jassem_polish_2003, strutynski_gramatyka_1998, gussmann_phonology_2007}.

With regard to soft and hard sibilants and affricates, contact with Polish seems to be the direct and sole factor responsible for their introduction to Wymysiöeryś \citep{zak_influence_2016, andrason_polish_2014, andrason_szkic_2014, andrason_vilamovicean_2015, Andrason2021}. This can be inferred from the absence of those two series in West Germanic languages, on the one hand, and their particular stability in Wymysiöeryś in lexical borrowings from Polish, on the other hand.\footnote{However, the two series of sibilants and affricates are not restricted to the vocabulary borrowed from Polish. They can also be used in genuine Germanic lexemes.} For the remaining types of consonantal surplus, Polish seems to have (significantly) strengthened and/or accelerated tendencies that are typologically common and that had operated (at least marginally) language- or family-internally.

Although the presence of the alveolo-palatal nasal [ȵ] may partially be attributed to the Polish influence, being evident in a large number of Polish loanwords, in which the original sound [ȵ] is rendered as such, it seems to coincide with language-internal processes. In genuine Wymysiöeryś vocabulary, [ȵ] typically derived from \textit{\'{ŋ}} [ŋ\textsuperscript{j}] (itself a reflex of an original cluster \textit{ng}/\textit{nc} [ŋ]) or arose in cases where \textit{n} was followed by palatal sounds \textit{i}, \textit{j}, or \textit{ć} \citep{Andrason2021}. The palatalization of the velar nasal [ŋ] to [ŋ\textsuperscript{j}] constitutes a recurrent cross-linguistic tendency. It occurred in the Szynwałd/Bojków (Schönwald) dialect, closely related to Wymysiöeryś, which suggests a dialectal – family-internal – development (cf. \citealt[98--99]{gusinde_vergessene_1911}). Even though articulatory proximity may motivate the development from [ŋ\textsuperscript{j}] to [ȵ], this change occurred only after World War II, coinciding with the increased presence of the Polish language in Wilamowice (see that it was still written \textit{\'{ŋ}} by \citealt{kleczkowski_dialekt_1920} and \citealt{mojmir_worterbuch}). The palatalization of \textit{n} in palatal contexts, which had already taken place before the war \citep{kleczkowski_dialekt_1920}, is a common cross-linguistic phenomenon. Strong palatalization tendencies affecting \textit{n} operated in Eastern diphthongized Silesian dialects, sometimes even more widely than in Wymysiöeryś (\citealt[32, 41]{waniek_zum_1880}, \citealt[39--40]{von_unwerth_schlesische_1908}, \citealt[96--98, 115, 144]{gusinde_vergessene_1911}, \citealt{Andrason2021}).\footnote{It also occurred in contexts where \textit{n} appeared after a short vowel and before dental consonants. Compare \textit{k'eńt'} `children' in Szynwałd/Bojków with \textit{kynt} in Wymysiöeryś (\citealt[98]{gusinde_vergessene_1911}, \citealt[116]{kleczkowski_dialekt_1920}) -- contrary to Polish.}

Similarly, the presence of other palatal sounds in Wymysiöeryś may be attributed to Polish influence as well as language- and family-internal processes. That is, Polish might have fortified palatalizing tendencies that were already operating in the Wymysiöeryś language and its Silesian relatives. As a result, the visibility of palatal(ized) consonants was intensified, their central status in the phonetic and phonological system was established, and new palatalization rules were introduced to those already operating (see Section \ref{sec:wymsorys:4.6}).

The development of the labialized velar approximant [w] from the velarized alveolar lateral approximant [ɫ] has also resulted from two drifts: language-ex\-ter\-nal and language-internal \citep{andrason_polish_2014, andrason_vilamovicean_2015, Andrason2021, zak2019pewnym}. Polish has significantly intensified and perhaps accelerated the process whose foundations were already in place (for a similar view consult \citealt[234]{selmer_velarization_1933}). On the one hand, the change seems to imitate an analogous development operating in Polish, i.e. the replacement of [ɫ] by [w], known under the term \textit{wałczenie}. The process appeared in Polish dialects in the 16\textsuperscript{th} and \textsuperscript{17}th century. At the turn of the 19\textsuperscript{th} and the 20\textsuperscript{th} century, it spread beyond dialects to the standard language, where it became the norm in the second half of the 20\textsuperscript{th} century (\citealt[372]{urbanczyk_encyklopedia_1991}, \citealt[28]{gussmann_phonology_2007}).\footnote{Currently, the pronunciation of \textit{ł} as [ɫ] is perceived as ``an affectation'' \citep[28]{gussmann_phonology_2007}. More regularly, it occurs only in east-southern dialects (\citealt[146]{dubisz_dialekty_1995}; see also \citealt[46--47]{nitsch_dialekty_1957}, \citealt{zak2019pewnym}).} Chronologically, the change of [ɫ] to [w] in Wymysiöeryś coincides with the period of the full generalization of [w] in Standard Polish, which is also the time where the Polonization of Wilamowice increased substantially. On the other hand, the development of [ɫ] to [w] is found in other Central East (colonial) German varieties. It was, for example, attested in Lower Silesian and diphthongized Silesian dialects (\citealt[35]{von_unwerth_schlesische_1908}, \citealt[105]{gusinde_vergessene_1911}, \citealt[233--234]{selmer_velarization_1933}). In the Szynwałd/Bojków dialect, it had been established by the beginning of the 20\textsuperscript{th} century (\citealt[104--105]{gusinde_vergessene_1911}, \citealt[125, 161--162]{kleczkowski_dialekt_1920}).\footnote{However, as in Wymysiöeryś, the change that took place in Szynwałd/Bojków is attributed to Polish influence; specifically, to the Polish Silesian variety used in the Upper Silesian coal basin and industrial region, where [ɫ] had earlier developed into [w] (\citealt[156]{nitsch_dialekty_1909}, \citealt[104--105]{gusinde_vergessene_1911}, \citealt[126]{kleczkowski_dialekt_1920}).} The same process could thus have been carried on in Wymysiöeryś. Furthermore, the development of [ɫ] into [w] has occurred in other members of the Germanic family: varieties of Swiss German dialects, Thuringian dialects, Lusatian dialects, East-Low German dialects, Franconian dialects, and Low Franconian dialects (\citealt{selmer_velarization_1933}, \citealt[1111--1112]{besch_dialektologie_1983}, \citealt{leemann_diffusion_2014}).\footnote{Often, however, the vocalic pronunciation of \textit{l} in German varieties is regarded as influenced by Romance and Slavonic languages \citep[235--238, 243]{selmer_velarization_1933}.} It is indeed common from a cross-linguistic perspective, featuring not only in Slavonic and Germanic, but also in Romance and other language phyla \citep{zak2019pewnym}.

\subsection{Consonantal length}\label{sec:wymsorys:4.5}

Polish contains geminated or long consonants. They occur in an intervocalic and word-initial position (\citealt[241]{gussmann_phonology_2007}, \citealt[82]{wagiel_fonematyka_2016}). Since a number of minimal pairs may be identified, geminated consonants play a phonemic role, at least peripherally \citep[82]{wagiel_fonematyka_2016}.

Length is a pervasive – both synchronically and diachronically – feature of Germanic languages \citep[74--79]{harbert_germanic_2007}. Geminate consonants arose in old and medieval Germanic languages, both in the Northern and Western branches, where they acquired a systemic relevance \citep[74--75]{harbert_germanic_2007}. Subsequently, various languages underwent changes and long consonants have often been simplified \citep[75--78]{lass_phonology_1992, harbert_germanic_2007}). This degemination is visible in the development from Middle High German to Modern Standard German and many other West Germanic languages (\citealt[76--78]{harbert_germanic_2007}, \citealt{schmidt_einfuhrung_2017}). In modern languages, only North Germanic exhibits genuine long consonants \citep[78--79]{harbert_germanic_2007}.

Rather than deriving directly from contact with Polish, the consonantal length in Wymysiöeryś most likely constitutes an inherited Germanic property, as it existed in Middle High German – the diachronic source of Wymysiöeryś. The Polish language could however have contributed to the maintenance of long consonants, thus preventing the developments that have taken place in many other modern West Germanic languages and German varieties. 

\subsection{Palatalization}\label{sec:wymsorys:4.6}

Polish exhibits various types of palatalizing processes and a wide range of palatal\-i\-zation-based oppositions. Polish has been viewed as one of ``the most highly palatalized'' languages in the entire Slavonic branch \citep[165]{sussex_slavic_2006}, the one that attests to ``a more advanced state of [\ldots] palatalization than any of the other'' members of this language family \citep[165]{sussex_slavic_2006}. Indeed, the contrast between palatal(ized) consonants and non-palatal(ized) consonants – generally referred to as ``soft'' and ``hard'' respectively (\citealt[244]{urbanczyk_encyklopedia_1991}, \citealt[43--44]{strutynski_gramatyka_1998}) – underpins not only the sound system of Polish but also the language's morphology. Crucially, for all consonants, there is a corresponding palatal(ized) consonant, either at a phonemic or a phonetic level (\citealt[687--690]{rothstein_polish_1993}, %\citealt[42--44, 54, 72--73, 77--78]{strutynski_gramatyka_1998},
\citealt[165--166]{sussex_slavic_2006}, \citealt[4--7]{gussmann_phonology_2007}).\footnote{For an exhaustive list of ``soft'' and ``hard'' consonants consult \citet{strutynski_gramatyka_1998}. The phonemic status of palatal(ized) consonants is related to the status of the vowels \textit{i} and \textit{y} ~(\citealt[77--78]{strutynski_gramatyka_1998}; \citealt[167]{sussex_slavic_2006}). Regarding phonological and morphophonemic aspects of palatalization in Polish see \citet{gussmann_phonology_2007}.}

Although certain types of palatalization have operated in the Germanic family, and palatal(ized) sounds feature relatively prominently in Dutch, Frisian, and Afrikaans (\citealt[529]{hoekstra_frisian_1994}, \citealt[482]{donaldson_afrikaans_1994}, \citealt{van_der_hoek_palatalization_2010}), as well as in Icelandic and Faroese (\citealt[193--195]{barnes_faroese_nodate}, \citealt[48--49]{harbert_germanic_2007}), palatalization is not as essential a component of the Germanic sound system as, for example, aspiration. Its role in the phonetics and phonology of West Germanic languages is certainly less fundamental than is the case of Slavonic languages (see \citealt[48--49]{harbert_germanic_2007}). Crucially, German (see Section \ref{sec:wymsorys:3.9}) and most of its dialects fail to exploit palatalization and palatal(ized) consonants to an extent that would be comparable to that attested in Polish (and in Wymysiöeryś). As attested at the beginning of the 20\textsuperscript{th} century, German dialects exhibited a slightly more palatalization-oriented character than Standard Modern German (\citealt[38--40, 53--54, 60, 71]{von_unwerth_schlesische_1908}).\footnote{Apparently, the strongest palatal effects were found in diphthongized dialects, to which Wymysiöeryś belonged.}

Given the peripheral status of palatalization in German varieties and West Germanic languages in contrast to its central position in Polish and Slavonic languages, it is highly probable that the extensive use of palatal(ized) consonants in Wymysiöeryś and the central position of palatalization in its phonetic and phonological system, may be attributed to contact with Polish (see a similar conclusion in \citealt[15]{kleczkowski_dialekt_1920} and \citealt[136]{zak_influence_2016}). This proposal is consistent with the scenario posited for Yiddish which acquired a wide array of palatal(ized) consonants most likely under the influence of Slavonic languages (\citealt[394]{jacobs_yiddish_1994}, \citealt[26]{harbert_germanic_2007}). Furthermore, two types of palatalizing processes seem to have been transferred directly from Polish, being absent in other colonial Central East German varieties: (a) regressive palatalization, i.e. a palatal(ized) pronunciation of consonants due to the presence of subsequent front vowels (contrary to the progressive palatalization typical of Silesian German; see next paragraph) and (b) a palatalizing process analogous to the so-called fourth palatalization, i.e. the development of \textit{ky}/\textit{ke} [k] and \textit{gy}/\textit{ge} [g] into \textit{ki/kje} [c] and \textit{gi/gje} [ɟ], respectively (\citealt[136]{zak_influence_2016}, see \citealt[124--129]{dejna_dialekty_1973}, \citealt[244]{urbanczyk_encyklopedia_1991}, \citealt[146--147]{dlugosz-kurczabowa_gramatyka_2006}).

However, although Wymysiöeryś and Polish currently exhibit similar sets of palatal(ized) consonants and regressive palatalization operates both in Wymysiöeryś and Polish, the two systems are not identical. The most relevant difference pertains to the manner with which various palatal(ized) consonants emerged. In genuine Wymysiöeryś vocabulary, the palata(ized) realization of the consonant was – and still often is – conditioned by the vowel that precedes it \citep[125]{kleczkowski_dialekt_1920} rather than by the vowel that follows, which is typical of Polish. The same principle governed palatalization in all Silesian German dialects thus revealing a firm family-internal tendency \citep[71]{von_unwerth_schlesische_1908}.\footnote{In a further contrast to Polish, in Silesian German – including the variety of Szynwałd/Bojków – palatalization operated spontaneously before a dental consonant, either plosive, nasal, or lateral (\citealt[38--39, 68--69]{von_unwerth_schlesische_1908}, \citealt[98]{gusinde_vergessene_1911}).}

Overall, Polish might have fortified palatalizing tendencies that were already operating in the Wymysiöeryś language and its Silesian relatives. As a result, the visibility of palatal(ized) consonants was intensified, their central status in the phonetic and phonological system was established, and new palatalization rules were added to those already existing.

\subsection{Onset clusters}\label{sec:wymsorys:4.7}


Polish exhibits rich phonotactics, tolerating complex consonant clusters in onset positions (\citealt{gussmann_phonology_2007}, \citealt[567]{zydorowicz_consonant_2010}, \citealt{dziubalska-kolaczyk_production_2014}, \citealt[101]{zydorowicz_he_2017}) – a property that is characteristic of the Slavonic family, in general \citep{sussex_slavic_2006}. Given that both the length of clusters and the number of combinations is ``impressive'' \citep[101]{zydorowicz_he_2017}, Polish is considered as ``one of the most permissive languages'' as far as phonotactics are concerned \citep[62]{kijak_polish_2008}. With regard to length, onset clusters tolerate maximally four elements (\citealt[565]{zydorowicz_consonant_2010}, \citealt{dziubalska-kolaczyk_production_2014}, \citealt[98]{zydorowicz_he_2017}). With regard to combinatority, 231 types of doubles, 165 triples, and 15 quadruples are found in onsets in Polish (\citealt{bargielowna_grupy_1950}, \citealt[565--567]{zydorowicz_consonant_2010}, %see also \citealt{dukiewicz_rodzaje_1980}, \citealt{dobrogowska_srodglosowe_1984,dobrogowska_word_1990,dobrogowska_word_1992},
\citealt[107--108]{zydorowicz_he_2017}). The richness of Polish phonotactics is not only quantitative but also concerns the qualitative properties of clusters. That is, Polish allows for onset clusters that exhibit falling sonority profiles (e.g. [rt]) and clusters with unchanged sonority values (the so-called plateau clusters; e.g. [fsx\nobreakdash-]) in addition to those with rising sonority (e.g. [tr]) \citep[104]{dukiewicz_rodzaje_1980, zydorowicz_he_2017}. Accordingly, sequences that are ``ill-formed'' from the perspective of the sonority scale (e.g. [rt], [rdz], [pstr]) are often tolerated \citep[104]{zydorowicz_he_2017}.

Even though onset clusters in Germanic languages can be complex \citep[33]{harbert_germanic_2007, van_oostendorp_germanic_2019}, their complexity is lower than in Polish and Slavonic languages (cf. \citealt{kucera_comparative_1968} who contrast German with Russian and Czech). Overall, only bi- and tri-segmental clusters are allowed in onsets \citep[34--36]{van_oostendorp_germanic_2019}. Even biconsonantal onsets, the most permissive ones, exhibit various combinatory restrictions \citep{harbert_germanic_2007, van_oostendorp_germanic_2019}. For instance, onsets with a glide as their second element, onsets composed of sibilants and voiced obstruents, and the clusters [tl] and [dl] are generally disallowed \citep{van_oostendorp_germanic_2019}.\footnote{In contrast, all of these onset clusters are allowed in Polish.} The most permissive language as far as phonotactics are concerned, is Yiddish \citep{van_oostendorp_germanic_2019} – likely due to Slavonic influence. Three-segmental onsets are even more restricted and mainly appear with [s] and [ʃ] as the first element. With a few exceptions involving [s] and [ʃ], two- and three-consonant onsets must comply with sonority hierarchy (\citealt[68, 73]{harbert_germanic_2007}, \citealt{van_oostendorp_germanic_2019}). This compliance is larger than what one observes in Polish.

The greater qualitative and quantitative restrictions exhibited by onsets in Germanic languages than is the case in Polish, as well as the fact that the most complex Wymysiöeryś onsets appear in Polish loanwords suggests the contact-induced increase in the complexity of onsets in Wymysiöeryś. 

\subsection{Module-global perspective}\label{sec:wymsorys:4.8}

The discussion in Section \ref{sec:wymsorys:4.1}--Section \ref{sec:wymsorys:4.7} suggests that most of the surplus of complexity exhibited by Wymysiöeryś in the sound-system module can be attributed to contact with Polish. In case of four features (i.e. nasalization, consonants, palatalization, and onset clusters) contact with Polish is the principle reason for the complexity attested, although in some instances, enhancing the (more or less visible) tendencies already operating at a language- or family-internal level. In case of one feature (i.e. consonantal length), the Polish influence is secondary – it is the family-internal genetic drift that is the primary factor motivating the complexity surplus observed. Lastly, in case of two features (i.e. triphthongs and vocalic sonorants), Polish has not contributed, even minimally, to the complexity surplus exhibited by Wymysiöeryś.

Although contact with Polish has most likely contributed to the complexification of Wymysiöeryś, it is also responsible for its simplification in certain aspects. As explained in Section \ref{sec:wymsorys:3.10}, in Germanic, the distinction between /p/, /t/, and /k/ and /b/, /d/, and /g/ involves primarily the feature of tenseness \citep{jessen_phonetics_1998} or spread glottis \citep[44]{harbert_germanic_2007}. Its typical acoustic effect is aspiration (\citealt{iverson_aspiration_1995, iverson_glottal_1999, iverson_laryngeal_2003, iverson_germanic_2008}, \citealt[44]{harbert_germanic_2007}). Indeed, the spread-glottis principle -- referred to as Germanic enhancement \citep[44]{iverson_laryngeal_2003} -- with aspiration effects one of the fundamental and ``inherent'' rules governing the sound system of Germanic languages. It has been operating since the development of the proto-language, being responsible for a series of changes and developments (\citealt[44]{iverson_laryngeal_2003}, \citealt[2--4]{iverson_germanic_2008}).~%; see also \citealt[44]{harbert_germanic_2007}). 
The spread-glottis principle and aspiration are absent in Wymysiöeryś. This absence is most likely due to interaction with Polish where tenseness and aspiration have never operated. Overall, however, the contribution of contact with Polish to the complexity of Wymysiöeryś is by far more positive than negative. 

\section{Conclusion}\label{sec:wymsorys:5}

This contribution demonstrates that Wymysiöeryś – a severely endangered moribund minority language – exhibits remarkable complexity. Therefore, its severe endangerment and moribund status are not correlated with structural simplicity – at least, in the variety used by fluent speakers.\footnote{As explained in Section \ref{AAsect1}, the semi-speakers of Wymysiöeryś, who did not learn the language properly in intergenerational transmission and rarely (if ever) use it, exhibit radical simplification and impoverishment. However, they have no bearing on general language use, the transmission of Wymysiöeryś to the younger generations, and the language's structure overall.} The surplus of complexity is largely attributable to the transfer of elements from the dominant code, Polish. This confirms the view of language contact as not only having simplifying effects on languages, but also as contributing to their complexification – even in the situation of seemingly imminent language death.

The analysis of local complexities pertaining to diverse phonetic{\slash}phonological features (monophthongs, diphthongs, triphthongs, vocalic length, vocalic sonorants, nasality, consonants, consonantal length, palatalization, aspiration, onset and coda clusters) and their subsequent combination into a global relational value demonstrate the following: (a) locally, the complexity of Wymysiöeryś is typically superior or equal to that of Middle High German and Modern Standard German; (b) module-globally, the complexity of Wymysiöeryś is greater than the overall complexity of Middle High German and Modern Standard German; (c) both locally and globally, the surplus of information exhibited by Wymysiöeryś – and, thus, the positive difference in complexity when compared with the two control languages – can, in its largest part, be attributed to the contact with Polish. That is, by assimilating various Slavonic properties, and simultaneously maintaining its inherited or internally developed Germanic traits, the sound system of Wymysiöeryś is richer than the systems of its mother and at least some of its sister languages.

While this research only demonstrates the contact-induced complexification of the sound-system module of Wymysiöeryś, it is likely that a similar increase in complexity would be observed in other modules, whether morphology, syntax, or vocabulary. The likelihood of such complexifications is motivated by the general trend exhibited by Wymysiöeryś, namely the simultaneous maintenance of the Germanic foundation and its enhancement by Polish elements – a trend that goes beyond accidental complexity fluctuations \citep{Andrason2021}. However, since in any given language, the complexities of different modules generally need not coincide, this hypothesis must be verified in future studies.

\sloppy
\printbibliography[heading=subbibliography,notkeyword=this]
\end{document}
