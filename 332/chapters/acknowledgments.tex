\addchap{Acknowledgments}
\begin{refsection}

The impulse for \textit{Contemporary research in minority and diaspora languages of Europe} arose from a panel entitled   ``European minority and diaspora languages'', organized by the two editors. The session was part of the 10th International Conference on Language Variation in Europe (ICLAVE-10), held in Leeuwarden, the Netherlands in 2019. Contributors to that panel offered presentations outlining the wide variety of linguistic variation in diaspora and minoritized languages within and outside of Europe. They presented a broad array of content which touched upon phonetic, phonological, morphological, syntactic and even semantic variation. This was a spirited and engaging event that extended well past the allotted time -- but things really picked up in the post conference dinner which inspired several parallel discussions on diaspora languages that lasted late into the night. Those exchanges enabled us to distill some overarching themes across different contexts: Phenomena such as diachronic and synchronic tendencies in variation alongside sociolinguistic and ethnolinguistic factors that are connected with contact phenomena, among others. In the early hours of the following morning, we converged on the idea of this book. Although not every contributor to the book was present at the ICLAVE-10 event, nor is every presenter in that panel included in this book, it was that discussion which served as the impulse for this book. Thank you to all those who stayed up late mulling over language change and contact, and thank you also to all those who participated in the panel and to the conference organizers of ICLAVE-10 at the Fryske Akademy.

It should go without saying that we are grateful to all the contributors to this book who shared their expertise in their own chapter and in the book overall. Collaborating on a collection of articles like this one with so many authors is no easy endeavor -- especially when at any given moment, many contributors are doing fieldwork or on sabbatical (or navigating the new academic landscape carved by the coronavirus pandemic). Therefore, we are grateful for the sustained collaboration of all authors who prioritized the writing, revisions, and final edits of their respective chapter. Likewise, we are very appreciative of the anonymous reviewers and copy editors who helped ensure the scientific quality and consistency in style and form. Their input helped transform a collection of essays into a consistent and coherent book. We are also very grateful to Sebastian Nordhoff at Language Science Press who answered innumerable questions about the production process. Finally, a big thank you to Sjors Weggeman, student assistant at the University of Groningen, who provided critical support with the final preparation of the manuscript. 

The preparation and publication of this book was funded by the EU Horizon 2020 research and innovation program under the Marie Skłodowska-Curie grant agreement No 778384. It was developed within the project \textit{Minority Languages, Major Opportunities. Collaborative Research, Community Engagement and Innovative Educational Tools} (COLING).

\end{refsection}
