\documentclass[output=paper]{langsci/langscibook} 
\title{Grammaticalization of tense\slash aspect\slash\newlineCover mood marking in Yucatec Maya} 
\author{%
 Christian Lehmann\affiliation{University of Erfurt}
}
\ChapterDOI{10.5281/zenodo.823244} %will be filled in at production

\abstract{%
Maybe the most pervasive among the changes analyzable as cases of grammaticalization in the languages of the Yucatecan branch of the Mayan stock is the formation of auxiliaries that allow finer tense\slash aspect\slash mood distinctions than the status suffixes inherited from Proto-Mayan. It has been continually productive since colonial times. While this amounts to a replacement of the status system, it follows strictly language-internal patterns. And while the source constructions form a rather heterogeneous set, they converge onto a common TAM auxiliary pattern in Modern Yucatecan.}

\maketitle
\begin{document}

% \textsf{\textbf{Keywords}}\textsf{:} grammaticalization, \isi{periphrastic} aspect, \isi{auxiliary}, \isi{motion-cum-purpose construction}, focused progressive

\section{Introduction}\label{sec:lehmann:1}
This study is devoted to the grammaticalization of auxiliaries in \ili{Yucatec} \ili{Maya}, whose functional side is the formation of a complex \isi{tense}\slash aspect\slash mood (TAM) system. In this, it aims at fulfilling several purposes at once. It is, in the first place, a contribution to a historical grammar of \ili{Yucatec} \ili{Maya}. To this end, it brings together a large set of data, contextualized in their historical situation. A side effect of this enterprise is a diachronic perspective on the system of present-day \ili{Yucatec} \ili{Maya}, which may, as usual, open an additional, viz. dynamic, dimension of understanding it. On the other hand, the analysis tries to systematize the facts in terms of a theory of grammaticalization so that they may become comparable with relevant facts of other languages. To secure understanding for the non-specialist, some elements of \ili{Mayan} grammar will be explained in \sectref{sec:lehmann:3}.



Some of the data used are actually in a diachronic relationship, viz. data from the history of \ili{Yucatec} \ili{Maya}. Most of the data of other \ili{Mayan} languages belong to recent stages of their evolution. Following established methodology, they will be projected onto the diachronic axis and be taken to represent stages of a development.



A word is necessary on the orthography. \ili{Yucatec} \ili{Maya} has had distinctive vowel length and tone at least for the period of its documented history, although it does not share tone with any of its sisters. Moreover, the glottal stop and /h/ are phonemes, and both can form a syllable coda. Since all of this is alien to \ili{Spanish}, the orthography of the Colonial \ili{Yucatec} \ili{Mayan} %%%(CYM) %%% deleted at Lehmann's request
sources hides important phonological information. These phonological properties have been marked consistently in the orthography only from the second half of the twentieth century on. For this reason and in order to facilitate diachronic comparison to the non-specialist, examples from Colonial \ili{Yucatec} \ili{Maya} are first quoted literally from the sources and then coupled with a representation in contemporary scientific orthography (which is, alas, not the one adopted currently by Mexican authorities; s. \citealt{Lehmann2015}). %%%Likewise, since examples of other \ili{Yucatecan} languages are taken from different sources, they are first quoted in the writing used in their source and then converted into the system employed for \ili{Yucatec} in order to facilitate comparison.%%% Deleted at Lehmann's request


\section{Prehistory and history of Yucatec Maya}\label{sec:lehmann:2}

The \ili{Mayan} languages of today are spoken in a culture area called Mesoamerica. Some of the \ili{Mayan} languages are so dissimilar that they may have branched off from the common stock as early as 2000 BC. The \ili{Yucatecan} branch was the second to separate from the rest of the \ili{Mayan} family. This took place during preclassical times in terms of \ili{Mayan} history, at the latest about 1000 BC. Both genetically and geographically, the closest neighbor is the branch of the Ch'olan languages, which are clearly mutually unintelligible with the \ili{Yucatecan} languages. The \ili{Yucatecan} languages are spoken on the peninsula of Yucatán and in more southern regions of the lowland in Belize, the Petén region of Guatemala and the Mexican state of Chiapas. The internal subdivision of this branch is relatively recent. It has the form shown in %Diagram 1. 
\figref{fig:lehmann:1}.


\begin{figure}
\caption{The Yucatecan branch of the Mayan languages}\label{fig:lehmann:1}  
%%%\includegraphics[width=\textwidth]{figures/a7Lehmann-img1.png}
\begin{forest} for tree={grow=east, draw, anchor=west}
[\ili{Yucatecan} [\ili{Mopán},tier=word] [~  [\ili{Itzá}, tier=word] [~ [\ili{Lacandón},tier=word] [\ili{Yucatec}]]  ] ]
\end{forest}
\end{figure}

\ili{Mopán} on the one hand and the other \ili{Yucatecan} varieties are hardly mutually intelligible and are commonly regarded as different languages. The latter three varieties do not differ more from each other than British and American English. The period of their separation does not exceed a few hundred years and is, thus, far shorter than the period of separation of the dialects of \ili{German}, British English or \ili{Italian}. They are mutually intelligible and should be regarded as dialects of one language rather than as distinct languages.


\ili{Mopán} split off at the end of the first millennium AD. The \ili{Itzá} people apparently emigrated from the peninsula to the Petén in the fourteenth century, although keeping contact with \ili{Yucatec} Mayas. The \ili{Lacandón} people, too, are Mayas of Yucatán who retreated from the peninsula into the woods of Chiapas in order to avoid contact with the Mexican civilization. The closest relative of \ili{Yucatec} is (Southern) \ili{Lacandón}. It is a dialect that split off the main variety in the 18\textsuperscript{th} century and preserves some archaic traits. The periods of the history of \ili{Yucatec} \ili{Maya} itself may be depicted as in %Diagram 2.
\figref{fig:lehmann:2}.

\begin{figure}
\caption{Periods of Yucatec language history}\label{fig:lehmann:2}
% \small
% \begin{tabularx}{\textwidth}{p{5mm}|p{5mm}|p{5mm}|p{5mm}|p{5mm}|p{5mm}|p{5mm}|p{5mm}|p{5mm}|p{5mm}|p{5mm}|p{5mm}|p{5mm}|p{5mm}|p{5mm}|p{5mm}|p{5mm}|}
% % \lsptoprule
% 
% \multicolumn{2}{|c|}{Proto- Maya} & 
% \multicolumn{7}{|p{35mm}|}{Proto- Yucatecan} & 
% \multicolumn{5}{|p{25mm}|}{Pre-Columbian Yucatec} & 
% Colonial \ili{Yucatec} & 
% \multicolumn{2}{|c|}{Modern Yucatec}\\
%  \\
% & \multicolumn{2}{|c|}{ {}-1500} & \multicolumn{2}{|c|}{ {}-1000} & \multicolumn{2}{|c|}{ {}-500} & \multicolumn{2}{|c|}{ 0} & \multicolumn{2}{|c|}{ 500} & \multicolumn{2}{|c|}{ 1000} & \multicolumn{2}{|c|}{ 1500} & \multicolumn{2}{|c|}{ 2000}\\
% % \lspbottomrule
% \end{tabularx}
% 
% \newlength{\clfactor}
% \setlength{\clfactor}{6mm}
% \newcommand{\clbox}[2]{\fbox{\rule{2pt}{2cm}\parbox{#1\clfactor}{#2}}}
% \newcommand{\yearbox}[2]{\fbox{\parbox{#1\clfactor}{#2}}}
% 
% \clbox{2}{Proto-Maya}%
% \clbox{7}{Proto-\ili{Yucatecan}\\}%
% \clbox{5}{Pre-Columbian Yucatec}%
% \clbox{1}{Colonial Yucatec}%
% \clbox{2}{Modern Yucatec}
\chronoperiodecoloralternation{langscicol1,langscicol4,langscicol3,langscicol2} 
\startchronology[startyear=-2500,endyear=2017,startdate=false,stopdate=false,width=.9\textwidth]
\chronoperiode[datesstyle=\scriptsize,textstyle=\small]{-2500}{-1500}{Proto-Maya}
\chronoperiode[datesstyle=\scriptsize,textstyle=\small]{-1500}{250}{Proto-Yucatecan}
\chronoperiode[datesstyle=\scriptsize,textstyle=\small]{250}{1500}{\parbox[t]{3cm}{Pre-Columbian\\Yucatec}}
\chronoperiode[datesstyle=\scriptsize,textstyle=\clrotate]{1500}{1750}{Colonial Yuc.}
\chronoperiode[datesstyle=\scriptsize,textstyle=\clrotate]{1750}{2017}{Modern Yuc.}
\stopchronology
\end{figure}

The inscriptions and codices of the Pre-Columbian \ili{Mayan} culture span a period from roughly 250 to 1500 AD. They represent some Ch'olan language and are therefore relatively close to Pre-Columbian \ili{Yucatec}. However, the glyphic writing as it has been deciphered up to now does not represent the morphology of the language very well, so that for our purposes, written documentation of the language starts with the \ili{Spanish} conquest.


\ili{Yucatec} \ili{Maya} has been historically well attested since the early times of \ili{Spanish} colonization.\footnote{While most of the hieroglyphic texts appear to represent the Ch'olan branch, one or another of the surviving codices, which probably stem from the fifteenth century AD, may be in \ili{Yucatec}.} This period of the language history is called Colonial \ili{Yucatec} \ili{Maya}, often also Classical \ili{Yucatec} \ili{Maya}. Apart from having a longer documented history than most Amerindian languages, \ili{Yucatec} \ili{Maya} also boasts a set of early grammars and dictionaries as shown in %Diagram 3.
\figref{fig:lehmann:3}.

\begin{figure}
\caption{Colonial grammars and dictionaries of Yucatec Maya}\label{fig:lehmann:3}
% 
% \begin{tabularx}{\textwidth}{XXXXXXXXXXXXX} &  & \ili{Spanish} conquest of Yucatan &  & Diccionario de Motul & Coronel &  &  & San Buenaventura &  & Beltrán de Santa Rosa &  & \\
% \lsptoprule
% &  & 1546 &  & 1577 & 1620 &  &  & 1684 &  & 1746 &  & \\
%  &  &  &  &  &  &  &  &  &  &  &  & \\
% \multicolumn{2}{l}{ 1500} & \multicolumn{2}{l}{ 1550} & \multicolumn{2}{l}{ 1600} & \multicolumn{2}{l}{ 1650} & \multicolumn{2}{l}{ 1700} & \multicolumn{2}{l}{ 1750} & 1800\\
% \lspbottomrule
% \end{tabularx}

\chronoperiodecoloralternation{langscicol1,langscicol4,langscicol3,langscicol2} 
\startchronology[startyear=1500,stopyear=1800,startdate=false,stopdate=false,width=.9\textwidth]
\chronoperiode[datesstyle=\scriptsize,textstyle=\small]{1500}{1600}{}
\chronoperiode[datesstyle=\scriptsize,textstyle=\small]{1600}{1700}{}
\chronoperiode[datesstyle=\scriptsize,textstyle=\small]{1700}{1800}{}
\chronoevent[datesstyle=\scriptsize,textstyle=\clrotate]{1546}{\parbox{2.5cm}{\raggedright \ili{Spanish} conquest\\of Yucatan}}
\chronoevent[datesstyle=\scriptsize,textstyle=\clrotate]{1577}{\parbox{2.5cm}{\raggedright Diccionario de Motul}}
\chronoevent[datesstyle=\scriptsize,textstyle=\clrotate]{1620}{\parbox{2.5cm}{\raggedright Coronel}              }
\chronoevent[datesstyle=\scriptsize,textstyle=\clrotate]{1684}{\parbox{2.5cm}{\raggedright San Buenaventura}      }
\chronoevent[datesstyle=\scriptsize,textstyle=\clrotate]{1746}{\parbox{2.5cm}{\raggedright Beltrán de Santa Rosa}  }
\stopchronology
\vspace*{2\baselineskip}
\end{figure}

The earliest source is the \textit{Diccionario de Motul}\footnote{The \textit{Diccionario} or \textit{Calepino de Motul} was first published in \citet{MartínezHernández1929}. In the examples, it is referred to as \textit{Motul}.}, which some assume to be compiled around 1577.\footnote{Since its first published edition, the manuscript of the \textit{Diccionario de Motul} has been attributed to Fray Antonio de Ciudad Real (1551--1617) and been dated to 1577. Now he may well be the author, the more so as he is known to have worked on \ili{Mayan} language and culture until his death. However, he started living in Mérida only in 1573; and in 1577 he was 26 years old. Consequently, he either is not the author (but only a compiler of material gathered by others), or the year of completion must be much later. \citet[164--168]{Hanks2010} discusses the problem extensively and essentially pleads agnostic.} The earliest grammars – and still among the earliest sources of data for \ili{Yucatec} \ili{Maya} – are Coronel (1620)\footnote{In quotations, I use \citet{MartínezHernández1929} for the page numbering, as it reproduces the pagination of the original edition; but I quote the text from the (more reliable, but unpaginated) online edition of \url{http://www.famsi.org/reports/96072/coronelgmr.htm}. (The critical edition of \citealt{Coronel1998} was not available to me).} and San Buenaventura (1684). In the course of the eighteenth century, Colonial \ili{Yucatec} \ili{Maya} passed into Modern \ili{Yucatec} \ili{Maya} (MYM). Beltrán (1746) is assumed to mark the transition between the two stages \citep[4]{Smailus1989}.
\nocite{Ciudad1929}



Thus, the documented history of \ili{Yucatec} \ili{Maya} begins with colonial documents of the 16\textsuperscript{th} century. Its prehistory is indirectly represented in \ili{Mayan} hieroglyphic writing and may be accessed by internal reconstruction and historical comparison with cognate languages. Data from the other \ili{Yucatecan} languages are from the second half of the 20\textsuperscript{th} century. \ili{Lacandón} preserves some archaic traits, lending thus additional support to reconstructions.


\largerpage
Given all this, reconstruction of Proto-\ili{Yucatecan} is %%%therefore %% Deleted at Lehmann's request
in a comparatively favorable methodological situation. Not only can we reconstruct the diachrony by comparing four languages which are related closely enough to elucidate each other but different enough to provide variation which may be projected onto the diachronic axis. We also have 450 years of documented history in the case of \ili{Yucatec}, which can substantiate or falsify our diachronic hypotheses to some extent. Thirdly, grammarians have described different stages of the history for the same period, sometimes even noting explicitly grammaticalization phenomena observable at their time. Under such circumstances, responsible diachronic analysis may reach back approximately 1,000 years, which is about the point where Proto-\ili{Yucatecan} began to split up.


\section{Typological sketch of Yucatecan languages}\label{sec:lehmann:3}
All \ili{Mayan} languages are very much alike in their morphological and syntactic structure,\footnote{A recent typological overview of the \ili{Mayan} family is in  \citet{GrinevaldPeake2012}.} with some of the more principled differences being taken up below. The lexemes and the grammatical morphemes filling the structural slots are generally cognate within each of the subfamilies, while there are great differences among the subfamilies in this respect. Con\-sequently, while the \ili{Yucatecan} languages form a homogeneous group, this subgroup differs from other subgroups of the \ili{Mayan} family chiefly in the individual lexical and grammatical morphemes and, to a lesser extent, in grammatical structure. We will here focus on the grammatical structure of the \ili{Yucatecan} subfamily and mention deviations from Proto-\ili{Mayan} \textit{suo loco}.



Apart from numeral classifiers, the typologically notable features of the word-class system are limited to the subclassification of the major classes. Both nouns and verbs are subclassified according to relationality: absolute and relational nouns differ in morphology and syntax similarly as do \isi{intransitive} and \isi{transitive} verbs. If the %%%root
valency of a stem includes a place for such an additional actant, then there is a pronominal index for it. If a clause lacks such an actant (no matter whether represented by an NP), the base must be derelationalized. And vice versa for an absolute or monovalent base. Moreover, besides pure verbs, there is a closed class of verboids which share all morphological and syntactic properties with verbs except that they do not inflect for status (\sectref{sec:lehmann:4.4}) and therefore do not combine with an \isi{auxiliary} (\sectref{sec:lehmann:4.5}).



\ili{Mayan} languages lack the category of case throughout. They do have a productive category of prepositions – most of them denominal in origin – but very few primary prepositions; and the \ili{Yucatecan} languages have only one fully grammaticalized \isi{preposition}, \textit{ti’} \textsc{Loc}, which marks the \isi{indirect object} (as in \ref{ex:lehmann:41} and \ref{ex:lehmann:42}b below), local and other concrete relations.\footnote{corresponding both etymologically and functionally to Ch'olan \textit{tyi}} Under these conditions, structural relations of modification are underdeveloped; the syntax is characterized by government. All dependency constructions are head-marking: indexes cross-reference the subject and \isi{direct object} of a verb, the \isi{possessor} in a nominal construction and the complement of a \isi{preposition}. The index is obligatory, the nominal dependent is optional. The verb with its cross-reference indexes, possibly preceded by an \isi{auxiliary} (s. \sectref{sec:lehmann:4.2}), constitutes a full clause. No nominal or pronominal constituents are necessary.



Alignment of fundamental syntactic relations was \isi{ergative} in Proto-\ili{Mayan}. Some \ili{Mayan} subfamilies have preserved this alignment to a large extent. The \ili{Yucatecan} languages show traces of syntactic ergativity in focus constructions \citep{Bricker1981}; but otherwise ergativity is restricted to a split in the index paradigm of the \isi{intransitive} \isi{predicate} conditioned by status, to which we return in \sectref{sec:lehmann:4.1}.



The morphology is characterized by a medium degree of synthesis. Most affixes are suffixes. Most of the morphology is agglutinative; still, there are, especially in \ili{Yucatec} \ili{Maya}, several internal modifications. While declension is comparatively simple, verbs inflect for many \isi{conjugation} categories. One of these must be singled out from the start as it plays an important role in subsequent sections: The first morpheme after the (simple or derived) verb stem is a so-called \isi{status suffix}, which comprises the subcategories of \isi{dependent status} proper, aspect and mood. It is illustrated by the dependent \isi{incompletive} suffix in \REF{ex:lehmann:3} below. Word formation includes compounding and \isi{derivation}, both in the nominal and in the verbal sphere. The entire verb \isi{derivation} is based on \isi{transitivity}: every stem is either \isi{transitive} or \isi{intransitive}; and this determines the allomorphy of \isi{conjugation} categories, especially of the status morphemes.



\ili{Mayan} languages lack a \isi{copula}.\footnote{Colonial \ili{Yucatec} \ili{Maya} features a suffix \textit{{}-h} \textsc{Cop}, exemplified in \REF{ex:lehmann:22}, which verbalizes nominal \isi{predicates}.} The \isi{word order} must have been left-branch\-ing in some remote pre-historic epoch. This is the environment in which the morphological categories marked by verb suffixes (s. \sectref{sec:lehmann:4.4}), and possibly the phrase-initial nominal determiners and modifiers, too, originated. The proto-language then switched to right-branching syntax; Proto-\ili{Mayan} was right-branching. To this day, \ili{Mayan} languages are left-branching or juxtapositive only in the nominal syntax, as shown in \tabref{tab:lehmann:2}; the rest of the syntax is\largerpage right-branching, as detailed in \tabref{tab:lehmann:1}.\clearpage

\begin{table}

\begin{tabular}{ll}
\lsptoprule

\isi{predicate} & subject\\
verb & actant\\
verbal complex & adjunct\\
\isi{auxiliary} & clause core\\
nominal group & relative clause\\
nominal group & nominal \isi{possessor}\\
\isi{preposition} & complement NP\\
conjunction & clause\\
\lspbottomrule
\end{tabular}
\caption{Right-branching constructions}
\label{tab:lehmann:1}
\end{table}


\begin{table}

\begin{tabular}{ll}
\lsptoprule

short \isi{adverb} & verb\\
adjective attribute & noun\\
numeral & numeral classifier\\
numeral complex & nominal group\\
determiner & nominal group\\
\lspbottomrule
\end{tabular}
\caption{Non-right-branching constructions}
\label{tab:lehmann:2}
\end{table}

(The vague wording of the \tabref{tab:lehmann:2} heading reflects the fact that some dependency relations inside the NP (or DP) are less than clear.) One might add to \tabref{tab:lehmann:2} the \isi{clitic} pronominal index preceding a verb or a possessed nominal and cross-referencing the subject or the \isi{possessor}, resp. (i.e. the “Set A” index of \sectref{sec:lehmann:4.1}).


Marked information structure provides for two sentence-initial positions to be occupied by main constituents, viz. the position of left-dislocated topical constituents and the \isi{focus position}. The maximum configuration was dubbed LIPOC (language-independent preferred order of constituents) in \citet[189ff]{Dik1981} and may be represented by %Diagram 4. 
\figref{fig:lehmann:4}. \REF{ex:lehmann:1} is an example.



\begin{figure}
\caption{Extended sentence structure}\label{fig:lehmann:4}

\begin{tabular}{lll}
[ left-dislocated topic & [ focus & \isi{extrafocal} clause ] ]\\
\end{tabular}
\end{figure}

\newpage
\ea\label{ex:lehmann:1}
Modern \ili{Yucatec} \ili{Maya}\\
\sn[]{\hspace{-1.5ex}\gll le      chaan  lak    he'l=a'\\
     \textsc{dem}    little    bowl    \textsc{prsv=r1}\\}
\sn[]{\hspace{-1.5ex}\gll  in      kiik        síih-mah-il                  ten\\
\textsc{a.1.sg} elder.sister  give.as.present-\textsc{prf-dep(b.3.sg)}  me\\}
\glt ‘this little clay bowl, my elder sister gave it to me’ (ACC\_0039)
\z



The left-dislocated constituent is marked by a referential \isi{enclitic}, R1 in \REF{ex:lehmann:1}. The paradigm contains an element (R3) which functions as a topicalizer if the deixis is neutral.\footnote{The \ili{Yucatecan} languages differ in the details. \ili{Itzá} continues Pre-Columbian grammar in allowing the topicalizer \textit{{}-e'/-eh} to follow – directly or at a distance – the \isi{deictic} clitics (\citealt[14f]{Hofling1991}). \ili{Lacandón} lacks the entire paradigm of referential clitics, including the topicalizer.} The focus itself (\textit{in kiik} in \REF{ex:lehmann:1}) is not marked, but the \isi{extrafocal} clause is marked by a dependent \isi{status suffix}, \textit{{}-il} in \REF{ex:lehmann:1} (s. \sectref{sec:lehmann:4.4}).


\section{Verbal categories}\label{sec:lehmann:4}

In this section, we will pursue the fate of some categories in the functional domain of \isi{tense}/aspect/mood in the \ili{Yucatecan} languages. The starting point will be Colonial \ili{Yucatec} \ili{Maya} as documented in the sources enumerated in \sectref{sec:lehmann:2}.


\subsection{Pronominal indexes}\label{sec:lehmann:4.1}

All \ili{Mayan} languages have at least three sets of personal pronominal \isi{formatives}. All but one of these paradigms are \isi{clitic} or bound and function as cross-reference indexes; the last is a set of independent personal pronouns. The main paradigms of bound indexes are called Set A and Set B in \ili{Mayan} linguistics. The functions of the pronominal sets are as follows:


\begin{itemize}
\item 
indexes of Set A cross-reference the \isi{possessor} of a nominal group and the actor of the \isi{transitive verb}. Moreover, in the split-subject marking languages including those of the \ili{Yucatecan} branch, they cross-reference the subject of an \isi{intransitive verb} in some verbal statuses (\sectref{sec:lehmann:4.4}). Thus, the syntactic function alignment based on the distribution of set A  is accusative.
\item 
indexes of Set B cross-reference the subject of a non-verbal clause and the undergoer of the \isi{transitive verb}. In the split-subject marking languages, Set B also cross-references the  \isi{intransitive subject} in the complementary subset of statuses. Thus, the syntactic function alignment based on the distribution of Set B is \isi{ergative}.

\item 
The free pronouns are reinforced forms of Set B forms. They appear as the complement of a \isi{preposition}, as left-dislocated topic and in \isi{focus position}. Some languages including \ili{Yucatec} \ili{Maya} have \isi{enclitic} variants which function as \isi{indirect object}, as does \textit{ten} in \REF{ex:lehmann:1}.
\end{itemize}

The labels “Set A" and “Set B" originate in the times of American structuralism. They are deliberately obscure and mnemonically unhelpful. We will nevertheless have to use them because the functions which might provide more practical labels are heterogeneous. At any rate, it may be helpful to bear in mind the following equivalences with more familiar labels of interlinear glossing:


\begin{itemize}
\item 
A = SBJ/POSS
\item 
B = ABS.
\end{itemize}

\tabref{tab:lehmann:3} shows the Modern \ili{Yucatec} forms of sets A and B. For 1\textsuperscript{s}\textsuperscript{t} person pl., the exclusive form is given. All of these pronominal elements are free forms at the stage of Proto-\ili{Maya}. The parenthesized glides are conditioned by a vowel-initial host of the pronominal index.



\begin{table}
\caption{Pronominal paradigms in Modern Yucatec Maya}

\begin{tabular}{llll}
\lsptoprule
     & & \cla{A} & \clb{B}\\
\midrule
 sg. & 1 & in (w-) & {}-en\\
     & 2 & a (w-) & {}-ech\\
     & 3 & u (y-) & ${\emptyset}$\\
\tablevspace
 pl. & 1 & k(a) & {}-o’n\\
     & 2 & a (w-)… -e’x & {}-e’x\\
     & 3 & u (y-)…-o’b & {}-o’b\\
\lspbottomrule
\end{tabular}
\label{tab:lehmann:3}
\end{table}

In all \ili{Mayan} languages, the Set A index precedes the possessed nominal, cross-referencing the \isi{possessor}. \REF{ex:lehmann:2} provides representative examples of the indexes with verbs:\newpage

 
\ea\label{ex:lehmann:2}
Modern \ili{Yucatec} \ili{Maya}\\
\ea 
\gll h    bin-\clb{ech}\\
  \textsc{pfv}  {go(\textsc{cmpl)-b.2.sg}}\\
\glt ‘you went’

\ex 
\gll t=\cla{u}      t'an-\clb{ech}\\
\textsc{pfv=a.3}  call(\textsc{cmpl)-b.2.sg}\\
\glt ‘he called you’
\z
\z 


The examples are in the \isi{completive status}, which triggers \isi{ergative} marking in all \ili{Mayan} languages. The Set A index immediately precedes the \isi{transitive verb}. The Set B index is a suffix to the verb.


In the \ili{Yucatecan} languages, Set A forms belong to a species of enclitics which are not banned from initial position. If they follow a word in the same phrase, they form a phonological unit with it. Since they syntactically depend on what they precede, they cliticize to what is, in grammatical terms, the wrong side. In the examples, clisis of Set A forms is marked by an equal sign (although some of the sources mistakenly write them as prefixes).


\subsection{Verbal clause structure}\label{sec:lehmann:4.2}

Tense, aspect and mood are verbal categories and therefore possible only in verbal clauses. Other kinds of \isi{predicates} have to be verbalized if these categories are to be specified. Therefore, we can narrow down the analysis to the verbal clause. With some simplification, the verbal complex has the structure shown in \figref{fig:lehmann:5}. \REF{ex:lehmann:3} is a \isi{transitive} finite verbal complex.



\begin{figure}
\caption{Transitive verbal complex}\label{fig:lehmann:5}
\begin{tabular}{c|ccc} 
\multicolumn{4}{c}{verbal complex}\\
& \multicolumn{3}{c}{finite verb}\\
 \cla{index A} & verb stem & -status & \clb{-index B}\\ 
\end{tabular}
\end{figure}

\ea\label{ex:lehmann:3}

Colonial \ili{Yucatec} \ili{Maya}\\
      u    ppaticech\\
\gll   u    p’at-ik{}-ech\\
  \textsc{a.3}  leave-\textsc{dep.incmpl-b.2.sg}\\
\glt ‘(that) he leaves you’ (Motul s.v. \textit{Hun c}\textit{h}\textit{ilbac})
\z



The basic clause structure is “\isi{predicate} – subject". If it is a verbal \isi{predicate}, the verbal complex of \figref{fig:lehmann:5} comes first, then follow the free complements and adjuncts. The most elementary independent verbal clause at the stage of Colonial \ili{Yucatec} consists of a verbal complex in \isi{completive status} and its dependents, as in \REF{ex:lehmann:4}.


\ea\label{ex:lehmann:4}

Colonial \ili{Yucatec} \ili{Maya}\\
      u    kamah          nicte  in      mehen\\
\gll   \cla{u}     k'am-ah-\clb{${\emptyset}$}      nikte'   \cla{in}     mehen\\
\textsc{a.3}  get-\textsc{cmpl-b.3.sg}  flower  \textsc{a.1.sg}  son\\
\glt ‘my son got the flower (i.e. got married)’ (Motul s.v. \textit{kamnicte})
\z


Already in Colonial \ili{Yucatec}, many verbal clauses are introduced by a formative which codes \isi{tense}, aspect or mood and which we will call an \isi{auxiliary} (see \sectref{sec:lehmann:4.5} for discussion of the appropriateness of this term). In Modern \ili{Yucatec}, this is the default for independent verbal clauses. At this stage, the verbal complex with its dependents as illustrated by \REF{ex:lehmann:4} only forms a \textsc{clause core}, while an independent declarative verbal clause generally (except in perfect status) requires an \isi{auxiliary} in front of it. \figref{fig:lehmann:6} formalizes this construction. The second clause of \REF{ex:lehmann:5} illustrates it with the recent past \isi{auxiliary}.



\begin{figure}
\caption{Verbal clause}\label{fig:lehmann:6}

\resizebox{\textwidth}{!}{\begin{tabular}{|c|c|c|c|c|c|c|}
\hline
\multicolumn{7}{|c|}{verbal clause}\\
\cline{2-6}
& \multicolumn{5}{c|}{ verbal clause core} & \\\cline{2-5}
& \multicolumn{4}{c|}{ verbal complex} &  & \\\cline{3-5} 
&  & \multicolumn{3}{c|}{ finite verb} &  & \\
\isi{auxiliary} & \cla{index A} & verb stem & -status & \clb{-index B} & dependents & referential \isi{clitic}\\
\cline{1-2}
\end{tabular}}
\end{figure}


\ea\label{ex:lehmann:5}
Modern \ili{Yucatec} \ili{Maya}\\
\gll     \cla{In}        watan=e'               mina'n                       way=e';         táant                   =\cla{u}    bin=e'.\\
	\textsc{a.1.sg}   wife=\textsc{top}    \textsc{neg.exist(b.3.sg)}    here=\textsc{r3} \textsc{rec.pst}  \textsc{=a.3}    go\textsc{(incmpl)=r3}\\
\glt ‘My wife isn't here; she just left.’ (BVS\_05-01-36.2)
\z



The last element in \figref{fig:lehmann:6} is the referential \isi{clitic} conditioned by some of the auxiliaries, the recent past \isi{auxiliary} being one of these.


\subsection{Nominalization}\label{sec:lehmann:4.3}

\ili{Mayan} languages generally lack an \isi{infinitive}. The verb has a set of non-finite forms, some with nominal (incl. adjectival), some with adverbial function. Here we are concerned only with bare deverbal nouns, so-called action nouns, and with the processes which do no more than convert a verbal into a nominal constituent.



From \isi{intransitive verb} bases, action nouns are formed by two such processes. For \isi{agentive} \isi{intransitive} verbs, the verb stem also serves as an \isi{action noun} stem, as in \textit{óok’ot} ‘dance’ and \textit{meyah} ‘work’. For inactive \isi{intransitive} verbs, an \isi{action noun}, or rather a process noun, is formed by suffixing a morpheme \textit{{}-Vl} to the verb root, where V is a copy of the \isi{root vowel}, as in \textit{wen-el} ‘sleep (n.)’ and \textit{kóoh-ol} ‘arrival’. Action nouns of \isi{intransitive} bases are optionally possessed by their underlying subject, as in \textit{in meyah} ‘my work’ and \textit{u wenel} ‘his sleep’. \REF{ex:lehmann:6} provides examples of \isi{intransitive} action nouns. (\ref{ex:lehmann:6}a), with an \isi{agentive} stem, lacks an index, while \#b and \#c show a Set A index in \textit{genitivus subjectivus} function.



\ea\label{ex:lehmann:6}
Colonial \ili{Yucatec} \ili{Maya}\\
\ea 
ti    canan\\
\gll   ti’    kanáan\\
\textsc{loc}  watch\\
\glt ‘for watching’ \citep[14v]{SanBuenaventura1684}
\ex 
et  hazac      ech      ti    in        hanal\\
\gll   ethas-ak-ech          ti’    =in      han-al\\
just.in.time-past-\textsc{b.2.sg}    \textsc{loc}  \textsc{=a.1.sg}    eat-\textsc{dep}\\
\glt ‘you arrived just in time (to meet me) at having my meal’ \citep[§299, p.132]{Beltrán1746}
\ex 
in      káti        a    benel\\
\gll  in      k’áat-ih    a    ben-el\\
  \textsc{a.1.sg}  want-\textsc{cfp}    \textsc{a.2}  go-\textsc{dep}\\
\glt ‘I want you to go’ \citep[51]{Coronel1620}
\z
\z 



\ea\label{ex:lehmann:7}
Colonial \ili{Yucatec} \ili{Maya}\\
\ea 
v    kin    ocçah\\
\gll   u    k’iin    ook-s-ah\\  
  \textsc{a.3}  day    enter-\textsc{caus-introv}\\
\glt ‘(it is) the sowing season’ \citep[56]{Coronel1620}
\ex 
in      káti        a    cámbeçic        in      mehén\\
\gll   in      k’áat-ih    a    kanbes-ik-${\emptyset}$    in      mehen\\
\textsc{a.1.sg} want-\textsc{cfp}   \textsc{a.2}  teach-\textsc{dep-b.3.sg}  \textsc{a.1.sg}  son\\
\glt ‘I want you to teach my son’ \citep[50]{Coronel1620}
\z
\z 

If the verbal base is \isi{transitive}, there are two possibilities. The first consists in introverting the base, i.e. detransitivizing it by suppressing the \isi{direct object} position. Once this is done, the stem is nominalized like an \isi{agentive} \isi{intransitive verb} stem, which means that the introversive stem also serves as an \isi{action noun}. Examples based on \isi{transitive} roots are \textit{xok} ‘read’ – \textit{xook} (read{\textbackslash}\textsc{introv}) ‘reading, study’ and \textit{k’ay} ‘sing’ – \textit{k’aay} ‘singing, song’. For derived \isi{transitive} stems, introversion is marked by the suffix \textit{{}-ah}: \textit{kambes} ‘teach’ – \textit{kambes-ah} (teach-\textsc{introv}) ‘teaching’ (as in \REF{ex:lehmann:74} below), \textit{hets’kun} ‘settle’ – \textit{hets’kunah} ‘settlement’. Such a form also appears in (\ref{ex:lehmann:7}a). The other possibility of nominalizing a \isi{transitive} base consists in providing it with the dependent \isi{status suffix} \textit{{}-ik} and accompanying it by the Set A and Set B indexes for subject and object. This is shown in (\ref{ex:lehmann:7}b).

The two nominalizing suffixes \textit{{}-Vl} and \textit{{}-ik} are glossed as \isi{dependent status} in (\ref{ex:lehmann:6}–\ref{ex:lehmann:7}). They will become \isi{incompletive} suffixes on their way to Modern \ili{Yucatec}. The appearance of the Set A index in front of the \isi{nominalized verb} is conditioned by rules of syntax which will not be detailed here. It suffices to note the following: In Modern \ili{Yucatec} \ili{Maya}, this element is missing (under coreference) from the purpose part of the \isi{motion-cum-purpose construction} if its verb is \isi{intransitive}, and occasionally also if it is \isi{transitive}. This will be taken up in \sectref{sec:lehmann:4.8}. In \ili{Lacandón}, \isi{incompletive} verbal complexes without a Set A index are widely used in nominalizations, as in \REF{ex:lehmann:8}.
 

\ea
\label{ex:lehmann:8}
\ili{Lacandón}\\
\gll   Ten  ti’   met-ik   baalche’,   Yum-eh.\\
I  \textsc{loc}  make-\textsc{incmpl}  honey.beer  lord-\textsc{voc} \\
\glt ‘I am for making honey beer, my lord.’ \citep[28]{Bruce1974} 
\z

The subordination of the nominalized verbal construction by the all-purpose \isi{preposition} \textit{ti’} illustrated by \REF{ex:lehmann:6} and \REF{ex:lehmann:8} deserves special attention. If the clause thus subordinated follows the \isi{main clause}, it may be a purpose clause. This is still so in Modern \ili{Yucatec} and \ili{Lacandón}, witness (\ref{ex:lehmann:9}–\ref{ex:lehmann:10}).


\ea\label{ex:lehmann:9} 
Modern \ili{Yucatec} \ili{Maya}\\
\sn[]{\hspace{-1.5ex}\gll       Meet      hum-p'éel  léech\\          
  make\textsc{(imp)}  one-\textsc{cl.inan}  trap\\}
\sn[]{\hspace{-1.5ex}\gll ti'    =k      léech-t-ik        le    haaleh=a'!\\
\textsc{loc}  \textsc{=a.1.pl}    trap-\textsc{trr-incmpl} \textsc{dem} paca=\textsc{r1}\\}
% % % Modern \ili{Yucatec} \ili{Maya}\\
% % % \gll Meet hum-p'éel léech ti' =k léech-t-ik le haaleh=a'!\\
% % % make\textsc{(imp)} one-\textsc{cl.inan} trap   \textsc{loc}  \textsc{=a.1.pl} trap-\textsc{trr-incmpl}  \textsc{dem} paca=\textsc{r1}\\ 
\glt ‘Make a trap for us to trap this paca!’ (RMC\_1993)
\z

\newpage
\ea\label{ex:lehmann:10}
\ili{Lacandón}\\
\gll       ts'a'      ten  t=in        wil-ik\\
  give\textsc{(imp)}  me  \textsc{loc=a.1.sg}  see-\textsc{incmpl}\\
\glt `give it to me for me to see' \citep[63]{Bruce1968}
\z


If, however, the \isi{subordinate clause} precedes the \isi{main clause}, the same \isi{preposition} instead conveys simultaneity of the situation of the \isi{main clause} with the background situation of the \isi{subordinate clause}. This is illustrated by \REF{ex:lehmann:11} with an \isi{intransitive} \isi{nominalized verb}. \REF{ex:lehmann:12}, with a \isi{transitive} one, shows that this reading also occurs if the \isi{nominalized clause} is postposed.


\ea\label{ex:lehmann:11}
Modern \ili{Yucatec} \ili{Maya}\\
\sn[]{\hspace{-1.5ex}\gll       hach  bin    t=u      t'úub-ul              k'iin=e'\\
  really  \textsc{quot} \textsc{loc=a.3} submerge{\textbackslash}\textsc{deag-incmpl}  sun/day=\textsc{top}\\}
 \sn[]{\hspace{-1.5ex}\gll táan    y-isíins-a'l          =u    yatan  yuum        ahaw\\
 \textsc{prog }    \textsc{a.3}{}-bathe-\textsc{incmpl.pass}  =\textsc{a.3}    wife    master/father  chief\\}
\glt ‘Exactly at sunset, the chief's wife was washed’ (HK'AN\_502)
\z

\ea\label{ex:lehmann:12}
Modern \ili{Yucatec} \ili{Maya}\\
\sn[]{\hspace{-1.5ex}\gll     Ki'mak  wáah  bin    y-óol    yuum        ahaw\\ 
  happy    \textsc{int}     \textsc{quot  }   \textsc{a.3}{}-mind  master/father  chief\\}
 \sn[]{\hspace{-1.5ex}\gll   t=u        yil-ik!\\
 \textsc{loc=a.3} see-\textsc{incmpl(b.3.sg)}\\}
\glt ‘How happy was the king to see him!’ (HK'AN\_527)
\z

We will meet this construction again at the genesis of the \isi{progressive aspect} (\sectref{sec:lehmann:4.7.3}).


\subsection{Status}\label{sec:lehmann:4.4}\largerpage[2]
In all \ili{Mayan} languages, the verb has a suffixal slot for a category called status, which comprises the subcategories of \isi{dependent status} proper, aspect and mood. These suffixes belong to the earliest layer of the diachrony (they must antedate the introduction of right-branching \isi{word order} in Proto-\ili{Mayan}) and are completely grammaticalized. This implies that they mostly lack a clear semantic function and are instead conditioned by the construction. While the category of status itself and most of its subcategories are shared among \ili{Mayan} languages, there is a great deal of heteromorphy among them, just as most statuses display a complicated allomorphy within each language.

All of verbal morphology and syntax depends on \isi{transitivity}. Every verb stem is either \isi{transitive} or \isi{intransitive}, and this can only be changed by derivational means.\footnote{Already \citet[§§107 and 150--158]{Beltrán1746} is quite explicit about this (cf. \REF{ex:lehmann:42} below), although his orthography represents neither tone nor the glottal stop, both of which play an important role in the morphological processes manipulating \isi{transitivity} \isi{distinctions}.} Transitivity is the major factor in conditioning allomorphy in status morphemes. The core of the paradigm of status morphemes for finite forms is shown in \tabref{tab:lehmann:4}, which presents the forms in colonial orthography. For lack of relevance to our discussion, \tabref{tab:lehmann:4} omits the \isi{imperative}, the perfect (only available for \isi{transitive} verbs, anyway) and some \isi{intransitive} \isi{conjugation} classes. “V" represents a copy of the \isi{root vowel}; “/" and parentheses indicate allomorphy.

\begin{table}
\caption{Status conjugation of Colonial Yucatec Maya}
\begin{tabular}{llcccc}
\lsptoprule
\multicolumn{2}{r}{ stem class} & \multicolumn{2}{c}{ intransitive} & \multicolumn{2}{c}{ transitive}\\
status & aspect/mood & basic & derived & \multicolumn{1}{c}{ basic} & derived\\
\midrule
plain & \isi{subjunctive} & {}-Vc & {}-n-ac & \multicolumn{1}{c}{ {}-Vb} & (-e)\\
& \isi{completive} & (-i) & {}-n(-ah)(-i) & \multicolumn{2}{c}{ {}-ah}\\
\tablevspace 
dependent & \isi{subjunctive} & \multicolumn{2}{c}{ {}-ebal} & \multicolumn{2}{c}{ {}-ic}\\
& \isi{completive} & {}-ci & {}-n-ici & \multicolumn{2}{c}{ {}-(i)ci/-i\footnotemark{}}\\
& \isi{incompletive} & \multicolumn{2}{c}{ {}-Vl} & \multicolumn{2}{c}{ {}-ic}\\ 
\lspbottomrule
\end{tabular}
\label{tab:lehmann:4}
\end{table}
\footnotetext{The allomorph \textit{{}-i} appears if the subject is the focus constituent of a cleft-construction.}

Transitive finite forms are preceded by Set A clitics and followed by Set B suffixes as shown in \figref{fig:lehmann:5}. Intransitive verbs, instead, take Set B suffixes in the plain forms, but Set A clitics in dependent forms. The \isi{finite verb} forms in \tabref{tab:lehmann:5} illustrate the status \isi{conjugation} of \tabref{tab:lehmann:4} for an \isi{intransitive} and a \isi{transitive} example verb.\footnote{The sources do not provide examples for all persons, so that some of the forms entered in \tabref{tab:lehmann:5} are constructed by the grammarians’ rules rather than primary data.}


\begin{table}
\caption{Examples of finite verb complexes in Colonial Yucatec Maya}

\begin{tabularx}{\textwidth}{llQp{4cm}}
\lsptoprule

\multicolumn{2}{r}{ stem class} & \isi{intransitive}  & \isi{transitive}\\ 
status & \multicolumn{1}{l}{aspect/mood} & (basic) &  (derived)\\
\midrule
plain & \isi{subjunctive} & {\itshape cim-ic-en}\newline 
		      ‘(that) I die’ & {\itshape in cambes-ech}\newline
					‘(that) I teach you’\\
& \isi{completive} & {\itshape cim(-i)-en}\newline 
		‘I died’ & {\itshape in cambes-ah-ech}\newline 
			    ‘I taught you’\\
\tablevspace
dependent & \isi{subjunctive} & {\itshape in cim-ebal}\newline
			  ‘(that) I may die’ & {\itshape in cambes-ic-ech}\newline 
						‘(that) I may teach you’\\
& \isi{completive} & {\itshape in cim-ci}\newline 
		‘(that) I died’ & {\itshape in cambes-ic-i-ech}\newline
				  ‘(that) I taught you’\\
& \isi{incompletive} & {\itshape in cim-il}\newline
		  ‘(that) I die’ & {\itshape in cambes-ic-ech}\newline
				    ‘(that) I teach you’\\
\lspbottomrule
\end{tabularx}
\label{tab:lehmann:5}
\end{table}

\largerpage[1.25]
{\interfootnotelinepenalty=10000 In the \ili{Yucatecan} languages, aspect plays a more important role than \isi{tense}. In Colonial \ili{Yucatec}, there is one grammaticalized \isi{tense}, the suffixal perfect (illustrated by \REF{ex:lehmann:1} above). Past time is optionally marked by the \isi{adverb} \textit{cuchi} (i.e. \textit{kuchih}) ‘formerly’ (Modern \ili{Yucatec} \ili{Maya} \textit{ka’ch-il}), but is otherwise implied by most occurrences of the \isi{completive aspect} (as in \REF{ex:lehmann:4}), which is essentially \isi{perfective}.\footnote{Traditional terminology in \ili{Mayan} linguistics designates as \isi{completive} vs. \isi{incompletive} what could also be called \isi{perfective} vs. \isi{imperfective}, were it not for the auxiliaries to be mentioned below, which go under the latter terms. See \citet{Vinogradov2016} for an attempt at semantically characterizing these two values of the status category.} Future is one of the senses of \isi{subjunctive} status and optionally coded by auxiliaries which we will come to in subsequent subsections.}

\largerpage Dependent status is used in the \isi{extrafocal} clause of a cleft-sentence (as in \REF{ex:lehmann:18} below) and in certain complement clauses, examples of which may be seen in (\ref{ex:lehmann:47}–\ref{ex:lehmann:48}) (b). Dependent status is, in fact, more frequent in the texts than plain status, especially in the \isi{incompletive}. It appears every time that the full verb is preceded by another main constituent or by an \isi{auxiliary}. Among the dependent statuses, the default is the \isi{incompletive}. As a matter of fact, the \isi{incompletive} dependent morphemes are nothing else than the nominalizers for \isi{intransitive} and \isi{transitive} verbs already reviewed in \sectref{sec:lehmann:4.3}.%
%Das wirft die Frage auf, wo die anderen dependenten Formen herkommen. Evtl. sind sie gar nicht transitiv; cf. Buenaventura p. 17f und Beltrán §§175, 241.
 These are the forms that we will meet most frequently in the \isi{periphrastic} constructions to be analyzed below. The \isi{completive} and \isi{subjunctive} dependent forms involve a high degree of syncretism, hardly occur in the texts, and even the colonial grammarians are not sure about their form and function. Some of the forms fossilize, but the two subcategories themselves disappear as the status category reaches the stage of the modern \ili{Yucatecan} languages. In other words, (apart from the perfect) the values of the status category in Modern \ili{Yucatec} are ‘\isi{subjunctive}’ and ‘\isi{completive}’\largerpage (erstwhile: plain) and ‘\isi{incompletive}’ (erstwhile: dependent).

There are more respects in which the paradigm of \tabref{tab:lehmann:4} is unstable. Its basic form, and the only form that a simple declarative sentence can be based on, is the plain \isi{completive}.\footnote{It seems that \ili{Mayan} languages are among those in which perfective aspect\is{perfective} is the default aspect for verbal clauses.{\widowpenalty = 10000}} All the other status forms occur in extended or complex or non-declarative sentences. The plain status obviously lacks the \isi{incompletive} subcategory. This means that any kind of imperfective aspect\is{imperfective} – and as we shall see, much semantic differentiation is possible here – requires marking beyond the paradigm of \tabref{tab:lehmann:4}, which entails complex constructions involving dependent statuses. The situation is similar in the other \ili{Mayan} languages. All of them have an \isi{incompletive} or imperfective aspect\is{imperfective}. There is, however, great heteromorphy; and mostly the syntactic conditions are as in the \ili{Yucatecan} branch, viz. an \isi{auxiliary} is needed in addition to the status morpheme \citep{Vinogradov2014}.

Colonial grammars start the description of verbal morphology with a category called present which involves \isi{incompletive} status. It will be analyzed extensively in \sectref{sec:lehmann:4.9}. It is a rather complex \isi{periphrastic} construction which is not at all basic to the system. It figures so prominently in the grammars essentially on account of a methodological mistake on the part of the grammarians (s. p. \pageref{lehmann:methomistake}). The first to recognize this is \citet[§§60, 172]{Beltrán1746}. He tentatively adduces as present a cleft-construction again containing the \isi{incompletive} \isi{dependent status}, which we must forego here.

The status paradigm is alive to this day, but given its high degree of grammaticalization, it is fragile. Several endings appear only \textit{in pausa} and are syncopated otherwise \citep[§§135--147]{Beltrán1746}. Some of the allomorphy is utterly complicated, syncretistic and constantly exposed to variation. For instance, while the \isi{subjunctive} of root transitives ends in \textit{{}-Vb} for \citet{SanBuenaventura1684}, \citet[§112]{Beltrán1746} says that this is now out of use, and the ending is \textit{{}-e} (as it used to be for derived transitives).

\subsection{Periphrastic aspects}\label{sec:lehmann:4.5}
There is a small set of syntagmatic positions at the left clause boundary, i.e. following any left-dislocated topic as shown in \figref{fig:lehmann:4} and immediately preceding the clause core. These positions may be plotted as in \figref{dig:lehmann:7}:\footnote{The left-dislocated topic of \figref{fig:lehmann:4} precedes (all the positions shown in) \figref{dig:lehmann:7}. The rest of \figref{fig:lehmann:4} is a cleft-construction. However, a focused constituent may also precede a full clause, as shown in \figref{dig:lehmann:7}.}\largerpage[2] 

\begin{figure}
\caption{Clause-initial syntagmatic positions}\label{dig:lehmann:7}
\begin{tabular}{cccc}
\midrule
 a & b & c & \multirow{4}{3cm}{Verbal Clause Core}\\
 Conjunction & Focus & Auxiliary & \\
\cmidrule{1-3}
\multicolumn{3}{c}{d} & \\
\multicolumn{3}{c}{Superordinate Predicate} & \\
\midrule 
\end{tabular}
\end{figure}

\begin{enumerate}[label=\alph*.]
\item The Conjunction slot may be occupied by conjunctions and other sentence-initial particles, as the conjunction in (\ref{ex:lehmann:16}b) and the negator of (\ref{ex:lehmann:20}b) and (\ref{ex:lehmann:41}).
\item The Focus slot may be occupied by focused constituents, as in \REF{ex:lehmann:18}.
\item The Auxiliary slot may be occupied by grammaticalized auxiliaries, such as \REF{ex:lehmann:28}.
\item Instead of all of this, a verbal clause core may be preceded by a superordinate \isi{predicate} like the \isi{phase verb} in \REF{ex:lehmann:47}, the \isi{modal} verboid in \REF{ex:lehmann:23} and one of the non-grammaticalized auxiliaries to be analyzed in \sectref{sec:lehmann:4.7}. While the \isi{positional} relation between any of the elements of \#a – \#c and the verbal clause core appears to be the same as the \isi{positional} relation between such a superordinate \isi{predicate} and the verbal clause core, the syntactic relation is different, since the superordinate \isi{predicate} is not, of course, a constituent of the clause in question, but rather takes the clause core as a dependent, as shown in  \figref{dig:lehmann:9} below.
\end{enumerate}

Distributional relations between elements of the three classes shown in \figref{dig:lehmann:7} are complex, involving several conditions of mutual exclusion. In any case, none of the three slots is occupied obligatorily, and most frequently only one of them is occupied. As a consequence,  any of the four kinds of elements mentioned in \#a – \#d may form a binary construction with an ensuing clause core. This is a structural pattern apparently inherited from Proto-\ili{Mayan}. It is an important presupposition for a reanalysis by which any such element may be reinterpreted as an \isi{auxiliary}. As we will see, elements occupying slots \#b – \#d are, in fact, frequently so reanalyzed.\footnote{In terms of \citet[esp. 511--513 and 535f]{Bisang1991}, the \isi{auxiliary position} of \figref{dig:lehmann:7} is an “attractor position”, that is, a position which acts as a melting-pot for material recruited from different sources and grammaticalized in this position.}

Since the material ending up in the Auxiliary position of \figref{dig:lehmann:7} is so heterogeneous, its relation to the rest of the clause differs accordingly, and consequently the constructions with slot fillers of the four above kinds are syntactically different. The differences are reflected morphologically on the full verb, which depending on the construction is in the dependent \isi{incompletive}, the \isi{completive} or the \isi{subjunctive} status. As we will be concentrating on such constructions in which the element in question gets grammaticalized to an \isi{auxiliary}, the result is that the \isi{auxiliary} conditions the status. \figref{fig:lehmann:8} takes up \figref{fig:lehmann:6} and in addition visualizes this dependency.

\begin{figure}
\caption{Syntagmatic relation between auxiliary and status}\label{fig:lehmann:8}
\resizebox{\textwidth}{!}{\begin{tabular}{|c|c|c|c|c|c|c|}
\hline
\multicolumn{7}{|c|}{verbal clause}\\
\cline{2-6}
& \multicolumn{5}{|c|}{ verbal clause core} & \\\cline{2-5}
& \multicolumn{4}{|c|}{ verbal complex} &  & \\\cline{3-5} 
&  & \multicolumn{3}{|c|}{ finite verb} &  & \\
\tikz[remember picture,baseline] \node[anchor=base] (lehmannAux) {auxiliary}; & \cla{index A} & verb stem & \tikz[remember picture,baseline] \node[anchor=base] (lehmannStatus) {-status}; & \clb{-index B} & dependents & \tikz[remember picture, baseline] \node[anchor=base] (lehmannClitic) {referential clitic};\\
\multicolumn{7}{|c|}{~}\\
\multicolumn{1}{|c}{~} & \multicolumn{2}{c}{\tikz[remember picture,baseline] \node[anchor=base] (lehmannCond) {\itshape conditions};} & \multicolumn{4}{c|}{~}\\
% % \cline{1-1} % Deleted on Lehmann's request
\end{tabular}
\begin{tikzpicture}[remember picture,overlay]
\draw (lehmannAux) |- (lehmannCond.west); \draw[-{Stealth[]}] (lehmannCond) -| (lehmannStatus); \draw[-{Stealth[]}] (lehmannCond) -| (lehmannClitic);
\end{tikzpicture}}
\end{figure}

The first thing to be noted about \figref{fig:lehmann:8} is that the full verb is finite. This is a peculiarity of \ili{Yucatecan} \isi{periphrastic} constructions whose diachronic explanation will become clear in the following sections. As already shown in \figref{fig:lehmann:6}, in the \ili{Yucatecan} languages, the pronominal indexes do not combine with the \isi{auxiliary}, but with the full verb. Thus, the \isi{auxiliary} deserves its name only insofar as it carries \isi{tense}/aspect/mood information. Person and number, however, are marked on the full verb, and consequently it is indeed finite. The discussion of the applicability of the \isi{auxiliary} concept to this class of \isi{formatives} will be taken up in \sectref{sec:lehmann:4.10.2}.

There is in \ili{Yucatec} a large variety of tenses, aspects and moods that are coded in the initial position of \figref{fig:lehmann:8}.\footnote{An extensive list of relevant markers appears in  \citet[ch. 1.2f.]{BriceñoChel2006}} None of the colonial grammars provides a systematic account of them. There are at least two reasons for this. Firstly, these grammars depend on the model of \ili{Latin} grammar, which almost totally lacks auxiliaries, \isi{conjugation} being essentially synthetic. Secondly, virtually none of the auxiliaries of Colonial \ili{Yucatec} \ili{Maya} is inherited and, thus, firmly entrenched in the system. While the clause-initial \isi{auxiliary} is a Pan-\ili{Mayan} category, practically all of the extant \isi{formatives} of this category emerge at the time of the first colonial grammarians. With the exception of the \isi{auxiliary} described in \sectref{sec:lehmann:4.9}, none of the incipient auxiliaries made its way into their \isi{conjugation} paradigms; instead, they throw those that they are aware of into the basket of particles. They do, however, use them in their examples.

The following subsections will pursue the grammaticalization of the subset of the \isi{tense}/aspect/mood auxiliaries of \ili{Yucatec} \ili{Maya} shown in \tabref{tab:lehmann:13}. This is less than half of the auxiliaries actually in use. Among the ones missing from \tabref{tab:lehmann:13} are three past time auxiliaries (recent [illustrated by \REF{ex:lehmann:5}], relative and remote past), the \isi{obligative}, potential and volitive moods illustrated below in \REF{ex:lehmann:23} and a commissive or assurative future. For a subset of these, the origin is unknown. None of the auxiliaries to be discussed here triggers the final referential \isi{clitic} mentioned in \sectref{sec:lehmann:4.2}, so it will be left out of consideration. The last column of \tabref{tab:lehmann:13} indicates the status that the auxiliaries trigger on the full verb. By this criterion, there are four structural subclasses of auxiliaries and four different \isi{auxiliary} constructions, each illustrated by one example in \REF{ex:lehmann:13}.


\begin{table}
\begin{tabular}{lll}
\lsptoprule
form & function & status conditioned\\
\midrule 
t-/h- & \isi{perfective} & \isi{completive}\\
\tablevspace
k- & \isi{imperfective} & \isi{incompletive}\\
\tablevspace
táan & progressive & \multirow{4}{*}{subjunctive} \\
ts'o'k & terminative & \\
yan & debitive/future & \\
bíin & predictive future & \\
\tablevspace
bin … ka'h & \isi{immediate future} & \isi{incompletive}/\isi{subjunctive}\\
\lspbottomrule
\end{tabular}
\caption{Some Yucatec tense/aspect/mood auxiliaries}
\label{tab:lehmann:13}
\end{table}


\ea\label{ex:lehmann:13}
Modern \ili{Yucatec} \ili{Maya}\\
\ea 
\gll h    lúub-en\\
  \textsc{pfv}  fall(\textsc{cmpl)-b.1.sg}\\
\glt ‘I fell’
\ex 
\gll k=in        lúub-ul\\
  \textsc{ipfv=a.1.sg}  fall-\textsc{incmpl}\\
\glt ‘I fall’
\ex 
\gll bíin  lúub-uk-en\\
\textsc{fut}  fall-\textsc{subj-b.1.sg}\\
\glt ‘I will fall’
\ex 
\gll bin      =in      ka’h    lúub-ul\\
\textsc{imm.fut}  =\textsc{a.1.sg}    do      fall-\textsc{incmpl}\\
\glt ‘I am going to fall’
\z
\z 

From this presentation, it appears that the categories in question are coded twice, both by the introductory \isi{auxiliary} and by the status morpheme. The question naturally arises why each \isi{auxiliary} goes with a different status. This problem will be analyzed in the following subsections. We will see that all the auxiliation constructions come about by grammaticalization, but that they originate from different sources.

Another difference between the statuses strikes the eye: Some of them have the \isi{intransitive subject} represented by a Set A index, while others have it represented by a Set B index. This is the alignment split already mentioned in \sectref{sec:lehmann:3}. Although it is not the main object of the ensuing analyses, these will nevertheless contribute to its understanding.

An item of methodology in the analysis of the grammaticalization of these auxiliaries is to be introduced here. At the point when an item is recruited to fill the clause-initial syntactic position, it is a word or even a phrase. Continuing grammaticalization then reduces auxiliaries to bound morphemes (illustrated by (\ref{ex:lehmann:13}a) and (\ref{ex:lehmann:13}b)). There are two tests for the structural status of an \isi{auxiliary}. First, as in many languages, the answer to a polar interrogative in \ili{Mayan} involves repeating the main \isi{predicate} with positive or negative polarity. From this we can derive a test to determine the main \isi{predicate} of a sentence. In principle, in a configuration like \figref{fig:lehmann:8}, either the \isi{auxiliary} or the finite full verb may be the main \isi{predicate}. The \isi{auxiliary}, however, can be the main \isi{predicate} only if it is a word. As we shall see, at the beginning of the process, the \isi{auxiliary} does indeed constitute the answer to a \isi{polar question}, while with advanced grammaticalization, this is no longer possible, and a short version of the verbal clause appears instead. The second test on the status of the \isi{auxiliary} involves the placement of \isi{enclitic} particles. Some of them occupy Wackernagel’s position. They may therefore immediately follow the \isi{auxiliary} if this is a word; and otherwise they must follow the full verb. One might think that the Set A indexes, which are \isi{enclitic} to the \isi{auxiliary}, already provide this test. However, these coalesce with the \isi{auxiliary} once this forfeits its word status and therefore become useless for the test.

\subsection{Auxiliation based on modification: from hodiernal past to perfective}\label{sec:lehmann:4.6}
As explained in \sectref{sec:lehmann:4.4} and illustrated by \REF{ex:lehmann:4}, the Colonial \ili{Yucatec} \ili{Maya} \isi{completive status} is the only one that a simple independent declarative clause may be based on (i.e. without the need for an \isi{auxiliary}).\footnote{Of course, \isi{imperative} sentences lack an \isi{auxiliary}, too.} This means, at the same time, that such clauses have little marking in comparison with all other \isi{tense}/aspect/mood categories appearing in independent sentences. Moreover, the \isi{completive} has zero allomorphs in several contexts. These may be the result of a phonological process, viz. syncope of the vowels appearing in the \isi{completive} line of \tabref{tab:lehmann:4} if this suffix is followed by a vowel; or else the overt allomorphs may be grammatically restricted to the position \textit{in pausa}.\footnote{The \isi{completive} endings are absent before a following vowel in \ili{Lacandón}, too. \citet[§3.3]{Coon2010} reports similar facts about Ch'ol.} Thus, the \isi{transitive} \isi{completive} suffix of \REF{ex:lehmann:14} and (\ref{ex:lehmann:20}b) would be zero in informal speech (as it would be in a Modern \ili{Yucatec} \ili{Maya} version of these examples); and likewise the \isi{intransitive} \isi{completive} suffix appearing in \REF{ex:lehmann:24} would normally be zero, as it is in \REF{ex:lehmann:42} from Colonial \ili{Yucatec} \ili{Maya}, in (\ref{ex:lehmann:13}a) from Modern \ili{Yucatec} \ili{Maya} and in \REF{ex:lehmann:15}.

\ea\label{ex:lehmann:14}
Colonial \ili{Yucatec} \ili{Maya}\\
      u    chabtahon            Dios\\
\gll   u    ch’ab-t-ah-o’n        dios\\
\textsc{a.3}  create-\textsc{trr-cmpl-b.1.pl}    god\\
\glt ‘god created us’ (Motul s.v. \textit{c}\textit{h}\textit{ab.tah.t})
\z

\ea\label{ex:lehmann:15}
\ili{Itzá}\\
\gll       Ka'  lub’(-ih)      ah  tikin    che'-eh ...  \\
%Could also be a' DET.
  then  fall-\textsc{cmpl(b.3.sg)}  \textsc{m} dry    wood-\textsc{top}\\
\glt `Then the dry tree fell ...’ (\citealt{Hofling1991}, 12:30)
\z

Anyway, the result is that many \isi{completive} verbal complexes occurring in texts reduce to verb stems provided with indexes. One might expect that such a formally weak category is ripe for reinforcement or renewal. This expectation will be only partially fulfilled.

In Colonial \ili{Yucatec} \ili{Maya}, the \isi{completive} clause can be marked for hodiernal \isi{completive}.\footnote{It is hodiernal past according to \citet[41f]{Coronel1620} and \citet[35r]{SanBuenaventura1684}, although in \citet[41]{Smailus1989} it is characterized as remote or anterior past. The treatment in Coronel is part of the section on \isi{dependent status}. The first examples of hodiernal past in plain status are in \citet{SanBuenaventura1684}.} This is achieved by the particle \textit{ti'} ‘there’ (or its prevocalic bound allomorph \textit{t-}), which may start out in the Focus position of \figref{dig:lehmann:7}, but anyhow ends up in the \isi{auxiliary position}. \REF{ex:lehmann:16} shows the simple plain \isi{completive} for an \isi{intransitive} (\#a) and a \isi{transitive} (\#b) verb. The two parts form minimal pairs with the \#a and \#b sentences of \REF{ex:lehmann:17}, which show the hodiernal \isi{completive}.\largerpage[2]

\ea\label{ex:lehmann:16}
Colonial \ili{Yucatec} \ili{Maya}\\
\ea 
Bini           Fiscal    ti     yotoch     ku,\\
\gll   bin-ih          fiscal    ti’    y-otoch    k’uh\\
go-\textsc{cmpl(b.3.sg)}  inspector  \textsc{loc}  \textsc{a}.3-house    god\\
\glt ‘The inspector went to the church’
\ex 
ca         vhaɔah     palalob\\
\gll   káa  =u    hats’-ah    paal-alo’b\\
\textsc{conj}  \textsc{=a.3}    beat-\textsc{cmpl}    child-\textsc{pl}\\
\glt ‘and beat the children’ \citep[23r-v]{SanBuenaventura1684}
\z
\z 

\ea\label{ex:lehmann:17}
Colonial \ili{Yucatec} \ili{Maya}\\
\ea 
ti      bini            padre\\
\gll  ti’      bin-ih          padre\\
  \textsc{hod}    go-\textsc{cmpl(b.3.sg)}  father\\
\glt ‘the father (reverend) went today / has gone’


\ex 
tin        haɔah      paal\\
\gll   t=in        hats'-ah    paal\\
\textsc{hod=a.1.sg}  beat-\textsc{cmpl}    child\\
\glt ‘I beat the child today / have beaten the child’ \citep[35r]{SanBuenaventura1684}
\z
\z 

Two facts should be noted: First, the \textit{ti’} functioning as \isi{auxiliary} here is based on the word \textit{ti’}, which is syntactically ambiguous between an \isi{adverb} and a \isi{preposition}. The \isi{adverb} is a deictically neutral local \isi{demonstrative} meaning ‘there’. The \isi{preposition} \textit{ti’} \textsc{Loc} appears in (\ref{ex:lehmann:16}a) and is seen to subordinate a \isi{nominalized verbal complex} in (\ref{ex:lehmann:6}a) and \REF{ex:lehmann:8} (\sectref{sec:lehmann:3}). The word occurs in both of these functions in \REF{ex:lehmann:58} below. While the \isi{preposition} governs the constituent following it and therefore presupposes \isi{dependent status} on it if it is based on a verbal construction, the \textit{ti’} presently at stake does not do this. The \isi{completive} morph in the verbal clause core remains unaffected by the addition of the \isi{auxiliary} in \isi{clause-initial position}. Consequently, this \isi{auxiliary} is based on the \isi{adverb}, not on the \isi{preposition}. The semantic shift from ‘there’ to \textsc{hodiernal} is obviously a metaphor from space to time. Second, the \isi{auxiliary} is the same for \isi{intransitive} and \isi{transitive} verbs.\footnote{In Ch'ol, the \isi{perfective} \isi{auxiliary} is \textit{tsa’} (shortened to \textit{tyi}) both for \isi{transitive} and \isi{intransitive} verbs.}

The specification of hodiernal past is possible in \isi{dependent status}, too:\footnote{\citet[17r]{SanBuenaventura1684} contends that the hodiernal past may trigger \isi{dependent status}, and gives two examples of it. These are probably due to conditions as obtain in (\ref{ex:lehmann:18}b).} the \#a sentence of \REF{ex:lehmann:18} illustrates simple \isi{completive}, the \#b sentence is its hodiernal counterpart. Here, too, the \isi{completive} morph is the same in both cases.\footnote{\citet[41]{Coronel1620} postulates a contrast between \isi{dependent status} suffixes for simple and hodiernal \isi{completive}; but this finds no support elsewhere.}\newpage


\ea\label{ex:lehmann:18}
Colonial \ili{Yucatec} \ili{Maya}\\
%\gll 
\ea 
 bal  v      chun  a      háɔci?\\
\gll ba'l  =u    chuun  =a    hats'-k-ih\\
  what  =\textsc{a.3}    ground  =\textsc{a.2}    beat-\textsc{dep-cmpl(b.3.sg)}\\
\glt ‘why did you beat her?’

%\gll 
\ex 
bal  v      chun  ta          háɔci?\\
\gll   ba'l  =u    chuun  t=a        hats'-k-ih\\
what  =\textsc{a.3}    ground  \textsc{hod=a.2}    beat-\textsc{dep-cmpl(b.3.sg}\textsc{)}\\
\glt ‘why have you beaten her?’ \citep[42]{Coronel1620}
\z
\z 

The hodiernal \isi{completive} is already highly grammaticalized in Colonial \ili{Yucatec} \ili{Maya}.\footnote{In translating it into English, one has the choice of either rendering the specific semantics and consequently using \textit{today} or else rendering the degree of grammaticity and thus using the perfect.} Already in \citet{Coronel1620}, some \isi{completive} examples introduced by \textit{ti’} are translated as \isi{simple past}. For instance, \REF{ex:lehmann:19} is translated as “Quien vino?”

\ea\label{ex:lehmann:19}
Colonial \ili{Yucatec} \ili{Maya}\\
      Macx   ti     tali?\\
\gll  makx  ti’    taal-ih\\
who    \textsc{hod}  come-\textsc{cmpl(b.3.sg)}\\
\glt ‘Who has come?’ \citep[48]{Coronel1620}
\z

In \citegen{Beltrán1746} examples – e.g. §§264f \textit{(t) luben} – the \isi{completive aspect} appears variously with and without the aspect \isi{auxiliary} \textit{t-}, with the same \ili{Spanish} translation \textit{caí} ‘I fell’ and no comment on any semantic difference. In §36, he admits that, in front of \isi{intransitive} verbs, the \textit{t} is “semipronunciada”, and establishes the variation taken up below. Apparently, the hodiernal component has disappeared, and what we now have is a \isi{perfective} \isi{auxiliary}, reduced to the phoneme \textit{t}, as in \REF{ex:lehmann:63} below, and therefore regularly univerbated with the following \isi{enclitic} Set A index, as evidenced by (\ref{ex:lehmann:17}b) and (\ref{ex:lehmann:18}b). In Modern \ili{Yucatec}, the \isi{perfective} \isi{auxiliary} has become obligatory with \isi{transitive} verbs in \isi{completive status}.


As for the tests for word status of this \isi{auxiliary}, it cannot be host to an \isi{enclitic} particle and cannot constitute the answer to a \isi{polar question}. The latter may be inferred from \REF{ex:lehmann:20}, where the answer has to contain the full verb.\newpage


\ea\label{ex:lehmann:20} \let\eachwordone=\small \let\eachwordtwo=\small
Colonial \ili{Yucatec} \ili{Maya}\\
\ea
ti      kamchijnech          ua.    l.    ta        kamah      ua  a      chij?\\
\gll   ti'     k'am-chi'-n-ech         wáa     \textit{o}:   t=a       k'amah   wáa   =a     chi'\\
\textsc{hod}    get-mouth-\textsc{cmpl-b.2.sg }    \textsc{int}    or:  \textsc{hod=a.2}  get-\textsc{cmpl}   \textsc{int}   \textsc{=a}.2    mouth \\
\glt ‘Have you had breakfast?’

\ex 
Ma  tin        kamah    in        chi.\\
\gll   ma   t=in         k'am-ah   =in       chi'\\
\textsc{neg} \textsc{hod=a.1.sg}  get-\textsc{cmpl}  =\textsc{a.1.sg}   mouth\\

  l.    ma    ti      kamchijnen.\\
\gll   \textit{o}:    ma'     ti’     k'am-chi'-n-en\\
or:  \textsc{neg }    \textsc{hod}    get-mouth-\textsc{cmpl-b.1.sg}\\
\glt ‘I have not had breakfast’ (Motul s.v. \textit{kamchij})
\z
\z 

If \textit{ti’} did start out in the Focus position of \figref{dig:lehmann:7}, anyhow it has lost focus function by the start of the documented history of \ili{Yucatec} \ili{Maya}, witness such examples as (\ref{ex:lehmann:18}b), where it follows the focus constituent. This is, then, the only \isi{auxiliary} which has already lost word status at the stage of Colonial \ili{Yucatec} and become a bound morpheme.

In\isi{transitive} \isi{completive} verbs get a Set B index suffixed, as seen, for example, in \REF{ex:lehmann:19}. The monophonematic \isi{auxiliary} therefore hits directly on the verb, which may start with a consonant, as in \REF{ex:lehmann:63}. \ili{Yucatec} has a phonological rule which converts /t/ into /h/ in front of /t/. An extended version of this rule may have applied to the \isi{perfective} \isi{auxiliary}. At any rate, this \isi{auxiliary} has an allomorph \textit{h} with \isi{intransitive} verbs. A preconsonantal /h/, however, generally disappears in \ili{Yucatecan}. The \textit{h} to be seen in (\ref{ex:lehmann:13}a) is optional both in speaking and in writing, but is mostly absent, as it is in \REF{ex:lehmann:15} and (\ref{ex:lehmann:16}a). One may speculate that what manifests itself in such cases is an uninterrupted continuation of the plain \isi{completive} of Colonial \ili{Yucatec} \ili{Maya}. This may be hard to settle. At any rate, since the hodiernal feature present at the beginning disappears, the result of the entire \isi{grammaticalization process} is a weak reinforcement of the inherited \isi{completive status}.

The picture of the \ili{Yucatecan} languages with regard to this \isi{auxiliary} is heterogeneous. \ili{Mopán} shows no trace of a \isi{perfective} \isi{auxiliary}, which may reflect the original situation illustrated by \REF{ex:lehmann:16}. \ili{Lacandón} has independent declarative clauses in \isi{completive status} with and without an \isi{auxiliary}. The latter is illustrated by \REF{ex:lehmann:21} (from the epic style).\newpage

\ea\label{ex:lehmann:21}
\ili{Lacandón}\\
\gll       K=u      yen-s-ik          =u    yok  lu’m    Hachäkyum    y-a’l-ah:\\
  \textsc{ipfv=a.3}  lower-\textsc{caus-incmpl }   \textsc{=a.3}    foot  earth    Hachäkyum    \textsc{a.3}{}-say-\textsc{cmpl}\\
\glt ‘When Hachakyum set his foot onto the land, he said:’ (\citealt[111]{Bruce1968} {\textasciitilde} \citeyear[19]{Bruce1974})
\z

No process is known by which the perfective aspect\is{perfective} \isi{auxiliary} would reduce to zero in such a context. Consequently, this may be a functional opposition like the one illustrated by (\ref{ex:lehmann:16}–\ref{ex:lehmann:17}). In \ili{Itzá}, the \isi{completive} only appears to be used with the \isi{perfective} \isi{auxiliary}. In both of these latter languages, the distribution of the allomorphs is essentially the same as in \ili{Yucatec}, except that the allomorph for \isi{intransitive} verbs is always zero.\footnote{\ili{Lacandón} has a subordinator combining with \isi{completive aspect}, viz. \textit{kahin} ‘when’ \citep[100]{Bruce1968}, corresponding to \ili{Yucatec} (\textit{le}) \textit{ka’h}. While the \ili{Yucatec} subordinator combines with the \isi{perfective} \isi{auxiliary}, the \ili{Lacandón} one apparently does not.} 

The \isi{perfective} is the only \isi{tense}/aspect/mood \isi{auxiliary} of the \ili{Yucatecan} branch that cooccurs with \isi{completive status}. The internal syntax of the hodiernal \isi{completive} construction which is its source differs from all the other \isi{auxiliary} constructions. The clause core does not depend on the \isi{auxiliary}, but is, instead, modified by it. There are, of course, many more adverbs which occupy the \isi{focus position} of \figref{dig:lehmann:7} and which, being mere modifiers, do not trigger any changes on the verb. However, in a language whose syntax is heavily based on government, a modifying construction is not a productive source for the grammaticalization of auxiliaries. The \isi{perfective} remains a loner as regards both the source of the \isi{auxiliary} and the status conditioned (or rather, conserved) by it on the verb. However, as we shall see, the more recent grammaticalization paths converge with it into a common paradigm.

\subsection{Auxiliation based on complementation}\label{sec:lehmann:4.7}
\subsubsection{Basics}\label{sec:lehmann:4.7.1}
Given that any dependents follow the verb, the \isi{subordinate clause} follows the \isi{main clause}. Of importance for complex syntax and especially for auxiliation is a kind of complex construction consisting of a \isi{main clause} core and a \isi{complement clause} core. The main \isi{predicate} may be a nominal or verbal one. It is in any case monovalent and therefore has no dependents beside the \isi{complement clause}. The latter functions as the subject of a verbal, and as the (“\isi{possessive}”) complement of a nominal main \isi{predicate}. This presupposes its nominalization, and therefore it is in \isi{incompletive} \isi{dependent status}. Given the categorial polymorphy of the main \isi{predicate}, this is simply categorized by its destination, viz. as an \isi{auxiliary} to come, in \figref{dig:lehmann:9}. This is construction \#d of the set enumerated in \sectref{sec:lehmann:4.5} which shares a syntactic slot in front of the clause core. It is illustrated by \REF{ex:lehmann:22}.

\begin{figure}
\caption{Subject complementation}\label{dig:lehmann:9}
\begin{forest} forked edges,
 [sentence, l sep=2\baselineskip
  [\isi{auxiliary} to come,edge label={node[near end, left,font=\itshape]{main predicate}}]
  [dependent clause core,edge label={node[near end,right,font=\itshape]{subject}}]
 ]
\end{forest}
\end{figure}


\ea\label{ex:lehmann:22}
Colonial \ili{Yucatec} \ili{Maya}\\
       çebhi         in       canic         maya    than\\
\gll  séeb-h-ih      =in      kan-ik        maaya    t’aan\\
fast=\textsc{cop-cmpl }  \textsc{=a.1.sg}   learn-\textsc{incmpl}    \ili{Maya}    speech\\
\glt ‘I learnt \ili{Maya} quickly’ (lit.: ‘it was quick that I learnt \ili{Maya}’) \citep[52]{Coronel1620}
\z

From an SAE point of view, the full verb in the dependent clause core may appear to be the main \isi{predicate}, which several SAE languages would modify by such peripheral concepts as the fastness of \REF{ex:lehmann:22}. A language like \ili{Maya}, generally averse to modification, prefers the alternative of having the peripheral \isi{predicate} govern the central \isi{predication} (cf. \citealt{Lehmann1990strategien} for this typological relationship). \REF{ex:lehmann:23} illustrates the construction with \isi{modal} verboids.

\ea\label{ex:lehmann:23}
Colonial \ili{Yucatec} \ili{Maya}\\
\ea 
v    nah      a      benél\\
\gll   u    nah      =a    ben-el\\
\textsc{a.3}  decorum\rmfnm{} =\textsc{a.2}   go-\textsc{incmpl}\\
\glt ‘you ought to go’ \citep[69]{Coronel1620}\\
\ex 
Vchuc    inbeelticlo\\
\gll   uuchuk  =in      beelt-ik      =lo'\\
possible  =\textsc{a.1.sg}    make-\textsc{incmpl}   \textsc{=r2}\\
\glt ‘I can do that’ \citep[18v]{SanBuenaventura1684}\footnotetext{lit. ‘what befits you / your obligation’, \ili{Spanish} \textit{conviene}}\newpage
\ex 
tac       in       xee\\
\gll   taak      =in      xeeh\\
prompted  =\textsc{a.1.sg}   vomit{\textbackslash}\textsc{introv(incmpl)}\\
\glt ‘I have/want to vomit’ \citep[§299, p.146]{Beltrán1746} 
\z
\z 

As already indicated in \sectref{sec:lehmann:4.5}, the complement construction resembles the cleft construction in having the main constituent in the same \isi{clause-initial position}. An important difference between the two constructions consists in the fact that the \isi{subordinate clause} of the former is just a \isi{nominalized clause}. Its status marking is the \isi{incompletive} \isi{dependent status}, with non-past reference. The \isi{extrafocal} clause, instead, may be in any \isi{dependent status} and thus have any time reference.


As the following subsections will show, this construction is the model for a number of auxiliaries. The clause-initial slot attracts not only \isi{intransitive} verbs, but also verboids, nouns and denominal adverbs. The construction, however, remains essentially the same: in all the constructions of \sectref{sec:lehmann:4.7}, the clause core depends on the initial element.


\subsubsection{From habitual to imperfective aspect}\label{sec:lehmann:4.7.2}

The inherited \isi{imperfective} was renewed in Colonial \ili{Yucatec} \ili{Maya}.\footnote{All \ili{Mayan} languages have an \isi{imperfective} \isi{auxiliary}, but the forms are very different. For instance, Ch'ol has \textit{muk’}, shortened to \textit{mi}; Q’eqchi’ has \textit{nak-}; and so on. See \citet{Vinogradov2014}.} At the beginning of this process, there is a set of words, apparently denominal in origin, which compete for the \isi{auxiliary position}. Three of these appear in \REF{ex:lehmann:24}, listed as synonymous in the colonial grammar. The first is \textit{lic(il)}, which has a variant \textit{lac} and must be a root with the meaning ‘this time span’, although it is no longer found in the texts as such. The second of these auxiliaries is \textit{tamuk}, a \isi{preposition} and conjunction meaning ‘during, while’. The third is \textit{ualac} ‘this time’. Both \textit{lik} and \textit{walak} survive in present-day \ili{Yucatec} in a form adverbialized by the suffix \textit{{}-il}.\footnote{The form \textit{licil} is treated extensively in \citet{Coronel1620}, and on p. 46 he does assign it a habitual meaning. Otherwise, \textit{licil} subordinates a clause similar in function to an oblique relative clause. Modern successors are \ili{Yucatec} \textit{ka’likil} ‘at the time, while’ and \ili{Itzá} \textit{kil} ‘when’ \citep[26]{Hofling1991}. Acatec \ili{Maya} has \textit{chi} {\textless} \textit{ki}.}



\ea\label{ex:lehmann:24}
Colonial \ili{Yucatec} \ili{Maya}\\
       cimçabi                 in       yum\\
\gll        kim-s-a’b-ih            =in      yuum\\
die-\textsc{caus-pass-cmpl(b.3.sg)}  =\textsc{a.1.sg}    master/father\\\newpage

  tilic          /  tamuk  /  ti    válac     v       hanál\\
\gll  ti’    lik      /  tamuk’  /  ti’    walak    =u    han-al\\
\textsc{loc} this.span  /  while  /  \textsc{loc}  this.time  =\textsc{a.3}    eat-\textsc{incmpl}\\
\glt ‘my father was killed while eating’ \citep[57]{Coronel1620}
\z



In \ili{Yucatec}, the competition among the three \isi{formatives} will be won by \textit{lic}. The \isi{preposition} \textit{ti} subordinating it can already be omitted, as in \REF{ex:lehmann:25}.


\ea\label{ex:lehmann:25}
Colonial \ili{Yucatec} \ili{Maya}\\
%\gll 
      lic  u      dzocol      a      hanal      ca      tacech              uaye\\
\gll   lik   =u     ts'o’kol     =a    han-al      káa     tal-ak-ech        way=e'\\
span  =\textsc{a.3}   end-\textsc{incmpl }   \textsc{=a.2}   eat-\textsc{incmpl}  \textsc{conj}    come-\textsc{subj-b.2.sg}    here=\textsc{r3}\\
\glt ‘when you have eaten, you should come here’ (Motul s.v. \textit{ca}{\textsubscript{6}})
\z



The clause introduced by \textit{lic} may also be independent; then the originally temporal construction may have a habitual sense (cf. \citealt[67]{Coronel1620}), clearly visible in \REF{ex:lehmann:26}.


\ea\label{ex:lehmann:26}
Colonial \ili{Yucatec} \ili{Maya}\\
%\gll 
      lic  in        uenel        tamuk    in        hanal\\
\gll         lik  =in      wen-el        tamuk'    =in      han-al\\
\textsc{hab}  \textsc{=a.1.sg}    sleep-\textsc{incmpl}  while    =\textsc{a.1.sg}    eat-\textsc{incmpl}\\
\glt ‘I usually fall asleep while eating’ (Motul s.v. \textit{lic}\textsubscript{2})
\z



By further grammaticalization, the morpheme functions as a mere \isi{imperfective} \isi{auxiliary}, as in \REF{ex:lehmann:27}.


\ea\label{ex:lehmann:27}
Colonial \ili{Yucatec} \ili{Maya}\\
%\gll 
      lic bin      a      haɔic      a      paalil    tu  men    u        tuz.  –\\
\gll   lik=bin    =a    hats’-ik    =a    paal-il    tumen    =u    tuus\\
\textsc{ipfv=quot}  \textsc{=a.2}    beat-\textsc{incmpl}  \textsc{=a.2}   child-\textsc{rel}  because  =\textsc{a.3}    lie{\textbackslash}\textsc{introv}\\
\glt ‘They say you (habitually) beat your boy because he lies.’

%\gll 
  lic. lici.\\
\gll        lik(-ih)\\
\textsc{ipfv-cfp} \\
\glt ‘Yes.’ (Motul s.v. \textit{lici lic})
\z



It may be noted that the two occurrences of the particle in \REF{ex:lehmann:27} fulfill the conditions of the two tests for word status introduced in \sectref{sec:lehmann:4.5}: the particle is, at this stage, syntactically independent. However, there already exists a shortened variant \textit{c(i)}, apparently in free variation, as in the dialogue of \REF{ex:lehmann:28}:


\ea\label{ex:lehmann:28}
Colonial \ili{Yucatec} \ili{Maya}\\
%\gll 
\ea bal    ca        uoktic?\\
\gll   ba’l    k=a      wook’-t-ik\\
what    \textsc{ipfv=a.2}  weep-\textsc{trr-incmpl}\\
\glt ‘What are you crying for?’

%\gll 
\ex 
in      kéban    lic    uoktic.\\
\gll   in      k’eban    lik    w-ook’-t-ik\\
\textsc{a.1.sg}  sin      \textsc{ipfv }   \textsc{a.1.sg}{}-weep-\textsc{trr-incmpl}\\
\glt‘It is for my sins that I am crying.’ \citep[67]{Coronel1620}
\z
\z 


One and a half centuries later, \textit{lic} is still found in the same contexts, as shown in (\ref{ex:lehmann:29}–\ref{ex:lehmann:30}).\footnote{\citet[§299, p.140]{Beltrán1746} also mentions \textit{liclili} (\textit{likil-ili’}) with the meaning ‘customarily, so it is always’, which is a reinforcement of the same particle by the identifying suffix \textit{-ili’}.}\largerpage


\ea\label{ex:lehmann:29}
Colonial \ili{Yucatec} \ili{Maya}\\
%\gll 
 \sn[]{\hspace{-1.5ex}\glll     tilic      ú      tzicic        Dios  Pedroe,\\
        ti’-lik    =u    tsik-ik        dios    Pedro=e’\\
\textsc{loc}{}-span  =\textsc{a.3}    obey-\textsc{incmpl}    god    Peter=\textsc{r3}\\}
%\gll 
 \sn[]{\hspace{-1.5ex}\glll bin  ú      {chuc  olt}          dzabilah\\
        bíin  =u    chuk-óol-t          ts’abilah\\
\textsc{fut}  \textsc{=a.3}   attain-mind-\textsc{trr(subj)}  grace\\}
\glt ‘as long as Peter obeys god, he will attain grace’ \citep[§261]{Beltrán1746}
\z


\ea\label{ex:lehmann:30}
Colonial \ili{Yucatec} \ili{Maya}\\
%\gll 
  Lic    ua    ú      hanal      kohane? –    Lic.\\
\gll   lik    wáah  =u    han-al      k’oha’n=e’    lik\\
\textsc{ipfv    int    =a.3}   eat-\textsc{incmpl}  sick=\textsc{r3}      \textsc{ipfv}\\
\glt ‘Does the sick person eat? – He does.’ \citep[§299, p.140]{Beltrán1746} 
\z



As \REF{ex:lehmann:30} proves, at this stage, \textit{lic} still stands both of the tests of syntactic independence. However, the status of its shortened variant \textit{c(i}), ‘very common’ according to \citet[§101]{Beltrán1746}, is already ambivalent.\footnote{\citeauthor{Beltrán1746} dedicates a section (95) to \textit{lic}(\textit{il}), attributing a habitual function to it, and another section (101) to \textit{ci}, attributing present \isi{tense} function to it, without noting any connection between the two.} It can still serve as host to a following \isi{enclitic}, as in the \#a version of the variants offered in \REF{ex:lehmann:31}.


\ea\label{ex:lehmann:31}
Colonial \ili{Yucatec} \ili{Maya}\\
%\gll 
\ea 
ci    bin    in        yacuntic\\
\gll   ki    bin    =in      yáakunt-ik\\
\textsc{ipfv}  \textsc{quot}    \textsc{=a.1.sg}    love-\textsc{dep}\\

%\gll 
\ex 
cin          yacuntic    bin\\
\gll   k=in          yáakunt-ik  bin\\
\textsc{ipfv=a.1.sg}    love-\textsc{dep  }    \textsc{quot}\\
\glt ‘it is said that I love him’ \citep[§246]{Beltrán1746} 
\z
\z


On the other hand, the particle already optionally univerbates with the \isi{enclitic} A index, as evidenced by the \#b version (separate combinations of \textit{ci in/a/u} in \citealt[§131]{Beltrán1746}). Beltrán uses the reduced \isi{auxiliary} \textit{c(i)} in his own examples when aspect is not at stake, thus, in order to choose unmarked aspect (as in \REF{ex:lehmann:31} and \textit{passim}). This is already today’s situation: The \isi{auxiliary} only survives in its one-phoneme form \textit{k}, obligatorily univerbates with the Set A index and carries aspectual information only in contrast with more specific auxiliaries.



Thus, the \isi{imperfective} \isi{auxiliary} becomes a bound monophonematic form just like the older \isi{perfective} \isi{auxiliary} seen in \sectref{sec:lehmann:4.6}. The opposition between \isi{perfective} and imperfective aspect\is{imperfective} emerges as a minimal one both in formal and in functional terms. It becomes the core of the extensive TAM \isi{auxiliary} paradigm indicated in \tabref{tab:lehmann:13}.



We come to the \isi{imperfective} auxiliaries of the other \ili{Yucatecan} languages. Both in \ili{Itzá} and in \ili{Lacandón}, imperfective aspect\is{imperfective} is marked by the same formative \textit{k} as in \ili{Yucatec}.\footnote{Its analysis as a \isi{future marker} in \citet[61]{Bruce1968} must be due to a confusion with the future subordinator \textit{k}(\textit{en}).} However, \ili{Lacandón} shows more variation. On the one hand, the formative is optional \citep[62]{Bruce1968}, imperfective aspect\is{imperfective} then being marked only by the \isi{incompletive} \isi{status suffix}, as in \REF{ex:lehmann:32}. Especially in Chan K’in Viejo’s terse epic style, an \isi{incompletive} verbal complex often constitutes an independent sentence, as in \REF{ex:lehmann:33}.


\ea\label{ex:lehmann:32}
\ili{Lacandón}\\
%\gll 
  \gll    K’ayyum  =u    häts’-ik      Cham-Bol\\
   K’ayyum    =\textsc{a.3}    beat-\textsc{incmpl}    Chan-Bor\\
\glt ‘Kayum beats Chan Bor’ \citep[105]{Bruce1968}
\z


\ea\label{ex:lehmann:33}
\ili{Lacandón}\\
%\gll 
  \gll    In      want-ik-ech      Yum-eh.\\
   \textsc{a.1.sg}  help-\textsc{incmpl-b.2.sg}  lord-\textsc{voc}\\
\glt ‘I (will) help you, my lord.’ \citep[26]{Bruce1974}
\z



The most plausible analysis of this construction is that the \isi{auxiliary} has been reduced to zero.\footnote{An alternative, and less plausible, account would be to assume that \ili{Lacandón} uses the nominalized constructions of \sectref{sec:lehmann:4.3} as independent sentences, in which case the change would instantiate insubordination. Note that this is not analogous to the \ili{Lacandón} use of the \isi{completive} without \isi{auxiliary}, discussed in  \sectref{sec:lehmann:4.6}, since the \isi{completive} construction at its origin was independent without an \isi{auxiliary}.} This is, then, an example of complete grammaticalization within half a millennium.



On the other hand, there is a formative \textit{k(ah)} which functions as a temporal conjunction. It may be illustrated by \REF{ex:lehmann:21}, repeated here as \REF{ex:lehmann:34}.


\ea\label{ex:lehmann:34}
\ili{Lacandón}\\
%\gll 
  \gll    K=u      yen-s-ik          =u    yok  lu’m    Hachäkyum    y-a’l-ah:\\
  \textsc{ipfv=a.3}  lower-\textsc{caus-incmpl}   \textsc{=a.3}    foot  earth    Hachäkyum     \textsc{a.3-}say\textsc{{}-cmpl}\\
\glt ‘When Hachakyum set his foot onto the land, he said:’ (\citealt[111]{Bruce1968} {\textasciitilde} \citeyear[19]{Bruce1974})
\z

The initial \textit{k} is glossed as ‘\isi{imperfective}’. It might as well be glossed as ‘when’.\footnote{This is actually the gloss provided by \citet{Bruce1974}.} The \ili{Yucatecan} languages have a rather large set of subordinating \isi{formatives} which start with or at least contain a /k/. Occupying the position indicated in \figref{dig:lehmann:7} of \sectref{sec:lehmann:4.5}, some of them allow a following \isi{auxiliary}. Recall that the Colonial \ili{Yucatec} \ili{Maya} formative \textit{lik}(\textit{il}), which yields the \ili{Yucatec} \isi{imperfective} \isi{auxiliary}, is first mostly found in temporal clauses. The exact relationship between the \isi{imperfective} auxiliaries and these conjunctions remains to be sorted out.

In \ili{Mopán}, the alternate \isi{auxiliary} \textit{walak} was chosen, which appears in \REF{ex:lehmann:35}.

\ea\label{ex:lehmann:35}
\ili{Mopán}\\
%\gll 
\gll     walak  =ti      ad-ik\\
  \textsc{hab }    \textsc{=a.1.pl}    say-\textsc{incmpl}\\
\glt ‘we always say it’ \citep[129]{Danziger2011}
\z



As may be seen, this is less grammaticalized, both functionally and formally, than its original competitors in the sister languages.


\subsubsection{Progressive aspect}\label{sec:lehmann:4.7.3}

The progressive itself is a Proto-\ili{Mayan} category. In Colonial \ili{Yucatec} \ili{Maya}, it is based on the relational noun \textit{tan} (\textit{táan}),\footnote{The progressive function of this morpheme may be inherited from Proto-\ili{Maya}; some languages, including Kaqchiquel, have plausible cognates.} illustrated in (\ref{ex:lehmann:36}–\ref{ex:lehmann:37}) in its lexical meaning ‘front, middle’.



\ea\label{ex:lehmann:36}
Colonial \ili{Yucatec} \ili{Maya} \\
%\gll 
       tan    cah\\
\gll       táan    kah\\
middle  village\\
\glt ‘(in) the village center’ \citep[§299, p.147]{Beltrán1746} 
\z


\ea\label{ex:lehmann:37}
Colonial \ili{Yucatec} \ili{Maya}\\
%\gll 
      tutan          Dios\\
\gll   t=u      táan    dios\\
\textsc{loc=a.3}  front    god\\
\glt ‘in front of god’ \citep[39v]{SanBuenaventura1684}
\z



\REF{ex:lehmann:37} shows the regular syntactic construction which is natural for a noun designating a spatial region, viz. preceded by a \isi{possessive} Set A \isi{clitic}\footnote{The only Set A index ever attested in this construction is \textit{u} A.3. This leads to the interpretation made explicit in the literal translation of \REF{ex:lehmann:38} and to the gloss `middle'. If the \isi{clitic} could have been of first person, then the other meaning of \textit{táan}, viz. `front', would appear to underlie the construction: `in front of me/us, P is happening'.} and governed by the default \isi{preposition} \textit{ti’} \textsc{Loc}. The same configuration is also at the source of its aspectual use. The full form \textit{tután} is only mentioned in \citealt[47]{Coronel1620}, but not illustrated in the sources. The earliest evidence lacks the \isi{preposition}. (\ref{ex:lehmann:38}–\ref{ex:lehmann:39}) illustrate the incipient progressive function for \isi{intransitive} and \isi{transitive} verbs, respectively \REF{ex:lehmann:38} is obviously a variant of \REF{ex:lehmann:24}.


\ea\label{ex:lehmann:38}
Colonial \ili{Yucatec} \ili{Maya}\\
%\gll 
      vtán       v       hanál       in       yum,\\
\gll         u     táan   =u     han-al       =in       yuum\\
\textsc{a.3 }  middle  \textsc{=a.3}    eat-\textsc{incmpl}  =\textsc{a.1.sg}   master/father\\

%\gll 
  ca     cimçabi\\
\gll        káa     kim-s-a’b-ih\\
\textsc{conj}    die-\textsc{caus-pass-cmpl(b.3.sg)}\\
\glt ‘my father was in the middle of eating when he was killed’ or: ‘while my father was eating, he was killed’ \citep[57]{Coronel1620}
\z


\ea\label{ex:lehmann:39}
Colonial \ili{Yucatec} \ili{Maya}\\
%\gll 
      Vtan            incambecic      paal,\\
\gll   u    táan  =in      kambes-ik      paal\\
\textsc{a.3}  middle  =\textsc{a.1.sg}    teach-\textsc{dep.incmpl}  child\\

%\gll 
  ca      xolhi            tu         pix.\\
\gll   káa    xol-hih          t=u       píix\\
\textsc{conj}    kneel-\textsc{cmpl(b.3.sg) }  \textsc{loc=a.3}   knee\\
\glt ‘While I was teaching the child, he knelt down.’ \citep[9Br]{SanBuenaventura1684}
\z



The original construction with the subordinating \textit{ti’} and its further evolution are, at any rate, completely analogous to the \isi{imperfective} \textit{ti’ lik} seen in \REF{ex:lehmann:24}: It follows the pattern of \figref{dig:lehmann:9}, where the full verb of the \isi{complement clause} is in the \isi{incompletive} \isi{dependent status}. Initially, the new \isi{auxiliary} is typically used in complex sentences, where the progressive clause provides the background for the event of the \isi{main clause}, as clearly shown by (\ref{ex:lehmann:38}–\ref{ex:lehmann:39}). However, and again like the \isi{imperfective}, the progressive also appears in monoclausal sentences as (\ref{ex:lehmann:40}–\ref{ex:lehmann:41}). \REF{ex:lehmann:41} features, already at Coronel’s time, a further reduced form of the \isi{auxiliary}, where the original \isi{possessive} \isi{clitic} preceding \textit{táan} is no longer there.\footnote{Since Colonial \ili{Yucatec} \ili{Maya}, there has been a complex form \textit{ma’táan} of the negator \textit{ma’}, which according to \citet[83]{Coronel1620} triggers the \isi{incompletive} of \isi{intransitive} and the \isi{subjunctive} of \isi{transitive} verbs. It is certainly present in \REF{ex:lehmann:41}. It is not clear whether it contains the morpheme \textit{táan} presently at stake.}


\ea\label{ex:lehmann:40}
Colonial \ili{Yucatec} \ili{Maya} \\
%\gll 
      U    tan    in        beeltic\\
\gll   u    táan    =in      beel-t-ik\\
\textsc{a.3}   \textsc{prog }    \textsc{=a.1.sg}    make-\textsc{trr-dep.incmpl}\\
\glt ‘I am (in the middle of) doing it’ \citep[37r]{SanBuenaventura1684}
\z


\ea\label{ex:lehmann:41}
Colonial \ili{Yucatec} \ili{Maya}\\
%\gll 
      ma  tan    a      túbul        ten\\
\gll   ma’  táan    =a    tu’b-ul        ten\\
\textsc{neg}   \textsc{prog   }  \textsc{=a.2}    escape-\textsc{incmpl}  me\\
\glt
‘I am not going to forget you’ \citep[34]{Coronel1620}
\z



\citet[§261]{Beltrán1746} includes \textit{utan} in the list of particles adopted from his predecessors, but in his own examples he only uses the reduced form \textit{tan}. Seeking to render the \ili{Spanish} progressive (“gerundio”) in \ili{Maya}, he offers, among other alternatives, the pair of examples in \REF{ex:lehmann:42}, which illustrates, at the same time, the morphological correlates of the \isi{transitivity} contrast:


\ea\label{ex:lehmann:42}
Colonial \ili{Yucatec} \ili{Maya}\\
%\gll 
\ea 
tan    in        tzeec,          ca      lub        kuna\\
\gll        táan    =in      tse’k          káa    lúub        k’u-nah\\
\textsc{prog}    \textsc{=a.1.sg}   preach\textsc{(incmpl)} \textsc{conj}    fall\textsc{(cmpl)}    god-house\\
\glt ‘I was preaching, there the church collapsed’
\ex 
%\gll 
tan    in        tzeectic          ú      than    Dios    tiob,\\
\gll  táan    =in      tse’k-t-ik          =u    t’aan  dios    ti’-o’b\\
\textsc{prog  }   \textsc{=a.1.sg}    preach-\textsc{trr-incmpl }  \textsc{=a.3}   word    god    \textsc{loc-3.pl}\\

%\gll 
  ca      cim        Joan\\
\gll   káa    kim        Juan\\
\textsc{conj}    die\textsc{(cmpl)}    John\\
\glt ‘I was preaching god’s word to them, there John died’ \citep[§262]{Beltrán1746} 
\z
\z 


As may be seen, this is now just a \isi{progressive aspect}. \REF{ex:lehmann:43} illustrates the test on susceptibility of serving as the host to a \isi{clitic} particle, with positive result for contemporary \ili{Yucatec} \ili{Maya}.


\ea\label{ex:lehmann:43}
Modern \ili{Yucatec} \ili{Maya}\\
%\gll 
\gll      Táan  wáah  =a    bin?\\
  \textsc{prog  }   \textsc{int }    \textsc{=a.2}    go(\textsc{incmpl)}\\
\glt ‘Are you going (leaving)?’ (Hnazario\_406)
\z



In its further development, and again in analogy with the development of the \isi{imperfective} \isi{auxiliary} as illustrated by (\ref{ex:lehmann:31}b) above, the \isi{progressive auxiliary} coalesces with the Set A index which regularly follows it. The full form of the \isi{auxiliary} survives essentially in writing and, in the oral mode, in cases like \REF{ex:lehmann:43}. The \isi{coalescence} is a process in two phases. At first, the product of the merger of \textit{táan} with the three singular indexes \textit{in, a, u} is \textit{tíin, táan, túun}, as illustrated by \REF{ex:lehmann:44}.


\ea\label{ex:lehmann:44}
Modern \ili{Yucatec} \ili{Maya}\\
\gll Túun    tsikbal.\\
  \textsc{prog:a.3}  tell(\textsc{incmpl})\\
\glt ‘He was talking.’ \citep[48]{Monforte2011}
\z


This is, however, just a transitional stage rarely represented in writing. In the end, these forms are shortened to \textit{tin, ta, tu} (cf. \citealt[24f]{BriceñoChel2006}), as in \REF{ex:lehmann:45}.


\ea\label{ex:lehmann:45}
Modern \ili{Yucatec} \ili{Maya}\\
\gll       T=u      sáas-tal\\
  \textsc{prog=a3}  dawn-\textsc{fient.incmpl}\\

\gll   káa    h    téek          líik'        y-ich    hun-túul    le      peek'=o'\\
\textsc{conj }    \textsc{pfv}  for.a.moment  rise(\textsc{cmpl) } \textsc{a.3}{}-eye    one-\textsc{cl.an  }   \textsc{dem}    dog=\textsc{r2}\\
\glt ‘It was dawning when one of the dogs suddenly rose his glance’ (hts'oon\_310.1)
\z



In the syntactic configuration illustrated by \REF{ex:lehmann:45}, the progressive clause specifies a situation holding in the background simultaneously with the event described by the following %%% main
clause. This is functionally equivalent with the combination described in \sectref{sec:lehmann:4.3} (cf. \ref{ex:lehmann:6}b and \ref{ex:lehmann:11}–\ref{ex:lehmann:12}), where a \isi{nominalized clause} subordinated by \textit{ti’} serves as background information for the \isi{main clause}. In fact, since the products of the merger of the \isi{preposition} and of the \isi{progressive auxiliary} with the following Set A index are homonymous, the two constructions are not easily distinguished. It may be assumed that the (much older) model of the nominalized construction played a role in the rather radical reduction of the \isi{auxiliary} complex.



By the same token, the reduced variant of the \isi{progressive auxiliary} becomes homonymous with the \isi{perfective} \isi{auxiliary}. The two aspects, however, do not thereby become homonymous, since the progressive conditions \isi{incompletive} status, while the \isi{perfective} conditions \isi{completive status}; and these two are distinct for all verbs (cf. \citealt{Lehmann2014}, §3.4.2). This convergence of two aspectual auxiliaries constitutes an important contribution to the maintenance of the status category, which otherwise might have been grammaticalized to zero (cf. \sectref{sec:lehmann:4.4}).



It remains to note that the progressive with \textit{tan} is a Pan-\ili{Yucatecan} construction; see \citet[93, 97]{Bruce1968} for \ili{Lacandón}, \citet[30]{Hofling1991} for \ili{Itzá} and \citet[125]{Danziger2011} for \ili{Mopán}. In \ili{Itzá} and \ili{Lacandón}, the reduced forms are as the above-mentioned intermediate forms of \ili{Yucatec} \REF{ex:lehmann:44}. The full form \textit{tan in wilik} is in free variation with the reduced form of \REF{ex:lehmann:46} \citep[61, 97]{Bruce1968}.\largerpage[2]


\ea\label{ex:lehmann:46}
\ili{Lacandón} \\
%\gll 
      tin          wilik\\
\gll   tan=in        wil-ik\\
\textsc{prog=a.1.sg}    see-\textsc{incmpl}\\
\glt `I am seeing it' \citep[34]{Bruce1968}
\z


Thus, the \isi{progressive auxiliary} becomes a bound monophonematic form just like the \isi{imperfective} \isi{auxiliary} seen in \sectref{sec:lehmann:4.7.2}.


\subsubsection{Terminative aspect}\label{sec:lehmann:4.7.4}

The first constituent of \figref{dig:lehmann:9} is filled by a noun in the cases reviewed in the two preceding sections. This is, however, not the most fertile \isi{grammaticalization path} for auxiliaries. Apart from \isi{modal} verboids, the most important subclass of \isi{intransitive} predicators to fill this position are phase verbs. The central \ili{Yucatec} phase verbs are \textit{ho'p'} ‘start’ and a set of verbs including \textit{ch'en}, \textit{ts'o'k}, \textit{haw}, \textit{nik} all meaning ‘end’. They are normally impersonal (see already \citealt[34f]{Coronel1620}). Personal use is possible with a few of them, but does not generate auxiliaries. In the impersonal construction, actancy is coded on the dependent verb; with some marginal exceptions, there is in \ili{Yucatecan} no “raising”.\footnote{Ch'ol has the same construction; see \citet[239]{Aulie1998}. According to \citet[§5.2]{Coon2010} the Ch’ol auxiliaries which trigger \isi{incompletive} status do allow raising of the absolutive \isi{enclitic}.} \REF{ex:lehmann:47} and \REF{ex:lehmann:48} illustrate the construction for \textit{ho'p'} ‘start’ and \textit{ts'o'k} ‘finish’, respectively. Whether or not the \isi{main clause} is clefted (\#a vs. \#b examples), the dependent verb is in the \isi{incompletive} \isi{dependent status}.\footnote{\citet[89]{Smailus1989} claims it to be in the \isi{subjunctive}. However, although crucial evidence, with an \isi{intransitive} dependent verb, appears to be rare, \citealt[35]{Coronel1620} does have \textit{maytoh ts'o'kok in menyali'} ‘I have not yet finished working’, with \textit{menyal} in the \isi{incompletive}.}



\ea\label{ex:lehmann:47}
Colonial \ili{Yucatec} \ili{Maya}\\
%\gll 
\ea 
hoppi      in      beeltic\\
\gll ho'p'-ih          =in      beel-t-ik\\
  start-\textsc{cmpl(b.3.sg)}   =\textsc{a.1.sg}    make-\textsc{trr-dep.incmpl}\\
\glt ‘I have begun to do it’ \citep[53]{Coronel1620}

%\gll 
\ex 
çamal    v      hoppol      in        ɔibtic\\
\gll   sáamal    =u    ho'p'-ol    =in      ts'íib-t-ik\\
tomorrow  =\textsc{a.3}    start-\textsc{incmpl} \textsc{=a.1.sg}   write-\textsc{trr-dep.incmpl}\\
\glt ‘tomorrow I will start writing it’ \citep[35]{Coronel1620}
\z
\z

\ea\label{ex:lehmann:48}
Colonial \ili{Yucatec} \ili{Maya}\\
%\gll 
\ea  ɔoci         incanic\\
\gll   ts'o'k-ih          =in      kan-ik\\
  end-\textsc{cmpl(b.3.sg)}    =\textsc{a.1.sg}    learn-\textsc{dep.incmpl}\\
\glt ‘I finished learning / have learnt it’ \citep[17r]{SanBuenaventura1684}  \newpage
%\gll 
\ex 
çamal    v      ɔócol      in        canic\\
\gll   sáamal    =u    ts'o'k-ol    =in      kan-ik\\
tomorrow  =\textsc{a.3}    end-\textsc{incmpl} \textsc{=a.1.sg}   learn-\textsc{dep.incmpl}\\
\glt ‘tomorrow I will finish learning it’ \citep[35]{Coronel1620}
\z
\z 


The \isi{phase verb} \textit{ts’o’k} ‘finish’ shown in \REF{ex:lehmann:48} combines with aspect auxiliaries just like any full lexical verb, e.g. in \REF{ex:lehmann:25}. It continues this life up to the present day. In \REF{ex:lehmann:49}, it regularly goes into the \isi{subjunctive} required by the construction, and only the translation suggests its \isi{auxiliary function}.


\ea\label{ex:lehmann:49}
Modern \ili{Yucatec} \ili{Maya} \\
\gll       le      kéen        ts'o'k-ok    =u    pa't-al=e'\\
  \textsc{dem}    when.\textsc{impf}  finish-\textsc{subj }  \textsc{=a.3}    form{\textbackslash}\textsc{pass-incmpl=top}\\

\gll   k=u      ts'a'bal              píib ...\\
\textsc{impf=a.3}  put/give:\textsc{incmpl.pass}  underground.oven\\
\glt ‘When they have been formed, they are put into the earth-oven ...’ (chaak\_028)
\z


\ea\label{ex:lehmann:50}
Modern \ili{Yucatec} \ili{Maya}\\
%\gll 
\gll      beey  túun    ts'o'k-ol      =u    kuxtal  le      p'us-o'b=o'\\
  thus    then    finish-\textsc{incmpl } \textsc{=a.3}   life    \textsc{dem}    hunchback-\textsc{pl=r2}\\
\glt ‘This then was the end of the life of the P'uz.’ (chem\_ppuzoob\_011)
\z

\REF{ex:lehmann:50} displays a symptom of grammaticalization: the \isi{phase verb} is in the \isi{incompletive}, but it lacks both the introductory \isi{imperfective} \isi{auxiliary} and the Set A index. This suggests that even in the construction at hand, where the \isi{main clause} comprises more than just the \isi{phase verb}, the latter fulfills an \isi{auxiliary function}, with the form \textit{kuxtal} in its subject not just being an abstract noun, but rather the verbal \isi{predicate} of the dependent clause core (a case of the zero nominalization described in \sectref{sec:lehmann:4.3}).

This \isi{grammaticalization process} starts in the colonial period. The seventeenth century grammars adduce the phase verbs \textit{ɔoc} ‘finish’ and \textit{hopp} ‘begin’ only in order to mention their regular impersonal or personal construction as illustrated by \REF{ex:lehmann:25} and (\ref{ex:lehmann:47}–\ref{ex:lehmann:48}) above. It is in the eighteenth century that the ongoing grammaticalization of the third person \isi{completive} form \textit{ts’o’k}\footnote{The grammaticalization of \textit{ho'p'} to an \isi{auxiliary} will not be described here. Both in \ili{Yucatec} and in \ili{Itzá} \citep[105]{Hofling1991}, it is common in narratives and reports to mark a new situation.} could no longer escape a critical linguist’s ear. Beltrán, writing his grammar in Mérida in 1742, observes the expansion of the use of \textit{ts’o’k} as \isi{auxiliary} in vogue at his time (§§85f), notes that it is a partial competitor to the (firmly established) \isi{perfective}, quotes some \isi{periphrastic} forms which are actually in use up to today and opposes violently both to this fashion and to the idea that \textit{ts’o’k} means ‘already’ (which it does in its function as terminative \isi{auxiliary}). His verdict is that the \isi{perfective} is formed without \isi{auxiliary} or “better” with the \isi{auxiliary} \textit{t-} (of \sectref{sec:lehmann:4.6} above), while \textit{ts’o’k} means ‘finish’ and nothing else.

The form of this verb which occupies the \isi{clause-initial position}, becoming, thus, a component of the \isi{grammaticalization path}, is the \isi{completive} form triggered by perfective aspect\is{perfective}, as in \REF{ex:lehmann:51} (where \textit{meyah} – just like \textit{kuxtal} in \REF{ex:lehmann:50} – can be an \isi{intransitive verb} with the zero allomorph of the \isi{incompletive} or an abstract noun).


\ea\label{ex:lehmann:51}
Modern \ili{Yucatec} \ili{Maya}\\
\gll       (h)  ts’o’k            =in      meyah\\
  \textsc{pfv}  finish(\textsc{cmpl:b.3.sg)} \textsc{=a.1.sg}    work\\
\glt ‘my work ended = I finished working = I have already worked’ (\citealt[84]{BriceñoChel2000terminar})
\z



In the sequel, the \isi{perfective} \isi{auxiliary} is omitted. In fact, by the evidence of \REF{ex:lehmann:48}, the grammaticalization of \textit{ts’o’k} probably started at a time when the \isi{completive} alone could make a \isi{perfective} clause. Otherwise, however, the new \isi{auxiliary} can maintain its full form even in the colloquial style. It passes the two tests on word status up to the present day, as evidenced by \REF{ex:lehmann:52}.


\ea\label{ex:lehmann:52}
Modern \ili{Yucatec} \ili{Maya}\\
\ea
\gll   Ts'o'k            wáah  =in      bo'l-t-ik        =in        p'aax?\\
  finish(\textsc{cmpl:b.3.sg)}  \textsc{int}    =\textsc{a.1.sg}    pay-\textsc{trr-incmpl}   \textsc{=a.1.sg}    debt\\
\glt ‘Have I paid my debt?’

%\gll 
\ex
\gll  Ma'    ts'o'k-ok=i'.\\
\textsc{neg}    finish-\textsc{subj=negf}\\
\glt ‘No (you haven't).’ (hnazario\_375f)
\z
\z

There is, however, a reduced form in addition to the full form, although not as widely used as the reduced form of the \isi{progressive auxiliary}. The \isi{auxiliary} is then reduced to its initial consonant and coalesces with the Set A \isi{clitic}, as shown by \REF{ex:lehmann:53} (cf. \citealt[87f]{BriceñoChel2000terminar}).\newpage

\ea\label{ex:lehmann:53}
Modern \ili{Yucatec} \ili{Maya}\\
\gll       ts'=in        w-a'l-ik      te'x\\
 \textsc{term=a.1.sg}    \textsc{0}{}-say-\textsc{incmpl}  you.all\\
\glt ‘I have told it to you’ (muuch\_340)
\z

The terminative is a kind of perfect and therefore in partial competition with the inherited suffixal perfect. They share the semantic component that the situation designated is finished at topic time. Their semantic difference lies in the implication of this fact. The \ili{Yucatec} perfect implies that the agent has the result of his action at his disposal, while the terminative focuses on the crossing of the end boundary of the situation, which may be counter to expectations.\footnote{Terminative aspect is incompatible with a temporal \isi{adverb} in the same clause (s.  \citet[82f]{BriceñoChel2000terminar}})

Like the progressive, \textit{ts’o’k} is a Pan-\ili{Yucatecan} \isi{auxiliary}. Its \ili{Lacandón} form is \textit{ts’ok};\footnote{According to \citet[81, 93, 99]{Bruce1968} the function is immediate past.} \REF{ex:lehmann:54} is an example.

\ea\label{ex:lehmann:54}
\ili{Lacandón}\\
\gll       Ts’ok  =u    me(n)t-i(k)    k’ax, ...\\
  \textsc{term } \textsc{=a.3}     make-\textsc{incmpl}  woods\\
\glt ‘He had made the woods, …’ \citep[24]{Bruce1974}
\z

Likewise in \ili{Itzá}, \textit{tz'o'k} is used in terminative function, as shown by \REF{ex:lehmann:55}:

\ea\label{ex:lehmann:55}
\ili{Itzá} \\
\gll      Tz’o’k-i(h)        =u    man    ka’-p’eel      k’in, ...\\
 \textsc{term-cmpl(b.3.sg)}   \textsc{=a.3}   pass    two-\textsc{cl.inan}    day\\
\glt ‘Two days had passed, ...’ (\citealt{Hofling2006}, 12:39)
\z

Besides this, \ili{Itzá} has grammaticalized another \isi{phase verb} to a terminative \isi{auxiliary}, viz. the verb \textit{ho'm} (\citealt{Hofling1991}: 25, 65), whose original meaning is ‘wane, abate’.

As an aside, it may be mentioned that the \isi{phase verb} \textit{ts’o’k} in the imperfective aspect\is{imperfective} is also the grammaticalization source of a paratactic conjunction that is very widely used in the colloquial register of Modern \ili{Yucatec} \ili{Maya}, as witnessed by the monotonous repetition in \REF{ex:lehmann:56}.\largerpage[2]

\ea\label{ex:lehmann:56}
Modern \ili{Yucatec} \ili{Maya}\\
\gll       K=u       ts'o'k-ol=e'                   k=in               p'o'-ik;\\
  \textsc{ipfv=a.3}  finish-\textsc{incmpl=top } \textsc{ipfv=a.1.sg}  wash-\textsc{incmpl}\\
\glt ‘Then I wash it;’\newpage

\gll   k=u          ts'o'k-ol=e'          \\
\textsc{ipfv=a.3}  finish-\textsc{incmpl=top}\\

\gll k=in                       ts'a'-ik              t=eh        k'áak'=o’ ...\\
      \textsc{ipfv=a.1.sg}  put/give-\textsc{incmpl} \textsc{loc=dem}    fire=\textsc{r2}\\

\glt ‘then I put it on fire ...’ (chakwaah\_03f)
\z

The phrase \textit{ku ts’o’}\textit{k}\textit{ole’} is commonly reduced to \textit{ts’o’(h)le’}, the loss of the \isi{auxiliary} complex being due to grammaticalization, while the shrinking of the verb form follows regular phonological processes.

\subsubsection{From existential via debitive to future tense}\label{sec:lehmann:4.7.5}

The existential \isi{predicate} in the \ili{Yucatecan} languages during their entire documented history is the \isi{intransitive} verboid \textit{yaan}, illustrated by \REF{ex:lehmann:57}.

\ea\label{ex:lehmann:57}
Colonial \ili{Yucatec} \ili{Maya}\\
      yan    cutz\\
\gll  yaan  kuts\\
\textsc{exist}  turkey\\
\glt `there are turkeys' (\citealt[§199]{Beltrán1746})
\z


Apart from predicating sheer existence, \textit{yaan} is also the locational \isi{copula}, as in \REF{ex:lehmann:58}.


\ea\label{ex:lehmann:58}
Colonial \ili{Yucatec} \ili{Maya} \\
%\gll 
      tij     yan    ti     yotoch\\
\gll   ti'      yaan    ti'    y-otoch\\
there    \textsc{exist} \textsc{loc  } \textsc{a.3}{}-home\\
\glt `there he is at his home' \citep[35v]{SanBuenaventura1684}
\z

Furthermore, the canonical construction coding ascription of possession is obtained by substituting a possessed nominal for the central actant of \textit{yaan}, as in \REF{ex:lehmann:59}.\footnote{Interestingly, \citet[§199]{Beltrán1746}  makes the not unreasonable claim that the verboid \textit{yaan} lacks the first and second persons in the existential and \isi{possessive} uses. However, the first example offered by the \textit{Diccionario de Motul} s.v. \textit{yan} features just the second person in the existential use.}\newpage

\ea\label{ex:lehmann:59}
Modern \ili{Yucatec} \ili{Maya} \\
\gll       yaan  =in      nah-il\\
  \textsc{exist}   \textsc{=a.1.sg}    house-\textsc{rel}\\
\glt  `I have got a house' (muuch\_274)
\z

Once a \isi{nominalized verbal complex} is substituted for the possessum of the ascription of possession, a debitive construction results. Just as the possessum is ascribed to its \isi{possessor} in \REF{ex:lehmann:59}, so the obligation is ascribed to the actor of the \isi{nominalized verbal complex} in \REF{ex:lehmann:60}.

\ea\label{ex:lehmann:60}
Modern \ili{Yucatec} \ili{Maya}\\
\gll      ba'l=e'        yan  =a    bo'l-t-ik-en\\
  however=\textsc{top}   \textsc{deb  } \textsc{=a.2}    pay-\textsc{trr-incmpl-b.1.sg}\\

\glt `however, you must pay me' (hala'ch\_084)
\z

This use is not found in the colonial sources and is documented only in the modern \ili{Yucatecan} languages. In \ili{Itzá}, the construction is the same as in \ili{Yucatec} \citep[25]{Hofling1991}. In \ili{Lacandón}, the dependent clause core is introduced by the subordinator \textit{ti’}, as shown by \REF{ex:lehmann:61}.

\ea\label{ex:lehmann:61}
\ili{Lacandón} \\
\gll       yan    ti'    =a    kaxt-ik      =u    hel\\
  \textsc{deb }    \textsc{loc}  \textsc{=a.2}    search-\textsc{incmpl } \textsc{=a.3}    replacive\\
\glt ‘you have to look for another one’ \citep[81]{Bruce1968}
\z
%ti’ tritt in zwei debitiven Beispielen auf; unerklärt; vgl. immerhin E79.
%Zudem: məna’ t=in meyah (NEG.EXIST LOC=A.1.SG work) ‘no tengo (con) que trabajar’ (Bruce p. 83)


The most recent development, first documented in the 20\textsuperscript{th} century oral register, is a pure future without debitive connotations, as in \REF{ex:lehmann:62}, where the speaker articulates what he thinks will certainly happen.


\ea\label{ex:lehmann:62}
Modern \ili{Yucatec} \ili{Maya}\\
\gll       yan  =u    kaxt-ik-ech          =a    taatah\\
  \textsc{deb}   \textsc{=a.3}    search-\textsc{incmpl-b.2.sg } \textsc{=a.2}    father\\
\glt `your father will search you' (hnazario\_402.1)
\z

This construction is currently ousting the (much older) predictive future (\sectref{sec:lehmann:4.8}), which gets pushed back into the formal register.

\subsection{Auxiliation based on motion cum purpose: predictive future}\label{sec:lehmann:4.8}

The \isi{motion-cum-purpose construction} is a regular syntactic construction in the \ili{Yucatecan} branch. It is a complex clause core starting with an oriented \isi{motion verb} followed by a verbal clause core in the \isi{subjunctive}, the latter coding the purpose. As long as nothing precedes the \isi{motion verb}, the core verb is in plain status \isi{subjunctive}, as in \REF{ex:lehmann:63}.\footnote{If the future clause as introduced by \textit{bin} is an \isi{extrafocal} clause, as in \REF{ex:lehmann:67} and \REF{ex:lehmann:69}, the full verb goes into dependent = \isi{incompletive} status.}

\ea\label{ex:lehmann:63}
Colonial \ili{Yucatec} \ili{Maya}\\
      t    binén          in        cimez          uacax\\
\gll   t    bin-en          =in      kim-es          wakax\\
\textsc{pfv}  go(\textsc{cmpl)-b.1.sg }  \textsc{=a.1.sg}    die-\textsc{caus(subj)}    cow\\
\glt ‘I went to kill cows’ (\citealt{Beltrán1746}, §110)
\z

The central verbs of oriented motion (‘go’, ‘come’, ‘pass’) become irregular in their \isi{conjugation} on their way to Modern \ili{Yucatec}. Specifically, they lose the \textit{{}-Vl} suffix which marks their nominalization and would be expected in their \isi{incompletive} status (see also \REF{ex:lehmann:81} below). Moreover, the verb \textit{ben} ‘go’ becomes \textit{bin} in \ili{Yucatec}, while in the other \ili{Yucatecan} languages it becomes \textit{bel}. The changed forms appear both with their lexical meaning ‘go’ and as auxiliaries.

The \isi{motion-cum-purpose construction} with \textit{bin} as \isi{motion verb} is grammaticalized to a future in the \ili{Yucatecan} branch. \citet{Coronel1620} already calls it “futuro” and provides examples of it. \citet[§299, p. 128]{Beltrán1746} lists \textit{bin} as “partícula de futuro", giving examples (\ref{ex:lehmann:64}–\ref{ex:lehmann:65}) for the \isi{intransitive} and \isi{transitive} construction, respectively (\ref{ex:lehmann:29} is another example; see \tabref{tab:lehmann:4} for the allomorphs).

\ea\label{ex:lehmann:64}
Colonial \ili{Yucatec} \ili{Maya}\\
      bin  bolnacén        dzedzetàc\\
\gll   bíin  bo’l-nak-en      ts’e’ts’etak\\
\textsc{fut}  pay-\textsc{subj-b.1.sg}  little.by.little\\
\glt ‘I shall pay little by little’ \citep[§299, p. 149]{Beltrán1746} 
\z
\newpage

\ea\label{ex:lehmann:65}
Colonial \ili{Yucatec} \ili{Maya}\\
      caix      u      tancoch  in        hanale,\\
\gll         kayx    =u    táankoch  =in      haanal=e’\\
although  =\textsc{a.3}    half      =\textsc{a.1.sg}    meal=\textsc{r3}\\

  bin   in       ziib                 tech\\
\gll        bíin  =in      síih-ib              tech\\
\textsc{fut}   \textsc{=a.1.sg}    present-\textsc{subj(b.3.sg)}    you\\
\glt ‘although it is half of my meal, I’ll give it to you’ \citep[§299, p.129]{Beltrán1746} 
\z

The core verb keeps the \isi{subjunctive} of the \isi{source construction}.\footnote{\label{fnt:ftn43}In Modern \ili{Yucatec} \ili{Maya}, the \isi{motion-cum-purpose construction} itself diverges from its source by having the \isi{intransitive verb} in the \isi{incompletive} instead of the \isi{subjunctive} status.} The \isi{motion verb} complex has been reduced to the root of the \isi{motion verb}. This becomes impersonal like all the other auxiliaries and, in \ili{Yucatec} and \ili{Lacandón}, undergoes an idiosyncratic phonological change: the vowel of the \isi{auxiliary} \textit{bin} (not of the lexical verb!) is lengthened and gets high tone in \ili{Yucatec}. This may be due to analogy with the \isi{progressive auxiliary} \textit{táan}, but may also be regarded as the expression counterpart of the grammatical change. At any rate, the impersonalization and morphological impoverishment of the \isi{auxiliary} comes under paradigmaticization and may be ascribed to analogical pressure from the older auxiliation constructions analyzed in §4.7.\footnote{Ch'orti' (a Ch'olan language, thus closely affiliated to \ili{Yucatecan}) has the same impersonal construction with an etymologically unrelated verb meaning `go'.} \REF{ex:lehmann:66} illustrates the construction for both an \isi{intransitive} and a \isi{transitive verb}.


\ea\label{ex:lehmann:66}
Modern \ili{Yucatec} \ili{Maya}\\
\gll       Bíin    suu-nak          yéetel  bíin  =in      wil-eh.\\
  \textsc{fut}    return-\textsc{subj(b.3.sg)}  and    \textsc{fut}   \textsc{=a.1.sg}    see-\textsc{subj(b.3.sg)}\\
\glt ‘He will come back and I will see him.' (xipaal\_032)
\z

This \isi{future construction} finds its place in the \isi{tense}/aspect/mood paradigm at the side of three other futures, viz. the debitive future (\sectref{sec:lehmann:4.7.5}), the \isi{immediate future} (\sectref{sec:lehmann:4.9}) and an assurative future not analyzed here. It does not become an \isi{immediate future}, as so many futures based on the \isi{motion-cum-purpose construction} do in other languages. Neither does it contrast with the \isi{immediate future} on the time axis, as can be inferred from examples like \REF{ex:lehmann:65}. Instead, it bears a feature of neutral, objective prediction, which may be related to the impersonality of its \isi{auxiliary} and which opposes it to the other three futures. Since this semantic component matters less in what is going to happen next, time reference is often to a remote future. But this is only a favorable circumstance, not a condition for the appropriateness of a prediction.

We find the predictive future at an intermediate stage of grammaticalization. On the one hand, the reduction process mentioned above proves that it is grammaticalized to some extent already at the stage of Colonial \ili{Yucatec} \ili{Maya}. \REF{ex:lehmann:67} provides evidence in the same sense, as it shows that the construction is compatible with an additional, preceding focus constituent.

\ea\label{ex:lehmann:67}
Colonial \ili{Yucatec} \ili{Maya}\\
      bay  bin  v      cíbic        Dios  teex\\
\gll   bay  bíin  =u    kib-ik        Dios  te’x\\
thus  go    =\textsc{a.3}    do-\textsc{dep.incmpl}  god    you.\textsc{pl}\\
\glt ‘thus will god do with you’ (\citealt[72]{Coronel1620} = \citealt[24r]{SanBuenaventura1684})
\z

On the other hand, the predictive future \isi{auxiliary} stands the \isi{clitic} placement test to this day:
\ea\label{ex:lehmann:68}
Modern \ili{Yucatec} \ili{Maya}\\
\gll       bíin  wáah  p'áat-ak-en      hun-p'éel      k'iin      he'bix-ech=a'\\
  \textsc{fut }  \textsc{int}    stay-\textsc{subj-b.1.sg}  one-\textsc{cl.inan}    sun/day  ever:how-\textsc{b.2.sg=r1}\\
\glt ‘will I become like you one day?’ (xipaal\_092)
\z

The predictive \isi{future construction} is, again, Pan-\ili{Yucatecan}. \ili{Lacandón} conserves a variant of it which is structurally identical to the \isi{motion-cum-purpose construction}, to be seen in \REF{ex:lehmann:69}.

\ea\label{ex:lehmann:69}
\ili{Lacandón}\\
\gll       way  k=u                 bin    p’at-al                t=in        meyah\\
            here  \textsc{ipfv=a.3}  go    stay-\textsc{incmpl} \textsc{loc=a.1.sg}  work\\
\glt ‘it will stay here for my work’ \citep[42]{Bruce1974}
\z

However, it also has the reduced \isi{auxiliary construction} like \ili{Yucatec}, as in \REF{ex:lehmann:70}.

\ea\label{ex:lehmann:70}
\ili{Lacandón} \\
      b’ihn   a-kihn-s-${\emptyset}$-een\\
\gll   bíin    =a    kíin-s-en\\
  \textsc{fut  }   \textsc{=a.2}    die-\textsc{caus(subj)-b.1.sg}\\
\glt ‘you will kill me’ \citep[247]{Bergqvist2011}
\z

\ili{Itzá} again has the full \isi{motion-cum-purpose construction} with future function, to be seen in \REF{ex:lehmann:71}:

\ea\label{ex:lehmann:71}
\ili{Itzá}\\
\gll       way=e’  k=in b’el      =in pak’-t-eech\\
  here=\textsc{r3}  \textsc{ipfv=a.1.sg} go  =\textsc{a.1.sg}    wait-\textsc{trr(subj)-b.2.sg}\\
\glt ‘here I’m going to await you’ (\citealt{Hofling1991}, 15:126)
\z

The origin of the predictive \isi{future construction} is the \isi{motion-cum-purpose construction}. It differs from the other \isi{tense}/aspect/mood auxiliaries analyzed in \sectref{sec:lehmann:4.6}–\sectref{sec:lehmann:4.7} in that the emerging marker – the verb `go' grammaticalized to a \isi{future marker} – does not originally occupy the \isi{clause-initial position} described at the beginning of \sectref{sec:lehmann:4.5} and instead is the remnant of a complete superordinate clause. However, the canonical model for an \isi{auxiliary construction} is \figref{fig:lehmann:6}: the \isi{auxiliary} is monomorphematic, impersonal and occupies the \isi{clause-initial position}. In its grammaticalization, the \isi{motion-cum-purpose construction} is forced into the Procrustean bed of the verbal clause expanded by an initial position, which is the template for the \isi{auxiliary construction}. This is, thus, a clear example of grammaticalization guided by analogy.

\subsection{Auxiliation based on focused progressive: immediate future}\label{sec:lehmann:4.9}

As noted in \sectref{sec:lehmann:4.5}, the \isi{clause-initial position} is a melting-pot for constituents of very different kinds, among them the focus. We now come to an auxiliation strategy originating in a \isi{focus construction}, more specifically, in a verb-\isi{focus construction}. From there, we get to the \isi{immediate future} in two steps: First, on the basis of the verb `go' in focus, a focused progressive is formed. Second, this strategy applies to the `go' verb of the \isi{motion-cum-purpose construction} to form the \isi{immediate future} of its purpose component.

Putting the lexical main verb of a clause into its \isi{focus position} requires filling the gap that it leaves in the \isi{extrafocal} clause by a verb meaning ‘do’.\footnote{See \citet[§~4.3]{Lehmann2008} for a comprehensive account of the underlying information structure and the \ili{Yucatec} development.} For this purpose, Colonial \ili{Yucatec} \ili{Maya} used a verb \textit{cib} ‘do’ which is totally irregular and defective. \tabref{tab:lehmann:7} presents the forms adduced in \citet[71f]{Coronel1620}.

\begin{table}
\caption{Partial paradigm of Colonial Yucatec Maya \textit{cib} ‘do’}

\begin{tabular}{ll}
\lsptoprule
 category &  form\\
\midrule
{[fossilized status]} & cah\\
\isi{completive} & cibah\\
\isi{subjunctive} & cib (not \textit{c}\textit{ibib}!)\\
\isi{incompletive} dependent & cibic\\
\lspbottomrule
\end{tabular}
\label{tab:lehmann:7}
\end{table}

Already \citet[§§209f]{Beltrán1746} doubts this paradigm and contends that the verb is defective, being reduced to a “present” form \textit{cah}. The verb is rarely found in a simple \isi{transitive} clause to code the meaning ‘do, make’;\footnote{One of the rare examples is \REF{ex:lehmann:67} above, featuring dependent \isi{incompletive} status.} the lexicon offers other verbs with this meaning. Instead, it is used almost exclusively in focus constructions. A relatively straightforward one appears in \REF{ex:lehmann:72}.

\ea\label{ex:lehmann:72}
Colonial \ili{Yucatec} \ili{Maya}\\
%\gll 
      balamil    u      cah  pedro\\
\gll   balam-il    =u    ka'h  Pedro\\
tiger-\textsc{advr} \textsc{=a.3}    do    Peter\\
\glt ‘Peter makes the tiger / Peter is like a tiger’ (lit.: ‘tiger-like is what Peter does’; Motul s.v. \textit{cah}\textit{\textsubscript{3}})
\z

At the stage of Colonial \ili{Yucatec}, the verb is indispensable as a pro-verb in the verb \isi{focus construction}. The paradigm shown in \tabref{tab:lehmann:7} is illustrated by \REF{ex:lehmann:73}.

\ea\label{ex:lehmann:73}
Colonial \ili{Yucatec} \ili{Maya} \\
%\gll 
\ea hanál      v      cah\\
\gll   han-al      =u    ka'h\\
eat-\textsc{incmpl} \textsc{=a.3}    do\\
\glt ‘he is eating’

%\gll 
\ex 
hanál      v      cibah\\
\gll   han-al      =u    kib-ah\\
eat-\textsc{incmpl } \textsc{=a.3}   do-\textsc{cmpl}\\
\glt ‘he was eating’

%\gll 
\ex hanal      bin  v      cib\\
\gll   han-al      bíin  =u    kib\\
  eat-\textsc{incmpl}  go    =\textsc{a.3}    do(\textsc{subj})\\
\glt ‘he is going to eat’ (\citealt[71]{Coronel1620}; cf. \citealt[23v]{SanBuenaventura1684})
\newpage
%\gll 
\ex lúbul      tu        cibah\\
\gll lúub-ul    t=u      kib-ah\\
fall-\textsc{incmpl} \textsc{hod=a.3}  do-\textsc{cmpl}\\
\glt‘he fell (earlier today)’ \citep[71]{Coronel1620}
\z
\z 

As suggested by the translations of (\ref{ex:lehmann:73}a--c), the same construction functions as a progressive in Colonial \ili{Yucatec} \ili{Maya}. As a matter of fact, it figures much more prominently in colonial grammars than the simpler progressive with the \isi{auxiliary} \textit{táan} (\sectref{sec:lehmann:4.7.3}). All of them start their account of the \isi{conjugation} with the \isi{periphrastic} construction based on \textit{ka’h}, calling it the “presente”. \REF{ex:lehmann:74} completes the example series with a \isi{transitive verb}.

\ea\label{ex:lehmann:74}
Colonial \ili{Yucatec} \ili{Maya}\\
%\gll 
      cámbeçah          in          cah  ti      pálalob\\
\gll   kambes-ah          =in        ka'h  ti'      paal-alo'b\\
teach-\textsc{introv(incmpl)} \textsc{=a.1.sg}      do    \textsc{loc}    child-\textsc{pl}\\
\glt ‘I am teaching the children’ \citep[72]{Coronel1620}
\z

While all of the examples (\ref{ex:lehmann:72}–\ref{ex:lehmann:74}) are focus constructions, there are a number of peculiarities. First, if these were standard cleft sentences, the pro-verb of the \isi{extrafocal} clause would have to be in \isi{dependent status}. While this is hard to know for the irregular forms \textit{ka’h} (\ref{ex:lehmann:73}a) and \textit{bíin} (\ref{ex:lehmann:73}c), the forms of (\ref{ex:lehmann:73}b) and (\ref{ex:lehmann:73}d) appear to be forms of the plain status. Second, while any constituent can be focused without its form being thereby affected in any way, a \isi{finite verb} cannot; it must be nominalized. Therefore, the focused verbs in (\ref{ex:lehmann:73}–\ref{ex:lehmann:74}) show the nominalizing suffixes introduced in \sectref{sec:lehmann:4.3}, identical with \isi{incompletive} (dependent) status. Third, the process is relatively unproblematic with \isi{intransitive} verbs, as in \REF{ex:lehmann:73}, as their only actant is identical with the subject of \textit{ka’h} and may thus safely be suppressed by the nominalization. Things are more complicated with \isi{transitive} focused verbs, as in \REF{ex:lehmann:74}. The purpose of the verb-\isi{focus construction} is to put the verb into focus. Consequently, its dependents remain in the \isi{extrafocal} clause. Therefore, the verb is detransitivized before it is nominalized. The internal syntax of the \isi{extrafocal} clause is adapted, too: what was the \isi{direct object} of the focused verb becomes a prepositional object \citep[§172]{Beltrán1746}. The verb \isi{focus construction} is, consequently, marked with plurivalent verbs.

{\interfootnotelinepenalty=100000 The \isi{progressive aspect} views what the verb designates as an ongoing situation that the referent of the subject is in. Consequently, the functional locus of the \isi{progressive aspect} is in \isi{intransitive} verbs.\footnote{Evidence for this is provided, \textit{inter alia}, by the documented history of the evolution of the \isi{progressive aspect} in English and in substandard \ili{German}; see \citealt[section 3.2]{Lehmann1991}.}   The verb \isi{focus construction} is therefore well suited to get grammaticalized into a} \isi{progressive aspect}.\footnote{The \isi{progressive aspect} of other languages has a similar origin; cf., e.g., \citet{Güldemann2003} for \ili{Bantu}.} The resulting construction may be dubbed \textsc{focused progressive} (as in \citealt{Lehmann2008}). Two symptoms of the grammaticalization of the focused \isi{progressive construction} in Colonial \ili{Yucatec} \ili{Maya} will be mentioned: First, its susceptibility to nominalization by coercion, i.e. by having it depend on the \isi{preposition} \textit{ti’}, as in \REF{ex:lehmann:75}.

\ea\label{ex:lehmann:75}
Colonial \ili{Yucatec} \ili{Maya}\\
      ti    cimil      in        cah\\
\gll   ti’    kim-il      =in      ka’h\\
  \textsc{loc}  die-\textsc{incmpl} \textsc{=a.1.sg}    do\\
\glt ‘at/by my being ill’ \citep[58]{Coronel1620}
\z

Second, since the action feature of the basic meaning of \textit{kib} is lost, it combines even with passive verbs, as in \REF{ex:lehmann:76}:

\ea\label{ex:lehmann:76}
Colonial \ili{Yucatec} \ili{Maya}\\
      tzicil           in       cah\\
\gll   tsi’k-il          =in      ka’h\\
obey{\textbackslash}\textsc{pass-incmpl} \textsc{=a.1.sg}    do\\
\glt ‘I am (being) obeyed’ \citep[11v]{SanBuenaventura1684}
\z

Modern \ili{Yucatec} \ili{Maya} has a verb-\isi{focus construction}, too, but it is not as central to the \isi{conjugation} paradigm as the focused progressive appears to be in the grammars of Colonial \ili{Yucatec} \ili{Maya}. This has two totally unrelated reasons. The first is that the Colonial \ili{Yucatec} \ili{Maya} construction is much more grammaticalized than is the Modern \ili{Yucatec} \ili{Maya} verb \isi{focus construction}, which was renewed with the lexical verb \textit{beet/meent} ‘make’ (seen in \REF{ex:lehmann:40} above). The modern counterpart to (\ref{ex:lehmann:73}d) would consequently be \REF{ex:lehmann:77}.

\ea\label{ex:lehmann:77}
Modern \ili{Yucatec} \ili{Maya}\\
\gll       lúub-ul    t=u      meet-ah\\
  fall-\textsc{incmpl} \textsc{pfv=a.3}  make-\textsc{cmpl}\\
\glt ‘fall was what he did’ ({\textasciitilde} ‘all of a sudden, he fell’)
\z

\label{ref:present}The Colonial \ili{Yucatec} \ili{Maya} construction is clearly a kind of \isi{progressive aspect}, which the Modern \ili{Yucatec} \ili{Maya} construction is not; it is rather more of a thetic construction fit for all-new-utterances. The second reason for its prominence in the colonial grammars is a methodological one: The category is not nearly as frequent in the texts as it is in the grammars. The explanation is not hard to find:\label{lehmann:methomistake} The grammarians needed to fill up the \isi{conjugation} paradigms presupposed by \ili{Latin} grammar (\citealt{Hanks2010}: 214f). If one looks for a present \isi{tense} in Colonial \ili{Yucatec} \ili{Maya}, the closest analog would appear to be the imperfective aspect\is{imperfective} described in \sectref{sec:lehmann:4.7.2}. This, however, originates in complex sentences, whereas here an isolated verb form was needed. In a decontextualized sentence reduced to a \isi{finite verb}, all of the emphasis is on the \isi{finite verb}. Which provokes a verb-\isi{focus construction}.

On its way into the modern \ili{Yucatecan} languages, the pro-verb \textit{cib} is further fossilized; only the form \textit{cah/ka’h} occurs in a couple of contexts. This is ousted from its function as a pro-verb in verb-focus constructions by the lexical verb \textit{beet/meent} illustrated in \REF{ex:lehmann:77}. \textit{Ka’h} survives in this function only in the formulaic pattern illustrated by \REF{ex:lehmann:78}.

\ea\label{ex:lehmann:78}
Modern \ili{Yucatec} \ili{Maya}\\
\gll       Chéen  uk’ul              =u    ka’h.\\
  only    drink:\textsc{introv(incmpl) } \textsc{=a.3}    do\\
\glt ‘drinking is all he does / he only drinks (all the time)’ \citep[77]{BriceñoChel1998}
\z

Neither is the focused progressive with \textit{ka’h} further grammaticalized to a plain progressive. As we have seen in \sectref{sec:lehmann:4.7.3}, the \isi{progressive construction} which gets established involves a different \isi{auxiliary}. Instead, verb focusing is applied to the \isi{motion-cum-purpose construction} analyzed in \sectref{sec:lehmann:4.8}. What is put into \isi{focus position} is the verb \textit{benel/binel/bin} ‘go’, while the purpose part of the construction is left behind in the \isi{extrafocal} clause core. The resultant specific construction is, thus, a merger of the focused progressive with the \isi{motion-cum-purpose construction}. (\ref{ex:lehmann:79}–\ref{ex:lehmann:80}) illustrate it with an \isi{intransitive} and \isi{transitive} full verb, respectively.

\ea\label{ex:lehmann:79}
Colonial \ili{Yucatec} \ili{Maya}\\
      benel      in        cah    ti      hanal\\
\gll   ben-el      =in      ka’h    ti’      han-al\\
go-\textsc{incmpl } \textsc{=a.1.sg}    do      \textsc{loc}    eat-\textsc{incmpl}\\
\glt ‘I am going to eat’ \citep[50]{Coronel1620}
\z

\ea\label{ex:lehmann:80}
Colonial \ili{Yucatec} \ili{Maya}\\
      Binel      in        cah    incambez              palalob.\\
\gll   bin-el      =in      ka’h    =in      kan-bes        paal-alo’b\\
go-\textsc{incmpl } \textsc{=a.1.sg}    do      =\textsc{a.1.sg}   learn-\textsc{caus(subj)}  child-\textsc{pl}\\
\glt ‘I am going to teach the children.’ \citep[9Br]{SanBuenaventura1684}
\z

As already observed above (fn. \ref{fnt:ftn43}), on its way to Modern \ili{Yucatec} \ili{Maya}, the \isi{motion-cum-purpose construction} develops an asymmetry conditioned by the \isi{transitivity} of the full verb which persists into the focused progressive: A \isi{transitive} full verb \REF{ex:lehmann:80} is in the \isi{subjunctive} motivated by the \isi{motion-cum-purpose construction}, while the status of an \isi{intransitive} full verb \REF{ex:lehmann:79} is the \isi{incompletive}, which is diachronically the pure nominalized form (\sectref{sec:lehmann:4.3}). This is in consonance with the latter being subordinated by the \isi{preposition} \textit{ti}.\footnote{The documentary situation is such that this latter change appears earlier in the focused progressive than in the \isi{motion-cum-purpose construction} proper.} Again at the stage of Colonial \ili{Yucatec} \ili{Maya}, the binary contrast between \textit{bin} ‘go’ and \textit{tal} ‘come’ is yet maintained in their grammaticalization, as proved by \REF{ex:lehmann:81}. Observe, by the way, the third person on the pro-verb, obviously in analogy to the third person in the \isi{phase verb} construction of \sectref{sec:lehmann:4.7.1}.

\ea\label{ex:lehmann:81}
Colonial \ili{Yucatec} \ili{Maya}\\

\ea tal(el)        v      cah    in        botic      in        ppax\\
\gll   tal(-el)        =u    ka’h    =in      bo’t-ik      =in      p’aax\\
come-\textsc{incmpl} \textsc{=a.3}    do      =\textsc{a.1.sg}    pay-\textsc{incmpl} \textsc{=a.1.sg}    debt\\
\glt ‘I would like to pay my debt’ \citep[69]{Coronel1620}
\z
\z

Further reduction of the paradigm, however, leads to the consequence that the only verb possible in the Modern \ili{Yucatec} \ili{Maya} focused \isi{motion-cum-purpose construction} is \textit{bin}, and the construction only survives in the modern \isi{immediate future}, illustrated by the \isi{intransitive} and \isi{transitive} sentences of \REF{ex:lehmann:82}.

\ea\label{ex:lehmann:82}
Modern \ili{Yucatec} \ili{Maya}\\
\ea 
\gll bin      =in      ka'h  xíimbal      ti'    le    chaan  kaah ...=e'\\
  \textsc{imm.fut} \textsc{=a.1.sg}   do    walk(\textsc{incmpl) } \textsc{loc } \textsc{dem}  little    village  =\textsc{r3}\\
\glt ‘I am going to walk to that little village’ (hts'on\_016)\\
\ex
\gll bin      =in      ka'h  =in      xíimba-t        yuum          ahaw\\
\textsc{imm.fut} \textsc{=a.1.sg}   do    =\textsc{a.1.sg}    walk-\textsc{trr(subj)}    master/father  chief\\
\glt ‘I am going to visit the chief’ (hts'on\_020)
\z
\z 

The \isi{preposition} \textit{ti’} no longer shows up in this construction in Modern \ili{Yucatec} \ili{Maya}. And as in the focused progressive \REF{ex:lehmann:76}, the full verb does not need to be an \isi{agentive} verb, as shown by (\ref{ex:lehmann:83}–\ref{ex:lehmann:84}).

\ea\label{ex:lehmann:83}
Modern \ili{Yucatec} \ili{Maya}\\
\gll       bin      =in      ka'h  kíim-il\\
  \textsc{imm.fut } \textsc{=a.1.sg}   do    die-\textsc{incmpl}\\
\glt ‘I am going to die’ (FCP 395)
\z

\ea\label{ex:lehmann:84}
Modern \ili{Yucatec} \ili{Maya}\\
\gll       bin      =u    ka'h-o'b  suut      ba'ba'l-il-o'b\\
  \textsc{imm.fut} \textsc{=a.3}   do-3.\textsc{pl}  turn{\textbackslash}\textsc{introv}  demon-\textsc{advr-pl}\\
\glt ‘they were becoming demons’ (hnazario\_415.5)
\z

By desemanticization, the semantic component of motion has disappeared, and what remains is only the direct tie between present topic time and future event time. \textit{Bin … ka’h} is now a complex \isi{auxiliary} with the value of \isi{immediate future}.

\REF{ex:lehmann:85} can serve for the \isi{clitic} placement test:

\ea\label{ex:lehmann:85}
Modern \ili{Yucatec} \ili{Maya}\\
\gll       Behe'la'=e'  bin      =in      ka'h  wáah  túun  =in      kíins-ech?\\
  today=\textsc{r3  } \textsc{imm.fut } \textsc{=a.1.sg}    do    \textsc{int}    then  =\textsc{a.1.sg}   kill-\textsc{b.2.sg}\\
\glt ‘And now I shall kill you?’ (hk'an\_610)
\z

It shows that – in contrast with the \textit{bíin} of the predictive future – the first component of the \isi{discontinuous auxiliary} cannot be host to a \isi{clitic}, but the second component can. This is in consonance with the reduction processes to be analyzed in a moment and argues for the structural unity of the \isi{discontinuous auxiliary}.

The structure of this auxiliation is peculiar within the grammar of \ili{Yucatec} \ili{Maya} in several respects. First, this is the only \isi{auxiliary} which conditions different statuses on the full verb depending on the latter’s \isi{transitivity}, as is shown by \REF{ex:lehmann:82}. This is a reflection of the blending of two different constructions at its origin: The \isi{subjunctive} on the \isi{transitive verb} is a reflection of the \isi{motion-cum-purpose construction}, which requires this status for the purpose verb. The \isi{incompletive} on the \isi{intransitive verb} is its nominalized form, which in turn is required by the \isi{preposition} which originally governed this verbal core. It only remains to find out why the \isi{intransitive} morphology reflects the verb-\isi{focus construction}, while the \isi{transitive} morphology reflects the \isi{motion-cum-purpose construction}.

Secondly, \textit{bin … ka’h} is the only \isi{discontinuous auxiliary} of the language. What is more, the real \isi{auxiliary} in the construction is the component \textit{ka’h}. This, however, does not occupy the \isi{clause-initial position} taken by all the other auxiliaries of the language. This position is, instead, taken by a verb which has the role of a full verb in the \isi{source construction}. Thirdly, while \textit{bin} is impersonal like all the other auxiliaries, \textit{ka’h} is the only one with personal inflection. As a consequence, with \isi{transitive} full verbs, the subject is cross-referenced twice \citep[82]{BriceñoChel1998}, as is apparent from examples like (\ref{ex:lehmann:82}b). There is, consequently, much redundancy in this auxiliation. In the colloquial register of Modern \ili{Yucatec}, the full forms are rarely used. They are normally reduced in phonologically irregular ways, and there is currently much variation in this respect. \citeauthor{BriceñoChel1998} (\citeyear[82]{BriceñoChel1998}, \citeyear{BriceñoChel2000ir}:88f, \citeyear{BriceñoChel2006}: §§1.2, 1.3) notes the fusion of \textit{bin in/a/u ka’h} into \textit{nika’h/naka’h/nuka’h}, as in (\ref{ex:lehmann:86}a). If the full verb is \isi{transitive} and therefore preceded by a Set A index, the \textit{ka’h} of the \isi{auxiliary} coalesces with it, as in \#b.

\ea\label{ex:lehmann:86}
Modern \ili{Yucatec} \ili{Maya}\\
\ea
\gll  Ni-ka’h          meyah    t=in        kool.\\
  \textsc{imm.fut{\textbackslash}a.1.sg-}do  work      \textsc{loc=a.1.sg}  milpa\\
\glt ‘I am going to work on my cornfield.’ (\citealt[88]{BriceñoChel2000ir})

\ex
\gll   Ni-k=in                hant          bak’\\
\textsc{imm.fut{\textbackslash}a.1.sg}{}-do=\textsc{a.1.sg}  eat:\textsc{trr(subj)}  meat\\
\glt ‘I am going to eat meat’ (\citealt[99]{BriceñoChel2000ir})
\z
\z

Other idiosyncratic mergers occur in a variant of the construction in which the \textit{ka’h} component takes Set B indexes. Using this variant with a \isi{transitive verb} leads to cross-referencing the subject three times. The reduction processes applied in this context disguise this to a certain extent. Thus, the first syllable of the complex \isi{auxiliary} in \REF{ex:lehmann:87} contains the vowel of the 1\textsuperscript{st} person sing. Set A \isi{clitic}.

\ea\label{ex:lehmann:87}
Modern \ili{Yucatec} \ili{Maya}\\
\gll       mi-ka'h-en            =in      wa'l       te'x ...\\
  \textsc{imm.fut{\textbackslash}a.1.sg}{}-do-\textsc{b.1.sg } \textsc{=a.1.sg}   say(\textsc{subj})  you.\textsc{pl}\\
\glt ‘I’m going to tell you ...’ (FCP\_043)
\z

However, contractions with clitics of other persons may also contain an \textit{i}, so that the interim result of these changes is an \isi{auxiliary} which takes Set B suffixes to cross-reference the subject of the clause core. In cases like \REF{ex:lehmann:87} it leads to doubling, quite untypical of the language. The only comment one may make on the situation is that before a construction becomes a fixed grammaticalized inflected form, much variation occurs.

The other \ili{Yucatecan} languages, too, have developed an \isi{immediate future} on the basis of a focused progressive of the \isi{motion-cum-purpose construction} involving their cognates of \ili{Yucatec} \textit{bin} ‘go’. \REF{ex:lehmann:88} shows the focused progressive with the defective pro-verb in \ili{Lacandón}, which here already assumes an imminent future function (Bruce S. 1968: 80, 101):

\ea\label{ex:lehmann:88}
\ili{Lacandón}\\
\gll       ok’ol        =u    kah\\
  weep-\textsc{incmpl } \textsc{=a.3}    do\\
\glt ‘he is about to cry’ \citep[80]{Bruce1968}
\z

Applying this to the \isi{motion verb} of the \isi{motion-cum-purpose construction} already illustrated by \REF{ex:lehmann:69} yields the \ili{Lacandón} \isi{immediate future}. Just as in \ili{Yucatec}, reduction of the \isi{immediate future} construction involves merger of the Set A index preceding the \isi{transitive} full verb with the \isi{auxiliary} \textit{kah} immediately preceding it. Thus, \textit{kah=in/a/u} yields \textit{kin/ka/ku} \citep[95, 101]{Bruce1968}, as in \REF{ex:lehmann:89} (where \textit{kah} must be a variant of \textit{k=a} [do=A.2]) and (\ref{ex:lehmann:90}a).

\ea\label{ex:lehmann:89}
\ili{Lacandón}\\
\gll       Bin  =a    kah  päy-e      lu’um-o’,\\
  go    =\textsc{a.2}   do    carry-\textsc{subj}    earthling-\textsc{pl}\\
\glt
‘You are going to take the earthlings with you,’ \citep[76]{Bruce1968}
\z

As an alternative to the construction of \REF{ex:lehmann:69}, an \isi{intransitive} purpose clause may be introduced by the \isi{preposition} \textit{ti}, as in (\ref{ex:lehmann:90}b). This may be seen as a direct continuation of the Colonial construction represented by \REF{ex:lehmann:79} and is furthermore in analogy with the debitive construction illustrated by \REF{ex:lehmann:61}.

\ea\label{ex:lehmann:90}
\ili{Lacandón}\\
    \ea
\gll  bin      =in      k=in        wuk’-ik\\ 
  \textsc{imm.fut } \textsc{=a.1.sg}   do=\textsc{a.1.sg}   drink-\textsc{incmpl}\\
\glt ‘I am going to drink it’

\ex
\gll  bin      =in      kah  t=in        wuk’-ul\\
\textsc{imm.fut } \textsc{=a.1.sg}   do    \textsc{loc=a.1.sg}  drink-\textsc{incmpl}\\
\glt ‘I am going to drink’ \citep[101]{Bruce1968}
\z
\z

In \REF{ex:lehmann:91} from \ili{Itzá}, the verb \textit{b’el} ‘go’ is the full verb occupying the \isi{focus position} in a simple verb-\isi{focus construction}.

\ea\label{ex:lehmann:91}
\ili{Itzá}\\
\gll        (B’el)      =u    ka’a    ich  =u    kool.\\
  go(\textsc{incmpl) } \textsc{=a.3}    do/go  in    =\textsc{a.3}    milpa\\
\glt ‘He is going to his cornfield.’ \citep[90f]{BriceñoChel2000ir}
\z

If \textit{b’el} is the \isi{motion verb} of a \isi{motion-cum-purpose construction}, an \isi{intransitive verb} in the purpose clause is subordinated by the \isi{preposition} \textit{ti}, as in \REF{ex:lehmann:92}, while a \isi{transitive verb}, as in \REF{ex:lehmann:93}, is in the \isi{subjunctive}.

\ea\label{ex:lehmann:92}
\ili{Itzá}\\
\gll        (B’el)  =u    ka’a    ti      han-al.\\
  go      =\textsc{a.3}    do/go  \textsc{loc}    eat-\textsc{incmpl}\\
\glt ‘He is going to eat.’ \citep[91]{Bruce1968}
\z

\ea\label{ex:lehmann:93}
\ili{Itzá} \\
\gll      U-ka'ah    =u    b’et-eh      =u-yotoch\\
  \textsc{a.3}{}-do/go    =\textsc{a.3} make-\textsc{subj } \textsc{=a.3}{}-home\\
\glt `He is going to make his home' (\citealt{Hofling1991}, 1:5)
\z

The peculiarity here is that since occurrence of the defective verb \textit{ka’a} is all but limited to the construction with \textit{b’el} in focus,\footnote{\citet[17]{Hofling1991} does present an example with \textit{ka’a} as the main verb meaning ‘do’. A similar construction in Ch'ol employs the cognate verb \textit{cha’l} ‘do’ (\citealt{Coon2010}, §3.1).} it assumes the sense of ‘go’ by syntagmatically mediated coding \citep{Lehmann2014}. Consequently, \textit{b’el} becomes redundant and may be omitted. This is true not only for the \isi{immediate future} developed from the \isi{motion-cum-purpose construction}, but also for the simple verb-\isi{focus construction} of \REF{ex:lehmann:91}.\footnote{This is mentioned in \citet{BriceñoChel2000ir}, but not in \citet{Hofling1991}.}

The facts of \ili{Mopán}, finally, are similar. \REF{ex:lehmann:94} illustrates the simple verb-\isi{focus construction}.

\ea\label{ex:lehmann:94}
\ili{Mopán}\\
\gll      T’an  =in-ka’aj.\\
  speak  =\textsc{a.1.sg-}do\\
\glt ‘I am speaking.’ \citep[154]{Hofling2011}
\z

\REF{ex:lehmann:95} shows the \isi{immediate future} construction with an \isi{intransitive} full verb in the second person (cf. \citealt{Hofling2011}:153). The \#a and \#b examples represent the full and reduced variants, resp. The same relationship holds between (\ref{ex:lehmann:96}a) and (b), where the pronominal \isi{enclitic} preceding the \isi{transitive verb} is involved in the contraction, too. As may be seen, contraction of the \isi{auxiliary} with the Set A index works similarly as in the \ili{Yucatec} \REF{ex:lehmann:86}. Moreover, the \isi{intransitive verb} of \REF{ex:lehmann:95} is in the \isi{incompletive} status and subordinated by \textit{ti}, while the \isi{transitive verb} of \REF{ex:lehmann:96} is in the \isi{subjunctive}.

\ea\label{ex:lehmann:95}
\ili{Mopán} \\
 \ea
\gll   Bel        =a    ka’a    ti    wäy-el.\\
  go(\textsc{incmpl)} \textsc{=a.2}   do      \textsc{loc}  sleep-\textsc{incmpl}\\
\glt ‘You are going to sleep.’
\ex
\gll  B=a-ka’a    ti    wäy-el.\\
  go=\textsc{a.2}{}-do  \textsc{loc}  sleep-\textsc{incmpl}\\
\glt ‘You’re going to sleep.’
\z
\z

\ea\label{ex:lehmann:96}
\ili{Mopán}\\

\ea
\gll  Bel      =in      ka’a  =in      koykin          =a    nene’e\\
  \textsc{imm.fut } \textsc{=a.1.sg}   do    =\textsc{a.1.sg}    lay.down(\textsc{subj})    =\textsc{dem}  baby\\
\glt ‘I am going to lay the baby to sleep’

\ex
\gll  B=i(n)-k=in              koykin          =a    nene’e\\
  \textsc{imm.fut=a.1.sg}{}-do=\textsc{a.1.sg}    lay.down(\textsc{subj})    =\textsc{dem}  baby\\
\glt ‘I’m going to lay the baby to sleep’ (\citealt[95]{BriceñoChel2000ir})
\z
\z

The languages of the \ili{Yucatecan} branch share all the essential properties of the \isi{immediate future} auxiliation: the \isi{discontinuous auxiliary}, the multiple cross-reference to the subject and the asymmetry of status marking of the full verb conditioned by its \isi{transitivity}, which reflects the contamination of two different syntactic constructions operative at the origin of this auxiliation. All four languages reduce this complex \isi{auxiliary construction}; but as the processes operative here are not phonologically regular, they also differ among the languages.

The grammaticalization of the construction is a process in two main phases:

\begin{itemize}
\item[a.] 
verb \isi{focus construction} {\textgreater} focused progressive

\item[b.]
focused progressive of \isi{auxiliary} ‘go’ {\textgreater} (simple) \isi{immediate future}.
\end{itemize}

More in detail, the following minimal steps compose the process:

\begin{itemize}
\item  The \isi{motion verb} \textit{bin} ‘go’ is semantically bleached; the movement component disappears.
\item  The \isi{incompletive} or \isi{subjunctive} verb remaining in the \isi{extrafocal} clause is reinterpreted as the main verb.
\item The internal structure of the complex “\textit{bin} set\_A\_index \textit{ka’h}" is blurred. By being forced into the Procrustean bed of the initial position, it is reanalyzed as a discontinuous \isi{immediate future} \isi{auxiliary} with internal inflection.
\item The whole sentence ceases to be complex; it is reinterpreted as a single clause.
\item Whatever may have remained of the focal emphasis on the initial verb vanishes; the construction becomes open to different information structures that may be superimposed.
\end{itemize}

The model of this complex reanalysis is the structure of the simple fully finite clause of \figref{fig:lehmann:6}, in which the initial \isi{auxiliary} combines with the \isi{enclitic} subject pronoun and is followed by the verbal complex (as, e.g., in (\ref{ex:lehmann:17}b)). The result of the change conforms to that model to the extent possible for a \isi{discontinuous auxiliary}.

\subsection{Auxiliation in Yucatecan languages}\label{sec:lehmann:4.10}

The inherited suffixal system, where a minimum \isi{aspect system} is coded as part of the status category, is renewed, in the period from Proto-\ili{Yucatecan} to Modern \ili{Yucatec}, by a large paradigm of aspectual auxiliaries. The sources of these auxiliaries are of different categories and form different syntactic constructions with the clause core. This explains the different status categories that they condition on the full verb. Conditioning them, they render them largely redundant. The new categories mark relatively fine \isi{distinctions} not only of aspectual, but also of temporal and \isi{modal} categories.

\subsubsection{Syntactic relations}\label{sec:lehmann:4.10.1}

The new set of auxiliaries is structurally completely different from the inherited suffixal status-aspect-mood system. Since it owes its origin essentially to grammaticalization, it is based on syntactic rules operative at the time of its formation. There are four syntactic constructions at work:

\begin{itemize}
\item[a.] an \isi{adverb} modifying the verbal clause (core) following it and leaving its status marking intact

\item[b.]
complementation, where a relational noun, an impersonal \isi{phase verb} or \isi{modal} verboid takes a verbal clause core in the dependent (subsequently \isi{incompletive}) status as its complement

\item[c.] 
the \isi{motion-cum-purpose construction}, where a verb of directed motion is followed by its purpose complement, represented by a verbal clause core in the \isi{subjunctive}

\item[d.] 
the verb-\isi{focus construction}, which puts the main verb of the clause into \isi{focus position}, leaving behind in the \isi{extrafocal} clause a pro-verb with all the dependents of the focused verb.
\end{itemize}

The primary structural division of this set contrasts construction \#a with constructions \#b – \#d. Construction \#a is mono-clausal from the beginning. The \isi{auxiliary} to-be bears a modifying relation to the clause core, which is syntactically independent. Constructions \#b – \#d transcend the simple clause; \#b and \#c are biclausal, \#d is clefted. In these, the \isi{auxiliary} to-be constitutes the \isi{main clause}, while the clause core depends on it. As a consequence, auxiliation strategy \#a leaves the syntactic relations in the clause core intact, while strategies \#b – \#d require some degree of nominalization of the clause core.

\largerpage[1]
This difference has consequences for the configuration of basic syntactic relations in the clause core. These do not concern the \isi{transitive} subject. Since Proto-\ili{Mayan}, this has been cross-referenced in all \ili{Mayan} languages by the same Set A indexes which also cross-reference the \isi{possessor}. This produces the \isi{ergative} pattern of alignment shown by the cross-reference indexes. Since it appears primarily in \isi{completive status}, which is semantically \isi{perfective}, one may plausibly assume that assignment of \isi{possessive} marking to the \isi{transitive} subject stems, in its turn, from a pre-historic nominalization process. Be that as it may, the subordination of the clause core with auxiliation strategies \#b – \#d again requires nominalization of the clause core. Since the underlying \isi{transitive} subject is already marked by a \isi{possessive} relation, the \isi{intransitive subject} now remains to be affected. This is why, in all tenses and aspects except the \isi{perfective}, and also excepting \isi{subjunctive} mood, it is marked by Set A indexes. The result is a rather peculiar form of aspect-conditioned split subject marking, which occurs in \isi{intransitive}, not in \isi{transitive} clauses.\footnote{\interfootnotelinepenalty=10000 If the analysis proposed in \citet[§6]{Coon2010} is accepted for \ili{Yucatecan}, the indexing pattern would be \isi{ergative} throughout, because what looks like accusative marking in almost all aspects actually occurs in subordinate clauses.}

\subsubsection{Grammaticalization of the auxiliary}\label{sec:lehmann:4.10.2}
Although the four constructions are clearly distinct, they share a \isi{clause-initial position} which becomes the melting-pot for the aspectual and \isi{modal} \isi{formatives} recruited from different sources. Paths \#a, \#c and \#d have been followed only once each in history; path \#b has been the most prolific one.

Since the process of renewal and grammaticalization of auxiliaries has not finished, the paradigm is open and heterogeneous both in functional and in structural terms. In contemporary \ili{Yucatec}, while all of the auxiliaries occupy the same structural position, the older ones are bound while the more recent ones are independent. And although several of them stem from verbs, they share the property of leaving \isi{conjugation} categories to the full verb while remaining uninflected themselves. This is true with the single exception of the \isi{immediate future} \isi{auxiliary}, which is idiosyncratic in many respects.

The grammaticalization of auxiliaries evidences a process of clause union: it shrinks an original biclausal construction into a monoclausal one. This is perhaps clearer in \ili{Mayan} languages, with their preference for verb-initial position and for impersonal constructions, than in many other languages. The many \isi{auxiliary} constructions of the \ili{Yucatecan} languages occupy all conceivable positions on a continuum from a complex sentence consisting of a matrix and a \isi{complement clause} down to a one-clause sentence. Once the matrix \isi{predicate} in initial position has been grammaticalized to an \isi{auxiliary}, one might think that the construction is monoclausal. However, a simple test like the form of the answer to a \isi{polar question} reveals that the \isi{auxiliary} keeps being the main \isi{predicate}.\footnote{While it is clear that the non-\isi{completive status} suffixes are explained diachronically by the syntactic relation between the \isi{auxiliary} and the full verb complex, \citet[§2.3.3]{Coon2010} insists that non-\isi{completive status} clauses in Ch'ol are synchronically subordinate to the \isi{auxiliary}. The \isi{imperfective} and progressive auxiliaries, which are at stake here, do have a few more verbal properties than the \ili{Yucatecan} \isi{imperfective} \isi{auxiliary}.} Only after the \isi{auxiliary} coalesces with the subject cross-reference index is it an irremovable part of a unitary clause.

The \isi{coalescence} of the \isi{auxiliary} with the following \isi{enclitic} subject index is especially interesting. In SAE languages, the \isi{auxiliary} is an element that hosts the \isi{conjugation} categories of a \isi{finite verb}, the most important of these being person and number. These are just the categories that the \ili{Yucatecan} \isi{auxiliary} lacks. Instead of denying it \isi{auxiliary} status on these grounds,\footnote{\citet[§4.13]{Andrade1955} has a rather extensive discussion on the applicability of this term to the \isi{formatives} in question.} it is intriguing to observe that, as a consequence of purely phonological enclisis, it coalesces with the subject indexes which syntactically accompany the following full verb, ending up as a morphologically complex form which codes not only \isi{tense}, aspect and mood, but also person and number like an SAE \isi{auxiliary}. However, the morphology – or maybe rather, the phonology – here is treacherous and not transparent to the syntax: even if merged with the preceding \isi{auxiliary}, the pronominal index clearly forms a syntactic constituent not with it, but with the following verb, as shown in \figref{fig:lehmann:6} and proved by numerous examples like \REF{ex:lehmann:4}, (\ref{ex:lehmann:6}c) and (\ref{ex:lehmann:32}–\ref{ex:lehmann:33}).

Although according to available descriptions, the complex of \isi{auxiliary} plus Set A index is prefixed to the full verb in other \ili{Mayan} languages, this has not happened in the \ili{Yucatecan} languages. First of all, the \isi{enclitic} status of the Set A index does not favor its univerbation with the material following it. Moreover, given the configuration “set\_A\_index X", neither X nor this binary configuration is categorially uniform, since X may either be the head of this syntagma or may be a modifier of a head which is yet to follow (an adjective in a noun phrase or an \isi{adverb} in a verbal complex). Consequently, although the \isi{auxiliary} forms a phonological complex with the Set A index in many cases, there are syntactic obstacles to the univerbation of this complex with the verb of the following verbal complex.\footnote{Some analysts (e.g. \citealt{Hofling1991}: 25; \citealt{Pye2009}: 266) claim the aspect auxiliaries to be prefixes. They are definitely not, in none of the \ili{Yucatecan} languages. \citet[37]{Hofling1991} keeps this analysis up by declaring the adverbs which may occur between the pronominal \isi{clitic} and the verb to be “incorporated into the verb”. This, however, is not so, witness the \isi{conjugation} shown by verbs preceded by such adverbs: the stems do not become complex by this combination, which shows that it is a syntactic construction.}

The grammaticalization of TAM in \ili{Yucatecan} languages is a clear example of convergence of grammaticalization paths starting from different sources. The convergence is fostered, if not forced, by a rather rigid syntactic framework that a clause must fit in: First,  an element that has scope over a verbal clause core must precede it. Although there are three distinct structural positions preceding a verbal clause core, their neutralization and merger into only one position is already predestined by the structure of \figref{dig:lehmann:7}. Second, all of the operators that may occupy this position are impersonal. With these two constants to begin with, practically the only variable is the syntactic relation between the initial element and the clause core. This then determines the status to be chosen on the full verb. Since this variation in status is conditioned rather than informative, it could, in principle, be leveled out with ongoing grammaticalization. However, phonological reduction has rendered a subset of aspect auxiliaries homonymous. These aspects can then only be distinguished by the different status categories that they condition. This, in turn, prevents, for the time being,\largerpage the disappearance of the status category.

The methodological lesson from the above for synchronic grammatical description is the following. Although all of the auxiliaries occupy the same structural position immediately preceding the clause core and although we are dealing with \isi{periphrastic} constructions, a description which aims to account for the status forms of the full verb which accompany the diverse initial aspectual “particles” has to make explicit the syntactic relations between the initial element and the clause core. This, in turn, is facilitated if the grammaticalization source\largerpage of these elements is taken into account.

\section{Conclusion}\label{sec:lehmann:5}
While many of the grammatical \isi{formatives} in the \ili{Mayan} languages are etymologically unrelated, their functional categories and their structural properties are often identical. For instance, \ili{Yucatecan} and Ch'olan languages share a large portion of the system of TAM auxiliaries; and these appear in the same structural position in all of these languages. What is more: They share particular aspects such as the \isi{perfective}, \isi{imperfective}, progressive etc.; but the morphemes appearing in these functions are unrelated. One must infer from this picture that the \ili{Mayan} languages have been very conservative, over the millennia, as to their grammatical structure, and have limited themselves to renewing the \isi{formatives} from time to time.

In view of the fact that grammaticalization is again and again hawked as a process of linguistic change, one must emphasize again and again that it is a process of linguistic variation both on the synchronic and on the diachronic axes. Moreover, history is always more complicated than diachrony: Variants that succeed each other on a dimension of grammaticalization co-occur synchronically, both within one language and across sister languages. And what would be a unitary source of a grammaticalized construction if one had to reconstruct it, with consideration of historical data turns out to be a set of variants and competing constructions that contributed in shaping the construction in question.

\section*{Abbreviations} 
\begin{multicols}{2}
\begin{tabbing}
\textsc{rec.pst}\hspace{1em} \=  possessive/subject function\kill
\textsc{a} \>   {possessive}/subject function\\
\textsc{advr} \>   adverbializer\\
\textsc{an} \>   animate\\
\textsc{b} \>   absolutive function\\
\textsc{caus} \>   {causative}\\
\textsc{cfp} \>   clause final particle\\
\textsc{cl} \>   classifier\\
\textsc{cmpl} \>   {completive}\\
\textsc{conj} \>   conjunction\\
\textsc{cop} \>   {copula}\\
% % % \textsc{cym} \>   Colonial \ili{Yucatec} \ili{Maya}\\
\textsc{deag} \>   deagentive\\
\textsc{deb} \>   debitive\\
\textsc{dem} \>   {demonstrative}\\
\textsc{dep} \>   {dependent status}\\
\textsc{exist} \>   existent\\
\textsc{fut} \>   future\\
\textsc{hab} \>   habitual\\
\textsc{hod} \>   hodiernal past\\
\textsc{imm} \>   immediate (future)\\
\textsc{inan} \>   inanimate\\
\textsc{incmpl} \>   {incompletive}\\
\textsc{int} \>   interrogative\\
\textsc{introv} \>   introversive\\
\textsc{m} \>   masculine\\
% % % \textsc{mym} \>   Modern \ili{Yucatec} \ili{Maya}\\
\textsc{neg} \>   negator\\
\textsc{negf} \>   negator, final part\\
\textsc{pass} \>   passive\\
\textsc{pfv} \>   {perfective}\\
\textsc{pl} \>   plural\\
\textsc{prf} \>   perfect\\
\textsc{prog} \>   progressive\\
\textsc{prsv} \>   presentative\\
\textsc{quot} \>   quotative\\
rc  \> referential {clitic}\\
\textsc{r1} \>   {clitic} of 1\textsuperscript{st}   person deixis\\
\textsc{r2} \>   {clitic} of 2\textsuperscript{nd}   person deixis\\
\textsc{r3} \>   non-{deictic} referential {clitic}\\
\textsc{rec.pst} \>   recent past\\
\textsc{sg} \>   singular\\
\textsc{subj} \>   {subjunctive}\\
\textsc{tam} \>   {tense}/aspect/mood\\
\textsc{term} \>   terminative\\
\textsc{top} \>   topic\\
\textsc{trr} \>   transitivizer\\
\textsc{voc} \>   vocative \\ 
\end{tabbing}
\end{multicols}
 
{\sloppy
\printbibliography[heading=subbibliography,notkeyword=this]
}
\end{document}