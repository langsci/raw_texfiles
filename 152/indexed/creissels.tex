\documentclass[output=paper]{langsci/langscibook} 
\title{Copulas originating from the imperative of ‘see\slash look’ verbs in Mande languages} 
\shorttitlerunninghead{Copulas originating from the imperative of ‘see\slash look’ verbs in Mande}
\author{%
Denis Creissels\affiliation{University of Lyon}  
}
\ChapterDOI{10.5281/zenodo.823228} %will be filled in at production

\abstract{This paper analyzes Mande data that suggest a grammaticalization path leading from the imperative of ‘see\slash look’ verbs to ostensive predicators (i.e. words functionally similar to French \textit{voici}, Italian \textit{ecco}, or Russian \textit{vot}), and further to copulas. Clear cases of copulas cognate with ‘see\slash look’ verbs are found in several branches of the Mande family, and there is convincing evidence that they did not develop from the semantic bleaching of forms originally meaning ‘is seen\slash found’ (another plausible grammaticalization path leading from ‘see’ verbs to copulas), but from the routinization of the ostensive use of the imperative of ‘see\slash look’. Comparison of the Mande data with the Arabic data provided by \citet{Taine-Cheikh2013} shows however that this is not the only possibility for imperatives of ‘see\slash look’ verbs to grammaticalize into copulas, since in the Arabic varieties in which the imperative form of ‘see’ has become a plain copula, the most plausible explanation is that a modal/discursive particle resulting from the grammaticalization of the imperative of ‘see’ has undergone a process of semantic bleaching in the context of an equative or locational predicative construction that initially included no overt predicator.}
\maketitle
\begin{document}
  
  

% \textbf{Keywords}: Grammaticalization, \isi{ostensive} \isi{predicator}, \isi{copula}, \ili{Mande}, Arabic
\section{Introduction}\label{sec:creissels:1}

The \isi{grammaticalization path} leading from the \isi{imperative} of ‘see\slash look’ verbs to \isi{ostensive} predicators or to copulas is not mentioned in the inventory of grammaticalization processes provided by \citet{Heine2002}, and ‘see\slash look’ verbs are not mentioned as a possible source of copulas in general accounts of non-verbal \isi{predication} such as \citet{Hengeveld1992} or \citet{Pustet2003} either. However, French \textit{voici}/\textit{voilà} constitute a well-known example of the grammaticalization of the \isi{imperative} of a ‘see’ verb as an \isi{ostensive} \isi{predicator}, and additional examples can be found for example among \ili{Chadic} languages (see \citet[380--382]{Hellwig2011} on \ili{Goemai}; \citet[468--469]{Jaggar2001} and \citet[181--182]{Newman2000} on \ili{Hausa}). The possibility that the \isi{imperative} form of a ‘see\slash look’ verb grammaticalizes as a \isi{copula} has been recognized so far in two language families, and in one of these two cases, the first stage in this evolution is the reanalysis of the \isi{imperative} of a ‘see\slash look’ verb as an \isi{ostensive} \isi{predicator}:
(a)  As discussed by \citet{Taine-Cheikh2013}, in Arabic languages, the grammaticalization of the \isi{imperative} form of verbs cognate with Classical Arabic \textit{raʔā} ‘see’ has developed in different directions, with the creation of a \isi{copula} as one of its possible outcomes.\footnote{The grammaticalization of the \isi{imperative} of ‘see’ into an \isi{ostensive} particle, and further into a \isi{copula}, in some Arabic varieties, was already briefly signaled by \citet[43]{Rubin2005}.}
(b)  As observed by \citet{Westermann1930}, \citet{Monteil1939}, \citet{Heydorn1940-1941}, \citet{Heydorn1949-1950}, \citet{Welmers1974}, \citet{Creissels1981}, and \citet{Tröbs2003}, \ili{Mande} languages provide evidence that copulas may result from the evolution of \isi{ostensive} predicators whose origin is the \isi{imperative} of a ‘see’ verb.
This is however not the only possible type of evolution resulting in the creation of a \isi{copula} or an existential verb from a ‘see’ verb. Cross-linguistically, the translation equivalents of ‘see’ may be polysemous verbs expressing the meanings commonly expressed in English as ‘find’ or ‘get’, and it is easy to imagine a process of semantic bleaching converting a form meaning ‘is found’ into a locational \isi{copula}. As rightly observed by a reviewer, in \ili{Sanskrit} the root VID ‘see/know’ (from Indo-European *\textit{weid}) in passive form (\textit{vid-ya-te}) was used in the classical language with the meaning ‘there is’, and more generally, the pathway (IS\_SEEN{\textasciitilde})IS\_FOUND > LOCATIONAL COPULA (or variants thereof)\footnote{The same reviewer signals that the pathway APPEAR > COPULA/EXISTENTIAL VERB is unambiguously attested in some varieties of \ili{Tibetan}, where the reflexes of Classical \ili{Tibetan} \textit{snang} ‘appear’ are used as \isi{copula} and evidential marker \citep{Suzuki2012}.} may be more common cross-linguistically than the creation of copulas from the \isi{imperative} form of ‘see\slash look’.
In this article, after clarifying the notion of the \isi{ostensive} \isi{predicator} (\sectref{sec:creissels:2}) and providing some background information on \ili{Mande} languages, and in particular on \ili{Mande} \isi{predicative} constructions (\sectref{sec:creissels:3}), I present comparative data on copulas originating from ‘see\slash look’ verbs in \ili{Mande} languages (\sectref{sec:creissels:4}). \sectref{sec:creissels:5} compares the \ili{Mande} data with the Arabic data provided by \citet{Taine-Cheikh2013}. In \sectref{sec:creissels:6}, I discuss the details of two possible grammaticalization paths whereby the \isi{imperative} of a ‘see\slash look’ verb may be converted into a \isi{copula}. In \sectref{sec:creissels:7}, I discuss evidence against the alternative hypothesis according to which the Arabic and \ili{Mande} copulas analyzed in this article might have resulted from the pathway (IS\_SEEN{\textasciitilde})IS\_FOUND > LOCATIONAL COPULA. \sectref{sec:creissels:8} summarizes the main conclusions.

\section{Ostensive predicators}\label{sec:creissels:2}

I define \textit{\isi{ostensive} predicators} as grammatical words or expressions whose combination with a noun phrase constitutes the core of clauses aiming to draw the attention of the addressee to the presence of some entity in the situation within which the speaker-addressee interaction takes place (speech situation), such as French \textit{voici}, English \textit{here is}, \ili{Italian} \textit{ecco}, \ili{Russian} \textit{vot}, etc. 
Ostensive predicators are more commonly called ‘presentative particles’ \citep{Petit2010}, but this term is ambiguous in two respects: on the one hand, ‘presentative’ is sometimes used as an equivalent of ‘existential’, and on the other hand, the label ‘presentative particle’ is sometimes used for words that have a different distribution (in particular, for interjections).
  Ostensive predicators entail meanings typically expressed by copulas: identification of a referent, and presence of a referent at some place. They differ from copulas in two crucial respects: the \isi{deictic} component of their meaning, and syntactic constraints following from the particular \isi{illocutionary force} they carry. The argument of an \isi{ostensive} \isi{predicator} must be located in the speech situation, and \isi{ostensive} clauses can be neither negated nor questioned, since their function is to draw the addressee’s attention to an obvious fact. In this respect, some similarity can be recognized between \isi{ostensive} clauses and exclamatory clauses. 
  In addition to their use in clauses that consist of just the \isi{ostensive} \isi{predicator} and a noun phrase, \isi{ostensive} predicators often occur with the same \isi{deictic} meaning in constructions in which they combine with a \isi{complement clause} – ex. (\ref{ex:creissels:1}b), or in constructions that can be described as including a secondary \isi{predication} (or ‘small clause’) – ex. (\ref{ex:creissels:1}c).

\ea%1
    \label{ex:creissels:1}
    \ili{French}\\
   \ea
    \gll   Voici  nos  amis.\\
      \textsc{ost}  our  friends\\
      \glt  ‘Here are our friends.’\\
   \ex
    \gll   Voici  que  nos  amis  arrivent.\\
      \textsc{ost}  \textsc{comp}  our  friends  arrive\\
      \glt { ‘Behold, our friends are coming.’ (lit. ‘Here is that our friends are coming!’)}\\
   \ex
    \gll   Voici  nos  amis  qui  arrivent.\\
      \textsc{ost}  our  friends  \textsc{rel}  arrive\\
      \glt {‘Behold, our friends are coming.’ (lit. ‘Here are our friends that are coming!’)}
\z
\z


\section{Verbal predication and copulas in Mande languages}\label{sec:creissels:3}

\subsection{Some background information about the Mande language family}\label{sec:creissels:3.1}

The \ili{Mande} \isi{language family} includes about 50--60 languages (depending on\linebreak whether relatively close varieties are counted as distinct languages or dialects of a single language) whose common ancestor is evaluated as dating back 5000--6000 years. The unity of the \ili{Mande} \isi{language family} was recognized very early in the history of African linguistics, because of its remarkable typological homogeneity. Its validity as a genetic grouping is uncontroversial, but the nature of its relationship to other language families of Subsaharan Africa remains an open question. The \ili{Mande} \isi{language family} was included by Greenberg in the Niger-Congo phylum, but the evidence supporting this decision is rather slim, and the Niger-Congo affiliation of \ili{Mande} is considered questionable by many specialists – on this question, see \citet{Dimmendaal2011}. A simplified version of the current classification of \ili{Mande} languages is given in \figref{tab:creissels:2} with the names of the languages mentioned in this article in italics.


\begin{figure}
\caption{The Mande language family (adapted from \citealt{Vydrin2009})}
 \label{tab:creissels:2}
\fbox{\begin{tabularx}{\textwidth}{lXX}    
South-East \ili{Mande} & South \ili{Mande}  & Dan\\
& & Guro \\
& & Mano\\
& & etc.\\
& East \ili{Mande} & Bisa\\
& & San \\
& & Busa\\
& & etc.\\
West \ili{Mande} & \ili{Soninke}-\ili{Bozo} & \textit{Soninke}\\
           &              & \textit{\ili{Bozo} languages} \\
&  Bobo-Samogo & Bobo \\
& & Dzuun\\
& & etc.\\
& Central  & \textit{\ili{Manding} languages}\\
& & Jogo-Jeri\\
& & Kono-Vai\\
& & etc.\\
& Soso-South-West-\ili{Mande} & Soso-Jalonka\\
& & \textit{South-West \ili{Mande} languages} \\
& & (\ili{Mende}, Kpelle, Loma, etc.)\\
\end{tabularx}}
\end{figure}


For more details on the internal classification of \ili{Mande} languages, see \citet{Vydrin2009}.

\subsection{Verbal predication in Mande languages}\label{sec:creissels:3.2}

In \ili{Mande} languages, verbal \isi{predication} can be schematized as S~(O)~V~(X).\footnote{S = subject, O = object, V = verb, X = oblique.} No variation is possible in the linear order of constituents. Predicative constructions with two or more terms encoded in the same way as the patient of typical monotransitive verbs (so-called ‘multiple object constructions’) are not possible.
In \ili{Mande} languages, an important characteristic of verbal \isi{predication} is the existence of paradigms of grammatical words (or clitics), called \textit{\isi{predicative} markers} in the Mandeist tradition, occupying a fixed position immediately after the subject. They express TAM, \isi{transitivity} and polarity \isi{distinctions}, either by themselves or in interaction with morphological variations of the verb. The division of labor between \isi{predicative} markers and suffixal or tonal verb inflection varies greatly from one \ili{Mande} language to another.
  For example, in \ili{Soninke}, the paradigm of \isi{predicative} markers includes (among others) \textit{má} ‘\isi{completive}, negative’, \textit{dà} ‘\isi{transitivity} marker’, and the locational \isi{copula} \textit{wá} (negative \textit{ntá}), which in combination with verbs in the gerundive fulfills the function of \isi{incompletive} \isi{auxiliary} – ex. \REF{ex:creissels:3}. Verb inflexion is limited to the gerundive suffix \textit{-n\'{V}} (where \textit{V} represents a copy of the preceding vowel), and a tonal alternation by which an entirely L contour substitutes for the inherent tonal contour of the verbal lexeme. The slot for \isi{predicative} markers (immediately after the subject NP) is left empty if one of the following two combinations of values is intended: ‘\isi{intransitive}, \isi{completive}, positive’ or ‘\isi{intransitive}, \isi{imperative} singular, positive’; in all other cases an overt \isi{predicative marker} must be present.

\ea%3
    \label{ex:creissels:3}    
   \ili{Soninke} (pers. doc.)\\

   \ea
    \gll   Ké  yúgó  xàrá. \\
      \textsc{dem}  man  study\\
      \glt {‘This man studied.’}\\

   \ex
    \gll   Ké  yúgó  má  xàrà.\\
      \textsc{dem}  man  \textsc{cpl.neg}  study\textsuperscript{\tiny L}\\
      \glt {‘This man did not study.’}\\

   \ex
    \gll   Lémínèn  dà  í  hàabá  ŋàrí.\\
      child-\textsc{d}  \textsc{tr}  \textsc{refl}  father  see\\
      \glt {‘The child saw his father.’}\\

   \ex
    \gll   Lémínèn  má  í  hàabá  ŋàrì.\\
      child-\textsc{d}  \textsc{cpl.neg}  \textsc{refl}  father  see\textsuperscript{\tiny L}\\
      \glt {‘The child did not see his father.’}\\

   \ex
    \gll   Ó  wá  táaxú-nú  dàagó-n  kànmá.\\
      \textsc{1pl}  \textsc{loccop}  sit-\textsc{ger}  mat-\textsc{d}  on\\
      \glt {‘We will sit on the mat.’}\\

   \ex
    \gll   \'N  dà  dòròkê-n  qóbó  sáxà-n  ŋá. \\
      \textsc{1sg}  \textsc{tr}  dress-\textsc{d}  buy  market-\textsc{d}   \textsc{postp}\\
      \glt {‘I bought a dress at the market.’}\\

   \ex
    \gll   \'N  ntá  dòròké  qòbò-nò  án  dà. \\
      \textsc{1sg}  \textsc{loccop.neg}  dress  buy-\textsc{ger}\textsuperscript{\tiny L}  \textsc{2sg}  for\\
      \glt {‘I will not buy a dress for you.’}\\


\z
\z


The rigid constituent order is crucial for the recognition of grammatical relations. In \ili{Mande} languages, the flagging of core syntactic terms is either totally inexistent, or very marginal. As regards argument indexation, some \ili{Mande} languages have subject indexes attached or incorporated to the \isi{predicative marker} (never to the verb itself), the others have no subject indexation at all. A mechanism that can be described as object indexation is found only in some languages in which the third person singular object pronoun has fused with the verb, and in which ‘third person singular object’ is encoded by a modification of the initial of the verb – see \REF{ex:creissels:16} below.

\subsection{Copulas in Mande languages}\label{sec:creissels:3.3}

In most \ili{Mande} languages, non-copular \isi{equative} or locational clauses (i.e. \isi{equative} or locational clauses without any explicit \isi{predicator}) are marginal. Equative \isi{predication} and locational \isi{predicative} constructions in \ili{Mande} languages can be schematized as S \textsc{cop} X. S is an unflagged NP sharing with the subject of verbal \isi{predication} its obligatoriness and its \isi{clause-initial position}. X shares with the obliques in verbal \isi{predication} the following two properties: it follows the \isi{predicative} element, and its most common form is that of an adpositional phrase, even in \isi{equative predication}.\footnote{In\largerpage \ili{Mande} languages, the second term of \isi{equative predication} is commonly flagged by means of the posposition that marks ‘functive’ phrases (i.e. the equivalent of \textit{as}{}-phrases in English) in verbal \isi{predication} – see \citet{Creissels2014}.{\clubpenalties 1 10000 \brokenpenalty 0 \par}} The position occupied by the \isi{copula} is comparable to that occupied by the verb in \isi{intransitive} verbal \isi{predication}, and in terms of possible syntactic operations, copular clauses are not different from \isi{intransitive} verbal clauses. The only difference is that the copulas have no inflexion, and do not combine with \isi{predicative} markers, which makes it impossible for copular clauses to express the TAM variations expressed by verb inflection and \isi{predicative} markers in canonical verbal \isi{predication}. The use of the verbs ‘become’ (in the case of \isi{equative predication}) and ‘be found’ (in the case of locational \isi{predication}) constitutes the usual strategy to bypass this impossibility.
Typically, \ili{Mande} languages have (at least) two distinct positive copulas: an \isi{equative copula} and a locational \isi{copula}. In the negative, they may correspond to two distinct negative copulas, as in \ili{Soninke} – ex. \REF{ex:creissels:4}, but it may also happen that the same negative \isi{copula} is used in \isi{equative} and locational \isi{predication}. As a rule, negative copulas bear no resemblance to their positive counterparts.

\ea%4
    \label{ex:creissels:4}    
   \ili{Soninke} (pers. doc.)\\
   \ea
    \gll   Ké  yúgó  nì  tàgé-n  ñà  yí.\\
      \textsc{dem}  man  \textsc{eqcop}  blacksmith-\textsc{d}  \textsc{foc}   \textsc{postp} \\
      \glt {‘This man is a blacksmith.’}\\
   \ex
    \gll   Ké  yúgó  hètí  tàgé  yì. \\
      \textsc{dem}  man  \textsc{eqcop}.\textsc{neg}  blacksmith  \textsc{postp}\\
      \glt {‘This man is not a blacksmith.’}\\

   \ex
    \gll   Múusá  wá  kónpè-n  dí. \\
      Moussa  \textsc{loccop}  room-\textsc{d}  in\\
      \glt {‘Moussa is in the room.’}\\
   \ex
    \gll   Múusá  ntá  kónpè-n  dí. \\
      Moussa  \textsc{loccop}.\textsc{neg}  room-\textsc{d}  in\\
      \glt {‘Moussa is not in the room.’}\\
\z
\z


\subsection{Copulas in auxiliary function}\label{sec:creissels:3.4}

As already illustrated in (\ref{ex:creissels:3}e) and (\ref{ex:creissels:3}g) above, it is common in \ili{Mande} languages that locational copulas in \isi{incompletive} \isi{auxiliary function} combine with verbs, in constructions that lend themselves to a straightforward analysis according to which the \isi{copula} fulfilling this function occupies the slot for \isi{predicative} markers. In some \ili{Mande} languages (for example, in \ili{Soninke}), the distinction between the use of copulas as \isi{predicative} markers in verbal \isi{predication}, and periphrases in which the complement of the \isi{copula} is a \isi{nominalized verb}, is quite clear-cut, but in some others, this distinction may be more or less problematic. This is not unexpected, since diachronically, periphrases in which the complement of the \isi{copula} is a \isi{nominalized verb} are a source from which constructions with copulas in \isi{predicative marker} function can develop.

\section{‘See/look’ verbs, ostensive predicators, and copulas in Mande languages}\label{sec:creissels:4}

Ostensive clauses formally analyzable as \isi{imperative} clauses headed by a ‘see\slash look’ verb are common in \ili{Mande} languages. Clear cases of copulas originating from the \isi{imperative} of ‘see\slash look’ verbs can be found in Southwestern \ili{Mande} languages and in the \ili{Manding} dialect cluster. Moreover, there is some evidence that the locational \isi{copula} of \ili{Soninke} might have the same origin.

\subsection{Copulas originating from ‘see\slash look’ verbs in Southwestern Mande}\label{sec:creissels:4.1}

Southwestern \ili{Mande} is a group of closely related languages including \ili{Mende}, Loko, Kpelle, Loma, Zialo, and \ili{Gbandi}. A common root *\textit{káa} ‘see’ can be reconstructed for Proto-Southwestern-\ili{Mande} (Valentin Vydrin, p.c.). In Kpelle, \textit{ka} ‘see’ is also an \isi{ostensive} \isi{predicator}, a locational \isi{copula} and a \isi{progressive auxiliary} – ex. \REF{ex:creissels:5}. A similar situation is found in \ili{Looma} \citep{Sadler2006} and \ili{Gbandi} \citep{Heydorn1940-1941}.

\ea%5
    \label{ex:creissels:5}
   \ili{Kpelle} (\citealt[3, 10, 11, 12]{Westermann1930})\\

   \ea
    \gll   Ku  ŋaloŋ  ka  bɛlɛi  mu.\\
      \textsc{1pl}  man  see  house  in\\
      \glt {‘We saw a man in the house.’}\\

   \ex
    \gll   I  seŋkau  ka!\\
      \textsc{2sg}  money  \textsc{ost}\\
      \glt {‘Here is your money!’}\\

   \ex
    \gll   Ŋaloŋ  ka  bɛlɛi  mu.\\
      man  \textsc{cop}  house  in\\
      \glt {‘The man is in the house.’}\\

   \ex
    \gll   Nɛni  ka  pai.\\
      woman  \textsc{prog}  come\\
      \glt {‘The woman is coming.’}\\
  \z
\z

In addition to the coincidence between Kpelle \textit{ka} ‘see’, \textit{ka} \isi{ostensive} \isi{predicator}, \textit{ka} locational \isi{copula}, and \textit{ka} \isi{progressive auxiliary}, Westermann observes that the behavior of the NP preceding the locational \isi{copula} \textit{ka} or the progressive marker \textit{ka} is different from the behavior of subjects in other \isi{predicative} constructions, and the explanation he put forward is that the subject of the locational \isi{copula} and the subject of verbs in the \isi{progressive construction} were originally the object of \textit{ka} ‘see’ in the \isi{imperative}: “the \textit{ka} in form No 3 [i.e. in the \isi{progressive construction}] is perhaps the verb \textit{ka} to see, so that the form really means ‘see me coming’, ‘see him coming’, etc.” \citep[11]{Westermann1930}. In other words, the \isi{grammaticalization path} analyzed in this paper was explicitly put forward for the first time in Westermann’s description of Kpelle.

\subsection{Copulas originating from ‘see\slash look’ verbs in Manding}\label{sec:creissels:4.2}

\ili{Manding} is a dialect cluster included in the Central sub-branch of the Western branch of the \ili{Mande} family. The analysis of \ili{Manding} as a single macro-language including some relatively divergent dialects, or as a set of distinct although closely related languages, is an open question. \ili{Manding} varieties share a root for ‘see’ found as \textit{yé} or \textit{jé}, depending on the individual varieties, and a root for ‘look’ found as \textit{félé}, \textit{f\'{ɛ}l\'{ɛ}}, or very similar forms. As illustrated in \tabref{tab:creissels:1}, the use of the \isi{imperative} of ‘look’ as an \isi{ostensive} \isi{predicator} is pervasive across \ili{Manding} varieties, and most of them have a similar use of the \isi{imperative} of ‘see’. As will be discussed in \sectref{sec:creissels:6.1.2}, in several \ili{Manding} varieties, \textit{félé} {\textasciitilde} \textit{f\'{ɛ}l\'{ɛ}} seems to be involved in an incipient \isi{grammaticalization process} that could lead to the emergence of a new \isi{copula}, but in all the \ili{Manding} varieties for which I have the relevant data, copula-like uses of \textit{félé} {\textasciitilde} \textit{f\'{ɛ}l\'{ɛ}} are only sporadic. As regards \textit{yé} {\textasciitilde} \textit{jé} ‘see’, there are \ili{Manding} varieties (for example, Sédhiou \ili{Mandinka}) in which no grammaticalized use of this verb can be found, but most \ili{Manding} varieties use \textit{yé} {\textasciitilde} \textit{jé} either as a locational \isi{copula} (and \isi{incompletive} \isi{auxiliary}), as an \isi{equative copula}, or both.


\begin{table}
 \caption{Grammaticalized uses of ‘see’ and ‘look’ in four Manding varieties}
\label{tab:creissels:1}
\begin{tabularx}{\textwidth}{lXXXX}
\lsptoprule
& Sédhiou  \ili{Mandinka}  & Dantila  \ili{Maninka} & Bamako  Bambara  & Kita \ili{Maninka}\\
\midrule
‘see’& \textit{jé} & \textit{jé} & \textit{yé} &  \textit{yé}\\
‘look’  &  \textit{félé} & \textit{félé} &  \textit{fl\'{ɛ}} & \textit{félé}\\
\isi{ostensive} \isi{predicator} &   \textit{félé} &  \textit{félé/jé} & \textit{fl\'{ɛ} / yé} &  \textit{félé / yé}\\
\isi{equative copula}   &  \textit{mú} &  \textit{mú} & \textit{dòn / yé} & \textit{lè / yé}\\
locational \isi{copula}  &  \textit{bé} &  \textit{bé/jé} & \textit{b\'{ɛ}} & \textit{yé}\\
\isi{incompletive} \isi{auxiliary}  &  \textit{bé} &  \textit{bé/jé} & \textit{b\'{ɛ}} & \textit{yé}\\
\lspbottomrule
\end{tabularx}
\end{table}


Kita \ili{Maninka} illustrates the case of a \ili{Manding} variety with the maximum range of grammaticalized uses of \textit{yé} {\textasciitilde} \textit{jé} ‘see’. Note that, in \REF{ex:creissels:6}, the notation of tone and nasality is phonetic, and only tones contrasting with the tone of the preceding syllable are explicitly noted, which means that \textit{yé} may be transcribed as \textit{yè}, \textit{ɲé}, \textit{ye}, etc. depending on the context.

\ea
    \label{ex:creissels:6}
   \ili{Kita Maninka} \citep[19, 78, 79, 87, 88]{Creissels2009}\\  
   \ea
    \gll   Sékù  dí  tùbabu  náni  ye  kunùn.\\
      Sékou  \textsc{cpl}  European  four  see  yesterday\\
      \glt {‘Sékou saw four Europeans yesterday.’}\\
   \ex
    \gll   Móngon  ɲè!\\
      mango.\textsc{d}  \textsc{ost}\\
      \glt {‘Here is a mango!’}\\
   \ex
  \gll {Nénè}  {yé}  {Kìta.}\\
           cold.\textsc{d}  \textsc{cop}  Kita\\
      \glt {‘It is cold in Kita.’}\\
   \ex
    \gll   Kóngò  ye  n  na.\\
      hunger.\textsc{d}  \textsc{cop}  \textsc{1sg}   \textsc{postp} \\
      \glt {‘I am hungry.’ (lit. ‘Hunger is in me’)}\\
   \ex
  \gll \textit{Nònilí}  \textit{ye̍}  \textit{ku-jogu̍}  \textit{lè}  \textit{di.}\\
        insult.\textsc{d}  \textsc{cop}  thing-bad.\textsc{d}  \textsc{foc}   \textsc{postp} \\
      \glt {‘Insult is a bad thing.’}\\
  \ex 
  \gll \textit{Músa}  \textit{ye}  \textit{ɲo̍}  \textit{sène-la.}\\
   Moussa  \textsc{cop}  millet.\textsc{d}  cultivate-\textsc{inf}\\   
      \glt {‘Moussa cultivates millet.’}\\ 
   \ex
    \gll   Sán  nà-dó  yè.\\
      rain.\textsc{d}  come-\textsc{ger}  \textsc{cop}\\
      \glt {‘Rain is coming.’}
\z
\z


\citet{Heydorn1949-1950} describes a similar situation in \ili{Manya} (a \ili{Manding} variety spoken in Liberia), and explicitly states that “Wie im \ili{Bandi} und verwandten Sprachen ein deutlicher Zusammenhang zwischen ‘sein’ und ‘sehen’ besteht, so scheint dies auch im \ili{Manya}, wo ‘sehen’ \textit{ye̦} heisst, der Fall zu sein.” ("In \ili{Bandi} and related languages there is a clear relationship between ‘be’ and ‘see’, and apparently the same is true for \ili{Manya}, where ‘see’ is \textit{ye}̣".) \citep[57]{Heydorn1949-1950}.

\subsection{The locative copula and the verb ‘see’ in Soninke}\label{sec:creissels:4.3}

The resemblance between the \ili{Soninke} verb \textit{wàrí} (or \textit{ŋàrí}, \textit{ŋèrí}) and the locational \isi{copula} \textit{wá} (also used in \isi{incompletive} \isi{auxiliary function}) is not very great, and might be due to mere chance. However, evidence of a possible etymological link is provided by the data of Azer, a now-extinct \ili{Soninke} variety. \citet[42--44]{Monteil1939} mentions the existence of variants of the locational \isi{copula}/\isi{incompletive} \isi{auxiliary} such as \textit{wari}, \textit{war}, \textit{wri}, and explicitly states that he considers this \isi{copula}/\isi{auxiliary} as a grammaticalized form of ‘see’.

\section{Comparison with the grammaticalization of ‘see’ in Arabic}\label{sec:creissels:5}

In this section, I summarize the data on the grammaticalization of ‘see’ in Arabic languages that have been presented and analyzed in detail by \citet{Taine-Cheikh2013}, emphasizing the commonalities and differences with \ili{Mande} languages that are directly relevant to the topic of this article.{}\footnote{In addition to Catherine \citeauthor{Taine-Cheikh2013}’s (2013) article, this paper has benefited from the discussions I had with her about the Arabic data analyzed in her article, and I want to express my gratitude to her.}
An important specificity that distinguishes the \isi{predicative} system of most Arabic varieties from that of most \ili{Mande} languages is the systematic use of \isi{equative} or locational \isi{predicative} constructions including no overt predicators –  \REF{ex:creissels:7}.\newpage

\ea
    \label{ex:creissels:7}
   \ili{Classical Arabic}\\
   \ea
    \gll   al-waladu  ṣa\.gīru-n.\\
       \textsc{def}-boy  small-\textsc{indef}\\
      \glt {‘The boy is small.’}\\
   \ex
    \gll   al-waladu  fī  l-madrasat-i.\\
       \textsc{def}-boy  in  \textsc{def}-school-\textsc{gen}\\
      \glt {‘The boy is at school.’}
\z
\z


The grammaticalization of \textit{raʔā} ‘see’, in particular in its \isi{imperative} form, is a very common phenomenon across Arabic varieties. This verb is preserved in literary Arabic, but in most modern Arabic varieties, only grammaticalized forms of \textit{raʔā} have subsisted, and the verb most commonly used in the sense of ‘see’ is \textit{šāf}. A detailed analysis can be found in \citet{Taine-Cheikh2013}. In the present article, I concentrate on the aspects that are directly relevant to the current discussion.
  Plain \isi{ostensive} predicators cognate with \textit{raʔā} are not very common across Arabic varieties. However, Ḥassāniyya (the variety spoken in Mauritania) and a few other varieties illustrate this possibility –  \REF{ex:creissels:8}\ and \REF{ex:creissels:9}.\footnote{Interestingly, an \isi{ostensive} \isi{predicator} \textit{ša} originating from \textit{šāf} (the verb most commonly used in the sense of ‘see’ in modern Arabic varieties) is attested in Syrian Arabic  \citep[115]{Stowasser1964}.}

  \ea
    \label{ex:creissels:8}    
   \langinfo{Ḥassāniyya}{}{\citealt{Taine-Cheikh2013}}\\   
  \gll ṛâˤi  xṛûv!  \\
   {\textsc{ost}}  {lamb}  \\
  \glt {{‘Here is a lamb!’}}\\ 
\z


\ea
    \label{ex:creissels:9}    
   \langinfo{Yâfiˤ, Yemen}{}{\citealt[336--337]{Vanhove2010}}\\   
  \gll raˤ ar-rābˤeh!  \\
   \textsc{ost}  \textsc{def}-jug  \\
  \glt {‘Here is the jug!’}
\z

Particles expressing not only simultaneity (‘right now’), but also various \isi{modal} or discursive values derivable from an original \isi{ostensive} meaning, constitute the commonest outcome of the grammaticalization of \textit{raʔā} across Arabic varieties. Their contribution to the meaning of the clause can be variously rendered in English as ‘indeed’, ‘really’, ‘certainly’, ‘don’t you see that...?’, ‘and then’, ‘this is a fact’, ‘you must know that...’, ‘I remind you that...’ etc. To the best of my knowledge, this \isi{grammaticalization path} has no equivalent in \ili{Mande} languages.
Equative or locational clauses including an element whose etymon is the \isi{imperative} of \textit{raʔā} ‘see’ are common across Arabic varieties. However, in most cases, this element is syntactically optional, and its presence implies a marked \isi{modal} or discursive value, as illustrated by \REF{ex:creissels:10} for a variety from the South of the Arabic Peninsula. Crucially, in such cases, this element can be added with the same value to verbal clauses. Consequently, it would not be correct to identify it as a \isi{copula}. Although this is not easy to reflect in the translation of isolated examples, it is clear from the comments in the original source that, in this Arabic variety, \textit{raˁ} cannot be analyzed as an integral part of a particular type of \isi{predicative} construction, and is rather an optional particle used to emphasize a precise fact or a sudden appearance and to express the reason behind something, or the consequences of an event.

  \ea
    \label{ex:creissels:10}
\langinfo{Datînah Arabic}{}{\citealt[485, 486, 488]{Landberg1909}}\\  
\ea
    \gll   raˁ=nī  ˁawaḍ.\\
       \textit{raˁ}=\textsc{1sg}  ˁawaḍ\\
      \glt {(‘Je suis ˁAwaḍ, moi.’)
‘Me, I am ˁAwaḍ.’}\\
   \ex
    \gll   raˁ=ak  fi  arḍ  ˁöleh.\\
       {raˁ}=\textsc{2sg}  in  country  ˁOlah\\
      \glt {(‘[...] c’est que tu [es] dans le pays des ˁOlah.’)
‘That’s because you [are] in the country of the Olah.’}\\ 
   \ex
    \gll   raˁ  em=maṭar  y-ehđil.\\
       {raˁ}  \textsc{def}=rain  \textsc{3m.incpl}-drizzle.\textsc{sg}\\
      \glt {(‘Voilà que la pluie tombe fine.’)
‘There goes the rain drizzling.’}\\
\z
\z


A \isi{plain copula} originating from the \isi{imperative} of \textit{raʔā} can only be found in Algerian Arabic, and more precisely in the variety spoken in Algiers. This was already observed by \citet[252]{Cohen1912}, and \citet{Boucherit2002} confirms that, in the \isi{equative} and locational clauses of Algiers Arabic, \textit{ra} does not express the values carried by its cognates in most other Arabic varieties, and can be analyzed as the suppletive present form of a \isi{copula} whose past form is \textit{kān.}

\ea%
    \label{ex:creissels:11}
\langinfo{Algiers Arabic}{}{\citealt[62]{Boucherit2002}}\\
 \gll    ra=ni  fi=l=kuzina.\\
   \textsc{cop}=\textsc{1sg}  in=\textsc{def}=kitchen\\
  \glt {  (‘Je suis dans la cuisine.’)
  ‘I am in the kitchen.’}
\z


\section{From the imperative of ‘see\slash look’ verbs to copulas}\label{sec:creissels:6}

\subsection{The grammaticalization path SEE\slash LOOK\textsubscript{imper} > OSTENSIVE PREDICATOR > COPULA}\label{sec:creissels:6.1} 

This \isi{grammaticalization path} is strongly suggested by the \ili{Mande} data, since the creation of \isi{ostensive} predicators from the \isi{imperative} of ‘see’ or ‘look’ is very common in \ili{Mande} languages, and in the \ili{Mande} languages that have copulas cognate with a verb ‘see’, the same form is used as an \isi{ostensive} \isi{predicator}, and has no other use that could constitute an intermediate stage in this \isi{grammaticalization path}. 

\subsubsection{SEE\slash LOOK\textsubscript{imper} > OSTENSIVE PREDICATOR}\label{sec:creissels:6.1.1} 

As already mentioned in \sectref{sec:creissels:1}, the grammaticalization of the \isi{imperative} of the ‘see’ or ‘look’ verbs as \isi{ostensive} predicators is common cross-linguistically. 
  The creation of \isi{ostensive} predicators from the \isi{imperative} of ‘see’ or ‘look’ boils down to the routinization of an \isi{ostensive} use of the \isi{imperative} of ‘see’ or ‘look’. In this use, \textit{See/look at X!} is not interpreted in its literal meaning of an invitation to see\slash look at the referent of X, but as expressing awareness of the presence of the referent of X in the speech situation. Since uttering \textit{See X!} or \textit{Look at X!} in their literal meaning entails the presence of the referent of X, the routinization of the \isi{ostensive} use of the \isi{imperative} of a ‘see\slash look’ verb can be viewed as the semanticization of a pragmatic entailment. 
  At an early stage of the evolution, there is no formal manifestation of the development of an \isi{ostensive} reading of the \isi{imperative} of ‘see’ or ‘look’, but subsequent changes may introduce formal \isi{distinctions}. For example, in French, it is obvious that \textit{Me voici!} comes from a construction whose equivalent in Modern French would be \textit{Vois-moi ici!}, but synchronically, the position of the object index (which in Modern French cannot precede the verb in the \isi{imperative} positive), and the \isi{coalescence} of \textit{vois + ici} into \textit{voici}, distinguish the \isi{ostensive} \isi{predicator} from the \isi{imperative} of ‘see’. However, the persistence of the ambiguity is possible too. For example, in \ili{Mandinka} (and other \ili{Manding} varieties), \textit{\'{Ŋ} félé} {\textbar}\textsc{1sg} look{\textbar} is ambiguous between its literal meaning ‘Look at me!’ and the \isi{ostensive} reading ‘Here I am!’.
  
\subsubsection{OSTENSIVE PREDICATOR > COPULA}\label{sec:creissels:6.1.2} 

Ostensive predicators entail meanings typically expressed by copulas: identification of a referent, and presence of a referent at some place. They differ from copulas in two crucial respects: 

\begin{itemize}
 \item[(a)] the argument of an \isi{ostensive} \isi{predicator} must be located in the speech situation;
\item[(b)] \isi{ostensive} clauses express a particular type of speech act (drawing the addressee’s attention to an obvious fact) distinct from plain assertion, and consequently do not lend themselves to operations such as negation, questioning, or relativization. 
\end{itemize}

Consequently, the relaxation of these constraints, manifesting the loss of the \isi{deictic} component of \isi{ostensive} \isi{predication} and the reinterpretation of \isi{ostensive} clauses as plain assertive clauses, is crucial in the evolution from the status of \isi{ostensive} \isi{predicator} to that of \isi{copula}.
  
  Interestingly, in \ili{Mande} languages, in addition to copulas analyzable as a result of the grammaticalization of an \isi{ostensive} marker, it is possible to find \isi{ostensive} markers that still cannot be analyzed as having grammaticalized into copulas, but which already occur more or less sporadically in contexts implying the weakening of their \isi{deictic} component or the bleaching of their particular \isi{illocutionary force}.
  
  For example, contrary to \textit{yé {\textasciitilde} jé} ‘see’, which has become a \isi{copula} in many \ili{Manding} varieties, I am aware of no \ili{Manding} variety with a true \isi{copula} cognate with \textit{félé {\textasciitilde} f\'{ɛ}l\'{ɛ}} ‘look’. However, in Kita \ili{Maninka}, \isi{ostensive} clauses lend themselves to relativization, which implies the cancellation of the particular \isi{illocutionary force} normally carried by \textit{hélé} in its use as an \isi{ostensive} \isi{predicator} – ex. \REF{ex:creissels:12}.

  \ea%12
    \label{ex:creissels:12}    
   \langinfo{Kita Maninka}{}{\citealt[82]{Creissels2009}}\\
   \ea
    \gll   Wórì  hele!\\
       money.\textsc{d}  look\\
      \glt {literal meaning ‘Look at the money!’ (\isi{imperative}),
can also be interpreted as ‘Here is the money!’ (\isi{ostensive} reading)} 
 \ex
    \gll   Wórì  mín  hèle,  ǒ  tà!\\
       money.\textsc{d}  \textsc{rel}  look  \textsc{dem}  take\\
      \glt ‘Take the money that is here!’
lit. ‘Look\textsubscript{imper} at which money, take that!’ 
\z
\z


Note that my consultant for Kita \ili{Maninka} accepted this use of \textit{hélé} in \isi{ostensive} \isi{predicator} function, but rejected other manipulations on \isi{ostensive} clauses (for example, questioning) which would have been expected to be accepted if \isi{ostensive} \textit{hélé}{}-clauses had been fully reinterpreted as plain assertive clauses.
  Similarly in \ili{Mandinka}, as \REF{ex:creissels:13} illustrates, the sporadic occurrence of \textit{félé} in contexts incompatible with the \isi{deictic} value normally implied by \textit{félé}: this sentence was extracted from a story about a village very far from the place where the story was recorded, which means that a plain \isi{locative} \isi{copula} could substitute for such an occurrence of the \isi{ostensive} \isi{predicator} without any difference in meaning.

  \ea%13
    \label{ex:creissels:13}
\langinfo{Mandinka}{}{\citealt[158]{Creissels2013}}\\
   \gll Jálájúw-òo félé lòo-ríŋ jěe hání bǐi.\\
     jala\_tree-\textsc{d}  look  stand-\textsc{res}  there  even  today\\
  \glt {  ‘Up to the present day, a \textit{jala}{}-tree stands there.’
  (lit. ‘Look at a \textit{jala}{}-tree standing there even today!’)}\\ 
\z

Similar observations can be made about \ili{Soninke} \textit{háyí} ‘look’ and \ili{Bozo} \textit{xai} ‘see’. In \ili{Soninke}, the \isi{imperative} of \textit{háyí} is routinely used as an \isi{ostensive} \isi{predicator}, but it is also sporadically found in contexts in which its \isi{deictic} component or its special \isi{illocutionary force} cannot be maintained, which points to an incipient process whose outcome could be the creation of a new locational \isi{copula}. For example, my \ili{Soninke} consultant accepts the use of \textit{háyí} in interrogative clauses such as those in \REF{ex:creissels:14}, which force a reading of \textit{háyí} as a mere locational \isi{copula}. 

\ea%14
    \label{ex:creissels:14}   
  \ili{Soninke} (pers. doc.)    
   \ea
    \gll   À  háyí  màní  ñàa-nà?\\
       \textsc{3sg}  look  what  do-\textsc{ger}\textsuperscript{\tiny L}\\
      \glt {‘What is he doing?’ (lit. ‘Look\textsubscript{imper} at him doing what?’)} 
   \ex
    \gll   À  háyí  sòxò-nó  bà?\\
       \textsc{3sg}  look  cultivate-\textsc{ger}  \textsc{q}\\
      \glt {‘Is he cultivating?’ (lit. ‘Look\textsubscript{imper} at him cultivating?’)} 
   \ex
    \gll   Kó  háyí  sòxò-nò?\\
       who  look  cultivate-\textsc{ger}\textsuperscript{\tiny L}\\
      \glt {‘Is he cultivating?’ (lit. ‘Look\textsubscript{imper} at whom cultivating?’)} 
\z
\z


As regards \ili{Bozo}, \citeauthor{Blecke1996}’s description of \ili{Tigemaxo} suggests that, in this \ili{Bozo} variety, there is a similar relationship between the locational \isi{copula} \textit{ga} (which incidently might well be cognate with the root \textit{*káa} reconstructed for Southwestern \ili{Mande}) and \textit{xai} ‘see’. \citet[206 et seq.]{Blecke1996} not only mentions an \isi{ostensive} use of the \isi{imperative} of \textit{xai} ‘see’, he also repeatedly insists on the possibility of substituting \textit{xai} for the locational \isi{copula} \textit{ga}.

\subsection{The grammaticalization path SEE\slash LOOK\textsubscript{imper} > MODAL\slash DISCURSIVE PARTICLE > COPULA}\label{sec:creissels:6.2} 

The \isi{grammaticalization path} discussed in \sectref{sec:creissels:6.1} is consistent with the \ili{Mande} data, but it does not provide a satisfactory explanation of the Arabic data, since plain \isi{ostensive} predicators cognate with \textit{raʔā} ‘see’ are not very common in Arabic. Given the pervasiveness of \isi{modal} or discursive particles cognate with \textit{raʔā}, it seems more plausible that the \isi{copula} \textit{ra} found in Algerian Arabic results from the reanalysis of such a particle in \isi{equative} and locational constructions that initially included no overt \isi{predicator}.

  Across Arabic varieties, irrespective of the presence of an \isi{ostensive} \isi{predicator} cognate with \textit{raʔā}, \isi{modal} or discursive particles cognate with \textit{raʔā} can be added to \isi{equative} or locational clauses including no overt \isi{predicator} exactly in the same way as they are added to verbal clauses, with the same semantic implications, as already illustrated by Ex. \REF{ex:creissels:10} above. Ex. \REF{ex:creissels:15} provides an additional illustration.
  
\ea%15
    \label{ex:creissels:15}
     \langinfo{Ḥassāniyya}{}{\citealt{Taine-Cheikh2013}}\\
   \gll \textrm{(}a\textrm{)}ṛâ=ni merîḍ.\\
   {(a)ṛā}=\textsc{1sg}  sick\textsc{.m.sg}\\
  \glt ‘I remind you that..., remember that I am sick.’\\ 
\z

The use of \textit{ra} as a \isi{plain copula} in Algerian Arabic (illustrated by Ex. \REF{ex:creissels:11} above) is most probably due to a process of semantic bleaching that affected \textit{ra} in a construction originally similar to \REF{ex:creissels:15}, leading to its reanalysis as a \isi{plain copula}. This hypothesis is supported by the fact that, in Algerian Arabic, \textit{ra} occurs in \isi{equative} and locational clauses without any particular semantic or discursive implication, but is still found in verbal clauses with values similar to those found in other Arabic varieties.

\section{An alternative grammaticalization path from ‘see’ verbs to copulas}\label{sec:creissels:7} 

This discussion of the grammaticalization of ‘see\slash look’ verbs into copulas would not be complete if another possible \isi{grammaticalization path} from ‘see’ verbs to copulas were not mentioned and confronted with the \ili{Mande} data. The point is that, cross-linguistically, as already mentioned in the introduction, the translation equivalents of ‘see’ may be polysemous verbs expressing the meanings commonly expressed in English as ‘find’ or ‘get’. This means that some forms of such verbs may be found with meanings such as ‘is found’ or ‘is available’, i.e. meanings very close to those typically expressed by locational copulas. Consequently, ‘see’ verbs and copulas can be diachronically related in at least three different ways:

\begin{itemize}
 \item[(a)] SEE\textsubscript{imper} /LOOK\textsubscript{imper} > OSTENSIVE PREDICATOR > COPULA
 \item[(b)] SEE\textsubscript{imper} /LOOK\textsubscript{imper} > MODAL/DISCURSIVE PARTICLE > COPULA
 \item[(c)] (IS\_SEEN{\textasciitilde})IS\_FOUND > LOCATIONAL COPULA
\end{itemize}

Given the rich verbal morphology of Arabic, and in particular the clear-cut distinction between subject and object indexation, there can be no doubt that the grammaticalized uses of ‘see’ described by \citet{Taine-Cheikh2013} developed from the \isi{imperative} form of this verb. For example, in Algiers Arabic, the subject of the suppletive form of the \isi{copula} resulting from the grammaticalization of the \isi{imperative} of ‘see’ is indexed by the suffixes used in canonical verbal \isi{predication} to index objects, which supports the hypothesis that the subject of the present form of the \isi{copula} is a former object that has been reanalyzed. By contrast, for \ili{Mande} languages, it is necessary to discuss the evidence supporting the hypothesis that, as assumed in the previous sections, copulas cognate with ‘see’ verbs in \ili{Mande} languages were created according to path (a) rather than path (c).

  In the case of Southwestern \ili{Mande} languages (see \sectref{sec:creissels:4.1}), conclusive evidence can be found in the systems of consonant alternations affecting the initials of nouns and verbs. The point is that, in Southwestern \ili{Mande} languages, the boundary between object NPs and verbs in \isi{transitive} \isi{predication} is characterized by sandhi phenomena that do not occur at the boundary between subject NPs and verbs in \isi{intransitive} \isi{predication}. Consequently, if a \isi{copula} had been created according to path (c), its subject would have been already a subject in the \isi{source construction} involving a verb ‘be found’, and it would therefore be expected to behave as a normal subject with respect to its interaction with the initial consonant of the verb. By contrast, if a \isi{copula} has been created according to path (a), its subject is historically a reanalyzed object. Consequently, the subject of a \isi{copula} created according to path (a) can be expected to retain the type of interaction with the initial of the verb which normally characterizes objects, and this is precisely what can be observed.
  
  For example, in Kpelle, in \isi{intransitive} constructions in which the verb is immediately preceded by its subject, the third person singular pronoun is realized as a distinct segment, and the initial consonant of the verb does not change – ex. (\ref{ex:creissels:16}a--b). In \isi{transitive} constructions, with the object NP in immediate preverbal position, the third person singular object manifests itself by a change in the initial consonant (and the tone) of the verb – ex. (\ref{ex:creissels:16}c--f), and the same phenomenon is observed with the third person singular subject of the \isi{copula} cognate with ‘see’ – ex. (\ref{ex:creissels:16}g--h).
  
\ea%16
    \label{ex:creissels:16}
    
   \langinfo{Kpelle}{}{\citealt[4, 11, 21]{Westermann1930}}  
\ea
    \gll   Kú  pâ.\\
  \textsc{1pl}  come\\
  \glt {‘We came.’}
\ex
    \gll   È  pà.\\
  \textsc{3sg}  come\\
  \glt ‘He, she, it came.’ 
\ex
    \gll   Loa  tíe!\\
  hole  dig\\
  \glt {‘Dig a hole!’} 
\ex
    \gll   Díe!\\
  \textsc{3sg}.dig\\
  \glt ‘Dig it!’
\ex
    \gll   Dì  kú  kâ.\\
  \textsc{3pl}  \textsc{1pl}  see\\
  \glt {‘They saw us.’}
\ex
    \gll   Dí  gà.\\
  \textsc{3pl}  \textsc{3sg}.see\\
  \glt {‘They saw him.’} 
\ex
    \gll   Kú  ká  bɛ.\\
  \textsc{1pl}  \textsc{cop}  here\\
  \glt {‘We are here.’} 
\ex
    \gll   Gà  bɛ.\\
  \textsc{3sg}.\textsc{cop}  here\\
  \glt {‘He is here.’}
\z
\z


This is certainly why Westermann, who was the first to mention the \isi{imperative} of ‘see’ as a plausible origin of copulas and \isi{incompletive} auxiliaries in \ili{Mande} languages, did not hesitate in putting forward this analysis of the Kpelle data.
  Things are less straightforward in \ili{Manding}, since in \ili{Manding} languages, the distinction between subjects and objects has no morphological correlate. However, evidence supporting the choice of path (a) can be found in \ili{Manding}, too.
  A first observation is that, in \ili{Manding}, the \isi{imperative} positive is the only \isi{tense} in which the verb does not combine with an overt TAM marker (either suffixed to the verb or immediately following the subject). Consequently, the fact that the \ili{Manding} copulas cognate with ‘see’ show no trace of TAM marking supports the hypothesis that they originate from an \isi{imperative} form.
  A second observation is that there is no reason why a \isi{copula} resulting from the semantic bleaching of a verb ‘be found, be available’ should not have a negative form created in a parallel way from the negative form of the same verb. By contrast, the meaning carried by \isi{ostensive} predicators makes them incompatible with negation. Consequently, the fact that no negative \isi{copula} cognate with ‘see’ is found in \ili{Manding} supports the hypothesis that the \ili{Manding} copulas cognate with ‘see’ were created according to path (a).
  
\section{Conclusion}\label{sec:creissels:8} 

In this article, I have tried to show that, in the \ili{Mande} \isi{language family}, clear cases of copulas cognate with ‘see\slash look’ verbs are found at least in Southwestern \ili{Mande} languages and in the \ili{Manding} dialect cluster, and I have discussed evidence that they did not develop from the semantic bleaching of forms originally meaning ‘is seen\slash found’, but from the routinization of the \isi{ostensive} use of the \isi{imperative} of ‘see\slash look’. By comparing the \ili{Mande} data with the Arabic data provided by \citet{Taine-Cheikh2013}, I have tried to show that this is however not the only possibility for imperatives of verbs ‘see\slash look’ to grammaticalize into copulas. In the Arabic varieties in which the \isi{imperative} form of ‘see’ has become a \isi{plain copula}, the most plausible explanation is that a \isi{modal}/\isi{discursive particle} resulting from the grammaticalization of the \isi{imperative} of ‘see’ has undergone a process of semantic bleaching in the context of an \isi{equative} or locational \isi{predicative} construction that initially included no overt \isi{predicator}. 
\largerpage[3]

\section*{Abbreviations}
\begin{tabularx}{.55\textwidth}{@{}>{\scshape}lQ}
comp & complementizer\\
 cop & {copula}\\
 cpl & {completive}\\
 d & default determiner\\
 dem & {demonstrative}\\
 eqcop & {equative copula}\\
 foc & focus marker\\
 gen & genitive\\
 ger & gerundive\\
 indef & indefinite\\
 inf & {infinitive}\\
 neg & negative\\
 l & replacive morphotoneme ‘low’\\
\end{tabularx} 
\begin{tabularx}{.45\textwidth}{>{\scshape}lX@{}}
 loccop & locational {copula}\\
 m & masculine\\
 ost & {ostensive} {predicator}\\
 pl & plural\\
 postp & multipurpose postposition\\
 prog & progressive\\
 q & interrogative particle\\
 refl & reflexive\\
 rel & relativizer\\
 res & resultative\\
 sg & singular\\
 tr & {transitivity} marker\\
\end{tabularx} 
 
{\sloppy
\printbibliography[heading=subbibliography,notkeyword=this]
}
\end{document}