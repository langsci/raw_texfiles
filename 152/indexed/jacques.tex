\documentclass[output=paper]{langsci/langscibook}  
\author{%
Guillaume Jacques\affiliation{CNRS-CRLAO-INALCO-EHESS (Paris)}
}
\ChapterDOI{10.5281/zenodo.823226} %will be filled in at production
 \title{The origin of comitative adverbs in Japhug}

\abstract{The aim of this paper is threefold. First, it provides a description of the morphological and syntactic properties of comitative adverbs in Japhug and other Gyalrong languages, a class of adverbs derived from nouns by a combination of prefixation and reduplication. Second, it argues that they result from a two-step derivation, first from noun into proprietive denominal verb, then from that verb into a participial form. The resulting form is later reanalyzed as a single morphological derivation from the noun. Third, this paper contributes to the study of language contact within the Gyalrongic group by showing how one of the two processes for building comitative adverbs in Japhug is borrowed from the neighbouring Tshobdun language.}

\maketitle 
  
\begin{document} 

 
  
% \textbf{Keywords}: Comitative, denominal verbalization, \isi{proprietive}, inalienable possession, nominalization, \isi{participle}s, reduplication, \ili{Japhug}

\section*{Introduction}
This paper discusses the origin of \isi{comitative} adverbs in \ili{Japhug} and other \ili{Gyalrong} languages. These adverbs, only attested in the core \ili{Gyalrong} languages, are a relatively recent innovation in these languages, and provide an interesting case study to investigate the origin of \isi{comitative} constructions in the world's languages.

The paper contains four sections.  First, I provide background information on \ili{Japhug} and the other \ili{Gyalrongic} languages. Second, I describe the morphological expression of possession in \ili{Japhug} nouns, which must be taken into consideration in all types of denominal derivations, including that of \isi{comitative} adverbs. Third, I discuss the morphological and syntactic properties of \isi{comitative} adverbs. Fourth, I propose a grammaticalization hypothesis to account for their origin, involving comparison with the closely related \ili{Tshobdun} language, and show that the pathway in question has not previously been proposed for \isi{comitative} markers.

 \section{Japhug and Gyalrongic languages} 
\ili{Japhug} (in \ili{Chinese} Chapu \zh{茶堡}) is a \ili{Gyalrong} language spoken in Mbarkham county, Rngaba prefecture, Sichuan, China. The present study is based on the \ili{Kamnyu} dialect, whose location is indicated in Figure \ref{fig:kamnyu}. In addition to \ili{Japhug}, there are three other \ili{Gyalrong} languages, \ili{Tshobdun} (in \ili{Chinese} Caodeng \zh{草等}, \citealt{jackson03caodeng}), Zbu (aka \ili{Showu}, in \ili{Chinese} Ribu \zh{日部}, see \citealt{jackson04showu, gongxun14agreement}) and Situ, the language with the greatest number of speakers and dialectal variation (\zh{四土}, \citealt{linxr93jiarongen, huangsun02, prins11kyomkyo}). The \ili{Gyalrong} languages in turn belong to the \ili{Gyalrongic} branch of Trans-Himalayan, which also includes Stau and \ili{Khroskyabs} (see \citealt{jackson00puxi} and \citealt{lai15person}).

\begin{figure}
\caption{Location of Kamnyu village}
\label{fig:kamnyu}
\includegraphics{figures/mapKamnyu2.pdf}
\end{figure}

The \ili{Gyalrong} languages, unlike most other languages of the Trans-Himalayan family, are polysynthetic languages with a rich derivational and \isi{inflectional} verbal morphology (\citealt{jacques12incorp, jackson14morpho}) and direct-inverse indexation (\citealt{delancey81direction, jackson02rentongdengdi, jacques10inverse, gongxun14agreement}), which are argued to be of proto-Trans-Himalayan origin (\citealt{delancey10agreement, jacques12agreement}). This morphology is typologically unusual in being mainly prefixing despite \ili{Gyalrong} languages having strict verb-final \isi{word order} (\citealt{jacques13harmonization}).

 \section{Inalienably possessed nouns} 
\ili{Japhug} nouns can be divided into inalienably possessed nouns (IPN) and non-inalienably possessed nouns (NIPN). IPNs differ from NIPNs in that they require the presence of one of the \isi{possessive} prefixes (Table \ref{tab:possessive}), while NIPNs can appear as their bare stem without any \isi{possessive} prefix. The IPN / NIPN distinction is not completely predictable: although all body parts and kinship terms are IPNs, we also find nouns referring to (but not all) clothes (\ipa{tɯ-ŋga} `clothes', \ipa{tɯ-xtsa} `shoes', etc), some implements (\ipa{tɤ-mkɯm} `pillow'), and abstract concepts (\ipa{tɯ-sɯm} `thought', \ipa{tɯ-ʑɯβ} `sleep', \ipa{tɯ-pʰɯ} `price', \ipa{tɯ-nŋa} `debt', \ipa{tɯ-kʰɯr} `official position', etc). Note that IPNs can refer to entities or properties that are not necessarily permanently and definitively associated with the \isi{possessor}, as is the case with clothes and concepts like `debt' or `official position', but that are not freely removable at least during a period of time (the time of being awake in the case of `clothes', the time of sleeping in the case of `pillow', the period between contracting the debt and repaying it in the case of `debt', etc).

When no definite \isi{possessor} is present, IPNs take one of the indefinite \isi{possessive} prefixes \ipa{tɤ--} or \ipa{tɯ--}. The citation form of IPNs is built by combining one of the indefinite prefixes with the noun stem (\ipa{tɤ-lu} `milk', \ipa{tɯ-ŋga} `clothes', \ipa{tɤ-rpɯ} `uncle', \ipa{tɯ-ci} `water'). The distribution of the prefixes \ipa{tɤ--} vs \ipa{tɯ--} is lexically determined.  When a specific \isi{possessor} is present, the indefinite prefix is replaced by the appropriate \isi{possessive} prefix (\ipa{ɯ-lu} `her/its milk (from her nipple)', \ipa{a-ŋga} `my clothes', \ipa{nɤ-rpɯ} `your uncle', \ipa{ɯ-ci} `its juice'). 

Although the generic \isi{possessive} prefix \ipa{tɯ--} is homophonous with one of the indefinite \isi{possessive} prefixes, the two are semantically distinct (compare \ipa{tɤ-se} \textsc{indef.poss}-blood `blood' with \ipa{tɯ-se} \textsc{genr.poss}-blood `one's/people's blood').

\begin{table} 
\caption{Possessive prefixes }\label{tab:possessive}
\begin{tabular}{lllllllll} 
\lsptoprule
 Prefix & Person\\
\midrule
\ipa{a--}  & \textsc{1sg} \\
\ipa{nɤ--}  & \textsc{2sg}\\
\ipa{ɯ--}  & \textsc{3sg}\\
\midrule
\ipa{tɕi--}  &  \textsc{1du} \\
\ipa{ndʑi--}  & \textsc{2du} \\	
\ipa{ndʑi--}  & \textsc{3du} \\	
\midrule
\ipa{i--}  & \textsc{1pl} \\
\ipa{nɯ--}  & \textsc{2pl} \\
\ipa{nɯ--}  & \textsc{3pl} \\
\midrule
\ipa{tɯ--},  \ipa{tɤ--} & indefinite \\
\ipa{tɯ--}   &  generic \\
\lspbottomrule
\end{tabular}
\end{table}

It is possible to turn an IPN into a NIPN by prefixing a definite \isi{possessive} prefix to the indefinite one, as in \ipa{ɯ-tɤ-lu} \textsc{3sg.poss-indef.poss}-milk `his milk (to drink)', \ipa{ɯ-tɯ-ci} \textsc{3sg.poss-indef.poss}-water `its water (of irrigated water, to a plant)'.\footnote{A similar phenomenon is reported in \ili{Tshobdun} (\citealt[140]{jackson98morphology})} NIPNs cannot take indefinite \isi{possessive} prefixes. However, they are compatible with the human generic \isi{possessor prefix} \ipa{tɯ--}, as in \REF{ex:tWlaXtCha}, where the nouns \ipa{kha} `house' and \ipa{laχtɕha} `thing' are NIPNs.

\noindent\parbox{\textwidth}{\begin{exe}
\ex \label{ex:tWlaXtCha}
\gll  \ipa{wuma}  	\ipa{ʑo}  	\ipa{tɯ-kha}  	\ipa{cho}  	\ipa{tɯ-laχtɕha}  	\ipa{ra}  	\ipa{sɯ-ɴqhi}.  \\
really \textsc{emph} \textsc{genr.poss}-house and \textsc{genr.poss}-thing \textsc{pl} \textsc{caus}-be.dirty:\textsc{fact} \\
\glt `(Flies) make one's house and one's things dirty.' (25 akWzgumba, 62)
\end{exe}}


 
 \section{Comitative derivation} 
In \ili{Japhug}, adverbs meaning `having X' or `together with X' can be productively built from various types of nouns.\footnote{Comitative adverbs in \ili{Japhug} have been briefly mentioned in a previous publication (\citealt[51]{jacques08}), but this paper is the first detailed description of this \isi{derivation} and its uses.} In this section, I first describe the morphological processes involved in the \isi{derivation} from noun to \isi{adverb}, and then provide an overview of the use of these adverbs in context.

\subsection{Morphology}
Comitative adverbs are formed by reduplicating the last syllable of the noun stem and prefixing either \ipa{kɤ́--} or \ipa{kɤɣɯ--}, as in examples such as \ipa{χɕɤlmɯɣ} `glasses' $\Rightarrow$ \ipa{kɤ́-χɕɤlmɯ\tld{}lmɯɣ} / \ipa{kɤɣɯ-χɕɤlmɯ\tld{}lmɯɣ} `together with glasses'.\footnote{\ili{Japhug} \ipa{χɕɤlmɯɣ} `glasses' is a loanword from \ili{Tibetan} \ipa{ɕel.mig}; note that reduplication disregards morpheme boundaries (\ipa{χɕɤl} `glass' (\ili{Tibetan} \ipa{ɕel}) is also attested in \ili{Japhug}). } No semantic difference between the \isi{comitative} adverbs in \ipa{kɤ́--} and those in \ipa{kɤɣɯ--} has been detected; both are fully productive and can be built from the same nouns.

When the base noun is an IPN, it is possible to build a \isi{comitative adverb} with the indefinite \isi{possessor prefix} or with the bare stem. For instance, from \ipa{tɤ-rte} `hat' one can derive both \ipa{kɤ́-rtɯ\tld{}rte} / \ipa{kɤɣɯ-rtɯ\tld{}rte} `with his/her hat' and \ipa{kɤ́-tɤ-rtɯ\tld{}rte} /  \ipa{kɤɣɯ-tɤ-rtɯ\tld{}rte} `with a hat' the latter bearing the indefinite \isi{possessor prefix} \ipa{tɤ--}. The inalienable/non-alienable distinction is present in these forms: \ipa{kɤ́-rtɯ\tld{}rte} / \ipa{kɤɣɯ-rtɯ\tld{}rte} means `wearing one's hat'  \REF{ex:kAGWrtWrte}, while \ipa{kɤ́-tɤ-rtɯ\tld{}rte} /  \ipa{kɤɣɯ-tɤ-rtɯ\tld{}rte} implies that one is not wearing the hat (\ref{ex:kAGWtArtWrte}).



\begin{exe}
\ex \label{ex:kAGWrtWrte}
\gll \ipa{kɤɣɯ-rtɯ\tld{}rte} 	\ipa{ʑo} 	\ipa{kha} 	\ipa{ɯ-ŋgɯ} 	\ipa{lɤ-tɯ-ɣe} 	\\
\textsc{comit}-hat \textsc{emph} house \textsc{3sg}-inside \textsc{pfv}-\textsc{2}-come[II] \\
\glt `You came inside the house with your hat (on).' (You were expected to take it off before coming in)
\end{exe}

\begin{exe}
\ex \label{ex:kAGWtArtWrte}
\gll  \ipa{laχtɕha} 	\ipa{kɤɣɯ-tɤ-rtɯ\tld{}rte} 	\ipa{ʑo} 	\ipa{ta-ndo}  \\
thing \textsc{comit-indef.poss}-hat \textsc{emph} \textsc{pfv}:\textsc{3}→\textsc{3}'-take \\
\glt `He took the things together with the hat.' (Not wearing it)
\end{exe}

Cognates of the \ili{Japhug} \isi{comitative} adverbs have been reported in other \ili{Gyalrong} languages, in particular \ili{Tshobdun} \ipa{ko--} (\citealt[107]{jackson98morphology}), and the \isi{comitative adverb} \isi{derivation} can thus be reconstructed back at least to proto-\ili{Gyalrong}. However, given the dearth of data on languages other than \ili{Japhug} (in particular in terms of text examples), little external data will be discussed in this paper. A full comparative assessment of the hypotheses laid out here will have to await the publication of fully-fledged grammatical descriptions of all \ili{Gyalrong} languages. 

Comitative adverbs, in any case, appear to be unattested outside of the core \ili{Gyalrong} languages (even in \ili{Khroskyabs}, their closest relative, see \citealt{lai13affixale}), and are probably one of the many common \ili{Gyalrong}  morphological innovations.

\subsection{Syntactic uses} 

The \isi{comitative adverb} can either follow (\ref{ex:kAjWjaR}) or precede (\ref{ex:kArnWrna}, \ref{ex:kAthAlwWlwa}) the noun over which it has scope. Alternatively, a \isi{comitative adverb} can occur without a corresponding overt noun (\ref{ex:kAsnWsno}). However, if the noun is overt, the \isi{comitative adverb} is contiguous to the NP to which it belongs. 

The NP in question can either correspond to the P (\ref{ex:kAjWjaR}, \ref{ex:kAthAlwWlwa}), the S (\ref{ex:kArnWrna}, \ref{ex:kAsnWsno}) or even the A (\ref{ex:kArJWrJit.kW}). This last option is not attested in the text corpus, but speakers have no trouble producing sentences of this type.

\begin{exe}
\ex \label{ex:kAjWjaR}
\gll
\ipa{tɤ-sno}  	\ipa{kɤ́-jɯ\tld{}jaʁ}  	\ipa{nɯ}  	\ipa{lu-ta-nɯ}  \\
\textsc{indef.poss}-saddle \textsc{comit}-hand \textsc{dem} \textsc{ipfv}-put-\textsc{pl} \\
\glt `(Then), they put the saddle with its handles.'
\end{exe}

\begin{exe}
\ex \label{ex:kArnWrna}
\gll
\ipa{pɣɤkhɯ}  	\ipa{nɯ}  	\ipa{ɯ-ku}  	\ipa{nɯnɯ}  	\ipa{lɯlu}  	\ipa{tsa}  	\ipa{ɲɯ-fse,}  	\ipa{ɯ-mtsioʁ}  	\ipa{ɣɤʑu}  	\ipa{ma}  \ipa{kɤ́-rnɯ\tld{}rna}  	\ipa{lɯlu}  	\ipa{ɯ-tɯ-fse}  	\ipa{ɲɯ-sɤre}  	\ipa{ʑo.}  \\
owl \textsc{dem} \textsc{3sg.poss}-head \textsc{dem} cat a.little \textsc{sens}-be.like \textsc{3sg.poss}-beak exist:\textsc{sens} a.part.from \textsc{comit}-ear cat \textsc{3sg.nmlz:degree}-be.like \textsc{sens}-be.extremely/be.funny \textsc{emph} \\
\glt `The owl's head looks a little like that of a cat; apart from the fact that it has a beak, it looks very much like a cat with its ears.'
\end{exe}


\begin{exe}
\ex \label{ex:kAthAlwWlwa}
\gll \ipa{kɤ́-thɤlwɯ\tld{}lwa}  	\ipa{ɯ-zrɤm}  	\ipa{ra}  	\ipa{kɯnɤ}  	\ipa{chɯ́-wɣ-ɣɯt}  	\ipa{pjɯ́-wɣ-ji}  	\ipa{ri}  	\ipa{maka}  	\ipa{tu-ɬoʁ}  	\ipa{mɯ́j-cha}  \\
\textsc{comit}-earth \textsc{3sg.poss}-root \textsc{pl} also \textsc{ipfv-inv}-bring \textsc{ipfv-inv}-plant but at.all \textsc{ipfv}-come.out \textsc{neg:sens}-can \\
\glt `Even if one takes its root with earth (around it) and plants it, it cannot grow.'
\end{exe}


\begin{exe}
\ex \label{ex:kAsnWsno}
\gll \ipa{kɤ́-snɯ\tld{}sno}  	\ipa{ʑo}  	\ipa{kɤ-rŋgɯ}  \\
\textsc{comit}-saddle \textsc{emph} \textsc{pfv}-lie.down \\
\glt `(The horse) slept with its saddle.' (elicited)
\end{exe}

\begin{exe}
\ex \label{ex:kArJWrJit.kW}
\gll 
\ipa{lɯlu} 	\ipa{kɤ́-rɟɯ\tld{}rɟit} 	\ipa{ra} 	\ipa{kɯ} 	\ipa{ʑo} 	\ipa{βʑɯ} 	\ipa{to-ndza-nɯ.} \\
cat \textsc{comit}-offspring \textsc{pl}  \textsc{erg} \textsc{emph} mouse \textsc{ifr}-eat-\textsc{pl} \\
\glt `The cat and its young ate the mouse.' (elicited)
\end{exe}

Nouns incorporated into \isi{comitative} adverbs lose their nominal status and cannot be determined by relative clauses (including attributive adjectives), numerals or \isi{demonstrative}s. In a sentence such as \REF{ex:kAGWNkhWNkhor} for instance, the attributive participial relative [\ipa{kɯ\tld{}kɯ-ŋɤn}] `all the ones who are evil' does not determine \ipa{kɤɣɯ-ŋkhɯ\tld{}ŋkhor} `with his subjects', a syntactic structure which would correspond to the translation `with all his evil subjects'. Rather, it determines the head noun together with the \isi{comitative adverb}, i.e.  \ipa{rɟɤlpu} \ipa{kɤɣɯ-ŋkhɯ\tld{}ŋkhor} `the king with his subjects', which implies the translation given below.

\begin{exe}
\ex \label{ex:kAGWNkhWNkhor}
\gll \ipa{rɟɤlpu}  	\ipa{kɤɣɯ-ŋkhɯ\tld{}ŋkhor}  	[\ipa{kɯ\tld{}kɯ-ŋɤn}]  	\ipa{ʑo}  	\ipa{to-ndo}  	\ipa{tɕe,}  	\ipa{tɕendɤre}  	\ipa{kɯ-mɤku}  	\ipa{nɯ}  	\ipa{sɤtɕha}  	\ipa{kɯ\tld{}kɯ-sɤ-scit}  	\ipa{ʑo}  	\ipa{jo-tsɯm}  	\ipa{ɲɯ-ŋu}  	\ipa{ri}  	\ipa{kɯ-maqhu}  	\ipa{tɕe,}  	\ipa{kɯ\tld{}kɯ-sɤɣ-mu}  	\ipa{ʑo}  	\ipa{jo-tsɯm}  	\ipa{tɕe}  \\
king \textsc{comit}-subjects \textsc{total}\tld{}\textsc{nmlz}:\textsc{s\slash a}-be.bad \textsc{emph} \textsc{ifr}-take \textsc{lnk}  \textsc{lnk} \textsc{nmlz}:\textsc{s\slash a}-be.before \textsc{dem} place \textsc{total}\tld{}\textsc{nmlz}:\textsc{s\slash a}-\textsc{deexp}-be.happy \textsc{emph} \textsc{ifr}-take.away \textsc{sens}-be \textsc{lnk} \textsc{nmlz}:\textsc{s\slash a}-be.after \textsc{lnk} \textsc{total}\tld{}\textsc{nmlz}:\textsc{s\slash a}-\textsc{deexp}-fear \textsc{emph} \textsc{ifr}-take.away \textsc{lnk} \\
\glt `She took the king and his subjects, all the evil ones; in the beginning she took them to nice places, but later she took them to fearful places.' (slobdpon)
\end{exe}


\section{Grammaticalization pathway} 
In this section, I first present the \isi{proprietive} denominal \isi{derivation} in \ipa{aɣɯ--} and the infinitival and participial prefix \ipa{kɯ--}. Then, I show that in fact \isi{comitative} adverbs are synchronically formally ambiguous with the \isi{infinitive} and the \textsc{s\slash a}-\isi{participle} of \isi{proprietive} denominal verbs in some contexts. Finally, I propose that \isi{comitative} adverbs derive diachronically from the participial forms of \isi{proprietive} denominal verbs, and were then extended to other contexts after reanalysis.


\subsection{Denominal derivation}
\ili{Japhug} has a rich array of denominal prefixes (\citealt{jacques14antipassive}). One of these prefixes, \ipa{aɣɯ--}, derives \isi{stative} \isi{intransitive} verbs from both inalienably possessed and non-inalienably possessed nouns. As illustrated by the examples in Table \ref{tab:aGW}, verbs derived with the prefix have meanings such as `having X', `producing a lot of X' or `having the same X ' (with plural S). The noun stem is sometimes reduplicated, especially for the first of these meanings.
 
\begin{table} 
\caption{The denominal prefix \ipa{aɣɯ--}}\label{tab:aGW}
\resizebox{\columnwidth}{!}{
\begin{tabular}{lllllllll} 
\lsptoprule
Base noun & Meaning & Denominal verb & Meaning\\
\midrule
\ipa{tɯ-ɣli} & excrement, manure & \ipa{aɣɯ-ɣli} & producing a lot of manure (of pigs) \\
\ipa{tɤ-lu} & milk &\ipa{aɣɯ-lu} & producing a lot of milk (of cows) \\
\ipa{tɯ-mɲaʁ} & eye & \ipa{aɣɯ-mɲaʁ} & having a lot of holes \\
\ipa{tɯ-ɕnaβ} & snot & \ipa{aɣɯ-ɕnɯ\tld{}ɕnaβ} & be  slimy \\
\ipa{ɯ-mdoʁ} & colour & \ipa{aɣɯ-mdoʁ} & having the same colour \\
\ipa{tɯ-sɯm} & thought & \ipa{aɣɯ-sɯm} & get along well \\
\ipa{smɤn} & medicine &  \ipa{aɣɯ-smɤn} & have a medical effect \\
\ipa{tɯ-ɕna} & nose &  \ipa{aɣɯ-ɕnɯ\tld{}ɕna} & having a keen sense of smell \\
\lspbottomrule
\end{tabular}}
\end{table}

In some cases, the semantic relationship between the base noun and the derived verb is more metaphorical and not predictable. For instance, from the noun \ipa{tɯ-jaʁ} `hand' one can derive either \ipa{aɣɯ-jɯ\tld{}jaʁ} `having a lot of hands' (of a bug), while the non-reduplicated form \ipa{aɣɯ-jaʁ} means `who steals anything (that comes near his hand)'.

\subsection{S/A participle and infinitive}

In \ili{Japhug}, \isi{stative} verbs (including the denominal verbs in \ipa{aɣɯ--} presented in the previous section) have two homophonous non-finite forms with a prefix \ipa{kɯ--}, the \textsc{s\slash a}-\isi{participle} (\citealt[5]{jacques14antipassive}) and the \isi{infinitive}.\footnote{The morphological evidence for distinguishing between \isi{participle} and \isi{infinitive} is clearer with dynamic verbs, whose \isi{infinitive} in \ipa{kɤ--} differs from the \textsc{s\slash a}-\isi{participle}. Cognates of the \isi{participle} and the \isi{infinitive} \ipa{kɯ--} are found in all \ili{Gyalrong} languages, with only minor differences (see in particular \citealt{sun14generic}).} The \isi{participle} appears mainly in participial relatives (including all forms corresponding to attributive adjectives in European languages), as in example (\ref{ex:kWpWpe}).

\begin{exe}
\ex \label{ex:kWpWpe}
\gll
\ipa{tɕheme} 	\ipa{ci} 	\ipa{kɯ-pɯ\tld{}pe} 	\ipa{kɯ-mpɕɯ\tld{}mpɕɤr,} 	\ipa{nɤ-ɕɣa} 	\ipa{kɯ-xtɕɯ\tld{}xtɕi} 	\ipa{ʑo} 	\ipa{a-nɯ-tɯ-ɤβzu} 	\ipa{smɯlɤm}  \\
girl a \textsc{nmlz:\textsc{s\slash a}-emph}\tld{}be.good \textsc{nmlz:\textsc{s\slash a}-emph}\tld{}be.beautiful  \textsc{2sg.poss}-tooth \textsc{nmlz:\textsc{s\slash a}-emph}\tld{}be.small \textsc{emph} \textsc{irr-pfv}-\textsc{2}-become prayer \\
\glt `May you become a nice and beautiful girl with short teeth.' (Slobdpon, 261)
\end{exe}

The \isi{infinitive} is used (by some speakers) as the citation form of verbs, and appears in some types of complement clauses and manner subordinate clauses (\citealt[271--272; 321--325]{jacques14linking}), as in \REF{ex:kWpWpe2} where \ipa{kɯ-pɯ\tld{}pe}, meaning here `nicely', is a manner \isi{subordinate clause} comprising a single verb.
 \begin{exe}
\ex \label{ex:kWpWpe2}
\gll \ipa{ɕɤr} 	\ipa{tɕe} 	\ipa{ʁzɤmi} 	\ipa{ni} 	\ipa{kɯ-pɯ\tld{}pe} 	\ipa{ʑo} 	\ipa{ɕ-ko-nɯ-rŋgɯ-ndʑi}  \\
evening \textsc{lnk} husband.and.wife \textsc{du} \textsc{inf:stat-emph}\tld{}good \textsc{emph} \textsc{transl-ifr-auto}-lie.down-\textsc{du} \\
\glt `In the evening, the husband and the wife laid down in bed nicely.'
\end{exe}

%  \begin{exe}
%\ex \label{ex:kWxtCi}
%\gll  \ipa{tʂu} 	\ipa{kɯ-xtɕi} 	\ipa{nɯ} 	\ipa{tɕu} 	\ipa{ʑo} 	\ipa{jo-ɕe} \\
%path \textsc{nmlz}:\textsc{s\slash a}-be.small \textsc{dem} \textsc{loc} \textsc{emph} \textsc{ifr}-go \\
%\glt He went by the small path. (the fox 03, 66)
%\end{exe}

\subsection{Potential ambiguity}
Due to the sandhi rule according to which \ipa{kɯ--} combined with \ipa{a-}initial verbs yields /\ipa{kɤ--}/ in \ili{Japhug} (\citealt{jacques04these}), \textsc{s\slash a}-\isi{participle}s or \isi{infinitive} forms of denominal verbs in \ipa{aɣɯ--} are formally homophonous with \isi{comitative} adverbs in \ipa{kɤɣɯ--}. For example, the form \ipa{kɤɣɯrtɯrtaʁ}  `together with its branches' from \ipa{tɤ-rtaʁ} `branch' is identical to the \isi{participle} \ipa{kɤɣɯrtɯrtaʁ} `the one which has many branches' found in example (\ref{ex:kAGWrtWrtaR}).

  \begin{exe}
\ex \label{ex:kAGWrtWrtaR}
\gll   
  \ipa{si} 	\ipa{kɯ-ɤɣɯrtɯrtaʁ} 	\ipa{ki} 	\ipa{kɯ-fse} 	\ipa{ɲɯ-ɕar-nɯ} \\
  tree \textsc{nmlz}:\textsc{s\slash a}-have.many.branches this \textsc{nmlz}:\textsc{s\slash a}-be.this.way \textsc{ipfv}-search-\textsc{pl} \\
\glt `They are searching for a tree which has a lot of branches like this.' (NOT: `a tree with its branches' in this particular context)
\end{exe}

Examples \REF{ex:kAGWrJWrJit} and \REF{ex:kAGWrJWrJit2} present a minimal pair contrasting the \isi{comitative adverb}  `with his/her children' on the one hand and the \isi{participle}  `having many children' on the other hand (both derived from the possessed noun  \ipa{tɤ-rɟit} `child').

\begin{exe}
\ex \label{ex:kAGWrJWrJit}
\gll   
\ipa{iɕqha} 	\ipa{tɕʰeme} 	\ipa{nɯ} 	\ipa{kɯ-ɤɣɯrɟɯrɟit} 	\ipa{ci} 	\ipa{pɯ-ŋu}  \\
the.aforementioned woman \textsc{dem} \textsc{nmlz}:\textsc{s\slash a}-have.many.children \textsc{indef} \textsc{pst.ipfv}-be \\
\glt `This woman had a lot of children.'
\end{exe}

\begin{exe}
\ex \label{ex:kAGWrJWrJit2}
\gll   
\ipa{kɤɣɯ-rɟɯ\tld{}rɟit} 	\ipa{ʑo} 	\ipa{jo-nɯ-ɕe-nɯ} \\
\textsc{comit}-children \textsc{emph} \textsc{ifr-vert}-go-\textsc{pl} \\
\glt `She/They went back with their children.'
\end{exe}



\subsection{Reanalysis}
The formal ambiguity between the \isi{comitative} adverbs on the one hand, and the \isi{participle}s and infinitives of \ipa{aɣɯ--} denominal verbs on the other hand, together with the semantic proximity of the two forms, raise the question of their potential historical relatedness.

An obvious possibility is that \isi{comitative} adverbs originate from the reanalysis of the \textsc{s\slash a}-\isi{participle}s of reduplicated \ipa{aɣɯ--} denominal verbs. Ambiguous sentences like (\ref{ex:kAGWrtWrtaR}) actually constitute the pivot constructions which allow reanalysis in contexts where both \isi{proprietive} (`having X') and \isi{comitative} (`with X') interpretations were possible.


  \begin{exe}
\ex \label{ex:kAGWrtWrtaR2}
\gll   
  \ipa{si} 	\ipa{kɤɣɯrtɯrtaʁ} \ipa{ɲɯ-ɕar-nɯ} \\
  tree \textsc{nmlz}:\textsc{s\slash a}-have.many.branches//\textsc{comit}-branch \textsc{ipfv}-search-\textsc{pl} \\
\glt `They are searching for a tree which has a lot of branches' $\Rightarrow$ `They are searching for a tree and/with its branches'
\end{exe}

Starting from such ambiguous sentences, the \isi{comitative adverb} was extended to nouns without a corresponding \isi{proprietive} denominal verb. In addition, \isi{comitative} adverbs incorporating the indefinite \isi{possessive} prefix were created (such as \ipa{kɤɣɯ-tɤ-rtɯ\tld{}rte} `with his hat'). Forms of this type are clearly distinct from infinitives or \isi{participle}s of denominal verbs, as indefinite \isi{possessive} prefixes are always deleted during  denominal \isi{derivation}.


I therefore propose the pathway (\ref{kAGW-pathway}) to account for \isi{comitative} adverbs in \ipa{kɤɣɯ--} in \ili{Japhug}:

 \begin{exe}
\ex \label{kAGW-pathway}
 \glt  \textsc{noun} + \textsc{property denominal derivation} + \isi{infinitive}/\isi{participle} → \textsc{comitative}
\end{exe} 

Among the possible origins of \isi{comitative} markers,  \citet[91, 139, 287]{heine-kuteva02} includes nouns meaning `comrade' or verbs such as `follow' and `take' and makes no mention of \isi{proprietive} markers. However, \citet{sutton76having} has noted etymological connections between \isi{proprietive} and \isi{comitative} markers in several languages of Australia, and although none of the standard references on \isi{comitative} constructions (\citealt{stassen00and, stolz06comitative, arkhipov09comitative}) explicitly mention a pathway \textsc{proprietive} → \textsc{comitative}, they all notice the close functional relationship between these two categories, notably in languages of Australia, where both \isi{comitative} and \isi{proprietive} cases may exist in the same language (for instance, Djabugay, see \citealt{patz91djabugay}).
 

The pathway presented above accounts well for the \isi{comitative} adverbs of the type \ipa{kɤɣɯ--}, but does not explain the \ipa{kɤ́--} variant, which is actually more common in the corpus.

The \isi{comitative adverb} marker \ipa{kɤ́--} is anomalous in \ili{Japhug} in being among the very few prefixes attracting stress, a feature that could indicate fusion of two syllables (for instance, the negative sensory marker \ipa{mɯ́j-} probably results historically from the fusion of the negative \ipa{mɯ--} and the sensory prefix \ipa{ɲɯ--}).

If the sound laws of \ili{Japhug} (\citealt{jacques04these}) are applied in reverse, the prefix \ipa{kɤɣɯ--} would go back to pre-\ili{Japhug} *\ipa{kɐwə--}. We know that in \ili{Tshobdun}, *\ipa{wə} regularly corresponds to \ipa{o}. It is in particular the case of the inverse prefix \ipa{o--} (\citealt{jackson02rentongdengdi}) which originates from proto-\ili{Gyalrong} *\ipa{wə}. Through vowel fusion (which also occurs with the inverse prefix), \ipa{ko--}, the actual form of the \isi{comitative} prefix (\citealt[107]{jackson98morphology}), is the expected outcome of *\ipa{kɐwə--}. We can therefore safely conclude that (1) the \isi{comitative} prefixes \ipa{kɤɣɯ--} in \ili{Japhug} and \ipa{ko--} in \ili{Tshobdun} are cognate and (2) that the grammaticalization in \REF{kAGW-pathway} took place before the split of \ili{Japhug} and \ili{Tshobdun}, and can be reconstructed at least to their common ancestor.

The \isi{comitative} prefix \ipa{kɤ́--} in \ili{Japhug}, on the other hand, makes no sense from a \ili{Japhug}-internal perspective. A possible way to explain it, however, is to suppose \textit{borrowing} from \ili{Tshobdun} \ipa{ko--}. \ili{Japhug}, and especially the \ili{Kamnyu} variety described in the present paper, has borrowed a few nouns from \ili{Tshobdun}, as shown by forms such as \ipa{qro} `ant', \ipa{qaliaʁ} `eagle' and \ipa{tɯɟo} `demon' instead of expected *\ipa{qroʁ}, *\ipa{qarɟaʁ} (attested in some dialects of \ili{Japhug}) and *\ipa{tɯʑu}, following the sound laws set out in \citet{jacques04these}.

Borrowing of \ili{Tshobdun} \ipa{ko--} as \ili{Japhug} \ipa{kɤ́--} is not surprising phonologically. The stress on the prefix in \ili{Japhug} is probably a trace of the stress on that prefix in pre-\ili{Tshobdun}, lost due to the strong tendency of \ili{Gyalrong} languages to stress the final or penultimate syllable (\citealt{jackson05yingao}). The vowel \ipa{ɤ} rather than \ipa{o} is a consequence of the fact that derivational prefixes in \ili{Japhug} are subject to strong phonotactic constraints: the only possible vowels are either \ipa{ɤ} or \ipa{ɯ} (and \ipa{a}, but only in the case of stem-initial \ipa{a--}).

The borrowing hypothesis also accounts for the absence of any discernible difference in function between the two \isi{comitative} prefixes in \ili{Japhug}.

\section{Conclusion} 
The contribution of this paper is threefold. First, it provides the first detailed description of \isi{comitative} adverbs in any \ili{Gyalrong} language. Second, it shows that \isi{language contact} between \ili{Gyalrong} languages is not restricted to the lexicon, but actually also involves clear cases of borrowing of grammatical morphemes. Third, it provides an example of evolution with clear \isi{directionality} from \textsc{proprietive} to \textsc{comitative}.

\section*{Acknowledgements}
I would like to thank Andrej Malchukov and an anonymous reviewer for useful comments, and Thomas Pellard for devising the map. The examples are taken from a corpus that is progressively being made available on the Pangloss archive (\citealt{michailovsky14pangloss}). This research was funded by the HimalCo project (\textsc{anr-12-corp-0006}) and  the Labex Empirical Foundations of Linguistics (\textsc{anr\slash cgi}).
 
{\sloppy
\printbibliography[heading=subbibliography,notkeyword=this]
}
\end{document}

