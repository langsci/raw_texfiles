\documentclass[output=paper,colorlinks,citecolor=brown]{langscibook}
\ChapterDOI{10.5281/zenodo.7053359}
\title{Introduction} 
\author{Andreas Hölzl\affiliation{University of Potsdam} and Thomas E. Payne\affiliation{University of Oregon}}

\abstract{This introduction briefly presents the Tungusic languages, discusses their classification from a meta-perspective, and outlines the contents of the eight individual contributions to this volume.}

\IfFileExists{../localcommands.tex}{
   \addbibresource{../localbibliography.bib}
   % add all extra packages you need to load to this file

\usepackage{tabularx,multicol}
\usepackage{url}
\urlstyle{same}

\usepackage{listings}
\lstset{basicstyle=\ttfamily,tabsize=2,breaklines=true}

\usepackage{langsci-basic}
\usepackage{langsci-optional}
\usepackage{langsci-lgr}
\usepackage{langsci-osl}
% \usepackage{./langsci/styles/langsci-lgr}
% \usepackage{./langsci/styles/langsci-osl}
% \usepackage{langsci-gb4e}

\usepackage{tikz}
\usetikzlibrary{patterns,calc}
\pgfdeclarepatternformonly{south east lines}{\pgfqpoint{-0pt}{-0pt}}{\pgfqpoint{3pt}{3pt}}{\pgfqpoint{3pt}{3pt}}{
    \pgfsetlinewidth{0.6pt}
    \pgfpathmoveto{\pgfqpoint{0pt}{3pt}}
    \pgfpathlineto{\pgfqpoint{3pt}{0pt}}
    \pgfpathmoveto{\pgfqpoint{.2pt}{-.2pt}}
    \pgfpathlineto{\pgfqpoint{-.2pt}{.2pt}}
    \pgfpathmoveto{\pgfqpoint{3.2pt}{2.8pt}}
    \pgfpathlineto{\pgfqpoint{2.8pt}{3.2pt}}
    \pgfusepath{stroke}}
    
\usepackage{stmaryrd}
\usepackage{wasysym}
\usepackage{multirow}
\usepackage{caption}
\usepackage{subcaption}
\usepackage{mathrsfs}
\usepackage{qtree}

\usepackage{linguex}


   %pminos do not split footnotes
% \interfootnotelinepenalty=10000 %Footnote in Laporte chapters has to be split SN


%\DeclareIndexNameFormat{default}{%
%\nameparts{#1}%
%\usebibmacro{index:name}%
%{\index[names]}%
%{\namepartfamily}%
%{\namepartgiveni}%
% {}% L1
% {}% L2
%{\namepartprefix}% generates spurious space L3
%{\namepartsuffix}% generates spurious space L4
%}

%  {\DeclareIndexNameFormat{default}{%
%     \usebibmacro{index:name}{\index[names]}{#1}{#3}{#5}{#7}}}

%\DeclareIndexNameFormat{default}{%
%  \usebibmacro{index:name}{\sindex[nom]}{#1}{#3}{#5}{#7}}

%\DeclareIndexNameFormat{default}{%
%  \usebibmacro{index:name}{\sindex[person]}{#1}{#3}{#5}{#7}}
%\DeclareIndexNameFormat{default}{%
%\nameparts{#1} \usebibmacro{index:name}{\sindex[person]]}{\namepartfamily}{‌​\namepartgiven}{\nam‌​epartprefix}{\namepa‌​rtsuffix}}

%\newcommand{\smiley}{:)}

%\renewbibmacro*{index:name}[5]{%
%\usebibmacro{index:entry}{#1}%
%{\iffieldundef{usera}{}{\thefield{usera}\actualoperator}\mkbibindexname{#2}{#3}{#4}{#5}}}

% \newcommand{\noop}[1]{}

%remove for final
%\overfullrule=1mm

\newcommand{\tobi}[2]}}
\renewcommand{\S}[1]{\tobi{#1}{\textsc{*}}}

% this volume references
% puts: [this volume]
% already defined: \citetv
%\newcommand{\citepv}[1]{(\citeauthor{#1} \citeyear*{#1} [this volume])}
\newcommand{\citealtv}[1]{\citeauthor{#1} \citeyear*{#1} [this volume]}

%parentheses around example number
\newcommand{\pref}[1]{(\ref{#1})}

% in-text examples

\newcommand{\lnex}[1]{\textit{#1}} %target lang word
\newcommand{\lnlit}[1]{(lit.: `#1')} %literal reading
\newcommand{\lnlat}[1]{(#1)} % latinization
\newcommand{\lntrans}[1]{`#1'} %translation
\newcommand{\lnexl}[2]%
{\lnex{#1}{} \lnlat{#2}} % ex with latinization
\newcommand{\lnexlat}[3]{\lnex{#1}{} \lnlat{#2}{} \lntrans{#3}} % ex with latinization and tranl.

%ch01
\newcommand{\co}[1]{\mbox{\textbf{#1}}}

%ch09

\newcommand{\cyrbulg}[1]{\begin{otherlanguage*}{bulgarian}#1\end{otherlanguage*}}


%ch10
\newcommand{\nlp}{{\small NLP}}
\newcommand{\mwe}{{\small MWE}}
\newcommand{\rae}{{\small RAE}}
\newcommand{\lvc}{{\small LVC}}
\newcommand{\pos}{{\small P}o{\small S}}
%\newcommand{\todo}[1]{ \textcolor{red}{#1} }

%\renewcommand{\labelenumi}{\theenumi}
%\ainamefmt{{vv}{ll}{, ff}{, jj}} % fullname

\newcommand{\biberror}[1]{{\color{red}#1}}

\newcommand{\osenovaitem}{--~}
   %% hyphenation points for line breaks
%% Normally, automatic hyphenation in LaTeX is very good
%% If a word is mis-hyphenated, add it to this file
%%
%% add information to TeX file before \begin{document} with:
%% %% hyphenation points for line breaks
%% Normally, automatic hyphenation in LaTeX is very good
%% If a word is mis-hyphenated, add it to this file
%%
%% add information to TeX file before \begin{document} with:
%% %% hyphenation points for line breaks
%% Normally, automatic hyphenation in LaTeX is very good
%% If a word is mis-hyphenated, add it to this file
%%
%% add information to TeX file before \begin{document} with:
%% \include{localhyphenation}
\hyphenation{
    Beck-man
    Ngu-yen
    back-chan-nel
    back-chan-nels
    mo-not-o-nous
    ste-reo-typ-i-cal
}

\hyphenation{
    Beck-man
    Ngu-yen
    back-chan-nel
    back-chan-nels
    mo-not-o-nous
    ste-reo-typ-i-cal
}

\hyphenation{
    Beck-man
    Ngu-yen
    back-chan-nel
    back-chan-nels
    mo-not-o-nous
    ste-reo-typ-i-cal
}

   \boolfalse{bookcompile}
   \togglepaper[1]%%chapternumber
}{}

\begin{document}
\maketitle

\section{Tungusic languages}

Tungusic (sometimes Manchu-Tungusic) is an endangered language family that encompasses approximately twenty languages located in Siberia and northern China (e.g., \citealt{Janhunen1996, Janhunen2005, Janhunen2012a}). These languages are distributed over an enormous area that ranges from the Yenisey River and Xinjiang in the west to the Kamchatka Peninsula and Sakhalin in the east. They extend as far north as the Taimyr Peninsula and, for a brief period, could even be found in parts of Central and South China (e.g., \citealt{HölzlHölzl2019b}). Tungusic-speaking peoples played an important role in the history of Northeast and East Asia and were the founders of several large empires, such as the Jin (1115--1234) and Qing dynasties (1636--1912). Recent years have seen considerable interest in this language family. Tungusic linguistics is an extremely active field of study that produced hundreds of new studies in recent years (see, for example, the references listed in \citealt{Hölzl2021b}). However, the field is also very fragmented with studies being written in several languages, from a wide range of scholarly traditions. Research on Tungusic languages has been published, among others, in \ili{Chinese}, \ili{Czech}, \ili{English}, \ili{French}, \ili{German}, \ili{Hungarian}, \ili{Italian}, \ili{Japanese}, \ili{Korean}, \ili{Latin}, \ili{Manchu}, \ili{Polish}, and \ili{Russian}. Many important contributions and entire languages have gone almost unnoticed because of language barriers or the limited availability of some publications. This volume is an attempt to bring researchers from different backgrounds together to provide an open-access publication in \ili{English} that is freely available to all scholars in the field. The volume emphasizes the diachronic dimension of Tungusic, tracing the development of the language family from prehistory and the earliest attestations, but also includes synchronic descriptions. This introduction briefly introduces the Tungusic languages, presents some recently published and previously overlooked data, and summarizes the individual contributions.


\section{Classification and terminology}

Tungusic is a top-level language family. The branching structure is open to discussion (see, e.g., \citealt{WhaleyOskolskaya2020} and references therein), but most accounts agree on four mid-level groupings. These are comparable to branches of \ili{Indo-European}, such as \ili{Germanic}, \ili{Italic}, or \ili{Slavic}, but there is no universally accepted terminology yet. Following \citet{Janhunen2012a}, the groups are referred to as \ili{Ewenic}, \ili{Udegheic}, \ili{Nanaic}, and \ili{Jurchenic}. These terms, based on the languages \ili{Even} (Ewen), \ili{Udihe} (Udeghe), \ili{Nanai}, and \ili{Jurchen}, respectively, are also used in this introduction and the contribution by \textcitetv{chapters/04}. They are also briefly addressed in \textcitetv{chapters/05} and \textcitetv{chapters/08}. Some of the terms are also used by other contributions in this volume (e.g., \citetv{chapters/03}; \citetv{chapters/08}; \citetv{chapters/09}). \ili{Jurchenic} (e.g., \citealt{Janhunen1996}) and \ili{Nanaic} (e.g., \citealt{Georg2004}) already have a relatively long history. For \ili{Udegheic}, Janhunen (e.g., \citeyear{Janhunen2015}) sometimes uses the term \ili{Orochic}, based on the closely related language \ili{Oroch} instead of \ili{Udihe}. \ili{Jurchenic} is also referred to as Manchuric in \citet{Alonso2010}, \citet{Jang2020}, \textcitetv{chapters/05}, or \textcitetv{chapters/08}, a name derived from the \ili{Manchu} language. In the Japanese tradition, the groups are indicated with the help of Roman numerals from I to IV (e.g., \citealt{Ikegami1974}; \citealt{Kazama2003}) that will be used alongside Janhunen’s terminology here.

Many alternative terminologies have been proposed. For instance, \ili{Ewenic} is often called \ili{Northern Tungusic} (e.g., \citetv{chapters/02}; \citetv{chapters/05}) while this name is reserved by Janhunen for a proposed group that includes \ili{Udegheic} and \ili{Ewenic}. Furthermore, many \ili{Ewenic} languages of China are spoken as far south as \ili{Nanaic} or \ili{Udegheic}. A hypothetical branch encompassing \ili{Udegheic} and \ili{Nanaic} is sometimes called Amuric\il{Amuric group of Tungusic} (e.g., \citetv{chapters/05}). But following \citet{Janhunen1996}, \ili{Amuric} is also often used as a label for varieties of \ili{Nivkh}. \citet[5]{Doerfer1978} also employs the terms Northern branch for \ili{Ewenic} (showing a secondary split into a Northeastern\il{Northeastern Tungusic} and a Northwestern\il{Northwestern Tungusic} group) and Southern branch for \ili{Jurchenic}, but Central Eastern\il{Central Eastern Tungusic} group for \ili{Udegheic} as well as Central Western\il{Central Western Tungusic} group for \ili{Nanaic}, illustrating that \ili{Udegheic} and \ili{Nanaic} are believed to belong to one branch. \ili{Southern Tungusic} in turn is used by Janhunen for a group that consists of \ili{Nanaic} and \ili{Jurchenic}. Given that these terminologies presuppose specific classifications of Tungusic that are not accepted by all researchers, a more neutral terminology is needed. Such a terminology is proposed in Table \ref{Introtable:1}.

\begin{table}
\begin{tabular}{ l l l }
  \lsptoprule
Numbers	& Names	& Languages\\
  \midrule
I	& \ili{Ewenic}	& \ili{Even} (Ewen), \ili{Evenki} (Ewenki) …\\
II	& \ili{Udegheic}/\ili{Orochic}	& \ili{Udihe} (Udeghe), \ili{Oroch}, …\\
III	& \ili{Nanaic}	& \ili{Nanai}, …\\
IV	& \ili{Jurchenic}/\ili{Manchuric}	& \ili{Jurchen}, \ili{Manchu}, …\\
  \lspbottomrule
\end{tabular}
\caption{Theory-neutral terminology for the four Tungusic groups}
\label{Introtable:1}
\end{table}

\begin{sloppypar}
While the four groups can be considered a common ground for most approaches, their internal classification and higher-level relations are a matter of ongoing debate. Within \ili{Ewenic}, for instance, \ili{Negidal} is assumed to be closely related to \ili{Evenki} in \citet{Doerfer1978} or \textcitetv{chapters/02}, but to the language \ili{Even} in \textcitetv{chapters/08}. The internal classification of the entire \ili{Udegheic} branch (e.g., \ili{Udihe}, \ili{Oroch}) is investigated in the contribution by \textcitetv{chapters/07}, demonstrating a historical continuum, while \ili{Oroch} problematically is not grouped with \ili{Udihe} in \citet{Oskolskayaetal2022}. The relationship of \ili{Ewenic} languages as spoken in Russia (i.e., \ili{Even}, \ili{Evenki}, \ili{Negidal}) is briefly addressed in \textcitetv{chapters/02} and \textcitetv{chapters/06}. \ili{Evenki} dialects situated around the \ili{Chinese}-\ili{Russian} border (particularly \il{Evenki!Khamnigan}Khamnigan Evenki and \ili{Nercha Evenki}) are discussed in \textcitetv{chapters/05}.
\end{sloppypar}

The internal structure and relationship of the four mid-level groups also face problems through family-internal language mixing. This can be illustrated with the language \ili{Kilen} that is variously classified as \ili{Jurchenic} (\citealt{Oskolskayaetal2022}, included into the category \ili{Hezhe}), mixed but basically \ili{Nanaic} (\citetv{chapters/04}), \ili{Udegheic} (\citealt{Kazama2003}, referred to as \ili{Hezhe}), or as “missing link” between \ili{Udegheic} and \ili{Ewenic} (\citealt{Kazama1998}, referred to as \ili{Kilen} or \ili{Hezhen}).\footnote{Due to the official classification, varieties of \ili{Kilen} (Chinese \textit{qileng} \zh{奇楞}, a mixed language) and \ili{Hezhen} (Chinese \textit{hezhen} \zh{赫真}, a form of southern \ili{Nanai}) are classified as dialects of the \ili{Hezhe} \zh{赫哲} language in China (e.g., \citealt{AnJun1986}). This is similarly problematic as the term “\ili{Ewenke}” for several \ili{Ewenic} languages (see below and \citetv{chapters/05}).} Similar difficulties exist, among others, for Kur-Urmi Nanai (or \ili{Kili}) and \ili{Ussuri} (or Bikin) \ili{Nanai} that are classified as mixed but basically \ili{Nanaic} in \textcitetv{chapters/04}, but as related to \ili{Jurchenic} in \citet{Oskolskayaetal2022}, whereas \citet{Kazama2003} classifies \ili{Kili} as \ili{Ewenic}. There is no simple solution to these problems. \citet[4f.]{Doerfer1978} attempted to solve such obstacles by assuming transitional varieties between the four subgroups. But they are perhaps best considered mixed languages (e.g., \citealt[6]{Janhunen2012a}) that are the result of complex secondary interactions and different types of admixture of the four groups around the confluence of the Amur, Sungari, and Ussuri rivers. Dialect mixture and language contact are universal problems of historical linguistics for which Tungusic languages might prove a valuable natural experiment for future studies (e.g., \citealt{Eppsetal2013}; \citealt{McMahon2013}).

There is currently no generally agreed-upon higher-level classification of Tungusic. Logically speaking, four groups can stand in five types of relationships with each other (Table \ref{Introtable:2}). Three of these represent cases of a twofold primary split, and the other two are cases of three- and fourfold splits, respectively. The exact age and internal diversity of the four groups are irrelevant for this purely topological approach.

\begin{table}
\begin{tabular}{ l l l l }
  \lsptoprule
Primary split	& Type	& Topological schema	& Possibilities\\
  \midrule
Twofold	& Type 1	& [A B] [C D]	& 3\\
	& Type 2	& A [B C D]	& 4\\
	& Type 3	& A [B [C D]]	& 12\\
Threefold	& Type 4	& [A B] C D	& 6\\
Fourfold	& Type 5	& A B C D	& 1\\
  \lspbottomrule
\end{tabular}
\caption{Logical possibilities for the classification of Tungusic}
\label{Introtable:2}
\end{table}

\begin{figure}
\begin{subfigure}[b]{.5\textwidth}\centering
    \begin{forest}  for tree={grow'=east,delay={where content={}{shape=coordinate}{}}},   forked edges  
    [{}
        [
            [Branch A]
            [Branch B]
        ]
        [
            [Branch C]
            [Branch D]
        ]
    ]
    \end{forest}
    \caption{\label{tree:Type1}Type 1}
\end{subfigure}\begin{subfigure}[b]{.5\textwidth}\centering
    \begin{forest}  for tree={grow'=east,delay={where content={}{shape=coordinate}{}}},   forked edges  
    [{}
        [Branch A]
        [
            [Branch B]
            [Branch C]
            [Branch D]
        ]
    ]
    \end{forest}
    \caption{\label{tree:Type2}Type 2}
\end{subfigure}\bigskip\\\begin{subfigure}[b]{.5\textwidth}\centering
    \begin{forest}  for tree={grow'=east,delay={where content={}{shape=coordinate}{}}},   forked edges  
    [{}
        [Branch A]
        [
            [Branch B]
            [
                [Branch C]
                [Branch D]
            ]
        ]
    ]
    \end{forest}
    \caption{\label{tree:Type3}Type 3}
\end{subfigure}\begin{subfigure}[b]{.5\textwidth}\centering
    \begin{forest}  for tree={grow'=east,delay={where content={}{shape=coordinate}{}}},   forked edges  
    [{}
        [
        [Branch A]
        [Branch B]
        ]
        [Branch C]
        [Branch D]
    ]
    \end{forest}
    \caption{\label{tree:Type4}Type 4}
\end{subfigure}\bigskip\\\begin{subfigure}[b]{.5\textwidth}\centering
    \begin{forest}  for tree={grow'=east,delay={where content={}{shape=coordinate}{}}},   forked edges  
    [{}
        [Branch A]
        [Branch B]
        [Branch C]
        [Branch D]
    ]
    \end{forest}
    \caption{\label{tree:Type5}Type 5}
\end{subfigure}
\caption{Possible topologies}
\end{figure}

Altogether there are 26 logical possibilities for the topology of the Tungusic tree. Only a few of these have been proposed or are widely represented in the literature. For instance, a split into four separate branches (Type 5), sometimes attributed to \citet{Ikegami1974}, is not accepted by any current approach. Types 2 and 4 do not appear to be accepted either but remain theoretically possible.

Recent classifications only diverge from each other by few variables, two of which are included here. First, they differ with respect to the position of \ili{Udegheic} that is either grouped with \ili{Ewenic} or with \ili{Nanaic}. Second, they disagree whether \ili{Jurchenic} is the first branch to diverge from all other branches or is somehow related to \ili{Nanaic}. Including only these two variables allows a meta-classification of Tungusic as illustrated in Table \ref{Introtable:3}.

\begin{table}
\begin{tabular}{ l l l }
  \lsptoprule
& \ili{Jurchenic} as first branch	& \ili{Jurchenic} related to \ili{Nanaic}\\
  \midrule
\ili{Udegheic} + \ili{Nanaic}	& A: IV [I [II III]]	& B: I [IV [II III]]\\
\ili{Udegheic} + \ili{Ewenic}	& C: IV [III [I II]]	& D: [I II] [III IV]\\
  \lspbottomrule
\end{tabular}
\caption{A simplified meta-classification of Tungusic}
\label{Introtable:3}
\end{table}

\begin{figure}
\begin{subfigure}[b]{.5\textwidth}\centering
\begin{forest}  for tree={grow'=east,delay={where content={}{shape=coordinate}{}}},   forked edges  
[{}
    [\ili{Jurchenic} (IV)]
    [
        [\ili{Ewenic} (I)]
        [
            [\ili{Udegheic} (II)]
            [\ili{Nanaic} (III)]
        ]
    ]
]
\end{forest}
\caption{Classification A\label{tree:A}}
\end{subfigure}\begin{subfigure}[b]{.5\textwidth}\centering
\begin{forest}  for tree={grow'=east,delay={where content={}{shape=coordinate}{}}},   forked edges  
[{}
    [\ili{Ewenic} (I)]
    [
        [\ili{Jurchenic} (IV)]
        [
            [\ili{Udegheic} (II)]
            [\ili{Nanaic} (III)]
        ]
    ]
]
\end{forest}
\caption{Classification B\label{tree:B}}
\end{subfigure}\bigskip\\\begin{subfigure}[b]{.5\textwidth}\centering
\begin{forest}  for tree={grow'=east,delay={where content={}{shape=coordinate}{}}},   forked edges  
[{}
    [\ili{Jurchenic} (IV)]
    [
        [\ili{Nanaic} (III)]
        [
            [\ili{Udegheic} (II)]
            [\ili{Ewenic} (I)]
        ]
    ]
]
\end{forest}
    \caption{\label{tree:C}Classification C}
\end{subfigure}\begin{subfigure}[b]{.5\textwidth}\centering
\begin{forest}  for tree={grow'=east,delay={where content={}{shape=coordinate}{}}},   forked edges  
[{}
    [
    [\ili{Ewenic} (I)]
    [\ili{Udegheic} (II)]
    ]    
    [
    [\ili{Nanaic} (III)]
    [\ili{Jurchenic} (IV)]
    ]
]
\end{forest}
    \caption{Classification D\label{tree:D}}
\end{subfigure}
\caption{Recent classifications}
\end{figure}

Three of these represent cases of Type 3 (classifications A, B, C) and one of Type 1 (classification D). All four classifications agree on some points that are, however, explained differently. The well-known similarities between \ili{Nanaic} and \ili{Udegheic} can theoretically be described by shared innovations (classifications A and B) or by convergence (classification D and perhaps C, e.g. \citealt{Georg2004}; \citealt[112]{Alonso2017}). The widely acknowledged differences between \ili{Jurchenic} and the rest of Tungusic can be explained by an early branching (classifications A and C) or by different types of contact with non-Tungusic languages, such as \ili{Koreanic}, \ili{Mongolic}, \ili{Para-Mongolic}, and \ili{Sinitic} (classification D and perhaps B, e.g., \citealt{Vovin2006}; \citealt{Hölzl2018c}).

Some previous studies slightly disagree with the classification into four subgroups. For instance, \citegen{Vovin1993} tree resembles classification A but assumes that \ili{Even} forms a separate branch after the split of \ili{Jurchenic} and before the diversification of the rest of Tungusic. But \citet[1103]{Vovin2009} later accepted classification D as proposed by \citet{Georg2004}. Most recent approaches can be categorized according to the meta-classification in Table \ref{Introtable:3}. For example, \citet{Robbeets2015} is a proponent of classification A while \citet{Doerfer1978}, although skeptical about tree diagrams, argues for classification B. \citet{Kazama2003} and \citet{Pevnov2017} follow classification C. \citet{Georg2004}, \citet{Janhunen2012a}, and \textcitetv{chapters/04} accept classification D that groups \ili{Ewenic} with \ili{Udegheic} into a Northern and \ili{Nanaic} with \ili{Jurchenic} in a \ili{Southern Tungusic} branch. Some approaches remain undecided or allow more than one possibility. For instance, \citet[91]{WhaleyOskolskaya2020} identified classification B as the most likely scenario with classification A also being supported by their study whereas \citet{Oskolskayaetal2022} tend towards classification D but leave the possibility for an early branching of \ili{Jurchenic} open.

Whichever classification will eventually be supported by the most evidence, provided that the four groups and the tree model are accepted as a basis, it must be one of the 26 in Table \ref{Introtable:2} and probably one of the only four possibilities shown in Table \ref{Introtable:3}. All previous classifications are likely to be the object of future revisions due to the development of new methodologies and in the light of newly available data.

\section{Availability of new data}

Tungusic linguistics has produced several outstanding works, such as the classical comparative dictionary by \citet{Cincius1975} that can be considered a milestone in the field. However, it is by now over 45 years old and appeared just before new data became available on languages spoken in China starting from the end of the 1970s, not to mention that the Tungusic languages in Russia have also been increasingly well described over the last decades. \citet{Cincius1975} represents only about half of the linguistic varieties (doculects) that are available by now. It has been supplemented by newer comparative dictionaries, such as \citet{Kazama2003}, \citet{DoerferKnueppel2004}, or \citet{Chaoke2014}, but these do not cover all varieties either. A comprehensive review of all available data is beyond this brief introduction that limits itself to briefly presenting some new monographs on \ili{Ewenic} languages in China from the last couple of years and some previously overlooked \ili{Jurchenic} languages described during the 1980s.

A comprehensive classification of \textsc{Ewenic} necessarily includes varieties located in Russia (e.g., \ili{Arman}, \ili{Even}, \ili{Evenki}, \ili{Negidal}) and in China. Except for the dialects of \ili{Oroqen}, which is called \textit{Elunchun} \zh{鄂伦春} in \ili{Chinese}, \ili{Ewenic} languages in China are collectively referred to as \textit{\ili{Ewenke}} \zh{鄂温克}, a cover term for various dialects of \ili{Solon} and \ili{Evenki} (e.g., \citealt{Tsumagari1992}; \citealt{Janhunen1996}; \citetv{chapters/05}). Several grammars and dictionaries of \ili{Ewenic} languages spoken in China, many of which were previously underdescribed, have been published over the course of the last couple of years. Recent monographs include, but are not restricted to, two grammars and texts of “\ili{Aoluguya} Ewenke” (\ili{Aoluguya}/Yakut Evenki, \citealt{ChaokeSirenbatu2016}; \citealt{Hasibateer2016}; \citealt{WengChaoke2016}), text collections and a grammar of “Tonggusi \ili{Ewenke}” (\il{Evenki!Khamnigan}Khamnigan/Tungus Evenki, \citealt{ChaokeKajia2016}; \citealt{DuoChaoke2016}), a comprehensive dictionary of “Elunchun” covering several \ili{Oroqen} dialects (\citealt{HanMeng2019}), an extensive phonology of “\ili{Ewenke}” (\ili{Huihe} Solon, \citealt{Wurigexiletu2018}), texts and a dictionary of “\ili{Arong Ewenke}” (\citealt{ChaokeKalina2017}), texts and a grammar of “\ili{Dula’er Ewenke}” (\citealt{Najia2017}; \citealt{ChaokeNajia2017}), a dictionary of “\ili{Nehe Ewenke}” (\citealt{ChaokeKajia2017}) etc. A detailed classification of the latter three varieties remains to be done. \citet{Chaoke2017} is a comparative dictionary of \ili{Huihe} Solon, \il{Evenki!Khamnigan}Khamnigan/Tungus Evenki (“\ili{Morigele}” dialect), and \ili{Aoluguya}/Yakut Evenki.

Apart from some relics, \ili{Ewenic} languages are unique among Tungusic in preserving an intervocalic *-g-, one common argument for classification B. Table \ref{Introtable:4} contains examples from the newly published sources. In some \ili{Ewenic} languages, the \textit{-g-} is realized as a fricative or approximant, e.g. \ili{Aoluguya Evenki} [bæːʁɑ] ‘moon’ (\citealt{Hasibateer2016}), and in a few the \textit{-g-} disappeared entirely, leading to the emergence of diphthongs and long vowels as in other Tungusic languages. This can be observed, among others, in one \il{Evenki!Khamnigan}Khamnigan Evenki dialect (\ili{Urulyungui} \textit{tee-}, Borzya \textit{tege-} ‘to sit’, \citetv{chapters/05}), in \ili{Oroqen}, but also in the language referred to as “\ili{Arong Ewenke}” that was recorded in Chabaqi \zh{查巴奇} in Inner Mongolia (\citealt{ChaokeKalina2017}). This language, tentatively classified as \ili{Solon} in \textcitetv{chapters/04}, also exhibits some features reminiscent of \ili{Solon} dialects, such as the developments of geminates from consonant clusters. For instance, the cluster \textit{-rg-} changed to \textit{-gg-} in the word \textit{iggə} ‘tail’ but is preserved in \textit{irgi} ‘brain’ (cf. \ili{Aoluguya Evenki} \textit{irgə} {\textasciitilde} \textit{irgi} ‘tail’, \textit{irgə} ‘brain’, \ili{Huihe} Solon \textit{iggi} ‘tail’, \textit{iiggi} ‘brain’, \citealt{Chaoke2017}). The dialects of \ili{Solon}, \ili{Oroqen}, and \ili{Evenki} show an intricate pattern of family resemblances and interaction that is still incompletely understood (e.g., \citealt{Whaleyetal1999}; \citetv{chapters/05}). This growing number of publications, although difficult to access for the wider public outside of China, represents important progress in the description of the dwindling dialectal diversity of \ili{Ewenic}.

\begin{table}
\begin{tabular}{ l l l l l l l }
  \lsptoprule
Variety	& fire	& four	& moon	& summer	& sun	& to sit\\
  \midrule
\ili{Aoluguya} Ewenke	& to\textbf{g}o	& di\textbf{g}in	& bee\textbf{g}(a)	& dʒo\textbf{g}a	& ʃi\textbf{g}un	& tə\textbf{g}ə-\\
\ili{Dula’er Ewenke}	& to\textbf{g}	& di\textbf{g}in	& bia\textbf{g}a	& dʒo\textbf{g}o	& ʃi\textbf{g}uŋ	& tə\textbf{g}ə-\\
\ili{Huihe} Ewenke	& to\textbf{g}	& di\textbf{g}iŋ	& bee\textbf{g}a	& dʒo\textbf{g}	& ʃi\textbf{g}ʉŋ	& tə\textbf{g}ə-\\
\ili{Morigele} Ewenke	& to\textbf{g}	& di\textbf{g}in	& bee\textbf{g}a	& dʒu\textbf{g}a	& ʃi\textbf{g}un	& tə\textbf{g}ə-\\
\ili{Nehe Ewenke}	& to\textbf{g}	& di\textbf{g}in	& bia\textbf{g}	& dʒua\textbf{g}	& ʃi\textbf{g}uŋ	& tə\textbf{g}-\\
\ili{Arong Ewenke}	& too	& dijin	& bia, be	& dʒona	& ʃiwən	& tə\textbf{g}ə-\\
\ili{Oroqen} (Elunchun)	& t‘ɔɔ	& tijin	& peja	& dʒuwaa	& ɕiwun	& t‘əə-\\
  \lspbottomrule
\end{tabular}
\caption{Examples for intervocalic \textit{-g-} in some Ewenic varieties of China (\citealt{Chaoke2017}; \citealt{ChaokeKajia2017}; \citealt{ChaokeKalina2017}; \citealt{HanMeng2019}; \citealt{Najia2017})}
\label{Introtable:4}
\end{table}

The \textsc{Jurchenic} branch is of special importance for the history of Tungusic. If classifications A or C should be correct, \ili{Jurchenic} represents the oldest branch of Tungusic. It is the largest branch in terms of speakers historically and currently. It has produced three distinct writing systems and by far contains the oldest and most numerous records among all Tungusic languages. Today, the last representative of \ili{Jurchenic} with many speakers is \ili{Sibe} (Xibe) that is increasingly well described in both its written (e.g., \citealt{Stary2017}) and spoken forms (e.g., \citealt{Jang2020}; \citealt{JangPayne2018}; \citealt{Zikmundová2013a}). Despite being studied longest, \ili{Jurchenic} is sometimes reduced to \ili{Jurchen}, \ili{Manchu}, and \ili{Sibe}. However, \ili{Jurchen} is a cover term for at least two different varieties (e.g., \citealt{Kiyose2000}), \textcitetv{chapters/09} points out dialectal differences within \ili{Sibe} (see also \citealt{Zheng2019}), and there is a large number of spoken \ili{Manchu} dialects that were recorded in places such as \ili{Aihui} (e.g., \citealt{Shirokogoroff1924}; \citealt{WangQingfeng2005}), \ili{Lalin} (e.g., \citealt{MuYejun1986b}; \citealt{Ma1997c}; \citealt{Wang2001}; \citealt{Aixinjueluo2014}), \ili{Sanjiazi} (e.g., \citealt{JinQizong1981}; \citealt{Enhebatu1995}; \citealt{KimJuwon2008}; \citealt{DaiGuangyu2012}), \ili{Yanbian} (e.g., \citealt{ZhaoJie2000}), or \ili{Yibuqi} (e.g., \citealt{ZhaoJie1989}). In addition, there are at least three outlying \ili{Jurchenic} varieties called \ili{Alchuka}, \ili{Bala}, and \ili{Kyakala} that were already described in the 1980s but overlooked in comparative studies of Tungusic (Table \ref{Introtable:5}). These three varieties are probably extinct and have mostly been recorded by a scholar named Mu Yejun (also called Mu’ercha Yejun or Mu’ercha Anbulonga). To avoid confusion, \citet[90]{HölzlHölzl2019a} introduce the names “\ili{Chinese Kyakala}” for the \ili{Jurchenic} and “\ili{Russian Kyakala}” for the \ili{Udegheic} variety with that name (on which see \citetv{chapters/07}). The descriptions suffer from inexact transcriptions, some typographic errors, and problematic analyses, but appear to be genuine. At least some of the data have been confirmed through independent recordings (see also \citealt{Ma1997a, Ma1997b, Ma1997c, Ma1997d}).

\begin{table}
\begin{tabular}{p{2,8cm} p{8cm}}
  \lsptoprule
Variety	& Main studies and sources\\
  \midrule
\ili{Alchuka} & \citealt[72]{MuYejun1981}; \citeyear{MuYejun1985}; \citeyear{MuYejun1986a}; \citeyear{MuYejun1986b}; \citeyear{MuYejun1987}; \citeyear{MuYejun1988b}; \citealt{Ikegami1994}; \citeyear[321–343]{Ikegami1999}; \citealt[passim]{Aixinjueluo2014}; \citealt{Hölzl2017a, Hölzl2020, Hölzlpreprint}\\
\ili{Bala} & \citealt{MuYejun1984, MuYejun1987, MuYejun1988a}; \citealt[321–343]{Ikegami1999}; \citealt{Lietal2018}; \citealt{Hölzl2020a, Hölzl2021a}\\
\ili{Chinese Kyakala} & \citealt{MuMa1983}; \citealt{MuMu1983}; \citealt{MuMeng1986}; \citealt{MuYejun1987}; \citealt{Gu2018}; \citealt{Hölzl2018b}; \citealt{HölzlHölzl2019a}\\
  \lspbottomrule
\end{tabular}
\caption{Three outlying Jurchenic varieties}
\label{Introtable:5}
\end{table}

The term \textit{Manchuric} (with an \textit{r}) as a synonym for \ili{Jurchenic} (Table \ref{Introtable:1}) should not be confused with \textit{Manchuic} (without the \textit{r}) as used by \citet{Hölzl2017a} for one of three hypothetical subgroups of \ili{Jurchenic}/Manchuric, the others being \ili{Alchukaic} and \ili{Balaic}. These have been tentatively proposed in analogy to \citegen{Janhunen2012a} \ili{Ulchaic} subbranch of \ili{Nanaic} that includes \ili{Uilta} and \ili{Ulcha}. \ili{Manchuic} is a cover term for one variety of \ili{Jurchen} described during Ming dynasty \citep{Kane1989}, \ili{written Manchu} (including \ili{written Sibe}), and spoken \ili{Manchu} dialects recorded in Northeastern (e.g., \ili{Aihui}, \ili{Lalin}/\il{Manchu!Jing}Jing, \ili{Sanjiazi}, \ili{Yanbian}, or \ili{Yibuqi} Manchu) and Northwestern China (i.e., spoken \ili{Sibe}). Following \textcitetv{chapters/09}, this last group of Manchurian and Jungarian spoken \ili{Manchu} dialects that is closely related to the written language can be called \ili{Bannermen Manchu} (\textit{qiren manyu} \zh{旗人满语} in \ili{Chinese}).

\ili{Alchuka}, \ili{Bala}, and \ili{Chinese Kyakala}, although all three are sometimes referred to as “\ili{Manchu}”, do not seem to belong to \ili{Bannermen Manchu} (e.g., \citealt{MuYejun1987}; \citealt{Hölzl2017a}; \citealt{HölzlHölzl2019a}; \citetv{chapters/09}). They are characterized by several significant retentions and innovations in phonology, lexicon, and grammar. For instance, all three exhibit cases that lack the sound change \textit{p} > \textit{f} found in \ili{written Manchu} and all \ili{Manchu} dialects, e.g. \ili{Alchuka} \textit{p‘ut‘ia-mei}, \ili{Bala} \textit{p‘ut‘ihiaŋ-mi}, \ili{Manchu} \textit{fucihiya-mbi} ‘to cough’. Of the three languages, \ili{Alchuka} and \ili{Kyakala} could be more closely related, although the latter appears to show an additional substrate from \ili{Udegheic} or perhaps \ili{Nanaic}, e.g. the ocean spirit \textit{taimu} \zh{泰木} (\ili{Udihe} \textit{temu}, \ili{Nanai} \textit{temu}). \ili{Bala} seems to be intricately connected to another \ili{Jurchen} variety, but a comprehensive comparison and evaluation is still wanting (e.g., \citealt{Kiyose1977, Kiyose2000}; \citealt{MuYejun1987}). Both show a number of peculiarities that are otherwise rare or unattested in other \ili{Jurchenic} languages, e.g. \ili{Bala} \textit{asəi}, \ili{Jurchen} <asui> \zh{阿隨} ‘\textsc{neg.ex}’ (but \ili{Manchu} \textit{akū}). \ili{Bala} has an additional admixture from at least one non-\ili{Jurchenic} language, possibly \ili{Kilen} (e.g., the word for ‘name’, \citetv{chapters/04}). \ili{Alchuka}, \ili{Bala}, and \ili{Chinese Kyakala} furthermore show influence from \ili{Bannermen Manchu} or \ili{written Manchu} as well as complex dialectal and sociolectal variation that remain to be investigated. Together, these three varieties illustrate that the \ili{Jurchenic} branch of Tungusic is much more diverse and complex than many previous studies assumed. \ili{Alchuka}, \ili{Bala}, and \ili{Chinese Kyakala} exhibit archaic features that are highly relevant for the prehistory of Tungusic and the reconstruction of \ili{Jurchen}. Their significance cannot be emphasized enough and could be comparable to that of \ili{Chuvash} and \ili{Khalaj} among the \ili{Turkic} languages.


\section{Overview of this volume}

This volume is based on a workshop held in 2018 at the 51st Annual Meeting of the Societas Linguistica Europaea (SLE) in Tallinn. It includes studies presented at the workshop and a few newly submitted ones. Altogether, it contains eight contributions from ten different scholars and several different countries. All papers were reviewed by three to four people. The contributions cover all branches of Tungusic (Table \ref{Introtable:6}), a wide range of linguistic features, and very different opinions concerning the classification, reconstruction, and cultural background of Tungusic. Some of the contributions are based on first-hand data collected during fieldwork, in some cases from the last speakers of a given language (see \citetv{chapters/02} on \ili{Negidal}; \citetv{chapters/03} on \ili{Uilta}; \citetv{chapters/07} on \ili{Udihe} and \ili{Oroch}).

\begin{table}
\begin{tabular}{ l l }
  \lsptoprule
Tungusic languages	& Studies in this volume\\
  \midrule
\ili{Ewenic}	& Aralova \& Pakendorf; Khabtagaeva; Klyachko\\
\ili{Udegheic}	& Perekhvalskaya\\
\ili{Nanaic}	& Czerwinski\\
\ili{Jurchenic}	& Zikmundová\\
\ili{Pan-Tungusic}	& Hölzl; Robbeets \& Oskolskaya\\
  \lspbottomrule
\end{tabular}
\caption{An overview of the contributions in this volume}
\label{Introtable:6}
\end{table}

In their contribution entitled \textit{The causal-noncausal alternation in the \ili{Northern Tungusic} languages of Russia}, \textbf{Natalia} \textbf{Aralova} and \textbf{Brigitte} \textbf{Pakendorf} investigate causative constructions in three endangered \ili{Northern Tungusic} languages of the \ili{Ewenic} branch – \ili{Even}, \ili{Evenki}, and \ili{Negidal}. They look at morphological causative/non-causative alternations for 20 verbal meanings in the three languages. For each meaning, the possibilities are marked causative, marked non-causative, equipollence (both alternations marked), or zero marking. They find that equipollence is the dominant strategy in \ili{Even} and \ili{Negidal}, whereas in \ili{Evenki} the logical possibilities are more evenly distributed. This paper contributes significantly to ongoing theoretical discussions of the typology of voice and valence related constructions in the world’s languages.

Based on data drawn from published sources spanning over 100 years and fieldwork among the last five speakers of the \ili{Nanaic} language \ili{Uilta}, \textbf{Patryk} \textbf{Czerwinski} presents a concise and typologically informed overview of the tense system. In his contribution entitled \textit{Tense and insubordination in \ili{Uilta} (Orok)}, he emphasizes the role of insubordination and verbalization in the emergence of finite verbal categories in all three temporal domains (past, present, future) and illustrates differences between the Northern and Southern dialects. The study is an important contribution in the grammatical description of this critically endangered language and substantially adds to our understanding of diachronic processes in the verbal domain of Tungusic that can also be applied to many other languages.

In \textit{‘What’s your name?’ in Tungusic and beyond}, \textbf{Andreas} \textbf{Hölzl} investigates what is referred to as the personal name question (PNQ). The study that is inspired by Frame Semantics and Construction Grammar presents a detailed cross-linguistic analysis of the PNQ that forms the basis of the analysis of the question in Tungusic languages. He identifies two main types that make use of an equational copula (Type A) and a speech act verb (Type B), respectively. Based on a global sample of about 50 languages, he describes several dimensions of variation, such as the use of different interrogatives, the marking of possession, politeness, the presence or absence of a copula, the valency of the speech act verb, etc. Including data from all Tungusic languages, he shows that the PNQ in \ili{Proto-Tungusic} was of Type A and points out changes that have occurred in the individual languages through language contact.

The contribution by \textbf{Bayarma} \textbf{Khabtagaeva} entitled \textit{On some shared and distinguishing features of \ili{Nercha} and \il{Evenki!Khamnigan}Khamnigan Evenki dialects} is an addition to the author’s recent monograph \citep{Khabtagaeva2017}. The study compares data of the probably extinct \ili{Nercha} Evenki dialect \citep{Castrén1856} with modern data from \il{Evenki!Khamnigan}Khamnigan Evenki obtained through fieldwork and some of the available literature \citep{Janhunen1991}. It also includes comparative data from a wide range of other Tungusic languages. Through lexical and phonological similarities, she shows a close connection between the two varieties. For instance, she finds that the two varieties share the word \textit{düčin} ‘40’ of \ili{Mongolic} origin that has a different form or is entirely absent in other \ili{Ewenic} varieties. The study furthermore points out cases of lexical borrowing from different \ili{Mongolic} languages, \ili{Russian}, and \ili{Solon}.

Placeholder words are items that speakers use to signal that they don’t know or can’t remember the correct word for something. Examples in \ili{English} include “whatchamacallit” and “thingamajig”. In \textit{Functions of placeholder words in Evenki}, \textbf{Elena} \textbf{Klyachko} looks at placeholder words in terms of their morphological and syntactic behavior. In addition to providing valuable background information on \ili{Evenki} varieties, including their morphological characteristics, Klyachko’s study finds that placeholder words can substitute for items in almost any word class. As such they reflect the morphological character of the word they replace. A detailed discourse study of the use of placeholder words is included, showing that they have additional uses beyond the expected placeholder function. For example, they can be used as hesitation particles, and as discourse initiators.

\ili{Udihe} is a highly endangered group of Tungusic varieties spoken in the \ili{Russian} far east. Varieties of \ili{Udihe} are famous for their multiple series of vowels, including short, long, laryngealized and sometimes pharyngealized sets. \textit{From consonant to tone: Laryngealized and pharyngealized vowels in Udihe} by \textbf{Elena} \textbf{Perekhvalskaya} contains detailed discussion of the special political and sociolinguistic history of the various \ili{Udegheic} varieties. Valuable spectrographic data from all recorded varieties, including data on allegro vs. full modes of pronunciation, forms the core of Perekhalskaya’s contribution. One major conclusion is that inter-variety variation in vowel inventories is explained on the basis of contrasting prosodic patterns.

In \textit{\ili{Proto-Tungusic} in time and space}, \textbf{Martine} \textbf{Robbeets} and \textbf{Sofia} \textbf{Oskolskaya} address some of the fundamental and important problems of Tungusic linguistics concerning the age, original location, and classification. They summarize and discuss the results of a recent Bayesian analysis of the Tungusic languages \citep{Oskolskayaetal2022} that identifies a form of classification D as the most likely scenario but leaves the possibility of an early branching of \ili{Jurchenic} open. They assume a rough age of \ili{Proto-Tungusic} at the beginning of the first millennium CE. Based on the modern distribution of the Tungusic languages and comparison with recent results from archaeology and genetic analyses of modern and prehistoric populations, they argue for a location of the \ili{Proto-Tungusic} homeland somewhere around lake Khanka. They furthermore speculate that a hypothetical form of pre-\ili{Proto-Tungusic} might have been spoken by incoming farmers that interacted with the distant ancestors of the modern \ili{Nivkh} several millennia before \ili{Proto-Tungusic} times.

With 20,000 or more native speakers, the \ili{Jurchenic} language \ili{Sibe} is the only modern Tungusic language that is not yet seriously endangered. There is a long-standing controversy over the ethnic identity of the \ili{Sibe} people and the linguistic lineage of the \ili{Sibe} language. Some, mostly linguists and outsiders to the culture, consider the spoken language to be a variety of \ili{Manchu}. Others, in particular many \ili{Sibe} speakers, consider the language and culture to be distinct from \ili{Manchu}, arguing partly on the basis of a large number of words and concepts with clear origin in the \ili{Khorchin} Mongol language. In her contribution, \textit{Historical language contact between \ili{Sibe} and Khorchin}, \textbf{Veronika} \textbf{Zikmundová} investigates several \ili{Mongolic} features of \ili{Sibe} and concludes that indeed \ili{Sibe} is genetically closely related to \ili{Manchu}, but that the \ili{Mongolic} features can be explained on the basis of documented historical contact with \ili{Khorchin} Mongol in the 15th and 16th centuries CE.
  
% \section*{Abbreviations}
% \begin{tabularx}{.45\textwidth}{lQ}
% ... & \\
% ... & \\
% \end{tabularx}
% \begin{tabularx}{.45\textwidth}{lQ}
% ... & \\
% ... & \\
% \end{tabularx}

% \section*{Acknowledgements}

%\section*{Contributions}


{\sloppy\printbibliography[heading=subbibliography,notkeyword=this]}
\end{document}
