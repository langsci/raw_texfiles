\documentclass[output=paper,colorlinks,citecolor=brown]{langscibook}
\ChapterDOI{10.5281/zenodo.7053371}

\author{Elena Perekhvalskaya\affiliation{LLACAN CNRS, France}}
\title{From consonant to tone: Laryngealized and pharyngealized vowels in Udihe}

\abstract{This article gives a comprehensive analysis of laryngealized and pharyngealized vowels in the Udihe language. Their realization in different Udihe varieties is considered, and their etymology is traced. The classification of Udihe dialects is also discussed. The presence of pharyngalized vowels is one of the most important features that distinguishes the northern dialect cluster  from the southern one. The loss of pharyngealized vowels has led to changes in the morphology and syntax of the dialects of the southern cluster. %
% 
The analysis provides a basis for a complete picture of a dialectal continuum, which includes dialects of Udihe and the closely related Oroch language. The internal mechanisms of the dialectal continuum are presented, taking into account types of pronunciation in neighbouring varieties.}

\IfFileExists{../localcommands.tex}{
   \addbibresource{../localbibliography.bib}
   % add all extra packages you need to load to this file

\usepackage{tabularx,multicol}
\usepackage{url}
\urlstyle{same}

\usepackage{listings}
\lstset{basicstyle=\ttfamily,tabsize=2,breaklines=true}

\usepackage{langsci-basic}
\usepackage{langsci-optional}
\usepackage{langsci-lgr}
\usepackage{langsci-osl}
% \usepackage{./langsci/styles/langsci-lgr}
% \usepackage{./langsci/styles/langsci-osl}
% \usepackage{langsci-gb4e}

\usepackage{tikz}
\usetikzlibrary{patterns,calc}
\pgfdeclarepatternformonly{south east lines}{\pgfqpoint{-0pt}{-0pt}}{\pgfqpoint{3pt}{3pt}}{\pgfqpoint{3pt}{3pt}}{
    \pgfsetlinewidth{0.6pt}
    \pgfpathmoveto{\pgfqpoint{0pt}{3pt}}
    \pgfpathlineto{\pgfqpoint{3pt}{0pt}}
    \pgfpathmoveto{\pgfqpoint{.2pt}{-.2pt}}
    \pgfpathlineto{\pgfqpoint{-.2pt}{.2pt}}
    \pgfpathmoveto{\pgfqpoint{3.2pt}{2.8pt}}
    \pgfpathlineto{\pgfqpoint{2.8pt}{3.2pt}}
    \pgfusepath{stroke}}
    
\usepackage{stmaryrd}
\usepackage{wasysym}
\usepackage{multirow}
\usepackage{caption}
\usepackage{subcaption}
\usepackage{mathrsfs}
\usepackage{qtree}

\usepackage{linguex}


   %pminos do not split footnotes
% \interfootnotelinepenalty=10000 %Footnote in Laporte chapters has to be split SN


%\DeclareIndexNameFormat{default}{%
%\nameparts{#1}%
%\usebibmacro{index:name}%
%{\index[names]}%
%{\namepartfamily}%
%{\namepartgiveni}%
% {}% L1
% {}% L2
%{\namepartprefix}% generates spurious space L3
%{\namepartsuffix}% generates spurious space L4
%}

%  {\DeclareIndexNameFormat{default}{%
%     \usebibmacro{index:name}{\index[names]}{#1}{#3}{#5}{#7}}}

%\DeclareIndexNameFormat{default}{%
%  \usebibmacro{index:name}{\sindex[nom]}{#1}{#3}{#5}{#7}}

%\DeclareIndexNameFormat{default}{%
%  \usebibmacro{index:name}{\sindex[person]}{#1}{#3}{#5}{#7}}
%\DeclareIndexNameFormat{default}{%
%\nameparts{#1} \usebibmacro{index:name}{\sindex[person]]}{\namepartfamily}{‌​\namepartgiven}{\nam‌​epartprefix}{\namepa‌​rtsuffix}}

%\newcommand{\smiley}{:)}

%\renewbibmacro*{index:name}[5]{%
%\usebibmacro{index:entry}{#1}%
%{\iffieldundef{usera}{}{\thefield{usera}\actualoperator}\mkbibindexname{#2}{#3}{#4}{#5}}}

% \newcommand{\noop}[1]{}

%remove for final
%\overfullrule=1mm

\newcommand{\tobi}[2]}}
\renewcommand{\S}[1]{\tobi{#1}{\textsc{*}}}

% this volume references
% puts: [this volume]
% already defined: \citetv
%\newcommand{\citepv}[1]{(\citeauthor{#1} \citeyear*{#1} [this volume])}
\newcommand{\citealtv}[1]{\citeauthor{#1} \citeyear*{#1} [this volume]}

%parentheses around example number
\newcommand{\pref}[1]{(\ref{#1})}

% in-text examples

\newcommand{\lnex}[1]{\textit{#1}} %target lang word
\newcommand{\lnlit}[1]{(lit.: `#1')} %literal reading
\newcommand{\lnlat}[1]{(#1)} % latinization
\newcommand{\lntrans}[1]{`#1'} %translation
\newcommand{\lnexl}[2]%
{\lnex{#1}{} \lnlat{#2}} % ex with latinization
\newcommand{\lnexlat}[3]{\lnex{#1}{} \lnlat{#2}{} \lntrans{#3}} % ex with latinization and tranl.

%ch01
\newcommand{\co}[1]{\mbox{\textbf{#1}}}

%ch09

\newcommand{\cyrbulg}[1]{\begin{otherlanguage*}{bulgarian}#1\end{otherlanguage*}}


%ch10
\newcommand{\nlp}{{\small NLP}}
\newcommand{\mwe}{{\small MWE}}
\newcommand{\rae}{{\small RAE}}
\newcommand{\lvc}{{\small LVC}}
\newcommand{\pos}{{\small P}o{\small S}}
%\newcommand{\todo}[1]{ \textcolor{red}{#1} }

%\renewcommand{\labelenumi}{\theenumi}
%\ainamefmt{{vv}{ll}{, ff}{, jj}} % fullname

\newcommand{\biberror}[1]{{\color{red}#1}}

\newcommand{\osenovaitem}{--~}
   %% hyphenation points for line breaks
%% Normally, automatic hyphenation in LaTeX is very good
%% If a word is mis-hyphenated, add it to this file
%%
%% add information to TeX file before \begin{document} with:
%% %% hyphenation points for line breaks
%% Normally, automatic hyphenation in LaTeX is very good
%% If a word is mis-hyphenated, add it to this file
%%
%% add information to TeX file before \begin{document} with:
%% %% hyphenation points for line breaks
%% Normally, automatic hyphenation in LaTeX is very good
%% If a word is mis-hyphenated, add it to this file
%%
%% add information to TeX file before \begin{document} with:
%% \include{localhyphenation}
\hyphenation{
    Beck-man
    Ngu-yen
    back-chan-nel
    back-chan-nels
    mo-not-o-nous
    ste-reo-typ-i-cal
}

\hyphenation{
    Beck-man
    Ngu-yen
    back-chan-nel
    back-chan-nels
    mo-not-o-nous
    ste-reo-typ-i-cal
}

\hyphenation{
    Beck-man
    Ngu-yen
    back-chan-nel
    back-chan-nels
    mo-not-o-nous
    ste-reo-typ-i-cal
}

   \boolfalse{bookcompile}
   \togglepaper[7]%%chapternumber
}{}

\begin{document}
\maketitle

\section{Introduction}

\ili{Udihe} (Udeghe, Udege) is a highly endangered Manchu-Tungusic language spoken in the southern part of the \ili{Russian} Far East. The \ili{Udihe} live in Khabarovskij Krai (districts: Imeni Lazo, Nanaisky) and Primorskij Krai (districts: Terneiskij, Požarskij, Krasnoarmejskij), and also in the Jewish Autonomous Region. The original name is \textit{Udihe} or \textit{Udie}.\footnote{Until the 1920s, the \ili{Udihe} did not have a common self-designation but used clan names, usually derived from names of rivers.} The official \ili{Russian} name is \textit{Udegeiskij jazyk}.\footnote{The name \textit{Udege} is the transmission of the self-designation \textit{Udihe}: the pharyngealized element was perceived as a consonant /γ/ and was written down with the Cyrillic letter <г>, which reflected the local \ili{Russian} pronunciation of /γ/ as a fricative consonant. In literary \ili{Russian} <г> denotes a plosive consonant. “The \ili{Russian} form \textit{Udege} is based, in a certain sense, on a phonetic misunderstanding” \citep[5]{Kormushin1998}.} In linguistic literature it is also known as \textit{udeiskij} (Evgenij Šneider \citeyear{Shneider1936}) and \textit{udyxeiskij} (Igor Kormušin \citeyear{Kormushin1998}). In the 2010 census \ili{Udihe} named themselves as: \textit{Udie}, \textit{Ude}, \textit{Udegeitsy}, \textit{Udexe}, and \textit{Udexeitsy}.

According to the latest censuses\footnote{For an analysis of the census data, see \citet{Perekhvalskaya2016}.} (1989, 2002 and 2010), the number of \ili{Udihe} is constantly decreasing from 1,902 in 1989 to 1,496 in 2010. In 2010, 620 people were registered in the Khabarovskij Krai; 793 people lived in the Primorskij Krai. An additional 83 \ili{Udihe} were registered outside of these territories, including 42 people in the Jewish Autonomous Region. The census data also reflect the steady decline of the language: according to the 1989 census, \ili{Udihe} was spoken by 462 people, in 2002 it was 227 people, and in 2010 it was only 103 people. The 2010 census shows a sharp drop in the \ili{Udihe} competence in the Khabarovskij Krai (from 96 to 16 people).

Traditionally, the \ili{Udihe} were semi-nomads, moving within a limited territory, each along a particular river and its tributaries, thereby forming territorial groups which usually consisted of several families. The territorial groups are mostly named after the corresponding rivers: (1) \ili{Kur-Urmi}, (2) \ili{Samarga}, (3) \ili{Anjuj}, (4) \ili{Xungari}, (5) \ili{Xor}, (6) \ili{Bikin}, (7) \ili{Iman}, and (8) Sea shore (Namunka). In the 1930s, the \ili{Udihe} were compelled to become sedentary: each territorial group was settled in a specially built permanent settlement: Kukan (\ili{Kur-Urmi}), Bira (\ili{Anjuj}), Kun (\ili{Xungari}), Agzu (\ili{Samarga}), Gvasjugi (\ili{Xor}), Sjain, Mitaxeza and Olon (\ili{Bikin}), Sančixeza (\ili{Iman}). The less numerous Sea shore \ili{Udihe} were dispersed. At present, the largest \ili{Udihe} settlements are: Agzu (Terneiskij district), constituting about 80\% of the population of the village; Gvasjugi (Imeni Lazo district; 65\% of the population), Krasnyj Jar and Olon (Požarskij district; 55\% of the population). Each territorial group is characterized by a specific language variety. Dialectal differences primarily concern phonetics and vocabulary, and to a lesser extent morphology and syntax.

Morphologically, \ili{Udihe} is an agglutinative language; the agglutination is combined with elements of fusion mainly in verb paradigms. The root, both nominal and verbal, always occupies the extreme left position in a word; it is followed by derivational and inflectional suffixes, which form a chain that can number up to six or seven (in the case of verb forms). In addition to synthetic forms the verb system contains analytic constructions with auxiliary verbs. The verbal negative construction consists of a negative verb and a main verb without any specific con-negative suffixes (for more details see \citealt{Hölzl2015}).

The peculiarity of \ili{Udihe} inside the Manchu-Tungusic group is largely due to its phonetics and phonology, primarily the existence of several series of vowels. The northern dialect cluster has four series of vowels: short, long (including diphtongoids), pharyngealized and laryngealized; the southern dialect cluster has three series: short, long and laryngealized. The phonological interpretation of these vowels is controversial (see \citealt{NikolaevaTolskaya2001}: 39--41).

The present article contains a comprehensive analysis of these vowels in \ili{Udihe} dialects. It is shown that they developed out of tri-phonemic complexes of the V-C-V type, which are found in the closely related \ili{Oroch} language.

When considering complex vowel phonemes, the phonological system of each territorial variety (dialect) is regarded as independent \citep{Trudgill1985}. In each variety the full mode and the allegro modes of pronunciation are taken into account, which makes it possible to show that, roughly, the allegro mode of one variety corresponds to the full pronunciation mode of another variety which, in turn, creates a new allegro mode, etc.

The objectives of the article are 1) to give an overview of \ili{Udihe} dialects and their clusters; 2) to display the anatomy of the “dialect continuum” by comparison of the modes of pronunciation in each territorial variety; 3) to show the relative character of the synchrony/diachrony dichotomy in a language description; 4) to demonstrate one of the mechanisms of tonogenesis in a previously atonal language. 


\section{The Udihe: Areal groups and dialects}

\subsection{Udihe and Oroch}

The \ili{Udihe} language area borders with \ili{Nanai}, \ili{Ulcha} and Ewenki\il{Evenki}, as well as, historically, \ili{Manchu} dialects. \ili{Udihe} had rather intensive contacts with these languages. Thus, \ili{Kur-Urmi} Udihe situated in traditional Ewenki\il{Evenki} territory underwent significant influence of the latter. \ili{Bikin} Udihe and Bikin\il{Ussuri} \ili{Nanai} (\ili{Kilen}) acquired a number of similar features (\citealt{Perekhvalskaya2001}). The linguistic border between \ili{Udihe}, on one hand, and \ili{Nanai} or Ewenki\il{Evenki}, on the other, are clear cut. Neither speakers nor linguists hesitate in attributing a variety to one or the other of these languages.

The situation of \ili{Udihe} and \ili{Oroch} is different.\footnote{In addition, there is \ili{Kilen} on the \ili{Chinese} side, which has been heavily influenced by \ili{Udihe} or \ili{Oroch} (for details see \citealt{Hölzl2018b}). \ili{Negidal}, most probably, also had an \ili{Oroch} substrate (\citealt{Pevnov2012}).} As there are no definite linguistic criteria for distinguishing “language” and “dialect”, it is worth considering the ethnic identity of \ili{Udihe} and \ili{Oroch} speakers. 

Traditionally, the \ili{Oroch} lived along the sea coast and the \ili{Tumnin} river. Their territory borders the \ili{Anjuj} and \ili{Xungari} Udihe area in the West and the \ili{Samarga} Udihe in the South (see Figure \ref{fig:7:1}). Culturally, the \ili{Udihe} and \ili{Oroch} are rather close. While the \ili{Nanai}, who lived along large rivers Amur and Ussuri, were mainly fishermen, the \ili{Udihe} and \ili{Oroch} travelled along small taiga rivers, their main occupation being hunting; fishing and gathering were secondary occupations. Their neighbours in the North, the Ewenki\il{Evenki}, were reindeer breeders; neither \ili{Oroch} nor \ili{Udihe} were engaged in breeding.

\begin{figure}
\includegraphics[width=0.9\textwidth]{figures/Udihe_Oroch_Kyakala.pdf}
\caption{Udihe territorial groups\label{fig:7:1}}
\end{figure}

Previously, the \ili{Udihe} and \ili{Oroch} as well as other local ethnic groups had “clan identity”. “…the ethnonym \ili{Udihe} (Udie) has been used since the 1930s. Before there was no general ethnic designation. Each areal group had its own self nomination: \textit{huŋgakə} on \ili{Xungari}, \textit{bikiŋkə} on \ili{Bikin}, \textit{uniŋka} on \ili{Anjuj} and so on” (\citealt[142]{Suliandzigaetal2003}). The \ili{Oroch} had no general ethnonym either. The \ili{Udihe} call them \textit{namuŋka} ‘sea shore dwellers’. This name was also used for the \ili{Udihe} living along small rivers that flow into the sea further to the south, so \ili{Oroch} clans were not distinguished from the \ili{Udihe}.

The first reseachers did not separate the \ili{Udihe} and the \ili{Oroch}, which, apparently, reflected the real state of affairs. In the absence of a common self-designation, these people\slash peoples were called \textit{Orochen} (\textit{\ili{Orochon}}) by the Russians. This name was given to the indigenous population living along the coast of the Tatar Strait and the Sea of Japan, by Jean-François de Lapérouse (\citealt[142]{Shrenk1883}). This term is essentially erroneous, since it goes back to the Manchu-Tungusic word for reindeer, \textit{oro(n)}. Neither the \ili{Udihe} nor the \ili{Oroch} were engaged in reindeer herding. Nevertheless, this ethnonym was used for some time.\footnote{The term \textit{\ili{Orochon}}, referring to both the \ili{Udihe} and the \ili{Oroch} together, was used in all geographical, statistical, and other documents of the late 19th and early 20th centuries (see, for example, \citealt{Shrenk1883}; \citealt{Nadarov1887}; \citealt{Margaritov1888}; \citealt{Protodjakonov1888}; \citealt{Przevalskij1990}). It is worth mentioning that in \ili{Iman} this designation is still used referring to the \ili{Udihe}, being perceived as pejorative.} The \textit{\ili{Orochon}} were considered a separate ethnic group, along with the Gold (i.e., \ili{Nanai}), Tungus (i.e., Ewenki\il{Evenki} and sometimes \ili{Even}) or Gilyak (i.e., \ili{Nivkh}).

In the modern scientific literature, the term \textit{Udihe} appears for the first time in Sergej Brailovskij’s work (\citealt{Brailovskij1901}). He used the autonym of one of the groups of northern \ili{Udihe}. Brailovskij also introduced the term \textit{Tazy}\footnote{The term \ili{Tazy} goes back to \ili{Chinese} \zh{鞑子} \textit{dázi} ‘local resident of Primorye’; the word was already attested many hundreds of years ago in \ili{Chinese} sources (\citealt[116]{Hölzl2018b}). \ili{Tazy} is an ethnic group of Tungus-Manchu origin who have lost their native language and use a northern dialect of \ili{Chinese}. \ili{Tazy} were settled in the village of Mikhailovka, Olginskij district; about the \ili{Tazy} language situation, see \citet{BelikovPerekhvalskaya1994}.} as a synonym for \ili{Udihe}. However, he did not separate the \ili{Udihe} and the \ili{Oroch}, and used the term \textit{\ili{Oroch} – Udihe}, and \textit{Tazy} as synonymous. In the late 1920s, the campaign to change ethnonyms of \ili{Russian} minorities was launched in the country. Old ethnonyms were assumed to be derogatory and were replaced usually by self-designations of respective peoples. Thus, Gold became \ili{Nanai}, Gilyak became \ili{Nivkh}, Tungus became Ewenki\il{Evenki}, Lamut became \ili{Even}, etc. The Orochon were divided into three groups: \ili{Oroch}, \ili{Udihe}, and \ili{Tazy}. This subdivision was apparently worked out by the famous geographer Vladimir Arseniev (\citealt{Arseniev1947}), who worked in the area.

This division is now universally recognized, and these ethnonyms are included in the list of \ili{Russian} minorities. They were also recorded in Soviet passports as “nationality”. At present, when these languages are on the verge of extinction, and people themselves firmly know their “nationality”, this separation became reality. Still, the question arises how these idioms actually correlate.

The first dictionaries and other linguistic data on \ili{Udihe} (\citealt{Protodjakonov1888}; \citealt{Leontovich1898}; \citealt{Nadarov1887}; \citealt{Margaritov1888}; \citealt{Schmidt1928}), as well as a generalizing work of Brailovskij (\citealt{Brailovskij1901}), did not separate \ili{Oroch} and \ili{Udihe} words.\footnote{There are newly found data on early \ili{Oroch} (\citealt{Alonso2017}).} However, Brailovskij compared the data that he personally collected with words of other territorial groups, and came to the conclusion that the southern \ili{Udihe} clans which had undergone \ili{Chinese} influence were different from other groups. He combined northern \ili{Udihe} (in modern terminology) and \ili{Oroch}. At the same time, Brailovskij insisted on the cultural and linguistic unity of all “\ili{Oroch}-\ili{Udihe}”. The same was the point of view of Peter Schmidt (\citealt{Schmidt1928}). The anthropologist Viktor Lar’kin also considered \ili{Oroch} and \ili{Udihe} two dialects of the same language, and divided \ili{Udihe} into several sub-dialects (\citealt[5]{Lar’kin1959}). \ili{Udihe} and \ili{Oroch} have been considered separate languages since the 1930s, beginning with works by Evgenij Šneider (\citealt{Shneider1936, Shneider1937}), Valentin Avrorin and Elena Lebedeva (\citealt{AvrorinLebedeva1978}).

Regardless of whether \ili{Udihe} and \ili{Oroch} should be considered closely related languages or distant dialects of the same language, the fact remains that their territorial varieties form a dialect continuum. The \ili{Xadi} (coastal) variety of \ili{Oroch} is close to northern \ili{Udihe}. The frontier dialect (\ili{Koppi} variety) is described as either the most southern dialect of \ili{Oroch} (\citealt{AvrorinLebedeva1978}), or the most distant dialect of \ili{Udihe} (\citealt{Kormushin1998}). In fact, here the “official” border between \ili{Oroch} and \ili{Udihe} just coincides with the administrative border between the Khabarovskij and Primorskij Krai. Since the mouth of the \ili{Koppi} river administratively is a part of the Khabarovskij Krai, local “Orochons” received the passport designation “\ili{Oroch}” and are officially the \ili{Oroch}.
Until recently, the linguistic position of the \ili{Koppi} variety remained unclear. In 2010, together with Natalia Kuznetsova, we conducted a study of the \ili{Koppi} variety. Based on these data, I came to the conclusion that the \ili{Koppi} variety combines features of \ili{Oroch} and \ili{Udihe}, being a transition from the northern dialects of \ili{Udihe} to coastal varieties of \ili{Oroch}. However, it shows more properties characteristic of \ili{Oroch}. One of the main features being the preservation of etymological V-q-V and V-h-V complexes.


\subsection{Udihe and Kekar (Kyakala)}\largerpage

Previously, the \ili{Udihe} were also known as \textit{Kekar} (\textit{Kyakala}, \textit{Kyakar} or \textit{Kiyakara} in \ili{Manchu}). “The \ili{Oroch} call them \textit{Ude} or \textit{Kekar}, they call \ili{Oroch} \textit{Pæ}” (\citealt{Emelianov1927}). However, Ude and \ili{Kekar} were not used as complete synonyms and referred not to one and the same but to two closely related ethnic groups. In 1998, Igor Kormušin wrote:

\begin{quote}\sloppy
Anthropologist Paul Schmidt in 1915 mentioned a remarkable fact, which did not attract due attention. Classifying the Manchu-Tungusic ethnic groups, he wrote that \ili{Oroch} consist of three tribes: \ili{Oroch}, Kyakar and \ili{Udihe}. The term «Kyakar» [...] is preserved in \ili{Udihe} in the form \textit{kǣ’}\footnote{Note that Kormušin used the apostrophe to mark laryngealized vowels after (not before) the character: \textit{kǣ}’ (Shn. \textit{k’eæ}, Sim. \textit{Ki’a}).}  (< \textit{keka(r)}).\footnote{Janhunen has argued that it goes back to the word for ‘edge’, *kira > kija > kae (\citeyear{Janhunen2012a}). However, \textit{K’eæ} ‘clan \textit{Kae’} and \textit{keæ} ‘edge’ are not homonyms. \textit{K’eæ} contains the laryngealized /’eæ/ which points to the historical change VqV > VʔV > V’V. It is mostly probable that the sequence *keka transformed into \textit{k’eæ}.} As the legend says, there was also a legendary clan of the same name, which branched into several \ili{Udihe} clans, localized mainly along the southern sea coast: \textit{Amuliŋka}, \textit{Geuŋka}, etc. If one takes into account that the \textit{Udi}\footnote{\textit{udi} might be a word from the \ili{Manchu} branch of Tungusic: \ili{Manchu} \textit{weji}, \ili{Alchuka} \textit{udi}, \ili{Bala} \textit{udi} ‘forest’. It seems there is no other Tungusic language that has a cognate of this word \citep[121--122]{Hölzl2018b}.} clan participated in the formation of \ili{Ulcha}, and therefore should be localized much further to the north, then one should conclude that \textit{Udi} and the \textit{\ili{Kekar}} correspond to the «Northern» and the «Southern» components of the \ili{Udihe} ethnos respectively... (\citealt[11--12]{Kormushin1998}, my translation – E.P.)
\end{quote}

The anthropologist Anatoliy Startsev suggested that initially there were three \ili{Udihe} clans: \textit{Udie}, \textit{Kæ} and \textit{Piaŋka} (\citealt{Startsev2004}). According to \citet{Lar’kin1959} the large \textit{Kæ} clan divided into several clans: \textit{Kančuga} (Kancuga), \textit{Geonka}, \textit{Kuinka} and \textit{Suanka}. It is worth pointing out that in \ili{Xor} (“\ili{Udihe} proper”) there were only two clans: \textit{Kjalundzjuga} (Kælunǯuga) and \textit{Kimonko} (Kimoŋko). The clan names \textit{Kančuga}, \textit{Geonka}, \textit{Kuinka}, \textit{Suanka} are usual among the \ili{Bikin} Udihe and \textit{K’æ} was registered in \ili{Iman}.

It may be concluded that two distinct groups, \ili{Udihe} and \ili{Kekar}, were classified as one “nation” which is now called \ili{Udihe} (or Udeghe, Udie). \textit{Udihe} corresponds to the northern dialect cluster (\ili{Xor} and \ili{Anjuj} varieties); \textit{\ili{Kekar}} corresponds to the southern dialect cluster (\ili{Bikin}, \ili{Iman} and \ili{Samarga} varieties). Very roughly, it can be said that northern dialects (“\ili{Udihe}”) are closer to \ili{Oroch}.


\subsection{Udihe areal groups and dialects}

\subsubsection{Overview}

Traditionally, the \ili{Udihe}, being semi-nomads were spread across a fairly large territory: about a thousand kilometers from north to south. First researchers, geographers, and anthropologists (see, for example, \citealt[V, 81]{Arseniev1947}) indicated that dialectal differences in \ili{Udihe} were so significant that the \ili{Udihe} from different territorial groups hardly understood each other. However, modern studies showed that, with all the differences, the \ili{Udihe} dialects are mutually intelligible (\citealt{Simonov1988}; \citealt{Perekhvalskaya2010}). Still, differences between \ili{Udihe} dialects are not insignificant, and the mutual understanding between the dialects does not mean that they have identical systems (\citealt[21--23]{Trudgill1985}).

Traditionally, \ili{Udihe} dialects were named according to the river basins where they were spoken. Hunter-gatherer groups roamed within the basin of one river and acquired their specific language variety. The language of a larger areal group, however, was not uniform. Thus, \ili{Udihe} clans living along the \ili{Bikin}-river occupied specific smaller areas (along smaller rivers), and their language had specific features. There are still differences in the speech of those who came from the camps of Mitaxeza, Sjain, Olon, Sigou, Ulunga, Toholo, etc.

By the beginning of the 20th century, there were the following \ili{Udihe} groups (Table \ref{table:7.1}): \ili{Kur-Urmi}, \ili{Xor}, \ili{Anjuj}, \ili{Xungari} (now Gur), \ili{Samarga}, \ili{Bikin}, \ili{Iman} (now Bolshaja Ussurka).

\begin{table}
\begin{tabular}{ l l l }
  \lsptoprule
Cover terms & Subgroups & Dialects\\
  \midrule
 & &	\ili{Kur-Urmi}\\
\cmidrule(lr){2-3}
 & &	\ili{Xor}\\
 & Northern (\ili{Udihe}) &	\ili{Anjuj}\\
\ili{Udihe} & &  (\ili{Xungari})\\
\cmidrule(lr){2-3}
 & &	\ili{Iman}\\
 & Southern (\ili{Kyakala}) &	\ili{Bikin}\\
 & &	\ili{Samarga}\\
\midrule
 & &	\ili{Koppi}\\	
\ili{Oroch} & &	\ili{Xadi}\\
 & &	\ili{Tumnin}\\
  \lspbottomrule
\end{tabular}
\caption{Udihe and Oroch dialects}
\label{table:7.1}
\end{table}

In the 1930s, the \ili{Udihe} were forcibly made sedentary: each areal group was settled in a specially built permanent settlement: Kukan (\ili{Kur-Urmi} Udihe), Bira (\ili{Anjuj}), Kun (\ili{Xungari}), Agzu (\ili{Samarga}), Gvasjugi (\ili{Xor}), Sjain, Mitaxeza and Olon (\ili{Bikin}), Sančixeza (\ili{Iman}). The less numerous sea-shore \ili{Udihe} were dispersed. In the 1960s and 70s in the course of the “consolidation of villages” campaign, smaller \ili{Udihe} villages were liquidated: Bira (\ili{Anjuj}), Sančixeza (\ili{Iman}), Sjain and Mitaxeza (\ili{Bikin}). The \ili{Bikin} Udihe resettled in the new \ili{Udihe} village of Krasnyj Jar; and \ili{Anjuj} and \ili{Iman} Udihe were resettled into neighboring \ili{Russian} villages. Therefore, a significant number of the \ili{Udihe} were dispersed and came into daily contact with speakers of \ili{Russian}. In the late 1930s the \ili{Kur-Urmi} Udihe village of Kukan became a place of exile of political prisoners. After the building of the Khabarovsk-Sovgavan’ railway, Kun, the village of the \ili{Xungari} Udihe, became a railway station. Soon the \ili{Udihe} were an insignificant part of the population in these villages.

At present, the largest \ili{Udihe} settlements are: Agzu (Terneiskij district), where they constitute about 80\% of the population of the village; Gvasjugi (Imeni Lazo district): 65\% of the population, Krasnyj Jar and Olon (Požarskij district): 55\% of the village population.

Each territorial group was characterized by a specific language variety. From a linguistic point of view, there are significant similarities between the \ili{Iman} and \ili{Bikin} dialects, on the one hand, and between the \ili{Xor} and \ili{Anjuj} dialects, on the other. They form the northern \ili{Udihe} dialect cluster (\ili{Xor} and \ili{Anjuj} varieties), and the southern \ili{Udihe} cluster (\ili{Bikin} and \ili{Iman} varieties). \ili{Samarga} displays mixed features; however, it seems to be historically closer to the southern (\ili{Kekar}) group. As for the \ili{Kur-Urmi} dialect, it was heavily influenced by Ewenki\il{Evenki}. Orest Sunik expressed the idea of the proximity of \ili{Samarga} and \ili{Xungari} varieties (\citealt[231]{Sunik1968}). According to Sunik, three dialect groups were distinguished in \ili{Udihe}: \ili{Iman}-\ili{Bikin}, \ili{Xor}-\ili{Anjuj} and \ili{Samarga}-\ili{Xungari}. This statement cannot be verified because the \ili{Xungari} variety has been completely lost and no data on it were published. From a purely a geographic point of view, the \ili{Xungari} dialect should be placed in the northern cluster. Therefore, I will contrast the northern group (\ili{Anjuj}, \ili{Xor}) and the southern group (\ili{Bikin}, \ili{Iman}, \ili{Samarga}) (\ref{tree}).

\begin{figure}
\caption{\label{tree}Classification of Udihe and Oroch varieties}
\begin{forest}  for tree={grow'=east},   forked edges
[
    [{Oroch}
        [{Koppi}]
        [{Xadi}]
        [{Tumnin}]
    ]
    [{Udihe}
        [Northern\\(Udihe),align=center
            [{Anjuj}]
            [{Xor}]
        ]
        [{Kur-Urmi}]
        [Southern\\(Kyakala),align=center
            [{Samarga}]
            [, delay={where content={}{shape=coordinate}{}}[{Bikin}]
             [{Iman}]
            ]
        ]
    ]
]
\end{forest}
\end{figure}
\il{Oroch}
\il{Koppi}
\il{Xadi}
\il{Tumnin}
\il{Udihe}
\il{Anjuj}
\il{Xor}
\il{Kur-Urmi}
\il{Kyakala}
\il{Samarga}
\il{Bikin}
\il{Iman}
% \begin{figure}
% \includegraphics[width=0.6\textwidth]{figures/Perekhvalsakya2.png}
% \caption{}
% \label{fig:7:2}
% \end{figure}


\subsubsection{Dialect continuum}

The linguistic reality is more complicated than the division of language into two dialect clusters. Territorial varieties of \ili{Udihe} and \ili{Oroch} form a “dialect continuum”. Neighboring varieties are linguistically rather close to each other, while the extreme points show significant differences. Moving from one variety to another, one can observe the gradual loss of certain linguistic features and the appearance of other features. This concerns all language levels: phonetics and phonology, morphology and syntax, vocabulary. In this article I will focus on the concrete realization of complex vowels in different varieties of \ili{Udihe}.

\section{Data}

This work is based on the study of the following data:

\begin{enumerate}
    \item Dictionaries

    \begin{itemize}
        \item Dictionaries of \ili{Oroch} (\citealt{Schmidt1928}; \citealt{AvrorinLebedeva1978});
        \item Dictionaries of \ili{Udihe} (\citealt{Nadarov1887}; \citealt{Shneider1936}; \citealt{Kormushin1998}; \citealt{SimonovKialundziuga1998}; \citealt{Girfanova2001}).
        \item Tungus-Manchu Comparative Dictionary. Materials for the etymological dictionary (\citealt{Cincius1975P}).
    \end{itemize}


    \item Descriptions

    \begin{itemize}
        \item Linguistic descriptions of \ili{Udihe} varieties (\citealt{Shneider1936}; \citealt{Sunik1968}; \citealt{Simonov1988}; \citealt{Kormushin1998}; \citealt{Nikolaeva2000}; \citealt{NikolaevaTolskaya2001}; \citealt{Hölzl2018b}).
    \end{itemize}


    \item Field data gathered in a number of field trips

    \begin{itemize}\sloppy
        \item \ili{Tumnin} and \ili{Xadi} variety of \ili{Oroch}, Khabarovskij Krai (2001, 2010); main speakers: Anatolij Namunka, Inna Akunka.
    
        \item \ili{Xor} variety, Khabarovskij Krai (1989, 2011, 2013, 2014, 2018); main speakers: Valentina Kjalundzjuga, Nelli Andreeva (Kimonko), Polina Sun (Kjalundzjuga), Tatjana Dvoinova (Sulaindziuga).
    
        \item \ili{Koppi} variety, Khabarovskij Krai (2006, 2010); main speaker Alexandr Ivaščenko (Pudza).
    
        \item \ili{Samarga} variety, Primorskij Krai (1989, 2010, 2011); main speakers: Anatolij Kamandiga, Tatjana Kaza.
    
        \item \ili{Bikin} variety, Primorskij Krai (more than 10 trips in the period 1984--2011); main speakers Aleksandr Kančuga, Nadežda Kukčenko, Aleksandr Pionka, Lydia Simanchuk, Susan Geonka, Anna Kančuga, Andrej Suanka, Evdikija Kjalundziga.
    
        \item \ili{Iman} variety, Primorskij Krai (1989, 2010); main speakers: Ksenia Ajanka, Antonina Suanka, Aleksandr Kjalundziga.
    
        \item Field data were gathered with the help of Vladimir Belikov, Fatima Eloeva, Albina Girfanova, Irina Nikolaeva, Maria Tolskaya, Vlada Baranova, Kirill Maslinskij, Ksenia Viktorova, Valentin Vydrin, Natalia Kuznetsova. 
    \end{itemize}

\end{enumerate}


\section{Pharyngealized and laryngealized vowels in Udihe varieties}

\subsection{Udihe vowels}

The peculiarity of \ili{Udihe} inside the Manchu-Tungusic group is largely due to its phonetics and phonology, and especially the existence of several series of vowels. Laryngealized (glottalized) and pharyngealized (aspirated) vowels are features that clearly distinguish \ili{Udihe} from \ili{Oroch} and other Manchu-Tungusic languages (\citealt[58]{Zinder1948}; \citealt{Cincius1949}). However, the pharyngealized vowels in \ili{Udihe} prove to be less stable than the laryngealized ones. \ili{Bikin} and \ili{Iman} varieties have lost pharyngealized vowels completely, while in \ili{Samarga} they are kept only in some root morphemes. They are fully preserved only in the \ili{Xor} dialect. On the contrary, laryngealized vowels are preserved in all varieties, although their specific realization may differ significantly (\citealt{Shneider1936}; \citealt{Simonov1988}; \citealt{Nikolaeva2000}). In fact, the concrete realization of laryngealized vowels is one of the important features which distinguish \ili{Udihe} dialects.


One of the most significant features that distinguish northern and southern varieties is the lack of pharyngealized long vowels in southern \ili{Udihe} (\ref{example7.2}):

\ea
    \label{example7.2}
\begin{tabular}{ l l l l }
& ‘fire’ & & ‘button’\\
\ili{Xor} &    \textit{tō} & {\textasciitilde} & \textit{toʰo}\\
\ili{Bikin} &	\textit{tō} & {\textasciitilde} & \textit{tō}\\
\end{tabular}
\z


\begin{table}[h]
\begin{tabular}{ l l l l l l l l l l }
  \lsptoprule
Description & \multicolumn{9}{l}{Vowels}\\
  \midrule
Short &	a & o & u & ə & i & e & æ & ɵ & (y)\\
Long & ā &	ō & ū	& ə̄ & ī &	iə & eæ & yɵ & yi\\
Long laryngealized & ‘a & ‘o & & (‘ə)\footnote{Laryngealized /’ə/ is postulated by Nikolaeva and Tolskaya on the basis of one verb form: the perfect stem for verbs of the type \textit{ətətə-} ‘to work’ – \textit{ətət’ə} ‘he has worked’ (\citealt[40]{NikolaevaTolskaya2001}). However, this phoneme has a very narrow scope: it does not occur in any other position.} & & & & & \\
Long pharyngealized & aʰa & oʰo & uʰu & əʰə & iʰi & (iʰe) & eʰæ & yʰɵ & yʰi\\
  \lspbottomrule
\end{tabular}
\caption{Vowel inventory of the Xor and Anjuj varieties (\citealt[83--86]{Shneider1936}; \citealt{Simonov1988})}
\label{table:7.2}
\end{table}


\begin{table}[h]
\begin{tabular}{ l l l l l l l l l l }
  \lsptoprule
Description & \multicolumn{9}{l}{Vowels}\\
  \midrule
Short &	a & o & u & ə & i & e & æ & ɵ & (y)\\
Long & ā &	ō & ū	& ə̄ & ī &	ē & ǣ & ɵ̄ & \\
Long laryngealized & ‘a & ‘o & & (‘ə) & & & & & \\
  \lspbottomrule
\end{tabular}
\caption{Vowel inventory of the Bikin, Iman and Samarga varieties}
\label{table:7.3}
\end{table}

\subsection{Interpretation of the Udihe complex vowels}

The interpretation of the \ili{Udihe} vocalic complexes, laryngealized and pharyngealized, has long been controversial. Trubetskoi’s phonology counts several dozen vowel phonemes in \ili{Udihe}, as it was presented by \citet{Shneider1936}; in some other works the phonemic status of these complex vowels is not clarified (\citealt{Nikolaeva2000}).

I regard suprasegmentals as an independent tier (\citealt{Goldsmith1976}). Therefore, I postulate the appearance of the suprasegmental tier as a compensation for the simplification of the segmental tier. In \ili{Xor}, \ili{Bikin} and \ili{Iman} varieties it was the representation of the suprasegmental tier that underwent changes while the segment tier remained unchanged. The concrete realizations of pharyngealized and laryngealized long vowels differ according to the variety and also according to the mode of speech.


\subsection{Pharyngealized long vowels}

\subsubsection{Etymology}

\ili{Udihe} pharyngealized vowels go back to a combination of three phonemes, V-s-V. In root morphemes:\footnote{Hereinafter, the correspondences are given according to \citet{Cincius1975P}.}

\ea
    \label{example7.3}
\ili{Udihe} \textit{aʰanta} ‘woman’ {\textasciitilde} \ili{Oroch} \textit{asa}, Pl. \textit{asa-nta}; \ili{Negidal} \textit{asīxān} ‘girl’; Ewenki\il{Evenki} \textit{asī} ‘female’;
\z

\ea
    \label{example7.4}
\ili{Udihe} \textit{aʰikta} ‘fir-tree’ {\textasciitilde} \ili{Oroch} \textit{asikta}, Ewenki\il{Evenki} \textit{asikta}, \textit{acēkān,} \ili{Nanai} \textit{xasikta} ‘fir-tree’;
\z

\ea
    \label{example7.5}
\ili{Udihe} \textit{iʰi} ‘larch tree’ {\textasciitilde} \ili{Oroch} \textit{isi}, \ili{Negidal} \textit{isi}, \ili{Manchu} \textit{isi}, Ewenki\il{Evenki} \textit{ismama} ‘dry larch tree’;
\z

\ea
    \label{example7.6}
\ili{Udihe} \textit{oʰɵ, woʰɵ} ‘deer-leg fur’ {\textasciitilde} \ili{Oroch} \textit{ōso}, \ili{Negidal} \textit{ōsa}, Ewenki\il{Evenki} \textit{ōsa} ‘deer-leg fur’;
\z

\ea
    \label{example7.7}
\ili{Udihe} \textit{uʰəndə, wuʰəndə} ‘to throw’ {\textasciitilde} Ewenki\il{Evenki} \textit{us\={ə}nd\={ə}} ‘to throw (weapon)', \ili{Negidal} \textit{us{ə}} ‘spear’.
\z

\noindent In suffixes:

\ea
    \label{example7.8}
\ili{Udihe} \textit{b{ə}liʰ{ə}} ‘girl (in tale)’ < \textit{*b{ə}li-s{ə}} {\textasciitilde} \ili{Negidal} \textit{b{ə}lin}, Ewenki\il{Evenki} \textit{b{ə}lin} ‘hysteria’;
\z

\ea
    \label{example7.9}
\ili{Udihe} \textit{tumiʰ{ə}} ‘path in mountains’ < \textit{*tumi-s{ə}} {\textasciitilde} \ili{Negidal} \textit{tuasan} ‘footprint’, Ewenki\il{Evenki} \textit{tū} ‘to step on’.
\z

\ea
    \label{example7.10}
\ili{Udihe} \textit{{ɲ}amaʰi} ‘warm’ < \textit{{ɲ}ama-si} {\textasciitilde} \ili{Oroch} \textit{{ɲ}ama, {ɲ}ama-si,} \ili{Nanai} \textit{{ɲ}ama}, Ewenki\il{Evenki} \textit{{ɲ}ama} ‘warm’. 
\z

It is worth noting that the transition V-s-V → VʰV in root morphemes took place mostly when this complex was at the beginning of the word, in other words, when the root began with a vowel (see examples above). Later, a prosthetic glide could appear \textit{oʰ{ɵ}} > \textit{woʰ{ɵ}} ‘deer-leg fur’; \textit{iʰi} > \textit{jiʰi} ‘larch tree’, which already happened early in \ili{Udihe}.

When a consonant was at the beginning of the word, this transition often did not take place, cf.: 

\ea
    \label{example7.11}
\ili{Udihe} \textit{gaja} ‘waterfowl, duck’ {\textasciitilde} \ili{Oroch} \textit{gasa} ‘waterfowl’, \ili{Negidal} \textit{gasa} ‘swan’, Ewenki\il{Evenki} \textit{gasa} ‘crane’; 
\z

\ea
    \label{example7.12}
\ili{Udihe} \textit{k{ə}j{ə}} ‘word, language’ {\textasciitilde} \ili{Oroch} \textit{x{ə}s{ə}} ‘word’, \ili{Negidal} \textit{x{ə}s{ə}} ‘word, language’, \ili{Nanai} \textit{x{ə}s{ə}} ‘word, order’, \ili{Manchu} \textit{x{ə}s{ə}} ‘order, command’.\footnote{Šneider gives the forms \textit{gaʰæ} ‘duck’ and \textit{k{ə}ʰi{ə}} ‘word’, which are not confirmed by modern material. I did not find such forms in any of the varieties.}
\z

Some words which have a consonant before the pharyngealized vowel do not have a convincing Tungus etymology: 

\ea
    \label{example7.13}
\textit{g{ə}ʰ{ə}} ‘bad’, \textit{düʰi} ‘brain’, \textit{{ʒ}aʰi} ‘wild boar’, \textit{t{ə}ʰu} ‘all’. 
\z

\noindent Or go back to different complexes:

\ea
    \label{example7.14}
\ili{Udihe} \textit{toʰo} ‘button’ {\textasciitilde} \ili{Negidal} \textit{toxon}, \ili{Nanai} \textit{toχ\~{o}} ‘button’, \ili{Manchu} \textit{toχon} ‘metallic button’.
\z

Intervocalic \textit{-s-} in \ili{Udihe} goes back to \textit{-č-}: \textit{asa-} ‘to fit’ {\textasciitilde} \ili{Oroch} \textit{ača-}, \ili{Nanai} \textit{ača-} ‘to come to’; Ewenki\il{Evenki} \textit{arča-} ‘to meet’. 

\subsubsection{Realization}\largerpage[2]

Pharyngealization in different \ili{Udihe} varieties can be realized as: a) a break of the sound by aspiration; b) breathy voice phonation, c) a “clean” long vowel. The concrete realizations of pharyngealized long vowels differ according to the variety and also according to the mode of speech. 

\begin{table}
\begin{tabular}{ *6{l} }
  \lsptoprule
& \multicolumn{2}{c}{Oroch} & \multicolumn{3}{c}{Udihe}\\
  \cmidrule(lr){2-3}
  \cmidrule(lr){4-6}
Varieties & \multicolumn{1}{c}{\shadecell {\ili{Xadi}, Tumnin}}\\
& & \multicolumn{2}{c}{\shadecell {\ili{Koppi} (transitional)}} &\\
& & & \multicolumn{2}{c}{\shadecell {\ili{Xor}, \ili{Anjuj} (\ili{Udihe})}}\\
& & & & \multicolumn{2}{c}{\shadecell {\ili{Bikin}, \ili{Iman} (\ili{Kyakala})}}\\\cmidrule(lr){2-2}
\cmidrule(lr){3-3}
\cmidrule(lr){4-4}
\cmidrule(lr){5-5}
\cmidrule(lr){6-6}
Stages & \multicolumn{1}{c}{V-s-V} & \multicolumn{1}{c}{V-h-V} & \multicolumn{1}{c}{VʰV} & \multicolumn{1}{c}{V̤̄} & \multicolumn{1}{c}{V̄}\\
  \lspbottomrule
\end{tabular}
\caption{Types of realizations of pharyngealized long vowels in different Udihe varieties and Oroch. Comments:
V-s-V and V-h-V: sequences of three segments;
VʰV: long vowels interrupted by aspiration;
V̤̄: long vowel with pharyngealized phonation (“breathy voice”);
V̄: long vowel.\label{table:7.4}}
\end{table}

\tabref{table:7.4} shows that each \ili{Udihe} variety is characterized by two different modes of pronunciation: the full mode (FM) which is shown in the cell to the left and the allegro mode (AM) in the right cell.

Taking into account different tempo modes in each variety, \tabref{table:7.4} shows that the allegro mode of pronunciation of one variety corresponds to the full mode of pronunciation of the neighboring one, which produces a new allegro mode. It demonstrates the internal “anatomy” of the dialect continuum. 

From a phonological point of view, the VhV sequence with a weakened consonant in the intervocal position is of particular interest. Acoustically it is a long vowel interrupted by aspiration. Its phonemic interpretation, however, can be twofold, depending on the variety analyzed.

In \ili{Koppi}, this is an optional pronunciation variant characteristic of the allegro mode; the full mode of pronunciation is V-h-V (sequence of three phonemes). In the speech of Alexandr Ivashchenko, a \ili{Koppi} speaker, sequences of this type were pronounced as three syllables in the full mode of pronunciation. In order to clarify a word Ivashchenko could chant it, clearly dividing these sequences into three syllables \textit{abdæha}\footnote{Hereinafter, aside from specific phonetic realizations, \ili{Udihe} words are given in Šneider’s writing system.} ‘leaf (of a tree)’ [ab.dɛæ.ha]. However, in the allegro mode, the V-h-V sequence contracted into a long vowel, interrupted by a brief aspiration [ab.dɛæʰa]. See the following pronunciation of the word /abdæha/ ‘leaf (tree)’ in allegro (left) and full (right) modes of pronunciation. 

% \begin{figure}
% \includegraphics[width=0.7\textwidth]{figures/Perekhvalsakya3.png}
% \caption{Koppi dialect, speaker Akexandr Ivashchenko: [abd{ɛ}æʰa], [abd{ɛ}æha] ‘leaf’}
% \label{fig:7:3}
% \end{figure}

\begin{figure}
\includegraphics[width=0.85\textwidth]{figures/Perekhvalskaya_3_new.jpg}
\caption{Koppi dialect, speaker Akexandr Ivashchenko: [abd{ɛ}æʰa], [abd{ɛ}æha] ‘leaf’}
\label{fig:7:3}
\end{figure}

Similar observations were made by Igor Kormušin:

\begin{quote}\sloppy 
In the fully marked type of pronunciation, if the vowels surrounding the pharyngeal consonant are similar, they are pronounced with equal length and, in fact, form two syllables with \textbf{\textit{h}} being voiced: \textit{ahanta} (a-{ḩ}an-ta) ‘woman’, \textit{gehe} (ge-{ḩ}e) ‘bad’, \textit{oloho} (o-lo-{ḩ}o) ‘boiled fish’, \textit{ihi} (i-{ḩ}i) ‘larch’. In the fully normal type of pronunciation, \textbf{\textit{h}} is articulated simultaneously with the second vowel, becoming a pharyngeal overtone in its initial part; at the same time, the pharyngeal consonant is fused with the previous vowel, so that a single complex sound is formed; as a result, the syllable border is aligned differently, combining two syllables into one: aʰanta, geʰe, oloʰo, iʰi. [...] in the normally abbreviated type of pronunciation, the surrounding vowels fuse into a long one, the pharyngeal consonant following it [...] aʰnta, geʰ, oloʰ, iʰ. This pronunciation creates conditions for the deletion of \textbf{\textit{h}} [...] (\citealt[64--65]{Kormushin1998}, my translation – E.P.). 
\end{quote}

Kormušin distinguished three pronunciation modes: fully marked, normally full and normally abbreviated. They correspond to chant, full style and allegro mode. 

According to my data, none of the varieties exhibit coexistence of all the types of pronunciation that Kormušin singled out. Most likely, the researcher combined phenomena observed in different varieties.

Evgenij Šneider who worked in the 1930s with \ili{Anjuj} Udihe interpreted the sequence VhV (full style in \ili{Anjuj}) not as a sequence of two syllables, but as a long vowel interrupted by aspiration. 

\begin{quote}
Of course, \textit{h} in this sound complex is not an independent consonant [...] When comparing \ili{Udihe} words with pharyngealized vowels with words of the same meaning in other Manchu-Tungusic languages, it turns out that [...] the two-syllable combination became monosyllabic, i.e., the transformation process s (ş) > h ({ḩ}) was accompanied by the contraction of the pair of identic vowels. This resulted in the emergence of a new category of vowels, for example \textit{aha}{}- ‘to catch up’ (Ewenki\il{Evenki} \textit{asa}{}-); \textit{imaha} ‘snow’ (\ili{Oroch} \textit{imasa}); \textit{i{ḩ}i} ‘larch’ (\ili{Oroch}, \ili{Manchu} \textit{işi}) [...] (\citealt[10--11]{Shneider1937}, my translation – E.P.).
\end{quote}

\subsubsection{Realization of pharyngealized vowels}

\subsubsubsection{Koppi}

The two types of pronunciation, V-h-V and VhV, seem to be characteristic of the northernmost dialects of \ili{Udihe}: \ili{Koppi}, and, apparently, \ili{Xungari}. Most likely, at the end of the 19th century pronunciation of pharyngealized complexes as three segments V-h-V was also characteristic for \ili{Xor} Udihe. In Nadarov’s work we find \textit{яга} (jaga) ‘eye’ (Shn. \textit{jehæ}, Sim. \textit{jâ}), \textit{нюге} (niuge) ‘nose’ (Shn. \textit{ŋyh{ɵ}}, Sim. \textit{ŋiê}), \textit{того} (togo) ‘button’ (Shn. \textit{toho}, Sim. \textit{tô}). It is not clear what kind of sound was represented by the Cyrillic letter «Г»; most likely it was a pharyngeal consonant, possibly voiced. In some cases, Nadarov did not note it, cf. another variant of the word ‘eye’ \textit{я} (ja), \textit{нiама} (niama) ‘leather jacket’ (Shn. \textit{nehæma}, Sim. \textit{ñâma} ‘leather’).\footnote{Simonov suggested that Nadarov recorded pharyngeal (h) only at the rhythmic boundaries of the word, but this does not explain the presence of doublets in Nadarov’s list of words: ‘eye’ \textit{я} and \textit{яга;} ‘nose’ \textit{нюгу} and \textit{нiонё} \citep{Nadarov1887}.}

\subsubsubsection{Xor}

In \ili{Xor} Udihe a “new category of vowel” was formed. In the 1930s, the \ili{Udihe} on the \ili{Xor} River apparently pronounced VhV in the full pronunciation mode, and VʰV in the allegro mode. The full pronunciation mode of \ili{Xor} Udihe was the basis of “literary” \ili{Udihe}, in which several textbooks for primary school were published. Simonov, who worked with \ili{Xor} Udihe since the late 1970s, noted at that period the VʰV variant was the full mode, and pharyngealized vowels were pronounced V̤̄ in allegro mode:

\begin{quote}
    Pharyngealized vowels are pronounced with a sharp increase in intensity towards the end of the phonation. [...] When the aspiration is present, it is not in the middle of the vowel, but is superimposed on its entire second half. (\citealt[52]{Simonov1988}, my translation – E.P.)
\end{quote}

Simonov presented to the speakers words with a pharyngealized vowel, pronounced in two syllables: “words *je.hæ (instead of \textit{jâ} ‘eye’); *a.han.ta (instead of \textit{ânta} ‘woman’); *imo.ho (instead of \textit{imô} ‘fat’) were simply not understood by speakers” (\citealt[52]{Simonov1988}, my translation – E.P.).\footnote{This consideration was the reason for changing the type of writing for \ili{Xor} Udihe made by Simonov; he introduced circumflex “\^{v}” to mark breathy voice phonation (aspiration): \textit{imô} {\textasciitilde} Shn. \textit{imoho}.}

In 2006, only one type of pronunciation of pharyngealized vowels was observed in the \ili{Xor} variety. With the most complete pronunciation mode, a separate word could be pronounced as VʰV. However, even in this case, aspiration appears also after the vowel, cf. Figure \ref{fig:7:4}.

% \begin{figure}
% \includegraphics[width=0.7\textwidth]{figures/Perekhvalsakya4.png}
% \caption{Xor variety. Speaker Valentina Kjalundzjuga: \textit{umaha} [uma{̤}a{̤}h] ‘bone marrow’}
% \label{fig:7:4}
% \end{figure}

\begin{figure}
\includegraphics[width=0.85\textwidth]{figures/Perekhvalskaya_4_new.jpg}
\caption{Xor variety. Speaker Valentina Kjalundzjuga: \textit{umaha} [uma{̤}a{̤}h] ‘bone marrow’}
\label{fig:7:4}
\end{figure}

Figure \ref{fig:7:4} shows that the final part loses vocalic characteristics turning into an aspiration. The final complex consists of a long vowel (250 milliseconds), the duration of which is almost twice of the initial short vowel (u). Compare pronouncing by the same speaker of the Accusative case form of the same word: \textit{umahawa} (Figure \ref{fig:7:5}). 

% \begin{figure}
% \includegraphics[width=0.7\textwidth]{figures/Perekhvalsakya5.png}
% \caption{Xor variety. Speaker Valentina Kjalundzjuga: \textit{umahawa} [umaawa{̤}h] ‘bone marrow (Acc)’}
% \label{fig:7:5}
% \end{figure}

\begin{figure}
\includegraphics[width=0.85\textwidth]{figures/Perekhvalskaya_5_new.jpg}
\caption{Xor variety. Speaker Valentina Kjalundzjuga: \textit{umahawa} [umaawa{̤}h] ‘bone marrow \textsc{acc})’}
\label{fig:7:5}
\end{figure}

Figure \ref{fig:7:5} shows that the long vowel /ā{̤}/ in \textit{umaha} has lost its pharyngealized quality, but the aspiration appears at the absolute end of the word. Such a transfer of aspiration to the end of the phonetic word may be an individual characteristic of the speaker, but most likely it reflects the pronunciation of pharyngealized vowels in the \ili{Xor} variety. Kormušin also pointed out such a realization of pharyngealized vowels.\footnote{Nikolaeva questioned the possibility of such a realization: if pharyngealization is realized at the end of the word, it contradicts the basic phonotactic rules of \ili{Udihe}. However, it is not a consonant, but a pharyngealized vowel \citep{Nikolaeva2000}. This is an additional consideration in favor of the interpretation of these complexes as single phonemes.}

Compare the realization of these two words in \figref{fig:7:6}. The principles of Autosegmental Phonology \citep{Goldsmith1976} explain this by the independent character of the suprasegmental level. Phonation characterizes the whole word and not any particular segment and is realized at the end of the word. 

% \begin{figure}
% \includegraphics[width=0.7\textwidth]{figures/Perekhvalsakya6.jpg}
% \caption{Xor variety. Speaker Valentina Kjalundzjuga: \textit{umaha} [umaa{̤}h] ‘bone marrow’; \textit{umahawa} [umaawah] ‘bone marrow (Acc)’}
% \label{fig:7:6}
% \end{figure}


As pointed out by Kormušin, such a pronunciation creates the conditions for a loss of aspiration. This happened primarily with pharyngealized vowels in the final position, as in the examples above. Apparently, the loss of pharyngealization occurred primarily in non-root morphemes. The pharyngealization, therefore, was lost in the personal markers 2\textsc{sg}, \textsc{pl}: \textit{-i{̤}} > \textit{-i};  \textit{-u{̤}} > \textit{-u}; and in past tense suffixes: \textit{-a{̤}- / -o{̤}- / -{ə}{̤}- > -a- / -o- / -{ə}-}.

\begin{figure}
\includegraphics[width=0.85\textwidth]{figures/Perekhvalskaya_6_new.jpg}
\caption{Xor variety. Speaker Valentina Kjalundzjuga: \textit{umaha} [umaa{̤}h] ‘bone marrow’; \textit{umahawa} [umaawah] ‘bone marrow (\textsc{acc})’}
\label{fig:7:6}
\end{figure}


\begin{table}
\begin{tabular}{ p{1.5cm} p{1.5cm} l l l }
  \lsptoprule
Form & Oroch &  \ili{Xor} Udihe FM & \ili{Xor} Udihe AM, & Translation\\
& & & \ili{Bikin} Udihe & \\
  \midrule
1\textsc{sg} & təgə-wi, & təgə-i & təgə-i & ‘my shirt’\\
& təgə-ji, & & & \\
& təgə-i & & & \\
2\textsc{sg} & təgə-si & təgə-i̤ & təgə-i & ‘your shirt’\\
1\textsc{pl.excl} & təgə-mu, & təgə-u & təgə-u & ‘our shirt’\\
&  təgə-wu & & & \\
2\textsc{pl} & təgə-su, & təgə-ṳ & təgə-u & ‘your shirt’\\
& təgə-hu & & & \\
  \lspbottomrule
\end{tabular}
\caption{Personal possessive forms in Udihe varieties and in Oroch. The Oroch data are taken from \citet{AvrorinLebedeva1968} and \citet{AvrorinBoldyrev2001}. In both sources, alternative forms are given without comments. I suggest that forms which are closer to Udihe are characteristic of the Xadi dialect which is more innovative.}
\label{table:7.5}
\end{table}

In the \ili{Xor} variety, this loss is characteristic of the allegro mode of pronunciation. Auxiliary and negative verbs, being the most frequent ones, were also “erased” as the full mode of their pronunciation was replaced by the allegro mode. Besides, in the speech of younger \ili{Xor} variety speakers, pharyngealized vowels are totally lost and have been replaced with pure long vowels in all positions. 

Published \ili{Xor} Udihe texts demonstrate incoherence in marking the pharyngealized vowels, which reflects differences in modes of pronunciation. An example are folklore texts recorded mainly with the \ili{Xor} Udihe \citep{Simonovetal1998}. In the vast majority of past tense forms, pharyngealization is not marked in suffixes, cf. in text No. 1: \textit{oloktoni} ( < *\textit{olokto-ho-ni}) ‘s/he cooked’; \textit{andalati} ( < *\textit{andala-ha-ti}) ‘they made friends’; \textit{alasieni} ( < *\textit{alasi-h{ə}-ni}) ‘s/he waited’ \citep[74]{Simonovetal1998}. Similarly, pharyngealized vowels in personal suffixes are also not marked in these texts. And in root morphemes, pharyngealized vowels are sequentially marked.

\subsubsubsection{Samarga}

In the \ili{Samarga} variety, pharygealization is kept only in some root morphemes. 

\subsubsubsection{Bikin and Iman}

Pharyngealized vowels are completely lost in \ili{Bikin} and \ili{Iman}, where the corresponding complexes are pronounced as clear long vowels, cf. the word for ‘leaf’: \ili{Oroch} [abdasa]; \ili{Koppi} variety [abd{ə}ha] (FM) {\textasciitilde} [abd{ə}ʰa] (AM); \ili{Xor} Udihe [abdæʰe] (FM) {\textasciitilde} [abdæ{̤}e{̤}] (AM); \ili{Bikin} Udihe [abdææ]. While in \ili{Xor} the loss of pharyngealized vowels is a recent phenomenon, and elder speakers still pronounce them at least under the full mode of pronunciation, in the \ili{Bikin} and \ili{Iman} varieties, pharyngealization was not characteristic for the speech of people born in the 1920--1930s. This means that pharyngealized vowels were lost at least a hundred years ago. In the \ili{Bikin} variety, the etymological pharyngealized vowels were replaced by long ones, and there is a tendency for these vowels to become short (\citealt{Nikolaeva2000}: 115--116; \citealt{Tsumagari2012}). 

In sum, the data presented show that \ili{Udihe} varieties present different stages of one process: weakening of the consonant in the intervocalic position with the substitution of segment units by suprasegmental ones.

\subsubsection{Loss of pharyngealization and its effects in morphology}

Loss of pharyngealization had a significant impact on the morphological system of the southern dialects. The main consequence of the loss of pharyngealized vowels here was the formal coincidence of possessive suffixes of the first and second person singular and plural (Exclusive form) for vowel-final stems; cf. data in \tabref{table:7.6}. 

\begin{table}
\begin{tabular}{ l l l l l }
  \lsptoprule
& \multicolumn{2}{c}{Nominative} & \multicolumn{2}{c}{Accusative} \\
  \cmidrule(lr){2-3}\cmidrule(lr){4-5}
& \ili{Xor} & \ili{Bikin} & \ili{Xor} & \ili{Bikin}\\
  \midrule
1\textsc{sg} & kusigә-i & \textbf{(bi) kusigә-i}
% &  ‘my knife’
&  kusigә-wә-i & \textbf{kusigә-wә-i}\\
2\textsc{sg} & kusigә-i̤ & \textbf{(si) kusigә-i}
% & ‘your knife’
& 	kusigә-wә-i̤ & 	\textbf{kusigә-wә-i}\\
3\textsc{sg} & kusigә-ni & kusigә-ni
% & ‘his/her knife’
& kusigә-wә-ni & kusigә-wә-ni\\
1\textsc{pl.excl} & kusigә-u & \textbf{(bu) kusigә-u}
% & ‘our (excl) knife’
& kusigә-wә-u & \textbf{kusigә-wә-u}\\
1\textsc{pl.incl} & kusigә-fi & kusigә-fi
% & ‘our (incl) knife’
& kusigә-wә-fi & kusigә-wә-fi\\
2\textsc{pl} & kusigә-ṳ & \textbf{(su) kusigә-u}
% & ‘your (pl) knife’
& kusigә-wә-ṳ & \textbf{kusigә-wә-u}\\
3\textsc{pl} & kusigә-ti & kusigә-ti
% & ‘their knife’
& kusigә-wә-ti & kusigә-wә-ti\\
  \lspbottomrule
\end{tabular}
\caption{Fragment of the paradigm of the personal possessive conjugation of the noun \textit{kusigə} ‘knife’ in Bikin and in Xor Udihe. Forms merged in \ili{Bikin} Udihe are bold.}
\label{table:7.6}
\end{table}

In southern \ili{Udihe}, in order to clarify the “possessor”, personal pronouns are used. While in northern \ili{Udihe} the use of personal pronouns indicates emphasis, in southern \ili{Udihe} it is neutral. Therefore, southern \ili{Udihe} displays a greater degree of analyzability. 

\subsection{Laryngealized long vowels}

\subsubsection{Etymology}

\ili{Udihe} laryngealized vowels go back to the V-q-V complex, which was a three-phoneme combination and is present in many Tungusic languages, cf.: 

\ea
    \label{example7.15}
\ili{Udihe} \textit{um’a} ‘fishing hook’ {\textasciitilde} \ili{Oroch} \textit{um{ə}k{ə}(n)}; \ili{Negidal} \textit{umaxa/umaka}; Ewenki\il{Evenki} \textit{umika;} 
\ex
    \label{example7.16}
\ili{Udihe} \textit{l’o-} ‘to hang’ {\textasciitilde} \ili{Oroch} \textit{lō/loko} [loqo]; \ili{Negidal} \textit{loko-}; \ili{Solon} \textit{loko-;} \ili{Kilen} \textit{loqo-}; Ewenki\il{Evenki} \textit{loko-}; 
\ex
    \label{example7.17}
\ili{Udihe} \textit{x’ai}\footnote{In sequences *a-q-i the first vowel holds the phonation: [a\textsuperscript{{ʔ}}ai] or [a{̰}a{̰}i], in practical writing: \textit{‘ai.}} ‘liver’ {\textasciitilde} \ili{Oroch} \textit{x{ākin}} [x{āqin]}; \ili{Negidal} \textit{x{āx}{ɩ}n}; \ili{Kilen} \textit{xakī}; Ewenki\il{Evenki} \textit{hakin};
\ex
    \label{example7.18}
Bik. \ili{Udihe} \textit{g’ata} ‘cranberry’ {\textasciitilde} \ili{Oroch} \textit{gākta/gakka} [gaqːa]; \ili{Negidal} \textit{gaxakta}; Ewenki\il{Evenki} \textit{gakakta}; 
\ex
    \label{example7.19}
\ili{Udihe} \textit{g’ai} ‘crow’ {\textasciitilde} \ili{Oroch} \textit{gāki} [gāqi]; \ili{Negidal} \textit{gāxi}; Ewenki\il{Evenki} \textit{gāki};
\ex
    \label{example7.20}
\ili{Udihe} \textit{kakt’a} ‘half’ {\textasciitilde} \ili{Negidal} \textit{kaltaka/kaltaxa}; Ewenki\il{Evenki} \textit{kaltaka}; \ili{Solon} \textit{xaltaxa};
\ex
    \label{example7.21}
\ili{Udihe} \textit{s’ou} ‘scoop’ {\textasciitilde} Ewenki\il{Evenki} \textit{sokowun/hokowun}; cf. \ili{Oroch} \textit{soko-} [soqo], \ili{Negidal} \textit{soxo-} ‘to scoop’.
\z

It should be noted that only the uvular variant [q] of the phoneme /k/ transformed into the glottal stop and further created the creaky voice phonation. The velar [k] was preserved in \ili{Udihe} as [x] and [k]:

\ea
    \label{example7.22}
\ili{Udihe} \textit{joxo} ‘kettle’ {\textasciitilde} \ili{Oroch}. \textit{iko}; \ili{Negidal} \textit{īx\={ə}}; Ewenki\il{Evenki} \textit{īk\={ə}}.
\ex
    \label{example7.23}
\ili{Udihe} \textit{j{ə}x{ə}} ‘to sing’ {\textasciitilde} \ili{Oroch} \textit{ik{ə}}; \ili{Negidal} \textit{ix\={ə}}; Ewenki\il{Evenki} \textit{ik\={ə}.}
\ex
    \label{example7.24}
\ili{Udihe} \textit{cik{ə}-} ‘to urinate’ {\textasciitilde} \ili{Negidal} \textit{cix\={ə}n}; \ili{Kilen} \textit{cik\={ə}}; Ewenki\il{Evenki} \textit{cik\={ə}n-}.
\z

In \ili{Oroch}, in accordance with an assimilation rule, the uvular allophone [q] occurs only after the vowels [a] and [o]. In other cases, the velar [k] appears. Apparently, a similar rule was also present in \ili{Udihe}. The uvular [q] then transformed into the glottal stop. This explains why the series of laryngealized vowels in \ili{Udihe} is limited to \textit{‘o} and \textit{‘a}. 

Evgeny Šneider, on the basis of general system considerations, postulated the presence of the entire set of laryngealized long vowels, both simple and diphthongoids \citep[83]{Shneider1936}. As Simonov showed, this does not correspond to the linguistic reality \citep{Simonov1988}.

It is worth noting that “non-etymological” laryngealized vowels sporadically appear after the plosives \textit{b, p, c}, if followed by the vowel \textit{a}, cf.: 

\ea
    \label{example7.25}
\ili{Udihe} \textit{b’ata} ‘boy’ {\textasciitilde} \ili{Oroch} \textit{bātu, baturi} ‘strong warrior’, \ili{Nanai} \textit{ātor} ‘strong warrior’; \ili{Manchu} \textit{baturu} ‘hero, warrior’;
\ex
    \label{example7.26}
\ili{Udihe} \textit{c’aligi} ‘white’ {\textasciitilde} \ili{Oroch} \textit{čām} ‘white’; \ili{Nanai} \textit{čāɣǯã} ‘white’.\footnote{However, consider Ewenki\il{Evenki} \textit{bagadi} ‘strong, brave’, proto-\ili{Mongolian} *\textit{ba{ɣ}atur} and proto-\ili{Mongolian} \textit{*ča{ɣ}ān} ‘white’. Based on a comparison of \ili{Udihe} forms with historically attested \ili{Mongolian} ones (as given by \citealt{Cincius1975P}), it may be imagined that \ili{Udihe} retained a more ancient form. However, \ili{Udihe} laryngealized vowels originate from \textit{-q-} and not \textit{-g}/\textit{ɣ-}.}
\z

It is also noticeable that many \ili{Udihe} words with laryngealized vowels do not have a reliable Tungisic etymology. Often they are attested only in \ili{Udihe}: \textit{‘ana} ‘boat ‘, \textit{d’a} ‘cotton wool’; \textit{gob’o} ‘fly’, \textit{‘asa} ‘bay’; \textit{t’aŋki} ‘middle’, \textit{s’ai} ‘salt’ and others. Still, these words are known in all \ili{Udihe} dialects. The etymology of some other words is not very convincing, e.g. \textit{od’o} ‘grandfather’ is compared with \ili{Oroch} \textit{{ə}di{ɣ}i}; \ili{Ulcha} \textit{{ə}d{ə}k{ə}(n)} ‘father-in-law’ which is doubtful. 

It may be supposed that \ili{Udihe} had undergone influence of a substrate or adstrate non-Tungusic language which was also the source of non-Tungusic loanwords.

\subsubsection{Types of realization}

Laryngealization in different \ili{Udihe} varieties can be realized as: a) a break of the sound by the glottal stop; b) creaky voice phonation, c) the increased intensity in combination with the low/falling tone. The flattening effect of laryngealization is observed in all \ili{Udihe} varieties. However, only in the most innovative varieties of \ili{Bikin} and \ili{Iman}, it became the main (and in \ili{Iman} Udihe the only) distinctive feature [+tone]. Thus, these varieties should be classified as tonal, which is untypical for Manchu-Tungusic languages.

The concrete realizations of laryngealized long vowels differ according to the variety and also according to the mode of speech.

\begin{table}
\begin{tabular}{ *6{l} }
  \lsptoprule
& \multicolumn{2}{c}{{Oroch}} & \multicolumn{3}{c}{{Udihe}}\\
  \cmidrule(lr){2-3}\cmidrule(lr){4-6}
Varieties & \multicolumn{1}{c}{\shadecell {\ili{Xadi}, Tumnin}}\\
& & \multicolumn{2}{c}{\shadecell {\ili{Koppi} (transitional)}} &\\
& & & \multicolumn{2}{c}{\shadecell {\ili{Xor}, \ili{Anjuj} (\ili{Udihe})}}\\
& & & & \multicolumn{2}{c}{\shadecell {\ili{Bikin}, \ili{Iman} (\ili{Kyakala})}}\\
\cmidrule(lr){2-2}\cmidrule(lr){3-3}\cmidrule(lr){4-4}\cmidrule(lr){5-5}\cmidrule(lr){6-6}
Stages & \multicolumn{1}{c}{V-q-V} & \multicolumn{1}{c}{V-ʔ-V} & \multicolumn{1}{c}{VˀV} & \multicolumn{1}{c}{V̰̄} & \multicolumn{1}{c}{V̄̀}\\
  \lspbottomrule
\end{tabular}
\caption{Types of realizations of laryngealized long vowels in different Udihe varieties and Oroch. Comments:
V-q-V and V-ʔ-V: sequences of three segments; 
VˀV: long vowels interrupted by a glottal stop;
V̰̄: long vowel with laryngealized phonation (“creaky voice”);
V̄̀: long vowel with falling (low) tone.\label{table:7.7}}
\end{table}


\tabref{table:7.7} shows the two modes of pronunciation: the full mode (FM) which is shown in the cell to the left and the allegro mode (AM) in the right cell.

\ili{Udihe} varieties and the closely related \ili{Oroch} language represent changes of certain phonetic complexes “from consonant to tone”; each variety representing a certain stage of this process. The innovation was spreading, roughly, in the direction from north to south: \ili{Oroch} → \ili{Koppi} variety → \ili{Xor} and \ili{Anjuj} varieties → \ili{Bikin} and \ili{Iman} varieties.

Cf. the word for ‘dog’: \ili{Oroch} [inaqi]; \ili{Koppi} variety FM [ɩnæ̰ʔi], AM [inə\textsuperscript{ʔ}i]: \ili{Xor} variety FM [inæ\textsuperscript{ʔ}ai], AM [inə̰ḛi]; \ili{Bikin} variey FM [inə̰ḛi], AM [inə̀èi].

\subsubsection{Realization of pharyngealized vowels}

To study the realizations of laryngealized vowels in different \ili{Udihe} varieties is a difficult task when based on written sources. In the case of pharyngealized vowels written sources provide more or less reliable information, but laryngealized vowels are written with an apostrophe uniformly by all researchers; this spelling hides rather different types of realization. 

\subsubsubsection{Oroch and Koppi}

According to Avrorin and Lebedeva, in \ili{Oroch} the phoneme /k/ is realized as uvular [q] in the position after /a/, /ä/, /o/, between identical vowels, or before /i/ (\citealt{AvrorinLebedeva1978}). This is also characteristic of the \ili{Koppi} variety. 

It should be noted that in slower speech, an aspiration [h] is clearly heard between the vowel and the uvular [q]: [naʰqi] ‘dog’, [beæʰqa] ‘river’, [araʰqi] ‘strong spirit’, [gaʰqi] ‘crow’. A variant realization is a pause before [q], which is perceived as a “long stop”: [maa\textsuperscript{{ʔ}}qi]. Here (\textsuperscript{{ʔ}}q] represents a preglottalized consonant. See Figures \ref{fig:7:7} and \ref{fig:7:8} on the pronunciation of the words [beæʰqa] and [maa\textsuperscript{{ʔ}}qi] in \ili{Koppi}.

% \begin{figure}
% \includegraphics[width=0.7\textwidth]{figures/Perekhvalsakya7.png}
% \caption{Koppi variety. Speaker Alexandr Ivaščenko: [beæʰqa] ‘river’}
% \label{fig:7:7}
% \end{figure}

\begin{figure}
\includegraphics[width=0.85\textwidth]{figures/Perekhvalskaya_7_new.jpg}
\caption{Koppi variety. Speaker Alexandr Ivaščenko: [beæʰqa] ‘river’}
\label{fig:7:7}
\end{figure}

% \begin{figure}
% \includegraphics[width=0.7\textwidth]{figures/Perekhvalsakya8.png}
% \caption{Koppi variety. Speaker Alexandr Ivaščenko: [maa\textsuperscript{{ʔ}}qi] ‘there is no’}
% \label{fig:7:8}
% \end{figure}

\begin{figure}
\includegraphics[width=0.85\textwidth]{figures/Perekhvalskaya_8_new.jpg}
\caption{Koppi variety. Speaker Alexandr Ivaščenko: [maa\textsuperscript{{ʔ}}qi] ‘there is no’}
\label{fig:7:8}
\end{figure}

The spectrogram of [beæʰqa] ‘river’ shows aspiration after a long diphthongoid [eæ], then there is a gap followed by the stop [q]. The spectrogram of [maa\_qi] ‘there is no’, seems to present no aspiration, but the silence zone before the stop lasts for more than 70 milliseconds. 

In other phonetic contexts the phoneme /k/ is realized as a velar [k]: [uk{ə}h{ə}] ‘door’: \ili{Bikin} \textit{uk{ə̄}} ‘door, doorway’. The \ili{Koppi} speaker never pronounced [q] as [{ʔ}]. Apparently, this pronunciation is not typical for the \ili{Oroch} language, including the \ili{Koppi} dialect. Obviously, we are dealing with a sequence of independent phonemes, and not with a complex sound in this variety.

\subsubsubsection{Xor}

Šneider described laryngealized vowels of \ili{Xor} Udihe as having a stop interrupting the vowel; Lev R. Zinder and Margarita I. Matusevich\footnote{The results of the study of the \ili{Udihe} phonetic system, carried out in the 1930s by Lev Zinder and Margarita Matusevich in the laboratory of experimental phonetics of Leningrad State University, were not published. Partially they were included in \citet{Zinder1948} and \citet{Kormushin1998}.} showed that this stop occurs closer to the beginning of the vowel \citep{Zinder1948}. At present, these vowels are pronounced in allegro mode with “creaky voice” phonation. See two variants of \textit{in’ai} ‘dog’ pronounced by the same speaker within the same recording session. At first, the speaker clarified the word (full mode); later, she pronounced it more “carelessly” (allegro mode). It is worth noting that the speech tempo remained almost the same; it was the intensity of pronunciation and the tonal pattern that changed. 

% \begin{figure}
% \includegraphics[width=0.7\textwidth]{figures/Perekhvalsakya9.png}
% \caption{Xor variety. Speaker Valentina Kjalundzjuga: in’ai ‘dog’: FM [{ɩ}næ{̰}\textsuperscript{{ʔ}}ai], AM [in{ə}{̰}e{̰}i]}
% \label{fig:7:9}
% \end{figure}

\begin{figure}
\includegraphics[width=0.85\textwidth]{figures/Perekhvalskaya_9_new.jpg}
\caption{Xor variety. Speaker Valentina Kjalundzjuga: \textit{in’ai} ‘dog’: FM [{ɩ}næ{̰}\textsuperscript{{ʔ}}ai], AM [in{ə}{̰}e{̰}i]}
\label{fig:7:9}
\end{figure}

In Figure \ref{fig:7:9}, the creaky phonation zone can be seen in AM pronunciation. It should be noted that in FM the part of the vowel before the stop is also pronounced with creaky phonation.

These observations confirm the conclusions made by Zinder and Matusevich. Indeed, there is a pronunciation variant when the vowel is broken by a stop. Still, at present the most common way to pronounce a “laryngealized vowel” in \ili{Xor} Udihe (FM) is when the vowel is not broken with a stop, but with a glottal approximant, cf. the utterance by V.T. Kjalundzjuga of the word \textit{bul’a} ‘ash’, see \figref{fig:7:10}.

% \begin{figure}
% \includegraphics[width=0.7\textwidth]{figures/Perekhvalsakya10.png}
% \caption{Xor variety. Speaker Valentina Kjalundzjuga: \textit{bul’a} ‘ash tree’ [b{ʋ}la{ʔ}a{̰}]}
% \label{fig:7:10}
% \end{figure}

\begin{figure}
\includegraphics[width=0.85\textwidth]{figures/Perekhvalskaya_10_new.jpg}
\caption{Xor variety. Speaker Valentina Kjalundzjuga: \textit{bul’a} ‘ash tree’ [b{ʋ}la{ʔ}a{̰}]}
\label{fig:7:10}
\end{figure}

Thus, in \ili{Xor} Udihe, three types of pronunciation of laryngealized vowels coexist: a) the vowel is interrupted by a stop; 2) by an approximant; 3) the vowel bears creaky voice phonation. However; the creaky phonation does not characterise the whole vowel, and happens in the place where the stop would have been pronounced under another mode of pronunciation.

\subsubsubsection{Bikin, Iman}\largerpage

In varieties of the southern dialect cluster laryngealized vowels with glottal stop are not found. Specific realizations of laryngealized vowels are in fact a diagnostic feature that differentiate local varieties of \ili{Bikin} Udihe. The \ili{Udihe} came from different smaller camps before settling down in the village of Krasnyj Jar, and they still retain certain speech differences. Unfortunately, at present, it is difficult to make a detailed study of these varieties due to the poor preservation of the language and the small number of speakers. However, our language consultants distinguish people speaking \textit{Sjain}, \textit{Olon}, \textit{Ulunga}, \textit{Mitahiza}, \textit{Sigou} and other varieties. Basing on the data that I have, it may be concluded that \ili{Udihe} varieties located upstream the \ili{Bikin}-river were more conservative in vocalism, and the most innovative one being Olon, the village lowest downstream.

The realization of laryngealized vowels in \ili{Bikin} Udihe varies significantly. It may be a) a laryngeal spirant; b) creaky voice phonation; c) a sharp increase of intensity of the vowel in combination with a low tone. The latter [c] is typical primarily for people from Olon. Consider realizations of laryngealized vowels: laryngeal spirant and creaky voice phonation with a speaker of the Sigou variety.

% \begin{figure}
% \includegraphics[width=0.7\textwidth]{figures/Perekhvalsakya11.png}
% \caption{Bikin variety. Speaker Lydia Simanchuk: \textit{‘ana} ‘boat’ [{ɦ}an{ɐ}]}
% \label{fig:7:11}
% \end{figure}

Figure \ref{fig:7:11} shows that the laryngealized vowel is in the beginning of the word and is realized as a laryngeal spirant, clearly visible on the spectrogram. The vowel [ā] is long and carries the creaky phonation. The next vowel [{ɐ}] is more close. The example presented concerns the full pronunciation mode in \ili{Bikin} variety; in the allegro mode, laryngealized vowels are never pronounced as a laryngeal spirant.\largerpage

% \begin{figure}
% \includegraphics[width=0.7\textwidth]{figures/Perekhvalsakya12.png}
% \caption{Bikin variety. Speaker Lydia Simanchuk: \textit{g’ai} ‘crow’ [g{ɐ}{̰}{ɛ}i]}
% \label{fig:7:12}
% \end{figure}

\begin{figure}
\includegraphics[width=0.85\textwidth]{figures/Perekhvalskaya_11_new.jpg}
\caption{Bikin variety. Speaker Lydia Simanchuk: \textit{‘ana} ‘boat’ [{ɦ}an{ɐ}]}
\label{fig:7:11}
\end{figure}

\begin{figure}
\includegraphics[width=0.85\textwidth]{figures/Perekhvalskaya_12_new.jpg}
\caption{Bikin variety. Speaker Lydia Simanchuk: \textit{g’ai} ‘crow’ [g{ɐ}{̰}{ɛ}i]}
\label{fig:7:12}
\end{figure}

In Figure \ref{fig:7:12}, the laryngealized vowel occurs in a CV syllable. It has an articulation that is more close and carries the creaky phonation.

It is interesting that in some \ili{Bikin} variants the laryngealized element, as an independent suprasegmental unit, can change its position in the word. It can be located at the beginning of the vowel (see Figures \ref{fig:7:11}, \ref{fig:7:12}), and it can also move to the end of the vowel (see Figure \ref{fig:7:13}). It may be an individual characteristic of the speaker, or, perhaps, characteristic of a local variety. Figure \ref{fig:7:12} presents the example of phonation realized at the second part of the vowel.

% \begin{figure}
% \includegraphics[width=0.7\textwidth]{figures/Perekhvalsakya13.png}
% \caption{Bikin variety. Speaker Nadežda Kukčenko: \textit{b’æsa} ‘small river’ [beæ{̰}xa]}
% \label{fig:7:13}
% \end{figure}

\begin{figure}[p]
\includegraphics[width=0.85\textwidth]{figures/Perekhvalskaya_13_new.jpg}
\caption{Bikin variety. Speaker Nadežda Kukčenko: \textit{b’æsa} ‘small river’ [beæ{̰}xa]}
\label{fig:7:13}
\end{figure}

\begin{figure}[p]
\includegraphics[width=0.8\textwidth]{figures/Perekhvalskaya_14_new.jpg}
\caption{Bikin variety, speaker Alexandr Kančuga: \textit{od’o} ‘grandfather’ [ɔ̀dɔ̀]}
\label{fig:7:14}
\end{figure}

\begin{figure}[p]
\includegraphics[width=0.8\textwidth]{figures/Perekhvalskaya_15_new.jpg}
\caption{Bikin variety, speaker Alexandr Kančuga: \textit{b’oto} ‘ligneous mushroom’ [bɔ̀tɔ́]}
\label{fig:7:15}
\end{figure}

\begin{figure}[p]
\includegraphics[width=0.85\textwidth]{figures/Perekhvalskaya_16_new.jpg}
\caption{Bikin variety. Speaker Alexandr Pionka: [baam\textsuperscript{i}] ‘I met’}
\label{fig:7:16}
\end{figure}

Figure \ref{fig:7:13} shows a decrease in pitch on the laryngealized vowel. This peculiarity of pronouncing laryngealized vowels in the \ili{Xor} variety was noted by Galina Radchenko, who conducted an experimental study of the phonetics of this variety \citep[37]{Radchenko1988}. Radchenko explained this phenomenon by the tone-lowering effect of laryngealization. This is even more obvious in the \ili{Bikin} variety. Consider the pronunciation of \textit{od’o} ‘grandfather’ by a speaker of the Sjain variety in \figref{fig:7:14}.

% \begin{figure}
% \includegraphics[width=0.6\textwidth]{figures/Perekhvalsakya14.png}
% \caption{Bikin variety. Speaker Alexandr Kančuga: \textit{od’o} ‘grandfather’ [ɔ̀dɔ̀]}
% \label{fig:7:14}
% \end{figure}


Figure \ref{fig:7:14} shows a word pronounced in the full mode with creaky phonation. It shows that the laryngealized vowel is characterized by a high intensity and lowering of the pitch.

In the \ili{Xor} variety low pitch was a side-effect of vowel laryngealization. In the \ili{Bikin} variety, due to the gradual loss of creaky phonation in allegro mode, low pitch accompanied by a high intensity of pronunciation became the main distinctive feature of laryngealized vowels in some idiolects. Consider the following example: the word \textit{b’oto} ‘ligneous mushroom’ pronounced by the same speaker.

% \begin{figure}
% \includegraphics[width=0.6\textwidth]{figures/Perekhvalsakya15.png}
% \caption{Bikin variety. Speaker Alexandr Kančuga: \textit{b’oto} ‘ligneous mushroom’ [bɔ̀tɔ́]}
% \label{fig:7:15}
% \end{figure}


The examples in Figures \ref{fig:7:14} and \ref{fig:7:15} present different tones (pitch movements): ɔ̀-ɔ̀ and ɔ̀-ɔ́.

Tone raising on the second syllable as shown in Figure \ref{fig:7:14} was described by Šneider in 1936 who interpreted it as an exponent of musical accent in \ili{Udihe} \citep[92]{Shneider1936}. See details in \citet[134--137]{Nikolaeva2000}. This interpretation seems erroneous, since accent (stress) is connected with the hierarchy of syllables in a word. In \ili{Udihe}, a word is characterized rather by a melodic pattern, which is closer to tone than to stress. Thus, we may conclude that in the vocalic systems of southern \ili{Udihe}, tonal systems are under formation. This is most obvious in bi-syllabic and polysyllabic words; however, it is also characteristic of monosyllabic words which have at least two moras \citep{Simonov1988}. Consider the following example: the verb \textit{b’aami} ‘I met’ pronounced by a speaker of the Olon variety which is the most innovative one and where phonation was lost. 

The laryngealized vowel is realized by a sharp raise of intensity together with low tone, in other words, on the suprasegmental level. The change “from consonant to tone” is complete.

% \begin{figure}
% \includegraphics[width=0.6\textwidth]{figures/Perekhvalsakya16.png}
% \caption{Bikin variety. Speaker Alexandr Pionka: [baam\textsuperscript{i}]}
% \label{fig:7:16}
% \end{figure}

\section{Discussion}

Juha Janhunen suggested the appearance of tonal distinctions in \ili{Udihe} are due to the \ili{Chinese} influence \citep{Janhunen1999}. His argument could be summarized as follows: a) \ili{Udihe} is the southernmost of the Tungusic languages, and it was in contact with \ili{Chinese} which is tonal; b) generally, tones in many Asian languages have arisen as “suprasegmental compensation” for the loss of segment sequences; c) four types of vowels of \ili{Udihe} correspond to four tones of \ili{Chinese}, as also noted by \citet[104]{Radchenko1988}. Janhunen pointed out that \ili{Chinese} tones also have complex realizations and are characterized not only by changes in pitch, but also by the duration and the presence of different types of phonation. There are certain objections to this explanation. 

First, the contraction of the V-s-V and V-q-V segment chains into a single complex vowel is already characteristic of the \ili{Koppi} transitional dialect. And there was no \ili{Chinese} influence in \ili{Koppi}.

Second, the four types of vowels which correspond, according to Janhunen, to four tones of \ili{Chinese} are found only in varieties of the northern dialect cluster. And, these are the varieties which were much less affected by \ili{Chinese} influence than \ili{Bikin} and \ili{Iman} Udihe. Indeed, many features of the southern varieties can be explained by intensive contact with \ili{Chinese} (for more details see \citet{Perekhvalskaya2001}. However, the influence of \ili{Chinese} manifested itself, rather in the general trend to analytism, which was also noted by \citet[83--84]{Tsumagari2012} and in a certain “simplification” of the system: alignment of paradigms by analogy, etc.

Still, the origin of \ili{Udihe} vocalism is hard to explain. Using Edward Sapir’s term “drift”, it can be said that in \ili{Udihe} and in neighbouring \ili{Oroch} varieties there was an influence of a certain “constant factor”. This had to be some peculiarities of articulation that were not characteristic of other Manchu-Tungusic languages. A large amount of \ili{Udihe} common words are of non-Tungusic origin: \textit{g{ə}ʰ{ə}} ‘bad’, \textit{duʰi} ‘brain’, \textit{{ʒ}aʰi} ‘wild boar’, \textit{t{ə}ʰu} ‘all’, \textit{‘ana} ‘boat ‘, \textit{d’a} ‘cotton wool’; \textit{gob’o} ‘a fly’, \textit{‘asa} ‘bay’; \textit{t’aŋki} ‘middle’, \textit{s’ai} ‘salt’, \textit{kæfakta} ‘firewood’ as well as the word \textit{asasa} ‘thank you’ and some others. On a rather cautious assumption, \ili{Udihe} was influenced by a non-Tungusic language, previously present in this area, but not \ili{Chinese}.

\section{Conclusions}

In the dialects of the southern cluster, three types of vowels correspond to the four types of vowels characteristic of the northern dialect cluster of \ili{Udihe}.

The decrease in pitch on a laryngealized vowel is characteristic of all \ili{Udihe} varieties, but in the dialects of the northern cluster the low pitch was a side-effect of vowel laryngealization. In \ili{Bikin} and \ili{Iman}, due to the gradual loss of creaky phonation in allegro mode, low pitch became the main distinctive feature of laryngealized vowels in some idiolects.

\begin{sloppypar}
All \ili{Udihe} varieties are characterized by a specific prosodic structure of the word. The word has minimally two moras, and consist of an initial and final rhythmic part that differ by suprasegmental pattern. This was noted by researchers of \ili{Udihe} before \citep{Simonov1988}. Still it was often interpreted in terms of “stress” (accent): Šneider and Sunik wrote that an \ili{Udihe} word has two stresses, one of which falls on the initial syllable of the word, and the other on the final syllable (\citealt{Shneider1936}; \citealt{Sunik1968}). The term stress or accent is not appropriate here. It is the prosodic structure of the word that is contrastive: words with similar segmental chains can differ by their prosodic structure.
\end{sloppypar}

Contrastive prosodic patterns depend on the presence of a laryngealized vowel and on its place in the word. Contrastive prosodic patterns are, in fact, linguistic tones. Thus, \ili{Udihe} and especially its southern varieties became a tonal language of the type of languages with low tone density, like Scandinavian dialects or Latvian.

Further research is hindered by the fact that suprasegmental patterns in modern versions of \ili{Udihe} are lost due to the influence of \ili{Russian}.

The study of \ili{Udihe} varieties shows how conventional the line between synchronic and diachronic descriptions of language can be. A synchronic description and comparison of modern varieties can shed light on the history of these varieties.

The study of the \ili{Udihe} dialect continuum reveals the internal mechanisms of language change. It becomes obvious that in all territorial varieties of the \ili{Udihe} language, similar trends acted, but in different areas they appeared with different degrees of intensity. The internal “structure” of the dialect continuum has been demonstrated: the allegro-style of one dialect corresponds to the full style of the neighbouring dialect, which produces a new allegro-style, and so on.

It is easy to see that each dialect is an independent system that is not reducible to the system of another dialect. At the same time, mutual understanding between speakers of different dialects is preserved and can be quite easy.

As a result of these considerations, it becomes clear that the idea of the unity of “language” (the concept of “such and such a language”) in the absence of codification is often misleading and causes disputes among linguists. “One language” is an abstraction. In reality, there are specific systems – idiolects that can be combined into dialects, language varieties and separate languages. However, the higher the taxon, the more likely it is that the various systems are combined. The foregoing does not apply only to cases where the “language” means a codified norm.\largerpage[2]


\section*{Abbreviations}

\begin{tabularx}{.55\textwidth}{@{}lQ}
Sim. & data from \citet{SimonovKialundziuga1998}\\
Shn. & data from \citet{Shneider1936}\\
AM & allegro mode of pronunciation\\
FM & full mode of pronunciation\\
V & any vowel\\
\end{tabularx}\begin{tabularx}{.45\textwidth}{lQ@{}}
C & any consonant\\
V̰ & vowel with creaky voice phonation\\
V̀ & vowel bearing low tone\\
V̤ & vowel with breathy voice phonation\\
\end{tabularx}

\section*{Acknowledgements}

I use this opportunity and thank all the participants of my field trips to the \ili{Russian} Far East. I would like to express my particular gratitude to \name{Natalia}{Kuznetsova}, who helped to collect high-quality data on the phonetics of \ili{Udihe} varieties, as well as to \name{Kirill}{Maslinskij}, who greatly helped in the interpretation of the material.

The study was supported by the \ili{Russian} Science Foundation, grant 20-18-00250 “Tonal languages of the world: on-line data base and atlas”.

{\sloppy\printbibliography[heading=subbibliography,notkeyword=this]}
\end{document}
