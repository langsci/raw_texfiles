\documentclass[output=paper,colorlinks,citecolor=brown]{langscibook}
\ChapterDOI{10.5281/zenodo.7053367}
\author{Bayarma Khabtagaeva\affiliation{University of Naples L’Orientale, Department of Asian, African and Mediterranian Studies}}
%\ORCIDs{}

\title{On some shared and distinguishing features of Nercha and Khamnigan Ewenki dialects}

\abstract{The present paper is a brief addition to the author’s recent monograph (\citealt{Khabtagaeva2017}) which deals with Mongolic elements in Ewenki dialects (Barguzin, Nercha, Baunt and North-Baikal) spoken in the territory of Buryatia, Russia. Today Nercha Ewenki is no longer spoken. During the initial years of Soviet rule (ca. 1918--1932) some Nercha speakers crossed the border into Manchuria, China, and today their descendants are speakers of Manchurian Khamnigan Ewenki. The aim of this paper is to find out similarities and differences between the extinct Nercha Ewenki dialect and Manchurian Khamnigan Ewenki.\\\\
\keywords{Nercha Ewenki, Khamnigan Ewenki, etymology, Mongolic loanwords}
}


\IfFileExists{../localcommands.tex}{
   \addbibresource{../localbibliography.bib}
   \usepackage{langsci-optional}
\usepackage{langsci-gb4e}
\usepackage{langsci-lgr}

\usepackage{listings}
\lstset{basicstyle=\ttfamily,tabsize=2,breaklines=true}

%added by author
% \usepackage{tipa}
\usepackage{multirow}
\graphicspath{{figures/}}
\usepackage{langsci-branding}

   
\newcommand{\sent}{\enumsentence}
\newcommand{\sents}{\eenumsentence}
\let\citeasnoun\citet

\renewcommand{\lsCoverTitleFont}[1]{\sffamily\addfontfeatures{Scale=MatchUppercase}\fontsize{44pt}{16mm}\selectfont #1}
  
   %% hyphenation points for line breaks
%% Normally, automatic hyphenation in LaTeX is very good
%% If a word is mis-hyphenated, add it to this file
%%
%% add information to TeX file before \begin{document} with:
%% %% hyphenation points for line breaks
%% Normally, automatic hyphenation in LaTeX is very good
%% If a word is mis-hyphenated, add it to this file
%%
%% add information to TeX file before \begin{document} with:
%% %% hyphenation points for line breaks
%% Normally, automatic hyphenation in LaTeX is very good
%% If a word is mis-hyphenated, add it to this file
%%
%% add information to TeX file before \begin{document} with:
%% \include{localhyphenation}
\hyphenation{
affri-ca-te
affri-ca-tes
an-no-tated
com-ple-ments
com-po-si-tio-na-li-ty
non-com-po-si-tio-na-li-ty
Gon-zá-lez
out-side
Ri-chárd
se-man-tics
STREU-SLE
Tie-de-mann
}
\hyphenation{
affri-ca-te
affri-ca-tes
an-no-tated
com-ple-ments
com-po-si-tio-na-li-ty
non-com-po-si-tio-na-li-ty
Gon-zá-lez
out-side
Ri-chárd
se-man-tics
STREU-SLE
Tie-de-mann
}
\hyphenation{
affri-ca-te
affri-ca-tes
an-no-tated
com-ple-ments
com-po-si-tio-na-li-ty
non-com-po-si-tio-na-li-ty
Gon-zá-lez
out-side
Ri-chárd
se-man-tics
STREU-SLE
Tie-de-mann
}
   \boolfalse{bookcompile}
   \togglepaper[2]%%chapternumber
}{}

\shorttitlerunninghead{Nercha and Khamnigan Ewenki dialects}
\begin{document}
\maketitle

\section{Introduction}

The present paper is a brief addition to the author’s recent monograph (\citealt{Khabtagaeva2017}) which focuses on \ili{Mongolic} loanwords in Ewenki dialects spoken in the territory of Buryatia, Russia. The idea to write this paper was motivated by the author’s trip in September 2017, to carry out fieldwork among \ili{Mongolic} and Tungusic people in Manchuria.

The main goal of the published monograph (\citealt{Khabtagaeva2017}) was to clarify the status of \ili{early Mongolic} (i.e. non-\ili{Buryat}) and later \ili{Mongolic} or \ili{Buryat} layers in Ewenki, with the main finding being that almost all phonetic characteristics of \ili{Mongolic} loanwords in Ewenki dialects coincide with \ili{Khamnigan Mongol}. This implies that an “early” \ili{Mongolic} language related to Modern \ili{Khamnigan Mongol}\footnote{Nowadays \ili{Khamnigan Mongol} has three dialects, which are close to each other linguistically but differ geographically. \ili{Khamnigan Mongol} is spoken in three different countries. (1) The Trans-Baikalian or Onon Khamnigan dialect is spoken in the Chita Province of Russia, and in several Regions of the \ili{Buryat} Aga National District. (2) The \ili{Khamnigan Mongol} dialect of Mongolia is spoken in the northeastern region of Mongolia in Khentei Province and in Dornod Province. The Khamnigan dialect of Dadal sum of Khentei Province was investigated by Uray- Kőhalmi. (3) The Manchurian Khamnigan dialect is spoken in the northeastern region of China, in the Hulunbuir district in the Ewenki Autonomous Arrow of the Old Bargut Banner.\il{Khamnigan Mongol}

\ili{Khamnigan Mongol} is an endangered \ili{Mongolic} language, its speakers total approximately 2,600 persons: Onon Khamnigans number 600, Manchurian Khamnigans 1,500, while Khamnigan Mongols of Mongolia 530 speakers (for more details and references, see \citealt{Khabtagaeva2017}: 49).} was spoken in the Transbaikalian territory before the \ili{Buryat} tribes arrived here, and this language had a considerable effect on Ewenki dialects in the earlier stages of borrowing (\citealt{Khabtagaeva2017}: 200--201).

The introductory part of \citet{Khabtagaeva2017} provides a brief overview of the Ewenki dialects of Buryatia (\ili{Barguzin}, \ili{Nercha}, \ili{Baunt} and \ili{North-Baikal}), their language status, common phonetic and semantic features and differences among them. However, in contrast with other Ewenki dialects, \ili{Nercha} Ewenki is not spoken any more (\citealt{Khabtagaeva2017}: 34--35). In my published monograph I very briefly mentioned the \il{Evenki!Khamnigan}Khamnigan Ewenki people, but I did not explicitly connect them with the \ili{Nercha} Ewenki people. The fieldwork among the \il{Evenki!Khamnigan}Khamnigan Ewenki people (September 2017, Hulunbuir, China) has proven my early assumptions to be correct.

The aim of this paper is to compare the lexical material of \ili{Nercha} Ewenki published by \citet{Castrén1856} with the Manchurian \il{Evenki!Khamnigan}Khamnigan Ewenki data published by \citet{Janhunen1991} and our fieldwork materials.

\section{Ewenki dialects}

The Ewenki\il{Evenki} language belongs to the Tungusic language family, traditionally believed to form the \ili{Altaic} language family together with the \ili{Turkic} and \ili{Mongolic} languages. Although the classification of Tungusic languages is not definitive, Tungusic languages are traditionally divided into two branches (for more details, see \citealt{Khabtagaeva2017}: 17--18). The northern branch\il{Northern Tungusic} includes 51 dialects and subdialects of Ewenki\il{Evenki}, Ewen\il{Even} or Lamut, \ili{Negidal}, etc. The southern branch\il{Southern Tungusic} is divided into two groups. The Manchuric group consists of \ili{Jurchen} or Old \ili{Manchu}, \ili{Manchu}, and its sole living member \ili{Sibe} {\textasciitilde} Sibo (Xibe\il{Sibe} {\textasciitilde} Xibo). The Amuric group\il{Amuric group of Tungusic} includes \ili{Nanai}, \ili{Ulcha}, Orok, \ili{Oroch}, and \ili{Udihe}.\footnote{A new classification of Tungusic languages was recently proposed by \citet[16]{Janhunen2012a}, where the northern branch includes the \ili{Ewenic} group as well as the \ili{Udegheic} group, while the southern branch consists of the \ili{Nanaic} and \ili{Jurchenic} groups. Accordingly, the Tungusic languages are divided into two branches. The \textsc{Northern Tungusic} branch includes the \ili{Ewenic} group: a) Siberian \ili{Ewenic} (Ewen, \ili{Arman}, Ewenki\il{Evenki}, \ili{Negidal}, Orochen\il{Oroqen} and \ili{Urulga} dialect of \il{Evenki!Khamnigan}Khamnigan Ewenki); and b) Manchurian \ili{Ewenic} (\ili{Mankovo} dialect of \il{Evenki!Khamnigan}Khamnigan Ewenki, \ili{Nonni Solon}, \ili{Hailar} Solon and \ili{Ongkor} Solon). The \ili{Udegheic} group includes Udeghe\il{Udihe} and \ili{Oroch}. The \textsc{\ili{Southern Tungusic}} branch contains two groups: a) the \ili{Nanaic} (\ili{Nanai}, \ili{Kili} and \ili{Kilen}) and \ili{Ulchaic} (\ili{Ulcha} and Orok) group and b) the \ili{Jurchenic} group (\ili{Jurchen}, \ili{Manchu} and \ili{Sibe}).}

The Ewenki people live in Russia, China and Mongolia,\footnote{A group of Ewenkis of unknown size also lives near Lake Buir in Northeastern Mongolia.} scattered over a vast territory. \citet[130]{Janhunen1997} suggests a differentiation of the Ewenki people into two groups: (1) the \ili{Siberian} Ewenki in Russia and (2) the Manchurian Ewenki in China.

\begin{enumerate}
    \item In Russia, the Ewenki people live in small groups of a few thousand people, far from each other. They number approximately 37,100 (Ewenki statistics in 2010).\footnote{The Republic of Yakutia – 21,008; the Krasnoyarsk Region – 4,372; the Khabarovsk Region – 4,101; the Republic of Buryatia – 2,974; the Province of Amur – 1,481; the Zabaikalsk Region~– 1,387; the Province of Irkutsk – 1,272; the Province of \ili{Sakhalin} – 209 and other Provinces – 312. On the geographical position of the Ewenki dialects in Russia, see the appended map in \citegen{Vasilevic1958} dictionary.} The Ewenki\il{Evenki} language in Russia has 51 dialects, which can be grouped into northern, southern and eastern branches (\citealt{Atknine1997}: 115; \citealt{Bulatova2002}: 270--271). Literary or \ili{Standard Ewenki} is based on the Stony \ili{Tunguska} dialect, which belongs to the southern branch (for more details, see \citealt{Khabtagaeva2017}: 19--23);\il{Evenki!Northern}\il{Evenki!Eastern}\il{Evenki!Southern}
    
    \item In Northeastern China, along the rivers Mergel and \ili{Hailar}, in the Ewenki Autonomous Banner of the Hulun Buir Region at least four historically and linguistically distinct groups live in close proximity:

\begin{enumerate}
    \item the \ili{Solon} Ewenkis – the largest group (25,000 or 90\% of the Ewenki by nationality in official statistics);\footnote{Historically they are a satellite group of the \ili{Dagur}. Just like the Dagurs, the \ili{Solon} Ewenkis used to live in the \ili{Zeya} basin north of the Middle Amur, from where the Qing government relocated them to other parts of Manchuria in 1654. Today they live in four different places. One place is the \ili{Zeya} basin in Russia (\citealt{Bulatova1987}), while the other three are in Manchuria, China: along the \ili{Nonni} basin in Nehe country, in the Ewenki Autonomous Banner of Hulun Buir, and in Ili Region of Xinjiang (\citealt{Janhunen1997}: 130--131).}
    
    \item the Orochens – another major group of Ewenki\il{Evenki} speakers in China forming four geographical and dialectal subgroups, including approximately 7,000 people;\footnote{Their ancestors moved to China from the regions north of the Amur during the 18th century and nowadays settled the two Khingan Ranges (\citealt{Janhunen1997}: 131--132).}
    
    \item the Manchurian Reindeer Tungus or \textit{Yakute}, officially classified together with the \ili{Solon} as \textit{Ewenke}, numbering about 200 individuals;\footnote{They live in the region of the river Jiliuhe in the Hulun Buir Province, and are culturally close to the Orochen\il{Oroqen} and different from the \ili{Solon} (\citealt{Janhunen1997}: 132).}
    
    \item the \il{Evenki!Khamnigan}Khamnigan Ewenkis – the last Ewenki group in China that numbers about 10,000 people (for more details, see below).
\end{enumerate}
\end{enumerate}

While they differentiate themselves from each other, most groups are erroneously called \textit{Ewenke} by the administration and some \ili{Chinese} linguists (\citealt{Janhunen1997}: 130--131).


\section{Nercha Ewenki}

The homeland of \ili{Nercha} Ewenki people was the southeastern part of Transbaikalia. Today the territory is situated in the Aga \ili{Buryat} National District of Chita Province. Politically and geographically it is not Buryatia, but it is the place where \ili{Buryat} people have lived for a long time.

We do not have any current information about \ili{Nercha} Ewenki speakers. The dialect is likely extinct, with only historical and ethnographic materials available. \citet{Uvarova2006} focuses on historical facts, the social structure, and some cultural features of 18th to 20th century \ili{Nercha} Ewenki peoples, with no examination of their language. The historical materials, including statistics, different government ordinances, and laws, were collected from various archives in Russia, originating mostly from the 19th century and author’s fieldwork material collected in the 1970s (\citealt{Uvarova2006}: 10--13). According to Uvarova, by the early 20th century the \ili{Nercha} Ewenki people merged with the \ili{Buryat} and \ili{Russian} populations of Transbaikalia. In the 1970 census, only 32 persons indicated the Ewenki\il{Evenki} language as their mother tongue (\citealt{Uvarova2006}: 9; 122). The total assimilation with the Buryats was completed by the 1980s.

% Some facts about the \ili{Nercha} Ewenki people can be learned from Tugolukov’s 1975 paper.
Based on archive materials, \citet{Tugolukov1975} characterizes the traditional culture and religion of \ili{Nercha} \textit{murčen}s, i.e. ‘horse breeders’, provides statistical data on the Ewenki tribes in the 18th and 19th centuries, and describes various historical facts connected with the Gantimur dynasty. The ethnic history of the \ili{Nercha} Ewenki people is closely linked with the name Gantimur, who was a leading representative of the \ili{Nercha} \textit{murčen} people. From the middle of 17th century he started to pay tribute to Russia. His oldest son Katana converted to Christianity and was presented to the court of \ili{Russian} Tsar Petr Alekseevich. From the end of the 17th century \ili{Nercha} \textit{murčen}s started to guard the \ili{Russian}-\ili{Chinese} border. The exact origin of Gantimur is unclear. According to various archive sources (for details, see \citealt{Tugolukov1975}: 98--103), he was Tungus or \ili{Dagur}. \citet[101--102]{Tugolukov1975} concludes that Gantimur was of Tungusic origin but “Dagurified”, which may be confirmed with Gantimur being of \ili{Nercha} Ewenki origin from the Dulikagir tribe, i.e. the original Ewenki tribe which is not \ili{Mongolic}.

Here it also needs to be considered, as noted correctly by \citet[16]{Janhunen1991}, that the Transbaikalian aboriginal population represents a complex mixture of \ili{Mongolic} and Tungusic elements. According to \citet[16]{Janhunen1991}, roughly one half of the \il{Evenki!Khamnigan}Khamnigan Ewenki clans had Tungusic-speaking ancestors, while the ancestors of the other half were \ili{Mongolic}-speaking. Therefore, it cannot be excluded that Gantimur was of either Tungusic or \ili{Mongolic} origin. It is important to mention that the \ili{Nercha} Ewenkis, like other Ewenki people, had no right to marry a person of the same tribe until the ninth generation, which may have led to their assimilation with \ili{Mongolic} people. In Tugolukov’s opinion (\citeyear{Tugolukov1975}: 109), before the 12th or 13th century, the Ewenki people were reindeer breeders and later assimilated to \ili{Mongolic} people. Besides intermarriage, another reason for assimilation was the change of lifestyle from reindeer breeding to horse breeding.\footnote{Ewenki legends tell us that when they came out with their reindeers to the steppe, they were forced to change their lifestyle because of the absence of reindeer moss (\citealt{Tugolukov1975}: 106).}

The full list of \ili{Nercha} Ewenki tribes and their numbers in 1762 and 1823 are given by \citet[93]{Tugolukov1975}, whose data includes 14 tribes in the following order: \textit{Balikagir} (87 persons), \textit{Bajagir} (477 persons), \textit{Wakasil} (42 persons), \textit{Gunow} (258 persons), \textit{Dolot} (126 persons), \textit{Dulikagir} (436 persons), \textit{Konur} (171 persons), \textit{Lunikir} (304 persons), \textit{Namyat} (512 persons), \textit{Počegor} (248 persons), \textit{Sortoc} (240 persons), \textit{Uzon} (358 persons), \textit{Ulyat} (140 persons), and \textit{Čemčagir} (346 persons). The speakers of the Borzya dialect of Manchurian \il{Evenki!Khamnigan}Khamnigan Ewenki comprise the \textit{Balkiegid}, \textit{Bayagiid}, \textit{Čimčagiid}, \textit{Duligaad} and \textit{Marugiid} clans, while the speakers of the \ili{Urulyungui} dialect include the \textit{Namied}, \textit{Altaŋganuud}, \textit{Čibčinüüd}, \textit{Jaltood}, \textit{Koonud}, \textit{Dulaad}, \textit{Galjood}, \textit{Ulied} and \textit{Üjeed} (\citealt{Janhunen1991}: 14). The forms with the final consonant \textit{-d} (\textit{-gid}/\textit{-gad}, \textit{-Ad}, \textit{-nuud}) in the clan names are possibly connected with the \ili{Mongolic} plural forms. Thus, if we compare all the above-mentioned clan names, the common clans of \ili{Nercha} and \il{Evenki!Khamnigan}Khamnigan Ewenki people are \textit{Bajagir}, \textit{Dulikagir}, \textit{Čemčagir}, \textit{Konur}, \textit{Ulyat} and \textit{Namyat}.

It seems that most of the \ili{Nercha} Ewenkis were the ancestors of Manchurian \il{Evenki!Khamnigan}Khamnigan Ewenkis, who crossed the border into Manchuria and China during the initial years of Soviet rule (ca. 1918--1932), moving with the ancestors of Khamnigan Mongols and Shinehen Buryats for “a better life” (\citealt{Janhunen1997}: 130).

\begin{sloppypar}
The first linguist who worked on the \ili{Nercha} Ewenki dialect with native consultants was M. A. \citet{Castrén1856}. His work was translated into \ili{Russian} by Ye. I. Titov and published as the appendix in his \textit{Tungus–\ili{Russian} dictionary} (\citealt{Titov1926}). Titov was the second and possibly last researcher of \ili{Nercha} Ewenki. Titov met with people from the Bultegir and Turuyagir clans (\citealt{Titov1926}: ix) and claimed that the people spoke similar dialects. Today the people of the Turuyagir clan belong to the \ili{Baunt} Ewenki people (\citealt[29]{Khabtagaeva2017}), while the people of the Bultegir clan were mentioned among the \ili{Dagur} people (\citealt{Vasilevic1969}: 265). The lexical material in \citegen{Vasilevic1969} \textit{Ewenki\il{Evenki}–\ili{Russian} dictionary} was probably collected from Castrén’s and Titov’s works.\footnote{During my comparison of the materials of Castrén’s and Vasilevič’s dictionaries, I noticed that Castrén’s transcription does not always coincide with Vasilevič’s. For instance, the consonant \textit{c} in Vasilevič’s dictionary is incorrectly indicated as \textit{č} and I had to correct it in my own materials (\citealt{Khabtagaeva2017}).}
\end{sloppypar}

Another important fact is that the histories of \ili{Nercha} Ewenkis and Transbaikalian Khamnigan Mongols are closely related to each other. The two groups were likely often confused in \ili{Russian} official documents and were considered to be \textit{Tungus} and later \textit{Ewenki\il{Evenki}}. For instance, the \ili{Russian} anthropologist \citet[77]{Tal’ko1904} wrote that the number of \textit{Ewenkis} who adopted Buddhism exceeded the number of Buryats. Or when we read that at the beginning of the 19th century there were six Buddhist monasteries in the \ili{Urulga} territory built by \textit{Ewenkis} (\citealt{Galdanova1983}: 41), we have to suppose that they mean the \ili{Nercha} Ewenki and Transbaikalian (or Onon) \ili{Khamnigan Mongol} people.


\section{Khamnigan Ewenki}

Speakers of Manchurian \il{Evenki!Khamnigan}Khamnigan Ewenki of China use two separate Ewenki dialects, both distinct from all other known Ewenki\il{Evenki} dialects and also relatively different from each other. The first one is the Borzya dialect, referring to the Upper Borzya river on the \ili{Russian} side, while the second one is the \ili{Urulyungui} dialect, referring to the river \ili{Urulyungui} also on the \ili{Russian} side (\citealt{Janhunen1991}: 11--12; \citeyear{Janhunen1997}: 132--133). Nowadays, geographically both rivers are situated in Transbaikalia, where the \ili{Nercha} Ewenki people formerly lived.

As \citet[132--133]{Janhunen1997} states, this group is ethnolinguistically the most atypical one, in that it is more or less congruous with the population speaking the \ili{Khamnigan Mongol} language. They are bilingual in Ewenki and \ili{Khamnigan Mongol}. Mongol is the dominant community language, while Ewenki\il{Evenki} is mainly used as an additional means of communication within many families. The main languages of interethnic communication between \il{Evenki!Khamnigan}Khamnigan Ewenkis and Khamnigan Mongols is \ili{Khamnigan Mongol},\footnote{As other \ili{Mongolian} speakers (e.g. \ili{Buryat}, \ili{Dagur}, \ili{Ordos}, \ili{Khorchin}, Kharchin, etc.) in Inner Mongolia, Khamnigan Mongols and \il{Evenki!Khamnigan}Khamnigan Ewenkis speak Standard \ili{Mongolian}.} so the Khamnigan Mongols do not speak Ewenki. As Janhunen noted in the 1990s (\citeyear{Janhunen1991}: 11--15), \ili{Khamnigan Mongol} is a stable and homogeneous variety, showing no essential variation within the community. Ewenki\il{Evenki} is destined eventually to lose its remaining role as a family language. By contrast, \ili{Khamnigan Mongol} may well further strengthen its position as the principal community language in the Mergel region (for details, see \citealt{Janhunen1997}: 130).

When interviewing the \il{Evenki!Khamnigan}Khamnigan Ewenki people during our fieldwork, we observed a slight shift in self-identification compared to Janhunen’s description. Our Khamnigan informants mostly emphasized their Ewenki\il{Evenki} affiliation, stating that Ewenki is probably the original language of the community while \ili{Khamnigan Mongol} (termed [\textit{evenkilig mongol üge}] by them) was adopted later “somewhere in Russia”.  At the same time, the speakers supposed that the \textit{Boorǰi} variety existed earlier and was the original language of the \il{Evenki!Khamnigan}Khamnigan Ewenki community while \textit{Namieetii} was a Mongolized variety adopted by them in Manchuria. \textit{Boorǰi} is a Borzya dialect, while \textit{Namieetii} is connected to the clan \textit{Namied}, which is listed among the \il{Evenki!Khamnigan}Khamnigan Ewenki tribes of \ili{Mongolic} origin and as speakers of the \ili{Urulyungui} dialect (\citealt{Janhunen1991}: 16, 14). As has been mentioned earlier, besides the \textit{Namied}, the speakers of the \ili{Urulyungui} dialect include the \textit{Altaŋganuud}, \textit{Čibčinüüd}, \textit{Koonuud}, \textit{Dulaad}, \textit{Galǰood}, \textit{Ulied} and \textit{Üǰeed} clans, while the speakers of Borzya dialect comprise the \textit{Balkiegid}, \textit{Bayagiid}, \textit{Čimčagiid}, \textit{Duligaad} and \textit{Marugiid} clans (\citealt{Janhunen1991}: 14).

The slight change of \il{Evenki!Khamnigan}Khamnigan Ewenkis in “self-classification” as \textit{Ewenki} people in recent years may be connected to the recent promotion of Ewenki culture in China and governmental support for the endangered Ewenki culture, which enables the Khamnigans to profit from their Ewenki identity. The idea of the unity of the three Ewenki branches of China has been promoted in various spheres. For instance, in \ili{Hailar} we had an opportunity to meet with the \ili{Solon} Ewenki scholar Do Dorji, who is the chief-editor of \textit{Ewenki-Chinese} (\citeyear{Dorji1998}) and \textit{Ewenki-Mongol} (\citeyear{Dorji2013}) dictionaries where he treats the “three Ewenki branches” together, taking \ili{Solon} as a base. Also, our 58-year-old informant told us that her mother tongue is very close to the \ili{Solon}, Orochen\il{Oroqen} and Yakut Ewenki\il{Evenki} varieties. When we asked about their connection with the \ili{Nercha} Ewenki people, our informant replied that she did not hear about them recently but knows that they came from the \ili{Russian} side and her parents were fluent in \ili{Russian}.\footnote{The personal names of our informant’s parents were unique: the mother’s name was Darima (a typical \ili{Buryat} or \ili{Khamnigan Mongol} name, it does not exist among other \ili{Mongolic} people such as \ili{Khalkha}, \ili{Oirat}, Dariganga, etc.), the father’s name was Prank (cf. \ili{Russian} \textit{Frank}) and the uncle’s name was Mark.} It is an interesting fact that our \il{Evenki!Khamnigan}Khamnigan Ewenki informants (a 59 year-old man and a 58 year-old woman) are fluent in both Ewenki (\textit{Boorǰi} and \textit{Namieetii}) varieties and \ili{Khamnigan Mongol}. As they told us, both Ewenki varieties are very close to each other, but \textit{Boorǰi} is the ‘original’ Ewenki\il{Evenki}, while \textit{Namieetii} is “mixed and primitive”, i.e. \textit{Namieetii} has more \ili{Mongolic} and \ili{Russian} elements.

It is important to mention the religion of the \il{Evenki!Khamnigan}Khamnigan Ewenki people. As our informants told us, the ‘original’ religion was Christianity, after migration from Russia in the village where the \il{Evenki!Khamnigan}Khamnigan Ewenki people lived there was one church with a priest, but during the Cultural Revolution in China in 1960-s and 70-s they were forced to ‘give up’ their religion and became atheists. Now the \il{Evenki!Khamnigan}Khamnigan Ewenki people believe in shamanism, regularly visit shamans, and perform shamanistic rites.

A brief grammatical sketch of \il{Evenki!Khamnigan}Khamnigan Ewenki was provided by \citet{Janhunen1991}.


\section{Comparative analysis of Nercha Ewenki and Khamnigan Ewenki materials}

Linguistically, the \ili{Nercha} dialect belongs in the southern sibilant group, representing the hissing type (\textit{s-}, \textit{VsV})\footnote{The main criterion used in the classification of the dialects is the fate of the \ili{Common Tungusic} consonant *\textit{s} in initial and intervocalic positions. In the three branches there appear the representations \textit{h}, \textit{s} and \textit{š}. E.g. \ili{Common Tungusic} ‘ear’ and ‘woman’ in northern\il{Evenki!Northern} group (spirant \textit{h-}, \textit{VhV}) are \textit{hēn} and \textit{ahī}, in southern\il{Evenki!Southern} (sibilant \textit{s-}, \textit{VsV} and \textit{š-}, \textit{VšV} are \textit{sēn} / \textit{šēn} and \textit{asī} / \textit{ašī}, in eastern\il{Evenki!Eastern} (sibilant and spirant \textit{s-}, \textit{VhV}) are \textit{sēn} and \textit{ahī}, respectively (\citealt{Khabtagaeva2017}: 20).} (\citealt{Atknine1997}: 115; \citealt{Bulatova2002}: 270--271; \citealt{Khabtagaeva2017}: 19--20). The \il{Evenki!Khamnigan}Khamnigan Ewenki variety also shares this phonetic feature.\footnote{According to \citegen{Castrén1856} and \citegen{Janhunen1991} materials, Common Ewenki \textit{sele} ‘iron’: \ili{Nercha} Ewenki, \il{Evenki!Khamnigan}Khamnigan Ewenki \textit{sele}; Common Ewenki \textit{asī} ‘woman’: \ili{Nercha} \textit{āśi}, \il{Evenki!Khamnigan}Khamnigan Ewenki \textit{asī}, \il{Evenki!Common}Common Ewenki \textit{ēsa} ‘eye’: \ili{Nercha} Ewenki \textit{īsa} {\textasciitilde} \textit{ēsa}, \il{Evenki!Khamnigan}Khamnigan Ewenki (\ili{Urulga}) \textit{iesa}, (Borzya) \textit{īsa}, etc.}

The following provides a list of common Tungusic\il{Common Tungusic} words in the \ili{Nercha} Ewenki and Manchurian \il{Evenki!Khamnigan}Khamnigan Ewenki dialects from \citegen{Castrén1856} and \citegen{Janhunen1991} works, and from our fieldwork material. The \ili{Nercha} Ewenki dialect includes the \ili{Urulga} and \ili{Mankovo} subdialects, while \il{Evenki!Khamnigan}Khamnigan Ewenki includes Borzya and \ili{Urulyungui}. The extinct \ili{Mankovo} subdialect of \ili{Nercha} Ewenki corresponds to Borzya, while the extinct \ili{Urulga} subdialect of \ili{Nercha} Ewenki has a close relation with \ili{Urulyungui} in Manchuria (\citealt{Janhunen1991}: 12). Since the influence from \ili{Solon} Ewenki language is assumed, the comparative data from the Hulunbuir \ili{Solon} dialect is added (\citealt{Dorji1998}; \citeyear{Dorji2013}; \citealt{Chaoke2014bB}). Additionaly, the data of Orochen\il{Oroqen} or \ili{Oroqen} of Hulunbuir (\citealt{Chaoke2014a}), \ili{Siberian} Ewenki dialects (\citealt{Vasilevic1958}) and other Tungusic languages are added (\citealt{Cincius1975B}; \citealt{Hauer1952}; \citealt{Stary1990}; \citealt{Zikmundová2013a}).


\subsection{Shared lexicon}

In most cases, \ili{Nercha} Ewenki and \il{Evenki!Khamnigan}Khamnigan Ewenki have the common Tungusic\il{Common Tungusic} vocabulary, which is also present in other Ewenki dialects.

\ea
A list of potential cognates shared by \ili{Nercha} and \il{Evenki!Khamnigan}Khamnigan Ewenki:

\begin{xlist}
    \ex Kinship terms:

\begin{xlist}
    \ex ‘elder brother’: \ili{Nercha} Ewenki, \il{Evenki!Khamnigan}Khamnigan Ewenki \textbf{\textit{akin}};

    cf. \ili{Solon} Ewenki \textit{ahiŋ}; Orochen\il{Oroqen} \textit{akin}; \ili{Siberian} Common Ewenki \textit{akīn};

    \textit{other Northern Tungusic}: Lamut \textit{akan}; \ili{Negidal} \textit{ahin}; \textit{\ili{Southern Tungusic}}: \ili{Nanai}, \ili{Ulcha}, \ili{Udihe} \textit{aga}; \ili{Oroch} \textit{aki}; Orok \textit{aka}; \ili{Manchu} \textit{agu}; \ili{Sibe} \textit{aʁů\textsuperscript{n}} 
    
    (\citealt{Castrén1856}: 71a; \citealt{Janhunen1991}: 73; \citealt{Dorji1998}: 14; \citealt{Chaoke2014a}: 160; \citealt{Vasilevic1958}: 21a; \citealt{Cincius1975B} 1: 23; \citealt{Hauer1952} 1: 14; \citealt{Zikmundová2013a}: 204);


    \ex ‘younger brother’: \ili{Nercha} Ewenki, \il{Evenki!Khamnigan}Khamnigan Ewenki \textbf{\textit{nekün}};
    
    cf. \ili{Solon} Ewenki \textit{nǝhuŋ}; Orochen\il{Oroqen} \textit{nekun}; \ili{Siberian} Ewenki: \ili{Sakhalin} \textit{nekūn}; \ili{Podkamennyj}, \ili{May}, \ili{Tokko}, \ili{Tommot}, \ili{Urmi}, \ili{Uchur}, \ili{Chulman} \textit{nekē};

    \textit{other Northern Tungusic}: Lamut \textit{nu}; \ili{Negidal} \textit{nekun {\textasciitilde} nehun}; \textit{\ili{Southern Tungusic}}: \ili{Nanai}, \ili{Ulcha}, Orok \textit{neu}; \ili{Udihe} \textit{neŋu}; \ili{Oroch} \textit{neku}; \ili{Manchu} \textit{non}
    
    (\citealt{Castrén1856}: 85; \citealt{Janhunen1991}: 24; \citealt{Dorji1998}: 480a; \citealt{Chaoke2014a}: 161; \citealt{Vasilevic1958}: 302a; \citealt{Cincius1975B} 1: 617b-618; \citealt{Hauer1952} 3: 720); 


    \ex ‘daughter-in-law’: \ili{Nercha} Ewenki, \il{Evenki!Khamnigan}Khamnigan Ewenki \textbf{\textit{kükin}};

    cf. \ili{Solon} Ewenki \textit{hühiŋ}; \ili{Siberian} Ewenki: \ili{Podkamennyj}, Yerbogochen\il{Yerbogachyon}, \ili{Barguzin}, \ili{Zeya}, \ili{Ilimpeya}, \ili{May}, \ili{Tokko}, \ili{Tommot}, \ili{Uchur} \textit{kukīn};

    \textit{other Northern Tungusic}: Lamut \textit{köken}; \ili{Negidal} \textit{kukin {\textasciitilde} kuhin}; Remaining lgs. \textit{n.a.}\footnote{The abbreviation \textit{n.a.} means that the form is not available, it may be present but not found in the considered dictionaries.}
    
    (\citealt{Castrén1856}: 81; \citealt{Janhunen1991}: 23; \citealt{Dorji1998}: 303a; \citealt{Vasilevic1958}: 217a; \citealt{Cincius1975B} 1: 425b); 


    \ex ‘mother’: \ili{Nercha} Ewenki (\ili{Urulga}), \il{Evenki!Khamnigan}Khamnigan Ewenki (\ili{Urulyungui}) \textbf{\textit{enin}};

    cf. \ili{Solon} Ewenki \textit{ǝniŋ}; Orochen\il{Oroqen} \textit{enin}; \ili{Siberian} Common Ewenki \textit{eńin};

    \textit{other Northern Tungusic}: Lamut \textit{eńin}; \ili{Negidal} \textit{enin}; \textit{\ili{Southern Tungusic}}: \ili{Nanai}, \ili{Ulcha}, \ili{Udihe}, \ili{Oroch}, Orok \textit{eni}; \ili{Manchu} \textit{eniyen} ‘female moose’; \ili{Sibe} \textit{ǝńi}
    
    (\citealt{Castrén1856}: 73; \citealt{Janhunen1991}: 23; \citealt{Dorji1998}: 179a; \citealt{Chaoke2014a}: 160; \citealt{Vasilevic1958}: 562a; \citealt{Cincius1975B} 2: 456; \citealt{Hauer1952} 1: 253; \citealt{Zikmundová2013a}: 210);

\end{xlist}


    \ex Names of body parts:

\begin{xlist}
    \ex ‘mouth’: \ili{Nercha} Ewenki, \il{Evenki!Khamnigan}Khamnigan Ewenki \textbf{\textit{amŋga}};

    cf. \ili{Solon} Ewenki \textit{amma}; Orochen\il{Oroqen} \textit{amŋa}; \ili{Siberian} Common Ewenki \textit{amŋa};

    \textit{other Northern Tungusic}: Lamut \textit{amŋa {\textasciitilde} amga}; \ili{Negidal} \textit{amŋa}; \textit{\ili{Southern Tungusic}}: \ili{Nanai}, \ili{Ulcha}, \ili{Udihe} \textit{aŋma}; \ili{Oroch} \textit{amma}; Orok \textit{amŋa {\textasciitilde} aŋma}; \ili{Manchu} \textit{aŋga}; \ili{Sibe} \textit{an} 
    
    (\citealt{Castrén1856}: 72; \citealt{Janhunen1991}: 51; \citealt{Dorji1998}: 27b; \citealt{Chaoke2014a}: 159; \citealt{Vasilevic1958}: 28b; \citealt{Cincius1975B} 1: 38; \citealt{Hauer1952} 1: 52; \citealt{Zikmundová2013a}: 205); 

    \ex ‘beard’: \ili{Nercha} Ewenki, \il{Evenki!Khamnigan}Khamnigan Ewenki \textbf{\textit{gurgakta}};

    cf. \ili{Solon} Ewenki \textit{guggatt}; \ili{Siberian} Common Ewenki \textit{gurgakta};

    \textit{other Northern Tungusic}: Lamut \textit{gurgat}; \ili{Negidal} \textit{goigakta}; \textit{\ili{Southern Tungusic}}: \ili{Nanai} \textit{gogakta}; \ili{Udihe} \textit{gugakta}; \ili{Oroch} \textit{gugahta}; Orok \textit{gudahta}; Remaining lgs. \textit{n.a}.
    
    (\citealt{Castrén1856}: 82; \citealt{Janhunen1991}: 40; \citealt{Dorji1998}: 230a; \citealt{Vasilevic1958}: 97b; \citealt{Cincius1975B}: 173b);

    \ex ‘skin’: \ili{Nercha} Ewenki, \il{Evenki!Khamnigan}Khamnigan Ewenki \textbf{\textit{nanda}};

    cf. \ili{Solon} Ewenki \textit{nanda}; Orochen\il{Oroqen} \textit{nana}; \ili{Siberian} Ewenki: \ili{Podkamennyj}, \ili{Barguzin}, Yerbogochen\il{Yerbogachyon}, \ili{Zeya}, \il{Evenki!Nepa}Nepa, \ili{Sakhalin}, \ili{Tokko}, \ili{Urmi}, \ili{Chumikan} \textit{nanna}; \ili{Ilimpeya}, \ili{May}, \ili{North-Baikal}, \ili{Uchur} \textit{nanda}; \ili{Sym} \textit{nandra};

    \textit{other Northern Tungusic}: Lamut \textit{nanra {\textasciitilde} nanda}; \ili{Negidal} \textit{nana}; \textit{\ili{Southern Tungusic}}: \ili{Nanai}, \ili{Ulcha} \textit{nanta}; \ili{Udihe} \textit{ńehe}; \ili{Oroch} \textit{naha {\textasciitilde} nasa {\textasciitilde} niha}; Orok \textit{natta {\textasciitilde} nata}; \ili{Manchu}, \ili{Sibe} \textit{n.a.} 
    
    (\citealt{Castrén1856}: 85; \citealt{Janhunen1991}: 49; \citealt{Dorji1998}: 473b; \citealt{Chaoke2014a}: 158; \citealt{Vasilevic1958}: 276b; \citealt{Cincius1975B} 1: 583b); 

    \ex ‘neck’: \ili{Nercha} Ewenki \textbf{\textit{nikimna}} \textit{{\textasciitilde}} \textbf{\textit{nikinma}}; \il{Evenki!Khamnigan}Khamnigan Ewenki\\ \textbf{\textit{nikimne}};

    cf. \ili{Solon} Ewenki \textit{niham}; Orochen\il{Oroqen} \textit{nikimna}; \ili{Siberian} Ewenki: \ili{Podkamennyj}, \il{Evenki!Nepa}Nepa, \ili{Upper Lena}, \ili{North-Baikal}, \ili{Tungir}, \ili{Zeya}, \ili{Aldan}, \ili{Urmi}, \ili{Ayan}, \ili{Sakhalin} \textit{nikinma}; Yerbogochen\il{Yerbogachyon} \textit{nikinma {\textasciitilde} nikimŋa}; \il{Evenki!Nepa}Nepa \textit{nikinmńa}; \ili{Barguzin} \textit{nikin}; \ili{Ilimpeya}, \ili{North-Baikal}, \ili{Uchur} \textit{nikimda};

    \textit{other Northern Tungusic}: Lamut \textit{ńiken}; \ili{Negidal} \textit{nihma}; \textit{\ili{Southern Tungusic}}: \ili{Ulcha} \textit{ńikin}; Orok \textit{nikimńa}; Remaining lgs. \textit{n.a.}

    (\citealt{Castrén1856}: 85; \citealt{Janhunen1991}: 49; \citealt{Dorji1998}: 487b; \citealt{Chaoke2014a}: 159; \citealt{Vasilevic1958}: 291b; \citealt{Cincius1975B} 1: 591);
\end{xlist}

    \ex Names of animals:

\begin{xlist}
    \ex ‘fish’: \ili{Nercha} Ewenki, \il{Evenki!Khamnigan}Khamnigan Ewenki \textbf{\textit{oldo}};

    cf. \ili{Solon} Ewenki \textit{n.a}.; Orochen\il{Oroqen} \textit{olo}; \ili{Siberian} Ewenki: \ili{Podkamennyj}, \il{Evenki!Nepa}Nepa, Yerbogochen\il{Yerbogachyon}, \ili{Tungir}, \ili{Zeya}, \ili{Aldan}, \ili{Urmi}, \ili{Chumikan}, \ili{Sakhalin} \textit{ollo}; \ili{Ilimpeya}, \ili{North-Baikal}, \ili{Uchur}, \ili{Upper Lena} \textit{oldo}; \ili{Sym} \textit{oldro};

    \textit{other Northern Tungusic}: Lamut \textit{olra}; \ili{Negidal} \textit{olo}; \textit{\ili{Southern Tungusic}}: \ili{Nanai} \textit{olo}; \ili{Ulcha}, Orok \textit{holto} ‘cooked fish’; \ili{Udihe} \textit{oloho} ‘cooked fish’; \ili{Oroch} \textit{okto < *olto}; \ili{Manchu}, \ili{Sibe} \textit{n.a.} 
    
    (\citealt{Castrén1856}: 75; \citealt{Janhunen1991}: 23; \citealt{Chaoke2014a}: 156; \citealt{Vasilevic1958}: 320a; \citealt{Cincius1975B} 2: 14); 

    \ex ‘kind of duck’: \ili{Nercha} Ewenki, \il{Evenki!Khamnigan}Khamnigan Ewenki \textbf{\textit{tarmi}};

    cf. \ili{Solon} Ewenki \textit{n.a}.; \ili{Siberian} Common Ewenki \textit{tarmī} ‘drake’

    \textit{other Northern Tungusic}: Lamut ; \ili{Negidal} ; \textit{\ili{Southern Tungusic}}: \ili{Nanai}, \ili{Ulcha} \textit{tarmi}; \ili{Udihe} \textit{tanmi}; \ili{Oroch} \textit{tajmi}; Orok \textit{n.a.}; \ili{Manchu} \textit{tarmin}; \ili{Sibe} \textit{n.a.}
    
    (\citealt{Castrén1856}: 86; \citealt{Janhunen1991}: 48; \citealt{Vasilevic1958}: 388b; \citealt{Cincius1975B} 2: 169a; \citealt{Hauer1952} 3: 891);

    \ex ‘goose’: \ili{Nercha} Ewenki, \il{Evenki!Khamnigan}Khamnigan Ewenki \textbf{\textit{nuŋnakī}};

    cf. \ili{Solon} Ewenki \textit{nunnahi}; Orochen\il{Oroqen} \textit{niunnaki}; \ili{Siberian} Ewenki: \ili{Podkamennyj}, \ili{Ayan}, \ili{Aldan}, \ili{Barguzin}, Yerbogochen\il{Yerbogachyon}, \ili{Ilimpeya}, \ili{May}, \ili{Sakhalin}, \ili{Tommot}, \ili{Tungir}, \ili{Urmi}, \ili{Uchur}, \ili{Chumikan} \textit{ńuŋńakī}; \il{Evenki!Nepa}Nepa, \ili{Tokmin} \textit{numńakī};

    \textit{other Northern Tungusic}: Lamut \textit{n.a.}; \ili{Negidal} \textit{ńoŋńahi}; \textit{\ili{Southern Tungusic}}: \ili{Nanai} \textit{ńoŋńa}; \ili{Ulcha} \textit{ńuŋńa}; \ili{Udihe} \textit{ńuŋńai}; \ili{Oroch} \textit{ńuŋńahi}; Orok \textit{nuŋna {\textasciitilde} nuŋńa}; \ili{Manchu} \textit{nioŋniyaha}; \ili{Sibe} \textit{n.a.}
    
    (\citealt{Castrén1856}: 86; \citealt{Janhunen1991}: 51; \citealt{Dorji1998}: 497b; \citealt{Chaoke2014a}: 156; \citealt{Vasilevic1958}: 307b; \citealt{Cincius1975B} 1: 646b; \citealt{Hauer1952} 2: 707);

    \ex ‘squirrel’: \ili{Nercha} Ewenki (\ili{Mankovo}), \il{Evenki!Khamnigan}Khamnigan Ewenki (Borzya) \textbf{\textit{ülükī}};

    cf. \ili{Solon} Ewenki \textit{uluhi}; \ili{Siberian} Ewenki: \ili{Podkamennyj}, \il{Evenki!Nepa}Nepa, Yerbogochen\il{Yerbogachyon}, \ili{Ilimpeya}, \ili{Barguzin}, \ili{Tungir}, \ili{Zeya}, \ili{Aldan}, \ili{Uchur}, \ili{Urmi}, \ili{Sakhalin} \textit{ulukī}; \ili{Ayan} \textit{olokī};

    \textit{other Northern Tungusic}: Lamut \textit{uliki}; \ili{Negidal} \textit{oluki}; \textit{\ili{Southern Tungusic}}: \ili{Nanai} \textit{hulu}; Orok, \ili{Ulcha} \textit{holo}; \ili{Udihe} \textit{olohi}; \ili{Oroch} \textit{oloki}; \ili{Manchu}, \ili{Sibe} \textit{ulhu}
    
    (\citealt{Castrén1856}: 78; \citealt{Janhunen1991}: 52; \citealt{Chaoke2014bB}: 42; \citealt{Vasilevic1958}: 440a; \citealt{Cincius1975B} 2: 263b; \citealt{Hauer1952} 3: 957; \citealt{Stary1990}: 92);

\end{xlist}

    \ex Inanimate nature:

\begin{xlist}
    \ex ‘river’: \ili{Nercha} Ewenki, \il{Evenki!Khamnigan}Khamnigan Ewenki \textbf{\textit{bira}};

    cf. \ili{Solon} Ewenki \textit{bera}; Orochen\il{Oroqen} \textit{bira}; \ili{Siberian} Common Ewenki \textit{bira};

    \textit{other Northern Tungusic}: Lamut \textit{bira}; \ili{Negidal} \textit{bija}; \textit{\ili{Southern Tungusic}}: \ili{Nanai}, \ili{Ulcha} \textit{bira}; \ili{Udihe} \textit{b\textsuperscript{j}eæsa}; \ili{Oroch} \textit{biaka}; Orok \textit{n.a.}; \ili{Manchu}, \ili{Sibe} \textit{bira}
    
    (\citealt{Castrén1856}: 95; \citealt{Janhunen1991}: 23; \citealt{Dorji1998}: 71b; \citealt{Chaoke2014a}: 153; \citealt{Vasilevic1958}: 56a; \citealt{Cincius1975B} 1: 84; \citealt{Hauer1952} 1: 96; \citealt{Zikmundová2013a}: 206);

\end{xlist}

    \ex Names of metals:

\begin{xlist}
    \ex ‘iron’: \ili{Nercha} Ewenki, \il{Evenki!Khamnigan}Khamnigan Ewenki \textbf{\textit{sele}};

    cf. \ili{Solon} Ewenki \textit{sǝl}; Orochen\il{Oroqen} \textit{sele}; \ili{Siberian} Ewenki: \ili{Podkamennyj}, \il{Evenki!Nepa}Nepa, \ili{Barguzin}, \ili{Tungir}, \ili{Zeya}, \ili{Uchur}, \ili{Urmi}, \ili{Aldan}, \ili{Chumikan}, \ili{Ayan}, \ili{Sakhalin} \textit{sele}; Yerbogochen\il{Yerbogachyon}, \ili{Ilimpeya}, Vilyui\ili{Vilyuy} \textit{hele}; \ili{Sym}, \ili{North-Baikal} \textit{šele};

    \textit{other Northern Tungusic}: Lamut \textit{hel}; \ili{Negidal} \textit{sele}; \textit{\ili{Southern Tungusic}}: \ili{Nanai}, \ili{Ulcha}, \ili{Udihe}, \ili{Oroch}, Orok, \ili{Manchu} \textit{sele}; \ili{Sibe} \textit{selei [ǰugūn]} ‘railroad’ 
    
    (\citealt{Castrén1856}: 91; \citealt{Janhunen1991}: 23; \citealt{Dorji1998}: 592b; \citealt{Chaoke2014a}: 163; \citealt{Vasilevic1958}: 376a; SSTMJa 2: 140; Hauer 3: 778; \citealt{Stary1990}: 76);

\end{xlist}

    \ex Names of plants:

\begin{xlist}
    \ex ‘tree’: \ili{Nercha} Ewenki, \il{Evenki!Khamnigan}Khamnigan Ewenki \textbf{\textit{mō}};

    cf. \ili{Solon} Ewenki \textit{mo}; Orochen\il{Oroqen} \textit{mō}; \ili{Siberian} Common Ewenki \textit{mō};

    \textit{other Northern Tungusic}: Lamut, \ili{Negidal} \textit{mō}; \textit{\ili{Southern Tungusic}}: \ili{Nanai}, \ili{Ulcha}, \ili{Udihe}, \ili{Oroch}, Orok, \ili{Manchu} \textit{mō}; \ili{Sibe} \textit{mo}
    
    (\citealt{Castrén1856}: 96; \citealt{Janhunen1991}: 30; \citealt{Dorji1998}: 444b; \citealt{Chaoke2014a}: 153; \citealt{Vasilevic1958}: 254b; \citealt{Cincius1975B} 1: 540; \citealt{Zikmundová2013a}: 218);

    \ex ‘pine tree’: \ili{Nercha} Ewenki, \il{Evenki!Khamnigan}Khamnigan Ewenki \textbf{\textit{ǰagda}};

    cf. \ili{Solon} Ewenki \textit{ǰadda {\textasciitilde} ǰagda}; \ili{Siberian} Common Ewenki \textit{ǰagda};

    \textit{other Northern Tungusic}: Lamut, \ili{Negidal} \textit{ǰagda}; \textit{\ili{Southern Tungusic}}: \ili{Nanai} \textit{ǰāgda}; \ili{Ulcha} \textit{ǰagda}; \ili{Udihe}, \ili{Oroch}, Orok \textit{n.a.}; \ili{Manchu} \textit{ǰaqdan}; \ili{Sibe} \textit{n.a.}
    
    (\citealt{Castrén1856}: 93; \citealt{Janhunen1991}: 23; \citealt{Dorji1998}: 350a; \citealt{Vasilevic1958}: 146b; \citealt{Cincius1975B} 1: 242);

    \ex ‘steppe lily’: \ili{Nercha} Ewenki, \il{Evenki!Khamnigan}Khamnigan Ewenki: Borzya \textbf{\textit{ǰōgta}};

    cf. \ili{Solon} Ewenki \textit{ǰōtt}; \ili{Siberian} Ewenki: \ili{Barguzin} \textit{ǰokta};

    \textit{\ili{Southern Tungusic}}: \ili{Manchu} \textit{ǰoqtoda {\textasciitilde} ǰoqtonda}; Remaining lgs. \textit{n.a}.
    
    (\citealt{Castrén1856}: 93; \citealt{Janhunen1991}: 42; \citealt{Dorji1998}: 379b; \citealt{Vasilevic1958}: 112b; \citealt{Cincius1975B} 1: 262);

\end{xlist}

    \ex Time measures:

\begin{xlist}
    \ex ‘tomorrow’: \ili{Nercha} Ewenki, \il{Evenki!Khamnigan}Khamnigan Ewenki \textbf{\textit{timī}};

    cf. \ili{Solon} Ewenki \textit{timašiŋ}; Orochen\il{Oroqen} \textit{timāna}; \ili{Siberian} Common Ewenki \textit{tïmānī};

    \textit{other Northern Tungusic}: Lamut \textit{t’em’en}; \ili{Negidal} \textit{t’emana}; \textit{\ili{Southern Tungusic}}: \ili{Nanai} \textit{čimaj}, \ili{Ulcha}, \ili{Udihe} \textit{tïmani}; \ili{Oroch} \textit{timaki}; Orok \textit{čimani}; \ili{Manchu} \textit{čimari}; \ili{Sibe} \textit{čimar} 
    
    (\citealt{Castrén1856}: 87; \citealt{Janhunen1991}: 29; \citealt{Dorji1998}: 689b; \citealt{Chaoke2014a}: 155; \citealt{Vasilevic1958}: 410b; \citealt{Cincius1975B} 2: 181; \citealt{Zikmundová2013a}: 207);

    \ex ‘day’: \ili{Nercha} Ewenki: \ili{Urulga} \textit{inaŋ}, Man’kovo \textit{ineŋī}; \il{Evenki!Khamnigan}Khamnigan Ewenki \textbf{\textit{ineŋī}};

    cf. \ili{Solon} Ewenki \textit{inǝŋ} ‘afternoon’; Orochen\il{Oroqen} \textit{iniyi}; \ili{Siberian} Ewenki: \ili{Podkamennyj}, \il{Evenki!Nepa}Nepa, \ili{Ilimpeya} \textit{ineŋ} ‘noon; south’; \ili{Sym}, \ili{Barguzin}, \ili{Nercha}, \ili{Tungir}, \ili{Zeya}, \ili{Urmi}, \ili{Chumikan}, \ili{Ayan}, \ili{Sakhalin} \textit{ineŋi} ‘day’;

    ‘day, in the daytime’: \textit{other Northern Tungusic}: Lamut \textit{ineŋ}; \ili{Negidal} \textit{ineŋi}; \textit{\ili{Southern Tungusic}}: \ili{Nanai} \textit{inie}; \ili{Udihe}, \ili{Oroch} \textit{ineŋi}; Orok \textit{inuŋi}; \ili{Ulcha} \textit{ineŋni}; \ili{Manchu} \textit{ineŋgi}; \ili{Sibe} \textit{ǝnǝŋ} ‘today’ 
    
    (\citealt{Castrén1856}: 74; \citealt{Janhunen1991}: 57; \citealt{Dorji1998}: 333b; \citealt{Chaoke2014a}: 154; \citealt{Vasilevic1958}: 175a; \citealt{Cincius1975B} 1: 318; \citealt{Hauer1952} 2: 499; \citealt{Zikmundová2013a}: 210);

\end{xlist}

    \ex Buildings and their parts:

\begin{xlist}
    \ex ‘door’: \ili{Nercha} Ewenki, \il{Evenki!Khamnigan}Khamnigan Ewenki \textbf{\textit{ürke}};

    cf. \ili{Solon} Ewenki \textit{ükkǝ}; Orochen\il{Oroqen} \textit{urke}; \ili{Siberian} Common Ewenki \textit{urke};

    \textit{other Northern Tungusic}: Lamut \textit{urke}; \ili{Negidal} \textit{ujke}; \textit{\ili{Southern Tungusic}}: \ili{Nanai} \textit{ujke}; \ili{Ulcha} \textit{uče}, \ili{Udihe} \textit{uke {\textasciitilde} uče}; \ili{Oroch} \textit{ukke}; Orok \textit{ute}; \ili{Manchu} \textit{uče}; \ili{Sibe} \textit{uči} 
    
    (\citealt{Castrén1856}: 78; \citealt{Janhunen1991}: 23; \citealt{Dorji1998}: 727a; \citealt{Chaoke2014a}: 162; \citealt{Vasilevic1958}: 453a; \citealt{Cincius1975B} 2: 286; \citealt{Hauer1952} 3: 942; \citealt{Zikmundová2013a}: 223);

\end{xlist}

    \ex Names of foods:

\begin{xlist}
    \ex ‘meat’: \ili{Nercha} Ewenki, \il{Evenki!Khamnigan}Khamnigan Ewenki \textbf{\textit{ülde}};

    cf. \ili{Solon} Ewenki \textit{üldǝ}; Orochen\il{Oroqen} \textit{ule}; \ili{Siberian} Ewenki: \ili{Podkamennyj}, \il{Evenki!Nepa}Nepa, Yerbogochen\il{Yerbogachyon}, \ili{Tokmin}, \ili{Upper Lena}, \ili{Barguzin}, \ili{Vitim}, \ili{Tungir}, \ili{Zeya}, \ili{Aldan}, \ili{Urmi}, \ili{Ayan}, \ili{Sakhalin} \textit{ulle}; \ili{Sym} \textit{uldre}; \ili{Ilimpeya}, \ili{North-Baikal}, \ili{Uchur} \textit{ulde}; \ili{Tokma} \textit{unle}; \ili{Ayan} \textit{ulre};

    \textit{other Northern Tungusic}: Lamut \textit{ulre}; \ili{Negidal} \textit{ule}; \textit{\ili{Southern Tungusic}}: \ili{Nanai} \textit{ulikse}; \ili{Ulcha} \textit{ulse}; \ili{Udihe} \textit{ulehe}; \ili{Oroch} \textit{ulese}; Orok \textit{ulise}; \ili{Manchu}, \ili{Sibe} \textit{n.a}.
    
    (\citealt{Castrén1856}: 78; \citealt{Janhunen1991}: 41; \citealt{Dorji1998}: 728a; \citealt{Chaoke2014a}: 160; \citealt{Vasilevic1958}: 439b; \citealt{Cincius1975B} 2: 262);

\end{xlist}

    \ex Clothes, utensils:

\begin{xlist}
    \ex ‘footwear, shoes’: \ili{Nercha} Ewenki, \il{Evenki!Khamnigan}Khamnigan Ewenki \textbf{\textit{unta}};

    cf. \ili{Solon} Ewenki \textit{unta}; \ili{Siberian} Common Ewenki \textit{unta};

    \textit{other Northern Tungusic}: Lamut \textit{unta}; \ili{Negidal} \textit{onta}; \textit{\ili{Southern Tungusic}}: \ili{Nanai} \textit{ota}; \ili{Ulcha}, \ili{Udihe}, \ili{Oroch} \textit{unta}; Orok \textit{utta}; \ili{Manchu}, \ili{Sibe} \textit{n.a}.
    
    (\citealt{Castrén1856}: 77; \citealt{Janhunen1991}: 49; \citealt{Dorji1998}: 530b; \citealt{Vasilevic1958}: 448b; \citealt{Cincius1975B} 2: 275);

    \ex ‘knife’: \ili{Nercha} Ewenki: Man’kovo \textbf{\textit{üťi}} \textit{{\textasciitilde} \textbf{üči}}; \il{Evenki!Khamnigan}Khamnigan Ewenki \textbf{\textit{üči}};

    cf. \ili{Solon} Ewenki \textit{n.a.}; \ili{Siberian} Ewenki: \il{Evenki!Nepa}Nepa, \ili{Uchur}, \ili{Urmi}, \ili{Sakhalin}, \ili{Chumikan} \textit{ut-} ‘to wind, twist, twirl’;

    ‘to fix, to repair, to mend’: \textit{other Northern Tungusic}: Lamut \textit{ut- {\textasciitilde} uč-}; \ili{Negidal} \textit{ute-}; \textit{\ili{Southern Tungusic}}: \ili{Nanai} \textit{ute-} ‘to quilt clothes, blanket’; \ili{Ulcha} \textit{uteče} ‘seam’; Remaining lgs. \textit{n.a}. 
    
    (\citealt{Castrén1856}: 78; \citealt{Janhunen1991}: 43; \citealt{Vasilevic1958}: 456b; \citealt{Cincius1975B} 2: 293);

\end{xlist}

    \ex Abstract nouns connected with human life:

\begin{xlist}
    \ex ‘name’: \ili{Nercha} Ewenki, \il{Evenki!Khamnigan}Khamnigan Ewenki \textbf{\textit{gerbī}};

    cf. \ili{Solon} Ewenki \textit{gǝrbi {\textasciitilde} gǝbbi}; \ili{Siberian} Common Ewenki \textit{gerbī};

    \textit{other Northern Tungusic}: Lamut \textit{gerbe}; \ili{Negidal} \textit{gelbi}; \textit{\ili{Southern Tungusic}}: \ili{Nanai} \textit{gebu} ← \ili{Manchu}; \ili{Ulcha}, Orok \textit{gelbu}; \ili{Udihe} \textit{gegbi}; \ili{Oroch} \textit{gebbi}; \ili{Manchu} \textit{gebu}; \ili{Sibe} \textit{gǝf} 
    
    (\citealt{Castrén1856}: 81; \citealt{Janhunen1991}: 40; \citealt{Dorji1998}: 205; \citealt{Vasilevic1958}: 100b; \citealt{Cincius1975B} 1: 180; \citealt{Hauer1952} 1: 339; \citealt{Zikmundová2013a}: 211);

\end{xlist}

    \ex Numbers:

\begin{xlist}
    \ex ‘one’: \ili{Nercha} Ewenki, \il{Evenki!Khamnigan}Khamnigan Ewenki \textbf{\textit{umun}};

    cf. \ili{Solon} Ewenki \textit{ǝmuŋ}; Orochen\il{Oroqen} \textit{emun}; \ili{Siberian} Ewenki: \ili{North-Baikal}, \ili{Tokma}, \ili{Tungir} \textit{emūn}; Remaining dialects \textit{umūn};

    \textit{other Northern Tungusic}: Lamut \textit{umen}; \ili{Negidal} \textit{omon}; \textit{\ili{Southern Tungusic}}: \ili{Nanai} \textit{em}; \ili{Ulcha}, \ili{Udihe}, \ili{Oroch} \textit{omo}; Orok \textit{umūke}; \ili{Manchu} \textit{emu}; \ili{Sibe} \textit{ǝm}
    
    (\citealt{Castrén1856}: 77; \citealt{Janhunen1991}: 76; \citealt{Dorji1998}: 174b; \citealt{Chaoke2014a}: 169; \citealt{Vasilevic1958}: 444b; \citealt{Cincius1975B} 2: 270; \citealt{Hauer1952} 1: 247; \citealt{Zikmundová2013a}: 209);

    \ex ‘two’: \ili{Nercha} Ewenki, \il{Evenki!Khamnigan}Khamnigan Ewenki \textbf{\textit{ǰǖr}};

    cf. \ili{Solon} Ewenki, Orochen\il{Oroqen} \textit{ǰūr}; \ili{Siberian} Common Ewenki \textit{ǰǖr};

    \textit{other Northern Tungusic}: Lamut \textit{ǰur}; \ili{Negidal} \textit{ǰul}; \textit{\ili{Southern Tungusic}}: \ili{Nanai} \textit{ǰuer}; \ili{Ulcha} \textit{ǰuel}; \ili{Udihe}, \ili{Oroch} \textit{ǰu}; Orok \textit{dū}; \ili{Manchu} \textit{ǰuwe}; \ili{Sibe} \textit{ǰu}
    
    (\citealt{Castrén1856}: 94; \citealt{Janhunen1991}: 76; \citealt{Dorji1998}: 391b; \citealt{Chaoke2014a}: 169; \citealt{Vasilevic1958}: 143a; \citealt{Cincius1975B} 1: 276; \citealt{Hauer1952} 2: 563; \citealt{Zikmundová2013a}: 216);

    \ex ‘three’: \ili{Nercha} Ewenki, \il{Evenki!Khamnigan}Khamnigan Ewenki \textbf{\textit{ilan}};

    cf. \ili{Solon} Ewenki \textit{ilaŋ}; Orochen\il{Oroqen} \textit{ilan}; \ili{Siberian} Ewenki: \ili{Aldan}, \ili{Zeya}, \ili{Ilimpeya} \textit{yelan}; Remaining dial. \textit{ilan};

    \textit{other Northern Tungusic}: Lamut \textit{yelan}; \ili{Negidal} \textit{yelan {\textasciitilde} ilan}; \textit{\ili{Southern Tungusic}}: \ili{Nanai} \textit{yelan}; \ili{Ulcha}, \ili{Udihe}, \ili{Oroch}, Orok \textit{ila}; \ili{Manchu} \textit{ilan}; \ili{Sibe} \textit{yila\textsuperscript{n}} 
    
    (\citealt{Castrén1856}: 74; \citealt{Janhunen1991}: 76; \citealt{Dorji1998}: 326a; \citealt{Chaoke2014a}: 169; \citealt{Vasilevic1958}: 162b; \citealt{Cincius1975B} 1: 305; \citealt{Hauer1952} 2: 490; \citealt{Zikmundová2013a}: 113);

    \ex ‘four’: \ili{Nercha} Ewenki, \il{Evenki!Khamnigan}Khamnigan Ewenki \textbf{\textit{digin}};

    cf. \ili{Solon} Ewenki \textit{diγiŋ}; Orochen\il{Oroqen} \textit{diyin}; \ili{Siberian} Common Ewenki \textit{dïgin};

    \textit{other Northern Tungusic}: Lamut, \ili{Negidal} \textit{dïgin}; \textit{\ili{Southern Tungusic}}: \ili{Nanai}, \ili{Ulcha} \textit{duin}; \ili{Udihe} \textit{dï}; \ili{Oroch} \textit{dī}; Orok \textit{ǰīn}; \ili{Manchu} \textit{duin}; \ili{Sibe} \textit{duyi\textsuperscript{n}}
    
    (\citealt{Castrén1856}: 90; \citealt{Janhunen1991}: 76; \citealt{Dorji1998}: 130a; \citealt{Chaoke2014a}: 170; \citealt{Vasilevic1958}: 127b; \citealt{Cincius1975B} 1: 204; \citealt{Hauer1952} 1: 217; \citealt{Zikmundová2013a}: 113);

    \ex ‘six’: \ili{Nercha} Ewenki \textit{nüŋün {\textasciitilde}} \textbf{\textit{ńüŋün}}; \il{Evenki!Khamnigan}Khamnigan Ewenki \textbf{\textit{nüŋün}};

    cf. \ili{Solon} Ewenki \textit{niŋuŋ}; Orochen\il{Oroqen} \textit{niuŋun}; \ili{Siberian} Ewenki: \ili{North-Baikal} \textit{ńugun}; Remaining dial. \textit{ńuŋun};

    \textit{other Northern Tungusic}: Lamut \textit{ńuŋi}; \ili{Negidal} \textit{ńuŋī}; \textit{\ili{Southern Tungusic}}: \ili{Nanai}, \ili{Ulcha} \textit{ńuŋgu(n)}; \ili{Udihe}, \ili{Oroch} \textit{ńuŋu}; Orok \textit{ńuŋg’ē}; \ili{Manchu} \textit{niŋgun}; \ili{Sibe} \textit{ńiŋu\textsuperscript{n}} 
    
    (\citealt{Castrén1856}: 86; \citealt{Janhunen1991}: 76; \citealt{Dorji1998}: 490b; \citealt{Chaoke2014a}: 170; \citealt{Vasilevic1958}: 308a; \citealt{Cincius1975B} 1: 647; \citealt{Hauer1952} 2: 703; \citealt{Zikmundová2013a}: 113);

    \ex ‘seven’: \ili{Nercha} Ewenki, \il{Evenki!Khamnigan}Khamnigan Ewenki \textbf{\textit{nadan}};

    cf. \ili{Solon} Ewenki \textit{nadaŋ}; Orochen\il{Oroqen} \textit{nadan}; \ili{Siberian} Common Ewenki \textit{nadan};

    \textit{other Northern Tungusic}: Lamut, \ili{Negidal} \textit{nadan}; \textit{\ili{Southern Tungusic}}: \ili{Nanai}, \ili{Ulcha}, \ili{Udihe}, \ili{Oroch}, Orok \textit{nada}; \ili{Manchu} \textit{nadan}; \ili{Sibe} \textit{nadǝ\textsuperscript{n}}
    
    (\citealt{Castrén1856}: 85; \citealt{Janhunen1991}: 76; \citealt{Dorji1998}: 469b; \citealt{Chaoke2014a}: 170; \citealt{Vasilevic1958}: 273b; \citealt{Cincius1975B} 1: 576; \citealt{Hauer1952} 2: 684; \citealt{Zikmundová2013a}: 113);
\end{xlist}


    \ex Qualitative adjectives:

\begin{xlist}
    \ex ‘good’: \ili{Nercha} Ewenki, \il{Evenki!Khamnigan}Khamnigan Ewenki \textbf{\textit{aja}};

    cf. \ili{Solon} Ewenki \textit{aja}; Orochen\il{Oroqen} \textit{aji}; \ili{Siberian} Common Ewenki \textit{aja};

    \textit{other Northern Tungusic}: Lamut \textit{aj}; \ili{Negidal} \textit{aja}; \textit{\ili{Southern Tungusic}}: \ili{Nanai}, \ili{Ulcha}, \ili{Udihe}, \ili{Oroch}, Orok \textit{aja}; \ili{Manchu}, \ili{Sibe} \textit{n.a}.
    
    (\citealt{Castrén1856}: 71; \citealt{Janhunen1991}: 95; \citealt{Dorji1998}: 49a; \citealt{Chaoke2014a}: 165; \citealt{Vasilevic1958}: 44a; \citealt{Cincius1975B} 1: 18);

    \ex ‘bad’: \ili{Nercha} Ewenki, \il{Evenki!Khamnigan}Khamnigan Ewenki \textbf{\textit{erǖ}};

    cf. \ili{Solon} Ewenki \textit{ǝru}; Orochen\il{Oroqen} \textit{eru}; \ili{Siberian} Ewenki: \ili{Podkamennyj}, \il{Evenki!Nepa}Nepa, Yerbogochen\il{Yerbogachyon}, \ili{Ilimpeya}, \ili{Tokma}, \ili{Tungir}, \ili{Zeya}, \ili{Aldan}, \ili{Uchir} \textit{erū}; \ili{Podkamennyj}, Yerbogochen\il{Yerbogachyon}, \ili{Ilimpeya} \textit{urū};

    \textit{other Northern Tungusic}: \textit{n.a}.; \textit{\ili{Southern Tungusic}}: \ili{Nanai} \textit{eru}; \ili{Ulcha} \textit{orkin}; \ili{Oroch}, Orok \textit{orke}; \ili{Manchu}, \ili{Sibe} \textit{n.a}.
    
    (\citealt{Castrén1856}: 73; \citealt{Janhunen1991}: 52; \citealt{Dorji1998}: 185a; \citealt{Chaoke2014a}: 165; \citealt{Vasilevic1958}: 566a; \citealt{Cincius1975B} 2: 465--466);

    \ex ‘warm’: \ili{Nercha} Ewenki, \il{Evenki!Khamnigan}Khamnigan Ewenki \textbf{\textit{nama}};

    cf. \ili{Solon} Ewenki \textit{namaddi} (< *\textit{namagdi} < \textit{ńama+gdi} \ili{Solon} denominal noun/adjective suffix); Orochen\il{Oroqen} \textit{niama}; \ili{Siberian} Common Ewenki \textit{ńama};

    \textit{other Northern Tungusic}: Lamut \textit{ńam}; \ili{Negidal} \textit{ńamagdï}; \textit{\ili{Southern Tungusic}}: \ili{Nanai}, \ili{Ulcha}, \ili{Oroch} \textit{ńama}; \ili{Udihe} \textit{ńamahi}; Orok \textit{ńamauli}; \ili{Manchu}, \ili{Sibe} \textit{n.a.}
    
    (\citealt{Castrén1856}: 85; \citealt{Janhunen1991}: 57; \citealt{Dorji1998}: 471a; \citealt{Chaoke2014a}: 166; \citealt{Vasilevic1958}: 310b; \citealt{Cincius1975B} 1: 630--631);

    \ex ‘other’: \ili{Nercha} Ewenki, \il{Evenki!Khamnigan}Khamnigan Ewenki \textbf{\textit{hüntü}};

    cf. \ili{Solon} Ewenki \textit{ǝntu}; \ili{Siberian} Ewenki: \ili{Podkamennyj}, \il{Evenki!Nepa}Nepa, Yerbogochen\il{Yerbogachyon}, \ili{Ilimpeya}, \ili{Barguzin}, \ili{Tungir}, \ili{Zeya}, \ili{Aldan}, \ili{Uchur}, \ili{Urmi}, \ili{Chumikan}, \ili{Sakhalin} \textit{huŋtu}; \ili{Sym} \textit{uŋtu};

    \textit{other Northern Tungusic}: Lamut \textit{hunte}; \ili{Negidal} \textit{heŋte {\textasciitilde} huŋtu}; \textit{\ili{Southern Tungusic}}: \ili{Udihe} \textit{hoŋto}; \ili{Oroch} \textit{hoŋto {\textasciitilde} honto}; \ili{Nanai}, \ili{Ulcha}, Orok \textit{n.a}.; \ili{Manchu} \textit{enču}; \ili{Sibe} \textit{unč\textsuperscript{u}} ‘different’
    
    (\citealt{Castrén1856}: 83; \citealt{Janhunen1991}: 49; \citealt{Dorji1998}: 180b; \citealt{Vasilevic1958}: 496a; \citealt{Cincius1975B} 2: 349--350; \citealt{Hauer1952} 1: 249; \citealt{Zikmundová2013a}: 224);

\end{xlist}


    \ex Demonstratives:

\begin{xlist}
    \ex ‘this’: \ili{Nercha} Ewenki \textbf{\textit{er}}, \il{Evenki!Khamnigan}Khamnigan Ewenki: Borzya \textbf{\textit{eri}} \textit{{\textasciitilde}} \textbf{\textit{er}};

    cf. \ili{Solon} Ewenki \textit{ǝri}; Orochen\il{Oroqen} \textit{eri}; \ili{Siberian} Common Ewenki \textit{er {\textasciitilde} eri};

    \textit{other Northern Tungusic}: Lamut \textit{er}; \ili{Negidal} \textit{ej}; \textit{\ili{Southern Tungusic}}: \ili{Nanai}, \ili{Ulcha} \textit{ej}; \ili{Udihe}, \ili{Oroch} \textit{eji}; Orok \textit{er {\textasciitilde} eri}; \ili{Manchu} \textit{ere}; \ili{Sibe} \textit{er}
    
    (\citealt{Castrén1856}: 73; \citealt{Janhunen1991}: 69; \citealt{Dorji1998}: 183b; \citealt{Chaoke2014a}: 164; \citealt{Vasilevic1958}: 564a; \citealt{Cincius1975B} 2: 460--461; \citealt{Hauer1952} 1: 255; \citealt{Zikmundová2013a}: 210);

    \ex ‘that’: \ili{Nercha} Ewenki: \ili{Mankovo} \textbf{\textit{tar}}, \ili{Urulga} \textbf{\textit{tari}} \textit{{\textasciitilde}} \textbf{\textit{tara}}; \il{Evenki!Khamnigan}Khamnigan Ewenki: Borzya \textbf{\textit{tari}} \textit{{\textasciitilde}} \textbf{\textit{tar}};

    cf. \ili{Solon} Ewenki, Orochen\il{Oroqen} \textit{tari}; \ili{Siberian} Common Ewenki \textit{tar {\textasciitilde} tarā {\textasciitilde} tari};

    \textit{other Northern Tungusic}: Lamut \textit{tar}; \ili{Negidal} \textit{taj}; \textit{\ili{Southern Tungusic}}: \ili{Nanai} \textit{tej}; \ili{Ulcha} \textit{tï}; \ili{Udihe} \textit{tei}; \ili{Oroch} \textit{tī}; Orok \textit{tari}; \ili{Manchu} \textit{tere}; \ili{Sibe} \textit{tǝr}
    
    (\citealt{Castrén1856}: 86; \citealt{Janhunen1991}: 69; \citealt{Dorji1998}: 666a; \citealt{Chaoke2014a}: 164; \citealt{Vasilevic1958}: 387b; \citealt{Cincius1975B} 2: 164--165; \citealt{Hauer1952} 3: 903; \citealt{Zikmundová2013a}: 222);

\end{xlist}


    \ex Interrogatives and quantifiers:

\begin{xlist}
    \ex ‘which’: \ili{Nercha} Ewenki, \il{Evenki!Khamnigan}Khamnigan Ewenki \textbf{\textit{abgū}};

    cf. \ili{Solon} Ewenki \textit{awu} ‘who’; \ili{Siberian} Ewenki: \ili{Tokma}, \ili{Zeya}, \ili{Aldan} \textit{abgū}; \ili{Podkamennyj}, \il{Evenki!Nepa}Nepa, \ili{North-Baikal}, \ili{Barguzin}, \ili{Zeya}, \ili{Uchir}, \ili{Urmi}, \ili{Chumikan}, \ili{Sakhalin} \textit{awgū}; \il{Evenki!Nepa}Nepa \textit{awawū}; \ili{Urmi} \textit{awagū};

    \textit{other Northern Tungusic}: Lamut \textit{awgida}; \ili{Negidal} \textit{awwu {\textasciitilde} awgu {\textasciitilde} au}; \textit{\ili{Southern Tungusic}}: \ili{Nanai}, \ili{Ulcha} \textit{hawuj}; \ili{Udihe}, \ili{Oroch} \textit{n.a}.; Orok \textit{hāwu}; \ili{Manchu}, \ili{Sibe} \textit{ai} 
    
    (\citealt{Castrén1856}: 72; \citealt{Janhunen1991}: 71; \citealt{Dorji1998}: 48b; \citealt{Vasilevic1958}: 13b; \citealt{Cincius1975B} 1: 4; \citealt{Hauer1952} 1: 15; \citealt{Zikmundová2013a}: 204; see also \citealt{Hölzl2018a}: 315--330);

    \ex ‘how many’: \ili{Nercha} Ewenki: \ili{Urulga} \textbf{\textit{adi}}, Man’kovo \textbf{\textit{adī}}; \il{Evenki!Khamnigan}Khamnigan Ewenki \textbf{\textit{adī}};

    cf. \ili{Solon} Ewenki, Orochen\il{Oroqen} \textit{adi}; \ili{Siberian} Ewenki: \il{Evenki!Nepa}Nepa, Yerbogochen\il{Yerbogachyon}, \ili{Upper Lena}, \ili{North-Baikal}, \ili{Barguzin}, \ili{Tungir}, \ili{Zeya}, \ili{Aldan}, \ili{Uchur}, \ili{Urmi}, \ili{Sakhalin} \textit{ad\={ï}};

    \textit{other Northern Tungusic}: Lamut, \ili{Negidal} \textit{adï}; \textit{\ili{Southern Tungusic}}: \ili{Nanai}, \ili{Ulcha} \textit{hadu}; \ili{Udihe}, \ili{Oroch} \textit{adï}; Orok \textit{n.a.}; \ili{Manchu} \textit{udu}; \ili{Sibe} \textit{ut} 
    
    (\citealt{Castrén1856}: 72; \citealt{Janhunen1991}: 71; \citealt{Dorji1998}: 10a; \citealt{Chaoke2014a}: 165; \citealt{Vasilevic1958}: 18a; \citealt{Cincius1975B} 1: 14--15; \citealt{Hauer1952} 3: 944; \citealt{Zikmundová2013a}: 224);

    \ex ‘many’: \ili{Nercha} Ewenki: Man’kovo, \il{Evenki!Khamnigan}Khamnigan Ewenki \textbf{\textit{kete}};

    cf. \ili{Solon} Ewenki \textit{hǝtǝ} ‘extremely’; \ili{Siberian} Ewenki: \ili{Podkamennyj}, \il{Evenki!Nepa}Nepa, Yerbogochen\il{Yerbogachyon}, \ili{Ilimpeya}, \ili{Tokma}, \ili{Upper Lena}, \ili{Tungir}, \ili{Aldan}, \ili{Uchur}, \ili{Chumikan} \textit{kete};

    \textit{other Tungusic}: \ili{Nanai} \textit{ketu}; \ili{Ulcha} \textit{kete {\textasciitilde} ketu {\textasciitilde} ket}; \ili{Udihe} \textit{ketu}; \ili{Oroch} \textit{ketu}; Orok \textit{ketette} ‘a little bit’; other Tungusic \textit{n.a.}
    
    (\citealt{Castrén1856}: 79; \citealt{Janhunen1991}: 42; \citealt{Chaoke2014bB}: 510; \citealt{Vasilevic1958}: 231b; \citealt{Cincius1975B} 1: 455--456);

\end{xlist}


    \ex Verbs:

\begin{xlist}
    \ex ‘to find’: \ili{Nercha} Ewenki, \il{Evenki!Khamnigan}Khamnigan Ewenki \textbf{\textit{baka-}};

    cf. \ili{Solon} Ewenki \textit{baha-}; \ili{Siberian} Common Ewenki \textit{baka-};

    \textit{other Northern Tungusic}: Lamut \textit{baq-}; \ili{Negidal} \textit{baha-}; \textit{\ili{Southern Tungusic}}: \ili{Nanai} \textit{bā-}; \ili{Ulcha} \textit{bā- {\textasciitilde} baqa-}; \ili{Udihe} \textit{b’a-}; Orok \textit{bā- {\textasciitilde} baqqa-}; \ili{Oroch} \textit{bā-}; \ili{Manchu} \textit{baha-}; \ili{Sibe} \textit{n.a.}
    
    (\citealt{Castrén1856}: 94; \citealt{Janhunen1991}: 82; \citealt{Dorji1998}: 53b; \citealt{Vasilevic1958}: 48a; \citealt{Hauer1952} 1: 66; \citealt{Cincius1975B} 1: 66--67);

    \ex ‘to come’: \ili{Nercha} Ewenki, \il{Evenki!Khamnigan}Khamnigan Ewenki \textbf{\textit{eme-}};

    cf. \ili{Solon} Ewenki \textit{ǝmǝ-}; \ili{Siberian} Common Ewenki \textit{eme-};

    \textit{other Northern Tungusic}: Lamut \textit{em-}; \ili{Negidal} \textit{eme-}; \textit{\ili{Southern Tungusic}}: \ili{Nanai} \textit{eme-}; \ili{Udihe} \textit{eme-}; \ili{Oroch} \textit{emegi-} ‘to return’; \ili{Ulcha}, Orok, \ili{Manchu}, \ili{Sibe} \textit{n.a}.
    
    (\citealt{Castrén1856}: 73; \citealt{Janhunen1991}: 66; \citealt{Dorji1998}: 173b; \citealt{Vasilevic1958}: 558a; \citealt{Cincius1975B} 2: 452);

    \ex ‘to step’: \ili{Nercha} Ewenki, \il{Evenki!Khamnigan}Khamnigan Ewenki \textbf{\textit{girkü-}};

    cf. \ili{Solon} Ewenki \textit{n.a.}; \ili{Siberian} Ewenki: \ili{Podkamennyj}, \ili{North-Baikal}, \ili{Barguzin} \textit{gerku-}; Remainig dial. \textit{girku-};

    \textit{other Northern Tungusic}: Lamut \textit{gerka-}; \ili{Negidal} \textit{gīhu-}; \textit{\ili{Southern Tungusic}}: \ili{Nanai}, \ili{Ulcha}, Orok \textit{giran-}; \ili{Udihe} \textit{geæna-}; \ili{Oroch} \textit{gia-}; \ili{Manchu} \textit{giru-}; \ili{Sibe} \textit{guri-} ‘to move’
    
    (\citealt{Castrén1856}: 82; \citealt{Janhunen1991}: 25; \citealt{Vasilevic1958}: 89b; \citealt{Cincius1975B} 1: 154--155; \citealt{Hauer1952} 1: 360; \citealt{Zikmundová2013a}: 212);

    \ex ‘to read’: \ili{Nercha} Ewenki: \ili{Urulga}; \il{Evenki!Khamnigan}Khamnigan Ewenki: Borzya \textbf{\textit{taŋir-}};

    cf. \ili{Solon} Ewenki \textit{n.a}.; \ili{Siberian} Common Ewenki \textit{taŋ-};

    \textit{other Northern Tungusic}: Lamut, \ili{Negidal} \textit{taŋ-}; \textit{\ili{Southern Tungusic}}: \ili{Nanai} \textit{taon-}; \ili{Ulcha}, Orok \textit{taun}; \ili{Udihe}, \ili{Oroch} \textit{taŋi-}; \ili{Manchu}, \ili{Sibe} \textit{n.a.} 
    
    (\citealt{Castrén1856}: 86; \citealt{Janhunen1991}: 25; \citealt{Vasilevic1958}: 386a; \citealt{Cincius1975B} 2: 161);

    \ex ‘to milk’: \ili{Nercha} Ewenki, \il{Evenki!Khamnigan}Khamnigan Ewenki: Borzya \textbf{\textit{śiri-}};

    cf. \ili{Solon} Ewenki \textit{sirǝ-} ‘to wring out (a towel)’; \ili{Siberian} Ewenki: \ili{Podkamennyj}, \il{Evenki!Nepa}Nepa, \ili{Tokma}, \ili{Barguzin}, \ili{Tungir}, \ili{Aldan}, \ili{Uchur}, \ili{Urmi}, \ili{Chumikan}, \ili{Ayan}, \ili{Sakhalin} \textit{sir-} ‘to reap’; Yerbogochen\il{Yerbogachyon}, \ili{Ilimpeya} \textit{hir-}; \ili{North-Baikal} \textit{širi-};

    \textit{other Northern Tungusic}: Lamut \textit{her-}; \ili{Negidal} \textit{sij-}; \textit{\ili{Southern Tungusic}}: \ili{Nanai} \textit{sere-}; \ili{Ulcha} \textit{siri-}; \ili{Udihe}, \ili{Oroch} \textit{sī-}; Orok \textit{sere-}; \ili{Manchu} \textit{siri-}; \ili{Sibe} \textit{n.a.}
    
    (\citealt{Castrén1856}: 84; \citealt{Janhunen1991}: 80; \citealt{Chaoke2017B}: 126; \citealt{Vasilevic1958}: 356b; \citealt{Cincius1975B} 2: 93; \citealt{Hauer1952} 3: 806);

    \ex ‘to know’: \ili{Nercha} Ewenki, \il{Evenki!Khamnigan}Khamnigan Ewenki \textbf{\textit{sā-}};

    cf. \ili{Solon} Ewenki, Orochen\il{Oroqen} \textit{sā-}; \ili{Siberian} Ewenki: \ili{Podkamennyj}, \il{Evenki!Nepa}Nepa, \ili{Tokma}, \ili{Barguzin}, \ili{Nercha}, \ili{Tungir}, \ili{Zeya}, \ili{Aldan}, \ili{Uchur}, \ili{Urmi}, \ili{Chumikan}, \ili{Ayan}, \ili{Sakhalin} \textit{sā-}; Yerbogochen\il{Yerbogachyon}, \ili{Ilimpeya} \textit{hā-}; \ili{Sym}, \ili{North-Baikal} \textit{šā-};

    \textit{other Northern Tungusic}: Lamut \textit{ha-}; \ili{Negidal} \textit{sa-}; \textit{\ili{Southern Tungusic}}: \ili{Nanai}, \ili{Ulcha}, \ili{Udihe}, \ili{Oroch}, Orok \textit{sā-}; \ili{Manchu}, \ili{Sibe} \textit{sa-}
    
    (\citealt{Castrén1856}: 84; \citealt{Janhunen1991}: 79; \citealt{Dorji1998}: 573b; \citealt{Chaoke2014a}: 169; \citealt{Vasilevic1958}: 340a; \citealt{Cincius1975B} 2: 49--51; \citealt{Hauer1952} 3: 764; \citealt{Zikmundová2013a}: 220);

    \ex ‘to return’: \ili{Nercha} Ewenki; \il{Evenki!Khamnigan}Khamnigan Ewenki \textbf{\textit{mučū-}};

    cf. \ili{Solon} Ewenki \textit{musū-}; \ili{Siberian} Common Ewenki \textit{mučū-};

    \textit{other Northern Tungusic}: Lamut \textit{muču-}; \ili{Negidal} \textit{močo-}; \textit{\ili{Southern Tungusic}}: \ili{Nanai} \textit{močogo-}; \ili{Ulcha} \textit{mučuǰu- {\textasciitilde} mučuǰi-}; Orok \textit{moto-}, \textit{mutudu-}; \ili{Udihe}, \ili{Oroch}, \ili{Manchu}, \ili{Sibe} \textit{n.a.} 
    
    (\citealt{Castrén1856}: 96; \citealt{Janhunen1991}: 30; \citealt{Dorji1998}: 455a; \citealt{Vasilevic1958}: 266b; \citealt{Cincius1975B} 1: 562);

    \ex ‘to wash’: \ili{Nercha} Ewenki \textbf{\textit{silki-}} \textit{{\textasciitilde} śilki-}; \il{Evenki!Khamnigan}Khamnigan Ewenki \textbf{\textit{silki-}};

    cf. \ili{Solon} Ewenki \textit{silgi}-; Orochen\il{Oroqen} \textit{šilki-}; \ili{Siberian} Ewenki: \ili{Podkamennyj}, \il{Evenki!Nepa}Nepa, \ili{Tungir}, \ili{Zeya}, \ili{Aldan}, \ili{Uchur}, \ili{Urmi}, \ili{Sakhalin} \textit{silki-}; Yerbogochen\il{Yerbogachyon}, \ili{Ilimpeya} \textit{hilki-};

    \textit{other Northern Tungusic}: Lamut \textit{helka-}; \ili{Negidal} \textit{silki-}; \textit{\ili{Southern Tungusic}}: \ili{Nanai} \textit{silko-}; \ili{Ulcha} \textit{silču-}; \ili{Udihe} \textit{sik-}; \ili{Oroch} \textit{sikki-}; Orok \textit{siltu-}; \ili{Manchu} \textit{silgi-}
    
    (\citealt{Castrén1856}: 84; \citealt{Janhunen1991}: 92; \citealt{Dorji1998}: 615; \citealt{Chaoke2014a}: 169; \citealt{Vasilevic1958}: 352b; \citealt{Cincius1975B} 2: 84b; \citealt{Hauer1952} 3: 794);

    \ex ‘to forget’: \ili{Nercha} Ewenki: \ili{Urulga}, \il{Evenki!Khamnigan}Khamnigan Ewenki \textbf{\textit{omŋo-}};

    cf. \ili{Solon} Ewenki \textit{ommo-}; Orochen\il{Oroqen} \textit{omŋo-}; \ili{Siberian} Ewenki: \il{Evenki!Nepa}Nepa, \ili{Tokma} \textit{ommo-}; Yerbogochen\il{Yerbogachyon}, \ili{Ilimpeya}, \ili{Sym} \textit{omgo-}; Remaining dial. \textit{omŋo-};

    \textit{other Northern Tungusic}: Lamut \textit{omŋa}-; \ili{Negidal} \textit{omŋo-}; \textit{\ili{Southern Tungusic}}: \ili{Nanai}, Orok \textit{omgo-}; \ili{Ulcha} \textit{oŋbo-}; \ili{Udihe} \textit{oŋmo-}; \ili{Oroch} \textit{ommo-}; \ili{Manchu} \textit{oŋgo-}; \ili{Sibe} \textit{onů-}
    
    (\citealt{Castrén1856}: 76; \citealt{Janhunen1991}: 51; \citealt{Dorji1998}: 507a; \citealt{Chaoke2014a}: 169; \citealt{Vasilevic1958}: 322b \citealt{Cincius1975B} 2: 17; \citealt{Hauer1952} 3: 738; \citealt{Zikmundová2013a}: 219).
\end{xlist}

\end{xlist}
\z

\ea
The next group includes some \ili{Nercha} Ewenki words without long vowels in the last syllable, which probably can be explained by the peculiarities of Castrén’s transcriptions from \citeyear{Castrén1856}. The \il{Evenki!Khamnigan}Khamnigan Ewenki data are noted down with long vowels as other Ewenki dialects:

\begin{xlist}
    \ex  ‘son’: \ili{Nercha} Ewenki \textbf{\textit{omolgi}}; \il{Evenki!Khamnigan}Khamnigan Ewenki \textbf{\textit{omolgī}};

    cf. \ili{Solon} Ewenki \textit{omolǝ} ‘grandson’ (← \ili{Manchu} \textit{omolo}); Orochen\il{Oroqen} \textit{omolie}; \ili{Siberian} Ewenki: \ili{Podkamennyj}, Yerbogochen\il{Yerbogachyon}, \ili{Ilimpeya}, \ili{North-Baikal}, \ili{Barguzin}, \ili{Tungir}, \ili{Zeya}, \ili{Aldan}, \ili{Uchur}, \ili{Urmi}, \ili{Chumikan}, \ili{Ayan}, \ili{Sakhalin} \textit{omolgī};

    \textit{other Northern Tungusic}: Lamut \textit{omolgo}; \ili{Negidal} \textit{omolgi}; \textit{\ili{Southern Tungusic}}: \ili{Udihe} \textit{omolo}; \ili{Oroch} \textit{omolī}; \ili{Nanai}, \ili{Ulcha}, Orok \textit{n.a.}; \ili{Manchu} \textit{omolo}
    
    (\citealt{Castrén1856}: 76; \citealt{Janhunen1991}: 40; \citealt{Dorji1998}: 507a; \citealt{Chaoke2014a}: 161; \citealt{Vasilevic1958}: 322b; \citealt{Cincius1975B} 2: 17b; \citealt{Hauer1952} 3: 736); 

    \ex  ‘worm’: \ili{Nercha} Ewenki \textbf{\textit{kulikan}}; \il{Evenki!Khamnigan}Khamnigan Ewenki \textbf{\textit{kulikān}};

    cf. \ili{Solon} Ewenki \textit{kulihaŋ}; Orochen\il{Oroqen} \textit{kulikan} ‘insect’; \ili{Siberian} Common Ewenki \textit{kulikān < kulin} ‘snake’ \textit{+tkĀn} Ewenki diminutive suffix;

    \textit{other Northern Tungusic}: Lamut \textit{qulin {\textasciitilde} quličān {\textasciitilde} qolisān {\textasciitilde} kuličan {\textasciitilde} quličān} ‘mosquito’; \ili{Negidal} \textit{kolixān {\textasciitilde} kulikān} ‘worm, bug’; \textit{\ili{Southern Tungusic}}: \ili{Oroch} \textit{kulæ} ‘worm (\textit{common name for worms, snakes, caterpillars})’; \ili{Udihe} \textit{kuliga} ‘id.’; \ili{Ulcha} \textit{qoli} ‘kind of aquatic insect’, \textit{qula} ‘worm’; Orok \textit{qola {\textasciitilde} qolia {\textasciitilde} qoliγa} ‘insect, worm’; \ili{Nanai} \textit{qolã} ‘worm; caterpillar; insect’; Remaining lgs. \textit{n.a}. 
    
    (\citealt{Castrén1856}: 81; \citealt{Janhunen1991}: 42; \citealt{Dorji1998}: 402a; \citealt{Chaoke2014a}: 157; \citealt{Vasilevic1958}: 218a; \citealt{Cincius1975B} 1: 428b);

    \ex  ‘elk’: \ili{Nercha} Ewenki \textbf{\textit{tōki}}; \il{Evenki!Khamnigan}Khamnigan Ewenki: Borzya \textbf{\textit{tōki}} \textit{{\textasciitilde}} \textbf{\textit{tōkī}};

    cf. \ili{Solon} Ewenki \textit{tōhi}; \ili{Siberian} Ewenki: \ili{Ilimpeya}, \ili{Sym}, \ili{Tokma}, \ili{North-Baikal Tungir}, \ili{Zeya}, \ili{Aldan}, \ili{Uchur}, \ili{Urmi}, \ili{Ayan}, \ili{Chumikan}, \ili{Sakhalin} \textit{tōkī}; \ili{Khingan} \textit{towekī};

    \textit{other Northern Tungusic}: Lamut, \ili{Negidal} \textit{tōki}; \textit{\ili{Southern Tungusic}}: \ili{Nanai}, \ili{Ulcha} \textit{tō}; Orok \textit{tō {\textasciitilde} toγo}; \ili{Udihe} \textit{n.a}.; \ili{Oroch} \textit{tōki}; \ili{Manchu} \textit{toho}; \ili{Sibe} \textit{n.a.}
    
    (\citealt{Castrén1856}: 87; \citealt{Janhunen1991}: 103; \citealt{Dorji1998}: 695b; \citealt{Vasilevic1958}: 391a; \citealt{Cincius1975B} 2: 191--192; \citealt{Hauer1952} 3: 909);

    \ex  ‘year’: \ili{Nercha} Ewenki: \ili{Urulga} \textbf{\textit{aŋańi}}, Man’kovo \textbf{\textit{aŋani}}; \il{Evenki!Khamnigan}Khamnigan Ewenki: Borzya, \ili{Urulyungui} \textbf{\textit{anŋanī}};

    cf. \ili{Solon} Ewenki \textit{annani}; Orochen\il{Oroqen} \textit{aŋŋani}; \ili{Siberian} Common Ewenki \textit{anŋanī};

    \textit{other Northern Tungusic}: Lamut \textit{anŋan}; \ili{Negidal} \textit{ańgani}; \textit{\ili{Southern Tungusic}}: \ili{Nanai} \textit{ajŋańa}; \ili{Ulcha} \textit{ańan}; \ili{Udihe} \textit{aŋan(i)}; \ili{Oroch} \textit{aŋŋani}; Orok \textit{anani}; \ili{Manchu} \textit{aniya}; \ili{Sibe} \textit{ań} 
    
    (\citealt{Castrén1856}: 71; \citealt{Janhunen1991}: 51; \citealt{Dorji1998}: 34b; \citealt{Chaoke2014a}: 154; \citealt{Vasilevic1958}: 32a; \citealt{Cincius1975B} 1: 43--44; \citealt{Hauer1952} 1: 53; \citealt{Zikmundová2013a}: 204); 

    \ex  ‘autumn’: \ili{Nercha} Ewenki \textbf{\textit{bolońi}}; \il{Evenki!Khamnigan}Khamnigan Ewenki \textbf{\textit{bolonī}};

    cf. \ili{Solon} Ewenki \textit{bolonn}; Orochen\il{Oroqen} \textit{bolo}; \ili{Siberian} Ewenki: \ili{Podkamennyj}, \il{Evenki!Nepa}Nepa \textit{bolonī}; \ili{Sakhalin}, \ili{Chumikan} \textit{bolorī}; Yerbogochen\il{Yerbogachyon}, \ili{Tungir} \textit{bolo};

    \textit{other Northern Tungusic}: Lamut \textit{bolani}; \ili{Negidal} \textit{bolonī}; \textit{\ili{Southern Tungusic}}: \ili{Nanai}, \ili{Ulcha}, \ili{Udihe}, \ili{Oroch}, Orok \textit{bolo}; \ili{Manchu}, \ili{Sibe} \textit{bolori} 
    
    (\citealt{Castrén1856}: 95; \citealt{Janhunen1991}: 40; \citealt{Dorji1998}: 78; \citealt{Chaoke2014a}: 154; \citealt{Vasilevic1958}: 60a; \citealt{Cincius1975B} 1: 92; \citealt{Hauer1952} 1: 110; \citealt{Zikmundová2013a}: 207);

    \ex  ‘summer’: \ili{Nercha} Ewenki \textbf{\textit{ǰugańi}}; \il{Evenki!Khamnigan}Khamnigan Ewenki \textbf{\textit{ǰuganī}};

    cf. \ili{Solon} Ewenki \textit{ǰuγann}; Orochen\il{Oroqen} \textit{ǰuga}; \ili{Siberian} Ewenki \textit{ǰūγanī};

    \textit{other Northern Tungusic}: Lamut \textit{ǰugani}; \ili{Negidal} \textit{ǰowani}; \textit{\ili{Southern Tungusic}}: \ili{Nanai} \textit{ǰoa}; \ili{Ulcha}, \ili{Udihe}, \ili{Oroch} \textit{ǰua}; Orok \textit{ǰuwa}; \ili{Manchu} \textit{ǰuwari} 
    
    (\citealt{Castrén1856}: 93; \citealt{Janhunen1991}: 40; \citealt{Dorji1998}: 380a; \citealt{Chaoke2014a}: 154; \citealt{Vasilevic1958}: 138b; \citealt{Cincius1975B} 1: 268; \citealt{Hauer1952} 2: 563);

    \ex  ‘winter’: \ili{Nercha} Ewenki \textbf{\textit{tügeńi}}; \il{Evenki!Khamnigan}Khamnigan Ewenki \textbf{\textit{tügenī}};

    cf. \ili{Solon} Ewenki \textit{tüγünn}; Orochen\il{Oroqen} \textit{tuwe}; \ili{Siberian} Common Ewenki \textit{tugenī};

    \textit{other Northern Tungusic}: Lamut, \ili{Negidal} \textit{tuweni}; \textit{\ili{Southern Tungusic}}: \ili{Nanai}, \ili{Ulcha}, \ili{Udihe}, \ili{Oroch} \textit{tue}; Orok \textit{tuwe}; \ili{Manchu}, \ili{Sibe} \textit{tuweri} 
    
    (\citealt{Castrén1856}: 89; \citealt{Dorji1998}: 708b; \citealt{Chaoke2014a}: 154; \citealt{Vasilevic1958}: 397a; \citealt{Cincius1975B} 2: 204b; \citealt{Hauer1952} 3: 939; \citealt{Stary1990}: 91);

    \ex  ‘wine’: \ili{Nercha} Ewenki \textbf{\textit{araki}}; \il{Evenki!Khamnigan}Khamnigan Ewenki \textbf{\textit{arakī}};

    cf. \ili{Solon} Ewenki \textit{arki}; Orochen\il{Oroqen} \textit{araki}; \ili{Siberian} Common Ewenki \textit{arakī};

    \textit{other Northern Tungusic}: Lamut \textit{arïgï} ← Yakut; \ili{Negidal} \textit{ayahī}; \textit{\ili{Southern Tungusic}}: \ili{Udihe} \textit{ayi}; \ili{Nanai}, \ili{Ulcha}, \ili{Oroch}, Orok \textit{araki}; \ili{Manchu} \textit{arki}; \ili{Sibe} \textit{erk};

    Tungusic ← \ili{Mongolic} ← \ili{Turkic} ← \ili{Arabic}
    
    (\citealt{Castrén1856}: 71; \citealt{Janhunen1991}: 31; \citealt{Dorji1998}: 43b; \citealt{Chaoke2014a}: 163; \citealt{Vasilevic1958}: 34a; \citealt{Cincius1975B} 1: 48; \citealt{Hauer1952} 1: 58; \citealt{Zikmundová2013a}: 209);
\end{xlist}
\z

\ea
The following \ili{Nercha} Ewenki example can also be explained by Castrén’s transcription, the \il{Evenki!Khamnigan}Khamnigan Ewenki of Manchuria as other Ewenki dialects have a long vowel in the last syllable of word:

\ea ‘woman’: \ili{Nercha} Ewenki \textbf{\textit{āśi}}; \il{Evenki!Khamnigan}Khamnigan Ewenki \textbf{\textit{asī}};

    cf. \ili{Solon} Ewenki \textit{ase}; Orochen\il{Oroqen} \textit{aši}; \ili{Siberian} Ewenki: \ili{Podkamennyj}, \il{Evenki!Nepa}Nepa, \ili{Vitim} \textit{asī}; Yerbogochen\il{Yerbogachyon}, \ili{Ilimpeya}, \ili{Barguzin}, \ili{Tungir}, \ili{Zeya}, \ili{Aldan}, \ili{Uchur}, \ili{Urmi}, \ili{Chumikan}, \ili{Ayan}, \ili{Sakhalin} \textit{ahī}; \ili{Sym}, \ili{North-Baikal}, \ili{Baunt} \textit{ašī};

    \textit{other Northern Tungusic}: Lamut, \ili{Negidal} \textit{asi}; \textit{\ili{Southern Tungusic}}: \ili{Nanai}, \ili{Ulcha}, Orok \textit{asi}; \ili{Udihe} \textit{a\textsuperscript{h}}\textit{anta}; \ili{Oroch} \textit{asa}; \ili{Manchu} \textit{aša} ‘elder brother’s wife’
    
    (\citealt{Castrén1856}: 72; \citealt{Janhunen1991}: 45; \citealt{Dorji1998}: 45a; \citealt{Chaoke2014a}: 160; \citealt{Vasilevic1958}: 38a; \citealt{Cincius1975B} 1: 55; \citealt{Hauer1952} 1: 61);
\z
\z

\ea
The change of Tungusic vowel *\textit{u} > \textit{o} in \ili{Nercha} Ewenki:

\ea ‘five’: \ili{Nercha} Ewenki \textbf{\textit{toŋa}}, \il{Evenki!Khamnigan}Khamnigan Ewenki \textbf{\textit{tunŋa}} \textit{{\textasciitilde}} \textbf{\textit{tunna}};

    cf. \ili{Solon} Ewenki \textit{tuŋa}; Orochen\il{Oroqen} \textit{tuŋŋa}; \ili{Siberian} Ewenki: \ili{Ilimpeya}, \ili{Tokma}, \ili{Nercha}, \ili{Zeya}, \ili{Aldan} \textit{tonŋa}; Remaining dial. \textit{tunŋa};

    \textit{other Northern Tungusic}: Lamut \textit{tuńŋan}; \ili{Negidal} \textit{tońŋa}; \textit{\ili{Southern Tungusic}}: \ili{Nanai} \textit{tojŋa}; \ili{Ulcha} \textit{tunǰa}; \ili{Udihe}, \ili{Oroch} \textit{tuŋa}; Orok \textit{tunda}; \ili{Manchu}, \ili{Sibe} \textit{sunǰa}
    
    (\citealt{Castrén1856}: 88; \citealt{Janhunen1991}: 76; \citealt{Dorji1998}: 703; \citealt{Chaoke2014a}: 170; \citealt{Vasilevic1958}: 401b; \citealt{Cincius1975B} 2: 214; \citealt{Hauer1952} 3: 830; \citealt{Zikmundová2013a}: 221);

\z
\z

\ea
The palatalization of vowels in \ili{Nercha} and \il{Evenki!Khamnigan}Khamnigan Ewenki:

\ea  ‘milk’: \ili{Nercha} Ewenki \textbf{\textit{ükümńi}}; \il{Evenki!Khamnigan}Khamnigan Ewenki \textbf{\textit{ükün}};

    cf. \ili{Solon} Ewenki \textit{uhuŋ {\textasciitilde} əkuŋ}; Orochen\il{Oroqen} \textit{ukun}; \ili{Siberian} Ewenki: \ili{Podkamennyj}, \ili{Ilimpeya}, \ili{Tokma} \textit{ukunmī}; \il{Evenki!Nepa}Nepa, Yerbogochen\il{Yerbogachyon}, \ili{Upper Lena}, \ili{North-Baikal}, \ili{Barguzin}, \ili{Tungir}, \ili{Aldan}, \ili{Ayan} \textit{ukumnī};

    \textit{other Northern Tungusic}: Lamut \textit{ukeń {\textasciitilde} ukuń}; \ili{Negidal} \textit{ukuńi}; \textit{\ili{Southern Tungusic}}: \ili{Nanai} \textit{ukuń}; \ili{Ulcha} \textit{kuen}; \ili{Udihe} \textit{kośo}; \ili{Oroch} \textit{okon}; Orok \textit{kō(n) {\textasciitilde} kū(n)}; \ili{Manchu}, \ili{Sibe} \textit{n.a.}
    
    (\citealt{Castrén1856}: 77; \citealt{Janhunen1991}: 24; \citealt{Chaoke2014bB}: 184; \citealt{Chaoke2014a}: 163; \citealt{Vasilevic1958}: 435a; \citealt{Cincius1975B} 2: 255);

    \item ‘to die’: \ili{Nercha} Ewenki \textbf{\textit{bü}}\textit{{}-}; \il{Evenki!Khamnigan}Khamnigan Ewenki \textbf{\textit{büde}}\textit{{}-};

    cf. \ili{Solon} Ewenki, Orochen\il{Oroqen}, \ili{Siberian} Common Ewenki \textit{bu}{}-;

    \textit{other Northern Tungusic}: Lamut ; \ili{Negidal} \textit{bu-}; \textit{\ili{Southern Tungusic}}: \ili{Nanai} \textit{bu{}- {\textasciitilde} buj-}; Orok, \ili{Ulcha} \textit{bu- {\textasciitilde} bul-}; \ili{Udihe} \textit{bude-}; \ili{Oroch} \textit{bude- {\textasciitilde} bukki-}; \ili{Manchu} \textit{bude- {\textasciitilde} buče-}; \ili{Sibe} \textit{beš- {\textasciitilde} bǝč(ǝ)-}
    
    (\citealt{Castrén1856}: 96; \citealt{Janhunen1991}: 79; \citealt{Dorji1998}: 95; \citealt{Chaoke2014a}: 169; \citealt{Vasilevic1958}: 62b; \citealt{Cincius1975B} 1: 98--99; \citealt{Hauer1952} 1: 117; \citealt{Zikmundová2013a}: 206);

\z
\z

\ea
The change of original Tungusic nasal consonant *\textit{ŋ-} > \textit{n-} in \ili{Nercha} and \il{Evenki!Khamnigan}Khamnigan Ewenki as in \ili{Solon} Ewenki:

\ea ‘hand’: \ili{Nercha} Ewenki \textbf{\textit{nāla}} \textit{{\textasciitilde}} \textbf{\textit{nala}}; \il{Evenki!Khamnigan}Khamnigan Ewenki \textbf{\textit{nāla}};

    cf. \ili{Solon} Ewenki \textit{nāl}; Orochen\il{Oroqen} \textit{ŋāla}; \ili{Siberian} Ewenki: Yerbogochen\il{Yerbogachyon}, \ili{Ayan} \textit{nāle}; Remaining all dialects \textit{ŋāle};

    \textit{other Northern Tungusic}: Lamut \textit{ŋāl}; \ili{Negidal} \textit{ŋāla {\textasciitilde} ŋala}; \textit{\ili{Southern Tungusic}}: \ili{Nanai}, \ili{Ulcha}, \ili{Udihe} \textit{ŋala}; Orok \textit{ŋāla}; \ili{Oroch} \textit{ŋāla {\textasciitilde} ŋala}; \ili{Manchu} \textit{gala}; \ili{Sibe} \textit{gal}
    
    (\citealt{Castrén1856}: 85, 83; \citealt{Janhunen1991}: 49; \citealt{Dorji1998}: 467b; \citealt{Chaoke2014a}: 159; \citealt{Vasilevic1958}: 278b; \citealt{Cincius1975B} 1: 656; \citealt{Hauer1952} 1: 331; \citealt{Zikmundová2013a}: 213);

    \ex ‘who?’: \ili{Nercha} Ewenki \textbf{\textit{nī}} \textit{{\textasciitilde}} \textbf{\textit{ńī}}; \il{Evenki!Khamnigan}Khamnigan Ewenki \textbf{\textit{nī}};

    cf. \ili{Solon} Ewenki \textit{ni} ‘which one’; Orochen\il{Oroqen} \textit{ni}; \ili{Siberian} Ewenki: \ili{Podkamennyj}, Yerbogochen\il{Yerbogachyon}, \ili{Ilimpeya}, \ili{Sym}, \ili{Tungir}, \ili{Zeya}, \ili{Uchur}, \ili{Urmi}, \ili{Chumikan}, \ili{Sakhalin} \textit{ŋī}; \ili{Upper Lena}, \ili{North-Baikal}, \ili{Barguzin}, \ili{Chumikan} \textit{nī};

    \textit{other Northern Tungusic}: Lamut, \ili{Negidal} \textit{nī {\textasciitilde} ŋī}; \textit{\ili{Southern Tungusic}}: \ili{Nanai} \textit{uj {\textasciitilde} ui}; \ili{Ulcha} \textit{ŋui {\textasciitilde} uj}; Orok \textit{ŋuji {\textasciitilde} ŋuj {\textasciitilde} ŋui}; \ili{Udihe} \textit{nī}; \ili{Oroch} \textit{ńī}; \ili{Manchu}, \ili{Sibe} \textit{n.a.} 
    
    (\citealt{Castrén1856}: 85; \citealt{Janhunen1991}: 29; \citealt{Chaoke2014bB}: 340; \citealt{Chaoke2014a}: 165; \citealt{Vasilevic1958}: 280a; \citealt{Cincius1975B} 1: 660).

\z
\z
\subsection{Phonetic differences between Nercha and {Khamnigan} {Ewenki}}

There are some phonetic differences between \ili{Nercha} and \il{Evenki!Khamnigan}Khamnigan Ewenki dialects, which are possibly results of change in \il{Evenki!Khamnigan}Khamnigan Ewenki in the 20th and 21st centuries.

\ea
The deletion of Tungusic initial consonant *\textit{h-} (< Proto Tungusic *\textit{p-}) in \il{Evenki!Khamnigan}Khamnigan Ewenki. In some cases \textit{h-} is sporadically preserved in the Borzya subdialect of \il{Evenki!Khamnigan}Khamnigan Ewenki (\citealt{Janhunen1991}: 46). The consonant \textit{h-} was already deleted in the extinct \ili{Mankovo} subdialect, while we find it in the extinct \ili{Urulga} and do not see it in \ili{Urulyungui}, the related subdialect of Manchuria. The initial \textit{h-} disappeared in \ili{Solon} Ewenki and Orochen\il{Oroqen} too:

\ea ‘daughter, young girl’: \ili{Nercha} Ewenki: \ili{Urulga} \textbf{\textit{hunāt}}; Man’kovo \textbf{\textit{unāt}}; \il{Evenki!Khamnigan}Khamnigan Ewenki: Borzya \textbf{\textit{hunāǰi}},\footnote{The Borzya form \textit{hunāǰi} ‘daughter, young girl’ is a hybrid word: the preservation of initial \textit{h-} is a heritage, while the rest is close to \ili{Solon} Ewenki \textit{unaǰi.}} \ili{Urulyungui} \textbf{\textit{unād}};

    cf. \ili{Solon} Ewenki \textit{unaǰi}; Orochen\il{Oroqen} \textit{unāǰi}; \ili{Siberian} Ewenki: \ili{Podkamennyj}, \il{Evenki!Nepa}Nepa, \ili{Ilimpeya}, \ili{Barguzin}, \ili{Zeya}, \ili{Uchur}, \ili{Urmi}, \ili{Chumikan}, \ili{Sakhalin} \textit{hunāt}; \ili{Tokma} \textit{honāt}; Yerbogochen\il{Yerbogachyon}, \ili{Tungir}, \ili{Aldan} \textit{sunāt};

    \textit{other Northern Tungusic}: Lamut \textit{hunāǰ {\textasciitilde} hunāt}; \ili{Negidal} \textit{honāt}; \textit{\ili{Southern Tungusic}}: \ili{Nanai} \textit{pondaǰo}; \ili{Ulcha} \textit{pundaǰu}; \ili{Udihe}, \ili{Oroch} \textit{hunaǰi}; Orok \textit{pundado {\textasciitilde} pundadu}; \ili{Manchu}, \ili{Sibe} \textit{n.a}.
    
    (\citealt{Castrén1856}: 77, 83; \citealt{Janhunen1991}: 43; \citealt{Dorji1998}: 529b; \citealt{Chaoke2014a}: 161; \citealt{Vasilevic1958}: 495b; \citealt{Cincius1975B} 2: 347);

    \ex ‘blanket’: \ili{Nercha} Ewenki: \ili{Urulga} \textbf{\textit{hulda}}, Man’kovo \textbf{\textit{ulda}}; \il{Evenki!Khamnigan}Khamnigan Ewenki: Borzya \textbf{\textit{(h)ulda}}; \ili{Urulyungui} \textbf{\textit{ulda}};

    cf. \ili{Solon} Ewenki \textit{ulda}; \ili{Siberian} Ewenki: \ili{Podkamennyj}, \il{Evenki!Nepa}Nepa, Yerbogochen\il{Yerbogachyon}, \ili{Tokma}, \ili{Barguzin}, \ili{Tungir}, \ili{Zeya}, \ili{Aldan}, \ili{Urmi}, \ili{Chumikan}, \ili{Sakhalin} \textit{hulla}; \ili{Ilimpeya}, \ili{Uchur}, \ili{Chumikan} \textit{hulda}; \ili{Ilimpeya}, \ili{North-Baikal}, \ili{Ayan} \textit{ulda};

    \textit{other Northern Tungusic}: Lamut \textit{hulra}; \ili{Negidal} \textit{hola}; \textit{\ili{Southern Tungusic}}: \ili{Nanai} \textit{polta}; Orok, \ili{Ulcha} \textit{pulta}; \ili{Udihe} \textit{hulaha}; \ili{Oroch} \textit{hukta}; \ili{Manchu}, \ili{Sibe} \textit{n.a.}
    
    (\citealt{Castrén1856}: 76; \citealt{Janhunen1991}: 52; \citealt{Dorji1998}: 526a; \citealt{Vasilevic1958}: 493b; \citealt{Cincius1975B} 2: 345);

    \ex ‘road’: \ili{Nercha} Ewenki: \ili{Urulga} \textbf{\textit{hokto}}, Man’kovo \textbf{\textit{okto}}; \il{Evenki!Khamnigan}Khamnigan Ewenki: Borzya \textbf{\textit{(h)ogto}};

    cf. \ili{Solon} Ewenki \textit{otto}; Orochen\il{Oroqen} \textit{okto}; \ili{Siberian} Ewenki: \ili{Podkamennyj}, \il{Evenki!Nepa}Nepa, Yerbogochen\il{Yerbogachyon}, \ili{Ilimpeya}, \ili{Sym}, \ili{Tokma}, \ili{Barguzin}, \ili{Tungir}, \ili{Zeya}, \ili{Aldan}, \ili{Uchur}, \ili{Urmi}, \ili{Ayan}, \ili{Sakhalin} \textit{hokto}; \ili{Aldan} \textit{sokto}; \ili{Upper Lena}, \ili{North-Baikal}, \ili{Chumikan} \textit{okto};

    \textit{other Northern Tungusic}: Lamut \textit{hōt}; \ili{Negidal} \textit{hokto}; \textit{\ili{Southern Tungusic}}: \ili{Nanai}, \ili{Ulcha}, Orok \textit{pokto}, \ili{Udihe}, \ili{Oroch} \textit{hokto}; \ili{Manchu} \textit{oktoron} ‘hare tracks’
    
    (\citealt{Castrén1856}: 83; \citealt{Janhunen1991}: 47; \citealt{Dorji1998}: 515a; \citealt{Chaoke2014a}: 163; \citealt{Vasilevic1958}: 484a; \citealt{Cincius1975B} 2: 331; \citealt{Hauer1952} 3: 732); 

    \ex ‘red’: \ili{Nercha} Ewenki \textbf{\textit{ularin}}; \il{Evenki!Khamnigan}Khamnigan Ewenki: Borzya \textbf{\textit{hularīn}}, \ili{Urulyungui} \textbf{\textit{ularīn}};

    cf. \ili{Solon} Ewenki \textit{ulariŋ}; Orochen\il{Oroqen} \textit{ularin}; \ili{Siberian} Ewenki: \ili{Podkamennyj}, \ili{Tokma}, \ili{Tungir}, \ili{Vitim}, \ili{Zeya}, \ili{Uchur}, \ili{Urmi}, \ili{Chumikan}, \ili{Ayan}, \ili{Sakhalin} \textit{hularīn}; \ili{Upper Lena}, \ili{North-Baikal}, \ili{Vitim} \textit{ularin};

    \textit{other Northern Tungusic}: Lamut \textit{hulańa}; \ili{Negidal} \textit{holajin}; \textit{\ili{Southern Tungusic}}: \ili{Nanai} \textit{folg\={æ} {\textasciitilde} forg\={æ}}; \ili{Ulcha}, \ili{Udihe} \textit{hulaligi}; \ili{Oroch}, Orok \textit{n.a.}; \ili{Manchu} \textit{fulgian}; \ili{Sibe} \textit{fulaʁů\textsuperscript{n}} 
    
    (\citealt{Castrén1856}: 76; \citealt{Janhunen1991}: 46; \citealt{Dorji1998}: 525; \citealt{Chaoke2014a}: 167; \citealt{Vasilevic1958}: 493a; \citealt{Cincius1975B} 2: 343; \citealt{Hauer1952} 1: 314; \citealt{Zikmundová2013a}: 211);

    \ex ‘big’: \ili{Nercha} Ewenki: \ili{Urulga} \textbf{\textit{hegdi}}, Man’kovo \textbf{\textit{egdi}}; \il{Evenki!Khamnigan}Khamnigan Ewenki: Borzya \textbf{\textit{hegdi}}, \ili{Urulyungui} \textbf{\textit{egdi}};

    cf. \ili{Solon} Ewenki \textit{ǝgdǝ {\textasciitilde} ǝdduγ}; Orochen\il{Oroqen} \textit{egdeŋe}; \ili{Siberian} Ewenki: \ili{Podkamennyj}, \il{Evenki!Nepa}Nepa, Yerbogochen\il{Yerbogachyon}, \ili{Ilimpeya}, \ili{Barguzin}, \ili{Tungir}, \ili{Zeya}, \ili{Aldan}, \ili{Uchur}, \ili{Urmi}, \ili{Chumikan}, \ili{Sakhalin} \textit{hegdi};

    \textit{other Northern Tungusic}: Lamut \textit{egde}; \ili{Negidal} \textit{egdi}; \textit{\ili{Southern Tungusic}}: \ili{Nanai}, Orok \textit{egǰi}; \ili{Ulcha}, \ili{Udihe}, \ili{Oroch} \textit{egdi}; \ili{Manchu}, \ili{Sibe} \textit{n.a.} 
    
    (\citealt{Castrén1856}: 82, 72; \citealt{Janhunen1991}: 47; \citealt{Dorji1998}: 150b; \citealt{Chaoke2014a}: 165; \citealt{Vasilevic1958}: 503b; \citealt{Cincius1975B} 2: 359);
\z
\z

\ea
The fluctuation of initial vowel \textit{u-} {\textasciitilde} \textit{i-} in \ili{Nercha} and \il{Evenki!Khamnigan}Khamnigan Ewenki dialects:

\ea ‘to drink’: \ili{Nercha} Ewenki: \ili{Urulga} \textbf{\textit{umi-}}, Man’kovo \textbf{\textit{imi-}}; \il{Evenki!Khamnigan}Khamnigan Ewenki: Borzya \textbf{\textit{imi-}}, \ili{Urulyungui} \textbf{\textit{um-}};

    cf. \ili{Solon} Ewenki \textit{omi- {\textasciitilde} imo}{}-; Orochen\il{Oroqen} \textit{imo}{}-; \ili{Siberian} Common Ewenki \textit{um}{}-;

    \textit{other Northern Tungusic}: Lamut \textit{n.a.}; \ili{Negidal} \textit{om-}; \textit{\ili{Southern Tungusic}}: \ili{Nanai} \textit{omi-}; Orok, \ili{Ulcha}, \ili{Udihe} \textit{umi}{}-; \ili{Oroch} \textit{imi-}; \ili{Manchu} \textit{omi-}; \ili{Sibe} \textit{emi-}
    
    (\citealt{Castrén1856}: 77; \citealt{Janhunen1991}: 98; \citealt{Chaoke2014bB}: 392; \citealt{Chaoke2014a}: 167; \citealt{Vasilevic1958}: 441b; \citealt{Cincius1975B} 2: 266; \citealt{Hauer1952} 3: 735; \citealt{Zikmundová2013a}: 209);

\z
\z
\pagebreak

\ea
The diphthongization of long vowel \textit{ī} > \textit{ie} in the \ili{Urulyungui} subdialect of Khamnigan

\ea ‘eye’: \ili{Nercha} Ewenki: \ili{Urulga} \textbf{\textit{yēsa}}, Man’kovo \textbf{\textit{yīsa}}; \il{Evenki!Khamnigan}Khamnigan Ewenki: Borzya \textbf{\textit{yīsa}}, \ili{Urulyungui} \textbf{\textit{iesa}};

    cf. \ili{Solon} Ewenki \textit{īsa}; Orochen\il{Oroqen} \textit{yiesa}; \ili{Siberian} Ewenki: \ili{Podkamennyj}, \il{Evenki!Nepa}Nepa \textit{ēsa}; Yerbogochen\il{Yerbogachyon}, \ili{Ilimpeya}, \ili{Upper Lena}, \ili{Barguzin}, \ili{Tungir}, \ili{Zeya}, \ili{Aldan}, \ili{Uchur}, \ili{Urmi}, \ili{Ayan}, \ili{Chumikan}, \ili{Sakhalin} \textit{ēha}; \ili{Sym}, \ili{North-Baikal} \textit{ēša};

    \textit{other Northern Tungusic}: Lamut \textit{yasal}; \ili{Negidal} \textit{ēsa}; \textit{\ili{Southern Tungusic}}: \ili{Nanai} \textit{nasal}; \ili{Ulcha} \textit{isal}; \ili{Udihe} \textit{yeha}; \ili{Oroch} \textit{isa}; Orok \textit{īsa}; \ili{Manchu} \textit{yasa}; \ili{Sibe} \textit{yas} 
    
    (\citealt{Castrén1856}: 75; \citealt{Janhunen1991}: 34; \citealt{Dorji1998}: 320a; \citealt{Chaoke2014a}: 159; \citealt{Vasilevic1958}: 567b; \citealt{Cincius1975B} 1: 291b-292b; \citealt{Hauer1952} 3: 1015; \citealt{Zikmundová2013a}: 225);

    \ex ‘ear’: \ili{Nercha} Ewenki: \ili{Urulga} \textit{śen}; \il{Evenki!Khamnigan}Khamnigan Ewenki: Borzya \textit{sīn}, \ili{Urulyungui} \textbf{\textit{sien}};

    cf. \ili{Solon} Ewenki \textit{sǝŋ}; Orochen\il{Oroqen} \textit{šien}; \ili{Siberian} Ewenki: \ili{Podkamennyj}, \ili{Ayan}, \ili{Aldan}, \ili{Barguzin}, \ili{Upper Lena}, \ili{Zeya}, \il{Evenki!Nepa}Nepa, \ili{Sakhalin}, \ili{Tokma}, \ili{Tungir}, \ili{Urmi}, \ili{Uchur}, \ili{Chumikan} \textit{sēn}; Yerbogochen\il{Yerbogachyon}, \ili{Ilimpeya} \textit{hēn}; \ili{Sym}, Nort-Baikal \textit{šēn};

    \textit{other Northern Tungusic}: Lamut, \ili{Negidal} \textit{sen}; \textit{\ili{Southern Tungusic}}: \ili{Nanai} \textit{siã}; \ili{Ulcha}, Orok \textit{sēn}; \ili{Udihe} \textit{n.a}.; \ili{Oroch} \textit{s\={æ}}; \ili{Manchu} \textit{šan}; \ili{Sibe} \textit{sa\textsuperscript{n}} 
    
    (\citealt{Castrén1856}: 84; \citealt{Janhunen1991}: 34; \citealt{Dorji1998}: 588a; \citealt{Chaoke2014a}: 159; \citealt{Vasilevic1958}: 347b; \citealt{Cincius1975B} 2: 70b-71b; \citealt{Hauer1952} 3: 843; \citealt{Zikmundová2013a}: 220); 

    \ex ‘moon’: \ili{Nercha} Ewenki \textit{bēga}; \il{Evenki!Khamnigan}Khamnigan Ewenki: Borzya \textit{bīga}, \ili{Urulyungui} \textbf{\textit{biega}};

    cf. \ili{Solon} Ewenki \textit{bēγa}; Orochen\il{Oroqen} \textit{biēga}; \ili{Siberian} Ewenki: \il{Evenki!Nepa}Nepa, \ili{Tokma} \textit{bēwa}; Remaining dial. \textit{bēga};

    \textit{other Northern Tungusic}: Lamut \textit{beg}; \ili{Negidal} \textit{bega}; \textit{\ili{Southern Tungusic}}: \ili{Nanai} \textit{bia}; \ili{Ulcha}, Orok \textit{bē}; \ili{Udihe} \textit{beæ}; \ili{Oroch} \textit{b\={æ}}; \ili{Manchu} \textit{biya}
    
    (\citealt{Castrén1856}: 95; \citealt{Janhunen1991}: 34; \citealt{Dorji1998}: 70a; \citealt{Chaoke2014a}: 152; \citealt{Vasilevic1958}: 52b; \citealt{Cincius1975B} 1: 78; \citealt{Hauer1952} 1: 100); 

    \ex ‘what’: \ili{Nercha} Ewenki \textit{ēkun}, \ili{Mankovo} \textit{īkun}; \il{Evenki!Khamnigan}Khamnigan Ewenki: Borzya \textit{ikun}, \ili{Urulyungui} \textbf{\textit{iekun}};

    cf. \ili{Solon} Ewenki \textit{ǝγu}; Orochen\il{Oroqen} \textit{ikun}; \ili{Siberian} Ewenki: Yerbogochen\il{Yerbogachyon}, \ili{Ilimpeya} \textit{īkūn}; \ili{Upper Lena} \textit{yākūn}; Remaining dial. \textit{ēkūn};

    \textit{other Northern Tungusic}: Lamut \textit{\={æ}k}; \ili{Negidal} \textit{ēhun}; \textit{\ili{Southern Tungusic}}: \ili{Udihe} \textit{y’eu}; \ili{Oroch} \textit{yaw}; Orok, \ili{Nanai}, \ili{Ulcha} \textit{n.a.}; \ili{Manchu} \textit{ya} 
    
    (\citealt{Castrén1856}: 73; \citealt{Janhunen1991}: 34; \citealt{Dorji1998}: 154b; \citealt{Chaoke2014a}: 165; \citealt{Vasilevic1958}: 551b; \citealt{Cincius1975B} 1: 286--287; \citealt{Hauer1952} 3: 1002);

\z
\z


\ea
The development of a secondary long vowel from \textit{VgV} in \ili{Urulyungui} subdialect of \il{Evenki!Khamnigan}Khamnigan Ewenki and in Orochen\il{Oroqen}:

\ea ‘to sit’: \ili{Nercha} Ewenki \textit{tege-}, \il{Evenki!Khamnigan}Khamnigan Ewenki: \ili{Urulyungui} \textbf{\textit{tē}}\textit{{}-}, Borzya \textit{tege-};

    cf. \ili{Solon} Ewenki \textit{tǝ$\gamma ǝ$}{}-; Orochen\il{Oroqen} \textit{tē}{}-; \ili{Siberian} Ewenki: \ili{Tokma}, \ili{North-Baikal} \textit{tē-}; \ili{Podkamennyj}, \il{Evenki!Nepa}Nepa, Yerbogochen\il{Yerbogachyon}, \ili{Ilimpeya}, \ili{Sym}, \ili{Barguzin}, \ili{Tungir}, \ili{Zeya}, \ili{Aldan}, \ili{Uchur}, \ili{Urmi}, \ili{Ayan}, \ili{Sakhalin} \textit{teγe}{}-;

    \textit{other Northern Tungusic}: Lamut \textit{teg-}; \ili{Negidal} \textit{tege-}; \textit{\ili{Southern Tungusic}}: \ili{Nanai}, \ili{Ulcha}, \ili{Udihe}, \ili{Oroch}, Orok \textit{tē-}; \ili{Manchu}, \ili{Sibe} \textit{te-}
    
    (\citealt{Castrén1856}: 87; \citealt{Janhunen1991}: 99; \citealt{Dorji1998}: 674a; \citealt{Chaoke2014a}: 168; \citealt{Vasilevic1958}: 418a; \citealt{Cincius1975B} 2: 226--228; \citealt{Hauer1952} 3: 899; \citealt{Zikmundová2013a}: 222);
\z
\z


\ea
The voicing of consonant clusters *\textit{rk}, *\textit{ks}, *\textit{kt} in \il{Evenki!Khamnigan}Khamnigan Ewenki:

\ea ‘fist’: \ili{Nercha} Ewenki \textit{nurka}; \il{Evenki!Khamnigan}Khamnigan Ewenki \textbf{\textit{nurga}};

    cf. \ili{Solon} Ewenki, \ili{Siberian} Ewenki \textit{n.a.};

    \textit{other Northern Tungusic}: \ili{Negidal} \textit{nelga {\textasciitilde} nojga}; \textit{\ili{Southern Tungusic}}: \ili{Ulcha} \textit{ńugǰa}, \ili{Oroch} \textit{nugga}; \ili{Manchu} \textit{nuǰan}; Remaining lgs. \textit{n.a.}
    
    (\citealt{Castrén1856}: 86; \citealt{Janhunen1991}: 23; \citealt{Cincius1975B} 1: 590a; \citealt{Hauer1952} 3: 722);

    Tungusic ← \ili{Mongolic}: \ili{Middle Mongol}: MNT \textit{nodurqa}; HY, Muq. \textit{nudurqa}; \ili{Literary Mongolian} \textit{nidurγa}; Onon \ili{Khamnigan Mongol} \textit{nidurga}, cf. \textit{nyudarga} (← \ili{Buryat}); Dadal-sum Khamnigan \textit{nidurγa}; \ili{Dagur} \textit{nyɔdruγw}; \ili{Buryat} \textit{nyudarga};\il{Khamnigan Mongol}

    \ili{Mongolic} ← \ili{Turkic} *\textit{ńïdru}{}-: cf. \ili{Old Turkic} \textit{yïðruq} ‘fist’ \citep[120]{Khabtagaeva2017};

    \ex ‘evening’: \ili{Nercha} Ewenki \textit{śikśe}; \il{Evenki!Khamnigan}Khamnigan Ewenki \textbf{\textit{sigsenī}};

    \ili{Siberian} Ewenki: \ili{Uchur}, \ili{Urmi}, \ili{Ayan}, \ili{Chumikan}, \ili{Sakhalin} \textit{sikse} ‘in the evening’;

    cf. \ili{Solon} Ewenki \textit{dolbon}; Orochen\il{Oroqen} \textit{dolbo}; \ili{Nercha} Ewenki \textit{dolboni} ‘night’;

    \textit{other Northern Tungusic}: Lamut \textit{hīsečin}; \ili{Negidal} \textit{sikse}; \textit{\ili{Southern Tungusic}}: \ili{Udihe} \textit{sikie}; \ili{Nanai}, \ili{Ulcha}, \ili{Oroch} \textit{sikse}; Orok \textit{šekše}; \ili{Manchu} \textit{sikse} ‘yesterday’; \ili{Sibe} \textit{čǝksǝ}

    (\citealt{Castrén1856}: 84; \citealt{Janhunen1991}: 45; \citealt{Dorji1998}: 133a; \citealt{Chaoke2014a}: 155; \citealt{Vasilevic1958}: 351a; \citealt{Cincius1975B} 2: 81; \citealt{Hauer1952} 3: 793; \citealt{Zikmundová2013a}: 207);

    \ex ‘to meet’: \ili{Nercha} Ewenki \textit{uktu-}; \il{Evenki!Khamnigan}Khamnigan Ewenki \textbf{\textit{ugtu}-};

    cf. \ili{Solon} Ewenki \textit{otto-}; \ili{Siberian} Ewenki \textit{n.a.};

    \textit{other Northern Tungusic}: \ili{Negidal} \textit{oktul-}; \textit{\ili{Southern Tungusic}}: \ili{Ulcha} \textit{oktoli- {\textasciitilde} uktuli-}, \ili{Oroch} \textit{uktul-}; Orok \textit{uktulli-}; Remaining lgs. \textit{n.a}. 

    (\citealt{Castrén1856}: 76; \citealt{Janhunen1991}: 26; \citealt{Dorji1998}: 515b; \citealt{Cincius1975B} 2: 254b)

    ← \ili{Mongolic}: \ili{Middle Mongol}: MNT \textit{uqtu-} \textit{{\textasciitilde} uqdu-}; HY \textit{uqtu}- {\textasciitilde} \textit{uγtu-}; \ili{Literary Mongolian} \textit{uγtu-} ‘to greet, to meet, to welcome’; Onon \ili{Khamnigan Mongol} \textit{ugta-}

    (← \ili{Buryat}); \ili{Dagur} \textit{ort}-; \ili{Buryat} \textit{ugta}- \citep[139]{Khabtagaeva2017};
\z
\z

\ea
The additional sound at the end of Borzya subdialect of \il{Evenki!Khamnigan}Khamnigan Ewenki as in Orochen\il{Oroqen} or \ili{Solon}, which proves their influence:

\ea ‘head’: \ili{Nercha} Ewenki \textit{dil}; \il{Evenki!Khamnigan}Khamnigan Ewenki: \ili{Urulyungui} \textit{dil}, Borzya \textbf{\textit{dili}};

    cf. \ili{Solon} Ewenki \textit{del}; Orochen\il{Oroqen} \textit{dili}; \ili{Siberian} Common Ewenki \textit{dïl};

    \textit{other Northern Tungusic}: Lamut, \ili{Negidal} \textit{dil}; \textit{\ili{Southern Tungusic}}: \ili{Nanai}, Orok \textit{ǰili}; \ili{Ulcha}, \ili{Udihe}, \ili{Oroch} \textit{dili}; \ili{Manchu}, \ili{Sibe} \textit{n.a.}
    
    (\citealt{Castrén1856}: 90; \citealt{Janhunen1991}: 27; \citealt{Dorji1998}: 127b; \citealt{Chaoke2014a}: 158; \citealt{Vasilevic1958}: 128b; \citealt{Cincius1975B} 1: 205b-206a, see also \citealt{Hölzl2018b}: 129 for some discussion);

    \ex ‘house’: \ili{Nercha} Ewenki \textit{ǰū}, \il{Evenki!Khamnigan}Khamnigan Ewenki: \ili{Urulyungui} \textit{ǰū}, Borzya \textbf{\textit{ǰūg}};

    cf. \ili{Solon} Ewenki \textit{ǰu}, cf. \textit{ǰūγ} (Ivanovskij); Orochen\il{Oroqen} \textit{ǰū}; \ili{Siberian} Common Ewenki \textit{ǰū};

    \textit{other Northern Tungusic}: Lamut \textit{ǰū}; \ili{Negidal} \textit{ǰō}; \textit{\ili{Southern Tungusic}}: \ili{Nanai} \textit{ǰōg}; \ili{Ulcha} \textit{ǰūg}; \ili{Udihe} \textit{ǰugdi}; \ili{Oroch} \textit{ǰug}; Orok \textit{duku}; \ili{Manchu}, \ili{Sibe} \textit{n.a.}
    
    (\citealt{Castrén1856}: 94; \citealt{Janhunen1991}: 41; \citealt{Dorji1998}: 385a; \citealt{Chaoke2014a}: 162; \citealt{Vasilevic1958}: 138a; \citealt{Cincius1975B} 1: 266);
\z
\z


\ea
In some cases the initial *\textit{ŋ-} through \textit{n-} is further deleted in \ili{Nercha} and \il{Evenki!Khamnigan}Khamnigan Ewenki as in \ili{Southern Tungusic} languages:
\ea  ‘dog’: \ili{Nercha} Ewenki: Man’kovo \textbf{\textit{inakin}}; \il{Evenki!Khamnigan}Khamnigan Ewenki: \ili{Urulyungui} \textbf{\textit{inakin}}; Borzya \textit{ninakin} (← \ili{Solon});

    cf. \ili{Solon} Ewenki \textit{ninihiŋ}; Orochen\il{Oroqen} \textit{ŋanakin}; \ili{Siberian} Ewenki: \ili{Podkamennyj}, \il{Evenki!Nepa}Nepa, Yerbogochen\il{Yerbogachyon}, \ili{Sym}, \ili{Tungir}, \ili{Urmi}, \ili{Sakhalin} \textit{ŋinakin}; \ili{Upper Lena} \textit{ninakin};

    \textit{other Northern Tungusic}: Lamut \textit{ŋin {\textasciitilde} ŋen}; \ili{Negidal} \textit{ninahin {\textasciitilde} ŋinahin}; \textit{\ili{Southern Tungusic}}: \ili{Nanai} \textit{inda}; \ili{Ulcha} \textit{iŋda}; \ili{Udihe} \textit{ińai}; \ili{Oroch} \textit{inaki}; Orok \textit{nina {\textasciitilde} ŋina}; \ili{Manchu} \textit{indahûn}; \ili{Sibe} \textit{yindaʁůn}
    
    (\citealt{Castrén1856}: 74; \citealt{Janhunen1991}: 49; \citealt{Dorji1998}: 491a; \citealt{Chaoke2014a}: 158; \citealt{Vasilevic1958}: 280b; \citealt{Cincius1975B} 1: 661; \citealt{Hauer1952} 2: 498; \citealt{Zikmundová2013a}: 225);

    \ex ‘to go’: \ili{Nercha} Ewenki: Man’kovo \textit{nene- {\textasciitilde} ŋene-}; \il{Evenki!Khamnigan}Khamnigan Ewenki: Borzya \textit{nene-}, \ili{Urulyungui} \textbf{\textit{ene}}{}-;

    cf. \ili{Solon} Ewenki \textit{nǝnǝ-}; \ili{Siberian} Ewenki: \ili{Tokma}, \ili{Tungir} \textit{gene-}; Remaining dial. \textit{ŋene-};

    \textit{other Northern Tungusic}: Lamut \textit{ŋen-}; \ili{Negidal} \textit{ŋene- {\textasciitilde} gene-}; \textit{\ili{Southern Tungusic}}: \ili{Nanai} \textit{ene-}; \ili{Ulcha}, \ili{Udihe}, \ili{Oroch}, Orok \textit{ŋene-}; \ili{Manchu} \textit{gene-}; \ili{Sibe} \textit{gǝn(ǝ)-}
    
    (\citealt{Castrén1856}: 85; \citealt{Janhunen1991}: 85; \citealt{Dorji1998}: 483b; \citealt{Vasilevic1958}: 284a; \citealt{Cincius1975B} 1: 669--671; \citealt{Hauer1952} 1: 344; \citealt{Zikmundová2013a}: 212);
\z
\z

\ea
The change of the consonant \textit{VŋV} > \textit{VgV} in \il{Evenki!Khamnigan}Khamnigan Ewenki:


\ea ‘he/she’: \ili{Nercha} Ewenki \textit{nuŋan}; \il{Evenki!Khamnigan}Khamnigan Ewenki \textbf{\textit{nugan}};

    cf. \ili{Solon} Ewenki \textit{nugaŋ}; Orochen\il{Oroqen} \textit{nugan}; \ili{Siberian} Ewenki: \ili{Podkamennyj}, \il{Evenki!Nepa}Nepa, Yerbogochen\il{Yerbogachyon}, \ili{Ilimpeya}, \ili{Tungir}, \ili{Zeya}, \ili{Aldan}, \ili{Uchur}, \ili{Urmi}, \ili{Sakhalin} \textit{nuŋan}; \ili{Tokma}, \ili{Upper Lena} \textit{noan}; \ili{North-Baikal} \textit{noan {\textasciitilde} nuan}; \ili{Sym} \textit{nugan};

    \textit{other Northern Tungusic}: Lamut \textit{noŋen}; \ili{Negidal} \textit{noŋan}; \textit{\ili{Southern Tungusic}}: \ili{Nanai} \textit{ńoani}; \ili{Ulcha} \textit{nān}, \textit{nāni {\textasciitilde} nōni}; \ili{Udihe} \textit{nuan}; \ili{Oroch} \textit{nuań {\textasciitilde} nuańi {\textasciitilde} nuŋańi}; Orok \textit{nōni}; \ili{Manchu}, \ili{Sibe} \textit{n.a.}
    
    (\citealt{Castrén1856}: 86; \citealt{Janhunen1991}: 49; \citealt{Chaoke2014bB}: 338; \citealt{Chaoke2014a}: 164; \citealt{Vasilevic1958}: 299b; \citealt{Cincius1975B} 1: 611);
\z
\z

\pagebreak
\ea
The change of \textit{VwV} > \textit{VbV} in \il{Evenki!Khamnigan}Khamnigan Ewenki:
\ea  ‘to catch’: \ili{Nercha} Ewenki \textit{ǰawa-}; \il{Evenki!Khamnigan}Khamnigan Ewenki \textbf{\textit{ǰaba}}\textit{{}-};

    cf. \ili{Solon} Ewenki, Orochen\il{Oroqen} \textit{ǰawa-}; \ili{Siberian} Ewenki: \ili{Ilimpeya}, \ili{Uchur} \textit{ǰaba-}; Remaining dial. \textit{ǰawa}{}-;

    \textit{other Northern Tungusic}: Lamut \textit{ǰaw-}; \ili{Negidal} \textit{ǰawa-}; \textit{\ili{Southern Tungusic}}: \ili{Nanai}, \ili{Ulcha} \textit{ǰapa-}; \ili{Udihe}, \ili{Oroch} \textit{ǰawa-}; Orok \textit{dapa- {\textasciitilde} dappa-}; \ili{Manchu} \textit{ǰafa-}; \ili{Sibe} \textit{ǰaf- {\textasciitilde} ǰavǝ-}
    
    (\citealt{Castrén1856}: 93; \citealt{Janhunen1991}: 90; \citealt{Dorji1998}: 361b; \citealt{Chaoke2014a}: 168; \citealt{Vasilevic1958}: 145b; \citealt{Cincius1975B} 1: 240--241; \citealt{Hauer1952} 2: 510; \citealt{Zikmundová2013a}: 215);
\z
\z

\ea
The reduplication of the consonant \textit{-ŋ-} in \il{Evenki!Khamnigan}Khamnigan Ewenki:
\ea ‘tongue’: \ili{Nercha} Ewenki \textit{iŋi}; \il{Evenki!Khamnigan}Khamnigan Ewenki \textbf{\textit{iŋŋi}};

    cf. \ili{Solon} Ewenki \textit{iŋi}; \ili{Siberian} Ewenki: \ili{Podkamennyj}, \ili{Tokma}, \ili{North-Baikal}, \ili{Barguzin}, \ili{Tungir}, \ili{Zeya}, \ili{Sakhalin} \textit{inni}; \ili{Barguzin}, \ili{North-Baikal}, \ili{Chumikan}, \ili{Zeya}, \ili{Aldan} \textit{inŋi}; \ili{Chumikan}, \ili{Ayan} \textit{ilŋi};

    \textit{other Northern Tungusic}: Lamut \textit{\textsuperscript{i}}\textit{enŋe}; \ili{Negidal} \textit{ińni {\textasciitilde} ińŋi}; \textit{\ili{Southern Tungusic}}: \ili{Nanai} \textit{siŋmu {\textasciitilde} sirmu}; \ili{Ulcha} \textit{sińu}; Orok \textit{sinu}; \ili{Udihe} \textit{iŋi}; \ili{Oroch} \textit{iŋi {\textasciitilde} iŋŋi}; \ili{Manchu} \textit{ileŋgu}
    
    (\citealt{Castrén1856}: 74; \citealt{Janhunen1991}: 52; \citealt{Janhunen1991}: ; \citealt{Dorji1998}: 334b; \citealt{Vasilevic1958}: 174a; \citealt{Cincius1975B} 1: 316; \citealt{Hauer1952} 2: 492).
\z
\z


\subsection{Solon Ewenki influence on Khamnigan Ewenki}

There are many words in \il{Evenki!Khamnigan}Khamnigan Ewenki which were borrowed in Manchuria from \ili{Solon} Ewenki. In most cases the \ili{Solon} influence is observable in the Borzya subdialect.

\ea
The change of affricate *\textit{č} > \textit{t} as in \ili{Solon} Ewenki:

    \ea ‘\ili{Russian}’: \ili{Nercha} Ewenki, \il{Evenki!Khamnigan}Khamnigan Ewenki (\ili{Urulyungui}) \textit{lūča}; cf. \il{Evenki!Khamnigan}Khamnigan Ewenki (Borzya) \textbf{\textit{lūta}}

    ← \ili{Solon} Ewenki \textit{lūt}; cf. \ili{Siberian} Ewenki: \ili{Zeya} \textit{lōča}; \ili{Ayan}, \ili{Aldan}, \ili{May}, \ili{Tommot}, \ili{Uchur} \textit{ńūča}; Remaining dial. \textit{lūča};

    \textit{other Northern Tungusic}: Lamut \textit{ńūči}; \ili{Negidal} \textit{lōča}; \textit{\ili{Southern Tungusic}}: \ili{Nanai} \textit{loča}; \ili{Ulcha} \textit{luča {\textasciitilde} nuča}; \ili{Udihe} \textit{lusa}; \ili{Oroch} \textit{luča}; Orok \textit{lūt’a {\textasciitilde} luča}; \ili{Manchu} \textit{loča} ‘demon, devil’; \ili{Sibe} \textit{n.a.} 
    
    (\citealt{Castrén1856}: 84; \citealt{Janhunen1991}: 98; \citealt{Dorji1998}: 417a; \citealt{Vasilevic1958}: 242a; \citealt{Cincius1975B} 1: 513b; \citealt{Hauer1952} 2: 626);
\z\z


\ea
The appearance of words with a final unstable consonant \textit{-n}\footnote{Vasilevič and Cincius indicate the final -n as a possessive suffix (\citealt{Vasilevic1958}: 26a; \citealt{Cincius1975B} 1: 34b).} as in \ili{Solon} Ewenki and Orochen\il{Oroqen}:

\ea ‘father’: \ili{Nercha} Ewenki: \ili{Urulga} \textit{ama}, Man’kovo \textit{amā}; cf. \il{Evenki!Khamnigan}Khamnigan Ewenki \textbf{\textit{amin}}

    ← \ili{Solon} Ewenki \textit{amiŋ}; Orochen\il{Oroqen} \textit{amin}; cf. \ili{Siberian} Common Ewenki \textit{amā};

    \textit{other Northern Tungusic}: Lamut \textit{amā}; \ili{Negidal} \textit{amaj}; \textit{\ili{Southern Tungusic}}: \ili{Nanai}, \ili{Ulcha}, \ili{Udihe} \textit{amin}; \ili{Oroch} \textit{ama}; Orok \textit{ama {\textasciitilde} amma}; \ili{Manchu} \textit{ama}; \ili{Sibe} \textit{amǝ}
    
    (\citealt{Castrén1856}: 72; \citealt{Janhunen1991}: 23; \citealt{Dorji1998}: 27a; \citealt{Chaoke2014a}: 160; \citealt{Vasilevic1958}: 26a; \citealt{Cincius1975B} 1: 34b; \citealt{Hauer1952} 1: 39; \citealt{Zikmundová2013a}: 204);

    \ex ‘carriage’: \ili{Nercha} Ewenki \textit{terge}; cf. \il{Evenki!Khamnigan}Khamnigan Ewenki \textbf{\textit{tergēn}}

    ← \ili{Solon} Ewenki \textit{tǝγunn}; Orochen\il{Oroqen} \textit{tergen}; cf. \ili{Siberian} Ewenki: \ili{Barguzin}, \ili{Ayan} \textit{terge}; \ili{Manchu}, \ili{Sibe} \textit{sejen};\footnote{\name[Hölzl, Andreas]{Hölzl} (p.c., 2020) drew my attention to the relationship of \ili{Manchu} and \ili{Sibe} forms with the \ili{Mongolic} word. According to \citet{Norman1977}, a cognate \textit{-rg-} regularly yields to \textit{-ǰ-} in \ili{Manchu}, e.g. \ili{Proto-Tungusic} *\textit{bargīlā}, cf. Ewenki \textit{bargīlā} {\textasciitilde} \ili{Manchu} \textit{baǰila} ‘on the other side’, \ili{Proto-Tungusic} \textit{herga(kta)}, cf. Ewenki \textit{irgakta} {\textasciitilde} \ili{Manchu} \textit{iǰa} ‘gadfly’; \ili{Proto-Tungusic} *\textit{tuŋa}. There are also some \ili{Mongolic} loanwords as \textit{songgo-} ‘to choose’ → \ili{Manchu} \textit{sonǰo-} ‘id.’; \ili{Mongolic} \textit{tergen} ‘cart, vehicle’ → \ili{Manchu} \textit{seǰen}; \ili{Mongolic} \textit{torγan} ‘silk’ → \ili{Manchu} \textit{suǰe} (the last two \ili{Mongolic} loanwords are examined in this paper).} \textit{other Tungusic}: \textit{n.a}.

% deleted:, cf. Ewenki \textit{tunŋa} {\textasciitilde} \ili{Manchu} \textit{sunǰa}, etc.

    (\citealt{Castrén1856}: 87; \citealt{Janhunen1991}: 100; \citealt{Dorji1998}: 676b; \citealt{Chaoke2014a}: 163; \citealt{Vasilevic1958}: 424a; \citealt{Cincius1975B} 2: 238; \citealt{Hauer1952} 3: 776; \citealt{Stary1990}: 76);

    Tungusic ← \ili{Mongolic}: \ili{Middle Mongol}: MNT \textit{terge(n)}; HY, ‘Phags-pa, Muq. \textit{tergen}; \ili{Literary Mongolian} \textit{terge} ‘vehicle; cart, wagon, carriage; car; rook (\textit{in chess})’; Manchurian and Onon \ili{Khamnigan Mongol} \textit{terge}; \ili{Dagur} \textit{tǝrǝγ}; \ili{Buryat} \textit{terge} (\citealt{Khabtagaeva2017}: 134; see also \citealt{Doerfer1985}: 104);
\z
\z


\ea
The change of vowel *\textit{o} > \textit{u} as in \ili{Solon} Ewenki:
\ea  ‘fingernail’: \ili{Nercha} Ewenki \textit{ośikta}; \il{Evenki!Khamnigan}Khamnigan Ewenki (\ili{Urulyungui}) \textit{osigta}; cf. \il{Evenki!Khamnigan}Khamnigan Ewenki (Borzya) \textbf{\textit{usigta}}

    ← *\textit{usikta}: \ili{Solon} Ewenki \textit{usitt}; cf. \ili{Siberian} Ewenki: \ili{Podkamennyj}, \il{Evenki!Nepa}Nepa \textit{osīkta}; Yerbogochen\il{Yerbogachyon}, \ili{Tungir}, \ili{Zeya}, \ili{Aldan}, \ili{Uchir}, \ili{Urmi}, \ili{Sakhalin} \textit{ohīkta}; \ili{Sym}, \ili{North-Baikal} \textit{ošīkta}; \ili{Ayan} \textit{ōtta};

    \textit{other Northern Tungusic}: Lamut \textit{oste}; \ili{Negidal} \textit{ōtta}; \textit{\ili{Southern Tungusic}}: \ili{Ulcha} \textit{husta}; \ili{Udihe} \textit{waikta}; \ili{Oroch} \textit{hosi-} ‘to scratch’; \ili{Nanai}, Orok \textit{hosikta}; \ili{Manchu} \textit{usiha}; \ili{Sibe} \textit{uśiχa {\textasciitilde} ušχa}
    
    (\citealt{Castrén1856}: 76; \citealt{Janhunen1991}: 98; \citealt{Dorji1998}: 535a; \citealt{Vasilevic1958}: 328b; \citealt{Cincius1975B} 2: 26b; \citealt{Hauer1952} 3: 973; \citealt{Zikmundová2013a}: 224);
\z
\z


\ea
The insertion of long vowels in the first syllable of words as in \ili{Solon} Ewenki:
\ea ‘when’: \ili{Nercha} Ewenki \textit{alī}; cf. \il{Evenki!Khamnigan}Khamnigan Ewenki \textbf{\textit{āli}}

    ← \ili{Solon} Ewenki \textit{āli}; cf. \ili{Siberian} Ewenki: \ili{Vitim}, \ili{Aldan}, \ili{Uchir} \textit{alī};

    \textit{other Northern Tungusic}: Lamut \textit{ālik} ‘once upon a time’; \ili{Negidal} \textit{āli}; \textit{\ili{Southern Tungusic}}: \ili{Nanai}, \ili{Ulcha}, Orok \textit{hāli}; \ili{Udihe} \textit{ali}; \ili{Oroch} \textit{āli}; \ili{Manchu}, \ili{Sibe} \textit{n.a}.
    
    (\citealt{Castrén1856}: 71; \citealt{Janhunen1991}: 32; \citealt{Dorji1998}: 2b; \citealt{Vasilevic1958}: 25a; \citealt{Cincius1975B} 1: 32);

\z
\z

The loanwords from \ili{Khamnigan Mongol} also belong here, the long vowel is possibly inserted under \ili{Solon} Ewenki influence (\citealt{Janhunen1991}: 100):

\ea ‘medicine, drug’: \ili{Nercha} Ewenki \textit{n.a}.; \il{Evenki!Khamnigan}Khamnigan Ewenki: \ili{Urulyungui} \textit{n.a.}; Borzya \textbf{\textit{ēm}};

    cf. \ili{Solon} Ewenki \textit{ǝŋ}; other Tungusic lgs. \textit{n.a}. (\citealt{Janhunen1991}: 100; \citealt{Dorji1998}: 156a);

    Tungusic ← \ili{Mongolic}: \ili{Middle Mongol}: HY, Muq. \textit{em}; \ili{Literary Mongolian} \textit{em}; Manchurian and Onon \ili{Khamnigan Mongol}, \ili{Buryat} \textit{em}; \ili{Dagur} \textit{ǝm};

    \ili{Mongolic} ← \ili{Turkic}: cf. \ili{Old Turkic} \textit{äm} ‘remedy’;

    \ex ‘silk’: \ili{Nercha} Ewenki \textit{n.a}.; \il{Evenki!Khamnigan}Khamnigan Ewenki \textbf{\textit{tōrga}};

    cf. \ili{Solon} Ewenki \textit{tōγo}; \ili{Siberian} Ewenki: \ili{Podkamennyj}, \il{Evenki!Nepa}Nepa, Yerbogochen\il{Yerbogachyon}, \ili{Ilimpeya}, \ili{Tokma}, \ili{Upper Lena}, \ili{Zeya}, \ili{Uchur}, \ili{Urmi}, \ili{Chumikan}, \ili{Sakhalin} \textit{tōrgā}; \ili{Manchu} \textit{suje}; other Tungusic lgs.: \textit{n.a}.
    
    (\citealt{Janhunen1991}: 100; \citealt{Dorji1998}: 695a; \citealt{Cincius1975B} 2: 199b; \citealt{Hauer1952} 3: 824);

    ← \ili{Mongolic}: \ili{Middle Mongol}: MNT \textit{torqan}; ZY \textit{turγa}; Muq. \textit{torqa}; \ili{Literary Mongolian} \textit{torγan}; Manchurian \ili{Khamnigan Mongol} \textit{torgo}; Onon \ili{Khamnigan Mongol} \textit{torgo(n)}; \ili{Dagur} \textit{tɔrγ\textsuperscript{w}}; \ili{Buryat} \textit{torgo(n)} 
    
    (\citealt{Khabtagaeva2017}: 135; see also \citealt{Doerfer1985}: 94);

    \ex ‘marten’: \ili{Nercha} Ewenki \textit{n.a.}; \il{Evenki!Khamnigan}Khamnigan Ewenki \textbf{\textit{sōlugī}}

    ← \ili{Solon} Ewenki \textit{sōlgǝ {\textasciitilde} sōlŋǝ}; cf. \ili{Siberian} Ewenki: \ili{Podkamennyj}, \ili{Baunt}, \ili{Barguzin}, \ili{North-Baikal}, \ili{Sakhalin}, \ili{Tokko}, \ili{Tommot}, \ili{Tungir}, \ili{Urmi} \textit{soloŋgō}; \ili{Sakhalin} \textit{solga}; Yerbogochen\il{Yerbogachyon} \textit{honoŋgo};

    \textit{other Tungusic}: \ili{Nanai} \textit{sol’u}; \ili{Udihe} \textit{selue}; \ili{Manchu} \textit{solohi}; Remaining lgs. \textit{n.a.} 
    
    (\citealt{Janhunen1991}: 100; \citealt{Dorji1998}: 632a; \citealt{Vasilevic1958}: 362b; \citealt{Cincius1975B} 2: 109a; \citealt{Hauer1952} 3: 813);

    ← \ili{Mongolic}: \ili{Middle Mongol}: MNT \textit{solangqa}; \ili{Literary Mongolian} \textit{solongγa}; Onon \ili{Khamnigan Mongol} \textit{solongo}; \ili{Dagur} \textit{n.a.}; \ili{Buryat} \textit{holongo}
    
    (\citealt{Khabtagaeva2017}: 128; see also \citealt{Doerfer1985}: 39--40; \citealt{Rozycki1994}: 187);
\z
% \z


\ea
\ili{Solon} Ewenki lexical items:


\ea ‘sun’:

\ea \il{Evenki!Khamnigan}Khamnigan Ewenki (Borzya) \textbf{\textit{sigün}}

    ← \ili{Solon} Ewenki \textit{siguŋ}; cf. \ili{Siberian} Ewenki: \ili{Zeya}, \ili{Aldan}, \ili{Khingan}, \ili{Uchur} \textit{sigun}; \textit{other Northern Tungusic}: \ili{Negidal} \textit{siwun {\textasciitilde} sigun}; \textit{\ili{Southern Tungusic}}: \ili{Nanai} \textit{siu}; \ili{Ulcha} \textit{siun {\textasciitilde} sun}; \ili{Udihe} \textit{sūn}; \ili{Oroch} \textit{seun}; Orok \textit{šun}; \ili{Manchu} \textit{šun}; \ili{Sibe} \textit{šu\textsuperscript{n}};

    \ex \ili{Nercha} Ewenki \textit{dilacā}; \il{Evenki!Khamnigan}Khamnigan Ewenki (\ili{Urulyungui}) \textit{dilacā {\textasciitilde} gilacā};

    cf. Orochen\il{Oroqen} \textit{diliča}; \ili{Siberian} Ewenki: \ili{Podkamennyj}, \il{Evenki!Nepa}Nepa, Yerbogochen\il{Yerbogachyon}, \ili{Ilimpeya}, \ili{Upper Lena}, \ili{North-Baikal}, \ili{Tungir}, \ili{Aldan}, \ili{Uchur}, \ili{Urmi}, \ili{Chumikan}, \ili{Sakhalin} \textit{dïlačā}; \textit{other Tungusic lgs}: Lamut \textit{dilača}
    
    (\citealt{Castrén1856}: 90; \citealt{Janhunen1991}: 98; \citealt{Dorji1998}: 607b; \citealt{Chaoke2014a}: 152; \citealt{Vasilevic1958}: 350b, 128a; \citealt{Cincius1975B} 2: 78; 1: 206a; \citealt{Hauer1952} 3: 867; \citealt{Zikmundová2013a}: 222);
\z

    \ex ‘new’: \ili{Nercha} Ewenki \textit{n.a.}; \il{Evenki!Khamnigan}Khamnigan Ewenki \textbf{\textit{irkekīn}}

    ← \ili{Solon} Ewenki \textit{irkǝhin {\textasciitilde} ikkiŋ}; Orochen\il{Oroqen} \textit{irkin}; cf. \ili{Siberian} Ewenki: \ili{Sakhalin} \textit{irkekīn} ‘new, fresh’;

    \textit{other Northern Tungusic}: Lamut ; \ili{Negidal} \textit{ihihīn {\textasciitilde} īhin {\textasciitilde} ihēhin}; \textit{\ili{Southern Tungusic}}: \ili{Nanai} \textit{sikū}; \ili{Ulcha} \textit{sičeun}; \ili{Udihe} \textit{sike}; \ili{Oroch} \textit{ikken}; Orok \textit{sitew {\textasciitilde} siteu}; \ili{Manchu} \textit{iče}; \ili{Sibe} \textit{ičǝ} 
    
    (\citealt{Janhunen1991}: 52; \citealt{Dorji1998}: 339b; \citealt{Chaoke2014a}: 166; \citealt{Vasilevic1958}: 178a; \citealt{Cincius1975B} 1: 328a; \citealt{Hauer1952} 2: 483; \citealt{Zikmundová2013a}: 215)
\z
\z


\ea
There are some \ili{Russian} loanwords in \il{Evenki!Khamnigan}Khamnigan Ewenki which were investigated during our fieldwork. These were possibly borrowed before the migration from Russia,\footnote{\ili{Russian} loanwords in \ili{Khamnigan Mongol} were analyzed in detail by \citet{Gruntov2015}. These loanwords are possibly also present in \il{Evenki!Khamnigan}Khamnigan Ewenki due to the fact that Russians lived together with \ili{Khamnigan Mongol} and \il{Evenki!Khamnigan}Khamnigan Ewenki people until the 1960s as one of our \il{Evenki!Khamnigan}Khamnigan Ewenki informants told us.} but are not found in Castrén’s monograph:
\ea ‘book’: \il{Evenki!Khamnigan}Khamnigan Ewenki \textbf{\textit{kinīska}}

    ← \ili{Russian} \textit{knížka} ‘small book’ < \textit{kniga} +\textit{ka} \ili{Russian} diminutive suffix;

    \ex ‘pig’: \il{Evenki!Khamnigan}Khamnigan Ewenki \textbf{\textit{čūske}}

    ← \ili{Russian} \textit{čúška} `piglet';

    \ex ‘bread’: \il{Evenki!Khamnigan}Khamnigan Ewenki \textbf{\textit{būlke}}

    ← \ili{Russian} \textit{búlka} ‘roll, white loaf’;

    \ex ‘candy, sweet’: \il{Evenki!Khamnigan}Khamnigan Ewenki \textbf{\textit{hampyētke}}

    ← \ili{Russian} \textit{konfétka < konfeta} +\textit{ka} \ili{Russian} diminutive suffix ← \ili{German} ← \ili{Latin};

    \ex ‘sugar’: \il{Evenki!Khamnigan}Khamnigan Ewenki \textbf{\textit{sākar}}

    ← \ili{Russian} \textit{sáhar} ← \ili{Arabic}; etc.
\z
\z


\subsection{Mongolic loanwords}

The \ili{Mongolic} loanwords in \ili{Nercha} and \il{Evenki!Khamnigan}Khamnigan Ewenki can be divided into two groups. The first one includes the \ili{Mongolic} loanwords peculiar to both dialects, while the second group contains loanwords borrowed at a different time.

\subsubsection{Mongolic loanwords peculiar to both dialects}


\ea
Most \ili{Mongolic} loanwords were borrowed into other Tungusic languages too:

\ea ‘person, man’: \ili{Nercha} Ewenki, \il{Evenki!Khamnigan}Khamnigan Ewenki \textbf{\textit{beye}};

    cf. \ili{Solon} Ewenki \textit{bǝyǝ}; Orochen\il{Oroqen} \textit{beye}; \ili{Siberian} Common Ewenki \textit{beye};

    \textit{other Northern Tungusic}: Lamut \textit{bey}; \ili{Negidal} \textit{beye}; \textit{\ili{Southern Tungusic}}: \ili{Nanai}, \ili{Ulcha}, \ili{Udihe}, \ili{Oroch}, Orok \textit{beye}; \ili{Manchu} \textit{beye}; \ili{Sibe} \textit{bey} 
    
    (\citealt{Castrén1856}: 94; \citealt{Janhunen1991}: 59; \citealt{Dorji1998}: 67b; \citealt{Chaoke2014a}: 158; \citealt{Vasilevic1958}: 73b; \citealt{Cincius1975B} 1: 122a; \citealt{Hauer1952} 1: 89; \citealt{Zikmundová2013a}: 206)

    ← \ili{Mongolic} ‘body, organism’: \ili{Middle Mongol}: MNT \textit{beye {\textasciitilde} be’e}; HY, ‘Phags-pa, Muq., Rasulid \textit{beye};

    \ili{Literary Mongolian} \textit{beye}; Manchurian, Onon and \ili{Mongolian} \ili{Khamnigan Mongol} \textit{beye}; Dadal-sum Khamnigan \textit{biye}; \ili{Dagur} \textit{bǝy}; \ili{Buryat} \textit{beye}\il{Khamnigan Mongol}
    
    (\citealt{Khabtagaeva2017}: 64; see also \citealt{Doerfer1985}: 20; \citealt{Rozycki1994}: 29);

    \ex ‘maral deer’: \ili{Nercha} Ewenki, \il{Evenki!Khamnigan}Khamnigan Ewenki \textbf{\textit{bugu}};

    cf. \ili{Solon} Ewenki \textit{buγu}; \ili{Siberian} Ewenki: \ili{Barguzin}, \ili{Upper Lena}, \ili{Vitim} \textit{buγu}; \ili{Zeya} \textit{buγ}; \ili{Ayan} \textit{buγe}; \ili{Sakhalin}, \ili{Urmi}, \ili{Uchur} \textit{buγuj};

    \textit{other Northern Tungusic}: Lamut \textit{n.a.}; \ili{Negidal} \textit{bočan}; \textit{\ili{Southern Tungusic}}: \ili{Nanai} \textit{bočā}; \ili{Ulcha} \textit{buča {\textasciitilde} boča}, \ili{Oroch} \textit{buča}; \ili{Manchu} \textit{buhû}; \ili{Sibe} \textit{bukun ihan} ‘mountain antilope’
    
    (\citealt{Castrén1856}: 95; \citealt{Janhunen1991}: 24; \citealt{Dorji1998}: 84a; \citealt{Vasilevic1958}: 64a; \citealt{Cincius1975B} 1: 101b; \citealt{Hauer1952} 1: 119; \citealt{Stary1990}: 9);

    ← \ili{Mongolic}: \ili{Middle Mongol}: MNT, HY \textit{buqu}; ZY, Muq. \textit{buγu}; Rasulid \textit{bu\.{g}a}; \ili{Literary Mongolian} \textit{buγu} ‘a male deer, stag’; Onon \ili{Khamnigan Mongol} \textit{bugu}; \ili{Dagur} \textit{bɔγ\textsuperscript{w}}; \ili{Buryat} \textit{buga}
    
    (\citealt{Khabtagaeva2017}: 67; see also \citealt{Doerfer1985}: 78; \citealt{Rozycki1994}: 37);

    \ex ‘type of duck’: \ili{Nercha} Ewenki, \il{Evenki!Khamnigan}Khamnigan Ewenki \textbf{\textit{aŋgir}};

    cf. \ili{Solon} Ewenki \textit{aŋgir}; \textit{other Northern Tungusic}: \ili{Negidal} \textit{ani}; \textit{\ili{Southern Tungusic}}: \ili{Nanai} \textit{āŋgi}; \ili{Manchu} \textit{aŋgir}; Remaining lgs. \textit{n.a}. (\citealt{Castrén1856}: 71; \citealt{Janhunen1991}: 51; \citealt{Chaoke2014bB}: 52; \citealt{Vasilevic1958}: 32b; \citealt{Cincius1975B} 1: 43b);

    Tungusic ← \ili{Mongolic}: \ili{Middle Mongol}: MNT, ZY, HY \textit{anggir}; \ili{Literary Mongolian} \textit{anggir}; Onon \ili{Khamnigan Mongol}, \ili{Buryat} \textit{angir}; \ili{Dagur} \textit{n.a}.;

    \ili{Mongolic} ← \ili{Turkic}:\footnote{\ili{Turkic}: Yakut \textit{andï {\textasciitilde} annï} ‘scoter, pochard; black duck’ → \ili{Siberian} Ewenki: \ili{Ayan}, \ili{Uchur} \textit{anni {\textasciitilde} andi} ‘black duck’.} cf. \ili{Old Turkic} \textit{aŋït} ‘a rather large bird predominantly red; the ruddy goose (\textit{Anas casarca})’
    
    (\citealt{Khabtagaeva2017}: 59; see also \citealt{Doerfer1985}: 68; \citealt{Rozycki1994}: 19); 

    \ex ‘to roar’: \ili{Nercha} Ewenki, \il{Evenki!Khamnigan}Khamnigan Ewenki \textbf{\textit{barkirā-}};

    cf. \ili{Solon} Ewenki \textit{baggera-}; \ili{Siberian} Ewenki: \textit{n.a}; Remaining Tungusic lgs. \textit{n.a}.
    
    (\citealt{Castrén1856}: 94; \citealt{Janhunen1991}: 31; \citealt{Dorji1998}: 53a; \citealt{Cincius1975B} 1: 75b); 

    ← \ili{Mongolic}: \ili{Middle Mongol} \textit{n.a}.; \ili{Literary Mongolian} \textit{barkira-}; Onon \ili{Khamnigan Mongol} \textit{barkir-}; \ili{Dagur} \textit{n.a.}; \ili{Buryat} \textit{barxir-} (\citealt{Khabtagaeva2017}: 63; see also \citealt{Doerfer1985}: 101);

    \ex ‘to think’: \ili{Nercha} Ewenki, \il{Evenki!Khamnigan}Khamnigan Ewenki \textbf{\textit{bodo-}};

    cf. \ili{Solon} Ewenki, Orochen\il{Oroqen} \textit{bodo-}; \ili{Siberian} Ewenki \textit{n.a.};

    \textit{other Tungusic}: \ili{Nanai}, \ili{Ulcha}, \ili{Udihe} \textit{bodo-}; \ili{Oroch} \textit{budu-}; Orok \textit{boddo- {\textasciitilde} bodo-}; \ili{Manchu} \textit{bodo-}; \ili{Sibe} \textit{bot- {\textasciitilde} bod(ǝ)-}; Remaining lgs. \textit{n.a}.
    
    (\citealt{Castrén1856}: 95; \citealt{Janhunen1991}: 101; \citealt{Dorji1998}: 75b; \citealt{Chaoke2014a}: 168; \citealt{Cincius1975B} 1: 88a; \citealt{Hauer1952} 1: 104; \citealt{Zikmundová2013a}: 207);

    ← \ili{Mongolic}: \ili{Middle Mongol}: \textit{n.a}.; \ili{Literary Mongolian} \textit{bodo-}; Manchurian \ili{Khamnigan Mongol} \textit{bod-}; Onon \ili{Khamnigan Mongol} \textit{bodo-}; \ili{Dagur} \textit{bɔd-}; \ili{Buryat} \textit{bodo-} 
    
    (\citealt{Khabtagaeva2017}: 65; see also \citealt{Doerfer1985}: 78; \citealt{Rozycki1994}: 33);
\z
\z

\ea
The \ili{Mongolic} loanwords in \ili{Nercha} and \il{Evenki!Khamnigan}Khamnigan Ewenki with \ili{Khamnigan Mongol} features
\ea The disappearance of the \ili{Mongolic} consonant \textit{q-} through *\textit{χ-}, which points to an early period of borrowing:

\ea ‘twenty’: \ili{Nercha} Ewenki, \il{Evenki!Khamnigan}Khamnigan Ewenki \textbf{\textit{orin}};

    cf. \ili{Solon} Ewenki \textit{uriŋ}; Orochen\il{Oroqen} \textit{urin}; \ili{Siberian} Ewenki: \ili{Barguzin} \textit{orin};

    \textit{other Northern Tungusic}: \ili{Negidal} \textit{ojin}; \textit{\ili{Southern Tungusic}}: \ili{Nanai} \textit{hori}; \ili{Ulcha}, Orok \textit{hori(n)}; \ili{Udihe} \textit{waji {\textasciitilde} uai}; \ili{Oroch} \textit{oi}; \ili{Manchu} \textit{orin}; \ili{Sibe} \textit{ori\textsuperscript{n}} 
    
    (\citealt{Castrén1856}: 75; \citealt{Janhunen1991}: 23; \citealt{Dorji1998}: 534a; \citealt{Chaoke2014a}: 170; \citealt{Vasilevic1958}: 326b; \citealt{Cincius1975B} 2: 24; \citealt{Hauer1952} 3: 740; \citealt{Zikmundová2013a}: 219);

    Tungusic ← \ili{Mongolic}: \ili{Middle Mongol}: MNT, HY, ‘Phags-pa, Leiden, Muq. \textit{qorin}; \ili{Literary Mongolian} \textit{qorin}; Manchurian \ili{Khamnigan Mongol} \textit{kori(n)}; Onon \ili{Khamnigan Mongol} \textit{xori(n)}; Dadal-sum Khamnigan \textit{χori}; \ili{Mongolian} Khamnigan \textit{orin}; \ili{Buryat} \textit{xori(n)}\il{Khamnigan Mongol}
    
    (\citealt{Khabtagaeva2017}: 123; see also \citealt{Doerfer1985}: 81; \citealt{Rozycki1994}: 169);

    \ex ‘thumb’: \ili{Nercha} Ewenki, \il{Evenki!Khamnigan}Khamnigan Ewenki \textbf{\textit{ürügün}};

    cf. \ili{Solon} Ewenki \textit{ǝruguŋ}; \ili{Siberian} Ewenki: \ili{Upper Lena}, \ili{Chumikan} \textit{urugun}; \ili{Barguzin} \textit{huruwūn}; \ili{Aldan}, \ili{Sakhalin}, \ili{Urmi}, \ili{Uchur} \textit{hurugun};

    \textit{other Northern Tungusic}: Lamut \textit{huregen}; \ili{Negidal} \textit{hojeŋen}; \textit{\ili{Southern Tungusic}}: \ili{Udihe} \textit{hue}; \ili{Oroch} \textit{hōŋo(n)}; \ili{Manchu} \textit{urhun}
    
    (\citealt{Castrén1856}: 78; \citealt{Janhunen1991}: 46; \citealt{Chaoke2014bB}: 153; \citealt{Cincius1975B} 2: 354b; \citealt{Hauer1952} 3: 969); 

    ← \ili{Mongolic} *\textit{χurugun}: \ili{Middle Mongol}: MNT \textit{quru’u(n)}; HY \textit{quru’un}; Leiden \textit{qurūn}; Muq. \textit{qurūn {\textasciitilde} χurūn}; Rasulid \textit{qurūn}; \ili{Literary Mongolian} \textit{quruγun} ‘finger, toe; finger-like’; Manchurian \ili{Khamnigan Mongol} \textit{kurū(n)}; Onon \ili{Khamnigan Mongol} \textit{xurū}; \ili{Mongolian} Khamnigan \textit{xurguon}; \ili{Buryat} \textit{xurgan}; \ili{Dagur} \textit{xɔr\={ɔ}} \citep[90]{Khabtagaeva2017};
\z

    \ex The preservation of original \ili{Mongolic} *\textit{ti} which later became \textit{či}:
\ea ‘thirty’: \ili{Nercha} Ewenki: \ili{Urulga} \textit{gučin}, Man’kovo \textbf{\textit{gutin}}; \il{Evenki!Khamnigan}Khamnigan Ewenki: Borzya \textbf{\textit{gutin}}, \ili{Urulyungui} \textit{gučin};

    cf. \ili{Solon} Ewenki \textit{gutiŋ}; \ili{Siberian} Ewenki: \textit{n.a.} instead \textit{ilan ǰār};

    \textit{other Tungusic}: \ili{Nanai} \textit{goči}; \ili{Ulcha}, \ili{Oroch} \textit{guti(n)}; \ili{Manchu} \textit{gûsin}; \ili{Sibe} \textit{goźi\textsuperscript{n}}; Remaining lgs. \textit{n.a.} 
    
    (\citealt{Castrén1856}: 82; \citealt{Janhunen1991}: 23, 76; \citealt{Dorji1998}: 235a; \citealt{Cincius1975B} 1: 175b; \citealt{Hauer1952} 2: 397; \citealt{Zikmundová2013a}: 213);

    ← \ili{Mongolic}: \ili{Middle Mongol}: MNT, ‘Phags{}-pa \textit{qučin}; Muq. \textit{γučin}; Rasulid \textit{qučin}; \ili{Literary Mongolian} \textit{γučin}; Manchurian and Onon \ili{Khamnigan Mongol} \textit{guči(n)}; Dadal-sum \ili{Khamnigan Mongol} \textit{γutš’i}; \ili{Dagur} \textit{gɔč} {\textasciitilde} \textit{gɔčin}; \ili{Buryat} \textit{guša(n)}
    
    (\citealt{Khabtagaeva2017}: 87; see also \citealt{Doerfer1985}: 79; \citealt{Rozycki1994}: 95);

    \ex ‘to write’: \ili{Nercha} Ewenki \textbf{\textit{biťi-}} \textit{{\textasciitilde} biči-}; \il{Evenki!Khamnigan}Khamnigan Ewenki \textit{biči-};

    cf. \ili{Solon} Ewenki \textit{biti-}; \ili{Siberian} Ewenki: Upper Amur \textit{biči-};

    \ex ‘letter, script’: \textit{other Northern Tungusic}: \ili{Negidal} \textit{bitehe}; \textit{\ili{Southern Tungusic}}: \ili{Nanai} \textit{bičhe}; \ili{Ulcha} \textit{bithe}; \ili{Udihe} \textit{n.a.}; \ili{Oroch} \textit{bitige {\textasciitilde} bithe}; Orok \textit{bičihe}; \ili{Manchu} \textit{bithe}; \ili{Sibe} \textit{bitkǝ}
    
    (\citealt{Castrén1856}: 95; \citealt{Janhunen1991}: 101; \citealt{Dorji1998}: 74b; \citealt{Cincius1975B} 1: 86; \citealt{Hauer1952} 1: 98; \citealt{Zikmundová2013a}: 206);

    Tungusic ← \ili{Mongolic} \textit{*biti-}: \ili{Middle Mongol}: MNT, ZY, HY, Muq., Rasulid \textit{bičik} ‘script’; \ili{Literary Mongolian} \textit{biči-} ‘to write’; Manchurian and Onon \ili{Khamnigan Mongol} \textit{biči-}; \ili{Dagur} \textit{n.a.}; \ili{Buryat} \textit{beše-};

    \ili{Mongolic} ← \ili{Turkic}: \ili{Old Turkic} \textit{biti-} ‘to write’ (\citealt{Khabtagaeva2017}: 65; see also \citealt{Doerfer1985}: 76; \citealt{Rozycki1994}: 31);
\z
\pagebreak
    \ex The preservation of the vowels \textit{-u-} and \textit{-ü-} in the last syllable:

\ea ‘butter, oil’: \ili{Nercha} Ewenki (Man’kovo) \textbf{\textit{tosun}}, (\ili{Urulga}) \textbf{\textit{tohun}}; \il{Evenki!Khamnigan}Khamnigan Ewenki \textbf{\textit{tosun}};

    cf. \ili{Solon} Ewenki \textit{n.a.}; \ili{Siberian} Ewenki: \ili{North-Baikal} \textit{tosun}; \ili{Aldan}, \ili{Barguzin} \textit{tohun}; \textit{other Tungusic}: \textit{n.a}.
    
    (\citealt{Castrén1856}: 88; \citealt{Janhunen1991}: 24; \citealt{Vasilevic1958}: 395b; \citealt{Cincius1975B} 2: 201a);

    ← \ili{Mongolic}: \ili{Middle Mongol}: MNT, HY, ‘Phags{}-pa, Leiden, Muq. \textit{tosun}; Rasulid \textit{ṭosun}; \ili{Literary Mongolian} \textit{tosun}; Manchurian \ili{Khamnigan Mongol} \textit{tohun}; Onon \ili{Khamnigan Mongol} \textit{tosu(n) {\textasciitilde} toso(n)}; Dadal-sum \ili{Khamnigan Mongol} \textit{t`osu}; \ili{Dagur} \textit{tɔs}; \ili{Buryat} \textit{toho(n)} \citep[134]{Khabtagaeva2017};

    \ex ‘age’: \ili{Nercha} Ewenki \textit{n.a.}; \il{Evenki!Khamnigan}Khamnigan Ewenki \textbf{\textit{nasun}};

    cf. \ili{Solon} Ewenki \textit{n.a.}; \ili{Siberian} Ewenki: \ili{Barguzin} \textit{nahun}; \textit{other Tungusic}: \textit{n.a}.
    
    (\citealt{Janhunen1991}: 24; \citealt{Cincius1975B} 1: 587a);

    ← \ili{Mongolic}: MNT \textit{nasu}; ‘Phags-pa \textit{nasu {\textasciitilde} nasun}; Leiden, Muq. \textit{nasun}; \ili{Literary Mongolian} \textit{nasun}; Manchurian \ili{Khamnigan Mongol} \textit{nahun}; Onon \ili{Khamnigan Mongol} \textit{nasu(n) {\textasciitilde} nasa(n)}; \ili{Mongolian} Khamnigan \textit{nasu}; \ili{Dagur} \textit{nas}; \ili{Buryat} \textit{naha(n)} (\citealt{Khabtagaeva2017}: 119; see also \citealt{Doerfer1985}: 127);

    \ex ‘bovine’: \ili{Nercha} Ewenki \textbf{\textit{ükür}}; \il{Evenki!Khamnigan}Khamnigan Ewenki: \ili{Urulyungui} \textbf{\textit{ükür}}; Borzya \textbf{\textit{hükür}};

    cf. \ili{Solon} Ewenki \textit{n.a.}; Orochen\il{Oroqen} \textit{ukur}; \ili{Siberian} Ewenki: \ili{Barguzin}, \ili{Zeya}, \ili{Aldan}, \ili{Khingan}, \ili{Uchur} \textit{hukur}; \textit{other Tungusic}: \textit{n.a}.
    
    (\citealt{Castrén1856}: 83; \citealt{Janhunen1991}: 46; \citealt{Chaoke2014a}: 157; \citealt{Vasilevic1958}: 491b; \citealt{Cincius1975B} 2: 341);

    Tungusic ← \ili{Mongolic}: \ili{Middle Mongol}: MNT \textit{hüker}; ZY \textit{üger}; HY \textit{hüger}; ‘Phags{}-pa, Leiden, Muq. \textit{hüker}; Rasulid \textit{üker}; \ili{Literary Mongolian} \textit{ükür}; Manchurian \ili{Khamnigan Mongol} \textit{üker}; Onon \ili{Khamnigan Mongol} \textit{üker {\textasciitilde} ökör} (← \ili{Khalkha}); Dadal-sum \ili{Khamnigan Mongol} \textit{ük`ür}; \ili{Dagur} \textit{xukur}; \ili{Buryat} \textit{üxer};

    \ili{Mongolic} *\textit{ükür} ‘bovine animal, ox, cow’ < \textit{hükür} ← \ili{Bulgar Turkic} \textit{*hökür}: cf. \ili{Old Turkic} \textit{öküz} ‘ox’ ← Tokharian\il{Tocharian} (\citealt{Khabtagaeva2017}: 89; see also \citealt{Doerfer1985}: 67);

\z

    \ex The preservation of \ili{Middle Mongol} intervocalic \textit{q} which later is voiced:

\ea ‘hedgehog’: \ili{Nercha} Ewenki, \il{Evenki!Khamnigan}Khamnigan Ewenki \textbf{\textit{dorokon}};

    cf. \ili{Solon} Ewenki \textit{n.a.}; \ili{Siberian} Ewenki: \ili{Podkamennyj}, \ili{Barguzin} \textit{dorokon}; \ili{Zeya} \textit{dorogun}; \ili{Chumikan} \textit{toraku}; \ili{Urmi} \textit{torokōn};

    \textit{other Northern Tungusic}: Lamut \textit{n.a.}; \ili{Negidal} \textit{doho}; \textit{\ili{Southern Tungusic}}: \ili{Nanai} \textit{doro}; \ili{Ulcha} \textit{doro(n)}, \ili{Oroch} \textit{dogo}; \ili{Udihe}, Orok \textit{n.a.}; \ili{Manchu} \textit{dorgon}; \ili{Sibe} \textit{dorhon} 
    
    (\citealt{Castrén1856}: 90; \citealt{Janhunen1991}: 40; \citealt{Vasilevic1958}: 122a; \citealt{Cincius1975B} 1: 217a; \citealt{Hauer1952} 1: 210; \citealt{Stary1990}: 19);

    ← \ili{Mongolic}: \ili{Middle Mongol}: Leiden \textit{dorkon}; \ili{Literary Mongolian} \textit{doroγon}, \ili{Mongolian} \ili{Khamnigan Mongol} \textit{dorokon}; Onon \ili{Khamnigan Mongol} \textit{dorogon}; \ili{Dagur} \textit{n.a.}; \ili{Buryat} \textit{dorgon} (\citealt{Khabtagaeva2017}: 80; see also \citealt{Doerfer1985}: 39; \citealt{Rozycki1994}: 62);

\z

    \ex The preservation of \ili{Mongolic} *\textit{ni}:

\ea ‘homeland’: \ili{Nercha} Ewenki \textit{n.a.}; \il{Evenki!Khamnigan}Khamnigan Ewenki \textbf{\textit{nitug}};

    cf. \ili{Solon} Ewenki, \ili{Siberian} Ewenki \textit{n.a.}; \textit{other Tungusic}: \textit{n.a}. \citep[24]{Janhunen1991};

    ← \ili{Mongolic}: \ili{Middle Mongol}: MNT \textit{nuntuq {\textasciitilde} nutuq}; HY \textit{nuntuq}; Muq. \textit{nutuq}; \ili{Literary Mongolian} \textit{nituγ}; Manchurian \ili{Khamnigan Mongol} \textit{nitug {\textasciitilde} nutug}; Onon \ili{Khamnigan Mongol} \textit{nitug}; \ili{Dagur} \textit{nɔtɔg}; \ili{Buryat} \textit{nyutag};

    \ex ‘sin’: \ili{Nercha} Ewenki, \il{Evenki!Khamnigan}Khamnigan Ewenki \textbf{\textit{nigul}};

    cf. \ili{Solon} Ewenki \textit{niw​ul}; \ili{Siberian} Ewenki: \ili{Upper Lena} \textit{niŋul}; \textit{other Tungusic}: \textit{n.a}. (\citealt{Castrén1856}: 85; \citealt{Janhunen1991}: 24; \citealt{Chaoke2014bB}: 316; \citealt{Cincius1975B} 1: 589a);

    ← \ili{Mongolic}: \ili{Middle Mongol}: ‘Phags-pa \textit{ni’ül}; \ili{Literary Mongolian} \textit{niγul}; Manchurian \ili{Khamnigan Mongol} \textit{nigül}; Onon \ili{Khamnigan Mongol} \textit{nügel} (← \ili{Buryat}); \ili{Dagur} \textit{nugul}; \ili{Buryat} \textit{nügel} \citep[120]{Khabtagaeva2017};

\z
    \ex The development of the \ili{Mongolic} sequence \textit{egü} > \textit{ē} as in Manchurian \ili{Khamnigan Mongol}:\footnote{E.g. Manchurian \ili{Khamnigan Mongol} \textit{kēgen} ‘child’ {\textasciitilde} \ili{Mongolic} ‘girl’: \ili{Literary Mongolian} \textit{keüken}; \ili{Buryat} \textit{xǖxen}; Manchurian \ili{Khamnigan Mongol} \textit{tēke} ‘history’ {\textasciitilde} \ili{Mongolic}: \ili{Literary Mongolian} \textit{teüke}; \ili{Buryat} \textit{tǖxe}; Manchurian \ili{Khamnigan Mongol} \textit{dē} ‘younger brother’ {\textasciitilde} Mongoli: \ili{Literary Mongolian} \textit{degüü}; \ili{Buryat} \textit{dǖ}, etc. (\citealt{Janhunen1990}: 28).}

\ea ‘mare’: \ili{Nercha} Ewenki \textbf{\textit{gēk}}; \il{Evenki!Khamnigan}Khamnigan Ewenki: Borzya \textbf{\textit{gēg}}, \ili{Urulyungui} \textbf{\textit{gē}};

    cf. \ili{Solon} Ewenki \textit{gǝ}; \ili{Siberian} Ewenki: \ili{Vitim} \textit{gēγ}; \ili{Barguzin} \textit{gog}; \ili{Upper Lena} \textit{gēn};

    \textit{other Tungusic}: \ili{Manchu} \textit{geo} 
    
    (\citealt{Castrén1856}: 81; \citealt{Janhunen1991}: 41; \citealt{Dorji1998}: 207a; \citealt{Vasilevic1958}: 84a; \citealt{Cincius1975B} 1: 145; \citealt{Hauer1952} 1: 345);

    ← \ili{Mongolic}: \ili{Middle Mongol}: MNT, HY, Muq. \textit{ge’ün}; \ili{Literary Mongolian} \textit{gegüü}; Manchurian \ili{Khamnigan Mongol} \textit{gē}; \ili{Mongolian} Khamnigan \textit{gökü}; Onon \ili{Khamnigan Mongol}, \ili{Buryat} \textit{gǖ}; \ili{Dagur} \textit{gǝu} (\citealt{Khabtagaeva2017}: 52; see also \citealt{Doerfer1985}: 102; \citealt{Rozycki1994}: 88);

\z
    \ex The preservation of \ili{Mongolic} \textit{VgV}, which became a secondary long vowel in Modern Mongol:

\ea ‘goat’: \ili{Nercha} Ewenki, \il{Evenki!Khamnigan}Khamnigan Ewenki \textbf{\textit{imagan}};

    cf. \ili{Solon} Ewenki \textit{imaγaŋ}; \ili{Siberian} Ewenki \textit{imagan}: \ili{Barguzin} ‘goat’; \ili{Uchur}, \ili{Urmi}, \ili{Sakhalin} ‘bastard calf’;

    \textit{other Northern Tungusic}: Lamut \textit{n.a.}; \ili{Negidal} \textit{imaja}; \textit{\ili{Southern Tungusic}}: \ili{Nanai}, \ili{Ulcha}, \ili{Udihe}, \ili{Oroch} \textit{ima}; Orok \textit{n.a.}; \ili{Manchu} \textit{imahû} ‘Capricorn’; \ili{Sibe} \textit{n.a.} 
    
    (\citealt{Castrén1856}: 75; \citealt{Janhunen1991}: 100; \citealt{Dorji1998}: 330b; \citealt{Vasilevic1958}: 167b; \citealt{Cincius1975B} 1: 312b; \citealt{Hauer1952}: 497);

    ← \ili{Mongolic}: \ili{Middle Mongol}: MNT \textit{ima’a}; HY \textit{ima’an}; Muq. \textit{ima’an {\textasciitilde} imān}; Rasulid \textit{imān}; \ili{Literary Mongolian} \textit{imaγan}; Manchurian \ili{Khamnigan Mongol} \textit{imā(n)}; Onon \ili{Khamnigan Mongol} \textit{yamā(n)} (← \ili{Buryat}); \ili{Mongolian} Khamnigan \textit{imagān}; \ili{Dagur} \textit{imā}; \ili{Buryat} \textit{yamā(n)} (\citealt{Khabtagaeva2017}: 90; see also \citealt{Doerfer1985}: 37;  \citealt{Rozycki1994}: 116);

    \ex ‘antelope’: \ili{Nercha} Ewenki \textit{n.a.}; \il{Evenki!Khamnigan}Khamnigan Ewenki \textbf{\textit{ǰegerēn}};

    cf. \ili{Solon} Ewenki \textit{dʒǝgǝrǝŋ} ‘\ili{Mongolian} gazelle’; \ili{Siberian} Ewenki \textit{n.a.};

    \textit{other Tungusic} ‘roe deer, wild goat’: \ili{Udihe} \textit{ǰeli}; \ili{Manchu} \textit{ǰeren}; Remaining lgs. \textit{n.a.}
    
    (\citealt{Janhunen1991}: 100; \citealt{Chaoke2014bB}: 40; \citealt{Cincius1975B} 1: 282b; \citealt{Hauer1952} 2: 530);

    Tungusic ← \ili{Mongolic}: \ili{Middle Mongol}: HY, Muq. \textit{ǰēren}; \ili{Literary Mongolian} \textit{ǰeger{}-e(n)}; Manchurian \ili{Khamnigan Mongol} \textit{ǰēre(n)}; Onon \ili{Khamnigan Mongol} \textit{dzēr} (← \ili{Khalkha}); \ili{Dagur} \textit{ǰǝrǝn}; \ili{Buryat} \textit{zēren};

    \ili{Mongolic} ← \ili{Turkic}: cf. \ili{Old Turkic} \textit{yägrän} ‘gazelle’ (\citealt{Doerfer1985}: 136; \citealt{Rozycki1994}: 122);

    \ex ‘camel’: \ili{Nercha} Ewenki \textit{n.a.}; \il{Evenki!Khamnigan}Khamnigan Ewenki \textbf{\textit{temegēn}};

    \ili{Solon} Ewenki \textit{tǝmǝgǝŋ}; Orochen\il{Oroqen} \textit{temegen}; \ili{Siberian} Ewenki: \ili{Barguzin} \textit{temegēn};

    \textit{other Tungusic}: \ili{Nanai}, \ili{Oroch} \textit{teme}; \ili{Manchu} \textit{temen}; Remaining lgs. \textit{n.a}. 
    
    (\citealt{Janhunen1991}: 100; \citealt{Chaoke2014bB}: 56; \citealt{Chaoke2014a}: 157; \citealt{Cincius1975B} 2: 235a; \citealt{Hauer1952} 3: 899);

    Tungusic ← \ili{Mongolic}: \ili{Middle Mongol}: MNT \textit{teme’en}; ZY \textit{te[m]mē}; HY \textit{teme’en}; Muq. \textit{temēn}; Rasulid \textit{temēn}; \ili{Literary Mongolian} \textit{temegen}; Manchurian and Onon \ili{Khamnigan Mongol} \textit{temē(n)}; Dadal-sum \ili{Khamnigan Mongol} \textit{t`ɛmē}; \ili{Mongolian} Khamnigan \textit{temegēn}; \ili{Dagur} \textit{tǝmǝ}; \ili{Buryat} \textit{temē(n)} (\citealt{Khabtagaeva2017}: 133; see also \citealt{Doerfer1985}: 77--78; \citealt{Rozycki1994}: 206);

    \ili{Mongolic} ← \ili{Turkic}: cf. \ili{Old Turkic} \textit{täβäy} ‘camel’;
\z
\z
\z


\subsubsection{Mongolic loanwords borrowed at a different time}

Castrén’s \ili{Nercha} Ewenki material includes the \ili{Mongolic} “early stage” loanwords which are present in other Tungusic languages as well, while Manchurian \ili{Khamnigan Mongol} has \ili{Mongolic} loanwords borrowed more recently.

\ea
The final \ili{Mongolic} consonant *\textit{-l} is presented as \textit{-n} in all Ewenki dialects:


\ea  ‘saddle’: \ili{Nercha} Ewenki: \ili{Urulga}  \textbf{\textit{emegen}}, Borzya \textit{emēl};

    cf. \il{Evenki!Khamnigan}Khamnigan Ewenki \textit{emegēl} ← \ili{Solon} Ewenki \textit{ǝmǝgǝl};

    cf. \ili{Siberian} Ewenki \textit{emegen}: Yerbogochen\il{Yerbogachyon}, \ili{Upper Lena}, \ili{Barguzin}, \ili{Tungir}, \ili{Ayan} ‘saddle’; \ili{Zeya}, \ili{Aldan}, \ili{Uchur}, \ili{Urmi}, \ili{Ayan}, \ili{Sakhalin} ‘pack saddle’; \ili{Podkamennyj}, \ili{Ilimpeya} ‘men’s hunting saddle’;

    \textit{other Tungusic}: Lamut \textit{emgun}; Orok \textit{emē(n) {\textasciitilde} emegē(n)}; \ili{Manchu} \textit{eŋgemu}; \ili{Sibe} \textit{ǝmǝŋ}; Remaining lgs. \textit{n.a.}
    
    (\citealt{Castrén1856}: 73; \citealt{Janhunen1991}: 100; \citealt{Dorji1998}: 172; \citealt{Vasilevic1958}: 558b; \citealt{Cincius1975B} 2: 452b; \citealt{Hauer1952} 1: 252; \citealt{Zikmundová2013a}: 210);

    Tungusic ← \ili{Mongolic}: \ili{Middle Mongol}: MNT \textit{eme’el}; ZY, Muq., Rasulid \textit{emēl}; \ili{Literary Mongolian} \textit{emegel}; Manchurian \ili{Khamnigan Mongol} \textit{emēl}; Onon \ili{Khamnigan Mongol} \textit{emēl {\textasciitilde} ömȫl}; \ili{Dagur}, \ili{Buryat} \textit{emēl} (\citealt{Khabtagaeva2017}: 83; see also \citealt{Doerfer1985}: 21; \citealt{Rozycki1994}: 70);

\z 
\z
\ea
The secondary long vowels have not developed yet in \ili{Nercha} Ewenki and other Ewenki dialects from sequence \textit{VgV}, while in \il{Evenki!Khamnigan}Khamnigan Ewenki they have, as in Modern Mongol:
\ea  ‘young’: \ili{Nercha} Ewenki \textbf{\textit{ǰalaf}}; cf. \il{Evenki!Khamnigan}Khamnigan Ewenki: \ili{Urulyungui} \textbf{\textit{ǰalō}}, Borzya \textbf{\textit{ǰalau}}

    ← \ili{Solon} Ewenki \textit{ǰalu}; cf. \ili{Siberian} Ewenki: \ili{Aldan}, \ili{Barguzin}, \ili{Upper Lena}, \ili{Zeya}, \ili{Tungir}, \ili{Uchur} \textit{ǰalaw}; \textit{other Tungusic}: \textit{n.a}. (\citealt{Castrén1856}: 93; \citealt{Janhunen1991}: 30, 54; \citealt{Dorji1998}: 356b; \citealt{Vasilevic1958}: 147a; \citealt{Cincius1975B} 1: 245a);
    ← \ili{Mongolic}: \ili{Middle Mongol}: MNT \textit{\v{j}ala’ui}; Muq. \textit{\v{j}ala’ū {\textasciitilde} \v{j}alū}; Ist. \textit{\v{j}alau}; Rasulid \textit{\v{j}alawu}; \ili{Literary Mongolian} \textit{ǰalaγu}; Onon \ili{Khamnigan Mongol} \textit{dzalū}; Dadal-sum \ili{Khamnigan Mongol} \textit{džalalgan} ‘boy’; \ili{Dagur} \textit{\v{j}al\={ɔ}}; \ili{Buryat} \textit{zalū} (\citealt{Khabtagaeva2017}: 93; see also \citealt{Doerfer1985}: 127);

\z
\z
\ea
The original Tungusic words with initial and intervocalic *s in \ili{Nercha} Ewenki and Manchurian \il{Evenki!Khamnigan}Khamnigan Ewenki were preserved, while in \ili{Mongolic} loanwords this sound was pharyngealized in the \ili{Urulga} sub-dialect and preserved in the \ili{Mankovo} sub-dialect of \ili{Nercha} Ewenki (\citealt{Khabtagaeva2017}: 175). Thus, the pharyngealization must have happened in a later period. More likely these loanwords were borrowed from the ancestors of the \ili{Urulyungui} subdialect of \ili{Khamnigan Mongol}:
\ea ‘rope, loop, lasso’: \ili{Nercha} Ewenki (\ili{Mankovo}) \textbf{\textit{desün}}; (\ili{Urulga}) \textbf{\textit{dehün}};

    cf. \il{Evenki!Khamnigan}Khamnigan Ewenki, \ili{Solon} Ewenki, \ili{Siberian} Ewenki \textit{n.a.}; \textit{other Tungusic lgs. n.a.} \citep[89]{Castrén1856};

    ← \ili{Mongolic}: \ili{Middle Mongol}: ZY \textit{dēsü}; HY \textit{de’esün}; Muq. \textit{dēsün}; Rasulid \textit{dēsün}; \ili{Literary Mongolian} \textit{degesün}; Onon \ili{Khamnigan Mongol} \textit{dēsün}; \ili{Dagur} \textit{dǝs}; \ili{Buryat} \textit{dēhe(n)} \citep[80]{Khabtagaeva2017}; 

    \ex ‘fly’: \ili{Nercha} Ewenki (\ili{Mankovo}) \textbf{\textit{ilāsun}}; (\ili{Urulga}) \textbf{\textit{ilāhun}}; \il{Evenki!Khamnigan}Khamnigan Ewenki \textbf{\textit{ilāsun}};

    cf. \ili{Solon} Ewenki \textit{ilā}; \ili{Siberian} Ewenki \textit{n.a.}; \textit{other Tungusic lgs.} \textit{n.a}. (\citealt{Castrén1856}: 74; \citealt{Janhunen1991}: 105; \citealt{Dorji1998}: 324b; \citealt{Cincius1975B} 1: 306b);

    ← \ili{Mongolic}: \ili{Literary Mongolian} \textit{ilaγasun}; Manchurian \ili{Khamnigan Mongol} \textit{ilāhun}; Onon \ili{Khamnigan Mongol} \textit{ilā}; Dadal-sum \ili{Khamnigan Mongol} \textit{ilā}; \ili{Dagur} \textit{xilā} ‘horsefly’; \ili{Buryat} \textit{ilāha(n)} \citep[90]{Khabtagaeva2017};

    cf. Orochen\il{Oroqen} \textit{dilkān}, other Ewenki dial. \textit{dilkēn};

\z
\z
\ea
There is one \ili{Mongolic} loanword which is absent in other Tungusic languages, yet must have been borrowed at an early stage:
\ea ‘forty’: \ili{Nercha} Ewenki, \il{Evenki!Khamnigan}Khamnigan Ewenki \textbf{\textit{düčin}};

    in other Ewenki dialects \textit{dïgin ǰār}; \ili{Solon} Ewenki \textit{n.a.}\footnote{\ili{Solon} \textit{dǝhi} \citep[346]{Chaoke2014bB} ← \ili{Manchu} \textit{dehi}.}
    
    (\citealt{Castrén1856}: 90; \citealt{Janhunen1991}: 76);

    ← \ili{Mongolic}: \ili{Middle Mongol}: MNT \textit{döčin}; ZY \textit{düčin}; HY, Muq., Ist., Rasulid \textit{döčin}; \ili{Literary Mongolian} \textit{döči(n)}; Manchurian \ili{Khamnigan Mongol} \textit{düči(n)}; Onon \ili{Khamnigan Mongol} \textit{düči(n)}; Dadal-sum \ili{Khamnigan Mongol} \textit{dötš`i}; \ili{Dagur} \textit{duč}; \ili{Buryat} \textit{düše(n)} \citep[81]{Khabtagaeva2017}.
\z
\z


\section{Conclusion}
\begin{sloppypar}
As expected, the material examined shows a close connection between \ili{Nercha} and \il{Evenki!Khamnigan}Khamnigan Ewenki. Most of the vocabulary coincides, yet in several cases the other Ewenki dialects have differing forms (e.g. \textit{nekün} ‘younger brother’, \textit{nuŋnakī} ‘goose’, \textit{timī} ‘tomorrow’, \textit{nama} ‘warm’, etc.) or even lack some words (e.g. \textit{düčin} ‘forty’). These facts argue strongly for a common linguistic background between the two varieties. A separate group of vocabulary items includes the \ili{Nercha} Ewenki words which have some phonetic differences from \il{Evenki!Khamnigan}Khamnigan Ewenki, though these variants can possibly be explained by Castrén’s transcriptions from 1856. However, there are some \il{Evenki!Khamnigan}Khamnigan Ewenki words that changed phonetically recently, under the influence of \ili{Solon} Ewenki – another neighboring Tungusic language of Manchuria. These include words with the unstable consonant \textit{-n} (e.g. \textit{amin} ‘father’, \textit{tergen} ‘carriage’) and secondary long vowels in the first syllable of words (e.g. \textit{āli} ‘when’). These vowels also appear in the \ili{Mongolic} loanwords (e.g. \textit{ēm} ‘medicine, drug’, \textit{tōrga} ‘silk’). In addition, some \ili{Solon} Ewenki words were borrowed by \il{Evenki!Khamnigan}Khamnigan Ewenki after the migration from Russia (e.g. \textit{sigün} ‘sun’, \textit{irkekīn} ‘new’). More likely, the \ili{Russian} loanwords in \il{Evenki!Khamnigan}Khamnigan Ewenki were borrowed at an early time but were not noted by Castrén in his \ili{Nercha} Ewenki material. According to phonetic criteria (e.g. the preservations of the vowel *\textit{u} in the last syllable, \ili{Middle Mongol} \textit{VqV}, and sequences *\textit{ni} and *\textit{si}), most of the \ili{Mongolic} loanwords in \ili{Nercha} and \il{Evenki!Khamnigan}Khamnigan Ewenki dialects match each other. These facts also support a common linguistic heritage of the two varieties. Due to our short and preliminary fieldwork among the \il{Evenki!Khamnigan}Khamnigan Ewenki people in 2017, further fieldwork is necessary for a morphological and syntactic analysis, which I was not able to present here.
\end{sloppypar}

\section*{Citation of data}

\begin{tabular}{@{}l@{~}l@{}}
Dadal-sum \ili{Khamnigan Mongol} & = \citet{Uray1959}\\
HY & = \citet{Mostaert1977}\\
Leiden & = \citet{Poppe1927}\\
\ili{Literary Mongolian} & = \citet{Lessing1996}\\
Manchurian \ili{Khamnigan Mongol} & = \citet{Janhunen1990}\\
MNT & = \citet{Haenisch1939}\\
\ili{Mongolian} Khamnigan & = \citet{Rincen1968}\\
Muq. & = \citet{Poppe1938} \\
Onon \ili{Khamnigan Mongol} & = \citet{Damdinov2015}\\
ʼPhags-pa & = \citet{Tumurtogoo2010}\\
Rasulid & = \citet{Golden2000}\\
% SSTMJa & = \citet{Cincius1975B}\\
ZY & = \citet{Kara1990}\\
\end{tabular}


\section*{Acknowledgements}

This paper is supported by the Alexander von Humboldt Foundation. I would like to thank my colleagues and friends from Charles University Prague, Dr. \name{Veronika}{Zikmundová} and Dr. \name{Veronika}{Kapišovská}, with whom I conducted the fieldwork. Also, I would like to thank all our consultants and contacts in Hulunbuir, China. I am also grateful to Dr. \name{Andreas}{Hölzl} (University of Zurich) for his helpful and invaluable remarks.

{\sloppy\printbibliography[heading=subbibliography,notkeyword=this]}
\end{document}
