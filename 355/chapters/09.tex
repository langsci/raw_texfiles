\documentclass[output=paper,colorlinks,citecolor=brown]{langscibook}
\ChapterDOI{10.5281/zenodo.7053375}
\title{Historical language contact between Sibe and Khorchin} 
\author{Veronika Zikmundová\affiliation{Charles University}}

\abstract{The Sibe of Xinjiang have been recognized as speakers of a Manchu variety by linguists. However, for the Sibe speakers themselves, the situation is more complicated. For certain reasons, the Sibe often present themselves as a group whose historical origins are different from the Manchus. Several mentions occur in historical sources about Sibe being vassals to the Khorchin Mongols before “becoming Manchus”. This has been used among the arguments for the non-Manchu identity of the Sibe.

In recent years, academic discussion has focused on the ethnic identity of the Manchus, and, to a lesser extent, also on the position of the Sibe in relation to the Manchus. In this paper I try to select out features of possible Khorchin, i.e. eastern Mongolian, origin, in Sibe which may have come from direct language contact. I discuss several morphological features of Mongolic origin which seem not to be shared by other Manchu varieties, and one remarkable Sibe feature of Khorchin origin (the emphatic prefix \textit{me-}). In addition, I mention the existence of lexical evidence of direct contact which is found in more conservative layers of Sibe vocabulary. Another question concerns the significance of this evidence for imagining the Sibe history. The linguistic situation in central Manchuria during the period concerned (15th–16th centuries) suggests that if the shared features indeed come from this period, they may rather be remnants of an extinct linguistic environment characterized by intense Mongolic-Tungusic contacts than of bilateral contact between two distinct groups – Khorchins and Sibe.}

\begin{document}
\maketitle

\section{Overview}\label{section9.1}\largerpage

Central Manchuria has been the home of many \ili{Mongolic}- and \ili{Jurchenic}-speak\-ing\footnote{The term \textit{Jurchenic} was coined by Janhunen (\citeyear[154]{Janhunen1996}) as a term comprising both the documented \ili{Jurchen} varieties and other, undocumented southern Tungusic languages whose existence Janhunen thus suggests. It seems convenient to use this term to refer to the branch of Tungusic languages which includes the extinct \ili{Jurchen} varieties and their successor languages – \ili{written Manchu} and several spoken \ili{Manchu} varieties. These have been known under the names of \ili{Alchuka}, \ili{Bala}, \ili{Lalin}, \ili{Aihui}, \ili{Sanjiazi}, \ili{Yibuqi} and \ili{Sibe}. Another little documented language, the \ili{Manchu Kyakala}, has recently been suggested as belonging to this branch \citep{HölzlHölzl2019a}.} communities and the site of multiple and multi-layer contacts between these groups for several centuries.\footnote{\citet[96--110]{Janhunen1996} describes the setting of Manchuria during the Ming and Qing rule with several case studies of migrations and contact events, which show the ethnic and linguistic complexity in the area and enable us to estimate analogous, insufficiently documented migrations and language contact events.} During the period of the Yuan and Ming rule, namely between the 14th and 16th centuries, many demographic shifts happened which were probably followed by important changes in the linguistic situation, such as the growth of \ili{Mongolic} influence in the area. These shifts supposedly resulted in new, both massive and small-scale, \ili{Jurchenic}-\ili{Mongolic} language contacts (cf. \citealt[97]{Janhunen1996}). Most of these contact events are little, if at all, documented. However, in 20th century China, one of these little documented events received particular attention and different interpretations. This was the historical fact of the (probably) \ili{Jurchenic}-speaking \ili{Sibe} being vassals of the \ili{Khorchin} Mongols. The present article is concerned with this contact event, its contexts and interpretations.

Modern \textsc{Sibe} is a \ili{Jurchenic} diaspora language which has often been classified as an oral variety of \ili{Manchu}. It is related to the other oral \ili{Manchu} varieties which have been discovered in Manchuria during the 20th century. \ili{Sibe} is spoken by 10,000--20,000 individuals in several localities close to the north-western border of China, detached by some 4000 kilometers from their original homeland in Manchuria.\footnote{For descriptions of spoken \ili{Sibe} see, for example, \citet{Norman1974}, \citet{Jang2008}, \citet{Zikmundová2013b}, \citet{Kogura2018}.} \textsc{Khorchin}, an eastern (Manchurian) variety of \ili{Mongolian}, currently has about a million speakers who inhabit a large area of eastern Inner Mongolia, Jilin and Heilongjiang.

In the 16th and early 17th century (before the Qing administrative re-or\-ga\-ni\-za\-tion of Manchuria), most \ili{Jurchenic}-speaking communities were grouped into several \ili{Jurchen} tribal confederacies.\footnote{For an overview of the pre-Qing organization of the \ili{Jurchen} tribes see, for example, \citet[98--100]{Janhunen1996}.} Historical sources relate that in the same period, the \ili{Sibe} were subject to the \ili{Mongolic} \ili{Khorchin} tribe and only in the 1690s were united with the rest of \ili{Jurchenic} speakers (see \sectref{section9.2.1}). Linguistically, modern \ili{Sibe} and modern \ili{Khorchin} share certain features which may have originated at the time of their mutual contact during the Ming dynasty.

This article is an attempt to examine these similarities in their socio-linguistic and historical contexts and suggest an interpretation of their significance for \ili{Sibe} studies. Further, I take the narrative of the historical \ili{Sibe}-\ili{Khorchin} contact and the search for possible linguistic evidence about it as a starting point for an attempt to outline some important traits of the linguistic situation in Central Manchuria before the 18th century.

First, in \sectref{section9.2}, the historical context of the supposed \ili{Sibe}-\ili{Khorchin} language contact is summarized and the political and socio-linguistic background of modern \ili{Sibe} historioghraphy is mentioned. I suggest that the period of historically documented pre-Qing contacts between the \ili{Sibe} and the Khorchins has been assigned particular importance in the argumentation for ethnic origins distinct from those of the Manchus. In \sectref{section9.3}, the actual parallels in phonetics and morphology are listed. These are based, for the most part, on fieldwork data. Here I only mention features which \ili{Sibe} shares with \ili{Khorchin} and which are either not attested, or are marginal, in the other documented \ili{Manchu} varieties. \sectref{section9.4} gives examples of \ili{Mongolic} loanwords in \ili{Sibe} which are not documented in the other \ili{Manchu} varietes. Some of them are \ili{Mongolic} in general while others belong exclusively to the cultural sphere of the Manchurian Mongols. In the concluding part I discuss what these shared features can tell us about the linguistic situation in pre-Qing central Manchuria.

I suppose that the selected features may have resulted from a direct \ili{Mongolic} influence on \ili{Sibe} which was more intensive than the general \ili{Mongolic} influence to which other \ili{Manchu} varieties were exposed. However, concerning further interpretations of these shared features, they can be attributed both to pre-\ili{Manchu} contact with \ili{Khorchin} and to later contact with other \ili{Mongolic} languages – Daur\il{Dagur}, \ili{Jungarian Chakhar} and \ili{Öölöd}. Independent internal developments cannot be ruled out either. Most importantly, in the light of historical data, it seems more plausible to interpret the shared features as remnants of a generally more \ili{Mongolic}-influenced \ili{Jurchenic} milieu which was otherwise lost due to language standardization, than as a proof of the historical \ili{Sibe}-\ili{Khorchin} contact.

\section{Methodology}
In search for the \ili{Sibe}-\ili{Khorchin} analogies, mainly corpora of \ili{Sibe} and \ili{Khorchin} fieldwork data were used. The \ili{Sibe} part of these data, collected by myself in the Xinjiang \ili{Sibe} communities mainly with the purpose of grammar description, comes from the period between 1993 and 2009. The \ili{Khorchin} part\footnote{The \ili{Khorchin} data comprise approximately 10 hours of lengthy interviews on historical and cultural topics.} was collected between 2004--2015 both by the local consultant Bai Xiaomei and by myself. The \ili{Khorchin} data were not elicited with the purpose of grammar description and therefore do not cover the whole \ili{Khorchin} grammar which leaves some room for as yet undiscovered shared grammatical features. Additionally, if not stated otherwise, I use \ili{Khalkha} \ili{Mongolian} and \ili{Sanjiazi} Manchu\footnote{The village of \ili{Sanjiazi} (Fuyu county) is one of the last locations in Heilongjiang where a form of \ili{Manchu} is still spoken by several elderly individuals.} data from my own fieldwork collections.

Distinctions between \ili{Sibe} and \ili{written Manchu} have been described, above all, by Jang Taeho (\citeyear{Jang2008}). During my work on \ili{Sibe} grammar description I tried to systematically note features which not only distinguished \ili{Sibe} from \ili{written Manchu}, but which seemed likely to be of \ili{Mongolic} origin. I subsequently searched for these features in the materials of \ili{Khorchin} on one hand, and in other spoken \ili{Manchu} materials on the other. I selected those features which are shared with \ili{Khorchin} and, at the same time, either not attested or – compared to \ili{Sibe} – marginal in the oral Manchurian varieties of \ili{Manchu}.

In order to draw a plausible interpretation of the selected shared features I attempted to systematize the available information about the linguistic history of the area concerned and align the historical mentions of pre-Qing \ili{Sibe} and \ili{Khorchin} with more general patterns of developments in Ming Central Manchuria. Further, it seemed to be important to assess the value of the official \ili{Sibe} historiography and its accent on the non-\ili{Jurchen} origins of the \ili{Sibe} for the interpretation of the \ili{Sibe}-\ili{Khorchin} contact history. Fortunately, recently published works such as \citet{Zhuangsheng2019} and \citet{Sárközi2019} offer a much-needed insight into the motivation of the indigenous \ili{Sibe} historiography.

\section{The historical and socio-linguistic background of the Sibe-Khorchin language contact}\label{section9.2}

Below I give basic data about the two languages involved in the supposed language contact episode, including some historical facts that pertain to the general linguistic situation in the area and time concerned. I also note the socio-historical contexts of the official self-presentation of the modern \ili{Sibe} people as a group of non-\ili{Jurchen} origin.

\subsection{Sibe}\label{section9.2.1}

At present, two groups of people in China at two different locations are officially recognized as members of the \ili{Sibe} ethnic group. The larger of these groups inhabits certain areas in Northeastern China (Manchuria) and are speakers of \ili{Mandarin}. The smaller group of \ili{Sibe}\footnote{The ancestors of this group were moved from Manchuria to Xinjiang in 1764 as soldiers of the \ili{Manchu} army with the task of manning the frontier garrisons on the border with Russia. For detailed accounts of the history of the 	Xinjiang \ili{Sibe} see, for example, \citet{Sárközi2019} or \citet{Zhuangsheng2019}.}, some 30,000 individuals, live in the most faraway corner of China – the Ili valley on the border with Kazakhstan. These \ili{Sibe} are not only more-or-less fluent speakers of a \ili{Manchu} variety, but also preservers of a specific Manchurian culture. This paper is concerned with the latter – Xinjiang or Jungarian – \ili{Sibe}\footnote{The term \textit{Jungarian Sibe} is employed by \citet[49]{Janhunen1996}.} group.

Comparative data from other living or recently extinct \ili{Manchu} varieties (e.g. \citealt{WangQingfeng2005Z}; \citealt{ZhaoJie1989Z}; \citealt{MuYejun1985Z, MuYejun1986Za, MuYejun1986Zb, MuYejun1987Z, MuYejun1988Z}; \citealt{HölzlHölzl2019a}) allow \ili{Sibe} to be classified as one of the Bannermen \ili{Manchu}\footnote{Cf. e.g. \citet{ZhaoJie1989Z}. \ili{Chinese} authors use the term \textit{Qiren Manyu} ‘\ili{Bannermen Manchu}’ to distinguish the standard \ili{Manchu} language from the varieties used in communities of \ili{Manchu}/\ili{Jurchen} civilians whose language was not subject to so intensive standardization, such as \ili{Alchuka} or \ili{Bala}.} varieties together with \ili{Sanjiazi} Manchu, \ili{Aihui} Manchu, \ili{Yibuqi} Manchu and \ili{Lalin}/\il{Manchu!Jing}Jing Manchu. Historically, these varieties, in contrast to other modern \ili{Jurchenic} languages, seem to have been forms of a standard spoken language used in \ili{Manchu} military garrisons. Knowledge of \ili{written Manchu}, which was widespread in the Manchurian garrisons as well as in the Xinjiang \ili{Sibe} enclaves, is probably responsible for the relatively little diversity among all \ili{Bannermen Manchu} varieties. Most of the differences between \ili{Sibe} and written \ili{Manchu} (cf. \citealt{Jang2008}) are in fact shared by \ili{Sanjiazi}, \ili{Aihui} and \ili{Yibuqi} and may therefore be interpreted in terms of differences between the spoken language on one hand and the written form on the other, rather similar to the difference between written (Classical) Mongolian\il{Literary Mongolian} and the modern spoken forms of \ili{Mongolian}. Furthermore, similar to the situation in \ili{Mongolian}, it may be assumed that, besides reflecting an earlier shape of the spoken language, some of the features in written \ili{Manchu} may be orthographic conventions rather than of records of the actual pronunciation.\footnote{An example of this – the difference between the notation and the actual pronunciation of the \ili{Manchu} past tense forms – was analysed by Kubo Tomoyuki in his lecture (Charles University Oct 4 2019). It should also be noted that the \ili{Manchu} writing system, similar to the \ili{Mongolian} script, ignores most allophones of the spoken forms.}

This homogeneity of the \ili{Bannermen Manchu} varieties notwithstanding, several distinctions exist between \ili{Sibe} on the one hand and the other \ili{Manchu} varieties on the other. These distinctions comprise phonetic, morphosyntactic and lexical features. Some of these features are likely to have originated in contact with \ili{Mongolic} languages.

\subsubsection{The historical background of the Sibe}

The \ili{Sibe} are first found in Central Manchuria, in the areas of Qiqihar and historical Bedune (the modern Fuyu city). The first substantial evidence about them is a note about the inclusion of the \ili{Sibe} into the \ili{Manchu} military system in 1692, found in the Records of Girin (\citealt[51]{Zhuangsheng2019}; \citealt[8]{Sárközi2019}). In noting this event, the source gives the retrospective detail that \ili{Sibe} and Gūwalca\footnote{The Gūwalca (known as \textit{Khuulchin} in \ili{Mongolian} sources, cf. \citealt[232--233]{Ujeed2013}) are mentioned together with the \ili{Sibe} in the early Qing period. By the 19th century they have disappeared, possibly due to merger with the \ili{Sibe}. Their language is not documented at all but they are generally considered to be linguistically related to the \ili{Sibe} (\ia{Zhuangsheng}Zhuangsheng, p.c. August 2019).} had been \ili{Khorchin} vassals. The transfer of \ili{Sibe} and Gūwalca from the \ili{Khorchin} under direct \ili{Manchu} administration was mediated by the Second Neichi Toyin\footnote{For a detailed description of the activities of the Second Neichi Toyin (1671--1703), a successor and re-incarnation of the famous Buddhist missionary to the eastern Mongols, the First Neichi Toyin, see \citet{Ujeed2013}.}, in whose biography the description of the event is given (\citealt[232--233]{Ujeed2013}). This is the historical base of the narrative about the \ili{Sibe} vassalage to the \ili{Khorchin}. Except for these accounts, other brief mentions confirm the relationship of the \ili{Sibe} and Gūwalca to the \ili{Khorchin}  (\citealt[35]{Gorelova2002}) – namely the account of the battle of Gure (1593) when \ili{Sibe} and Gūwalca fought together with the Khorchins and the Hūlun Jurchens against Nurhaci, and a mention of the \ili{Sibe} and Gūwalca as \ili{Khorchin} vassals in the biography of the all-important Buddhist missionary to the \ili{Khorchin}, the First Neichi Toyin (between 1636 and 1653, cf. \citealt[36]{Heissig1980}).\largerpage

Especially the account of the Battle of Gure places the \ili{Sibe} into the context of the Hūlun Jurchens, about who \citet[65]{Crossley2006} writes: “The majority of Hūluns were \ili{Jurchen} in origin but by the late 1500s spoke a distinct dialect, with a much larger portion of \ili{Mongolian} loan-words, and among them were found a very high incidence of \ili{Mongolian} names, marriage into \ili{Mongolian}-speaking lineages (either \ili{Khorchin} or \ili{Kharachin}), and extensive acculturation with the \ili{Khorchin} or \ili{Kharachin} populations generally.” The \ili{Khorchin} and \ili{Kharachin} were, in their majority, descendants of the Ujiyed and Uriangkhan Mongols respectively (see below).

Consequently, the \ili{Sibe}, together with the Gūwalca, were probably involved in the intensive contact processes on the borders between the \ili{Mongolic}- and Tungusic-dominated parts of Manchuria (\citealt[98--99]{Janhunen1996}). The historical accounts of the event of incorporation of the \ili{Sibe} and the Gūwalca into the \ili{Manchu} banners state that these two groups were related to the Jurchens. These people, whatever their political status was, can thus probably be taken as representatives of \ili{Jurchenic} groups of the \ili{Mongolic}-influenced area. They were acculturated by Mongols who, in their turn, were linguistically and culturally Tungusic-influenced, and themselves were, in part, Mongolized Tungusic speakers (see below). Interestingly, \citet[65]{Crossley2006}\footnote{\citet[65]{Crossley2006} quotes the source \textit{Huangqing kaiguo fanglüe} 3.3a. written by Agui et al. For a brief description of the ethnic setting of Central Manchuria in late Ming based on contemporary sources see \citet[64--66]{Crossley2006}.} notes that “the Jurchens of Nurgaci’s time used the word \textit{Mongol} (\textit{monggo}) for the Hūluns”, which could have likewise influenced the traditional self-perception of the \ili{Sibe}\footnote{This tradition of viewing the \ili{Jurchenic} groups of central Manchuria as Mongols may also stand behind the appellation “Sibege Mongols” for a sinicized group of Manchurian \ili{Sibe} mentioned in \citet[225--227]{Lattimore1935}.}. In 1636--1638, the \ili{Sibe}, together with the Gūwalca, the Daur\il{Dagur} and possibly other originally Hūlun groups (cf. \citealt[69--70]{Crossley2006}), were incorporated into the newly created Mongol Eight Banners, to be transferred to the \ili{Manchu} Eight Banners in 1692.\largerpage

While the abovementioned historical sources confirm the fact that the \ili{Sibe} were \ili{Khorchin} subjects, they do not give details about this relationship and its duration. It is, however, clear that \ili{Sibe} lived in a \ili{Mongolic}-influenced environment for two or three centuries before becoming \ili{Manchu} bannermen. After becoming \ili{Manchu} army soldiers, they were divided into several groups and relocated into several military garrisons in Manchuria and Inner Mongolia (\citealt[36]{Gorelova2002}). There they were organized into the \ili{Sibe} banners. Initially, the Gūwalca had their own banners but later were probably merged into the \ili{Sibe} banners (\ia{Zhuangsheng}Zhuangsheng, p.c. August 2019), in this way disappearing from history. In different garrisons the \ili{Sibe} came into contact with different – Tungusic and \ili{Mongolic} – speakers.  As \ili{Manchu} bannermen they probably participated in the processes described by \citet[9--12]{Atwood2005}, and others. These processes involved, on one hand, intensive merging which resulted in the common millieu of \ili{Manchu} bannermen, also known as Qizu, literally ‘Banner ethnic group’, in the beginning of the 20th century, cf. \citet{Chengzhi2021}. High prestige of Standard \ili{Manchu} was one of several important traits of this milieu. On the other hand, identification with particular banners created the notions of \ili{Sibe}, \ili{Solon}, Daur\il{Dagur} and other groups based on administrative affiliation rather than origin and language. Thus “\ili{Sibe}” in the Qing period largely referred to people affiliated with the \ili{Sibe} banners which could include people of different linguistic background. The thus constituted \ili{Sibe} identity was distinct from that of the Manchus and rather close to that of the Daur\il{Dagur}, \ili{Solon} and Butha (cf. \citealt[85]{Elliott2001}). In 1764, 1000 individual soldiers were picked up from different \ili{Sibe} banners (\citealt[9]{Sárközi2019}) and with their families were transferred to their present location in Xinjiang. Closer study of these developments leads historians to question the continuity between the pre-Qing \ili{Sibe} and the modern \ili{Sibe} in Xinjiang (e.g. \citealt[257--268]{Chengzhi2012}).

During the Qing period \ili{Sibe} came into close contact with other \ili{Mongolic} groups, such as the Daur\il{Dagur}, the Chakhar or the \ili{Öölöd}. Nevertheless, Standard \ili{Manchu} became their first language. Throughout the Qing rule and until modern times, \ili{Sibe} have been known for their solid \ili{Manchu} skills (\citealt[51]{Zhuangsheng2019}).

\subsubsection{The socio-linguistic background of the narrative about the non-Jurchenic origins of the Sibe and of the Sibe-Khorchin contacts}\largerpage

In the beginning of the 20th century, the fact that the \ili{Sibe} people in the vicinity of \textit{Ghulja} (\ili{Mongolian} \textit{Ili hot}, \ili{Chinese} \textit{Yining shi}) spoke \ili{Manchu} had been widely recognized by the speakers themselves (e.g. \citealt{Donjina1989}; \citealt[10--12]{Porter2018}), as well as by foreign travelers and researchers (e.g. \citealt{Kałużyński1987}). Historical sources confirm that \ili{Sibe} spoke \ili{Manchu} as at least one of their languages during the whole Qing era (\citealt[51]{Zhuangsheng2019}). However, in 1990, when I visited the Xinjiang \ili{Sibe} community for the first time, any relationship to Manchus was generally denied in the official discourse among \ili{Sibe} intellectuals. The language of the \ili{Sibe} was called \textit{Sibe}. Moreover, several of my \ili{Sibe} consultants were suggesting that \ili{Sibe} originally spoke a \ili{Mongolic} or \ili{Mongolic}-related language. The remarkable difference between the \ili{written Manchu} language (known by many in the older generation of \ili{Sibe}) and spoken \ili{Sibe}\footnote{This difference involves not only features which seem to reflect diachronic processes such as vowel reduction or consonant weakening, but also features which call for other interpretations such as dialectal variation (namely in lexicon and morphology). In the 1990s the \ili{Sibe} were generally not aware that many of these distinctions were shared by the oral varieties of Manchuria.} was mentioned in support of this idea. \ili{Sibe} was presented as a language on its own, distinct from \ili{Manchu}. Publications influential in \ili{Sibe} society described \ili{Sibe} culture without the \ili{Manchu} context and studies of \ili{Sibe} history argued for an ethnic origin distinct from that of the Manchus.\footnote{The basic comprehensive description of \ili{Sibe} folk culture is \textit{Xibozu minsu – \ili{Sibe} uksurai an tacin} (\citealt{He1989}), the main description of \ili{Sibe} ethnic history was \textit{Xibozu jianshi/\ili{Sibe} uksurai šolokon suduri} \citep{Wu1985}.}

As \citet[58--70]{Zhuangsheng2019} has shown, this narrative came into being at the beginning of the 20th century and became essential in the context of the creation of the 55 ethnic minorities during the 1970s. Evidence for a distinct origin and a history as an ethnic group of its own was required in order to be officially recognized as an ethnic minority and enjoy the advantages associated with this status. Another reason why the \ili{Sibe} strongly denied common origins with the Manchus was the persecution of ethnic Manchus which started in Republican China and continued into the PRC period. \citet[58--71]{Zhuangsheng2019} describes how the \ili{Sibe} intellectuals worked on collecting historical evidence for writing a \textit{\ili{Sibe} history}. He concludes (\citeyear[71--72]{Zhuangsheng2019}) that \ili{Sibe} as a political or ethnic entity indeed occur in historical sources since early 17th century. However, the whole narrative about their relationship to the presumably \ili{Mongolic}-related \ili{Xianbei} and their early history since the 3rd century\footnote{The official \ili{Sibe} history uses several unclear mentions found in oral tradition to argue that the ethnonym \textit{Sibe} is related to the name of the \ili{Xianbei}, a presumably nomadic group from western Manchuria which ruled over the \ili{Mongolian} grasslands in the 2nd century. The \ili{Xianbei} language has been most often interpreted as \ili{Mongolic} (e.g. \citealt[281]{Janhunen2010}). This hypothetical \ili{Xianbei} connection of the \ili{Sibe} has been used in support of the argumentation for a non-\ili{Jurchen} origin of the \ili{Sibe}.} was made up without any historical basis, and with very little background in oral tradition. This narrative has become part of the modern \ili{Sibe} identity.

\subsubsection{A story of a “different original language”: The case of the \textit{jivš} language}\largerpage[2]

The story of the extinct \textit{jivš} language is an example of a detail from \ili{Sibe} oral tradition that became an important part of the \ili{Sibe} “ethnic narrative” and (linguistic) self-consciousness as a non-\ili{Manchu} group.{\interfootnotelinepenalty=10000\footnote{As for the possible identity of this enigmatic language, the Inner \ili{Mongolian} linguist Otgonchecheg suggested a connection to the Chipchin (\ili{Bargu}: \textit{šivšin}), an exonym used for the Old \ili{Bargu} (a \ili{Buryat}-related \ili{Mongolic} group) during the Qing. Otgonchecheg, who did fieldwork in Chabchal in order to collect data of the \textit{jivš} language, did not publish her research due to the lack of evidence. From a historical point of view it is plausible that a group of Chipchin \ili{Bargu} bannermen was incorporated into the \ili{Sibe} banners. However, the \ili{Sibe} scholar \citet{Su1984}, based on his fieldwork in the Fifth banner, maintains that the term \textit{jivš gisun} referred merely to a layer of \ili{Khorchin} loanwords which was thicker in some groups of \ili{Sibe} than in others. Su Deshan, following a “folk” explanation, interprets the word \textit{jivš} as ‘double, additional’ and the term \textit{jivš gisun} as ‘additional words, synonyms’. Small pieces of evidence from more recent fieldwork (\ia{Guo, Junxiao}Guo Junxiao, \ia{Chengzhi}Chengzhi, p.c. September 2020) suggest that the notion of \textit{jivš gisun} is still remembered in the Fifth banner, currently pointing to a mixture of \ili{Mongolian} loanwords and Literary \ili{Manchu} expressions which are marginal, though not entirely unknown, among the rest of the Chabchal speakers. Guo Junxiao, a \ili{Sibe} speaker (p.c. 2020) describes \textit{jivš gisun} as a group of “unfamiliar, \ili{Mongolian}-sounding words” while the unpublished data collected by Chengzhi (2020) include lexical items such as \textit{saxaxuri} ‘whitish’ (< \ili{written Manchu} \textit{sahahūri}) and \textit{xurdun} ‘quick’ (< \ili{written Mongol} \textit{qurdun}, \ili{Khorchin} \textit{xurden}, vs. \ili{Sibe} \textit{xudun}, \ili{written Manchu} \textit{hūdun}).}}

The inhabitants of the Fifth banner, one of the eight administrative units of Chabchal, speak \ili{Sibe} with a (for a native speaker) remarkably different pronunciation. The difference supposedly consists of lesser reduction and generally greater closeness to \ili{written Manchu}. \ili{Sibe} speakers from other banners often quote the example of the \ili{written Manchu} word \textit{aliyaha} ‘waited’  which is pronounced as \textit{aliaxa} in the Fifth banner but \textit{alixe} in the rest of Chabchal. Oral tradition explains this by saying that \ili{Sibe} of the Fifth banner were originally speakers of a different language and therefore were taught Standard \ili{Manchu} as a new language. This caused their pronunciation in the spoken language being closer to the literary language. Oral tradition calls their original language \textit{jivš gisun} ‘the \textit{jivš} language’ (written form \textit{jibsi gisun}), and holds that it had disappeared by the end of the 19th century. Different ‘folk’ hypotheses exist about this language, such as that \textit{jivš gisun} was a “Mongol language, perhaps something like \ili{Khorchin} or Daur\il{Dagur}” or that it was a “secret language which consisted of repeating every word twice.” (fieldwork data February 1995). Moreover, now and then a statement is heard or read that \textit{jivš gisun} was the original language of the \ili{Sibe}.\il{Jibsi language}
 
Whatever the historical roots of the \textit{jivš} case, it has become part of the popular narrative of \ili{Sibe} indentity. \ili{Even} today the statement about \textit{jivš} as the original language of the \ili{Sibe}, attributed to a source called “minjian” (folk), is repeated on \ili{Sibe} social media,\footnote{\textit{musei te gisuremaha gisun oci manju gisun inu, musei da gisun oci jibsi gisun, manju gisun waka} `the language we speak now is \ili{Manchu}, but our original language is the Jibsi language, not \ili{Manchu}'. (E.g. \url{http://blog.sina.com.cn/s/blog_4aa943a1010008yv.html}. Last access 28.10.2020.)} which testifies to its lasting popularity.\il{Jibsi language}

\subsection{Khorchin}

\ili{Khorchin} Mongol, spoken by close to a million of speakers and thus being the largest and most influential \ili{Mongolian} dialect after \ili{Khalkha}, is less researched than \ili{Sibe}. The \ili{Khorchin} speech community differs from most other \ili{Mongolian} speech communities in that it has a long tradition of sedentary or semi-sedentary life-style.  Two important descriptions of \ili{Khorchin} are Bayančogtu (2002) and Caidengduoerji (2014), the latter being an unpublished dissertation.\footnote{Other studies and materials of \ili{Khorchin} include, for example, \citet{Brosig2014a, Brosig2014b} and \citet{Yamakoshi2015}.}

At present, \ili{Khorchin} is spoken over a large territory in Inner Mongolia and the neighboring provinces of Jilin and Heilongjiang. The locations with the greatest concentration of speakers are the administrative unit of Tongliao City and the Hinggan League in Inner Mongolia. The varieties spoken in these two areas slightly differ from each other. \ili{Khorchin} is close to two other large eastern \ili{Mongolian} varieties – \ili{Kharachin} and \ili{Baarin} – and the three, including a number of their sub-varieties, share some important differences from the rest of \ili{Mongolian}. The Tongliao variety, in particular, is hardly intelligible to speakers of most other modern \ili{Mongolian} languages.

However, the available descriptions of \ili{Khorchin} present a picture of a rather regular variety of modern \ili{Mongolian} and do not give sufficient explanation for the mutual unintelligibility with standard varieties such as \ili{Khalkha}.

In my observation, two main factors may be responsible for the surface difference of Tongliao \ili{Khorchin} from other modern \ili{Mongolian} varieties. First, \ili{Khorchin} retains, with certain exceptions such as the loss of the vowel \textit{ö}, the general phonological structure that goes back to Proto-\ili{Mongolic} (e.g. \citealt[4]{Janhunen2003a}). However, extensive processes on the phonetic level such as consonant weakening, vowel shifts and vowel reduction fundamentally change its shape in speech. 
Second, \ili{Khorchin} in most rural areas is profoundly influenced by \ili{Chinese} with which it has been in close contact for several centuries. \ili{Chinese} influence is mostly manifested in syntax  (e.g. paratactic constructions instead of chains of clauses connected by non-finite verbal forms, which are typical for most other modern \ili{Mongolian} languages) and vocabulary. Depending on the topic and circumstances, the speech of a \ili{Khorchin} speaker may consist of about fifty percent of words of \ili{Chinese} origin. These features are not readily seen in the descriptions but are important for shaping the performance of \ili{Khorchin} speakers which then radically differs from the speech of, for example, a \ili{Khalkha} speaker.

\subsubsection{Historical background of the Khorchin}

The \ili{Khorchin} population seems to have initially been composed of two main elements. The first, the most important according to \ili{Khorchin} historians, and the one which gave the group its name and proclaimed identity, is the Mongol noble lineage descended from Khasar and their subjects. In the 13th century Khasar, the younger brother of Genghis Khan, was granted the lands around Lake Hulun and the Ergune river as an appanage, hence approximately the area of the modern administrative unit of Hulunbuir.\footnote{It is often difficult to establish the precise location of the lands of particular nomadic peoples in this period. In the case of the Khasar lineage, however, archaeologists have interpreted at least two important sites in the Ergune valley as towns built by Khasar’s descendants (e.g. \citealt[227--227]{Kradin2018}).} During the Ming dynasty, probably in connection to the period of internal conflicts in Mongolia (\citealt[29]{Caidengduoerji2014}), the main part of the Khorchins crossed the Khingan mountains to the east and settled in the Nonni valley where they became the overlords of the local Mongol population. The local Mongols, the second important – and probably more numerous – element in the composition of Khorchins, were the Ujiyed of the Fuyu Guard\footnote{The Fuyu guard, situated close to the present Qiqihar in the Nonni valley, was one of the three “loose rein” guards (the Fuyu guard of the Ujiyed people, the Taining guard of the Ongniuts and the Döyin guard of the Uriangkhan) established in Manchuria by the Ming. The “Guards” were groups of former subjects of the Yuan empire who were identified as Mongols and after the fall of the Yuan rule became tributaries of the new Ming dynasty \citep[536]{Atwood2004}.} \citep[306]{Atwood2004}. The Fuyu guard was one of the Three Guards – administrative units in Manchuria loosely controlled by the \ili{Chinese} (Ming) court. The population of the Three Guards was referred to as either Mongol or Uriangkhan, but comprised, besides Mongols, groups of Tungusic origin.\footnote{\citet[64]{Crossley2006} refers to the Ming authors Xiao Daheng and Ye Xianggao for a definition of “Mongols” in the Ming era, concluding that: “[...] some \ili{Mongolian}-speaking communities were not nomadic but agricultural; many groups who migrated with “Mongols” were speakers of \ili{Turkic} or Tungusic languages; many living among the Mongols were Han or the descendants of Han, who had been taken by the hundreds of thousands by eastern Mongol raiders in northern China.”} Therefore, in imagining the linguistic situation during the Ming, it seems important that the population of the Three guards, which later\footnote{For the detailed descriptions of the migrations of the Three Guards and their mixing with other Mongols see \citet[304, 410]{Atwood2004}.} “became the ancestors of many eastern Inner \ili{Mongolian} peoples” \citep[35]{Atwood2004}, was probably largely homogenous in terms of language and culture\footnote{The Three Guard Mongols were mostly sedentary and practiced agriculture \citep[535]{Atwood2004}.} which contained elements of Tungusic origin (\citealt[82]{Crossley2006}). In addition to this picture, the Three Guards were geographically close to the former Khitan territories, and their settlement in the area probably goes back to times when Khitans still existed as a distinct entity. Therefore a certain \ili{Khitan} influence on \ili{Khorchin} cannot be excluded.

Consequently, the remarkable features shared by the eastern \ili{Mongolian} dialects – \ili{Khorchin}, \ili{Kharachin} and \ili{Baarin} – may in fact have originated in the language of the Three Guard Mongols who have been continually exposed to local Manchurian influences since as early as the Yuan period.

Since the 15th century the Khorchins often intermarried with Jurchens (\citealt[65]{Crossley2006}). Since their arrival they started migrating from the Nonni valley southwards, into their present territory in the Liao valley. According to a contemporary account of a \ili{Korean} observer, they were “dressed in furs, with their felt yurts on wagons, moving their herds toward appropriate pastures. Many, he noted, were also agricultural and would sow fields in the spring to which they expected to return in the fall to reap a meager crop of wheat or millet.” (\citealt[66]{Crossley2006}). During the Qing period the Khorchins took over the Liao valley and thanks to their alliance with the Manchus politically dominated the area. At the same time groups of outsiders settled on this territory and were integrated and assimilated by the Khorchins (\citealt[37]{Caidengduoerji2014}). These immigrants were both large groups of Manchus and \ili{Chinese} and smaller groups or individuals of other ethnic origin such as \ili{Sibe}, Ewenki, or Koreans. In the beginning of the 20th century the \ili{Khorchin} area became one of the main targets of the Qing New Policies, which involved an unrestricted immigration of Han \ili{Chinese} and further sedentarization of the local Mongols. \ili{Even} during the 20th century, however, many immigrants kept adopting the \ili{Khorchin} language and culture. 

\section{Evidence of Sibe-Khorchin contacts}\label{section9.3}

In this section I list some shared features of modern \ili{Sibe} and modern \ili{Khorchin}, which may have resulted from mutual contacts between the ancestors of the two modern groups. These features, in my opinion, indeed point in the direction of direct contact of some kind. Historically and linguistically, these features remain open to different interpretations. When taking into consideration the available evidence about “ethnic” and “linguistic” mobility in Manchuria, especially within the Eight Banners,\footnote{In Qing-time Manchuria large-scale migrations and resettlements are documented, such as the abovementioned resettlement of \ili{Sibe}, \ili{Khorchin} migrations, or the massive Daur\il{Dagur} and \ili{Solon} migration into the Qiqihar area in the 17th century. In addition, evidence of countless shifts of small groups and individuals among the Qing garrisons is scattered across historical sources. Another factor important for linguistic developments are frequent intermarriages among members of different banners which were supported by the strict rules of exogamy in Tungusic-speaking groups. Among these, intermarriages between \ili{Sibe} and \ili{Manchu} bannermen seem to have been common (\ia{He, Rongwei}He Rongwei, p.c. June 2020). Intermarriages between \ili{Khorchin} and \ili{Manchu} speakers are generally known to have been frequent (\ia{Shuangshan}Shuangshan, p.c. August 2015). If we take the longest-surviving “banner society” – that of Hulun Buir – as a model for the linguistic situation in the Manchurian Banner communities, we may assume that not only many bilingual couples lived in the Banners but most of the bannermen were, to a certain degree, familiar with other languages. The supposed constant language contact between the \ili{Sibe} and \ili{Manchu} bannermen and the Khorchins rules out the possibility of independent developments of these languages and any clear-cut evidence for the earlier direct contacts between the \ili{Sibe} and the \ili{Khorchin}.

It also needs to be taken into account that the available data of spoken Manchurian \ili{Manchu} represent tiny pieces of a once broad continuum of local varieties, and that much of the data available were collected from semi-speakers and rememberers, and thus cannot supply a complete picture of Manchurian \ili{Manchu}.} it is rather clear that it is impossible to entirely separate the linguistic developments in \ili{Sibe} from the other \ili{Manchu} varieties. However, the features listed below are central and massive in \ili{Sibe} while, if attested, marginal in Manchurian \ili{Manchu}.

\subsection{Manchu influence on Khorchin?}

For obvious reasons – namely the absence of any \ili{Sibe} data before the 20th century – any specifically \ili{Sibe} influence on \ili{Khorchin} cannot be determined. In the context of the historical developments described above, strong influence of Manchurian Tungusic varieties might be expected. Quite surprisingly, however, little influence is seen on the lexical level. While \ili{Chinese} loanwords form a significant part of the \ili{Khorchin} vocabulary, \ili{Manchu} loanwords do not seem to excess several tens. Words used in everyday life such as \textit{lah} for the brick bed (Chin. \textit{kang}) from \ili{Manchu} \textit{nahan} or kinship terms such as \textit{eme} for mother (\ili{Manchu} \textit{eme}) have been noted by native linguists (\citealt[25]{Bayančoγtu2002}). Some \ili{Manchu} loanwords are connected to shamanic practices, such as \textit{samaan} ‘shaman’ from \ili{Manchu} \textit{saman}, \textit{sarg} ‘home altar’ from \ili{Manchu} \textit{sarha} or the verb \textit{magsi-} ‘to perform shamanic dance’ from \ili{Manchu} \textit{maksi-} ‘to dance.’ On the level of morphology and morphosyntax, the general typological similarity of \ili{Manchu} and \ili{Mongolian} makes it difficult to single out instances of mutual influence. 

The small number of \ili{Jurchenic} loanwords in general may, at least partly, be attributed to the standardization forces during the Qing dynasty which affected \ili{Mongolian} (proper)\footnote{In contrast to \ili{Mongolian} proper, the \ili{Mongolic} Daur\il{Dagur} langauge was not affected by standardization, instead borrowing many \ili{Manchu} words.} speakers not less than \ili{Manchu} speakers. In spite of the fact that the \ili{Mongolian} script was invented before the Yuan times, it became widely used only since the 16th century with the spread of Buddhism, accompanied by translations of literary works into Classical \ili{Mongolian}. At the same time, original compositions of didactic and other character were written and read in \ili{Mongolian}-speaking societies. The influence of Classical \ili{Mongolian} could have brought the vocabulary of the (politically) Mongol groups of Manchuria closer to other \ili{Mongolian} varieties (\citealt[83]{Crossley2006}).

In terms of contact features, research into phonetic peculiarities of \ili{Khorchin} and their relationship to the language environment of Manchuria may prove more rewarding. It seems worthwhile to analyze \ili{Khorchin} phonetic and phonological differences from other \ili{Mongolian} varieties in the context of other eastern \ili{Mongolic} idioms (\ili{Baarin}, \ili{Buryat}, Daur\il{Dagur}), in the context of \ili{Manchu} varieties, Manchurian \ili{Mandarin} and possibly even the of language of the \ili{Korean} minority of China. 

Below I just note two features which are similarly typical for \ili{Sibe} among \ili{Manchu} varieties as for \ili{Khorchin} among Mongol dialects and may therefore be added among the candidates for results of direct \ili{Sibe}-\ili{Khorchin} language contact.

\subsection{Shared phonetic developments in Khorchin and Sibe}

Generally speaking, \ili{Sibe} and \ili{Khorchin} are phonetically strikingly similar, which seems to be caused for the most part by the Manchurian influence on \ili{Khorchin}. For example, \ili{Khorchin} is perhaps the only \ili{Mongolian} variety where the intervocalic cluster \textit{ŋg} is pronounced as syllable-initial [ŋ], as in [moŋol] ‘Mongol’. Still, however, two of the shared features may be interpreted as results of phonetic processes that Manchurian \ili{Manchu} has avoided.

\subsubsection{Change of closing diphthongs into opening diphthongs}\label{section9.3.3}

In \ili{Sibe}, the equivalent of the \ili{written Manchu} diphthong \textit{ai} is often pronounced as \textit{iä}, e.g. written \ili{Manchu} \textit{bayimbi}\footnote{Unlike the pronunciation in spoken varieties, academic pronunciation of \ili{written Manchu} unpacks the diphthong.} [pajmbi] vs. \ili{Sibe} \textit{biäm} [pjɛm] ‘to look for’, etc. This is valid for approximately half of the reflections of the written \ili{Manchu} \textit{ai}. The rest either remains as \textit{äi}/\textit{ai} or is monophthongized. Some instances of retention of the closing diphthong are in the word-initial position (e.g. written \ili{Manchu} \textit{ai}, \ili{Sibe} \textit{ai} ‘what’), others come after uvulars (e.g. \ili{written Manchu} \textit{kaicambi}, \ili{Sibe} \textit{qaicem}/\textit{qacim} ‘to shout’), or apparently belong to a more literary style (e.g. written \ili{Manchu} \textit{saikan}, \ili{Sibe} \textit{saiken} ‘beautiful’). In other cases such as the \ili{written Manchu} \textit{baita}, \ili{Sibe} \textit{bäit} there is no immediately apparent reason. The “reversal” also took place in a few cases of the closing diphthong oi (e.g. written \ili{Manchu} \textit{boihon}, \ili{Sibe} \textit{bioxun} ‘dust’). These changes fit into the context of the overall phonetic tendencies in \ili{Sibe} (vowel raising and fronting, e.g. \ili{written Manchu} \textit{omimbi}, \ili{Sibe} \textit{eimim}/\textit{iemim} ‘to drink’).

In contrast to \ili{Sibe}, in the spoken Manchurian varieties of \ili{Sanjiazi}, \ili{Aihui} and \ili{Yibuqi} monophthongization of the written \ili{Manchu} diphthongs occurs (e.g. \ili{written Manchu} \textit{sain}, \ili{Sanjiazi} \textit{sän} ‘good’), but there are no cases of “reversal” of the diphthongs.

Unlike Manchurian \ili{Manchu} but quite similarly to \ili{Sibe}, \ili{Khorchin} has a strong tendency towards vowel fronting and raising (\citealt[60--61]{Janhunen2012b}). Closing diphthongs of \ili{written Mongol} (which are either retained or monophthongized in the central \ili{Mongolian} varieties such as \ili{Khalkha}) are, at least in some \ili{Khorchin} varieties, almost regularly reversed, e.g. written \ili{Khalkha} \textit{naim}, \ili{Khorchin} \textit{nie:m} ‘eight’ or written \ili{Khalkha} \textit{meiren}, \ili{Khorchin} \textit{mie:rin} (title of an official). The reversal may involve change of vowel quality such as written \ili{Khalkha} \textit{xoit}, \ili{Khorchin} \textit{xie:t} ‘north’.

\citet[45]{Janhunen2012b} notes that the tendency towards vowel fronting is seen in \ili{Mongolian} in general but this process has been most complete in the eastern dialects including \ili{Khorchin}. Similarly, reversal of diphthongs occasionally happens in other \ili{Mongolian} varieties but has become regular in \ili{Khorchin}. The described feature of \ili{Sibe} may therefore be interpreted as a diachronic change that happened during the period of influence of the eastern \ili{Mongolian} phonetic environment but was halted when the \ili{Sibe} left this particular environment.

\subsubsection{Dissimilation of the cluster \textit{čx}}

There is another phonetic development that occurs in \ili{Sibe} and \ili{Khorchin} but is found neither in other \ili{Manchu} varieties, nor in any other \ili{Mongolian} variety. In spoken \ili{Sibe} the consonant clusters \textit{čk} and \textit{čx}, which result from vowel elision, often change into the sequence \textit{šk}, e.g. \ili{written Manchu} \textit{tacikū}, \ili{Sibe} \textit{tačqu}/\textit{tašqu} ‘school’ or written \ili{Manchu} \textit{tacihabi}, \ili{Sibe} \textit{tačxei}/\textit{tašqei} ‘studied’. The dissimilated forms are used in quick and less careful speech, while the careful pronunciation retains the original consonants. In \ili{Khorchin}, the cluster \textit{čx} in the \ili{Mongolian} deverbal suffix \textit{-čix-}/\textit{-čx-} (quick or intensive action) in quick speech is sometimes dissimilated in a similar way. e.g. \textit{yavšgen}/\textit{yavčxen} cf. written \ili{Khalkha} \textit{yavčixna} ‘will leave’. While this may be just a parallel development, it certainly contributes to the similarity of the two languages.

\subsection{Potential Khorchin influence on Sibe grammar}

In the next part I list those features of \ili{Sibe} grammar which have analogies in \ili{Khorchin} and are not shared by, or are marginal in, the other oral \ili{Manchu} varieties. 

\subsubsection{The emphatic prefix \textit{mV-} (used with deictics)}

\subsubsubsection{The prefix \textit{mV-} in Sibe}

\ili{Sibe} has the element \textit{me-}/\textit{mu-} which is added to the beginning of some deictic expressions. Generally it adds emphasis to the deictics and is possibly best translated as ‘just, exactly’, sometimes ‘the very’. Its use is often analogous to the \ili{Chinese} particle \textit{jiù} ‘just, exactly’, sometimes also ‘the same’.

The prefix  is at least partly productive. Below I list forms encountered in my fieldwork material with examples:

\begin{itemize}
\item  \textit{mere} ‘exactly this’, from \textit{ere} ‘this’ (\ref{example9.1}).

\ea
    \label{example9.1}\il{Sibe}
    \gll mere 		jilgan mim-be 		eme 	diower 	amxe-we-xa-qv.\\
    just.this 	sound 1\textsc{sg-acc} 	one 	night   sleep-\textsc{caus-ptcp.pfv-neg}\\
    \glt `It was exactly this thing which did not let me sleep the whole night.'\\
    \z

\item  \textit{metere} ‘exactly that’ from \textit{tere} ‘that’ (\ref{example9.2}).

\ea
    \label{example9.2}\il{Sibe}
    \gll metere 		baite-we 	giser-maie.\\
    just.that 	matter-\textsc{acc} 	speak-\textsc{prog}\\
    \glt `This is exactly what I am speaking about; I am speaking about the same thing.'\\
    \z

\item  \textit{merange} ‘exactly like this’ from \textit{erange} ‘like this’ (\ref{example9.3}).

\ea
    \label{example9.3}\il{Sibe}
    \gll min-i 		uwe=da 	merange.\\
    1\textsc{sg-gen} 	fate=\textsc{foc} 		just.like.this\\
    \glt `This is exactly what my fate is (I cannot change it).'\\
    \z

\item  \textit{meterange} ‘exactly like that’ from \textit{terange} ‘like that’ (\ref{example9.4}).

\ea
    \label{example9.4}\il{Sibe}
    \gll meterange=da 		are!\\
    Just.like.that=\textsc{foc} 		write.\textsc{imp}\\
    \glt `Write it exactly in that way!/ Just write it in that way!'\\
    \z

\item  \textit{meske} ‘just this much’ from \textit{eske} ‘this much’ (\ref{example9.5}).

\ea
    \label{example9.5}\il{Sibe}
    \gll bilxa=ni 		meske 		ma.\\
    neck=3\textsc{sg.poss}		just.this.much	thick\\
    \glt `His neck is just this thick. (This form is usually used when demonstrating the degree of something with a gesture.)'\\
    \z

\item  A lexicalized expression formed in the same way is \textit{mujaqen} ‘just now’ from \textit{jaqen} ‘a while ago’.
\end{itemize}

The form \textit{mere} ‘exactly this’ is further used as means of emphasis with different types of expressions, both with deictics (\ref{example9.6}) and with other words (\ref{example9.7}), (\ref{example9.8}).  In this case it rather adds emphasis to the whole sentence than to its determinandum.

\ea
    \label{example9.6}\il{Sibe}
    \gll mere 		ewade=da 	yinde!\\
    \textsc{emph} 	here=\textsc{foc} 	stay.overnight.\textsc{imp}\\
    \glt `Just stay here overnight! (emphasis for the sake of expressing hospitality).'\\
    \z

\ea
    \label{example9.7}\il{Sibe}
    \gll mere 	xancide 	min-i 		bo-de	 		ji-xei.\\
    \textsc{emph}	recently	1\textsc{sg-gen}	house-\textsc{dat.loc} 	come-\textsc{pst}\\
    \glt `He came to my place during the very last couple of days.'\\
    \z

\ea
    \label{example9.8}\il{Sibe}
    \gll mere 	feksi-m.\\
    \textsc{emph} run-\textsc{npst}\\
    \glt `He took a flight/ immediately started running.' (emphasis in storytelling)\\
    \z

\noindent This feature is very likely borrowed from \ili{Khorchin}, where the element \textit{m(V)-} has an analogous function.

\subsubsubsection{The prefix \textit{mV-} in Khorchin}

According to \citet[148--151]{Bayančoγtu2002}, in \ili{Khorchin} this prefix is fully productive with demostratives. In his description the author gives a list of more than 120 possible forms. Below I give examples from my fieldwork material:

\begin{itemize}
\item \textit{men}/\textit{mun} ‘exactly this’ from en ‘this’ (\ref{example9.9}).

\ea
    \label{example9.9}\il{Khorchin}
    \gll Tongliao-nii 	laajii-gii 	men 		dotor 	avšir-č.baina.\\
    Tongliao-\textsc{gen}	 waste-\textsc{acc}		just.this	inside	 bring-\textsc{prs.prog}\\
    \glt `It is (exactly) inside this (fence) they are bringing the waste from Tongliao.'\\
    \z

\item \textit{meter} ‘exactly that’ from \textit{ter} ‘that’ (\ref{example9.10}). 

\ea
    \label{example9.10}\il{Khorchin}
    \gll meter 		modon.eel 	šii.\\
    just.that 	\textsc{pn} 		\textsc{emph}\\
    \glt `It was that very Modon eel.'\\
    \z

\item \textit{miim} ‘just like this’ from \textit{iim} ‘like this’ (\ref{example9.11}).

\ea
    \label{example9.11}\il{Khorchin}
    \gll huu 	miim 		miim 		budun.\\
    all	just.like.this 	just.like.this 	thick\\
    \glt `They were all just this thick.’ (showing)'\\
    \z

\item \textit{mitiim} ‘just like that’ from \textit{tiim} ‘like that’ (\ref{example9.12}).

\ea
    \label{example9.12}\il{Khorchin}
    \gll mitiim 		sanaa-tai 	ir-jee.\\
    just.like.that  	idea-\textsc{com} 	come-\textsc{pst}\\
    \glt `I came exactly with this idea in mind (I came exactly for this purpose).'\\
    \z

\item \textit{mengeed} ‘(doing) in this very way’= ’just like this’ from \textit{engeed} ‘(doing) in this way’ (\ref{example9.13}).

\ea
    \label{example9.13}\il{Khorchin}
    \gll mengeed 		neg 	tangs 	mod 	ux-jee.\\
    just.in.this.way	one	row	tree	die-\textsc{pst}\\
    \glt `And in this very way the whole row of trees died.'\\
    \z

\item \textit{metgej} ‘(doing) in that very way’ from \textit{tegej} ‘(doing) in that very way’ (\ref{example9.14}).

\ea
    \label{example9.14}\il{Khorchin}
    \gll metgej 			or-j 			ir-sen 			šdee.\\
    just.in.this.way 	enter-\textsc{cvb.ipfv}	come-\textsc{ptcp.pfv} \textsc{emph}\\
    \glt `This is the very road we took on the way here.'\\
    \z

\item \textit{mudii} ‘exactly this much’ from \textit{udii} ‘this much’ (\ref{example9.15}).

\ea
    \label{example9.15}\il{Khorchin}
    \gll mudii 			gonjgoil-son.\\
    just.this.much  	be.oblong-\textsc{ptcp.pfv}\\
    \glt `(Its shape was) oblong, this long (showing).'\\
    \z

\item \textit{mendegu}/\textit{mundugu} ‘exactly from here’ from \textit{ende-gu} [here-\textsc{nmlzr}] ‘coming from here, local’ (\ref{example9.16}).

\ea
    \label{example9.16}\il{Khorchin}
    \gll nienie-nii 		ug 		suugaal 	ger 	bol 	mende-gu 		ii?\\
    grandmother-\textsc{gen} 	original 	seat 		home 	\textsc{top} 	just.here-\textsc{nmlzr} 	\textsc{q}\\
    \glt `Grandmother, are you originally from this very place?'\\
    \z
\end{itemize}

The forms listed above are mostly found in eastern \ili{Mongolian} dialects, even though in recent years they started being occasionally used by speakers of other Inner \ili{Mongolian} varieties. The word \textit{meter}, which is also used as a filler, is so prominent that Mongols in some other parts of Inner Mongolia used to mock \ili{Khorchin} soldiers by calling them \textit{Meteruud} ‘the Meters’.

This element \textit{mV-} has most probably evolved from the \ili{Mongolian} emphatic pronoun \textit{mön} (\ili{written Mongol} ‘the same, just this’, \citet[51]{Poppe2006}, \il{Mongolic}Proto-Mon\-gol\-ic ‘the very, the same’, \citet[20]{Janhunen2003a}. In modern \ili{Mongolian} proper it has been mostly used as an (often emphatic) copula, e.g.

\ea
    \label{example9.17}\il{Mongolian}
    \gll bi Dorj mön.\\
    1\textsc{sg} Dorj \textsc{cop}\\
    \glt `I am (indeed) Dorj.'\\
    \z

\noindent and as an emphatic particle, e.g.

\ea
    \label{example9.18}\il{Mongolian}
    \gll Ulaanbaatar 	utaa-güi 	bol 	mön 	goyo.\\
    Ulaanbaatar 	smog-\textsc{priv} 	\textsc{top} 	\textsc{ptc} 	nice\\
    \glt `It would be really nice if Ulaanbaatar was without smog.'\\
    \z

While combining the particle \textit{mön} with deictics is occasionally found in many of the modern Mongol varieties (e.g. \ili{Khalkha} \textit{mön ter xün} ‘that very person’), its grammaticalization into a kind of prefix has only taken place in \ili{Khorchin} and the adjacent eastern \ili{Mongolian} varieties. In other spoken \ili{Manchu} varieties mainly the form \textit{meter} is attested \citep[155]{WangQingfeng2005Z} but seems to be marginal compared to its massive use in \ili{Sibe}. Another interesting question is that of the \ili{Sibe} word \textit{menjang} ‘indeed, truly’ which is used in positions corresponding to the use of the word \textit{mön} in \ili{Mongolian}. This expression is attested in \ili{written Manchu} in the form \textit{mujangga}. No plausible \ili{Jurchen} etymology for this word seems to be at hand, therefore a connection to the \ili{Mongolian} form \textit{mön} may be considered. In the whole, the above-mentioned \ili{Sibe} set of emphatic deictic expressions is one of the candidates for a proof of direct and intensive contact between the ancestors of modern \ili{Khorchin} and \ili{Sibe}.

\subsubsection{Replacement of personal pronouns with demostratives}

Grammars of \ili{written Manchu} give the 3rd person pronouns as \textit{i} (3\textsc{sg}) and \textit{ce} (3\textsc{pl}) which are regularly inflected for case. In \ili{Manchu} texts, especially in the more “natural” ones such as historical narratives the demonstrative plural forms \textit{ese} ‘these’ (singular \textit{ere} ‘this’) and \textit{tese} ‘those’ (singular \textit{tere} ‘that’) are used more frequently than \textit{ce}. As plural forms\footnote{In \ili{Manchu} only nouns denoting people, deities or ghosts are marked for number.} they are generally reserved for human or human-like beings, thus being in fact personal pronouns. In the oral \ili{Manchu} varieties (\ili{Sanjiazi}, \ili{Aihui} and \ili{Yibuqi}) the 3rd person plural pronoun \textit{ce} has been completely replaced by an oral form of \textit{tese} (\citealt[52]{WangQingfeng2005Z} \textit{tetse}, \citealt[123]{ZhaoJie1989Z} \textit{ts’etse}, etc.). A form derived from the 3rd  person singular pronoun \textit{i} is, however, attested in all three varieties: \ili{Sanjiazi}: \textit{yin}, \ili{Aihui} \textit{i} \citep[52]{WangQingfeng2005Z}, \ili{Yibuqi} \textit{ji} \citep[189]{ZhaoJie1989Z}.  These forms are noted as used along with the demonstrative \textit{tere}/\textit{tele} ‘that’.

In \ili{Mongolic}, already in the \ili{Middle Mongol} period the Proto-\ili{Mongolic} 3rd person pronouns \textit{i} (singular) and \textit{a} (plural) have been generally replaced by the demonstratives \textit{ene}/\textit{tere} for singular and \textit{ede}/\textit{tede} for plural \citep[72]{Rybatzki2003}.

In \ili{Sibe} the 3rd person pronouns are not attested at all, even though knowledge of the literary language and thus also of the forms \textit{i} and \textit{ce} was widespread till the 20th century.

Hence, the tendency towards replacement of 3rd person pronouns by demonstratives exists not only in \ili{Mongolic}, but also in \ili{Manchu}. Systematic usage of personal pronouns in \ili{written Manchu} may be regarded as a conservative feature and is being abandoned in less canonical \ili{Manchu} writing. The process, however, is on half-way in Manchurian \ili{Manchu} while it has been completed in \ili{Sibe}.

Admittedly, this is a cross-linguistically common process and does not tell anything about the \ili{Khorchin}-\ili{Sibe} contacts. However, it is still possible that a direct influence of a \ili{Mongolic} vernacular on \ili{Sibe} has accelerated the change that was already underway in the spoken \ili{Manchu} varieties -- the complete loss of the \ili{Manchu} pronominal form and its replacement with demonstratives which are, moreover, almost homophonous in Monglian and \ili{Manchu}.

\subsubsection{Possessive clitics and Sibe phrasal possession}

\ili{Sibe} has a system of possessive clitics which resemble the \ili{Mongolian} possessive clitics and do not occur in any other \ili{Manchu} variety. Their function is similar, specifically, to \ili{Khorchin}. Much in the same way as in most modern \ili{Mongolian} languages including \ili{Khorchin}, the 3rd person possessive clitic functions as a definite marker or a topicalizer (cf. \citealt{Hölzl2017b}).

Furthermore, \ili{Sibe} uses the 3rd person possessive clitic to express possession in a way which resembles the prototypical Tungusic head-marked possessive phrases (cf. \citealt[45]{Gorelova2002}).

\subsubsubsection{Phrasal possession and definite marking in Manchu}

In \ili{written Manchu} the principal way to express possession and association is marking on the dependent which then takes the genitive (or genitive-in\-stru\-men\-tal) suffix, e.g. \textit{min-i bithe} [1\textsc{sg-gen} book] ‘my book’; \textit{morin-i uju} [horse-\textsc{gen} head] ‘horse’s head’ or ‘horse head’; \textit{tacikū-i sefu} [school-\textsc{gen} teacher] ‘teacher of the school/school teacher’. Written \ili{Manchu} has no possessive clitics.

In the spoken Manchurian varieties possession may be dependent-marked, which is obligatory if the possessor is a pronoun. In other cases juxtaposition is common. However, while no possessive clitics are attested in the available materials, \ili{Sanjiazi} uses the genitive marker \textit{-ning} (< \ili{written Manchu} marker of independent definite form \textit{=ningge}) as a possessive and definite marker in the same way as \ili{Sibe} uses the 3rd person possessive clitic \textit{=ni}, e.g.

\ea
    \label{example9.20}\il{Sanjiazi}
    \gll ame-ning  		yawe-xei.\\
    father-3\textsc{sg.poss} 	go-\textsc{pst}\\
    \glt `His father/the father left.'\\
    \z

\subsubsubsection{Possessive markers in Mongolian}

Most \ili{Mongolian} varieties have a set of possessive markers which go back to reconstructed genitive forms of the Proto-\ili{Mongolic} personal pronouns (Table \ref{example9.1}).

\begin{table}
\begin{tabular}{ l l l }
  \lsptoprule
& Singular &	Plural\\
  \midrule
1\textsuperscript{st} person &	*mi.n-U &	*bida.n-u\\
2\textsuperscript{nd} person &	*ci.n-U &	*ta.n-u\\
3\textsuperscript{rd} person &	*i.n-U &	*a.n-u\\
  \lspbottomrule
\end{tabular}
\caption{Proto-Mongolic personal pronouns (\citealt[18]{Janhunen2003a})}
\label{table:9.1}
\end{table}

While in some Mongol varieties such as \ili{Buryat} and \ili{Oirat} these pronouns have been grammaticalized into possessive suffixes, others, like \ili{Khalkha} and \ili{Khorchin}, use slightly modified forms of the 1st and 2nd person possessive pronouns as clitics. Since the 3rd person possessive pronouns have been replaced by demonstratives, the system of possessive clitics has been supplemented with a “neutralized reflex of the original pronominal genitives” (\citealt[92]{Janhunen2003b}) – the form \textit{ni}. Consequently, the \ili{Khalkha} possessive clitics are the ones shown in Table \ref{table:9.2}.


\begin{table}
\begin{tabular}{ l l l l }
  \lsptoprule
& \thead{Possessive\\ enclitic} &	\thead{Possessive\\ pronoun} &	\thead{Personal\\ pronoun}\\
  \midrule
1\textsc{sg} &	miny &	minii &	bi\\
2\textsc{sg} &	ciny &	cinii &	ci\\
3\textsc{sg} &	ny & &		-  (*i)\\
1\textsc{pl} &	maany &	manai &	bid\\
2\textsc{pl} &	tany &	tanai &	ta\\
3\textsc{pl} &	ny & &		-  (*a)\\
  \lspbottomrule
\end{tabular}
\caption{Khalkha possessive enclitics, possessive pronouns and personal pronouns (\citealt[164]{Svantesson2003})}
\label{table:9.2}
\end{table}

\begin{sloppypar}
In \ili{Khalkha}, all the enclitics are alternatively used to express possession along with the basic dependent-marked noun phrases. The choice of a clitic instead of a pronoun in genitive form may have semantic, stylistic or modality reasons, e.g. \textit{min-ii eej} [1\textsc{sg-gen} mother] ‘my mother (neutral)’ vs. \textit{eej=miny} [mother=2\textsc{sg.poss}] ‘my mother (expressing emotional attachment)’. The enclitics may be used instead of pronominal genitives in all functions of the latter, i.e. possession, association, whole-part relationship (cf. \citealt[262]{Dixon2010b}). They also determine postpositions or indicate the agent in relative clauses.
In \ili{Khorchin} the frequency of clitics slightly differs from other \ili{Mongolian} varieties: the 3rd person enclitic\textit{ =ni} [en] is frequent, closely followed by the 2nd person singular enclitic \textit{šini} [ʃin]. In contrast, the rest, 1st person and 2nd person plural enclitics, are rare.
\end{sloppypar}

Examples of possessive enclitics in \ili{Khorchin}:

\ea
    \label{example9.21}\il{Khorchin}
    \gll ger=ni 		dalan 		šagaan 	nohoi-tee.\\
    house=3\textsc{sg.poss} 	seventy 	white 		dog-\textsc{com}\\
    \glt `In their house there were seventy white dogs.'\\
    \z

\ea
    \label{example9.22}\il{Khorchin}
    \gll ger=šini 	  	bol 	bain 		aa 	šii?\\
    house=2\textsc{pl.poss} 	\textsc{top}	wealthy	EMPH \textsc{q}\\
    \glt `Was your house(hold) wealthy?'\\
    \z

% % \noindent \textbf{Clitics as definite markers}:\todo{Sectioning!}
\begin{sloppypar}
In most modern Mongol varieties, possessive clitics are used in functions whose common denominator is probably best described as definiteness (Janhunen: “deictic determinants connected with the category of definiteness”). In some cases they “refer to the discourse situation” (\citealt[93]{Janhunen2003b}).  The 3rd person and 2nd person singular possessive clitics are the most common in this function. In \ili{Khorchin}, only the latter two seem to be used as definite markers, e.g.:
\end{sloppypar}


\ea
    \label{example9.23}\il{Khorchin}3rd person possessive enclitic \textit{=ni} (\ref{example9.23})\\
    \gll ter 	olson 		yum=ni 		ertnii, 		uldsen=ni 	bol 	suulernii.\\
    that 	bamboo 	thing=\textsc{def} 		ancient, 	the.rest=\textsc{def}	\textsc{top} 	later\\
    \glt `The one made of bamboo is ancient, the rest of them is more recent.'\\
    \z 

\ea
    \label{example9.24}\il{Khorchin}2nd person singular possessive clitic \textit{=šini} (\ref{example9.24})\\
    \gll ter 	uise-d=šini 			iim 	terg 	gue.\\
    that 	times-\textsc{dat.loc}=\textsc{def} 	such 	cart  	\textsc{neg.ex}\\
    \glt `In those times there were no such carts.'\\
    \z

\subsubsubsection{Sibe possessive clitics}

In \ili{Sibe} a set of possessive clitics exists which for the 1st and 2nd persons are almost identical with possessive pronouns. In the 3rd person the form \textit{ni} is used which can be interpreted either as having evolved from the \ili{Manchu} 3rd person possessive pronoun \textit{ini} or as a \ili{Mongolian} borrowing. However, while the 3rd person clitic is frequent and the 2nd person singular clitic occurs sporadically, the rest of the forms is rather rare.

\begin{table}
\begin{tabular}{ l l l }
  \lsptoprule
& Singular &	Plural\\
  \midrule
1 &	mini &	moni\\
2 &	sini &	soni\\
3 &	ni &	ni\\
  \lspbottomrule
\end{tabular}
\caption{Possessive enclitics in Sibe}
\label{table:9.3}
\end{table}

Examples of possessive clitics in \ili{Sibe}:

\ea
    \label{example9.25}\il{Sibe}
    \gll bo=ni		    	ambu.\\
    house=3\textsc{sg.poss} 	big\\
    \glt `His house is big.'\\
    \z

\ea
    \label{example9.26}\il{Sibe}
    \gll jaqe-we=sini 		bierxe!\\
    thing-\textsc{acc}=2\textsc{sg.poss}	collect\\
    \glt `Take your belongings!'\\
    \z

In \ili{Sibe} only the 3rd person possessive clitic is used as a \textsc{definite marker}, e.g.

\ea
    \label{example9.27}\il{Sibe}
    \gll nane=ni 	ji-xe			na?\\
    person=\textsc{def} 	come-\textsc{ptcp.pfv}	\textsc{q}\\
    \glt `Has the person arrived?'\\
    \z

Besides the function of definite marker the \ili{Sibe} marker \textit{ni} is also used as a kind of \textsc{topic marker}, e.g.

\ea
    \label{example9.28}\il{Sibe}
    \gll Tana=ni 	terang 	baite 	icxia-qu.\\
    Tana=\textsc{top}	such	matter 	arrange-\textsc{neg}\\
    \glt `Tana would not do such things. (As for Tana, she would not do such things.)'\\
    \z

\subsubsubsection{The case of ‘head-marked’ possession in Sibe}

In \ili{Sibe}, the \ili{Manchu}-type marking on the dependent is obligatory when the possessor is referred to by a pronoun, e.g. \textit{sin-i bo} [2\textsc{sg-gen} house] ‘your house’. In other cases it is used alternatively with simple juxtaposition (e.g. \textit{tašqu sewe} [school teacher] ‘teacher of the school/school teacher’), the latter being more frequent. However, the head of possessive phrases is very often (additionally) marked by the 3rd person possessive clitic \textit{=ni}. In such cases the clitic may be interpreted either as a topic marker (\ref{example9.29}) or/and as emphasizing definiteness (\ref{example9.30}), the boundaries between the two meanings being rather vague.  

\ea
    \label{example9.29}\il{Sibe}
    \gll min-i 		age=ni 	tese-maqe 	yavu-qu.\\
    1\textsc{sg-gen}	brother=\textsc{top}	3\textsc{pl-ins}	go-\textsc{neg}\\
    \glt `As for my brother, he does not maintain contacts with them.'\\
    \z

\ea
    \label{example9.30}\il{Sibe}
    \gll honin 	uju=ni 		yecin.\\
    sheep 	head=\textsc{def}	black\\
    \glt `The head of the sheep is black.'\\
    \z

This type of constructions, which has no correspondence in any \ili{Manchu} variety, is so frequent and remarkable in \ili{Sibe} that it resembles the head-marked possessive phrases in the non-\ili{Jurchenic} Tungusic languages. In contrast to the latter, however, the marker \textit{=ni} is always optional in \ili{Sibe}.

While such type of phrases occurs neither in \ili{written Manchu} nor in the Manchurian oral varieties, in \ili{Mongolian} we find structurally similar constructions. Possessive phrases often have additional marking on the head which at the same time implies greater definiteness, e.g.

\ea
    \label{example9.31}
    \ili{Khalkha}\\
    \gll Ganaa.g-iin eej=ni 			emch.\\
    Ganaa-\textsc{gen} mother=3\textsc{sg.poss} 	doctor\\
    \glt `Ganaa’s mother is a doctor.'\\
    \z

In \ili{Mongolian}, simple juxtaposition is marginal in expressing possession which makes ‘head-marked’ possessive constructions of the \ili{Sibe} type rare. However, constructions with similar structure still occur:

\ea
    \label{example9.32}
    \ili{Khalkha}\\
    \gll eej 	bie=ni 			muu 	baina.\\
    mother	body=3\textsc{sg.poss}	bad	\textsc{cop}\\
    \glt `Mother is sick (literally: Mother her body is bad).'\\
    \z

 The existence of possessive clitics in \ili{Sibe} constitutes a remarkable typological difference from \ili{written Manchu}. The clitics are formed and used in a way that is almost identical with that of \ili{Khorchin}. On the first sight, ‘head-marked’ possession does not exist in \ili{Mongolian}. In fact, however, structurally similar possessive phrases occur in colloquial \ili{Mongolian}. No such possessive phrases seem to have been attested in any other \ili{Manchu} variety.


\subsubsection{The limiting clitic \textit{=li}\label{Section9.3.3.5}}

In \ili{Sibe}, the main means for expressing limitation is the clitic \textit{=li}.\footnote{The \ili{Mongolic} origin of the \ili{Sibe} limitation marker was suggested by Norikazu \citet{Kogura2020}.} It can follow any sentence member, e.g.

\ea
    \label{example9.33}\il{Sibe}
    \gll bi=li 		gene-m.\\
    1\textsc{sg}=\textsc{lim} 	go-\textsc{npst}\\
    \glt `Only I will go.'\\
\ex
    \label{example9.34}\il{Sibe}
    \gll eme=li 	nane 	ji-xei.\\
    one=\textsc{lim} 	person come-\textsc{pst}\\
    \glt `Only one person arrived.'\\
\ex
    \label{example9.35}\il{Sibe}
    \gll eme 	nane 	dudu-r=li 		orun 	bi-xei.\\
    one 	person 	lie-\textsc{ptcp.ipfv=lim} 	place 	be-\textsc{pst}\\
    \glt `There was space for only one person to lie.'\\
\ex
    \label{example9.36}\il{Sibe}
    \gll uculu-m 		bana-qv, 	qaici-m=li.\\
    sing-\textsc{cvb.ipfv} 	be.able-\textsc{neg} 	shout-\textsc{npst=lim}\\
    \glt `They cannot sing, they only shout.'\\
    \z

In most modern \ili{Mongolic} languages including \ili{Khalkha} and \ili{Khorchin} the clitic \textit{lV} (<  Classical \ili{Mongolian} \textit{la}/\textit{le}) is used in much the same way, but typically does not determine the predicate, e.g. 

\ea
    \label{example9.37}
    \ili{Khalkha}\\
    \gll bi=l 		yav-na.\\
    1\textsc{sg=lim} 	go-\textsc{npst}\\
    \glt `Only I will go.'\\
\ex
    \label{example9.38}
    \gll neg=l 		xun	 ir-sen.\\
    one=\textsc{lim} 	person 	 come-\textsc{pst}\\
    \glt `Only one person arrived.'\\
    \z

In \ili{written Manchu}, postpositions such as \textit{-i teile}, e.g. \textit{emu niyalma-i teile} ‘only one person’, are used as means of postnominal\footnote{Besides postnominally used expressions, both \ili{Manchu} varieties and \ili{Mongolian} employ adverbs to express limitation. These adverbs (e.g. written \ili{Manchu} \textit{damu}, \ili{Sibe} \textit{dame}, \ili{Khalkha} \ili{Mongolian} \textit{zövxön}) usually stand in the beginning of a sentence, and always come before the noun which they determine, e.g. \ili{written Manchu} \textit{damu emu niyalma} ‘only one person’, \ili{Mongolian} \textit{zövxön neg xün} ‘only one person’. These adverbs are often used together with postnominal limitation as means of emphasis, e.g. \ili{written Manchu} \textit{damu emu niyalma-i teile} ‘only one person’, \ili{Khalkha} \textit{zövxön neg l xün} ‘only one person’.} limitation, and no clitic with similar meaning seems to be attested. Likewise, any similar clitic does not seem to be attested in the Manchurian spoken \ili{Manchu} varieties, wherefore the \ili{Sibe} clitic \textit{=li} is likely to be a borrowing from a \ili{Mongolic} language.

\subsection{Absence of the Manchu directional (itive and ventive) suffixes \textit{-nV-} and \textit{-nji-}}

Written \ili{Manchu} has a large set of deverbal suffixes, most of which have lost their productivity in the spoken varieties. However, in \ili{Sanjiazi}, \ili{Aihui} and \ili{Yibuqi} two of the deverbal suffixes are highly productive – the suffix \textit{-nji-} ‘to come to do something’ and \textit{-nV-} ‘to go to do something’, e.g. \ili{written Manchu} \textit{ala-na-ha}, \ili{Sanjiazi} \textit{ale-na-xe} ‘went to tell’.

In \ili{Sibe} these suffixes have completely lost their productivity. Instead, multiverb expressions are used to convey similar meanings, e.g. \textit{ale-me gene-xei} [tell-\textsc{cvb.ipfv} go-\textsc{pst}] ‘went to tell’, or \textit{gene-me ale-xei} [go-\textsc{cvb.ipfv} tell-\textsc{pst}] ‘went and told.’

\ili{Mongolian} has no directional deverbal suffixes and the meanings ‘go to do’ and ‘come to do’ are expressed by multiverb constructions, e.g. \textit{hele-heer ir-sen} [tell-\textsc{cvb.purp} come-\textsc{pst}]. 

Multiverb constructions are frequent and preferred in many languages in the area. A tendency towards replacing deverbal suffixes by multiverb chains in \ili{Sibe} is not surprising. Perhaps more surprising is the retention of productivity of the deverbal suffix in Manchurian \ili{Manchu}. Still, however, the different developments may have been prompted by the different language environment.


\section{Lexical borrowings}\label{section9.4}

In addition to the possibly contact-induced features in \ili{Sibe} grammar, there is a small-scale but interesting evidence of direct contacts with \ili{Mongolic} languages in the \ili{Sibe} lexicon.

The vocabulary of modern spoken \ili{Sibe} is almost identical with that of \ili{written Manchu}, the main difference being a larger number of \ili{Chinese} loanwords. In addition, several \ili{Russian}, \ili{Uyghur} and \ili{Kazakh} loanwords are used. Although colloquial \ili{Sibe} contains a large amount of \ili{Mongolian} loanwords, most of them are also found in \ili{written Manchu} and therefore do not testify to any specific \ili{Sibe}-\ili{Mongolian} contacts.\footnote{ In general, any search for lexical borrowings is complicated by the nature of \ili{Manchu}-\ili{Mongolian} language contacts which involved not only interactions of spoken varieties, but also the sphere of written translations between \ili{Manchu} and Mongol\il{written Mongol}, which were often done by native speakers of \ili{Mongolic} varieties. There exist many bilingual texts written in the form of interlinear translations. The \ili{Manchu} parts of these bilingual texts usually contain a greater portion of Mongol(ic)\il{Mongolic} loanwords than other types of \ili{Manchu} texts, which are mostly synonyms to original \ili{Manchu} words or \ili{Chinese} loanwords. Once used in written documents, these \ili{Mongolic} loanwords also entered \ili{Manchu} dictionaries, even though their actual use may have been limited.}

Several lexical items such as \textit{kurwo} for ‘bridge’ (\ili{written Manchu} \textit{doohan}) from \ili{Mongolian} \textit{xöörög} (\ili{written Mongol} \textit{kögerge}) ‘bridge’ seem to be restricted to \ili{Sibe}.

While the modern colloquial language hardly yields any lexical evidence of \ili{Sibe}-\ili{Mongolic} contact, in more archaic layers of the lexicon there exist \ili{Mongolian} loanwords related to Buddhism, shamanism and what may be called “folk religion” which are not found in other \ili{Manchu} varieties. Some of these terms are still in use while others are only found in written sources.

\subsection{Buddhist terminology and the language of Buddhist monks}

Historical sources mention the adoption of Tibetan Buddhism by the \ili{Sibe} during the period of their vassalage to the Khorchins. Until the 1930s a Buddhist monastery existed in Chabchal with approximately fourty monks. The language of recitation was Classical \ili{Mongolian}. The language of the monks contained many \ili{Mongolian} Buddhist terms for which nowadays \ili{Manchu} words or \ili{Chinese} loanwords are used. Examples of such pairs are \textit{sumu} (< \ili{written Mongol} \textit{süme}) vs. \textit{miao} (< \ili{Chinese} \textit{miao}) ‘Buddhist temple, monastery’, or \textit{burkan baksi} (< \ili{written Mongol} \textit{burqan bagsi}) vs. \textit{fišk} (\ili{Manchu} \textit{fucihi}\footnote{The \ili{Manchu} word \textit{fucihi} has been interpreted as a borrowing from \ili{Korean} by \citet[259]{Vovin2006}.}) ‘Buddha’. However, judging, among others, from the recording of recitation of a Buddhist text by a \ili{Sibe} monk \citep{Zhuangsheng2018}, the local \ili{Oirat} Mongol tradition of \ili{Mongolian} recitation preserved among the Öölöds of Ili should also be considered as a possible source of the use of \ili{Mongolian} in \ili{Sibe} Buddhist tradition.

\subsection{Shamanic terminology}

Modern \ili{Sibe} in Xinjiang consider shamanic traditions to be their ‘original’ religion. In the construction of their ethnic culture, ‘shamanism’ is assigned key importance. Several influential publications give detailed and normative descriptions of the pantheon, system of rituals and main types of ritualists considered to belong to the concept of ‘shamanism’.\footnote{For descriptions of \ili{Sibe} shamanic traditions see e.g. \citet{Sárközi2013} or \citet{Harris2005}.} The descriptions were accomplished based on fieldwork among family members of shamans, accounts of eyewitnesses and texts written by shamans since the 19th century. These texts, intended as handbooks for shaman disciples and containing mostly invocation texts with few comments and explanations, are the main source of \ili{Mongolic} loanwords which seem to be found exclusively in \ili{Sibe} (cf. \citealt{Zikmundová2013b}).

The so far indentified \ili{Mongolic} loanwords in \ili{Sibe} shamanic texts are the following:


\begin{description}
\item[\textit{elci},] a ritualist specialized in healing children’s diseases, in particular smallpox < \ili{Mongolian} \textit{elc(in)} ‘messenger’, in \ili{Khorchin} ritual practice also a type of ritualist

\item[\textit{deoci},] a ritualist specialized in exorcist rituals connected to the ‘ghost disease’\footnote{ ‘ghost disease’, \ili{Sibe} \textit{yivaxen niungku}, \ili{Khorchin} \textit{ad uvšin}, is a term for a specific type of spirit possession occuring mainly in women (cf. \citealt{Zikmundová2013b})} < \ili{Khorchin} \textit{duuci} a person assigned a role of ‘singer’ in healing the ‘ghost disease’

\item[\textit{deole-},] to perform the exorcist ritual in healing the ‘ghost disease’ < \ili{Khorchin} \textit{duul-} ‘to sing’; to assist the exorcist ritual by singing

\item[\textit{kuri},] a ritual implement in the shape of a building used during the exorcist ritual < Mongol \textit{xüree} ‘circle, temple, monastery’

\item[\textit{altan kuri},] refrain of a song used during the exorcist ritual in healing the ‘ghost disease’ < \ili{Khorchin} \textit{altan xüree} ‘golden circle/golden temple’, name of a ritual procedure used during the exorcist ritual in healing the ‘ghost disease’.
\end{description}

All but one of the above \ili{Mongolic} loanwords pertain to a single type of shamanic ritual – healing a certain type of spirit possession. The ritual was apparently borrowed by \ili{Sibe} from the eastern Mongols, most probably Khorchins, where it existed in several elaborated variants until the Cultural Revolution. The original \ili{Mongolian} ritual, known as \textit{andai}, is unique for Khorchins and their immediate neighbors. The \ili{Sibe} version of the ritual is simplified and shortened.


\section{Conclusions: The “reality” of Sibe-Khorchin contacts}

For reasons that may be called political, the ethnic history of the \ili{Sibe} – speakers of a \ili{Manchu} (Tungusic-\ili{Jurchenic}) variety – has been a much discussed topic in China. As part of the official narrative, the pre-Qing contacts of the \ili{Sibe} with \ili{Khorchin} Mongols are being mentioned – a fact recorded by a few brief notes in historical documents. The \ili{Sibe} are said to have been vassals of the Khorchins before the 1690s. After 1764, when the ancestors of modern \ili{Sibe} speakers were moved to Xinjiang, no more contacts between \ili{Sibe} and \ili{Khorchin} Mongols took place. The \ili{Sibe}-\ili{Khorchin} contact narrative has been used, together with popular views with some background in oral tradition, to argue for a non-\ili{Jurchen}, possibly Mongol-related origin of the \ili{Sibe}. It has gradually become part of the self-consciousness of modern Xinjiang \ili{Sibe}. The question has also triggered academic discussion on this topic. 

In this paper I tried to select shared features in \ili{Sibe} and \ili{Khorchin} which are not, or marginally, documented in other varieties of spoken \ili{Manchu} and therefore may testify to a specific contact history. Since no diachronic data for either \ili{Sibe} or \ili{Khorchin} are available, modern spoken \ili{Sibe} and modern \ili{Khorchin} materials were used. Additionally, lexical data from a written source are mentioned that testify to certain cultural exchange between the \ili{Sibe} and the Khorchins.

The collected features mostly apply to morphology and one of them, the emphatic prefix \textit{me-} is typical for spoken \ili{Sibe} and eastern \ili{Mongolian}. The latter, together with the shared shamanic terminology, and possibly also the shared phonetic features, seem to testify to a direct and lively linguistic and cultural exchange between the \ili{Sibe} and the ancestors of modern \ili{Khorchin}. The rest of the mentioned analogies have less clear implications: Being more or less typical for all modern \ili{Mongolic} languages, they may be features of a linguistic area where multiple \ili{Mongolic} and Tungusic languages influenced each other.

A short overview of historical facts with connection to the linguistic situation in the Qiqihar region during the Ming is given as a broader context of the documented \ili{Sibe}-\ili{Khorchin} contacts. These facts show that the main contact language of the \ili{Sibe} was not the language of the Khorchins which arrived from the \ili{Mongolian} plateau in the mid-16th century but rather the language of the Ujiyed. The Ujiyed were a Mongolized Tungusic group whose presence in the Qiqihar region dates back to early Ming, or even Yuan, times. Together with two other groups – the Uriangkhan and the Ongniud – these local Mongols may have already spoken a disctinct dialect with “eastern” features when the Khorchins arrived and merged with them. The described shared morphological features and lexical borrowings, however scanty, seem to point towards a \ili{Mongolic} influence that was stronger and longer-lasting on the ancestor of modern \ili{Sibe} than on the ancestors of the other spoken \ili{Manchu} varieties. In this context, another important and rather \ili{early Mongolic} contact language of the \ili{Sibe} – the Daur\il{Dagur} – needs to be examined in the future.

Another question posed in this paper is the significance of the shared linguistic features in imagining \ili{Sibe} history. The areas around modern Qiqihar and Fuyu, where the \ili{Sibe} lived, were bordering the homeland of the Hūlun Jurchens who are thought to have spoken a \ili{Mongolic}-influenced \ili{Jurchen} variety during the Ming period. The whole area was controlled by the Mongolized Ujiyed and the Hūlun Jurchens were even referred to as Mongols by other Jurchens. This suggests an image of the \ili{Sibe} as linguistic representatives of this broader \ili{Mongolic}-influenced \ili{Jurchenic} community. 

The linguistic developments of \ili{Sibe} during the Qing period fall out of the scope of this paper. It is, however, important to mention that the period of linguistic diversity during the Ming was effectively ended by the subsequent standardization processes, which, for the \ili{Sibe}, begun with their incorporation into the \ili{Manchu} Eight Banners in 1692. The latter affected both \ili{Mongolic} and \ili{Jurchenic} languages. Introduction of Buddhism to the \ili{Khorchin} Mongols, accompanied by spread of literature in general, brought about literacy in Classical \ili{Mongolian}. For the \ili{Jurchenic} part, standardization efforts of the \ili{Manchu} ruling strata is a generally acknowledged fact. Both \ili{Literary Mongolian} and Classical \ili{Manchu} enjoyed high prestige. Spread of Classical \ili{Mongolian} may be one of the factors that brought \ili{Khorchin} vocabulary and grammar closer to the central \ili{Mongolian} varieties. The local \ili{Jurchenic} varieties probably became extinct after the incorporation of the speakers, including \ili{Sibe}, into the \ili{Manchu} military units where their spoken varieties were gradually replaced by forms of Standard \ili{Manchu}. 

The question remains whether the described features of \ili{Mongolic} origin in \ili{Sibe} may be considered remains of traditional diglossia in a standard \ili{Manchu} language and an older, \ili{Mongolic}-influenced \ili{Jurchenic} variety. Information received from \ili{Sibe} speakers (e.g. \ia{Guo, Qing}Guo Qing, p.c. August 2009) suggests that in the colloquial language of some elderly speakers \ili{Mongolic} synonyms to \ili{Manchu} lexemes are frequent and some of them seem not to be found in \ili{Manchu} dictionaries, such as the verb \textit{amere-} ‘to rest, to sleep’ (cf. \ili{written Manchu} \textit{erge-}, \ili{Sibe} \textit{erxe-}, \ili{Khorchin} \textit{amer-} ‘to rest, to sleep’). It is worth mentioning that most of the studies of \ili{Sibe} were conducted on the basis of material gathered from speakers with high level of literacy in \ili{Manchu}. Any research of the reported non-standard features has not yet been conducted.


\section*{Abbreviations}

\begin{multicols}{2}
\begin{tabbing}
\textsc{dat.loc}\hspace{1ex}\= dative-locative\kill
1\textsc{sg} \> 1st person singular\\
1\textsc{pl} \> 1st person plural\\
2\textsc{sg} \> 2nd person singular\\
2\textsc{pl} \> 2nd person plural\\
3\textsc{sg} \> 3rd person singular\\
3\textsc{pl} \> 3rd person plural\\
\textsc{com} \> comitative\\
\textsc{cop} \> copula\\
\textsc{cvb} \> converb\\
\textsc{dat.loc} \> dative-locative\\
\textsc{def} \> definite marker\\
\textsc{emph} \> marker of emphasis\\
\textsc{foc} \> focus marker\\
\textsc{gen} \> genitive\\
\textsc{ipfv} \> imperfective\\
\textsc{lim} \> limitation marker\\
\textsc{neg} \> negation\\
\textsc{nmlz} \> nominalizer\\
\textsc{npst} \> non-past tense\\
\textsc{poss} \> possessive clitic\\
\textsc{pn} \> place name\\
\textsc{pfv} \> perfective\\
\textsc{priv} \> privative\\
\textsc{prog} \> progressive\\
\textsc{purp} \> purposive\\
\textsc{pst} \> past tense\\
\textsc{ptc} \> particle\\
\textsc{ptcp} \> participle\\
\textsc{top} \> topic marker\\
\textsc{q} \> question marker
\end{tabbing}
\end{multicols}

\section*{Acknowledgements}

I am indebted to Bai Xiaomei, then a student at the Inner Mongolian National University of Tongliao, for the collection of narratives in the Khorchin Northern Banner in 2015. I am grateful to \name{Andreas}{Hölzl} for the idea on \sectref{section9.3.3}. I further thank \name{Andreas}{Hölzl}, \name{Benjamin}{Brosig}, two anonymous reviewers and Jichang Lulu for their helpful suggestions and comments.\largerpage

{\sloppy\printbibliography[heading=subbibliography,notkeyword=this]}
\end{document}
