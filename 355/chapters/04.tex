\documentclass[output=paper,colorlinks,citecolor=brown]{langscibook}
\ChapterDOI{10.5281/zenodo.7053365}
\author{Andreas Hölzl\affiliation{University of Potsdam}}
%\ORCIDs{}

\title{‘What’s your name?’ in Tungusic and beyond}

\abstract{This study investigates questions about personal names, i.e. questions corresponding to \textit{What’s your name?} in English. This potentially universal type of question is referred to as the personal name question (PNQ). The study sketches the typological variation found in the PNQ from a cross-linguistic perspective and analyzes the synchronic typology and diachronic development of the PNQ in Tungusic, a small but important language family spoken in Northeast Asia.

Cross-linguistically, two main types of PNQs are attested. Type A is an equational copula sentence (e.g., \textit{What is your name?}) while Type B contains a speech act verb (e.g., \textit{What are you called?}). Tungusic shows a tendency for Type A but, because of contact languages such as Mongolian and Russian, also has instances of Type B. One of several other dimensions of variation among the world’s languages is the kind of interrogative used in PNQs. Tungusic languages originally used an interrogative meaning ‘who’ (literally \textit{Who is your name?}). The use of ‘what’ in several languages located in the south and of ‘how’ in many languages in the north can be attributed to influence from Chinese, Russian, and other languages.

Historical accounts of Tungusic are usually restricted to individual items (e.g., *\textit{si} ‘you (\textsc{sg})’ *\textit{gärbü} ‘name’, *\textit{ŋüi} ‘who’, e.g. \citealt{Benzing1956}), but rarely are larger expressions reconstructed to Proto-Tungusic. This study shows that the Proto-Tungusic PNQ as one idiom can be plausibly reconstructed as *\textit{si(n-i)} \textit{gärbü-si} \textit{ŋüi}? ‘2\textsc{sg(.obl-gen)} name-2\textsc{sg.poss} who’. Most deviations in modern languages can be explained by contact with surrounding languages.\\
\\
\keywords{personal name question, typology, Tungusic, reconstruction, frames, construction grammar}
}


\IfFileExists{../localcommands.tex}{
   \addbibresource{../localbibliography.bib}
   \usepackage{langsci-optional}
\usepackage{langsci-gb4e}
\usepackage{langsci-lgr}

\usepackage{listings}
\lstset{basicstyle=\ttfamily,tabsize=2,breaklines=true}

%added by author
% \usepackage{tipa}
\usepackage{multirow}
\graphicspath{{figures/}}
\usepackage{langsci-branding}

   
\newcommand{\sent}{\enumsentence}
\newcommand{\sents}{\eenumsentence}
\let\citeasnoun\citet

\renewcommand{\lsCoverTitleFont}[1]{\sffamily\addfontfeatures{Scale=MatchUppercase}\fontsize{44pt}{16mm}\selectfont #1}
  
   %% hyphenation points for line breaks
%% Normally, automatic hyphenation in LaTeX is very good
%% If a word is mis-hyphenated, add it to this file
%%
%% add information to TeX file before \begin{document} with:
%% %% hyphenation points for line breaks
%% Normally, automatic hyphenation in LaTeX is very good
%% If a word is mis-hyphenated, add it to this file
%%
%% add information to TeX file before \begin{document} with:
%% %% hyphenation points for line breaks
%% Normally, automatic hyphenation in LaTeX is very good
%% If a word is mis-hyphenated, add it to this file
%%
%% add information to TeX file before \begin{document} with:
%% \include{localhyphenation}
\hyphenation{
affri-ca-te
affri-ca-tes
an-no-tated
com-ple-ments
com-po-si-tio-na-li-ty
non-com-po-si-tio-na-li-ty
Gon-zá-lez
out-side
Ri-chárd
se-man-tics
STREU-SLE
Tie-de-mann
}
\hyphenation{
affri-ca-te
affri-ca-tes
an-no-tated
com-ple-ments
com-po-si-tio-na-li-ty
non-com-po-si-tio-na-li-ty
Gon-zá-lez
out-side
Ri-chárd
se-man-tics
STREU-SLE
Tie-de-mann
}
\hyphenation{
affri-ca-te
affri-ca-tes
an-no-tated
com-ple-ments
com-po-si-tio-na-li-ty
non-com-po-si-tio-na-li-ty
Gon-zá-lez
out-side
Ri-chárd
se-man-tics
STREU-SLE
Tie-de-mann
}
   \boolfalse{bookcompile}
   \togglepaper[4]%%chapternumber
}{}

\begin{document}
\maketitle

\section{Introduction} 

\begin{flushright}
\begin{tabular}{@{}r@{~}l@{}}
    \textsc{faust}: & What is thy name?\\
    \textsc{mephistopheles}: & A question small, it seems,\\
                    & For one whose mind the Word so so much despises;\\
                    & Who, scorning all external gleams,\\
                    & The depths of being only prizes.\\
    \multicolumn{2}{r}{(Johann Wolfgang \citealt{vonGoethe2018})}
\end{tabular}
% Faust: What is thy name?\\
% Mephistopheles: A question small, it seems,\\
% For one whose mind the Word so so much despises;\\
% Who, scorning all external gleams,\\
% The depths of being only prizes.\\
\end{flushright}
\bigskip
\noindent This study investigates what will be referred to as the \textit{personal name question} (PNQ), i.e. a question about the name of a person, more specifically of an addressee (or second person), such as \textit{What’s your name?} in \ili{English}. Almost every natural language seems to have a conventional way of expressing this question. But despite being a question that occurs in textbooks of many languages, there has been surprisingly little cross-linguistic research on this topic. Even \textit{The Oxford handbook of names and naming} \citep{Hough2016} only devotes a brief section to this topic \citep[26]{VanLangendonck2016}. Not many grammatical descriptions mention PNQs and even fewer address it as a topic in its own right. There are some noticeable exceptions, such as \citet[8, 19]{Mushin1995}, who noted that Australian languages often employ a personal interrogative meaning ‘who’ in questions about names. \citet[509f.]{Blust2013} made a similar observation about \ili{Austronesian} languages. The following examples, therefore, literally mean ‘Who is your name?’ (see also \citealt{Hölzl2014}; \citealt{Gil2018}).\footnote{Throughout the paper, examples without translation can be translated into \ili{English} as ‘What is your name?’ or as an answer thereto.}

\ea
    \label{example4.1}
    \ili{Yankunytjatjara} (\ili{Pama-Nyungan}; \citealt[19]{Mushin1995})\\
    \gll nyuntu	ini \textbf{ngana}-nya?\\
    	 2\textsc{sg.nom} name who-\textsc{nom.name}\\
\ex
    \label{example4.2}
    \ili{Ngaju Dayak} (\ili{Austronesian}; \citealt[510]{Blust2013})\\
    \gll \textbf{eweh} ara-m?\\
    who name-2\textsc{sg.poss}\\
    \z

Many other languages, such as \ili{Aymara} spoken in southern Peru or \ili{Badaga} in India, behave like \ili{English} and use an interrogative with the meaning ‘what’ instead.

\ea
    \label{example4.3}
    Muylaq’ \ili{Aymara} (\ili{Aymaran}; \citealt{Coler2014}: 402)\\
    \gll ¿\textbf{kuna}	suti-ni-ʋ-rak(i)-ta-st(i)?\\
    what name-\textsc{att-cop.v-ad-2sim-q}\\
\ex
    \label{example4.4}
    \ili{Badaga} (\ili{Dravidian}; \citealt{Balakrishnan1999}: 214)\\
    \gll ninna hesaru \textbf{e:na}?\\
    2\textsc{sg.gen} name what\\
\z

Some languages, such as \ili{Tok Pisin} spoken in Papua New Guinea or \ili{Wulai Atayal} on Taiwan, allow the use of both ‘who’ and ‘what’.

\ea
    \label{example4.5}
    \ili{Tok Pisin} (\ili{English}-based creole; \citealt{Wurm1985}: 345)\\
    \gll \textbf{husat}/\textbf{wanem}	nem		bilong	yu?\\
    who/what			name	\textsc{gen}		2\textsc{sg}\\
\ex
    \label{example4.6}
    \ili{Wulai Atayal} (\ili{Austronesian}; \citealt[293]{Huang1996}: 293)\\
    \gll \textbf{imaʔ}/\textbf{naluʔ}		laluʔ=suʔ?\\
    who/what		name=2\textsc{sg.poss}\\
\z

This variation is also addressed in \citet[61–94, passim]{Idiatov2007}, who, among other things, investigated “\textsc{name}-questions” in a large sample of languages. This kind of question is broadly defined, however, and not restricted to the question about personal names. According to \citet[47]{Idiatov2007}, the question is based on “non-prototypical combinations of values” because it combines the features \textsc{thing}, \textsc{identification}, and \textsc{proper name} (as an expected answer). Prototypical combinations, on the contrary, are said to be \textsc{person}, \textsc{identification}, \textsc{proper name} for ‘who’ (e.g., \textit{Who are you?} \textit{I’m Mike.}) and \textsc{thing}, \textsc{classification}, and \textsc{common noun} for ‘what’ (e.g., \textit{What is this?} \textit{This is a book.}). Following \citet{Idiatov2007}, the fact that some languages like Aymara use ‘what’ and others, such as \ili{Ngaju Dayak}, ‘who’ in questions about names is a result of the non-prototypical combination of these features that allows both choices. An alternative explanation of the variation, among other things based on the ambiguous nature of the concept \textsc{name} itself, will be proposed in this study.
The use of other interrogatives, such as \textit{jak} ‘how’ in \ili{Polish} (asking about the \textsc{manner}), is argued to be an “avoidance strategy” (\citealt[61]{Idiatov2007}). This is a feature common in, but not restricted to, European languages.\il{languages of Europe}

\ea
    \label{example4.7}
    \ili{Polish} (\ili{Indo-European})\\
    \gll \textbf{Jak}		masz			na		imię?\\
    how		have.2\textsc{sg.prs.ind}	on		name\\
\ex
    \label{example4.8}
    \ili{Manam} (\ili{Austronesian}; \citealt[406]{Lichtenberk1983})\\
    \gll ará-m			\textbf{báʔara}?\\
    name-2\textsc{sg.poss}		how\\
    \z

For some reason, the focus of previous studies has been on the choice of the interrogative in the PNQ. Apart from \citet[63–67]{Idiatov2007}, few studies address morphosyntactic patterns on how questions about names are expressed cross-linguistically. But the PNQ also varies on many other dimensions, including the marking of possession, politeness, the presence or absence of a copula, the valency of the speech act verb and many more. These typological features of the PNQ are addressed in \sectref{section4.2}.

The underlying theoretical background of this study is loosely based on a general form of Frame Semantics and Construction Grammar, especially as it can be applied to historical and areal phenomena (e.g., \citealt{Fillmore1985}; \citealt{Langacker2008}; \citealt{Hilpert2014}; \citealt{Trousdale2014}; \citealt{Lefebvre2015}; \citealt{Hölzl2018c}). Construction Grammar is built on the idea that the lexicon and the grammar of a language are not clearly distinct, but form a continuum of constructions of different size and complexity. Crucially, idioms and fixed expressions, including the PNQ, are considered constructions in their own right. Construction Grammar allows for partial analyzability and different levels of schematicity. In \ili{English}, for instance, \textit{What’s your name?} is not only a conventional expression, but is at the same time analyzable as an instantiation of more abstract constructions, including \textit{what’s} X, where X refers to an open slot. The questions \textit{What’s this?} and \textit{What’s the problem?} are other instantiations of this partially schematic construction.

This study investigates the personal name question in the Tungusic language family, which allows a detailed analysis of the individual constructions involved in the expression of the question. Tungusic is a small language family of up to twenty different languages spoken in Northeast Asia, especially eastern Russia and northern China. Data from all attested Tungusic languages are included in the study. Its internal classification is a matter of dispute, but four different subgroups can unmistakably be identified. Following \citet{Janhunen2012a}, these will be referred to as \ili{Ewenic}, \ili{Udegheic}, \ili{Nanaic}, and \ili{Jurchenic}. According to one view (e.g., \citealt{Georg2004}; \citealt{Janhunen2012a}), the former two together form the \ili{Northern Tungusic} languages while the latter can be referred to as \ili{Southern Tungusic} (Table \ref{table:4.1}). The discussion of the Tungusic PNQ in \sectref{section4.4} is divided into subsections on each of the four subgroups. Tungusic is an especially rewarding language family for this study due to the relatively high variability of the personal name question, especially in terms of the interrogative used.

\begin{table}
\begin{tabular}{ p{1.7cm} p{1.7cm} p{7.5cm} }
  \lsptoprule
Branch & Subbranch & Languages\\
  \midrule
Northern Tungusic &	Ewenic & 	Arman, \ili{Even} (Ewen), \ili{Evenki} (Ewenki), \ili{Oroqen}, \ili{Solon}, \ili{Negidal}, ...\\
\cmidrule(lr){2-3}
 & Udegheic\slash\ili{Orochic} & Oroch, \ili{Udihe} (Udeghe), ...\\
 \midrule
Southern Tungusic &	Nanaic  & Kilen*, \ili{Kili}*, \ili{Nanai}, \ili{Samar}, \ili{Ulcha}, \ili{Uilta}, \ili{Ussuri} Nanai, ...\\
\cmidrule(lr){2-3}
 & Jurchenic\slash\ili{Manchuric} & Alchuka, \ili{Bala}, \ili{Jurchen} A, \ili{Jurchen} B, \ili{Kyakala}, \ili{written Manchu}, spoken \ili{Manchu} (e.g., \ili{Sibe}), ...\\
  \lspbottomrule
\end{tabular}
\caption{Possible classification of the Tungusic languages (e.g., \citealt{Georg2004}; \citealt{Janhunen2012a}); *languages with highly mixed affiliation}
\label{table:4.1}
\end{table}\il{Ewenic}\il{Arman}\il{Udegheic}\il{Oroch}\il{Nanaic}\il{Kilen}\il{Jurchenic}\il{Alchuka}\il{Southern Tungusic}\il{Northern Tungusic}

Previous diachronic accounts of Tungusic languages usually focused on phonological, morphological, and lexical aspects (e.g., \citealt{Benzing1956}; \citealt{Doerfer1978} among many others), but have rarely addressed larger expressions. However, similar to lexical items, it is possible to identify \textsc{cognate constructions} in related languages and, therefore, to reconstruct larger constructions to proto-languages (e.g., \citealt{Barðdal2013}). A superficial survey of the personal name question in several \ili{Romance} languages can illustrate this concept.

\ea \label{example4.9}
\ea \ili{French} Comment t’appelles-tu?\\
\ex \ili{Italian} Come ti chiami?\\
\ex \ili{Portuguese} Como te chamas?\\
\ex \ili{Romanian} Cum te cheamă?\\
\ex \ili{Spanish} ¿Cómo te llamas?\\
\z
\z

Of the five languages mentioned, all can make use of a similar construction with the same elements, e.g. the interrogative \textit{come} ‘how’ in initial position, followed by the personal pronoun \textit{ti} ‘2\textsc{sg.obl}’, and an inflected second person singular present indicative form of the verb \textit{chiamare} ‘to call’ in \ili{Italian} (see also \ref{example4.27}). Only \ili{French} has a different verb (\textit{appeler}). Apart from phonological differences, there are also differences in the verbal morphology (e.g., an enclitic personal pronoun \textit{tu} in \ili{French}, see also \ref{example4.19}). Nevertheless, the overall similarity suggests that earlier stages of \ili{Romance} also had a construction out of which the constructions in the individual languages might have developed.\footnote{A proofreader pointed out that Brazilian Portuguese also has an innovative construction: \textit{Como você se chama?}} Changes in the Tungusic PNQ construction and how it can be reconstructed to the proto-language will be addressed in \sectref{section4.4} and \sectref{section4.5}.

This paper has five sections, including this introduction. \sectref{section4.2} sketches a typology of the personal name question from a cross-linguistic perspective. \sectref{section4.3} introduces the semantic background of the question from a frame semantic point of view. \sectref{section4.4} addresses the expression of the question in Tungusic languages. \sectref{section4.4.1} discusses the second person forms and the genitive, \sectref{section4.4.2} gives an overview of the word for ‘name’, and \sectref{section4.4.3} to \sectref{section4.4.6} investigate the PNQ in the four subbranches of Tungusic. The discussion in \sectref{section4.5} reconstructs the PNQ to \ili{Proto-Tungusic} (\sectref{section4.5.1}) gives some conclusions (\sectref{section4.5.2}).


\section{The personal name question from a cross-linguistic perspective}\label{section4.2}

Personal names are probably a universal or near-universal property of human cultures. An exception could be the \ili{Matsigenka} in Peru, where “personal names are of little significance” (\citealt[10]{Johnson2003}). A similar case can be observed in Venezuela, which also illustrates culture-specific functions of personal names:

\begin{quote}
    The \ili{Panare}, for example, have five personal names for men and seven for women. They are all based on physical characteristics, like ‘big eyes’, ‘cutie’, ‘big one’, ‘lopsided one’ etc. Individuals are more likely to be referred to by kinship and locality, e.g., grandfather of Camana (a place), child of sister, brother (anyone in one’s male peer group), etc. Also, people have different ‘names’ throughout their lifetime. Before about age three, children are just known as ‘baby’. When it looks like they are going to survive, they are given a childhood name. Then when they come of age (ready to marry) they get their adult name. They may also have a \ili{Spanish}-based name if they are baptised. But none of these ‘names’ are really used all that much as names in the way Europeans use names. Maybe the Christian names come closest. [...] If you ask a \ili{Panare} person ‘What is your name?’ (in \ili{Spanish}) you would only get their Christian name in response. (\name{Thomas E.}{Payne}, p.c. 2020)
\end{quote}

\noindent To my knowledge all Tungusic cultures have personal names. As a rule, \ili{Russian} and \ili{Chinese} naming practices can also be found among speakers of Tungusic languages today. Culture-specific details, such as the use of derogatory names among the Manchus (\citealt{Alonso2012}) or the reference to rivers for the self-identification among the \ili{Evenki} (\citealt{Lavrillier2006}), seem to play no significant role for the expression of the PNQ among Tungusic languages. A discussion of specific meanings or functions of names goes beyond the scope of the present paper.

The PNQ could also be a universal or near-universal property, but is expressed differently from language to languages. Cross-linguistically, however, only a limited number of different constructional types is attested (e.g., \citealt[63–67]{Idiatov2007}). This section gives a brief overview of the typological variation attested in the expression of the PNQ emphasizing those aspects that are relevant for the classification of Tungusic (see also \citealt{Idiatov2007} and \citealt{Gil2018}).

The question ‘What is your name?’ is part of a question-answer sequence, such as in the following well-known \ili{Russian} dialogue of the explorer Vladimir Arsen'ev with his later friend Dersu Uzala, a member of the Tungusic-speaking Nanai people.

\ea
\label{example4.10}
\textit{Tebja kak zovut? Sprosil ja neznakomca.}\\
\textit{Dersu Uzala, otvečal on.}\\
“What is your name?” I asked the stranger.\\
“Dersu Uzala,” he answered. (\citealt{Arsenjev1921}, \citeyear[18]{Arsenjev2016})
\z

\noindent More specifically, the sequence consists of a content question with an interrogative, in this case \ili{Russian} \textit{kak} ‘how’ (see also \ref{example4.26}), that is taken up again in the elliptic answer in the form of a personal name, i.e. \textit{Dersu Uzala}.

Pragmatically speaking, there are, of course, many different ways of achieving the same overall meaning as a PNQ, for instance by using an imperative form of a speech act verb (e.g., \citealt[254]{Schulze2007}). The following is an example from the Tungusic language \ili{Evenki} (similar to \textit{State your name!}).

\ea
    \label{example4.11}
    \ili{Evenki} (Nedjalkov 1997: 148)\\
    \gll si.n-ngi-ve		gerbi-ve		mi.ne-ve		gu:-kel!\\
    2\textsc{sg.obl-gen-acc}	name-\textsc{acc}	1\textsc{sg.obl-acc}		say-2\textsc{sg.imp}\\
	\glt ‘Tell me your name!’
\z

\noindent In certain contexts, even the word \textit{Name!} alone could already be sufficient.

But not only is this much less polite than a question, but cross-linguistically it also is not the usual way of putting the question. \textsc{Conventionality} is key in the investigation of the personal name question. While every language is certainly capable of asking for the name of a person, the universal tentatively proposed here is that almost every language might have a conventional way of expressing it.

In some languages, such as \ili{German}, there are several different ways of putting the question. As in \ili{Evenki}, an imperative of a speech act verb can be used in certain contexts, for instance when giving vent to one’s impatience.

\ea
    \label{example4.12}
    \ili{German}\\
    \gll Sag			mir		(schon)		dein-en			Name-n!\\
    say.\textsc{imp.sg}	\textsc{1sg.dat}	already		2\textsc{sg.gen-m.sg.acc}	name-\textsc{m.sg.acc}\\
    \glt ‘Tell me your name (already)!’
    \z

Given a certain context, it is also possible to jokingly ask whether somebody actually has a name. Because we know that (in our culture) everybody has a name, we draw the conclusion, by means of pragmatic inference and the intention of being informative, that the appropriate answer to the question is the specific name rather than the answer yes.

\ea
    \label{example4.13}
    \ili{German}\\
    \gll Hast			du		ein-en		Name-n?\\
    have.2\textsc{sg.prs.ind}	2\textsc{sg}		a-\textsc{m.sg.acc}	name-\textsc{m.sg.acc}\\
    \glt ‘Do you have a name?’
    \z

However, \ili{German} has two more conventional ways of expressing the question (\ref{example4.14}) that in most situations would be preferred to the stylistically marked ones above.

\ea
    \label{example4.14}
    \ili{German}\\
    \ea
    \gll Was/Wie	ist		dein				Name?\\
    what/how		is		2\textsc{sg.gen.m.sg.nom}	name.\textsc{m.sg.nom}\\

    \ex
    \gll Wie		heiß-t				du?\\
    how		be.called-2\textsc{sg.prs.ind}	2\textsc{sg}\\
    \z
\z

Conventionality could theoretically be measured by text frequency, but, given that there are no large corpora for Tungusic languages, this method is inapplicable. Most texts that are available to me only contain the question too few times (if at all) to allow any conclusions. The pragmatic approach followed in this study is mostly impressionistic. It is based on the information available in grammar books, dictionaries, some texts, and the information from experts on individual languages.

Cross-linguistically, there are two main ways of expressing this special type of content question that correspond to the two most conventional expressions in \ili{German} above (\ref{example4.14}). Consider the following examples from \ili{Mandarin} and their \ili{English} translation:\pagebreak

\ea
    \label{example4.15}
    \ili{Mandarin} (\ili{Sino-Tibetan})\\
    \ea
    \gll [nǐ	de		míngzi]		shì		shénme?\\
    	2\textsc{sg} 	\textsc{attr}		name	\textsc{cop} what\\
    \glt ‘What is your name?’

    \ex
    \gll nǐ		jiào		[shénme		(míngzi)]?\\
    2\textsc{sg} 	call		what		(name)\\
    \glt ‘What are you called?’
    \z
\z

Both examples are directed at a second person and contain an interrogative. Example (\ref{example4.15}a) is a copula construction that equates ‘your name’ (the copula subject) with the interrogative (the copula complement, \citealt{Dixon2010a}) while example (\ref{example4.15}b) contains a speech act verb. These two types of constructions will be referred two as \textsc{Type A} and \textsc{Type B}, respectively.

Both patterns have several subtypes. Type A, for instance, can take at least two different forms in which the interrogative is either used as an argument of its own (your name = what, see \ref{example4.16}) or as an attribute of the noun meaning ‘name’ (you = what name, see \ref{example4.17}). These will be referred to as \textsc{Type A.1} and \textsc{Type A.2}.

\ea
    \label{example4.16}
    Jammu/Kashmir \ili{Burushaski} (\ili{Burushaski}; \citealt[148]{Munshi2006})\\
    \gll [um-e		gu-yik]			besen	d̪̪ila?\\
    2\textsc{sg-gen}		2\textsc{sg.poss}-name		what		be.\textsc{prs.3sg[abst]}\\
\ex
    \label{example4.17}
    \ili{Sanjiazi} Manchu (Tungusic; \citealt[239]{Dai2012})\\
    \gll ˈɕi			[ˈai			ˈkəvə]?\\
    2\textsc{sg}			what			name\\
\z

Both types of the personal name question refer to a \textsc{second person}. In many languages, this is overtly marked by a personal pronoun (both types), a possessive marker that also encodes person (especially Type A, see \ref{example4.18}), or verbal agreement (especially Type B, see \ref{example4.19}).

\ea
    \label{example4.18}
    \ili{Nungon} (\ili{Trans-New Guinea}; \citealt[469]{Sarvasy2017}: 469)\\
    \gll gok		maa-ya			numa?\\
    2\textsc{sg}		name-2\textsc{sg.poss}		who\\
\ex
    \label{example4.19}
    \ili{French} (\ili{Indo-European})\\
    \gll Comment	t’=appelles=tu?\\
    how			2\textsc{sg.obl}=call.2\textsc{sg.prs.ind=2sg}\\
    \z

\pagebreak In languages with egophoricity, second person can also be encoded indirectly with the help of the \textit{anticipation rule} (\citealt[245]{TournadreLaPolla2014}). In such languages, an egophoric marker usually refers to a first person, but in questions can also refer to a second person because the perspective of the addressee is taken.

\ea
    \label{example4.20}
    \ili{Karlong Mongghul} (\ili{Khitano-Mongolic}; \citealt{Faehndrich2007}: 114)\\
    \gll ʨɨ-nɨ		nara		ja:n-\textbf{i:}?\\
    2\textsc{sg-gen}		name		what-\textsc{ego}\\
\ex
    \label{example4.21}
    \ili{Wutun} (\ili{Sino-Tibetan}; \citealt{Sandman2016}: 295)\\
    \gll ni-de		minze-li		ma		sho-\textbf{yek}?\\
    2\textsc{sg-attr}		name-\textsc{loc}		what		say-\textsc{ego}\\
    \z

Among Tungusic languages, only \ili{Sibe} has been claimed to possess some sort of grammaticalized egophoric system (\citealt{Li1984}), but to my knowledge, this does not include any marking that would be relevant for the PNQ.

Both types of PNQs usually contain an \textbf{interrogative}. A potential exception to this generalization is the language \ili{Wari’} spoken in Brazil that uses demonstratives instead. \ili{Jahai} appears to make use of a polar question that also lacks an interrogative (see also \citealt{Gil2018}).

\ea
    \label{example4.22}
    \ili{Wari’} (\ili{Chapacuran}; \citealt{EverettKern1997}: 40)\\
    \gll \textbf{cain’}		\textbf{cain’}		ne	wixi-um		ne?\\
    that.\textsc{n.dist}	that.\textsc{n.dist}	3\textsc{n}	name-2\textsc{s}		\textsc{rec.pst}\\
\ex
    \label{example4.23}
    \ili{Jahai} (\ili{Austroasiatic}; \citealt{Schebesta1928}: 808, 821)\\
    \gll \textbf{ha}	kenmoˀ	pai?\\
    \textsc{q}	name	you\\
    \glt Literally: ‘Your name?’
    \z

For Tungusic, only examples with interrogatives are attested. As seen in the Introduction, the kind of interrogative in the name question also differs from language to language. Cross-linguistically, the two most common categories of interrogatives to be found in this question are \textsc{thing} (\textit{what}, e.g. \ili{English}) and \textsc{person} (\textit{who}, e.g. \ili{Tigre}, \ili{Pazih}), both of which are attested among Tungusic languages.

\ea
    \label{example4.24}
    \ili{Tigre} (\ili{Afroasiatic}; \citealt{Elias2014}: 227)\\
    \gll \textbf{man}	tu			səmetka?\\
    who		\textsc{cop.3sg.m}	name.2\textsc{sg.poss.m}\\
    \glt Literally: ‘Who is your name?’
\ex
    \label{example4.25}
    \ili{Pazih} (\ili{Austronesian}; \citealt{LiTsuchida2001}: 44, 46)\\
    \gll \textbf{ima}		langat	pai		siw?\\
    who		name	\textsc{q}		2\textsc{sg.nom}\\
    \z

This variation certainly has several causes, only some of which can be addressed here. In most Tungusic languages, the use of a given interrogative can be explained with language contact. But this does not explain why different interrogatives can be used in the first place.

Table \ref{table:4.2} sketches what can be assumed to be some prototypical features of the two interrogatives from a cross-linguistic perspective, although there are language-specific boundaries (based on \citealt{Nau1999}: 148; \citealt{Croft2003}: 130; \citealt{Idiatov2007}: 18).\largerpage

\begin{table}
\begin{tabular}{ l l l l }
  \lsptoprule
        &   \textsc{person} (\textit{who})   &               &		\textsc{thing} (\textit{what})\\
  \midrule
referentiality  &	identification  &               &		classification\\
animacy         &	human           &	animate     &	    inanimate\\
word class      &   pronoun         &   proper name	&       common noun\\
  \lspbottomrule
\end{tabular}
\caption{Tentative prototypical combinations of features for ‘who’ and ‘what’. What is referred to as “word class” is not identical to \citegen{Idiatov2007} feature “expected answer” that is assumed to be “proper name” for ‘who’. Instead, this refers to the word class of the interrogative itself.}
\label{table:4.2}
\end{table}

The frequent use of ‘who’ in PNQs might be explained by the fact that it is a question about an identification of a specific person (\textit{Who are you? I’m Bill.}), but not a classification (\textit{What is that? That is an airplane.}). The two other features are located on well-known typological scales, i.e. pronoun > proper name > common noun and human > animate > inanimate. Perhaps because a PNQ asks about a proper name that is located in the middle of the first of these two scales, ‘who’ (often an interrogative pronoun) and ‘what’ (often an interrogative noun) can both be used. Another factor for the variation might be the ambiguous nature of the concept \textsc{name} itself. First, some languages, such as \ili{Great Andamanese}, treat a name as if it was a body part (\citealt{Abbi2013}: 80). Second, a name can also be metaphorically conceptualized as a \textsc{thing} that can be possessed (e.g., \textit{I \textbf{have} a book}/\textit{name}, \textit{\textbf{my} book}/\textit{name}). Third, a name can also metonymically stand for the \textsc{person} itself (e.g., \textit{I \textbf{am} Mike}). The first interpretation might allow both ‘who’ and ‘what’ (animate entity), the second favors the use of ‘what’ (inanimate entity), the last of ‘who’ (human being). This represents a slight difference with respect to \citegen[47]{Idiatov2007} account that assumes that a name generally is a type of \textsc{thing}.

The use of a \textsc{manner} (\textit{how}) or other interrogative, such as \textit{come} in \ili{Italian} or \textit{comment} in \ili{French}, is less frequent and can possibly be explained with avoidance (\citealt{Idiatov2007}: 61). This seems to be relatively frequent in southern, central and eastern Europe, but can also be found in other languages (e.g., \citealt{Gil2018}).\il{languages of Europe}

\ea
    \label{example4.26}
    \ili{Russian} (\ili{Indo-European})\\
    \gll \textbf{Kak}		tebja		zovut?\\
    how		2\textsc{sg.acc}		call.3\textsc{pl.prs.ind}\\
    \glt Literally: ‘How do they call you?’
\ex
    \label{example4.27}
    \ili{Italian} (\ili{Indo-European})\\
    \gll \textbf{Come}	ti			chiam-i?\\
    how		2\textsc{sg.obl}		call-2\textsc{sg.prs.ind}\\
    \glt Literally: ‘How do you call (yourself)?’
    \z

\noindent As will be shown in \sectref{section4.4}, many Tungusic languages appear to have calqued the use of a \textsc{manner} interrogative on the basis of \ili{Russian}, i.e. the European pattern spread towards the East.\il{languages of Europe}

An interrogative in both types of PNQs may be \textsc{focused}. Cross-linguistically, there are different means of focusing an interrogative. A strategy common, for instance, in \ili{Japonic} languages is the use of a morphosyntactic marker.

\ea
    \label{example4.28}
    \ili{Tarama Miyako} (\ili{Japonic}; \citealt{Aoi2015}: 417)\\
    \gll naa=ju=ba		\textbf{nuu}=ti=\textbf{ga}			ïï=ga?\\
    name=\textsc{acc=top}	what=\textsc{quot=foc}		say={q}\\
    \z

\noindent Except for, perhaps, \ili{Uilta}, this is not attested in the Tungusic PNQs. Another way of focusing the interrogative is through fronting, also called (full) \textit{wh-movement}, as in \ili{English}. In Northeast Asia, few languages exhibit this syntactic phenomenon. An indication of fronting is the comparison of the PNQ with its answer. If the personal name appears in the same position as the interrogative (i.e., \textit{in situ}), there is no fronting involved.

\ea
    \label{example4.29}
    \ili{English}\\
    \ea 
    \textit{\textbf{What} is [your name]?}\\
    \ex 
    \textit{[My name] is \textbf{Anna}.}\\
    \z
\ex
    \label{example4.30}
    \ili{Mandarin}\\
    \ea
    \gll [nǐ		de		míngzi]		shì		\textbf{shénme}?\\
    2\textsc{sg}		\textsc{attr}		name		\textsc{cop}		what\\

    \ex
    \gll [wǒ		de		míngzi]		shì		\textbf{ānnà}.\\
    1\textsc{sg}		\textsc{attr}		name		\textsc{cop}		\textsc{pn}\\
    \z
\z

\ili{Northern Tungusic} languages are among the very few exceptions with occasional sentence-initial interrogatives in Northeast Asia (\citealt{Dryer2013}; \citealt{Hölzl2018a}). \ili{Ewenic} languages also exhibit other focus positions that are more central for the PNQ. Some Tungusic languages have adopted the European pattern through \ili{Russian}.\il{languages of Europe}

Type A, and sometimes Type B also, contains a \textsc{dummy} noun meaning ‘name’. Obviously, there is no generalization on what phonological form this noun has cross-linguistically. It is necessary to distinguish between chance resemblance, a common inheritance, and mutual contact. \ili{German} \textit{Name} and \ili{English} \textit{name}, for instance, are similar due to a common \ili{Germanic} origin. The similarity to \ili{Uralic}, e.g. \ili{Finnish} \textit{nimi}, can perhaps best be explained by \ili{Indo-European} influence (e.g., \citealt{Anthony2007}: 95). In many other cases, similarities between individual words, such as \ili{Persian} \textit{nām}, \ili{Kurux} \textit{naːme}, \ili{Japanese} \textit{namae}, or \ili{Papuan Malay} \textit{nama}, is probably the result of chance.

\ea
    \label{example4.31}
    \ili{Papuan Malay} (\ili{Austronesian}; \citealt{Kluge2017}: 623)\\
    \gll kam		pu		\textbf{nama}	siapa{\textasciitilde}siapa?\\
    2\textsc{pl}		\textsc{poss}		name	who{\textasciitilde}\textsc{pl}\\
    \glt ‘What are your names?’
    \z

\begin{sloppypar}
In a few languages, the dummy noun can fuse with other elements. For instance, in the \ili{Austronesian} language \ili{Kilivila}, the dummy noun \textit{yaga} ‘name’ (\citealt{Senft1986}: 420) fused with an interrogative to form the complex stem \textit{amyaga-} ‘what is the name of’ (\citealt{Senft1986}: 187), which is the basis of the PNQ \textit{amyagam?} that contains a possessive marker \textit{-m} ‘2\textsc{sg.poss}’ (\citealt{Senft1986}: 52).
\end{sloppypar}

Interrogatives are often reinforced with other elements, such as basic nouns, e.g. \ili{Italian} \textit{che cosa} ‘what thing > what’ (e.g., \citealt{Diessel2003}; \citealt{Hölzl2018a}). \ili{Tok Pisin} \textit{wanem} ‘what’ seen in (\ref{example4.5}) is a contraction of \ili{English} \textit{what} and \textit{name} (\citealt{Wurm1985}: 210). This reinforcement suggests that the concept \textsc{name} is considered, at least by the speakers of this language, a very basic category equivalent to \textsc{thing}.

Depending on the grammar of the individual languages, the dummy noun can belong to a certain class (e.g., animacy, gender, noun class). For instance, it has male gender in \ili{German} and in the following construction in the Sepik language \ili{Abau}. In the South American language \ili{Panare}, it is marked for inanimateness and invisibility.

\ea
    \label{example4.32}
    \ili{Abau} (Sepik; \citealt{Lock2011}: 227)\\
    \gll hwon-o		uru		po-\textbf{ho}?\\
    2\textsc{sg-gen}		name 	\textsc{q-gl.m}\\
\ex
    \label{example4.33}
    \ili{Panare} (Cariban; \citealt{Payne2013}: 66)\\
    \gll sïnka	\textbf{mën}	a-yiche?\\
    how		\textsc{in.invis}	2-name\\
    \z

\noindent In Tungusic, there is no such classification of the dummy noun.

Some languages have more than one dummy that can enter the question. In Standard \ili{Korean}, for instance, there is a distinction between neutral \textit{ilum} and honorific \textit{sengham} (\citealt{Song2005}: 95).

\ea
    \label{example4.34}
    \ili{Korean} (\ili{Koreanic}; \citealt{Sohn1999}: 418)\\
    \ea
    \gll \textbf{ilum}	i		mwe	yey-yo?\\
    name	\textsc{nom}		what	be-\textsc{pol}\\

    \ex
    \gll \textbf{sengham}	i		ettehkey		toy-sey-yo?\\
    name.\textsc{hon}		\textsc{nom}		how			become-\textsc{hon-pol}\\
    \glt Literally: ‘How does your name become?’
    \z
\z

\noindent In this language, the two nouns are part of different constructions. Example (\ref{example4.34}a) is said to a child or teenager and (\ref{example4.34}b) is the honorific version. Individual Tungusic languages only have one dummy noun.

An additional distinction in Type A is whether languages make use of an overt \textsc{copula} or not. While some languages, such as \ili{Sumerian} (\ref{example4.35}), require an overt copula, others, such as \ili{Kurux} (\ref{example4.36}) and many Tungusic languages, do not.

\ea
    \label{example4.35}
    \ili{Sumerian} (\citealt{Black2007}: 21)\\
    \gll aba-\textbf{m}			mu-zu?\\
    what-3\textsc{sg.cop}		name-2\textsc{sg.poss}\\
\ex
    \label{example4.36}
    \ili{Kurux} (\ili{Dravidian}; \citealt{Kobayashi2017}: 242)\\
    \gll niŋg-hay	ender	naːme?\\
    2\textsc{sg-gen} 		what		name\\
    \z

In Type A languages, there is an additional \textsc{possessive relationship}, which, depending on the language, can be dependent-marked (e.g., \ili{Mongsen Ao}, \ref{example4.37}), head-marked (e.g., \ili{Teiwa}, \ref{example4.38}), double marked (e.g., \ili{Turkish}, \ref{example4.39}), or unmarked (e.g., \ili{Nihali}, \ref{example4.40}).\footnote{The PNQ in \ili{Mongsen Ao} can also be expressed with ‘what’.}

\ea
    \label{example4.37}
    \ili{Mongsen Ao} (\ili{Sino-Tibetan}; \citealt{Coupe2007}: 99)\\
    \gll \textbf{nə}			tə-niŋ		sə́páʔ?\\
    2\textsc{sg.poss}		\textsc{rl}-name		who\\
\ex
    \label{example4.38}
    \ili{Teiwa} (\ili{Trans-New Guinea}; \citealt{Klamer2010}: 239)\\
    \gll \textbf{ha}-yit			amidan?\\
    2\textsc{sg.poss}-name		what\\
\ex
    \label{example4.39}
    \ili{Turkish} (\ili{Turkic})\\
    \gll sen-\textbf{in}		ad-\textbf{ın}			ne?\\
    2\textsc{sg-gen}		name-2\textsc{sg.poss}		what\\
\ex
    \label{example4.40}
    \ili{Nihali} (\citealt{Nagaraja2014}: 116)\\
    \gll ne			jumu	na:n?\\
    2\textsc{sg}		name	what\\
\z


\noindent All four types are attested in Tungusic.

In those languages that have \textsc{possessive classification}, there is an additional distinction that refers to the class of the word for ‘name’. In \ili{Mongsen Ao}, for example, the “relational prefix” \textit{tə-} that is seen in (\ref{example4.37}) is usually found on body parts and kinship terms (\citealt{Coupe2007}: 84). In \ili{Mandarin}, \textit{míngzi} ‘name’ belongs to the set of nouns that is obligatorily possessed with a genitive marker \textit{de}. This marker can be absent with kinship terms. A language that makes a distinction into several different possessive classes is \ili{Great Andamanese}.

\ea
    \label{example4.41}
    \ili{Great Andamanese} (\citealt{Abbi2013}: 181, 270)\\
    \gll ŋ=\textbf{er}=liu			\textbf{a}=ʃyu		bi?\\
    2\textsc{sg=cl2}=name	\textsc{cl}1=who		\textsc{cop}\\
    \z

In this language, the word \textit{liu} ‘name’ takes the class 2 possessive marker \textit{ɛr=} {\textasciitilde} \textit{er=} (\citealt{Abbi2013}: 80, 140, 161) that otherwise attaches to “major body parts that pertain to the ‘head’, ‘brain’, ‘neck’, ‘face’, ‘arms’, ‘thigh’, ‘calf’, ‘knee’ and ‘bones.’” (\citealt{Abbi2013}: 141). In addition, the personal interrogative has the class 1 possessive marker \textit{a-} also found on words referring to the mouth and kinship terms, such as mother. As will be shown below, the Tungusic possessive classification marker cannot enter the PNQ.

In Type B constructions, there is variance in the type of \textsc{speech act verb} that is involved. Apart from the language-specific semantics, the most important variation concerns the valency of the verb. In \ili{German}, \textit{heißen} ‘to be called’ is an intransitive verb and \textit{nennen} ‘to call’ is a transitive verb. In \ili{Mandarin}, \textit{jiào} is an ambitransitive verb that can be either intransitive or transitive (Table \ref{table:4.3}).

\begin{table}
\begin{tabular}{ l l l }
  \lsptoprule
& Intransitive (+ \textsc{name}) &	Transitive (+ \textsc{name})\\
  \midrule
\ili{German}      &	heißen  &	-\\
            &	-       &	nennen\\
\ili{Mandarin}    &   jiào    &	jiào\\
  \lspbottomrule
\end{tabular}
\caption{Valency of speech act verbs in German and Mandarin. In \ili{German}, the transitive or causative use of \textit{heißen} is archaic.}
\label{table:4.3}
\end{table}

\ea
    \label{example4.42}
    \ea
    \ili{Mandarin} (\ili{Sino-Tibetan})\\
    \gll wǒ		\textbf{jiào}		ānnà.\\
    1\textsc{sg}		call		\textsc{pn}\\
    \glt ‘I am called Anna.’ (intransitive)

    \ex
    \gll tā		\textbf{jiào}		wǒ		ānnà.\\
    3\textsc{sg}		call		1\textsc{sg}		\textsc{pn}\\
    \glt ‘(S)he calls me Anna.’ (transitive)
    \z
\z

\noindent \ili{English} requires a passive, a reflexive, or a third person plural dummy agent in order to use the verb to call as an intransitive verb, e.g. \textit{he is called Joe}, \textit{he calls himself Joe}, \textit{they call him Joe}. A reflexive or a passive of a speech act verb are also possible in \ili{German}.

\ea
    \label{example4.43}
    \ili{German} (\ili{Indo-European})\\
    \gll Wie		nenn-st			du		dich?\\
    how		call-2\textsc{sg.prs.ind}	2\textsc{sg}		2\textsc{sg.refl}\\
    \glt Literally: ‘How do you call yourself?’\footnote{This is identical to the original of the question in the quotation from Goethe above.}
\ex
    \label{example4.44}
    \ili{German} (\ili{Indo-European})\\
    \gll Wie		wirst			du		genannt?\\
    how		get.2\textsc{sg.prs.ind}	2\textsc{sg}		called\\
    \glt Literally: ‘How are you called?’
\z

\noindent An impersonal construction is also attested in other languages with Type B constructions.

\ea
    \label{example4.45}
    \ili{Beng} (\ili{Mande}; \citealt{Paperno2014}: 17)\\
    \gll ouo				mi		si		\textbf{po}?\\
    3\textsc{pl.hab.aff}		2\textsc{sg}		call.\textsc{l}	what\\
    \glt Literally: ‘What do they call you?’
\z

\noindent Changing of valency, reflexives or impersonals are not attested in the few cases of Type B constructions in Tungusic.

\textsc{Politeness} is a dimension of variation that plays a larger or smaller role for both types of PNQs depending on the language. In \ili{German}, there is a two-way politeness distinction that affects the choice of the pronoun and, consequently, the verbal ending. Instead of the usual \textit{du} ‘you (\textsc{sg})’, the polite pronoun \textit{Sie} ‘you (\textsc{sg.pol})’ is used. Both have suppletive case forms.

\ea
    \label{example4.46}
    \ili{German} (\ili{Indo-European})\\
    \gll Wie		heiß-en					Sie?\\
    how		be.called-2\textsc{sg.prs.ind.pol}	2\textsc{sg.pol}\\
\ex
    \label{example4.47}
    \ili{German} (\ili{Indo-European})\\
    \gll Was/Wie	ist		Ihr					Name?\\
    what/how		is		2\textsc{sg.pol.gen.sg.m.nom}	name.\textsc{sg.m.nom}\\
\z

While \ili{German} makes use of the same two constructions, there are languages that change the whole construction according to the politeness register. Two such languages that had contact with Tungusic languages are \ili{Korean} (see above) and \ili{Mandarin}. \ili{Mandarin}, apart from the other expressions mentioned throughout this section has the following honorific form that is based on a different pattern.

\ea
    \label{example4.48}
    \ili{Mandarin} (\ili{Sino-Tibetan})\\
    \gll nín			guì			xìng?\\
    2\textsc{sg.hon}		honorable		surname\\
\z

In \ili{Koreanic} languages, apart from the use of a different construction seen above, there is also a distinction in the question marker.

\ea
    \label{example4.49}
    Soviet \ili{Korean} (\ili{Koreanic}; \citealt{King1987}: 253, 269, slightly adjusted)\\
    \ea
    \gll irimi	misi-ge-ja?\\
    name	what-thing-\textsc{q.plain}\\

    \ex
    \gll irimi	misi-ge-mdu?\\
    name	what-thing-\textsc{q.pol}\\
    \z
\z

Politeness could also have led to some exceptions from the proposed universal that all languages have a conventionalized way of expressing the PNQ. Jiaomuzu \ili{Gyalrong} in China, for instance, tends “to avoid direct address”, including questions about names. However, even in this language it is possible to ask a PNQ in a polite way:

\ea
    \label{example4.50}
    Jiaomuzu \ili{Gyalrong} (\ili{Sino-Tibetan}; \citealt{Prins2017}: 343)\\
    \gll nənɟo	tʰi		tə-rɲu-n			ko?\\
    2\textsc{sg}		what		2-be.called-2\textsc{sg}		\textsc{anx}\\
    \glt ‘Please, do tell me what is your name?’
    \z

\noindent Overall, Tungusic languages have few grammaticalized expressions for politeness.\footnote{While some \ili{Koreanic} question markers that show politeness disctinctions were possibly borrowed by the \ili{Jurchenic} branch of Tungusic (\citealt{Hölzl2018a}: 213), their exact function in \ili{Jurchenic} still remains unclear.}


\section{The personal name frame}\label{section4.3}

The semantic side of a construction, like that of a lexical item, can be represented by what is often referred to as a frame (e.g., \citealt{Fillmore1985}). This section introduces the \textit{personal name frame} (PNF) that could be the basis for the personal name question. This frame can be illustrated with dialogues from the Tungusic language \ili{Sibe}.

\ea
    \label{example4.51}
    \ili{Sibe} (\citealt{Jin1993}: 3)\\
    \ea
   \textbf{ tʂunfu/Chunfu:}\\
    \gll ɕi		χodʐ=na?\\
    2\textsc{sg}		good=\textsc{q}\\
    \glt ‘How are you?

    \ex
    \gll nənə-m			mi.n-j			bəji-v  bəji-d		əmdan	taqə-vɨ-ki.\\
    first-\textsc{cvb.ipfv}		1\textsc{sg.obl-gen}		self-\textsc{acc}   self-\textsc{dat}		once		know-\textsc{caus-des}\\
    \glt ‘First, let me introduce myself.’

    \ex
    \gll mi.n-b			tʂunfu	ʂɨ-m.\\
    1\textsc{sg.obl-acc}		\textsc{pn}		say-\textsc{ipfv}\\
    \glt ‘My name is Chunfu.’

    \ex
    \textbf{tʂaŋmiŋ/Changming:}\\
    \gll ɕi.n-b			taqə-m			mutu-xuŋ	bi		ursun   bailə-m.\\
    2\textsc{sg.obl-acc}		know-\textsc{cvb.ipfv}		can-\textsc{ptcp.pfv}	1\textsc{sg}		very    rejoice-\textsc{ipfv}\\
    \glt ‘I am very happy to meet you.’

    \ex
    \gll mi.n-j			gəvə-v		tʂaŋmiŋ	ʂɨ-m.\\
    1\textsc{sg.obl-gen}		name-\textsc{acc}		\textsc{pn}		say-\textsc{ipfv}\\
    \glt ‘My name is Changming.’
    \z
\z

\noindent While this brief dialogue does not contain the personal name question, it is arguably located in a very similar type of situation. While the direct question about the name is avoided by Chunfu, Changming, by means of pragmatic inference, draws the conclusion that, given Chunfu’s introduction, it is appropriate to say one’s own name in response. In a similar albeit more direct way, one can add a truncated question at the end of one’s own introduction:

\ea
    \label{example4.52}
    \ili{Sibe} (Jin Ning 1993: 3)\\
    \gll mi.n-j		gəvə-v		sarasu	ʂɨ-m.	\textbf{ɕi}	\textbf{ni}?\\
    1\textsc{sg.obl-gen}	name-\textsc{acc}		\textsc{pn}		say-\textsc{ipfv}	2\textsc{sg}	\textsc{q}\\
    \glt ‘My name is Sarasu. What’s yours?’
    \z

\noindent As another example consider the following dialogue:

\ea
    \label{example4.53}
    \ili{Sibe} (\citealt{Jin1993}: 4)\\
    \ea
    \textbf{dʐaluʂan/Zhalushan:}\\
    \gll ɕi		mi.n-d			əmdan	taqə-və-∅!\\
    2\textsc{sg}		1\textsc{sg.obl-dat}		once		know-\textsc{caus-imp}\\
    \glt ‘Would you introduce me to him please?’

    \ex
    \textbf{bəkdəsu/Bekdesu:}\\
    \gll bi		so.n-j			dʐu		nanə-v		əmdan	taqə-vɨ-ki.\\
    1\textsc{sg}		2\textsc{pl.obl-gen}		two		person-\textsc{acc}	once	know-\textsc{caus-des}\\
    \glt ‘Allow me to introduce you.’

    \ex
    \gll ər	əmkən=ni		ɢoɕiŋa	sɨ-m			{\normalfont [...]}\\
    this	one=3\textsc{sg.poss}		\textsc{pn}		say-\textsc{ipfv}\\
    \glt ‘This is Gosinga.’
    \z
\z

In this case, the situation involves not two, but three persons. Apart from the two people making the acquaintance (Zhalusan and Gosinga), there is a third mediating person (Bekdesu).

All three situations above are based on the common background knowledge that everybody has a name. The same is obviously true for the personal name question. But this is only part of the larger personal name frame that contains several subevents and roles tentatively listed in Table \ref{table:4.4}.\footnote{The list presented in Table \ref{table:4.4} is probably not exhaustive and the individual subevents could be slightly different depending on the cultural background. For instance, in some societies names can also be removed from a person (e.g., \citealt{Moutu2013}: 147). Apart from giving, a name can evolve through a process known as onymization (\citealt{VanLangendonck2016}: 33). Future studies will have to revise the personal name frame accordingly.}

\begin{table}
\begin{tabular}{ l l l }
  \lsptoprule
        &	Subevents                   &	Roles\\
  \midrule
1   &	\textsc{giving a name}                   &	\textsc{namee}, \textsc{name}, \textsc{namer}\\
2   &	\textsc{having a name}                   &	\textsc{possessor}, \textsc{name}\\
3   &	\textsc{acquainting}	\\
3a  &	2 persons: \textsc{asking, telling}      &	\textsc{asker}, \textsc{addressee}, \textsc{name}\\
3b  &	3 persons: \textsc{introducing}          &	\textsc{introducer}, \textsc{person} A,\\
    &                                   &   \textsc{person} B, \textsc{name}s\\
4   &	\textsc{knowing a name}                  &	\textsc{knower}, \textsc{known}, \textsc{name}\\
    &   (common ground)                 &\\
5   &	\textsc{calling by name}                 &	\textsc{caller}, \textsc{called}, \textsc{name}\\
  \lspbottomrule
\end{tabular}
\caption{The personal name frame and its subparts. The dummy noun meaning ‘name’ is not listed, but is optionally present in all subevents (based on \citealt{Hölzl2014})}
\label{table:4.4}
\end{table}

First, most people do not usually chose their names on their own, but are given the name by somebody else, such as their parents. In this case, there are three different roles, the person giving the name (\textsc{namer}), the personal name given (\textsc{name}), and the person being named (\textsc{namee}). There are culture- and language-specific conventions and examples for each of these subevents. In this case, this could be a baptism, the acceptance of a new name during a religious initiation, or the change of one’s own name in court.

Second, everybody has or owns a name. Here the roles are the person having the name (\textsc{possessor}), and the name (\textsc{name}). Cross-linguistically, this frame is usually expressed with possessive relationships, e.g. \textit{\textbf{her} name} (attributive possession), \textit{she \textbf{has} a beautiful name} (predicative possession). But because a name is not a concrete and tangible object, these expressions are based on an underlying conceptual metaphor that \textsc{ideas are objects} (\citealt{Lakoff1999}: 124f.). This can also be seen in other expressions, e.g. \textit{my plan} or \textit{to have a plan}.\footnote{In addition, the conceptual metonymies that the \textsc{name stands for a person} and that the \textsc{face stands for a person} are often combined with this, e.g. in a passport. For instance, when looking at a photo of a person’s face it is possible to say \textit{This is Sam}.} A culture-specific case can be found among the \ili{Iatmul} in Papua New Guinea who “believe that there is a mystical connection between a name and its bearer” (\citealt{Moutu2013}: 147).

Third, there are at least two subevents for making the acquaintance of a person that correspond to the two dialogues from \ili{Sibe} above. These include either two persons (three roles: \textsc{asker}, \textsc{addressee}, \textsc{name}) or three persons (four roles: \textsc{introducer}, \textsc{person} A, \textsc{person} B, \textsc{name}s).

Fourth, after giving a name or after having made the acquaintance of a person, one has the knowledge of that person’s name. This subevent has three roles, the person knowing the name (\textsc{knower}), the person whose name is known (\textsc{known}), and the name (\textsc{name}). Knowing other people’s names is part of the common ground. Forgetting somebody’s name can lead to severe social awkwardness. Depending on the society, a certain amount of control can for instance be associated with knowing a person’s name.

Fifth, when knowing a person’s name, one (the \textsc{caller}) can refer to that person (\textsc{called}) by his or her name (\textsc{name}), either in a direct address (vocative) or in the third person. The name theoretically identifies the exact individual. Depending on the type of naming in a given culture, namesakes can lead to more or less problems (see \citealt{Moutu2013}: 145ff. for an extreme example). Conversely, one person can have several different names. In certain cases, uttering a specific name can be a taboo.

The PNQ is part of the acquainting subevent, more specifically subevent 3a, but is based on several aspects of the personal name frame. Questions of Type A combine 3a with subevent 2 (\textsc{having a name}), and Type B with subevent 5 (\textsc{calling by name}). There is a mapping of the roles of the two combined subevents (Table \ref{table:4.5}). In addition to the roles, the three subevents also contain semantic relations not specified above that can be indicated as \textsc{ask} (a type of question), \textsc{call} (a form of speech act), and \textsc{have} (a possessive relationship), respectively.

\begin{table}
\begin{tabular}{ l l l l }
  \lsptoprule
Type    &	Subevent 1  &	Subevent 2      &	Combined Roles/Relations\\
  \midrule
Type A  &	\textsc{acquainting} &   \textsc{having a name}   &	\textsc{asker} = ∅,\\
        &   (2 persons)	&                   &   \textsc{addressee} = \textsc{possessor}\\
        &               &                   &   \textsc{name} = \textsc{name}\\
        &               &                   &   \textsc{ask} + \textsc{have}\\
Type B  &   \textsc{acquainting} &   \textsc{calling by name} &	\textsc{asker} = \textsc{caller}\\
        &   (2 persons) &                   &   \textsc{addressee} = \textsc{called}\\
        &               &                   &   \textsc{name} = \textsc{name}\\
        &               &                   &   \textsc{ask} + \textsc{call}\\
  \lspbottomrule
\end{tabular}
\caption{Combinations of subevents and roles in the two main PNQ types}
\label{table:4.5}
\end{table}

Using \citegen[66]{Langacker2008} terminology, one could say that different PNQs highlight or \textit{profile} different aspects of the underlying frame that functions as a base. For instance, even though Type B does not necessarily refer to subevent 2 (\textsc{having a name}), a speaker must still be aware of it in order to ask the question in the first place.


\section{‘What’s your name?’ in Tungusic}\label{section4.4}

The question ‘What is your name?’ has been recorded for the majority of the Tungusic languages and in a considerable number of dialects. To the best of my knowledge, the PNQ is not documented in \ili{Arman}, \ili{Bala}, \ili{Lalin}/\il{Manchu!Jing}Jing Manchu, the two \ili{Jurchen} varieties, and \ili{Kili} (Kur-Urmi Nanai). However, for all these languages, similar constructions or at least individual words, such as ‘name’ are attested. Only for \ili{Chinese} \ili{Kyakala} there is no information on the PNQ at all.

As expected, Tungusic languages show a certain amount of variation in how they express the question. Nevertheless, all constructions exhibit a cognate of the Tungusic word for ‘name’. This word functions as some kind of \textit{anchor} around which all PNQs are built. One example with the optional \ili{Mongolic} word apart, no other word for ‘name’ is attested in these constructions. This lexical item is addressed in \sectref{section4.4.2}.


\subsection{Second person pronoun and genitive}\label{section4.4.1}

All Tungusic languages preserve cognates of \ili{Proto-Tungusic} *\textit{si} ‘you (\textsc{sg})’ (e.g., \citealt{Benzing1956}: 109). There are some well-known phonological changes, such as \textit{s} > \textit{ɕ} before \textit{i} in some \ili{Jurchenic} varieties, or \textit{s} > \textit{h} in some \ili{Even} dialects. The \textsc{personal pronoun} can often be absent and is less central for the personal name question. Apart from \ili{Jurchenic}, Tungusic languages also employ a grammaticalized version of this personal pronoun as possessive marker as in the following example from \ili{Ulcha} (\ref{example4.54}) (see \citealt{Ikegami1985} for details):

\ea
    \label{example4.54}
    \ili{Ulcha} (\citealt{Angina1993}: 3)\\
    \gll si(ə)	gəlbu-\textbf{si}			nguj?\\
    2\textsc{sg}		name-2\textsc{sg.poss}		who\\
    \z

In \ili{Proto-Tungusic}, the personal pronoun *\textit{si} has an oblique form *\textit{si.n-}, for example for the genitive *\textit{si.n-i}. The presence of the \textit{-n-} in oblique forms is a phenomenon found throughout the pronominal system of Tungusic and neighbouring languages, such as \ili{Mongolic}. The genitive is retained, for example, in \ili{written Manchu} \textit{si.n-i} ‘2\textsc{sg.obl-gen}’ and \textit{suwe.n-i} ‘2\textsc{pl.obl-gen}’. In some languages the genitive \textit{-i} changed to \textit{-u} in the plural pronouns due to a progressive vowel assimilation, e.g. \ili{Uilta} \textit{si.n-\textbf{i}} ‘your (\textsc{sg})’, but \textit{su.n-\textbf{u}} ‘your (\textsc{pl})’ (\citealt{Tsumagari2009b}: 7). In a few languages, for example in \ili{Even} (\textit{hi.n}) and \ili{Bala} (\textit{ɕi.n}), the oblique form was retained in genitive function, although the genitive itself was lost. In several other languages, such as \ili{Udihe}, the genitive was functionally lost, but still functions as a stem for the possessive forms, e.g. \textit{si.n-\textbf{i}-ŋi} ‘yours (\textsc{sg})’, \textit{su.ñ-\textbf{u}-ŋu} ‘yours (\textsc{pl})’ (\citealt{NikolaevaTolskaya2001}: 336). In some languages, possessive forms of this sort developed a meaning similar to a genitive (e.g., \ref{example4.11}), which led to a probably erroneous reconstruction of the genitive in \citet[79]{Benzing1956}.


\subsection{The Tungusic word for ‘name’}\label{section4.4.2}

Traditionally, the Tungusic word for ‘name’ is reconstructed as *\textit{gärbü} (\citealt{Benzing1956}: 49). While this reconstruction is reasonably robust, it is slightly misleading as the reconstructed *\textit{ä} must actually have been pronounced as schwa [ə], as in the majority of the modern languages. \citet[40]{Janhunen1991}, perhaps based on \il{Evenki!Khamnigan}Khamnigan Evenki \textit{gərbii}, reconstructs Tungusic *\textit{gerbüü} with a long vowel in the second syllable. While a long vowel can also be found in other \ili{Evenki} dialects, for example \ili{Sakhalin} Evenki \textit{gərbī} (\citealt{BulatovaCotrozzi2004}) or \ili{Nercha} Evenki \textit{gərbī} \parencitetv{chapters/05}, this seems to be an innovation rather than a retention. Cognates of *\textit{gärbü} ‘name’ are collected, among others, in \citet{Schmidt1923a, Schmidt1923b, Schmidt1928a, Schmidt1928b}, \citet[49]{Benzing1956}, \citet[180f.]{Cincius1975}, \citet[143]{Lie1978}, \citet[68]{Kazama2003}, \citet[336]{Doerferetal2004}, or \citet[300f.]{Chaoke2014eH}.

The earliest recordings of Tungusic are in \ili{Jurchen}, which is a cover term for at least two different varieties that, for lack of better terms and in analogy to similar cases such as \ili{Tocharian}, can be called \ili{Jurchen} A (*\textit{gebu} \zh{革卜}, \citealt{Kiyose1977}) and \ili{Jurchen} B (*\textit{gebu} \zh{革不}, \citealt{Kane1989}). The word recorded for these two varieties of \ili{Jurchen} are identical to \ili{written Manchu} \textit{gebu}, which is attested from the 17th century onward (e.g., \citealt{Norman2013}). Apart from \ili{Jurchen} and \ili{Manchu}, some of the oldest records of the word for ‘name’ have been made for \ili{Evenki} and \ili{Even}. For instance, at the beginning of the 18th century \citet[654]{Witsen1705} mentioned \ili{Evenki} \textit{gerbisch} ‘your name’, which can be analyzed as \textit{gerbi-ʃ} ‘name-2\textsc{sg.poss}’. \citet[169]{Pallas1786} listed \textit{gorbi}/горби for \ili{Evenki} dialects and \textit{gerbi-nʺ}/гербинъ for \ili{Even}. A form \textit{garbi-n} was recorded in 1808 by Koshewin (\citealt{Klaproth1817}: 224). To mention but some more examples, the word has been recorded as \textit{gärbî} or \textit{garbi-n} in 1810 by Spassky (\citealt{Castrén1856}: 107, 128). Schiefner already correctly equated \ili{Evenki} \textit{gärbî} with \ili{Manchu} \textit{gebu} (see \citealt{Castrén1856}: x). Two of the earliest recordings of the word in \ili{Nanai} (specifically the Ussuri dialect)\il{Ussuri} in the 19th century are \textit{gerbi}/герби or \textit{gerbu}/гербу (\citealt{Brylkin1861}) and \textit{gorbi-ni} (\citealt{Venjukov1862}; \citealt{Alonso2011}: 20). The \ili{Nanai} form \textit{ǵerbú} listed in the dictionary by \citet{Grube1900} was also collected around the middle of the 19th century. For many other languages, data are only available from the 20th century onward.

The reconstructed *\textit{ü} in *\textit{gärbü} ‘name’ underwent a regular sound change to \textit{i} in \ili{Northern Tungusic} languages (\ili{Ewenic} and \ili{Udegheic}) and to \textit{u} in \ili{Southern Tungusic} (\ili{Nanaic} and \ili{Jurchenic}), e.g. \ili{Oroqen} \textit{gərbi}, \ili{Oroch} \textit{gəbbi}, but \ili{Nanai} \textit{gərbu}, \ili{Manchu} \textit{gebu}. The same sound change can be seen in the interrogative *\textit{ŋüi} ‘who’, e.g. \ili{Oroqen} \textit{nii}, \ili{Oroch} \textit{n’ii}, but \ili{Nanai} \textit{uj} (\ili{Uilta} \textit{ŋui}), \ili{Manchu} \textit{we} (see also \citealt{Hölzl2018a}: 314). Only \ili{Even} (\textit{gərbə}), \ili{Arman} (\textit{gerbụ}, \textit{gurbu}), and one recording of \ili{Oroqen} or \ili{Solon} (\textit{gerbu} in \citealt{Ivanovskiy1982}: 1) might represent special cases in \ili{Northern Tungusic}. However, other recordings of \ili{Oroqen} and \ili{Solon} as well as the \ili{Even} form \textit{gerbi-} recorded by \citet[169]{Pallas1786} contain the expected \textit{i} (cf. also \ili{Arman} \textit{ŋii} and \ili{Even} \textit{ŋi(i)} ‘who’). Apart from that, there have been several language-specific developments. The \textit{r} has been, probably regularly, lost in \ili{Jurchenic} (e.g., \ili{Lalin}/\il{Manchu!Jing}Jing Manchu \textit{gəbu}) and changed to \textit{l} in several languages around the lower Amur, including \ili{Uilta} (\textit{gəlbu}), \ili{Ulcha} (\textit{gəlbu}), and \ili{Lower Negidal} (\textit{gölbi} [gəlbi], \citealt{Schmidt1923a}: 18, \textit{gilbi} with additional regressive vowel assimilation, \citealt{KhasanovaPevnov2003}: 7). The \textit{l} is already attested in data collected at the beginning of the 20th century, i.e. \ili{Uilta} \textit{gylbṓ-ni}/\textit{gylbú(-ni)}, \ili{Ulcha} \textit{gýlbu} in Piłsudski (\citealt{Majewicz2011}: 258, 817) and \ili{Ulcha} \textit{gölbu} [gəlbu] in \citet[251]{Schmidt1923b}. The consonant cluster *\textit{rb}, possibly via *\textit{lb}, developed into a cluster \textit{db} in \ili{Upper Negidal} (\textit{gədbi}, \name{Natalia}{Aralova} p.c. 2019), \textit{gb} in \ili{Bikin} Udihe (\textit{gegbi}), and into the geminate \textit{bb} in \ili{Oroch} (\textit{gəbbi}). \ili{Huihe} Solon \textit{gəbbi} also has a geminate, but other \ili{Solon} dialects preserve the consonant cluster \textit{-rb-}, e.g. \ili{Ongkor} Solon \textit{ge̮rbi} (\citealt{Aalto1977}: 63). These are mostly regular changes with parallels, for example, in the cluster *\textit{lb} as in \ili{Proto-Tungusic} *\textit{dolba} ‘night’, e.g. \ili{Manchu} \textit{dobo-(ri)}, \ili{Bikin} Udihe \textit{dogbo}, \ili{Oroch} \textit{dobbo} etc. (\citealt{Benzing1956}: 46; \citealt{Kazama2003}: 50; \citealt{Doerferetal2004}: 234).\footnote{Some languages show a slightly different pattern for *\textit{lb}. For instance, one subgroup of \ili{Jurchenic} preserved a reflex of the \textit{l}, i.e. \ili{Bala} \textit{dɔlɔbɔ} (\citealt{MuYejun1987H}: 17), \ili{Jurchen} A \zh{多羅斡} [duo luo wo] (\citealt{Kiyose1977}: 101), etc.} In a few recordings, an epenthetic vowel seems to have been inserted (either by the speakers themselves or the researchers) to avoid the consonant cluster (e.g., \ili{Oroqen} or \ili{Solon} \textit{geribé} in \citealt{Ivanovskiy1982}: 1, \ili{Uilta} \textit{geribu} in \citealt{Nakanome1928}: 52). The consonant cluster as such is preserved in several \ili{Ewenic} (e.g., \ili{Evenki} \textit{gərbi}) and \ili{Nanaic} languages (e.g., \ili{Samar} \textit{görbu} [gərbu], \citealt{Schmidt1923a}). In \ili{Jurchenic}, the final vowel was sometimes lost and the \textit{b} underwent regular intervocalic spirantization in several \ili{Manchu} dialects both in Dzungaria (e.g., \ili{Sibe} \textit{gəv(ə)}) and Manchuria (e.g., \ili{Aihui} Manchu \textit{gəvo} {\textasciitilde} \textit{govo}, \ili{Yibuqi} Manchu \textit{kowə}, Shenyang \ili{Manchu} \textit{gef(u)}, \ili{Sanjiazi} Manchu \textit{gəwu}). \ili{Alchuka} represents a special case not only in \ili{Jurchenic}, but in all of Tungusic due its occasional loss of the initial consonant, i.e. ?\textit{əɔwɔ} (\citealt{MuYejun1986H}: 14). While the word has also been recorded as \textit{gəbu} (\citealt{MuYejun1987H}: 14), the form ?\textit{əɔwɔ} is not necessarily an error (although the \textit{ə} is potentially a misprint for \textit{g}). The language is known to have lost word initial consonants and exhibited a certain amount of internal variation that is poorly understood. Similar variation is known from other dialects, such as that from \ili{Sanjiazi}. As opposed to the form \textit{gəwu} in \citet{KimJuwon2008} that was collected in 2005/06, \citet{Enhebatu1995H} in 1961 recorded the form \textit{gɯ:bu} instead. While some of the discrepancies are probably a mere byproduct of the transcription (e.g., \textit{ɯ} instead of \textit{ə}), there are certainly also actual differences in the forms, for example the presence or absence of spirantization. For \ili{Chinese} \ili{Kyakala}, no cognate of the word for ‘name’ appears to have been recorded (\citealt{Hölzl2018b}; \citealt{HölzlHölzl2019a}).

Some languages, in addition to the autochthonous reflex of *\textit{gärbü}, have borrowed the \ili{Manchu} word, but with a special semantics (e.g., \citealt{Benzing1956}: 18, 49; \citealt{Alonso2011}: 27; \citetv{chapters/05}, Table \ref{table:4.6}). This led to doublets, such as \ili{Udihe} \textit{gegbi} ‘name’ vs. \textit{gebu} ‘honor’ (\citealt{NikolaevaTolskaya2001}). The latter word must represent a borrowing because an intervocalic \textit{b} is otherwise only retained in \ili{Jurchenic} (e.g., \citealt{Benzing1956}: 34).

\begin{table}
\begin{tabular}{ l l l }
  \lsptoprule
Language        &	Name    &	Source\\
  \midrule
\ili{Kili}            &	gəbu    &	\citealt{Sunik1958}: 170\\
\ili{Nanai}           &	gəbu    &	\citealt{Benzing1956}: 18\\
\ili{Oroch}           &	gəbu    &	\citealt{AvrorinLebedeva1978H}: 175\\
\ili{Udihe}           &	gebu    &	\citealt{NikolaevaTolskaya2001}: 930\\
\ili{Ulcha}           &	gəbu    &	\citealt{Benzing1956}: 18\\
\ili{Ussuri} Nanai    &	gəbə̄    & 	 \citealt{Sem1976}: 150\\
  \lspbottomrule
\end{tabular}
\caption{Manchu \textit{gebu} ‘name’ in other Tungusic languages}
\label{table:4.6}
\end{table}

The \ili{Manchu} borrowing in other Tungusic languages usually has a slightly different meaning, such as ‘honor’, which makes it less important for the purposes of this study. A similar doublet can be found, for instance, in \ili{Kili} (Kur-Urmi Nanai), i.e. \textit{gərbi} ‘name’ (\citealt{Sunik1958}: 116) vs. \textit{gəbu} ‘honor, authority, respect’ (\citealt{Sunik1958}: 170). But in this case, both forms are a borrowing from another language. Apart from \ili{Kili}, also \ili{Bala}, \ili{Kilen}, and \ili{Ussuri} Nanai must have borrowed the word for ‘name’ from a \ili{Northern Tungusic} and more exactly an \ili{Ewenic} language. For \ili{Bala}, this was misinterpreted by \citet[17]{MuYejun1988H} as an autochthonous development. But clearly, the words are from a form similar or identical to \ili{Evenki} (see Table \ref{table:4.7}). If these were not borrowings, in all four languages the final vowel should be an \textit{u} as in \ili{Manchu} \textit{gebu} or \ili{Nanai} \textit{gərbu}.\footnote{Additionally, the \textit{r} would perhaps have to be absent in the \ili{Bala} form as in \ili{Manchu} \textit{gebu}, although \ili{Bala} is more conservative than \ili{Manchu} in this particular feature, e.g. \ili{Bala} \textit{bardi-}, \ili{Manchu} \textit{banji-} ‘to live’, \ili{Bala} \textit{dɔrdi-}, \ili{Manchu} \textit{donji-} ‘to hear’ (\citealt{MuYejun1987H}, slightly corrected).} \citet[12]{Brylkin1861} recorded both \textit{gerbi} (borrowed) and \textit{gerbu} (autochthonous) among the \ili{Ussuri} Nanai.

\begin{table}
\begin{tabular}{ l l l }
  \lsptoprule
Language        &	Name        &	Source\\
  \midrule
\ili{Bala}            &	gərbi       &	\citealt{MuYejun1987H}: 14\\
\ili{Kilen}	        &   gerbi       &   \citealt{Dong2016}: 337\\
                &   gərbi       &	\citealt{AnJun1986H}: 96\\
\ili{Kili}	        &   gərbi       &	\citealt{Kazama2003}: 68\\
	            &   gərbi       &	\citealt{Sunik1958}: 116\\
\ili{Ussuri} Nanai	&   gərb’i      &	\citealt{Sem1976}: 150\\
	            &   gorbi-ni    &	\citealt{Venjukov1862}\\
	            &   gerbi       &	\citealt{Brylkin1861}: 12\\
  \lspbottomrule
\end{tabular}
\caption{The Ewenic word for ‘name’ (e.g., Evenki \textit{gərbi}) in {Southern Tungusic}}
\label{table:4.7}
\end{table}
\il{Southern Tungusic}

In many languages, *\textit{gärbü} is the basis for the derivation of verbs, e.g. \ili{Manchu} \textit{gebu-le-} ‘to name, to call by name’, \ili{Uilta} \textit{gəlbullee-} ‘to give a name to’, \ili{Udihe} \textit{gegbi-si-} ‘to call’, \ili{Evenki} \textit{gerbi-te-} ‘to be named’ etc., but these are not often encountered in the personal name question.

Among Tungusic languages, only \ili{Jurchenic} has a gender-like distinction. \ili{Even} in \ili{Jurchenic}, this is restricted to a few nouns that show an ablaut phenomenon, e.g. \ili{Manchu} \textit{haha} ‘man’, \textit{hehe} ‘woman’. The \ili{Manchu} word \textit{gebu} ‘name’ does not belong to this set of nouns.

All branches of Tungusic except for \ili{Jurchenic} have a limited system of possessive classification, making use of what is usually referred to as alienable possessive marker, e.g. \ili{Udihe} \textit{-ŋi}, \ili{Uilta} \textit{-ŋu} etc. For instance, the noun \textit{dili} ‘head’ in \ili{Udihe} can be used with and without \textit{-ŋi} (\citealt{NikolaevaTolskaya2001}: 135). The word for ‘name’ does not belong to the set of nouns that can be marked with the suffix, i.e. it is probably not conceptualized as alienable.


\subsection{Ewenic}\label{section4.4.3}

The question is known from all \ili{Ewenic} languages, with the exception of \ili{Arman}. Almost all \ili{Ewenic} examples below are copula sentences (Type A). In \ili{Even}, two different patterns are attested, but both contain the same interrogative meaning ‘who’. Consider the following two question-answer sequences:

\ea \label{example4.55}
    Eastern Even\il{Even!East} (\ili{Beryozovka}; \citealt{KimJuwon2011}: 189, corrected)\\
    \ea
    \gll hi		\textbf{ŋi}		gərbə-s?\\
    2\textsc{sg}		who		name-2\textsc{sg.poss}\\

    \ex
    \gll mi.n			gərbə-w			\textbf{garpʊk}.\\
    1\textsc{sg.obl.gen}		name-1\textsc{sg.poss}		\textsc{pn}\\
    \z
\ex \label{example4.56}
    \ili{Lamunkhin} Even (\name{Natalia}{Aralova} p.c. 2019)\\
    \ea
    \gll hi		gərbə-s			\textbf{ɲiː}?\\
    2\textsc{sg}		name-2\textsc{sg.poss}		who\\

    \ex
    \gll bi		gərbə-w			\textbf{taisiya}.\\
    1\textsc{sg}		name-1\textsc{sg.poss}		\textsc{pn}\\
    \z
\z

\noindent In both examples, the question makes use of the nominative form of the personal pronoun. In \ili{Lamunkhin} Even, not even the answer exhibits the genitive. Notably, only the interrogative, but not the personal name of the answer can stand before the word for ‘name’. Because the person is already marked on the head noun, the personal pronoun can be absent in \ili{Even} and, as will be seen, in several other Tungusic languages.

Given the overall similarity of \ili{Arman} to \ili{Even}, the question might have been very similar as well. The individual elements of the \ili{Even} examples above have the following form in \ili{Arman}: \textit{ṣi} ‘2SG’, \textit{nịị} ‘who’, \textit{gerbụ}, \textit{gurbu} ‘name, title etc.’, \textit{-s}/\textit{-SI}/\textit{-čI} ‘-2\textsc{sg.poss}’ (\citealt{Doerferetal2013}: 28, 133, 138, 228, 302f., transcription slightly changed). Consequently, the question might have been something like *\textit{ṣi nịị gerbụ-s?} or *\textit{ṣi gerbụ-s nịị?} (constructed). However, only the following example with a verb derived from \textit{gerbụ} is attested in the material available to me:

\ea
    \label{example4.57}
    \ili{Arman} (\citealt{Doerferetal2013}: 30, transcription slightly changed)\\
    \gll tẹẹmịị		tẹẹk		gerbụụtte		kamčidalal’ǰi.\\
    therefore		now		call.\textsc{nfut}[3\textsc{pl}]		\textsc{pn}\\
    \glt ‘Therefore, they now call themselves Kamchadals.’
    \z

The same possibility of the interrogative to precede or follow the word for ‘name’ as in \ili{Even} is also observed in \ili{Evenki}. The following example from the \ili{Sakhalin} dialect has the interrogative after the word for ‘name’ (the same can be found in \citealt{Konstantinova1964H}: 41). As early as the 19th century an example with a preposed interrogative has been recorded.
% \footnote{See also \url{http://evengus.ru/en/phrasebook/acquaintance/}, 03.05.2019.}

\ea
    \label{example4.58}
    \ili{Sakhalin} Evenki (\citealt{BulatovaCotrozzi2004}: 58)\\
    \gll gərbī-s			\textbf{ŋī}?\\
    name-2\textsc{sg.poss}		who\\
\ex
    \label{example4.59}
    \ili{Viljuj} Evenki (\citealt{Maack1859}: xviii; \citealt{Schiefner1878}: 144)\\
    \gll \textbf{ni}		gärbi-s?\\
    who		name-2\textsc{sg.poss}\\
\z

\noindent The absence of the personal pronoun (\textit{si} in \citealt{Maack1859}: xix, \textit{sī} in \citealt{BulatovaCotrozzi2004}: 58) is also attested in \ili{Even}.

\ea
    \label{example4.60}
    \ili{Even} (\citealt{Idiatov2007}: 307)\\
    \gll gerbe-s			\textbf{ɲi}?\\
    name-2\textsc{sg.poss}		who\\
\ex
    \label{example4.61}
    \ili{Even} (\citealt{Benzing1955}: 176)\\
    \gll \textbf{ŋī}		gə́rbə-s?\\
    who		name-2\textsc{sg.poss}\\
\z

This can also be observed in other \ili{Evenki} recordings, such as the following example from the \il{Evenki!Eastern}Eastern dialect:

\ea
    \label{example4.62}
    \il{Evenki!Eastern}Eastern Evenki (\citealt{Makarova1999}: 16)\\
    \ea
    \gll \textbf{ŋiː}		gərbiː-s?\\
    who		name-2\textsc{sg.poss}\\

    \ex
    \gll biː		gərbiː-v			\textbf{ando}.\\
    1\textsc{sg}		name-1\textsc{sg.poss}		\textsc{pn}\\
    \z
\z

\noindent Similar to \ili{Even} above, the interrogative stands in a focus position before the dummy noun while the personal name in the answer follows. Seemingly, the same asymmetry of the question and the answer has also been recorded for \ili{Aoluguya} Evenki in China.\footnote{The analysis by \citet[278]{Hasibateer2016H} is \textit{ɕini}, i.e. \textit{ɕi.n-i} `2\textsc{sg.obl-gen}', which leads to an example without interrogative, which is unlikely.}\largerpage[-1]

\ea
    \label{example4.63}
    \ili{Aoluguya} Evenki (\citealt{Hasibateer2016H}: 278)\\
    \ea
    \gll ɕi		\textbf{ni}		gərbi-ɕi?\\
    2\textsc{sg}		who		name-2\textsc{sg.poss}\\

    \ex
    \gll bi		gərbi-w			\textbf{məre}.\\
    1\textsc{sg}		name-1\textsc{sg.poss}		\textsc{pn}\\
    \z
\z

By comparing \ili{Even} and \ili{Evenki} dialects with the close relative \ili{Oroqen} in China, a very similar pattern with the interrogative in second position can sometimes be observed.

\ea
    \label{example4.64}
    \ili{Evenki} (\citealt{Boldyrev2000}: 134)\\
    \gll si		\textbf{ŋi}		gərbi-s?\\
    2\textsc{sg}		who		name-2\textsc{sg.poss}\\
\ex
    \label{example4.65}
    \ili{Oroqen} (\citealt{Chaoke2014aH}: 8)\\
    \gll shi		\textbf{ni}		gerbi-shi?\\
    2\textsc{sg}		who		name-2\textsc{sg.poss}\\
\z

\noindent This suggests a relatively high age of this phenomenon among \ili{Ewenic} languages.

All examples given so far contain a cognate of the Tungusic interrogative *\textit{ŋüi} ‘who’. The same interrogative can also be found in the personal name question of some \ili{Udegheic} and \ili{Nanaic} varieties, but not in \ili{Jurchenic}. Apart from \ili{Even} and \ili{Evenki}, many \ili{Ewenic} languages also employ different interrogatives. In most \ili{Solon} dialects, *\textit{ŋüi} has been replaced by a selective interrogative meaning ‘which (one)’ that is also found in the personal name question.

\ea
    \label{example4.66}
    \ili{Huihe} Solon (\citealt{Tsumagari2009a}: 15)\\
    \gll si.n-ii			gebbi-si			\textbf{aawu}?\\
    2\textsc{sg.obl-gen}		name-2\textsc{sg.poss}		who\\
\z

\noindent This latter construction has an exact parallel in the following \ili{Dagur} example, although the use of the nominative \textit{šiː} ‘you (\textsc{sg})’ is also possible.

\ea
    \label{example4.67}
    Tacheng \ili{Dagur} (\ili{Khitano-Mongolic}; \citealt{Yu2008}: 173)\\
    \gll šin				nər-šin			\textbf{anja}?\\
    2\textsc{sg.obl.gen}		name-2\textsc{sg.poss}		who\\
    \z

\noindent Both \ili{Solon} and \ili{Dagur} have an innovative personal interrogative that replaced Tungusic *\textit{ŋüi} ‘who’ and \ili{Mongolic} *\textit{ken} ‘who’, respectively. This innovation in \ili{Solon} appears to have later spread to \ili{Oroqen}. This interrogative is already attested in the recordings by Ivanovskiy from the end of the 19th century that are usually taken to represent \ili{Solon} (e.g., \citealt{Lie1978}).

\ea
    \label{example4.68}
    Butkha \ili{Solon} (\citealt{Ivanovskiy1982}: 1)\footnote{What is tentatively transcribed as \textit{-g-} here remains partly unclear.}\\
    \gll geribé		\textbf{agó}?\\
    name		who\\
    \z

\noindent Unlike \ili{Huihe} Solon, however, no geminate can be found in the word \textit{geribé} ‘name’. In fact, Ivanovskiy mentions three additional expressions, all of which appear to be closer to \ili{Oroqen} than \ili{Solon}:

\ea
    \label{example4.69}
    ``\ili{Manegir}'' (\citealt{Ivanovskiy1982}: 1)
    \ea
    \gll ší.n-i			gerbu 		\textbf{ní}?\\
    2\textsc{sg.obl-gen}		name		who\\

    \ex
    \gll \textbf{ni}		gerbu	bí-či?\\
    who		name	\textsc{cop-?prs}\\

    \ex
    \gll \textbf{jému}	gerbi-čí?\\
    which	name-\textsc{poss}\\
    \z
\z

\noindent Notably, two of the examples still have a cognate of *\textit{ŋüi} ‘who’ that shows the same syntactic behavior as in \ili{Even} and \ili{Evenki}. Alternatively, \textit{neré} `name' is said to be used in (\ref{example4.69}a), which is the \ili{Mongolic} word (see examples \ref{example4.20}, \ref{example4.67}, \ref{example4.73}, \ref{example4.119}, \ref{example4.125}).

% The final \textit{-u} is unexpected but could be the result of an incoherent transcription. A \ili{Nanaic} borrowing is implausible.

Examples (\ref{example4.69}a) and (\ref{example4.69}b) are also similar to \ili{Even} and \ili{Evenki}, although they appear to lack a possessive marker. The second example is one of the few examples among Tungusic languages that has an overt copula in a Type A construction. A copula is also present in a more recent example from \ili{Oroqen} that shares the absence of the possessive marker as well as the interrogative of the last example (\ref{example4.69}c) from Ivanovskiy.

\ea
    \label{example4.70}
    \ili{Xunke} Oroqen (\citealt{ZhangLiZhang1989}: 141)\\
    \gll ɕi:		\textbf{jEma}	gərbi	bi-ɕi-ni?\\
    2\textsc{sg}		which	name	\textsc{cop-prs-3sg}\\
    \z

\noindent Phonological differences apart, the following two \ili{Oroqen} sentences are identical to (\ref{example4.69}c) (see also \ref{example4.106} from \ili{Kilen}). Some Ewenic languages, such as Oroqen, use the comitative or possessive suffix instead of the second person possessive marker. These are sometimes difficult to differentiate.

\ea
    \label{example4.71}
    \ili{Gankui} Oroqen (\citealt{Sa1981}: 51)\footnote{The <q> in this transcription is based on the \ili{Chinese} Pinyin system, where it stands for [tɕʰ].}\\
    \gll \textbf{yam}		gerbi-qi?\\
    which	name-\textsc{poss}\\
\ex
    \label{example4.72}
    \ili{Shengli} Oroqen (\citealt{HanMeng1993}: 303)\\
    \gll \textbf{jeema}	kərpi-tʃ‘i?\\
    which	name-\textsc{poss}\\
\z

\citet[3]{Ivanovskiy1982} mentions two \ili{Dagur} examples, one of which contains a selective interrogative that might have influenced the choice and position of the interrogative in \ili{Oroqen}, although the two are probably not etymologically related.

\ea
    \label{example4.73}
    \ili{Dagur} (\ili{Khitano-Mongolic}; \citealt{Ivanovskiy1982}: 3)\\
    \gll si		\textbf{jamár}	neré?\\
    2\textsc{sg}		which	name\\
    \z

The same interrogative as in \ili{Oroqen} is also found in an example from \ili{Negidal}, albeit in a different syntactic position. This is not the same variation as observed for \textit{ŋüi} ‘who’, however, because this selective interrogative has an attributive function if preceding the dummy noun. In other words, we are dealing with a Type A.1 construction in \ili{Negidal} (\ref{example4.74}), but with a Type A.2 construction in \ili{Oroqen} (\ref{example4.69}c, \ref{example4.70}, \ref{example4.71}, \ref{example4.72}).

\ea
    \label{example4.74}
    \ili{Lower Negidal} (\citealt{Kazama2002}: 80)\\
    \gll sii		gilbi-si			\textbf{eema}?\\
    2\textsc{sg}		name-2\textsc{sg.poss}		which\\
    \z

\ili{Oroqen} and \ili{Evenki} dialects in China also make use of a \textsc{thing} interrogative, potentially influenced by languages such as \ili{Manchu} or \ili{Chinese}. The following two examples likewise are instances of Type A.1 (\ref{example4.75}) and Type A.2 (\ref{example4.76}), respectively:

\ea
    \label{example4.75}
    \il{Evenki!Khamnigan}Khamnigan Evenki (\citealt{Tsumagari1992}: 96)\\
    \gll ər-nii	gərbii		\textbf{ikun}?\\
    this-\textsc{gen}	name		what\\
    \glt ‘What is the name of this?’
\ex
    \label{example4.76}
    \il{Evenki!Khamnigan}Khamnigan Evenki (\citealt{ChaokeKajia2016H}: 9)\\
    \gll su			\textbf{ikon}	gərbi-tsi			wee?\\
    2\textsc{pl}		what		name-\textsc{poss}		\textsc{cq}\\
\z

\noindent The use of the interrogative \textit{ikun} in (\ref{example4.75}) might be due to the fact that it does not refer to the name of a person.

One \ili{Solon} dialect employs \textit{oni} ‘how’, which might be due to \ili{Russian} influence (see \ref{example4.26}). Given that this interrogative cannot be used attributively, the example contains fronting as in other \ili{Ewenic} languages.

\ea
    \label{example4.77}
    Arong \ili{Solon}\il{Arong Ewenke} (\citealt{ChaokeKalina2017H}: 17)\\
    \gll ʃi		\textbf{oni}		gəbbi-ʃe?\\
    2\textsc{sg}		how		name-\textsc{poss}\\
    \z

The use of \textsc{manner} interrogatives is more common in \ili{Udegheic} and \ili{Nanaic} but can also be observed in one recording of \ili{Negidal}. In the following examples, the interrogative \textit{oːn} either stands in the unexpected sentence-initial position even before the personal pronoun or in the same position as the proper name in the answer.

\ea
    \label{example4.78}
    \ili{Upper Negidal} (\name{Natalia}{Aralova} p.c. 2019)\\
    \ea
    \gll \textbf{oːn}		si		gədbi-s?\\
    how		2\textsc{sg}		name-2\textsc{sg.poss}\\

    \ex
    \gll si		gədbi-s			\textbf{oːn}?\\
    2\textsc{sg}		name-2\textsc{sg.poss}		how\\

    \ex
    \gll bi		gədbi-β			\textbf{Antonina}\\
    1\textsc{sg}		name-1\textsc{sg.poss}		\textsc{pn}\\
    \z
\z

\noindent The sentence-initial position of the interrogative in front of the pronoun, which is otherwise unattested in the PNQ in Tungusic, is clearly due to \ili{Russian} influence and is a typical European feature (\citealt{Dryer2013}).\il{languages of Europe}

\ili{Oroqen} and \ili{Solon} have been more strongly influenced by \ili{Mongolic} languages than most other \ili{Ewenic} language. In both languages, there is an alternative Type B construction that is often found in answers to the personal name question. The Type A.2 construction, as in \ili{Jurchenic}, lacks the genitive in \ili{Oroqen}.

\ea
    \label{example4.79}
    \ili{Nanmu} Oroqen (\citealt{Chaoke2007}: 140, corrected)\\
    \ea
    \gll ʃi			\textbf{ikon}		gərbi-tʃe?\\
    2\textsc{sg}			what			name-\textsc{poss}\\

    \ex
    \gll mi.ŋi		gərbi-wi			\textbf{tumbutʃə}	gunən.\\
    1\textsc{sg.obl.gen}	name-1\textsc{sg.poss}		\textsc{pn}			say.3\textsc{sg}\\
    \z
\ex
    \label{example4.80}
    \ili{Solon} (\citealt{Chaokeetal2014}: 8)\\
    \ea
    \gll shi.n-i			gebbi-shi		\textbf{awu}?\\
    2\textsc{sg.obl-gen}		name-2\textsc{sg.poss}		who\\

    \ex
    \gll mi.n-i		gebbi-wi			\textbf{...}			gʉnɵŋ.\\
    1\textsc{sg.obl-gen}	name-1\textsc{sg.poss}		(\textsc{pn})			say.3\textsc{sg}\\
    \z
\z

This construction appears to be impossible in the PNQ with the transitive verb \textit{gun-} ‘to say’ in \ili{Evenki} and other \ili{Ewenic} languages. Another Type B construction, although calqued from \ili{Russian}, is found in \ili{Negidal}. Similar to the \ili{Arman} example above, the verb is derived from the word \textit{gədbi} ‘name’.

\ea
    \label{example4.81}
    \ili{Upper Negidal} (\name{Natalia}{Aralova} p.c. 2019)\\
    \gll mi.nə-βə			gədbitʨə		\textbf{Ton’a}\\
    1\textsc{sg.obl-acc}		call.\textsc{nfut[3pl]}		\textsc{pn}\\
    \z

\noindent \citet{Ivanovskiy1982} recorded an answer without a speech act verb.

\ea
    \label{example4.82}
    ``\ili{Manegir}'' \citep[1]{Ivanovskiy1982}\\
    \gll mi.n-í		gerbú		\textbf{...}\\
    1\textsc{sg.obl-gen}	name		(\textsc{pn})\\
\z

\noindent Although ellipsis cannot be ruled out, this might be additional evidence that the Type B construction is a recent innovation in these languages.


\subsection{Udegheic}\label{section4.4.4}

For both \ili{Oroch} and \ili{Udihe} several different expressions have been recorded. Except for the following Type B example, \ili{Udegheic} makes use of copula sentences. Example (\ref{example4.83}a) from \ili{Udihe} seems to be entirely based on \ili{Russian} while the answer (\ref{example4.83}b) is similar to \ili{Ewenic} languages and represents the original Tungusic construction.

\ea
    \label{example4.83}
    \ili{Udihe} (\citealt{Tsumagari2011}: 81, 85)\\
    \ea
    \gll si.n-awa		ono		gegbi-si-ti?\\
    2\textsc{sg.obl-acc}	how		name-\textsc{v-3pl}\\

    \ex
    \gll bii		gegbi-i			Tausima.\\
    1\textsc{sg}		name-1\textsc{sg.poss}		\textsc{pn}\\
    \z
\z

Some of the oldest examples for \ili{Udegheic} have been recorded around 1900 by Brailovski. Schmidt corrected the sentences, but misinterpreted \textit{ņi} ‘who’ in (\ref{example4.84}) as a possessive marker. It is an interrogative that derives from *\textit{ŋüi} instead.

\ea[]{\label{example4.84}
    \ili{Oroch} (Bochi river; \citealt{Schmidt1928a}: 20, from Brailovski, corrected)\\
    \gll si		gabi	ņi?\\
    2\textsc{sg}		name	who\\}
\ex[?]{\label{example4.85}
    \ili{Udihe} (\ili{Samarga} river; \citealt{Schmidt1928a}, from Brailovski, corrected)\\
    \gll si		gabi	jav?\\
    2\textsc{sg}		name	what\\}
\ex[]{
    \label{example4.86}
    \ili{Udihe} (Nakhtu river; \citealt{Schmidt1928a}, from Brailovski, corrected)\\
    \gll 	si		gagbi	jau?\\
    2\textsc{sg}		name	what\\}
\z

It is unclear whether the last example (\ref{example4.86}) might contain a fused second person possessive marker \textit{-(h)i} (< *\textit{-si}) as in the following modern examples from the \ili{Khor} and \ili{Bikin} dialects (see also \citetv{chapters/07}, on intervocalic \textit{s} and its reflexes in \ili{Udegheic}):

\ea
    \label{example4.87}
    \ili{Khor} Udihe (\name{Elena}{Perekhvalskaya}, p.c. 2019)\\
    \gll si		gəgbi-\textbf{hi}			\textbf{j’əu}?\\
    2\textsc{sg}		name-2\textsc{sg.poss}		what\\
\ex
    \label{example4.88}
    \ili{Bikin} Udihe (\citealt{NikolaevaTolskaya2001}: 804)\\
    \gll si		gegbi-\textbf{i}			\textbf{j’eu}?\\
    2\textsc{sg}		name-2\textsc{sg.poss}		what\\
\z

The use of a personal interrogative (\ili{Udihe} \textit{ni(i)}, \ili{Oroch} \textit{n’ii}) seems to be much more restricted than in \ili{Ewenic} and \ili{Nanaic}. Apart from \textit{j’ə-u} ‘what’ (\textit{ja-v} and \textit{ja-u} in Brailovski), which is cognate with \ili{Oroqen} \textit{i-kon}, and \il{Evenki!Khamnigan}Khamnigan Evenki \textit{i-kun} or \textit{i-kon} above, \ili{Udihe} can also employ \textit{ono} (< *\textit{oni}) ‘how’ in the same construction.

\ea
    \label{example4.89}
    \ili{Udihe} (\citealt{Tsumagari2006}: 6)\\
    \gll sii		gegbi-i			\textbf{ono}?\\
    2\textsc{sg}		name-2\textsc{sg.poss}		how\\
    \z

\ili{Oroch} also uses a cognate of this interrogative. In the following example, there is an additional overt copula that is not usually found in the \ili{Udihe} examples (see \sectref{section4.4.5} on \ili{Nanaic}). As in \ili{Ewenic}, the personal pronoun can be absent.

\ea
    \label{example4.90}
    \ili{Oroch} (\citealt{AvrorinLebedeva1978H}: 175)\\
    \gll gəbbi-si			\textbf{ōn’i}		bi?\\
    name-2\textsc{sg.poss}		how		\textsc{cop}\\
    \z

\noindent While \ili{Oroch} also has a construction without a copula, according to one author a different interrogative meaning ‘how’ can be employed.

\ea
    \label{example4.91}
    \ili{Oroch} (\citealt{Lopatin1957}, corrected)\\
    \gll si		gabы-si			\textbf{yavanká}/\textbf{yanká}?\\
    2\textsc{sg}		name-2\textsc{sg.poss}		how\\
    \z

In sum, the \ili{Udegheic} PNQ shows a strong tendency for Type A and more specifically Type A.1. As opposed to \ili{Ewenic}, Type A.2 is not attested and one Type B construction in \ili{Udihe} can be plausibly explained by \ili{Russian} influence. Apart from this example, fronting of the interrogative is absent in the \ili{Udegheic} PNQ.


\subsection{Nanaic}\label{section4.4.5}

\citet{Brylkin1861} very early recorded the following question among the \ili{Ussuri} Nanai:

\ea
    \label{example4.92}
    \ili{Ussuri} Nanai (\citealt{Brylkin1861}: 21)\footnote{The \ili{Russian} translation was \textit{kakʺ nazyvaetsja?} ‘How is (it) called?’}\\
    \gll gerbi-si			\textbf{xamaca}?\\
    name-2\textsc{sg.poss}		which\\
    \z

\begin{sloppypar}
\noindent This interrogative (\textit{χamača} ‘which (one)’ in \citealt{Sem1976}: 62) is not attested in any other Tungusic PNQ. The question appears to be otherwise unattested for \ili{Kili}\footnote{For convenience, \ili{Kili} and \ili{Kilen} are discussed in this subsection, but they exhibit many features from other Tungusic languages.} and \ili{Ussuri} Nanai. But for both languages similar constructions have been recorded.
\end{sloppypar}

\ea
    \label{example4.93}
    \ili{Kili} (\citealt{Sunik1958}: 116, 122, shortened)\\
    \gll asi-ni			gərbi-ni\\
    woman-3\textsc{sg.poss}	name-3\textsc{sg.poss}\\
    \glt ‘the name of his wife’
    \z

\noindent This example from \ili{Kili} also suggests that a Type A construction might have been used. A PNQ in the third person is attested for \ili{Ussuri} Nanai.

\ea
    \label{example4.94}
    \ili{Ussuri} Nanai (\citealt{Sem1976}: 38)\\
    \gll s’i		am’ɪ-s’ɪ		gərb’i-n’i		\textbf{χaɪ}	χala-n’i			\textbf{χaɪ}?\\
    2\textsc{sg}		father-2\textsc{sg.poss}	name-3\textsc{sg.poss}		what		clan-3\textsc{sg.poss}		what\\
    \glt ‘What’s your father’s name and what’s his surname?’
    \z

A similar case, but with a personal interrogative borrowed from \ili{Northern Tungusic} can be found in \ili{Kilen}.

\ea
    \label{example4.95}
    \ili{Kilen} (\citealt{Dong2016}: 49, slightly modified)\footnote{<x> stands for [ɕ].}\\
    \gll xi	hale		\textbf{ni},		gerbi	\textbf{ni}?\\
    2\textsc{sg}	clan		who		name	who\\
    \z

According to \citet[241]{Schmidt1928b}, northern \ili{Nanai} (\ili{Samar}) has similar questions without a possessive marker, but in the reverse order, perhaps based on \ili{Manchu} influence. The questions about the clan name in all three languages probably represent cultural influence from \ili{Manchu} and seem to contain the loanword \textit{hala} ‘clan’.\footnote{Ewenic languages of Manchuria also have similar expressions, e.g. Oroqen \textit{shi ikun kal?} `What is your surname?' \citep[9]{Chaoke2014aH}.}

\ea
    \label{example4.96}
    \ili{Samar} (\citealt{Schmidt1928b}: 241)\\
    \ea
    \gll \textbf{xai}		ḡörbu?\\
    what		name\\

    \ex
    \gll \textbf{xai}		xala?\\
    	what		clan\\
    \z
\z

The personal name question in \ili{Ussuri} Nanai might have been *\textit{s’i gərb’i-s’i χaɪ?} (constructed) as in the following \ili{Nanai} example. In \ili{Nanai}, however, both \textit{xaj} ‘what’ and \textit{uj} ‘who’ can be employed (\ili{Ussuri} Nanai \textit{ui}):

\ea
    \label{example4.97}
    \ili{Nanai} (\citealt{Avrorin1959}: 274)\\
    \gll si		gərbu-si			\textbf{xaj}/\textbf{uj}?\\
    2\textsc{sg}		name-2\textsc{sg.poss}		what/who\\
    \z

\noindent The latter example has an exact equivalence in \ili{Ulcha}.

\ea
    \label{example4.98}
    \ili{Ulcha} (\citealt{Schmidt1923b}: 235)\\
    \gll si		gölbu-si			\textbf{uji}?\\
    2\textsc{sg}		name-2\textsc{sg.poss}		who\\
    \z

\ili{Nanai} has several different possibilities of expressing the question. Apart from the construction above, there is one influenced by \ili{Russian} making use of a \textsc{manner} interrogative.

\ea
    \label{example4.99}
    \ili{Nanai} (\citealt{KoYurn2011}: 151)\\
    \gll swə		gərbu-su		\textbf{xo:ni}	bi?\\
    2\textsc{pl}		name-2\textsc{pl.poss}		how		\textsc{cop}\\
    \glt ‘What is your (\textsc{sg.pol}) name?’
    \z

\noindent An almost identical example with a copula is found in \ili{Ulcha}.

\ea
    \label{example4.100}
    \ili{Ulcha} (\citealt[3]{Angina1993})\\
    \gll si.n			gəlbu-si			\textbf{xon}		bi-ni?\\
    2\textsc{sg.obl.gen}	name-2\textsc{sg.poss}		how		\textsc{cop-3sg}\\
    \z

In answers, \ili{Nanai} has more or less the same construction as in \ili{Ewenic} and \ili{Udegheic} with the personal name following the word for ‘name’:

\ea
    \label{example4.101}
    \ili{Nanai} (\citealt{KoYurn2011}: 151)\\
    \gll mi		gərbu-i			\textbf{tanja}.\\
    1\textsc{sg}		name-1\textsc{sg.poss}		\textsc{pn}\\
    \glt ‘My name is Tanja.’
    \z

\ili{Uilta} is special among \ili{Nanaic} languages in showing a regular content question marker that is unattested in the rest of Tungusic and might be a \ili{Nivkh} borrowing (\citealt{Hölzl2018a}: 39, 302–305).

\ea
    \label{example4.102}
    \ili{Uilta} (\citealt{Nakanome1928}: 52; \citealt{Ikegami1997}: 67)\\
    \ea
    \gll sî.n-i			geribu-si		\textbf{hai}=\textbf{ga}?\\
    2\textsc{sg.obl-gen}	name-2\textsc{sg.poss}	what=\textsc{cq}\\

    \ex
    \gll si.n-i			gəlbu-si			\textbf{xai}=\textbf{gaa}?\\
    2\textsc{sg.obl-gen}	name-2\textsc{sg.poss}	what=\textsc{cq}\\
    \z
\z

In another recording, an example from \ili{Uilta} uses a personal interrogative. This suggests that the same synchronic variation as in \ili{Nanai} might be present. The genitive is obligatory in the southern\il{Southern Uilta} dialect but absent in the northern\il{Northern Uilta} (\name{Patryk}{Czerwinski}, p.c. 2020).

\ea
    \label{example4.103}
    \ili{Uilta} (\citealt{Ozolniya2001}: 72)\\
    \gll si		gəlbu-si			\textbf{ŋui}=\textbf{ɣə}?\\
    2\textsc{sg}		name-2\textsc{sg.poss}		who=\textsc{cq}\\
\z

But all three examples share the special question marker \textit{=KA(A)} that is only attested in \ili{Uilta}. This question marker is also found in the following example that contains the interrogative \textit{xooni} ‘how’ (cognate of \ili{Solon} \textit{oni}, \ili{Negidal} \textit{oːn}, \ili{Udihe} \textit{ono}, \ili{Oroch} \textit{ōn’i}, \ili{Nanai} \textit{xo:ni}, and \ili{Ulcha} \textit{xon} above).

\ea
    \label{example4.104}
    \ili{Uilta} (\name{Patryk}{Czerwinski}, p.c. 2019)\\
    \gll \textbf{xooni}=\textbf{ka}	naa			gəlbu-ni?\\
    how=\textsc{cq}		\textsc{interj}		name-3\textsc{sg.poss}\\
    \glt ‘But what’s its name?’
    \z

\noindent As in \ili{Negidal}, the sentence-initial position of the interrogative is probably based on \ili{Russian}.

In \ili{Kilen}, another special case in \ili{Nanaic}, one example has been recorded that differs in its interrogative from all the other Tungusic languages. Semantically, however, \textit{yanemi} is a \textsc{manner} interrogative and might have been directly or indirectly influenced by \ili{Russian}. The stem \textit{ya-} ‘what, which’ is cognate with \ili{Oroqen} \textit{i(-kon)}, \ili{Udihe} \textit{j’ə(-u)} etc. The combination of the dummy noun with the speech act verb also suggests some \ili{Chinese} influence.

\ea
    \label{example4.105}
    \ili{Kilen} (\citealt{Dong2016}: 37)\footnote{\textit{xn} with initial [ɕ-] goes back to \textit{si.n-i}.}\\
    \gll xn			gerbi-xi			\textbf{ya-ne-mi}		hudarewye?\\
    2\textsc{sg.obl.gen}	name-2\textsc{sg.poss}		what-\textsc{v-cvb.ipfv}	call\\
    \z

\noindent Another \ili{Kilen} example has an equivalent in \ili{Oroqen} (\sectref{section4.4.3}). In fact, not only the dummy noun \textit{gerbi}, but also the interrogative \textit{yama} is from \ili{Ewenic}.

\ea
    \label{example4.106}
    \ili{Kilen} (\citealt{Chaoke2014b}: 8)\\
    \gll shi		\textbf{yama}	gerbi-shi?\\
    2\textsc{sg}		which	name-2\textsc{sg.poss}\\
    \z

\ili{Nanaic}, like \ili{Ewenic} and \ili{Udegheic}, has a tendency for Type A.1. Isolated Type A.2 constructions in \ili{Samar} and \ili{Kilen} are most likely based on \ili{Jurchenic} or \ili{Ewenic} influence. Similar to \ili{Ewenic}, the genitive is only occasionally attested in the PNQ. Fronting is almost entirely absent and based on the \ili{Russian} pattern.


\subsection{Jurchenic}\label{section4.4.6}

Although the person is not marked on the head noun, the personal pronoun can also be absent in \ili{Jurchenic} languages. According to one source, \ili{Manchu} can make use of a personal interrogative \textit{we} ‘who’.

\ea
    \label{example4.107}
    \ili{Manchu} (\citealt{Avrorin2000}: 113)\\
    \gll si.n-i		gebu	?\textbf{we}?\\
    2\textsc{sg.obl-gen}	name	who\\
    \z

\noindent However, this appears to be a mistake, perhaps based on the author’s knowledge of \ili{Nanai}, as all other sources invariably give the interrogative \textit{ai} ‘what’ instead. This interrogative is cognate with the \ili{Nanaic} form encountered above, e.g. \ili{Uilta} \textit{xai}. In \ili{Sibe}, an optional question marker can attach at the end of the PNQ.

\ea
    \label{example4.108}
    \ili{Sibe} (\citealt{Sameng2010}: 447)\footnote{In this example, <x> also stands for [ɕ].}\\
    \gll xi.n-ǐ			gev		\textbf{ai}=ye?\\
    	2\textsc{sg.obl-gen}	name	what=\textsc{q}\\
    \z

Apart from the universal use of this interrogative, \ili{Manchu} dialects seemingly show the same variation as the \ili{Ewenic} languages. The interrogative can precede or follow the noun, the personal pronoun can be absent, and it can take a genitive if the interrogative is postposed. But \ili{Jurchenic} has a tendency for preposed interrogatives.

\ea
    \label{example4.109}
    \ili{Yibuqi} Manchu (\citealt{ZhaoJie1989H}: 127)\\
    \gll \textbf{ɛi}		kowə?\\
    what		name\\
\ex
    \label{example4.110}
    \ili{Aihui} Manchu (\citealt{WangQingfeng2005H}: 208)\\
    \gll ɕi		\textbf{ɛ}		gəvo?\\
    2\textsc{sg}		what		name\\
\ex
    \label{example4.111}
    \ili{Sanjiazi} Manchu (\citealt{Enhebatu1995H}: 39)\\
    \gll ɕi		\textbf{ai}		gɯ:bu?\\
    2\textsc{sg}		what		name\\
\z

\noindent Furthermore, these are Type A.2 constructions in which the interrogative stands attributively to the dummy noun. There is no fronting as in \ili{Ewenic}.

\ili{Manchu} in \ili{Yanbian} close to the North \ili{Korean} border is only preserved in some isolated words and expressions among which there is the following:\footnote{The meaning of \textit{keci} is not clear. It could theoretically correspond to \ili{Manchu} \textit{se-ci} ‘say-\textsc{cvb.cond}’, but this is problematic on phonological grounds. It could also corresponds to \ili{Manchu} \textit{o-ci} ‘become-\textsc{cvb.cond}’, which can be a topic marker. \ili{Alchuka} is known to have an occasional initial \textit{k-} in this word, i.e. \textit{(k)ɔ-} (\citealt{MuYejun1986H}). A connection to \ili{Mongolian} \textit{g(e)-} ‘to say’ is unlikely.}

\ea
    \label{example4.112}
    \ili{Yanbian} Manchu (\citealt{ZhaoJie2000H}: 19)\\
    \gll \textbf{ai}			hala		(keci)?\\
    what			surname		?\\
    \glt ‘What’s your surname (clan name)?’
    \z

While the same expression \textit{ai hala} is also attested in classical \ili{Manchu} (e.g., \citealt{Hauer2007}: 217), the \textit{Qingwen Qimeng}, one of the most influential descriptions of \ili{Manchu}, also contains the following example with reversed word order:
		
\ea
    \label{example4.113}
    \ili{Manchu} (\citealt{Wuge1730}: vol. 2; \citealt{Wylie1855}: 82)\\
    \gll hala		\textbf{ai}?\\
    surname		what\\
    \z

\noindent According to the same source, questions about personal names have the same structure with the interrogative following the noun.

\ea
    \label{example4.114}
    \ili{Manchu} (\citealt{Wuge1730}: vol. 2; \citealt{Wylie1855}: 82)\\
    \gll gebu		\textbf{ai}?\\
    name		what\\
    \z

\noindent According to \name{Veronika}{Zikmundová} (p.c., 2019), this postposed position of the interrogative is impossible in spoken \ili{Sibe}. As seen above, it is also not very common in other \ili{Manchu} dialects.

One special example that contains two copies of the word for ‘name’ (\ili{written Manchu} \textit{gebu}) is attested for \ili{Sanjiazi} Manchu.

\ea
    \label{example4.115}
    \ili{Sanjiazi} Manchu (\citealt{Enhebatu1995H}: 39)\\
    \gll ɕin		gɯ:bu	[\textbf{ai}		gɯ:bu]?\\
    2\textsc{sg.obl.gen}		name	what		name\\
    \z

\noindent In a similar example from \ili{Sibe} that is strongly influenced by the written language, the noun \textit{nalma} ‘person’ (\ili{written Manchu} \textit{niyalma}) can occur twice. In this case, ‘what name’ seems to function as an attribute to ‘person’.

\ea
    \label{example4.116}
    \ili{Sibe} (\citealt{Kałużyński1977}: 23)\\
    \gll ere		nalma	[\textbf{ai̯}		gebu	nalma]?\\
    this		person	what		name	person\\
    \glt ‘What is this person’s name?’
    \z

\noindent The sentence thus literally means ‘A what-named person is this person?’

A major difference of \ili{Jurchenic} with respect to most other Tungusic languages is the widespread use of questions of Type B. An occasional affricatization of \textit{s} (\ili{Manchu} \textit{se-} ‘to say’) seen in the following \ili{Sibe} example is also attested in other \ili{Jurchenic} varieties (see also \citealt{Chaoke2014d}: 8).

\ea
    \label{example4.117}
    \ili{Sibe} (\citealt{Chaoke2006}: 206)\\
    \gll ʂi.n-i			gəvə-v		\textbf{ai}		dʐi-m?\\
    2\textsc{sg.obl-gen}		name-\textsc{acc}		what		say-\textsc{ipfv}\\
    \z

\noindent In the following parallel from \ili{written Manchu} the optional accusative has been added.

\ea
    \label{example4.118}
    \ili{Manchu} (\citealt{He2009}: 21)\\
    \gll si.n-i			gebu(-be)	\textbf{ai}		se-mbi?\\
    2\textsc{sg.obl-gen}		name(-\textsc{acc})	what		say-\textsc{ipfv}\\
    \z

\noindent \citet[259]{Vovin2006} argues that \ili{Manchu} \textit{se-} is a \ili{Koreanic} loanword. Admittedly, \textit{se-} is unattested outside of \ili{Jurchenic} and has all the hallmarks of being a borrowing. But \ili{Manchu} \textit{se-} has almost exactly the same range of functions as \ili{Mongolian} \textit{g(e)-} ‘to say’ (\citealt{Janhunen2012b}: 283–285). On phonological grounds it cannot be a direct borrowing from \ili{Mongolian}, but the underlying construction in the PNQ is almost identical to the one in \ili{Jurchenic}. Consider the following answer to a PNQ.

\ea
    \label{example4.119}
    \ili{Mongolian} (\citealt{Janhunen2012b}: 283)\\
    \gll mi.n-ii			ner-iig		delger+maa		ge-deg.\\
    1\textsc{sg.obl-gen}		name-\textsc{acc}		\textsc{pn}				say-\textsc{ptcp.hab}\\
    \glt ‘My name is Delgerma.’
    \z

\noindent This parallel with the same word order and the same functional elements suggests that the \ili{Jurchenic} PNQ has been calqued from \ili{Mongolian}, but the similarities of the verbs go beyond this construction.

In both languages, this intransitive (+ \textsc{name}) speech act verb here has a lexical function but is otherwise frequently used in grammatical functions, for example as a quotative. Depending on how the quotative is embedded into the sentence, it can have different forms that have parallels in both languages. For example, \ili{Mongolian} \textit{ge-deg} ‘say-\textsc{ptcp.hab}’, functionally corresponds to \ili{Manchu} \textit{se-re} ‘say-\textsc{ptcp.ipfv}’ and can function as an attribute to a following noun or can take case markers. \ili{Mongolian} \textit{g-e.j} ‘say-\textsc{cvb.ipfv}’ functionally corresponds to \ili{Manchu} \textit{se-me} ‘say-\textsc{cvb.ipfv}’ and is used adverbially (e.g., \citealt{Janhunen2012b}: 283). While these parallels cannot rule out a potential \ili{Koreanic} origin of the \ili{Jurchenic} verb, they nevertheless illustrate a much more intimate connection with \ili{Mongolic}.

For instance, \textit{se-} does not have the function of a speech act verb, but that of a quotative in the following example that contains the main verb \textit{hūla-} ‘to call’.

\ea
    \label{example4.120}
    \ili{Manchu} (\citealt{Schluessel2014})\\
    \gll [si.n-i	gebu-be		ai]		se.me	hūla-mbi?\\
    2\textsc{sg.obl-gen}		name-\textsc{acc}		what		\textsc{quot}	call-\textsc{ipfv}\\
    \z

\noindent In the following construction, the same verb is used, but without quotative.

\ea
    \label{example4.121}
    \ili{Sanjiazi} Manchu (\citealt{KimJuwon2008}: 161)\\
    \gll si		aj		gəwu	xola-m?\\
    2\textsc{sg}		what		name	call-\textsc{ipfv}\\
    \z

\noindent In the former sentence, the entire part \textit{sini gebu-be ai} is embedded by means of the quotative \textit{se.me}. In the latter example, the question is not embedded. This example is most likely based on the \ili{Chinese} construction (e.g., \ref{example4.15}b) but it also resembles the \ili{Solon} and \ili{Oroqen} answers in \sectref{section4.4.3}.

While the PNQ is unknown in \ili{Bala}, the words \textit{ɕi} ‘you (\textsc{sg})’, \textit{ɕin} ‘your (\textsc{sg})’, \textit{gərbi} ‘name’, and perhaps \textit{a(i)-} ‘what’ are all attested (\citealt{MuYejun1987H}: 14, 25, 31). As seen above, the word \textit{gərbi} is of \ili{Northern Tungusic} origin and must have been transmitted through a form of southern \ili{Nanai}, such as \ili{Kilen}.

The sentence is not attested in \ili{Alchuka} and \ili{Lalin}/\il{Manchu!Jing}Jing Manchu either. However, a similar construction in the third person has the following form:

\ea
    \label{example4.122}
    \ili{Alchuka} (\citealt{MuYejun1986H}: 14)\\
    \gll t‘ə.rə-i		?əɔwɔ	int‘uki.\\
    that-\textsc{gen}		name	\textsc{pn}\\
    \glt ‘His name is Yentugi.’
\ex
    \label{example4.123}
    \ili{Lalin} Manchu (\citealt{AixinjueluoYingsheng1987a}: 14)\\
    \gll te.re-i		gebu	yintuhi.\\
    that-\textsc{gen}		name	\textsc{pn}\\
    \glt ‘His name is Yentugi.’
\z

\noindent As seen before, the dummy noun was also recorded as \textit{gəbu} for \ili{Alchuka}. The cognate of \ili{written Manchu} \textit{ai} ‘what’ has the form \textit{(k)ai} or \textit{ei} in \ili{Alchuka} and \textit{ai} in \ili{Lalin}/\il{Manchu!Jing}Jing Manchu. Written \ili{Manchu} \textit{si} ‘you (\textsc{sg})’ and \textit{sin-i} ‘your (\textsc{sg})’ correspond to \ili{Alchuka} \textit{ɕi}/\textit{ɕin-i} and \ili{Lalin}/\il{Manchu!Jing}Jing Manchu \textit{si}/\textit{sin-i}. Written Manchu \textit{se-} ‘to say’ has the form \textit{ts‘ə-} in \ili{Alchuka} and \textit{se-} in \ili{Lalin}/\il{Manchu!Jing}Jing Manchu (\citealt{MuYejun1986H}; \citealt{AixinjueluoYingsheng1987a}).\il{written Manchu}

The earliest recordings of Tungusic are in \ili{Jurchen}, but to the best of my knowledge the sentence is not attested in these materials either. In \ili{Jurchen} B, the second person pronoun is attested as *\textit{ši} \zh{失}, the genitive as *\textit{-i} \zh{亦}, and the word ‘name’ as *\textit{gebu} \zh{革不} (\citealt{Kane1989}: 270, 272, 356). In \ili{Jurchen} A, the second person pronoun apparently is not attested, but the equivalences of \ili{Manchu} \textit{min-i} ‘my’ and \textit{gebu} ‘name’ have the forms *\textit{min-i} \zh{密你} and *\textit{gebu} \zh{革卜}, respectively (\citealt{Kiyose1977}: 138, 140, 145). It is likely that a comparable range of different constructions as in modern varieties of \ili{Manchu} might have been present in these languages.

\ili{Jurchenic} has several examples of all three types of constructions, Type A.1, Type A.2, and Type B. As seen above, Tungusic has otherwise few cases of A.2 and even fewer of Type B. \ili{Jurchenic} is also the only subbranch of Tungusic that does not use the personal interrogative in the PNQ. The speech act verb \textit{se-} found in Type B constructions is also unattested in other Tungusic languages. \ili{Jurchenic} lost head-marked possession and has extended the scope of the genitive to elements other than the speech act participants. All of these features can best be explained by an unusually strong impact from other languages, such as \ili{Khitano-Mongolic} and perhaps \ili{Koreanic} (e.g., \citealt{Vovin2006}), rather than with an early branching of \ili{Jurchenic} (e.g., \citealt{Kazama2003}). As has been shown, the \ili{Jurchenic} Type B construction is clearly a calque from \ili{Mongolian}.


\section{Discussion}\label{section4.5}

\subsection{The (re)construction in Proto-Tungusic}\label{section4.5.1}

A personal name question must have already existed in \ili{Proto-Tungusic}. The only element that all Tungusic languages without exception have in common in the PNQ is a cognate of the word *\textit{gärbü} ‘name’. The second person pronoun *\textit{si}, which also functions as a possessive marker *\textit{-si} in languages outside of \ili{Jurchenic}, can be absent in some constructions, but is also attested in all Tungusic languages. The genitive form can be reconstructed as *\textit{si.n-i}.

The interrogative is the element of the question that exhibits the most variation. However, apart from \ili{Jurchenic}, all three other subbranches of Tungusic have at least some examples with a cognate of the interrogative *\textit{ŋüi} ‘who’. No other interrogative has such as wide distribution in the PNQs of Tungusic. Instances of *\textit{Kooni} ‘how’ are also found in \ili{Ewenic}, \ili{Udegheic}, and \ili{Nanaic}, but this widespread usage can be more plausibly explained with \ili{Russian} influence all over the northern half of the Tungusic-speaking areas. The use of Tungusic *\textit{Kai}\footnote{Given the uncertainty of the initial, the abstract label *\textit{K-} is used in this reconstruction (e.g., \citealt{Hölzlpreprint}).} in both \ili{Nanaic} (e.g., \ili{Uilta} \textit{xai}) and \ili{Jurchenic} (e.g., \ili{Manchu} \textit{ai}) could indicate that this is a \ili{Southern Tungusic} innovation, although it is much more pervasive in \ili{Jurchenic} than in \ili{Nanaic} and likely due to language contact. Other interrogatives, such as *\textit{ja-} ‘which’, can only be found in very few languages (e.g., \ili{Oroqen} \textit{i(-kon)}, \ili{Udihe} \textit{j’e(-u))}.

\begin{table}
\begin{tabular}{ l p{1.2cm} p{1.2cm} p{1.2cm} p{1.2cm} l }
  \lsptoprule
Language    &	*ŋüi ‘who’  &	*Kai ‘what’ &	*ja- ‘which’    &	*Kooni ‘how’    &	other\\
  \midrule
\ili{Arman}    &	?    &	?    &	?    &	?    &	?\\
\ili{Even}    &	+    &	−    &	−    &	−    &	−\\
\ili{Evenki}    &	+    &	−    &	+    &	−    &	−\\
\ili{Oroqen}    &	+    &	−    &	+    &	−    &	which (one)\\
\ili{Solon}    &	−    &	−    &	−    &	+    &	which (one)\\
\ili{Negidal}    &	?    &	−    &	−    &	+    &	which (one)\\
\midrule
\ili{Oroch}    &	+    &	−    &	?+    &	+    &	how\\
\ili{Udihe}    &	−    &	−    &	+    &	+    &  −\\
\midrule
\ili{Kilen}    &	+    &	?    &	?    &	?    &	how, which (one)\\
\ili{Kili}    &	?    &	?    &	?    &	?    &	?\\
\ili{Nanai}    &	+    &	+    &	−    &	+    &	−\\
\ili{Samar}    &	?    &	+    &	?    &	?    &	?\\
\ili{Ulcha}    &	+    &	−    &	−    &	+    &	−\\
\ili{Uilta}    &	+    &	+    &	−    &	+    &	−\\
U. Nanai\il{Ussuri}    &	−    &	+    &	−    &	−    &	which (one)\\
\midrule
\ili{Alchuka}    &	?    &	?    &	?    &	?    &	?\\
\ili{Bala}    &	?    &	?    &	?    &	?    &	?\\
\ili{Jurchen} A    &	?    &	?    &	?    &	?    &	?\\
\ili{Jurchen} B    &	?    &	?    &	?    &	?    &	?\\
\ili{Kyakala}    &	?    &	?    &	?    &	?    &	?\\
sp. \ili{Manchu}    &	−    &	+    &	−    &	−    &	−\\
sp. \ili{Sibe}    &	−    &	+    &	−    &	−    &	−\\
w. \ili{Manchu}    &	?−    &	+    &	−    &	−    &	−\\
  \lspbottomrule
\end{tabular}
\caption{Overview of the interrogatives used in the Tungusic PNQs, including dialects and historical data mentioned in the discussion}
\label{table:4.8}
\end{table}

The use of ‘who’ in the North and of ‘what’ in the South is part of a general areal division between languages around Siberia and Mongolia on the one hand and the surrounding languages (e.g., parts of Europe, China, Japan) on the other (e.g., \citealt{Idiatov2007}; \citealt{Gil2018}). \ili{Proto-Tungusic} most likely was part of an area with ‘who’ and due to contact with \ili{Chinese} and other languages changed its typological profile in the South. The increasing use of ‘how’ in the North is based on the \ili{Russian} construction that represents a pattern found in many European languages.\il{languages of Europe}

The reconstruction of the \ili{Proto-Tungusic} PNQ depends on the internal classification of Tungusic. If \ili{Jurchenic} is considered the oldest branch of the language family (e.g., \citealt{Kazama2003}), the presence of a second person possessive marker could well be a later innovation in the non-\ili{Jurchenic} branch. But \ili{Jurchenic} preserves some traces of the personal markers that must have been present earlier. For instance, \citet[7]{Doerfer1978} observed that ordinal numerals in some Tungusic languages are ultimately derived from what appears to be a third person plural possessive marker (Table \ref{table:4.9}). The possessive form is preserved, for example, in \ili{Udihe}, e.g. \textit{neŋu-ti} ‘their younger sibling’ (\citealt{NikolaevaTolskaya2001}: 107). In \ili{Udihe}, a case marker can occasionally precede the ordinal marker, which might be a relic of its origin as a possessive marker, e.g. \textit{nada} ‘seven’, \textit{nadä-ma-ti} ‘seventh (\textsc{acc}) (\citealt{NikolaevaTolskaya2001}: 424). The syllable *\textit{ti} that is still recorded as such in \ili{Alchuka} regularly changed to \textit{ci} in \ili{Manchu} (e.g., \textit{nadan}, \textit{nada-\textbf{ci}}).

\begin{table}
\begin{tabular}{ l l l l }
  \lsptoprule
Language    &	three       &	third   &   -3\textsc{pl.poss}\\
  \midrule
\ili{Alchuka}     &	ila(-n)     &	ila-\textbf{t‘i}    &   -\\
\ili{Manchu}      &	ila-n       &   ila-\textbf{ci}     &   -\\
\midrule
\ili{Kilen}       &	ila-n       &   ila-n-\textbf{tin}  &   -ti\\
\ili{Udihe}       &	ila         &   ile-n-\textbf{ti}   &   -ti\\
  \lspbottomrule
\end{tabular}
\caption{Ordinal markers in Alchuka (\citealt{MuYejun1986H}), and Manchu, Kilen (\citealt{ZhangZhangDai1989}), and Udihe (\citealt{NikolaevaTolskaya2001})}
\label{table:4.9}
\end{table}

\noindent This strongly speaks in favor of head-marking (e.g., head-marked possession) being present in \ili{Proto-Tungusic}.

Given the presence of Type A constructions throughout the entire language family, \ili{Proto-Tungusic} must have been of the same type (Table \ref{table:4.10}). Type B is restricted to few examples, most of which can be found in \ili{Jurchenic}. For instance, as seen before, the typical \ili{Jurchenic} question containing a speech act verb (\ili{Manchu} \textit{se-}) is clearly calqued from the \ili{Mongolian} pattern (\sectref{section4.4.6}). Apart from the use of a personal interrogative, the construction is almost a perfect match.

\ea
    \label{example4.124}
    \ili{Sibe} (\citealt{Zikmundová2013a}: 138)\footnote{\ili{Sibe} \textit{śin} goes back to \textit{si.n-i} ‘2\textsc{sg.obl-gen}’. \ili{Jurchenic} also has sentence-final content question marking that is, however, not usually attested in the PNQs.}\\
    \gll śin			gəvə-f		ai		zə-mie?\\
    2\textsc{sg.obl.gen}	name-\textsc{acc}		what		say-\textsc{ipfv}\\
\ex
    \label{example4.125}
    \ili{Mongolian} (elicited in \ili{May} 2019)\\
    \gll či.n-ii		ner-iig		xen		ge-deg=ve?\\
    2\textsc{sg.obl-gen}	name-\textsc{acc}		who		say-\textsc{ptcp.hab=cq}\\
    \z

\noindent Content question marking as in this \ili{Mongolian} example is a feature absent from most Tungusic languages (\citealt{Hölzl2018a}: 286–312). In those languages that have this feature, such as \ili{Jurchenic} languages, \il{Evenki!Khamnigan}Khamnigan Evenki, or \ili{Uilta}, this is clearly an innovation. Consequently, \ili{Proto-Tungusic} most likely did not have content question marking either. All Type B constructions can plausibly be explained with language contact.

\begin{table}
\begin{tabular}{ l cc }
  \lsptoprule
Language    &	Type A (copula)  &	Type B (speech act verb)\\
  \midrule
\ili{Arman}    &	?    &	?\\
\ili{Even}    &	+    &	−\\
\ili{Evenki}    &	+    &	−\\
\ili{Oroqen}    &	+    &	−\\
\ili{Solon}    &	+    &	−\\
\ili{Negidal}    &	+    &	−\\
\midrule
\ili{Oroch}    &  +    &	−\\
\ili{Udihe}    &	+    &	+\\
\midrule
\ili{Kilen}    &	+    &	+\\
\ili{Kili}    &	?+    &	?\\
\ili{Nanai}    &	+    &	−\\
\ili{Samar}    &	+    &	−\\
\ili{Ulcha}    &	+    &	−\\
\ili{Uilta}    &	+    &	−\\
U. Nanai\il{Ussuri}    &	+    &	−\\
\midrule
\ili{Alchuka}    &	?+    &	?\\
\ili{Bala}    &	?    &	?\\
\ili{Jurchen} A    &	?    &	?\\
\ili{Jurchen} B    &	?    &	?\\
\ili{Kyakala}    &	?    &	?\\
sp. \ili{Manchu}    &	+    &	+\\
sp. \ili{Sibe}    &	+    &	+\\
w. \ili{Manchu}    &	+    &	+\\
  \lspbottomrule
\end{tabular}
\caption{The type of PNQs in Tungusic languages}
\label{table:4.10}
\end{table}

In conclusion, the most likely reconstruction for the proto-Tungusic personal name question is perhaps the following Type A, more specifically Type A.1, construction with an optional pronoun and an optional genitive.\footnote{Very similar constructions to this one reconstructed to Tungusic can be found in some surrounding languages. These cannot be addressed here for reasons of space (see, e.g., \ref{example4.39}).}

\ea
    \label{example4.126}
    \ili{Proto-Tungusic}\\
    \gll *(si(n-i))		gärbü-si			ŋüi?\\
    2\textsc{sg.obl-gen}		name-2\textsc{sg.poss}		who\\
    \z

\begin{sloppypar}
\noindent All four subbranches of Tungusic have direct descendants of this construction, such as the following from \ili{Even} (with optional pronoun \textit{ḥi} `2\textsc{sg}', \textit{ḥin} `2\textsc{sg.obl(.gen)}') and \ili{Manchu}.
\end{sloppypar}

\ea
    \label{example4.127}
    \ili{Even} (\citealt{Doerferetal1980}: 304, modified transcription)\\
    \gll 		gerbe-s			ŋịị?\\
    name-2\textsc{sg.poss}		who\\
\ex
    \label{example4.128}
    \ili{Manchu} (\citealt{Haenisch1961}: 73)\\
    \gll si.n-i			gebu			ai?\\
    2\textsc{sg.obl-gen}		name			what\\
\z

\noindent Some languages, such as \ili{Manchu}, have introduced a new interrogative into the construction, replacing the original *\textit{ŋüi}. \ili{Jurchenic} has generally lost the possessive marker *\textit{-si}, at the same time generalizing the genitive.

One can suspect that the Tungusic construction above was based on a more schematic construction that has the following form, X being a pronoun, Y a possessive ending, and Z a proper name or the interrogative *\textit{ŋüi}: *(X(\textit{n-i})) \textit{gärbü}-Y Z. The genitive might have been restricted to first and second person pronouns. Only \ili{Jurchenic} has third person pronouns that can take a genitive (singular \textit{i.n-i}, plural \textit{ce.n-i} in \ili{Manchu}) and it remains an open question whether this represents a \ili{Proto-Tungusic} pattern that was replaced everywhere else or is also an innovation in \ili{Jurchenic} (e.g., \citetv{chapters/09}). The use of the genitive on elements other than the pronouns is probably a \ili{Jurchenic} innovation that later spread to a few other Tungusic languages.

\ea
    \label{example4.129}
    \ili{Manchu} (\citealt{AixinjueluoYingsheng1987a}: 14)\\
    \gll te.re-i			gebu			yentugi.\\
    that-\textsc{gen}			name			\textsc{pn}\\
    \z

\noindent Another instantiation of the schematic construction can be observed in the following answer from \ili{Even}.

\ea
    \label{example4.130}
    \ili{Even} (\citealt{Doerferetal1980}: 304)\\
    \gll mị.n			gerbe-w			Anna.\\
    1\textsc{sg.obl}(.\textsc{gen})		name-1\textsc{sg.poss}		\textsc{pn}\\
    \z

\noindent The preposed interrogative as in the following \ili{Aoluguya} Evenki example (Type A.1) appears to be restricted to \ili{Ewenic} (found in \ili{Even}, \ili{Evenki}, \ili{Oroqen}, and \ili{Solon} in \sectref{section4.4.3}).

\ea
    \label{example4.131}
    \ili{Aoluguya} Evenki (\citealt{ChaokeSirenbatu2016H}: 1)\\
    \gll ʃi		[ni]		gərbi-tʃi\\
    2\textsc{sg}		who		name-\textsc{poss}\\
    \z

\noindent This also illustrates another innovation in parts of Ewenic, which is the use of the comitative or possessive suffix (\textit{gərbi-tʃi} `with/having a name'), replacing the second person possessive marker in the PNQ (\textit{gərbi-ʃi} `your name', \citealt{ChaokeSirenbatu2016H}: 5).

Seemingly similar expressions in \ili{Jurchenic} (see \ref{example4.17} and \sectref{section4.4.6}) cannot be based on the same construction because the interrogative (\ili{Manchu} \textit{ai}) functions as an attribute to the dummy noun (\ili{Manchu} \textit{gebu}) (Type A.2).

\ea
    \label{example4.132}
    \ili{Manchu} (\ili{Sanjiazi}; \citealt{Chaoke2014c}: 8)\\
    \gll shi		[ayi		gewe]?\\
    2\textsc{sg}		what		name\\
    \z

\noindent The personal interrogative in \ili{Evenki} cannot, however, stand attributively to a noun (\citealt{Nedjalkov1997}: 215). The interrogative, therefore, must be interpreted as an argument of its own that stands in some sort of focus position that is specific to \ili{Ewenic}. In \ili{Evenki}, interrogatives often are sentence-initial, but there is another construction: “Much more rarely, they appear in the second position after the subject or the object of the question in cases when these components are stressed.” (\citealt{Nedjalkov1997}: 7f.) This must be considered an early innovation of \ili{Ewenic} languages.


\subsection{Conclusion: Construction and frame}\label{section4.5.2}

This study has investigated a potentially universal property of human language, the personal name question (PNQ, ‘What’s your name?’). While the focus was on Tungusic languages, several typological dimensions of variation were discussed from a global perspective. Cross-lingusitically, there are two main types of PNQs that contain an equational copula (Type A) and a speech act verb (Type B), respectively. Tungusic languages show a tendency for Type A, although the \ili{Jurchenic} subbranch due to language contact also has many instances of Type B. On the basis of the PNQ in the individual Tungusic languages, the PNQ in \ili{Proto-Tungusic} has been reconstructed as an instance of Type A. This reconstruction lacks a copula but contains a personal interrogative *\textit{ŋüi} ‘who’, an optional personal pronoun *\textit{si} ‘you (\textsc{sg})’ (oblique *\textit{si.n-}) with optional genitive *\textit{-i}, and a dummy noun *\textit{gärbü} ‘name’ that functions as a host for head-marked possessive affixes. The basis for the apparent split between head-marking on the one hand and double marking on the other remains unclear for now.

Generally, personal name questions can be said to be semantically based on what has been called the personal name frame (\sectref{section4.3}) that has several subevents, each with its individual roles. The Tungusic Type A construction highlights or profiles the subevents of \textsc{having a name} and \textsc{acquainting}. The whole expression is the result of a complex interaction of the individual frames and constructions (\figref{fig:4:1}).

\begin{figure}
\includegraphics[width=\textwidth]{figures/Hoelzl-Name-schema.pdf}
\caption{The interaction of frames and constructions in the Proto-Tungusic PNQ (figure created by the author)}
\label{fig:4:1}
\end{figure}

% could we make a new figure as tikzpicture?

% \begin{figure}
% \vspace*{\baselineskip}
% \begin{tikzpicture}

% \end{tikzpicture}
%     \vspace*{\baselineskip}
% \caption{The interaction of frames and constructions in the Proto-Tungusic PNQ}
% \label{fig:4:1}
% \end{figure}


In the schematic construction, X is an open slot for a pronoun, Y for a possessive ending corresponding to X, and Z for a proper name or the interrogative *\textit{ŋüi}. CS and CC stand for copula subject and copula complement, respectively (\citealt{Dixon2010a}). The dummy noun *\textit{gärbü} ‘name’ is head and the personal pronoun *\textit{si} ‘you (\textsc{sg})’ is the dependent. Dotted lines indicate that a given element is identical in the schematic and in the specific construction, e.g. the genitive remains *\textit{-i}. Dotted arrows show the filling of an open slot with a certain element, e.g. of X with the pronoun *\textit{si} ‘you (\textsc{sg})’. Arrows from the frames to the constructions indicate the place of realization of roles and relations. In some cases, multiple realization is possible, e.g. of the possessor as both the personal pronoun and possessive affix. Finally, dashed arrows are used for roles and relations that are only indirectly coded in the construction. In this example, the role of the person asking is only indirectly represented by the second person elements. The interrogative force of the question, here tentatively indicated with the semantic relation ASK, has no overt morphosyntactic expression but is indirectly encoded in the interrogative and perhaps a special intonation contour that is difficult to reconstruct given the scarcity of data from modern languages.


\section*{Abbreviations}

PNQ stands for \textit{personal name question} and PNF for \textit{personal name frame}. Abbreviations follow the general convention. Special grammatical abbreviations include:\smallskip\\

\noindent\begin{tabularx}{.5\textwidth}{@{}lQ}
\textsc{abst} & abstract, non-concrete (cf. \citealt{Munshi2006})\\
\textsc{anx} & anxiousness \citep{Prins2017}\\
\textsc{cq} & content question marker\\
\end{tabularx}%
\begin{tabularx}{.50\textwidth}{lQ@{}}
\textsc{gl} & general topic \citep{Lock2011}\\
\textsc{l} & low tone form \citep{Paperno2014}\\
\textsc{sim} & simple (unmarked) tense \citep{Coler2014}\\
\end{tabularx}



\section*{Acknowledgements}

This paper is dedicated to the memory of Prof. \name{Wolfgang}{Schulze} (1953–2020).

I want to thank \name{Veronika}{Zikmundová}, \name{Patryk}{Czerwinski}, \name{Elena}{Perekhvalskaya}, \name{Natalia}{Aralova}, and \name[Payne, Thomas E.]{Tom}{Payne} for their valuable comments on Sibe, Uilta, Udihe, Negidal, and Panare, respectively.

{\sloppy\printbibliography[heading=subbibliography,notkeyword=this]}
\end{document}
