\documentclass[output=paper,colorlinks,citecolor=brown]{langscibook}
\ChapterDOI{10.5281/zenodo.7053369}

\author{Elena Klyachko\affiliation{Higher School of Economics \& Institute of Linguistics, Russian Academy of Sciences}}
\title{Functions of placeholder words in Evenki}

\abstract{Placeholders are used to fill in the pause when the speaker has forgotten the exact word. They have the syntactic properties of the word the speaker cannot recall (the target word). Studying placeholders is thus important for understanding how discourse works. However, the area has been much understudied, especially for low-resource languages, due to the lack of oral corpora. This paper fills in this lacuna for the Evenki language. It describes the functions of placeholders and their grammatical properties, drawing on data from oral corpora and elicitation. More specifically, it looks into the transfer of grammatical features from the target word to the placeholder. Dialectal distribution of placeholders and their correlates in other Tungusic languages are also discussed.}

\IfFileExists{../localcommands.tex}{
   \addbibresource{../localbibliography.bib}
   \usepackage{langsci-optional}
\usepackage{langsci-gb4e}
\usepackage{langsci-lgr}

\usepackage{listings}
\lstset{basicstyle=\ttfamily,tabsize=2,breaklines=true}

%added by author
% \usepackage{tipa}
\usepackage{multirow}
\graphicspath{{figures/}}
\usepackage{langsci-branding}

   
\newcommand{\sent}{\enumsentence}
\newcommand{\sents}{\eenumsentence}
\let\citeasnoun\citet

\renewcommand{\lsCoverTitleFont}[1]{\sffamily\addfontfeatures{Scale=MatchUppercase}\fontsize{44pt}{16mm}\selectfont #1}
  
   %% hyphenation points for line breaks
%% Normally, automatic hyphenation in LaTeX is very good
%% If a word is mis-hyphenated, add it to this file
%%
%% add information to TeX file before \begin{document} with:
%% %% hyphenation points for line breaks
%% Normally, automatic hyphenation in LaTeX is very good
%% If a word is mis-hyphenated, add it to this file
%%
%% add information to TeX file before \begin{document} with:
%% %% hyphenation points for line breaks
%% Normally, automatic hyphenation in LaTeX is very good
%% If a word is mis-hyphenated, add it to this file
%%
%% add information to TeX file before \begin{document} with:
%% \include{localhyphenation}
\hyphenation{
affri-ca-te
affri-ca-tes
an-no-tated
com-ple-ments
com-po-si-tio-na-li-ty
non-com-po-si-tio-na-li-ty
Gon-zá-lez
out-side
Ri-chárd
se-man-tics
STREU-SLE
Tie-de-mann
}
\hyphenation{
affri-ca-te
affri-ca-tes
an-no-tated
com-ple-ments
com-po-si-tio-na-li-ty
non-com-po-si-tio-na-li-ty
Gon-zá-lez
out-side
Ri-chárd
se-man-tics
STREU-SLE
Tie-de-mann
}
\hyphenation{
affri-ca-te
affri-ca-tes
an-no-tated
com-ple-ments
com-po-si-tio-na-li-ty
non-com-po-si-tio-na-li-ty
Gon-zá-lez
out-side
Ri-chárd
se-man-tics
STREU-SLE
Tie-de-mann
}
   \boolfalse{bookcompile}
   \togglepaper[6]%%chapternumber
}{}

\begin{document}
\maketitle

\section{Introduction}

\subsection{Placeholder words}

% replaced \ul with \underline since all special characters disappeared in the PDF

In conversation, speakers can employ a number of devices in case they hesitate or have forgotten the exact word. “Non-silence devices” which are used to fill in the pause are called fillers. More specifically, fillers “fulfilling the syntactic projection” of a phrase (in contrast to interjections) are called placeholders (see \citealt{Fox2010} for a discussion of the terms). For many languages, corpus-based placeholder studies may be difficult. Firstly, placeholder words were usually omitted in older published materials, which are not accompanied by audio. Secondly, the very technique of writing down texts without speech recorders (such as asking the speaker to dictate) may have forced the speakers to use fewer placeholder words.

\subsection{The Evenki language}

In this work, a study of placeholder words in the \ili{Evenki} language is performed. \ili{Evenki} is an endangered Tungusic language spoken in Russia, China and Mongolia. In Russia, there are fewer than 5000 speakers (\citealt{Census2010}). For China, a number of 11 000 is given in Ethnologue (\citeyear{Ethnologue2019}) but we should take into account that the traditional Chinese classification counts \ili{Solon}, \ili{Aoluguya} and \il{Evenki!Khamnigan}Khamnigan Evenki as \ili{Evenki} dialects whereas \ili{Oroqen} is considered a separate language (\citealt{Tsumagari1992}). However, \ili{Oroqen} is actually closer to the \ili{Russian} Evenki dialects (as well as \ili{Aoluguya} and \il{Evenki!Khamnigan}Khamnigan Evenki) than \ili{Solon}. Therefore, if we count only dialects of \ili{Evenki} proper in China, there are roughly 2500 \ili{Oroqen} speakers (\citealt{WhaleyLi2000}), fewer than 200 \ili{Aoluguya} speakers (\citealt{Tsumagari1992}), and fewer than 1000 \il{Evenki!Khamnigan}Khamnigan Evenki speakers (\citealt{Whaley1998}). As regards Mongolia, the \il{Evenki!Khamnigan}Khamnigan Evenki language seems to be extinct. This paper addresses the \ili{Evenki} dialects of Russia due to the lack of oral speech corpora from China or Mongolia.

The \ili{Evenki} language is spread over a huge territory and comprises numerous dialects, which are quite different from each other. \citet{Vasilevich1948} provides a classification of the \ili{Evenki} dialects spoken in the former USSR, dividing them into three groups: \il{Evenki!Northern}Northern, \il{Evenki!Southern}Southern, and \il{Evenki!Eastern}Eastern (see Figure \ref{fig:6:1} for a map).

\begin{figure}
\includegraphics[width=\textwidth]{figures/Klyachko.png}
\caption{Evenki dialects of Russia (based on \citealt{Vasilevich1948}, redrawn by Nadezhda Mamontova)}
\label{fig:6:1}
\end{figure}

\subsection{Notes on Evenki morphology}

In this paper, placeholders are analyzed from the morphological point of view. Therefore, a brief introduction into the \ili{Evenki} morphology will be given here.

\ili{Evenki} is an agglutinating language with rich derivational and inflectional morphology. Nedjalkov lists the following morphological classes in \ili{Evenki} (\citealt[139--140]{Nedjalkov1997}):

\begin{itemize}
  \item nouns
  \item pronouns
  \item verbs
  \item adjectives
  \item postpositions (usually nominal stems with a locative meaning, having possessive suffixes)
  \item numerals/quantifiers
  \item particles (including what is called an adverb in traditional classifications)
\end{itemize}

A nominal wordform has the template shown in Table \ref{table:6.1}. For example:

% % \todo{Rephrase this sentence to accomodate positioning of Tables}


\ea
    \label{example6.1}
    \gll bəjə\textsubscript{1}{ }   {∅}\textsubscript{2}{ }		-ŋi\textsubscript{3}{ }	-l\textsubscript{4}{ }	-ba\textsubscript{5}{ }	-w\textsubscript{6}\\
    person {}	\textsc{-ind.poss}	\textsc{-pl}	\textsc{-acc} \textsc{-1sg.poss}\\
    \glt `<told about> my people'\\
\ex
    \label{example6.2}
    \gll əwiː\textsubscript{1}{ } -wu\textsubscript{2}{ } {∅}\textsubscript{3}{ }		-r\textsubscript{4}{ }	-ə\textsubscript{5}{ }	-tin\textsubscript{6}\\
    	play	\textsc{-nmlz} {}		\textsc{-pl}	\textsc{-accin}	\textsc{-3pl.poss}\\
    \glt `<made> toys for them’
    \z

Verbal wordforms can be finite or non-finite (participles and converbs), depending on whether it can be the only verbal form in an independent clause.\footnote{Actually, there are rare cases of non-finite forms used independently in oral speech.} A finite verb form has the following template (Table \ref{table:6.2}).\footnote{Some aspect affixes can precede voice affixes. There can be several aspect affixes in a verb form (Tables \ref{table:6.2}--\ref{table:6.3}).}

\begin{table}
\captionsetup{margin=.025\linewidth}
\begin{floatrow}
\ttabbox{\begin{tabular}{ ll }
  \lsptoprule
1 & stem                  \\
2 & derivational affixes  \\
3 & alienable possession  \\
4 & number                \\
5 & case                  \\
6 & inalienable possession\\
\lspbottomrule
\end{tabular}}
{\caption{Nominal template (\citealt[15]{BulatovaGrenoble1999})\label{table:6.1}}}

\ttabbox{%
\begin{tabular}{ ll }
  \lsptoprule
1 & stem                \\
2 & derivational affixes\\
3 & voice               \\
4 & aspect              \\
5 & mood+tense          \\
6 & mood+number         \\
  \lspbottomrule
\end{tabular}}
{\caption{Finite verbal template (\citealt[27]{BulatovaGrenoble1999})\label{table:6.2}}}
\end{floatrow}
\end{table}


\noindent For example:

\ea
    \label{example6.3}
    \gll t͡ʃa\textsubscript{1}{ }	-ti\textsubscript{2}{ }	-pkaː\textsubscript{3}{ }	-l\textsubscript{4}{ }	-də\textsubscript{5}{ }	-n\textsubscript{6}{ }	=daː\\
    tea	\textsc{-vblz}	\textsc{-caus}	\textsc{-inch}	\textsc{-nfut}	-3\textsc{sg}	\textsc{-foc}\\
    \glt `She started to give tea to drink.'\\
    \z

Non-finite verb forms can have personal or number endings, which depends on the actual participial or converbial form itself. A non-finite verb form has therefore the scheme shown in Table \ref{table:6.3}.

\begin{table}
\begin{tabular}{ ll }
\lsptoprule
1 & stem \\
2 & derivational affixes \\
3 & voice \\
4 & aspect \\
5 & participial or converbial affix \\
6 & person + number/ number \\
7 & case (for participles) \\
8 & inalienable possession (for participles)\\
\lspbottomrule
\end{tabular}
\caption{Non-finite verbal template\label{table:6.3}}
\end{table}


(\ref{example6.4}) is an example of a converb with no personal endings, (\ref{example6.5}) is a converb with personal endings, and (\ref{example6.6}) is a participle.
\ea
    \label{example6.4}
    \gll boriː\textsubscript{1}{ }    {∅}\textsubscript{2}{ }		-maːt\textsubscript{3}{ }	-t͡ʃə\textsubscript{4}{ }	-nə\textsubscript{5}{ }	-l\textsubscript{6}\\
    share   {}		\textsc{-recip}	\textsc{-ipfv}	\textsc{-cvsim}	\textsc{-pl}\\
    \glt `<They live> sharing with each other.'\\
    \z


\ea
    \label{example6.5}
    \gll hokori\textsubscript{1}{ }	-w\textsubscript{2}{ }	{∅}\textsubscript{3}{ }	{∅}\textsubscript{4}{ }	-rak\textsubscript{5}{ }	-tin\textsubscript{6}\\
    lose	\textsc{-pass}	{}	{}	\textsc{-cvcond}	-3\textsc{pl}\\
    \glt `if they are lost.'\\
    \z

\ea
    \label{example6.6}
    \gll boγolo\textsubscript{1}	{∅}\textsubscript{2}{ }	{∅}\textsubscript{3}{ }		-t\textsubscript{4}{ }	-t͡ʃə\textsubscript{4}{ }	-riː\textsubscript{5}{ } {∅}\textsubscript{6}{ }		-duk\textsubscript{7}{ }	-tin\textsubscript{8}\\
    deprive.of.share {} {}	\textsc{-dur}	\textsc{-ipfv}	\textsc{-psim} {}	\textsc{-abl}	\textsc{-3pl.poss}\\
    \glt `because of their depriving of the share.'\\
    \z

All wordforms can be followed by a clitic (as in \ref{example6.3}).


\subsection{Aims}

The aims of this work are as following:

\begin{itemize}
  \item Describe various placeholder words in the \ili{Evenki} language. To my knowledge, no comprehensive work of such kind has ever been done.
  \item Define the functions of the placeholder words and compare them to the functions observed typologically.
\end{itemize}

\subsection{Methods}

The study is mainly based on a corpus of texts, which have been recorded, transcribed and analyzed by a group of linguists, including the author of this work (Siberian-Lang corpus \citeyear{Corpus2019}). The material was recorded in 2007–2018 in Tomsk oblast, Krasnoyarsk krai and Irkutsk oblast. The corpus comprises mainly texts recorded from speakers of the \il{Evenki!Northern}Northern dialect group. A corpus of oral \ili{Evenki} texts recorded in Krasnoyarsk krai by Nadezhda Mamontova in 2014 is also used (Corpora IEA \citeyear{Corpora2019}). Another source is descriptive grammars of the \ili{Evenki} language: \citet{Konstantinova1964}, \citet{Nedjalkov1997}, and \citet{BulatovaGrenoble1999}. Both corpora are more focused on the \il{Evenki!Northern}Northern and \il{Evenki!Southern}Southern Evenki dialects, while data on the \il{Evenki!Eastern}Eastern dialects is scarce. Furthermore, the corpora of related Tungusic languages have been studied.

I conducted several elicitation experiments in Krasnoyarsk krai, Irkutsk oblast, and Khabarovsk krai. The design was straightforward: the speakers were given sentences containing placeholders and asked whether the sentences sounded acceptable and which words could be used instead. However, when analyzing the elicitation results, it should be taken into account that the status of the placeholder words is very low among the speakers. They are often referred to as “slips of the tongue” or “just insertions to connect the words together”. Sometimes a speaker says about a particular placeholder that there is no such word, although they still use it in their own speech. Still, some speakers recall “people using these words in the past when telling something”, and even try to distinguish the meanings of the placeholders.

\section{Placeholder words in Evenki}

In this section, placeholder words in \ili{Evenki} will be described in detail according to the following plan, which roughly follows \citet{Podlesskaya2010}.

\begin{enumerate}
    \item \textit{Functions as a placeholder:} The placeholder can be used in various pragmatic circumstances, which will be further described in greater detail.

    \item \textit{Restrictions on the target word:} The target word is the word the speaker cannot recall. Therefore, it is replaced with a placeholder. Typologically, there can be placeholders which replace only words of specific classes (e. g., only nouns, or only proper nouns).

    \item \textit{Functions other than those of a placeholder:} Data from other languages show that some placeholder stems can also be used as interjections or hesitation markers without actually replacing any target word.

    \item \textit{Mirroring the grammatical shape of the target word:} The placeholder can copy inflectional and/or derivational affixes from the target word, thus reflecting its structure. \citet{Podlesskaya2010} calls this mirroring. The target word is then sometimes recalled, so we can compare its surface realization to that of the placeholder. Typological studies show that there is sometimes no mirroring at all. In other cases, mirroring can be full (copying all affixes) or partial (copying only some of them).
    
    \item \textit{Frequency:} We can leverage corpus data to understand how frequent a particular placeholder is, and whether it depends on the dialect, on the speaker’s age, gender, or fluency in \ili{Evenki}, as well as on the genre of the text.
    
    \item \textit{Dialectal distribution of a placeholder:} As shown above, the \ili{Evenki} comprises various dialects that may display different behaviour in terms of placeholders.
    
    \item \textit{Possible source and comparative data from related languages:} In typological literature on placeholders, they are often traced to a pronoun or to a noun meaning ‘thing’. It is therefore crucial to trace the etymology of the placeholder, if possible. \ili{Evenki} is a \ili{Northern Tungusic} language, so I will use available data from other Northern as well as \ili{Southern Tungusic} languages to indicate the etymology of common placeholders.
\end{enumerate}

In all examples, the placeholder word will be put in \textbf{bold}, whereas the correspondent target phrase will be \uline{underlined}. In translations, “whatsitsname” and “do that thing” will be used.


\subsection{\textit{aŋə} / \textit{aŋi}}

In the \ili{Evenki} grammars, \textit{aŋi} is described as a placeholder, though this exact term is not always used. In \citet[265]{Konstantinova1964} it is called a demonstrative particle meaning ‘whatsitsname, something’. In \citet[24, 26]{BulatovaGrenoble1999} it is classified as an interrogative pronoun as well as a placeholder, and its use in both nominal and verbal roots is described. In the corpus texts, it is pronounced as either \textit{aŋə} or \textit{aŋi}. It seems to be more frequently pronounced as \textit{aŋə} when it is used independently, without any affixes. Furthermore, the stem is sometimes shortened to \textit{aŋ}, without the final vowel. However, (\ref{example6.10}) shows that the final vowel is not just a connecting or epenthetic vowel (otherwise the form would be *\textit{aŋ-tikiː} and not \textit{aŋi-tkiː}). Prosodically, \textit{aŋi} is often followed with a pause. However, this can be justified by the speaker actively trying to recall the target word. Generally speaking, intonation in \ili{Evenki} is understudied (see, for example \citealt{Morozova2019}). Therefore, I will not go into greater detail regarding intonation.


\subsubsection{Functions as a placeholder}

\textit{aŋi} is used widely if the speaker cannot recall the exact word to ensure the fluency of the narrative. For example, in (\ref{example6.7}) there are two instances of \textit{aŋi} for two nouns, which are both repaired on the spot. In (\ref{example6.8}), the first occurrence of the placeholder is repaired but the second is not.

\ea
    \label{example6.7}
    \gll \textbf{aŋi-l}	\uline{bəjə-l}	\textbf{aŋi-l-tin}	\uline{igi-l-tin}	doːldi-w-ra-∅\\
    whatsitsname-\textsc{pl}	person-\textsc{pl}	whatsitsname-\textsc{pl}-\textsc{3pl.poss}	voice-\textsc{pl}-\textsc{3pl.poss}	hear-\textsc{pass}-\textsc{nfut}-3\textsc{pl}\\
    \glt `The whatsitsnames, the voices of whatsitsnames, of people were heard.' (G. K. Lapuko, Tura, 2008)\\
    \z

\ea
    \label{example6.8}
    \gll umnoː	\textbf{aŋi-wa}	\uline{bagdakə-wə}	\textbf{aŋi-}∅\textbf{-m}\\
    once	whatsitsname-\textsc{acc}	wild.reindeer-\textsc{acc}	whatsitsname-\textsc{nfut}-1\textsc{sg}\\
    \glt `I once did that thing to [killed] a whatsitsname, a wild reindeer.’ (S. M. Andreyeva, Strelka-Chunya, 2007)\\
    \z


\subsubsection{Restrictions on the target word}

As (\ref{example6.9}), (\ref{example6.10}), and (\ref{example6.11}) show, \textit{aŋi} can substitute for both nouns (including proper nouns) and verbs. 

\ea
    \label{example6.9}
    \gll həwəkiː	hələ	\textbf{aŋi-l-duk}	\uline{sʲita-l-duk}	o-ďa-fkiː   bi-sʲə	bəjə-l-bə\\
    god	\textsc{intj}	whatsitsname-\textsc{pl}-\textsc{abl}	clay-\textsc{pl}-\textsc{abl}	make-\textsc{ipfv}-\textsc{phab}    be-\textsc{pant}	person-\textsc{pl}-\textsc{acc}\\
    \glt `God was making people out of whatsitsname, out of clay.' (V. Kh. Yoldogir, Chiringda, 2007)\\
    \z

\ea
    \label{example6.10}
    \gll nuŋartin=tə	\textbf{aŋi-tkiː}	huru-t͡ʃoː-tin	tar \uline{moskwa-tkiː}	ďuː-laː-wər=tə\\
    3\textsc{pl}=\textsc{foc}	whatsitsname-\textsc{all}	go.away-\textsc{pst}-3\textsc{pl}	that    Moscow-\textsc{all}	house-\textsc{loc.all}-\textsc{rfl.pl}=\textsc{foc}\\
    \glt `They went to whatsitsname, to Moscow, home.’ (G. K. Lapuko, Tura, 2008)\\
    \z

\ea
    \label{example6.11}
    \gll    t͡ʃaŋit	tar	t͡ʃaŋit-pa	tarə  \textbf{aŋi-waːt}	\uline{t͡ʃok-naː-γaːt}\\
    bandit	that	bandit-\textsc{acc}	that.\textsc{acc} whatsitsname-\textsc{imper}.1\textsc{pl.incl} 	kill-\textsc{prgrn-imper}.1\textsc{pl.incl}\\
    \glt `Let us do that thing, let us go and kill that bandit (=bear).' (S. M. Andreyeva, Strelka-Chunya, 2007)\\
    \z

In (\ref{example6.12}), it replaces an adjective: the speaker could not come up with the \ili{Evenki} word and switched to \ili{Russian}.

\ea
    \label{example6.12}
    \gll əməgən=ta	on	\textbf{aŋə}	\uline{skolskij}\\
    saddle=\textsc{foc}	how	whatsitsname	slippery.R\\
    \glt `The saddle is how, whatsitsname, slippery.' (I. K. Uvachan, Tutonchany, 2008)\\
    \z

There are no examples of \textit{aŋi} replacing a numeral, a quantifier, or a postposition in our corpus or in the IEA RAS corpus.

In (\ref{example6.13}), \textit{aŋi} may be considered to be replacing an adverb \textit{ďuga} ‘in summer’. It is the only example of that kind in our corpus.

\ea
    \label{example6.13}
    \gll hulakiː-l	koŋnomo-l	\textbf{aŋi}	\uline{ďuga}	o	ďuga    \textbf{aŋi-wkiː-l}\\
    fox-\textsc{pl}	black-\textsc{pl}	whatsitsname	in.summer	\textsc{intj}	in.summer   whatsitsname-\textsc{phab}-\textsc{pl}\\
    \glt `Black foxes whatsitsname, in summer, oh, in summer they usually do that thing.' (L. V. Mikhaylova, Tura, 2008)\\
    \z

In (\ref{example6.14}), \textit{aŋi} has the same affixes as the personal pronoun following it (note that the 3rd person pronoun form in \ili{Evenki} has a possessive affix historically, which behaves just like a normal possessive suffix in nominal forms). However, it would be strange for a placeholder to replace a personal pronoun. Perhaps, the speaker wanted to say “When we were going past her grave…” and then said simply “When we were going past her…”. It is the only example in our corpus where the speaker uses a pronoun to “repair” the placeholder.

\ea
    \label{example6.14}
    \gll tara	\textbf{aŋi-liː-n}	\uline{nuŋan-duliː-n}	ŋənə-ďə-wun eːkun=məl	təpkə-l-də-n\\
    that.\textsc{acc}	whatsitsname-\textsc{prol}-\textsc{3sg.poss}	3\textsc{sg}-\textsc{prol}-\textsc{3sg.poss}	go-\textsc{psim}-1\textsc{pl.excl} what=\textsc{indef}	shout-\textsc{inch}-\textsc{nfut}-3\textsc{sg}\\
    \glt `When we were going past whatsitsname, past her, something started to shout.' (G. K. Lapuko, Tura, 2008)\\
    \z

\subsubsection{Functions other than those of a placeholder}

\textit{aŋi} is sometimes used as an interjective hesitation marker as in \citet{HayashiYoon2010}, when the speaker cites the direct speech of a character:

\ea
    \label{example6.15}
    \gll tuŋ 	ɲikə-rə-n=daː	gun-ə-n 	\textbf{aŋi}\\
    thus	do-\textsc{nfut}-3\textsc{sg}=\textsc{foc}	say-\textsc{nfut}-3\textsc{sg}	whatsitsname\\
    \glt `Having done this, (he) said: whatsitsname….' (V. Kh. Yoldogir, Chiringda, 2007)\\
    \z

Sometimes, \textit{aŋi} is used at the beginning of a new sentence (\ref{example6.16}) or at the end of a sentence (\ref{example6.17}) with seemingly no syntactic role or any actual placeholder function, being an interjection, marking hesitation and/or introducing a new topic.

\ea
    \label{example6.16}
    \gll \textbf{aŋe}	ďur	anŋaɲiː-l	uʐə	ŋənə-rə-∅\\
    whatsitsname	two	year-\textsc{pl}	already.R	go-\textsc{nfut}-3\textsc{pl}\\
    \glt `Well, two years have already passed.' (I. I. Tsurkan, \ili{Yerbogachyon}, 2016)\\
    \z

\ea
    \label{example6.17}
    \gll ŋinakin-tin	əmə-hi-lďi-sʲa	\textbf{aŋe}\\
    dog-\textsc{3pl.poss}	come-\textsc{incep}-\textsc{comit}-\textsc{pant}	whatsitsname\\
    \glt `Their dog came with them, well.' (V. Kh. Yoldogir, Chiringda, 2007)\\
    \z

In some of these examples, \textit{aŋ} has a focus marker \textit{=kə}:

\ea
    \label{example6.18}
    \gll ə-kəldu	ɲikagda	ə-kəldu	ɲiː-wə=dəː	\textbf{aŋ=kə}   abiʐat-tə\\
    \textsc{neg-imper.2pl}	never.R	\textsc{neg-imper.2pl}	who-\textsc{acc=foc}	whatsitsname=\textsc{foc} offend.R=\textsc{pneg}\\
    \glt `Never, well, never offend anybody.' (T. A. Bogdanova, Potapovo, 2011)\\
    \z

According to \citet[300]{Idiatov2007}, who follows \citet[24]{BulatovaGrenoble1999}, \textit{aŋi} can be used as an interrogative word. However, such usages are lacking in our corpus.


\subsubsection{Mirroring the grammatical shape of the target word}

The questions of this section are: whether wordforms with \textit{aŋi} can have all possible slots filled in; and which slots are copied from the target word. I must emphasize the fact that we cannot be 100\% sure that the word recalled by the speaker is actually the target word. However, it will be our assumption. First, I will look into the slots of nominal and verbal wordforms. For nominal wordforms, there are no examples of \textit{aŋi} taking the alienable possession suffix in our corpus. However, there are no examples where the target word is then recalled and actually has the alienable possession affix, either. Other slots can also be filled in. For example, in (\ref{example6.19}), a derivational intensifier affix is used together with the case and number suffixes.

\ea
    \label{example6.19}
    \gll irəktə-l-ə	\textbf{aŋi-kaːkuː-r-ə}	o-ďa-n\\
    larch-\textsc{pl}-\textsc{accin}	whatsitsname-\textsc{ints}-\textsc{pl}-\textsc{accin}	make-\textsc{futcnt}-3\textsc{sg}\\
    \glt `He will make larches, whatsitsnames….' (S. P. Mukto, Uchami, 2014)\\
    \z

In our corpus, intensifiers are the only non-inflectional affixes which are used in nominal \textit{aŋi} wordforms.

As regards verbs, there are no examples of the derivational slot (including intensifiers) filled in for the \textit{aŋi} verbal wordforms. Furthermore, the voice slot also remains empty in the corpus examples, although there are elicited sentences where the speaker uses a wordform with a non-empty voice slot (\ref{example6.20}).

\ea
    \label{example6.20}
    \gll ə-doː-tin	\textbf{aŋi-ďə-rə}	\uline{isə-w-ďə-rə=doː} loku-sa-ďa-ra=daː\\
    \textsc{neg}-\textsc{cvpurp}-3\textsc{pl}	whatsitsname-\textsc{ipfv}-\textsc{pneg}	see-\textsc{pass}-\textsc{ipfv}-\textsc{pneg}=\textsc{foc}  hang-\textsc{stat}-\textsc{ipfv}-\textsc{pneg}=\textsc{foc}	\\
    \glt `So that they will not do that thing, be seen, hang.' (S. P. Mukto, Uchami, 2014)\\
    \z

Aspect and mood/tense slots are, on the contrary, often filled. In (\ref{example6.21}), \textit{aŋi} has non-empty aspect and tense slots, and in (\ref{example6.22}) the aspect and the mood (imperative) slots are filled.

\ea
    \label{example6.21}
    \gll patom	bu	luhu	\textbf{aŋi-ŋnə-rə-w}	luhu\\
    then.R	1\textsc{pl.excl}	all.the.time	whatsitsname-\textsc{hab}-\textsc{nfut}-1\textsc{pl.excl}	all.the.time\\
    \glt `Then we would all the time do that thing.' (I. K. Uvachan, Tutonchany, 2008)\\
    \z

\ea
    \label{example6.22}
    \gll ďəm-muː-l-mi	\textbf{aŋi-ŋna-kal}	guː-səː	əri-ŋ-mə-w  tugeː	\uline{sʲiwu-ŋna-kal}\\
    eat-\textsc{des}-\textsc{inch}-\textsc{cvcond}	whatsitsname-\textsc{hab}-\textsc{imper}.2\textsc{sg}	say-\textsc{pant}	this-\textsc{ind.poss}-\textsc{acc}-1\textsc{sg.poss}    so	lick-\textsc{hab}-\textsc{imper}.2\textsc{sg}\\
    \glt `If you get hungry, he said, do that thing, lick this your <paw> so.' (V. K. Udygir, Ekongda, “The man and the bear-relative”) (IEA RAS\footnote{  \url{http://corpora.iea.ras.ru/corpora/describe_text.php?id=43}})\\
    \z


There are examples of participial (\ref{example6.23}) and converbial (\ref{example6.24}) forms with \textit{aŋi}:

\ea
    \label{example6.23}
    \gll bi	tar	doːldiː-∅-m	\textbf{aŋi-ďə-ri-l-wə}  \uline{buːɲiː-ďə-ri-l-wə}	straʃ	ŋəːləwsʲi=koː\\
    1\textsc{sg}	that	hear-\textsc{nfut}-1\textsc{sg}	whatsitsname-\textsc{ipfv}-\textsc{psim}-\textsc{pl}-\textsc{acc} howl-\textsc{ipfv}-\textsc{psim}-\textsc{pl}-\textsc{acc}	horrible.\textsc{slip}.R	horrible=\textsc{foc}\\
    \glt `I heard doing that thing, howling, [it was] horrible.' (S. M. Andreyeva, Strelka-Chunya, 2007)\\
    \z

\ea
    \label{example6.24}
    \gll eː-ja=wəl	eː-ďə-nə	horol-ďo-fkiː	taduː=wər   ŋaːlə-l-ďi-ji	\textbf{aŋ-ďa-na}\\
    what-\textsc{accin}=\textsc{indef}	what-\textsc{ipfv}-\textsc{cvsim}	whirl-\textsc{ipfv}-\textsc{phab}	there=\textsc{rfl}.\textsc{pl} arm-\textsc{pl}-\textsc{instr}-\textsc{rfl}	whatsitsname-\textsc{ipfv}-\textsc{cvsim}\\
    \glt `Doing something, he is whirling there, doing that thing with his arms.' (V. N. Udygir, Ekongda, 2007)\\
    \z

Clitic slots can be filled in \textit{aŋi} nominal (\ref{example6.25}) and verbal (\ref{example6.26}) wordforms.

\ea
    \label{example6.25}
    \gll \textbf{aŋi-l=daː}	ɲi	\uline{əmkə-n=dəː}	asʲin	bi-soː-n\\
    whatsitsname-\textsc{pl}=\textsc{foc}	not.R	cradle-\textsc{3sg.poss}=\textsc{foc}	\textsc{neg}	be-\textsc{pst}-3\textsc{sg}\\
    \glt `He had no whatsitsnames, no cradle.' (S. P. Mukto, Uchami, 2014)\\
    \z

\ea
    \label{example6.26}
    \gll tug=doː	eːku-r=wal	nuŋar-wa-tin	\textbf{aŋi}-\tS\textbf{a=hin}\tS\textbf{a} \uline{huru-p-t͡ʃo}\\
    so=\textsc{foc}	what-\textsc{pl}=\textsc{indef}	3\textsc{pl-acc-3pl.poss}	whatsitsname-\textsc{pant}=\textsc{prob} go.away-\textsc{tr}-\textsc{pant}\\
    \glt `So something must have done that thing to them, carried them away.' (G. K. Lapuko, Tura, 2008)\\
    \z

% for some reason, the t͡ʃ cannot be in boldface; using \tS instead in this case (see localcommands)

As demonstrated by previous examples, \textit{aŋi} can take nominal or verbal suffixes, mirroring the shape of the target. (\ref{example6.11}) shows that the mirroring can be partial: the inflectional affix (\textit{-waːt} ‘\textsc{imper.1pl.incl}’) is copied whereas the derivational one (\textit{-naː} ‘\textsc{prgrn}’) is not. However, there are some examples where \textit{aŋi} is used with no suffixes at all. In (\ref{example6.27}), both strategies are followed. It is worth noting that the same speaker also uses verbal affixes with \textit{aŋi} in other examples.

\ea
    \label{example6.27}
    \gll bi	nuŋanman	\textbf{aŋi} 	\uline{sabira-∅-m}	i	kuŋakan     \textbf{aŋ-duː}	hapoki-kaːn-tikiː	rezin-tikiː 	\uline{resinowij-duː}   hisʲi-hi-ŋnə-∅-m\\
    1\textsc{sg}	3\textsc{sg}.\textsc{acc}	whatsitsname	gather-\textsc{nfut}-1\textsc{sg}	and.R	child   whatsitsnam-\textsc{dat.loc}	boot-\textsc{atten}-\textsc{all}	rubber-\textsc{all}	rubber-\textsc{dat.loc}   shove-\textsc{incep}-\textsc{hab}-\textsc{nfut}-1\textsc{sg}\\
    \glt `I whatsitsname, gathered it (the antenna) and put it into whatsitsname, child’s rubber boot.' (L. D. Utukogir, Khantayskoye Ozero, 2011)\\
    \z

In (\ref{example6.28}), it is hard to distinguish between the placeholder and the interjective use of \textit{aŋi}.

\ea
    \label{example6.28}
    \gll eːkun 	ta-wər	gun-ďə-rə-n	\textbf{aŋi}	lutʃa-l	kokoldo-l-tin	zə\\
    what 	that-\textsc{rfl}.\textsc{pl}	say-\textsc{ipfv}-\textsc{nfut}-3\textsc{sg}	whatsitsname	\ili{Russian}-\textsc{pl}	mitten-\textsc{pl}-\textsc{3pl.poss}	\textsc{foc}.R\\
    \glt `What’s that? – he says. – (It’s) whatsitsname, Russians’ mittens.' (L. A. Yeryomina speaking to M. D. Turskaya, Khantayskoye Ozero, 2011)\\
    \z

If we denote the suffix set of \textit{aŋi} with \textbf{\textit{AS}} and the suffix set of the target word with \textbf{\textit{TS}}, we can theoretically consider the following cases:



\begin{enumerate}
    \item AS = TS (the sets are equal)
    \item AS \(\subseteq\) TS, AS ≠ TS (AS is a strict subset of TS)
    \item TS \(\subseteq\) AS, AS ≠ TS (TS is a strict subset of AS)
    \item AS \(\cap\) TS = ∅ (AS and TS have nothing in common)
    \item not(AS \(\subseteq\) TS), not(TS \(\subseteq\)  AS), AS \(\cap\) TS ≠ ∅ (there is a non-empty intersection of suffixes in AS and TS)
\end{enumerate}

\tabref{example6.4} shows the distribution of these cases in our corpus for nominal and verbal forms separately.

\vfill
\begin{table}[H]
\begin{tabularx}{\textwidth}{ l Q rr rr }
  \lsptoprule
& & \multicolumn{2}{c}{nominal forms} &	\multicolumn{2}{c}{verbal forms}\\
  \midrule
1. &	equal sets &	85 & (45.2\%) &	51 & (37.8\%)\\
2. &	AS is a strict subset of TS &	18 & (9.6\%) &	19 & (14.1\%)\\
3. &	TS is a strict subset of AS &	9 & (4.8\%) &	6 & (4.4\%)\\
4. &	AS and TS have nothing in common &	0 &  & 0& 	\\
5. &	there is a non-empty intersection of suffixes in AS and TS &	0 &	& 0\\
6. &	other cases (no target form, slips of the tongue etc.) &	76 & (40.4\%) &	59 & (43.7\%)\\
\midrule
 & total &	188 & (100\%) &	135 & (100\%)\\
  \lspbottomrule
\end{tabularx}
\caption{Suffix mirroring according to the corpus}
\label{table:6.4}
\end{table}\vfill\pagebreak

We compare the suffix sets only in case the \ili{Evenki} target word was actually used. Therefore, the cases when the speaker did not actually pronounce the target word or shifted to \ili{Russian} are included into “other cases”. However, even when the target word is lacking, the placeholder and interjective uses of \textit{aŋi} can usually be distinguished with the help of \textit{aŋi} forms and the context, such as the speaker’s explanations in \ili{Russian}.

It can be seen that full mirroring occurs in most cases. The cases of partial mirroring can be explained with several reasons:

\begin{description}
  \item [Case 1:] The suffix which is absent from the placeholder form can be better described as derivational rather than inflectional. For example, in (\ref{example6.29}) the causative suffix in \textit{juːbdoːn} ‘to make go out’ lacks in the placeholder:

\ea
    \label{example6.29}
    \gll toːliː	dolboː	baldiː-ŋahiː-w 	amiː-m 	gənnoː-saː-n umukoːn	atirkaːnmə 	minə	baldiː-ďa-rakiː-w   \textbf{aŋi-daː-n}	juː-b-doː-n\\
    then	at.night	be.born-\textsc{cvsim}-1\textsc{sg}	father-1\textsc{sg.poss}	fetch-\textsc{pst}-3\textsc{sg} one	old.woman-\textsc{acc}	1\textsc{sg}.\textsc{acc}	be.born-\textsc{ipfv}-\textsc{cvcond}-1\textsc{sg} whatsitsname-\textsc{cvpurp}-3\textsc{sg}	go.out-\textsc{caus}-\textsc{cvpurp}-1\textsc{sg}\\
    \glt `Then, at night, when I was born, my father went to fetch one old woman so that she would do that thing, make me go out. (A. I. Pankagir, Ekongda, 2007)'\\
    \z

  \item [Case 2:] The placeholder can have intensifier and diminutive suffixes, serving the aims of emphasis (\ref{example6.30}, \ref{example6.31}).

\ea
    \label{example6.30}
    \gll it͡ʃə-t-mi=ka	tuγi	\textbf{aŋi-kaːkun}	tarə	it͡ʃə-t-mi=doː gun-ďəŋoː-n	fsʲigda	bəjə	gun-ďəŋoː-n tar	wojennij	nuŋan\\
    see-\textsc{dur}-\textsc{cvcond}=\textsc{foc}	so	whatsitsname-\textsc{ints}	тот.\textsc{acc} see-\textsc{dur}-\textsc{cvcond}=\textsc{foc} say-\textsc{fut}-3\textsc{sg}	always.R	person	say-\textsc{fut}-3\textsc{sg} that	military	3\textsc{sg}\\
    \glt `When someone sees – (he is) very whatsitsname, when someone sees, they will say… A person will always say that he is a military man.' (G. K. Lapuko, Tura, 2008)\\
    \ex
    \label{example6.31}
    \gll a	tar	tuliː-gido-n	talu	aŋi \textbf{aŋi-sʲi-kaːkuːn}	bi-fkiː	tar\\
    and.R	that	outside-\textsc{side}-\textsc{3sg.poss}	birch.bark	whatsitsname    whatsitsname-\textsc{atr}-\textsc{ints}	be-\textsc{phab}	that\\
    \glt `And on the outside there is usually birch bark whatsitsname, with whatsitsname.' (S. P. Mukto, Uchami, 2014)\\
    \z

  \item [Case 3:] the meaning of the suffix which is absent from the target form is incorporated in the word stem. For example, in verbal forms the most frequently omitted suffixes are the so-called inchoative, inceptive, durative, and stative aspects (called subaspects in \citealt[30]{BulatovaGrenoble1999}). In (\ref{example6.32}) the target word does not have the inchoative suffix, whereas the placeholder has it:

\ea
    \label{example6.32}
    \gll taduk	\textbf{aŋ}	nuŋan	\textbf{aŋi-l-da-n}  \uline{himuːrga-ra-n} 	tar 	ʃaman	ʃamani-tkaːn=tə\\
    then	whatsitsname	3\textsc{sg}	whatsitsname-\textsc{inch}-\textsc{nfut}-3\textsc{sg}  become.silent-\textsc{nfut}-3\textsc{sg} 	that 	shaman 	shaman-\textsc{child}=\textsc{foc}\\
    \glt `Then she started doing that, became silent, that shaman, little shaman.' (G. K. Lapuko, Tura, 2008)\\
    \z
\end{description}

We can suppose that the stem \textit{himuːrga-} ‘become.silent’ already has an inchoative meaning, so it is not necessary to use the inchoative suffix. However, to prove this, a separate survey on the lexical restrictions for the stems in question should be carried out.

Finally, the target word used by the speaker may sometimes be not the target word originally intended. Thus, partial mirroring can show the speaker’s doubts, whereas the original intention cannot be retrieved.


\subsubsection{Frequency}

In our data, \textit{aŋi} is quite frequent, occurring 350 times in a corpus of about 27,700 running words, i. e. about 12.6 times per one thousand words. This is much higher than the rates cited in \citet{Podlesskaya2010} (5–6.7 per thousand), which may be explained by the lack of proficiency in some speakers. Actually, most speakers do not use the \ili{Evenki} language in their daily life, and text generation presents difficulties for some of them, with lexical production being more challenging than following grammar rules. Many passive \ili{Evenki} speakers have no trouble declining a noun or conjugating a verb, including participial or converbial forms. However, recalling the exact lexemes demands much more effort from them. As a result, texts produced by such speakers might be grammatically correct but have nearly all meaning words replaced by placeholders.


\subsubsection{Dialectal variation}

According to the corpus, \textit{aŋi} is used in the dialects of the \il{Evenki!Southern}Southern and \il{Evenki!Northern}Northern dialect groups: \ili{Sym}, Podkamennya \ili{Tunguska}, and \ili{Ilimpeya} dialects. However, the word seems to be absent from the Far Eastern Tugur-\ili{Chumikan} and \ili{Sakhalin} dialects: it does not occur in texts, and the speakers do not recognize it in context. We have little spoken data from other \il{Evenki!Eastern}Eastern dialects.


\subsubsection{Possible source and evidence from related languages}

In \citet[299--302]{Idiatov2007}, the functions of \textit{aŋi} as both a placeholder word and an interrogative pronoun are discussed. The author also states a hypothesis about its origin, tracing it to an old genitive form of a word originally meaning ‘thing’ or, alternatively, “a fossilized genitive of the … ‘what’ root” (which can be found in other interrogative pronouns).
\textit{aŋi} can also be found in the Udeghe\il{Udihe} language, a relative of \ili{Evenki} (\citealt[361, 362]{NikolaevaTolskaya2001}). In Udeghe\il{Udihe}, the target words for \textit{aŋi} can be both verbs and nouns (including proper names), and \textit{aŋi} tends to mirror the grammatical shape of the target. Furthermore, it can function as an indefinite pronoun.

In \ili{Uilta} (Orok), a Southern Tungusic language, \textit{aŋŋu} is a placeholder word (\citealt[301]{Idiatov2007}, citing \citealt[I: 45]{Cincius1975K}). According to our \ili{Uilta} field data, its target words can be both verbs and nouns, just like in Udeghe\il{Udihe}, and it also has the mirroring feature.


\subsection{\textit{uŋun}}

\textit{uŋun} is a named-entity placeholder. To my knowledge, this stem has not been reported in \ili{Evenki} grammars yet.

\subsubsection{Functions as a placeholder}

In the texts, \textit{uŋun} substitutes proper nouns: names of people (\ref{example6.33}, \ref{example6.34}) or animals in tales (\ref{example6.35}, \ref{example6.36}), as well as geographical terms (\ref{example6.37}, \ref{example6.38}).

\ea
    \label{example6.33}
    \gll \textbf{uŋun}	tare	wot	gusʲə-ja	\textbf{aŋi-tin} am	\uline{ďeduʃka-tin}	haː-∅-ndә	kosin-mo\\
    whatsitsname2	that	so.R	Gusya-\textsc{coll}	whatsitsname-\textsc{3pl.poss}  \textsc{slip}	grandfather-\textsc{3pl.poss}	know-\textsc{nfut}-2\textsc{sg}	Kosin-\textsc{acc}\\
    \glt `Whatshisname, do you know, whatsitsname, the grandfather of the Gusya’s family, Kosin?' (P. K. Pankagir speaking to V. P. Khukochar, Tutonchany, 2008)\\
    \z

\ea
    \label{example6.34}
    \gll bəjə	\textbf{uŋun-mə}	\uline{dəwit-pa}	haː-∅-ndə\\
    person	whatsitsname2-\textsc{acc}	David-\textsc{acc}	know-\textsc{nfut}-2\textsc{sg}\\
    \glt `Friend, do you know whatshisname, David?' (L. F. Utukogir speaking to A. D. Chempogir, Khantayskoye Ozero, 2011)\\
    \z

\ea
    \label{example6.35}
    \gll tar	huru-rə-∅	dəgi	huwuli-n	\textbf{uŋun-dulaː}	tarə    \uline{ďantakiː-laː}\\
    that	go.away-\textsc{nfut}-3\textsc{pl}	bird	\textsc{all}-\textsc{3sg.poss}	whatsitsname2-\textsc{loc.all}	that    wolverine-\textsc{loc.all}\\
    \glt `So all birds went to whatshisname, to the wolverine.' (V. Kh. Yoldogir, Chiringda, 2007)\\
    \z

\ea
    \label{example6.36}
    \gll tar	guː-sə	\textbf{uŋun}	\uline{heːŋaːn}	guː-səː\\
    that	say-\textsc{pant}	whatsitsname2	burbot	say-\textsc{pant}\\
    \glt `So said whatshisname, the burbot said. (it is not clear whether \textit{uŋun} refers to the burbot or to the other character, the fox)’ (“Burbot and fox”, V. T. Yoldogir, Chiringda, 2011 (IEA RAS)\footnote{  \url{http://corpora.iea.ras.ru/corpora/describe_text.php?id=35}})\\
    \z

\ea
    \label{example6.37}
    \gll eː	nu	ər-tikiː zə	\textbf{uŋun-tikiː}	bi-nə   bi-rkə-∅	nawerna	ərə	walok-tuk\\
    \textsc{intj}	\textsc{intj}	this-\textsc{all}	\textsc{foc}	whatsitsname2-\textsc{all}	be-\textsc{cvsim}    be-\textsc{prob}-3\textsc{sg}	perhaps	that	Valyok-\textsc{abl}\\
    \glt `Yes, it <the settlement discussed previously> was perhaps in the direction of whatsitsname, in the direction from Valyok.' (L. A. Yeryomina speaking to M. D. Turskaya, Khantayskoye Ozero, 2011)\\
    \z

\ea
    \label{example6.38}
    \gll \textbf{uŋun-duk}	ə	\textbf{aŋi-l}	ďa-li-n \textbf{aŋi-duk}	\uline{gulə-l-duk}	əmə-rə-∅=dəː    isʲə-noː[-rə-∅]	\textbf{aŋi-laː}\\
    whatsitsname2-\textsc{abl}	\textsc{intj}	whatsitsname-\textsc{pl}	relative-\textsc{pl}-\textsc{3sg.poss}   whatsitsname-\textsc{abl}	house-\textsc{pl}-\textsc{abl}	come-\textsc{nfut}-3\textsc{pl}=\textsc{foc}   see-\textsc{prgrn}[-\textsc{nfut}-3\textsc{pl}]	whatsitsname-\textsc{loc.all}\\
    \glt `His whatsitsname, relatives came from whatsitsname, village, came and went to see to whatsitsname.' (G. K. Lapuko, Tura, 2008)\\
    \z

There is actually one example from Mutoray (\il{Evenki!Southern}Southern dialect group) where \textit{uŋun} is probably a placeholder for a common noun meaning ‘hole in the ice’ and not a proper noun. However, it is hard to judge from the context as the speaker does not actually pronounce the word. Importantly, the narrator tells the tale in the presence of her husband, who makes comment to what she says, so this can also be regarded as a sort of a dialogue:

\ea
    \label{example6.39}
    \gll i	tɨma	nu	bira	\textbf{uŋun=kə}	həgdɨ-məmə  i	globokaj	oj	suŋta	\uline{muː}\\
    and.R	in.morning.\textsc{slip}	\textsc{intj}.R	river	whatsisname2=\textsc{foc}	big-\textsc{ints}    and.R	deep.R	\textsc{intj}	deep	water	\\
    \glt `In the morning, well, the river, whatsitsname [a hole in the ice], large and deep water.' (I. M. Yastrikova, Mutoray, 2007)\\
    \z

It can be seen from most of these examples that, when \textit{uŋun} is used in dialogues, the speaker often asks the interlocutor to help recall the missing target. This brings \textit{uŋun} into a sharp contrast with \textit{aŋi}. With \textit{aŋi}, the target word can usually be restored from the context, even if not pronounced. With \textit{uŋun}, it is important to recall the exact name of a person or a place. This is perhaps the reason for its being used in dialogues with the inhabitants of the same settlement, who have the same background knowledge. We can say that \textit{aŋi} is a placeholder for a word, whereas \textit{uŋun} is a placeholder for the notion: \textit{aŋi} helps make the narrative fluent acting like a joker, whereas \textit{uŋun} brings the interlocutor’s attention to the word being missed.

Similar conclusions on the interactional use of a placeholder in Estonian have been made in \citet{Keevallik2010}.


\subsubsection{Restrictions on the target constituent}

As shown before, the target constituent is a proper noun.

\subsubsection{Functions other than those of a placeholder}

None have been found.

\subsubsection{Mirroring the grammatical shape}

\textit{uŋun} mirrors the grammatical shape of a noun, copying case markers. Proper nouns in \ili{Evenki} do not usually have alienable or inalienable possession markers. Plural number markers are theoretically possible but rare. Therefore, there is no surprise that \textit{uŋun} has no number or possession markers in our data.


\subsubsection{Frequency}

In our data, \textit{uŋun} is quite rare, occurring only 13 times in a corpus of about 27,700 running words. As shown above, it occurs mainly in dialogues between several \ili{Evenki} speakers, and such dialogues are rare in our corpus. In the IEA RAS corpus, it only occurs 2 times in a corpus of 121,286 running words (the majority of the texts are, however, written texts), both times referring to the name of an animal in a tale.

\subsubsection{Dialectal variation}

In our corpus, \textit{uŋun} is only found in the \ili{Ilimpeya} dialect texts (\il{Evenki!Northern}Northern dialect group), and in Podkamennaya \ili{Tunguska} texts (\il{Evenki!Southern}Southern dialect group). However, a speaker from Nakanno (Irkustk oblast, \ili{Yerbogachyon} dialect, \il{Evenki!Northern}Northern dialect group) recalled this word being used in the past by elderly people, although she was not entirely sure. Speakers of the Tugur-\ili{Chumikan} dialect (Khabarovsk krai, in the Far East of Russia, \il{Evenki!Eastern}Eastern dialect group) did not accept the word.

\subsubsection{Possible source and evidence from related languages}

We cannot trace the origin of \textit{uŋun}, and it is not mentioned in the comparative dictionary (\citealt{Cincius1975K}).

In \ili{Negidal}, a close relative of \ili{Evenki}, \textit{uŋun} is used as a general purpose placeholder for both nominal and verbal stems, mirroring the target word grammar, e. g. in a text from the \ili{Negidal} corpus \citep{Pakendorf2017}:\footnote{\url{https://elar.soas.ac.uk/Record/MPI1084918}. The original glossing of \textit{uŋun} as \textsc{hesit} is preserved.}

\ea
    \label{example6.40}
    \ili{Negidal}\\
    \gll net	baka-ja-βun	\textbf{uŋun-ma}\\
    no.R	find-\textsc{nfut}-1\textsc{pl.excl}	\textsc{hesit}-\textsc{acc}\\
    \glt `No, we found a whatchamacallit.' (A. V. Kazarova, Vladimirovka, 2017)\\
    \z

% changed glossing for consistency, e.g. 1EXCL.PL to 1PL.EXCL

\ea
    \label{example6.41}
    \ili{Negidal}\\
    \gll iʨe-mi	hoŋte	mesto-duki-n	tak	   moʐno	\textbf{uŋun}	məjga-ʨa	možno   t͡ʃto	rjukzak=to	minə-βə 	muː-duk-in  \textbf{uŋun-ʨa}	\uline{ɟaβu-ʨa-ʨa}\\
    see-\textsc{ss.cond}	other	place.R-\textsc{abl}-3\textsc{sg}	so.R    be.able.to.R	\textsc{hesit}	think-\textsc{pst}	be.able.to.R    that.R	rucksack.R=\textsc{ptl}.R	1\textsc{sg.obl}-\textsc{acc}.\textsc{def}	water-\textsc{abl}-3\textsc{sg}  \textsc{hesit}-\textsc{pst}	take-\textsc{res}-\textsc{pst}\\
    \glt `Looking from the side one could think that it was the rucksack that was holding me up.' (A. V. Kazarova, Vladimirovka, 2017)\\
    \z

In the dictionary of \ili{Even}, another close relative (\citealt[271]{Robbek2005}), \textit{uŋ} is glossed as “interjection” which has the meaning “pause”. \citet{Matić2008} shows that it is typical for the Eastern Even\il{Even!East} dialects. Arkady Taraboukine, a native speaker of \ili{Even} born in \ili{Beryozovka} and living in Anyuysk, gave the following examples of how it could be used.

\ea
    \label{example6.42}
    \ili{Even}\\
    \gll ťiɲiw	bi	bəri-ri-w	\textbf{uɲ-u}	\uline{halka-w}\\
    yesterday	1\textsc{sg}	lose-\textsc{nfut}-1\textsc{sg}	whatsitsname2-\textsc{acc}	hammer-\textsc{acc}\\
    \glt `Yesterday I lost whatsitsname, a hammer.' (A. Taraboukine, \ili{Beryozovka}, 2020)\\
    \z

	\ea
    \label{example6.43}
    \ili{Even}\\
    \gll bi	\textbf{uŋ-ďi-m}	\uline{mərgət-t͡ʃi-m}\\
    1\textsc{sg}	whatsitsname2-PRS-1\textsc{sg}	think-\textsc{prs}-1\textsc{sg}\\
    \glt `I am doing that thing, thinking.' (A. Taraboukine, \ili{Beryozovka}, 2020 )\\
    \z

Therefore, in \ili{Even}, just like in \ili{Negidal}, \textit{uŋ} is used as a general purpose placeholder for both nominal and verbal stems, mirroring the grammatical features of the target word.

We have no information of the stem \textit{uŋ(un)} being used in \ili{Southern Tungusic} languages.

To sum up, the \textit{uŋ(un)} stem can be found in \ili{Northern Tungusic} languages with its function ranging from a general placeholder in \ili{Even} and \ili{Negidal} to a proper noun placeholder in some \ili{Evenki} dialects. Interestingly, it was not found in the Tugur-\ili{Chumikan} dialect (at least in elicitation experiments), otherwise quite close to the \ili{Even} language both geographically and linguistically. In all these languages, \textit{uŋ(un)} mirrors the grammatical features of the target word.


\subsection{\textit{eː(kun)}}

\textit{eː(kun)} is an interrogative pronoun meaning ‘what/who’. \textit{eː(kun)} can also have a shortened stem \textit{eː-} (\citealt[137]{Konstantinova1964}), mostly in oblique forms. In \citet{Poppe1977} as well as in \citet[I: 286]{Cincius1975K}, \textit{-kun} is considered to be a morpheme, with \textit{eː} being the original stem. According to \citet[303--308]{Idiatov2007}, it can refer to objects, animals and to humans but only when questioning their “kind” (for example, their belonging to a clan). The meaning is different in various dialects, with \ili{Vanavara} dialect (\il{Evenki!Southern}Southern dialect group) speakers more approving of its referring to humans. Indefinite and negative pronouns are formed from the interrogative pronominal base (\citealt[25]{BulatovaGrenoble1999}). \textit{eː-}/\textit{ə-} is also the stem of the question verb ‘what to do?’

\subsubsection{Functions as a placeholder}

\textit{eː(kun)} serves as a placeholder for both nominal and verbal stems:

\ea
    \label{example6.44}
    \gll ďukt͡ʃa	\textbf{eːkun-ma-n}	ham-na-∅	\uline{urkə-wə-n}	ham-na-∅\\
    tent	what-\textsc{acc}-\textsc{3sg.poss}	close-\textsc{nfut}-3\textsc{pl}	door-\textsc{acc}-\textsc{3sg.poss}	close-\textsc{nfut}-3\textsc{pl}\\
    \glt `They close the tent’s whatsitsname, its door.' (N. G. Yegorova, \ili{Yerbogachyon}, 2016)\\
    \z

\ea
    \label{example6.45}
    \gll ə-wkiː	minə	\textbf{ə-ra}	\uline{əγi-fkoːn-ə}\\
    \textsc{neg}-\textsc{phab}	1\textsc{sg}.\textsc{acc}	what-\textsc{pneg}	play-\textsc{caus}-\textsc{pneg}\\
    \glt `She did not let me do that thing, play.' (V. I. Udygir, Kislokan, 2008)\\
    \z

\subsubsection{Restrictions on the target constituent}

In the examples considered, \textit{eː(kun)} can substitute for both nominal and verbal roots.

\subsubsection{Functions other than those of a placeholder}

The functions of \textit{eː(kun)} as an interrogative, indefinite or negative pronoun have already been discussed.

\subsubsection{Mirroring the grammatical shape of the target word}

The examples show that \textit{eː(kun)} mirrors the grammatical shape, sometimes partially. Due to the scarcity of the data, I will not discuss the percentage of partial vs full mirroring.
Like \textit{aŋi}, \textit{eː(kun)} is sometimes used as an interjective hesitation marker:

\ea
    \label{example6.46}
    \gll \textbf{eːkun=ka}	nawerna	ŋinaki-r	kiki-rka-l	kujiː-koːt-t͡ʃə-nə-l\\
    what=\textsc{foc}	perhaps.R	dog-\textsc{pl}	bite-\textsc{prob}-3\textsc{pl}	fight-\textsc{ints}-\textsc{ipfv}-\textsc{cvsim}-\textsc{pl}\\
    \glt `Well, perhaps the dogs bit <it>, when they were fighting.' (V. N. Saygotin, Bolshoye Sovetskoye Ozero, 2007)\\
    \z

An anonymous reviewer suggests that it could be a calque of \ili{Russian} \textit{chto zhe} что же ‘what so’, used sometimes as an interjective hesitation marker. I think it might be difficult to prove or confute it with little data on how discourse markers are generally calqued in \ili{Evenki}. However, I still think it is not so. \textit{Chto zhe} sounds quite formal, and is not wide-spread in \ili{Russian} colloquial speech. The speakers who use \textit{eː(kun)=ka}, though bilingual in \ili{Evenki} and \ili{Russian}, are not exposed much to the formal \ili{Russian} style. Actually, prosodically and functionally \textit{eː(kun)=ka} more closely resembles \ili{Russian} \textit{eto} это ‘this’ used very often as a hesitation marker, by native \ili{Evenki} speakers, too, when they are speaking \ili{Russian}.

Another important function of \textit{eː(kun)} is its use when listing several objects of a kind, at the end of such enumerations, e. g.:

\ea
    \label{example6.47}
    \gll muldiː-ka-r	ərəgəri-t	eː-wa=da    doku-ďa-miː=da	\textbf{eː-ďa-miː=da}\\
    not.be.able-\textsc{nmlz}-\textsc{pl}	at.all-\textsc{advz}	what-\textsc{acc}=\textsc{foc}    write-\textsc{ipfv}-\textsc{cvcond}=\textsc{foc}	what-\textsc{ipfv}-\textsc{cvcond}=\textsc{foc}\\
    \glt `(They were) not able at all to write anything or do such things.' (V. Kh. Yoldogir, Chiringda, 2007)\\
    \z

\ea
    \label{example6.48}
    \gll umukoː-riktə	\textbf{aŋi}	\uline{tar}	\uline{ahiː}	moːni-n ďuː-duː	bi-ďə-ri	tari-rikta	bi-ŋkiː-n	sat-tɨ-fkaːn-ďə-nə  \textbf{ə-ďə-nə}\\
    one-LIM	whatsitsname	that	woman	\textsc{rfl}-\textsc{3sg.poss}   tent-\textsc{dat.loc}	be-\textsc{ipfv}-\textsc{psim}	that-LIM	be-\textsc{pstiter}-3\textsc{sg}  tea-\textsc{vblz}-\textsc{caus}-\textsc{ipfv}-\textsc{cvsim}    what-\textsc{ipfv}-\textsc{cvsim}	\\
    \glt `Only one whatsitsname, that woman, who was in her own tent, only she gave tea to drink and did such things.' (E. K. Khukochar, Tura, 2014)\\
    \z

\ea
    \label{example6.49}
    \gll toʐə	ɲimŋakaːn=li	\textbf{eːkun=li}\\
    also.R	tale=\textsc{q}.R	what=\textsc{q}.R\\
    \glt `Also a tale or what.' (I. I. Tsurkan, \ili{Yerbogachyon}, 2016)\\
    \z

In (\ref{example6.50}), both stems of \textit{eː(kun)} are used: \textit{eːkun} as a placeholder and \textit{eː-} in the enumeration.

\ea
    \label{example6.50}
    \gll walok-tulaː	toʐə	\textbf{eːku-r-wa}	\uline{oldo-ŋi-l-wa}	\textbf{eː-l-wa} əmə-wu-pkiː-l	bi-t͡ʃo-l\\
    Valyok-\textsc{loc.all}	also.R	what-\textsc{pl}-\textsc{acc}	fish-\textsc{ind.poss}-\textsc{pl}-\textsc{acc}	what-\textsc{pl}-\textsc{acc} come-\textsc{tr}-\textsc{phab}-\textsc{pl}	be-\textsc{pant}-\textsc{pl}\\
    \glt `They also carried whatsitsname, fish and such to Valyok.' (L. A. Yeryomina speaking to M. D. Turskaya, Khantayskoye Ozero, 2011)\\
    \z

The enumeration function of \textit{eːkun} is close to what is described for Udeghe\il{Udihe} in \citet{TolskayaTolskaya2008}. In Udeghe\il{Udihe} a repetition of the verbal form with the interrogative `what' is used in the formation of open alternative questions.


\subsubsection{Frequency}

In our data, \textit{eːkun} and \textit{eː-} are used 32 times (out of 27,700 running words) in the function of placeholders.


\subsubsection{Dialectal variation}

\textit{eː(kun)} as a placeholder is used in texts from the Bolshoye Sovetskoye Lake, Sovrechka, Ekongda, and Kislokan (\ili{Ilimpeya} dialect, \il{Evenki!Northern}Northern group), \ili{Yerbogachyon} (\ili{Yerbogachyon} dialect, \il{Evenki!Northern}Northern group), \ili{Sym} and Bely Yar (\ili{Sym} dialect, \il{Evenki!Southern}Southern group), and \ili{Poligus} (\ili{Poligus} dialect, \il{Evenki!Southern}Southern group). There is an intersection between \textit{aŋi} and \textit{eːkun} areas, although texts from Bolshoye Sovetskoye Lake and Sovrechka lack the otherwise very frequent \textit{aŋi}, which suggests some dialectal variation.


 \subsubsection{Possible source and evidence from related languages}
 
\textit{ eː(kun)} can be both used as a normal question word and as a placeholder by the same speakers. When used as a placeholder or an interjective hesitation marker, the \textit{=ka} focus particle is sometimes attached, like in (\ref{example6.46}) or in (\ref{example6.51}):
 
 \ea
    \label{example6.51}
    \gll it͡ʃəː-rə-w 	\textbf{eːkun-ma}	\textbf{eːkun-ma=ka}	\uline{kiran-t͡ʃikaːn-mə}   toγo-t-t͡ʃə-riː-wə	ďagda-duː\\
    see-\textsc{nfut}-1\textsc{pl.excl}	what-\textsc{acc}	what-\textsc{acc}=\textsc{foc}	eagle-\textsc{child}-\textsc{acc} sit-\textsc{dur}-\textsc{ipfv}-\textsc{psim}-\textsc{acc}	pine-\textsc{dat.loc}		\\
    \glt `We saw whatsitsname, whatsitsname, a little eagle sitting on a pine.' (G. P. Boyarin, \ili{Sym}, 2009)\\
    \z

% corrected : to ː here and in the following

\section{Conclusions}

\ili{Evenki} speakers employ various placeholders that mirror the grammatical form of the target word.These placeholders have different discourse functions: \textit{aŋi} and \textit{eːkun} are general purpose placeholders which provide speech fluency, whereas \textit{uŋun} requires interaction from the interlocutor. There seems to be no difference between \textit{aŋi} and \textit{eːkun} when used as placeholders, although there is a hypothesis at a dialectal variation. \textit{aŋi} and \textit{eːkun} have usages other than those of a placeholder, which is typologically typical for placeholders, whereas \textit{uŋun} is only registered as a placeholder in our materials.

\textit{uŋun}, a placeholder with obscure etymology, has been found in western \ili{Evenki} dialects, in \ili{Even} and in \ili{Negidal}. The westernmost and easternmost idioms have no contact nowadays, which suggests an ancient origin of \textit{uŋun}. Interestingly, according to the considered materials, it is only in \ili{Evenki} that \textit{uŋun} has a special restriction on the target word, being a proper noun placeholder. \textit{aŋi} is also quite wide-spread, as it is present in western \ili{Evenki} dialects (i. e., in the Northern sub-branch of the Tungusic family\il{Northern Tungusic}) and in two languages of the Southern sub-branch:\il{Southern Tungusic} \ili{Uilta}, and Udeghe\il{Udihe}. However, there is quite little data on placeholders in Tungusic languages in general. It is urgent to study discourse and, specifically, the use of placeholders in the Tungusic languages, especially given their endangered status and the decline of communication in these languages.

According to the corpus data, there are some regularities in placeholders copying intensifier affixes from the target word but not other derivational affixes, or, for example, voice slots. Nevertheless, it should be studied in elicitation experiments whether such copying is theoretically possible. The corpus data also suggests restrictions on the part of speech of the target word even for general purpose placeholders (\textit{aŋi} and \textit{eːkun}), which should also be tested with elicitation. However, direct elicitation experiments for the placeholders proved to be inefficient due to the low status of these words. A different experiment design, such as asking to fill in the gap, should be attempted. Another important lacuna is the prosodic features of the placeholders. In this paper, I do not look into prosodic features of the placeholder verbs in great detail. It should also be studied using the available oral corpora with annotated multimedia content.


\section*{Non-standard abbreviations}

\ili{Russian} words are indicated with an R. Grammatical abbreviations include:

\begin{multicols}{2}
\begin{tabbing}
\textsc{ind.poss}\hspace{1ex} \= indirect (alienable) possession\kill
\textsc{accin} \> indefinite accusative\\
\textsc{atr} \> attributive\\
\textsc{futcnt} \> immediate future\\
\textsc{cvcond} \> conditional converb\\
\textsc{cvpurp} \> converb of purpose\\
\textsc{cvsim} \> converb of simultaneous \\ \> action\\
\textsc{incep} \> inceptive\\
\textsc{inch} \> inchoative\\
\textsc{ind.poss} \> indirect (alienable) \\ \> possession\\
\textsc{pant} \> anterior participle\\
\textsc{phab} \> habitual participle\\
\textsc{pneg} \> negative participle\\
\textsc{prgrn} \> peregrinative\\
\textsc{poss} \> possession\\
\textsc{psim} \> participle of simultaneous \\ \> action\\
\textsc{psiter} \> iterative past\\
\textsc{tr} \> transitivizer\\
\end{tabbing}
\end{multicols}



% changed PS1SG to 1SG.POSS etc.
% changed INDPS to IND.POSS
% changed PL(EXCL) to PL.EXCL etc.
% changed LOCALL to LOC.ALL
% changed DATLOC to DAT.LOC

% changed ś to sʲ
% changed ń to ɲ
% changed š to ʃ
% changed ī to iː
% changed ž to ʐ

\section*{Acknowledgements}

This work was supported by the RSF (grant no 17-18-01649). I would like to thank the anonymous reviewers and the editors of the present volume for their valuable remarks.

{\sloppy\printbibliography[heading=subbibliography,notkeyword=this]}
\end{document}
