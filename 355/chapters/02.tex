\documentclass[output=paper,colorlinks,citecolor=brown]{langscibook}
\ChapterDOI{10.5281/zenodo.7053361}

\author{Natalia Aralova\affiliation{Kiel University} and Brigitte Pakendorf\affiliation{Dynamique du Langage, UMR5596, CNRS \& Université de Lyon}}

\title[The causal-noncausal alternation in Northern Tungusic languages]
      {The causal-noncausal alternation in the Northern Tungusic languages of Russia}

\abstract{Languages differ widely in the way they code causal-noncausal alternations, in which a verb event is either presented as happening by itself (the noncausal event) or as being instigated by an external causer (the causal event). Some languages, such as English, tend not to make a morphological distinction; rather, the same form of certain verbs can express both a causal and a noncausal event, depending on the context. Other languages, such as Romanian or Russian, have a strong tendency to mark the noncausal event morphologically, while yet others, such as Turkish, tend to code the causal event with morphological means (\citealt{Haspelmath1993}).

We here investigate the causal-noncausal alternation in Even, Negidal, and Evenki, three Northern Tungusic languages spoken in the Russian Federation, in a cross-linguistic perspective. In these languages, morphological means for decreasing and increasing valency predominate, although equipollence – in which both forms are morphologically marked without one being derivable from the other – is a salient strategy for verbs of destruction.
Although we find broadly comparable coding patterns in these and other Tungusic languages that are similar to what is found in other languages of Northern Asia, there are numerous intriguing differences at a fine-grained level.\smallskip\\
\keywords{Northern Asia, valence, causative, anticausative, equipollence, form-to-frequency correspondence, Tungusic}
}


\IfFileExists{../localcommands.tex}{
   \addbibresource{../localbibliography.bib}
   \usepackage{langsci-optional}
\usepackage{langsci-gb4e}
\usepackage{langsci-lgr}

\usepackage{listings}
\lstset{basicstyle=\ttfamily,tabsize=2,breaklines=true}

%added by author
% \usepackage{tipa}
\usepackage{multirow}
\graphicspath{{figures/}}
\usepackage{langsci-branding}

   
\newcommand{\sent}{\enumsentence}
\newcommand{\sents}{\eenumsentence}
\let\citeasnoun\citet

\renewcommand{\lsCoverTitleFont}[1]{\sffamily\addfontfeatures{Scale=MatchUppercase}\fontsize{44pt}{16mm}\selectfont #1}
  
   %% hyphenation points for line breaks
%% Normally, automatic hyphenation in LaTeX is very good
%% If a word is mis-hyphenated, add it to this file
%%
%% add information to TeX file before \begin{document} with:
%% %% hyphenation points for line breaks
%% Normally, automatic hyphenation in LaTeX is very good
%% If a word is mis-hyphenated, add it to this file
%%
%% add information to TeX file before \begin{document} with:
%% %% hyphenation points for line breaks
%% Normally, automatic hyphenation in LaTeX is very good
%% If a word is mis-hyphenated, add it to this file
%%
%% add information to TeX file before \begin{document} with:
%% \include{localhyphenation}
\hyphenation{
affri-ca-te
affri-ca-tes
an-no-tated
com-ple-ments
com-po-si-tio-na-li-ty
non-com-po-si-tio-na-li-ty
Gon-zá-lez
out-side
Ri-chárd
se-man-tics
STREU-SLE
Tie-de-mann
}
\hyphenation{
affri-ca-te
affri-ca-tes
an-no-tated
com-ple-ments
com-po-si-tio-na-li-ty
non-com-po-si-tio-na-li-ty
Gon-zá-lez
out-side
Ri-chárd
se-man-tics
STREU-SLE
Tie-de-mann
}
\hyphenation{
affri-ca-te
affri-ca-tes
an-no-tated
com-ple-ments
com-po-si-tio-na-li-ty
non-com-po-si-tio-na-li-ty
Gon-zá-lez
out-side
Ri-chárd
se-man-tics
STREU-SLE
Tie-de-mann
}
   \boolfalse{bookcompile}
   \togglepaper[2]%%chapternumber
}{}

\begin{document}
\maketitle

\section{Introduction}\label{section2.1}

% finished, waiting for feedback by the authors

The alternation between a causal and a noncausal (sometimes more specifically called inchoative) form that certain verbs can undergo has drawn a lot of scientific attention, both from a formal perspective – with a focus on only one or two languages, mainly \ili{English} – and from a typological perspective based on cross-linguistic comparison (see, among many others, \citealt{Haspelmath1993}; \citealt{Nichols2004}; \citealt{Comrie2006}; \citealt{Schäfer2009}; \citealt{Koontz2009}; \citealt{Haspelmathetal2014}; \citealt{Levin2015}). The verbs involved in this kind of alternation form pairs
\begin{quote}
    which express the same basic situation […] and differ only in that the causative verb meaning includes an agent participant who causes the situation, whereas the inchoative verb meaning excludes a causing agent and presents the situation as occurring spontaneously. (\citealt[90]{Haspelmath1993})
\end{quote}

Intriguingly, not all verbs undergo this alternation: while ‘break’ does, ‘cut’ does not (cf. \citealt[653]{Schäfer2009}). Furthermore, languages differ greatly in the way they code causal-noncausal alternations (e.g. \citealt{Haspelmath1993}; \citealt{Nichols2004}). Thus, some languages, such as \ili{English}, tend not to make a morphological distinction; rather, the same form of some verbs\footnote{These are mainly patient-preserving labile verbs denoting a change of state, verbs of motion, and some psych verbs \citep[181--182]{Zúñiga2019}.} can express both a causal and a noncausal event, depending on the context, e.g., \ili{English} \textit{break} or \textit{melt}. Other languages have a strong tendency to mark the noncausal event morphologically, as seen by \ili{Romanian} \textit{se sparge} : \textit{sparge} and \ili{Russian} \textit{lomat’sja} : \textit{lomat’} ‘break’ and \ili{Romanian} \textit{se topi} : \textit{topi} and \ili{Russian} \textit{plavit’sja} : \textit{plavit’} ‘melt’. Here and throughout the paper the first verb of each pair is the noncausal member (i.e. an intransitive verb) and the second is the causal member (i.e. a transitive verb). A third type of languages, such as \ili{Turkish}, tends to code the causal event with morphological means,\footnote{That this is just a tendency and not an obligatory rule is shown by the fact that for ‘break’ \ili{Turkish} marks the noncausal event morphologically: \textit{kırıl-} : \textit{kır-} \citep[Appendix A7]{Haspelmathetal2014}.} as shown by the translation equivalents of ‘melt’ and ‘fill’: \textit{eri-} : \textit{erit-} and \textit{dol-} : \textit{doldur-}, respectively (\citealt{Haspelmathetal2014}: Appendices). When it is the noncausal member of the pair that is derived morphologically from the causal member, such as \ili{Negidal} \textit{ʨapʨaβ-} : \textit{ʨapʨa-} ‘break’, we will use the term \textit{anticausative coding}. In contrast, when it is the causal member of the pair that is morphologically derived, as in the \ili{Negidal} pair \textit{un-} : \textit{uniβkan-} ‘melt’, we will use the term \textit{causative coding}.

Non-morphological strategies found to express the causal-noncausal alternation are: 1) syntactic (or: periphrastic) causativization, such as \textit{cause to die} in \ili{English} (which falls outside the scope of this article); 2) ambitransitivity, as is common in \ili{English}, where so-called labile verbs can express both the causal and the noncausal event, as illustrated above with ‘break’ and ‘melt’; 3) suppletion (also called lexical causativization, \citealt[25]{Zúñiga2019}), where different roots are used to express the two events, such as \ili{English} \textit{die} vs. \textit{kill}; and 4) equipollence, where the causal-noncausal alternation is formally marked, but neither form can be analysed as being derived from the other. This can be illustrated with the \ili{Negidal} pair \textit{ɟəgdə-} : \textit{ɟəgdi-} ‘burn’, where the stem ending in \textit{-ə} is intransitive and that ending in \textit{-i} is transitive, and where the bare root \textit{ɟəgd-} does not exist.

These differences in coding have been explained by the so-called degree of spontaneity of the verb event, that is, to what extent an external causer is involved in the event:
\begin{quote}
    [E]vents that are placed on the spontaneous extreme of the scale would be those that can be perceived as internally caused. The occurrence of an external cause in these events is very unlikely. The externally caused events would correspond to a wider portion of the scale of spontaneous occurrence, including not just the events on the non-spontaneous extreme of the scale, but also those in the middle of the scale. \citep[4]{Samardžić2012}
\end{quote}

A different approach holds that form-frequency correspondences might account for the coding preferences \citep{Haspelmathetal2014}: where the noncausal member of a pair occurs more frequently, it will be the causal member that is coded overtly; conversely, if the causal member is used more often, it will be the noncausal member that is marked. In a further development, Haspelmath links the notion of degree of spontaneity to the form-frequency correspondence:
\begin{quote}
    Meanings higher on the spontaneity scale tend to require longer (and more analytic) causative markers because it is less common (and hence less expected) that one uses them in a causal context, so the speaker needs to make a greater coding effort to signal the causal meaning to the hearer. Conversely, meanings lower on the scale tend to have anticausative markers because it is less common and less expected to find them in a noncausal context, so speakers need to expend coding energy to signal the noncausal meaning. (\citeyear[57]{Haspelmath2016})
\end{quote}

An additional perspective concerning the actual use of causal vs. noncausal verbs in discourse takes pragmatic considerations into account, with the causal member of a pair being considered more informative in the description of events that involve an external causer (\citealt[77--78]{Levin2015} reporting on \citealt{Hovav2014}). Thus, speakers are assumed to choose a particular member of a causal-noncausal pair “based on their intentions, their perspective on the situation being described, and the discourse context” \citep[78]{Levin2015}.

The preferred means of coding the alternation has been shown to be relatively stable over time, at least in some European languages\il{languages of Europe} (e.g. \citealt[314--317]{Comrie2006}; \citealt[45]{Plank2015}). \citet{Nichols2018}, however, argues that in certain contact situations causative coding functions as an ``attractor'', that is, languages change their profile towards more causative coding. She explains this with causative coding being more iconic: the added semantic content (an agent who causes the event) is expressed by an added element in the verb form; furthermore, causatives can fairly straightforwardly grammaticalize out of phrases with the verb ‘make’. Finally, \citet{Creisselstoappear} points out that semantic changes can affect the coding of particular verb pairs. For example, in several \ili{sub-Saharan African languages}, the pair ‘go out/put out (a fire)’ exhibits a cross-linguistically rare suppletive strategy. This can be explained by the fact that it has lexicalized out of ‘die/kill’, and in doing so has maintained the suppletive coding strategy found for ‘die/kill’.

In this article, we describe the strategies used by the three \ili{Northern Tungusic} languages spoken in the Russian Federation, namely \ili{Even}, \ili{Evenki}, and \ili{Negidal}, from both a discourse frequency and functional perspective, and discuss them in the light of cross-linguistic studies and comparative data from other languages spoken in Eurasia. We base our study on a twenty-verb meaning list proposed by \citet{Creissels2018} specifically to investigate causal-noncausal alternations (\ref{example2.1}).

\ea
\label{example2.1}
boil; break; burn; close; run out/use up; dry; fall/drop; get wet/(make) wet; go out/extinguish; increase; melt; move (here: go/bring); open; rise/raise; split; spoil; spread; stop (of humans); turn over; twist
\z

As can be seen, most of the verbs in the list involve an inanimate S/O-argument upon which an animate A-argument can act in the causal state of affairs. In this, the list differs from those used in many of the preceding studies of the causal-noncausal alternation, such as \citet{Haspelmath1993} or \citet{Nichols2004}, which included verbs with both inanimate and animate undergoer, or \citet{Nichols2018}, which focusses on nine verb pairs with animate undergoer. The impact that the choice of verb meanings has on the results of the study will be addressed in \sectref{section2.4}.

The remainder of the paper is structured as follows: In the next section we briefly introduce the three languages on which this article is based and describe our data sources. In \sectref{section2.3} we describe the strategies these languages employ to code the causal-noncausal alternation, and in \sectref{section2.4} we discuss the differences in frequency and function of these strategies among the three languages. In \sectref{section2.5} we discuss the \ili{Northern Tungusic} data from a genealogical and cross-linguistic perspective, and in \sectref{section2.6} we investigate to what extent the form-to-frequency hypothesis set up by Haspelmath et al. holds for \ili{Even} and \ili{Negidal}. We end the paper with brief conclusions in \sectref{section2.7}.


\section{The languages and data}\label{section2.2}

Although there is as yet no consensus on the internal branching of the Tungusic family tree (compare, for example, the classifications in \citealt{Atknine1997} and \citealt{Janhunen2012a}), all classifications agree that \ili{Even}, \ili{Evenki} and \ili{Negidal} belong to one branch, which we here label with the traditional term ``\ili{Northern Tungusic}’’. Within this unit, \ili{Evenki} and \ili{Negidal} are more closely related to each other than either is to \ili{Even}. 

\ili{Even} and \ili{Evenki} are spoken by small communities scattered over a vast area of Siberia, from the Yenisey in the west to the Sea of Okhotsk in the east and from the Taimyr Peninsula in the north to northern China in the south. Evens and Evenks traditionally practised highly nomadic hunting and reindeer herding, with concomitant dispersal of the individual communities, resulting in a high degree of dialectal fragmentation. For \ili{Even}, we use both published dictionaries representing the so-called standard,\il{Standard} and a text corpus comprising data from mainly two dialects:\footnote{The corpus also includes a few texts collected from three speakers of the \ili{Tompo} dialect. We were unfortunately unable to treat the individual dialects separately due to lack of data.} \ili{Lamunkhin} Even spoken in the village of Sebjan-Küöl in central Yakutia and \ili{Bystraja} Even spoken in central Kamchatka. The total \ili{Even} corpus comprises largely monologues, especially autobiographical narratives and some folklore, but also includes a few conversations. Sixty-six speakers (44 women and 22 men) of varying proficiency and aged 11 to 78 years at the time of recording contributed to the corpus, which numbers approximately 90,000 words. For \ili{Evenki}, we base our study on published dictionaries; these represent largely the southern dialects that form the basis of the so-called standard language (cf. \tabref{table:2.1}).\largerpage

\ili{Negidal} used to be spoken by a very small population of traditional fishermen and hunters settled along the lower reaches of the Amgun’ river (a tributary of the Amur), and used to comprise two dialects (\citealt{Myl’nikova1931}; \citealt{KhasanovaPevnov2003}). Nowadays, however, the \ili{Lower Negidal} dialect is already extinct, and the Upper\il{Upper Negidal} dialect is spoken with varying proficiency by only five elderly women \citep{Pakendorf2018}.\footnote{Note that \citet{Pakendorf2018} list seven speakers; however, one of them (speaker 1 in their Table 1) passed away in April 2019, and another (speaker 5) passed away in February 2020.} Our study is based on three types of sources for \ili{Negidal} (cf. Table \ref{table:2.1}): 1) We elicited the list of 20 verb meanings with two speakers (one fluent, one less so), and 2) we used the \ili{Negidal}-\ili{Russian} dictionary appended in \citet{Cincius1982} to find lexemes that the speakers hadn’t been able to remember. 3) We searched for the verb meanings in a corpus of transcribed, translated, and glossed oral recordings of the Upper dialect\il{Upper Negidal} \citep{Pakendorf2017} numbering approximately 60,000 words at time of writing and comprising fairy tales, everyday stories, descriptions and procedural texts as well as some conversations. These recordings represent nine different speakers, eight women and one man, of whom four women cannot be considered fluent anymore. Five of the women are a mother and her four daughters, and the recordings provided by the mother (now deceased; see footnote 4) and her oldest still living daughter make up the bulk of the corpus. Table \ref{table:2.1} summarizes the data sources used for this investigation as well as the abbreviations used in the text to reference the languages.

\begin{table}
\begin{tabular}{ p{3.5cm}  p{4cm}  p{3.5cm} }
  \lsptoprule
Even (Evn) &	Negidal (Neg) &	Evenki (Evk)\\
  \midrule
\citet{Cincius1952} &	List of 20 verbs elicited with 2 speakers & \citet{Boldyrev1994}\\
\midrule
Verified with \citet{Robbek2005} &	Verified and completed using \citet{Cincius1982} &	Verified with \citet{Boldyrev2000A} and \citet{Myreeva2004}\\
\midrule
Dialectal corpora of oral narratives (c. 90k words) &	Corpus of oral narratives (\citealt{Pakendorf2017}; c. 60k words)	&\\
  \lspbottomrule
\end{tabular}
\caption{Data sources}
\label{table:2.1}
\il{Evenki}\il{Negidal}\il{Even}
\end{table}


\section{Strategies of coding the causal-noncausal alternation and further valency changes in Even, Negidal, and Evenki}\label{section2.3}

The most frequent strategy found in the \ili{Northern Tungusic} languages to code the causal-noncausal alternation is morphological marking, with equipollence being fairly common as well (especially in the domain of verbs of destruction, see below); in contrast, we found only few verb meanings in \ili{Negidal} and \ili{Evenki} where an ambitransitive pair coexists with at least one pair showing morphological derivation; see (\ref{example2.2}a, b) for a \ili{Negidal} example.

\ea
    \label{example2.2}
    \ea
    \ili{Negidal} (\citealt{Pakendorf2017}: GIK\_bear: 32--33)\\
    \gll taduk	məjgɑː-ja-n	iʨe-kte	ni=lə   \textbf{huki-sin-e-n=də}	ɟaɟa-ŋi-n	tiː	daga-ma-ʨa\\
    then	think-\textsc{nfut-3sg}	see-\textsc{hort.sg}	who=\textsc{foc} turn.around-\textsc{tam1-nfut-3sg=add}	bear-\textsc{poss-px.3sg}	thus	near-\textsc{vr-pst[3sg]}\\
    \glt ‘Then he thinks, let me see who it is. He turns around, and the bear [lit. his uncle] has [already] come close like this.’ /\\‘Потом думает, давай посмотрю, кто это. Поворачивается, а дядя (=медведь) уже вот подошел.’\\

\ex
    \ili{Negidal} (\citealt{Pakendorf2017}: GIK\_shuka: 13)\\
    \gll əsi=gdə	odin	odi-l-la-n	ogda-βa-βun	\textbf{huki-sin-e-n}\\
    now=\textsc{contr}	wind	blow-\textsc{inch-nfut-3sg}	boat-\textsc{acc-px.1pl.ex}	turn.around-\textsc{tam1-nfut-3sg}\\
    \glt ‘… suddenly the wind blew and turned the boat around.’ /\\‘… вдруг ветер подул, лодку повернул.’\\
    \z
\z

Although we did not find any suppletive pairs among the 20 verb meanings that form the basis of the study, ‘die’ and ‘kill’ are expressed suppletively in all three languages. While \ili{Negidal} and \ili{Evenki} share the same forms (\textit{bu-} ‘die’ vs. \textit{βaː-} ‘kill’), \ili{Even} has distinct items  (\ili{Lamunkhin} \textit{koke-}, \ili{Bystraja} \textit{ɲoːme-} ‘die’ vs. \textit{maː-} ‘kill’ for both dialects, see (\ref{example2.3}) for an illustration).\pagebreak

\ea
    \label{example2.3}
    \ili{Lamunkhin} Even (AAS\_elk\_17)\\
    \gll … kapkan-du	họr-ʨa	tọːki	himbiːr […]	tiːla-nikan \textbf{koke-ɟi-n}	goː-mi	nọŋan	pektereː-niken	\textbf{maː-ri-n}\\
    … trap.R-\textsc{dat}	get.caught-\textsc{pst.ptcp}	elk	\textsc{ptl}.Y {}	get.thin-\textsc{sim.cvb} die-\textsc{fut-3sg}	say-\textsc{cond.cvb}	3\textsc{sg}	shoot-\textsc{sim.cvb}	kill-\textsc{pst-3sg}\\
    \glt ‘… because an elk that has gotten caught in a trap […] will starve and die anyway, he shot and killed (it).’ /\\‘… потому что попавший на капкан лось все равно […] умрет, отощав, он убил, застрелив из ружья.’\\
    \z

Verbs of destruction in the \ili{Northern Tungusic} languages make notable use of equipollence to distinguish valency (transitive vs. intransitive) and Aktionsart (semelfactive vs. iterative), with different consonantal endings coding the distinct meanings (Table \ref{table:2.2}). This is most systematic in \ili{Even}, where four different endings are found, while in \ili{Negidal} the distinction between iterative and semelfactive transitives has largely been lost, although the distinction in Aktionsart has been retained for the intransitive forms. In \ili{Evenki}, the system appears to be at most vestigial, judging from the lack of mention in descriptions (\citealt{Konstantinova1964A}; \citealt{Nedjalkov1997}; \citealt{BulatovaGrenoble1999}; \citealt{Boldyrev2007}). The forms we provide in Table \ref{table:2.2} are extracted from examples in \citet{Myreeva2004} and \citet{Boldyrev2007}, and we indicate our uncertainty about our analysis with the added question marks. The suffix \textit{-rgA}, for example, is described by \citet[228]{Nedjalkov1997} as being a general anticausative morpheme, albeit one that is mostly used with verbs of destruction or change of state. In \ili{Negidal}, the cognate form \textit{-dgA} functions as a general anticausative as well, but with verbs of destruction it gets a specifically semelfactive reading. In this language, the ending \textit{-nA} occurs very rarely, with \textit{-l} generally expressing both iterative and semelfactive transitive events. Examples (\ref{example2.4}a--d) show the full system for the \ili{Negidal} verb \textit{kalta-} ‘split, halve’, one of the few for which a separate transitive-iterative form exists. Note that the root \textit{kalta-} does not exist by itself.

{\captionsetup{font={stretch=.8,small},width=.85\textwidth}
\begin{table}
\begin{tabular}{ l  l l l  l l l }
  \lsptoprule
& \multicolumn{3}{c}{transitive} &	\multicolumn{3}{c}{intransitive}\\\cmidrule(lr){2-4}\cmidrule(lr){5-7}
&    \ili{Even} & \ili{Negidal} &   \ili{Evenki} &	\ili{Even} &	\ili{Negidal} &	\ili{Evenki}\\\midrule
    iterative &	-k &	(-nA) &	?-gA &	-m &	-m &	?-m\\
    semelfactive &	-t &	-l &	?-li {\textasciitilde} -t, (-nA) &	-r &	-dgA &	?-rgA\\
  \lspbottomrule
\end{tabular}
\caption{Consonantal endings of verbs of destruction and their meanings}
\label{table:2.2}
\end{table}
}

\ea
    \label{example2.4}
    \ea
    \ili{Negidal} (\citealt{Pakendorf2017}: DIN\_preparing\_hide: 29)\\
    \gll tiː\_ɲekomi	\textbf{kaltal-la}	noŋan-ma-n	\textbf{kaltal-la}\\
    therefore	split\textsc{[tr.smlf]-nfut[3pl]}	3\textsc{sg-acc-px.3sg}	split\textsc{[tr.smlf]-nfut[3pl]}\\
    \glt ‘That is why they cut it (the hide) in half.’ /\\‘Поэтому (шкуру) разрезают на половину.’\\

% \pagebreak

\ex
    \ili{Negidal} (\citealt{Pakendorf2017}: TIN\_stingy\_man: 69)\\
    \gll gə	osi=gdə	noŋan-ma-n	halka-l-ʨaː moŋi-l-ʨaː	dajama-βa-n	ələ	\textbf{kaltanaː-ja-n}\\
    \textsc{dp}	now=\textsc{contr}	3\textsc{sg-acc-px.3sg}	to.hammer-\textsc{inch-pst[3sg]}    hit-\textsc{inch-pst[3sg]}	back-\textsc{acc-px.3sg}	nearly	split\textsc{[tr.iter]-nfut-3sg}\\
    \glt ‘… he immediately started to beat and hit him, he nearly split his back.’ /\\‘... он стал бить, колотить его палкой, спину чуть ему не переломил.’\\

\ex
    \ili{Negidal} (\citealt{Pakendorf2017}: TIN\_monokan: 66)\\
    \gll \textbf{kaltadga-ja-n} 	tik-kə-n	ŋɑːləβki	oje-la-n\\
    split\textsc{[intr.smlf]-nfut-3sg}	fall-\textsc{nfut-3sg}	wolf	 top-\textsc{loc-px.3sg}\\
    \glt ‘It split and fell on top of the wolf.’ /\\‘Треснула и упала на волка.’\\

\ex
    \ili{Negidal} (field data, 04.08.17)\\
    \gll est’	takie	moː-l	kotorye	maːn-tin	\textbf{kaltam-ma}\\
    exist.R	such.R	tree-\textsc{pl}	which.R	self-\textsc{px.3pl}	split\textsc{[intr.iter]-nfut[3pl]}\\
    \glt ‘There are such trees which split by themselves in several places’ /\\‘Есть такие деревья, которые сами по себе раскалываются в нескольких местах.’\\
    \z
\z

Table \ref{table:2.3} shows the major morphological means by which the \ili{Northern Tungusic} languages code valency changes, including the causal-noncausal alternation. As can be readily seen, in all three languages both transitive and detransitive derivation is achieved with a polysemous suffix comprising a labial (cf. \citealt[12]{Nedjalkov}; \citealt[299]{Pakendorf2020}; see \ref{example2.5}--\ref{example2.7}); this appears to have been strengthened with the erstwhile diminutive suffix \textit{-kAn} to form the causative suffix {\textit{-βkAn}} (cf. \citealt{LiWhaley2012}). 

\begin{table}
\begin{tabular}{ l  l  l  l }
  \lsptoprule
& \ili{Even} & \ili{Negidal} & \ili{Evenki}\\
  \midrule
(Anti)causativizing & -β/-u & -β & -β\\
Adversative-passive & -β/-u & [-β] & [-β/-mu]\\
Medio-passive &	-p/-b &	-p & -p/-β\\
Causative &	-βkAn & -βkAn & -βkAn\\
  \lspbottomrule
\end{tabular}
\caption{Major valency changing morphemes in Northern Tungusic}
\label{table:2.3}
\end{table}

The labial (anti)causativizing suffix plays a role in the causal-noncausal alternation, since it can express both causative coding (\ref{example2.5}a, b) and anticausative coding (\ref{example2.6}a, b). It also functions as a general marker of valency change, such as deriving passives (\ref{example2.7}a, b). In order to cover all these functions in one gloss, \citet[215]{Pevnov2007} calls it “ambivalent voice” in his analysis of this suffix. However, it should be noted that not all the functions are equally productive \citep{Nedjalkov1993}. 

\ea
    \label{example2.5}
    \ea
    \ili{Lamunkhin} Even (Krivoshapkina\_AX\_1930s\_055)\\
    \gll upeː-ɲɟe	hiβkeŋkeːn	tar	ʨajnika-n   \textbf{huje-l-ʨe-le-n}	ʨaj-u	oŋke-ʨe-l-ʨe\\
    grandmother-\textsc{aug.def}	silently	\textsc{dist}	teapot.R\textsc{-px.3sg} boil\textsc{-inch-pst.ptcp-loc-3sg}	tea.R-\textsc{acc}	pour-\textsc{tam2-inch-pst[3sg]}\\
    \glt ‘When the teapot started to boil, grandmother quietly started to pour tea.’ /\\‘Бабушка тихонько, когда вскипел чайник, начала разливать чай.’

\ex
    \ili{Lamunkhin} Even (Krivoshapkina\_Marta\_bear\_003)\\
    \gll aːŋŋa-riɟur	ʨaj-u	igin	\textbf{huj-u-t-ʨe-le-t}\\
    stop.for.night-\textsc{ant.cvb.pl}	tea.R-\textsc{acc}	etc.Y	boil-\textsc{val-tam2-pst.ptcp-loc-1pl}\\
    \glt ‘When we had spent the night, when we were making tea, …’ /\\‘Переночевав, когда мы вскипятили чай, …’
    \z
\ex
    \label{example2.6}
    \ea
    \ili{Bystraja} Even (Egorova\_RM\_Arishal\_127)\\
    \gll aha	meːn-ken	\textbf{man-u-waːt-ta-n}\\
    aha	\textsc{refl-dim.int}	finish-\textsc{val-gnr-nfut-3sg}\\
    \glt ‘Mhm, it disappears by itself.’ /\\‘Ага, сам исчезает.’

\pagebreak
\ex
    \ili{Bystraja} Even (Tylkanova\_Lidija\_Gavrilevna\_2\_054)\\
    \gll olra-β=da	\textbf{man-ra}\\
    fish-\textsc{acc=add}	finish-\textsc{nfut[3pl]}\\
    \glt ‘They finished the fish (i.e. they exterminated the fish)!’ /\\‘Рыбу закончили!’
    \z
\ex
    \label{example2.7}
    \ea
    \ili{Negidal} (\citealt{Pakendorf2017}: DIN\_game: 31)\\
    \gll ta-duk	geː	hutə-βa-n	\textbf{ɟep-pa-n}\\
    \textsc{dist-abl}	second	offspring-\textsc{acc-px.3sg}	eat-\textsc{nfut-3sg}\\
    \glt ‘… then he [the devil] eats the second child …’ /\\‘потом второго ребенка съедает …’\\

\ex
    \ili{Negidal} (\citealt{Pakendorf2017}: DIN\_Emeksikan: 380)\\
    \gll amban-du	\textbf{ɟepu-β-ʨa}	bi-ɟa-n\\
    devil-\textsc{dat}	eat-\textsc{val-pst.ptcp}	be-\textsc{fut-3sg}\\
    \glt ‘… probably he has been eaten by the devil, ….’ /\\‘наверно, амбан его съел (наверно, он амбаном съеден), …’\\
    \z
\z

\begin{sloppypar}
Although the polysemy covering both valency-increasing and -decreasing functions might at first glance seem counter-intuitive, it is cross-linguistically not uncommon, being attested in several languages of East Asia, such as \ili{Mongolian}, \ili{Japanese}, and \ili{Korean} (\citealt[83--84]{Kazama2004}; \citealt[226]{Zúñiga2019}); it is also a common phenomenon in the Tungusic languages (\citealt[122]{Benzing1955AP}; \citealt[123--130]{Sunik1962}). Recent studies have shown that the development is likely to have taken place from the causative to the passive function (\citealt{LiWhaley2012}; \citealt{JangPayne2014}; \citealt{Nedjalkov2014}).
\end{sloppypar}

The adversative-passive is a construction that “… creates an additional argument – just as the causative does” (\citealt[131]{Palmer1994}). Furthermore, in contrast to standard passives, the subject is not the promoted direct object of the active transitive verb, but is “… an entity affected by the situation, possibly not being its participant” (\citealt[906]{Kazenin2001}). This can be seen in the \ili{Even} example (\ref{example2.8}a, b), where (\ref{example2.8}a) shows that the addressee of the bivalent intransitive verb of speech \textit{tore-} ‘speak’ is marked with dative case (which might alternate with allative or be left unexpressed); in contrast, in the adversative construction (\ref{example2.8}b) the addressee is promoted to the subject position (as seen in the verbal subject agreement).

\ea
    \label{example2.8}
    \ea
    \ili{Lamunkhin} Even (beseda\_1626)\\
    \gll ebe-di-t	\textbf{tore-ɟi-p}	nọŋan-du-n\\
    \ili{Even}-\textsc{adjr-ins}	speak-\textsc{fut-1pl}	3\textsc{sg-dat-px.3sg}\\
    \glt ‘We'll speak in \ili{Even} to him.’ /\\‘Ему по-эвенски будем говорить.’\\

\ex
    \ili{Lamunkhin} Even (AEK\_childhood\_091)\\
    \gll tọbọr	goːn-teken	emie	\textbf{tore-β-gere-re-m}	tar	ahi-du\\
    this	say-\textsc{mult.cvb}	also.Y	speak-\textsc{advrs-hab-nfut-1sg}	that	woman-\textsc{dat}\\
    \glt ‘… and again that woman would scold me/says bad things at me.’ /\\‘… опять эта женщина на меня говорит.’\\
    \z
\z

The adversative-passive is a productive category in \ili{Even} (cf. \citealt[21--26]{Malchukov1995}), but in \ili{Evenki} (\citealt[220--222]{Nedjalkov1997}) and \ili{Negidal} (\ref{example2.9}a--b) it is restricted to environment verbs. As pointed out by \citet[3]{Nedjalkov}, in \ili{Evenki} the ad\-ver\-sa\-tive-pas\-sive construction “obligatorily include[s] an animate patient, i.e. the person who is subject to a certain atmospheric phenomenon considered as adversative to this person”, “while the base verbs do not contain any `animate' semantic roles in their predicate frames”.

\ea
    \label{example2.9}
    \ea
    \ili{Negidal} (\citealt{Pakendorf2017}: GIK\_2tatarskoe: 28)\\
    \gll bu	o-ŋati-βun	ŋənə-jə	uže	\textbf{dəlbə-ŋati-n}\\
    1\textsc{pl.ex}	\textsc{neg-deont-1pl.ex}	go-\textsc{neg.cvb}	already.R	fall(night)-\textsc{deont-3sg}\\
    \glt ‘We're not going, it's already getting night.’ /\\‘… мы не поедем, уже наступит ночь.’\\

\ex
    \ili{Negidal} (\citealt{Pakendorf2017}: GIK\_kljukva: 45)\\
    \gll noŋan	goje-βa	aː-ʨa-n	noŋan	ɟali-n  bit	\textbf{dəlbə-β-ʨa-lti}\\
    3\textsc{sg}	distance-\textsc{acc}	sleep-\textsc{pst-3sg}	3\textsc{sg}	because.of-\textsc{px.3sg} 1\textsc{pl.in}	fall(night)-\textsc{advrs-pst-1pl.in}\\
    \glt ‘‎‎‎She slept for a long time, because of her we were caught by the night.’ /\\‘Она долго спала, из-за неё нас застала ночь.’\\
    \z
\z

The medio-passive derivation results in constructions in which no agent is implied (compare \ref{example2.10}b with \ref{example2.10}a). In \ili{Even} and \ili{Negidal} this is marked by a labial stop rather than the labial fricative or glide used in the (anti)causitivizing and (adversative-)passive function, but in \ili{Evenki} \textit{-p} and \textit{-β} are used interchangeably, e.g. \textit{ula-} ‘make wet, moisten’ : \textit{ula-β} {\textasciitilde} \textit{ula-p-} ‘become wet’ (\citealt[13]{Nedjalkov}).

\pagebreak

\ea
    \label{example2.10}
    \ea
    \ili{Bystraja} Even (RME\_Arishal\_042)\\
    \gll ɲan	urke-β	\textbf{aŋa-ri-βun}\\
    and	door-\textsc{acc}	open-\textsc{pst-1pl.ex}\\
    \glt ‘And we opened the door.’ /\\‘И вот дверь открыли.’\\

\ex
    \ili{Bystraja} Even (RME\_Arishal\_20)\\
    \gll iami	urke	\textbf{aŋa-p-ta-n}\\
    \textsc{ptl}	door	open-\textsc{med-nfut-3sg}\\
    \glt ‘… suddenly the door opened.’ /\\‘… вдруг дверь открылась.’\\
    \z
\z

Finally, the causative marker \textit{-βkAn} derives causatives from both intransitives (in which the causee is marked by the accusative case, as illustrated in (\ref{example2.11}b) where the morpheme appears as the allomorph \textit{-ukeŋ-}) and transitives, with variation between dative- and accusative-marking for the causee (cf. \citealt[11]{Nedjalkov}; \citealt[207]{Pevnov2007}; \citealt[302]{Pakendorf2020}).

\ea
    \label{example2.11}
    \ea
    \ili{Lamunkhin} Even (AXK\_1930s\_125)\\
    \gll edu	tuŋŋan	nimeːr	bi-niken	\textbf{tegeʨ-ʨe-l}\\
    here	five	neighbor	be-\textsc{sim.cvb}	live-\textsc{pst-pl}\\
    \glt ‘… here they lived as five families..’ /\\‘… здесь жили они в пять семей.’\\

\ex
    \ili{Lamunkhin} Even (KKK\_history\_012)\\
    \gll ebe-sel-bu	ʨele-βu-tnen	omen    tor-du	\textbf{tegeʨ-ukeŋ-gel}	goːn-ʨe-l\\
    Even-\textsc{pl-acc}	all-\textsc{acc-px.3pl}	one earth-\textsc{dat}	live-\textsc{caus-hort.pl}	say-\textsc{pst-pl}\\
    \glt ‘Let's make the Evens all live in one place …, they said.’ /\\‘Давайте всех эвенов заставим жить на одном месте …’\\
    \z
\z

When both the (anti)causitivizing suffix \textit{-β} and the causative \textit{-βkAn} can be used to encode transitivization, the difference in meaning is one of direct vs. indirect causation, as illustrated by the following examples from \ili{Negidal} (\ref{example2.12}a--c). Here, the underived verb \textit{ŋənə-} (\ref{example2.12}a) expresses an animate agent moving of his own volition, while the derived verb \textit{ŋənə-β-} (\ref{example2.12}b) means to make something go by exerting direct, physical force, i.e. by carrying it, while \textit{ŋənə-βkan-} (\ref{example2.12}c) means to cause someone to go by exerting only indirect pressure, i.e. by requesting or commanding them to go.

\ea
    \label{example2.12}
    \ea
    \ili{Negidal} (\citealt{Pakendorf2017}: GIK\_2sluchaj: 23)\\
    \gll man-si	\textbf{ŋənə-kəl}	ɟul-la	bi	amar-gida-du-s	ŋənə-ɟa-β\\
    self-\textsc{px.2sg}	go-\textsc{imp.sg}	front-\textsc{loc}	1\textsc{sg}	behind-\textsc{side-dat-px.2sg}	go-\textsc{fut-1sg}\\
    \glt ‘Go first yourself, I will go behind you.’ /\\‘Сам иди впереди, я сзади буду идти. ’\\

\ex
    \ili{Negidal} (\citealt{Pakendorf2017}: DIN\_crow: 92)\\
    \gll taj	konɟe-βa	hena-laː-ja-n	ɟo-tki-j    \textbf{ŋənə-β-βə-n}\\
    \textsc{dist}	birchbark.box-\textsc{acc}	carry.on.back-\textsc{smlf-nfut-3sg}	house-\textsc{all-prfl.sg} go-\textsc{val-nfut-3sg}\\
    \glt ‘‎‎He hoisted the box on his back and brought it home.’ /\\‘‎‎Взял этот короб и понёс домой.’\\

\ex
    \ili{Negidal} (\citealt{Pakendorf2017}: APN\_DIN\_memories: 235)\\
    \gll nuŋan	əmə-dgi-je-n	munə(-βə)	\textbf{ŋənə-βkan-a}	kamenka-la\\
    3\textsc{sg}	come-\textsc{rep-nfut-3sg}	1\textsc{pl.ex-acc}	go-\textsc{caus-nfut[3pl]}	place.name-\textsc{loc}\\
    \glt ‘He comes back and they send us to Kamenka.’ /\\‘Он возвращается, и нас отправляют на Каменку.’\\
    \z
\z

These data confirm \citegen{Levshina2016} cross-linguistic observation that the morphological marking of indirect causation (here: \textit{-βkan}) is longer than that of direct causation (here: \textit{-β}; cf. \citealt[784--788]{Haiman1983}).

\begin{sloppypar}
Thus, to summarize this section, the \ili{Northern Tungusic} languages predominantly use morphological means to mark causal-noncausal alternations, although equipollence is common in particular with verbs of destruction. Ambitransitivity and suppletion are rare, and the latter does not occur among the 20 verb pairs which form the basis of the next section, namely the investigation of the patterns of use of the different strategies.
\end{sloppypar}

\section{Patterns of causal-noncausal alternation among the 20 verb pairs}\label{section2.4}

Table \ref{table:2.4} summarizes the different coding patterns found in the three languages for each of the verb pairs; for the actual forms see the Appendices~\ref{Appendix1}–\ref{Appendix3}.\footnote{The data files are also downloadable in .csv format from: \url{http://doi.org/10.5281/zenodo.3911606}.} In the table, nC stands for “noncausal”, C stands for “causal”, and the mathematical operator indicates the direction of derivation: nC > C “causal is derived from noncausal” (causative coding); nC < C “noncausal is derived from causal” (anticausative coding); nC ≈ C “noncausal and causal are equipollent”; nC = C “noncausal and causal are expressed by the same item” (i.e. the verb is labile). As mentioned in the preceding section, we did not find any suppletive verbs among the 20 meanings.

\begin{sloppypar}
Following the methodology of previous studies (\citealt{Haspelmath1993,Comrie2006}), in those cases where we found synonymous pairs with different coding, we included them all in the dataset. However, we excluded verbs with very narrow meanings, such as \ili{Negidal} \textit{boʨo(-β)-} ‘dry out’, which refers only to hides that dry out excessively during preparation and then become unworkable. The number of synonyms and different coding patterns can be quite large (for instance, ‘burn’ in \ili{Evenki} has four different coding patterns), because we tried to cover the dialectal variation and were rather inclusive in our choice. In these cases, we counted the coding patterns proportionally to their number (e.g. each pattern for ‘break’ in \ili{Even} counts as 0.5 and each pattern for ‘burn’ in \ili{Evenki} as 0.25; cf. the Appendices \ref{Appendix1}--\ref{Appendix3}).
\end{sloppypar}

It should be noted that our choice of meaning was partly determined by the \ili{Negidal} elicitation, with which we started our data collection. For instance, since the speakers were unable to give a translation equivalent of ‘move’ (of an inanimate object), we changed this meaning to ‘move (of an animate object)’, i.e. ‘go’. Furthermore, we attempted to include only ‘‘basic’’ meanings and excluded stems where the derivation seemed to provide additional semantic content. We thus excluded forms such as \ili{Negidal} \textit{ŋənəβkan-} ‘make someone go’ as the causative counterpart for \textit{ŋən-} ‘move (go)’, since the causative suffix \textit{-βkAn} adds a meaning of indirect causation, as explained above (\ref{example2.12}c). We also excluded Evn \textit{tikuken-} {\textasciitilde} Neg \textit{tikeβkan-} {\textasciitilde} Evk \textit{tikiβkəːn-} ‘make fall, drop intentionally, unload’, since this carries a meaning of voluntary, intentional action that is absent from ‘fall/drop’.

\begin{table}\small
\il{Even}\il{Negidal}\il{Evenki}
\begin{tabular}{ l p{2.5cm} p{2.5cm} p{2.5cm} }
  \lsptoprule
Verb meaning &	Even &	Negidal & 	Evenki\\
  \midrule
boil &	nC > C &	nC > C &	nC > C\\
\midrule
break &	nC < C &	nC < C &	nC < C\\
& nC  ≈ C & nC ≈ C &\\
\midrule
burn &	nC > C &	nC ≈ C &	nC ≈ C\\
& & & nC > C\\
& & & nC = C\\
& & & nC < C\\
\midrule
close &	nC < C & nC < C &	nC < C\\
& nC ≈ C	& &\\
\midrule
run out/use up &	nC < C &	nC < C &	nC < C\\
\midrule
dry &	nC ≈ C &	nC ≈ C &	nC ≈ C\\
\midrule
fall/drop &	nC > C &	nC ≈ C &	nC > C\\
& & & nC ≈ C\\
\midrule
get wet/make wet &	nC < C &	nC < C &	nC < C\\
\midrule
go out/put out &	nC > C &	nC > C (corpus) & nC < C\\
& & nC = C (elicit.)	&\\
\midrule
increase &	nC < C &	nC < C &	nC < C\\
\midrule
melt &	nC > C &	nC > C &	nC > C\\
\midrule
move (go) &	nC > C &	nC > C &	nC > C\\
\midrule
open &	nC < C &	nC < C &	nC < C\\
\midrule
rise/raise &	nC < C & nC = C & nC < C\\
& nC > C & nC > C & nC > C\\
\midrule
split &	nC ≈ C &	nC ≈ C & nC ≈ C\\
& & nC > C &\\
\midrule
spoil &	nC < C & nC < C & nC ≈ C\\
& nC > C & nC ≈ C &\\
\midrule
spread &	nC < C &	nC < C &	nC < C\\
\midrule
stop (of humans) &	nC > C &	nC > C &	nC > C\\
\midrule
turn over (around) &	nC ≈ C & nC = C & nC > C\\
& nC > C	& nC < C & nC < C\\
\midrule
bend (twist) &	nC < C &	nC < C &	nC < C\\
  \lspbottomrule
\end{tabular}
\caption{Coding patterns in causal-noncausal verb pairs}
\label{table:2.4}
\end{table}

Given the close relationship of the languages included here, it is not surprising that the patterns we find are overall quite similar, with 15 out of the 20 verb pairs showing the same coding pattern for at least one synonym in all three languages. In contrast, what is notable is that we do find differences in the patterns based on such a small sample of verbs. For instance, for the verb pair ‘fall/drop’, \ili{Negidal} uses equipollence to code the causal-noncausal alternation (\textit{tik-} : \textit{tibgu-}\footnote{Note that while \textit{tik-} : \textit{tibgu-} is synchronically equipollent, diachronically it is likely to be a causative derivation followed by metathesis: \textit{tibgu-} < *tigbu- < *tikbu- < *tiki-bu- (\name{Aleksander M.}{Pevnov} p.c., 28.06.2020).}), whereas \ili{Even} and \ili{Evenki} use causative coding (Evn \textit{tik-} : \textit{tikəβ-}, Evk \textit{tik-} : \textit{tikiβ-}). Furthermore, all three languages derive an indirect causative with the causative suffix \textit{-βkAn}, e.g. \ili{Negidal} \textit{tikeβkanas} ‘you made me fall’. In addition to the pan-Tungusic root \textit{tik-} (\citealt[87]{Sunik1962}), \ili{Evenki} has an equipollent non-cognate pair \textit{buru-} : \textit{buriː-}. In the case of ‘go out/put out (a fire)’, \ili{Even} and \ili{Negidal}\footnote{For the \ili{Negidal} pair ‘go out/put out’, we included two coding patterns in the dataset: one is found in the corpus and the other was obtained during elicitation.} have causative coding (Evn \textit{hiːβ-} {\textasciitilde} Neg \textit{siβ-} : Evn \textit{hiːβi-}/\textit{hiːβuken-} {\textasciitilde} Neg \textit{siβi-}) in contrast to the noncausative coding found in \ili{Evenki} (\textit{siːβ-} : \textit{siː-}). It appears as if \ili{Evenki} speakers reanalyzed the root-final \textit{-β} of the noncausal form as the (anti)causativizing morpheme and from this derived the causal form by dropping the labial. For ‘rise/raise’, \ili{Even} and \ili{Evenki} have a verb pair showing anticausative coding (Evn \textit{ugərəb-} : \textit{ugər-} and Evk \textit{ugiːriβ-} : \textit{ugiːr-}, respectively) where the \ili{Negidal} cognate is labile (\textit{ugi-}); in addition, all three languages have a synonymous pair with causative coding, but here only the \ili{Negidal} and \ili{Evenki} forms are cognate (Neg and Evk \textit{tukti-} : \textit{tuktiβ-} vs. Evn \textit{ojʨi-} : \textit{ojʨiβkan-}).

Some further pattern differences we find with respect to specific verbs in the dataset are:

\begin{itemize}
    \item The verb meaning is expressed by different lexemes, which nevertheless show the same coding pattern, e.g. ‘spread’, which shows consistently anticausative coding: \textit{girkəb-} : \textit{girkə-} in \ili{Even}, \textit{gildeβ-} : \textit{gilde-} in \ili{Negidal}, \textit{səktəβ-} : \textit{səktə-} in \ili{Evenki}.

    \item Differences in coding pattern correlate with differences in lexeme form, e.g. ‘burn’, which is coded causatively in \ili{Even}: \textit{dur-} : \textit{duruken-}, but which is expressed by the equipollent pair \textit{ɟəgdə-} : \textit{ɟəgdi-} in both \ili{Negidal} and \ili{Evenki}; in addition, \ili{Evenki} has three more synonyms, none of which are cognate to the \ili{Even} or \ili{Negidal} form, and each of which shows a different coding pattern (Appendix \ref{Appendix3}).

    \item The lexemes are cognate, but the coding patterns differ, e.g. ‘fall/drop’ and ‘go out/put out (of fire)’ discussed above, or ‘spoil’, which in \ili{Even} shows causative coding: \textit{mun-} : \textit{munuken-}, but which is expressed by the equipollent pair \textit{munu-} : \textit{muniː-} in \ili{Evenki}.
\end{itemize}

In Table \ref{table:2.5} we summarize the frequency of the different coding patterns for the three languages. While in \ili{Even} anticausative and causative coding occur with approximately equal frequency, in \ili{Negidal} and \ili{Evenki} anticausative coding predominates over causative coding. This is particularly pronounced for \ili{Evenki}, where anticausative coding is nearly twice as frequent as causative coding. 

\begin{table}
\begin{tabular}{ l  rr  r }
  \lsptoprule
Relation &	\ili{Even} &	\ili{Negidal} &	\ili{Evenki}\\
  \midrule
nC > C &	8.5 &	5.5 &	5.75\\
nC < C &	8\phantom{.5} &	8.5 &	10.25\\
nC ≈ C &	3.5 &	4.5 &	3.75\\
nC = C &	-- &	1.5 &	0.25\\
  \lspbottomrule
\end{tabular}
\caption{Frequency of different causal-noncausal relations in the Northern Tungusic languages (over 20 verb pairs)}
\label{table:2.5}
\end{table}

These results offer some counterevidence to the findings of \citet[180]{Nichols2004}, who state that “[f]rom eastern North America across the Bering Strait and through Siberia there is a large region marked by a strong preference for augmentation [i.e. causative coding]”. These differences in results are likely to be due to the different verb meanings included in the studies: as mentioned in the Introduction, \citet{Nichols2004} based their investigation on 18 verb pairs, of which nine have an animate undergoer such as ‘laugh’ or ‘sit’, and only nine have an inanimate undergoer and therefore partly overlap with the verb meanings included here.\il{languages of North America}\il{languages of Siberia}

The impact of the verb meanings on the coding patterns can be further seen from data on \ili{Even} and \ili{Evenki} presented in a recent follow-up study by \citet{Nichols2018}. This is based on only the nine verb meanings with animate undergoers from the original dataset: ‘laugh : make laugh/amuse’, ‘die : kill’, ‘sit : seat/make sit’, ‘eat : feed/give food’, ‘learn/know : teach’, ‘see : show’, be/become angry : anger’, ‘fear : frighten’, and ‘hide’. In this study, 63\% of the nine \ili{Evenki} verbs show causative coding vs. 50\% of the \ili{Even} verbs (\citealt[Table 6]{Nichols2018}). In our study with its mostly inanimate verbs, approximately 34\% (11/32) of all \ili{Evenki} verb pairs (i.e. counting over all synonyms) and about 38\% (12/32) of all \ili{Even} verb pairs show causative coding. When counting how many of the 20 verb meanings included in our study can be expressed with a causative derivation (irrespective of whether there are synonymous pairs using a different coding strategy), we find 40\% (8/20) verb meanings with causative coding in \ili{Evenki} and 50\% (10/20) in \ili{Even}. Not only are the overall proportions of causative coding generally lower in our study than those reported by Nichols (with the sole exception being the proportion of verb meanings in \ili{Even}), but the pattern is the opposite: in our data, \ili{Evenki} makes less use of causativization than \ili{Even}, while Nichols finds that it makes more use. 

To summarize this section, the preferred strategies of the \ili{Northern Tungusic} languages to code the causal-noncausal alternation are anticausativization and causativization, with a relatively high frequency of equipollence. Even though the languages are very closely related and the list of verb meanings is quite small, there are still noteworthy differences between them. However, the results of such studies depend considerably on the verb meanings they are based on as well as on the data bases used. For instance, the \ili{Evenki} dictionaries are much more extensive than the \ili{Negidal} dictionary and include many dialectal forms. In the following section, we compare the coding patterns found in \ili{Even}, \ili{Evenki}, and \ili{Negidal} to their Tungusic relatives as well as to other Eurasian languages.


\section{Northern Tungusic causal-noncausal alternations in a genealogical and cross-linguistic perspective}\label{section2.5}

\subsection{Cognates across Tungusic languages}

In the preceding section we already mentioned that in some cases cognate verbs show different coding patterns across the three \ili{Northern Tungusic} languages. Some further interesting patterns emerge when comparing \ili{Even}, \ili{Negidal}, and \ili{Evenki} with other Tungusic languages, namely \ili{Nanai}, \ili{Udihe}, and \ili{Manchu}, the data for which come from the World Atlas of Transitivity Pairs (\citeyear{Atlas2014}) with verification by specialists of these languages  (see \sectref{section2.5.2} for further details on this dataset). For instance, the equipollent final vowel change of noncausal \textit{-o}/\textit{-ə} : \textit{-i} (as found in \ili{Negidal} and \ili{Evenki} \textit{ɟəgdə-} : \textit{ɟəgdi-} ‘burn’ and \textit{olgo-} : \textit{olgi-} ‘dry’) is also found for the \ili{Nanai} cognates \textit{ɟəgdə-} : \textit{ɟəgɟi-} and \textit{holgo-} : \textit{holgi(ʨi)-} and for the putative \ili{Udihe} cognate \textit{ogo-} : \textit{wagi-} `dry'. Although this alternation is synchronically equipollent, etymologically it traces back to a causativizing pattern with the Tungusic causative *\textit{-gi} (\citealt[122]{Benzing1955AP}; \citealt[93]{Sunik1962}). However, \ili{Udihe} has regularized the causal form of ‘burn’ to \textit{ɟəgdə-βənə}, and \ili{Manchu} has regularized the causal form of ‘dry’ to \textit{olho-bu}, with both languages deriving the causal form with their regular causative suffix. \ili{Udihe} also derives the causal form of ‘turn (around, over)’ from the base root (\textit{kumtə-} : \textit{kumtə-βənə-}), while \ili{Negidal} and \ili{Evenki} treat the base root as causal and derive the noncausal form (\textit{kumtəβ-}~: \textit{kumtə-}).

Furthermore, some cases of semantic shift appear to have taken place. For example, the \ili{Nanai} word \textit{dasip-} : \textit{dasi-} means both ‘close’ and ‘cover’, while the \ili{Northern Tungusic} cognate Evn \textit{dasab-} {\textasciitilde} Neg, Evk \textit{dasiβ-} : Evn, Neg, Evk \textit{das-} means only ‘cover’, with a separate root (Evn \textit{homab-} {\textasciitilde} Neg \textit{samuβ-} {\textasciitilde} Evk \textit{soːmiβ-} : Evn \textit{hom-} {\textasciitilde} Neg \textit{sam-} {\textasciitilde} Evk \textit{soːm-}) meaning ‘close’. Likewise, the \ili{Nanai} word for ‘break’ \textit{kaltalip-} : \textit{kalta-} is cognate to the equipollent root \textit{kalta-} (Evn, Neg) {\textasciitilde} \textit{kakta-} (\ili{Udihe}) ‘split’. It is unclear whether this is a semantic shift from ‘split’ to ‘break’ in \ili{Nanai}, or whether it is an artefact of data collection (since ‘break’ and ‘split’ are very close in meaning).


\subsection{Causal-noncausal alternations across Eurasia}\label{section2.5.2}

For a cross-linguistic comparison of the \ili{Northern Tungusic} causal-noncausal alternation we also used data from the World Atlas of Transitivity Pairs (\citeyear{Atlas2014}). This Atlas contains information on coding patterns for 31 verb meanings based on \citet[104]{Haspelmath1993}. Thirteen verb meanings overlapped between our list of meanings (1) and that of \citet{Haspelmath1993}. However, we decided to exclude the meaning ‘put out/go out’, since we noticed that for the \ili{Even} WATP dataset the collected meaning was ‘exit’ and not ‘extinguish’. Since other contributors to the WATP may also have misunderstood the targeted meaning, we opted to exclude this from the dataset in order to ensure that we are indeed comparing the same meanings across languages. We thus used only twelve verb meanings per language for our cross-linguistic comparison: \textit{boil}, \textit{break}, \textit{burn}, \textit{close}, \textit{dry}, \textit{melt}, \textit{open}, \textit{rise/raise}, \textit{split}, \textit{spread}, \textit{stop}, \textit{turn over}. We included 60 languages of Eurasia in our comparison, as listed in the legend to Figure \ref{fig:2:1}. For each of them we counted the number of coding patterns in the same manner as shown in Table \ref{table:2.5} for \ili{Even}, \ili{Negidal} and \ili{Evenki}. It is important to mention that coding decisions might have had an impact on the counts. For example, in \ili{Evenki} we analyze the verb pair \textit{ula-p-} ‘get wet’ and \textit{ulaː-} ‘make wet’ as having anticausative derivation with morphonological vowel shortening in the root of the derived noncausal verb. But for \ili{Nanai} we followed the decision of  the WATP contributor Kazama, who coded the relation between \textit{kaltaa-} ‘break (intr.)’ and \textit{kaltali-} ‘break (tr.)’ as equipollent, since there is the pair \textit{xətu-ə-} : \textit{xətu-li-} ‘split’, where the final vowel in the intransitive verb is clearly a separate vowel, not length. This suggests that ‘break (intr.)’ might also be analysed as \textit{kalta-a-}, with the noncausal form in these equipollent pairs being marked by a mid-low vowel and the causal form being marked by \textit{-li}. The resulting frequency table was plotted on the map in Figure \ref{fig:2:1} in the form of pie charts reflecting the proportions of the different coding patterns in each language.

The coordinates for the languages were obtained mostly from Glottolog (\citealt{Hammarström2019}), with a few exceptions, such as \ili{Domaaki} and \ili{Burushaski}, which had completely overlapping pie charts and were plotted next to each other. For \ili{Even} we chose the location of Ola, which is the place where \ili{Standard} Even is spoken, even though our data come predominantly from the \ili{Lamunkhin} and \ili{Bystraja} dialects, and not the standard variety. We chose Ola since it is midway between the locations where the \ili{Lamunkhin} and \ili{Bystraja} dialects are spoken and it is also frequently used in typological maps to represent the location of \ili{Even} as a whole.

\begin{figure}
\includegraphics[width=\textwidth]{figures/Aralova1.jpg}
\caption{Causal-noncausal alternations in Eurasia; created with R (\citeyear{R2020}), based on data in WATP (\citeyear{Atlas2014})}
\label{fig:2:1}
\end{figure}
\il{Icelandic}\il{English}\il{French}\il{German}\il{Maltese}\il{Slovene}\il{Swedish}\il{Hungarian}\il{Lithuanian}\il{Romanian}\il{Greek}\il{Finnish}\il{Turkish}\il{Hebrew}\il{Georgian}\il{Armenian}\il{Batsbi}\il{Tzez}\il{Azerbaijani}\il{Avar}\il{Arabic}\il{Lezgian}\il{Russian}\il{Udmurt}\il{Turkmen}\il{Uzbek}\il{Tajik}\il{Kazakh}\il{Kyrgyz}\il{Punjabi}\il{Kashmiri}\il{Domaaki}\il{Burushaski}\il{Marathi}\il{Hindi-Urdu}\il{Nepali}\il{Meche}\il{Tiddim Chin}\il{Burmese}\il{Jinghpaw}\il{Amdo-Tibetan}\il{Thai}\il{Gyalrong}\il{Vietnamese}\il{Mongolian}\il{Khakas}\il{Evenki}\il{Mandarin}\il{Manchu}\il{Shuri Okinawan}\il{Korean}\il{Even}\il{Sakha}\il{Nanai}\il{Japanese}\il{Udihe}\il{Negidal}\il{Kita Akita Japanese}\il{Ainu}\il{Koryak}


In Figure \ref{fig:2:1} the \ili{Northern Tungusic} languages are labeled with 47 (\ili{Evenki}), 52 (\ili{Even}) and 57 (\ili{Negidal}). In general, they do not stand out in this picture, since they show the most common coding patterns – causativization and anticausativization as well as equipollence – roughly in the same proportion, with \ili{Negidal} additionally having a small proportion of ambitransitive verbs (for the meanings ‘rise/raise’ and ‘turn over’). \ili{Nanai} (labelled as 54) matches this distribution as well, whereas \ili{Udihe} (56) and \ili{Manchu} (49) show a stronger preference for causativization. With respect to the other languages of the region, the Tungusic languages seem to be rather typical in their marking of causal-noncausal relations: a similar pattern is found in \ili{Sakha} (53), \ili{Mongolian} (45), \ili{Ainu} (59) and the Shuri dialect of Okinawan (50).\il{Shuri Okinawan}

The Tungusic languages show a degree of homogeneity of marking the causal-noncausal alternation that is intermediate between that found within the \ili{Japonic} languages and that found in the \ili{Turkic} family. In the former, \ili{Shuri Okinawan} (50), Standard \ili{Japanese} (55), and the Kita-Akita dialect of \ili{Japanese} (58) show widely differing proportions of the three major coding patterns, while in the latter, languages as geographically distant as \ili{Turkish}, \ili{Azerbaijani}, \ili{Central Asian Turkic}, and \ili{Khakas} all show an overall very similar pattern of roughly equal proportions of anticausative and causative coding, with equipollence being very rare. Interestingly, \ili{Sakha} (Yakut) (53) shows a considerably higher proportion of equipollent coding than its \ili{Turkic} relatives, a feature that might be due to contact with Tungusic languages.

In general, as seen in Figure \ref{fig:2:1}, while causativization is a feature of Asia as a whole, being quite common in South Asia as well as in some languages of China, it gradually decreases from East to West: indeed, in Europe the only languages with a high proportion of causative strategies are non-\ili{Indo-European} (\ili{Finnish}, \ili{Hungarian}, \ili{Maltese}, \ili{Turkish}, some \ili{languages of the Caucasus}, and \ili{Udmurt}). Furthermore, as pointed out by \citet[180]{Nichols2004}, causativization extends beyond the Bering Strait into North America:\il{languages of Europe}\il{languages of North America}

\begin{quote}
    Northern Asia and North America, and to some extent also Central America-Mexico, favor augmentation [i.e. causativization] (and to a lesser extent double derivation [i.e. equipollence]) and disfavor reduction [i.e. anticausativization], ambitransitivity, and auxiliary change.
\end{quote}

To summarize this section, the \ili{Northern Tungusic} languages show quite similar coding strategies to their Tungusic relatives. Some of the patterns are clearly old in the Tungusic family, such as the final vowel alternation in equipollent stems, which goes back to an erstwhile causative pattern, while individual innovation can be shown to have played a role as well, such as the regular causative derivation of formerly equipollent stems in \ili{Udihe} or \ili{Manchu}. The \ili{Northern Tungusic} languages also do not stand out in areal perspective, making use of the most common strategies. To what extent these preferred patterns of coding might be explained by the form-to-frequency hypothesis will be addressed in the next section.


\section{The form-frequency correspondence in Even and Negidal}\label{section2.6}

As mentioned in the Introduction, in their paper \citet{Haspelmathetal2014} focus on the frequency-based motivation for the causal-noncausal alternation. Using large corpora of seven languages they test several predictions. Following their approach, we use data from our \ili{Even} and \ili{Negidal} corpora to test the form-to-frequency prediction, which states that unmarked forms are more frequent. This is formulated by \citet[597]{Haspelmathetal2014} as follows:

\begin{quote}
    In each language, in a causative verb pair, the causal member will be rarer than the noncausal member, while in an anticausative verb pair, the causal member will be more frequent than the noncausal member.
\end{quote}

In our count we did not consider the frequencies of labile verbs (one pair for ‘rise/raise’ and one pair for ‘turn over’ in \ili{Negidal}, both synonymous with morphologically marked pairs), nor did we consider equipollent verbs, as neither of these types is informative for this hypothesis. In \ili{Even}, the meaning ‘rise/raise’ is expressed with two synonymous verb pairs with opposite coding (see Appendix~\ref{Appendix1}), but in our corpus we find only one of these verbs with both causal and noncausal members (\textit{ojʨi-} ‘rise’ vs. \textit{ojʨiβkan-} ‘raise’). For this reason, we included only the causative coding in our count (see Appendix \ref{Appendix4}). The verb meaning ‘spoil’ was not found in either the \ili{Even} or the \ili{Negidal} corpus.

There are ten verb meanings in both \ili{Even} and \ili{Negidal} that clearly confirm the form-to-frequency prediction and only four and three, respectively, that do not. If we include those verb pairs where the difference in frequency is very small (only 1--2), so that we cannot say with certainty that one of the forms is truly more frequent than the other (see the cases in the table where “yes” is in brackets), the number of verb pairs confirming the form-to-frequency prediction rises to 12 in both languages. Our data thus do provide some support for the cross-linguistic tendency proposed by \citet{Haspelmathetal2014}.

\begin{sloppypar}
\citet{Haspelmathetal2014} suggest that the cross-linguistic tendency for deriving the less frequently used form might in individual languages be overridden by that language’s ``macro-type'', i.e a potentially strong preference for causative or anticausative coding (as exemplified by \ili{Romanian}, which has a distinct preference for anticausative coding and more verb pairs that disconfirm than confirm the prediction, \citealt[599]{Haspelmathetal2014}). In order to abstract away from such language-specific particularities they examine the frequencies of (non)causal uses independently from their coding. They test whether the proportion of noncausal verb uses correlates with the causative prominence scale proposed by \citet{Haspelmath1993}. The causative prominence scale ranks the verb meanings included in the study from the most causative-prominent to the most an\-ti\-caus\-a\-tive-prom\-i\-nent and reflects which verb meanings tend to be coded as causatives, and which tend to be coded as anticausatives, across the 21 languages included in Haspelmath’s study. \citet{Haspelmathetal2014} show that the ratio of noncausal uses over all occurrences of a particular verb meaning correlates significantly with the rank of a particular verb on the causative prominence scale: the verb meanings with the least causative prominence (i.e. those where the causal form is the basic form and it is the noncausal form which is derived) tend to have the least noncausal uses in the analysed corpora.
\end{sloppypar}

Since we lacked data for all of the verb meanings included by \citet{Haspelmathetal2014}, we did not replicate their test for \ili{Even} and \ili{Negidal}; rather, we followed the modified approach proposed by \citet{Seifart2019}, who reduce the list of verb meanings to six with different levels of causative prominence cross-linguistically: high (\textit{boil}, \textit{dry}), mid (\textit{turn}, \textit{burn}) and low (\textit{break}, \textit{open}). They modify the causative prominence scale by using data from WATP and by including some data from previous studies (\citealt{Haspelmath1993}; \citealt{Nichols2004}) as well as data from their own oral corpora of 14 understudied languages from South America and Papunesia. The results of \citet{Seifart2019} are quite consistent with those of \citet{Haspelmathetal2014}, notwithstanding the fact that they use a modified causative-prominence scale, fewer verb meanings, and much smaller corpora. Both studies confirm that for the verb meanings with lower causative prominence the corpus frequency of the noncausal event is lower, and vice versa, that when the causative prominence is high, the frequency of the noncausal event is higher.\il{languages of South America}\il{languages of Papunesia}

We test whether this tendency holds for the data in the \ili{Even} and \ili{Negidal} corpora by plotting the ratio of the noncausal uses over the total number of uses for each verb onto the typological causative prominence scale taken from \citegen{Seifart2019} study. The results are visualized in Figures~\ref{fig:2:2} and~\ref{fig:2:3}. It should be noted that this analysis can only be taken as indicative of tendencies of use in these languages, since it is based on rather few datapoints. 

\begin{figure}
\includegraphics[width=\textwidth]{figures/Aralova2.png}
\caption{Noncausal uses of six verbs in Even. For each verb the number of the noncausal uses over the total number of uses is shown in brackets; created with R (\citeyear{R2020})}
\label{fig:2:2}
\end{figure}

\begin{figure}
\includegraphics[width=\textwidth]{figures/Aralova3.png}
\caption{Noncausal uses of six verbs in Negidal. For each verb the number of the noncausal uses over the total number of uses is shown in brackets; created with R (\citeyear{R2020})}
\label{fig:2:3}
\end{figure}

\ili{Even} and \ili{Negidal} show different results. In \ili{Even}, the frequency of use of noncausal verb meanings does not increase with increasing rank on the typological causative prominence scale, while in \ili{Negidal} it does. The difference between the two patterns is caused by two verbs with mid and high causative prominence: ‘burn’ and ‘dry’. It is remarkable how differently ‘burn’ and ‘dry’ are used in the corpora of these closely related languages. In the \ili{Negidal} corpus, the ratio of noncausal usage is 77\% for ‘burn’ (62/81) and 68\% for ‘dry’ (34/50). In the \ili{Even} corpus, in contrast, only 33\% (35/106) of the occurrences of ‘burn’ have a noncausal meaning\footnote{Notably, ‘burn’ in \ili{Even} is also one of the few verbs in Appendix \ref{Appendix4} which does not confirm the form-to-frequency prediction.} and there are only about 30\% (4/13) of noncausal uses of ‘dry’. However, it should be noted that the \ili{Even} dialects show opposite patterns for ‘burn’: in the Lamunkhin dialect only \textasciitilde14\% (8/56) of the occurrences of ‘burn’ are noncausal, whereas in the Bystraja dialect \textasciitilde64\% (27/42) of the occurrences of this verb are noncausal. Thus it is the Lamunkhin dialect of \ili{Even} that patterns very differently from both its sister dialect and \ili{Negidal}. This underlines the high degree of lect-specificity of these patterns of usage.\il{Lamunkhin}\footnote{All the frequency differences we discuss here are significant: \ili{Negidal} vs. \ili{Even} ‘burn’: $\chi^2 =  33.119$, $p < 0.00001$; \ili{Negidal} vs. \ili{Even} ‘dry’: $\chi^2 =  4.5207$, $p = 0.03$ (also for Fisher’s exact test, $p = 0.02$); Lamunkhin vs. Bystraja ‘burn’: $\chi^2 = 24.001$, $p < 0.00001$. However, one should keep in mind that usage patterns depend to a large extent on the topic of the text as well as speaker idiosyncracies, and it is possible that the numbers would change if one were to include a wider range of texts and more speakers.}

Another observation concerns ‘boil’, a meaning with high causative prominence: in contrast to what is expected on typological grounds, this verb meaning has a rather low ratio of noncausal usage in both the \ili{Even} and the \ili{Negidal} corpora (23\% and 32\%, respectively\footnote{These values do not differ significantly: $\chi^2 = 0.20807$, $p = 0.6483$.}), and this is the only verb which disconfirms the form-to-frequency prediction in both \ili{Even} and \ili{Negidal} (see Appendix~\ref{Appendix4}). However, this low frequency of noncausal ‘boil’ is not exceptional cross-linguistically: in several languages of \citegen{Seifart2019} sample noncausal ‘boil’ occurs with zero or low frequency as well. One can speculate why this pattern emerges for ‘boil’ in several languages spoken in vastly different geographical regions, but not for other verbs with high causative prominence, such as ‘dry’ or ‘freeze’. Whereas freezing and drying can occur spontaneously in natural environments, completely spontaneous boiling is found only in thermal springs or in a volcano crater. Instead, for most boiling events there must be a human who initiates the process by putting a pot with water on a fire. Thus, purely spontaneous boiling is an infrequent event. However, there is a time lapse between the causal event (putting the pot on the fire) and the noncausal event (the water boiling), so that the actual boiling event might be conceptualized as spontaneous and be expressed with a noncausal base form. But in some languages, it seems, people tend to talk more about the causal event because that in general has to precede the noncausal, spontaneous boiling. In addition, in \ili{Negidal} the verb ‘boil’ appears to be lexicalizing to generalized ‘cook’ – which is of course a causal event and thus adds more causal uses.

To summarize this section, the causal-noncausal alternations in \ili{Negidal} and \ili{Even} confirm the form-to-frequency hypothesis formulated by \citet[597]{Haspelmathetal2014}: most verbs in our sample support the tendency that the derived member of a pair is rarer and the basic one is more frequent. However, some verbs which do not support this hypothesis turn out to be crucial for another prediction, namely that verbs which are higher on the causative prominence scale tend to have a higher ratio of noncausal usage, irrespective of their language-specific coding. The \ili{Negidal} data support this tendency, whereas the \ili{Even} data rather contradict it. In both \ili{Even} and \ili{Negidal}, as in some \ili{languages of South America} and Papunesia, the alternation pattern for ‘boil’ deviates from the expected one: this verb has a high rank on the causative prominence scale, but shows a low ratio of noncausal usage. This might be due to the characteristics of the boiling event, which generally needs to be initiated by a human causer, but which manifests itself only after a considerable amount of time.\il{languages of Papunesia}


\section{Conclusions}\label{section2.7}

To summarize, the \ili{Northern Tungusic} languages have a strong preference for morphological marking of the causal-noncausal alternation, with equipollence being a particularly salient strategy for verbs of destruction in \ili{Even} and \ili{Negidal}. Ambitransitivity and suppletion, in contrast, are very rare. This observation fits well with the fact that these languages are morphologically rich and express all manner of derivations with a variety of morphemes.

At a broad level the causal-noncausal alternation is fairly stable across languages, as shown by the similarity of the coding patterns found in the Tungusic and especially the \ili{Turkic} languages. This stability also emerges in the general Asian preference for causativization. However, at a fine-grained level many language-specific particularities emerge, as seen in the different patterns found for cognate verbs in the Tungusic languages, or in the widely different strategies preferred by the \ili{Japonic} lects included in the WATP dataset. 

Lastly, it should be noted that comparative work on the causal-noncausal alternation is rendered quite difficult due to the big impact that the choice of verb meanings and coding decisions can have; the cross-linguistic comparison discussed here should therefore be taken with a grain of salt. For instance, the comparison of our data with those of \citet{Nichols2018} has shown that the choice of verb meanings included in the study can have a notable impact on the preferred coding patterns determined for individual languages. Furthermore, it is not clear whether different studies always collected the same translation equivalents for all verb meanings, as seen by the fact that in our study we used ‘move (animate being)', i.e. `go’ rather than ‘move (inanimate object)’, or that Kazama obtained the translation equivalent of ‘go.out (exit)’ instead of ‘go.out (extinguish)’. In addition, coding decisions can also play a big role in the resulting overall pattern frequencies. Nevertheless, we hope that the overview of causal-noncausal alternations in \ili{Northern Tungusic} languages presented here can add some valuable observations about these understudied varieties to the areal and cross-linguistic research on this interesting feature.


\section*{Abbreviations}

\ili{Even}, \ili{Evenki}, and \ili{Negidal} are abbreviated as Evn, Evk, and Neg, respectively. \ili{Russian} and \ili{Sakha} (Yakut) copies are indicated with R and Y. Grammatical abbreviations used in the glosses are:

\begin{multicols}{2}
\begin{tabbing}
\textsc{deont}\hspace{1ex}\= deontic future\kill
1, 2, 3             \> person\\
\textsc{abl}        \> ablative\\
\textsc{acc}        \> accusative\\
\textsc{add}        \> additive\\
\textsc{adjr}       \> adjectivizer\\
\textsc{advrs}      \> adversative\\
\textsc{all}        \> allative\\
\textsc{ant}        \> anterior\\
\textsc{aug}        \> augmentative\\
\textsc{caus}       \> causative\\
\textsc{cond}       \> conditional\\
\textsc{contr}      \> contrastive\\
\textsc{cvb}        \> converb\\
\textsc{dat}        \> dative\\
\textsc{def}        \> definite\\
\textsc{deont}      \> deontic future\\
\textsc{dim}        \> diminutive\\
\textsc{dist}       \> distal (demonstrative)\\
\textsc{dp}         \> discourse particle\\
\textsc{ex}         \> exclusive\\
\textsc{foc}        \> focus\\
\textsc{fut}        \> future\\
\textsc{gnr}        \> generic\\
\textsc{hab}        \> habitual\\
\textsc{hort}       \> hortative\\
\textsc{imp}        \> imperative\\
\textsc{in}         \> inclusive\\
\textsc{inch}       \> inchoative\\
\textsc{ins}        \> instrumental\\
\textsc{int}        \> intensive\\
\textsc{intr}       \> intransitive\\
\textsc{iter}       \> iterative\\
\textsc{loc}        \> locative\\
\textsc{med}        \> medio-passive\\
\textsc{mult}       \> multiplicative\\
\textsc{neg}        \> negative\\
\textsc{nfut}       \> non-future\\
\textsc{pl}         \> plural\\
\textsc{poss}       \> marker of non-canonical \\
                    \> possession\\
\textsc{prfl}       \> reflexive-possessive\\
\textsc{pst}        \> past\\
\textsc{ptcp}       \> participle\\
\textsc{ptl}        \> (unspecified) particle\\
\textsc{px}         \> possessive suffix\\
\textsc{refl}       \> reflexive\\
\textsc{rep}        \> refactive (repetitive)\\
\textsc{sg}         \> singular\\
\textsc{sim}        \> simultaneous\\
\textsc{smlf}       \> semelfactive\\
\textsc{tam}        \> (unspecified) TAM-marker\\
                    \> (1 and 2 identify two different \\ \> morphemes)\\
\textsc{tr}         \> transitive\\
\textsc{val}        \> valency-changing suffix\\
\textsc{vr}         \> verbalizer\\
\end{tabbing}
\end{multicols}

\section*{Acknowledgements}

Very different versions of this paper were presented at the “Atelier morphosyntaxe” at the research unit “Dynamique du Langage”, Lyon, France, on 6 April 2018, and at the “Conference on Uralic, Altaic and Paleoasiatic Languages in the memory of A.P. Volodin” held at the Institute of Linguistic Studies RAS, Saint-Petersburg, on 6 December 2018. We thank the audiences of both events for their comments. We are also grateful to an anonymous reviewer and most especially to \name{Andreas}{Hölzl} and \name[Payne, Thomas E.]{Tom}{Payne}, whose detailed comments helped us improve the paper; needless to say, any remaining errors are our sole responsibility. Furthermore, we thank \name{Galina I.}{Kandakova} and \name{Antonina V.}{Kazarova} for their help in compiling the \ili{Negidal} dataset. This paper was written in 2019 when \name{Natalia}{Aralova} was a post-doctoral researcher at Dynamique du Langage funded by the Endangered Languages
Documentation Programme (ELDP), \url{www.eldp.net}. We are very grateful to ELDP for their generous support of our work on \ili{Negidal}. We also thank the LABEX ASLAN (ANR-10-LABX-0081) of Université de Lyon for its financial support within the program “Investissements d'Avenir” (ANR-11-IDEX-0007) of the French government operated by the National Research Agency (ANR).

We would furthermore like to express our gratitude to \name[Perekhvalskaya, Elena]{Elena}{Perexval’skaya}  (\ili{Udihe}), \name{Andreas}{Hölzl} (\ili{Manchu}), and \name[Oskolskaya, Sofia]{Sonya}{Oskol’skaya} (\ili{Nanai}) for checking the WATP datasets and for adding important information, such as synonymous verbal pairs.

\pagebreak

\begin{paperappendix}
\section{Causal-noncausal verb pairs in Even}\label{Appendix1} 

In the following tables, transitivity is abbreviated as “TR” (+: transitive, −: intransitive) and the coding pattern as “Coding”.

\begin{table}[H]
\fittable{\begin{tabular}{ l  l c l l}
\lsptoprule
\multicolumn{2}{c}{Verb meaning} & \\\cmidrule(lr){1-2}
English  & Russian &	TR &	Even verb &	Coding\\
\midrule
boil &	кипеть &	− &    huj &	nC > C\\
% \cline{2-4}
&	кипятить &	+ &	hujuː, hujuken &\\
\midrule
break &	(с)ломаться &	− &	hajubna, butar & 	nC < C\\
% \cline{2-4}
&	(с)ломать &	+ &	hajuː, but	 &\\
% \cline{2-4}
&	(с)ломаться &	− &	kabar, hokar, kaβar, koŋdar &	nC  ≈ C\\
% \cline{2-4}
&	(с)ломать &	+ &	kabal, hokak, kaβak, koŋdak	 &\\
\midrule
burn &	гореть &	− &	dur &	nC > C\\
% \cline{2-4}
&	сжечь &	+ &	duruː, duruken &\\
\midrule
close &	закрыться &	− &	homab &	nC < C\\
% \cline{2-4}
& 	закрыть & 	+ &	hom &\\
% \cline{2-4}
& 	закрыться &	− &	nipkəb &	nC < C\\
% \cline{2-4}
& 	закрыть & 	+ &	nipkə &\\
% \cline{2-4}
& 	закрыться &	− &	nipku &	nC  ≈ C\\
% \cline{2-4}
& 	закрыть & 	+ &	nipkə &\\
\midrule
run out/ &	израсходоваться &	− &	manuː &	nC < C\\
% \cline{2-4}
use up & 	израсходовать &	+ &	man &	\\
\midrule
dry &	сушиться (сохнуть) &	− &	olga & 	nC ≈ C\\
% \cline{2-4}
& 	сушить & 	+ &	olgi &\\
\midrule
fall/  &	падать &	− &	tik & 	nC > C\\
% \cline{2-4}
drop &	уронить &	+ &	tikəβ\footnote{It should be noted that we do not find the form \textit{tikəβ-} in our Even corpus, where we find only \textit{tikuken-}, derived with the causative suffix \textit{-βkAn}. The Even dictionaries don’t let us determine whether \textit{tikəβ-} indeed has only the basic meaning ‘drop’, but we assume so, since \textit{tikuken-} adds specific semantics of a voluntary, intentional action.} & \\
  \lspbottomrule
\end{tabular}}
\end{table}


\begin{table}[H]
\begin{tabular}{ l  l c l l}
\lsptoprule
\multicolumn{2}{c}{Verb meaning} & \\\cmidrule(lr){1-2}
English  & Russian &	TR &	Even verb &	Coding\\
\midrule
get wet/make &	промокнуть &	− &	ulab &	nC < C\\
%\cline{2-4}
wet, soak &	замочить &	+ &	ul	&\\
\midrule
go out/put out &	погаснуть &	− &	hiːβ &	nC > C\\
%\cline{2-4}
& 	потушить &	+ &	hiːβiː, hiβuːken\footnote{Note that we cannot be fully certain that the form \textit{hiβuːken-} does not add any additional semantic component, since we do not find this in our Even corpus, and the dictionaries do not let us determine the precise meaning.}	&\\
\midrule
increase &	прибавиться &	− &	haːβu &	nC < C\\
%\cline{2-4}
&	прибавить &	+ &	haːβ &\\
\midrule
melt & 	растаять &	− &	un, nen &	nC > C\\
%\cline{2-4}
 &	растопить &	+ &	umke, nemkat 	&\\
\midrule
move (go) &	идти &	− &	ŋən, hor/ur &	nC > C\\
%\cline{2-4}
 &	везти &	+ &	ŋənuː, horu/uru 	&\\
\midrule
open &	открыться &	− &	aːŋaːb &	nC < C\\
%\cline{2-4}
 &	открыть &	+ &	aːŋaː	&\\
\midrule
rise (ascend)/ &	подниматься &	− &	ugərəb &	nC < C\\
%\cline{2-4}
raise (carry up) &	поднять & 	+ &	ugər	&\\
%\cline{2-4}
 &	подниматься &	− &	ojʨi &	nC > C\\
%\cline{2-4}
 &	поднять & 	+ &	ojʨiβkan	&\\
\midrule
split &	расколоться &	− &	kaltar & 	nC ≈ C\\
%\cline{2-4}
 &	расколоть & 	+ &	kaltiː, kaltal	&\\
\midrule
spoil &	испортиться &	− &	hojib, haːjuːb &	nC < C\\
%\cline{2-4}
 &	испортить &	+ &	hoj, haːjuː	&\\
%\cline{2-4}
 &	испортиться &	− &	mun &	nC > C\\
%\cline{2-4}
 &	испортить &	+ &	munuken	&\\
\midrule
spread &	расстилаться &	− &	girkəb & 	nC < C\\
%\cline{2-4}
 &	расстилать &	+ &	girkə	&\\
\lspbottomrule
\end{tabular}
\label{table:2Appendix1b}
\end{table}

\begin{table}[H]
\begin{tabular}{ l  l c l l}
\lsptoprule
\multicolumn{2}{c}{Verb meaning} & \\\cmidrule(lr){1-2}
English  & Russian &	TR &	Even verb &	Coding\\
\midrule
stop (of &	остановиться &	− &	il & 	nC > C\\
% \cline{2-4}
humans) &	остановить &	+ &	iluːkan	&\\
\midrule
turn over  &	повернуться, перевернуться &	− &	hukəlbəŋʨi &	nC ≈ C\\
% \cline{2-4}
 (around)&	повернуть, перевернуть &	+ &	hukəsən	&\\
% \cline{2-4}
 &	повернуться, перевернуться &	− &	kumərkin &	nC > C\\
% \cline{2-4}
 &	повернуть, перевернуть &	+ &	kumərkimkeːn &\\
\midrule
bend  &	скрутиться (согнуться) &	− &	uʨib &	nC < C\\
% \cline{2-4}
(twist) &	скрутить (согнуть) &	+ &	ut	&\\
  \lspbottomrule
\end{tabular}
\label{table:2Appendix1c}
\end{table}

\section{Causal-noncausal verb pairs in Negidal}\label{Appendix2}

\begin{table}[H]
\begin{tabularx}{\textwidth}{ l  l c Q l}
\lsptoprule
\multicolumn{2}{c}{Verb meaning} & \\\cmidrule(lr){1-2}
English  & Russian &	TR &	Negidal verb &	Coding\\
\midrule
boil &	кипеть &	− &	huj &	nC > C\\
%\cline{2-4}
&	кипятить &	+ &	hujuβ &\\
\midrule
break &	(с)ломаться &	− &	ʨapʨaβ &	nC < C\\
%\cline{2-4}
 &	(с)ломать &	+ &	ʨapʨa\\
%\cline{2-5}
 &	(с)ломаться &	− &	tonŋodgə, tonŋam, kilgədgə, kilgam, boktadga, boktam &	nC  ≈ C\\
%\cline{2-4}
 &	(с)ломать &	+ &	tonŋol, kilgəl, boktal, boktanaː &\\
\midrule
burn &	гореть &	− &	ɟəgdə &	nC ≈ C\\
%\cline{2-4}
 &	сжечь &	+ &	ɟəgdi &\\
\midrule
close &	закрыться &	− &	samuβ &	nC < C\\
%\cline{2-4}
 &	закрыть & 	+ &	sam &\\
\midrule
run out/ &	израсходоваться &	− &	manaβ &	nC < C\\
%\cline{2-4}
 use up &	израсходовать &	+ &	mana &\\

  \lspbottomrule
\end{tabularx}
% \caption{Causal-noncausal verb pairs in Negidal}
\label{table:2Appendix2a}
\end{table}


\begin{table}[H]
\begin{tabularx}{\textwidth}{ l  l c Q l}
\lsptoprule
\multicolumn{2}{c}{Verb meaning} & \\\cmidrule(lr){1-2}
English  & Russian &	TR &	Negidal verb &	Coding\\
\midrule
dry &	сушиться (сохнуть) &	− &	olgo &	nC ≈ C\\
%\cline{2-4}
 &	сушить & 	+ &	olgi\\
\midrule
fall/  &	падать &	− &	tik &	nC ≈ C\\
%\cline{2-4}
 drop&	уронить &	+ &	tibgu &\\
\midrule
get wet/make &	промокнуть &	− &	olap &	nC < C\\
%\cline{2-4}
wet, soak &	замочить &	+ &	ola &\\
\midrule
go out/ &	погаснуть &	− &	siːβ (corpus) &	nC > C\\
%\cline{2-4}
 put out &	потушить &	+ &	siβi/siβu\newline (corpus) &\\
%\cline{2-5}
 &	погаснуть &	− &	siβ (elicitation) &	nC = C\\
%\cline{2-4}
 &	потушить &	+ &	siβ (elicitation) &	\\
\midrule
increase &	прибавиться &	− &	haβup &	nC < C\\
%\cline{2-4}
 &	прибавить &	+ &	haβ & 	\\
\midrule
melt &	растаять &	− &	un &	nC > C\\
%\cline{2-4}
 &	растопить &	+ &	uniβkan	 &\\
\midrule
move (go) &	идти &	− &	ŋənə &	nC > C\\
%\cline{2-4}
 &	везти &	+ &	ŋənəβ &	\\
\midrule
open &	открыться &	− &	niβ, aːŋaβ &	nC < C\\
%\cline{2-4}
 &	открыть &	+ &	niː, aːŋa &	\\
\midrule
rise (ascend) / &	подниматься &	− &	ugi &	nC = C\\
%\cline{2-4}
raise (carry up) &	поднять & 	+ &	ugi &	\\
\midrule
 &	подниматься &	− &	tukti &	nC > C\\
%\cline{2-4}
 &	поднять & 	+ &	tuktiβ &	\\
\midrule
split &	расколоться &	− &	dəlpədgə,\newline dəlpam, kaltadga, kaltam &	nC ≈ C\\
%\cline{2-4}
 &	расколоть & 	+ &	dəlpəl, kaltanaː, kaltal &	\\
%\cline{2-4}
 &	расколоться &	− &	dəlpədgə &	nC > C\\
%\cline{2-4}
 &	расколоть &	+ &	dəlpədgəβkan &	\\
  \lspbottomrule
\end{tabularx}
\label{table:2Appendix2b}
\end{table}


\begin{table}[H]
\begin{tabularx}{\textwidth}{ l  l c Q l}
\lsptoprule
\multicolumn{2}{c}{Verb meaning} & \\\cmidrule(lr){1-2}
English  & Russian &	TR &	Negidal verb &	Coding\\
\midrule
spoil &	испортиться &	− &	hajiβ / hajip & 	nC < C\\
%\cline{2-4}
 &	испортить &	+ &	haji &	\\
%\cline{2-4}
 &	испортиться &	− &	hajiβ / hajip &	nC ≈ C\\
%\cline{2-4}
 &	испортить &	+ &	haju &	\\
 \midrule
 spread &	расстилаться &	− &	gildeβ &	nC < C\\
%\cline{2-4}
 &	расстилать &	+ &	gilde &	\\
\midrule
stop (of &	остановиться &	− &	el &	nC > C\\
%\cline{2-4}
humans) &	остановить &	+ &	eleβkan &	\\
 \midrule
turn over  &	повернуться,  &	− &	kumtəβ,  &	nC < C\\
%\cline{2-4}
(around)  & перевернуться &  & kumtədgə \\
 &	повернуть, перевернуть &	+ &	kumtə	\\
%\cline{2-4}
 &	повернуться, перевернуться &	− &	hukil &	nC = C\\
%\cline{2-4}
 &	повернуть, перевернуть &	+ &	hukil &	\\
\midrule
bend (twist) &	скрутиться (согнуться) &	− &	oʨeβ &	nC < C\\
%\cline{2-4}
 &	скрутить (согнуть) &	+ &	ot &	\\
  \lspbottomrule
\end{tabularx}
\label{table:2Appendix2c}
\end{table}



\section{Causal-noncausal verb pairs in Evenki}\label{Appendix3}


\begin{table}[H]
\begin{tabular}{ l l c l l}
\lsptoprule
\multicolumn{2}{c}{Verb meaning} & \\\cmidrule(lr){1-2}
English  & Russian &	TR &	Evenki verb &	Coding\\
\midrule
boil &	кипеть &	− &	huju &	nC > C\\
%\cline{2-4}
 &	кипятить &	+ &	hujuβ &	\\
\midrule
break &	(с)ломаться &	− &	kapurga, sukʨarga &	nC < C\\
%\cline{2-4}
 &	(с)ломать &	+ &	kapu, sukʨa &	\\
\midrule
burn &	гореть &	− &	ɟegdə & 	nC ≈ C\\
%\cline{2-4}
 &	сжечь &	+ &	ɟegdiː & 	\\
%\cline{2-4}
 &	гореть &	− &	ilaβ &	nC < C\\
%\cline{2-4}
 &	сжечь &	+ &	ila &	\\
%\cline{2-4}
 &	гореть &	− &	lurgi &	nC = C\\
%\cline{2-4}
 &	сжечь &	+ &	lurgiː &	\\
%\cline{2-4}
 &	гореть &	− &	badara &	nC > C\\
%\cline{2-4}
 &	сжечь &	+ &	badaran &	\\
\midrule
close &	закрыться &	− &	soːmiβ &	nC < C\\
%\cline{2-4}
 &	закрыть  &	+ &	soːm &	\\
\midrule
run out / use up &	израсходоваться &	− &	manaβ &	nC < C\\
%\cline{2-4}
 &	израсходовать &	+ &	mana &	\\
\midrule
dry &	сушиться (сохнуть) &	− &	olgo &	nC ≈ C\\
%\cline{2-4}
 &	сушить  &	+ &	olgi	\\
\midrule
fall / drop &	падать &	− &	tik &	nC > C\\
%\cline{2-4}
 &	уронить &	+ &	tikiβ & 	\\
%\cline{2-4}
 &	падать &	− &	buru &	nC ≈ C\\
%\cline{2-4}
 &	уронить &	+ &	buriː &	\\
 \midrule
get wet / make &	промокнуть &	− &	ulap &	nC < C\\
%\cline{2-4}
wet, soak &	замочить &	+ &	ulaː &\\
\midrule
go out / put out &	погаснуть &	− &	siːβ &	nC < C\\
%\cline{2-4}
 &	потушить &	+ &	siː &	\\
 \midrule
increase &	прибавиться &	− &	haːβuβ &	nC < C\\
%\cline{2-4}
 &	прибавить &	+ &	haːβ &	\\
  \lspbottomrule
\end{tabular}
% \caption{Causal-noncausal verb pairs in Evenki}
\label{table:2Appendix3a}
\end{table}


\begin{table}[H]
\begin{tabularx}{\textwidth}{ l  l c Q l}
\lsptoprule
\multicolumn{2}{c}{Verb meaning} & \\\cmidrule(lr){1-2}
English  & Russian &	TR &	Evenki verb &	Coding\\
\midrule
melt &	растаять &	− &	uːn , ʨuːm &	nC > C\\
%\cline{2-4}
 &	растопить &	+ &	uːnŋiː, ʨuːmŋiː &	\\
\midrule
move (go) &	идти &	− &	suru, ŋənə &	nC > C\\
%\cline{2-4}
 &	везти &	+ &	suruβ, ŋənəβ &	\\
\midrule
open &	открыться &	− &	niːβ, aːŋaːβ &	nC < C\\
%\cline{2-4}
 &	открыть &	+ &	niː, aːŋaː &	\\
\midrule
rise (ascend) / &	подниматься & − &	ugiːriβ &	nC < C\\
%\cline{2-4}
raise (carry up) &	поднять & 	+ &	ugiːr &	\\
%\cline{2-4}
 &	подниматься &	− &	tukti &	nC > C\\
%\cline{2-4}
 &	поднять  &	+ &	tuktiβ &	\\
\midrule
split &	расколоться &	− &	dəlpərgə &	nC  ≈ C\\
%\cline{2-4}
 &	расколоть &	+ &	dəlpəliː &	\\
\midrule
spoil &	испортиться	 & − &	munu &	nC  ≈ C\\
%\cline{2-4}
 &	испортить &	+ &	muniː &	\\
\midrule
spread &	расстилаться &	− &	səktəβ &	nC < C\\
%\cline{2-4}
 &	расстилать &	+ &	səktə &	\\
\midrule
stop (of &	остановиться &	− &	il, tuːriːn &	nC > C\\
%\cline{2-4}
humans) &	остановить &	+ &	iliβkaːn, turinmu, tuːriːnmukəːn &	\\
\midrule
turn over  &	повернуться,  &	− &	horol &	nC > C\\
(around) & перевернуться & \\
%\cline{2-4}
 &	повернуть, перевернуть &	+ &	horoliβkaːn &	\\
%\cline{2-4}
 &	повернуться, перевернуться &	− &	kumtəβ &	nC < C\\
%\cline{2-4}
 &	повернуть, перевернуть &	+ &	kumtə &	\\
\midrule
bend (twist) &	скрутиться (согнуться) &	− &	uʨiːβ, mataβ &	nC < C\\
%\cline{2-4}
 &	скрутить (согнуть) &	+ &	uʨiː, mata &	\\
  \lspbottomrule
\end{tabularx}
\label{table:2Appendix3c}
\end{table}


% \pagebreak

\section{Corpus frequencies and coding patterns}\label{Appendix4}

Corpus frequencies and coding patterns for 20 verbs (beginning with the 12 that overlap with \citealt{Haspelmathetal2014}); conf.: confirmed, freq.: frequency, equi.: equipollent, antiC: anticausative, caus: causative.

\begin{table}
\begin{tabular}{ l rl rl ll }
  \lsptoprule
	& \multicolumn{2}{c}{Even} &	\multicolumn{2}{c}{Negidal} & \multicolumn{2}{c}{Hypothesis conf.?}\\\cmidrule(lr){2-3}\cmidrule(lr){4-5}\cmidrule(lr){6-7}
meaning &	freq. &	coding &	freq. &	coding &	\ili{Even} &	\ili{Negidal}\\
  \midrule
split.intr &	4 &	equi &	1 &	equi &	n/a &	n/a\\
split.tr &	5 &&		4 &			&&\\
\midrule
close.intr &	4 &	equi &	0 &	antiC &	n/a  &	yes\\
close.tr &	22 &&		8 &			&&\\
\midrule
break.intr &	7 &	equi &	16 &	equi &	n/a &	n/a\\
break.tr &	7 &&	9 &			&&\\
\midrule
open.intr &	6 &	antiC &	9 &	antiC &	yes &	yes\\
open.tr &	25 &&		43	 &		&&\\
\midrule
rise &	52 &	caus &	71 &	caus &	yes &	yes\\
raise &	1 &&	11 &			&&\\
\midrule
burn.intr &	35 &	caus &	62 &	equi &	no &	n/a\\
burn.tr &	71 &&		19 &			&&\\
\midrule
turn.intr &	4 &	caus &	7 &	antiC &	yes &	no\\
turn.tr &	0 &&		1 &			&&\\
\midrule
stop.intr &	6 &	caus &	5 &	caus &	yes &	yes\\
stop.tr &	0 &&		0 &			&&\\
\midrule
melt.intr &	4 &	caus &	4 &	caus &	yes &	yes\\
melt.tr &	0 &&		1 &			&&\\
\midrule
go.out &	9 &	caus &	8 &	caus &	yes &	yes\\
put.out &	2 &&		6 &			&&\\
  \midrule
dry.intr &	4 &	equi &	34 &	equi &	n/a &	n/a\\
dry.tr &	9 &&		16 &			&&\\
\midrule
boil.intr &	7 &	caus &	9 &	caus &	no &	no\\
boil.tr &	23 &&		19 &			&&\\
  \lspbottomrule
\end{tabular}
%   \caption{Corpus frequencies and coding patterns for 20 verbs (beginning with the 12 that overlap with \citealt{Haspelmathetal2014}); equi=equipollent, antiC=anticausative, caus=causative.}
\label{table:2Appendix4a}
\end{table}

\begin{table}
\begin{tabular}{ l rl rl ll }
  \lsptoprule
	& \multicolumn{2}{c}{Even} &	\multicolumn{2}{c}{Negidal} & \multicolumn{2}{c}{Hypothesis conf.?}\\\cmidrule(lr){2-3}\cmidrule(lr){4-5}\cmidrule(lr){6-7}
meaning &	freq. &	coding &	freq. &	coding &	\ili{Even} &	\ili{Negidal}\\
\midrule
run.out &	25 &	antiC &	22 &	antiC &	no &	no\\
use.up &	10 &&		11 &			&&\\
\midrule
fall &	118 &	caus &	95 &	caus &	yes &	yes\\
drop &	0 &		 & 3 &			&&\\
\midrule
get.wet &	12 &	antiC &	11 &	antiC &	no &	yes\\
make.wet &	7 &&		31 &			&&\\
\midrule
increase.intr &	0 &	antiC &	0 &	antiC &	yes &	(yes)\\
increase.tr &	9 &&		1 &			&&\\
\midrule
move(go) &	886 &	caus &	886 &	caus &	yes &	yes\\
make.move &	147 &&		107 &			&&\\
\midrule
spoil.intr &	2 &	caus & & &			(yes) &	\\
spoil.tr &	0 & &					&&\\
\midrule
spread.intr &	1 &	antiC &	0 &	antiC &	yes &	(yes)\\
spread.tr &	5 &&		2 &			&&\\
\midrule
bend.intr &	2 &	antiC &	1 &	antiC &	(yes) &	yes\\
bend.tr &	3 &&		9 &			&&\\
\midrule
Total YES & & & & &					10 (12) &	10 (12)\\
Total NO & & & & &					4 & 3\\
  \lspbottomrule
\end{tabular}
\label{table:2Appendix4b}
\end{table}
\end{paperappendix}

{\sloppy\printbibliography[heading=subbibliography,notkeyword=this]}
\end{document}
