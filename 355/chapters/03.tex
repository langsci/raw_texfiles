\documentclass[output=paper,colorlinks,citecolor=brown]{langscibook}
\ChapterDOI{10.5281/zenodo.7053363}
\author{Patryk Czerwinski\affiliation{University of Mainz}}
%\ORCIDs{}

\title{Tense and insubordination in Uilta (Orok)}

\abstract{The paper describes the tense category in Uilta, a critically endangered Tungusic language, from a functional and diachronic perspective. The functional analysis, based on the author’s fieldwork, provides a comprehensive typological description of the Uilta tense system. Similarly to other Tungusic languages, the diachronic development of this system and its current shape and complexity are largely the result of the processes of insubordination (replacement of finite verbal forms by non-finite forms in predicative use).}


\IfFileExists{../localcommands.tex}{
   \addbibresource{../localbibliography.bib}
   % add all extra packages you need to load to this file

\usepackage{tabularx,multicol}
\usepackage{url}
\urlstyle{same}

\usepackage{listings}
\lstset{basicstyle=\ttfamily,tabsize=2,breaklines=true}

\usepackage{langsci-basic}
\usepackage{langsci-optional}
\usepackage{langsci-lgr}
\usepackage{langsci-osl}
% \usepackage{./langsci/styles/langsci-lgr}
% \usepackage{./langsci/styles/langsci-osl}
% \usepackage{langsci-gb4e}

\usepackage{tikz}
\usetikzlibrary{patterns,calc}
\pgfdeclarepatternformonly{south east lines}{\pgfqpoint{-0pt}{-0pt}}{\pgfqpoint{3pt}{3pt}}{\pgfqpoint{3pt}{3pt}}{
    \pgfsetlinewidth{0.6pt}
    \pgfpathmoveto{\pgfqpoint{0pt}{3pt}}
    \pgfpathlineto{\pgfqpoint{3pt}{0pt}}
    \pgfpathmoveto{\pgfqpoint{.2pt}{-.2pt}}
    \pgfpathlineto{\pgfqpoint{-.2pt}{.2pt}}
    \pgfpathmoveto{\pgfqpoint{3.2pt}{2.8pt}}
    \pgfpathlineto{\pgfqpoint{2.8pt}{3.2pt}}
    \pgfusepath{stroke}}
    
\usepackage{stmaryrd}
\usepackage{wasysym}
\usepackage{multirow}
\usepackage{caption}
\usepackage{subcaption}
\usepackage{mathrsfs}
\usepackage{qtree}

\usepackage{linguex}


   %pminos do not split footnotes
% \interfootnotelinepenalty=10000 %Footnote in Laporte chapters has to be split SN


%\DeclareIndexNameFormat{default}{%
%\nameparts{#1}%
%\usebibmacro{index:name}%
%{\index[names]}%
%{\namepartfamily}%
%{\namepartgiveni}%
% {}% L1
% {}% L2
%{\namepartprefix}% generates spurious space L3
%{\namepartsuffix}% generates spurious space L4
%}

%  {\DeclareIndexNameFormat{default}{%
%     \usebibmacro{index:name}{\index[names]}{#1}{#3}{#5}{#7}}}

%\DeclareIndexNameFormat{default}{%
%  \usebibmacro{index:name}{\sindex[nom]}{#1}{#3}{#5}{#7}}

%\DeclareIndexNameFormat{default}{%
%  \usebibmacro{index:name}{\sindex[person]}{#1}{#3}{#5}{#7}}
%\DeclareIndexNameFormat{default}{%
%\nameparts{#1} \usebibmacro{index:name}{\sindex[person]]}{\namepartfamily}{‌​\namepartgiven}{\nam‌​epartprefix}{\namepa‌​rtsuffix}}

%\newcommand{\smiley}{:)}

%\renewbibmacro*{index:name}[5]{%
%\usebibmacro{index:entry}{#1}%
%{\iffieldundef{usera}{}{\thefield{usera}\actualoperator}\mkbibindexname{#2}{#3}{#4}{#5}}}

% \newcommand{\noop}[1]{}

%remove for final
%\overfullrule=1mm

\newcommand{\tobi}[2]}}
\renewcommand{\S}[1]{\tobi{#1}{\textsc{*}}}

% this volume references
% puts: [this volume]
% already defined: \citetv
%\newcommand{\citepv}[1]{(\citeauthor{#1} \citeyear*{#1} [this volume])}
\newcommand{\citealtv}[1]{\citeauthor{#1} \citeyear*{#1} [this volume]}

%parentheses around example number
\newcommand{\pref}[1]{(\ref{#1})}

% in-text examples

\newcommand{\lnex}[1]{\textit{#1}} %target lang word
\newcommand{\lnlit}[1]{(lit.: `#1')} %literal reading
\newcommand{\lnlat}[1]{(#1)} % latinization
\newcommand{\lntrans}[1]{`#1'} %translation
\newcommand{\lnexl}[2]%
{\lnex{#1}{} \lnlat{#2}} % ex with latinization
\newcommand{\lnexlat}[3]{\lnex{#1}{} \lnlat{#2}{} \lntrans{#3}} % ex with latinization and tranl.

%ch01
\newcommand{\co}[1]{\mbox{\textbf{#1}}}

%ch09

\newcommand{\cyrbulg}[1]{\begin{otherlanguage*}{bulgarian}#1\end{otherlanguage*}}


%ch10
\newcommand{\nlp}{{\small NLP}}
\newcommand{\mwe}{{\small MWE}}
\newcommand{\rae}{{\small RAE}}
\newcommand{\lvc}{{\small LVC}}
\newcommand{\pos}{{\small P}o{\small S}}
%\newcommand{\todo}[1]{ \textcolor{red}{#1} }

%\renewcommand{\labelenumi}{\theenumi}
%\ainamefmt{{vv}{ll}{, ff}{, jj}} % fullname

\newcommand{\biberror}[1]{{\color{red}#1}}

\newcommand{\osenovaitem}{--~}
   %% hyphenation points for line breaks
%% Normally, automatic hyphenation in LaTeX is very good
%% If a word is mis-hyphenated, add it to this file
%%
%% add information to TeX file before \begin{document} with:
%% %% hyphenation points for line breaks
%% Normally, automatic hyphenation in LaTeX is very good
%% If a word is mis-hyphenated, add it to this file
%%
%% add information to TeX file before \begin{document} with:
%% %% hyphenation points for line breaks
%% Normally, automatic hyphenation in LaTeX is very good
%% If a word is mis-hyphenated, add it to this file
%%
%% add information to TeX file before \begin{document} with:
%% \include{localhyphenation}
\hyphenation{
    Beck-man
    Ngu-yen
    back-chan-nel
    back-chan-nels
    mo-not-o-nous
    ste-reo-typ-i-cal
}

\hyphenation{
    Beck-man
    Ngu-yen
    back-chan-nel
    back-chan-nels
    mo-not-o-nous
    ste-reo-typ-i-cal
}

\hyphenation{
    Beck-man
    Ngu-yen
    back-chan-nel
    back-chan-nels
    mo-not-o-nous
    ste-reo-typ-i-cal
}

   \boolfalse{bookcompile}
   \togglepaper[2]%%chapternumber
}{}

\begin{document}
\maketitle

\section{Introduction}\label{Section3.1}

\subsection{The purpose and scope of this paper}\label{Section3.1.1}

The paper offers a comprehensive functional analysis of the tense system of the Tungusic language \ili{Uilta} (Orok), based largely on the author’s own fieldwork, and partially on existing descriptions.\footnote{\ili{Uilta} is the endonym and is strongly preferred by the community over the exonym Orok. Both terms are used in the literature.} The previous descriptions of the \ili{Uilta} tense system are either incomplete or contradicting, partially due to dialectal differences, as well as diachronic changes. The present analysis aims to account for those differences through different degrees and stages of the processes of insubordination (cf. \citealt{Evans_2007}).

Insubordination, the development of non-finite (participial) into finite (verbal) forms, is a prominent factor in the development of the TAM systems of Tungusic languages, which underwent repeated cycles of renewal of finite verbal forms through participles (\citealt{Malchukov_2013}). It will be shown that similar diachronic processes account for the current shape and peculiarities of the \ili{Uilta} tense system.


\subsection{Basic information about Uilta}\label{Section3.1.2}

While there is no universally accepted internal classification of Tungusic, most authors agree on placing \ili{Northern Tungusic} (represented by \ili{Even} and \ili{Evenki}) and \ili{Southern Tungusic} (the \ili{Jurchen}/\ili{Manchu} group) in separate branches, with the remaining groupings, \ili{Udegheic} and \ili{Nanaic}, variously assigned to one of the two branches, to a separate (Southeastern) branch, or to branches of their own (\citealt{Whaley_Oskolskaya_2020}). \ili{Uilta} is a member of the \ili{Nanaic} (sub-)branch. It is spoken exclusively on the island of Sakhalin, in the Russian Federation. This relative isolation from the rest of the family led to the development of a number of innovations not attested in the languages spoken on the mainland (\citealt{Pevnov_2016}).

The two \ili{Uilta} dialects, Northern\il{Northern Uilta} and Southern\il{Southern Uilta}, are mutually intelligible and historically formed a dialect continuum. The language is critically endangered, with five fluent speakers remaining, all in their seventies, of which four are speakers of the Northern dialect, centered on the village of Val in the Nogliki raion, and one of the Southern\il{Southern Uilta} dialect, in the city of Poronaysk.\footnote{Historically, different \ili{Uilta} clans lived as reindeer herders along different rivers on the east coast of central and northern Sakhalin, and migrated yearly between the coast and the mountains in the central part of the island. They were forcibly settled in the 1950s, around a collective farm in Val, Nogliki raion, and Yuzhnyj ostrov, Poronaysk.}

\ili{Uilta} has been in close areal contact with Sakhalin \ili{Nivkh} for at least 300 years (\citealt{Yamada_2010a}), and shares numerous features in the lexical and, to a lesser degree, grammatical domain (\citealt{Pevnov_2016}). Much later, from the mid-19th century onwards, it came into contact with \ili{Sakhalin} Evenki, a later entrant in the northern part of Sakhalin (\citealt{Yamada_2010a}). Contact with \ili{Sakhalin} Evenki accounts for a number of distinct features of the Northern dialect compared to the Southern\il{Southern Uilta} dialect of \ili{Uilta} (\citealt{Ikegami_1994}).


\subsection{Insubordination in Tungusic}\label{Section3.1.3}
\begin{sloppypar}
“Canonical” insubordination, as introduced into linguistic typology by \citet{Evans_2007}, involves “conventionalized main clause use of what, on prima facie grounds, appear to be formally subordinate clauses” (\citealt{Evans_2007}: 367). A variant of this process, labelled “verbalisation” in \citet{Malchukov_2013}, involves reanalysis of a nominal (participial) predicate into a verbal predicate. Both scenarios are illustrated below for \ili{Even} (\ili{Northern Tungusic}), after \citet{Malchukov_2013}.
\end{sloppypar}

\begin{description}
\item[{\normalfont Insubordination “proper”:}] Reanalysis of a sentential argument as a main clause: \textsc{[s part-agr.poss] [cop] → [s part-agr.poss] ∅ → [s] [v-agr.poss]}

\ea
\ili{Even}\\
\label{example3.1}
\ea
\gll [Bej-il	hör-ri-ten]				bi-d′i-n.\\
     man-\textsc{pl} go-\textsc{nfut(part)}-\textsc{3pl(poss)}
     be-\textsc{fut}-\textsc{3sg}\\
\glt `The men probably left.’ (Literally: ‘The men’s leaving will be.’)

\ex
\gll Bej-il	hör-ri-ten.\\
     man-\textsc{pl} go-\textsc{pst}-\textsc{3pl(poss)}\\
\glt `The men left.’ (\citealt{Malchukov_2013}: 182)
\z
\z

\item[{\normalfont Verbalization:}] Reanalysis of a nominal predicate into a verbal predicate:\\
\textsc{[s] [n/part] [cop] → [s] [v2 aux] (→ [s] [v]}).


\ea
\ili{Even}\\
\label{example3.2}
\ea
\gll Bej	[hör-če]	[bi-si-n].\\
     man
     go-\textsc{pfv}
     be-\textsc{pst}-\textsc{3sg}\\
\glt `The man was gone.’


\ex
\gll Bej	[hör-če	bi-si-n].\\
     man
     go-\textsc{pfv}
     be-\textsc{pst}-\textsc{3sg}\\
\glt `The man had left.’ (\citealt{Malchukov_2013}: 181)
\z
\z
\end{description}

In the first scenario, the subject complement clause followed by the existential verb is reanalysed as an independent clause. Typically for Tungusic, the non-finite complement clause has the form of a nominal possessive phrase. Possessive agreement on the participle indicates the subordinate subject. In (\ref{example3.1}b), the same participial form now forms the predicate of the verbal clause, but retains the (nominal) possessive subject agreement. In the second scenario, the nominal (participial) predicate followed by the existential verb is reanalysed as a periphrastic verbal (pluperfect) construction.

The two processes exemplified above for \ili{Even} led to gradual replacement of finite TAM forms by forms of participial origin in the verbal paradigms of all branches of Tungusic, and account for a number of peculiarities in their grammatical structures: weak distinction between nominal and verbal forms; inherent ambiguity of certain tense forms despite rich inventories of distinct markers; the presence of nominal (possessive) agreement paradigms in the (finite) verbal domain (\citealt{Malchukov_2013}).

Furthermore, as demonstrated by \citet{Robbeets_2009, Robbeets_2015} and \citet{Malchukov_Czerwinski_2020}, the process of replacement of finite forms by participles in repeated cycles of insubordination is prevalent in all ``\ili{Macro-Altaic}”\footnote{Here and elsewhere, ``\ili{Macro-Altaic}" is used as an areal-typological label, without any claims regarding genetic relatedness of the families in question (\ili{Turkic}, \ili{Mongolic}, Tungusic, \ili{Koreanic} and \ili{Japonic}).} languages, and its preponderance can be viewed as one of their characteristic features.

More broadly, as demonstrated by \citet{Malchukov_2013} and \citet{Malchukov_Czerwinski_2021}, this tendency is not limited to ``\ili{Macro-Altaic}", and instead constitutes an areal feature (diachronic isogloss) of Siberian languages generally, including \ili{Paleosiberian} (\ili{Chukotko-Kamchatkan}, \ili{Eskimo-Aleut}, \ili{Nivkh}, \ili{Yeniseian} and \ili{Yukaghir}) and \ili{Uralic} languages.

\begin{sloppypar}
The gradual replacement of finite (verbal) through non-finite (participial) forms leads to competition between old and new forms, often resulting in functional shifts in the relevant verbal categories. This is well documented for \ili{Southeastern Tungusic} languages (i.e. \ili{Udegheic} and \ili{Nanaic}, see \sectref{Section3.1.2} above), which all retain forms of both finite and participial origin, to varying degrees. As the imperfective and perfective participles acquire predicative function and general present/past meaning respectively, the erstwhile finite forms are pushed out from general present/past use and acquire direct evidential, and later affirmative-emphatic, meaning through a process known as markedness reversal (\citealt{Croft_2002}). In the past domain, the development from resultative through perfect to (indirect evidential) past is a universal grammaticalisation path, well-attested cross-linguistically (\citealt{Bybee_1994}). Competition between forms at each stage leads to further development from perfect to (non-witnessed) past to general past, and the parallel development of erstwhile finite forms from (unmarked) indicative first into direct evidential, and later into affirmative-emphatic. Different Tungusic languages display different stages of this development. This is illustrated below for \ili{Southeastern Tungusic} (\ili{Udegheic} and \ili{Nanaic}; Figure \ref{tab:evolution}, adapted from \citealt{Malchukov_2000}: 454).\footnote{The figure in \citet{Malchukov_2000} listed \ili{Uilta} as representing the final, fourth stage, based on a previous description. It was modified to reflect the fact that the finite past form is marginally retained in \ili{Uilta}, as per other descriptions and as confirmed by the present author (see \sectref{Section3.3.3} below).}
\end{sloppypar}\largerpage


% \begin{figure}
% \caption{Evolution of past tense forms in Southeastern Tungusic (adapted from \citealt{Malchukov_2000}: 454)}
% \label{tab:evolution}
%     \begin{tabular}{p{1.3cm}p{1.7cm}lp{1.8cm}lp{1.6cm}lp{1.7cm}}
% \hline
% & 1\textsuperscript{st} stage & & 2\textsuperscript{nd} stage & & 3\textsuperscript{rd} stage & & 4\textsuperscript{th} stage\\
% \hline
% Participial: & (Resultative) & > & Perfect/indirect evid. & > & Preterite & > &	General past\\
% Finite: & General past & > & Imperfect/direct evid. & > & Validational & > & ∅\\
% \hline
% & & & {\ili{Udihe}} & & {\ili{Nanai}}\\
% \hline
%     \end{tabular}
% \end{figure}

\begin{figure}
\includegraphics[width=1.0\textwidth]{figures/CzerwinskiFigure1.pdf}
\caption{Evolution of past tense forms in Southeastern Tungusic (adapted from \citealt{Malchukov_2000}: 454)}
\label{tab:evolution}
\end{figure}

This competition between forms, with the resulting functional shifts, occured in \ili{Uilta} in all three temporal domains, past, present and future, and is a key factor in understanding both the diachronic development and the current shape of the \ili{Uilta} tense system.
\il{Udihe}
\il{Nanai}

\sectref{Section3.2} of the paper outlines the \ili{Uilta} tense system. \sectref{Section3.2.1} lists previous descriptions, with the relevant information on the attested forms, the period of data collection and the dialect they pertain to. \sectref{Section3.3}, \sectref{Section3.4} and \sectref{Section3.5} provide functional analysis of the past, present and future tense forms respectively, as well as their diachronic development through different scenarios of insubordination. \sectref{Section3.6} provides a summary and conclusions.


\section{Uilta tense system}\label{Section3.2}

\subsection{Overview}

The contemporary \ili{Uilta} tense system consists of nine (Northern\il{Northern Uilta} dialect) or eight forms (Southern\il{Southern Uilta} dialect; the general future form in \textit{-li} is attested only in the Northern\il{Northern Uilta} dialect). They are listed below according to their origin. The forms in the right-hand column are the old finite forms. They are mono-functional, i.e. can be used exclusively as the predicate of a main clause, and take subject agreement of the verbal type (see below). The forms in the left-hand column, grammaticalised from the perfective, imperfective and future participles, are poly-functional (retain their function as participles/nominalisations on top of their function as the main clause predicate), and take agreement of the nominal (possessive) type.

\begin{table}
\caption{Tense forms in Uilta\label{table:1}}
 \begin{tabularx}{\textwidth}{ l  l  Q }
  \lsptoprule
& Participial & Finite\\
  \midrule
Past	  &   general past in \textit{-xAn}           & direct evid./affirm.-emph. past in \textit{-tAA}\\
&   pluperfect in \textit{-xA- bi-čči}         & \\
\midrule
Present  &   general present in +\textit{RI}          & direct evid./emph./mirative present in +\textit{RAkkA}\\
\midrule
Future &   general future in \textit{-li} (N dialect)          & immediate spontaneous future in +\textit{RIlA}\\
&  probable future in +\textit{RIli}          & likely/anticipated future in +\textit{RAŋA}\\
  \lspbottomrule
 \end{tabularx}
\end{table}

\subsection{Existing descriptions}\label{Section3.2.1}

Existing descriptions of tense in \ili{Uilta} go back over a hundred years (see \citealt{Yamada_2013} for a comprehensive overview). Table \ref{table:2}, adapted from \citet[90]{Yamada_2013}, lists them all, specifying which dialect they pertain to, the period of data collection, and the forms attested.

\begin{table}
\caption{Existing descriptions of tense in Uilta (adapted from \citealt{Yamada_2013}: 90)}
\label{table:2}
 \begin{tabular}{p{3.3cm}lllll}
  \lsptoprule
            Author &	Period &	Dialect &	Past &	Present &	Future\\
  \midrule
  Piłsudski  &	1900 &	S &	\textit{-xAn} &	+\textit{RI} &	–\\
  \citep{Majewicz_2011}\\\\
  \citealt{Nakanome_1917} &	1910 &	S &	-\textit{xAn} &	+\textit{RI} &	–\\\\
  \citealt{Magata_1981} &	1920--1930 &	S &	-\textit{xAn} &	+\textit{RI} &	+\textit{RIlA}\\
	& & & 		-\textit{tAA} &	 +\textit{RAkkA}	& \\\\
  \citealt{Petrova_1967} &	1930--1940 &	N &	-\textit{xAn} &	+\textit{RI} &	-\textit{llee}\\
	& & & & & 				+\textit{RIlA}\\\\
  \citealt{Ikegami_1959} &	1940--1950 &	S &	-\textit{xAn} &	+\textit{RI} &	+\textit{RIlA}\\
	& & &		-\textit{tAA} &	 +\textit{RAkkA} &	+\textit{RAŋA}\\
		& & & & & 			+\textit{RIli}\\\\
  \citealt{Tsumagari_2009} &	1940--1980 &	S &	-\textit{xAn} &	+\textit{RI} &	+\textit{RIlA}\\
	& & &		-\textit{tAA} &	 +\textit{RAkkA}	&\\\\
  \citealt{Ozolinja_2013} &	1990--2000 &	N &	-\textit{xAn} &	+\textit{RI} &	+\textit{RIlA}\\
  \lspbottomrule
 \end{tabular}
\end{table}

The next three sections describe the category of tense in \ili{Uilta} from a functional perspective, based on the author’s own fieldwork.\footnote{Unless otherwise stated, \ili{Uilta} data and findings come from the author’s own fieldwork.} The analysis by the author will be reconciled with existing descriptions, particularly with regard to diachronic development. It will be shown that this development is best explained through the processes of insubordination. This part is based on and expands on the work on insubordination in Tungusic by Malchukov (\citeyear{Malchukov_2000, Malchukov_2013}).


\section{Past domain}\label{Section3.3}

In the past domain, \ili{Uilta} has three forms, general past in \textit{-xAn}, pluperfect in \textit{\nobreakdash-xA- bi-čči} [-\textsc{pst}-\textsc{agr} be-\textsc{pfv}], and direct evidential/affirmative-emphatic in \textit{-tAA}. The forms in \textit{-xAn} and \textit{-xA- bi-čči} grammaticalised from the perfective participle in \textit{-xAn}, and retain the person/number agreement paradigm of the nominal (possessive) type, in contrast with the form in \textit{-tAA} which takes person/number agreement of the (finite) verbal type (Table \ref{table:3}).\footnote{Unlike most Tungusic languages, Uilta has no inclusive/exclusive first person plural distinction.}

\begin{table}
\caption{Person/number agreement paradigms of Uilta past tense forms.}
\label{table:3}
 \begin{tabular}{llll}
  \lsptoprule
    Number        & Person & \textit{-xAn} & \textit{-tAA}\\
  \midrule
  &   1\textsuperscript{st}  &   \textit{-xA-mbi}      & \textit{-tA-mmee}\\
  Singular  &   2\textsuperscript{nd}  &   \textit{-xA-si}      & \textit{-tA-ssee}\\
  &   3\textsuperscript{rd}  &   \textit{-xA-ni}      & \textit{-tAA}\\
  \midrule
  &   1\textsuperscript{st}  &   \textit{-xA-pu}      & \textit{-tA-ppOO}\\
  Plural  &  2\textsuperscript{nd}  &    \textit{-xA-su}      & \textit{-tA-ssOO}\\
  &   3\textsuperscript{rd}  &   \textit{-xA-či}      & \textit{-tAA-l}\\
  \lspbottomrule
 \end{tabular}
\end{table}

\subsection{General past in \textit{-xAn}}\label{Section3.3.1}\largerpage

The general past form in \textit{-xAn} is by far the most frequent past form, with the other two forms limited to specific contexts (see \sectref{Section3.3.2} and \sectref{Section3.3.3} below). In some conjugational classes the perfective participle/general past tense takes the form \textit{-či}. It is unclear whether the forms in \textit{-xAn} and \textit{-či} are cognate or heteroclitic (\citealt{Malchukov_2000}). The form in \textit{-xAn} is used in recent (\ref{example3.3}) and remote past contexts (\ref{example3.4}), and with punctual (\ref{example3.5}), durative (\ref{example3.6}) and habitual meanings (\ref{example3.7}):

\ea
\label{example3.3}
\gll Pakčira-du-xa-ni.			Siweeskə-bi		dəgǰitə.\\
     get.dark-\textsc{reiter}-\textsc{pst}-\textsc{3sg}
     candle-\textsc{1sg.poss}
     burn(\textsc{trans})+\textsc{1sg.hort}\\
\glt `It got dark. Let me light a candle.’\il{Uilta}
\ex
\label{example3.4}
\gll Bii		nuuči-ǰǰi			ŋəələ-xə-mbi		ŋinda-l-ǰi.\\
     \textsc{1sg}		little-\textsc{instr.refl}	fear-\textsc{pst}-\textsc{1sg}		dog-\textsc{pl}-\textsc{instr}\\
\glt `When I was little, I was afraid of dogs.’\il{Uilta}
\ex
\label{example3.5}
\gll Ča-du	bi-čči-ndulə-ni	bii	əigə-təkki			gumaaska	buu-xə-mbi.\\
     that-\textsc{loc}	be-\textsc{pfv(pst)}-\textsc{loc}-\textsc{3sg}	\textsc{1sg}	older.sister-\textsc{all.refl}	money		give-\textsc{pst}-\textsc{1sg}\\
\glt `Because he had lived there, I gave my sister money.’\il{Uilta}
\ex
\label{example3.6}
\gll Tari	ənu-či		narree		goroo			daputa-xa-či	okči-či-kku		duku-du.\\
     this	fall.ill-\textsc{pfv}	man+\textsc{acc}	long.time+\textsc{emph}	hold-\textsc{pst}-\textsc{3pl}	heal-\textsc{dur}-place	house-\textsc{loc}\\
\glt `They kept this sick man in the hospital for a long time.’\il{Uilta}
\ex
\label{example3.7}
\gll Niməri-ŋəssəə-wwee,			mittəi	aptauli-mba	tɵyɵ-xɵ-či.\\
     visit-\textsc{concur}.\textsc{pst}.\textsc{conv}-\textsc{1sg}	\textsc{1sg}.\textsc{all}	tasty-\textsc{acc}	treat-\textsc{pst}-\textsc{3pl}\\
\glt `When I visited [them], they always treated me to something tasty.’\il{Uilta}
\z

It is also the form most often used in narratives, as in (\ref{example3.8}):

\ea
\label{example3.8}
\gll Niiwənikəən		balǰi-xa-ndulli			xaali=ddaa	suunəə		ə-čči-ni ittəə.\\
     Niiwənikəən(\textsc{pn})	grow(\textsc{intr})-\textsc{pfv(pst)}-\textsc{loc}.\textsc{refl}	how=\textsc{foc}	sun+\textsc{acc}		\textsc{neg.aux}-\textsc{pst}-\textsc{3sg}
	see+\textsc{conneg}\\
\glt `When he was growing up, Niiwənikəən never saw the sun.’\il{Uilta}
\z

On top of its predicative use, the form in \textit{-xAn} retains its original use as the perfective participle (which in all Tungusic languages has double adnominal/nominal function; example \ref{example3.9}, cf. also examples \ref{example3.6}, \ref{example3.23} and \ref{example3.29}).\footnote{As in other Tungusic languages, participles are also the main strategy for relative clauses, both pre-nominal (cf. \ref{example3.42}) and internally headed (\ref{example3.9}, \ref{example3.49}), complement clauses (\ref{example3.32}) and, with oblique cases, one of the two strategies for adverbial clauses (\ref{example3.5}, \ref{example3.8}, \ref{example3.12}, \ref{example3.15}, \ref{example3.39}, \ref{example3.45}).}

\ea
\label{example3.9}
\gll Tari	puttə	iiwu-xə-mbə-ni		sundattaa	əni-ni			təldə-xə-ni.\\
     	that	child	bring.in-\textsc{pfv}-\textsc{acc}-\textsc{3sg}	fish+\textsc{acc}		mother-\textsc{3sg.poss}	fillet-\textsc{pst}-\textsc{3sg}\\
\glt `The mother filleted the fish that the son brought.’\il{Uilta}
\z


In line with its origin as the perfective participle, while firmly established as a general past tense form in predicative use, in a limited number of cases the meaning of \textit{-xAn} is closer to the resultative or perfect than a pure tense form (\citealt{Yamada_2013}: 98):

\ea
\label{example3.10}
\gll Nu,		əsi=ləkə		dəgdə-xə-či		əmbee.\\
     	\textsc{intj}		now=\textsc{top}	burn-\textsc{pst}-\textsc{3pl}		of.course\\
\glt `Well, now they have burnt of course.’ (\citealt{Yamada_2013}: 99)\il{Uilta}
\z


\subsection{Pluperfect in \textit{-xA- bi-čči}}\label{Section3.3.2}

The perfective participle form in \textit{-xAn} followed by the copula/existential verb in the past tense forms the periphrastic pluperfect, similar to other Tungusic languages:

\ea
\label{example3.11}
\gll Buu	gasa-ttai-ppoo		gubernaator	sinda-xa-ni		bi-čči.\\
     	\textsc{1pl}	village-\textsc{all}-\textsc{1pl}.\textsc{poss}	governor		come-\textsc{pst}-\textsc{3sg}	be-\textsc{pfv}\\
\glt `A governor had come to our village. [He had already left.]’\il{Uilta}
\z

In \ili{Uilta}, with atelic verbs, the same form can also be used to express past progressive meaning:

\ea
\label{example3.12}
\gll Bii	gyauli-du-wwee		bii	mapa-ŋu-bi			eekkuta-xa-ni		bi-čči.\\
     	\textsc{1sg}	row+\textsc{ipfv(pres)}-\textsc{loc}-\textsc{1sg}	\textsc{1sg}	old.man-\textsc{al}-\textsc{1sg.poss}	steer-\textsc{pst}-\textsc{3sg}		be-\textsc{pfv}\\
\glt `While I was rowing, my husband was steering.’\il{Uilta}
\z

Either the lexical verb or the copula can take subject agreement marking, i.e. both \textit{sinda-xa-ni bi-čči} [come-\textsc{pst-3sg} be-\textsc{pfv}] and \textit{sinda-xa bi-čči-ni} [come-\textsc{pfv} be-\textsc{pst-3sg}] are correct (but not *\textit{sinda-xa-ni bi-čči-ni} or *\textit{sinda-xa bi-čči}).


\subsection{Direct evidential/affirmative-emphatic past in \textit{-tAA}}\label{Section3.3.3}

The direct evidential/affirmative-emphatic past form in \textit{-tAA} is marginal in pres\-ent-day \ili{Uilta}. It does not appear naturally in narratives or dialogue, and all attestations were obtained through elicitation. It is used overwhelmingly in the third, occasionally in the second, and very rarely in the first person. In the third person, its main use is direct visual evidential as in (\ref{example3.13}):

\ea
\label{example3.13}
\gll Sii	ŋinda-si			bii	nakku-ŋŋoo-wwee		puktuu-təə.\\
     	\textsc{2sg}	dog-\textsc{2sg}.\textsc{poss}		\textsc{1sg}	chicken-\textsc{al}+\textsc{acc}-\textsc{1sg}	carry.away-\textsc{direvid}.\textsc{pst}.\textsc{3}\\
\glt `Your dog carried away my chicken.’ [The hearer cannot retort ‘it wasn’t my dog’ because the speaker saw it.]\il{Uilta}
\z

It can also combine direct evidential with emphatic meaning as in (\ref{example3.14}):

\ea
\label{example3.14}
\gll Ɵrɵɵ,	aya		bara	nari-sal. 		Əsi		sinda-taa-l				ulaa-ǰi.\\
     	\textsc{intj}		very		many	people-\textsc{pl}	now		come-\textsc{direvid}.\textsc{pst}.\textsc{3}-\textsc{pl}	reindeer-\textsc{instr}\\
\glt `Wow, how many people. They just came by reindeer.’\il{Uilta}
\z

Rarely, it can be used purely emphatically, without clear evidential connotation (although not incompatible with it), as in (\ref{example3.15}):

\ea
\label{example3.15}
\gll Seryozha			uumbu-čči-du-ni		sundatta		tarttəə		iktəmə-təə.\\
     	Seryozha(\textsc{pn})		fish-\textsc{pfv(pst)}-\textsc{loc}-\textsc{3sg}		fish			suddenly		bite-\textsc{direvid}.\textsc{pst}.\textsc{3}\\
\glt `When Seryozha was fishing, a fish suddently bit.’\il{Uilta}
\z

It is overwhelmingly used in immediate past (just witnessed) contexts, with adverbs like \textit{tarttəə} ‘there (emphatic), right now’. It is incompatible with indirect reported speech, only with direct reported speech as in (\ref{example3.16}):

\ea
\label{example3.16}
\gll Sergei		mittəi	uč-či-ni: 		“Attaa,		tari	nari	pastuuxi-tai	ŋənə-təə”.\\
     	Sergei(\textsc{pn})	\textsc{1sg}.\textsc{all}	say-\textsc{pst}-\textsc{3sg}	grandmother	this	man	herder-\textsc{all}	go-\textsc{direvid}.\textsc{pst}.\textsc{3}\\
\glt `Sergei said to me: “Grandma, he left to join the reindeer herders”.’\il{Uilta}
\z

In the second person, the form in \textit{-tAA} has affirmative-emphatic meaning as in (\ref{example3.17}), typically reinforced by the emphatic use of the adverb \textit{goči} ‘again, indeed’.



\ea
\label{example3.17}
\gll Sii	dəptu-tə-ssee			goči!\\
     	\textsc{2sg}	eat-\textsc{direvid}.\textsc{pst}-\textsc{2sg}	\textsc{emph}\\
\glt `You have already eaten though!’\il{Uilta}
\z

Finally, very rarely, the form in \textit{-tAA} can also be used in the first person, also with affirmative-emphatic meaning as in (\ref{example3.18}).

\ea
\label{example3.18}
\gll Buu	təə-wu-tə-ppɵɵ			goči		čaa	duwa-du		kartooskkaa.\\
     	\textsc{1pl}	sit-\textsc{trans}-\textsc{direvid}.\textsc{pst}-\textsc{1pl}	\textsc{emph}	that	summer-\textsc{loc}	potato+\textsc{acc}\\
\glt `We did plant potatoes that summer.’\il{Uilta}
\z


\subsection{Diachronic development of Uilta past tense forms}\label{Section3.3.4}

Earlier descriptions of \ili{Uilta} past tense forms (\citealt{Ikegami_1959}; \citealt{Tsumagari_2009}) describe the finite form in \textit{-tAA} as fully productive, with a complete person/number paradigm. Already at that stage it was restricted to direct evidential contexts (\citealt{Ikegami_1959}), and as is clear from the above description, it has become even more restricted in present-day \ili{Uilta}, with the participial form in \textit{-xAn} used predicatively in almost all contexts.

Together with the fact that the form in \textit{-xAn} retains resultative/perfect meaning (cf. example \ref{example3.10} above), this points to a diachronic development where the perfective participle gradually replaced the erstwhile finite form, through resultative and perfect stages. This mirrors the development observed in other Tungusic languages (cf. \citealt{Malchukov_2000}: 447), along a well-attested grammaticalisation path (\citealt{Bybee_1994}: 105).


\section{Present domain}\label{Section3.4}

In the present domain \ili{Uilta} displays competition between two forms, the general present form in +\textit{RI},\footnote{The form in +\textit{RI} has irregular conjugation and alternates between \textit{-ri}, \textit{-si}, \textit{-ǰi} and consonant reduplication and/or vowel reduplication and/or alternation. See \citet{Ikegami_1959} for a full breakdown of alternations by conjugational class of the verb stem. For this and other forms, irregular inflection is marked by a plus sign and capital letters throughout this paper (capitalised vowels indicate vowel harmony).} and the direct evidential/emphatic/mirative in +\textit{RAkkA}. The form in +\textit{RI} grammaticalised from the imperfective participle, while +\textit{RAkkA} is the original finite form.\footnote{\label{footnote8}+\textit{RA} is cognate with the Tungusic aorist form in \textit{-rA}. \textit{-rA} in combination with the emphatic particle in \textit{=k(k)A} is attested as an (emphatic) confirmative mood form in a number of Tungusic languages (\citealt{Malchukov_2000}: 458). In \ili{Uilta} the bare form in +\textit{RA} marks the lexical verb (glossed as connegative) in negative constructions with the inflected negative auxiliary in ə- (cf. examples \ref{example3.8}, \ref{example3.34}, \ref{example3.38}, \ref{example3.41} and \ref{example3.43}). In combination with other morphemes, it forms the direct evidential/mirative/emphatic in +\textit{RAkkA} (cf. \sectref{Section3.4.2}), the likely/anticipated future in +\textit{RAŋA} (\sectref{Section3.5.3}), and the different-subject imperfective conditional converb in +\textit{RAi} (cf. examples \ref{example3.34}, \ref{example3.41}). All forms in +\textit{RA} in \ili{Uilta} have irregular conjugations and alternate between -\textit{rA}, -\textit{si} and vowel reduplication and/or alternation and/or consonant reduplication. See \citet{Ikegami_1959} for details.} The person/number agreement paradigms for both forms are shown in Table \ref{table:4} (the form in +\textit{RAkkA} is only attested in the 3\textsuperscript{rd} person in my data).

\begin{table}
\caption{Person/number agreement paradigms of Uilta present tense forms}
\label{table:4}
 \begin{tabular}{llll}
  \lsptoprule
Number & Person & +\textit{RI} & +\textit{RAkkA}\\
  \midrule
&   1\textsuperscript{st}  &   +\textit{RI-wi}      & –\\
Singular &   2\textsuperscript{nd}  &   +\textit{RI-si}      & –\\
&   3\textsuperscript{rd}  &   +\textit{RI-ni}      & +\textit{RAkkA}\\
  \midrule
&   1\textsuperscript{st}  &   +\textit{RI-pu}      & –\\
Plural &  2\textsuperscript{nd}  &    +\textit{RI-su}      & –\\
&   3\textsuperscript{rd}  &   +\textit{RI-či}      & +\textit{RAkkA-l}\\
  \lspbottomrule
 \end{tabular}
\end{table}

\subsection{General present in +\textit{RI}}\label{Section3.4.1}

The form in +\textit{RI} is the most frequent present tense form, used in all present contexts except for direct evidential, emphatic and mirative, where the form in +\textit{RAkkA} is used instead (see below). It is used for events occurring at the moment of speaking as in (\ref{example3.19}), events occurring in the present generally (generic present) as in (\ref{example3.20}), habitual events (\ref{example3.21}), and general statements (\ref{example3.22}).

\ea
\label{example3.19}
\gll Bii	dəgǰitə			pukki-mbi.		J̌iŋ		nalmakta		see-ri-či.\\
     	\textsc{1sg}	burn+\textsc{hort}.\textsc{1sg}	fire-\textsc{1sg}.\textsc{poss}		very		mosquitoes	bite-\textsc{pres}-\textsc{3pl}\\
\glt `Let me start a fire. Mosquitoes are biting really bad.’\il{Uilta}
\ex
\label{example3.20}
\gll Bɵyɵ-mbɵ	uǰi-pissəə 				waa-ri-či.\\
     	bear-\textsc{acc}		rear-\textsc{cond}.\textsc{conv}.\textsc{ss}.\textsc{pl}		kill-\textsc{pres}-\textsc{3pl}\\
\glt `They rear the bear and then kill it.’\il{Uilta}
\ex
\label{example3.21}
\gll Nooni	purə-ttəi		puli-si-ni			waa-ŋda-su-si-ni.\\
     	\textsc{3sg}		taiga-\textsc{all}	walk-\textsc{pres}-\textsc{3sg}	kill-\textsc{ven}-\textsc{iter}-\textsc{pres}-\textsc{3sg}\\
\glt `He regularly goes into the taiga and hunts.’\il{Uilta}
\ex
\label{example3.22}
\gll Suwəə-ǰǰee	suunə	agbi-nǰi-ni,		pərxi-ǰǰee	tuu-ǰǰi-ni.\\
     	east-side		sun		appear-\textsc{pres}-\textsc{3sg}	west-side	fall-\textsc{reiter}+\textsc{pres}-\textsc{3sg}\\
\glt `The sun rises in the east and sets in the west.’\il{Uilta}
\z

The form in +\textit{RI} also appears in narratives as in (\ref{example3.23}), although less frequently than the general past form in \textit{-xAn}.

\ea
\label{example3.23}
\gll Wəədə-ptu-xə		əəktə	peeččila-gačči	təə-si-ni		moo		pəǰǰee-du-ni.\\
     	lose-\textsc{intr}-\textsc{pfv}	woman	lean-\textsc{ant}.\textsc{conv}	sit-\textsc{pres}-\textsc{3sg}	tree		under-\textsc{loc}-\textsc{3sg}.\textsc{poss}\\
\glt `The lost woman sat down leaning against the tree.’\il{Uilta}
\z

In the Southern\il{Southern Uilta} dialect, which lacks the general future form in \textit{-li} (see \sectref{Section3.5.2} below), the form in +\textit{RI} is also used for both near (\ref{example3.24}) and distant future events (\ref{example3.25}).
	
\ea
\label{example3.24}
\gll Ɵlɵ-pee				sittəi	buu-ri-wi,		ɵlɵ-pula			sundattaa.\\
     	cook-\textsc{cond}.\textsc{conv}.\textsc{ss}	\textsc{2sg}.\textsc{all}	give-\textsc{npst}-\textsc{1sg}	cook-\textsc{pass}.\textsc{pfv}	fish+\textsc{acc}\\
\glt `I will cook and give [it] to you, the cooked fish.’ (Southern\il{Southern Uilta} dialect)
\ex
\label{example3.25}
\gll Bii	mɵrɵ-čči-wi		xaali=ddaa	daayi	naa-tai		ŋənnee-wi.\\
     	\textsc{1sg}	think-\textsc{dur+npst}-\textsc{1sg}	when=\textsc{foc}	big		land-\textsc{all}		go+\textsc{npst}-\textsc{1sg}\\
\glt `I am thinking, I will go to the mainland someday.’ (Southern\il{Southern Uilta} dialect)
\z

On top of its predicative use as the main verbal present form, the form in +\textit{RI} retains its participial (adnominal)/nominal function, as in (\ref{example3.26}); cf. also example (\ref{example3.32}).

\ea
\label{example3.26}
\gll Pɵččɵ-nɵ-si-l=ddəə,		mičči-l=ddəə,			naa-wa 		xullee-l=ddəə.\\
     	jump-\textsc{iter}-\textsc{ipfv}-\textsc{pl}=\textsc{foc}	crawl+\textsc{ipfv}-\textsc{pl}=\textsc{foc}	earth-\textsc{acc}	burrow+\textsc{ipfv}-\textsc{pl}=\textsc{foc}\\
\glt `Those [insects and worms] that jump, those that crawl, and those that burrow in the ground.’\il{Uilta}
\z


\subsection{Direct evidential/emphatic/mirative present in +\textit{RAkkA}}\label{Section3.4.2}

The direct evidential/emphatic/mirative present form in +\textit{RAkkA}, while far more restricted than the general present in +\textit{RI}, is more frequent than the past evidential/affirmative-emphatic form in -\textit{tAA}, and occurs naturally in everyday speech (\citealt{Yamada_2013}: 114). Previously, it was reported to have 1) direct evidential and 2) experiential meaning, and a full person/number paradigm (\citealt{Ikegami_1959}). In present-day \ili{Uilta}, it is restricted to third person use, and to events witnessed by the speaker, at the moment of speech as in (\ref{example3.27}):

\ea
\label{example3.27}
\gll Xəwərə-kki	bɵyɵtɵɵ	daurakka.			Pauri-mi		aaptu-li-ni=yyuu,	xai=yyuu?\\
     	lagoon-\textsc{prol}	bear.cub	cross+\textsc{direvid}.\textsc{pres}.\textsc{3}	swim-\textsc{conv}	reach-\textsc{fut}-\textsc{3sg}=\textsc{q}	what=\textsc{q}\\
\glt `A bear cub is swimming across the lagoon. Is it going to make it or not?’\il{Uilta}
\z

Very occasionally, it is used in non-visual direct evidential contexts as in (\ref{example3.28}).

\ea
\label{example3.28}
\gll Tarree,		čoora	ui-sikkə.				Nari-sal	sindaakka-lee.\\
     	that+\textsc{emph}	bell		ring-\textsc{direvid}.\textsc{pres}.\textsc{3}	man-\textsc{pl}	come+\textsc{direvid}.\textsc{pres}-\textsc{pl}+\textsc{emph}\\
\glt `There, I can hear a bell. People are coming.’\il{Uilta}
\z

Typically, it combines direct evidential and emphatic meaning as in (\ref{example3.29}).

\ea
\label{example3.29}
\gll Ɵɵ,	sindaakka			tari	nari,	sokto-xo			čipal!\\
     	\textsc{intj}	come+\textsc{direvid}.\textsc{pres}.\textsc{3}	this	man		get.drunk-\textsc{pfv}		completely\\
\glt `There, this man is coming, completely drunk!’\il{Uilta}
\z

In some instances, the emphatic meaning is clearly more prominent, and the evidential function secondary at best, as in (\ref{example3.30}) and (\ref{example3.31}).

\ea
\label{example3.30}
\gll Nooni	mittəi	čii			puli-sikkə,			čii			puli-sikkə.\\
     	\textsc{3sg}		\textsc{1sg}.\textsc{all}	constantly	walk-\textsc{direvid}.\textsc{pres}.\textsc{3}	constantly	walk-\textsc{direvid}.\textsc{pres}.\textsc{3}\\
\glt `He constantly comes to me [won’t leave me alone].’\il{Uilta}
\ex
\label{example3.31}
\gll Bii	ulbaaxxoo-wwee		təitəəkkə				goči		ai	suddəəki		tari	əəktə.\\
     \textsc{1sg}	dress+\textsc{acc}-\textsc{1sg}.\textsc{poss}	wear+\textsc{direvid}.\textsc{pres}.\textsc{3}	\textsc{emph}	\textsc{intj}	shameless	this	woman\\
\glt `She’s wearing my dress, how shameless, this woman.’\il{Uilta}
\z

Finally, the form in +\textit{RAkkA} is used to express mirative meaning (the speaker’s surprise at unexpected revelation or new information), as in (\ref{example3.32}).

\ea
\label{example3.32}
\gll Ɵrɵɵi,	tari	nurreekka				goči		ləədənǰi-wə-ppɵɵ!\\
     \textsc{intj}		this	write+\textsc{direvid}.\textsc{pres}.\textsc{3}		\textsc{emph}	talk+\textsc{ipfv}-\textsc{acc}-\textsc{1pl}\\
\glt `Oh, it is recording what we are saying!’ [The informants realised that the recording device was on.]\il{Uilta}
\z

\subsection{The effect of insubordination on the Uilta present tense forms}\label{Section3.4.3}

Similar to what we observe in the past domain, the functional distribution of the two present forms in present-day \ili{Uilta} is consistent with the new form of participial origin gradually replacing the old finite form in most contexts, limiting it to direct evidential, emphatic and mirative uses. This mirrors the development in other languages of the \ili{Udegheic} and \ili{Nanaic} groups, where the participial forms, semantically neutral, pushed out the old verbal forms into direct evidential, validational and affirmative-emphatic uses, to varying degrees (markedness reversal). This process is typically further advanced in the past domain than in the present (Tense Hierarchy; \citealt{Malchukov_2000}).\footnote{\label{footnote9}\citet[450]{Malchukov_2000} postulates the Markedness Hierarchy according to which the process of replacement of unmarked finite forms through marked participial forms is further advanced in the past than the present domain, in the plural further than in the singular, and in the 3\textsuperscript{rd} further than in the 1\textsuperscript{st} and 2\textsuperscript{nd} person.} This is borne out by the fact that the present finite form in +\textit{RAkkA} is more frequent than the corresponding past form in -\textit{tAA} in present-day \ili{Uilta}. 

The fact that the form in +\textit{RAkkA} is restricted to third person use in present-day \ili{Uilta} is probably motivated by the fact that the third person is more congruous with direct evidential and mirative semantics.


\section{Future domain}\label{Section3.5}

The \ili{Uilta} future tense domain displays the clearest example of insubordination at work. There are three future tense forms in the Southern\il{Southern Uilta} dialect, two of finite and one of participial origin. The present-day Northern\il{Northern Uilta} dialect additionally features another participial form. It will be shown, through comparison with previous descriptions, that this new form replaced the old finite forms in most functional domains, to become the most productive future form in the Northern dialect.

The four forms are: general future in \textit{-li} (Northern\il{Northern Uilta} dialect only), immediate spontaneous future in +\textit{RIlA}, likely/anticipated future in +\textit{RAŋA}, and probable future in +\textit{RIli}. +\textit{RIlA} and +\textit{RAŋA} are pure verbal forms, i.e. can only be used as predicates of a main clause. They take person/number agreement of the verbal type. The forms in \textit{-li} and +\textit{RIli} are of participial origin, and retain their function as participles/nominalisations. They take agreement of the nominal (possessive) type, also in predicative use. The agreement paradigms for all four forms are presented in Table \ref{table:5}.

\begin{table}
\caption{Person/number agreement paradigms of Uilta future tense forms}
\label{table:5}
 \begin{tabular}{llllll}
  \lsptoprule
Number & Person & \textit{-li} (N dialect) & +\textit{RIli} & +\textit{RIlA} & +\textit{RAŋA}\\
  \midrule
  &   1\textsuperscript{st}  &   \textit{-li-wi}      & +\textit{RIli-wi} & +\textit{RIlA-mi} & +\textit{RAŋŋii}\\
  Singular  &  2\textsuperscript{nd}  &   \textit{-li-si}  & +\textit{RIli-si} & +\textit{RIlA-si}    & +\textit{RAŋA-si}\\
  &    3\textsuperscript{rd}  &   \textit{-li-ni}      & +\textit{RIli-ni} & +\textit{RIllAA} & +\textit{RAŋŋAi}\\
  \midrule
  &   1\textsuperscript{st}  &   \textit{-li-pu}      & +\textit{RIli-pu} & +\textit{RIlA-pu} & +\textit{RAŋA-pu}\\
  Plural  & 2\textsuperscript{nd}  &    \textit{-li-su}      & +\textit{RIli-su} & +\textit{RIlA-su} & +\textit{RAŋA-su}\\
  &  3\textsuperscript{rd} &   \textit{-li-či}      & +\textit{RIli-či} & +\textit{RIllAA-l} & +\textit{RAŋŋA-l}\\
  \lspbottomrule
 \end{tabular}
\end{table}


\subsection{Immediate spontaneous future in +\textit{RIlA}}\label{Section3.5.1}

The immediate spontaneous future form in +\textit{RIlA}, from the imperfective participle in +\textit{RI} plus \textit{-lA} (< *-\textit{lan}, of unknown origin; \citealt{Pevnov_2016}), is the most productive future form in the Southern\il{Southern Uilta} dialect, and the second most productive in the Northern\il{Northern Uilta} dialect, where it competes with the general future form in \textit{-li}.


In the Northern\il{Northern Uilta} dialect, +\textit{RIlA} is restricted to immediate future spontaneous contexts, as in (\ref{example3.33}), (\ref{example3.34}) and (\ref{example3.35}).

\ea
\label{example3.33}
\gll Sii	čaa	bičixxəə		tauuta-ssee,					bii	sittəi	kampeetka buu-rilə-mi.\\
    	\textsc{2sg}	that	book+\textsc{acc}	read+\textsc{cond}.\textsc{pfv}.\textsc{conv}.\textsc{ds}-\textsc{2sg}	\textsc{1sg}	\textsc{2sg}.\textsc{all}	candy
	give-\textsc{nearfut}-\textsc{1sg}\\
\glt `If you read this book, I will give you a candy.’
\ex
\label{example3.34}
\gll Sii	noo<mba>ni		ə-siyi-si							sommee			nooni pukči-lləə.\\
    	\textsc{2sg}	\textsc{3sg}<\textsc{acc}>		\textsc{neg}.\textsc{aux}-\textsc{cond}.\textsc{ipfv}.\textsc{conv}.\textsc{ds}-\textsc{2sg}	close+\textsc{conneg}	\textsc{3sg}
		jump-\textsc{nearfut.3}\\
\glt `If you don’t close it [the door], it [the cat] will run around.’
\ex
\label{example3.35}
\gll Kooppee		umi-gačči,		gayai-ǰǰila-mi.\\
    	coffee+\textsc{acc}	drink-\textsc{ant}.\textsc{conv}	stay.awake-\textsc{nearfut}-\textsc{1sg}\\
\glt `Having drunk coffee, [as a result] I won’t be able to sleep.’
\z
	
In the most detailed previous description (\citealt{Ikegami_1959}) the form in +\textit{RIlA} was characterised as expressing 1) near future, 2) future of which the speak\-er is sure, and 3) spontaneous action in the future. In all attestations of this form in my data, both conditions 1) and 3), namely short temporal distance and spontaneity, are met. Furthermore, the form is limited to very near, or immediate, future contexts. “Spontaneous” does not imply agent’s own volition, cf. example (\ref{example3.35}). The relevant distinction is between spontaneous, as in decided/realised on the spot, and planned, or otherwise predicted or predictable events. The form in +\textit{RIlA} is compatible with durative verbs as in (\ref{example3.34}) and (\ref{example3.35}), but for actions and states extending into the future, which conflict with its immediate future semantics, the form in \textit{-li} will be used instead (see below). Similarly, for the epistemic modal function reported previously, future that the speaker is sure of, the forms in \textit{-li} or +\textit{RIli} (see \sectref{Section3.5.2} and \sectref{Section3.5.4} below) will normally be used unless the use of the form in +\textit{RIlA} is specifically conditioned by immediate and spontaneous context.


\subsection{General future in \textit{-li}}\label{Section3.5.2}

In the Northern\il{Northern Uilta} dialect of \ili{Uilta}, the form in \textit{-li} is the most productive, general future form, with the other forms limited to their specific functions. It is used in all contexts that do not warrant the use of any of the other forms, immediate spontaneous future in +\textit{RIlA}, or the two marginal forms with epistemic modal semantics, +\textit{RIli} and +\textit{RAŋA} (see \sectref{Section3.5.3} and \sectref{Section3.5.4} below). For example, it is used for all planned future events, whether near (\ref{example3.36}) or distant (\ref{example3.37}).

\ea
\label{example3.36}
\gll Iigəri	sindauta-nnee,				buu		Naxulakka-tai	ŋənə-li-pu.\\
    	Igor(\textsc{pn})	come+\textsc{cond}.\textsc{pfv}.\textsc{conv}.\textsc{ds}-\textsc{3sg}	\textsc{1pl}		Nogliki-\textsc{all}		go-\textsc{fut}-\textsc{1pl}\\
\glt `When Igor comes, we will go to Nogliki.’ [already planned]\il{Uilta}
\ex
\label{example3.37}
\gll Ərkəə		nooni	xotto-du		isu-li-ni			duku-takki.\\
    	next.year		\textsc{3sg}		city-\textsc{abl}		come.back-\textsc{fut}-\textsc{3sg}	house-\textsc{all}.\textsc{refl}\\
\glt `Next year he will return home from the city.’\il{Uilta}
\z


It is also used for predicted or expected future outcomes (\ref{example3.38}), (\ref{example3.39}), or statements about the future that hold generally (\ref{example3.40}).


\ea
\label{example3.38}
\gll Məənə	boččoo-bi			əəxəktə-mi,		tari	andu-l-bi ə-mi=ddəə			xoǰǰee			o-li-si		taani.\\
    	own		face+\textsc{acc}-\textsc{refl}.\textsc{poss}	take.care-\textsc{conv}	this	work-\textsc{pl}-\textsc{refl}.\textsc{poss}
	\textsc{neg}.\textsc{aux}-\textsc{conv}=\textsc{foc}	finish+\textsc{conneg}	do-\textsc{fut}-\textsc{2sg}	likely\\
\glt `If you are preoccupied with your own face [looks], you won’t finish these works.’\il{Uilta}
\ex
\label{example3.39}
\gll Tari	xoosa-ŋu-l-bari						to-ǰǰi-ndulli,				əsi=ləkə 	utta-lu		o-li-pu.\\
    	this	reindeer.leg.hide-\textsc{al}-\textsc{pl}-\textsc{refl}.\textsc{poss}.\textsc{pl}	do-\textsc{reiter}+\textsc{ipfv(pres)}-\textsc{loc}.\textsc{refl}	now=\textsc{top}
	boot-\textsc{poss}	become-\textsc{fut}-\textsc{1pl}\\
\glt `As we process these reindeer leg hides, now we are going to be in possession of boots.’\il{Uilta}
\ex
\label{example3.40}
\gll Nəŋnə	boo-du		nama-li-ni.\\
    		spring	outside-\textsc{loc}	be.warm-\textsc{fut}-\textsc{3sg}\\
\glt `In the spring it is going to be warm outside.’\il{Uilta}
\z


It is also used instead of the form in +\textit{RIlA} for unplanned, spontaneous events if these are not temporally limited to the immediate future, as in (\ref{example3.41}).


\ea
\label{example3.41}
\gll “Sii	gaandu-ittaayi-si						məənə	puttə-bi,			bii	sindu gəsə		ə-li-wi				bee”,		unǰi-ni			nooni	sitəu mama-ŋu-ni.\\
    		\textsc{2sg}	go.after-\textsc{vol}+\textsc{cond}.\textsc{ipfv}.\textsc{conv}.\textsc{ds}-\textsc{2sg}	own		child-\textsc{refl}.\textsc{poss}	\textsc{1sg}	\textsc{2sg}.\textsc{loc}
	together	\textsc{neg}.\textsc{aux}-\textsc{fut}-\textsc{1sg}		be+\textsc{conneg}	say+\textsc{pres}-\textsc{3sg}	\textsc{3sg}		new
	wife-\textsc{al}-\textsc{3sg.poss}\\
\glt `“If you want to go and bring your child, I won’t live with you”, says his new wife.’\il{Uilta}
\z


Finally, as with the forms in \textit{-xAn} and +\textit{RI}, the form in \textit{-li} retains its participial (attributive/nominal) function as in (\ref{example3.42}).

\ea
\label{example3.42}
\gll Nooči	sinda-li-či		ulaa-l-ba			uidu-xə-či.\\
    		\textsc{3pl}		come-\textsc{fut}-\textsc{3pl}	reindeer-\textsc{pl}-\textsc{acc}	send-\textsc{past}-\textsc{3pl}\\
\glt `They₁ dispatched the reindeer by which they₂ are coming.’\il{Uilta}
\z


\subsection{Likely/anticipated future in +\textit{RAŋA}}\label{Section3.5.3}

The form in +\textit{RAŋA} (cf. footnote \ref{footnote8}) combines temporal and epistemic/deontic modal meaning, expressing future that the speaker considers very likely, for example through inference from past experience or common knowledge. In a previous description (\citealt{Ikegami_1959}) it was characterised as follows: 1) distant future, 2) possible future, 3) action in the future the doer is compelled or obliged to perform. In present-day \ili{Uilta}, this form has no inherent temporal distance value, its use being conditioned exclusively by its epistemic/deontic modal function.\footnote{I gloss this form as “distant future” throughout this paper in line with previous descriptions, and to distinguish it from other future forms.} It is exemplified below in the 1\textsuperscript{st}, 2\textsuperscript{nd} and 3\textsuperscript{rd} person use, expressing likelihood based on inference from circumstances (\ref{example3.43}), common knowledge (\ref{example3.44}), and past experience (\ref{example3.45}). Example (\ref{example3.46}) shows the use of the form in +\textit{RAŋA} in the deontic modal function (obligatoriness). As is clear from the below examples, it is not limited to distant future contexts.

\ea
\label{example3.43}
\gll Gəə,		balaa,		gata-mari		ə-ŋə-pu					kulpee. Kusal-ǰi		gitu-mari		ŋənneesu.\\
    		\textsc{intj}		fast+\textsc{emph}	pick-\textsc{conv}.\textsc{pl}		\textsc{neg}.\textsc{aux}-\textsc{distfut}-\textsc{1pl}		make.it.in.time+\textsc{conneg}
	fast-\textsc{instr}	walk-\textsc{conv\textsc}.\textsc{pl}	go+\textsc{hort}.\textsc{1pl}\\
\glt `Come on, faster, we won’t finish picking in time. Let’s walk faster.’
\ex
\label{example3.44}
\gll Čii			tagda-na-mi,			čii			tagda-na-mi, 	ənu-llɵɵŋə-si.\\
    		constantly	be.angry-\textsc{iter}-\textsc{conv}	constantly	be.angry-\textsc{iter}-\textsc{conv}
	be.ill-\textsc{inch}+\textsc{distfut}-\textsc{2sg}\\
\glt `If you are angry all the time, you will fall ill.’\il{Uilta}
\ex
\label{example3.45}
\gll Dolbo	puttə-ni			soŋŋee-du-ni			əni-ni			əmu-mi tooŋŋai.			Əmu-siŋŋəi.\\
    		night	child-\textsc{3sg}.\textsc{poss}	cry+\textsc{ipfv(pres)}-\textsc{loc}-\textsc{3sg}		mother-\textsc{3sg}.\textsc{poss}	rock-\textsc{conv}
	do+\textsc{distfut}.\textsc{3}		rock-\textsc{distfut}.\textsc{3}\\
\glt `During the night, when her child cries, the mother will be rocking him. She will rock him.’\il{Uilta}
\ex
\label{example3.46}
\gll Bii	xotto-ttoi		ŋənnəəŋŋii,			puyə-bi			ittəu-ndəəŋŋii.\\
    		\textsc{1sg}	city-\textsc{all}		go+\textsc{distfut}.\textsc{1sg}		wound-\textsc{1sg}.\textsc{poss}	show-\textsc{ven}+\textsc{distfut}.\textsc{1sg}\\
\glt `I ought to go to the city and have my wound looked at.’\il{Uilta}
\z


\subsection{Probable future in +\textit{RIli}}\label{Section3.5.4}

Similar to the form in +\textit{RAŋA}, the form in +\textit{RIli} (from imperfective participle +\textit{RI} plus future participle \textit{-li}) combines temporal and epistemic modal meaning, expressing future that the speaker considers probable (cf. also \citealt{Ikegami_1959}). It is usually accompanied by the adverb \textit{taani} ‘likely, probably’, as in (\ref{example3.47}) and (\ref{example3.48}).

\ea
\label{example3.47}
\gll Upa-ŋu-bi			dabgu-xa-ni,		ləpeeskə-buǰǰi			to-i-ni. 	Isu-pee				to-ǰǰeeli-ni				taani.\\
    		flour-\textsc{al}-\textsc{refl}.\textsc{poss}	prepare-\textsc{pst}-\textsc{3sg}	make.flatbread-\textsc{purp.refl}	do-\textsc{pres}-\textsc{3sg}
	come.back-\textsc{cond}.\textsc{conv}.\textsc{ss}	do-\textsc{reiter}+\textsc{probfut}-\textsc{3sg}	likely\\
\glt `She prepared the flour, intends to make flatbread. She will probably resume making it when she comes back.’\il{Uilta}
\ex
\label{example3.48}
\gll Yə	uilə-bi			xoǰi-gačči,		goi		uilə-bi			čai-wa	umi-pee				otokoo	to-ǰǰeeli-wi				taani.\\
    		this	work-\textsc{refl}.\textsc{poss}	finish-\textsc{ant}.\textsc{conv}	other	work-\textsc{refl}.\textsc{poss}	tea-\textsc{acc}
	drink-\textsc{cond}.\textsc{conv}.\textsc{ss}	later		do-\textsc{reiter}+\textsc{probfut}-\textsc{1sg}	likely\\
\glt `Having finished this work, after having tea, I will likely go back to doing the other work later.’\il{Uilta}
\z

Like the forms in \textit{-xAn}, +\textit{RI} and \textit{-li}, the form in +\textit{RIli} is ultimately of participial origin, and retains its attributive/nominal function, as in (\ref{example3.49}).

\ea
\label{example3.49}
\gll Nooni	aduli-bi				atu-ǰǰeeli-wa-ni					tari-sal	sinda-xa-či, tulə-du-xə-či.\\
    		\textsc{3sg}		fishing.net-\textsc{refl}.\textsc{poss}	remove-\textsc{reiter}+\textsc{probfut}-\textsc{acc}-\textsc{3sg}	that-\textsc{pl}	come-\textsc{pst}-\textsc{3pl}
	set-\textsc{reiter}-\textsc{pst}-\textsc{3pl}\\
\glt `They came and set again the fishing nets that he wanted to remove.’\il{Uilta}
\z


\subsection{The effect of insubordination on the Uilta future tense forms}\label{Section3.5.5}

The general future form in \textit{-li}, the most productive future form in the present-day Northern\il{Northern Uilta} dialect of \ili{Uilta}, is not attested in the previous descriptions before the 2000s (cf. Table \ref{table:2}). Moreover, it is not attested in the Southern\il{Southern Uilta} dialect, where the finite form in \textit{+RIlA} is the most productive future tense form, with the form in \textit{+RI} also extended to future use. The most comprehensive description of the Northern\il{Northern Uilta} dialect, by \citet{Petrova_1967}, does not mention the form in \textit{-li}, but briefly describes another future form in \textit{-llee}, not mentioned anywhere else. It is unclear whether the forms in \textit{-llee} and \textit{-li} are related, but consonant gemination with vowel lengthening is a prominent feature in \ili{Uilta}, frequently used for emphasis (cf. example \ref{example3.28}), but also for marking grammatical categories like accusative (cf. e.g. examples \ref{example3.6} and \ref{example3.33}). With some markers, e.g. the connegative form in +\textit{RA} (from Tungusic aorist in \textit{-rA}, cf. footnote \ref{footnote8}), there is free variation between geminated and ungeminated forms in some conjugations.

Nevertheless, the form in \textit{-li} features prominently in the most recent descriptions of the Northern\il{Northern Uilta} dialect (\citealt{Pevnov_2016}; \citealt{Yamada_2010b}; \citeyear{Yamada_2013}), as well as the data from fieldwork in recent years by the present author. It accounts for 60\% of all future forms in my data, with the form in +\textit{RIlA} at 40\%, and the other two forms being marginal. It apparently developed relatively recently in the Northern\il{Northern Uilta} dialect, and pushed out the older, finite forms in most functional domains: the old distant future form in +\textit{RAŋA} no longer displays the temporal distance value,  and is limited to epistemic modal uses; the form in +\textit{RIlA} is restricted to immediate future, spontaneous events. While tail-end languages are known to undergo substantial grammatical changes (\citealt{Harrison_Anderson_2008}),\footnote{“[L]ast generation speakers of endangered languages […] can and do introduce grammatical and phonological innovations, […] including changes resulting in both simplification and in greater complexity. It is often difficult to disentangle whether a particular change is driven by internal restructuring, contact induced change, obsolescence effects, or some combination of these.” (\citealt{Harrison_Anderson_2008}: 243 ff.).} this rather dramatic shift seems to be another manifestation of the tendency of Tungusic languages (and more broadly, languages of the ``\ili{Macro-Altaic}" areal-typological profile) to renew verbal forms through participles, through the processes of insubordination and verbalisation.


\section{Summary and conclusions}\label{Section3.6}\largerpage

As is clear from the above description, the processes of insubordination and verbalisation played a prominent role in the development of the \ili{Uilta} tense system. The gradual replacement of finite verbal forms through forms of participial origin, with the resulting functional shifts between old and new forms in the relevant verbal categories, is evident across all three temporal domains. In the past domain, the development of the perfective participle in \textit{-xAn} into the general past tense form, through resultative, perfect, and indirect evidential stages, mirrors the development in other Tungusic languages (\citealt{Malchukov_2000}: 447). \ili{Uilta} represents the last stage of this process as the form in \textit{-xAn} has no discernible evidential meaning; it functions as the general past tense form, with the resultative meaning only partially retained. The erstwhile finite form in \textit{-tAA} is marginally retained, with direct evidential and affirmative-emphatic (particularly in the first and second person, the third person being naturally more congruous with evidential meaning) functions, reflecting its development through the direct evidential and affirmative-emphatic stages, in competition with the finite form. Again, this mirrors the development in other languages of the \ili{Nanaic} and \ili{Udegheic} groups: as the participial forms replace the erstwhile finite forms, first in resultative/perfect, then indirect evidential use, the old past forms are restricted to the direct evidential function, and further develop affirmative-emphatic (validational) meaning (stages 2 and 3 in Figure \ref{tab:evolution} above).

Similarly, in the present domain, the participial form in +\textit{RI} replaced the old verbal form in +\textit{RAkkA} as the general present form, with the old form restricted to third person direct evidential, emphatic and, by extension, mirative uses. The fact that the form in +\textit{RAkkA}, although marginal and restricted to third person use, is still more frequent than the equivalent past form in \textit{-tAA} conforms to the Tense Hierarchy of the patterns of replacement of old verbal forms postulated in \citealt{Malchukov_2000}: 450).

Finally, in the future domain, the participial form in \textit{-li} pushed out the old finite forms in +\textit{RIlA} and +\textit{RAŋA} to become the most productive, general future tense form. This recent development, less advanced than in the past and present domains and limited to the Northern\il{Northern Uilta} dialect, is yet another example of the tendency of Tungusic languages to renew finite verbal forms through insubordination. It represents the most recent one in the history of repeated cycles of renewal of verbal forms through participles in Tungusic, with most finite forms, including the above forms in +\textit{RA}, ultimately of participial origin (\citealt{Robbeets_2009}).

\begin{sloppypar}
In fact, this tendency is not limited to Tungusic, with all languages of the ``\ili{Macro-Altaic}" areal-typological type repeatedly undergoing similar development, with some apparent parallels at the proto-languages stage as postulated by Robbeets (\citeyear{Robbeets_2009}; \citeyear{Robbeets_2015}), some evident in the diachronic development of individual families, and some still observed in the individual languages (\citealt{Malchukov_Czerwinski_2020}). Note, however, that this tendency is not limited to ``\ili{Macro-Altaic}", and instead constitutes a general areal feature of Siberian languages, including the \ili{Paleosiberian} and \ili{Uralic} languages (\citealt{Malchukov_2013}; \citealt{Malchukov_Czerwinski_2021}). In \ili{Uilta}, this process played a prominent role in the development, and is largely responsible for the current shape of the \ili{Uilta} tense system.
\end{sloppypar}

\section*{Abbreviations}
\begin{multicols}{2}
\begin{tabbing}
\textsc{nearfut}\hspace{1ex}\=near future\kill
\textsc{1} \> 1\textsuperscript{st} person\\
\textsc{2} \> 2\textsuperscript{nd} person\\
\textsc{3} \> 3\textsuperscript{rd} person\\
\textsc{abl} \> ablative (case)\\
\textsc{acc} \> accusative (case)\\
\textsc{agr} \> (person/number) agreement\\
\textsc{al} \> alienable (possession)\\
\textsc{all} \> allative (case)\\
\textsc{ant} \> anterior\\
\textsc{aux} \> auxiliary\\
\textsc{concur} \> concurrent\\
\textsc{cond} \> conditional\\
\textsc{conneg} \> connegative\\
\textsc{conv} \> converb\\
\textsc{cop} \> copula\\
% \textsc{dat} \> dative (case)\\
\textsc{direvid} \> direct evidential\\
\textsc{distfut} \> distant future\\
\textsc{ds} \> different subject\\
\textsc{dur} \> durative\\
\textsc{emph} \> emphatic\\
\textsc{foc} \> focus\\
\textsc{fut} \> future\\
\textsc{hort} \> hortative\\
\textsc{inch} \> inchoative\\
\textsc{intr} \> intransitiviser\\
\textsc{instr} \> instrumental (case)\\
\textsc{intj} \> interjection\\
\textsc{ipfv} \> imperfective\\
\textsc{iter} \> iterative\\
\textsc{loc} \> locative (case)\\
\textsc{n} \> nominalisation\\
\textsc{nearfut} \> near future\\
\textsc{neg} \> negative\\
\textsc{nfut} \> non-future\\
\textsc{npst} \> non-past\\
\textsc{part} \> participle\\
\textsc{pass} \> passive\\
\textsc{pfv} \> perfective\\
\textsc{pl} \> plural\\
\textsc{pn} \> proper name\\
\textsc{poss} \> possessive\\
\textsc{pres} \> present\\
\textsc{probfut} \> probable future\\
\textsc{pst} \> past\\
\textsc{purp} \> purposive\\
\textsc{q} \> question particle\\
\textsc{refl} \> reflexive\\
\textsc{reiter} \> reiterative\\
\textsc{s} \> subject\\
\textsc{sg} \> singular\\
\textsc{ss} \> same subject\\
\textsc{top} \> topic\\
\textsc{trans} \> transitive\\
\textsc{v} \> verb\\
\textsc{ven} \> venitive\\
\textsc{vol} \> volitive\\
\end{tabbing}
\end{multicols}

\section*{Acknowledgements}

I am very grateful to \name[Malchukov, Andrej L.]{Andrej}{Malchukov}, \name{Walter}{Bisang}, \name{Andreas}{Hölzl} and two anonymous reviewers for their valuable comments. I would like to express my deepest gratitude to my \ili{Uilta} informants, Elena A. Bibikova, Ljubovʹ R. Kitazima, Ljubovʹ N. Konusova, Irina G. Kurušina and Ljudmila X. Minato. This work was partially supported by the Laboratory Program for Korean Studies through the Ministry of Education of the Republic of Korea and the Korean Studies Promotion Service of the Academy of Korean Studies (AKS-2016-LAB-2250004).

{\sloppy\printbibliography[heading=subbibliography,notkeyword=this]}
\end{document}
