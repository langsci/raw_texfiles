\documentclass[output=paper,colorlinks,citecolor=brown]{langscibook}
\ChapterDOI{10.5281/zenodo.15471451}
\author{Ljudmila Geist\affiliation{University of Stuttgart}\orcid{0000-0001-7907-4958} and Joanna Błaszczak\affiliation{University of Wrocław}\orcid{0000-0002-8332-2827}}
\title{The fine-grained structure of the DP in Slavic: Evidence from the distributed plural hypothesis in Polish and Russian}
\abstract{In this chapter, building on \citet{Mathieu2014}, we put forth a split analysis of plural number. More precisely, we argue that functional morphemes can be polysemous. Different positions in the syntactic structure correlate with different interpretations. In other words, the plural is distributed along the syntactic spine and may realize three different functional heads in the fine-grained DP-structure: \textit{n}\textsuperscript{0}  (a low functional head), Div\textsuperscript{0} (a middle functional head), and  \#\textsuperscript{0} (a high functional head). We present evidence based on the plural marking in Russian and Polish showing that also in Slavic languages the plural comes in many guises, which justifies proposing different syntactic positions for its different semantic manifestations. We refer to this observation as the ``distributed plural hypothesis." Altogether we argue, based on the facts discussed in the chapter, that the assumption of a fine-grained DP-structure is justified in Russian and Polish, potentially also in other Slavic languages.

\keywords{DP, plural, countability, kind, classifier}
}




\begin{document}
\shorttitlerunninghead{The fine-grained structure of the DP in Slavic} % Kurzversion des Titels für die Kopfzeile
\maketitle

% %%%%%%%%%%%%%%%%%%%%%%%%%%%%%%%%%%%%%%%%%%%%%%%%%%%%%%%%%%%%%%%
% %%%%%%%%%%%%%%%%%%%%%%%%%%%%%%%%%%%%%%%%%%%%%%%%%%%%%%%%%%%%%%%

\section*{Preface}
This article is written in memory of our teacher and colleague Ilse Zimmermann. In her long scientific career, Ilse Zimmermann investigated various topics in a number of languages, including Slavic languages, in particular Russian but also Polish. One of the important research topics that she addressed was the semantics and syntax of nominal phrases; see \citet{Zimmermann1983,gb:Zimmermann1988,Zimmermann1991,Zimmermann2008}. The present work follows this tradition. We examine the structure of nominal phrases by looking at two Slavic languages: Russian and Polish.

In particular, in the present study we will address some issues which have already played an important role in the above-mentioned publications by Ilse Zimmermann. In this sense, we develop further (or deepen) Ilse Zimmermann's ideas. More specifically, we will attempt to provide evidence for a richer inventory of categories -- in our present work the categories in question are different guises of the plural. In this sense, what we are aiming at is to provide arguments for a more fine-grained structure of nominal phrases in Slavic languages.

Furthermore, just like \citet{gb:Zimmermann1988}, in the present study we will show that different meanings of a given category (here: the plural) can be attributed to different positions that a given formative can occupy in the syntactic structure (here: in the fine-grained nominal structure).
Finally, just like \citet{Zimmermann2008}, in the present study we investigate the polyfunctionality of certain elements (here: the plural) and their integration into the DP structure. However, we depart from the lexicalist view, which Ilse Zimmermann based her work on. The central tenet of the lexicalist view is the assumption that the morphosyntactic content of a word (word formation and rules of inflection) is determined in the lexicon. Words enter the syntax as complete atoms, and parts of a word cannot be altered by syntactic operations. We will take another view on the interaction between the lexicon and syntax, the so-called neo-constructionist view. Under this view, new words and word forms are built in the syntax. This view has become more prominent in recent years. In this study we follow the neo-constructionist approach since we think it can render a more transparent account of the phenomenon treated in this work -- the syntax and semantics of the plural. 

%-----------------------------------------------------

\section{Introduction and theoretical background}

It has been assumed in lexicalist approaches that knowledge about the meaning and grammar of a noun is stored in the lexicon and is independent of syntax (\citealt{DiSciulloWilliams1987}, \citealt{Zimmermann1983}, \citealt{gb:Zimmermann1988}, \citealt{Wasow1977}, among others). For instance, for the noun \textit{tree} the lexicon stores the information that it is a woody plant consisting of a root, a tall trunk rising from it, and a leafy crown. It is also coded in the lexicon that this noun is a count noun. This information suggests that it can be pluralized and combined with numerals, unlike, e.g., \textit{air} or \textit{water}, which are uncountable. The lexicalist approaches have been undermined in recent years by the observation that many nouns (\textit{stone}, \textit{cake}, \textit{rock}, \textit{judgment}) are flexible with respect to countability (\citealt{Rothstein2010}, \citealt{KissPelletier2014}, \citealt{Zamparelli2017}). In some syntactic contexts mass nouns can be interpreted as count and count nouns can be interpreted as mass in many languages. Thus, the syntactic context determines the interpretation of the noun as countable or uncountable. Much work on the interpretation of nouns suggests that this flexible behavior of nouns is the rule rather than the exception. This challenges the lexicalist view, according to which the mass/count distinction is lexically pre-established. To account for noun flexibility, neo-constructionist approaches (\citealt{Marantz1997}, \citealt{Borer2003}, \citealt{Borer2005}, \citealt{Kratzer2007}, among others) assume that countability is derived in the syntax by combining the non-countable nominal root with functional heads that host grammatical markers of countability. Different countability features are distributed along the syntactic spine. According to this view, being mass or count is a property of a whole DP. 

The syntactic projection of a noun phrase may include the following functional heads; see \figref{fig:1}. It should be noted that \#P and DivP are named differently by different researchers (cf. \citealt{AlexiadouStavrou2007}, \citealt{Borer2005}, \citealt{ChengSybesma1999,ChengSybesma2014}, \citealt{ChengZamparelli2017}, \citealt{Mathieu2012}, \citealt{Wiltschko2008}, \citealt{Zamparelli2000}, among others):

%\ea
%   \begin{forest}
%   for tree={s sep=1.2cm, inner sep=1, l=0}
%   [DP [D$^{0}$] [\#P [\#$^{0}$] [DivP [Div$^{0}$] [\textit{n}P [\textit{n}$^{0}$] [{[}$\sqrt{root}${]}] {\draw (.east) node[right=0.5cm]{$\longrightarrow$ descriptive content};}]{\draw (.east) node[right=0.5cm]{$\longrightarrow$ nominalization};}]{\draw (.east) node[right=0.5cm]{$\longrightarrow$ division/atomization};}]{\draw (.east) node[right=0.5cm]{$\longrightarrow$ quantization};}]{\draw (.east) node[right=0.5cm]{$\longrightarrow$ reference};}
%    \end{forest}
%\label{ex:geist:1}
%\z

\begin{figure}[ht]
   \begin{forest}
   for tree={s sep=1.2cm, inner sep=1, l=0}
   [DP [D$^{0}$] [\#P [\#$^{0}$] [DivP [Div$^{0}$] [\textit{n}P [\textit{n}$^{0}$] [{[}$\sqrt{root}${]}] {\draw (.east) node[right=0.5cm]{$\longrightarrow$ descriptive content};}]{\draw (.east) node[right=0.5cm]{$\longrightarrow$ nominalization};}]{\draw (.east) node[right=0.5cm]{$\longrightarrow$ division/atomization};}]{\draw (.east) node[right=0.5cm]{$\longrightarrow$ quantization};}]{\draw (.east) node[right=0.5cm]{$\longrightarrow$ reference};}
    \end{forest}
\caption{The fine structure of noun phrases}
\label{fig:1}
\end{figure}

Each functional layer hosts particular morphological elements that are endowed with a specific semantic function. The head D maps the whole phrase onto the referential argument. The DP-layer is a host of ``strong'' determiners such as definite articles and demonstratives. In articleless languages, D is phonologically empty, but it still bears a definiteness or indefiniteness feature depending on the context. Bare NPs in Russian (i.e., NPs without an overt article) are therefore analyzed as DPs (\citealt{Pereltsvaig2007}, \citealt{Geist2010}). The D-layer, although not realized by articles, serves as the semantic locus for definiteness, indefiniteness, and specificity. The type-shifting operators (\citealt{gb:Partee1987}) located here correspond to article semantics. We will not further elaborate on the DP level here. The present chapter focuses on the domain of the DP below D\textsuperscript{0}, namely \#P, DivP, and \textit{n}P. The \#P (also referred to as NumberP or QuantityP) is assumed to be the locus of the counting function/quantization. It encodes number properties (a feature [$\pm$pl]; cf. \citealt[307]{EmbickNoyer2007}). Numerals can be located in its specifier. The function of the Div(ision)P is to individuate nouns. In particular, Div\textsuperscript{0} performs division of the mass, denoted by the \textit{n}P \citep{Borer2005}. Below we will motivate the assumption of these structural layers. 

Roots like $\sqrt{dog}$ have been assumed to merely specify idiosyncratic aspects of the final semantic representation. The categorizing head \textit{n}\textsuperscript{0} is responsible for nominalization of uncategorized roots. In the influential analysis of noun phrases developed by \citet{Borer2005}, it is assumed that the basic interpretation of a nominalized root is a non-countable interpretation as mass. Crucial evidence for this assumption comes from the flexible behavior of nouns in different syntactic contexts. To explain the observed noun flexibility, Borer assumes that countability is not lexically encoded in the nominal root but is rather syntactically derived from the non-countable mass interpretation by combination with a type of classifier head Div that hosts grammatical markers of countability. 

In classifier languages such as Chinese, this function is performed by a numeral classifier. Chinese nouns are assumed to be mass nouns (\citealt{Chierchia1998}, \citealt{Krifka1995}). They cannot be directly combined with a numeral and counting needs to be mediated by a classifier. In languages such as English, countability is marked by number morphology. According to the analysis of basic noun denotations as mass, numeral classifiers and number morpho\-logy both indicate the presence of the classifier phrase, which relates to countable units (\citealt{Li2013}, \citealt{ChengSybesma1999}, \citealt{Chierchia1998}, among others for similar views, going back to \citealt{Greenberg1972}). As suggested by \citet{Kratzer2007}, in languages such as English, a noun that is usually categorized as a count noun can be conceived of as a noun with an incorporated classifier. Thus, in Chinese, a combination of a noun with a classifier such as \textit{zhi bi} behaves just like an English count noun in the plural, such as \textit{pencils}, in that they both indicate the presence of countable units:

\ea\label{ex:geist:2}
\ea
\gll san \minsp{[} zhi bi] \\ 
     three {} \textsc{cl} pen \\ 
\glt `three pens' \hfill (Chinese)
\label{ex:geist:2a}
\ex three pens \hfill (English) 
\label{ex:geist:2b}
\z\z

\newpage
\noindent The association of plural marking in English with the classifiers in Chinese argued for by \citet{Borer2005} is based on the assumption that the plural does not in fact mark quantity or a set of singulars, but rather the division of mass. Borer assumes that both numeral classifiers in classifier languages and the plural in English are exponents of the functional category Div\textsuperscript{0}. Div\textsuperscript{0} is an individualizer, the locus of a unit-marking function or atomization. 

However, the work on numeral classifiers in \citet{ChengSybesma2014} shows that in the Chinese languages Mandarin and Cantonese, two different types of numeral classifiers should be distinguished and only one type of classifier spells out Div\textsuperscript{0}. While in both Cantonese and Mandarin, the numeral classifier has to be added if the noun is combined with a numeral like in \REF{ex:geist:2}, in Cantonese, the classifier is also used with a noun if it occurs as a definite subject \REF{ex:geist:3a} or for reference to a kind \REF{ex:geist:3b}. As \citet{ChengSybesma2014} claim, in such contexts in Mandarin, a bare noun without a classifier is used instead; cf. \REF{ex:geist:4}.

\ea\label{ex:geist:3}
\ea
\gll bun\textsuperscript{2} / di\textsuperscript{1} syu\textsuperscript{1}  hai\textsuperscript{6}   ngo\textsuperscript{5}-ge\textsuperscript{3} \\
\textsc{cl}\textsuperscript{\textsc{sg}} {} \textsc{cl}\textsuperscript{\textsc{pl}} book be \textsc{1sg-poss} \\ 
\glt `The book/the books is/are mine.'
\label{ex:geist:3a}
\ex 
\gll di\textsuperscript{1} sai\textsuperscript{1}gwaa\textsuperscript{1} zau\textsuperscript{6} faai\textsuperscript{3} zyut\textsuperscript{6}zung\textsuperscript{2} laa\textsuperscript{3}\\
\textsc{cl}\textsuperscript{\textsc{pl}} watermelon \textsc{foc} soon extinct \textsc{sfp}\\
\glt `Watermelons will become extinct soon.’
%\ex \gll di\textsuperscript{1} wan\textsuperscript{4} zel-zyu\textsuperscript{6}  go\textsuperscript{3}taai\textsuperscript{3}yoeng\textsuperscript{4} \\
%\textsc{cl}\textsuperscript{\textsc{pl}} cloud block-\textsc{cont} {\textsc{cl} sun} \\
%\glt `The clouds are blocking out the sunlight.' 
\\ \hfill \citep[Cantonese;][253f.]{ChengSybesma2014}
\label{ex:geist:3b}
\z\z
  
\ea\label{ex:geist:4}
\ea
\gll shū shì wǒ-de \\
book be \textsc{1sg-poss} \\ 
\glt `The book/the books is/are mine.'
\label{ex:geist:4a}
\ex
\gll shīzi hěn kuài jiù huì juézhǒng \\
lion very quick then will be-extinct \\
\glt `Lions will be extinct very soon.' \\\hfill \citep[Mandarin;][252]{ChengSybesma2014}
\label{ex:geist:4b}
\z\z

\noindent Thus, the combination [Cl-N] in Cantonese and bare noun without a classifier in Mandarin have the same interpretation and distribution and may be assumed to be structurally the same. The classifier in Cantonese contributes individuation, while in Mandarin the noun can be individuated without a classifier. To account for this difference, \citet{ChengSybesma2014} assume that the classifier in Cantonese is merged low in the structure, right above the \textit{n}P in the Cl-u phrase (\textit{u} for ``unit-marking''). This explains the fact that the classifier is used in Cantonese in all contexts in which individuation or atomization plays a role. The Cl-u projection corresponds to DivP\textsuperscript{0} in Borer's system. The Mandarin classifier is different. Its function is not individuation or division but rather counting. \citeauthor{ChengSybesma2014} %Cheng and Sybesma 
assume that it is merged in another classifier phrase above Cl-u, in the Cl-c phrase, where \textit{c} stands for ``count-marking''. If the numeral occurs in the specifier of this projection, the head of the phrase needs to be phonologically realized and it is the count-marking classifier that spells out the head. The Cl-c phrase corresponds to \#P in Borer's system. Taking her system as a basis, we assume that the classifier in Mandarin realizes the \#-head, while the classifier in Cantonese spells out the Div-head, and in counting contexts, we assume, it may in addition spell out the \#-head. The fact that nouns in Mandarin may be individuated without a classifier can be explained by \textit{n}-to-Div movement. 

An additional argument for two classifier phrases comes from classifier languages in which the noun phrase can contain two classifiers. An example from Thai exemplifies this point. In \REF{ex:geist:5}, the classifier \textit{tua} appears first between the noun and the adjective, and in addition, the phonologically identical classifier also obligatorily occurs after the numeral.

\ea\label{ex:geist:5}
\gll maa tua yai song tua \\
dog \textsc{cl}	big two \textsc{cl} \\ 
\glt `two big dogs' \hfill \citep[Thai;][261]{ChengSybesma2014}
\z

\noindent The discussion of this example in \citet[261]{ChengSybesma2014} suggests that the first occurrence of the classifier is used for unithood, while the second one is for counting. 

As we have seen above, in English and Chinese, the same functional categories can be assumed to construct the noun phrase. But the content of functional categories can vary cross-linguistically (see \citealt{RitterWiltschko2009}). Thus, as we pointed out, for example, Div\textsuperscript{0} may be realized by a numeral classifier in Cantonese or by a classifying plural in English. And as we will see below, many other realizations of Div\textsuperscript{0} in other languages can be found. All language-specific exponents of Div\textsuperscript{0} have the same function -- to perform individuation. As far as the functional category \#\textsuperscript{0} is concerned, the literature mentions the counting plural and numeral classifiers as possible exponents. All these elements facilitate counting after individuation has taken place. 

In this work, we will elaborate on the issue of whether there are indications in Slavic languages which would justify the assumption of a fine-grained nominal structure below D as indicated in \figref{fig:1}. Do we have evidence for the classifier phrases DivP and \#P in such languages? We think that one can indeed answer this question positively. To do so we will further develop an idea in formal Slavic linguistics going back to the work of Ilse Zimmermann in the eighties (cf. \citealt{Zimmermann1983,gb:Zimmermann1988}). In this work, she argues for a richer syntactic inventory of categories, since such an enriched system allows for a better explanation of many syntactic, morphological, and semantic generalizations. More recent work on the DP-structure in Slavic languages, an overview of which is given in \citet{Geist2021}, shows that a more fine-grained syntactic structure has been put forward in this domain and additional functional layers below D can be substantiated in Slavic languages. In the present study we will discuss additional evidence for the fine-grained structure based on the plural marking in Russian and Polish. We will show that the plural comes in many guises, which justifies proposing different syntactic positions for its different semantic manifestations. We will refer to this observation as the ``distributed plural hypothesis.'' The goal of our study is to provide motivation for the distributed plural hypothesis in Polish and Russian. Our work can thus be regarded as a contribution to the broader cross-linguistic research on the fine-grained structure of nominal phrases. The arguments developed in this chapter can be taken as a guide for similar future research in other Slavic languages.

The remainder of the chapter is structured as follows. In \sectref{sec:geist:2}, we discuss the function of the plural as a divider, first giving a short general overview of the literature and facts (\sectref{sec:geist:2.1}) and then providing arguments based on linguistic material from Polish and Russian in support of this assumption (\sectref{sec:geist:2.2}). It is shown that the dividing plural may have a number-neutral reading ``one or more than one'' and that this reading is facilitated in non-referential contexts if the type of objects rather than the quantity of objects is at issue. \sectref{sec:geist:3} is devoted to the function of the plural as a counter. After a short general overview in \sectref{sec:geist:3.1}, in \sectref{sec:geist:3.2} singulatives in Russian and Polish -- their formation, function, and semantics -- are discussed in more detail. It is shown that, unlike the dividing plural, the plural formed from singulatives (a counting plural) has the meaning of an exclusive plural: it presupposes a set of particular objects and cannot refer to one object. These observations lead in \sectref{sec:geist:4.1} to the question of how many different plurals should be assumed in Polish and Russian. Our stance on this question is that in these languages plural number can be split across different syntactic projections. More precisely, we argue that, as proposed by \citet{Mathieu2014} for other languages, the heads \textit{n}\textsuperscript{0}, Div\textsuperscript{0}, and \#\textsuperscript{0} in the nominal spine in Polish and Russian are responsible for the different readings and distributions of the plural morphemes. To complete the overview of different functions of plurals, in \sectref{sec:geist:4.2} we discuss lexical plurals in general, and in \sectref{sec:geist:4.3} we focus on lexical plurals in Russian and Polish. It is shown that the plural of abundance, one type of lexical plural also available in Polish and Russian, displays idiosyncratic behavior. To account for these observations, we assume that the lexical plural formative is attached very low in the structure, in the process of the formation of the noun -- in \textit{n}\textsuperscript{0}. Finally, \sectref{sec:geist:5} concludes the chapter.

%-------------------------------------------------------

\section{Plural as a divider}\label{sec:geist:2}
\subsection{Plural does not always mean `more than one'}\label{sec:geist:2.1}

According to the traditional view on number, the singular refers to one entity, while the plural refers to more than one. However, a more complex view on the semantics of the plural has been emerging (\citealt{Krifka1989}, among others). It has been observed that in negative sentences, and conditional constructions, plural nouns in English are typically understood as referring not to a group of two or more individuals (exclusive plural interpretation, dominant in declarative affirmative sentences) but to any number of individuals, one or many, as long as it is not zero (number-neutral or inclusive plural interpretation). Consider \REF{ex:geist:6}--\REF{ex:geist:7}.

\ea\label{ex:geist:6}
Max didn't visit his relatives.\\
{[}Sentence is false if Max visited even a single relative.{]}
\z

\ea\label{ex:geist:7}
If you see any dogs, let me know.\\
{[}The speaker wants to be notified if at least one dog was seen.{]}
\z

\noindent Semantic work on the plural (\citealt{Sauerland2003}, \citealt{SauerlandYatsushiro2005}, \citealt{Spector2007}, \citealt{Zweig2009}, \citealt{BaleKhanjian2011}) explains the preferences for the inclusive or exclusive interpretation of the plural by the type of context: downward entailing environments, such as those above, facilitate the inclusive interpretation of the plural whereas in upward entailing contexts, the plural is interpreted exclusively.\footnote{Monotonicity is a logical property related to the direction of inferences associated with a given construction. A predicate which allows inferences from a subset to a superset is known as upward monotone, whereas a predicate allowing inferences in the opposite direction is called downward monotone. It should be noticed, however, that the monotonicity of interrogative sentences is problematic. For a discussion, see \citet{Giannakidou1998}, \citet{Gutierrez-Rexach1997}, \citet{Progovac1993}, or \citet{vanderWouden1997}. Similarly, the downward monotonic properties of conditionals have been questioned in the literature (see, among others, \citealt{Gajewski2011}, \citealt{Heim1984}, and \citealt{vonFintel1999}). For a general recent discussion, see \citet{GulgowskiBlaszczak2020}.} \citet{Grimm2013} shows that inclusive interpretation of the plural arises not only in downward entailing contexts, but can also occur in polar questions with \textit{have}, modal contexts, and also with objects of opaque verbs such as \textit{look for}:

\ea\label{ex:geist:8}
Do you have children? \\
{[}It can be answered ``yes'' truthfully even if the addressee of the question has just one child.{]}
\z
	
\ea\label{ex:geist:9}
I am looking for houses. \hfill \citep[249]{Grimm2013} \\
{[}If the speaker finds one house that is suitable, she/he will look no further.{]}
\z

\noindent According to \citet{Grimm2013}, the contrast between the inclusive and exclusive readings does not coincide with a contrast between upwards or downwards entailing environments but rather is built on whether the noun is interpreted as referring to particular objects, where quantity is relevant, or referring more generally to the type of object, where quantity is irrelevant. Contexts such as questions, negation, conditionals, and modal contexts allow for weak referential readings of nouns, where no reference to any particular quantity of objects is made. The noun may be non-specific and may not presuppose the existence of any particular referents. Following Grimm, we will refer to this reading of plural NPs as an \textsc{instance of a kind} reading to distinguish it from a \textsc{quantified object} reading, which we will characterize below. Grimm assumes that an instance of a kind reading is a type of generic reading. 

According to \citet{KrifkaTerMeulen1995}, two types of generic reference have to be distinguished: (i) kind reference:\footnote{\citet{Krifka1995} assumes that kinds are a subclass of a more general class of concepts: not all concepts need to be well-established, while kinds are assumed to be. In this study we will use the term \textit{kind} in a broader sense to also refer to not well-established kinds for simplicity.} generic NPs as arguments of kind-level predicates like \textit{be extinct, invent}, and \textit{be first cultivated} refer to kinds rather than to objects, cf. \textit{Dodos are extinct}; (ii) characterizing predication: generic sentences express some characterizing property of instances of a kind and may occur, for example, as subjects of predicates like \textit{contain} or as objects of predicates such as \textit{like} (\citealt[71f]{KrifkaTerMeulen1995}), cf. \REF{ex:geist:10}, or denotations of habitual activities as shown in \REF{ex:geist:11}. Different analyses of such NPs have been suggested in the literature (cf. \citealt{gb:Krifka2004} for an overview). Most of them assume that the generic (bare plural) NP introduces a variable that can be bound by a generic operator, which may be explicated by adverbs like \textit{typically} or \textit{usually}.

\ea\label{ex:geist:10}
\ea Peas (typically) contain a lot of fiber.
\label{ex:geist:10a}
\ex Paula (generally) likes novels. 
\label{ex:geist:10b}
\z\z

\ea\label{ex:geist:11}
Hella (usually) eats apples for breakfast.
\z 

\noindent However, as Grimm claims, generic reference may also arise in questions with existential \textit{have}, where no generic quantification is available. Following the assumption that inclusive or number-neutral interpretation of the plural is another type of generic interpretation, Grimm suggests an analysis of generic NPs in such contexts similar to the analysis of generic NPs in characterizing predication: such NPs introduce a variable over instances of a kind, but in addition, in their representation, there is an existentially bound variable for number, which gives rise to number-neutral inclusive interpretation.\footnote{\label{fn:InstanceKind} The formula in (i) represents the meaning of the bare plural \textit{dogs} in its inclusive interpretation. In the representation, the relation \cnst{r} is a realization relation between \textit{kind} and the instances of the kind at the level of objects. \cnst{ou} is an operator which, given a kind and a set of objects, provides a measure, $n$, of the number elements which qualify as instances of the kind. In questions with \textit{have}, modal contexts, and with opaque verbs like \textit{look for} the variable for instances of a kind may be bound by existential closure. 

\ea Instance of a kind reading (inclusive interpretation of plural):\\
\sib{dogs}: $\lambda i \, \lambda x \, \exists n \,[\cnst{r}_i (x, \textsc{dog}) \wedge \cnst{ou}_i (\textsc{dog})(x)=n]$ \hfill \citep[7]{Grimm2013}
\z}

Note, however, that even in questions the plural can receive an exclusive interpretation (`more than one') (see also footnote 1). Thus contexts facilitating the non-referential use of nouns do not automatically exclude the referential use of nouns and exclusive interpretation of the plural. In a context like \REF{ex:geist:12} from \citet{Grimm2013}, where the quantity of objects is made relevant, the exclusive reading in the question is even preferred.

\eanoraggedright\label{ex:geist:12} \textit{Scenario}: Several colleagues are looking for a meeting place. B has an office in the building:
\begin{xlist}
\exi{A:}{Do you have chairs in your office?}\label{ex:geist:12A:}
\exi{B:}{?Yes, one. / No, only one. \hfill \citep[9]{Grimm2013} \\ 
{[}The question cannot be answered ``yes'' truthfully if the addressee of the question has just one chair.{]} }\label{ex:geist:12B:}
\end{xlist}
\z
	
\noindent In this context the quantity of objects is relevant and the plural noun \textit{chairs} yields an exclusive reading. That the exclusive reading is not intrinsic to the plural of \textit{chairs} is shown in \REF{ex:geist:13}, where a question about chairs is used in the context of a furniture store.


\eanoraggedright\label{ex:geist:13} \textit{Scenario}: A person is looking for a blue chair for her bedroom. She goes to the small 	furniture store on the corner and asks the salesman:
\begin{xlist}
\exi{A:}{Do you have blue chairs?}\label{ex:geist:13A:}
\exi{B:}{Yes, one. / ?No, only one. \hfill \citep[9]{Grimm2013}}\label{ex:geist:13B:}
\end{xlist}
\z

\noindent In this case \textit{chairs} is weakly referential since the question under discussion is the availability of objects exemplifying the kind ``blue chair'' in the store. The cardinality of the objects is not at issue.

To conclude, following \citet{Grimm2013}, it can be assumed that plural nouns in contexts allowing weak reference as in \REF{ex:geist:6}--\REF{ex:geist:9} are in principle ambiguous between the instance of a kind reading and the quantified object reading. In the former reading, the plural is not interpreted as indicating a sum of particular objects, i.e., it does not have a counting function, but rather designates an object as a type of thing.

\citet{Borer2005} takes an extreme view that the plural, at least in English, is always used to divide or portion out undivided mass rather than to count objects. That the function of number is not always counting but sometimes just dividing was originally observed by \citet{Krifka1989}. Consider, for example, \REF{ex:geist:14}, in which the noun \textit{apple} has a plural form although there is only one apple involved in each case except for \REF{ex:geist:14c}.

\ea\label{ex:geist:14}
\ea 0.2 apples/*apple
\label{ex:geist:14a}
\ex 0.1 apples/*apple
\label{ex:geist:14b}
\ex 1.5 apples/*apple
\label{ex:geist:14c}
\ex 1.0 apples/*apple \hfill \citep[115]{Borer2005} 
\label{ex:geist:14d}
\z\z

\noindent Data like these point to the fact that plural does not need to refer to a group of objects, thus it serves as a divider rather than a counter. \citet{Borer2005} assumes that the plural with a dividing function in English is an exponent of the functional category Div\textsuperscript{0}. 

The assumption that the plural in English is always a divider and never a counter has been questioned in theoretical and empirical studies \citep{Alexiadou2019,Grimm2013}. These studies indicate that the plural is in principle ambiguous, and depending on the grammatical context and the discourse context may have a counting or dividing interpretation. In the neo-constructionist approach, which we advocate here, these different interpretations can be captured in terms of different attachment points in the syntactic tree. Plural merged under Div corresponds to what the semantic literature discussed above analyzes as an inclusive plural: such a plural indicates a generic instance-of-a-kind reading, where quantity is not relevant. As we will show in \sectref{sec:geist:3}, a counting plural that highlights the quantity of objects and has an exclusive reading must be realized higher in the structure, in the \#P.

In the next section, we turn to Polish and Russian and show that one kind of plural in these languages behaves very much like the English dividing plural. In \sectref{sec:geist:3}, we will compare this kind of plural with the plural of singulatives and show that it is a counting plural.

%---------------------

\subsection{Plural as a divider in Polish (and Russian)}\label{sec:geist:2.2}

In this subsection we will analyze the semantic and distributional properties of plural morphemes in Polish and Russian. What morphemes can serve to express the plural in these languages? 

In Polish, the most common plural endings are \textit{-y} (as in \textit{samochód}.\textsc{sg.m} `car' -- \textit{samochod-y}.\textsc{nv.pl} `cars', \textit{kobiet-a}.\textsc{sg.f} `woman' -- \textit{kobiet-y}.\textsc{nv.pl} `women') and \textit{-i} (as in \textit{ptak}.\textsc{sg.m} `bird' -- \textit{ptak-i}.\textsc{nv.pl} `birds', \textit{studentk-a}.\textsc{sg.f} `student (female)' -- \textit{studentk-i}.\textsc{nv.pl} `students (female)'). However, the plural form of some nouns is formed by adding other suffixes, for example, the suffix \textit{-e} as in \textit{przyjaciel}.\textsc{sg.m} `friend' -- \textit{przyjaciel-e}.\textsc{v.pl} `friends' or the suffix \textit{-a} as in \textit{okn-o}.\textsc{sg.n} `window' -- \textit{okn-a}.\textsc{nv.pl} `windows' (for a general synchronic overview of the different plural forms in Polish, see \citealt{Swan2002}; for the diachronic emergence, see \citealt{KlemensiewiczStanislaw1955} and \citealt{Rospond1979}).

In Russian, the plural morpheme added to the stem is \textit{-i} for stems ending in a ``weak'' (=palatalized) consonant and \textit{-y} for stems ending in a ``hard'' (=not palatalized) consonant. However, there are many exceptions to this general rule: some masculine and most neutral nouns combine with the plural suffix \textit{-a/-ja} (\textit{okn-ó}.\textsc{sg.n} `window' --  \textit{ókn-a}.\textsc{pl} `windows', \textit{učitel'}.\textsc{sg.m} `teacher' --  \textit{učitel-já}.\textsc{pl} `teachers') and masculine nouns ending in \textit{-an} have the plural suffix \textit{-e} (\textit{gražd-an-nin}.\textsc{sg.m} `citizen' -- \textit{gražd-an-e}.\textsc{pl} `citizens'); cf., e.g., for a synchronic overview \citet[169ff]{SvedovaV1990}, and for the diachronic emergence, see \citet{Miklosich1868}, \citet{Vondrak1928}, \citet{Kiparsky1963}, and 
\citet{Isacenko1962,Issatschenko1980, Issatschenko1983}.  

In this section we will explore whether the plural in Slavic languages may function as a divider. Since Russian appears to be very similar to Polish, in this section we will present only examples from Polish. 

First, as argued in \citet{Grimm2013}, in contexts in which the quantity of objects is not under discussion, the plural is preferably interpreted as an inclusive plural (`one or more'). As already explained above, this is so because in such cases the plural does not refer to particular entities, where the quantity would be relevant, but rather it refers to the kind of things designated by the noun. In the latter case, the quantity is not relevant (see \citealt[\S 1]{Grimm2013}). The examples in \REF{ex:geist:15}--\REF{ex:geist:17}, modeled on similar examples discussed by \citet{Grimm2013}, provide illustration of such contexts (see above \sectref{sec:geist:2.1}).

\eanoraggedright
\begin{xlist}
\exi{A:}{\gll Masz dzieci?\\
have.\textsc{2sg} children.\textsc{acc}\\  
\glt `Do you have children?'
\label{ex:geist:15A:}}
\exi{B:}{\gll Tak, mam jedno. / \minsp{\#} Nie, mam jedno. \\
yes have.\textsc{1sg} one {} {} no have.\textsc{1sg} one \\
\glt `Yes, I have one. / \#No, I have one.' \hfill (Polish)
\label{ex:geist:15B:}}
\end{xlist}\label{ex:geist:15}
\z

\noindent As B shows, the question in A can be answered ``yes'' truthfully even if the addressee of the question has just one child, which demonstrates that the plural noun is indeed compatible with an interpretation including just a single entity; see \citep{Grimm2013}.

\ea\label{ex:geist:16}
\gll Piotr nie widzial \minsp{(} żadnych) dzieci. \\ 
Peter \textsc{neg} saw {} no children.\textsc{gen} \\ 
\glt `Peter didn't see any children.' \hfill (Polish)\\
\z

\noindent Sentence \REF{ex:geist:16} is true even if it was only one child that Peter didn't see.

\ea\label{ex:geist:17}
\gll Jeśli państwo mają dzieci, podnieś rekę. \\ 
if you.\textsc{3pl} have.\textsc{3pl} children.\textsc{acc} raise hand.\textsc{acc} \\ 
\glt `If you have children, please raise your hand.' \hfill (Polish)\\
{[}In the context in question, persons with just one child would also be required to raise their hands.{]}
\z

\noindent Second, plural nouns can occur in characterizing sentences. This suggests that plurals in Polish may receive a generic reading.   

\ea\label{ex:geist:18}
\ea
\gll Jabłka zawierają witaminę C. \\
apples.\textsc{nv.pl.nom} contain vitamin C \\ 
\glt `Apples contain vitamin C.'\label{ex:geist:18a}
\ex 
\gll Anna uwielbia słowniki. \\
Anna loves dictionaries.\textsc{nv.pl.acc} \\
\glt `Anna loves dictionaries.'\label{ex:geist:18b}
\hfill (Polish)\z\z

\noindent Third, it can be shown that in some contexts the plural does not refer to a group consisting of more than one object, but rather serves as a type of classifier that, like numeral classifiers in Cantonese, just indicates the availability of discrete units for counting. In \REF{ex:geist:19} and \REF{ex:geist:20} it is shown that numeral \textit{zero} and fractions in Polish are also combined with nouns in the plural.\footnote{As an alternative to \REF{ex:geist:20a}, some speakers of Polish also allow the singular marking with 5.5, as shown in \REF{ex:geist:fn4}:
\ea\gll pięć i pół stołu \\
five and half table.\textsc{nv.sg.gen} \\
\glt `5.5 table' \label{ex:geist:fn4}
\hfill (Polish) \z
In this case, we assume that the noun has a generic interpretation; see also \citet{KwapiszewskiFuellenbach2021}.} In this case, the plural noun cannot denote a sum of objects, but rather a type.

\ea\label{ex:geist:19}
\ea\gll zero stołów \\
zero table.\textsc{nv.pl.gen} \\
\glt `zero tables' \label{ex:geist:19a}
\ex\gll zero chłopców \\
zero boy.\textsc{v.pl.gen} \\
\glt `zero boys' \label{ex:geist:19b}
\hfill (Polish)\z\z

\ea\label{ex:geist:20}
\ea\gll pięć i pół stołów \\
five and half table.\textsc{nv.pl.gen} \\
\glt `5.5 tables' \label{ex:geist:20a}
\ex\gll pięć i pół chłopców \\
five and half boy.\textsc{v.pl.gen} \\
\glt `5.5 boys' \label{ex:geist:20b}
\hfill (Polish)\z\z

\noindent To conclude, the above facts can be taken to indicate that in Polish (and Russian, which shows very similar behavior), the plural may serve as a divider: in this case it triggers a generic reading of the noun and facilitates an inclusive interpretation in non-referential contexts. This reading is facilitated in non-referential contexts if the type of objects rather than the quantity of objects is at issue.

%-------------------------------------------------------

\section{Plural as a counter}\label{sec:geist:3}

\subsection{Previous observations}\label{sec:geist:3.1}

In the previous section we have already mentioned that along with the dividing plural there must be at least one other type of plural, the counting plural. Plural morphemes attached to a count noun are in principle ambiguous between these two types of plural. In this section, we will introduce the idea of \citet{Mathieu2012,Mathieu2014} that some nouns can only be combined with a counting plural. In contrast to the dividing plural, nouns combined with a counting plural refer to a group of atoms and the plural can only have an exclusive reading. The counting plural is located in the head of \#P above DivP. \citet{Mathieu2012,Mathieu2014} argues that this special type of noun which can only combine with a counting plural but not with a dividing plural is a so-called singulative noun or singulative, i.e., a noun derived with a singulative suffix in languages that have a productive singulative formation.   

Languages such as Celtic, Semitic, Gur (Niger-Congo), and some American aboriginal languages use suffixes, mainly diminutive suffixes, to derive a denotation of singular unit from a noun denoting mass or collection. Because of their function, such suffixes have been called ``singulative suffixes.'' The typological literature points to incomplete productivity of singulativization \citep{Stolz2001}, which suggests that this process falls somewhere between inflection and derivation.  

Singulative formation is often accompanied by gender shift. The examples in \REF{ex:geist:21} from Breton, a Celtic language, exemplify this fact. The input to singulative formation in this language is collective nouns. Such nouns are morphosyntactically singular but have the semantics of plurals since they denote groups of objects viewed as a totality (see \citealt{Acquaviva2015}). They behave like mass nouns in that they cannot combine with numerals. As shown in \REF{ex:geist:21}, some genuine mass nouns, substance denotations, may also undergo singulative formation. The feminine suffix \textit{-enn} is added to divide the collection or mass. The output is a singular noun in feminine gender meaning a unit of counting/measuring.

\ea\label{ex:geist:21}
\ea\gll buzhug $\sim$ buzhug-enn \\
`worms' {} {`a worm'} \\ \hfill (Breton) 
\label{ex:geist:21a}
\ex\gll kraon $\sim$ kraon-enn \\
`walnut' {} {`a walnut'} \\ 
\glt \hfill (\citealt{Stump2005}, cited in \citealt{Mathieu2012}) 
\label{ex:geist:21b}
\z\z

\ea\label{ex:geist:22}
\ea\gll dour $\sim$ dour-enn \\
`water' {} {`drop of water'} \\ \hfill (Breton)
\label{ex:geist:22a}
\ex\gll glav $\sim$ glav-enn \\
`rain' {} {`drop of rain'} \\
\glt \hfill (\citealt{Stump2005}, cited in \citealt{Mathieu2012}) 
\label{ex:geist:22b}
\z\z

\noindent Since the singulative turns the noun denoting an aggregate or substance into a denotation of a unit, its contribution is similar to that of numeral classifiers in Cantonese. The observation that the singulative may perform the same function as classifiers goes back to \citet[26]{Greenberg1972}. Building on this observation, \citet{Mathieu2012,Mathieu2014} adds the singulative system to the theory of division. He argues that since singulative morphemes like classifiers perform division, both can be assumed to spell out the Div\textsuperscript{0} head. 

The fact that is crucial for the topic of this chapter, namely the issue of the plural, is that the singulative noun in the singular always refers to atoms and can never refer to sums. However, the singulative noun may be pluralized. The product of pluralization always denotes a sum of individuals and has an exclusive interpretation `more than one.' As Mathieu points out in \REF{ex:geist:23} from Ojibwe, an Algonkian language, the singulative noun in the plural is not neutral enough to be used in a question that asks about the existence or availability of some kind of objects, as is intended in \REF{ex:geist:23}. The question with a singulative noun in the plural rather asks about a quantity of objects and is inappropriate in such a question.

\ea{?*}{\gll {hai indik} burtogaalaat? \\
have.you oranges.\textsc{pl.f} \\ \hfill (Ojibwe)
\glt `Do you have oranges?' \hfill \citep[170]{Mathieu2014}}
\label{ex:geist:23}
\z
	   
\noindent To account for the meaning and distribution of the plural of singulatives, Mathieu assumes that this plural is not realized under Div. First, the Div\textsuperscript{0} is already realized by the singulative suffix. If the plural were always a divider realizing Div\textsuperscript{0}, it would be in complementary distribution with the singulative suffix. The fact that this plural and the singulative morpheme can co-occur suggests that they cannot simultaneously fill the same position in the structure. To account for the co-occurrence of the plural morpheme and the singulative morpheme and for the exclusive interpretation of the plural, Mathieu assumes that this plural is the realization of the \# head above DivP and it is a counting plural.\footnote{The semantic representation for nouns with the counting plural is given in \REF{ex:geist:QuantObj}; see footnote \ref{fn:InstanceKind} for the instance of a kind reading of bare plurals for comparison. The variable for number $n$ must be specified as being greater than 2.

\ea \textit{Quantified object reading {(}exclusive interpretation of plural{)}:}\\
\sib{dogs}: $\lambda i \, \lambda x \, \lambda n \, [ \cnst{r}_i (x, \textsc{dog}) \wedge \cnst{ou}_i (\textsc{dog})(x) = n ]$  \hfill \citep[7]{Grimm2013} 
\label{ex:geist:QuantObj}
\z

\noindent According to this representation, the bare plural \textit{dogs} designates particulars: the indefinite refers to a quantity of particular objects.} 

To conclude, contrary to \citet{Borer2005}, who claims that the sole function of the plural is division, or to \citet{Alexiadou2019}, who claims that the plural is always underspecified between the inclusive reading associated with dividing and the exclusive reading associated with counting, \citet{Mathieu2014} puts forth the hypothesis that the plural combined with a singulative noun in some languages unambiguously serves as a counter. The plural of singulative nouns is unambiguously exclusive, while the plural of underived simple nouns can be inclusive or exclusive depending on the grammatical context and discourse situation. The counting plural is realized in \#P, while the dividing plural is located in Div\textsuperscript{0}. 

% -------------------------------------------------------

\subsection{Singulative in Russian and Polish}\label{sec:geist:3.2}

Russian is a language with a relatively productive singulative formation (see \citealt{GeistInprep}). It has series of collective nouns denoting aggregates but also a few mass nouns that can be turned into individuals or portions via the use of a singulative morpheme.\footnote{We will focus on collective non-human nouns as the input for singulativization and will not discuss the derivation of singulatives from nouns denoting mass (\textit{krov'} `blood' \textit{krov-ink-a} `a drop of blood') and from human collective nouns.} Singulative morphemes include the original singulative suffix \textit{-in-}, the diminutive suffix \textit{-k-}, the combination of these two, as well as some less productive suffixes that can have a singulative function: \textit{-enk-}, \textit{-ičk-}, \textit{-ovic-}, and \textit{-šk-}. Singulative suffixes may target collective mass nouns denoting berries: \textit{klubnika} -- \textit{klubnič-in-a} `strawberries -- a strawberry', grains: \textit{ris} -- \textit{ris-in-a} `rice -- a grain of rice', precipitation: \textit{sneg -- snež-ink-a} `snow -- a snowflake', dried fruits: \textit{izjum -- izjum-in-a} `raisins -- a raisin', beans: \textit{fasol' -- fasol-in-a} `beans -- a bean', and granular aggregates: \textit{pyl' -- pyl-ink-a} `dust -- a speck of dust', but also human nouns denoting social and ethnic groups: \textit{graždan-e -- graždan-in} `citizens --  a citizen'. We will focus here on non-human collective mass nouns. With such nouns, singulativization is accompanied by a gender shift to the feminine form.  

\ea\label{ex:geist:24} \hspace{.5cm} collection/mass \phantom{I} singulative \hfill (Russian)
\ea\label{ex:geist:24a}
\gll  ris \phantom{XXXXi} ris-in-a \\
rice.\textsc{m.sg} {} rice-\textsc{sing-f.sg} \\ 
\glt `rice' \phantom{XXXXXXXi} `a grain of rice'
 \ex\label{ex:geist:24b}
\gll višnja \phantom{XXXi} višen-k-a \\
cherry.\textsc{f.sg} {} cherry-\textsc{sing-f.sg} \\
\glt `cherry fruit' \phantom{XXi} `a cherry'
\z\z

\noindent There is evidence for an analysis of the singulative morpheme in Russian as an exponent of Div\textsuperscript{0} and against analyzing it as the realization of the lower head \textit{n}\textsuperscript{0}. In the neo-constructionist accounts we follow here, \textit{n}Ps are uncountable and have abstract denotation, i.e., they denote a kind or a mass. As a diagnostic for such an abstract denotation, the occurrence of a nominal as an argument of kind-level predicates such as \textit{be extinct} or \textit{be first cultivated} has been used. Example \REF{ex:geist:25} shows that singulative nouns cannot serve as the subject of such predicates.

\ea{\#}{\gll Risinka byla vpervye kultivirovana v Tailande. \\
rice.\textsc{sing.f} was first cultivated in Thailand \\ \hfill (Russian)
\glt lit.: `A grain of rice was first cultivated in Thailand.'}
\label{ex:geist:25}
\z

\noindent That singulatives in Russian can only have an object-level interpretation and never a kind-level interpretation is also argued in \citet[329]{Trugman2013}. She claims that noun-adjective combinations with postponed adjectives like \textit{goroch posevnoj} `seed peas' can be used as names of kinds. However, as she points out, singulative nouns cannot combine with postponed adjectives since they are not names of kinds. 

\begin{table}
\centering
\begin{tabular}{lll} 
\lsptoprule
Collective & \multicolumn{2}{l}{Singulative in \textsc{sg}} \\
\midrule
goroch posevnoj & \#gorošina & posevnaja \\ 
pea.\textsc{sg} seed & \phantom{\#}pea.\textsc{sing.sg} & seed \\ 
`seed peas' & \multicolumn{2}{l}{`a seed pea'} \\ 
\lspbottomrule
\end{tabular}
\caption{Postponed adjectives as indicators of kind-reference in Russian}
\label{tab:postponed_Rus}
\end{table}

It has been assumed that modification of a kind with an adjective is done at a very low syntactic level. In the syntactic system assumed in this study, kind-level modification with postponed adjectives should apply at the \textit{n}P level, the lowest nominal syntactic level. Since this type of modification with singulative nouns is excluded, they cannot be analyzed as \textit{n}Ps. The unitizing semantics of the singulative morpheme in Russian can be captured if we assume that it is a realization of Div\textsuperscript{0}, and the singulative noun is a DivP. A kind-level modification cannot apply at this level.\footnote{\citet{ErschlerGeistKagan} claim that the singulative morpheme \textit{-in-} realizes not just the head Div\textsuperscript{0}, but rather the fused head Div\textsuperscript{0}/\textit{n}\textsuperscript{0}. To support this claim, they show that (i) the singulative \textit{-in-} does not allow nominalizing suffixes between the root and itself, and (ii) the singulative \textit{-in-} is in complementary distribution with other nominalizing suffixes realizing \textit{n}\textsuperscript{0}. For reasons of space, we cannot present these arguments in detail here and for simplicity will just assume that \textit{-in-} realizes Div\textsuperscript{0}.}

Singulative nouns in the singular denote one unit, but they may be pluralized, as shown in \REF{ex:geist:26}. They then denote a sum of units.

\ea\label{ex:geist:26}
plural of singulatives from non-human nouns \hfill (Russian)
\ea\label{ex:geist:26a}
\gll ris-in-y \\
rice-\textsc{sing-pl} \\
\glt `grains of rice'
\ex\label{ex:geist:26b}
\gll višen-k-i \\
cherry-\textsc{sing-pl} \\
\glt `cherries'
\z\z

\noindent Such plurals differ from the plurals of underived nouns that we considered in the previous section. First, pluralized singulatives cannot receive a generic interpretation in characterizing sentences like \REF{ex:geist:27}:

\ea{\#}{\gll Risiny obyčno soderžat mnogo mineral'nych veščestv. \\ 
rice.\textsc{sing.pl} usually contain much mineral substances \\ \hfill (Russian)
\glt lit.: `Grains of rice contain a lot of natural minerals.'}
\label{ex:geist:27}
\z

\noindent Second, in weak referential contexts like questions, pluralized singulatives cannot receive an instance of a kind reading but rather get a quantified object reading. In a context that makes a quantity interpretation relevant, as in \REF{ex:geist:28}, plural singulatives refer to a group of particular objects, i.e., they have an exclusive rather than inclusive reading.
\eanoraggedright\label{ex:geist:28} \textit{Scenario}: Ann needs some cherries to decorate a cake. Unfortunately, she doesn't have any. She asks her neighbor:
\begin{xlist}
\exi{A:}{\gll U tebja est' višenki ukrasit' tort?\\
at you exist cherry.\textsc{sing.pl} decorate.\textsc{inf} cake\\
\glt `Do you have cherries to decorate a cake?'}\label{ex:geist:28A:}
\exi{B:}{\gll Net, u menja est' tol'ko odna višenka. / \minsp{\#} Da, u menja est' odna višenka. \\ 
no at me exist only one cherry.\textsc{sing.sg} {} {} yes at me exist one cherry.\textsc{sing.sg}\\
\glt `No, I have only one.' / \textit{Not}: `Yes, I have one.'\hfill (Russian)}\label{ex:geist:28B:}
\end{xlist}
\z

\noindent The question with a pluralized singulative asks about the existence of a plurality of particular cherries. Since only one cherry is available, the affirmative answer is dispreferred. By contrast, in the context created in \REF{ex:geist:29} the quantity of cherries is irrelevant. The speaker is asking about the availability of some underspecified amount of the substance.

\eanoraggedright\label{ex:geist:29} \textit{Scenario}: Ann likes cherries and would like to buy about a kilo of cherries at the weekly market. She goes there and asks a saleslady:
\begin{xlist}
\exi{A:}{\gll U vas est' \minsp{\#} višenki / višnja?\\
at you exist {} cherry.\textsc{sing.pl} {} cherry\\
\glt `Do you have cherries?'\hfill (Russian)}
\end{xlist}
\z

\noindent As expected, the singulative noun in the plural is not appropriate in such a context, and the noun \textit{višnja} denoting a mass/collection would be used instead. 

\largerpage
Now we turn to Polish. The singulative formation in Polish is not as productive as in Russian (see \citealt{Szymanek2014} and the references cited therein) -- compounds are used more productively to denote one unit; cf. \textit{kiełek pszenicy} `grain of wheat' \citep[863]{GunkelZifonun2017}. A few examples which could be regarded as examples of singulatives in Polish are given in \REF{ex:geist:30}. All such examples are derived from collective or mass nouns by means of the otherwise diminutive suffixes \textit{-ek/-ka/ -ko} and \textit{-ik}.\footnote{\citet{Szymanek2014} discusses yet another suffix, \textit{-in}, which according to some scholars (e.g., \citealt{Wierzbicka2002}) also expresses singulative meaning in Polish. Polish \textit{-in} has the same origin as the Russian singulative suffix \textit{-in}, however in Polish this suffix is very restricted. It can be used to derive a singular form from the plural nouns designating social and ethnic groups. Some scholars, e.g., \citet{GrzegorczykowaWrobel1999}, analyze \textit{-in} as an inflectional rather than derivational suffix (for more details, see \citealt{GrzegorczykowaWrobel1999} and \citealt{Orzechowska1999}; for a critical discussion, see \citealt{Szymanek2014}).
\ea 
\ea	Amerykanie `Americans.\textsc{pl}' \rightarrow \ Amerykanin `an American' 
\ex mieszczanie `burghers.\textsc{pl}' \rightarrow \ mieszczanin `a burgher' 
\z\z

\noindent Such formations with \textit{-in} are also available in Russian. Although the behavior of \textit{-in} here is similar to that of the singulative \textit{-in} discussed above for Russian, such forms do not pattern with singulatives derived from non-human nouns: they cannot be pluralized in the same way. This ban on pluralization and the restriction to bases that refer to social and ethnic groups suggest that such an \textit{-in} suffix requires a different treatment.} As \citet{Szymanek2014} points out, the singulative meaning in Polish usually co-occurs with the diminutive function `small N.' In this regard singulatives can be analyzed in terms of semantic extension of the diminutive, as is, for example, proposed by \citet{Jurafsky1996}\footnote{More precisely, as \citet{Szymanek2014} points out, \citet{Jurafsky1996} proposes a structured polysemy model which is based on the notion of a radial category.} (see Szymanek for a critical discussion). In this sense, the singulative meaning in Polish arises as ``a by-product of diminutivization'' \citep{Szymanek2014}.\footnote{However, this does not mean that all diminutives will also have a singulative or individuating interpretation. For example, as \citet{Szymanek2014} points out, the diminutive suffix \textit{-ek} added to the mass noun \textit{żwir} `gravel, pebbles' does not result in the singulative meaning of `a single pebble' but rather it just encodes the diminutive sense of smallness (`small pebbles'). In this sense, both forms, \textit{żwir} `gravel, pebbles' and \textit{żwirek} `small pebbles', are mass nouns (see \citealt{Szymanek2014}.)}

The derived form is in principle ambiguous between the singulative and di\-min\-u\-tive readings. Under the singulative reading the form denotes just one unit of the mass or collection denoted by the noun. Importantly, under this reading the derived form can be pluralized, just as the singulative form in Russian. However, the singulative forms in Polish still differ from their Russian counterparts: the singulative formation in Polish is not accompanied by a gender shift. The derived form retains the original gender of the base noun. This could be taken to mean that although the singulative suffix is transparent to the gender of the base, it determines the inflection endings and therefore it can be analyzed as a head rather than a modifier (cf. \citealt{WiltschkoSteriopolo2008} for the head-modifier distinction between diminutive suffixes in some languages). 

\ea\label{ex:geist:30}
\hspace{.5cm} collection/mass \phantom{i} singulative \hfill (Polish)
\ea 
\gll groch \phantom{XXXXi} grosz-ek \\
pea.\textsc{sg.m} {} pea-\textsc{sing.sg.m} \\
 \glt `peas' \phantom{XXXXXXi} `a pea'\label{ex:geist:30a}
\ex
\gll pył \phantom{XXXX} pył-ek \\
dust.\textsc{sg.m} {} dust-\textsc{sing.sg.m}  \\
\glt `dust' \phantom{XXXXXXi} `a speck of dust'\label{ex:geist:30b}
\ex
\gll czekolad-a \phantom{X} czekolad-k-a \\
       chocolate-\textsc{\textsc{sg.f}} {} chocolate-\textsc{sing-\textsc{sg.f}} \\
 \glt `chocolate' \phantom{XXXi} `a chocolate'\label{ex:geist:30c}
\z\z 

\noindent Similarly to Russian, evidence from modification of singulative nouns in Polish can be used to indicate that they cannot refer to kinds and hence cannot be analyzed as \textit{n}Ps, which are kind-denoting. According to \citet{Wagiel2014}, nouns modified by prenominal adjectives can only receive an object-level interpretation, while nouns modified by postnominal adjectives are systematically ambiguous between kind-level and object-level readings. Applied to singulative nouns, the following pattern emerges: if combined with a singulative noun, only an object-level interpretation is possible, while the kind-level interpretation is inappropriate; see \tabref{tab:postponed_Pol}.  

\begin{table}
\centering
\begin{tabular}{lll} 
\lsptoprule
Collective & \multicolumn{2}{l}{Singulative in \textsc{sg}} \\
\midrule
groch włoski & \#groszek & włoski \\ 
pea.\textsc{sg} Italian & \phantom{\#}pea.\textsc{sing.sg} & Italian \\ 
`chick pea' & \multicolumn{2}{l}{`a chick pea' (under kind-level reading)} \\ 
\lspbottomrule
\end{tabular}
\caption{Postponed adjectives as indicators of kind-reference in Polish}
\label{tab:postponed_Pol}
\end{table}

\citet{Wagiel2014} argues that the modification on the kind-level applies at a lower structural level, the \textit{n}P, while object-level modifiers combine with the noun above \textit{n}P. This can explain why singulatives do not usually allow a kind-level modi\-fication: they preferably refer to particular individuals and not to kinds, like their counterparts in Russian. 

\newpage
Singulative nouns can be pluralized. The pluralized singulative form denotes a sum of objects and hence cannot receive a generic reading in characterizing sentences like \REF{ex:geist:31}.

\ea{\#}{\gll Groszki zawierają dużo błonnika. \\
pea.\textsc{sing.pl} contain much fiber \\ \hfill (Polish)
\glt lit.: `Small peas contain a lot of fiber.'}
\label{ex:geist:31}
\z

\noindent Consider questions. Just like in Russian, in Polish when the context makes a quantity interpretation relevant, as in \REF{ex:geist:32}, plural singulatives may be used, and they then refer to a group of particular objects, i.e., they have an exclusive rather than an inclusive reading. The answer confirming the possession of one single pea, as in \REF{ex:geist:32B}/iii, is inappropriate in the context in this question.

\eanoraggedright\label{ex:geist:32} \textit{Scenario}: Ania needs some peas to decorate her salad. Unfortunately, she doesn't have any. She asks her neighbor:
\begin{xlist}
\exi{A:}
\gll Czy masz \minsp{(} jakieś) groszki? \\ 
whether have.\textsc{2sg} {} some peas \\ \hfill (Polish)
\glt `Do you have some peas?' \label{ex:geist:32A:}
\exi{B:}{
    \begin{xlist}
        \exi{i.}{
        Tak, mam. \\ 
        `Yes, (I) have.'}
        \exi{ii.}{
        Tak, mam parę.\\
        `Yes, (I) have some / a couple.'}
        \exi{iii.}{
        \#Tak, mam jeden.\\
        \phantom{\#}`Yes, (I) have one.'}
    \end{xlist}
}\label{ex:geist:32B}
\end{xlist}
\z

\noindent The comparison between Polish and Russian reveals that singulative formation in Polish is not productive and often coincides with diminutivization. However, the few available singulatives in Polish pattern with singulatives in Russian in that they refer to units and cannot refer to kinds, and hence they can both be analyzed as exponents of the functional head Div\textsuperscript{0}. The analysis of singulative nouns as DivPs can account for their distribution and interpretation. In both languages, singulatives may be pluralized. Like in Russian, the pluralized singulatives in Polish have an exclusive (quantified object) interpretation: they denote a sum of particular objects. This can be used as evidence for the assumption that in Polish, like in Russian, the plural combined with singulative nouns is a counting plural, an exponent of the \#-head. 

%------------------------------------------------------------

\section{A split analysis of plural number}\label{sec:geist:4}
\largerpage
\subsection{How many plurals?}\label{sec:geist:4.1}

We start with an interim conclusion. As originally proposed by \citet{Borer2005}, the basic function of the plural is to divide undivided mass. Nouns in the plural of division do not denote sums of objects but rather a set of instances of a kind, as assumed by \citet{Mathieu2012} based on \citet{Grimm2013}. While sums of objects cannot consist of only one individual, a set of kind instances, under certain circumstances, can. Nouns combined with a plural of division are weakly referential and allow non-inclusive readings in non-referential contexts where quantity is not relevant. The plural of division surfaces under Div\textsuperscript{0}. Div is a functional category that serves as a locus for different elements that have the function of division, or more generally, individuation, performed on mass and collective nouns. Crosslinguistic investigations of division show that Div\textsuperscript{0} may have different realizations, i.e., besides the plural, at least numeral classifiers and singulative morphemes but also the indefinite article can spell out this category \citep{Mathieu2012,Mathieu2014}. 

We have seen that the attachment of different exponents of Div\textsuperscript{0} to the noun has different results: a dividing plural on the noun creates an instance of a kind interpretation, while the attachment of a singulative morpheme to the noun yields a denotation of a particular unit. This suggests that the product of the division performed by different morphemes need not be the same. 

Singulatives may pluralize and the fact that this plural only gets an exclusive interpretation suggests that the plural is not always the realization of Div\textsuperscript{0}. If a singulative is combined with the plural, it refers to a sum of particular objects and the plural serves as a counter -- this is the function attributed to the plural by traditional grammars. Following \citet{Mathieu2012,Mathieu2014}, it can be assumed that the counting plural surfaces under \#\textsuperscript{0}. We have also assumed that the counting plural can also combine with non-singulative nouns, even in English, in contexts where a particular quantity of objects is at issue. In this case the plural has an exclusive reading.  

Assuming that the dividing plural surfaces under Div\textsuperscript{0}, and the counting plural is a realization of \#\textsuperscript{0}, \textit{n} can be assumed to be a possible locus for idiosyncratic or expressive plural, also called ``lexical plural'' (see \citealt{Alexiadou2011,Mathieu2014}). Before we elaborate on the lexical plural, let us emphasize that the idea that the plural can be realized in different syntactic positions along the functional spine of a noun phrase can be represented as in \figref{fig:2}. 

\begin{figure}
\begin{forest}
    for tree={s sep=1.2cm, inner sep=1, l=0}
    [DP [\phantom{D$^{0}$}] [\#P [\#] [DivP [Div$^{0}$] [\textit{n}P [\textit{n}$^{0}$] [{[}$\sqrt{root}${]}]
    {\draw (.east) node[left=2.8cm]{Lexical/idiosyncratic plural
$\longrightarrow$};}]{\draw (.east) node[left=2.8cm]{Dividing/classifying
plural $\longrightarrow$};}]{\draw (.east) node[left=2.8cm]{Counting plural
$\longrightarrow$};}]{\draw (.east) node[left=2.8cm]{ };}]{\draw (.east)
node[left]{};}
\end{forest}
\caption{The distributed plural hypothesis for Russian and Polish (based on \citealt{Mathieu2014})}
\label{fig:2}
\end{figure}

As we have already mentioned at the beginning of \sectref{sec:geist:2.2}, there are different suffixes in Russian and Polish classified in traditional grammars as plural suffixes. All these suffixes are ambiguous between the three functions identified in \figref{fig:2}. We can account for the particular function of these suffixes in a given context via a difference in ``attachment height'': the heads \textit{n}\textsuperscript{0}, Div\textsuperscript{0}, and \#\textsuperscript{0} in the nominal spine are responsible for the different readings and distribution of the plural morphemes.\footnote{On methodological grounds, the fine-grained structure in \figref{fig:2} assumed within the neo-constructionist approach to DP-structure may account for the polyfunctionality of the plural morpheme. It has to be noted that such methodology for resolving ambiguity has also been used in lexicalist approaches. In her analysis of the polyfunctionality of \textit{kak} `how' in Russian, \citet{geist:Zimmermann2000} uses insertion into different syntactic positions to disambiguate it as an adverb, adjective, and as a conjunction.} In the next subsection we will focus on lexical plurals.

%----------------------------------

\subsection{Lexical plural: An overview}\label{sec:geist:4.2}

The lexical plural includes many types of plural mentioned in \citet{Corbett2000}, including the plural of abundance, the exaggerative plural, and the evasive plural. Lexical plurals are idiosyncratic in many respects. While the ``grammatical'' plural, which comprises a counting plural and a dividing plural, is productive, the idiosyncratic plural targets only a few roots. Besides idiosyncratic restrictions on its formation, the lexical plural has unpredictable meaning. It yields neither the interpretation `more than one x' nor the interpretation `one or more than one x', which are typical of the grammatical plural, but rather gives other non-compositional interpretations. Here we will focus on one type of lexical plural, the plural of abundance, sometimes also called the ``greater plural'' or ``global plural'' (see \citealt[30f.,]{Corbett2000}, \citealt[109ff.]{Acquaviva2008}), as analyzed in \citet{Alexiadou2011} for Greek. The plural of abundance is a plural that if attached to some mass noun gives the interpretation `a lot of/a great amount of'; cf. \REF{ex:geist:34}.

\ea\label{ex:geist:34}
\gll hithika nera / hithike nero sto patoma \\	
dripped water.\textsc{pl} {} dripped water on\_the floor \\
\glt `a lot of water' / `water dripped on the floor' \hfill \citep[Greek;][36]{Alexiadou2011}
\z

\noindent As Alexiadou notes, mass nouns in the plural of abundance in Greek are cumulative and cannot be combined with numerals. The combination of mass nouns with numerals generally requires a meaning shift from the domain of mass into the domain of objects yielding a sortal or portion interpretation of mass nouns, as illustrated in \REF{ex:geist:35}.

\ea\label{ex:geist:35}
There are five waters in the shop.\\ 
{[}5 types/bottles of water]
\z

\noindent The plural of abundance differs from the grammatical plural in \REF{ex:geist:35} since it preserves the mass interpretation and just adds the meaning of a huge quantity. In \REF{ex:geist:36} it is shown that the plural of abundance does not trigger domain shift. The noun in the plural of abundance remains a mass noun and cannot combine with numerals.

\ea{\#}{\gll Dio nera peftun apo to tavani. \\
two water.\textsc{pl} fall.\textsc{3pl} from the ceiling \\ %\hfill {\small (Greek; only ok under type shifting)}
\glt `two waters fall from the ceiling' (only acceptable under type shifting) \\ \hfill \citep[Greek;][35]{Alexiadou2011}}
\label{ex:geist:36}
\z

\noindent Besides these peculiarities, \citet{Alexiadou2019}, following \citet{Acquaviva2004}, also mentions restrictions concerning verbal predicates that can occur with the abundance plural. She observes that such plurals often occur with predicates of the \textit{spray/load} class, such as \textit{fall, spray, drip,} and \textit{gather}, because such contexts facilitate the conceptualization of a greater amount; see example \REF{ex:geist:36}, where the predicate \textit{peftun} `fall' is used. Nouns in the plural of abundance do not occur as objects of verbs like \textit{drink}; cf. \REF{ex:geist:37}.

\ea\label{ex:geist:37}
\gll ipia nero / \minsp{\#} nera \\ 
drank.\textsc{1sg} water {} {} water.\textsc{pl} \\ %\hfill {\small (Greek; plural only ok under type shifting)}
\glt `I drank water' (plural only acceptable under type shifting) \\ \hfill \citep[Greek;][36]{Alexiadou2011} 
\z

\noindent The above-identified properties of the plural of abundance point to its idiosyncratic behavior. This behavior is reminiscent of the distinction between two levels distinguished in Distributed Morphology: word level vs. root level; cf. \citet{Marantz2001}. Assuming this division, idiosyncratic processes result from affixation at the lowest structural level,  the root level. Merger with a root is highly restricted and leads to idiosyncratic meaning, while merger with a categorized root at a higher level is less restricted and the meaning of the categorized stem combined with the added morpheme is more transparent. Under this view, the abundance plural as a type of idiosyncratic plural must be part of the \textit{n}P. Alexiadou assumes that it is a realization of the nominal categorizing head \textit{n}\textsuperscript{0}: this plural combines with an uncategorized root to form a noun. Thus, while the grammatical plural realized in Div or \# attaches to something that is a noun, a lexical plural is added in the process of formation of a noun. Restrictions on the lexical plural (with respect to input-roots, verbal predicates, and interpretation) thus receive a natural explanation. Following \citet{Alexiadou2011}, we assume that the idiosyncratic plural is a derivational morpheme, an exponent of \textit{n}\textsuperscript{0}. Such an \textit{n}P denotes a non-countable mass. In the next section we will look at lexical plurals in Russian and Polish. 

%-----------------------------

\subsection{Lexical plural in Russian and Polish}\label{sec:geist:4.3}

As in other languages, pluralization of mass nouns is also possible in Russian and Polish. The grammatical plural of mass nouns performs division of mass into portions or sorts, yielding the so-called packaging and sortal reading of mass nouns cf. \REF{ex:geist:38} and \REF{ex:geist:39}. Since portions and sorts are objects rather than masses, we can assume that the plural shifts a noun from mass to count, facilitating the combination with numerals.

\ea\label{ex:geist:38}
\ea\gll Ivan vypil pjat' piv.  \\	
Ivan drank five beer.\textsc{pl.gen} \\ 
\glt `Ivan drank five beers.' [five portions of beer]\label{ex:geist:38a}
\ex\gll Zdes' prodajutsja desjat' različnyx vin. \\
here are\_sold ten different wine.\textsc{pl.gen} \\
\glt `Here they sell ten different wines.' [ten sorts of wine]\hfill (Russian)\label{ex:geist:38b}
\z\z

\ea\label{ex:geist:39}
\ea\gll Piotr wypił pięć piw. \\
Piotr drank five beer.\textsc{pl.gen} \\ 
\glt lit.: `Peter drank five beer.' [five portions of beer]
\label{ex:geist:39a}
\ex\gll W tym sklepie sprzedają dziesięć różnych win. \\
in this store sell.\textsc{3pl} ten different wine.\textsc{pl.gen} \\
\glt lit.: `In this store they sell ten different wines.' [ten sorts of wine] 
\label{ex:geist:39b}
\z\hfill (Polish)\z
   			
\noindent However, there are occurrences in which the mass reading is preserved even in pluralization, which we would identify as the plural of abundance. Such forms cannot be combined with numerals. 

\ea\gll Ravninu \minsp{\{} pokryvali snega / \minsp{*} pjat' snegov\}. \\
plain.\textsc{acc} {} covered snow.\textsc{pl.nom} {} {} five snow.\textsc{pl.gen} \\  
\glt `The plain was covered with a lot of snow.'\hfill (Russian)
\z

\noindent According to the \textit{Academy Grammar of Russian} \citep{geist:Svedova1980}, the plural of abundance in Russian only attaches to a few mass nouns: some substance nouns and some abstract nouns. With nouns denoting substances, the plural of abundance yields the meaning \textsc{a huge amount}, while with abstract nouns it triggers the interpretation of \textsc{high intensity}.

\ea\label{ex:geist:41}
Plural of abundance in Russian \citep[473]{geist:Svedova1980} 
\ea \textsc{a huge amount}\\\textit{voda} `water.\textsc{sg}' -- \textit{vódy} `water.\textsc{pl}'\\ \textit{pesok} `sand.\textsc{sg}' -- \textit{peskí} `sand.\textsc{pl}'\\ 	\textit{sneg} `snow.\textsc{sg}' -- \textit{snegá} `snow.\textsc{pl}' 
\ex \textsc{high intensity}\\  \textit{bol'} `pain.\textsc{sg}' -- \textit{bóli} `pain.\textsc{pl}'\\ \textit{múka} `torture.\textsc{sg}' -- \textit{múki} `torture.\textsc{pl}'
\z\z
	
\noindent The restriction on verbal predicates with which objects in the plural of abundance can occur seems to be similar to what has been mentioned in the literature for Greek and other languages. The relevant examples from the Russian National Corpus contain verbs such as \textit{orosit'} `irrigate' and \textit{projti} `pass', which facilitate or just are compatible with the interpretation `a huge amount' or `high intensity', cf. \REF{ex:geist:42}. Such uses are often found in poetry and the literary style. In the non-literary style as in \REF{ex:geist:43}, the plural of abundance sounds odd.

\ea\label{ex:geist:42}
\ea\gll Orosili bezvodnye peski Kara-Kuma {--} ėto ljudi truda. \\
irrigate.\textsc{pst.3pl} waterless sand.\textsc{pl.nom} Kara-Kum.\textsc{gen} {} this people labor.\textsc{gen} \\
\glt `The waterless sands of Kara-Kum have been irrigated by people of labor.' \hfill [RNC 1]
\label{ex:geist:42a}
\ex\gll Projdut doždi, sojdut snega \ldots \\
pass.\textsc{3pl} rain.\textsc{pl.nom} melt.\textsc{3pl} snow.\textsc{pl.nom} \\
\glt `The rains will pass, the snows will melt \ldots' \hfill [RNC 2] 
\label{ex:geist:42b}
\z\hfill (Russian)\z

\ea{\#}{\gll Anja kupila vódy dlja {detskogo sada}. \\
Anja bought water.\textsc{pl} for kindergarten \\ 
\glt Intended: `Anja bought a lot of water to drink for the kindergarten.'}\\\hfill (Russian)
\label{ex:geist:43}
\z
	
\noindent The occurrence of the plural of abundance in Polish is even more restricted than in Russian. Here preferably abstract nouns seem to occur in the plural of abundance and the plural has the meaning of `high intensity,' as illustrated below. The examples in \REF{ex:geist:44} and \REF{ex:geist:46} are taken from the National Corpus of Polish \citep{PrzepiorkowskiLewandowska-Tomaszczyk2012}.

\ea\label{ex:geist:44}
\gll A później przychodzi nowy rząd, który myśli, że pozjadał wszystkie rozumy {\ldots} i tak dalej. \\
and later comes new government which thinks that ate all mind.\textsc{pl.acc} {} and so further \\ 
\glt Lit.: `And then comes a new government that thinks it has eaten all the minds \ldots and so on.' \hfill (Polish) [NKJP 1]  
\z

\noindent Example \REF{ex:geist:44} contains the idiomatic expression \textit{pozjadać wszystkie rozumy} (lit.: `to eat all the minds'), which has a pejorative meaning. It means that someone thinks that he knows everything and is the wisest person, which is not in line with reality. Another example would be \REF{ex:geist:45}. It has the meaning that he has suffered great torment in his life. A similar corpus example is provided in \REF{ex:geist:46}.

\ea\label{ex:geist:45}
\gll W swoim życiu przeszedł przez męki.\\
in his life go.\textsc{pst.\textsc{sg.m}} through torment.\textsc{pl.acc} \\   
\glt `He has gone through severe torments in his life.'\hfill (Polish)
\z

\ea\label{ex:geist:46}
\gll lecz czasem można się okazać zbyt podatnym i słabym i wtedy trzeba przechodzić przez \minsp{``} męki'' zanim samemu się przekona; czasem warto jest zaufać komuś że nie warto brać 	tych \minsp{``} narkotyków'' żeby przekonać się że są śmiertelne. \\
but sometimes possible \textsc{refl} appear too vulnerable and weak and then necessary pass through {} torment.\textsc{pl.acc} before self.\textsc{dat} \textsc{refl} convinces sometimes worth is trust somebody.\textsc{dat} that \textsc{neg} worth take these {} drugs to convince \textsc{refl} that are lethal \\
\glt Lit.: `but sometimes you can be too vulnerable and weak, and then you have to go through ``torments'' before finding out for yourself; sometimes it is worth trusting someone that it is not worth taking these ``drugs'' to find out that they are deadly.' \hfill (Polish) [NKJP 2]  
\z

\noindent However, one can also find individual examples of the plural of abundance formed from a mass noun such as, for example, \textit{woda} `water'. The example below from the first book of Moses (Genesis 1, 2) illustrates this point.

\ea\label{ex:geist:47}
\gll Ziemia zaś była bezładem i pustkowiem: ciemność była nad powierzchnią bezmiaru wód, a Duch Boży unosił się nad wodami. \\
earth again was disorder.\textsc{ins} and desolation.\textsc{ins} darkness was above surface.\textsc{ins} vastness.\textsc{gen} water.\textsc{gen.pl} and spirit divine ascended \textsc{refl} above water.\textsc{ins.pl} \\ 
\glt Lit.: `And the earth was a disorder and a desolation: darkness was over the surface of the vastness of the waters, and the Spirit of God was hovering over the waters.' \hfill (Polish; Stary Testament, Księga Rodzaju 1,2)  
\z

\noindent The discussion of the examples above shows that the plural of abundance is also available in Russian and Polish and displays idiosyncratic behavior. It can be accounted for if we assume that it is attached very low in the structure, i.e., it is a realization of \textit{n}\textsuperscript{0}.  

%----------------------------------------------------------

\section{Conclusion}\label{sec:geist:5}

In this work, building on \citet{Mathieu2014}, we put forth a split analysis of plural number within the neo-constructionist approach. More precisely, we assume that the plural is distributed along the syntactic spine and may realize three different functional heads: \textit{n}\textsuperscript{0}  (a low functional head), Div\textsuperscript{0} (a middle functional head), and  \#\textsuperscript{0} (a high functional head) in the fine-grained DP-structure. Our analysis shows that disambiguation through syntactic insertion at different positions in the syntactic tree, which has only been used in lexicalist approaches (see \citealt{Zimmermann2008}) for disambiguation of lexemes, proves useful in neo-constructionist approaches for the explanation of the contribution of bound functional morphemes as well. On a more general level our analysis suggests that: 

\begin{itemize}
  \item Functional morphemes like the plural can be polysemous; their function may be differentiated by their position in the syntactic spine;
  \item The structure in \figref{fig:1} has the potential to be a universal structure for DPs; however, the content of the functional categories is language-specific;
  \item The phenomena in Russian and Polish which we discussed in this chapter justify the assumption of a fine-grained DP-structure in Slavic languages as well. However, more work is needed to identify the language-specific contents of the functional categories in this domain; 
  \item Other exponents of the functional categories in \figref{fig:1} have yet to be discovered. 
\end{itemize}
	
We believe that our analysis and discussion that we have offered in this work can serve to inform future studies investigating other Slavic languages and hence can contribute to reaching a better understanding of the puzzling category of plural.

%-----------------------------------------

\section*{Sources} 

 RNC 1: Russian National Corpus: \url{https://ruscorpora.ru/new/en/index.html}\\ Fridrix Gorenštejn. Kuča (1982) // ``Oktjabr'{''}, 1996
 
\noindent RNC 2: Russian National Corpus: \url{https://ruscorpora.ru/new/en/index.html}\\ Vesennij prizyv // ``Ėkran i scena'' [2004.05.06]

\noindent NKJP 1: The National Corpus of Polish, \url{http://nkjp.pl}\\ IJPPAN\_p00002604448, Polityka nr 2298

\noindent NKJP 2: The National Corpus of Polish, \url{http://nkjp.pl}\\ PELCRA\_6203010001504; Usenet -- \url{http://pl.sci.socjologia.narkive.com}

%---------------------------

\section*{Abbreviations}

\begin{tabularx}{.5\textwidth}{@{}lQ}
1 &first person \\
2 &second person \\
3 &third person \\
\textsc{acc} &accusative \\
\textsc{cl} &classifier \\
\textsc{cont} &continuous \\
\textsc{dat} &dative \\
\textsc{f} &feminine \\
\textsc{foc} &focus \\
\textsc{gen} &genitive \\
\textsc{inf} &infinitive \\
\textsc{ins} &instrumental \\
\end{tabularx}%
\begin{tabularx}{.5\textwidth}{lQ@{}}
\textsc{m} &masculine \\
\textsc{neg} &negation \\
\textsc{nom} &nominative \\
\textsc{nv} &non-virile \\
\textsc{pl} &plural \\
\textsc{poss} &possessive \\
\textsc{pst} &past \\
\textsc{refl} &reflexive \\
\textsc{sfp} &sentence-final particle \\
\textsc{sg} &singular \\
\textsc{sing} &singulative \\
\textsc{v} &virile 
%&\\ % this dummy row achieves correct vertical alignment of both tables
\end{tabularx}

% ---------------------------------

\section*{Acknowledgements}

I would like to thank Uwe Junghanns and Kerstin Schwabe for their comments on the draft version of this chapter, which helped me to improve the text. I am also grateful to Hagen Pitsch and Anastasiia Kurbaeva for technical support as well as Berit Gehrke and Radek Šimík for editorial work. Because of the death of Joanna after the first submission of the chapter, I have revised the chapter alone. My work on this chapter was funded by the Deutsche Forschungsgemeinschaft (DFG, German Research Foundation), via the project grant number GE 2136/3-1, project ``The fine structure of the Russian noun phrase: A comparative perspective'' (\url{https://gepris.dfg.de/gepris/projekt/445439335}).

%---------------------------------

\sloppy
\printbibliography[heading=subbibliography,notkeyword=this]

% SCHLAGWORTE

% \is{Cognition} %add "Cognition" to subject index for this page => add whereever necessary on page
% \il{Latin} %add "Latin" to language index for this page => add below example
\cleardoublepage
\end{document}
