\documentclass[output=paper]{langscibook}
\ChapterDOI{10.5281/zenodo.15471423}
% \bibliography{localbibliography}


\author{Ilse Zimmermann\affiliation{Zentralinstitut für Sprachwissenschaft der Akademie der Wissenschaften der DDR, Berlin}}
\title{Untersuchungen zum Verhältnis von Substantivgruppe und Nebensatz}  
\abstract{\noabstract}


% \usepackage{langsci-optional}
\usepackage{langsci-gb4e}
\usepackage{langsci-lgr}

\usepackage{listings}
\lstset{basicstyle=\ttfamily,tabsize=2,breaklines=true}

%added by author
% \usepackage{tipa}
\usepackage{multirow}
\graphicspath{{figures/}}
\usepackage{langsci-branding}

% 
\newcommand{\sent}{\enumsentence}
\newcommand{\sents}{\eenumsentence}
\let\citeasnoun\citet

\renewcommand{\lsCoverTitleFont}[1]{\sffamily\addfontfeatures{Scale=MatchUppercase}\fontsize{44pt}{16mm}\selectfont #1}
  

% \IfFileExists{../localcommands.tex}{%hack to check whether this is being compiled as part of a collection or standalone
%   \usepackage{langsci-optional}
\usepackage{langsci-gb4e}
\usepackage{langsci-lgr}

\usepackage{listings}
\lstset{basicstyle=\ttfamily,tabsize=2,breaklines=true}

%added by author
% \usepackage{tipa}
\usepackage{multirow}
\graphicspath{{figures/}}
\usepackage{langsci-branding}
https://www.overleaf.com/project/62d7faee58f36f3b25c7a70c
%   
\newcommand{\sent}{\enumsentence}
\newcommand{\sents}{\eenumsentence}
\let\citeasnoun\citet

\renewcommand{\lsCoverTitleFont}[1]{\sffamily\addfontfeatures{Scale=MatchUppercase}\fontsize{44pt}{16mm}\selectfont #1}
   
% \togglepaper[23]
% }{}https://www.overleaf.com/project/5faa78eaf7428e3e7bee4ea0

\begin{document}
\begin{otherlanguage}{german}
\maketitle

% The part below is from the generic LangSci Press template for papers in edited volumes. Delete it when you're ready to go.

%LINE FOR THE FOOTNOTE FROM THE HEADING\footnote{\label{fn0}I presented a shorter version of this paper at the second meeting of the Slavic Linguistics Society in Berlin in August 2007 and profited from the discussion. For contemporary Russian, I have consulted Elena Gorichneva, Nikolai Grettschak, Wladimir Klimonow, and Faina Pimenova. For Slovenian, I am indebted to Boštjan Dvořák. For discussion on German and for help in various respects I am grateful to Brigitta Haftka. Barbara Jane Pheby and Jean Pheby have helped me with the English translation of the examples. To Uwe Junghanns and a \textit{JSL} reviewer I owe many valuable suggestion}

% section 0
%\setcounter{section}{-1}
\section*{Aufgabenstellung} \label{sec:zi83:0}

Das Verhältnis von Sprache und Denken wird in diesem Beitrag unter dem\linebreak Aspekt der Beziehungen zwischen Laut- und Bedeutungsstrukturen sprachlicher Äußerungen einerseits und zwischen Bedeutungsstrukturen und der mentalen Repräsentation von Wissen und Überzeugungen der Kommunikationspartner andererseits angesprochen. Am Material der russischen Sprache der Gegenwart werden Nebensätze und korrespondierende Substantivgruppen als Sachverhaltsbezeichnungen mit unterschiedlichem Grad der Ausdrucksverdichtung und semantischen Spezifiziertheit vergleichend untersucht. Dabei werden zentrale\linebreak grammatiktheoretische Fragen über den Zusammenhang von Semantik, Syntax und Lexikon diskutiert.

Im Abschnitt 1 wird skizziert, welchen spezifischen Anteil das Lexikon und die syntaktischen Regeln bei der über mehrere Strukturebenen vermittelten Zuordnung von Laut- und Bedeutungsstrukturen haben und auf welche Weise die semantische und syntaktisch-morphologische Parallelität von Nebensätzen und ih\-rer substantivischen Entsprechungen zu erfassen ist. 

Im Abschnitt 2 wird verdeutlicht, daß Substantivgruppen mit einem Verbal- oder Adjektivabstraktum als Konstruktionskern in ihrer intensionalen Bedeutung mit vergleichbaren Nebensätzen nur partiell übereinstimmen und in ihrer referentiellen Bedeutung weniger festgelegt sind als diese. Einen Schwerpunkt bildet die Frage, wie substantivische Satzentsprechungen als Sachverhaltsbezeichnungen temporal und modal zu interpretieren sind und welche Rolle dabei Ge\-ge\-ben\-hei\-ten des unmittelbaren sprachlichen Kontexts bzw. Voraussetzungen be\-züg\-lich des gemeinsamen Wissens der Kommunikationspartner spielen.

Der Abschnitt 3 ist eine detaillierte Untersuchung der sprachlichen Kontexte, in denen durch Sätze resp. Substantivgruppen bezeichnete Sachverhalte als Tatsachen qualifiziert werden. In Gegenüberstellung von einfachen Behauptungen, Tatsachenbehauptungen und faktiven Präsuppositionen wird gezeigt, wie die das Bestehen von Sachverhalten betreffende Sprechereinstellung sprachlich zum Ausdruck kommt und worin die semantische und pragmatische Spezifik der verschiedenen Ausdrucksweisen besteht.

\largerpage[2]
%section 2
\section{Substantivische Sachverhaltsbezeichnungen und ihre Korrespondenz mit Nebensätzen} \label{sec:zi83:1}
\subsection{Substantivische Satzentsprechungen} \label{sec:zi83:1.1}

Obwohl Sätze und Substantivgruppen grundlegend verschiedene Einheiten des Sprachsystems sind, können sie sich in der Funktion begegnen, Sachverhalte zu bezeichnen. So ist es kein Zufall, daß Sätze mit Substantivgruppen in se\-man\-ti\-scher Hinsicht wie auch in den Ausdrucksmitteln weitgehend übereinstimmen können.

In vielen Fällen sind Substantivgruppen mit einem Verbal- oder Adjektiv\-ab\-strak\-tum als Konstruktionskern substantivische Entsprechungen von Ne\-ben\-sätzen.\footnote{Die Quellen für das Beispielmaterial sind in einem besonderen Verzeichnis angegeben. Als Informanten standen mir V. V. Nesterenko und A. P. Samraj zur Verfügung.}

\ea \label{ex:zi83:1}
    \ea \label{ex:zi83:1a}
    \gll To, čto pacient točno sobljudal ukazanija vrača, uskorilo ego vyzdorovlenie. \\
    das.\textsc{sg}.\textsc{n} dass Patient genau beachtete Anweisungen Arztes beschleunigte.\textsc{sg}.\textsc{n} seine Genesung \\
    \glt ‘Dass der Patient die Anweisungen des Arztes strikt einhielt, beschleunigte seine Genesung.’

\largerpage
    \ex \label{ex:zi83:1b}
    \gll Točnoe sobljudenie pacientom ukazanij vrača uskorilo ego vyzdorovlenie.\\
     genaue Beachtung.\textsc{sg}.\textsc{n}  Patient.\textsc{ins} Anweisungen Arztes beschleunigte.\textsc{sg}.\textsc{n} seine Genesung\\
       \glt ‘Die strikte Einhaltung der Anweisungen des Arztes durch den Patienten beschleunigte seine Genesung.’
    \z
\ex \label{ex:zi83:2}
    \ea \label{ex:zi83:2a}
    \gll Boris somnevaetsja v tom, čto naša komanda pobedit. \\
    Boris zweifelt in das.\textsc{loc} dass unsere Mannschaft siegt.\textsc{prs}.3\textsc{sg}.\textsc{pfv} \\
    \glt ‘Boris zweifelt daran, dass unsere Mannschaft siegen wird.’
    
    \ex \label{ex:zi83:2b}
    \gll Boris somnevaetsja v pobede našej komandy. \\
    Boris zweifelt in Sieg.\textsc{loc} unserer Mannschaft \\
    \glt ‘Boris zweifelt am Sieg unserer Mannschaft.’
    \z
\ex \label{ex:zi83:3}
    \ea \label{ex:zi83:3a}
    \gll Avtor opisyvaet, v kakix obstojatel’stvax pojavljajutsja i isčezajut letajuščie tarelki. \\ 
    Autor beschreibt in welchen Umständen.\textsc{loc}.\textsc{pl} erscheinen und verschwinden fliegende Untertassen \\
    \glt ‘Der Autor beschreibt, unter welchen Umständen fliegende Untertassen auftauchen und verschwinden.’
    
    \ex \label{ex:zi83:3b}
    \gll Avtor opisyvaet obstojatel’stva, v kotoryx pojavljajutsja i isčezajut letajuščie tarelki. \\
    Autor beschreibt Umstände in welchen.\textsc{loc}.\textsc{pl} erscheinen und verschwinden fliegende Untertassen\\
    \glt ‘Der Autor beschreibt die Umstände, unter denen fliegende Untertassen auftauchen und verschwinden.’
    
    \ex \label{ex:zi83:3c}
    \gll Autor opisyvaet obstojatel’stva pojavlenija i isčeznovenija letajuščich tarelok.  \\
     Autor beschreibt Umstände  Erscheinens und Verschwindens fliegender Untertassen \\
     \glt ‘Der Autor beschreibt die Umstände des Auftauchens und Verschwindens fliegender Untertassen.’ (\textit{Novye knigi})
    \z
\z

\noindent Solche Ausdrucksvarianten wie in den angeführten Beispielen bilden jeweils eine Äquivalenzklasse von Sachverhaltsbezeichnungen. Sie vereint die Proposition, die die betreffenden Konstruktionen ausdrücken und die der Identifizierung der bezeichneten Sachverhalte dient.\footnote{Zur Äquivalenzklassenbildung von Sachverhaltsbezeichnungen siehe \citet[93]{vendler1970say-what-you-think}. Zum Begriff Proposition und zu seiner Geschichte vgl. \citet[Kap. 1]{arutjunova1976ponjatie-propozicii-v-logike-i-lingvistike} und \citet{arutjunova1976predlozenie-i-ego-smysl:-logiko-semanticeskie-problemy}. Siehe auch unsere Ausführungen im Abschnitt \ref{sec:zi83:3.2}.} Offensichtlich handelt es sich hier um Propositionen, die in andere Propositionen eingebettet sind, so daß unsere Beispielsätze komplexe Propositionen ausdrücken, die ihrerseits komplexe Sachverhalte identifizieren. Namentlich für eingebettete Propositionen sind Substantivgruppen geeignete Ausdrucksalternativen zu Nebensätzen.

Wir übersehen nicht, daß mit den uns interessierenden substantivischen Syntagmen unter bestimmten Gegebenheiten auch Infinitiv- und Adverbialpar\-ti\-zi\-pi\-alkonstruktionen korrespondieren. Da es jedoch um die Frage geht, ob und inwieweit Wortverbindungen mit einem Verbal- resp. Adjektivabstraktum als syntaktischem Kern und vergleichbare nichtsubstantivische Einbettungen hinsichtlich ihrer Bedeutung und in den Ausdrucksmitteln übereinstimmen, ist es notwendig, den Nebensatz als Vergleichsobjekt zu wählen, denn in ihm kommen die verschiedenen semantischen und syntaktischen Äußerungskomponenten und Beziehungen am explizitesten und am eindeutigsten zum Ausdruck.

Nebensätze sind auch insofern das geeignetste Vergleichsobjekt für substantivische Sachverhaltsbezeichnungen, als sie wie diese keine selbständige kommunikative Funktion haben und sich darin von Hauptsätzen unterscheiden. Nur Hauptsätze repräsentieren -- 
bezogen auf einen konkreten Rahmen kommunikativer Interaktion -- einen bestimmten Sprachhandlungstyp.\footnote{Es versteht sich, daß Hauptsätze zusammen mit den in sie eingeschlossenen Nebensätzen einem bestimmten Sprachhandlungstyp zuzuweisen sind. Inwiefern Hauptsätze als Teile von Satzgefügen eine selbständige kommunikative Funktion haben, betrachte ich als offene Frage. Ihre Beantwortung hängt auch mit dem Problem zusammen, auf welche Weise Sätze und Satzverknüpfungen unter dem Gesichtspunkt ihrer kommunikativen Funktion zu (Teil-)Texten integriert werden. Siehe dazu \citet{motsch1981sprachhandlung-satz-und-text} sowie \citet{viehweger1983sequenzierung-von-sprachhandlungen-und-prinzipien-der-einheitenbildung-im-text}.} 

Da es uns bei dem Vergleich verbaler und substantivischer Sachverhaltsbezeichnungen nicht nur um den propositionalen Gehalt, um die einen Sachverhalt konstituierenden Gegenstände (im weitesten Sinne) und ihre Eigenschaften resp. Beziehungen zueinander geht, sondern auch um den Zeit- und Realitätsbezug von Sachverhalten (s. die Abschnitte \ref{sec:zi83:2} und \ref{sec:zi83:3}), betrachten wir nur solche Substantivgruppen, die wie in unseren Beispielen in der jeweiligen Satzstruktur substantivische Entsprechungen von Nebensätzen sind. In diesem Sinne enthält der Beispielsatz (\ref{ex:zi83:1b}) nur eine substantivische Satzentsprechung. Die Prozeßbezeichnungen \textit{ego vyzdorovlenie} (‘seine Genesung’) und die Gegenstandsbezeichnung \textit{ukazanija vrača} (‘die Anweisungen des Arztes’) bleiben außer Betracht, obwohl es sich in beiden Fällen um deverbale Fügungen handelt und obwohl Prozeßbezeichnungen wohl als ein spezieller Typ von Sachverhaltsbezeichnungen anzusehen sind.\footnote{Wo die Grenze liegt zwischen substantivischen Sachverhaltsbezeichnungen und substantivischen Handlungs-, Zustands- und Eigenschaftsbezeichnungen, betrachte ich als offene Frage. Ihre Beantwortung hängt u. a. davon ab, wie unbesetzte Argumentstellen zu interpretieren sind.}

Sofern also eine Substantivgruppe zu einer durch eine Proposition festgelegten Äquivalenzklasse von Sachverhaltsbezeichnungen gehört und in der jeweiligen Satzkonstruktion durch einen Nebensatz ersetzt werden kann und mit diesem in den lexikalischen Ausdrucksmitteln weitgehend übereinstimmt, wollen wir sagen, daß eine substantivische Satzentsprechung vorliegt. Und wir ergänzen sogleich: Es gibt substantivische Sachverhaltsbezeichnungen, die keine Satzentsprechung haben.

\subsection{Der Parallelismus verbaler und substantivischer Sachverhaltsbezeichnungen} \label{sec:zi83:1.2}

Im folgenden betrachten wir die semantische und syntaktisch-morphologische Verwandtschaft von Sachverhaltsbezeichnungen in Satzform und in entsprechender substantivischer
Gestalt.\footnote{Im folgenden sprechen wir abkürzend von verbalen resp. substantivischen Sachverhaltsbezeichnungen.}

Semantisch stimmen die zu vergleichenden Konstruktionen darin überein, daß sie die gleiche Proposition ausdrücken. Wenn nun der propositionale Gehalt in verbalen und entsprechenden substantivischen Sachverhaltsbezeichnungen durch gleiche resp. derivationell engstens verwandte lexikalische Mittel zum Ausdruck gebracht wird, ist auch eine weitgehende Parallelität der betreffenden Konstruktionen in ihrer syntaktischen Struktur und morphologischen Form zu erwarten.

Zunächst ist eine systematische Entsprechung zwischen den syntaktischen Kategorien und Konstituentenkonstellationen verbaler und substantivischer Sachverhaltsbezeichnungen festzustellen. Wie weit diese Korrespondenz geht, hängt davon ab, auf welcher Strukturebene die Konstruktionen zueinander ins Verhältnis gesetzt werden. Um es vorwegzunehmen: Ich gehe davon aus, daß in der Korrelierung von Laut- und Bedeutungsstrukturen sprachlicher Äußerungen die syntaktische Tiefenstruktur als vermittelnde Ebene anzunehmen ist (vgl. die Ausführungen im Abschnitt \ref{sec:zi83:1.3.1}). Hier ist die Korrespondenz verbaler und substantivischer Sachverhaltsbezeichnungen am größten.

Bezogen auf die Entsprechung von Nebensatz und Substantivgruppe in der Position des grammatischen Subjekts im Beispiel \REF{ex:zi83:1} ist von der folgenden (hier in verallgemeinerter Form angegebenen) syntaktischen Grundstruktur auszugehen:

\ea \label{ex:zi83:4} [ $_{\bar{X}}$ [$_{Y}$ točn-] [$_{X}$ sobljud-] [ $_{\bar{\bar{N}}_{1}}$pacient-] [ $_{\bar{\bar{N}}_{2}}$ukazanij- vrača]]
\z

\noindent Je nach Wortklassenzugehörigkeit $X$ des syntaktischen Kerns, d. h. des Prädikatworts des jeweiligen Syntagmas $\bar{X}$, ergeben sich die Belegungen der Kategorienvariablen resp. die syntaktischen Funktionen der Argumente des Prädikatworts, hier ${\bar{\bar{N}}_{1}}$ und ${\bar{\bar{N}}_{2}}$. Konkret: Ist das Prädikatwort der Einbettung wie in \REF{ex:zi83:1a} ein Verb (\textit{V}), ergibt sich für $\bar{X}$ die syntaktische Kategorie Verbalphrase ($\bar{V}$), $Y$ realisiert sich als Adverb (\textit{Adv}) und das syntaktische Argument ${\bar{\bar{N}}_{1}}$ kann als gram\-mati\-sches Subjekt gewählt werden, so daß ${\bar{\bar{N}}_{2}}$ als direktes Objekt fungiert.\footnote{Die Vereinigung der syntaktischen Argumente eines verbalen Prädikatworts unter der syntaktischen Kategorie Verbalphrase ist vergleichbar den Annahmen  \citeauthor{fillmore1968the-case-for-case}'s (\citeyear{fillmore1968the-case-for-case}) über die Anordnung der Aktanten relativ zum Verb. Zu den Regeln der Subjektselektion und zur Problematik der Transitivität/Intransitivität von Verben siehe \citet{zimmermann1978sintaksiceskie-funkcii-aktantov-zalog-i-perechodnost}.} Gehört das Prädikatwort der Klasse der Substantive (\textit{N}) an, ergibt sich für $\bar{X}$ Nominalphrase ($\bar{N}$) als Syntagmentyp, $Y$ erscheint als Adjektiv (\textit{A}) und die syntaktischen Argumente des Prädikatworts fungieren als Attribute. Das sind die grundlegenden innensyntaktischen Entsprechungen zwischen verbalen und sub\-stan\-ti\-vi\-schen Sachverhaltsbezeichnungen.

Das, was Verbalphrasen ($\bar{V}$) zu Sätzen ($\bar{\bar{V}}$) macht, sind die in \REF{ex:zi83:4} vernachlässigten temporalen und modalen Spezifizierungen (\textit{Aux}) und bei Nebensätzen die Konjunktion (\textit{Konj}).
Das, was Nominalphrasen (${\bar{N}}$) zu Substantivgruppen ($\bar{\bar{N}}$) macht, sind Demonstrativpronomen resp. im Deutschen der Artikel (\textit{Det}).\footnote{Das hier verwendete Typensystem syntaktischer Kategorien ist eng angelehnt an Vorstellungen, die von \citet{chomsky1970remarks-on-nominalization-} und \citet{jackendoff1974introduction-to-the-xbar-convention., jackendoff1977xbar-syntax:-a-study-of-phrase-structure} zur Parallelität von Sätzen und Substantivgruppen entwickelt wurden. Vereinfachend ist in der unter \REF{ex:zi83:4} angegebenen Struktur das „Attribut” \textit{točn-} syntaktisch auf eine Ebene mit dem Prädikatwort und dessen Argumenten gestellt. Es bleibt zu prüfen, inwieweit es angemessener wäre, das „Attribut” als Kokonstituente des durch das Prädikatwort und seine Argumente gebildeten Syntagmas anzusehen und somit aus dem $\bar{X}$-Verband herauszuheben. Siehe dazu \citet{jackendoff1977xbar-syntax:-a-study-of-phrase-structure}.
}

Als außensyntaktische Regularität ist hervorzuheben, daß substantivische Satzentsprechungen immer da auftreten, wo die korrespondierenden Nebensätze mittels des kataphorischen Pronomens \textit{to} (‘das’, ‘es’) in übergeordnete Konstruktionen eingebettet sind, oder es ist wie im Beispiel \REF{ex:zi83:3b} eine substantivische Einbettungsstütze vorhanden (vgl. auch die im Abschnitt \ref{sec:zi83:3} behandelten Fakt\hyp Einbettungen). Ich nehme deshalb an, daß Nebensätze, die mit substantivischen Sachverhaltsbezeichnungen korrespondieren, immer als Subkonstituente eines substantivischen Syntagmas ($\bar{\bar{N}}$) auftreten. Für Satzeinbettungen wie in \REF{ex:zi83:1a} und \REF{ex:zi83:2a} setze ich folgende syntaktische Basisstruktur an:

\ea{} \label{ex:zi83:5}
    [ \textsubscript{$\bar{\bar{N}}$}
        [\textsubscript{Det} to]
        [\textsubscript{$\bar{N}$}
            [\textsubscript{N} e]
        ]
        [\textsubscript{$\bar{\bar{V}}$} \thickspace{} Konj \thickspace{} Aux \thickspace{} $\bar{V}$]
    ]
\z

\noindent Die durch $e$ symbolisierte leere Kette vom Kategorientyp Substantiv ($N$) markiert diejenige syntaktische Position, in der substantivische Einbettungsstützen wie \textit{fakt} (‘Tatsache’) oder \textit{vopros} (‘Frage’) figurieren können. Damit ist eine wei\-tere grundlegende Gemeinsamkeit substantivischer Sachverhaltsbezeichnungen und entsprechender Nebensätze festgestellt: Beide Konstruktionen gehen als substantivische Syntagmen ($\bar{\bar{N}}$) in die übergeordnete Äußerungsstruktur ein. Im Abschnitt \ref{sec:zi83:1.3.3} kommen wir auf die in \REF{ex:zi83:5} angenommene Ausgangsstruktur für Satzeinbettungen zurück.

Zunächst wenden wir uns weiteren Übereinstimmungen zwischen Neben\-sät\-zen und ihren substantivischen Entsprechungen zu, die aus ihren semantischen und lexikalischen Gemeinsamkeiten resultieren.

Zu den Regularitäten von Sätzen, die sich in Substantivgruppen mit einem Verbal- oder Adjektivabstraktum als zentralem Glied wiederholen, gehört die Kor\-re\-spon\-denz der semantischen und syntaktischen Argumente des Prädikatworts, unabhängig von seiner Zugehörigkeit zur Klasse der Verben (resp. Adjektive) oder der Substantive.

Es wird davon ausgegangen, daß die Korrelierung der semantischen und syntaktischen Funktionen der von einem Prädikatwort abhängigen Substantivgruppen vermittels des Lexikons erfolgt und zwar durch Inbeziehungsetzung der semantischen und der syntaktischen Argumentstellen des Prädikatworts. Ferner wird angenommen, daß sich verbale (oder adjektivische) und entsprechende substantivische (nominalisierte) Prädikatwörter in dieser Hinsicht nicht unterscheiden. Die semantischen Funktionen der von einem Prädikatwort abhängigen Substantivgruppen sind durch die elementaren Prädikate der Bedeutungsstruktur des betreffenden Prädikatworts festgelegt, deren Argumente die Substantivgruppen ausdrücken. Die syntaktischen Funktionen ergeben sich aus den Positionen der Substantivgruppen relativ zum Prädikatwort und entsprechend dessen Wortklassenzugehörigkeit (siehe unsere Betrachtungen zur syntaktischen Grund\-struk\-tur \REF{ex:zi83:4} für die in den Beispielen unter \REF{ex:zi83:1} enthaltenen alternativen Sachverhaltsbezeichnungen). Die Repräsentation der semantischen Funktionen der substantivischen Argumente eines Prädikatworts in Gestalt besonderer syntaktischer Kategorien wie in Fillmore's Kasusgrammatik, s. \textcite{fillmore1968the-case-for-case}, ist nicht erforderlich.

\largerpage
Die wesentlich durch das Lexikon gesteuerten Konstellationen der Prä\-di\-kat\-wör\-ter und ihrer syntaktischen Ergänzungen haben auf der Ebene der syntaktischen Tiefenstruktur den Status von Basisdiathesen. Nur in ihnen ist der Bezug auf die semantische Strukturebene eindeutig.\footnote{Zum Konzept der Diathese und zur Unterscheidung von Basisdiathese und abgeleiteter Diathese \citet{zimmermann1978sintaksiceskie-funkcii-aktantov-zalog-i-perechodnost}.} Und hier ist auch die syntaktische Über\-ein\-stim\-mung zwischen verbalen und substantivischen Sachverhaltsbezeichnungen -- verglichen mit ihrer Oberflächenstruktur -- am größten.

Wie die unter \REF{ex:zi83:2} angeführten Beispiele zeigen, können Nebensätze und ihre substantivischen Entsprechungen auch in der Weglaßbarkeit bestimmter Argumente korrespondieren. Sowohl in \REF{ex:zi83:2a} wie auch in \REF{ex:zi83:2b} ist das zweite Argument von \textit{pobedit’} (‘siegen’) resp. \textit{pobeda} (‘Sieg’) nicht benannt, obwohl es in allgemeinster Form mitverstanden wird und in die semantische Struktur dieser Prädikatwörter eingeht. Die diesbezüglichen Parallelitäten und Unterschiede sind in den Lexikoneintragungen der betreffenden Prädikatwörter genau zu verzeichnen.\footnote{Das für das Deutsche erarbeitete „Wörterbuch zur Valenz und Distribution der Substantive” von \citet{sommerfeldt1977worterbuch-zur-valenz-und-distribution-der-substantive} trägt dem Gesichtspunkt der Weglaßbarkeit von Aktanten leider nicht systematisch Rechnung.} Ganz systematisch ist die Weglaßbarkeit der Agensphrase in verbalen und entsprechenden substantivischen Sachverhaltsbezeichnungen.

\begin{exe}
    \exp{ex:zi83:1} \label{ex:zi83:1'}
        \ea \label{ex:zi83:1'a}
        \gll To, čto ukazanija vrača točno sobljudalis’, uskorilo vyzdorovlenie pacienta. \\
        das.\textsc{sg}.\textsc{n} dass Anweisungen Arztes genau beacht.\textsc{pst}.\textsc{pl}.\textsc{refl}  beschleunigte.\textsc{sg}.\textsc{n}  Genesung  Patient.\textsc{gen} \\
        \glt ‘Dass die Anweisungen des Arztes strikt eingehalten wurden, beschleunigte die Genesung des Patienten.’
        
        \ex \label{ex:zi83:1'b}
        \gll Točnoe sobljudenie ukazanij vrača uskorilo vyzdorovlenie pacienta.\\
        genaue Beachtung.\textsc{sg}.\textsc{n}  Anweisungen.\textsc{gen}  Arztes beschleunigte.\textsc{sg}.\textsc{n}  Genesung  Patient.\textsc{gen}\\
            \glt ‘Die strikte Einhaltung der Anweisungen des Arztes beschleunigte die Genesung des Patienten.’
        \z
\end{exe}

\noindent Die Möglichkeiten, bestimmte Aktanten unspezifiziert zu lassen, sind in substantivischen Konstruktionen weitaus größer als in Sätzen, so daß Verbal- und Adjektivabstrakta in nominativen Ketten als quasipronominale Ausdrucksmittel der Herstellung von Koreferenzbeziehungen dienen können. Mit der für Verbal- und Adjektivabstrakta gegebenen Möglichkeit des Absehens von Aktanten, die in entsprechenden Satzkonstruktionen spezifiziert werden müßten, hängt auch ihr Auftreten in Funktionsverbgefügen wie \textit{oderzat’ pobedu} (‘einen Sieg erringen’) zusammen. Wir klammern die mit diesen Verwendungsweisen von Verbal- und Adjektivabstrakta zusammenhängenden Probleme aus unseren Betrachtungen aus.

Auffallend ist auch die weitgehende Übereinstimmung von Syntagmen mit einem verbalen resp. einem entsprechenden substantivischen Prädikatwort hinsichtlich der Form regierter Substantivgruppen.\footnote{Vgl. dazu \citet{zimmermann1967der-parallelismus-verbaler-und-substantivischer-konstruktionen-in-der-russischen-sprache-der-gegenwart}.} Vgl. die Beispiele \REF{ex:zi83:6} und \REF{ex:zi83:7}, in denen jeweils der zweite Aktant in der gleichen vom Prädikatwort regierten Form auftritt, die auch in den entsprechenden verbalen Konstruktionen erscheinen würde:

\ea \label{ex:zi83:6}
    \gll izbeganie učitelem neprijatnostej \\
    Vermeidung  Lehrer.\textsc{ins} Unannehmlichkeiten.\textsc{gen} \\
    \glt ‘die Vermeidung von Unannehmlichkeiten durch den Lehrer’
\ex \label{ex:zi83:7}
    \gll uxaživanie sosedki za našimi cvetami \\
     Pflege Nachbarin.\textsc{gen} für unsere Blumen.\textsc{ins} \\
    \glt ‘die Pflege unserer Blumen durch die Nachbarin’
\z

\noindent Auch hier sind die betreffenden Regularitäten wie auch die Abweichungen der Konstruktionsweise in den Lexikoneintragungen der jeweiligen Prädikatwörter zu verzeichnen. Zu den systematischen Nichtentsprechungen gehört, daß das direkte Objekt transitiver Verben in den entsprechenden substantivischen Fügungen als Genetivattribut auftritt (vgl. \REF{ex:zi83:1a} und \REF{ex:zi83:1b}). Ich nehme an, daß die Zuweisung von Merkmalen, die die Kasusform der syntaktischen Argumente eines Prädikatworts bestimmt, erfolgt, nachdem in verbalen Fügungen das grammatische Subjekt festgelegt ist und bevor Permutationen stattfinden, die nicht mit der Festlegung syntaktischer Funktionen von Substantivgruppen zusammenhängen.\footnote{Zu den Einzelheiten s. \citet{zimmermann1967der-parallelismus-verbaler-und-substantivischer-konstruktionen-in-der-russischen-sprache-der-gegenwart, zimmermann1978sintaksiceskie-funkcii-aktantov-zalog-i-perechodnost}.}

Auch im Hinblick auf die Permutierbarkeit der syntaktischen Argumente von Prädikatwörtern bestehen zwischen verbalen und substantivischen Sachverhaltsbezeichnungen Gemeinsamkeiten, die übrigens im Deutschen wegen der besonders starren Wortfolge in substantivischen Fügungen nicht gegeben sind. Vgl. die veränderte Abfolge der Aktanten im folgenden Beispiel %n \REF{ex:zi83:1''} 
gegenüber der -- wie ich annehme -- zugrunde liegenden kommunikativen Gliederung der betreffenden Syntagmen in \REF{ex:zi83:1}:

\begin{exe}
    \exi{(1$''$)} \label{ex:zi83:1''}
        \ea \label{ex:zi83:1''a}
        \gll To, čto ukazanija vrača točno sobljudal pacient, uskorilo ego vyzdorovlenie. \\
        das.\textsc{sg}.\textsc{n} dass Anweisungen.\textsc{acc}  Arztes genau beachtete.\textsc{sg}.\textsc{m} Patient.\textsc{sg}.\textsc{m} beschleunigte.\textsc{sg}.\textsc{n} seine Genesung \\
        \glt ‘Dass der Patient die Anweisungen des Arztes strikt einhielt, beschleunigte seine Genesung.’
        
        \ex \label{ex:zi83:1''b}
        \gll Točnoe sobljudenie ukazanij vrača pacientom uskorilo ego vyzdorovlenie.\\
        genaue Beachtung.\textsc{sg}.\textsc{n} Anweisungen.\textsc{gen} Arztes Patient.\textsc{ins} beschleunigte.\textsc{sg}.\textsc{n} seine Genesung\\
        \glt ‘Die strikte Einhaltung der Anweisungen des Arztes beschleunigte seine Genesung.’
        \z
\end{exe}

\noindent Den Einzelheiten muß genauer nachgegangen werden. Ich begnüge mich hier mit diesem Hinweis auf eine weitere Parallelität zwischen Sätzen und ihren substantivischen Entsprechungen.

Als nächstes wenden wir uns nun der Frage zu, wie der Parallelismus verbaler und substantivischer Sachverhaltsbezeichnungen in der Grammatik, d. h. in den Regeln der Laut-Bedeutungs-Zuordnung erfaßt wird.

\subsection{Grammatiktheoretische Grundannahmen über die Behandlung von Nominalisierungen} \label{sec:zi83:1.3}
\subsubsection{Zur Motivierung der syntaktischen Tiefenstruktur} \label{sec:zi83:1.3.1}

In meinen Untersuchungen folgte ich Grammatikmodellvorstellungen, wie sie im Rahmen der generativen Transformationsgrammatik entwickelt wurden. Als wissenschaftliche Rekonstruktion des Sprachsystems und als Hypothese über die Sprachkompetenz expliziert die generative Transformationsgrammatik in Form von Strukturbeschreibungen sprachlicher Äußerungen die Regularitäten des über mehrere Strukturebenen vermittelten Zusammenhangs von Laut und Bedeutung. Sie ist ein rekursiver Konstruktionsmechanismus, der für jeden in einer Sprache möglichen Satz spezifiziert, aus welchen Struktureinheiten er sich aufbaut und wie sich seine Bedeutung aus der Bedeutung seiner Bestandteile und ihrer
Relationen zusammensetzt.

Die generative Transformationsgrammatik zeichnet sich durch eine komplexe Betrachtungsweise der verschiedenen Strukturaspekte sprachlicher Äußerungen aus. In das System ihrer Komponenten schließt sie neben der Phonologie, der Morphologie und der Syntax auch das Lexikon und die Semantik ein. Sie stellt die Beschreibung der gegenseitigen Zuordnung von sprachlichen Signalstrukturen und gedanklichem Inhalt für Wörter, Wortgruppen und Sätze auf eine einheitliche theoretische Grundlage und ermöglicht, die systematischen Bezie\-hungen zwischen strukturell verwandten Konstruktionen zu erhellen und unterschiedlich explizite Ausdrucksvarianten in einen erklärenden Zusammenhang zu bringen.

Im Rahmen der generativen Transformationsgrammatik haben sich in den letz\-ten fünfzehn Jahren zwei Grammatikmodelle herausgebildet, die Interpretative Semantik und die Generative Semantik, deren Gegensatz in unterschiedlichen Auffassungen über das Verhältnis von Syntax und Semantik wurzelt. Ein kritischer Vergleich der Grundannahmen dieser beiden Grammatikkonzeptionen findet sich in \citet{pasch1983die-rolle-der-semantik-in-der-generativen-grammatik}.

Für das Grammatikmodell der Interpretativen Semantik ist die Annahme der syntaktischen Tiefenstruktur, einer zwischen der semantischen Repräsentation sprachlicher Äußerungen und ihrer syntaktischen Oberflächenstruktur vermittelnden Strukturebene, konstitutiv. Sie bildet die entscheidende Integrationsstelle von Semantik, Syntax und Lexikon.\footnote{Zum Wesen der syntaktischen Tiefenstruktur und zur Geschichte dieses Konstrukts s. \citet[Kap. 1]{ruzicka1980studien-zum-verhaltnis-von-syntax-und-semantik-im-modernen-russischen}.}

Sie ist diejenige Stelle in der Korrelierung von Inhalt und Ausdruck sprachlicher Äußerungen, wo die für eine bestimmte Sprache charakteristischen syntaktischen Kategorien zusammen mit den Lexikoneinheiten in die Konstituentenstruktur von Sätzen eingehen. Als hierarchisch und linear organisiertes Bezie\-hungs\-gefüge der einen beliebig komplexen Satz bildenden lexikalischen Einheiten ist die syntaktische Tiefenstruktur ein wesentlicher Angelpunkt für die In\-be\-zie\-hung\-setzung syntaktischer und semantischer Strukturen, u.a. der semantischen und syntaktischen Argumente von Prädikatwörtern, wovon im Abschnitt \ref{sec:zi83:1.2}. die Rede war. Sie enthält alle semantischen Informationen, die über das Lexikon in die Bedeutungsstruktur des jeweiligen Satzes eingehen,\footnote{In \citet[Abschnitt 3.]{pasch1983die-rolle-der-semantik-in-der-generativen-grammatik} ist die zentrale Rolle der Lexikoneinheiten in der Laut-Bedeutungs-Zuordnung ausführlich dargestellt.} und die grundlegenden Konstituentenkonstellationen, die für die Verknüpfung der Bedeutung le\-xi\-ka\-li\-scher Einheiten zur Bedeutung komplexer syn\-tak\-ti\-scher Einheiten relevant sind.

Gegenüber den auf der Ebene der syntaktischen Ober\-flä\-chen\-struktur in Erscheinung tretenden Möglichkeiten der Variation der inhaltlichen und syn\-tak\-ti\-schen Gliederung von Äußerungen bilden die in der syn\-tak\-ti\-schen Tiefenstruktur verankerten Aspekte der Bedeutungs- und Ausdrucksstruktur das Invariante verschiedenartig abgewandelter, semantisch, syntaktisch-morphologisch und lexikalisch verwandter sprachlicher Ausdrücke wie die von uns betrachteten verbalen und substantivischen Sachverhaltsbezeichnungen.

\largerpage
\subsubsection{Syntaktische Transformationen}\label{sec:zi83:1.3.2}

Die zwischen der syntaktischen Tiefenstruktur und der syntaktischen Ober\-flä\-chen\-struktur von Sätzen vermittelnden Regeln sind Transformationen. Transformationen bilden in der Interpretativen Semantik syntaktische Strukturen auf syntaktische Strukturen ab. Eine durch Transformationen spezifizierte Teilfolge von Konstituentenstrukturbäumen ($P_{t}$, $P_{t+1}$, ..., $P_{o-1}$, $P_{o}$) repräsentiert die transformationelle Ableitung (Derivation) eines Satzes. Entsprechend der Art und der Anzahl der in einer transformationellen Derivation explizierten Strukturunterschiede mißt sich der Abstand zwischen der syntaktischen Tiefenstruktur und der syntaktischen Oberflächenstruktur von Sätzen und auch der Verwandtschaftsgrad sprachlicher Ausdrücke mit einer gemeinsamen tiefenstrukturellen Basis.

Unter den syntaktischen Transformationen nehmen Umordnungen (Permutationen) von Konstituenten einen großen Raum ein. Mindestens zwei Arten von Permutationen sind zu unterscheiden: solche, die Substantivgruppen umordnen und deren syntaktische Funktion festlegen, und solche, die der aktuellen Gliederung von Äußerungen gemäß der kommunikativen Wichtung von Bedeutungseinheiten dienen und keinen Einfluß auf die syntaktischen Funktionen der Satzglieder haben.

Zu den Permutationen des ersten Typs gehört die Wahl des grammatischen Subjekts. Durch diese Regel wird eine für diese Funktion zulässige Substantivgruppe aus dem tiefenstrukturellen Verband des verbalen (oder adjektivischen) Prädikatworts und seiner syntaktischen Argumente herausgehoben und in die Spitzenposition der betreffenden Satzstruktur unter die unmittelbare Dominanz der Kategorie Satz ($\bar{\bar{V}}$) gebracht (vgl. die Ausführungen im Abschnitt \ref{sec:zi83:1.2}).\footnote{Bei Nebensätzen geht dem ausgewählten und in Spitzenposition gebrachten Subjekt die Konjunktion oder ein Syntagma mit einem pronominalen Fügewort voraus. Ich betrachte es als offene Frage, ob und unter welchen Gesichtspunkten es erforderlich ist, die Nebensätze einleitenden Ausdrücke von den übrigen Satzkonstituenten in der Strukturhierarchie abzuheben.} Diese Transformation hat in substantivischen Syntagmen keine Entsprechung. Hier operieren nur Permutationen des zweiten Typs. Wir müssen es hier bei diesen kurzen Feststellungen belassen.

Es erhebt sich nun für unser Thema, die Beziehung von Nebensätzen und ihnen entsprechenden Substantivgruppen, die entscheidende Frage, ob diese Aus\-drucks\-al\-ter\-na\-ti\-ven in einem transformationellen Ableitungsverhältnis stehen.

\subsubsection{Argumente für eine nichttransformationelle Behandlung von Wortbildungen} \label{sec:zi83:1.3.3}

Zur Erklärung der von uns hier verglichenen verbalen und substantivischen Sachverhaltsbezeichnungen wurden im Rahmen des Grammatikmodells der Interpretativen Semantik zwei sehr verschiedene Lösungsvorschläge entwickelt.

Der erste und ältere, transformationelle Erklärungsweg wurzelt in der Annahme, daß die betreffenden substantivischen Fügungen aus entsprechenden verbalen resp. adjektivischen Satzkonstruktionen transformationell abgeleitet sind und mit ihnen eine gemeinsame tiefenstrukturelle Basis haben.\footnote{Vor allem ist zu verweisen auf \citet{lees1960the-grammar-of-english-nominalizations-supplement-to-international-journal-of-american-linguistics-26}, \citet{vendler1964nominalizations} und \citet{hartung1964die-zusammengesetzten-satze-im-deutschen}. Vgl. auch \citet{zimmermann1967der-parallelismus-verbaler-und-substantivischer-konstruktionen-in-der-russischen-sprache-der-gegenwart, Zimmermann72Die-substantivische}.} Substanti\-vische Satzentsprechungen könnten, meinen ur\-sprüng\-li\-chen Vorstellungen zufolge, auf der Ebene der syntaktischen Tiefenstruktur die gleiche Repräsentation haben wie die in \REF{ex:zi83:5}, Abschnitt \ref{sec:zi83:1.2} für Nebensätze angegebene mit der Abweichung, daß die eingebettete Satzstruktur ($\bar{\bar{V}}$) keine Konjunktion (\textit{Konj}) und keine temporalen und modalen Spezifizierungen (\textit{Aux}) aufweist. Wir hätten also die gegenüber vergleichbaren Nebensätzen reduzierte Basisstruktur% \REF{ex:zi83:5'}
:

\begin{exe}
    \exp{ex:zi83:5} \label{ex:zi83:5'} [$_{\bar{\bar{N}}}$ [$_{Det}$ to] [$_{\bar{N}}$ [$_{N}$e]] [$_{\bar{\bar{V}}} \bar{V}$]]
\end{exe}

\noindent Bezogen auf die substantivische Sachverhaltsbezeichnung im Beispiel \REF{ex:zi83:1b} ergäbe sich die Struktur \REF{ex:zi83:8}:

\ea \label{ex:zi83:8} [$_{\bar{\bar{N}}}$ [$_{Det}$ to] [$_{\bar{N}}$ [$_{N}$e]] [$_{\bar{\bar{V}}}$ [$_{\bar{V}}$ [$_{Adv}$ točn-] [$_{V}$ sobljud-] [$_{\bar{\bar{N}}}$ pacient-] \newline [$_{\bar{\bar{N}}}$ ukazanij- vrača]]]]
\z

\noindent Die transformationelle Ableitung des Verbalabstraktums könnte so bewerkstelligt werden, daß das Verb aus der eingebetteten Satzstruktur ($\bar{\bar{V}}$) unter die unmittelbare Dominanz des einbettenden Nomens ($N$) gebracht wird, indem das Verb die leere Kette $e$ substituiert. In dieser Konfiguration könnte dann das entsprechende substantivierende Ableitungssuffix aus dem Lexikon eingesetzt werden. Dafür müßte man vorsehen, daß die Lexikoneintragung des jeweiligen verbalen oder adjektivischen Prädikatworts ein Merkmal enthält, das die Wahl des Ableitungssuffixes steuert. Dieses seinerseits müßte bezüglich des Kontextes gekennzeichnet sein, in welchem es einsetzbar ist.\footnote{Man müßte in der Merkmalmatrix der Lexikoneintragung für das Verb \textit{sobljudat’} (‘beachten’) für die richtige Wahl des Nominalisierungssuffixes das Merkmal [${+ enij} \; {- Nominalisierung}$] vorsehen. Das im Lexikon verzeichnete Suffix \textit{enij-}, das durch die Merkmale [$+ N$], [$- fem$], [$- mask$] gekennzeichnet ist, müßte außerdem das Kontextmerkmal [${+ enij} \; {- Nominalisierung}$\_] aufweisen, so daß es wie in \REF{ex:zi83:9} unter der unmittelbaren Dominanz der Kategorie Substantiv ($N$) als rechter Nachbar einer das Kontextmerkmal erfüllenden Einheit, in unserem Fall des Verbs \textit{sobljudat’}, auftreten kann.} Zu dieser hier grob skizzierten Nominalisierungstransformation gehört auch, daß das pronominale Element ($Det$) getilgt wird. Wir würden folgende transformierte Struktur erhalten:

\ea \label{ex:zi83:9} [$_{\bar{\bar{N}}}$ [$_{\bar{N}}$ [$_{N}$ [$_{V}$ sobljud-] [enij-]]] [$_{\bar{\bar{V}}}$ [$_{\bar{V}}$ [$_{Adv}$ točn-]  [$_{\bar{\bar{N}}}$ pacient-] \newline [$_{\bar{\bar{N}}}$ ukazanij- vrača]]]]
\z

\noindent Offensichtlich müßten weitere Transformationen und Strukturreduzierungen vor\-ge\-nom\-men werden um das Adverb zum adjektivischen Attribut des Verbalnomens und die substantivischen Aktanten zu seinen Attributen zu machen. Als resultierende Struktur müßte sich \REF{ex:zi83:10} ergeben (vgl. \REF{ex:zi83:4}):

\ea \label{ex:zi83:10} [$_{\bar{\bar{N}}}$ [$_{\bar{N}}$ [$_{A}$ točn-] [$_{N}$ [$_{V}$ sobljud-] [enij-]] [$_{\bar{\bar{N}}}$ pacient-]  [$_{\bar{\bar{N}}}$ ukazanij- vrača]]]
\z

\noindent Warum ist dieser Erklärungsansatz zu verwerfen? So adäquat er für die Ableitung von Verbal- und Adjektivabstrakta in substantivischen Sachverhaltsbezeichnungen erscheinen mag, er ist unzureichend für die Erklärung anderer Wortbildungen, die die Nominalisierung von Verben resp. Adjektiven -- wie in unserem Beispiel \REF{ex:zi83:1} -- sogar begleiten können.

Der angenommene, mit Nebensätzen weitgehend übereinstimmende tiefenstrukturelle Ansatz \REF{ex:zi83:8} weist zwar für das Verbalnomen einen kategoriell vorgeprägten Platzhalter auf, nicht jedoch für sein adjektivisches Attribut. Es müßte also eine Umkategorisierung vorgenommen werden, die Adverbiale wie \textit{točno} (‘genau’) in die Klasse der Adjektive überführt. Oder die Ausgangsstruktur \REF{ex:zi83:8} müßte mit einem weiteren Platzhalter, [$_{A}e$] als linker Nachbar von [$_{N}e$], an\-ge\-rei\-chert werden. Dieser wäre dann auch eine Basis für Wortbildungen wie das adjektivische Kompositum \textit{poslevoennyj} (‘Nachkriegs-’) in substantivischen Sachverhaltsbezeichnungen wie \textit{poslevoennaja vstreča druzej} (‘das Nachkriegstreffen der Freunde’). Man vergleiche die entsprechende verbale Konstruktion \REF{ex:zi83:11}, die die Temporalangabe in Form einer adverbiellen Bestimmung enthält:

\ea \label{ex:zi83:11}
\gll Druz’ja vstretilis’ posle vojny. \\
Freunde trafen.\textsc{refl} nach  Krieg.\textsc{gen} \\
\glt ‘Die Freunde trafen sich nach dem Krieg.’
\z

\noindent Es ist offensichtlich, daß unzählige ad hoc-Angaben in den lexikalischen Einheiten bezüglich ihrer Verträglichkeit mit Wortbildungsaffixen gemacht werden müßten, zusätzlich zu der Frage der Umkategorisierung, so daß dieses Verfahren, Wortbildungen transformationell zu erklären, zweifelhaft erscheint.

Es ist auch höchst fragwürdig, ob für deverbale Nomina in Funktionsverbgefügen wie \textit{oderžat’ pobedu} (‘einen Sieg erringen’) oder \textit{okazat’ pomošč’} (‘Hilfe erweisen’) eine tiefenstrukturelle Ableitungsbasis wie in %(\REF{ex:zi83:5'})
(5’) zu rechtfertigen wäre.

Für deverbale Bildungen wie \textit{ukazanija vrača} (‘die Anweisungen des Arztes’) in unserem Beispiel \REF{ex:zi83:1b} müßten in der betreffenden Ausgangsstruktur auch Vorkehrungen getroffen werden, um die spezielle Bedeutung des Verbalnomens zu erfassen. Dieses Erfordernis ergäbe sich schließlich für alle Wortbildungen, die gegenüber entsprechenden syntaktischen Fügungen mit Be\-deu\-tungs\-spe\-zia\-li\-sie\-run\-gen verbunden sind.\footnote{Siehe dazu die Ausführungen von \citet{motsch1983uberlegungen-zu-den-grundlagen-der-erweiterung-des-lexikons}.} \textit{Predmetnyj ukazatel’} heißt im Deutschen \textit{Sachregister}, \textit{ukazatel’ skorosti} – \textit{Geschwindigkeitsmesser}, \textit{ukazka} – \textit{Zeigestock} usw.

Der irreguläre Charakter von Wortbildungen, und zwar sowohl hinsichtlich der Ableitungsmittel wie auch hinsichtlich der Bedeutungen, weist deutlich auf die Andersartigkeit dieser sprachlichen Einheiten gegenüber regulären syntaktischen Konstruktionen und ihren Bedeutungen hin. Alles in allem: Es erscheint unangemessen, Wortbildungen mittels syntaktischer Transformationen zu er\-klä\-ren. \citet{motsch1977ein-pladoyer-fur-die-beschreibung-von-wortbildungen-auf-der-grundlage-des-lexikons, motsch1979zum-status-von-wortbildungsregularitaten} hat das ausführlich begründet.

Das Russische verfügt über kein regelhaftes Nominalisierungsverfahren, wie es im Englischen in Gestalt der Gerundivnominalisierungen wie \textit{the hunters’ shooting the deer, John’s rapidly driving the car} existiert. Diese regulären Bildungen sind nominale Verbalformen und als solche dem Verbalparadigma zuzurechnen, und die durch sie gebildeten Syntagmen gehören in den Erklärungsbereich der Syntax genau wie die entsprechenden Sätze. Das Russische hat nur vollständig dem Paradigma der Substantive angehörende Verbal- und Adjektivabstrakta mit den erwähnten Irregularitäten einschließlich zufälliger lexikalischer Lücken.

Alle hier vorgebrachten Einwände gegen eine transformationelle Ableitung substantivischer Sachverhaltsbezeichnungen und von Wortbildungen allgemein sind zugleich Argumente für den zweiten, neueren Erklärungsweg, der im Rahmen des Grammatikmodells der Interpretativen Semantik für die Behandlung von Nominalisierungen vorgeschlagen wurde.\footnote{Siehe vor allem \citet{chomsky1970remarks-on-nominalization-} und \citet{jackendoff1974introduction-to-the-xbar-convention., jackendoff1975morphological-and-semantic-regularities-in-the-lexicon, jackendoff1977xbar-syntax:-a-study-of-phrase-structure}.}

Gemäß diesem in einigen wesentlichen Punkten stark modifizierten Grammatikmodellentwurf haben verbale und substantivische Sachverhaltsbezeichnungen keine gemeinsame syntaktische Tiefenstruktur, und sämtliche Wortbildungen stehen im Lexikon als fertige Benennungseinheiten zur Verfügung. Dadurch werden die syntaktischen Regeln von vielen Unnatürlichkeiten befreit und die Grammatik insgesamt wird strengeren Beschränkungen unterworfen.\footnote{Siehe dazu \citet{jackendoff1972semantic-interpretation-in-generative-grammar.}.}

\largerpage
Auf dieser grammatiktheoretischen Basis nehme ich also an, daß Nebensätze wie in dem Beispiel \REF{ex:zi83:1a} auf der Ebene der syntaktischen Tiefenstruktur die in \REF{ex:zi83:5} in allgemeiner Form angegebene Repräsentation haben. Die substantivischen Entsprechungen von Nebensätzen haben ihre eigene tiefenstrukturelle Basis. Für die im Beispiel \REF{ex:zi83:1b} enthaltene substantivische Sachverhaltsbezeichnung ist sie die in \REF{ex:zi83:10} angegebene Struktur.\footnote{Die Frage, ob Wortbildungen wie das Verbalnomen in \REF{ex:zi83:10} eine interne syntaktische Struktur haben, wäre separat zu erörtern.}

Die zahlreichen semantischen, syntaktisch-morphologischen und le\-xi\-ka\-li\-schen Übereinstimmungen verbaler und substantivischer Sachverhaltsbezeichnungen, von denen wir im Abschnitt \ref{sec:zi83:1.2} gesprochen haben, werden bei dieser Lösungsva\-ri\-an\-te zu einem großen Teil durch Korrespondenzregeln zwischen derivationell verwandten Lexikoneinheiten erfaßt und durch entsprechende Verallgemeinerungen bezüglich der Anwendungsdomänen syntaktischer und morphologischer Re\-geln wie auch der semantischen Interpretationsregeln. Das Grammatikmodell der Interpretativen Semantik hat dafür inzwischen geeignete Beschreibungsmittel ent\-wi\-ckelt. Ihre Darstellung und Rechtfertigung, einschließlich ihrer Be\-wäh\-rung an den Fakten der russischen Sprache geht über die Aufgabenstellung dieses Beitrags hinaus.

\section{Die referentielle Bedeutung substantivischer Sachverhaltsbezeichnungen}\label{sec:zi83:2}

\subsection{Zur Unterscheidung von signifikativer und referentieller Bedeutung}\label{sec:zi83:2.1}

Um die Frage beantworten zu können, ob und inwieweit sich Nebensätze und ihre substantivischen Entsprechungen semantisch unterscheiden, ist es erforderlich, die Ebene der signifikativen (oder: intensionalen) Bedeutung und die Ebene der referentiellen Bedeutung sprachlicher Äußerungen auseinanderzuhalten. Mei\-ner Meinung nach führte unter anderem die ungenügende theoretische Reflexion dieses Unterschieds in der generativen Transformationsgrammatik dazu, daß für Sätze und ihre Nominalisierung eine gemeinsame syntaktische Tiefenstruktur und auf dieser Basis auch die gleiche Bedeutungsstruktur angenommen wurden.

Ich gehe davon aus, daß Nebensätze und ihre substantivischen Entsprechungen die gleiche referentielle Bedeutung haben können, jedoch in ihrer signifikativen Bedeutung nur partiell übereinstimmen. Diese Bedeutungsdifferenz widerspiegelt sich in entsprechenden Unterschieden der Ausdrucksstruktur der be\-tref\-fen\-den verbalen resp. substantivischen Sachverhaltsbezeichnungen.

Wie unsere Darlegungen im Abschnitt \ref{sec:zi83:1} deutlich gemacht haben, ist in semantischer Hinsicht der propositionale Gehalt der gemeinsame Nenner der verglichenen Ausdrucksvarianten. Die in Nebensätzen durch die finiten Verformen ausgedrückten Bedeutungsanteile sind in der Bedeutungsstruktur und in der Aus\-drucks\-struk\-tur der entsprechenden substantivischen Sach\-ver\-halts\-be\-zeich\-nun\-gen abwesend. Und zwar sind es vor allem die Spezifizierungen, die die temporale Einordnung und den Realitätsbezug des jeweiligen Sachverhalts ausmachen und ohne die die sich in Sätzen vollziehende Prädikation nicht möglich ist.

Ich nehme nun an, daß nicht nur Hauptsätze, sondern auch Nebensätze die folgenden grundlegenden Komponenten in ihrer signifikativen Bedeutungsstruktur aufweisen: (a) den propositionalen Gehalt ($pc$), der alle diejenigen Einheiten umfaßt, die die bezeichneten Sachverhalte zu identifizieren ge\-statten, (b) die temporale Einordnung ($T$) des jeweiligen Sachverhalts und (c) die propositionale Einstellung ($pa$) des Sprechers, die seine Sicht bezüglich des Wirklich- oder Möglich-Seins des betreffenden Sachverhalts beinhaltet.

Für $sem$, die signifikative Bedeutungsstruktur eines Satzes, ergibt sich folgendes Beziehungsgefüge:

\ea \label{ex:zi83:12} $sem = (pa \, (T(pc)))$
\z

\noindent Angewendet auf den Kontext der jeweiligen Äußerung, insbesondere den sprachlichen und situativen Kontext, zu dem als wesentliche Komponente das Wissens- und Überzeugungssystem der Kommunikationspartner gehört, kommt auf der Basis der signifikativen Bedeutung (der Satzbedeutung) die referentielle Bedeutung der betreffenden sprachlichen Äußerung (die Äußerungsbedeutung) zu\-stan\-de.

Die referentielle Bedeutung (das Denotat) einer sprachlichen Äußerung setzt sich zusammen aus dem durch $pc$ identifizierten und auf eine durch $T$ bestimmte Zeitspanne -- relativ zum Redemoment -- bezogenen Sachverhalt sowie aus der durch $pa$ charakterisierten Einstellung des Sprechers, die die Existenzweise des betreffenden Sachverhalts in seinem Bewußtsein bestimmt.\footnote{Ich verweise auf \citet{Bierwisch80Semantic-structure}, \citet{lang1983einstellungsausdrucke-und-ausgedruckte-einstellungen} und \citet{steube1980temporale-bedeutung-im-deutschen, steube1983indirekte-rede-und-zeitverlauf}.} Siehe dazu unsere Ausführungen im Abschnitt \ref{sec:zi83:3.2}.

Jede eine Äußerung mit der signifikativen Bedeutung $sem$ interpretierende Person wird in Abhängigkeit von dem Charakter der ausgedrückten Einstellung des Sprechers bezüglich des bezeichneten temporal eingeordneten Sachverhalts ihre eigene Position entwickeln, die ihre möglichen Reaktionen auf die betreffende Äußerung determiniert.\footnote{Vgl. dazu die sehr erhellenden Ausführungen von \citet{miller1976language-and-perception} im Rahmen ihrer operationalen Semantikauffassung. Siehe insbesondere 198 ff., 631 ff.} Dieser Aspekt der Interpretation ist besonders wichtig bei der Erfassung des kommunikativen Sinns sprachlicher Äußerungen, d. h. des Sprachhandlungstyps. Diesen Fragenkomplex können wir jedoch vernachlässigen, da er nur selbständige Sätze, nicht Nebensätze und ihre substantivischen Entsprechungen betrifft. Sofern allerdings Spre\-cher\-ein\-stel\-lun\-gen bezüglich des durch Nebensätze oder substantivische Sachverhaltsbezeichnungen ausgedrückten propositionalen Gehalts vorliegen, sind sie vom Adressaten in Rechnung zu stellen.

Es ist hier nicht beabsichtigt, die verschiedenen Ausdrucksmittel von Spre\-cher\-ein\-stel\-lun\-gen in Nebensätzen näher zu betrachten. Einschränkend muß jedoch festgestellt werden, daß $pa$ in $sem$ von Nebensätzen im Unterschied zu Hauptsätzen nicht spezifiziert zu sein braucht. Das heißt: der Sprecher kann bezüglich des Wirklich- oder Möglich-Seins des betreffenden Sachverhalts eine indifferente Position einnehmen.

Anders verhält es sich mit der temporalen Einordnung der Sachverhalte. Sie ist in Nebensätzen (wie auch in Hauptsätzen) immer ausgedrückt.\footnote{Es versteht sich, daß ich Infinitivkonstruktionen der verschiedensten Art nicht als Nebensätze betrachte. Für letztere sind finite Verbformen konstitutiv.}

Auch für substantivische Sachverhaltsbezeichnungen ist nicht ausgeschlossen, daß ihre signifikative Bedeutungsstruktur und entsprechend auch ihre Ausdrucksstruktur neben dem propositionalen Gehalt Temporalangaben und Sprechereinstellungen beinhaltet. Das betrifft aber nur solche Spezifizierungen, die in den entsprechenden Nebensätzen nicht durch die Kategorien des finiten Verbs ausgedrückt werden, sondern durch lexikalische Mittel, adverbielle Temporalbestimmungen und Modalwörter.\footnote{Wir konzentrieren uns hier auf den Ausdruck von Sprechereinstellungen, die das Wirklich- resp. Möglich-Sein der zur Rede stehenden Sachverhalte betreffen. Soweit ich sehe, haben nur sie -- und zwar in beschränktem Maße -- die Chance, auch in substantivischen Sachverhaltsbezeichnungen zum Vorschein zu kommen.}

 Substantivische Sachverhaltsbezeichnungen können Zeitangaben enthalten\linebreak oder selbst in Zeitbestimmungen eingehen. In den Beispielsätzen \REF{ex:zi83:13}--\REF{ex:zi83:15} benennt die Präposition resp. präpositionale Fügung, die der Substantivgruppe mit dem Verbalabstraktum jeweils vorausgeht, den Zeitbezug zu den im übergeordneten Satz angesprochenen Sachzusammenhängen. Das Beispiel \REF{ex:zi83:15} ist hinsichtlich der Reichweite der Temporalbestimmung \textit{v 1900 godu} (‘im Jahre 1900’) mehr\-deu\-tig. Diese Zeitangabe kann sich auf den letzten Relativsatz beziehen, könnte aber auch der in ihm enthaltenen substantivischen Sachverhaltsbezeichnung zu\-ge\-schrie\-ben werden.\footnote{Nicht umsonst merken die Herausgeber des Briefwechsels Lenins und Gorkis an, daß es für eine Reise Lenins nach Krasnojarsk in dem fraglichen Jahr keine Hinweise gebe.} Vgl. auch unsere Betrachtungen im Abschnitt \ref{sec:zi83:1.3.3} zu der temporal spezifizierten Sachverhaltsbezeichnung \textit{poslevoennaja vstreča druzej} (‘das Nachkriegstreffen der Freunde’).

\newpage
\ea \label{ex:zi83:13}
    \gll I ėto prozvučalo v ego golose, kogda on posle dolgogo molčanija skazal: -- Da, u nas poka tišina … . \\
    und das.\textsc{sg}.\textsc{n}  klang.\textsc{sg}.\textsc{n} in seiner Stimme als er nach langem Schweigen sagte {} ja bei uns vorerst Ruhe {}\\ 
    \glt ‘Und das war in seiner Stimme zu hören, als er nach langem Schweigen sagte: „Ja, bei uns ist es vorerst ruhig …“.’ (K. Simonov) 
    
\ex \label{ex:zi83:14}
    \gll Kniga osveščaet istoriju revoljucii v masštabe vsej strany so vremeni priezda v Rossii V. I. Lenina do prinjatija pervoj konstitucii RSFSR. \\
    Buch beleuchtet  Geschichte  Revolution.\textsc{gen} in Maßstab gesamten Landes seit  Zeit  Ankunft.\textsc{gen} in Russland V. I. Lenin.\textsc{gen}  bis Annahme.\textsc{gen} erste Verfassung.\textsc{gen} RSFSR.\textsc{gen} \\
    \glt ‘Das Buch beleuchtet die Geschichte der Revolution im ganzen Land seit der Ankunft Lenins in Russland bis zur Verabschiedung der ersten Verfassung des RSFSR.’ (\textit{Novye knigi})
    
\ex \label{ex:zi83:15}
    \gll Anjute, požalujsta, peredaj, čto filosofskaja rukopis’ poslana uže mnoj tomu znakomomu, kotoryj žil v gorodke gde my videlis’ pered moim ot’’ezdom v Krasnojarsk v 1900 godu.  \\
    Anja bitte übermittle dass  philosophisches Manuskript gesandt schon mir.\textsc{ins} jenem Bekannten der lebte in  Städtchen wo wir sahen.\textsc{refl} vor meiner Abreise nach Krasnojarsk in 1900 Jahr \\
    \glt ‘Anja, bitte, übermittle, dass das Philosophiemanuskript von mir bereits jenem Bekannten geschickt worden ist, der in dem Städtchen lebte, wo wir uns im Jahre 1900 vor meiner Abreise nach Krasnojarsk gesehen hatten.’ (V. I. Lenin)

\z

\noindent Was Ausdrücke für Sprechereinstellungen bezüglich des Wirklich- resp. Möglich-Seins von Sachverhalten angeht, so sind die Möglichkeiten für ihr Auftreten in substantivischen Satzentsprechungen sehr beschränkt. Modalwörter wie \textit{jakoby, budto by} (‘angeblich’) sind mit Bezug auf Attribute des Verbalnomens wie in \REF{ex:zi83:16} und \REF{ex:zi83:17} in substantivischen Sachverhaltsbezeichnungen anzutreffen. 

\ea \label{ex:zi83:16}
    \gll Branili Borisa za {budto by} častoe opazdyvanie. \\
     schalten.\textsc{pst}.3\textsc{pl} Boris wegen angeblich häufigen Zuspätkommens \\
    \glt ‘Man hat Boris gescholten, weil er angeblich häufig zu spät kommt.’

\newpage
\ex \label{ex:zi83:17}
    \gll Medsestra kritikovala pacienta za jakoby netočnoe sobljudenie ukazanij vrača. \\
     Krankenschwester kritisierte  Patienten wegen angeblich ungenauer Beachtung  Anweisungen.\textsc{gen}  Arzt.\textsc{gen} \\
    \glt ‘Die Krankenschwester kritisierte den Patienten, weil er angeblich die Anweisungen des Arztes nicht strikt befolgt hat.’
\z

\noindent Auch Konstruktionen wie \REF{ex:zi83:18}, in denen sich das Modalwort semantisch auf die Negationspartikel bezieht, wurden als sprachgerecht beurteilt. Dagegen wurden Fügungen ohne Negation wie in \REF{ex:zi83:19} nicht akzeptiert.

\ea \label{ex:zi83:18}
    \gll Medsestra kritikovala pacienta za jakoby nesobljudenie ukazanij vraca. \\
    Krankenschwester kritisierte  Patienten wegen angeblicher Nichtbeachtung  Anweisungen.\textsc{gen} Arzt.\textsc{gen} \\
    \glt ‘Die Krankenschwester kritisierte den Patienten, weil er angeblich die Anweisungen des Arztes nicht befolgt hat.’
\z

\ea[*]{ \label{ex:zi83:19}
    \gll Branili Borisa za {budto by} opazdyvanie. \\
    schalten.\textsc{pst}.3\textsc{pl} Boris wegen angeblich Zuspätkommens \\
    \glt Intendiert: ‘Man hat Boris gescholten, weil er angeblich zu spät kam/kommt.’}
\z

%Was Ausdrücke für Sprechereinstellungen bezüglich des Wirklich- resp. Möglich-Seins von Sachverhalten angeht, so sind die Möglichkeiten für ihr Auftreten in substantivischen Satzentsprechungen sehr beschränkt. Modalwörter wie \textit{jakoby, budto by} (,angeblich‘) sind mit Bezug auf Attribute des Verbalnomens wie in \REF{ex:zi83:16} und \REF{ex:zi83:17}

%\ea \label{ex:zi83:16}
%    \gll Branili Borisa za {budto by} častoe opazdyvanie. \\
%    {man hat gescholten} Boris wegen angeblich häufigen Zuspätkommens \\

%\ex \label{ex:zi83:17}
%    \gll Medsestra kritikovala pacienta za jakoby netočnoe sobljudenie ukazanij vrača. \\
%    {(die) Krankenschwester} kritisierte {(den) Patienten} wegen angeblich ungenauer Beachtung {(der) Hinweise} {(des) Arztes} \\
    
%\z

%in substantivischen Sachverhaltsbezeichnungen anzutreffen. Auch Konstruktionen wie \REF{ex:zi83:18},

%\ea \label{ex:zi83:18}
%    \gll Medsestra kritikovala pacienta za jakoby nesobljudenie ukazanij vraca. \\
%    {(die) Krankenschwester} kritisierte {(den) Patienten} wegen angeblicher Nichtbeachtung {(der) Hinweise} {(des) Arztes} \\
    
%\z

%in denen sich das Modalwort semantisch auf die Negationspartikel bezieht, wurden als sprachgerecht beurteilt. Dagegen wurden Fügungen ohne Negation wie in \REF{ex:zi83:19} nicht akzeptiert.

%\ea \label{ex:zi83:19}
%    \gll Branili Borisa za {budto by} opazdyvanie. \\
%    {man hat gescholten} Boris wegen angeblichen Zuspätkommens \\
    
%\z

\noindent Offenbar sind die Möglichkeiten, modale Ausdrücke, die Sprechereinstellungen beinhalten, auch in substantivischen Sachverhaltsbezeichnungen zu verwenden, im Deutschen weitgehender als im Russischen. Allerdings bestehen auch hier zahlreiche Beschränkungen. Im Gegensatz zu \textit{kaum}, \textit{hoffentlich} und wohl auch \textit{vermutlich} kann \textit{angeblich} als Attribut zu einem Verbal- oder Adjektivabstraktum unter Beibehaltung seiner semantischen Funktion auftreten. Für attributiv verwendbare Modalwörter wie \textit{wahrscheinlich}, \textit{möglich}, \textit{unbestreitbar} usw. wäre zu prüfen, ob sie in substantivischen Sachverhaltsbezeichnungen als Entsprechungen zu Satzadverbien, in vergleichbaren Sätzen gelten können oder aber attributive Entsprechungen zu prädikativ verwendeten Adjektiven sind oder ob sie be\-züg\-lich dieser beiden Interpretationen mehrdeutig sind. Vgl.:

\ea \label{ex:zi83:20} Die mögliche Abwesenheit des Direktors auf der Tagung darf deren Vorbereitung nicht beeinträchtigen.
\z

\newpage
\noindent Diese Fragen stellen sich analog auch fürs Russische. Wir können ihre sorgfältige Beantwortung hier nicht vornehmen.\footnote{Zum unterschiedlichen Status von Satzadverbien und entsprechender prädikativer Adjektive in der semantischen Struktur von Äußerungen s. unsere Ausführungen im Abschnitt \ref{sec:zi83:3.4.3}.} 

Es kam uns darauf an zu zeigen, daß substantivische Sachverhaltsbezeichnungen wie entsprechende Nebensätze lexikalisch ausgedrückte Spezifizierungen hinsichtlich des Zeit- und Realitätsbezugs der zur Rede stehenden Sachverhalte enthalten können. Wenn die diesbezüglichen Einordnungen von Sachverhalten in den betreffenden sprachlichen Äußerungen unausgedrückt bleiben, entsteht die Frage, ob und auf welcher semantischen Ebene von ihnen abgesehen wird und auf welcher Grundlage ein solches Absehen von bestimmten semantischen Kategorien möglich ist.

\subsection{Die Kontextabhängigkeit der referentiellen Bedeutung substantivischer Sachverhaltsbezeichnungen}\label{sec:zi83:2.2}

Entsprechend der von \citet[5]{Bierwisch80Semantic-structure} formulierten sehr strengen Beschrän\-kung für mögliche Grammatiken, d. h. für die Regeln der Laut-Bedeutungs-Zu\-ord\-nung, nehme ich an, daß in der signifikativen Bedeutungsstruktur sprachlicher Äußerungen nichts enthalten sein kann, was nicht auch -- vermittelt durch die syntaktischen, morphologischen und phonologischen Regeln der Grammatik einschließlich des Lexikons -- in der Ausdrucksstruktur seinen Niederschlag fin\-det.\footnote{Die mit dieser starken Hypothese verbundenen Implikationen für elliptische Ausdrücke verschiedenster
Art klammern wir hier völlig aus.} Für substantivische Sachverhaltsbezeichnungen wie in den im Abschnitt \ref{sec:zi83:1.1} angeführten Bei\-spie\-len \REF{ex:zi83:1b}, \REF{ex:zi83:2b} und \REF{ex:zi83:3c}, die keinerlei Indikatoren für die temporale Einordnung und den Realitätsbezug des jeweiligen Sachverhalts auf\-wei\-sen, wird deshalb in der signifikativen Bedeutungsstruktur nur der propositionale Gehalt $(pc)$ spezifiziert, während $pa$ und $T$ (vgl. \REF{ex:zi83:12}) unspezifiziert sind. Das bedeutet, daß sich Festlegungen im Hinblick auf die zeitliche Einordnung eines durch $pc$ identifizierten Sachverhalts und auf die Sprechereinstellung bezüglich des Bestehens des betreffenden Sachverhalts nur aus dem sprachlichen oder situativen Kontext ergeben können.

Sowohl der unmittelbare sprachliche Kontext innerhalb der betreffenden mög\-li\-cher\-wei\-se komplexen Satzstruktur wie auch bestimmte Teile des weiteren sprachlichen Kontexts können Informationen enthalten, die die referentielle Bedeutung temporal und modal unspezifizierter substantivischer Sachverhaltsbezeichnungen determinieren. Ich nehme an, daß Sachverhalte im Bewußtsein des Sprechers und auch jeder die betreffende Äußerung interpretierenden Person an bestimmte kognitive Einstellungen gebunden sind und auch auf bestimmte Zeitspannen resp. -punkte bezogen sind.

Diese Annahme wird unter anderem gestützt durch die relative Leichtigkeit, mit der Personen substantivische Sachverhaltsbezeichnungen in verbale Ent\-spre\-chun\-gen, d. h. in Sätze, umformen können, wobei die temporale Einordnung und oft auch die propositionale Einstellung deutlich zum Vorschein kommen. Dabei stützt sich die betreffende Person auf ihr Wissens- und Überzeugungssystem, in das der fragliche Sachverhalt eingebettet ist. In diesem Sinne können die im Abschnitt \ref{sec:zi83:1.1} angeführten Satzgefüge \REF{ex:zi83:1a}, \REF{ex:zi83:2a}, \REF{ex:zi83:3a} und \REF{ex:zi83:3b} als referentiell sy\-no\-ny\-me Ausdrucksalternativen der entsprechenden Äußerungen mit den substantivischen Sachverhaltsbezeichnungen angesehen werden.

Nicht selten allerdings machen Sprecher im Aufbau ihrer Äußerungen falsche Voraussetzungen bezüglich des aktuellen Wissens und der Überzeugungen der Kommunikationspartner. So kann es dazu kommen, daß Sachverhalte zeitlich nicht eingeordnet werden können und auch hinsichtlich ihres Wirklich-Seins mindestens nicht entsprechend der vom Sprecher eingenommenen Einstellung festgelegt werden können. Hier ist der besonders in stilistischen Arbeiten sehr häufig zu global erhobene Vorwurf gegen die Verwendung substantivischer Sachverhaltsbezeichnungen am Platze. Die durch sprachliche Äußerungen in kommunikative Interaktionszusammenhänge eingebrachten Sachverhalte müssen -- um im praktischen und theoretischen Handeln berücksichtigt werden zu können -- in das im Bewußtsein der Kommunikationspartner repräsentierte Weltbild eingeordnet werden können.

Es sei also folgendes festgehalten: Substantivische Satzentsprechungen sind in ihrer signifikativen Bedeutung sehr häufig hinsichtlich des Zeit- und Rea\-li\-täts\-be\-zugs der bezeichneten Sachverhalte unspezifiziert und sind dann in ihrer referentiellen Bedeutung weniger festgelegt als vergleichbare Nebensätze.

Es soll nun noch an einigen Beispielen demonstriert werden, wie der sprachliche Kontext zur referentiellen Interpretation substantivischer Sachverhaltsbezeichnungen beiträgt.


Im Gegensatz zu Verben wie \textit{somnevat’sja} (‘zweifeln’) im Beispiel \REF{ex:zi83:2}, Abschnitt \ref{sec:zi83:1.1}, das eine epistemische propositionale Einstellung seines Subjekts bezeichnet und für sein propositionales Objekt keinerlei Festlegungen für die zeitliche Einordnung des betreffenden Sachverhalts beinhaltet, implizieren Verben wie \textit{prosit’} (‘bitten’) und \textit{ždat’} (‘warten’) Nachzeitigkeit des durch ihr propositionales Objekt bezeichneten Sachverhalts relativ zum Referenzpunkt des Tempus im einbettenden Satz.\footnote{Zu den Regularitäten der consecutio temporum s. \citet{steube1980die-consecutio-temporum-im-deutschen-als-ausdruck-eines-komplexen-zuordnungsverhaltnisses-von-zeichenfolge-und-bedeutungsstruktur}.} Vgl.:

\ea \label{ex:zi83:21}
    \ea \label{ex:zi83:21a}
        \gll „Meždunarodnaja kniga“ prosit knigotorgovye firmy o tom, čtoby oni nemedlenno soobščili svoi zakazy. \\
         „Internationales Buch“ bittet  buchhandelnde Firmen um jenes dass.\textsc{cond} sie unverzüglich mitteilen ihre Bestellungen \\
         \glt ‘Der Verlag \textit{Meždunarodnaja kniga} bittet die Buchvertriebsfirmen, ihre Bestellungen unverzüglich aufzugeben.’
        
    \ex \label{ex:zi83:21b}
        \gll „Meždunarodnaja kniga“ prosit knigotorgovye firmy o nemedlennom soobščenii svoich zakazov. \\
         „Internationales Buch“ bittet buchhandelnde Firmen um unverzügliche Mitteilung ihrer Bestellungen.\textsc{gen} \\
         \glt ‘Der Verlag \textit{Meždunarodnaja kniga} bittet die Buchvertriebsfirmen um die unverzügliche Aufgabe ihrer Bestellungen.’
        
    \z
\ex \label{ex:zi83:22}
    \ea \label{ex:zi83:22a}
    \gll Mat’ ždet togo, čto rebenok vozvratitsja. \\
     Mutter wartet jenes.\textsc{gen} dass  Kind zurückkehrt \\
     \glt ‘Die Mutter wartet darauf, dass ihr Kind zurückkommt.’
    
    \ex \label{ex:zi83:22b}
    \gll Mat’ ždet togo, čtoby rebenok vozvratilsja. \\
    Mutter wartet jenes.\textsc{gen} dass.\textsc{cond}  Kind zurückkehr.\textsc{l-ptcp}.\textsc{sg}.\textsc{m}.\textsc{refl} \\
    \glt ‘Die Mutter wartet darauf, dass ihr Kind zurückkommen möge.’
    
    \ex \label{ex:zi83:22c}
    \gll Mat’ ždet vozvraščenija rebenka. \\
    Mutter wartet Rückkehr.\textsc{gen}  Kind.\textsc{gen} \\
    \glt ‘Die Mutter erwartet die Rückkehr ihres Kindes.’
    
    \z
\z

\largerpage[-1]
\noindent Zugleich beinhaltet die Bedeutung dieser jeweils ein propositionales Argument einbettenden Verben, daß der betreffende Sachverhalt als noch nicht bestehender, erst noch zu realisierender ‘Gegenstand’ verstanden wird. Der Konjunktiv des Nebensatzes in \REF{ex:zi83:21a} und \REF{ex:zi83:22b} resp. das futurische Tempus des Nebensatzes in \REF{ex:zi83:22a} bringt dies zum Ausdruck. Die entsprechenden Syntagmen mit dem Verbalabstraktum als Kern besagen selbst nichts über diese temporalen und modalen Charakteristika. Sie ergeben sich aus dem unmittelbaren sprachlichen Kontext, der Bedeutung des einbettenden Prädikatworts. Die in \REF{ex:zi83:22b} durch den Konjunktiv ausgedrückte optativische Modalität kommt unabhängig von dem einbettenden Verb als zusätzliche modale Bedeutungskomponente des Nebensatzes zustande. Wenn die substantivische Sachverhaltsbezeichnung in \REF{ex:zi83:22c} als referentiell synonyme Ausdrucksalternative zu dem Nebensatz in \REF{ex:zi83:22a} und in \REF{ex:zi83:22b} angesehen werden soll, ist im letzteren Fall diese zusätzliche modale Spezifik aus dem Kontext jenseits des betreffenden Satzes zu erschließen.\footnote{Es sei vermerkt, daß die im Beispielsatz \REF{ex:zi83:22c} enthaltene substantivische Satzentsprechung nicht nur hinsichtlich der in dem Nebensatz in \REF{ex:zi83:22a} und \REF{ex:zi83:22b} explizierten Modalitätsunterschiede mehrdeutig ist, sondern auch dadurch, daß dem Verbalnomen \textit{vozvraščenie} ein reflexives Verb mit der Bedeutung ‘zurückkehren’, ‘zurückkommen’ resp. das entsprechende kausative transitive Verb mit der Bedeutung ‘zurückgeben’, ‘zurückführen’ entspricht. Diese Mehrdeutigkeit von sehr vielen Verbalnomen liegt auf der Ebene ihrer signifikativen Bedeutung; ich habe dieses interessante Kapitel hier nicht zur
Sprache bringen können.}

Ebenso sind der Zeit- und Realitätsbezug des durch die Substantivgruppe \textit{Vaš priezd} (‘Ihr Kommen’) in der Satzfolge \REF{ex:zi83:23} ausgedrückten Sachverhalts nur aus dem durch den Aufforderungssatz \textit{Priezžajte sjuda} (‘Kommen Sie her’) gegebenen weiteren Kontext zu ermitteln. Aufforderungssätze implizieren, daß der betreffende Sachverhalt in Zukunft erst zu realisieren ist. Diese referentielle Interpretation vererbt sich auf die ko\-re\-fe\-ren\-te substantivische Sachverhaltsbezeichnung im nachfolgenden Text.

\ea \label{ex:zi83:23}
    \gll Priezžajte sjuda, do pory, poka škola ešče ne končilas’, posmotrite na rabočix, pogovorite s nimi.  Malo  ix. Da, no oni stojat Vašego priezda.  \\
    komm.\textsc{imp}.2\textsc{pl}  hierher bis  Zeit solange Schule noch nicht endete schau.\textsc{imp}.2\textsc{pl} auf  Arbeiter sprech.\textsc{imp}.2\textsc{pl}  mit ihnen wenig sie.\textsc{gen}.3\textsc{pl}  ja aber sie verdienen Ihr Kommen.\textsc{gen} \\
    \glt ‘Kommen Sie her, solange die Schulzeit noch nicht vorbei ist, schauen Sie auf die Arbeiter, sprechen Sie mit ihnen. Es sind nur wenige. Aber sie verdienen ja, dass Sie kommen.’ (\textit{A. M. Gor’kij})
\z

%\noindent absatz nach oben verschoben

%Ebenso sind der Zeitund Realitätsbezug des durch die Substantivgruppe vaš priezd (,Ihr Kommen‘) in der Satzfolge \REF{ex:zi83:23}

%\ea \label{ex:zi83:23}
%    \gll Priezžajte sjuda, do pory, poka škola ešče ne končilas’, posmotrite na rabočich, pogovorite s nimi. Malo {} ich. Da, no oni stojat Vašego priezda. \emph{(A. M. Gor’kij)} \\
%    {Kommen Sie} her vor {(der) Zeit} da {(die) Schule} noch {} {geendet hat} {schauen Sie} an {(die) Arbeiter} {sprechen Sie} mit ihnen wenige sind es ja aber sie verdienen Ihr Kommen \\
%\z

%ausgedrückten Sachverhalts nur aus dem durch den Aufforderungssatz \textit{Priezžajte sjuda} (,Kommen Sie her‘) gegebenen weiteren Kontext zu ermitteln. Aufforderungssätze implizieren, daß der betreffende Sachverhalt in Zukunft erst zu realisieren ist. Diese referentielle Interpretation vererbt sich auf die koreferente substantivische Sachverhaltsbezeichnung im nachfolgenden Text.

\noindent In vielen Fällen ist für die referentielle Interpretation substantivischer Sachverhaltsbezeichnungen entscheidend, daß der Sprecher den betreffenden Sachverhalt als existent ansieht. Es gibt einbettende Prädikatwörter wie \textit{soznavat’} (‘sich bewußt sein’) im Beispielsatz \REF{ex:zi83:24}, die diese propositionale Einstellung im\-pli\-zie\-ren. (Siehe dazu unsere ausführlichen Darlegungen im folgenden Abschnitt.)

\ea \label{ex:zi83:24}
    \gll Ja očen’ soznaju svoju nepodgotovlennost’ k ėtoj oblasti, mešajuščuju mne vystupat’ publično. \\
    ich sehr  bewusst mein Unvorbereitetsein.\textsc{acc}.\textsc{sg}.\textsc{f} zu diesem Gebiet  störend.\textsc{acc}.\textsc{sg}.\textsc{f} mir.\textsc{dat} auftreten.\textsc{infv} öffentlich \\
    \glt ‘Ich bin mir sehr bewusst, dass ich hierfür nicht vorbereitet bin. Das hindert mich, öffentlich aufzutreten.’ (\textit{V. I. Lenin})
\z

%In vielen Fällen ist für die referentielle Interpretation substantivischer Sachverhaltsbezeichnungen entscheidend, daß der Sprecher den betreffenden Sachverhalt als existent ansieht. Es gibt einbettende Prädikatwörter wie \textit{soznavat’} (,sich bewußt sein’) im Beispielsatz \REF{ex:zi83:24},

%\ea \label{ex:zi83:24}
%    \gll Ja očen’ soznaju svoju nepodgotovlennost’ k ėtoi oblasti, mesajuščuju mne vystupat’ publično. \emph{(V. I. Lenin)} \\
%    ich sehr {mir bewußt bin} meines Unvorbereitetseins auf dieses Gebiet {(das) hindert} mich aufzutreten öffentlich \\
%\z

%die diese propositionale Einstellung implizieren. (Siehe dazu unsere ausführlichen Darlegungen im folgenden Abschnitt.)

\noindent Die im Beispiel \REF{ex:zi83:25} enthaltene substantivische Sachverhaltsbezeichnung ist re\-fe\-ren\-ti\-ell mehrdeutig. Sie kann einem Nebensatz entsprechen, der mit der Konjunktion \textit{čto} (‘daß’) eingeleitet ist und die Einstellung des Sprechers impliziert, daß der betreffende Sachverhalt besteht, oder einem Nebensatz, der die Fragepartikel \textit{li} (‘ob’) aufweist, so daß die für Interrogativsätze charakteristische Moda\-lität bestünde, daß in einer gegebenen Situation unentschieden ist, ob der betreffende Sachverhalt besteht. Wie das Beispiel zeigt, kann eine Entscheidung zwischen diesen Interpretationsmöglichkeiten nur auf der Basis entsprechender Kenntnisse über den vorliegenden Sachverhalt bzw. über das vom Autoren der betreffenden Äußerung Gemeinte getroffen werden.\footnote{Obwohl für Lenin selbstverständlich außer Frage stand, daß N. A. Semaschko der SDAPR angehörte, kann er in seinem Brief an Gorki dennoch gemeint haben, daß Martow in seiner Verlautbarung die in der damaligen Situation wichtige Frage der Zugehörigkeit Semaschkos zur SDAPR geflissentlich offen gelassen hat.}

\ea \label{ex:zi83:25}
    \gll L. Martov pomestil v bernskoj s.-d. gazete „zajavlenie“, gde govorit, čto Semaško ne byl delegatom na Štutgartskom kongresse, a prosto žurnalistom. Ni slova o ego prinadležnosti k s.-d. partii.  \\
    L. Martov platzierte in  Berner sozialdemokratischen Zeitung  „Erklärung“ wo sagt.3\textsc{sg} dass Semaschko nicht war  Delegierter.\textsc{ins} auf Stuttgarter Kongress sondern einfach Journalist.\textsc{ins} kein Wort über seine Zugehörigkeit zur sozialdemokratischen Partei \\
    \glt ‘L. Martov platzierte in der Berner sozialdemokratischen Zeitung eine „Erklärung“, in der er sagt, dass Semaschko auf dem Stuttgarter Kongress kein Delegierter war, sondern nur als Journalist teilnahm. Kein Wort über seine Mitgliedschaft in der sozialdemokratischen Partei.’ (\textit{V. I. Lenin})

\z

\noindent Ohne diese Bezugnahme auf das Wissen und die Überzeugungen des Spre\-chers resp. der eine sprachliche Äußerung interpretierenden Person ist die Festlegung der referentiellen Bedeutung sprachlicher Ausdrücke nicht möglich. Für substantivische Sachverhaltsbezeichnungen schließt dies die Suche nach einem bestimmten Zeitbezug und nach einer Entscheidung bezüglich der Existenz des betreffenden Sachverhalts notwendigerweise ein.

In dem folgenden Abschnitt wird das mit der letzten Frage engstens zusammenhängende Problem, wann Sachverhalte die Geltung von Tatsachen haben, gründlich beleuchtet.

\section{Die Tatsachengeltung von Sachverhalten}\label{sec:zi83:3}

\subsection{Problemstellung und Einordnung}\label{sec:zi83:3.1}

Im Mittelpunkt der folgenden Betrachtungen stehen Sätze und ihnen entsprechende Substantivgruppen, deren referentielle Bedeutung Sachverhalte sind, die für den Sprecher als Tatsachen, d. h. als zu einem bestimmten Zeitpunkt rea\-li\-sier\-te Sachverhalte, gelten. Insbesondere geht es um den Beitrag des Nomens \textit{Tatsache} (russisch: \textit{fakt}) zur Bedeutung sprachlicher Äußerungen.\footnote{Als Beobachtungsbasis und Überprüfungsinstanz dienten mir die deutsche und die russische Sprache der Gegenwart. Ich illustriere meine Erörterungen hier vornehmlich an rus\-sisch\-spra\-chi\-gen Beispielen. Alle im Text gemachten Aussagen beziehen sich gleichermaßen auf \textit{Tatsache} und auf \textit{fakt}.} Es soll gezeigt werden, welche Beziehung zwischen der ausdrücklichen Klassifizierung eines Sachverhalts als Tatsache und der in Behauptungen und Präsuppositionen vorliegenden epistemischen propositionalen Einstellung des Sprechers besteht.\footnote{Siehe \citet{zimmermann1976der-standpunkt-des-sprechers-bezuglich-der-wahrheit-der-mitteilung, Zimmermann82Explizite-und}.} Dabei werden Sachverhaltsbezeichnungen verschiedener Form, Hauptsätze wie \REF{ex:zi83:26}, Nebensätze wie in den Satzgefügen \REF{ex:zi83:27a}, \REF{ex:zi83:28a} und \REF{ex:zi83:29a} und sub\-stan\-ti\-vi\-sche Entsprechungen von Nebensätzen wie in \REF{ex:zi83:27b}, \REF{ex:zi83:28b} und \REF{ex:zi83:29b}, zueinander ins Verhältnis gesetzt und hinsichtlich ihrer Bedeutungsäquivalenz verglichen.

\ea \label{ex:zi83:26}
    \gll Petr často otsutstvoval. \\
    Peter häufig  fehlte \\
    \glt ‘Peter hat häufig gefehlt.’
    
\ex \label{ex:zi83:27}
    \ea \label{ex:zi83:27a}
        \gll Čto Petr často otsutstvoval, ėto — fakt. \\
        dass Peter häufig fehlte das {}  Tatsache \\
        \glt ‘Dass Peter häufig gefehlt hat, das ist eine Tatsache.’
        
    \ex \label{ex:zi83:27b}
        \gll Častoe otsutstvie Petra — ėto fakt. \\
        häufiges Fehlen Peter.\textsc{gen} {} das  Tatsache \\
        \glt ‘Das häufige Fehlen von Peter ist eine Tatsache.’
        
    \z

\ex \label{ex:zi83:28}
    \ea \label{ex:zi83:28a}
        \gll Učitelja v svoej rabote učityvajut tot fakt, čto Petr často otsutstvoval. \\
        Lehrer.\textsc{pl} in ihrer Arbeit berücksichtigen jene Tatsache dass Peter häufig fehlte \\
        \glt ‘Die Lehrer berücksichtigen in ihrer Arbeit die Tatsache, dass Peter häufig gefehlt hat.’
        
    \ex \label{ex:zi83:28b}
        \gll Učitelja v svoej rabote učityvajut fakt častogo otsutstvija Petra. \\
         Lehrer.\textsc{pl} in ihrer Arbeit berücksichtigen Tatsache häufiges Fehlen.\textsc{gen} Peter.\textsc{gen} \\
         \glt ‘Die Lehrer berücksichtigen in ihrer Arbeit die Tatsache des häufigen Fehlens von Peter.’
        
    \z

\ex \label{ex:zi83:29}
    \ea \label{ex:zi83:29a}
        \gll Učitelja v svoej rabote učityvajut, čto Petr často otsutstvoval. \\
        Lehrer.\textsc{pl} in ihrer Arbeit berücksichtigen dass Peter häufig fehlte \\
        \glt ‘Die Lehrer berücksichtigen in ihrer Arbeit, dass Peter häufig gefehlt hat.’
        
    \ex \label{ex:zi83:29b}
        \gll Učitelja v svoej rabote učityvajut častoe otsutstvie Petra. \\
        Lehrer.\textsc{pl} in ihrer Arbeit berücksichtigen  häufiges Fehlen Peter.\textsc{gen} \\
        \glt ‘Die Lehrer berücksichtigen in ihrer Arbeit das häufige Fehlen von Peter.’
    
    \z
\z

\noindent Ausgangs- und Orientierungspunkt ist der Aufsatz „Fact“ von \citet{kiparsky1970fact}, deren Grundannahmen über den Zusammenhang von Semantik und Syntax eingebetteter Propositionen sich zwar in einigen wesentlichen Punkten als strittig und revisionsbedürftig erwiesen haben und auch bezüglich der für das Auftreten des Nomens \textit{Tatsache} typischen Kontexte weiter präzisiert wurden,\footnote{Es sei verwiesen auf \citet{poldauf1972fact-and-non-fact, poldauf1972factive-implicative-evaluative-predicates, poldauf1976fact-non-fact-and-the-place-of-phasal-and-some-other-expressions}, auf \citet{wilkinson1970factive-complements-and-action-complements} und \citet{norrick1978factive-adjectives-and-the-theory-of-factivity} sowie auf \citet{reis1977prasuppositionen-und-syntax}.} die jedoch für das Verständnis von Fakt-Einbettungen das Fundament gelegt haben.

\largerpage
Was das Wesen der signifikativen und der referentiellen Bedeutung sprachlicher Äußerungen anbelangt (siehe Abschnitt \ref{sec:zi83:2.1}), sind meine Überlegungen stark beeinflußt von Vorstellungen, die in der modelltheoretischen Semantik ent\-wi\-ckelt wurden,\footnote{Siehe u. a. \citet{montague1973the-proper-treatment-of-quantification-in-ordinary-english}, \citet{lewis1970general-semantics} und \citet{cresswell1973logics-and-languages}.} und von den fundierenden Ideen, die Frege zur Unterscheidung von Sinn und Bedeutung sprachlicher Zeichen beigetragen hat.\footnote{Siehe vor allem \citet{frege1892uber-sinn-und-bedeutung, frege1918der-gedanke.-eine-logische-untersuchung-in:-beitrage-zur-philosophie-des-deutschen-idealismus-i-1918/1919} sowie die Schriften aus seinem Nachlaß, insbesondere \citet{frege1897logik, frege1906einleitung-in-die-logik}. \\ Bezüglich der Würdigung und wissenschafts- sowie philosophiegeschichtlichen Einordnung Freges sei auf \citet{birjukov1959o-robotach-g.-frage-po-filosofskim-voprosam-matematiki, birjukov1960teorija-smysla-gotloba-frege, birjukov1965o-vzgljadach-g.-frege-na-rol-znakov-i-iscislenija-v-poznanii}, \citet{rieske1968philosophische-aspekte-der-fregeschen-begrundung-der-modernen-logik.-zur-120.-wiederkehr-des-geburtstages-von-g.-frege}, \citet{kreiser1973geschichte-und-logisch-semantische-probleme-des-wissenschaftlichen-werkes-freges} und auf \citet{kozlova1972filosofija-i-jazyk-kriticeskij-analiz-nekotorych-tendencij-evoljucii-pozitivizma-xx-v.} hingewiesen. Siehe auch die umfangreiche Bibliographie der Schriften Freges und der Schriften über ihn in \citet[270-298]{frege1973schriften-zur-logik.-aus-dem-nachlass-von-l.-kreiser}.} Unmittelbar anregend wirkten auch die von \citet{Bierwisch80Semantic-structure,bierwisch1974utterance-meaning-and-mental-states} vertretene Hypothese, daß die in den Sprachträgern intern repräsentierten Modelle der praktisch erfahrenen und kognitiv verarbeiteten Wirklichkeit den Denotatbereich sprach\-li\-cher Äußerungen bilden und als „mögliche Welten“ interpretierbar sind, sowie die Darlegungen von \citet{lang1979zum-status-der-satzadverbiale} zum Status der Satzadverbiale, die den Versuch enthalten, die Fregeschen Begriffsprägungen des Fassens eines Gedankens und der Anerkennung der Wahrheit eines Gedankens im Rahmen mo\-dell\-theo\-re\-ti\-scher und mit Blick auf die kognitive Psychologie interpretierter Semantikvorstellungen zu rekonstruieren.

Für die Klärung des Zusammenhangs von Präsuppositionen und Faktoren der kommunikativ-pragmatischen Aufgliederung sprachlicher Äußerungen in Topik und Fokus resp. in Thema und Rhema waren besonders die Arbeiten von \citet{kiefer1978factivity-in-hungarian, kiefer1978functional-sentence-perspective-and-presuppositions} und von \citet{Pasch78Topik-vs., pasch1983mechanismen-der-inhaltlichen-gliederung-von-satzen} hilfreich.

\subsection{Zum Begriff der Wahrheit}\label{sec:zi83:3.2}

Bezüglich des Begriffs der Wahrheit, der in unseren Erörterungen notwendigerweise eine wichtige Rolle spielen wird, sei hervorgehoben, daß Wahrheit im semantischen Sinne ausschließlich auf im Bewußtsein modellierte und als Abbilder der objektiven Realität existierende Weltzustände bezogen wird und immer an die urteilende Kraft der Sprachträger als erkennender Wesen gebunden ist. Die erkenntnistheoretische Fragestellung nach der Wahrheit als Übereinstimmungsrelation zwischen Sein und Bewußtsein liegt jenseits der Kompetenz der linguistischen Semantik.

Das hier Gesagte soll auf die spezifischen Aspekte aufmerksam machen, die die Fragestellung nach dem Wahrsein von Denkinhalten in der Semantik und in der marxistischen Erkenntnistheorie kennzeichnen, es zielt nicht auf eine absolute Grenzziehung zwischen beiden Disziplinen ab. Schon allein das, was eine Person auf Grund von Folgerungsbeziehungen zwischen Bewußtseinseinheiten für wahr halten muß, zusammen mit dem, was sie als wahr behauptet oder als wahr voraussetzt, betrifft in den Gesetzen des Denkens wirksame objektive Zusammenhänge und damit einen sehr wesentlichen Überschneidungsbereich, für den Erkenntnistheorie, Logik und Semantik gleichermaßen zuständig sind.

Ich nehme nun an, daß die im Bewußtsein einer Person repräsentierten und durch ihre urteilende Kraft als Tatsachen angesehenen Sachverhalte demjenigen Bereich ihres inneren Modells der Wirklichkeit angehören, der als Wissen dieser Person -- verstanden als System begründbarer Annahmen über die Welt -- charakterisiert werden kann und von ihren Vermutungen, Zweifeln, Hoffnungen, Wünschen usw. abgehoben ist. Ein solches Wissenssystem läßt sich auffassen als mögliche Welt, die die in den Äußerungen der betreffenden Person ausgedrückten und von ihr als wahr behaupteten resp. als wahr vorausgesetzten Propositionen und im Normalfall auch alle aus ihnen durch Folgerungsbeziehungen ableitbaren Propositionen erfüllt (wahr macht).

Das, was hier ‘Proposition’ heißt, entspricht etwa dem, was Frege ‘Sinn eines Satzes’ genannt hat, und ist zu verstehen als diejenige Komponente der signifikativen Bedeutung einer sprachlichen Äußerung, die die Art des Verweisens auf einen Sachverhalt spezifiziert, d. h. ein Kriterium liefert, mit dem sich ein ent\-spre\-chen\-der Sachverhalt ausgliedern (identifizieren) läßt.\footnote{In genau diesem Sinn charakterisiert \citet[618]{klix1971information-und-verhalten} Begriffsstrukturen als „Erkennungsvorschriften“.} Im Hinblick auf ihre Erfüllbarkeit ist eine Proposition $p$ auffaßbar als charakteristische Funktion $f_{K}(w)$, die die Klasse $K$ derjenigen möglichen Welten $w$ festlegt, die die Proposition erfüllen (wahr machen), d. h. bezüglich derer der durch die Proposition identifizierte Sachverhalt besteht. Kurz gesagt: $p$ definiert die Klasse $K = \{w|w \in p\}$.\footnote{Vgl. \citet{Materna76Propozice}, die auf der Basis einer modelltheoretischen Semantikkonzeption eine aus\-ge\-zei\-chne\-te Darstellung des Wesens der Proposition geben.}

Was nun das Urteil des Sprechers anbelangt, daß ein bestimmter Sachverhalt besteht, so bezieht es sich darauf, daß die diesen Sachverhalt identifizierende Proposition $p$ in der aktualen, zur Rede stehenden, durch den Sprecher geistig reproduzierten Welt $w_{a}$, erfüllt ist. Das heißt, es gilt die spezialisierte Erfüllungsrelation $w_{a} \in p$ (und damit $w_{a} \in \{w|w \in p\}$).


\subsection{Der Begriff der Tatsachengeltung}\label{sec:zi83:3.3}

Die das Bestehen von Sachverhalten betreffende Einstellung des Sprechers ist ihrem Wesen nach eine epistemische propositionale Einstellung und ließe sich ganz allgemein etwa mit „sicher sein, überzeugt sein“ umschreiben. Diese Behauptungssätzen innewohnende Bedeutungskomponente ist vergleichbar dem, was Frege ‘Urteil’ (oder ‘Urteilen’) nannte und -- offenbar auf dem Hintergrund seiner starken Orientierung auf die Objektivität der Wahrheit -- als ‘Anerkennung der Wahrheit eines Gedankens’ verstand.\footnote{Zum kritischen Verständnis der von Frege postulierten Objektivität der Gedanken s. \citet[XXII ff.]{kreiser1973geschichte-und-logisch-semantische-probleme-des-wissenschaftlichen-werkes-freges}. Zur Rekonstruktion der Fregeschen Begriffsprägungen ‘Sinn’ (‘Gedanke’), ‘Bedeutung’ (‘Wahrheit des Gedankens’) und ‘Urteil’ (‘Anerkennung der Wahrheit eines Gedankens’) im Rahmen der modelltheoretischen Semantik unter psychologischem Gesichtswinkel s. \citet{bierwisch1979satztyp-und-kognitive-einstellung, bierwisch1974utterance-meaning-and-mental-states}.}

Den subjektiven Charakter der betreffenden Position des Sprechers betonend, gehe ich davon aus, daß alle in sprachlichen Äußerungen als existierend behaupteten, als existierend vorausgesetzten und/oder explizit als Tatsachen klassifizierten Sachverhalte im Bewußtsein des jeweiligen Sprechers repräsentierte Sachzusammenhänge sind, die er als reale, nachweisbare Gegebenheiten, d. h. als Tatsachen, ansieht. In diesem Sinne spreche ich von ‘Tatsachengeltung’ eines Sachverhalts resp. von ‘Faktivität’ der entsprechenden Proposition.\footnote{Vgl. \citet{Zimmermann82Explizite-und}. Bezüglich des Begriffs der Tatsache wie auch des Begriffs des Sachverhalts verweise ich auf \citet{klaus1974philosophisches-worterbuch}, \citet{kondakov1971logiceskij-slovar-worterbuch-der-logik}, \citet{heitschm1969zur-bestimmung-eines-allgemeinen-begriffs-der-tatsache} und besonders auf \citet{vendler1967effects-results-and-consequences, vendler1967facts-and-events, vendler1970say-what-you-think, vendler1980telling-the-facts}, in denen analysiert wird, in welchen sprachlichen Kontexten eine Bezugnahme auf Tatsachen vorliegt, und immer wieder die drängende Frage auftaucht, in welchem Sinn eigentlich „facts are in the world“.}

Ich beschränke also den Begriff der Tatsachengeltung nicht auf die Einstellung des Sprechers, die für Behauptungssätze, die assertorische oder apodiktische (also nichtproblematische) Urteile\footnote{Eine ausführliche Charakterisierung der verschiedenen Urteilstypen gibt \citet{kondakov1971logiceskij-slovar-worterbuch-der-logik}.} ausdrücken, charakteristisch ist, sondern dehne ihn aus auf alle Fälle, wo ein in einer sprachlichen Äußerung benannter Sachverhalt durch den Sprecher als Tatsache aufgefaßt wird. Und das trifft ganz allgemein für den Bereich des Bewußtseins der Sprachträger zu, der ihr Wissen ausmacht, denn „eine Erkenntnis kommt dadurch zustande, daß ein Gedanke als wahr anerkannt wird“ \citet[227]{frege1924erkenntnisquellen-der-mathematik-und-der-mathematischen-naturwissenschaften} oder -- anders gesagt -- daß ein Sachverhalt als bestehend, d. h. als Tatsache, angesehen wird.

Es ist erwähnenswert, daß \citet[53]{frege1879begriffsschrift-eine-der-arithmetischen-nachgebildete-formelsprache-des-reinen-denkens-1879.} bei der Erläuterung seiner Begriffsschrift unterstreicht, daß es ihm in der Darstellung des Urteils nicht um die verschiedenen Möglichkeiten der inhaltlichen Aufgliederung von Sätzen in Subjekt und Prädikat geht, sondern für die Begriffsschrift allein der invariante „begriffliche Inhalt“ von Sätzen von Belang ist, „weil im Urteile hier nur das in Betracht kommt, was auf die möglichen Folgerungen Einfluß hat. Alles, was für eine richtige Schlußfolge nöthig ist, wird voll ausgedrückt…“.

Für unser Thema besonders aufschlußreich ist, daß Frege für den Ausdruck dessen, was der Urteilsstrich ‘—’ in seiner Begriffsschrift besagt, das Nomen \textit{Tatsache} in prädikativer Funktion in Erwägung zieht. Er schreibt: „Es läßt sich eine Sprache denken, in welcher der Satz: ,Archimedes kam bei der Eroberung von Syrakus um‘ in folgender Weise ausgedrückt würde: ,der gewaltsame Tod des Archimedes bei der Eroberung von Syrakus ist eine Thatsache‘. Hier kann man zwar auch, wenn man will, Subject und Prädicat unterscheiden, aber das Subject enthält den ganzen Inhalt, und das Prädicat hat nur den Zweck, diesen als Urteil hinzustellen. \textit{Eine solche Sprache würde nur ein einziges Prädicat für alle Urtheile haben, nämlich ,ist eine Thatsache’ ... Eine solche Sprache ist unsere Begriffssprache und das Zeichen |— ist ihr gemeinsames Prädicat für alle Urtheile.}“ (ebd., 53 f.)

Für die eingangs angeführten Beispielsätze \REF{ex:zi83:26}--\REF{ex:zi83:29b} kann nun folgende in unserem Zusammenhang wesentliche Übereinstimmung festgestellt werden: Sie bezeichnen einen seitens des Sprechers als Tatsache angesehenen Sachverhalt, nämlich daß Peter häufig gefehlt hat. Die Tatsachengeltung dieses Sachverhalts ist jedoch auf unterschiedliche Weise ausgedrückt. Es stellt sich die Frage, ob die vergleichbaren Äußerungen bzw. Äußerungsteile ihrem Sinn nach und/oder in ihrer referentiellen Bedeutung äquivalent sind. Die folgenden Überlegungen zum Status der Faktivität bewirkenden Faktoren in der Bedeutungsstruktur von Sätzen werden deutlich machen, inwieweit ich die aufgeworfene Frage mit Nein beantworte.

\subsection{Die Ausdrucksweise der Tatsachengeltung und ihre semantische Spezifik} \label{sec:zi83:3.4}

Worin besteht also die semantische Spezifik der Qualifizierung des in unseren Beispielsätzen bezeichneten Sachverhalts, daß Peter häufig gefehlt hat, als eines nach der Überzeugung des Sprechers existierenden Sachverhalts?

\subsubsection{Die einfache Behauptung} \label{sec:zi83:3.4.1}

Im Beispielsatz \REF{ex:zi83:26} ist das Bestehen des fraglichen Sachverhalts in elementarer Form behauptet, und zwar ist das Urteil des Sprechers ausgedrückt, daß die Charakteristik, häufig gefehlt zu haben, auf das als existent vorausgesetzte Individuum namens Peter zutrifft. Eine solche Zerlegung des propositionalen Gehalts von Sätzen in einen prädikativen, ergänzungsbedürftigen Teil (also eine Prädikatenkonstante) und in einen oder mehrere abgeschlossene Teile (Individuenkonstanten) liegt jedem Urteil zugrunde, durch das die Subsumtion der durch die Individuenkonstanten charakterisierten Gegenstände unter den durch die Prädikatenkonstante spezifizierten Begriff als zutreffend hingestellt wird.\footnote{Vgl. dazu \citet[35]{frege1891function-und-begriff.-vortrag-gehalten-in-der-sitzung-vom-9.-januar-1891-der-jenaischen-gesellschaft-fur-medizin-und-naturwissenschaft} und \citet[154 f.]{frege1914logik-in-der-mathematik} mit \citet[151 ff.]{katz1972semantic-theory} und \citet[59]{katz1977propositional-structure-and-illocutionary-force:-a-study-of-the-contribution-of-sentence-meaning-to-speech-acts}.}

Das, was die assertorische Modalität von Behauptungssätzen wie \REF{ex:zi83:26} -- und natürlich auch wie die übrigen im Abschnitt \ref{sec:zi83:3.1} angeführten Beispielsätze -- ausmacht, gehört zu dem Teil der semantischen Struktur der betreffenden Sätze, der den jeweiligen Satztyp festlegt, also Behauptungs-, Frage- und Aufforderungssätze unterscheidet. Die charakteristische Komponente assertorischer Modalität ist die das ‘Urteil’ des Sprechers beinhaltende epistemische propositionale Einstellung, daß er sicher ist, daß die Dinge sich so verhalten, wie sie durch den propositionalen Gehalt des jeweiligen Satzes dargestellt sind. Sie wird zusammen mit der kommunikativen Intention des Sprechers mit nichtlexikalischen Mitteln ausgedrückt, dem Indikativ und dem für Behauptungssätze charakteristischen Intonationsverlauf. Ihrem Wesen nach sind solche einen bestimmten Satztyp kennzeichnenden propositionalen Einstellungen spezifische modale Operatoren, die auf die referentielle Interpretation der betreffenden Sätze entscheidenden Einfluß haben\footnote{Der von \citet{lang1979zum-status-der-satzadverbiale} unterbreitete Vorschlag, in der semantischen Repräsentation von Sätzen Einstellungsoperatoren zu verankern, die Propositionen in einstellungsbewertete Komponenten der Äußerungsbedeutung überführen, trägt dem Stufenunterschied zwischen dem propositionalen Gehalt von Sätzen und den Faktoren der Satzmodalität Rechnung. \\ Vgl. auch \citet{bierwisch1979satztyp-und-kognitive-einstellung, Bierwisch80Semantic-structure}, \citet{Pasch82Illokutionare-Kraft} und \citet[Kap. 2 und 7]{jackendoff1972semantic-interpretation-in-generative-grammar.}, der zwi\-schen funktionaler und modaler Bedeutungsstrukturierung von Sätzen unterscheidet und zeigt, inwiefern die referentielle Bedeutung von Sätzen und Satzkonstituenten von bestimmten modalen Operatoren abhängig ist. Zur Behandlung der die verschiedenen Satztypen konstituierenden Modalitätsfaktoren werden allerdings nur einige Andeutungen gemacht (s. ebd., 3, 17, 318).} (siehe auch Abschnitt \ref{sec:zi83:2.1}).

\subsubsection{Die Tatsachenbehauptung} \label{sec:zi83:3.4.2}
\largerpage
In den Beispielsätzen: \REF{ex:zi83:27a} und \REF{ex:zi83:27b} bildet der in der Subjektphrase benannte Sachverhalt den Gegenstand, bezüglich dessen die Eigenschaft, eine Tatsache zu sein, als zutreffend ausgesagt wird. Der Gegenstand der Prädikation wird hier aufgefaßt als ein Ding, das die den betreffenden Sachverhalt ausmachende Spezifik als Eigenschaft hat. Es ist das, was Reichenbach %\citet{reichenbach1966elements-of-symbolic-logic} %REICHENBACH 
\textit{event-splitting} nennt und vom \textit{thing-splitting} als einer einen Sachverhalt in Gegenstand und Eigenschaft aufgliedernden Sehweise, wie sie im Satz \REF{ex:zi83:26} vorliegt, unterscheidet.\footnote{Siehe \citet[286 ff.]{reichenbach1966elements-of-symbolic-logic}.} \textit{Event-splitting} konstituiert einen besonderen Typ von Dingargument, auf das der gesamte propositionale Gehalt -- in unserem Fall der in der Subjektphrase der Sätze \REF{ex:zi83:27a} und \REF{ex:zi83:27b} ausgedrückten Proposition -- als komplexes Prädikat bezogen wird und bezüglich dessen Faktivität prädikativ aussagbar ist, eben als eine durch ein entsprechendes Lexem ausgedrückte Eigenschaft eines als Ding aufgefaßten Sachverhalts. Diese ganzheitliche Sehweise bezüglich eines Sachverhalts hat ihren charakteristischen Reflex in substantivischen Sachverhaltsbezeichnungen wie im Beispielsatz \REF{ex:zi83:27b}.

Der durch das Subjekt von Sätzen wie \REF{ex:zi83:27a} und \REF{ex:zi83:27b} beschriebene Gegenstand ist auch insofern besonderer Art, als seine Existenz nicht vorausgesetzt ist,\footnote{\textit{Tatsache} gehört wie auch \textit{wahr} zu den Lexemen, die implikative Prädikate ausdrücken. Die Existenz des durch das Argument dieser Prädikate charakterisierten Sachverhalts kann nicht unabhängig von Erfüllung (dem Wahrsein) der einbettenden, diese Prädikate einschließenden Proposition gegeben sein. Das heißt, der implikative Charakter dieser Prädikate schließt aus, daß die Existenz der durch ihre Argumente spezifizierten Sachverhalte vorausgesetzt ist. Zum Wesen implikativer Prädikate siehe
\citet{karttunen1971implicative-verbs, karttunen1972die-logik-englischer-pradikatkomplementkonstruktionen}.} sondern mit dem Prädikat definitiv als gegeben ausgesagt wird, als Einordnung des betreffenden Sachverhalts in die Klasse der Tatsachen. Nicht zuletzt dadurch, daß hier ein gedanklich erfaßter, hinsichtlich seines Bestehens zu be\-ur\-tei\-len\-der Sachverhalt und das betreffende Urteil des Sprechers einander in Subjekt- und in Prädikatfunktion gegenübergestellt sind und den propositionalen Gehalt dieser Sätze bilden, kommt der besondere Nachdruck zustande, mit dem der Sprecher hier ,einen gefaßten Gedanken als wahr anerkennt‘.

Ich betrachte die dem Nomen \textit{Tatsache}  (russisch: \textit{fakt}) entsprechende semantische Einheit als einstelliges Prädikat, das den durch sein Argument spe\-zi\-fi\-zier\-ten Sachverhalt als Wirklich-Seiendes, d. h. als Tatsache, klassifiziert. Dieses Prädikat fixiert das positive Resultat der Prüfung eines identifizierten Sachverhalts im Hinblick auf seine Existenz in der aktualen Welt. Wesentlich ist, daß in Tatsachenbehauptungen wie \REF{ex:zi83:27a} und \REF{ex:zi83:27b} anders als in einfachen Behauptungen wie \REF{ex:zi83:26} die Kennzeichnung eines Sachverhalts als Tatsache zur propositionalen Bedeutungskomponente der betreffenden Sätze gehört.

\subsubsection{Die Frage der semantischen Äquivalenz von einfachen Behauptungen und Tatsachenbehauptungen} \label{sec:zi83:3.4.3}
\largerpage

Was die aufgeworfene Frage anbelangt, inwieweit die hier verglichenen Sätze semantisch äquivalent sind, soll folgendes gesagt werden: Zweifellos ist es nicht möglich, einen der durch diese Sätze ausgedrückten Gedanken für wahr oder falsch anzusehen, ohne auch die anderen als wahr resp. falsch anzusehen, weil diese Sätze bezüglich des Bestehens des Basissachverhalts, daß Peter häufig ge\-fehlt hat, äquivalent sind. Mit dieser Eigenschaft begnügt sich die Logik. Eine auf die natürliche Sprache zugeschnittene Semantik muß jedoch die vorhandenen Bedeutungsunterschiede dieser Sätze ins Blickfeld nehmen.

Zunächst ist festzustellen, daß das Nomen \textit{Tatsache} (russisch: \textit{fakt}) dem Verb \textit{existieren} (russisch: \textit{suščestvovat’}) semantisch sehr ähnlich ist. Diese Lexeme drücken Metaprädikate aus, denen bestimmte Operationen auf einer Metaebene be\-züg\-lich des Vorhandenseins der in der Subjektphrase der betreffenden Sätze benannten oder beschriebenen Gegenstände resp. Sachverhalte in der aktualen Welt ent\-spre\-chen.\footnote{Zu den semantischen Besonderheiten von Existenzaussagen siehe \citet[205 ff., 264 ff.]{arutjunova1976predlozenie-i-ego-smysl:-logiko-semanticeskie-problemy} und \citet[25]{arutjunova1976referencija-imeni-i-struktura-predlozenija}.} Diese Metaprädikate implizieren, daß die Existenz der durch ihr Argument identifizierten Gegenstände (incl. Sachverhalte) irgendwie in Frage steht oder in Zweifel gezogen werden könnte, und sie beinhalten die definitive Festlegung, daß der betreffende Gegenstand in der aktualen Welt existent ist. Vgl.:

\ea \label{ex:zi83:30}
    \gll Spor ob universalijax suščestvuet. \\
    Streit um Universalien existiert \\
    \glt ‘Den Streit um Universalien gibt es.’
    
\ex \label{ex:zi83:31}
    \gll Spor ob universalijax est’ fakt. \\
    Streit um Universalien ist Tatsache \\
    \glt ‘Der Streit um Universalien ist eine Tatsache.’
    
\z

\noindent Eine solche Problematisierung der Tatsachengeltung eines Sachverhalts, wie sie für Sätze wie \REF{ex:zi83:30}, \REF{ex:zi83:31}, \REF{ex:zi83:27a} und \REF{ex:zi83:27b} typisch ist, liegt in Sätzen wie \REF{ex:zi83:26} nicht vor. In einfachen Aussagesätzen bleibt die Inbeziehungsetzung von Sach\-ver\-halts\-kon\-zep\-ten und den entsprechenden Gegebenheiten der aktualen Welt unausgesprochen, in Existenzaussagen resp. Tatsachenbehauptungen tritt sie -- signalisiert durch \textit{suščestvovat’} resp. \textit{fakt} -- offen zutage. Dadurch rückt auch der Sprecher als urteilendes Subjekt in den Vordergrund und Sätze wie \REF{ex:zi83:27a} und \REF{ex:zi83:27b} werden zu nachdrücklichen Urteilen und einem geeigneten Mittel der eindeutigen, jeden anderen Standpunkt ausschließenden Stellungnahme des Spre\-chers bezüglich des Bestehens eines bestimmten Sachverhalts in der aktualen Welt.

Diese Besonderheiten machen Äußerungen wie \REF{ex:zi83:27a} und \REF{ex:zi83:27b} dazu geeignet, als Metaurteile zu fungieren, d. h. unter bestimmten Umständen funktional in eine Reihe mit Äußerungen wie \REF{ex:zi83:32} zu treten, die, wie ich annehme, Metaurteile sind.\footnote{Zu Unterschieden zwischen den Prädikaten, die durch die Lexeme \textit{Tatsache} (russisch: \textit{fakt}) und \textit{wahr} (russisch: \textit{verno}) ausgedrückt werden, siehe \citet{Zimmermann82Explizite-und}. \\ Zu den Begriffen ‘Basisurteil’, ‘Metaurteil’ und entsprechend ‘Basissachverhalt’ und ‘Metasachverhalt’ s. \citet[415 f.]{povarov1960sobytijnyj-i-suzdenceskij-aspekty-logiki-v-svjazi-s-logiceskimi-zadacami-techniki}.}

\ea \label{ex:zi83:32}
    \gll Čto Petr často otsutstvoval, ėto verno. \\
    dass Peter häufig fehlte das wahr \\
    \glt ‘Dass Peter häufig gefehlt hat, ist wahr.’
\z

\noindent Das Substantiv \textit{Tatsache} hat als Prädikatsnomen ohne attributive Ergänzung auch viele Gemeinsamkeiten mit solchen Ausdrücken wie \textit{feststehen} (russisch: \textit{ne\-so\-mnen\-no}, \textit{bessporno}), die einen sehr hohen Grad von Gewißheit des Sprechers signalisieren, daß die Dinge sich so und nicht anders verhalten, wie sie durch den jeweiligen in Subjektposition figurierenden sprachlichen Ausdruck dargestellt sind, auf den sich diese Lexeme in Prädikatfunktion beziehen. Worin die Bedeutungsspezifik und die besonderen Verwendungsbedingungen dieser Wörter im Unterschied zu \textit{Tatsache} bestehen, bedarf der weiteren Klärung. Erwähnenswert ist, daß bei Anfangsstellung solcher die Tatsachengeltung von Sachverhalten betreffenden prädikativen Ausdrücke und bei nichttopikalischem Charakter des Subjekts die Nähe der betreffenden Satzgefüge zu einfachen Behauptungen wie \REF{ex:zi83:26} besonders groß ist.\footnote{Zum Begriff des Topiks s. \citet{Pasch78Topik-vs., pasch1983mechanismen-der-inhaltlichen-gliederung-von-satzen}.} Vgl.:

\ea \label{ex:zi83:33}
    \gll Fakt, čto Petr často otsutstvoval. \\
    Tatsache dass Peter häufig fehlte \\
    \glt ‘Es ist eine Tatsache, dass Peter häufig gefehlt hat.’
    
\ex \label{ex:zi83:34}
    \gll Bessporno, čto Petr často otsutstvoval. \\
    unstrittig dass Peter häufig fehlte \\
    \glt ‘Es ist unstrittig, dass Peter häufig gefehlt hat.’
    
\z

\noindent Trotzdem erscheint es mir nicht gerechtfertigt, solche komplexen Sätze mit den entsprechenden einfacheren Behauptungen ohne lexikalisch ausgedrückte Ge\-wiß\-heits\-mo\-da\-li\-tät semantisch gleichzusetzen. Sie sind auch zu unterscheiden von Sätzen wie %\REF{ex:zi83:33'} und \REF{ex:zi83:34'}
(33’) und (34’), in denen die Ausdrücke der Ge\-wiß\-heits\-mo\-da\-li\-tät als parenthetische Ausdrücke (Satzadverbiale) auftreten.

\begin{exe}
    \exp{ex:zi83:33} \label{ex:zi83:33'}
        \gll Fakt, Petr často otsutstvoval.\\
        Tatsache Peter häufig fehlte \\
        \glt ‘Es ist eine Tatsache: Peter hat häufig gefehlt.’
        
        
    \exp{ex:zi83:34} \label{ex:zi83:34'}
        \gll Bessporno, Petr často otsutstvoval. \\
        unstrittig Peter häufig fehlte \\
        \glt ‘Es ist unstrittig: Peter hat häufig gefehlt.’
\end{exe}

\noindent Es ist ein wesentlicher semantischer Unterschied, ob eine propositionale Einstellung des Sprechers durch ein Prädikatsnomen oder aber durch entsprechende Parenthesen (Satzadverbiale) resp. durch die verbalen Modi und die Satzintonation ausgedrückt ist. Im ersten Fall liegt eine in den propositionalen Gehalt der betreffenden Äußerungen gehörende Prädikatzuschreibung, d. h. eine be\-stimm\-te Klassifizierung eines ‘Gedankens’ resp. eines Sachverhalts, vor. Und als solche gehört sie zu denjenigen signifikativen Bedeutungskomponenten, die den durch die betreffende Äußerung bezeichneten Sachverhalt zu identifizieren gestatten. Im zweiten Fall ist die betreffende Einstellung des Sprechers das mit seiner Äußerung über seine Überzeugungen (oder auch emotionalen Zustände) An\-ge\-zeig\-te (oder: ‘Ausgedrückte’) und ist als solches nicht Bestandteil des propositionalen Gehalts: Sie dient nicht der Identifizierung des durch die betreffende Äußerung bezeichneten Sachverhalts. Sie gehört nicht zu dem mit dieser Äu\-ße\-rung ‘Ge\-sag\-tem’.\footnote{\citet{lang1983einstellungsausdrucke-und-ausgedruckte-einstellungen} unterscheidet prinzipiell das mit einer Äußerung ‘Gesagte’ von dem mit ihr ‘Ausgedrückten’ und differenziert in der Bedeutungsstruktur von Äußerungen zwischen propositional und nicht-propositional repräsentierten Einstellungen.}

Auf diesem Hintergrund vertrete ich die Ansicht, daß sich Tatsachenbehauptungen wie \REF{ex:zi83:27a} und \REF{ex:zi83:27b} von einfachen Behauptungen wie \REF{ex:zi83:26} sowohl in ihrer signifikativen und referentiellen Bedeutung wie auch in pragmatischer Hinsicht unterscheiden.

\subsubsection{Attributive Ergänzungen zu \textit{Tatsache}} \label{sec:zi83:3.4.4}

Auch bei attributiver Ergänzung des Prädikatnomens \textit{Tatsache} (russisch: \textit{fakt}) bleibt seine Eigenbedeutung, Sachverhalte als objektiv Gegebenes, Wirklich-Sei\-en\-des zu qualifizieren, erhalten. In Sätzen wie \REF{ex:zi83:35}--\REF{ex:zi83:38} ist \textit{fakt} nicht einfach weglaßbar oder durch solche Substantive wie \textit{obstojatel’stvo} (‘Umstand’) oder \textit{delo} (‘Sache’), denen die \textit{fakt} eigene modale Bedeutungskomponente fehlt, ersetzbar.

\ea \label{ex:zi83:35}
    \gll To, čto Petr často otsutstvoval, ėto — fakt, kotoryj učitelja dolžny učityvat’ v svoej rabote. \\
    das dass Peter häufig fehlte das {} Tatsache die Lehrer.\textsc{pl} müssen berücksichtigen in ihrer Arbeit \\
    \glt ‘Dass Peter häufig gefehlt hat, ist eine Tatsache, die die Lehrer in ihrer Arbeit berücksichtigen müssen.’
    
\ex \label{ex:zi83:36}
    \gll Postojannoe dviženie […] okružajuščego nas material’nogo mira — fakt, dannyj nam v čuvstvennom vosprijatii.  \\
    ständige Bewegung {} umgebender.\textsc{gen} uns materieller.\textsc{gen} Welt.\textsc{gen} {}  Tatsache gegeben uns in  sinnlicher Wahrnehmung  \\
    \glt ‘Die ständige Bewegung in der uns umgebenden materiellen Welt ist eine Tatsache, die wir mit unseren Sinnen erfahren.’ (\textit{D. P. Gorskij})
    
\ex \label{ex:zi83:37}
    \gll Suščestvovanie moe est’ fakt, {samyj nesomnennyj} i lično dlja menja črezvyčajno važnyj.  \\
    Existenz meine ist  Tatsache   unbezweifelbarste und persönlich für mich äußerst wichtige \\
    \glt ‘Meine Existenz ist eine Tatsache -- die unbezweifelbarste (von allen) und eine, die für mich persönlich äußerst wichtig ist.’ (\textit{A. N. Tolstoj})
    
\ex \label{ex:zi83:38}
    \gll To, čto slova v sostave predloženij priobretajut značenija, kotoryx u nix net vne predloženij, dostatočno očevidnyj i izvestnyj fakt. \\
    das dass  Wörter in Bestand Sätze.\textsc{gen} erwerben Bedeutungen die bei ihnen nicht  außerhalb Sätze.\textsc{gen} genügend offensichtliche und bekannte Tatsache \\
    \glt ‘Dass Wörter in Sätzen Bedeutungen erlangen, die sie außerhalb der Sätze nicht haben, ist eine hinreichend offensichtliche und bekannte Tatsache.’ (\textit{V. M. Solncev})

\z

\noindent Mindestens das das Urteil des Sprechers bekräftigende und seinen Wahrheits\-anspruch akzentuierende Moment, daß es sich bei dem jeweils im Subjekt benannten Sachverhalt um eine objektive, im Handeln und Urteilen in Rechnung zu stellende Gegebenheit handelt, ginge verloren, verzichtete man in solchen prädikativen Fügungen auf \textit{fakt}.

In solchen Sätzen mit einem attributiv ergänzten Prädikatsnomen bezieht sich dieses wie auch die Attribute auf den in der Subjektgruppe benannten ‘Gegenstand’. Die betreffenden Äußerungen drücken das ‘Urteil’ des Sprechers aus, daß dieser ‘Gegenstand’ in den durch das komplexe Prädikatsnomen charakterisierten Klassendurchschnitt fällt.

Daß das Substantiv \textit{Tatsache} ein semantisch und syntaktisch geeigneter Auf\-hän\-ger für verschiedene Spezifizierungen eines als bestehend betrachteten Sach\-ver\-halts ist, zeigt sich auch bei nichtprädikativer Verwendung dieses Nomens wie in den Beispielen \REF{ex:zi83:39}--\REF{ex:zi83:41}.

\ea \label{ex:zi83:39}
    \gll V sovremennoj nam dejstvitel’nosti prixoditsja sčitat’sja s tem nepreložnym faktom, čto mašina vxodit v žizn’ obščestva {tak že}, kak v svoe vremja vošla v nee pis’mennost’. \\
    in gegenwärtiger uns.\textsc{dat}  Wirklichkeit muss rechnen.\textsc{infv} mit der unwiderlegbaren Tatsache dass  Maschine eintritt in  Leben  Gesellschaft.\textsc{gen} ebenso wie zu ihrer Zeit eintrat.\textsc{pst} in es  Schrift \\
    \glt ‘In der jetzigen Wirklichkeit muss man die unwiderlegbare Tatsache berücksichtigen, dass die Maschine für die Gesellschaft die gleiche Bedeutung hat wie ehemals die Schrift.’ (\textit{V.A. Zvegincev})

\newpage
\ex \label{ex:zi83:40}
    \gll My ne možem ignorirovat’ udivitel’nyj fakt, čto sotrudnik vse vremja molčit. \\
    wir nicht können ignorieren  verwunderliche Tatsache dass  Mitarbeiter ganze Zeit schweigt \\
    \glt ‘Wir können die erstaunliche Tatsache nicht ignorieren, dass der Mitarbeiter die ganze Zeit schweigt.’
    
\ex \label{ex:zi83:41}
    \gll Nesomnennyj fakt častogo otsutstvija Petra ešče ničego ne govorit o ego uspevaemosti. \\
     unbezweifelbare Tatsache  häufiges Fehlen.\textsc{gen} Peter.\textsc{gen} noch nichts nicht sagt über seine Lernerfolge \\
    \glt ‘Die unzweifelhafte Tatsache, dass Peter häufig fehlt, sagt noch nichts über seine Leistungen.’

\z

\noindent Hier haben wir komplexe Terme vor uns, in denen das Nomen \textit{fakt} den syntakti\-schen Kern der entsprechenden Substantivgruppen bildet und ausdrückt, daß der Sprecher den in dem Attributsatz resp. in dessen substantivischem Äqui\-valent benannten Sachverhalt als existent voraussetzt. Damit kommen wir zum Ver\-gleich unserer Beispielsätze \REF{ex:zi83:28a} bis \REF{ex:zi83:29b}.

\subsubsection{Faktive Präsuppositionen} \label{sec:zi83:3.4.5}

Während der Sprecher mit den Äußerungen \REF{ex:zi83:26}, \REF{ex:zi83:27a} und \REF{ex:zi83:27b} das Bestehen des Sachverhalts, daß Peter häufig gefehlt hat, behauptet, setzt er es in Sätzen wie \REF{ex:zi83:28a} bis \REF{ex:zi83:29b} voraus. Das heißt, der Sprecher nimmt die Existenz des be\-treffenden Sachverhalts in der aktualen Welt als gegeben genauso, wie er -- bezogen auf unsere Beispielsätze -- die Existenz von bestimmten Personen, die als Lehrer tätig sind, als gegeben nimmt. Die Annahme der Existenz dieser beiden ‘Gegenstände’ in der aktualen Welt ist Voraussetzung für das der Behauptung zugrunde liegende ‘Urteil’, daß die in den Sätzen \REF{ex:zi83:28a} bis \REF{ex:zi83:29b} genannte Beziehung zwischen diesen ‘Gegenständen’ in der aktualen Welt besteht.\footnote{Vgl. \citet[128 ff.]{katz1972semantic-theory} und \citet[88]{katz1977propositional-structure-and-illocutionary-force:-a-study-of-the-contribution-of-sentence-meaning-to-speech-acts}. \citet[47]{frege1892uber-sinn-und-bedeutung} sagt über die vor\-aus\-ge\-setz\-te Existenz der ‘Gegenstände’ der Prädikation, die für ihn die ‘Bedeutung’ der betreffenden Satzkomponenten ausmachen: „Wer eine Bedeutung nicht anerkennt, der kann ihr ein Prädikat weder zu- noch absprechen.“ Siehe auch ebd., 54 f., wo es heißt: „Wenn man etwas behauptet, so ist immer die Voraussetzung selbstverständlich, daß die einfachen oder zusammengesetzten Eigennamen eine Bedeutung haben.“ 

\noindent Und sowohl in Behauptungen wie auch in anderen Sprechakttypen sind die Voraussetzungen unabhängig von (insensitiv gegenüber) der jeweiligen illokutiven Kraft und spezifizieren Gelingensbedingungen für die betreffende kommunikative Handlung. Vgl. dazu \citet[97]{fillmore1969verbs-of-judging:-an-exercise-in-semantic-description}.} Das Erfülltsein der durch den Nebensatz resp. seine substantivische Entsprechung der Beispiele \REF{ex:zi83:28a}--\REF{ex:zi83:29b} ausgedrückten Proposition in der im Bewußtsein des Sprechers repräsentierten aktualen Welt $w_{a}$, ist also eine Voraussetzung für das Erfülltsein der durch die jeweiligen Satzgefüge ausgedrückten komplexen Proposition in der Welt $w_{a}$. Das ist das Kernstück dessen, was \citet{kiparsky1970fact} unter Faktivität verstehen, wenn sie sagen: „The speaker presupposes that the embedded clause expresses a true proposition, and makes some assertion about that proposition.“ (ebd., 147; vgl. auch 143, 148, 155, 162 f, 172.)

Das Nomen \textit{fakt} drückt in solchen Sätzen wie \REF{ex:zi83:28a} und \REF{ex:zi83:28b} oder auch wie \REF{ex:zi83:39}, \REF{ex:zi83:40} und \REF{ex:zi83:41} das Erfülltsein der jeweiligen eingebetteten Proposition aus, es zeigt an, daß der durch die betreffende Proposition identifizierte Sachverhalt im Bewußtsein des Sprechers als existierend registriert ist.

In Sätzen wie \REF{ex:zi83:28a}, \REF{ex:zi83:28b}, \REF{ex:zi83:39}, \REF{ex:zi83:40} und \REF{ex:zi83:41} liegt explizite Faktivität vor, während wir es in Sätzen wie \REF{ex:zi83:29a} und \REF{ex:zi83:29b} mit impliziter Faktivität der eingebetteten Pro\-po\-sition zu tun haben. Sie ergibt sich im gegebenen Kontext aus der Bedeutung des faktive Propositionen einbettenden (kurz: faktiven) Verbs \textit{učityvat’} (‘be\-rück\-sich\-tigen’). Die in den Beispielen \REF{ex:zi83:28a} und \REF{ex:zi83:28b} durch \textit{fakt} signalisierte Information ist somit redundant.

Anders verhält es sich in Satzgefügen wie \REF{ex:zi83:42} und \REF{ex:zi83:43}.

\ea \label{ex:zi83:42}
    \gll {Ėkzamenacionnaja komissija} v svojej ocenke podčerkivaet, čto Petr často otsutstvoval. \\
     Prüfungskommission in ihrer Beurteilung unterstreicht dass Peter häufig fehlte.\textsc{pst} \\
    \glt ‘Die Prüfungskommission unterstrich in ihrer Beurteilung, dass Peter häufig gefehlt hat.’
    
\ex \label{ex:zi83:43}
    \gll {Ėkzamenacionnaja komissija} v svojej ocenke ukazyvaet na to, čto Petr často otsutstvoval. \\
    Prüfungskommission in ihrer Beurteilung hinweist auf das.\textsc{acc} dass Peter häufig fehlte.\textsc{pst} \\
    \glt ‘Die Prüfungskommission weist in ihrer Beurteilung darauf hin, dass Peter häufig gefehlt hat.’
\z

\noindent Hier ist der Sprecher hinsichtlich der Tatsachengeltung des durch die eingebettete Proposition identifizierten Sachverhalts nicht festgelegt. Er kann sich von der Meinung der Prüfungskommission, daß der genannte Sachverhalt besteht, distanzieren, indem er die
jeweilige Äußerung z. B. mit der Feststellung \REF{ex:zi83:44} fortsetzt, ohne zu sich selbst in Widerspruch zu geraten.

\ea \label{ex:zi83:44}
    \gll No neverno, čto Petr často otsutstvoval. \\
    aber unwahr dass Peter häufig fehlte.\textsc{pst} \\
    \glt ‘Aber es ist nicht wahr, dass Peter häufig gefehlt hat.’
    
\z

\noindent Ein solcher Widerspruch entstünde allerdings, wenn der Behauptung \REF{ex:zi83:44} die Äußerung \REF{ex:zi83:45} oder \REF{ex:zi83:46} vorausginge.

\ea \label{ex:zi83:45}
    \gll {Ėkzamenacionnaja komissija} v svojej ocenke podčerkivaet tot fakt, čto Petr často otsutstvoval. \\
    Prüfungskommission in ihrer Beurteilung unterstreicht jene Tatsache dass Peter häufig fehlte.\textsc{pst} \\
    \glt ‘Die Prüfungskommission unterstreicht in ihrer Beurteilung die Tatsache, dass Peter häufig gefehlt hat.’

\ex \label{ex:zi83:46}
    \gll {Ėkzamenacionnaja komissija} v svojej ocenke ukazyvaet na tot fakt, čto Petr často otsutstvoval. \\
    Prüfungskommission in ihrer Beurteilung hinweist auf jene Tatsache dass Peter häufig fehlte.\textsc{pst} \\
    \glt ‘Die Prüfungskommission weist in ihrer Beurteilung auf die Tatsache hin, dass Peter häufig gefehlt hat.’
\z

\noindent In den Äußerungen \REF{ex:zi83:45} und \REF{ex:zi83:46} macht der Sprecher durch Verwendung des Nomens \textit{fakt} deutlich, daß er sich mit der Meinung der Prüfungskommission identifiziert, daß der Sachverhalt, daß Peter häufig gefehlt hat, eine objektive Gegebenheit ist. Hier ist das
Nomen \textit{fakt} also eine relevante Information über die Tatsachengeltung des betreffenden Sachverhalts für den Sprecher. Äußerungen wie \REF{ex:zi83:42} und \REF{ex:zi83:43} ohne diese Information sind referentiell mehrdeutig, sie lassen eine faktive (transparente) wie auch eine nichtfaktive (nichttransparente, opake) Lesart des Nebensatzes zu.\footnote{Zur Unterscheidung referentiell transparenter und referentiell nichttransparenter Kontexte vgl. \citet[261 ff.]{katz1972semantic-theory} und \citet[169]{seuren1975tussen-taal-en-denken:-een-bijdrage-tot-de-empirische-funderingen-van-de-semantiek.-deutsche-ubersetzung:-zwischen-sprache-und-denken:-ein-beitrag-zur-empirischen-begrundung-der-semantik}.}


Die durch Verben wie \textit{podčerkivat’} (‘unterstreichen’), \textit{ukazyvat’ na} (‘hinweisen auf’), \textit{ssylat’sja na} (‘Bezug nehmen auf’), \textit{isxodit’ iz} (‘ausgehen von’), \textit{upominat’} (‘erwähnen’), \textit{otmečat’} (‘hervorheben’) gebildeten Kontexte besagen nichts über die Faktivität der eingebetteten Proposition. Das heißt, die Bedeutung dieser Verben ist bezüglich der Faktivität der durch ihr Komplement ausgedrückten Proposition indifferent.\footnote{\citet[163]{kiparsky1970fact} bemerken zur Bedeutungsspezifik solcher Verben treffend: „On a deeper level, their semantic specifications include no specifications as to whether their complement sentences represent presuppositions by the speaker or not.“} Das Substantiv \textit{fakt} beseitigt die bestehende Mehrdeutigkeit und bringt den Standpunkt des Spre\-chers zum Ausdruck. Es zeigt an, daß der Sprecher den durch die eingebettete Proposition identifizierten Sachverhalt als existierend voraussetzt.

\largerpage
Im Gegensatz zu Prädikatausdrücken der Meinungsäußerung, des Glaubens und Schließens\footnote{Siehe dazu \citet{Zimmermann82Explizite-und}. 

\noindent Sehr aufschlußreich ist die Untersuchung von \citet{rivero1971mood-and-presupposition-in-spanish} am Spanischen zur Modalität eingebetteter Sätze. Die Autorin zeigt, daß die Wahl des Indikativs bzw. des Konjunktivs in Komplementsätzen von Verben wie \textit{creer} und \textit{parecer} bestimmt ist von der Position des Sprechers bezüglich der Wahrheit der durch den Komplementsatz ausgedrückten Proposition und daß Ausdrucksvariationen wie u. a. Negationshebung und Infinitivkonstruktionsbildung semantisch mit der konjunktivischen (nichtfaktiven) Satzeinbettung korrespondieren.} ist bei dem Verb \textit{wissen} (russisch: \textit{znat’}) der Sprecherstandpunkt bezüglich der Wahrheit der durch den Komplementsatz ausgedrückten Proposition immer mit im Spiel. Der Komplementsatz mit der Konjunktion \textit{daß} (russisch: \textit{čto}) bezeichnet einen Sachverhalt, der nach Meinung des Sprechers besteht bzw. eintreten wird\footnote{Inwieweit Propositionen, die zukünftige Sachverhalte identifizieren, Faktivität zuerkannt werden kann, ist eine nicht leicht zu beantwortende und meines Wissens nicht genau untersuchte Frage. Es scheint so, daß bei uneingeschränkter Gewißheit des Sprechers, daß der betreffende Sachverhalt eintreten wird, Faktivität gegeben ist und in entsprechenden Kontexten durch das Nomen \textit{Tatsache} expliziert werden kann.

\ea \label{ex:zi83:f53i} 
    \gll Direktor ne budet prisutstvovat’ na konferencii. Ėto -- fakt. On v komandirovke.\\
    Direktor nicht wird {anwesend sein} auf  Konferenz das {}  Tatsache er  auf  Dienstreise \\
    \glt ‘Der Direktor wird auf der Konferenz nicht anwesend sein. Das ist eine Tatsache. Er ist auf einer Dienstreise.’
    
\ex \label{ex:zi83:f53ii}
    \gll Fakt, čto direktor ne budet prisutstvovat’ na konferencii, izvesten ee organizatoram. \\
     Tatsache dass Direktor nicht wird {anwesend sein} auf  Konferenz bekannt ihren Organisatoren.\textsc{dat} \\
    \glt ‘Die Tatsache, dass der Direktor auf der Konferenz nicht anwesend sein wird, ist deren Organisatoren bekannt.’
\z

\noindent In diesem Zusammenhang sei angemerkt, daß das als faktiv geltende kognitive Verb \textit{wissen} (russisch: \textit{znat’}) ohne weiteres mit futurischen Komplementsätzen, die sogar die Gewißheit einschränkende Ausdrücke enthalten können, vereinbar ist.

\ea \label{ex:zi83:f53iii}
    \gll Učitel' znaet, čto Petr, verojatno, často budet otsutstvovat’. \\
     Lehrer weiß dass Peter wahrscheinlich häufig wird fehlen \\
    \glt ‘Der Lehrer weiß, dass Peter wahrscheinlich häufig fehlen wird.’
\z

\noindent In der philosophischen Literatur ist zu der hier aufgeworfenen Frage eine klare Antwort bei \citet[39]{wagner1974zur-marxistisch-leninistischen-wahrheitstheorie} zu finden, wo es heißt: „Mit ‘Bestehen’ eines Sachverhalts ist nicht nur dessen unmittelbar-gegenwärtige, aktuelle Existenz gemeint, auch Sachverhalte der Vergangenheit und Zukunft wollen wir ‘bestehende Sachverhalte’ nennen ...“.} so daß in Sätzen wie \REF{ex:zi83:47} eine faktive Satzeinbettung vorliegt.\footnote{Inwieweit es eine lexikalische Idiosynkrasie ist, daß \textit{wissen} im Unterschied zu \textit{bekannt sein} (russisch: \textit{izvestnyj}) keine über das Nomen \textit{Tatsache} vermittelte Satzeinbettung erlaubt, betrachte ich als offene Frage. Die Befunde über die Sprachgerechtheit solcher Sätze wie \REF{ex:zi83:f54i} sind schwankend.

\ea \label{ex:zi83:f54i}
    \gll {Ėkzamenacionnaja komissija} znaet tot fakt, čto Petr často otsutstvoval. \\
    Prüfungskommission kennt jene Tatsache dass Peter häufig fehlte.\textsc{pst} \\
    \glt ‘Die Prüfungskommission ist sich der Tatsache bewusst, dass Peter häufig gefehlt hat.’
\z
}

\ea \label{ex:zi83:47}
    \gll {Ėkzamenacionnaja komissija} znaet, čto Petr často otsutstvoval. \\
    Prüfungskommission weiß dass Peter häufig fehlte.\textsc{pst} \\
    \glt ‘Die Prüfungskommission weiß, dass Peter häufig gefehlt hat.’
\z

\noindent Eine überzeugende Bedeutungscharakterisierung für \textit{wissen} resp. \textit{znat’}, die auch Fragesatzeinbettungen berücksichtigt, steht noch aus.\footnote{Ich verweise auf Entwürfe, die von \citet[277 f.]{reichenbach1966elements-of-symbolic-logic} und \citet{vendler1980telling-the-facts} zur Bedeutungsanalyse von \textit{wissen} vorgelegt wurden, sowie auf einen Bericht von \citet{wuttich1976epistemische-logik} über Ansätze im Rahmen der epistemischen Logik. Siehe auch \citet{zybatow1983syntaktische-und-semantische-eigenschaften-der-komplementsatze-kognitiver-verben-des-modernen-russischen}.} Hervorhebenswert erscheint mir, daß \citet{frege1892uber-sinn-und-bedeutung} -- ausgehend von der Analyse von Satzgefügen mit dem Verb \textit{wähnen} im Hauptsatz -- die auch für \textit{wissen} wie auch für \textit{bekannt sein} und \textit{erkennen} geltend gemachte Feststellung trifft, „daß der Nebensatz ... eigentlich doppelt zu nehmen ist mit verschiedenen Bedeutungen, von denen die eine ein Gedanke, die andere ein Wahrheitswert ist.“ (ebd., 62 f.) Im Begriffssystem der modelltheoretischen Semantik ausgedrückt besagt das für die Bedeutung von Sätzen wie \REF{ex:zi83:47}, daß in der im Bewußtsein des Sprechers repräsentierten aktualen Welt sowohl die durch das Satzgefüge ausgedrückte komplexe Proposition wie auch die durch den Nebensatz ausgedrückte Proposition als erfüllt gelten und daß letztere auch bezüglich des Subjekts von \textit{wissen} als erfüllt gilt.\footnote{Bezüglich des Komplementsatzes von \textit{wähnen} ist das, wovon der Sprecher bzw. das Subjekt der durch dieses Verb bezeichneten propositionalen Einstellung überzeugt sind, diametral entgegengesetzt. \citet[62]{frege1892uber-sinn-und-bedeutung} trägt dem Rechnung, indem er den Sinn des betreffenden Satzgefüges als aus zwei Gedanken zusammengesetzt dargestellt, aus ‘a glaubt, daß S’ und aus ‘$\sim$S’.}

So kommt es bei prädikativen Ausdrücken, die eine epistemische propositionale Einstellung bezeichnen oder sie implizieren, häufig zu einer solchen Über\-la\-ge\-rung von Standpunkten verschiedener erkennender und urteilender Individuen im Hinblick auf die Wahrheit der durch den jeweiligen Komplementsatz oder durch dessen nominalisierte Entsprechung ausgedrückten Proposition. Immer dann, wenn ein (Neben-)Satz oder sein substantivisches Äquivalent im Fre\-ge\-schen %FREGEschen 
Sinne ‘einen Wahrheitswert bedeutet’, ist Faktivität der betreffenden Proposition gegeben. Und das heißt, daß der Sprecher der festen Ansicht ist, daß der bezeichnete Sachverhalt in der aktualen Welt besteht. Das Nomen \textit{Tatsache} (russisch: \textit{fakt}) signalisiert das da, wo es möglich ist, deutlich.

In Untersuchungen zur Faktivität ist mehrfach auf den engen Zusammenhang von Wissen und Wahrheit aufmerksam gemacht worden. \citet[187]{kiefer1978factivity-in-hungarian} betrachtet das Prädikat \textsc{wissen} „as the most basic feature of factives“ in dem Sinne, daß die durch lexikalische Einheiten induzierten faktiven Präsuppositionen in bestimmter grundlegender Weise mit Wissen zu tun haben. Ähnlich wie \citet[414 ff.]{zuber1977decomposition-of-factives} nimmt er an, daß die mit zweistelligen emotiven und evaluativen Verben bzw. Adjektiven verbundene faktive Präsupposition die als wahr angesehene Proposition $\textsc{wissen}(x, p)$ ist, die durch das Prädikat \textsc{wissen} ihrerseits die Wahrheit von $p$ präsupponiert.\footnote{\label{fn:zi83:kiefer}\citet[190]{kiefer1978factivity-in-hungarian} stellt die semantische Regel (81) auf, „Let $\textsc{emot}$ be a certain type of emotive predicate. Then, if $\textsc{emot}(x, p)$ or $\sim\textsc{emot}(x, p)$, then $\textsc{know}(x, p)$“, und führt erläuternd aus: „This means that $\textsc{emot}(x, p)$ presupposes $\textsc{know}(x, p)$, which, in turn, presupposes $p$. … If (81) holds, then it is clear that $p$ must be true in $W_{su}$. But since it cannot be the case that something be true in $W_{su}$ and false in $W_{s}$ if knowledge is involved (i.e. it cannot be the case that $\textsc{know}(x, p)$, x $\neq$ speaker, be true but $\textsc{know}(x, p)$, x $=$ speaker be false), it follows that $p$ must also be true in $W_{s}$.“ ($W_{s}$ und $W_{su}$ repräsentieren mögliche Welten des Sprechers resp. eines vom Sprecher verschiedenen Subjekts.)} Das heißt: Die Präsupponiertheit der Proposition $p$ wird hier vermittelt über die Proposition $\textsc{wissen}(x, p)$, die beide in der im Bewußtsein des Sprechers repräsentierten aktualen Welt als erfüllt gelten.

Es ist offensichtlich, daß dieser Vorschlag nur für faktive Lexeme mit einem personalen Subjekt (resp. einem entsprechenden Objekt) wirksam werden könn\-te. Für alle übrigen faktiven Lexeme (vgl. die Beispielsätze \REF{ex:zi83:56} und \REF{ex:zi83:57}) müßte ein anderer Weg beschritten werden, um faktive Präsuppositionen mit Wissen~-- insbesondere mit dem Wissens- und Überzeugungssystem des Sprechers -- in Beziehung zu bringen.

Indem wir uns im folgenden auf emotive Prädikatausdrücke beschränken,\footnote{Über den Zusammenhang von evaluativen Prädikatwörtern und Faktivität ihres propositionalen Komplements s. \citet{Zimmermann82Explizite-und}.} muß zu der vorgeschlagenen Analyse selbst angemerkt werden, daß nicht ohne weiteres klar ist, inwieweit Sätze mit negierten emotiven Verben die Schlußfolgerung zulassen, daß die durch das Subjekt benannte Person von dem durch das propositionale Komplement bezeichneten Sachverhalt überhaupt eine Ahnung hat. Vgl. die Sätze \REF{ex:zi83:48} und \REF{ex:zi83:49} sowie das von \citet[22]{gazdar1978eine-pragmatisch-semantische-mischtheorie-der-bedeutung} entlehnte Beispiel \REF{ex:zi83:50}:

\ea \label{ex:zi83:48}
    \gll Boris ne raduetsja, čto naša komanda pobedila. On ne interesuetsja sportom. \\
    Boris nicht freut.\textsc{refl} dass unsere Mannschaft siegte.\textsc{pst} er nicht interessiert.\textsc{refl} Sport.\textsc{ins} \\
    \glt ‘Boris freut sich nicht, dass unsere Mannschaft gewonnen hat. Er interessiert sich nicht für Sport.’
    
    
\ex \label{ex:zi83:49}
    \gll Boris ne raduetsja, čto naša komanda pobedila. On ničego ne znaet o rezul'tate sorevnovanij. No ja daže somnevajus’, čto naša komanda učastvovala v sorevnovanijax.\\
    Boris nicht freut.\textsc{refl} dass unsere Mannschaft siegte.\textsc{pst} er nichts nicht weiß über Ergebnis Wettkämpfe.\textsc{gen} aber ich sogar zweifle dass unsere Mannschaft teilnahm.\textsc{pst} in  Wettkämpfen.\textsc{loc} \\
    \glt ‘Boris freut sich nicht, dass unsere Mannschaft gewonnen hat. Er weiß nichts über den Ausgang der Wettkämpfe. Und ich zweifle sogar, dass unsere Mannschaft an den Wettkämpfen teilgenommen hat.’
    
\ex \label{ex:zi83:50} Hans bedauert nicht, beim Examen durchgefallen zu sein, weil er es in Wirklichkeit bestanden hat.

\z

\noindent Nur die Satzfolge \REF{ex:zi83:48} erlaubt die für den Sprecher gültige Schlußfolgerung, daß Boris vom Sieg der Mannschaft überzeugt ist (in Kiefers %\citeauthor{kiefer1978factivity-in-hungarian}s \citeyear{kiefer1978factivity-in-hungarian} %KIEFERS 
Verständnis: daß Boris weiß, daß die Mannschaft gesiegt hat). Dabei spielt keine Rolle, ob die Äußerung als einfache Behauptung oder als Bestreitung der Wahrheit einer entgegengesetzten Behauptung mit nachfolgender Begründung für deren Nichtwahrsein oder auch als (verkürzte) Konstatierung der logischen Unmöglichkeit der Wahrheit einer entsprechenden Annahme mit ergänzter Begründung dafür interpretiert wird. Die beiden letztgenannten Interpretationen sind wohl für Äußerungen wie \REF{ex:zi83:49} und \REF{ex:zi83:50} die einzig möglichen.\footnote{Vgl. dazu auch \citet[414]{zuber1977decomposition-of-factives} und \citet[183 ff.]{kiefer1978factivity-in-hungarian}.} Bei einer solchen Lesart negativer Sätze ist nur aus den Angaben des Sprechers, die er ergänzend zu einer als unzutreffend zurückgewiesenen Behauptung bzw. zu einer aus logischen Gründen ausgeschlossenen Annahme macht, direkt oder indirekt zu entnehmen, ob die aus der betreffenden Behauptung bzw. Annahme ableitbaren und nur in ihrem Rahmen gültigen Schlußfolgerungen wahre Propositionen bezüglich der im Bewußtsein des Sprechers repräsentierten aktualen Welt sind. In den Beispielen \REF{ex:zi83:49} und \REF{ex:zi83:50} trifft das für die mit den emotiven Verben verbundene Schlußfolgerung, daß Boris von dem Sieg der Mannschaft überzeugt ist bzw. daß Hans überzeugt ist, beim Examen durchgefallen zu sein, nicht zu. Für den Sprecher gelten diese Schlußfolgerungen nicht.

Auch über den Standpunkt des Sprechers bezüglich der Tatsachengeltung des durch das propositionale Komplement zweistelliger emotiver Verben und Adjektive bezeichneten Sachverhalts ist erst aus dem Kontext Klarheit zu gewinnen. Und das gilt nicht nur für negative Sätze wie in den Äußerungen \REF{ex:zi83:48}--\REF{ex:zi83:50}, sondern auch für affirmative Sätze mit emotiven lexikalischen Einheiten wie in dem Beispiel \REF{ex:zi83:51}.

\ea \label{ex:zi83:51}
    \gll Direktor školy udivljaetsja tomu, čto Petr často otsutstvoval. Odnako, Petr reguljarno poseščal uroki. \\
    Direktor  Schule.\textsc{gen} wundert.\textsc{refl} jenem.\textsc{dat} dass Peter häufig fehlte.\textsc{pst} jedoch Peter regelmäßig besuchte.\textsc{pst} Unterricht.\textsc{acc}.\textsc{pl} \\
    \glt ‘Der Schuldirektor wundert sich darüber, dass Peter häufig gefehlt hat. Jedoch hat Peter den Unterricht regelmäßig besucht.’
\z

\noindent Die Satzfolgen \REF{ex:zi83:49}--\REF{ex:zi83:51} verbieten die Schlußfolgerung, daß die durch das pro\-po\-si\-tio\-na\-le Komplement der Verben \textit{bedauern}, \textit{radovat’sja} (‘sich freuen’) und \textit{udi\-vljat’sja} (‘sich wundern’) ausgedrückte Proposition einen nach Überzeugung des Sprechers in der aktualen Welt bestehenden Sachverhalt identifiziert. Nur für das Beispiel \REF{ex:zi83:48} ist eine solche Interpretation zulässig.

Damit entsteht die Frage, wann bei den hier betrachteten lexikalischen Einheiten faktive propositionale Einbettungen vorliegen, ob diese Lexeme faktive Präsuppositionen induzieren und welche es gegebenenfalls sind.

Für die Beantwortung der ersten Frage haben wir bei emotiven Verben wie  \textit{radovat’sja} (‘sich freuen’) bzw. entsprechenden Adjektiven wie \textit{rad} (‘froh sein’) die Möglichkeit, Faktivität zu explizieren: Wo das betreffende propositionale Komplement vermittels des Nomens \textit{fakt} eingebettet werden kann, ohne daß sich die Bedeutung und die semantische Wohlgeformtheit der betreffenden Äußerung ändern, haben wir es mit Faktivität der propositionalen Einbettung zu tun. Das trifft für das Beispiel \REF{ex:zi83:48} zu, wie dessen Ausdrucksvariante \REF{ex:zi83:52} zeigt:

\ea \label{ex:zi83:52}
    \gll Boris ne raduetsja tomu faktu, čto naša komanda pobedila. On ne interesuetsja sportom. \\
    Boris nicht freut.\textsc{refl} jene  Tatsache.\textsc{dat} dass unsere Mannschaft siegte.\textsc{pst} er nicht interessiert.\textsc{refl} Sport.\textsc{ins} \\
    \glt ‘Boris freut sich nicht über die Tatsache, dass unsere Mannschaft gesiegt hat. Er interessiert sich nicht für Sport.’
\z

\noindent Die Äußerungen \REF{ex:zi83:49}--\REF{ex:zi83:51} enthalten Ansichten des Sprechers, mit denen die durch das Nomen \textit{Tatsache} resp. \textit{fakt} angezeigte Überzeugung des Sprechers, daß der durch das propositionale Adjunkt dieses Nomens bezeichnete Sachverhalt besteht, unvereinbar ist.

Nun ist die Einfügbarkeit des Nomens \textit{Tatsache} vor propositionalen Einbettungen zwar ein eindeutiges Indiz für deren Faktivität, besagt jedoch nichts darüber, ob die einbettenden Lexeme faktive Präsuppositionen induzieren. Das heißt: Aus der Vereinbarkeit von \textit{radovat’sja} und \textit{fakt} im Beispiel \REF{ex:zi83:52} und aus dessen Sy\-no\-nymie mit dem Beispiel \REF{ex:zi83:48} läßt sich nicht schließen, daß \textit{radovats’ja} ein inhärent faktives Verb ist. Sind also Sätze wie das erste Konjunkt in \REF{ex:zi83:48} -- als einfache Behauptung, nicht als Bestreitung einer als unzutreffend angesehenen Meinung verwendet -- im Hinblick auf Faktivität/Nichtfaktivität ihres propositonalen Komplements mehrdeutig? Oder anders: Ist für das propositionale Komplement emotiver Verben wie \textit{radovat’sja} (‘sich freuen’), \textit{udivljat’sja} (‘sich wundern’), \textit{trevožit’sja} (‘sich beunruhigen’) usw. eine nichttransparente, also nichtfaktive Leseart möglich?

Damit hängt auch die Frage zusammen, wie bei Lexemen die mit ihnen per Folgerung verbundene epistemische propositionale Einstellung der durch das je\-wei\-li\-ge Subjekt bezeichneten Person zu charakterisieren ist. Das Prädikat \textsc{wissen} ist nicht angebracht. Durch die Proposition $\textsc{wissen}(x, p)$, wobei $x$ und $p$ den Argumenten der betreffenden emotiven Prädikate entsprechen, wäre der Sprecher unausweichlich auf die Wahrheit der Proposition $p$ festgelegt (s. Anm. \ref{fn:zi83:kiefer}%57
). Das wäre aber nur für die propositionale Einbettung inhärent faktiver Le\-xe\-me zutreffend. In allen anderen Fällen, sowohl bei nichtfaktiven Verben und Adjektiven wie auch bei Le\-xe\-men, die bezüglich der Faktivität ihres propositionalen Komplements nicht festgelegt sind, ist die epistemische propositionale Einstellung des Subjekts (bzw. des entsprechenden Objekts), soweit sie für die Bedeutungscharakterisierung der betreffenden Lexeme in Betracht kommt, mit dem Prädikat \textsc{glauben}, d. h. als begründbares, starkes Glauben, zu charakterisieren. Der Sprecher ist durch dieses Prädikat -- im Gegensatz zu \textsc{wissen} -- bezüglich der Wahrheit des propositionalen Arguments nicht festgelegt.

Es ergäbe sich somit die höchst unbefriedigende Situation, daß die mit den betreffenden Verben und Adjektiven verbundene epistemische propositionale Einstellung ihres Subjekts einmal als Wissen und das andere Mal als starkes Glauben anzusehen wäre, je nach dem Festgelegtsein oder Nichtfestgelegtsein des Spre\-chers bezüglich der Tatsachengeltung des durch die propositionale Einbettung identifizierten Sachverhalts.

\newpage
Um es kurz zu sagen: Es erscheint mir geboten, anders als Kiefer die mit emotiven und auch evaluativen Verben und Adjektiven verbundene die epis\-te\-mi\-sche propositionale Einstellung des personalen Subjekts (resp. Objekts) betreffende Schlußfolgerung generell als $\textsc{glauben}_{c}(x, p)$ zu verstehen. Diese Annahme befindet sich in voller Übereinstimmung mit den Annahmen von \citet{frege1892uber-sinn-und-bedeutung} %FREGE 
zur Bedeutung von Satzgefügen mit emotiven und evaluativen propositionale Komplemente einbettenden Verben.\footnote{Nach \citet[52]{frege1892uber-sinn-und-bedeutung} haben Satzgefüge mit Verben wie \textit{sich freuen, bedauern, billigen, tadeln} im Hauptsatz „als Sinn nur einen einzigen Gedanken, und die Wahrheit des Ganzen schließt weder die Wahrheit noch die Unwahrheit des Nebensatzes ein.“ „Wenn Wellington sich gegen Ende der Schlacht bei Belle-Alliance freute, daß die Preußen kämen, so war der Grund seiner Freude eine Überzeugung. Wenn er sich getäuscht hätte, so würde er sich, solange sein Wahn dauerte, nicht minder gefreut haben, und bevor er die Überzeugung gewann, daß die Preußen kämen, konnte er sich nicht darüber freuen, obwohl sie in der Tat schon anrückten.“ (ebd., 52 f.).

\noindent Siehe auch \citet[14, Anm. 8 und 62, incl. Anm. 4]{norrick1978factive-adjectives-and-the-theory-of-factivity}.}

Bei Faktivität der propositionalen Einbettung teilt der Sprecher die Überzeugung der durch das Subjekt (resp. Objekt) emotiver Verben und Adjektive be\-zeich\-ne\-ten Person hinsichtlich des Bestehens des betreffenden Sachverhalts in der aktualen Welt. Ist er diesbezüglich einer anderen Auffassung oder unsicher, kann er das -- wie in dem Beispiel \REF{ex:zi83:51} -- auf geeignete Weise zum Ausdruck bringen.

Es erscheint mir trivial, aber dennoch nicht überflüssig, festzustellen, daß inhärent faktive Lexeme wie das kognitive Verb \textit{učityvat’} (‘berücksichtigen’) keine Distanzierung des Sprechers bezüglich der Tatsachengeltung des durch das be\-tref\-fen\-de propositionale Komplement bezeichneten Sachverhalts erlauben. Sätze wie \REF{ex:zi83:27a}--\REF{ex:zi83:29b} verbieten eine Textfortsetzung, wie sie das zweite Konjunkt im Beispiel \REF{ex:zi83:51} bildet. Auch die Einschaltung der modalen Partikel \textit{jakoby} (‘an\-geb\-lich’) in den Komplementsatz ist in diesen Sätzen nicht möglich.

Dieses Festgelegtsein des Sprechers bezüglich der Tatsachengeltung von Sach\-ver\-hal\-ten bei inhärent faktiven Lexemen und nicht zuletzt die Betrachtungen Freges zur ‘ungeraden Bedeutung’ von Nebensätzen\footnote{Siehe insbesondere \citet{frege1892uber-sinn-und-bedeutung}. Vgl. auch \citet[169 ff.]{seuren1975tussen-taal-en-denken:-een-bijdrage-tot-de-empirische-funderingen-van-de-semantiek.-deutsche-ubersetzung:-zwischen-sprache-und-denken:-ein-beitrag-zur-empirischen-begrundung-der-semantik}.} und die Arbeiten von \citet{kiefer1978factivity-in-hungarian, kiefer1978functional-sentence-perspective-and-presuppositions} zu Grundfragen der Faktivität und ihren differenzierten Ausdrucks\-mög\-lich\-keiten im Ungarischen lassen mich zweifeln, daß im Russischen und Deutschen emotive Lexeme wie \textit{radovat’sja} (deutsch: \textit{sich freuen}) oder \textit{rad} (deutsch: \textit{froh sein}), sofern sie ein vom Sprecher verschiedenes Subjekt (bzw. Objekt) haben können, als inhärent faktive Einheiten anzusehen sind. Die Mög\-lich\-keit der Mei\-nungs\-differenz zwischen dem Sprecher und dem Subjekt bezüglich der Tatsachengeltung des betreffenden Sachverhalts ist bei diesen Prädikaten so lange offen, wie sich der Sprecher nicht eindeutig festgelegt hat. Sind dem jeweiligen Diskurs keine gegenteiligen Ansichten des Sprechers zu entnehmen, ist es normal, der betreffenden propositionalen Einbettung eine faktive Leseart zuzuordnen. Es ist normal, anzunehmen, daß der Emotionen bewirkende Sachverhalt keine Einbildung ist.\footnote{Vgl. dazu \citet[477 f.]{rosenberg1975factives-that-arent-so-in}.

\citet[14]{gazdar1978eine-pragmatisch-semantische-mischtheorie-der-bedeutung} kündigt eine pragmatisch orientierte Behandlung faktiver Präsuppositionen an. Gemäß seinen Vorstellungen kommen die mit Verben wie \textit{bedauern} verbundenen „potentiellen“ Präsuppositionen nur zur Wirkung, d. h. werden sie zu „tatsächlichen“ Präsuppositionen, sofern sie „mit den anderen Implikationen, die der Satz im Kontext seiner Äußerung hat, verträglich sind.“ Als „potentielle“ faktive Präsupposition, die durch \textit{bedauern} in Satzbedeutungen eingebracht wird, sieht er an: „Der Sprecher weiß, daß …“. Sie kann in dem im Beispiel \REF{ex:zi83:50} gegebenen Kontext nicht wirksam werden.}

Auf die oben aufgeworfene Frage, ob das Beispiel \REF{ex:zi83:48} hinsichtlich einer faktiven resp. nichtfaktiven Leseart des von \textit{radovat’sja} abhängigen Komplement\-satzes mehrdeutig ist, kann jetzt so geantwortet werden: Wenn man die be\-tref\-fende Satzfolge als abgeschlossene Äußerung betrachtet, der seitens des Spre\-chers nichts mehr hinzugefügt wird, ist für die durch den Nebensatz ausgedrückte Proposition Faktivität anzunehmen, so daß die Äußerungen \REF{ex:zi83:48} und \REF{ex:zi83:52} als sy\-no\-ny\-me Ausdrucksvarianten gelten können.

Aus diesen Überlegungen ergibt sich, daß das Nomen \textit{Tatsache} (russisch: \textit{fakt}) als Einleitung eines propositionalen Komplements emotiver Verben und Adjektive ein weniger redundantes Ausdrucksmittel für Faktivität ist als bei inhärent faktiven Lexemen wie \textit{ucityvat’} (‘berücksichtigen’).

Es soll nun noch auf propositionale Einbettungen eingegangen werden, die Zustände, Prozesse, Ereignisse bewirkende Sachverhalte bezeichnen.

Dabei ist zunächst eine wesentliche Ergänzung zu den bisherigen Betrachtungen über emotive Prädikatwörter zu machen. Während intransitive emotive Verben wie \textit{radovat’sja} (‘sich freuen’), \textit{udivljat’sja} (‘sich wundern’) \textit{trevožit’sja} (‘sich beunruhigen’), \textit{ogorčat’sja} (‘betrübt sein’), \textit{vostorgat’sja} (‘sich begeistern’), \textit{serdit’sja} (‘sich ärgern’), \textit{pugat’sja} (‘Angst haben’) und viele andere -- wie ich annehme -- die Möglichkeit offenlassen, daß die die jeweiligen Emotionen em\-pfin\-den\-de Person einer puren Einbildung, einem Irrtum oder einer Falschmeldung erlegen ist, d. h. daß der die Empfindungen verursachende Sachverhalt nicht exi\-stiert, besagen die entsprechenden transitiven Verben, daß die betreffende Emotion durch einen realen, d. h. nach der Überzeugung des Sprechers bestehenden Sachverhalt hervorgerufen ist. Das heißt, durch diese transitiven Verben ist der Sprecher auf die Wahrheit der entsprechenden Proposition, die den Emotionen bewirkenden Sachverhalt identifiziert, festgelegt. Sätze wie \REF{ex:zi83:53} erlauben keine Textfortsetzung, in der wie im Beispiel \REF{ex:zi83:51} der Sprecher das Gegebensein des durch die eingebettete Proposition identifizierten Sachverhalts bestreitet. 

\ea \label{ex:zi83:53}
    \gll Direktora školy udivljaet tot fakt, čto Petr často otsutstvoval. \\
    Direktor.\textsc{gen}/\textsc{acc} Schule.\textsc{gen} wundert jene Tatsache.\textsc{nom} dass Peter häufig fehlte.\textsc{pst} \\
    \glt ‘Den Schuldirektor verwundert die Tatsache, dass Peter häufig gefehlt hat.’
\z

\noindent Für emotive Kausativa gilt genau wie für alle kausativen Verben, daß es sich um inhärent faktive Lexeme handelt. In ihrem Kontext ist die durch das Nomen \textit{Tatsache} (russisch: \textit{fakt}) angezeigte Information, daß der in der Subjektphrase benannte Sachverhalt vom Sprecher als bestehend angesehen wird, redundant.

%Sätze wie \REF{ex:zi83:53}

%\ea \label{ex:zi83:53}
%    \gll Direktora školy udivljaet tot fakt, čto Petr často otsutstvoval. \\
%    {(den) Direktor} {(der) Schule} wundert die Tatsache daß Peter häufig {gefehlt hat} \\

%\z

%erlauben keine Textfortsetzung, in der wie im Beispiel \REF{ex:zi83:51} der Sprecher das Gegebensein des durch die eingebettete Proposition identifizierten Sachverhalts bestreitet. Für emotive Kausativa gilt genau wie für alle kausativen Verben, daß es sich um inhärent faktive Lexeme handelt. In ihrem Kontext ist die durch das Nomen \textit{Tatsache} (russisch: \textit{fakt}) angezeigte Information, daß der in der Subjektphrase benannte Sachverhalt vom Sprecher als bestehend angesehen wird, redundant.

Obwohl sich intransitive und transitive emotive Verben hinsichtlich der Faktivität ihres propositionalen Komplements unterscheiden, können sie synonym verwendet werden unter der Voraussetzung, daß die an das transitive Verb notwendig gebundene faktive Präsupposition beim Auftreten des entsprechenden intransitiven Verbs unabhängig von diesem erfüllt ist. In dem im Beispiel \REF{ex:zi83:54} vorliegenden Kontext wird nicht nur die durch das Subjekt des transitiven emotiven Verbs \textit{ogorčat’} (‘betrüben’) ausgedrückte Proposition als wahr interpretiert, sondern auch die durch das Objekt des intransitiven emotiven Verbs \textit{udivljat’sja} (‘sich wundern’) ausgedrückte Proposition. 

\ea \label{ex:zi83:54}
    \gll Devočka udivilas’ ne tomu, čto vnutrennost’ mašiny byla obšita gladkim metallom, — ee ogorčilo otsutstvie okon, bez kotoryx rebenku ne myslimo udovol’stvie progulki.  \\
    Mädchen wunderte.\textsc{pst}.\textsc{refl} nicht jenem.\textsc{dat} dass Inneres.\textsc{nom}  Fahrzeug.\textsc{gen} war.\textsc{pst} verkleidet glattem.\textsc{ins} Metall.\textsc{ins} {} es.\textsc{acc} betrübte.\textsc{pst}.\textsc{sg}.\textsc{n}  Fehlen.\textsc{nom}.\textsc{sg}.\textsc{n} Fenstern.\textsc{gen} ohne welche.\textsc{gen}.\textsc{pl}  Kind.\textsc{dat} nicht denkbar.\textsc{sg}.\textsc{n} Vergnügen.\textsc{nom}.\textsc{sg}.\textsc{n}  Ausflugs.\textsc{gen} \\
    \glt ‘Das Mädchen wunderte sich nicht darüber, dass das Innere des Fahrzeugs mit glattem Metall verkleidet war, es war betrübt ob des Fehlens von Fenstern, ohne welche dem Kinde das Vergnügen eines Ausflugs nicht denkbar war.’ (\textit{L. Leonov})
\z


\largerpage[2]
\noindent Die Umstände, die in diesem Satz beschrieben sind, setzt der Autor als exi\-stent voraus, genau wie das Mädchen, und die Art des Betroffenseins des Kindes von diesen Umständen wird behauptend ausgesagt. Hier scheint also zwischen den Konstruktionen mit dem transitiven Verb resp. dem intransitiven Verb kein Unterschied zu bestehen; die Bedeutung des Satzes würde sich nicht ändern, begänne er so:
%In dem im Beispiel \REF{ex:zi83:54}

%\ea \label{ex:zi83:54}
%    \gll Devočka udivilas’ ne tomu, čto vnutrennost’ mašiny byla obšita gladkim metallom, — ee ogorčilo otsutstvie okon, bez kotorych rebenku ne myslimo {} udovol’stvie progulki. \emph{(L. Leonov)} \\
%    {(das) Mädchen} {wunderte sich} nicht darüber daß {(das) Innere} {(des) Fahrzeugs} war verkleidet {mit glattem} Metall {} es betrübte {(das) Fehlen} {(von) Fenstern} ohne die {für (ein) Kind} nicht denkbar ist {(das) Vergnügen} {(eines) Ausflugs} \\
    
%\z

%vorliegenden Kontext wird nicht nur die durch das Subjekt des transitiven emotiven Verbs \textit{ogorčat} (,betrüben‘) ausgedrückte Proposition als wahr interpretiert, sondern auch die durch das Objekt des intransitiven emotiven Verbs \textit{udivljat’sja} (,sich wundern‘) ausgedrückte Proposition. Die Umstände, die in diesem Satz beschrieben sind, setzt der Autor als existent voraus, genau wie das Mädchen, und die Art des Betroffenseins des Kindes von diesen Umständen wird behauptend ausgesagt. Hier scheint also zwischen den Konstruktionen mit dem transitiven Verb resp. dem intransitiven Verb kein Unterschied zu bestehen; die Bedeutung des Satzes würde sich nicht ändern, begänne er so:

\ea \label{ex:zi83:55}
    \gll Devočku udivilo ne to, čto … \\
    Mädchen.\textsc{acc} wunderte nicht jenes.\textsc{nom} dass  \\
    \glt ‘Das Mädchen wunderte nicht das, dass …’
\z

\noindent Die komplizierte Frage der Bedeutungsstruktur dieser Verbpaare wie auch das davon nicht zu trennende Problem des Verhältnisses von Transitivität und Intransitivität (Reflexivität) der betreffenden Verben können hier nicht erörtert werden.\footnote{In diesem Zusammenhang wäre auch zu der Arbeit von \citet[insbesondere 57 ff.]{norrick1978factive-adjectives-and-the-theory-of-factivity} Stellung zu nehmen.}

Nichtemotive kausative Verben wie in den Beispielen \REF{ex:zi83:56} und \REF{ex:zi83:57} gehören zu denjenigen inhärent faktiven Prädikatwörtern, für deren propositionales Argument, das in der Subjektphrase ausgedrückt ist, allein die epistemische propositionale Einstellung des Sprechers in Betracht kommt, der gemäß der jeweilige bestimmte Wirkungen hervorrufende Sachverhalt als Tatsache gilt. 

\ea \label{ex:zi83:56}
    \gll Neožidannoe vozvraščenie materi ešče bolee obostrilo i usililo ix [= brat’ev] želanie {vydelit’sja}. \\ 
    unerwartete Rückkehr.\textsc{nom}  Mutter.\textsc{gen} noch mehr verschärfte.\textsc{pst} und verstärkte.\textsc{pst} ihrer.\textsc{gen}  {} Brüder.\textsc{gen} Wunsch.\textsc{acc} unabhängig.machen.\textsc{inf}.\textsc{refl} \\
    \glt ‘Die unerwartete Rückkehr der Mutter verschärfte und verstärkte ihren \mbox{[= der Brüder]} Wunsch noch mehr, sich unabhängig zu machen.’ (\textit{M. Gor’kij})
    

\largerpage[2]
\ex \label{ex:zi83:57}
    \gll S roždeniem poslednego rebenka Ženni tjaželo zabolela i spustja četyre mesjaca umerla … smert’ ljubimoj dočeri slomila poslednie sily Marksa.  \\
    mit Geburt.\textsc{ins} letzten Kindes.\textsc{gen} Jenny.\textsc{nom}.\textsc{sg}.\textsc{f} schwer erkrankte.\textsc{pst}.\textsc{sg}.\textsc{f} und nach vier.\textsc{acc} Monaten.\textsc{gen} starb.\textsc{pst}.\textsc{sg}.\textsc{f} {} Tod.\textsc{nom}.\textsc{sg}.\textsc{f}  geliebter.\textsc{gen} Tochter.\textsc{gen} brach.\textsc{pst}.\textsc{sg}.\textsc{f}  letzte Kräfte.\textsc{acc} Marx.\textsc{gen} \\
    \glt ‘Nach der Geburt des letzten Kindes erkrankte Jenny schwer, und nach vier Monaten starb sie … der Tod der geliebten Tochter beraubte Marx seiner letzten Kräfte.’ (\textit{Ženskij kalendar’})
\z

\noindent Um kausale Zusammenhänge behaupten zu können, muß der Sprecher den als Beweggrund für etwas angesehenen ‘Gegenstand’ als existent voraussetzen.\footnote{Vgl. dazu \citet{vendler1967effects-results-and-consequences}.} Bei solchen Prädikatwörtern ohne personales Subjekt scheitert der oben skizzier\-te Vorschlag, faktive Einbettungen über das Wissen einer im Subjekt (oder Objekt) benannten, vom Sprecher verschiedenen Person zu erklären. Nach unserem Verständnis ist für Faktivität einer Proposition allein das Überzeugungssystem des Sprechers entscheidend.

%Nichtemotive kausative Verben wie in den Beispielen \REF{ex:zi83:56} und \REF{ex:zi83:57}

%\ea \label{ex:zi83:56}
%    \gll Neožidannoe vozvraščenie materi ešče bolee obostrilo i usililo ich [brat’ev] želanie {vydelit’sja. \hspace{10pt} \emph{(M. Gor’kij)}} \\
%    {(die) unerwartete} Rückkehr {(der) Mutter} noch mehr verschärfte und verstärkte ihren {[(der) Brüder]} Wunsch {sich unabhängig zu machen} \\
    
%\ex \label{ex:zi83:57}
%    \gll S roždeniem poslednego rebenka Ženni tjaželo zabolela i spustja četyre mesjaca umerla... smert’ ljubimoj dočeri slomila poslednie sily Marksa. \emph{(Ženskii kalendar’)} \\
%    mit {(der) Geburt} {(des) letzten} Kindes Jenny schwer erkrankte und nach vier Monaten starb {(der) Tod} {(der) geliebten} Tochter brach {(die) letzten} Kräfte Marx \\
    
%\z

%gehören zu denjenigen inhärent faktiven Prädikatwörtern, für deren propositionales Argument, das in der Subjektphrase ausgedrückt ist, allein die epistemische propositionale Einstellung des Sprechers in Betracht kommt, der gemäß der jeweilige bestimmte Wirkungen hervorrufende Sachverhalt als Tatsache gilt. Um kausale Zusammenhänge behaupten zu können, muß der Sprecher den als Beweggrund für etwas angesehenen ,Gegenstand‘ als existent voraussetzen.\footnote{Vgl. dazu \citet{vendler1967effects-results-and-consequences}.} Bei solchen Prädikatwörtern ohne personales Subjekt scheitert der oben skizzierte Vorschlag, faktive Einbettungen über das Wissen einer im Subjekt (oder Objekt) benannten, vom Sprecher verschiedenen Person zu erklären. Nach unserem Verständnis ist für Faktivität einer Proposition allein das Überzeugungssystem des Sprechers entscheidend.

\subsubsection{Zum Verhältnis faktiver und fokaler Präsuppositionen} \label{sec:zi83:3.4.6}
\largerpage
Schließlich sind noch einige Anmerkungen zum Verhältnis faktiver und fokaler Präsuppositionen erforderlich.

\citet{kiefer1978factivity-in-hungarian, kiefer1978functional-sentence-perspective-and-presuppositions} hat gezeigt, daß sich die Themafunktion und die Topikfunktion einer propositionalen Einbettung begünstigend auf deren faktive Interpretation auswirken. Ich habe diese Faktoren der inhaltlichen Gliederung von Sätzen in meinen Betrachtungen bewußt ausgeklammert. Die Annahmen des Sprechers bezüglich der Tatsachengeltung von Sachverhalten, auf die er in seinen Äußerungen referiert, sind unabhängig von der Aufgliederung von Sätzen in Thema und Rhema und in Topiks und Comment resp. in fokale Präsupposition und Fokus.\footnote{Zu der Gegenüberstellung von fokaler Präsupposition und Fokus s. \citet[Kap. 6]{jackendoff1972semantic-interpretation-in-generative-grammar.}, der explizit sagt (ebd., 246), daß fokale Präsuppositionen nicht notwendigerweise mit der Vor\-aus\-se\-tzung verbunden sind, daß die betreffenden Referenten existieren.} 

Es versteht sich, daß vorerwähnte, seitens des Sprechers behauptete propositionale Einbettungen wie in dem Beispiel \REF{ex:zi83:57} sowie in den Satzfolgen \REF{ex:zi83:58} und \REF{ex:zi83:59} Topikstatus haben und auf Sachverhalte referieren, die der Sprecher als exis\-tie\-rend ansieht.

\ea \label{ex:zi83:58}
    \gll … čelovek v obyčnyx uslovijax ne osoznaet sokraščenij svoej {serdečnoj myšcy}. No iz togo fakta, čto sokraščenija myšcy ne osoznajutsja, sovsem ne vytekaet, čto serdce rabotaet ne sokraščajas’. \\
    {} Mensch in gewöhnlichen Bedingungen.\textsc{loc} nicht wahrnimmt Kontraktionen.\textsc{acc} seines Herzmuskels.\textsc{gen} aber aus jener Tatsache.\textsc{gen} dass Kontraktionen.\textsc{nom} Muskels.\textsc{gen} nicht wahrnehmen.\textsc{prs}.3\textsc{pl}.\textsc{refl} überhaupt nicht folgt dass Herz.\textsc{nom} arbeitet nicht kontrahierend.\textsc{refl} \\
    \glt ‘Unter gewöhnlichen Bedingungen nimmt der Mensch die Kontraktionen seines Herzmuskels nicht wahr. Aber aus der Tatsache, dass die Muskelkontraktionen nicht wahrgenommen werden, folgt ganz und gar nicht, dass das Herz ohne Kontraktionen arbeitet.’ (\textit{L. N. Landa})
\z

\ea \label{ex:zi83:59}
    \gll Petr často otsutstvoval. {Ėkzamenacionnaja komissija} podčerknula ėto v svoej ocenke. \\
    Peter häufig fehlte.\textsc{pst}  Prüfungskommission unterstrich.\textsc{pst} dies.\textsc{acc} in ihrer Beurteilung \\
    \glt ‘Peter hat häufig gefehlt. Die Prüfungskommission stellte das  in ihrer Beurteilung heraus.’
\z

\noindent Nicht so klar liegen die Dinge in Sätzen wie \REF{ex:zi83:60}, wo der hier durch Großbuchstaben angezeigte Hauptakzent auf dem einbettenden Prädikatausdruck auf to\-pi\-ka\-lische Funktion der Satzeinbettung hinweist.

\ea \label{ex:zi83:60}
    \gll {Ėkzamenacionnaja komissija} v svojej ocenke PODČERKNULA, čto Petr často otsutstvoval. \\
    Prüfungskommission in ihrer Beurteilung unterstrich.\textsc{pst} dass Peter häufig fehlte.\textsc{pst} \\
    \glt ‘Die Prüfungskommission STELLTE in ihrer Beurteilung HERAUS, dass Peter häufig gefehlt hat.’
\z

%Es versteht sich, daß vorerwähnte, seitens des Sprechers behauptete propositionale Einbettungen wie in dem Beispiel \REF{ex:zi83:57} sowie in den Satzfolgen \REF{ex:zi83:58} und \REF{ex:zi83:59}

%\ea \label{ex:zi83:58}
%    \gll ...čelovek v obyčnych uslovijach ne osoznaet sokraščenij svoej {serdečnoj myšcy}. No iz togo fakta, čto sokraščenija myšcy ne osoznajutsja, sovsem ne vytekaet, čto serdce rabotaet ne {sokraščajas’. \hspace{5pt} \emph{(L. N. Landa)}} \\
%    {(der) Mensch} unter gewöhnlichen Bedingungen nicht {sich bewußt wird} {(der) Kontraktionen} seines Herzmuskels aber aus der Tatsache daß {(die) Kontraktionen} {(des) Muskels} nicht {bewußt werden} {ganz und gar} nicht folgt daß {(das) Herz} arbeitet nicht {sich kontrahierend} \\
    
%\ex \label{ex:zi83:59}
%    \gll Petr často otsutstvoval. {Ėkzamenacionnaja komissija} podčerknula čto v svoej ocenke. \\
%    Peter häufig {hat gefehlt} {(die) Prüfungskommission} {hat unterstrichen} dies in ihrer Beurteilung \\
%\z

%Topikstatus haben und auf Sachverhalte referieren, die der Sprecher als existierend ansieht. Nicht so klar liegen die Dinge in Sätzen wie \REF{ex:zi83:60},

%\ea \label{ex:zi83:60}
%    \gll {Ėkzamenacionnaja komissija} v svojej ocenke PODČERKNULA, čto Peter často otsutsvoval. \\
%    {(die) Prüfungskommission} in ihrer Beurteilung {hat unterstrichen} daß Peter häufig {gefehlt hat} \\
    
%\z

%wo der hier durch Großbuchstaben angezeigte Hauptakzent auf dem einbettenden Prädikatausdruck auf topikalische Funktion der Satzeinbettung hinweist. 
\noindent Damit ist aber über die Annahmen des Sprechers bezüglich der Tatsachengeltung des durch das topikalische Komplement bezeichneten Sachverhalts nichts festgelegt. Es kommt auf den Kontext an, in den der betreffende Satz eingereiht ist. Vgl. Beispiel \REF{ex:zi83:61}:

\begin{exe}
    \ex \label{ex:zi83:61}
        \begin{xlist}
            \exi{A:} 
                \gll Petr často otsutstvoval? \\
                Peter häufig fehlte.\textsc{pst} \\
                \glt ‘Hat Peter häufig gefehlt?’
                
            \exi{B:}
                \gll {Ėkzamenacionnaja komissija} v svojej ocenke PODČERKNULA, čto Petr často otsutstvoval. No ja ne uverena, čto ėto verno. \\
                 Prüfungskommission in ihrer Beurteilung unterstrich.\textsc{pst} dass Peter häufig fehlte.\textsc{pst} aber ich nicht  sicher dass dies wahr \\
                 \glt ‘Die Prüfungskommission STELLTE in ihrer Beurteilung HERAUS, dass Peter häufig gefehlt hat. Aber ich bin nicht sicher, dass das stimmt.’
                
        \end{xlist}
\end{exe}

\noindent Ganz offensichtlich ist Topikalität einer Konstituente nicht notwendig mit der Voraussetzung verbunden, daß ihr Referent existiert. Das gilt auch für the\-ma\-ti\-sche Konstituenten.

Ich beschränke mich hier auf diese kurzen Bemerkungen. Die angeschnittenen Fragen bedürfen eines gründlichen Studiums.\footnote{Ich verweise auf \citet{pasch1983mechanismen-der-inhaltlichen-gliederung-von-satzen}.}

\subsection{Zusammenfassung} \label{sec:zi83:3.5}
\largerpage[2]
Der den vorstehenden Darlegungen zugrunde gelegte Faktivitätsbegriff ist der Versuch einer Verallgemeinerung dessen, was \citet{kiparsky1970fact} unter faktiven Präsuppositionen verstanden. Er ist nicht gebunden an die Möglichkeit, das Substantiv \textit{Tatsache} (russisch: \textit{fakt}) zur Kennzeichnung von Faktivität zu verwenden. Er ist einzig und allein an das Gegebensein der epistemischen propositionalen Einstellung des Sprechers geknüpft, daß eine durch einen Satz oder eine seiner Ausdrucksalternativen ausgedrückte Proposition von ihm bezüglich der aktualen Welt als wahr angesehen wird. Die im Geltungsbereich dieser Einstellung des Sprechers liegenden Folgerungen aus den betreffenden Äußerungskomponenten haben den gleichen Status, seitens des Sprechers als wahr angesehene Propositionen zu sein.

„Ein Gedanke gelangt nie nackt an die Öffentlichkeit“, bemerkt \citet[209]{lang1979zum-status-der-satzadverbiale} sehr anschaulich. Er ist vor allem eingekleidet in eine propositionale Einstellung des Sprechers, die je nach der kommunikativen Intention, die mit der betreffenden Äußerung verfolgt wird, in systematischer Weise variiert. Ist diese propositionale Einstellung nicht die für Behauptungen charakteristische Position des Sprechers, nämlich starkes Glauben, muß dennoch diese letztere Einstellung -- gewissermaßen als Auffangbecken für Folgerungen aus den betreffenden Äußerungen, die für den Sprecher gültig sind -- auf geeignete Weise bereitgestellt werden (vgl. \citealt{morgan1969on-the-treatment-of-presupposition-in-transformational-grammar}).\footnote{Ernsthaft in Betracht zu ziehen wäre die Möglichkeit, die epistemische propositionale Einstellung des Sprechers für eingebettete Sätze in deren semantischer Struktur zu verankern (vgl. \citealt{zybatow1980uber-glaubensoperatoren}). Das müßte auch für Nominalphrasen ins Auge gefaßt werden, die substantivische Satzentsprechungen sind. Und von hier aus wäre zu überlegen, wie überhaupt die Position des Sprechers bezüglich der Gültigkeit der in Nominalphrasen kondensierten Prädikationen in der semantischen Struktur von Sätzen zu repräsentieren wäre. Anzunehmen, daß die Konjunktion \textit{daß} und entsprechend der bestimmte Artikel mehrdeutig seien, nämlich in Bezug auf eine transparente resp. nichttransparente Lesart der durch sie eingeleiteten Satzkonstituenten, wie \citet[166 ff.]{cresswell1973logics-and-languages} zur Unterscheidung von faktivem und nichtfaktivem \textit{that} verfährt, wäre nichts weiter als ein ad hoc-Verfahren. Viel angemessener erschiene, in den betreffenden Konstituenten einen Platz für einen modalen Operator vorzusehen, der vor allem markierte, d. h. Zweifel oder Widerspruch signalisierende Positionen des Sprechers bezüglich des Zutreffens der jeweiligen Prädikation beinhalten könnte. Zu diesbezüglichen Fakten s. \citet{zimmermann1978substantivverbande-als-satzentsprechungen-und-ihre-referentielle-bedeutung}.}

Alle an eine Äußerung gebundenen Propositionen $p$, für die die epistemische propositionale Einstellung des Sprechers $\textsc{glauben}_{c}$, zutrifft, sind nach meiner Auffassung faktive Propositionen. Aus diesem Überzeugungssystem des Spre\-chers ist kein Entweichen. Alle logischen Gesetze gelten in diesem epistemischen Rahmen, jede Interpretation von Äußerungen bezieht sich auf diesen Rahmen, es sei denn, es wird ein schwächerer Gewißheitsgrad signalisiert.

Auch wenn Faktivität von Propositionen hier auf das Überzeugungssystem des Spre\-chers bezogen wurde -- und zwar ausschließlich auf dieses -- und nicht wie in der Logik nur von der Wahrheit von Propositionen in Abstraktion von urteilenden Gedan\-ken\-trä\-gern gesprochen wurde, glaube ich dennoch nicht, mich aus\-schließ\-lich im Bereich der Pragmatik bewegt zu haben. Im Gegenteil: ich betrachte Faktivität von Propositionen als eine in die Semantik gehörende Seite der Bedeutung von Sätzen. Es sei denn, man geht dazu über, die referentielle Bedeutung sprachlicher Äußerungen der Pragmatik zuzuweisen.

Es ist evident, daß ich keinen Unterschied mache zwischen Sprecher-Prä\-sup\-po\-si\-tio\-nen und logischen Präsuppositionen, wie es \citet{norrick1978factive-adjectives-and-the-theory-of-factivity} vorsieht und meint, kognitive und nichtkognitive faktive Verben und Adjektive auf diese Weise unterscheiden zu müssen. Sofern es inhärent faktive Lexeme gibt -- und daran ist wohl nicht zu zweifeln --, ist der Sprecher auf die Tatsachengeltung des durch die betreffende Proposition identifizierten Sachverhalts festgelegt. Es gibt Le\-xe\-me, die bezüglich Faktivität/Nichtfaktivität ihres propositionalen Komplements unspezifiziert sind. Hier ist das Nomen \textit{Tatsache} ein nicht redundantes Signal dessen, daß der Sprecher den betreffenden Sachverhalt als Tatsache ansieht.

Da, wo das Nomen Tatsache nicht in indirekter Rede\footnote{Zu den komplizierten Verhältnissen des Bezugs von propositionalen Einstellungen in der indirekten Redewiedergabe s. \citet{steube1983indirekte-rede-und-zeitverlauf}.} oder in Ausdrücken wie \textit{x ščitaet, čto y -- fakt} (deutsch: ‘$x$ nimmt an, daß $y$ eine Tatsache ist’), \textit{x rassmatrivaet y faktom} (deutsch: ‘$x$ betrachtet $y$ als Tatsache’) auftritt, wobei $x$ nicht den Sprecher bezeichnet, drückt es auf eindeutige Weise aus, daß der Sprecher den betreffenden Sachverhalt als bestehend oder in Zukunft ganz sicher eintretend ansieht. Da, wo der Sprecher diese Einstellung nicht hat, ist das Nomen \textit{Tatsache} nicht am Platze. Anderslautende Auskünfte, insbesondere über die Substituierbarkeit des pronominalen Elements \textit{to} durch (\textit{tot}) \textit{fakt} (‘die Tatsache’) resp. durch \textit{(to) obstojatel’stvo} (‘der Umstand’) ohne jede Einschränkung sind ungenau.\footnote{Siehe \citet[693]{svedova1970grammatika-sovremennogo-russkogo-literaturnogo-jazyka}. Es fehlt der Hinweis, daß solche nominalen Ausdrücke wie \textit{tot fakt} (‘die Tatsache’) resp. \textit{to obstojatel’stvo} (‘der Umstand’) nur unter bestimmten Bedingungen zur Einbettung von Nebensätzen, die mit der Konjunktion \textit{čto} (‘daß’) eingeleitet sind, an die Stelle des pronominalen Elements \textit{to} treten können. In Sätzen wie \REF{ex:zi83:f69i} und \REF{ex:zi83:f69ii} ist eine solche Substitution nicht möglich. 

\ea \label{ex:zi83:f69i}
    \gll Boris somnevaetsja v tom, čto naša komanda pobedila. \\
    Boris zweifelt.\textsc{refl} in jenes.\textsc{loc}  dass unsere Mannschaft siegte.\textsc{pst} \\
    \glt ‘Boris zweifelt daran, dass unsere Mannschaft gesiegt hat.’
    
    
\ex \label{ex:zi83:f69ii}
    \gll {Redakcionnaja kollegija} nadeetsja na to, čto čitateli svoimi kritičeskimi zamečanijami pomogut avtoram v ix dal’nejšej rabote. \\
    Redaktionskollegium hofft.\textsc{refl} auf jenes.\textsc{acc} dass  Leser ihren kritischen Bemerkungen.\textsc{ins}  helfen.\textsc{prs}.3\textsc{pl}  Autoren.\textsc{dat}.\textsc{pl} in ihrer weiteren Arbeit.\textsc{loc} \\
    \glt ‘Das Redaktionskollegium hofft darauf, dass die kritischen Anmerkungen der Leser den Autoren in ihrer weiteren Arbeit helfen werden.’

\z

\noindent Diese substantivischen Fügungen sind eben semantisch nicht so entleert, wie das für das der Satzeinbettung dienende Pronomen \textit{to} konstatiert wird (s. ebd., 684), so daß es kein Zufall ist, daß sie zur Einbettung von Nebensätzen, die mit Fragewörtern oder mit \textit{kak} (‘wie’), \textit{čtoby} (‘daß’, ‘damit’) oder \textit{budto} (‘als ob’, ‘daß angeblich’) eingeleitet sind, nicht verwendet werden können. 

\noindent Die spezifischen Verwendungsbedingungen von \textit{(to) obstojatel’stvo} im Unterschied zu \textit{(tot) fakt} sind ein Kapitel für sich.}

Ich habe die Behandlung des Nomens \textit{Tatsache} als einheitliches Lexem sehr weit getrieben. Offenbar tendiert dieses Nomen in den einzelnen Sprachen in unterschiedlich starkem Maße zu einem rein syntaktischen Stützelement, wenn es nicht prädikativ verwendet wird. \citet[Bd. 2, 1900]{robert1960dictionaire-alphabetique-et-analogique-de-la-langue-fracaise:-tome-second} gibt für das Französische an, daß \textit{le fait que…} den Indikativ bzw. den Konjunktiv verlangt „selon le degré de réalité qui caractérise la proposition introduite par \textit{que}“.\footnote{Diese Information verdanke ich R. Pasch.} Den Einzelheiten muß nachgegangen werden. Fürs Deutsche und Russische jedenfalls trifft eine Bedeutungsentleerung dieses Substantivs als Einbettungsstütze nicht zu.\footnote{Vgl. die Feststellungen von \citet{arutjunovasokrovennaja-svjazka-k-probleme-predikativnogo-otnosenija} zur Bedeutung und kommunikativen Funktion des Nomens \textit{fakt}.}

Fragen der Repräsentation der Bedeutungsstruktur von Sätzen habe ich bei meinen Erörterungen unberücksichtigt gelassen, obwohl ich sie für wichtig erachte. Zu den wesentlichten in dieser Hinsicht zu klärenden Fragen zähle ich, welche Beziehung zwischen dem Nomen \textit{Tatsache} und modalen Operatoren einerseits\footnote{Hier wären auch das von \citet{jackendoff1971modal-structure-in-semantic-representation} und \citet[Kap. 7]{jackendoff1972semantic-interpretation-in-generative-grammar.} entwickelte Konzept modaler Operatoren und ihre Rolle in der referentiellen Interpretation von Sätzen und Satzkomponenten abzuwägen.} und dem Existenzoperator andererseit besteht und wie die Argumente der den prädikativ verwendeten Ausdrücken für Wahrsein bzw. für Tatsachesein entsprechenden Prädikate zu repräsentieren wären. Das sind nicht vorrangig Fragen nach einem geeigneten formalen Apparat, sondern -- wie mein Text zeigt -- auch offene Fragen zum Wesen der hier betrachteten sprachlichen Erscheinungen. Tatsache ist, daß noch viel zu tun bleibt.

\section*{Anmerkungen der Herausgeber}
Beispiele aus anderen Sprachen als dem Deutschen haben wir glossiert und übersetzt. In einigen Beispielen haben wir Glossen an die \textit{Leipzig Glossing Rules} an\-ge\-passt. 

\section*{Abkürzungen}
\begin{tabularx}{.5\textwidth}{@{}lQ}
        \textsc{2}&zweite Person\\
         \textsc{3}&dritte Person\\
        \textsc{acc}&Akkusativ\\
        \textsc{cond}&Konditionalmodus\\
        \textsc{dat}&Dativ\\
        \textsc{f}&Femininum\\
       \textsc{gen}&Genitiv\\
        \textsc{imp}&Imperativ\\
        \textsc{infv}&Infinitiv\\
        \textsc{ins}&Instrumental\\
        \textsc{loc}&Lokativ\\
\end{tabularx}%
\begin{tabularx}{.5\textwidth}{@{}lX@{}}
        \textsc{l-ptcp}&\textit{l}-Partizip\\
        \textsc{m}&Maskulinum\\
        \textsc{n}&Neutrum\\
        \textsc{nom}&Nominativ\\
        \textsc{pfv}&perfektiv\\
        \textsc{pl}&Plural\\
        \textsc{prs}&Präsens\\
        \textsc{pst}&Präteritum\\
        \textsc{refl}&Reflexiv\\
        \textsc{sg}&Singular\\
%&\\ % this dummy row achieves correct vertical alignment of both tables
\end{tabularx}

\nocite{Gorskij1966,Landa1966,Lenin-Gorkij-1969,nove-Knigi1967,Simonov1964,Slovar1957,Solncev1976,Zvegincev1976}

\defbibnote{quellen-1983-prenote}
           {Die Ziffer in den Klammern weist auf das entsprechende Beispiel hin.}

\printbibliography[
    heading=subbibliography,
    keyword={quellenverzeichnis1983},
    title={Quellenennachweis für das Beispielmaterial},
    prenote=quellen-1983-prenote
  ]
% 
% \noindent Gorskij, D. P. (1966). \textit{Problemy obščej metodologii nauk i dialektičeskoj logiki}. Moskva: Mysl', 9 (36) \newline
% Landa, L. N. (1966). \textit{Algoritmizacija v obučenii}. Moskva: Prosveščenie, 477 (58)\newline 
% Lenin, V. I. \& A. M. Gor'kij. 1969. \textit{Pis'ma, vospominanija, dokumenty}. 3. Auflage. Moskva, 20 (25), 24 (24), 47 (15) und 51 (23) \newline
% \textit{NOVYE KNIGI}, 1967, 38, Pos. 6 (14); 1970, 10, Pos. 56 (3c) \newline
% Simonov, K. M. 1964. \textit{Soldatami ne roždajutsja}. Moskva: Sovetskij pisatel', 9 (13) \newline
% \textit{Slovar' russkogo jazyka v četyrex tomax}. Moskva, Band I: 1957, 330 (56); Band IV: 1961, 749 (37) \newline
% Solncev, V. M. 1976. Otnositel’no koncepcii “glubinnoj struktury”. \textit{Voprosy ja\-zy\-ko\-zna\-nija} 5: 13--25, 19 (38) \newline
% %\textit{Ženskij kalendar'}. 1969. \newline
% Zvegincev, V. A. 1976. Naučno-texničeskaja revoljucija i lingvistika. \textit{Voprosy filosofii} 10: 55--66, 64 (39)

\printbibliography[heading=subbibliography,notkeyword={quellenverzeichnis1983}]

\end{otherlanguage}
\end{document}
