\documentclass[output=paper,colorlinks,citecolor=brown]{langscibook}
\ChapterDOI{10.5281/zenodo.15471421}

\title{Introduction}
\author{Łukasz Jędrzejowski\affiliation{University of Agder} and Uwe Junghanns\affiliation{University of Göttingen} and Kerstin Schwabe\affiliation{Leibniz Institute for the German Language, Mannheim} and Carla Umbach\affiliation{University of Cologne}}
% replace the above with you and your coauthors
% rules for affiliation: If there's an official English version, use that (find out on the official website of the university); if not, use the original
% orcid doesn't appear printed; it's metainformation used for later indexing

%%% uncomment the following line if you are a single author or all authors have the same affiliation
% \SetupAffiliations{mark style=none}

%% in case the running head with authors exceeds one line (which is the case in this example document), use one of the following methods to turn it into a single line; otherwise comment the line below out with % and ignore it
\lehead{Jędrzejowski et al.}

% replace the above with your paper title
%%% provide a shorter version of your title in case it doesn't fit a single line in the running head
% in this form: \title[short title]{full title}
\abstract{In this introductory chapter, we depict the history of the present volume and the structure of the papers collected in it. We give a general overview over major linguistic topics Ilse Zimmermann worked on, and outline her biography. Furthermore, selected statements of Ilse's students and colleagues are presented.}
% There is no abstract

% \DeclareCiteCommand{\fullcite}
%   {\usebibmacro{prenote}}
%   {\clearfield{author}%
%    \clearfield{labelyear}%
%    \usedriver
%      {\DeclareNameAlias{sortname}{default}}
%      {\thefield{entrytype}}\addperiod}
%   {\multicitedelim}
%   {\usebibmacro{postnote}}

 
\begin{document}
\maketitle

% Just comment out the input below when you're ready to go.
%For a start: Do not forget to give your Overleaf project (this paper) a recognizable name. This one could be called, for instance, Simik et al: OSL template. You can change the name of the project by hovering over the gray title at the top of this page and clicking on the pencil icon.

\section{Introduction}\label{sim:sec:intro}

Language Science Press is a project run for linguists, but also by linguists. You are part of that and we rely on your collaboration to get at the desired result. Publishing with LangSci Press might mean a bit more work for the author (and for the volume editor), esp. for the less experienced ones, but it also gives you much more control of the process and it is rewarding to see the quality result.

Please follow the instructions below closely, it will save the volume editors, the series editors, and you alike a lot of time.

\sloppy This stylesheet is a further specification of three more general sources: (i) the Leipzig glossing rules \citep{leipzig-glossing-rules}, (ii) the generic style rules for linguistics (\url{https://www.eva.mpg.de/fileadmin/content_files/staff/haspelmt/pdf/GenericStyleRules.pdf}), and (iii) the Language Science Press guidelines \citep{Nordhoff.Muller2021}.\footnote{Notice the way in-text numbered lists should be written -- using small Roman numbers enclosed in brackets.} It is advisable to go through these before you start writing. Most of the general rules are not repeated here.\footnote{Do not worry about the colors of references and links. They are there to make the editorial process easier and will disappear prior to official publication.}

Please spend some time reading through these and the more general instructions. Your 30 minutes on this is likely to save you and us hours of additional work. Do not hesitate to contact the editors if you have any questions.

\section{Illustrating OSL commands and conventions}\label{sim:sec:osl-comm}

Below I illustrate the use of a number of commands defined in langsci-osl.tex (see the styles folder).

\subsection{Typesetting semantics}\label{sim:sec:sem}

See below for some examples of how to typeset semantic formulas. The examples also show the use of the sib-command (= ``semantic interpretation brackets''). Notice also the the use of the dummy curly brackets in \REF{sim:ex:quant}. They ensure that the spacing around the equation symbol is correct. 

\ea \ea \sib{dog}$^g=\textsc{dog}=\lambda x[\textsc{dog}(x)]$\label{sim:ex:dog}
\ex \sib{Some dog bit every boy}${}=\exists x[\textsc{dog}(x)\wedge\forall y[\textsc{boy}(y)\rightarrow \textsc{bit}(x,y)]]$\label{sim:ex:quant}
\z\z

\noindent Use noindent after example environments (but not after floats like tables or figures).

And here's a macro for semantic type brackets: The expression \textit{dog} is of type $\stb{e,t}$. Don't forget to place the whole type formula into a math-environment. An example of a more complex type, such as the one of \textit{every}: $\stb{s,\stb{\stb{e,t},\stb{e,t}}}$. You can of course also use the type in a subscript: dog$_{\stb{e,t}}$

We distinguish between metalinguistic constants that are translations of object language, which are typeset using small caps, see \REF{sim:ex:dog}, and logical constants. See the contrast in \REF{sim:ex:speaker}, where \textsc{speaker} (= serif) in \REF{sim:ex:speaker-a} is the denotation of the word \textit{speaker}, and \cnst{speaker} (= sans-serif) in \REF{sim:ex:speaker-b} is the function that maps the context $c$ to the speaker in that context.\footnote{Notice that both types of small caps are automatically turned into text-style, even if used in a math-environment. This enables you to use math throughout.}$^,$\footnote{Notice also that the notation entails the ``direct translation'' system from natural language to metalanguage, as entertained e.g. in \citet{Heim.Kratzer1998}. Feel free to devise your own notation when relying on the ``indirect translation'' system (see, e.g., \citealt{Coppock.Champollion2022}).}

\ea\label{sim:ex:speaker}
\ea \sib{The speaker is drunk}$^{g,c}=\textsc{drunk}\big(\iota x\,\textsc{speaker}(x)\big)$\label{sim:ex:speaker-a}
\ex \sib{I am drunk}$^{g,c}=\textsc{drunk}\big(\cnst{speaker}(c)\big)$\label{sim:ex:speaker-b}
\z\z

\noindent Notice that with more complex formulas, you can use bigger brackets indicating scope, cf. $($ vs. $\big($, as used in \REF{sim:ex:speaker}. Also notice the use of backslash plus comma, which produces additional space in math-environment.

\subsection{Examples and the minsp command}

Try to keep examples simple. But if you need to pack more information into an example or include more alternatives, you can resort to various brackets or slashes. For that, you will find the minsp-command useful. It works as follows:

\ea\label{sim:ex:german-verbs}\gll Hans \minsp{\{} schläft / schlief / \minsp{*} schlafen\}.\\
Hans {} sleeps {} slept {} {} sleep.\textsc{inf}\\
\glt `Hans \{sleeps / slept\}.'
\z

\noindent If you use the command, glosses will be aligned with the corresponding object language elements correctly. Notice also that brackets etc. do not receive their own gloss. Simply use closed curly brackets as the placeholder.

The minsp-command is not needed for grammaticality judgments used for the whole sentence. For that, use the native langsci-gb4e method instead, as illustrated below:

\ea[*]{\gll Das sein ungrammatisch.\\
that be.\textsc{inf} ungrammatical\\
\glt Intended: `This is ungrammatical.'\hfill (German)\label{sim:ex:ungram}}
\z

\noindent Also notice that translations should never be ungrammatical. If the original is ungrammatical, provide the intended interpretation in idiomatic English.

If you want to indicate the language and/or the source of the example, place this on the right margin of the translation line. Schematic information about relevant linguistic properties of the examples should be placed on the line of the example, as indicated below.

\ea\label{sim:ex:bailyn}\gll \minsp{[} Ėtu knigu] čitaet Ivan \minsp{(} často).\\
{} this book.{\ACC} read.{\PRS.3\SG} Ivan.{\NOM} {} often\\\hfill O-V-S-Adv
\glt `Ivan reads this book (often).'\hfill (Russian; \citealt[4]{Bailyn2004})
\z

\noindent Finally, notice that you can use the gloss macros for typing grammatical glosses, defined in langsci-lgr.sty. Place curly brackets around them.

\subsection{Citation commands and macros}

You can make your life easier if you use the following citation commands and macros (see code):

\begin{itemize}
    \item \citealt{Bailyn2004}: no brackets
    \item \citet{Bailyn2004}: year in brackets
    \item \citep{Bailyn2004}: everything in brackets
    \item \citepossalt{Bailyn2004}: possessive
    \item \citeposst{Bailyn2004}: possessive with year in brackets
\end{itemize}

\section{Trees}\label{s:tree}

Use the forest package for trees and place trees in a figure environment. \figref{sim:fig:CP} shows a simple example.\footnote{See \citet{VandenWyngaerd2017} for a simple and useful quickstart guide for the forest package.} Notice that figure (and table) environments are so-called floating environments. {\LaTeX} determines the position of the figure/table on the page, so it can appear elsewhere than where it appears in the code. This is not a bug, it is a property. Also for this reason, do not refer to figures/tables by using phrases like ``the table below''. Always use tabref/figref. If your terminal nodes represent object language, then these should essentially correspond to glosses, not to the original. For this reason, we recommend including an explicit example which corresponds to the tree, in this particular case \REF{sim:ex:czech-for-tree}.

\ea\label{sim:ex:czech-for-tree}\gll Co se řidič snažil dělat?\\
what {\REFL} driver try.{\PTCP.\SG.\MASC} do.{\INF}\\
\glt `What did the driver try to do?'
\z

\begin{figure}[ht]
% the [ht] option means that you prefer the placement of the figure HERE (=h) and if HERE is not possible, you prefer the TOP (=t) of a page
% \centering
    \begin{forest}
    for tree={s sep=1cm, inner sep=0, l=0}
    [CP
        [DP
            [what, roof, name=what]
        ]
        [C$'$
            [C
                [\textsc{refl}]
            ]
            [TP
                [DP
                    [driver, roof]
                ]
                [T$'$
                    [T [{[past]}]]
                    [VP
                        [V
                            [tried]
                        ]
                        [VP, s sep=2.2cm
                            [V
                                [do.\textsc{inf}]
                            ]
                            [t\textsubscript{what}, name=trace-what]
                        ]
                    ]
                ]
            ]
        ]
    ]
    \draw[->,overlay] (trace-what) to[out=south west, in=south, looseness=1.1] (what);
    % the overlay option avoids making the bounding box of the tree too large
    % the looseness option defines the looseness of the arrow (default = 1)
    \end{forest}
    \vspace{3ex} % extra vspace is added here because the arrow goes too deep to the caption; avoid such manual tweaking as much as possible; here it's necessary
    \caption{Proposed syntactic representation of \REF{sim:ex:czech-for-tree}}
    \label{sim:fig:CP}
\end{figure}

Do not use noindent after figures or tables (as you do after examples). Cases like these (where the noindent ends up missing) will be handled by the editors prior to publication.

\section{Italics, boldface, small caps, underlining, quotes}

See \citet{Nordhoff.Muller2021} for that. In short:

\begin{itemize}
    \item No boldface anywhere.
    \item No underlining anywhere (unless for very specific and well-defined technical notation; consult with editors).
    \item Small caps used for (i) introducing terms that are important for the paper (small-cap the term just ones, at a place where it is characterized/defined); (ii) metalinguistic translations of object-language expressions in semantic formulas (see \sectref{sim:sec:sem}); (iii) selected technical notions.
    \item Italics for object-language within text; exceptionally for emphasis/contrast.
    \item Single quotes: for translations/interpretations
    \item Double quotes: scare quotes; quotations of chunks of text.
\end{itemize}

\section{Cross-referencing}

Label examples, sections, tables, figures, possibly footnotes (by using the label macro). The name of the label is up to you, but it is good practice to follow this template: article-code:reference-type:unique-label. E.g. sim:ex:german would be a proper name for a reference within this paper (sim = short for the author(s); ex = example reference; german = unique name of that example).

\section{Syntactic notation}

Syntactic categories (N, D, V, etc.) are written with initial capital letters. This also holds for categories named with multiple letters, e.g. Foc, Top, Num, etc. Stick to this convention also when coming up with ad hoc categories, e.g. Cl (for clitic or classifier).

An exception from this rule are ``little'' categories, which are written with italics: \textit{v}, \textit{n}, \textit{v}P, etc.

Bar-levels must be typeset with bars/primes, not with an apostrophe. An easy way to do that is to use mathmode for the bar: C$'$, Foc$'$, etc.

Specifiers should be written this way: SpecCP, Spec\textit{v}P.

Features should be surrounded by square brackets, e.g., [past]. If you use plus and minus, be sure that these actually are plus and minus, and not e.g. a hyphen. Mathmode can help with that: [$+$sg], [$-$sg], [$\pm$sg]. See \sectref{sim:sec:hyphens-etc} for related information.

\section{Footnotes}

Absolutely avoid long footnotes. A footnote should not be longer than, say, {20\%} of the page. If you feel like you need a long footnote, make an explicit digression in the main body of the text.

Footnotes should always be placed at the end of whole sentences. Formulate the footnote in such a way that this is possible. Footnotes should always go after punctuation marks, never before. Do not place footnotes after individual words. Do not place footnotes in examples, tables, etc. If you have an urge to do that, place the footnote to the text that explains the example, table, etc.

Footnotes should always be formulated as full, self-standing sentences.

\section{Tables}

For your tables use the table plus tabularx environments. The tabularx environment lets you (and requires you in fact) to specify the width of the table and defines the X column (left-alignment) and the Y column (right-alignment). All X/Y columns will have the same width and together they will fill out the width of the rest of the table -- counting out all non-X/Y columns.

Always include a meaningful caption. The caption is designed to appear on top of the table, no matter where you place it in the code. Do not try to tweak with this. Tables are delimited with lsptoprule at the top and lspbottomrule at the bottom. The header is delimited from the rest with midrule. Vertical lines in tables are banned. An example is provided in \tabref{sim:tab:frequencies}. See \citet{Nordhoff.Muller2021} for more information. If you are typesetting a very complex table or your table is too large to fit the page, do not hesitate to ask the editors for help.

\begin{table}
\caption{Frequencies of word classes}
\label{sim:tab:frequencies}
 \begin{tabularx}{.77\textwidth}{lYYYY} %.77 indicates that the table will take up 77% of the textwidth
  \lsptoprule
            & nouns & verbs  & adjectives & adverbs\\
  \midrule
  absolute  &   12  &    34  &    23      & 13\\
  relative  &   3.1 &   8.9  &    5.7     & 3.2\\
  \lspbottomrule
 \end{tabularx}
\end{table}

\section{Figures}

Figures must have a good quality. If you use pictorial figures, consult the editors early on to see if the quality and format of your figure is sufficient. If you use simple barplots, you can use the barplot environment (defined in langsci-osl.sty). See \figref{sim:fig:barplot} for an example. The barplot environment has 5 arguments: 1. x-axis desription, 2. y-axis description, 3. width (relative to textwidth), 4. x-tick descriptions, 5. x-ticks plus y-values.

\begin{figure}
    \centering
    \barplot{Type of meal}{Times selected}{0.6}{Bread,Soup,Pizza}%
    {
    (Bread,61)
    (Soup,12)
    (Pizza,8)
    }
    \caption{A barplot example}
    \label{sim:fig:barplot}
\end{figure}

The barplot environment builds on the tikzpicture plus axis environments of the pgfplots package. It can be customized in various ways. \figref{sim:fig:complex-barplot} shows a more complex example.

\begin{figure}
  \begin{tikzpicture}
    \begin{axis}[
	xlabel={Level of \textsc{uniq/max}},  
	ylabel={Proportion of $\textsf{subj}\prec\textsf{pred}$}, 
	axis lines*=left, 
        width  = .6\textwidth,
	height = 5cm,
    	nodes near coords, 
    % 	nodes near coords style={text=black},
    	every node near coord/.append style={font=\tiny},
        nodes near coords align={vertical},
	ymin=0,
	ymax=1,
	ytick distance=.2,
	xtick=data,
	ylabel near ticks,
	x tick label style={font=\sffamily},
	ybar=5pt,
	legend pos=outer north east,
	enlarge x limits=0.3,
	symbolic x coords={+u/m, \textminus u/m},
	]
	\addplot[fill=red!30,draw=none] coordinates {
	    (+u/m,0.91)
        (\textminus u/m,0.84)
	};
	\addplot[fill=red,draw=none] coordinates {
	    (+u/m,0.80)
        (\textminus u/m,0.87)
	};
	\legend{\textsf{sg}, \textsf{pl}}
    \end{axis} 
  \end{tikzpicture} 
    \caption{Results divided by \textsc{number}}
    \label{sim:fig:complex-barplot}
\end{figure}

\section{Hyphens, dashes, minuses, math/logical operators}\label{sim:sec:hyphens-etc}

Be careful to distinguish between hyphens (-), dashes (--), and the minus sign ($-$). For in-text appositions, use only en-dashes -- as done here -- with spaces around. Do not use em-dashes (---). Using mathmode is a reliable way of getting the minus sign.

All equations (and typically also semantic formulas, see \sectref{sim:sec:sem}) should be typeset using mathmode. Notice that mathmode not only gets the math signs ``right'', but also has a dedicated spacing. For that reason, never write things like p$<$0.05, p $<$ 0.05, or p$<0.05$, but rather $p<0.05$. In case you need a two-place math or logical operator (like $\wedge$) but for some reason do not have one of the arguments represented overtly, you can use a ``dummy'' argument (curly brackets) to simulate the presence of the other one. Notice the difference between $\wedge p$ and ${}\wedge p$.

In case you need to use normal text within mathmode, use the text command. Here is an example: $\text{frequency}=.8$. This way, you get the math spacing right.

\section{Abbreviations}

The final abbreviations section should include all glosses. It should not include other ad hoc abbreviations (those should be defined upon first use) and also not standard abbreviations like NP, VP, etc.


\section{Bibliography}

Place your bibliography into localbibliography.bib. Important: Only place there the entries which you actually cite! You can make use of our OSL bibliography, which we keep clean and tidy and update it after the publication of each new volume. Contact the editors of your volume if you do not have the bib file yet. If you find the entry you need, just copy-paste it in your localbibliography.bib. The bibliography also shows many good examples of what a good bibliographic entry should look like.

See \citet{Nordhoff.Muller2021} for general information on bibliography. Some important things to keep in mind:

\begin{itemize}
    \item Journals should be cited as they are officially called (notice the difference between and, \&, capitalization, etc.).
    \item Journal publications should always include the volume number, the issue number (field ``number''), and DOI or stable URL (see below on that).
    \item Papers in collections or proceedings must include the editors of the volume (field ``editor''), the place of publication (field ``address'') and publisher.
    \item The proceedings number is part of the title of the proceedings. Do not place it into the ``volume'' field. The ``volume'' field with book/proceedings publications is reserved for the volume of that single book (e.g. NELS 40 proceedings might have vol. 1 and vol. 2).
    \item Avoid citing manuscripts as much as possible. If you need to cite them, try to provide a stable URL.
    \item Avoid citing presentations or talks. If you absolutely must cite them, be careful not to refer the reader to them by using ``see...''. The reader can't see them.
    \item If you cite a manuscript, presentation, or some other hard-to-define source, use the either the ``misc'' or ``unpublished'' entry type. The former is appropriate if the text cited corresponds to a book (the title will be printed in italics); the latter is appropriate if the text cited corresponds to an article or presentation (the title will be printed normally). Within both entries, use the ``howpublished'' field for any relevant information (such as ``Manuscript, University of \dots''). And the ``url'' field for the URL.
\end{itemize}

We require the authors to provide DOIs or URLs wherever possible, though not without limitations. The following rules apply:

\begin{itemize}
    \item If the publication has a DOI, use that. Use the ``doi'' field and write just the DOI, not the whole URL.
    \item If the publication has no DOI, but it has a stable URL (as e.g. JSTOR, but possibly also lingbuzz), use that. Place it in the ``url'' field, using the full address (https: etc.).
    \item Never use DOI and URL at the same time.
    \item If the official publication has no official DOI or stable URL (related to the official publication), do not replace these with other links. Do not refer to published works with lingbuzz links, for instance, as these typically lead to the unpublished (preprint) version. (There are exceptions where lingbuzz or semanticsarchive are the official publication venue, in which case these can of course be used.) Never use URLs leading to personal websites.
    \item If a paper has no DOI/URL, but the book does, do not use the book URL. Just use nothing.
\end{itemize}

\noindent Ilse Zimmermann was an outstanding linguist, a meticulous researcher, a generous and helpful colleague, and an inspiration for  the next generation. She was a pioneer of Generative Grammar in Germany and made important contributions to the analysis of German and Slavic languages, predominantly Russian. Her fields of work were morphology, syntax, semantics, and the lexicon, thus covering an astonishingly wide part of the theoretical reconstruction of natural language grammar. When commenting on talks she was always to the point and sometimes unflattering but always friendly and constructive. Her doctoral students report her insistence on precise thinking, brevity, and clarity of language as well as genuine interest in their topics and impartiality towards other \mbox{methods} and schools and, above all, careful and empathetic supervision. This book is dedicated to her memory.

Ilse Zimmermann died in June 2020 at the age of nearly ninety two. The idea of publishing a volume in honor of her academic work emerged in the summer of 2018 following a colloquium to celebrate her 90th birthday and a visit to her wonderful garden. It took us (by then, Łukasz Jędrzejowski and Carla Umbach) some time to find a publisher for the Festschrift. Actually, what we had in mind was not a Festschrift but instead a combination of original papers by Ilse Zimmermann and guest contributions relating to these papers, and we finally succeeded in convincing Language Science Press (Open Slavic Linguistics) that this would be an attractive book. Unfortunately, Ilse Zimmermann died before we could discuss our plan with her -- the final confirmation by LangSci Press reached us on the same day as the news of her death. We decided to go ahead because we thought that the volume would be a wonderful posthumous tribute.

The contents of the volume include approximately half original articles by Ilse Zimmermann and half new articles by other linguists. The guest chapters in this volume are each closely related to Zimmermann's topics: Ljudmila Geist and Joanna Błaszczak propose a fine-grained structure of the DP in Slavic. In Gisbert Fanselow's chapter, the distribution of \textit{wh}-scope marking  is examined under a typological perspective. 
Hagen Pitsch suggests to reconstruct Slovak verbal inflection using a declarative morphology and presents assumptions on word-schemas and correspondences in language acquisition. He supports a word-form lexicon of the moderate type.
Boban Arsenijević argues in favor of the consideration that the reciprocal interpretation of `one'$+$`other' expressions in, e.g., Serbo-Croatian is not given by the composition of the meanings of the constituents but instead by pragmatic strengthening.
Uwe Junghanns demonstrates in his chapter that adverbials predicate over various partial situations which, he claims, is a strong argument for the semantic decomposition of causative verbs. 
Finally, Manfred Bierwisch provides an insight into the progression of the scholarly work by the researchers in the Structural Grammar Research Group constituting a structural part of the linguistic department at the German Academy of Sciences in Berlin that played a pioneering role in the development of linguistics in Germany. Ilse Zimmermann joined the group in the mid-1960s and conducted most of her research in this context. 

Before proceeding with the introduction we would like to commemorate three of our guest authors who are sadly no longer with us, Joanna Błaszczak (1971--2021), Gisbert Fanselow (1959--2022), and Manfred Bierwisch (1930--2024). 

The original articles were selected for various reasons. We did not try to establish a ranking in significance or quality, but instead chose papers which are of particular importance from our subjective point of view. All in all, they cover a wide range of topics and many years of work, the earliest of these papers dating back to 1983, and they make it clear that issues that are now highly topical have been the subject of linguistic research for a long time. Representational conventions change, of course, over a period of more than forty years. We tried to preserve the original way of presentation as far as possible and adapt it only where absolutely necessary for the understanding of the text.

\newpage
Generally speaking, Ilse Zimmermann was concerned with what she referred to as \textit{wechselseitige Zuordnung von Laut und Bedeutung}, that is the systematic correlation between sound and meaning in natural language. Her linguistic approach was based on a modular grammar in the Chomskyan tradition. Following work by Manfred Bierwisch, she assumed Semantic Form (SF) as the linguistic level where abstract meaning representations of basic linguistic items are combined according to rule schemata for deriving the meaning of complex expressions. Adhering to lexicalism, she suggested explicit and yet economical lexical entries as the basis for both morphosyntactic structure building and semantic amalgamation. Among her major contributions to linguistic theory was the introduction of various semantic templates that serve to adapt expressions to fit in the contexts in which they are used. In an elegant way she showed that the principle of compositionality can be upheld to a maximal degree. She was known for her insistence in linguistic discussions that any (morpho-)syntactic analysis must be translatable into adequate semantic representations, especially when such relationships seemed obscure or were not considered in the proposals of other linguists. Her persistence earned her respect and admiration. In what follows we present a list of major topics that she dealt with, as can be condensed from her publications.

\begin{enumerate}

\item[(i)] Lexicon: Lexicalism, lexical information, lexical entries, lexicon-syntax interface, stems and affixes, formation of inflected forms and its impact on argument structure, verbal nouns; see, e.g., \textcite{Zimmermann1988intro} in this volume.

\item[(ii)] Morphology: Approaches to morphology -- lexicalist and declarative models, inflectional morphology, decomposition of pronouns, proadverbialia and adverbial connectives, expression of verbal mood; see, e.g., \textcite{Zimmermann2002} in this volume.

\item[(iii)] Syntax: Sentence structure, functional categories, empty heads, con\-stit\-u\-ent clauses, clauses vs. noun phrases, nominalization and argument realization, adjectival constructions, participial constructions, syntax of comparison, noun phrase structure, structural case, clitics; see, e.g., \citeauthor{Zimmermann1983} (\citeyear{Zimmermann1983}, \citeyear{Zimmermann1991intro}, \citeyear{Zimmermann1997}) in this volume.

\item[(iv)] Semantics: Lexical meaning, decomposition, syntax–semantics interface, compositional semantics, functional categories, embedded clauses with and without correlates, semantic templates, modification, meaning contribution of lexical cases, reflexives, reciprocals, wh-expressions, sentence types, sentential mood, illocutions, speaker attitude, sentential adverbials, modality, worlds, factivity; see, e.g., \citeauthor{Zimmermann2008} (\citeyear{Zimmermann2008}, \citeyear{Zimmermann2016intro}, \citeyear{Zimmermann2018intro}, \citeyear{Zimmermann2020}) in this volume.

\end{enumerate}

\noindent (Note that the papers above, in fact, belong to more than one topic area. We have assigned them to one class each for reasons of simplicity. Readers interested in specific topics are referred to the list of publications included in this volume.)

One excellent example of Zimmermann's way of conducting research is her \citeyear{Zimmermann1997} paper in which she investigated adverbial phrases and clauses headed by (German) \textit{so} ‘so’, ‘such’ and \textit{wie} ‘how’, ‘like’ and developed a syntactic and semantic analysis of adverbial \textit{so}- and \textit{wie}-phrases emphasizing their proximity to modal sentence adverbials and speech comments. Zimmermann considered this paper as a summary and culmination point of several preceding studies on special aspects of the syntax and semantics of \textit{so}- and \textit{wie}-phrases. 

We would have liked to include another paper in this volume in which she compared subordinated clauses introduced by German \textit{wie} and Russian \textit{kak} ‘how’, ‘as’. This paper was the very last one she wrote except for the étude (her expression) on \textit{wie} in March 2020 (see \cite{Zimmermann2020} in this volume). The topic of her last paper was also the last topic of her last presentation (February 2020, University of Potsdam). Unfortunately we did not succeed in convincing the heirs to give their consent, which is a great pity. We are sure Ilse would have wanted to see her final paper published.

\addsec{Biographical details}
\largerpage
Ilse Zimmermann was born on 13 July 1928 in Ströbitz (now part of Cottbus, Brandenburg). She began her career in 1949 as a teacher of Russian at primary and secondary schools near Cottbus, then at secondary schools in Potsdam and Werneuchen. Later she taught at teacher training colleges (Pädagogische Hoch\-schu\-len) in Frankfurt/Oder and Potsdam. In 1980, after completing her postgraduate studies at the Potsdam College of Education and the Academy of Sciences in Berlin she received her doctorate with a dissertation titled \textit{Studien zur Beziehung von Syntax und Semantik} (‘Studies on the relationship between syntax and semantics’). From 1968 to 1988, she was a member of the Arbeitsgruppe Strukturelle Grammatik (‘Structural grammar research group’) and follow-up groups at the Academy of Sciences in Berlin, working together with Manfred Bierwisch and Ewald Lang. Academic work at the time was accompanied by numerous restrictions such as major difficulties in  accessing Western publications and publishing in the Western journals, and in participating in conferences in Western countries.

In the 1990s, Ilse Zimmermann held teaching assignments in  research centers at the universities of Lund, Leipzig and Stuttgart. She also had close contact with linguists at the Centre for General Linguistics in Berlin (now Leibniz-ZAS), which resulted in numerous publications. At the same time she supervised Master's and doctoral theses such as those of Kerstin Schwabe, Andreas Späth, and Ulrike Bischof, neé Huste. In 2003, the University of Potsdam awarded her an honorary doctorate. In July 2018, the University of Potsdam and the Leibniz-ZAS held a colloquium in honor of her 90th birthday, where she herself contributed a lecture on Spanish verb inflection. In December 2019, she was included in the \textit{ZAS Wall of Fame}, alongside other outstanding linguists associated with the Leibniz-ZAS.

Ilse Zimmermann died on 20 June 2020, at the age of almost ninety two. She greatly enriched the academic and sometimes also the private life of many linguists, including the editors of this volume. Some statements by students and colleagues are listed below.\medskip

\addsec{Remembering Ilse Zimmermann}

\noindent \textbf{Boban Arsenijević (Graz)}: Beim ersten sprachwissenschaftlichen Kolloquium, das ich während meiner Potsdamer Zeit 2014--2016 dort besuchte, fand ich es auffällig, dass keine KollegInnen aus Berlin kamen. Gisbert Fanselow sagte mir, dass dies eher die Regel als die Ausnahme sei, machte mich aber auch auf diese zerbrechlich wirkende Frau aufmerksam, von der er mir erzählte, dass sie zu so ziemlich jedem Vortrag, der in Potsdam stattfindet, kommt und sich aktiv in die Diskussion einbringt. Die Frau war natürlich Ilse Zimmermann, und in den nächsten zwei Jahren hatte ich Gelegenheit, nicht nur Gisberts Worte bestätigt zu finden, sondern Ilse auch gelegentlich auf einen Tee und eine Diskussion über die Einbettung von Sätzen zu treffen. \medskip

\noindent \textbf{Ulrike Bischof, neé Huste (Ringenhain)}: Üppig gelb blüht der gefüllte Ranunkelstrauch aus Ilses Garten in jedem Frühjahr und lässt mich an sie denken. Dabei ist es über 35 Jahre her. Als Betreuerin hatte sie mich und „meine russischen Verbalabstrakta“ unter ihre Fittiche genommen, als mich die Karl-Marx-Universität 1986/1987 für ein Jahr an das Zentralinstitut für Sprachwissenschaft der Aka\-de\-mie der Wissenschaften delegierte. Was war das für ein Privileg, in die ge\-dank\-li\-che Welt der Arbeitsgruppe Kognitive Linguistik eintauchen zu dürfen, an den Diskussionen teilzunehmen, wichtige Literatur kennenzulernen! Dazu gehörten auch Ermahnungen zum präzisen Denken, zur Kürze und Klarheit der Sprache. In diese Zeit fällt auch der XIV. Internationale Linguistenkongress in Berlin, mit dem sich die DDR kurz für die Welt öffnete. Meine beiden Doktormütter, Heide Crome-Schmidt und Ilse Zimmermann, waren für mich der nicht hinterfragte Beweis von Gleichberechtigung. Auch wenn ich später Syntax und Semantik gegen das Übersetzen und Dolmetschen eingetauscht habe, die Dankbarkeit für Ilses fordernd-motivierende und uneigennützige Begleitung meines Weges bleibt. \medskip

\noindent \textbf{Boštjan Dvořák (Berlin)}: Mit Ilse Zimmermann entschwand für mich der Garten Eden der Linguistik; die beflügelnden Treffen in ihrem erquickend schönen Gar\-ten und die spannenden Gespräche an ihrem gedeckten Tisch, ihre nie enden wollende Neugier, die Offenheit für Neues, Überraschendes und Ungewohntes, das Gespür für Regeln und Phänomene, ihre Fähigkeit und Mühe, die verschiedensten Beobachtungen, Ansichten und Theorien, die sich jede/r von uns einzeln einverleibte und sich an ihnen erfreute, kritisch und konstruktiv zu hinterfragen und mit beispielloser Sorgfalt und Verantwortung samt den Autoren und Autorinnen zu integrieren, in ein großes, allumfassendes Netz einzubauen, in dem jedes einzelne Objekt samt dem betrachtenden Subjekt seinen Platz fand und wirkte und die Forschung um der Forschung willen existieren durfte -- das Bewundern der sprachlichen und der natürlichen Strukturen -- ja, so klang das Hohe Lied der Wissenschaft. Die Freude, die sie an der forschenden Arbeit genoss und ausstrahlte, sie leuchtet aus den zahllosen wissenschaftlichen Beiträgen wie auch aus dem Bild der blühenden \textit{Rosa hugonis}, das sie mir zuletzt von ihrem Garten mit den Worten „Nun geht es auf Ostern zu“ zuschickte. \medskip

\noindent \textbf{Steven Franks (Bloomington, Indiana)}: Ilse Zimmermann was a regular and active participant at Slavic linguistics conferences. I looked forward to interacting with her at these meetings, listening to her insightful comments and suggestions to presentations in diverse areas of syntax, semantics, and morphology. I last met her at the Slavic Linguistics Society meeting in Potsdam in 2019, not long before her passing, and was fortunately able to include her posthumous contribution \textit{On pronouns relating to clauses} in the volume of selected proceedings (in honor of Peter Kosta). She was a prodigious scholar, whose work continues to have broad impact in bringing formal approaches to Slavic languages to the larger linguistic community, in Germany and beyond. Ilse Zimmermann was a significant force in advancing Slavic linguistics and she will be sorely missed by all who knew her. I am honored to be included in that number.

\noindent \textbf{Martin Haspelmath (Leipzig)}: Als ich im Frühjahr 1990 von Köln an die FU Berlin kam, war die Stadt in einem historischen Umbruch, und gerade für die Ostberliner:innen änderte sich sehr viel. Ich wusste, dass es dort sehr gute Linguistik gab, und ich machte manchmal Ausflüge zum Zentralinstitut für Sprachwissenschaft in Pankow. Als der FAS (Forschungsschwerpunkt Allgemeine Sprachwissenschaft) in Berlin Mitte gegründet wurde, wurde es noch leichter, diese Kontakte zu pflegen. Mit zu den aktivsten Leuten der Gruppen um Manfred Bierwisch und Wolfgang Ullrich Wurzel gehörte Ilse Zimmermann. Meine Agenda war damals eine andere (ich suchte eher nach Alternativen zur generativen Grammatik), während sich Bierwisch und sein Umfeld natürlich über die Freiheit freu\-ten, ihre formalen Ansätze weiterzuverfolgen. So hatte ich inhaltlich nicht viele Überschneidungen, aber es gab viele Begegnungen (auch später auf DGfS-Ta\-gun\-gen), und ich habe gerade Ilse Zimmermann in sehr positiver Erinnerung. \medskip

\noindent \textbf{Valéria Molnár (Malmö)}: Wir hatten an der Universität Lund das Privilegium, Ilse schon am Ende der 1980er Jahre kennenzulernen und sie während mehrerer Jahrzehnte als Gastprofessorin in Lund regelmäßig treffen zu können. Sie hat bei uns den ersten sehr geschätzten Semantikkurs noch vor dem Fall der Mauer gehalten und hat uns Doktoranden während unseres Studiums und auch in den späteren Phasen der Karriere unterstützt. Wir haben Ilse auch oft auf Konferenzen in Rendsburg getroffen, die von Inger (Rosengren) und Marga (Reis) organisiert wurden. Diese haben im Rahmen des von Inger und Marga geleiteten Projekts \textit{Sprache und Pragmatik} zu einer engen Zusammenarbeit und dem wohl\-be\-kannt\-en BRRZ (Brandt, Reis, Rosengren, Zimmermann \citeyear{Zimmermann1992a}) Artikel geführt. Ilse wurde ein wichtiges Mitglied unseres Instituts und war mit vielen von uns befreundet. Es war ein großes Erlebnis, Ilse in Potsdam zu besuchen und an ihrer wunderbaren Führung in der Stadt~-- auf dem „Linguistenpfad“ wandernd~-- teilnehmen zu können. Ihre Kenntnisse im Bereich der Kunst und Geschichte waren beeindruckend und ihre herzvolle Betreuung war unvergesslich. Herz\-li\-chen Dank, liebe Ilse! \medskip

\noindent \textbf{Susan Olsen (Berlin)}: Ilse Zimmermann habe ich im Jahre 1987 auf dem Internationalen Linguistenkongress in Ostberlin kennengelernt. So wertvoll der linguistische Kontakt für mich über die anschließenden etwa dreißig Jahre war, bleibt mir ebenso prägnant die private Zeit mit ihr in Erinnerung: ein gemeinsamer Genuss eines Live-Konzerts der 9. Symphonie Beethovens an einem schönen Sommerabend am Schloss Sanssouci, häufige Treffen zu Kaffee und selbstge\-ba\-cke\-nem Kuchen in ihrem Garten in Potsdam, auf den sie so stolz war und den sie mit fachmännischem Wissen pflegte. Unabhängig vom Wetter war sie für einen Bummel durch die Stadt Potsdam und ihre Umgebung, die sie so liebte, zu haben, wobei sie gerne über ihre persönlichen Erfahrungen in der DDR berichtete. In so vielen Hinsichten hat Ilse auf diese Weise mein Leben und das Leben anderer bereichert! \medskip

\noindent \textbf{Kilu von Prince (Düsseldorf)}: Ich habe Ilse Zimmermann als unaufdringliche, aber umso beeindruckendere Kollegin am ZAS kennengelernt. Sie fiel stets durch außergewöhnlich komplexe Gedanken und eine enorm breitgefächerte Expertise auf. Sie verband detailliertes Wissen auch über die jüngere Geschichte der deutschen Linguistik mit zeitloser Originalität. Besonders werde ich nie ver\-ges\-sen, wie sehr sie mir als junger Nachwuchswissenschaftlerin den Rücken ge\-stärkt hat, als ich noch Schwierigkeiten hatte, meine Ideen effizient zu kommunizieren. \medskip

\noindent \textbf{Inger Rosengren (Lund)}: Ich lernte Ilse kennen, als ich selbst einen Vortrag in Berlin hielt im Rahmen von \textit{Sprache und Pragmatik} (Lunder germanistische Forschungen). In der Diskussion nach dem Vortrag bat Manfred Bierwisch um das Wort und teilte mit, dass er eigentlich gar nicht verstanden habe, was ich hatte sagen wollen. Ehe ich antworten konnte, ergriff Ilse das Wort und sagte mit ihrer freundlichen Stimme zu Manfred, dass sie sehr gut verstanden habe, was ich sagen wollte, worauf sie dann eine Zusammenfassung gab, auf die Manfred jedoch nicht antwortete. Nach dem Vortrag schlug Wolfgang Motsch mir vor, Ilse nach Lund einzuladen. Das taten wir. Zusammen haben wir dann vieles geschrieben. Wir wussten sie sehr zu schätzen. \medskip

\noindent \textbf{Andreas Späth (Leipzig)}: Offenheit und Konsequenz, das vor allem verbinde ich mit Ilse Zimmermann. Mir hat immer imponiert, wie sie auch anderen Methoden und Schulen unvoreingenommen begegnete, Gemeinsamkeiten in den Erkenntniszielen suchte und fand. Konkurrenzdenken war ihr fremd. Ich behalte Ilse Zimmermann mit ihrem Streben nach Plausibilität und ihrem von Exaktheit ge\-präg\-ten Herangehen sowie mit ihrem Wohlwollen und ihrer immerwährenden Zuversicht in dankbarer Erinnerung. \medskip

\largerpage
\noindent \textbf{Dieter Wunderlich (Berlin)}: Ilse Zimmermann erwarb sich ganz besondere Verdienste für die Entwicklung der Linguistik in Berlin, Potsdam und ganz Deutschland. Sie stand für eine Art von Kontinuität in unserer Disziplin, die in dieser Art einmalig war. Darüber hinaus war Ilse eine ständige Förderin junger Linguistinnen und Linguisten. Eine große Zahl von Dissertationen, Diplomarbeiten und jungen Karrieren wären ohne ihre aufmerksame und fordernde Begleitung so nicht zustande gekommen. \medskip

\noindent Finally, the editors would like to express their great esteem and gratitude. \medskip

\noindent \textbf{Łukasz Jędrzejowski (Kristiansand)}: Ich habe Ilse Zimmermann im Januar 2010 in Wro\-cław bei einer Tagung kennengelernt. Sie war eine beeindruckende und belesene Person, die Zuhörerinnen und Zuhörer zu begeistern vermochte. Seit dieser Zeit trafen wir uns regelmäßig in ihrem Potsdamer Garten bei Tee und Kuchen und tauschten unsere Gedanken zu verschiedensten Themen aus. Niemand hat sich für meine persönliche Entwicklung so stark interessiert und sie so intensiv verfolgt wie Ilse Zimmermann. Ihre Anwesenheit in einer der ersten Reihen während meiner Disputation im Dezember 2015 bedeutete mir sehr viel. Für all das bin ich ihr sehr dankbar. \medskip

\noindent \textbf{Uwe Junghanns (Göttingen)}: Ilse Zimmermann hat mich persönlich und fachlich geprägt. Ihre überlegte, sachliche und bestimmte Art -- in linguistischen Diskussionen wie auch im privaten Kontakt -- begründete für mich ein Muster für menschliche Größe, wissenschaftliche Redlichkeit und Durchsetzungskraft. \medskip

\noindent \textbf{Kerstin Schwabe (Berlin/Mannheim)}: Ich lernte Ilse Zimmermann kennen, als ich Ende 1977 am Zentralinstitut für Sprachwissenschaft zu arbeiten begann. Sie war in der Arbeitsgruppe von Manfred Bierwisch, ich bei den Anglisten. 1983, als ich in die Arbeitsgruppe „Kommunikative Semantik“ von Wolfgang Motsch wechselte, wurde sie die Betreuerin meiner Dissertation \textit{Syntax und Semantik situativer Ellipsen}. Die Zusammenarbeit war sehr intensiv. Ich bin ihr überaus dankbar für all das, was ich durch sie gelernt, erfahren und erreicht habe. \medskip

\noindent \textbf{Carla Umbach (Berlin)}: Als Ilse Zimmermann mir -- damals noch Informati\-kerin, die vorsichtig die Fühler zur Linguistik ausstreckte -- einen Entwurf ihres Artikels zu \textit{so} und \textit{wie} gab, mit der Bemerkung „Das könnte Sie interessieren“,  hatte das maßgeblichen Einfluss darauf, dass ich schließlich Linguistin wurde. Ich bin ihr nach wie vor dankbar dafür. 

\largerpage
\defbibnote{IntroNote}{For a comprehensive list of publications by Ilse Zimmermann, see the correspondingly titled chapter below.}
\printbibliography[heading=subbibliography,notkeyword=this,prenote=IntroNote]


% \thispagestyle{empty}
% \hbox{}
% \vfill
\begin{figure}
%    \hbox{}\hfill
    \includegraphics[height=6.5cm]{figures/Zimmermann_Foto_final.jpg}\\

    {\small Ilse Zimmermann in her garden in 2018}%
%    \hfill%
%    \includegraphics[height=5.5cm]{figures/2.Baum.jpg}%
%    \hfill\hbox{}
\end{figure}
\vfill
\pagebreak

\end{document}
