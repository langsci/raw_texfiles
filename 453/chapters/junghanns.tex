\documentclass[output=paper]{langscibook}
\ChapterDOI{10.5281/zenodo.15471453}
\author{Uwe Junghanns\orcid{}\affiliation{University of Göttingen}}
\title{Adverbial modification with causative verbs}
\abstract{It is not unusual to render the meaning of verbs just by naming a monolithic semantic predicate and indicating its adicity. This approach does not uncover the differences between various verb classes, however. One type of evidence that speaks against such treatment is modification. In the present chapter, I will show that the various cases of adverbial modification with causative verbs require the decomposition of their lexical meaning. Consequently, the semantic representations must display a higher degree of granularity than \citet{Davidson1967a} suggested for action verbs. Adverbials may be anchored to variables that refer to cause, effect, or state. A dilemma emerges: Semantic integration of an adverbial modifier requires that the relevant variable be bound by a $λ$-operator (cf. \citealt{Higginbotham85On-semantics,Bierwisch1988,Stechow2012}). Only the highest-ranked referential argument variable can be $λ$-bound at the point of determining reference. I will offer a solution that is based on the modification template as originally proposed by \citet{Zimmermann1992}. The template, however, needs to be refined in order for technical implementation to work and, thus, cover the entire range of cases. The empirical data discussed in the present chapter come from the Slavic languages.}
\IfFileExists{../localcommands.tex}{
  \addbibresource{../localbibliography.bib}
  \usepackage{langsci-optional}
\usepackage{langsci-gb4e}
\usepackage{langsci-lgr}

\usepackage{listings}
\lstset{basicstyle=\ttfamily,tabsize=2,breaklines=true}

%added by author
% \usepackage{tipa}
\usepackage{multirow}
\graphicspath{{figures/}}
\usepackage{langsci-branding}

  
\newcommand{\sent}{\enumsentence}
\newcommand{\sents}{\eenumsentence}
\let\citeasnoun\citet

\renewcommand{\lsCoverTitleFont}[1]{\sffamily\addfontfeatures{Scale=MatchUppercase}\fontsize{44pt}{16mm}\selectfont #1}
  
  %% hyphenation points for line breaks
%% Normally, automatic hyphenation in LaTeX is very good
%% If a word is mis-hyphenated, add it to this file
%%
%% add information to TeX file before \begin{document} with:
%% %% hyphenation points for line breaks
%% Normally, automatic hyphenation in LaTeX is very good
%% If a word is mis-hyphenated, add it to this file
%%
%% add information to TeX file before \begin{document} with:
%% %% hyphenation points for line breaks
%% Normally, automatic hyphenation in LaTeX is very good
%% If a word is mis-hyphenated, add it to this file
%%
%% add information to TeX file before \begin{document} with:
%% \include{localhyphenation}
\hyphenation{
affri-ca-te
affri-ca-tes
an-no-tated
com-ple-ments
com-po-si-tio-na-li-ty
non-com-po-si-tio-na-li-ty
Gon-zá-lez
out-side
Ri-chárd
se-man-tics
STREU-SLE
Tie-de-mann
}
\hyphenation{
affri-ca-te
affri-ca-tes
an-no-tated
com-ple-ments
com-po-si-tio-na-li-ty
non-com-po-si-tio-na-li-ty
Gon-zá-lez
out-side
Ri-chárd
se-man-tics
STREU-SLE
Tie-de-mann
}
\hyphenation{
affri-ca-te
affri-ca-tes
an-no-tated
com-ple-ments
com-po-si-tio-na-li-ty
non-com-po-si-tio-na-li-ty
Gon-zá-lez
out-side
Ri-chárd
se-man-tics
STREU-SLE
Tie-de-mann
}
  \togglepaper[1]%%chapternumber
}{}

\begin{document}
\maketitle
%\shorttitlerunninghead{}%%use this for an abridged title in the page headers

\section{Introduction}

The investigation presented in this contribution is concerned with causative verbs, adverbial modifiers and their “anchors” (i.e. the targets of adverbial predication). As will be seen, the question of how to integrate adverbials in sentence semantics is of crucial importance. There must be more than one way of integration -- adverbial modifiers may target different entities, not only events as such.\footnote{Events as targets of adverbial modifiers were originally suggested by \citet{Davidson1967a}.} Looking from another angle, we find that adverbials may be used as a heuristic means to pin down the meaning structure of causative verbs.

\largerpage
An initial observation is that adverbials have the potential to be related to a diversity of aspects of what is communicated: the causing situation -- \REF{ex:junghanns:1a}--\REF{ex:junghanns:1b}, the agent -- \REF{ex:junghanns:1c}, the caused situation -- \REF{ex:junghanns:1d}, the duration of the resulting state \REF{ex:junghanns:1e}, or the degree, the extent of the resulting state – \REF{ex:junghanns:1f}.\footnote{The examples that are used to illustrate the points made in this contribution come from the Slavic languages. Languages are abbreviated as follows: BCS (Bosnian, Croatian, Serbian) · Bel(arusian) · Bg (Bulgarian) · Croat(ian) · Cz(ech) · LSorb (Lower Sorbian) · Mac(edonian) · OCS (Old Church Slavonic) · Po(lish) · Ru(ssian) · Serb(ian)  · Sk (Slovak) · Slvn (Slovenian) · Ukr(ainian) · USorb (Upper Sorbian). For examples that were extracted from websites, the date when the news report appeared is specified. Where no source is indicated, the examples were constructed and discussed with native speakers.}\textsuperscript{,}\footnote{In the example sentences, the adverbials that are relevant for the discussion are given in square brackets. The label XP is meant to cover different cases of realization, with the syntactic head X \textrm{${\in}$} \{P, N, Adv\}, cf., e.g., \REF{ex:junghanns:1a}--\REF{ex:junghanns:1c}. One should, however, bear in mind that there are good reasons to analyze all occurrences of adverbials as PPs. I refrain from discussing the details of such a unified analysis here.} The mapping from abstract sentence semantics to conceptual structure would have to be taken as intransparent, if the semantic decomposition of the involved causative verbs was not assumed.

\ea%1
    \label{ex:junghanns:1}
  \ea  \gll   [\textsubscript{XP} V pátek] 25letý útočník zabil jednoho policistu.\\
    ~ on Friday.\textsc{acc} 25\_year\_old attacker.\textsc{nom.sg.m} kill.\textsc{lp.sg.m} one.\textsc{gen/acc.sg.m} policeman.\textsc{acc.sg.m}\\
    \glt ‘On Friday a twenty-five-year-old attacker killed a police officer.’ \\\ \hfill (Cz, cf. novinky.cz 2021\_0403) \label{ex:junghanns:1a}

  \ex  \gll   Nacbank               [\textsubscript{XP} mynuloho tyžnja] prodav rekordnu  kil’kist’ valjuty.\\
    national\_bank.\textsc{nom.sg.m} ~                 last.\textsc{gen} week.\textsc{gen} sell.\textsc{lp.sg.m} record.\textsc{acc}  amount.\textsc{acc} convertible\_currency.\textsc{gen}\\
    \glt ‘Last week the National Bank sold a record sum of convertible currency.’ \hfill (Ukr, pravda.com.ua 2022\_0530) \label{ex:junghanns:1b}

  \ex \gll   Toj [\textsubscript{XP} namerno] go skrši prozorecot. \\
            \textsc{3sg.m.nom}   ~                  intentionally \textsc{cl.3sg.m.acc} break.\textsc{aor.3sg} window.\textsc{sg.m.def}\\
    \glt ‘He broke the window intentionally.’ \hfill (Mac) \label{ex:junghanns:1c}

  \ex  \gll   Vratata              se            zatvori                 [\textsubscript{XP} s trjasâk].\\
            door.\textsc{sg.f.def} \textsc{refl} close.\textsc{aor.3sg} ~                   with bang\\
    \glt ‘The door closed with a bang.’ \hfill (Bg) \label{ex:junghanns:1d}

  \ex  \gll   Neugomonnaja vražda nas razdelila [\textsubscript{XP} navsegda]!\\
    incessant.\textsc{nom.sg.f} enmity.\textsc{nom.sg.f} \textsc{1pl.acc}  divide.\textsc{lp.sg.f} ~  for\_good.\textsc{adv}\\
    \glt ‘Incessant enmity divided us for good!’ \hfill (Ru, Lermontov: \textit{Dva brata}) \label{ex:junghanns:1e}

  \ex  \gll   Siły rosyjskie [\textsubscript{XP} całkowicie] zniszczyły Donbas.\\
    force.\textsc{nom.pl} Russian.\textsc{nom.pl} ~ completely destroy.\textsc{lp.pl} Donbas.\textsc{acc}\\
    \glt ‘Russian (military) forces completely destroyed the Donbas.’ \\ \hfill (Po, tvn24.pl 2022\_0520) \label{ex:junghanns:1f}
\z
\z

\noindent It is important to emphasize that, in the semantic field of causativity, we have to distinguish between lexical causatives and causative constructions \citep{Kulikov2001}.\footnote{For lexical causatives, see, e.g., \citet{Fabricius-Hansen1991}. Causative constructions are dealt with by 
\citet{Song2011b,Song2011a}, inter alia.} The differences between the two types manifest themselves in the number of lexical items involved, the steps and complexity of syntactic structure building, and the way in which the semantic representation of the sentence is created.

The examples in \REF{ex:junghanns:2}--\REF{ex:junghanns:5} present verbs that are lexically causative. Since the Slavic languages are morphologically rich, one would expect causative verbs to be recognizable by a morphological marker. However, not every causative verb has an exponent of causativity, e.g., a suffix. Compare the (a) and (b)-examples in \REF{ex:junghanns:2}--\REF{ex:junghanns:5}. Evidently, derived causative verbs form only a subset of the entire set.
\largerpage[2]

\ea%2
    Russian: \label{ex:junghanns:2}
  \ea  vy/žeč’ (\textsc{pfv}/\textsc{ipfv}, vt) ‘burn’ < *žēg-ti\footnote{The convention of historical linguistics to mark reconstructed forms with an asterisk is used where applicable.}\textsuperscript{,}\footnote{{}-\textit{ti} is the infinitive marker in Proto-Slavic. The modern Slavic languages show variants of the marker that resulted from language-specific changes. The prefixes (separated by the slash) contribute perfective (grammatical) aspect.}
  \ex vy/suš-i-t’ (\textsc{pfv}/\textsc{ipfv}, vt) ‘dry’ < *sux- ‘dry’     
\z
\ex%3
    Polish: \label{ex:junghanns:3}
  \ea    u/kraść (\textsc{pfv}/\textsc{ipfv}, vt) ‘steal’ < *krad-ti
  \ex u/top-i-ć (\textsc{pfv}/\textsc{ipfv}, vt) ‘drown, immerse sb., sth.’ < *top-\footnote{Compare \textit{to-n}\textrm{\textit{ą}}\textit{{}-ć} (vi) ‘drown, sink, go under (in water)’ < *\textit{top-n}\textrm{\textit{ǫ}}\textit{{}-ti} \citep[60]{Leskien1922}.}     
\z
\ex%4
    Croatian: \label{ex:junghanns:4}
  \ea po/mes-ti (\textsc{pfv}/\textsc{ipfv}, vt) ‘sweep (out)’ < *met-ti
  \ex po/čist-i-ti (\textsc{pfv}/\textsc{ipfv}, vt) ‘clean’ < *čist- ‘clean’     
\z
\ex%5
    Slovenian: \label{ex:junghanns:5}
  \ea iz/bos-ti (\textsc{pfv}/\textsc{ipfv}, vt) ‘gouge out; sting’ < *bod-ti
  \ex iz/modr-i-ti (\textsc{pfv}/\textsc{ipfv}, vt) ‘make sb. wise’ < *mǫdr- ‘wise’ 
\z
\z

\noindent Causative constructions are syntactically complex structures in contrast to causative verbs, which are syntactic atoms.\footnote{A lexicalist approach to morphology is adopted here.} See examples \REF{ex:junghanns:6a}--\REF{ex:junghanns:6h}.

\ea%6
    \label{ex:junghanns:6}
  \ea \gll   Cěsar’ĭ sŭtvori jĭ slyšati rěčĭ sijǫ.\\
    emperor.\textsc{nom} make.\textsc{aor.3sg} \textsc{3sg.m.acc} hear.\textsc{inf} speech.\textsc{acc} this.\textsc{acc}\\
    \glt ‘The Emperor had him (Constantine) listen to this matter (lit. speech).’\\ \hfill (OCS, \textit{Vita Constantini}) \label{ex:junghanns:6a}

  \ex \gll   Evo ću ih učiniti da dođu i da se poklone pred nogama  tvojim.\\
    behold \textsc{aux}.\textsc{1sg} \textsc{cl.3pl.acc} make.\textsc{inf} \textsc{prtcl} come.\textsc{prs.3pl} and \textsc{prtcl} \textsc{refl} bow.\textsc{prs.3pl} before feet.\textsc{ins.pl} your.\textsc{ins.pl}\\
    \glt ‘Behold, I will make them come and bow before thy feet.’ \\ \hfill (Serb, Rev. 3:9)

  \ex \gll   Učinio ju je da u životu ponovo vidi smisao.\\
    make.\textsc{lp.sg.m} \textsc{cl.3sg.f.acc} \textsc{aux}.\textsc{3sg} \textsc{prtcl} in life.\textsc{loc} anew.\textsc{adv} see.\textsc{prs.3sg} sense.\textsc{acc}\\
    \glt ‘He brought about that she perceived her life as meaningful once again.’ \hfill (Croat, cf. jutarnji.hr 2022\_0715)

  \ex \gll Turecký prezident nechal ruského prezidenta čakať takmer minútu.\\
    Turkish.\textsc{nom.sg.m} president.\textsc{nom.sg.m} let.\textsc{lp.sg.m} Russian.\textsc{gen/acc.sg.m} president.\textsc{gen/acc.sg.m} wait.\textsc{inf} almost minute.\textsc{acc}\\
    \glt ‘The Turkish president let the Russian president wait for almost a minute.’ \hfill (Sk, startitup.sk 2022\_0720)

  \ex \gll  Żona zmusiła go do sprzedania PS5.\\
    wife.\textsc{nom.sg.f} force.\textsc{lp.sg.f} \textsc{cl.3sg.m.acc} into selling.\textsc{gen} {PS5.\textsc{gen}}\\
    \glt ‘His wife forced him to sell the PlayStation 5.’ \\ \hfill (Po, {tech.wp.pl 2020\_1130)}

  \ex \gll   Ty sy ju glucnu wucynił.\\
    \textsc{2sg.nom} \textsc{aux}.\textsc{2sg} \textsc{3sg.f.acc} happy.\textsc{acc.sg.f} make.\textsc{lp.3sg.m}\\
    \glt ‘You made her happy.’ \\ \hfill(LSorb, https://niedersorbisch.de/dnw/woerterbuch/machen)

  \ex \gll   Anonimnyj žurnalist podverg operu Šostakoviča   rezkoj kritike.\\
    anonymous.\textsc{nom.sg.m} journalist.\textsc{nom.sg.m} subject.\textsc{lp.sg.m} opera.\textsc{acc} Shostakovich.\textsc{gen} harsh.\textsc{dat} criticism.\textsc{dat}\\
    \glt ‘An anonymous journalist subjected Shostakovich’s opera to harsh criticism.’ \hfill (Ru)

  \ex \gll   Padstavili arhanizataraw perapisu pad rèprèsii.\\
    put.\textsc{lp.pl} organizers.\textsc{gen/acc.pl.m} census.\textsc{gen} under repression.\textsc{acc.pl}\\
    \glt ‘They subjected the organizers of the census to repressions.’ \\ \hfill(Bel, cf. bielarus.net 2020\_0603) \label{ex:junghanns:6h}
\z
\z
\largerpage[2]
\noindent Causativity in natural languages has been a long-standing topic in linguistics. See \citet{McCawley1968uwe,NedjalkovSilnickij1969a,NedjalkovSilnickij1969b,Fodor1970,Wierzbicka1975,Comrie1976,Farber1976,Shibatani1976ed,Shibatani1976,Shibatani2002ed,Talmy1976,Wali1981,KemmerVerhagen1994,LevinRappaportHovav1994,Wunderlich1997cause,Dixon2000,Kulikov2001,Bierwisch2002,Bierwisch2005,Wolffetal2002,Song2011b,Song2011a,NeelemanVandeKoot2012}, and many more. Work on causative verbs specifically in Slavic includes \citet{Batistic1978,Paduceva2001,Karlik2002,Bilandzija2014}, and \citet{Bondaruk2021}.

The empirical distinction between causative verbs and causative constructions cannot be under doubt. What has been debated, though, are the issues of whether the meaning of causative verbs is monolithic or not and whether causative verbs are lexical items and, hence, represented by a V-head in syntax or emerge by combining a number of heads whose nature is rather semantic than syntactic. For the latter approach see, e.g., \citet{McCawley1971,vonStechow1996,Alexiadouetal2006,Ramchand2008,Schaefer2008,Alexiadou2010}, or \citet{Bondaruk2021}, who have made assumptions that are based on ideas developed in the framework of generative semantics of the 1960s and revived by adherents of the theory of distributed morphology, cf. \citet{HalleMarantz1993}, amongst others.

The fact that we find both non-derived and derived causative verbs has implications for claims concerning the representation of their meaning. Some, but not all, verbs are recognizable as causative verbs through a morphological exponent, e.g., a suffix, see above, examples \REF{ex:junghanns:2}--\REF{ex:junghanns:5}. Hence, membership in the lexical class of causative verbs does not imply morphological signalling. From this it follows that  there must be a causative component in the meaning representation of the respective verbs as such. This leads to the conclusion that the meaning of causative verbs must be given in a decomposed form.

Now the question arises whether a coarse-grained analysis as in \REF{ex:junghanns:7} should suffice or whether we have to go further and suggest an even more fine-grained analysis.

\ea%7
    \label{ex:junghanns:7}
 [$x$ \textsc{cause} [\textsc{become} [\textsc{state} $y$]]]
\z

\noindent To answer this question is a major task of the present chapter. My claim is that an investigation into the issue of adverbial modification is essential for answering the question.

Two general aspects must be taken into account: (i)~The verbs encode a relation between a cause and an effect \citep{Bierwisch2005}. (ii)~Causal relations hold between events  (\citealt{Davidson1967b,Lewis1973}), or in more general terms: situations (\citealt{BarwisePerry1983}); see also \citet{McCawley1976}.

The task is to propose a precise analysis of the semantic structure of causative verbs, that is, to uncover enough structure in order to be able to adequately describe and explain the properties of the verbs and the sentences in which they appear. Prerequisites are a formal framework, explicit assumptions concerning the structure and interpretation of linguistic expressions, and an adequate semantic metalanguage. I will use a lambda-categorial language as developed in the \textit{Two-level semantics framework} (\citealt{Bierwisch1982,Bierwisch1986nature,Bierwisch2007}; for a general characterization of this framework see \citealt{Lang-Maienborn2011}).

Among the semantic ingredients are cause (causing situation), effect (caused situation), transition, source state, and target state. These ingredients imply that causative verbs belong to the so-called change-of-state verbs (\citealt{Dowty1979,Fabricius-Hansen1991,RappaportLevin2005}).

\largerpage
The lexical systems of the Slavic languages display relations between items that encode states, transitions that result in states, and caused transitions resulting in states, respectively. This might suggest a general way of building up meaning. See \tabref{tab:junghanns:1} and the paradigms in \REF{ex:junghanns:8} and \REF{ex:junghanns:9}.\footnote{\tabref{tab:junghanns:1} has been modelled after the one presented by \citet[43]{Bierwisch2005} for German.} What is exemplified here is not the only pattern, however.\footnote{See Tables \ref{tab:junghanns:2} and \ref{tab:junghanns:3} below for another pattern.}


\begin{table}
\fittable{%
\begin{tabular}{llll}
\lsptoprule
 {‘State’}   (\textsc{adj}) &  {‘Become State’}  (vi · \textsc{ipfv}/\textsc{pfv}) &  {‘Cause Become State’}  (vt) & \\
%\hhline%%replace by cmidrule{---~}
\midrule
 slěpŭ & oslĭ(p)nǫti & oslěpiti &  (OCS)\\
 slepoj & slepnut’ / oslepnut’ & oslepit’ &  (Ru)\\
 slipyj & slipnuty / oslipnuty & oslipyty &  (Ukr)\\
 sljapy & slepnuc’ / aslepnuc’ & asljapic’ &  (Bel)\\
 ślepy & ślepnąć / oślepnąć & oślepić &  (Po)\\
 slepý & slepnout / oslepnout & oslepit &  (Cz)\\
 slijep & slijepjeti / oslijepjeti & oslijepiti &  (Croat)\\
 slep & slepeti / oslepeti & oslepiti &  (Serb)\\
 sljap & slepeja, -eeš / oslepeja, -eeš & oslepja, -iš &  (Bg)\\
 ‘blind’ & ‘become blind’ & ‘make sb. blind’ & \\
\lspbottomrule
\end{tabular}%
}
\caption{One pattern of lexical relations}
\label{tab:junghanns:1}
\end{table}

\ea%8
    \label{ex:junghanns:8}
  \ea \gll   Odežda suxa.\\
    clothing.\textsc{nom.sg.f} dry.\textsc{nom.sg.f}\\
    \glt ‘The clothes are dry.’

  \ex \gll   Odežda vysoxla.\footnotemark{}\\
    clothing.\textsc{nom.sg.f} dry.\textsc{lp.sg.f}\\
    \footnotetext{According to \citet{Tixonov1985}, \textit{sux}{}- and \textit{sox}{}- are variants of one and the same root. The variants reflect an old ablaut relation, \citeauthor{Vasmer1953-1958} (1955: 704 and 1958: 54).}
    \glt ‘The clothes have dried.’

  \ex \gll   Solnce vysušilo odeždu.\\
    sun.\textsc{nom.sg.n} dry.\textsc{lp.sg.n} clothing.\textsc{acc.sg.f}\\
    \glt ‘The sun has dried the clothes.’ \hfill (Ru)
\z
\ex%9
    \label{ex:junghanns:9}
  \ea \gll   Bielizna jest sucha.\footnotemark{}\\
    laundry.\textsc{nom.sg.f} be.\textsc{prs.3sg} dry.\textsc{nom.sg.f}\\
  \footnotetext{\{-\textit{such}{}-, -\textit{sch}{}-, -\textit{susz}{}-, -\textit{sych}{}-\} are allomorphs of the root.}\\
    \glt ‘The laundry is dry.’

  \ex \gll   Bielizna wyschła.\\
    laundry.\textsc{nom.sg.f} dry.\textsc{lp.sg.f}\\
    \glt ‘The laundry has dried.’

  \ex \gll   Słońce wysuszyło bieliznę.\\
    sun.\textsc{nom.sg.n} dry.\textsc{lp.sg.n} laundry.\textsc{acc.sg.f}\\
    \glt ‘The sun has dried the laundry.’ \hfill (Po)
\z
\z

\noindent Although the relations appear systematic, it must be emphasized that corresponding lexical items do not exist in all cases. Also, the lexical relations are not necessarily identical in the various languages, if they exist at all. While, for example, English \textit{open} (adj.) > \textit{to open} (verbum intransitivum) > \textit{to open} (verbum transitivum) may exemplify the adjective > inchoative verb > causative verb derivation (albeit without any overt morphological reflex), in Slavic we see a different path for this lexical field, with the causative verb as the starting point for two processes: (i)~The adjective derives from the stem of the passive participle of the transitive causative verb, and (ii)~the inchoative verb (“decausative”) results from combining the causative verb with the reflexive marker (cf. \cite[174]{Paduceva2003}, \cite[292]{Fehrmannetal2014}).\footnote{A similar claim was made by \citet[9]{Reinhart2016}: “Unaccusative and one-place [verbs] originate as two-place predicates, and they are derived from their transitive alternate by a reduction operation.”} Morphological structure and complexity show this clearly. See Tables \ref{tab:junghanns:2} and \ref{tab:junghanns:3}.


\begin{table}[h]
\begin{tabular}{llll}
\lsptoprule
Causative verb & Passive Participle & \\
 (infinitive) & (stem) & {Adjective} & \\
\midrule
 otkryt’ & otkry-t- & otkry-t-yj &  (Ru)\\
 otevřít & otevř-en- & otevř-en-ý &  (Cz)\\
 otvoriti & otvor-en- & otvor-en-i & (BCS)\\
 ‘to open’ (vt) & ‘opened’ & ‘open’ & \\
\lspbottomrule
\end{tabular}
\caption{Adjective derivation}
\label{tab:junghanns:2}
\end{table}


\begin{table}[h]
\begin{tabular}{lll}
\lsptoprule
{Causative verb} & {Inchoative verb} & \\
\midrule
 otkryt’ & otkryt’-sja &  (Ru)\\
 otevřít & otevřít se &  (Cz)\\
 otvoriti & otvoriti se &  (BCS)\\
 open.\textsc{inf} & open.\textsc{inf.refl} & \\
 ‘to open’ (vt) & ‘to open’ (vi) & \\
\lspbottomrule
\end{tabular}
\caption{Decausative formation}
\label{tab:junghanns:3}
\end{table}

Importantly, we must be aware of the fact that the existence of related or relatable lexical items speaks neither for nor against the decomposition of the meaning of causative verbs. The semantic representation that results from decomposing a verb’s meaning is abstract (cf. \citealt{Bierwisch2011}: 324), it does not depend on the existence or non-existence of other lexical items, and it is irrelevant whether the object language has a syntactic construction with identical or similar meaning. The discussion whether or not the English verb \textit{kill} means ‘cause to die’ is obsolete -- \textit{kill} is a verb with an abstract meaning representation, whereas the object-language expression \textit{cause to die} is a syntactic construction comprising a number of lexical items; their abstract meanings yield, by composition, the abstract meaning of the whole.\footnote{The discussion had been ongoing since the 1960s, see, e.g., \citet{McCawley1968uwe,Fodor1970,Wierzbicka1975,Shibatani1975,Shibatani1976ed,Shibatani1976,Saksena1982,FanselowStaudacher1991}, and \citet{Kulikov2001}.}

Among the issues that need to be addressed in order to solve the question of how to represent the meaning of causative verbs are the following: (i)~What is the conceptual and semantic structure of the causing situation? (ii)~Is the “causer” always an agent? (iii)~What is the conceptual and semantic structure of the caused situation? (iv)~What is the formal correlate of the causal relation?

Interestingly, adverbials will contribute to the solution, since, as I claim, the decomposition of the lexical meaning of causative verbs is necessary for covering all types of adverbial modification occurring with them. What is needed is, in fact, a fine-grained semantic representation which reflects the partial situations involved and comprises a sufficient number of variables referring to those situations.

In what follows, an approach will be developed that, as I will argue, comes up to the goal of finding an adequate analysis for causative verbs and adverbial modifiers. In \sectref{sec:junghanns:2}, I will present and discuss proposals as to how the meaning of causative verbs is to be represented. \sectref{sec:junghanns:3} will be devoted to adverbials -- their syntax and semantics (\sectref{sec:junghanns:3.1}) as well as the issues of adverbials and states (\sectref{sec:junghanns:3.2}), adverbials and caused situations (\sectref{sec:junghanns:3.3}), and adverbials and causing situations (\sectref{sec:junghanns:3.4}). Problems will come to light that concern the impossibility of semantically integrating adverbials in certain cases, unless refinements of the apparatus are introduced the necessity of which follows from empirical observations and considerations of their impact. In \sectref{sec:junghanns:4}, I will introduce my proposal, namely refinements of \citeposst{Zimmermann1992} modification template and what the newly proposed templates effect in the course of semantic composition. By the extended and refined apparatus, I will argue, it is possible to cover not only the case of adverbial modification of the causing situation, but also the cases that, in the course of the discussion in \sectref{sec:junghanns:3}, have turned out to pose problems for the semantic integration of adverbials modifying the target state and the caused situation, respectively. \sectref{sec:junghanns:5} summarizes the discussion, lists the specific versions of the modification template, and shows in what form these versions can be generalized.

\section{Representation of verb meaning}
\label{sec:junghanns:2}

In the linguistic literature, the meaning of causative verbs is often rendered as a relation as simple as that in \REF{ex:junghanns:10}.

\ea%10
    \label{ex:junghanns:10}
  $\textsc{p} (x, y)$
\z

\noindent Accordingly, verbs like English \textit{open} or \textit{kill} and their equivalents in other languages would be represented as in \REF{ex:junghanns:11a} and \REF{ex:junghanns:12a}, respectively. Consequently, these representations would have to be taken as the basis for the computation of the meaning of sentences as in \REF{ex:junghanns:11b}--\REF{ex:junghanns:11c} and \REF{ex:junghanns:12b}--\REF{ex:junghanns:12c}.

\ea%11
    \label{ex:junghanns:11}
  \ea     $\textsc{open} (x, y)$ \label{ex:junghanns:11a}

  \ex \gll   Som list wó(t)cynił.\\
    \textsc{aux}.\textsc{1sg} letter.\textsc{acc} open.\textsc{lp.sg.m}\\
    \glt ‘I opened the letter.’ \\ \hfill(LSorb, https://niedersorbisch.de/dnw/woerterbuch/oeffnen)\label{ex:junghanns:11b} 

  \ex \gll   Sym wokno wočinił.\\
    \textsc{aux}.\textsc{1sg} window.\textsc{acc} open.\textsc{lp.sg.m}\\
    \glt ‘I opened the window.’ \hfill (USorb, \citealt{Breu2011}: 161, fn 14)\label{ex:junghanns:11c}
\z
\ex%12
    \label{ex:junghanns:12}
  \ea     $\textsc{kill} (x, y)$ \label{ex:junghanns:12a}

  \ex \gll  Kain swójogo bratša wusmjerśi.\\
            Cain.\textsc{nom.sg.m} his brother.\textsc{gen/acc} kill.\textsc{pst.3sg}\\
    \glt ‘Cain killed his brother.’ \hfill ({LSorb,} cf. New Testament, 1 John 3,12)\label{ex:junghanns:12b} 

  \ex \gll   Kain swojeho bratra zabi.\\
    Cain.\textsc{nom.sg.m} his brother.\textsc{gen/acc} kill.\textsc{pst.3sg}\\
    \glt ‘Cain killed his brother.’ \hfill (USorb, cf. New Testament, 1 John 3,12)\label{ex:junghanns:12c}
\z
\z

\noindent \REF{ex:junghanns:13}--\REF{ex:junghanns:16} show some examples from the literature.

\ea%13
    \label{ex:junghanns:13}
  \ea   $\textsc{butter} (x, y)$
  \ex Jones buttered the toast. (cf. \citealt{Davidson1980} [\citeyear{Davidson1967a}]: 107)
\z
\ex%14
    \label{ex:junghanns:14}
    \textsc{hit}\textsc{\textsubscript{HARRY, MARY}}
  (cf. \citealt{Fillmore1968}: 374)
\ex%15
    \label{ex:junghanns:15}
  \ea  (for which person \textit{x}, \textit{x} hit Bill)
  \ex Who hit Bill? (cf. \citealt{Chomsky81Lectures-on-government}: 324)
\z
\ex%16
    \label{ex:junghanns:16}
  \ea     $\textsc{break} (x (y))$
  \ex     The girl broke the window. (cf. \citealt{Grimshaw1990}: 24)
\z
\z

\noindent The above representations (and other similar ones) of the meaning of causative verbs are not adequate.\footnote{I concede that those semantic notations are, of course, legitimate in case issues are investigated for whose treatment more fine-grained analyses are not required. However, if lexical semantics proper is the object of investigation, then coarse-grained representations of verb meaning clearly are simplifications.} This is for the following three reasons: (i)~It is coarse-grained and does not render the event structure. (ii)~The way conceptual interpretation could be derived from it is non-transparent and, thus, unclear. (iii)~Different anchors of adverbial modification do not have formal counterparts -- they do not occur in the analysis and would have to be inferred. The notation is not complex enough: It names the verbal predicate and indicates its adicity (number of arguments). If this is all that is stated in a representation, claims that lexical meaning should not be decomposed at all (e.g., \citealt{Putnam1975}) would be strengthened.

There is a further deficiency: Any difference between actions, on the one hand, and natural occurrences and states, on the other, could not be detected; consider \REF{ex:junghanns:17} and \REF{ex:junghanns:18}, respectively.

\ea
    \label{ex:junghanns:17}
    \ea
\gll Sunce je dostiglo zenit.\\
  sun.\textsc{nom.sg.n} \textsc{aux}.\textsc{3sg} reach.\textsc{lp.sg.n} zenith.\textsc{acc}\\
\glt   ‘The sun reached the zenith.’ \hfill (BCS)
\ex $\textsc{reach} (x, y)$
\z
\ex%18
    \label{ex:junghanns:18}
     \ea
\gll Petâr običa Marija.\\
  Petâr love.\textsc{prs.3sg} Marija\\
    \glt ‘Petâr loves Marija.’ \hfill (Bg)
    \ex $\textsc{love} (x, y)$
\z
\z

\noindent Classifications of verbs would be the result of pure intuition; it would, de facto, have to suffice to simply declare a verbal lexeme as a member of a lexical class. Structural grammar, however, has to invoke features, e.g. functors, as the basis for class formation (cf. \citealt{Steube1988}, chapter 6).

 \citet[43]{Bierwisch2005} points out systematic relations between different types of predication. Proceeding from this we may assume a schema as the following.\footnote{I should like to emphasize that this is a semantic schema. It represents structured lexical semantics but not a hierarchy of abstract syntactic heads (for a syntactic approach, see, for example, \citealt{Ramchand2008}). As I will argue below, Slavic verbs are syntactic atoms and, therefore, inserted under the lexical V-head in syntax.}

\ea%19
    \label{ex:junghanns:19}
[\cnstx{cause} [\cnstx{transition} [\cnstx{state}]]]
\z

\noindent Based on this schema we get the following representations for \textit{kill} and \textit{open}, respectively.

\ea%20
    \label{ex:junghanns:20}
[[$x$ does something] \cnstx{cause} [\cnstx{become} $\neg$[$y$ is alive]]]\\
  (cf. \citealt{Dowty1979}: 91)
\ex%21
    \label{ex:junghanns:21}
 $\lambda y \, \lambda x \, \lambda s \, [[[\cnstx{act}\, x] \, [\cnstx{cause} \, [\cnstx{become} \, [\textsc{open}\, y]]]] \, s]$\\
  \citep[337]{Bierwisch2002}
\z

\noindent The assumption of semantic predicates like \cnstx{act} (\citealt{Bierwisch2002,Bierwisch2005}), \cnstx{do}-something (\citealt{Dowty1979,Levin-MalkaRappaport1995}), or \textit{do} as a realization of a “higher predicate of intentionality” with the action verb proper more deeply embedded \citep{Ross1972} provokes the following question: Is it always the case that an action occurs as the cause?

Many causative verbs co-occur not only with agents but also with non-volitional causers. Compare:

\ea%22
    \label{ex:junghanns:22}
  \ea \gll   Mladý chlapec zabil dievča.\\
    young.\textsc{nom.sg.m} guy.\textsc{nom.sg.m} kill.\textsc{lp.sg.m} girl.\textsc{acc.sg.n}\\
    \glt ‘A young man killed a girl.’ \\ \hfill(Sk, https://www.kkbagala.sk/texty/bitky-na-papieri-su-bez-krvi/)\label{ex:junghanns:22a}

  \ex \gll   Plyn zabil dievča.\\
    gas.\textsc{nom.sg.m} kill.\textsc{lp.sg.m} girl.\textsc{acc.sg.n}\\
    \glt ‘Gas killed a girl.’ \hfill (Sk, cf. cas.sk 2009\_0112)\label{ex:junghanns:22b}
\z
\z

\noindent That is why \citet[304]{Fehrmannetal2014} proposed abstracting over the semantic predicate in order to avoid the exclusion of non-actions. They treat the predicate variable $P$ as a semantic parameter.\footnote{At the level of Conceptual Structure the parameter is specified by way of an interpretation that is based on the context.}

\ea%23
    \label{ex:junghanns:23}
 $\lambda y \, \lambda x \, \lambda s \, [[[P\, x] [\cnstx{cause} \, [\cnstx{become} \, [\textsc{open}\, y]]]] \, s]$\\
  (\citealt{Fehrmannetal2014}: 304)
\z

\noindent As stated above, adverbial modification with causative verbs may relate to different situations -- the causing situation vs. the caused situation, leaving states for the moment aside.

The local and temporal adverbial in \REF{ex:junghanns:24} and \REF{ex:junghanns:25}, respectively, predicate over the causing situation. Thus, they are represented as predicates of $s$ in the formal notation. See below, \sectref{sec:junghanns:3.4}.

\ea%24
    \label{ex:junghanns:24}
\gll [\textsubscript{XP} Pid Izjumom] ukrajins'ki vojiny zbyly rosijs'kyj vertolit.\\
  ~ near Izyum.\textsc{ins}  Ukrainian.\textsc{nom.pl} soldier.\textsc{nom.pl} shoot\_down.\textsc{lp.pl} Russian.\textsc{acc} helicopter.\textsc{acc}\\
    \glt ‘Near Izyum Ukrainian soldiers shot down a Russian helicopter.’ \\ \hfill(Ukr, pravda.com.ua 2022\_0613)
\ex%25
    \label{ex:junghanns:25}
\gll Miroslav Polcar svou přítelkyni zavraždil [\textsubscript{XP} v březnu].\\
  Miroslav Polcar his girlfriend.\textsc{acc} murder.\textsc{lp.sg.m} ~ in March.\textsc{loc}\\
    \glt ‘Miroslav Polcar murdered his girlfriend in March.’ \\ \hfill(Cz, cf. blesk.cz 2014\_1022)
\z

\noindent The example in \REF{ex:junghanns:26} presents a complex verb that is the result of attaching the reflexive marker to the transitive verb \textit{začynic’} ‘close’. In the course of this process, the argument structure of the verb is affected, which is the basis for the decausative interpretation of the sentence.\footnote{The term “decausative” goes back to \citet{Paduceva2003} who considers the reflexively marked verb as the result of a derivation process. \citet{Fehrmannetal2014} support \citeauthor{Paduceva2003}’s view that the causative component of the underlying transitive verb persists. See below, \sectref{sec:junghanns:3.3}, for further discussion.} The modal adverbial \textit{cixa} ‘quietly’ that occurs in \REF{ex:junghanns:26} characterizes the caused situation, i.e. the transition to the target state.\footnote{The equivalent of ‘door’ is a \textit{plurale tantum} in the majority of the modern Slavic languages.}

\ea%26
    \label{ex:junghanns:26}
\gll Dzvery [\textsubscript{XP} cixa] začynilisja.\\
  door.\textsc{nom.pl.f} ~ quietly close.\textsc{lp.pl.refl}\\
    \glt ‘The door quietly closed.’ \hfill (Bel)
\z

\noindent In order to cover the various cases, the semantic representation in \REF{ex:junghanns:23} has to be improved. The variable $s'$ is introduced. It stands for the caused situation. The two situation variables -- $s$ and $s'$ -- are related by the \cnstx{cause} operator.\footnote{\citet{Davidson1967b,Dowty1979}, and \citet{Bierwisch2002,Bierwisch2005}, assume a \cnstx{cause} operator relating events (situations). “:” in \REF{ex:junghanns:27} represents the asymmetric logical conjunction, cf. \citet{Heidolph1992}. The \cnstx{inst} functor maps from the situation class to an instance of the class, cf. \citet[176]{Bierwisch1990cluster}.}

\ea%27
    \label{ex:junghanns:27}
$\lambda y \, \lambda x \, \lambda s \, [[s\, \cnstx{inst} \, [P\, x]] : [[s\, \cnstx{cause}\, s'] : [s'\, \cnstx{inst} \, [\cnstx{become} \, [\textsc{open}\, y]]]]]$\\
  \hfill (\citealt{Fehrmannetal2014}: 305)
\z

\noindent The assumption of the variable $s'$ %$s’$ 
provides the basis for explaining adverbials as predicates of the caused situation. We are now able to cover both cases of adverbial predication that have been discussed so far.

Have we already reached the final, optimal, representation of the meaning of causative verbs? Consider examples such as the following:

\ea%28
    \label{ex:junghanns:28}
\gll Stanciju metro \minsp{“} Park Pobedy” zakryli [\textsubscript{XP} na dva dnja].\\
  station.\textsc{acc} subway.\textsc{indecl} {} Park Pobedy close.\textsc{lp.pl}  ~ for two.\textsc{acc} day.\textsc{gen.sg}\\
    \glt ‘They closed the subway station “Park Pobedy” for two days.’ \\ \hfill(Ru, spbdnevnik.ru 2021\_0103)
\ex%29
    \label{ex:junghanns:29}
\gll Dálnici D5 na sedmém kilometru ve směru na Prahu v neděli večer [\textsubscript{XP} na zhruba dvě hodiny] uzavřela nehoda osobního auta a motorky.\\
  motorway.\textsc{acc} D5 at seventh.\textsc{loc} kilometer.\textsc{loc} in direction.\textsc{loc} to Prague.\textsc{acc} on Sunday.\textsc{acc} evening  ~ for about two.\textsc{acc} hour.\textsc{acc.pl} close.\textsc{lp.sg.f} accident.\textsc{nom.sg.f} passenger.\textsc{gen} car.\textsc{gen} and motorbike.\textsc{gen}\\
    \glt ‘An accident beween a passenger car and a motorbike led to the closure of the D5 motorway for about two hours at kilometer seven in the direction to Prague.’ \hfill(Cz, lidovky.cz 2021\_0627)
\z

\noindent In these examples, the adverbial relates to the resulting state (target state). It turns out that a variable referring to this state must be introduced. This variable is given as $z$ in the following representation.\footnote{The variable $z$ in \REF{ex:junghanns:30} stands for an abstract state argument. “$\approx$” is a junctor that relates $z$ to an expression that characterizes it; see \citet{Maienborn2003,Maienborn2005,Maienborn2007}, \citet{Geist2006}, and \citet{Pitsch2014}.}

\ea%30
    \label{ex:junghanns:30}
$\lambda y \, \lambda x \, \lambda s \, [[s\, \cnstx{inst} \, [P\, x]] : [[s\, \cnstx{cause}\, s'] : [s'\, \cnstx{inst}\, \cnstx{become} \; [z \approx [Q\, y]]]]]$
\z

\noindent Now we have arrived at a fine-grained meaning representation for causative verbs, with the three variables $s$, $s'$ and $z$ as potential anchors for adverbial modification.\footnote{Three variables are also suggested by \citet[85]{Koontz-Garboden2009} -- $v$ (causing eventuality), $e$ (caused change-of-state event), and $s$ (resulting state).} In a way, the analysis suggested here zooms in on the meaning structure.\footnote{The global reference of a causative verb to an event is not undermined: “A verb refers to one and only one […] event, irrespective of the complex structure of causatives and inchoatives involving causation, cause, effect, transition, source [state], and target state” \citep[11]{Bierwisch2005}.}

We turn next (in \sectref{sec:junghanns:3} below) to general aspects of adverbials and the non-trivial problem of how to properly integrate adverbials in the compositional semantics.

\section{Adverbials} \label{sec:junghanns:3}
\subsection{Syntax and semantics}
\label{sec:junghanns:3.1}

In principle, there are two types of adverbials: (i)~free adverbials (adjuncts) and (ii)~necessary adverbials (complements).\footnote{Apart from these, sentences can contain parenthetic adverbials. They have a special status and are thus not considered in this paper.} In the most common case, adverbials are expressions which can freely occur in sentences. They are phrases (XPs) adjoining to phrases in sentence structure.\footnote{\textit{PPO} in \REF{ex:junghanns:31} stands for \textit{Protypovitrjana oborona} ‘air defense’.}

\ea%31
    \label{ex:junghanns:31}
\gll PPO [[\textsubscript{XP} unoči] [[ zbyla 2 litaky RF] [\textsubscript{XP} nad  Dnipropetrovščynoju]]].\\
  air\_defense.\textsc{nom.sg.f} ~ in\_night ~ shoot\_down.\textsc{lp.sg.f} two planes.\textsc{acc} Russian\_Federation.\textsc{gen} ~ above Dnipro\_region.\textsc{ins}\\
    \glt ‘During the night, the (Ukrainian) air defense shot down two airplanes of the Russian Federation that were flying over the district of Dnipro.’ \\ \hfill(Ukr, pravda.com.ua 2022\_0317)
\ex%32
    \label{ex:junghanns:32}
\gll Ozbrojenyj čolovik [[\textsubscript{XP} u supermarketi v amerykans'komu misti Buffalo] \minsp{[} rozstriljav ščonajmenše 10 ljudej]].\\
  armed.\textsc{nom.sg.m} man.\textsc{nom.sg.m} ~ in supermarket.\textsc{loc} in American.\textsc{loc} city.\textsc{loc} Buffalo {} shoot.\textsc{lp.sg.m} at\_least ten people.\textsc{gen.pl}\\
    \glt ‘An armed man shot at least ten people in a supermarket in the American city of Buffalo.’ \hfill(Ukr, pravda.com.ua 2022\_0515)
\ex%33
    \label{ex:junghanns:33}
\gll [[\textsubscript{XP} Na drodze krajowej nr 83, pod Turkiem], \minsp{[} wiatr przewrócił na bok naczepę ciężarówki, w której były puste puszki na piwo]].\\
  ~ on road.\textsc{loc} state.\textsc{loc} No. 83 near Turek.\textsc{ins} {} wind.\textsc{nom.sg.m} overturn.\textsc{lp.sg.m} on side.\textsc{acc} trailer.\textsc{acc} truck.\textsc{gen} in which.\textsc{loc} be.\textsc{lp.pl} empty.\textsc{nom.pl.f} can.\textsc{nom.pl.f} for beer.\textsc{acc}\\
    \glt ‘On state road N\textsuperscript{o}. 83, near Turek, the trailer of a truck laden with empty beer cans was overturned by the wind.’ \hfill (Po, tvn24.pl 2017\_0303)
\ex%34
    \label{ex:junghanns:34}
\gll [[\textsubscript{XP} Na gaj paradi u Jerusalimu] \minsp{[} pripadnik radikalne jevrejske organizacije Haredi [[\textsubscript{XP} nožem] \minsp{[} je ranio šestoro ljudi]]]].\\
  ~  on gay parade.\textsc{loc} in Jerusalem.\textsc{loc} {} follower.\textsc{nom.sg.m} radical.\textsc{gen} Jewish.\textsc{gen} organization.\textsc{gen} Haredi ~ knife.\textsc{ins} {} \textsc{aux}.\textsc{3sg} injure.\textsc{lp.sg.m} six people.\textsc{gen.pl}\\
    \glt ‘At the gay parade in Jerusalem a follower of Haredi, a radical Jewish organization, wounded six people with a knife.’ \hfill (Serb, rts.rs 2015\_0730)
\z

\noindent In the special case, adverbials are necessitated by the semantics of the verb and, therefore, realized as complements in syntax (see, e.g., \citealt{Steube1988}: ch. 6.2, \citealt{Werkmann2003}: 52--53). Local, directional, and temporal expressions are instances of such adverbials.\footnote{By convention, an asterisk preceding round brackets means that the object-language expression must not be omitted.}

\ea%35
    \label{ex:junghanns:35}
\gll Piotr \minsp{[} mieszka *([\textsubscript{XP} w Warszawie])].\\
  Piotr {} live.\textsc{prs.3sg} ~ in Warsaw.\textsc{loc}\\
    \glt ‘Piotr lives in Warsaw.’ \hfill (Po)
\ex%36
    \label{ex:junghanns:36}
\gll Marija \minsp{[} otiva *([\textsubscript{XP} v grada])].\\
  Marija {} go.\textsc{prs.3sg} ~ to town.\textsc{def}\\
    \glt ‘Marija goes to town.’ \hfill (Bg)
\ex%37
    \label{ex:junghanns:37}
\gll Rebënok \minsp{[} rodilsja *([\textsubscript{XP} v prošlom godu])].\\
  child.\textsc{nom.sg.m} {} be\_born.\textsc{lp.sg.m.refl} ~ in last.\textsc{loc} year.\textsc{loc}\\
    \glt ‘The child was born last year.’ \hfill (Ru)
\z

\newpage
\noindent Semantically, adverbials serve as modifiers of situations, see example \REF{ex:junghanns:38}, or states, see \sectref{sec:junghanns:3.2}.\footnote{\citet{bartsch1972adverbialsemantik}, inter alia, distinguishes between VP-modifying and S(entence)-modifying adverbials. In the Slavic languages, the difference is not brought to light by different syntactic behaviour of the adverbials.}

\ea%38
    \label{ex:junghanns:38}
  \ea \gll   Policijata zatvori ulicata [\textsubscript{XP} zaradi bomba].\\
    police.\textsc{sg.f.def} close\textsc{.aor.3sg} street\textsc{.def} ~  because\_of bomb\\
    \glt ‘The police closed the street due to a bomb alert.’ \hfill (Bg)
  \ex $R\textsubscript{\cnstx{caus}} \,  (s, \, [${\exists}$y \, [[\textsc{bomb} \, y] : [\cnstx{quant} \, y = 1]]])$
\z
\z

\noindent Adverbial modifiers of nominal expressions restrict the denotation as in example \REF{ex:junghanns:39}.

\ea%39
    \label{ex:junghanns:39}
  \ea \gll   glasovi [\textsubscript{XP} iza zida]\\
    voice.\textsc{nom.pl} ~ behind wall.\textsc{gen.sg}\\
    \glt ‘voices behind the wall’ 
    \hfill{(Croat)}

  \ex ${[[[\textsc{voice} \, x] : [\cnstx{quant} \, x \, {\supseteq} \, 2]] : [x \, \textsc{behind the wall}]]}$
  \z
\z

\noindent Specific questions arise concerning the syntactic positions of adverbials. What positions do adverbials occupy in the underlying structure and in what positions do they occur at the surface? These questions are not trivial for languages with so-called free constituent order.

There has been a long-lasting discussion concerning the criteria and factors for adverbial placement, cf. \citet{Steinitz1969,maienborn1996situation,Alexiadou1997,Frey2003,FreyPittner1999,Cinque1999,Cinque2004}, inter alia. Many linguists adhere to the view of a rigid, semantically determined hierarchy of adverbials. No convincing evidence for this can be found in the Slavic languages. In these languages, there are almost no restrictions and adverbials of different types show up in many positions. The positions that we see at the surface reflect requirements of information structure rather than fixed syntactic placement. Relative scope can be an additional factor (cf. \citealt{Junghanns2002,Junghanns2006,Szucsich2002,Werkmann2003,Biskup2011}).

In Slavic languages, free adverbials are merged as adjuncts to VP but can occur in higher positions at the surface (cf. \citealt{Junghanns2002} and \citealt{Werkmann2003}, amongst others). A relatively low base-generated position can be shown even for so-called sentential adverbials  \citep[116--117]{Junghanns2006}. Compare the following examples:

\ea%40
    \label{ex:junghanns:40}
  \ea \gll   … abych                           mu\textsubscript{i} [\textsubscript{DP} ten zvláštní a pravdivý námet]\textsubscript{j} [[\textsubscript{SA} skutečně] \minsp{[} daroval \_\textsubscript{i} \_\textsubscript{j} ]].\\
             ~ in\_order\_to.\textsc{cond.1sg} \textsc{cl.3sg.m.dat}   ~ this.\textsc{acc} special.\textsc{acc} and true.\textsc{acc} story.\textsc{acc} ~ really  {} give\_as\_present.\textsc{lp.sg.m}\\
    \glt ‘… so that I really would give him this special and true story.’ \\ \hfill(Cz, \citealt{Junghanns2006}: 117)

  \ex \gll   Vždyt’ já\textsubscript{i} ho\textsubscript{j} [[\textsubscript{XP} jednou] [[\textsubscript{SA} opravdu] [[\textsubscript{XP} úplně] [ \_\textsubscript{i} opustím \_\textsubscript{j} ]]]].\\
    after\_all \textsc{1sg.nom} \textsc{cl.3sg.m.acc} ~ someday.\textsc{adv} ~ really ~ completely ~ ~ leave.\textsc{prs.1sg}\\
    \glt ‘After all, one day I really will leave him for good.’ \\ \hfill (Cz, \citealt{Junghanns2006}: 117)
\z
\z
%\todo[inline]{either use ``\textsc{3sg.dat}'' in glosses, or him, but not ``him.\textsc{dat}''}

\noindent How do adverbials get integrated in the course of semantic composition?

The operation effecting the integration of adjunct adverbials has been given different names -- theta identification \citep{Higginbotham85On-semantics}, unification of theta roles \citep{Bierwisch1988}, or predicate modification  \citep{Stechow2012}. The approach chosen by \citet{Zimmermann1992} is specific in that it involves a modification template. The differently named operations all yield the same result: logical conjunction of predicates.

In the following example we see in \REF{ex:junghanns:41b} the result of amalgamating the meaning of the VP (modificandum) and the meaning of the adverbial (modifier), irrelevant details omitted.

\ea%41
    \label{ex:junghanns:41}
  \ea \gll   Blesk [\textsubscript{XP} v Tatrách] zabil otca rodiny.\\
    lightning.\textsc{nom.sg.m} ~ in Tatra\_mountains.\textsc{loc.pl} kill.\textsc{lp.sg.m} father.\textsc{gen/acc} family.\textsc{gen}\\
    \glt ‘In the Tatra mountains, a bolt of lightning killed the father of a family.’ \hfill (Sk, noviny.sk 2016\_0727)\label{ex:junghanns:41a}

  \ex $λs \, [[s \, \cnstx{INST} \, [P \, …]] : [[s \, \cnstx{CAUSE} \, … ] : [[\cnstx{LOC} \, s] \,{\subset} \, [\cnstx{LOC} \,  \textsc{tatra}]]]]$ \label{ex:junghanns:41b}
\z
\z
%\todo[inline]{presumably, lightning is lexically specified for masculine, so this does not have to be added in the gloss}

\noindent Necessary adverbials are semantically integrated in a different way. Since they have the status of an argument of the verb, they replace a predicate variable $Q$ in the verb’s meaning representation \REF{ex:junghanns:42}, via  functional application. Technically this is achieved by lambda conversion. As a result we get conjoined predicates.

\ea%42
    \label{ex:junghanns:42}
  $λQ \, … \, λs \, [s \, \cnstx{inst} \, […] : [Q \, s]]$
\z

\noindent The interaction of syntax and semantics sketched here is transparent and non-redundant. In other approaches (e.g., \citealt{Schaefer2008}), syntax encodes event structure, proceeding from state via change of state to causation. In my opinion, this puts a semantic overload on syntax, however. One reason to reject such an approach is the integrity of the verbal lexeme in Slavic.\footnote{For a detailed argumentation against lexical decomposition in syntax see \citet{JaegerBlutner1999}.}\textsuperscript{,}\footnote{It might be claimed that syntactic decomposition would yield positions syntactically determining specific semantic differences. However, there is no evidence for adverbials in the assumed abstract positions. Since explicit semantic representations provide the basis for the respective interpretations, an over-articulated syntax is superfluous and should be ruled out by economy requirements.}

\subsection{Adverbials and states}
\label{sec:junghanns:3.2}

Adverbials can predicate over states. This is a rather restricted case. Durative adverbials are one type that can be used in this function. They indicate the time period for which the resulting state is claimed to last or has lasted.

\ea%43
    \label{ex:junghanns:43}
\gll Stanciju metro \minsp{“} Park Pobedy” zakryli [\textsubscript{XP} na dva dnja].\\
  station.\textsc{acc} subway.\textsc{indecl} {} Park Pobedy close.\textsc{lp.pl} ~ for two.\textsc{acc} day.\textsc{gen.sg}\\
    \glt ‘They closed the subway station “Park Pobedy” for two days.’ \\ \hfill(Ru, spbdnevnik.ru 2021\_0103)
\ex%44
    \label{ex:junghanns:44}
\gll Dálnici D5 [\textsubscript{XP} na zhruba dvě hodiny] uzavřela nehoda osobního auta a motorky.\\
  motorway.\textsc{acc} D5 ~ for about two.\textsc{acc} hour.\textsc{acc.pl} close.\textsc{lp.sg.f} accident.\textsc{nom.sg.f} passenger.\textsc{gen} car.\textsc{gen} and motorbike.\textsc{gen}\\
    \glt ‘An accident beween a passenger car and a motorbike led to the closure of the D5 motorway for about two hours.’ \hfill(Cz, lidovky.cz 2021\_0627)
\ex%45
    \label{ex:junghanns:45}
\gll Obor zavrel jaskyňu   [\textsubscript{XP} na večné časy].\\
  giant.\textsc{nom.sg.m} close.\textsc{lp.sg.m} cave.\textsc{acc} ~ for eternal.\textsc{acc.pl} time.\textsc{acc.pl}\\
    \glt ‘The giant closed the cave for good.’ \hfill (Sk)
\ex%46
    \label{ex:junghanns:46}
\gll Zavřu ti hubu [\textsubscript{XP} navždy]!\\
  close.\textsc{prs.1sg} \textsc{cl.2sg.dat} mouth.\textsc{acc} ~ for\_good\\
    \glt ‘I will close your mouth for good!’ \hfill (Cz)
\z

\noindent The classification of a given adverbial as durative is not always obvious. The interpretation of an irreversible and, hence, lasting state has to be brought out by inferences based on world knowledge. Compare:

\ea%47
    \label{ex:junghanns:47}
\gll Religija ego slabuju nervnuju sistemu ubila [\textsubscript{XP} okončatel’no] svoimi protivorečijami.\\
  religion.\textsc{nom.sg.f} his weak.\textsc{acc} nerve.\textsc{acc} system.\textsc{acc} kill.\textsc{lp.sg.f} ~ ultimately \textsc{refl.poss.pl.ins} contradiction.\textsc{ins.pl}\\
    \glt ‘Religion with its contradictions once and for all wrecked his weak nerves.’ \hfill (Ru, “Veseluxa” na 11 tysjač znakov 2015\_0314)
\z
%\todo{should svoimi not be glossed as \textsc{refl.poss.pl.inst}?}

\noindent A problem arises for the semantic integration of the adverbials that target resulting states. Although the version of the semantic representation of causative verbs adopted here posits a variable $z$ which, since it refers to the resulting state, is the target of the adverbial, the semantic operation effecting the required step in the meaning composition cannot apply because $z$ is not bound by a lambda operator \REF{ex:junghanns:48} and, therefore, in some sense inert.

\ea%48
    \label{ex:junghanns:48}
 $λy \, λx \, λs \, [[s \, \cnstx{inst} \, [P \, x]] : [[s \, \cnstx{cause} \, s'] : [s' \, \cnstx{inst become} \, [z \,${\approx}$ \, [Q \, y]]]]]$
\z

\noindent A solution to this problem will be proposed in \sectref{sec:junghanns:4}.

A special case is constitued by adverbial expressions of the kind shown in the following examples.

\ea%49
    \label{ex:junghanns:49}
\gll Zbrojni syly Ukrajiny [\textsubscript{XP} povnistju] znyščyly zalyšky rosijs'koji texniky na ostrovi Zmijinyj.\\
  armed.\textsc{nom.pl} forces.\textsc{nom.pl} Ukraine.\textsc{gen} ~ completely destroy.\textsc{lp.pl} remain.\textsc{acc.pl} Russian.\textsc{gen} equipment.\textsc{gen} on island.\textsc{loc} Zmijinyj \\
    \glt ‘The Ukrainian armed forces completely destroyed the equipment left behind by the Russians on Snake Island.’ \hfill (Ukr, Fokus, 2022\_0702)
\ex%50
    \label{ex:junghanns:50}
\gll Rosijany pered Dnem nezaležnosti Ukrajiny [\textsubscript{XP} suttjevo] zbil’šyly kil’kist’ raketonosijiv u Čornomu mori.\\
  Russian.\textsc{nom.pl} before day.\textsc{ins} independence.\textsc{gen} Ukraine.\textsc{gen} ~ substantially increase.\textsc{lp.pl} number.\textsc{acc} missile\_carrier.\textsc{gen.pl} on Black.\textsc{loc} Sea.\textsc{loc}\\
    \glt ‘Before Ukraine’s independence day, the Russians substantially increased the number of aeroplanes carrying missiles in the Black Sea region.’ \\ \hfill(Ukr, pravda.com.ua 2022\_0821)
\ex%51
    \label{ex:junghanns:51}
\gll Rosjanie [\textsubscript{XP} częściowo] zniszczyli budynek szpitala rejonowego. \\
 Russian.\textsc{nom}.\textsc{pl} ~ partially destroy.\textsc{lp}.\textsc{pl} building.\textsc{acc} hospital.\textsc{gen} regional.\textsc{gen} \\
    \glt ‘The Russians partially destroyed the building of the district hospital.’ \\ \hfill(Po, cf. wyborcza.pl, 2022\_0402)
\z

\noindent These adverbials name the degree (grade, extent, scale) as to which the implicit predicate that corresponds to the resulting state holds.

Although not obvious at first sight, the adverbial in \REF{ex:junghanns:52} belongs in this group too.

\ea%52
    \label{ex:junghanns:52}
\gll Muž nožem [\textsubscript{XP} smrtelně] zranil knihovnici.\\
  man.\textsc{nom.sg.m} knife.\textsc{ins} ~ deathly wound.\textsc{lp.sg.m} librarian.\textsc{acc.sg.f}\\
    \glt ‘A female librarian was stabbed with a knife to death by a man.’ \\ \hfill(Cz, zpravy.aktualne.cz 2015\_0525)
\z


\noindent Modification by degree expressions is a topic of its own and will be left aside here.

\subsection{Adverbials and caused situations}
\label{sec:junghanns:3.3}

Adverbial modification of the caused situation can be very clearly seen in decausatives.\footnote{Slavic decausatives represent a type of interpretation of linguistic structures containing a verb that is derived from a causative verb and occurs with a reflexive marker. The marker signals a change in the argument structure of the original verb (\citealt{Fehrmannetal2014}).}

\ea%53
    \label{ex:junghanns:53}
  \ea    \textit{Transitive structure} \\
    \gll Petr [\textsubscript{XP} v tomto okamžiku] otevřel dveře.\\
        Petr.\textsc{nom.sg.m} ~ in this.\textsc{loc} moment.\textsc{loc} open.\textsc{lp.sg.m} door.\textsc{acc.pl.f}\\
    \glt ‘In this moment, Petr opened the door.’ \hfill (Cz)

  \ex    \textit{Decausative} \\
   \gll Dveře se [\textsubscript{XP} v tomto okamžiku] otevřely.\\
    door.\textsc{nom.pl.f} \textsc{refl} ~ in this.\textsc{loc} moment.\textsc{loc} open.\textsc{lp.pl}\\
    \glt ‘In this moment, the door opened.’ \hfill (Cz) \label{ex:junghanns:53b}
\z
\z

\noindent In the case of \REF{ex:junghanns:53b}, what caused the transition to the door being open remains implicit. The adverbial contributes the temporal specification of the transition and, therefore, modifies the caused situation.

One can make the same point with modal adverbials.

\ea%54
    \label{ex:junghanns:54}
  \ea \gll   Vrata su se [\textsubscript{XP} tiho] zatvorila.\\
    door.\textsc{nom.pl.n} \textsc{aux}.\textsc{3pl} \textsc{refl} ~ quietly close.\textsc{lp.pl.n}\\
    \glt ‘The door quietly closed.’ \hfill (BCS)

  \ex \gll   Drzwi otworzyły się [\textsubscript{XP} ze skrzypieniem].\\
    door.\textsc{nom.pl} open.\textsc{lp.pl} \textsc{refl} ~ with creak.\textsc{ins.sg}\\
    \glt ‘The door opened with a creak.’ \hfill (Po)

  \ex \gll   Dzvery [\textsubscript{XP} bjazhučna] začynilisja.\\
    door.\textsc{nom.pl.f} ~ soundlessly close.\textsc{lp.pl.refl}\\
    \glt ‘The door closed without any noise.’ \hfill (Bel)
\z
\z

\noindent It is not the causing situation that happens ‘quietly’, ‘with-a-creak’, or ‘soundlessly’. Rather, the transition to the state gets characterized in this way.

The Semantic Form (SF) of causative verbs in \REF{ex:junghanns:55} has an anchor for those adverbials, namely the variable $s'$.

\ea%55
    \label{ex:junghanns:55}
  $λy \, λx \, λs \, [[s \, \cnstx{inst} \, [P \, x]] : [[s \, \cnstx{cause} \, s'] : [s' \, \cnstx{inst become} \, [z \, {\approx} \, [Q \, y]]]]]$
\z

\noindent Similarly as in the case of adverbial modification of states (see the preceding sub-section), the relevant variable is not lambda-bound. Due to this, the necessary step in the process of semantic composition is technically impossible. This problem is addressed in \sectref{sec:junghanns:4}.

\largerpage
The possibility of adverbial modification of the caused situation as in the examples given above does not constitute a special, exceptional case. We find adverbial modification also with inchoatives. These have a less complex semantic structure as compared with causative verbs. However, the situations interpreted for the inchoatives in a way resemble the caused situations in the more complex structures.

\ea%56
    \label{ex:junghanns:56}
  \ea \gll   [\textsubscript{XP} Cora wjacor] jo wumrěł Michail Gorbatšow.\\
   ~ yesterday evening \textsc{aux.3sg} die.\textsc{lp.sg.m} Mikhail Gorbachev \\
    \glt ‘Mikhail Gorbachev died yesterday evening.’ \\ \hfill(LSorb, rbb-online.de/radio 2022\_0831)

  \ex \gll   Britanskata kralica Elizabet II počina [\textsubscript{XP} v zamâka Balmoral].\\
    British.\textsc{sg.f.def} queen.\textsc{sg.f} Elizabeth II die.\textsc{aor.3sg} ~ in castle.\textsc{def} Balmoral\\
    \glt ‘The British Queen Elisabeth II died in Balmoral castle.’ \\ \hfill(Bg, 24chasa.bg 2022\_0908)

  \ex \gll   Kraljica je [\textsubscript{XP} danes] [\textsubscript{XP} mirno] umrla [\textsubscript{XP} na Balmoralu].\\
    queen.\textsc{nom.sg.f} \textsc{aux.3sg} ~ today ~ peacefully die.\textsc{lp.sg.f}  ~ in Balmoral.\textsc{loc}\\
    \glt ‘The Queen today died peacefully in Balmoral.’ \\ \hfill(Slvn, zurnal24.si 2022\_0908)

  \ex \gll   [\textsubscript{XP} Wčera] je kralowna Elisabeth II. [\textsubscript{XP} w Šotiskej] zemrěła.\\
    ~ yesterday \textsc{aux.3sg} queen.\textsc{nom.sg.f} Elizabeth II ~ in Scotland.\textsc{loc} die.\textsc{lp.sg.f}\\
    \glt ‘Yesterday afternoon Queen Elizabeth II died in Scotland.’ \\ \hfill(USorb, serbske-nowiny.de 2022\_0909)
\z
\z

\subsection{Adverbials and causing situations}
\label{sec:junghanns:3.4}

The semantic integration of adverbials modifying causing situations does not pose any problem at all. The variable $s$ provides the anchor for the adverbial’s meaning contribution. This variable is technically accessible, since it is lambda-bound, cf. \REF{ex:junghanns:57}.

\ea%57
    \label{ex:junghanns:57}
  $λy \, λx \, λs \, [[s \, \cnstx{inst} \, [P \, x]] : [[s \, \cnstx{cause} \, s'] : [s' \, \cnstx{inst become} \, [z \, {\approx} \, [Q \, y]]]]]$
\z

\noindent The relevant variables of the modificandum and the modifier get identified via application of the modification rule schema, see below. As a consequence, the adverbial is predicated over the causing situation. This can be illustrated as follows:

\ea%58
    \label{ex:junghanns:58}
\gll [\textsubscript{XP} V neděli večer] policie uzavřela dálnici D5.\\
 ~ on Sunday.\textsc{acc} evening police.\textsc{nom.sg.f} close.\textsc{lp.sg.f} motorway.\textsc{acc} D5\\
    \glt ‘On Sunday evening, the police closed the D5 motorway.’ \\ \hfill(Cz, adapted from lidovky.cz 2021\_0627)
\ex%59
    \label{ex:junghanns:59}
  \ea   \textit{Semantic value of the VP}\\
    $λs \, [[s \, \cnstx{inst} \, [P \, [\textsc{the police}]]] : [[s \, \cnstx{cause} \, s'] : [s' \, \cnstx{inst} \, [\cnstx{become}$ \\$[z \,{\approx} \, [\cnstx{not accessible} \, [\textsc{motorway d5} \, ]]]]]]] $ \label{ex:junghanns:59a}

  \ex \textit{Meaning representation for the temporal adverbial}\\
    $λx \, [\cnstx{time} \, x \, {\subset} \, \textsc{sunday evening}]$ \label{ex:junghanns:59b}

  \ex \textit{Modification template}\footnote{Compare \citet[256]{Zimmermann1992}. For more details on this, see \sectref{sec:junghanns:4}.}\\
   $ λQ\textsubscript{2} \,  λQ\textsubscript{1} \,  λs \,  [[Q\textsubscript{1} \,  s] : [Q\textsubscript{2} \,  s\textsubscript{} \,  ]]$  \label{ex:junghanns:59c}

  \ex \textit{Application of the template to the modifier}\\
      $λQ\textsubscript{2} \, λQ\textsubscript{1} \, λs \, [[Q\textsubscript{1} \, s] : [Q\textsubscript{2} \, s]] \, (λx \, [\cnstx{time} \, x \, {\subset} \, \textsc{sunday evening}]) \newline
    {\equiv} \,  λQ\textsubscript{1} \, λs \, [[Q\textsubscript{1} \, s] : [λx \, [\cnstx{time} \, x \, {\subset} \, \textsc{sunday evening}] \, s]] \newline
    {\equiv} \,  λQ\textsubscript{1} \, λs \, [[Q\textsubscript{1} \, s] : [\cnstx{time} \, s \, {\subset} \, \textsc{sunday evening}]]$ \label{ex:junghanns:59d}

  \ex \textit{Application of the preceding result to the modificandum}\\
      $λQ\textsubscript{1} \, λs \, [[Q\textsubscript{1} \, s]  : [\cnstx{time} \, s \, {\subset} \, \textsc{sunday evening}]]$\\ $(λs \, [[s \, \cnstx{inst} \, [P \, [\textsc{the police}]]] : [[s \, \cnstx{cause} \, s'] : [s' \, \cnstx{inst} \, [\cnstx{become} \, [z \, {\approx} \, [\cnstx{not accessible} \, [\textsc{motorway d5} \, ]]]]]]]) \newline
    {\equiv} \,  λs \, [[λs \, [[s \, \cnstx{inst} \, [P \, [\textsc{the police}]]] : [[s \, \cnstx{cause} \, s'] : [s' \, \cnstx{inst} \, [\cnstx{become} \, [z \, {\approx} \, [\cnstx{not accessible} \, [\textsc{motorway d5} \, ]]]]]]] \, s] : [\cnstx{time} \, s \, {\subset} \, \textsc{sunday evening}]] \newline
    {\equiv} \,  λs \, [[[s \, \cnstx{inst} \, [P \, [\textsc{the police}]]] : [[s \, \cnstx{cause} \, s'] : [s' \, \cnstx{inst} \, [\cnstx{become} \, [z \, {\approx} \, [\cnstx{not accessible} \, [\textsc{motorway d5} \, ]]]]]]] : [\cnstx{time} \, s \, {\subset} \, \textsc{sunday evening}]]$ \label{ex:junghanns:59e}
\z
\z

\noindent Adverbials of one special type are restricted to modification of the causing situation, namely Agent-oriented adverbials (AOA). An illustrating example is given below.

\ea%60
    \label{ex:junghanns:60}
\gll Petr [\textsubscript{XP} záměrně] zlomil tužku.\\
  Petr.\textsc{nom.sg.m} ~ intentionally break.\textsc{lp.sg.m} pencil.\textsc{acc}\\
    \glt ‘Petr intentionally broke the pencil.’ \hfill(Cz, \citealt{Karlik2002}: 413)
\z

\noindent The meaning for the Slavic equivalents of ‘intentionally’ may tentatively be represented as follows:

\ea%61
    \label{ex:junghanns:61}
  $λs \, [[\cnstx{agent} \, (x, \, s)] : [{\exists}w \, [\cnstx{purpose} \, (w, \, s)] : [\cnstx{have} \, (x, \, w)]]]$
\z

\noindent This meaning contribution gets integrated in the usual way by means of the template (compare above, \REF{ex:junghanns:59c}--\REF{ex:junghanns:59e}. This gives us:

\ea%62
    \label{ex:junghanns:62}
$λs \, [[[s \, \cnstx{inst} \, [P \, [\textsc{peter}]]] : [[s \, \cnstx{cause} \, s'] : [s' \, \cnstx{inst} \, [\cnstx{become}$\\ $[z \, {\approx} \, [\cnstx{not intact} \, [\textsc{the pencil}]]]]]]] : [[\cnstx{agent} \, (x, \, s)] : [{\exists}w \, [\cnstx{purpose} \, (w, \, s)] : [\cnstx{have} \, (x, \, w)]]]]$
\z

\noindent The variable $x$ remains a semantic parameter. It will be interpreted as coreferential with $[\textsc{peter}]$ at the level of Conceptual Structure.

In \sectref{sec:junghanns:2}, \REF{ex:junghanns:22a}--\REF{ex:junghanns:22b}, we saw that causative verbs can be used with both agents and non-agents in some cases. Where there is a co-occurrence of a causative verb and a non-agent, an AOA is excluded:

\ea%63 
    %Croatian: 
    \label{ex:junghanns:63}
  \ea \gll   Marko je [\textsubscript{XP} namjerno] otvorio vrata.\\
    Marko.\textsc{nom.sg.m} \textsc{aux.3sg} ~ intentionally open.\textsc{lp.sg.m} door.\textsc{acc.pl.n}\\
    \glt ‘Marko intentionally opened the door.’

  \ex \gll     Vjetar  je (*[\textsubscript{XP} namjerno]) otvorio vrata.\\
      wind.\textsc{nom.sg.m} \textsc{aux.3sg} ~ intentionally open.\textsc{lp.sg.m} door.\textsc{acc.pl.n}\\
    \glt ‘The wind opened the door.’\hfill (Croat)
\z
\z

\noindent Only in the case of volitional causation does the use of an AOA yield a well-formed expression. Exclusion of an AOA, thus, is a diagnostic for non-agents (non-volitional causation). Compare the following examples with volitional (\REF{ex:junghanns:64a}, \REF{ex:junghanns:65a}) and non-volitional (\REF{ex:junghanns:64b}--\REF{ex:junghanns:64c}, \REF{ex:junghanns:65b}--\REF{ex:junghanns:65c}) causation.

\ea%64
    %Ukrainian:
    \label{ex:junghanns:64}
  \ea \gll   Ukrajins'ki zaxysnyky vidbyly dev’’jat’ sprob nastupu okupantiv.\\
    Ukrainian.\textsc{nom.pl} defender.\textsc{nom.pl} repel.\textsc{lp.pl} nine.\textsc{acc} attempt.\textsc{gen.pl.f} attack.\textsc{gen} occupying\_force.\textsc{gen.pl.m}\\
    \glt ‘The Ukrainian defenders repelled nine attempted attacks by the (Russian) occupying forces.’\label{ex:junghanns:64a}

  \ex \gll   Dzerkalo vidbylo jiji oblyččja.\\
    mirror.\textsc{nom.sg.n} reflect.\textsc{lp.sg.n} her face.\textsc{acc}\\
\glt ‘The mirror reflected her face.’\label{ex:junghanns:64b}

  \ex \gll   Viter vidbyv korabel’ vid pryčalu.\\
    wind.\textsc{nom.sg.m} tear\_away.\textsc{lp.sg.m} ship.\textsc{acc} from pier.\textsc{gen}\\
    \glt ‘The wind tore the ship away from the pier.’\hfill (Ukr)\label{ex:junghanns:64c} 
\z
\ex%65
    %Ukrainian:
    \label{ex:junghanns:65}
  \ea \gll   Ukrajins'ki zaxysnyky [\textsubscript{XP} uspišno] vidbyly dev’’jat’ sprob nastupu okupantiv.\\
    Ukrainian.\textsc{nom.pl} defender.\textsc{nom.pl} ~ successfully repel.\textsc{lp.pl} nine.\textsc{acc} attempt.\textsc{gen.pl.f} attack.\textsc{gen} occupying\_force.\textsc{gen.pl.m}\\
    \glt ‘The Ukrainian defenders successfully repelled nine attempted attacks by the (Russian) occupying forces.’ \hfill(pravda.com.ua 2022\_0905)\label{ex:junghanns:65a}

  \ex \gll   Dzerkalo (*[\textsubscript{XP} uspišno]) vidbylo jiji oblyččja.\\
    mirror.\textsc{nom.sg.n} ~ successfully reflect.\textsc{lp.sg.n} her face.\textsc{acc}\\
    \glt ‘The mirror reflected her face.’\label{ex:junghanns:65b} 

  \ex \gll   Viter (*[\textsubscript{XP} uspišno]) vidbyv korabel’ vid pryčalu.\\
    wind.\textsc{nom.sg.m} ~ successfully tear\_away.\textsc{lp.sg.m} ship.\textsc{acc} from pier.\textsc{gen}\\
    \glt ‘The wind tore the ship away from the pier.’\hfill (Ukr)\label{ex:junghanns:65c} 
\z
\z

\section{Proposal}
\label{sec:junghanns:4}

The expression in \REF{ex:junghanns:66} represents -- omitting irrelevant details -- the result of semantically combining the SF of a VP (modificandum) with the SF of an adverbial (modifier).

\ea%66
    \label{ex:junghanns:66}
 \gll {$λs \,  [$} {${[s \, \cnstx{inst} \, … \, ]}$}  :  {${[… \, s \, … \, ]} \, ]$} \\
       {} modificandum {} modifier \\
\z

\noindent The lambda operator binds the situation variable $s$. The lambda-bound $s$ is the only active argument slot left. Thus, the variable $s$ may serve as an anchor for an adverbial.

\citet{Davidson1967a} used adverbial modification as an argument for the introduction into the theory of the event variable, enriching the set of arguments of verbal predicates.\footnote{Instead of the term “event variable”, the more general term “situation variable” is used in the present chapter.} However as we can conclude from our preceding discussion, adverbials anchored to the causing situation variable represent only one of the various cases of adverbial modification. This case does not pose any problem at all (see above, \sectref{sec:junghanns:3.4}). But we still have to find a solution for the other two cases -- modification of $s'$ and $z$, respectively.

The solution offered here involves the use of the semantic-template tool in a way similar to that originally suggested by \citet{Zimmermann1992} in her paper on modifiers. We need, however, a more flexible templatic structure in order to be able to cover all cases of adverbial modification. Below, in the course of discussing the problematic cases refined versions of \citeauthor{Zimmermann1992}’s original template will be proposed.

Templates are “silent” linguistic elements in that they contribute meaning but lack phonetic form, cf. \citet[275, fn 4]{Zimmermann1992}. Using templates has the effect of adapting linguistic expressions to the contexts in which they have to fulfill certain functions and cannot do so on their own.\footnote{Apart from modification, Zimmermann introduced templates for deriving case meanings \citep{Zimmermann2003cases}, the interaction of tense and mood \citep{Zimmermann2016mood}, and the integration of subordinate clauses \citep{Zimmermann2018subordinate}.}

\citeauthor{Zimmermann1992}’s original proposal is given in \REF{ex:junghanns:67}.\footnote{Here, Zimmermann used S(entence) and N(ame) as semantic types in the way proposed by \citet{Ajdukiewicz1935}. The notation can be easily adapted to a system using logical types like $t$ and $e$ (cf., e.g., \citealt{Stechow2012}).}

\ea%67
    \label{ex:junghanns:67}
 \textit{Modification template}\\
  $λQ\textsubscript{2} \, λQ\textsubscript{1} \, λx \, [[Q\textsubscript{1} \, x] : [Q\textsubscript{2} \, x]]$ \newline
  where $Q\textsubscript{1}, \, Q\textsubscript{2} \, {\in} \, S/N$ and $“\!\!:\!\!” \, \, {\in} \, ({\alpha}/{\alpha})/{\beta}$
  \citep[256]{Zimmermann1992}
\z

\noindent Unfortunately, the template in this form is insufficient for our purposes. We have to refine it in order to be able to technically implement the modification of $s'$ and $z$, respectively.

As a first step, the SF of causative verbs needs improvement. For the purpose of covering all types of adverbial modification, the SF must, besides binding $s$, provide options for lambda-binding of $z$ and $s'$. This is taken care of in \REF{ex:junghanns:68}.

\ea%68
    \label{ex:junghanns:68}
$λy \, λx \, (λz) \, (λs') \, λs \, [[s \, \cnstx{inst} \, [P \, x]] : [[s \, \cnstx{cause} \, s'] : [s' \, \cnstx{inst} \, [\cnstx{become}$ \\$[z \, {\approx} \, [Q \, y]]]]]]$
\z

\noindent Proceeding from the versatile SF proposed here, we will discuss each case in need of a solution in turn.

\subsection{Modification of $z$}
\label{sec:junghanns:4.1}

\ea%69
    \label{ex:junghanns:69}
\gll Policie uzavřela dálnici D5 [\textsubscript{XP} na dvě hodiny].\\
  police.\textsc{nom.sg.f} close.\textsc{lp.sg.f} motorway.\textsc{acc} D5 ~ for two.\textsc{acc} hour.\textsc{acc.pl}\\
    \glt ‘The police closed the D5 motorway for two hours.’ \\ \hfill(Cz, adapted from lidovky.cz 2021\_0627)

\newpage
\ex%70
    \label{ex:junghanns:70}
  \ea   \textit{Semantic value of the VP} \newline  $(λz)$\footnote{The round brackets mark the argument slot as optional. It will be activated only in case of need. The case arises when an adverbial is meant to predicate over $z$.} $λs \, [[s \, \cnstx{inst} \, [P \, [\textsc{the police}]]] : [[s \, \cnstx{cause} \, s'] : [s' \, \cnstx{inst} \, [\cnstx{become} \, [z \,{\approx} \, [\cnstx{not accessible} \, [\textsc{motorway d5}]]]]]]]$


  \ex     \textit{Meaning representation for the temporal adverbial}\\
    $λx \, [\cnstx{time interval} \, x \, = \, \textsc{two hours}]$ \\
  \ex    \textit{Modification template} \newline (1st refined version of \citeauthor{Zimmermann1992}’s (\citeyear{Zimmermann1992}: 256) original template)\\
    $λQ\textsubscript{2} \, λQ\textsubscript{1} \, λs \, [[[Q\textsubscript{1} \, z] \, s] : [Q\textsubscript{2} \, z]] \newline
    Q\textsubscript{1} \, {\in} \, S/N/N, \, Q\textsubscript{2} \, {\in} \, S/N$

  \ex \textit{Application of the template to the modifier}\\
      $λQ\textsubscript{2} \, λQ\textsubscript{1} \, λs \, [[[Q\textsubscript{1} \, z] \, s] : [Q\textsubscript{2} \, z]] \, (λx \, [\cnstx{time interval} \, x \, = \, \textsc{two hours}]) \newline
    {\equiv} \,  λQ\textsubscript{1} \, λs \, [[[Q\textsubscript{1} \, z] \, s] : [λx \, [\cnstx{time interval} \, x \, = \, \textsc{two hours}] \, z]] \newline
    {\equiv} \,  λQ\textsubscript{1} \, λs \, [[[Q\textsubscript{1} \, z] \, s] : [\cnstx{time interval} \, z \, = \, \textsc{two hours}]]$

  \ex \textit{Application of the preceding result to the modificandum} \\
      $λQ\textsubscript{1} \, λs \, [[[Q\textsubscript{1} \, z] \, s] : [\cnstx{time interval} \, z \, = \, \textsc{two hours}]] \, (λz \, λs \, [[s \, \cnstx{inst}$\\$[P \, [\textsc{the police}]]] : [[s \, \cnstx{cause} \, s'] : [s' \, \cnstx{inst} \, [\cnstx{become} \, [z \,{\approx} \, [\cnstx{not accessible} \, [\textsc{motorway d5} \, ]]]]]]]) \newline
    {\equiv} \, λs \, [[[λz \, λs \, [[s \, \cnstx{inst} \, [P \, [\textsc{the police}]]] : [[s \, \cnstx{cause} \, s'] : [s' \, \cnstx{inst} \, [\cnstx{become} \, [z \, {\approx} \, [\cnstx{not accessible} \, [\textsc{motorway d5} \, ]]]]]]] \, z] \, s] : [\cnstx{time interval} \, z \, = \, \textsc{two hours}]] \newline
    {\equiv} \,  λs \, [[λs \, [[s \, \cnstx{inst} \, [P \, [\textsc{the police}]]] : [[s \, \cnstx{cause} \, s'] : [s' \, \cnstx{inst} \, [\cnstx{become} \, [z \, {\approx} \, [\cnstx{not accessible} \, [\textsc{motorway d5} \, ]]]]]]] \, s] : [\cnstx{time interval} \, z \, = \, \textsc{two hours}]] \newline
    {\equiv} \,  λs \, [[[s \, \cnstx{inst} \, [P \, [\textsc{the police}]]] : [[s \, \cnstx{cause} \, s'] : [s' \, \cnstx{inst} \, [\cnstx{become} \, [z \, {\approx} \, [\cnstx{not accessible} \, [\textsc{motorway d5} \, ]]]]]]] : [\cnstx{time interval} \, z \, = \, \textsc{two hours}]]$
\z
\z

\subsection{Decausative: Modification of $s'$}
\label{sec:junghanns:4.2}

\ea%71
    \label{ex:junghanns:71}
\gll Vratata se zatvori [\textsubscript{XP} tixo].\\
  door.\textsc{sg.f.def} \textsc{refl} close.\textsc{aor.3sg} {} quietly\\
    \glt ‘The door closed quietly.’ \hfill (Bg)
\ex%72
    \label{ex:junghanns:72}
  \ea \textit{Meaning representation for the causative verb}\\
    $λy \, λx \, (λs') \, λs \, [[s \, \cnstx{inst} \, [P \, x]] : [[s \, \cnstx{cause} \, s'] : [s' \, \cnstx{inst} \,[\cnstx{become} \, [z \, {\approx} \, [\cnstx{not open} \, y]]]]]]$

  \ex \textit{Argument-blocking reflexive marker} \\
    $λP \, λx \, [P \, u \, x]$ (\citealt{Fehrmannetal2014}: 302)

  \ex \textit{Blocking of the internal argument}\\
      $λP \, λx \, [P \, u \, x] \, (λy \, λx \, (λs') \, λs \, [[s \, \cnstx{inst} \, [P \, x]] : [[s \, \cnstx{cause} \, s'] : [s' \, \cnstx{inst} \, [\cnstx{become} \, [z \, {\approx} \, [\cnstx{not open} \, y]]]]]]) \newline
    {\equiv} \,  λx \, [λy \, λx \, (λs') \, λs \, [[s \, \cnstx{inst} \, [P \, x]] : [[s \, \cnstx{cause} \, s'] : [s' \, \cnstx{inst} \, [\cnstx{become} \, [z \, {\approx} \, [\cnstx{not open} \, y]]]]]] \, u \, x] \newline
    {\equiv} \,  λx \, [λx \, (λs') \, λs \, [[s \, \cnstx{inst} \, [P \, x]] : [[s \, \cnstx{cause} \, s'] : [s' \, \cnstx{inst} \, [\cnstx{become} \, [z \,{\approx} \, [\cnstx{not open} \, u]]]]]] \, x] \newline
    {\equiv} \,  λx \, (λs') \, λs \, [[s \, \cnstx{inst} \, [P \, x]] : [[s \, \cnstx{cause} \, s'] : [s' \, \cnstx{inst} \, [\cnstx{become} \, [z \, {\approx} \, [\cnstx{not open} \, u]]]]]]$

  \ex \textit{Semantic value of the VP, including the meaning of the subject} 
  \newline
  $(λs') \, λs \, [[s \, \cnstx{inst} \, [P \, [\textsc{the door}]]] : [[s \, \cnstx{cause} \, s'] : [s' \, \cnstx{inst} \, [\cnstx{become} \, [z \, {\approx} \, [\cnstx{not open} \, u]]]]]]$\footnote{\textit{Vratata} ‘the door’ is the syntactic subject of the sentence. At the level of Conceptual Structure, the variable \textit{u} will be interpreted as having the same referent as the subject.}

  \ex \textit{Meaning representation for the modal adverbial}\\
    $λs^{*} \, [\textsc{quiet} \, s^{*}]$

  \ex \textit{Modification template}
    \newline 
    (2nd refined version of \citeauthor{Zimmermann1992}’s (\citeyear{Zimmermann1992}: 256) original template) \\
    $λQ\textsubscript{2} \, λQ\textsubscript{1} \, λs \, [[[Q\textsubscript{1} \, s'] \, s] : [Q\textsubscript{2} \, s']] \newline
    Q\textsubscript{1} \, {\in} \, S/N/N, \, Q\textsubscript{2} \, {\in} \, S/N$

  \ex \textit{Application of the template to the modifier} \\
      $λQ\textsubscript{2} \, λQ\textsubscript{1} \, λs \, [[[Q\textsubscript{1} \, s'] \, s] : [Q\textsubscript{2} \, s']] \, (λs^{*} \, [\textsc{quiet} \, s^{*}]) \newline
    {\equiv} \,  λQ\textsubscript{1} \, λs \, [[[Q\textsubscript{1} \, s'] \, s] : [λs^{*} \, [\textsc{quiet} \, s^{*}] \, s']] \newline
    {\equiv} \, λQ\textsubscript{1} \, λs \, [[[Q\textsubscript{1} \, s'] \, s] : [\textsc{quiet} \, s']]$

  \ex \textit{Application of the preceding result to the modificandum}\\
      $λQ\textsubscript{1} \, λs \, [[[Q\textsubscript{1} \, s'] \, s] : [\textsc{quiet} \, s']] \, (λs' \, λs \, [[s \, \cnstx{inst} \, [P \, [\textsc{the door}]]] : [[s \, \cnstx{cause} \, s'] : [s' \, \cnstx{inst} \,[\cnstx{become} \, [z \,{\approx} \, [\cnstx{not open} \, u]]]]]]) \newline
    {\equiv} \,  λs \, [[[λs' \, λs \, [[s \, \cnstx{inst} \, [P \, [\textsc{the door}]]] : [[s \, \cnstx{cause} \, s'] : [s' \, \cnstx{inst} \, [\cnstx{become} \, [z \, {\approx} \, [\cnstx{not open} \, u]]]]]] \, s'] \, s] : [\textsc{quiet} \, s']] \newline
    {\equiv} \,  λs \, [[λs \, [[s \,\cnstx{inst} \, [P \,[\textsc{the door}]]] : [[s \, \cnstx{cause} \, s'] : [s' \, \cnstx{inst} \, [\cnstx{become} \, [z \, {\approx} \, [\cnstx{not open} \, u]]]]]] \, s] : [\textsc{quiet} \, s']] \newline
    {\equiv} \,  λs \, [[[s \,\cnstx{inst} \, [P \, [\textsc{the door}]]] : [[s \,\cnstx{cause} \, s'] : [s' \,\cnstx{inst} \, [\cnstx{become} \, [z \, {\approx} \, [\cnstx{not open} \, u]]]]]] : [\textsc{quiet} \, s']]$
\z
\z

\noindent The adverbial, which is predicated over $s'$, has been integrated in a transparent, formally correct way. At the level of Semantic Form, $s'$ remains unbound and, hence, is a semantic parameter. Later at the level of Conceptual Structure, it will be existentially quantified by default.

The lambda operator that binds the variable $s$ will, in the course of further semantic amalgamation, be replaced by an existential quantifier. For details regarding this operation see \citet[486]{Zimmermann2009} and \citet[290]{Zimmermann2016mood}.

By assuming various versions of the modification template and the optional activation of argument slots through the binding of the respective variables by a lambda operator, it has become possible to provide solutions for the problematic cases of adverbial modification.

\section{Summary} \label{sec:junghanns:5}
\largerpage
Adverbials that appear as modifiers in sentences induced by causative verbs are not always linked to the causing situation. By closer inspection, it became clear that the caused situation and the resulting state may also be semantic anchors for adverbials, given the right contexts. Two issues arose: (i)~We were in need of determining the appropriate semantic representation for causative verbs, that is, uncovering enough structure to account for the empirical facts. (ii)~Technical means had to be found for the integration of adverbials predicating over (a)~caused situations and (b)~resulting states in the course of semantic composition.

In order to solve issue (i), I have defended the view that the meaning of causative verbs has to be decomposed, yielding a more fine-grained structure than the simple $P (x, y, e)$ analysis. Ultimately, the meaning representation must display three variables: $s$, $s'$ and $z$, referring, respectively, to the causing situation, the caused situation, and the target state.

\ea%73
    \label{ex:junghanns:73}
 \textit{Generalized meaning representation of causative verbs}\\
  $λy \, λx \, (λz) \, (λs') \, λs \, [[s \, \cnstx{inst} \, [P \, x]] : [[s \, \cnstx{cause} \, s'] : [s' \, \cnstx{inst} \,[\cnstx{become} \, [z \, {\approx} \, [Q \, y]]]]]]$
\z

\noindent This representation is sufficiently detailed, in that it displays all the potential anchors for adverbial modifiers.

\citeposst{Zimmermann1992} approach to modification was based on the assumption of a semantic template, which, among other things, accounted for the case of adverbials predicated over the event as a whole. The discussion in the present chapter showed that we have to go further than this. It became clear that it must be possible to activate the arguments $s'$ and $z$ and, thus, make them accessible for steps in the process of semantic composition. My proposal consists in the lambda-binding of the respective variables, cf. the representation in \REF{ex:junghanns:73}. Since the argument slots need not be opened in all cases, lambda-binding is optional, which is indicated by the use of round brackets. We are thus not forced to assume default deactivation in case the respective argument is not needed as an anchor, which is desirable for reasons of economy.

As I have argued, the solution for issue (ii) can be reached by refining \citeposst{Zimmermann1992} original template. We have ended up with three specific versions of the modification template. These are the following:

\ea%74
    \label{ex:junghanns:74}
  \ea     \textit{Version for} $s$ \\
    $λQ\textsubscript{2} \, λQ\textsubscript{1} \, λs \, [[Q\textsubscript{1} \, s] : [Q\textsubscript{2} \, s]] \newline
    Q\textsubscript{1}, \, Q\textsubscript{2} \, \in \,  S/N$ \newline
    see \citet[256]{Zimmermann1992}

  \ex     \textit{Version for} $s'$ \\
    $λQ\textsubscript{2} \, λQ\textsubscript{1} \, λs \,[[[Q\textsubscript{1} \, s'] \, s] : [Q\textsubscript{2} \, s']] \newline
    Q\textsubscript{1} \, \in \, S/N/N, \, Q\textsubscript{2} \, \in \, S/N$

  \ex     \textit{Version for} $z$ \\
    $λQ\textsubscript{2} \, λQ\textsubscript{1} \, λs \,[[[Q\textsubscript{1} \, z] \, s] : [Q\textsubscript{2} \, z]] \newline
    Q\textsubscript{1} \, \in \, S/N/N, \, Q\textsubscript{2} \, \in \, S/N$
\z
\z

\noindent We can generalize over the specific versions. This gives us the modification template in its final form:

\ea%75
    \label{ex:junghanns:75}
 \textit{Generalized modification template}\\
  $λQ\textsubscript{2} \, λQ\textsubscript{1} \, λs \, [ \, \textsubscript{α}([)\textsubscript{α} \,[Q\textsubscript{1} \,\textsubscript{α}(w])\textsubscript{α} \, s \, ] : [\, Q\textsubscript{2} \, \textsubscript{α}(w)\textsubscript{α} \,\textsubscript{{}-α}(s)\textsubscript{{}-α} \, ]] \newline
  Q\textsubscript{1} \, \in \, S/N\textsubscript{α}(/N)\textsubscript{α}, \, Q\textsubscript{2} \, \in \, S/N, \, w \, = \, s' \, \vee \, z$ 
\z

\noindent  If α-marked brackets apply, -α-marked brackets do not apply, and vice versa.

\largerpage
\section*{Abbreviations}
\begin{multicols}{2}
\begin{tabbing}
MMMM \= first\kill
\textsc{1} \>  first person\\%OK
\textsc{2} \>  second person\\%OK
\textsc{3} \>  third person\\%OK
\textsc{acc} \>  accusative case\\%OK
\textsc{adj} \>  adjective\\%OK
\textsc{adv} \>  adverb\\%OK
\textsc{aor} \>  aorist\\%OK
\textsc{aux} \>  auxiliary\\%OK
\textsc{cl} \>  clitic\\%OK
\textsc{cond} \>  conditional mood\\%OK
\textsc{dat} \>  dative case\\%OK
\textsc{def} \>  definite\\%OK
\textsc{f} \>  feminine\\%OK
\textsc{gen} \>  genitive case\\%OK
\textsc{indecl} \>  indeclinable\\%OK
\textsc{inf} \>  infinitive\\%OK
\textsc{ins} \>  instrumental case\\%OK
\textsc{ipfv} \>  imperfective aspect\\%OK
\textsc{loc} \>  locative case\\%OK
\textsc{lp} \>  \textit{l}-participle\\%OK
\textsc{m} \>  masculine\\%OK
\textsc{n} \>  neuter\\%OK
\textsc{nom} \>  nominative case\\%OK
\textsc{pfv} \>  perfective aspect\\%OK
\textsc{pl} \>  plural number\\%OK
\textsc{poss} \>  possessive\\%OK
\textsc{prs} \>  present tense\\%OK
\textsc{prtcl} \>  particle\\%OK
\textsc{pst} \>  past tense\\%OK
\textsc{refl} \>  reflexive marker\\%OK
\textsc{sa} \>  sentential adverbial\\
\textsc{sg} \>  singular number\\%OK
vi  \>  verbum intransitivum\\%OK
vt  \>  verbum transitivum
\end{tabbing}
\end{multicols}

\section*{Acknowledgements}

Before I met Ilse Zimmermann for the first time in person, I got acquainted with her work back in the 1980s when I was a student at the University of Leipzig. The way she presented linguistic problems and developed solutions made a deep impression on me. In 1989, she read a paper at a workshop that I attended. At this workshop, I presented a paper too, and she commented on it.  She did that in a very careful way, giving hints rather than full instruction. Then as well as in our discussions to follow through the years, I understood that her intention was not to take others by the hand and guide them step by step but to stimulate independent thinking. She set an example in various ways -- think for yourself, be clear in what you say, become a visible member of the community and stand your ground. I will always remember her with warmth and respect.

Thanks are due to Svitlana Adamenko, Krâstina Arbova, Elena Grimmig, Ma\-rian\-na Leonova, Olga Liebich, Małgorzata Małolepsza, Zrnka Meštrović, Lubomír Sůva, Lucia Vlášková, Viktor Zakar, and Rok Žaucer for checking object-language examples. I am grateful to Petr Biskup, Hagen Pitsch, Luka Szucsich, and Carla Umbach for helpful discussion, which stimulated me to try and make myself clearer where assumptions were not obvious. I am indebted to Elliott Lash for checking my English and suggesting improvements. Any remaining mistakes, errors, and shortcomings are my own responsibility.


\sloppy\printbibliography[heading=subbibliography,notkeyword=this]
\end{document}
