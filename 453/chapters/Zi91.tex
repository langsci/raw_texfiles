\documentclass[output=paper,colorlinks,citecolor=brown]{langscibook}
\ChapterDOI{10.5281/zenodo.15471429}

\author{Ilse Zimmermann\affiliation{Zentralinstitut für Sprachwissenschaft}}
\title{The “subject” in noun phrases: Its syntax and semantics}  
\abstract{\noabstract}


\begin{document}
\maketitle

% section 0
%\setcounter{section}{-1}
%\section{} \label{sec:zi91:0}

 \noindent Within the framework of \citet{Chomsky86Barriers}, the analysis of noun phrases initiated by \citet{Szabolosi81The-possessive-Hungarian, Szabolosi83Possessor-ran-away, Szabolosi87Functional-categories} and by \citet{Abney86Functional-elements, Abney87The-English} and on the basis of \posscitet{Bierwisch87Semantik-der, Bierwisch88On-the, Bierwisch89Event-nominalizations:} assumptions on semantics, I will consider the syntax and semantics of the so-called subject in noun phrases.\footnote{The present paper is the revised and expanded version of my talk given at the 11th annual conference of the DGfS in Osnabrück 1989 and of \citet{Zimmermann89The-subject}. I thank Oda Buchholz and Karl Erich Heidolph for stimulating discussion and Jörg-Peter Schultze for help with my
English.} I will give an account of its syntactic categorization and its various prenominal and postnominal positions in German, English and Russian. In order to avoid premature generalizations as to the position of the “subject” in noun phrases and to the syntactic and semantic interrelations of the “subject” and determiners I took into consideration data from Hungarian, Italian, Bulgarian and Albanian, too. It will be shown that the language-specific peculiarities of the “subject” are based on parameter setting which concerns the direction of government, the placement of genitive phrases and of constituents agreeing with the nominal head and to a large extent manifests itself in lexical properties of the particular language.

In section \ref{sec:zi91:1}, I will characterize the problems to be solved. In section \ref{sec:zi91:2}, the theoretical fundamentals of the following considerations are exposed. Section \ref{sec:zi91:3} is devoted to Szabolcsi's stimulating analysis of Hungarian noun phrases with “subjects”, which gives an account of their syntax and semantics. Section \ref{sec:zi91:4} discusses data from Italian, Bulgarian, Albanian. Section \ref{sec:zi91:5} deals with German, English and Russian and contains the proposed solution of the problem concerning the placement of “subject” phrases at the left periphery of noun phrases. Section \ref{sec:zi91:6} sums up.

% section 1
\section{} \label{sec:zi91:1}

The “subject” of noun phrases is by no means easy to identify. First of all, a terminological clarification seems to be in place. By “subject” (or “possessor”)\footnote{In the following, I will omit the quotation-marks.} I mean (a) the argument of deverbal or deadjectival and other relational nouns as \textit{Untersuchung} (‘investigation’), \textit{Zufriedenheit} (‘satisfaction’), \textit{Appetit} (‘appetite’) corresponding to the external argument of the verbal or adjectival base or of a periphrastic expression as \textit{untersuchen} (‘investigate’), \textit{untersucht werden} (‘be investigated’), \textit{zufrieden sein} (‘be satisfied’), \textit{Appetit haben} (‘have appetite’), (b) the possessive modifier of absolute nouns as \textit{Hund} (‘dog’). Characteristically, the case-marking of the subject in noun phrases -- as in sentences -- is structurally determined (details see below), regardless of whether the pertinent expression is a prepositional phrase as in \REF{ex:zi91:1}, a genitive phrase as in \REF{ex:zi91:2a} and \REF{ex:zi91:2b} or a possessive pronoun as in \REF{ex:zi91:2c} or \REF{ex:zi91:3} with the additional dative phrase.

\ea \label{ex:zi91:1}
   \ea \label{ex:zi91:1a} 
    \gll die drei Hunde von Peter / von ihm\\
        the three dogs of Peter {} of him\\
    
   \ex \label{ex:zi91:1b} 
   \gll von Peter / von ihm die drei Hunde\\
   of Peter {} of him the three dogs\\
   \z
  
    
\ex \label{ex:zi91:2}
    \ea \label{ex:zi91:2a}
    \gll \minsp{(} die) drei Hunde Peters\\
        {} the three dogs Peter's\\
        
    \ex \label{ex:zi91:2b} 
    \gll Peters drei Hunde\\
    Peter's three dogs\\
    
    \ex \label{ex:zi91:2c} 
    \gll seine drei Hunde\\
    his three dogs\\
    \z
\ex \label{ex:zi91:3} 
    \gll \minsp{(} dem) Peter seine drei Hunde\\
    {} the.\textsc{dat} Peter.\textsc{dat} his three dogs\\
\z


\noindent The examples \REF{ex:zi91:1}--\REF{ex:zi91:3} exhaust the possible expression types for
subjects in German noun phrases. They show that the subject is not
only the constituent which heads the chain of constituents in a noun
phrase and which has been characterized as to its syntactic function
by \REF{ex:zi91:4}, as in \citet{Chomsky65Aspects-of}.

\ea \label{ex:zi91:4} {[NP, NP]}
\z

\noindent Thus, the question arises whether the subject as conceived of here can be characterized functionally at some level of syntactic representation. This problem immediately concerns the semantic and syntactic interrelation of the prenominal and the postnominal subject expressions in \REF{ex:zi91:1}--\REF{ex:zi91:3}. Are the pertinent positions independent of each other? Or, are the constituents in these positions interconnected by co-indexing? The solution of this problem amounts to the question where the possessor is at home in deep structure. In this connection, one has to decide how the different semantic functions of subject expressions, i.e. their function as an argument of the nominal head or as modifier, must be reflected in deep structure and whether one has to reckon with the ambivalent function of an argument-adjunct in the sense of \citet{Grimshaw88Adjuncts-and}.

Another problem to be solved concerns the prenominal subject position. One must decide where the subject is located when the determiner is silent, as in \REF{ex:zi91:2b}, \REF{ex:zi91:2c}, and \REF{ex:zi91:3}. In view of the fact that the determiner and the subject phrase serve different semantic functions I will not accept the traditional view according to which the prenominal subject and the determiner were analyzed as being in complementary distribution (see \citealt{Chomsky65Aspects-of} and others). I assume that the subject and the determiner occupy different positions. Nevertheless, one has to decide whether the various prenominal subject phrases in \REF{ex:zi91:1b}, \REF{ex:zi91:2b}, \REF{ex:zi91:2c}, and \REF{ex:zi91:3} are in the same position, as is assumed in \citet{Bhatt90Die-syntaktische}. In this connection, one must answer the question why prepositional phrases can precede overt determiners whereas possessive pronouns and genitive phrases cannot. Compare:

\begin{exe}
    \exr{ex:zi91:1b} 
    \gll von Peter / von ihm die drei Hunde\\
   of Peter {} of him the three dogs\\
    
    \exi{(1c)} \label{ex:zi91:1c} 
        \gll \minsp{*} Peters / seiner die drei Hunde\\
        {} Peter's {} his.\textsc{gen} the three dogs\\
        
    \exi{(1d)} \label{ex:zi91:1d}
    \gll \minsp{*} seine die drei Hunde\\
    {} his.\textsc{nom} the three dogs\\
\end{exe}



\noindent Last not least, one must clarify which reference type the container noun phrase is of when the determiner is not realized phonologically and the subject is located in front of all other constituents of the noun phrase. And what is the case when a noun phrase with a subject is used predicatively, as in \REF{ex:zi91:5}?

\ea \label{ex:zi91:5}
    \ea \label{ex:zi91:5a} 
    \gll Peter und seine drei Hunde wurden schnell unsere Freunde.\\
    Peter and his three dogs  became quickly our friends\\
    \glt `Peter and his three dogs quickly became our friends.'
    
    \ex \label{ex:zi91:5b} 
    \gll Peter und seine drei Hunde wurden schnell Freunde von uns.\\
    Peter and his three dogs became quickly friends of us \\
    \glt `Peter and his three dogs quickly became  friends of ours.'\\
    \z
\z

\noindent I find it premature to assume that noun phrases with prenominal subject expressions are definite in reference, as do \citet{Olsen89AGReement-in} and \citet{Bhatt90Die-syntaktische} among others.

These questions in mind, I will characterize the theoretical fundamentals of the following considerations.

%section 2
\section{} \label{sec:zi91:2}

“Wenngleich sich die verschiedenen generativen Strömungen erheblich in ihren Analysen für sprachliche Phänomene, aber auch im psychologischen Anspruch unterscheiden, so einigt sie doch ein Bemühen: Über eine Analyse verschiedener Einzelsprachen soll eine Grammatiktheorie konstruiert werden, deren Axiome den Begriff ‘mögliche natürliche Sprache’ definieren ...”\footnote{``Although the various generative currents differ considerably in their analyses of linguistic phenomena, but also in their psychological claims, they are united by one effort: By analyzing different individual languages, a theory of grammar is to be constructed, the axioms of which define the notion 'possible natural language'..." (translation added by guest editors).}  \citep[93]{Fanselow88Aufspaltung-von}. The axioms of grammatical theory can be conceived of as universal grammar in the sense of \citet{Chomsky65Aspects-of, Chomsky81Lectures-on-government, Chomsky86Knowledge-of-language}, and the system of a particular language as an instance of universal grammar with specifically fixed parameters of possible variation.

As for syntax, the X-schema for admissible syntactic structures is a substantial universal. In its most general form it is \REF{ex:zi91:6}.

\ea \label{ex:zi91:6} $X^{i} \rightarrow \left\{ \begin{array}{c} ...\; X^{j} \;... \\ e \end{array} \right\} \; (i \geq j \geq 0)$
\z
\vspace{-0,3cm}
 \hspace{0,8cm}with ‘...’ being a sequence of maximal projections
\vspace{0,3cm} \newline
\noindent According to this schema, syntagmas are characterized as endocentric structures. It allows recursion, direct passing to lower projections and to the empty chain $e$. The value for $m$, the maximal projection of $X$, is 2 for functional categories. As for lexical categories, I assume that the value for $m$ is 2 for verbs and nouns, 1 otherwise.

The values for the variable $X$ in the $\overline{X}$-schema \REF{ex:zi91:6} are combinations of positively or negatively specified features of the sets indicated in \REF{ex:zi91:7} and \REF{ex:zi91:8}.

\ea \label{ex:zi91:7} Syntactic features: $\pm$V, $\pm$N, $\pm$D, $\pm$K, $\pm$Q, ...
\ex \label{ex:zi91:8} Morphosyntactic features: $\pm$1ps, $\pm$2ps; $\pm$pl; $\pm$masc, $\pm$fem; $\pm$governed, $\pm$oblique, $\pm$genitive, $\pm$poss; ...
\z

\noindent The various feature combinations serve the (morpho)syntactic classification of lexical items and are projected as lexical information into the structure of complex expressions \citep[see][]{Zimmermann87Syntactic-categorization, Zimmermann88Die-substantivische-Verwendung, Zimmermann88Wohin-mit-Affixen}.

I assume that the feature $+$poss characterizes possessive pronouns, the English possessor marker \textit{-s} and the genitive markers in German prenominal subject expressions as \textit{Peters} in \REF{ex:zi91:2b}. The postnominal subject phrases in \REF{ex:zi91:1} and \REF{ex:zi91:2} and the prenominal \textit{von}-phrase in \REF{ex:zi91:1b} are not categorized by $+$poss. The consequences of this classification will become clear below.

I will assume that \REF{ex:zi91:6} is valid for deep structure (DS), S-structure (SS), and for logical form (LF). Whether it applies to the syntactic surface structure (OS) as well seems to be a matter of parametric variation. Such a conception becomes feasible with \posscitet{Chomsky86Barriers} very general conditions on transformations, if one regards them as restrictions applying to transformations correlating DS, SS and LF, but not necessarily to the so-called stylistic transformations correlating SS and OS.

The $\overline{X}$-schema \REF{ex:zi91:6} and the assumptions concerning the category-specific depth of the various phrase types amount to the hypothesis that all languages to a certain degree are configurational. This hypothesis also applies to the interrelation of the syntactic layers constituted by functional and lexical heads, respectively. It is an empirically important question at which levels of syntactic representation the dominance of functional layers over lexical layers -- as indicated in \REF{ex:zi91:9} -- should be valid.

\ea \label{ex:zi91:9} {[\textsubscript{FP} $e$ [\textsubscript{F$'$} F [\textsubscript{LP} ... L ...]]] \newline where $F$ and $L$ are variables for functional and lexical categories, respectively}
\z

\noindent In accordance with \citet{Chomsky86Barriers}, and with \citet{Fukui86A-theory}, \citet{Abney86Functional-elements, Abney87The-English}, \citet{Haider87Deutsche-Syntax, Haider88Die-Struktur}, \citet{Felix88Structure-functional}, and others I regard sentences as maximal projections of the functional category C (in my system $+$D, $+$K) and referring noun phrases as maximal projections of the functional category D (in my system $+$D, $-$K).\footnote{For analyses of noun phrases with the functional DP-layer see \citet{Szabolosi81The-possessive-Hungarian, Szabolosi83Possessor-ran-away, Szabolosi87Functional-categories}, \citet{Hellan86The-headedness}, \citet{Delsing88The-Scandinavian, Delsing90A-DP-analysis}, \citet{Olsen88Agreement-und-Flexion, Olsen89AGReement-in, Olsen89Das-Possessivum, Olsen88Die-deutsch-Nominalphrase}, \citet{Lobel89Q-as-functional, Lobel88D-und-Q, Lobel90Functional-categories, Lobel90On-the-parametrization}, \citet{Bhatt89Parallels-in, Bhatt90Die-syntaktische, Bhatt90Kasuszuweisung-in}, \citet{Huste89Zur-Syntax, Huste89Zur-Topologie}, \citet{Bischof91Sachverhaltsbezeichnungen-des}, \citet{Freytag90Die-syntaktische, Freytag91Sachverhaltsbezeichnungen-des}.} I will assume the universal \REF{ex:zi91:10}.

\ea \label{ex:zi91:10} In referring syntagmas, the functional categories C and D close the projection of the lexical categories V and N, respectively.
\z

\noindent Semantically, \REF{ex:zi91:10} amounts to the assumption that the lexical categories V and N have a referential argument slot which needs binding by a pertinent operator characterizing the reference type of the syntagma as a whole. C- and D-entities, typically, serve this function.\footnote{See \citet{Zimmermann90Zur-Legitimierung}. For the decisive role of C as the functional head of sentences in determining the various sentence types with corresponding specific meanings see \citet{Brandt89Satzmodus-Modalitat}.} Whether there is more than one funtional layer dominating a lexical layer is an empirical question. (For further layers between CP and VP see \citealt{Pollock89Verb-movement} and \citealt{Chomsky89Some-notes} and between DP and NP see \citealt{Lobel89Q-as-functional, Lobel90Functional-categories, Lobel88D-und-Q, Lobel90On-the-parametrization}, \citealt{Delsing88The-Scandinavian, Delsing90A-DP-analysis}, \citealt{Bhatt90Die-syntaktische} and the discussion in \citealt{Zimmermann91Die-Syntax-Substantivgruppe}.) In the following considerations I will reckon for noun phrases with the DP layer dominating the NP layer, as is indicated in \REF{ex:zi91:11}.

\ea \label{ex:zi91:11} $[$\textsubscript{DP} $e$ [\textsubscript{D$'$} D \; [\textsubscript{NP} $e$ [\textsubscript{N$'$} ... (QP) ... N ...]]]]
\z

\noindent QP is considered as modifier in NP and not as an intermediate layer between DP and NP as in \citet{Zimmermann89The-subject, Zimmermann90Pranominale-Argument}.\footnote{See the syntactic and semantic analysis of quantifying expressions in French partitive and pseudopartitive constructions proposed by \citet{Burchert90Zur-Syntax}.}

Now, the $\overline{X}$-schema \REF{ex:zi91:6} with its instantiations like \REF{ex:zi91:11} is completed by further restricting principles, beyond the one in \REF{ex:zi91:10}. For our discussion, the following principles will be relevant.

\ea \label{ex:zi91:12} In S-structure, any phonologically nonempty noun phrase must be case-marked.
\ex \label{ex:zi91:13} A structurally case-marked argument of X can be licensed externally to X$'$ only if it is the most prominent argument of X. 
\ex \label{ex:zi91:14} In deep structure, the place of the internal arguments of a lexical head X coincides in the given language with the direction of government of the pertinent category X belongs to.
\ex \label{ex:zi91:15} If in a given language N governs rightwards, in S-structure phrases agreeing with N are located on its left and genitive phrases on its right.
\z

\noindent It will become clear in the following sections that the languages under consideration make different choices with respect to the implicational universal \REF{ex:zi91:15}, with characteristic consequences for the placement of the subject expressions.

As to semantics of linguistic expressions, I take the view that the level of semantic form (SF) belongs to the structural representations determined by grammar. This level correlates logical form and conceptual structure (see \citealt{Bierwisch87Semantik-der, Bierwisch88On-the, Bierwisch89Event-nominalizations:}). In following \citet{Higginbotham85On-semantics} and \citet{Bierwisch87Semantik-der, Bierwisch88On-the, Bierwisch89Event-nominalizations:}, I will discriminate three types of theta-role discharging: (a) saturation of argument slots of lexical heads, (b) unification of the external argument slot of a modifier with the referential argument slot of the modificandum, and (c) binding of the referential argument by an operator. (c) takes place in the DP-layer, (b) concerns the semantic integration of adjuncts, (a) in noun phrases is the semantic amalgamation of the nominal head with its internal arguments. With \citet{Williams81Argument-structure}, \citet{di1987definition} and \citet{Bierwisch88On-the, Bierwisch89Event-nominalizations:} I assume that nouns do not have an external argument. Deverbal and deadjectival nouns internalize the external argument of the verbal or adjectival base. Thus, in DS all arguments of N are represented as sisters of N (see the analyses by \citealt{Bischof91Sachverhaltsbezeichnungen-des} and \citealt{Freytag91Sachverhaltsbezeichnungen-des}).

With respect to the lexicon (see \citealt{Zimmermann87Die-Argumentstruktur, Zimmermann87Die-Rolle-Lexikons}) I assume that it brings in all information a language learner must know about the peculiarities of a given language beyond specifying the parameters of universal grammar (see \citealt{Felix88Structure-functional}). As will be shown below, the morphosyntactic feature $+$poss combined with adjectival agreement features or with the case feature $+$genitive plays a decisive discriminating role in determining the possible subject expressions of noun phrases in different languages.

As will become evident in the following analysis of noun phrases with subject expressions, I adhere to the so-called atomicity hypothesis with respect to the treatment of morphology (see \citealt{di1987definition}). As far as possible, I do not admit affix hopping or movement of heads to their affixes, as in \citet{Pollock89Verb-movement}, \citet{Chomsky86Barriers, Chomsky89Some-notes} and others. I assume that all morphological information is represented in the structure of lexical items as functional or lexical heads of the pertinent phrases (see \citealt{Zimmermann88Die-substantivische-Verwendung, Zimmermann88Wohin-mit-Affixen}). I will assume that case-marked phrases -- including $+$poss-marked ones -- must be licensed in SS. The same is true for phrases categorized by agreement features. These constituents in SS must take part in an agreement chain.

\largerpage
On this basis, I assume that subject expressions in German noun phrases in DS are located as postnominal adjuncts of N$'$ or as sisters of N depending on their function as modifier as in \REF{ex:zi91:1}--\REF{ex:zi91:3} or as argument, respectively. In SS, they can occur in prenominal positions, In SpecN or in SpecD, as will be shown below.\footnote{SpecX is the sister of X$'$ and the daughter of XP. See the corresponding positions in \REF{ex:zi91:11} occupied by the empty category $e$.} In \ref{sec:zi91:5.4}, I will discuss a more general possibility of the DS position of subject phrases.

%section 3
\section{} \label{sec:zi91:3}

In this section, I will consider \posscitet{Szabolosi87Functional-categories} pioneering analysis of Hungarian noun phrases with subject (possessor) expressions.\footnote{See also \citet{Szabolosi81The-possessive-Hungarian, Szabolosi83Possessor-ran-away}.} It seems possible to adapt it to the DP structure assumed here.

One essential peculiarity of the Hungarian noun phrase consists in the division of labor between two kinds of expressions with respect to the two semantic functions of the determiners, i.e. term building and reference type specification.\footnote{As will become clear below (see \REF{ex:zi91:20}, \REF{ex:zi91:21}), referring noun phrases semantically are considered as entities of type S/(S/N).} The Hungarian formative \textit{a}(\textit{z}) ‘the’ serves the first function. Determiners as \textit{ezen} `this', \textit{minden} `every', \textit{melyik} `which' serve the second one.\footnote{Semantically, this division of labor amounts to the separation of the inner operator $\widehat{P_{i}}$ from the SF of determiners like \textit{every}, \textit{the} etc. and its introduction by lambda abstraction. See \REF{ex:zi91:20d} and \REF{ex:zi91:20e} below.} \citet{Szabolosi87Functional-categories} assumes that there is one more functional layer in the structure of noun phrases, parallel to CP of sentences, which is constituted by the functional head CN, the nominal complementizer:

\ea \label{ex:zi91:16} $[$\textsubscript{CNP} [\textsubscript{CN$'$} CN DP]]
\z

\noindent Thus, one has one more SpecX, too. \citeauthor{Szabolosi87Functional-categories} exploits SpecCN as landing site for the Hungarian dative possessor. The nominative possessor remains within DP. This is shown in \REF{ex:zi91:17}. The structure given departs from \citeauthor{Szabolosi87Functional-categories}'s in several respects, as will become clear below.\footnote{\posscitet{Szabolosi87Functional-categories} analysis is given in \REF{ex:zi91:19}.}

\ea \label{ex:zi91:17}
    \ea \label{ex:zi91:17a}
        \gll $[$\textsubscript{CNP} [\textsubscript{CNP$_i$} Péter-nek [\textsubscript{CN$'$} [\textsubscript{CN} a(z)] [\textsubscript{DP} $t_i$ [\textsubscript{DP} [\textsubscript{D$'$} [\textsubscript{D} minden] [\textsubscript{NP} $t_i$ [\textsubscript{NP} [\textsubscript{N$'$} $t_i$ [\textsubscript{N} [\textsubscript{N} kalap] ja ]]]]]]]]] \\
        {} {} Péter-\DAT {} {} {} the {} {} {} {} {} every {} {} {} {} {} {} {} hat \textsc{poss}\\
        \glt
    \ex \label{ex:zi91:17b}
        \gll $[$\textsubscript{CNP} [\textsubscript{CN$'$} [\textsubscript{CN} a(z)] [\textsubscript{DP} [\textsubscript{CNP$_i$} a(z) Péter-$\emptyset$] [\textsubscript{D$'$} [\textsubscript{D} minden] [\textsubscript{NP} $t_i$ [\textsubscript{NP} [\textsubscript{N$'$} $t_i$ [\textsubscript{N} [\textsubscript{N} kalap] ja ]]]]]]]] \\
        {} {} {} the {} {} the Péter-\NOM {} {} {} every {} {} {} {} {} {} {} hat \textsc{poss}\\
        \glt {`Peter’s every hat'}
    \z
\z

\noindent The correctness of \citeauthor{Szabolosi87Functional-categories}'s important insight that determiners and possessors are not in complementary distribution is evident. It applies to $D$ entities and to the CN formative \textit{a}(\textit{z}) ‘the’ as well. In \REF{ex:zi91:17a} the dative possessor goes ahead of both \textit{a}(\textit{z}) ‘the’ and \textit{minden} `every'. In \REF{ex:zi91:17b}, the nominative possessor has its position between these entities. In both cases the possessor phrase agrees in person and number with the nominal head affixed by the possession marker. With respect to its case it seems to be legitimized in two alternative positions, the nominative in SpecD and the dative in SpecCN. From SpecCN the possessor can “run away from home” \citet{Szabolosi83Possessor-ran-away}:

\ea \label{ex:zi91:18} 
    \gll Péter-nek$_{i}$ láttam [$t_i$ a kalap-já-t] \\
    Peter-\textsc{dat} saw.1\textsc{sg} {} the {hat-\textsc{poss}-\textsc{acc}} \\
    \glt `I saw Peter's hat'
\z

\noindent There are no restrictions for both the dative and the nominative possessor. The only necessary assimilation of the CN context to the possessor phrase in SpecD is the elimination of the CN formative in case the possessor has the CN formative or some D expression at its beginning, as in \REF{ex:zi91:17b}.\footnote{In standard Hungarian, proper names are used without the term maker \textit{a}(\textit{z}) ‘the’.} The details of this rule need not concern us here (see fn. \ref{fn:zi91:15}).

My analysis in \REF{ex:zi91:17} differs from \citet{Szabolosi87Functional-categories} in four interrelated respects.

Firstly, I regard determiners like \textit{ezen} `this', \textit{minden} `every' etc. as functional heads of category D with the projections D$'$ and DP, as does \citet{Abney86Functional-elements, Abney87The-English}. \citet{Szabolosi87Functional-categories} reserves for them the category Art(icle) without projecting capacity:

\ea \label{ex:zi91:19} $[$\textsubscript{CN$''$} [\textsubscript{CN$'$} CN [\textsubscript{IN$''$} CN$''$ [\textsubscript{I$'$} \text{Art} [\textsubscript{$\overline{\text{IN}}$} N IN]]]]] \vspace{5pt}

(CN = nominal C, IN = nominal I, $\overline{\text{IN}}$ = complex predicate)
\z

\largerpage
\noindent Secondly, I do not consider the possession marker (IN in \REF{ex:zi91:19}) a projecting category to be compared with I(nfl), despite the agreement with the possessor. As is evident from \REF{ex:zi91:19}, \citet[171]{Szabolosi87Functional-categories} views IN as head of the complement of CN. IN has the feature $\pm$poss, AGR, comparable to the verbal I(nfl) with its features $\pm$tense, AGR. The characteristic of IN in comparison with I(nfl) consists in the semantic potential of $\pm$poss to equip the nominal subject with a theta role. I agree with \citet{Szabolosi87Functional-categories}, \citet{Anderson83Prenominal-genitive} and others that the theta role of the possessor phrase in noun phrases can have its origin in the semantics of an affix. But I hesitate to draw the conclusion that this affix is parallel to verbal I(nfl). Functional categories are semantically void of descriptive content and consequently unable to assign a theta role to an argument (see \citealt{Abney86Functional-elements, Abney87The-English}). \citeauthor{Szabolosi87Functional-categories}'s $-$poss IN will always be phonologically empty as in simple noun phrases without a possessor like \textit{ezen kalap} `this hat'. My revision lacks any basis for such a conclusion. I do not need IN in such cases.

Thirdly, my analysis provides a possessor position in N$'$. Therefore, I have more traces in \REF{ex:zi91:17} than \citet{Szabolosi87Functional-categories}. I restrict the semantic scope of the possession marker to the domain of N$'$. The affixed nominal head with its semantics -- enriched by one slot -- assigns a theta role to the possessor expression as in \REF{ex:zi91:17}, which is the head of a chain with traces.

Fourthly, as is clear from the foregoing, I do not regard the determiner (i.e. Art in \REF{ex:zi91:19}) as an entity which fills a semantic slot of the semantic structure of the possession marker. \citet{Szabolosi87Functional-categories} has given the neighbourhood of the subject to CN as in \REF{ex:zi91:17} a semantic basis in assuming that the possessive marker reserves in its argument structure a slot for a D entity with its semantics. My analysis avoids such an ad hoc assumption.

\largerpage
I believe in a deeply founded semantic and syntactic parallelism of C and D in specifying the reference type of the respective CP and DP and in term constitution (see \citealt{Zimmermann88Wohin-mit-Affixen}). And I hope that the structure in \REF{ex:zi91:17} proposed here for Hungarian noun phrases with a possessor takes account of the facts, including the relatedness of sentences and noun phrases as referring expressions.

\largerpage
In \REF{ex:zi91:20} I give the semantics of the Hungarian CNPs in \REF{ex:zi91:17}, ingredient by ingredient. \REF{ex:zi91:20a} shows the semantics of the possession marker.\footnote{Compare \citet[184 (32)]{Szabolosi87Functional-categories}.}\textsuperscript{,}\footnote{According to \citet{Melcuk73On-the-possessive-form}, \textit{kalapja} `hat' in \REF{ex:zi91:17} morphologically consists of the stem \textit{kalap-}, the possession marker \textit{-ja} and a zero morpheme for the third person singular. In view of the fact that $-$1ps, $-$2ps, $-$plural are the unmarked values for person and number, the absence of sound is natural. \textit{káz-a-i-m} (`house' - possession marker - plural marker - first person singular marker) exhibits marked realizations of the person and number feature. I will not go into the details here. In any case, the possession marker restricts the combinatory potential of the noun, morphologically and syntactically. \textit{az} (\textit{én}) \textit{házaim} `my houses' shows the agreement of the possessor \textit{én} `I' and the possessed noun with the morpheme \textit{-m} for first person singular.} The ‘:’ in the semantic form of \REF{ex:zi91:20} and in \REF{ex:zi91:21} is an asymmetric conjunction, a kind of attributor, and reads “such that”.\footnote{Guest editors' note: throughout this paper $\widehat{x}$ is used to express lambda abstraction, equivalent to  $\lambda x$.}

\ea \label{ex:zi91:20}
    \ea \label{ex:zi91:20a} possession marker: $\widehat{P}_{2} \; \widehat{P}_{1} \; \widehat{x} \; [{P}_{1} \; [\widehat{y} \; [[P_{2} \; x] : [x \; R \; y]]]]$
    \ex \label{ex:zi91:20b} \textit{kalap-} : $\widehat{x} \; [\textsc{hat} \; x]$
    \ex \label{ex:zi91:20c} \textit{Péter-} : $\widehat{P}_{1} \; [P_{1} \; \textsc{peter}]$
    \ex \label{ex:zi91:20d} \textit{minden} : $\widehat{P}_{2} \; [\forall x \; [[P_{2} \; x] \rightarrow [P_{3} \; x ]]]$
    \ex \label{ex:zi91:20e} \textit{a(z)} : $\widehat{p} \;  \widehat{P}_{3} \; [p]$
    
    (with $P_{i} \in S/N$, $R \in (S/N)/N$, ${P}_{1} \in S/(S/N)$)
    \z
\z

\noindent Successive functional application, $(e(d((a(b))(c))))$, results in \REF{ex:zi91:21}.

\ea \label{ex:zi91:21} $\widehat{P}_{3} \; [\forall x \; [[\textsc{hat} \; x] : [x \; R \; \textsc{peter}]] \rightarrow [P_{3} \; x]]$
\z

\noindent The theta role the term \textit{Péter} receives from the possession marker is based on an anonymous relation to another entity which will be specialized in the conceptual structure depending on context.\footnote{\label{fn:zi91:14}\textit{Az} (\textit{én}) \textit{olvasasom} `(the) my reading' is adduced in \citet[n. 5]{Melcuk73On-the-possessive-form} as an indication of the very abstract character of the possession relation. According to our conception of the interrelation of semantic structure and conceptual structure, the anonymous relation $R$ of \REF{ex:zi91:20a} in the above example will be identified with reading and the possessor with the reader in conceptual structure.} $(a(b))$ is the complex predicate (\citeauthor{Szabolosi87Functional-categories}'s \citeyear{Szabolosi87Functional-categories} $\overline{IN}$ in \REF{ex:zi91:19}) with the semantics of \textit{kalap-} `hat' and of the possession marker as composite parts. It goes without saying that the semantics of the possession marker is valid not only for Hungarian. As to the semantic structures in \REF{ex:zi91:20d} and \REF{ex:zi91:20e}, they, too, can be met in the semantics of other languages, provided there is a comparable division of labor as in Hungarian in specifying the reference type and term constituting.

Like \citet{Abney87The-English}, I will assume that the existence / nonexistence of a functional category CN with its projections CN$'$, CNP constitutes a parameter of UG. It seems reasonable to look at the entities of category D in languages without a CNP layer as mergers of CN and D, perhaps resulting from D to CN movement or in another perspective (see \citealt{Haider88Die-Struktur}) as matching projections.

I will now turn to other languages which do not exhibit a CNP layer and which do not mark the presence of a possessor by an affix and by agreement features on the lexical head of the container noun phrase.\footnote{\label{fn:zi91:15} In leaving the Hungarian CNPs, I wish to raise some objection to the distributional analysis of the Hungarian personal pronouns given by \citet{Reichert86Verteilung-und-Leistung}. I have got much insight from this work, but with respect to the cooccurrence of the personal pronouns like \textit{én} `I' and the term maker \textit{a}(\textit{z}) ‘the’, I was left disturbed. Why -- I asked myself -- is the (stressed) CP subject \textit{én} `I', whereas according to \posscitet{Reichert86Verteilung-und-Leistung} analysis (see op. cit.: 56ff.) -- the corresponding CNP subject (i.e. the nominative possessor) is [\textit{az én}]. Only from \posscitet{Szabolosi87Functional-categories} description of the facts I have learned that in CNPs, too, the first person subject is \textit{én}, whereas \textit{a}(\textit{z}) is the characteristic Hungarian term maker CN. [[\textit{a férfi}] \textit{háza}] `the man$'$s house' vs. [\textit{az} [[\textit{én}] \textit{házam}]] `my house' (literally: `the I house-my') seem to be the correct analyses, with one occurrence of \textit{a}(\textit{z}) in the first example being deleted in contact with the \textit{a}(\textit{z}) of the possessor phrase. I thank Elisabeth Komlósy, Gert Sauer and Petra Hauel for discussion about the topic.}

%section 4
\section{} \label{sec:zi91:4}

In this section, I will adduce data which have influenced my analysis of noun phrases with subject expressions and my decision to follow \citet{Szabolosi87Functional-categories} in assuming two different prenominal subject positions.

Many languages exhibit the linear word order totality pronoun -- demonstrative pronoun or determiner -- possessive pronoun -- numeral adjectival modifiers -- head noun. This again confirms the insight that determiners and possessor expressions are not in complementary distribution and serve different semantic functions. In accordance with the structure given in \REF{ex:zi91:11} above, I assume that determiners and demonstrative pronouns are instances of D. The totality pronouns can be adjoined to D. Numerals are the head of QP. As QP, adjectival modifiers are adjuncts of N$'$. As for possessive pronouns and certain other prenominal subject expressions, I assume that they occupy the SpecN position, in SS. Italian and Bulgarian are illuminating in this respect.\footnote{I rely on \citet{Honti14Italienische-Elementargrammatik} and \citet{Sauer21Italienische-Konversationsgrammatik} and on help by Christina Lange, for Italian, and on \citet{Walter87Lehrbuch-der-bulgarischen} and on help by Iva Petkova-Schick, for Bulgarian.}

\subsection{} \label{sec:zi91:4.1}

Italian confronts us with the fact that possessive pronouns agreeing with N occur immediately after the determiners. Pronominal phrases with the preposition \textit{di} `of' can do that, too:\footnote{The Italian relative pronoun \textit{cui} in \REF{ex:zi91:25} loses the preposition \textit{di} ‘of’ in prenominal position.}

\ea \label{ex:zi91:22} 
    \gll il / un / questo mio amico \\
    the {} a {} this my friend \\
    \glt
\ex \label{ex:zi91:23}
    \gll i miei tre nuovi libri \\
    the my three new books \\
    \glt
\ex \label{ex:zi91:24}
    \ea \label{ex:zi91:24a}
        \gll la sorella ed i suoi figli \\
        the sister and the \textsc{poss.ref} children \\
        \glt 
    \ex \label{ex:zi91:24b}
        \gll la sorella ed i figli di lei \\
        the sister and the children of her \\
        \glt 
    \ex \label{ex:zi91:24c} 
        \gll la sorella ed i di lei figli \\
        the sister and the of her children \\
        \glt
    \z
\ex \label{ex:zi91:25} 
    \gll l' uomo, la cui casa \\
    the man the  who.\textsc{gen} house \\
    \glt 
\ex \label{ex:zi91:26}
    \gll Giuseppe, mio amico \\
    Joseph my friend \\
    \glt
\ex \label{ex:zi91:27}
    \gll Giuseppe è mio amico \\
    Joseph is my friend \\
    \glt 
\z

\noindent It is not my aim to give a complete survey of the Italian subject phrases. I took those that are of significance for the present discussion. The examples show that determiners precede the pronominal subject phrases whereas numerals and adjective phrases follow them provided the subject does not figure after the N. Furthermore, unlike English, German, Russian and other languages, Italian has an overt differentiation between definite, indefinite and predicative resp. appositive noun phrases even with possessors.\footnote{In contrast to \citet{Delsing88The-Scandinavian}, \citet{Felix88Structure-functional} does not allow NPs to occur without functional head. He envisages a biunique selection of functional heads and their complement. Such a very restrictive conception amounts to saying that there are no D-less NPs, as there are no C-less IPs and no I-less VPs. With respect to constructions as in \REF{ex:zi91:26}--\REF{ex:zi91:27}, this would mean that in all cases one has to do with full DPs. I do not regard this unreasonable. There is one type of construction, the pseudopartitives (see \citealt{Lobel88D-und-Q, Lobel89Q-as-functional} and \citealt{Burchert90Zur-Syntax}), which seem to warrant Felix$'$ conception, though. Löbel argues that the complement of counter nouns as in \textit{zwei Glas süßer Wein} `two glasses of sweet wine' are NPs, not DPs.}

As to the deep structure position of subject expressions, I assume that it is in N$'$, after N. In many languages, this position is unrestricted whereas the prenominal occurrence of subject expressions is bound to severe restrictions (see \citealt{Bhatt90Die-syntaktische}). Thus, \REF{ex:zi91:28} will have the following structure:

\ea \label{ex:zi91:28} $[$\textsubscript{DP} [\textsubscript{D$'$} [\textsubscript{D} il] [\textsubscript{NP} [\textsubscript{DP} miei] [\textsubscript{N$'$} [\textsubscript{N$'$} [\textsubscript{QP} tre] [\textsubscript{N$'$} [\textsubscript{AP} nuovi] [\textsubscript{N$'$} [\textsubscript{N} libri]]]] $t_i$]]]]
\z

\noindent Like \textit{miei} `my' the prenominal phrase with \textit{di} `of' in (24c) and (25) figures in SpecN. Its trace is in the postnominal position.

Semantically, Italian possessive \textit{di}-phrases and the possessive pronouns will be treated absolutely parallel to the corresponding expressions in German, English, Russian, Bulgarian, Albanian and other languages (see \citealt{Zimmermann90Pranominale-Argument}). They all lack a possession marker on the noun. Therefore one has to decide where the theta role of the possessor expressions comes from. I propose the following: Genitive markers as Italian \textit{di}, German \textit{von}, English \textit{of}, Bulgarian \textit{na} or the corresponding morphological possessive genitive have a meaning which is comparable to the one of the Hungarian possession marker. The difference consists in the way the composite morphosyntactic neighbours of the respective formatives bring into play their semantics. Compare \REF{ex:zi91:20} with \REF{ex:zi91:30} where the semantic components of \REF{ex:zi91:29} are given:\footnote{I borrow the term ‘template’ from \citet{Bierwisch89Dimensional-Adjectives}. Templates enrich the semantic structure of the entity / entities they apply to, which often causes type shifting. The modification template combines a modifier, $Q_{2}$, with a modificandum, $Q_{1}$, such that their argument slots are unified.}

\ea \label{ex:zi91:29}
    \gll la casa di Pietro \\
    the house of Peter \\
    \glt
\ex \label{ex:zi91:30}
    \ea \label{ex:zi91:30a} possession marker: ${\widehat{P}}_{1} \; \widehat{x} \; [{P}_{1} \; [\widehat{y} \; [x \; R \; y]]]$
    \ex \label{ex:zi91:30b} \textit{Pietro}: ${\widehat{P}}_{1} [\widehat{P}_{1} \; \textsc{peter}]$
    \ex \label{ex:zi91:30c} \textit{casa}: $\widehat{x} \; [\textsc{house} \; x]$
    \ex \label{ex:zi91:30d} modification template: $\widehat{Q}_{2} \; \widehat{Q}_{1} \; \widehat{x} \; [[{Q}_{1} \; x]:[{Q}_{2} \; x]]$
    \ex \label{ex:zi91:30e} \textit{la}: $\widehat{P}_{2} \; \widehat{P}_{3} \; [\textsc{def} \; x \; [[P_{2} \;x]\wedge[P_{3} \;x]]]$
    \z
\z

\noindent Successive functional application, $(e((d(a (b))) (c)))$, gives \REF{ex:zi91:31}.

\ea \label{ex:zi91:31} $\widehat{P}_{3} \; [\textsc{def} \; x \; [[\textsc{house} \;x]:[x \; R \; \textsc{Peter}]]\wedge[P_{3} \; x]]$
\z 

\noindent The possessive pronouns have the possession marker meaning incorporated. This is shown in \REF{ex:zi91:33}. The meaning of \REF{ex:zi91:32} is given in \REF{ex:zi91:34}.

\ea \label{ex:zi91:32}
    \gll la mia casa \\
    the my house \\
    \glt
\ex \label{ex:zi91:33} \textit{mi-} : $\widehat{x}[x \; R  \; \textsc{speaker}]$
\ex \label{ex:zi91:34} $\widehat{P}_{3}[\textsc{def} \; x \; [[ \textsc{house} \; x]:[x \; R \; \textsc{speaker}]] \wedge [P_{3} \; x]]$
\z

\noindent All this is valid for the corresponding possessive pronouns and genitive markers of other languages, too.

One revision of the semantics of the possession marker in \REF{ex:zi91:20a}, \REF{ex:zi91:30a} and of the possessive pronouns as in \REF{ex:zi91:33} is necessary in order to account for the characteristic argument-adjunct semantics of possessors. \citet{Grimshaw88Adjuncts-and} discussed this phenomenon on principle (see also \citealt{Anderson83Prenominal-genitive}). The problem is this: The possessor phrases in \REF{ex:zi91:17}, \REF{ex:zi91:23}, \REF{ex:zi91:25}, \REF{ex:zi91:29}, \REF{ex:zi91:32} are modifiers with respect to the head noun, which is nonrelational in its semantics. On the other hand, the possessor phrases in \REF{ex:zi91:22}, \REF{ex:zi91:24}, \REF{ex:zi91:26}, \REF{ex:zi91:27} are arguments of the relational nominal head. The same is true for subject expressions as arguments of deverbal and deadjectival nouns (see below and fn. \ref{fn:zi91:14}, as well as \citet{Huste89Zur-Syntax} and \citet{Bischof91Sachverhaltsbezeichnungen-des}). In view of my assumption that the subject originates in N$'$, where, too, its semantics comes into play, I adapt the meaning of the possession marker and of the possessive pronoun to the possible argument status of the subject as follows:

\begin{exe}
    \exp{ex:zi91:20a} \label{ex:zi91:20p} possession marker: $(\widehat{P}_{2} \;{\widehat{P}}_{1} \; \widehat{x} \; [{P}_{1} \; [\widehat{y} \; [[P_{2} \; x] : [x \; R \; y]]]])$
    \exp{ex:zi91:30a} \label{ex:zi91:30apB} possession marker: $({\widehat{P}}_{1} \; \widehat{x} \; [{P}_{1} \; [\widehat{y} \; [x \; R \; y]]])$
    \exp{ex:zi91:33} \label{ex:zi91:33p} \textit{mi-} : $[({\widehat{P}}_{1} \; \widehat{x} \; [{P}_{1} \; [\widehat{y} \; [x \; R \; y]]]) \; [{\widehat{P}}_{4}[P_{4} \; \textsc{speaker}]]]$
\end{exe}

\noindent This amounts to regarding the possession semantics as a kind of semantic template associated with certain formatives (lexical items, affixes) which is activated only in cases of modification, i.e. where the head noun lacks a semantic slot for the possessor. This sleeping semantic potential of the possession meaning is indicated above by brackets. Thus, the genitive and the Hungarian possession morpheme on nouns can be semantically inactive (i.e. empty).\footnote{Let me emphasize that I will not have “dangling” prepositions as occur in many papers on English. English \textit{of}, German \textit{von}, French \textit{de}, Italian \textit{di}, Bulgarian \textit{na} are prepositions and -- as case markers -- can be semantically empty. The same is true with respect to the morphological genitive and to the English possessor morpheme \textit{-s}.} The same is true of the corresponding part in the semantics of possessive pronouns. This semantic inactivity occurs in contact with relational and deverbal or deadjectival nouns, whose argument structure overrides the very general possession meaning of the pertinent possessor phrases, as in \REF{ex:zi91:35} with the semantics \REF{ex:zi91:37} composed of the ingredients in \REF{ex:zi91:36}.

\ea \label{ex:zi91:35} 
\gll il mio amico \\
the my friend \\
\glt
\ex \label{ex:zi91:36}
    \ea \label{ex:zi91:36a} \textit{amico}: ${\widehat{P}}_{4} \; \widehat{x} [{P}_{4} \; [\widehat{y} \; [x \; \textsc{friend} \; y]]]$
    \ex \label{ex:zi91:36b} \textit{mi-}: $\widehat{P}_{4} \; [P_{4} \; \textsc{speaker}]$
    \ex \label{ex:zi91:36c} \textit{il}: $\widehat{P}_{2} \widehat{P}_{3} \; [\textsc{def} \; x \; [[P_{2} \; x] \wedge [P_{3} \; x]]]$
    \z
\ex \label{ex:zi91:37} $\widehat{P}_3 \; [\textsc{def} \; x \; [x \; \textsc{friend} \; \textsc{speaker}] \wedge [P_{3} \; x]]$
\z

\noindent As to the syntax of \REF{ex:zi91:29} and \REF{ex:zi91:35}, it should be mentioned that the subject phrase is in different places, depending on its function. The modifier will be a sister of N$'$ (as is the trace $t_i$ of the possessive pronoun in \REF{ex:zi91:28}) whereas the argument will be a sister of the nominal relational head. These differences in syntactic structure correspond to the difference in meaning, as indicated in \REF{ex:zi91:31} and \REF{ex:zi91:37}, respectively.

In section \ref{sec:zi91:5}, I will discuss another possibility of treating subjects in noun phrases, which will exploit \posscitet{Grimshaw88Adjuncts-and} conception of argument-adjuncts.

\subsection{} \label{sec:zi91:4.2}

I will now turn to Bulgarian, where, too, overt determiners occur in front of the DP, before possessive pronouns. Only totality pronouns -- as in other languages as well -- can go ahead.

\ea \label{ex:zi91:38}
    \gll vsički tezi moi / mi petdeset i dve xubavi knigi \\
    all these my {} me.\textsc{dat} fifty and two nice books \\
    \glt
\z

\noindent The totality pronoun is adjoined to D (with a trace in QP). The demonstrative pronoun is the head of DP. The possessive pronoun or the clitic dative personal pronoun occur in SpecN with a trace in N$'$. The numeral is the head of QP. The adjective is a prenominal adjunct. The noun heads NP. All this is absolutely parallel to the facts in Italian (see structure \REF{ex:zi91:28}).

One peculiarity of Bulgarian consists in the clitic realization of the definite article. The bearer of the clitic morpheme -- itself agreeing with the head in gender and number -- is the first member of the agreement chain spreading from D to N.

\ea \label{ex:zi91:39}
    \ea \label{ex:zi91:39a}
        \gll vsički-te moi / mi xubavi knigi \\
        all-the my {} me.\textsc{dat} nice books \\
        \glt
    \ex \label{ex:zi91:39b}
        \gll moite / \minsp{*} mi-te xubavi knigi \\
        my-the {} {} me.\textsc{dat}-the nice books \\
        \glt
    \ex \label{ex:zi91:39c}
        \gll petdeset i dve-te xubavi knigi \\
        fifty and two-the nice books \\
        \glt
    \ex \label{ex:zi91:39d}
        \gll xubavi-te \minsp{(} mi) knigi \\
        nice-the {} me.\textsc{dat} books\\
        \glt
    \ex \label{ex:zi91:39e}
        \gll knigi-te \minsp{(} mi) \\
        books-the {} me.\textsc{dat} \\
        \glt
    \z
\z

\noindent These clear facts fit very well into the syntactic structure assumed for Italian noun phrases with a possessor (see \REF{ex:zi91:28}). The only thing one has to take special care of is the placement of the clitics \textit{t-} `the' and \textit{mi} `me'. \textit{t-} combines with the first member of the agreement chain of DP in the surface structure OS. \textit{mi} combines with the first nonclitic DP constituent. I assume that in DS and SS the clitic \textit{t-} as the head of DP figures in D, in complementary distribution with the demonstrative pronoun.\footnote{\citet{Delsing88The-Scandinavian, Delsing90A-DP-analysis} gives a quite analogous analysis of DPs with clitic definiteness markers in Scandinavian languages.} As lexical item, the clitic \textit{t-} is equipped with agreement requirements as to its complement NP and to its word structure sister in OS as well. The first type of cooccurrence restriction is known as functional selection. The second type is an idiosyncratic cooccurrence restriction as to the sister in OS word structure.

The example in \REF{ex:zi91:40} where we have to do with two DPs -- one embedded within the other -- confirms all this. \REF{ex:zi91:41} and \REF{ex:zi91:42} show structures with coordinated adjective phrases. \REF{ex:zi91:42} indicates reference to two different objects by the twofold expression of definiteness on the coordinated adjective phrases.

\ea \label{ex:zi91:40}
    \ea \label{ex:zi91:40a}
        \gll gordij-at sâs svoi-te deca bašta \\
        proud-the of his-the children father \\
        \glt
    \ex \label{ex:zi91:40b}
        \gll gordij-at s deca-ta si bašta \\
        proud-the of children-the him.\textsc{dat} father \\
    \z
\ex \label{ex:zi91:41}
    \gll vkusno-to i pikantno jadene \\
    savoury-the and spicy dish \\
    \glt
\ex \label{ex:zi91:42}
    \gll bulgarskij-at i nemskij-at ezik \\
    Bulgarian-the and German-the language \\
    \glt
\z

\noindent An idiosyncratic cooccurrence restriction of the clitic dative possessors is the requirement of a definite OS sister. In indefinite noun phrases, one does not encounter them (compare \REF{ex:zi91:39d} and \REF{ex:zi91:43}).

\ea \label{ex:zi91:43} 
    \gll \minsp{*} xubavi mi knigi \\
    {} nice my books \\
    \glt
\z

\noindent A last lexical peculiarity of Bulgarian, which it shares with other Slavonic languages, is the existence of possessive adjectives, besides the possessive pronouns. Compare:

\ea \label{ex:zi91:44}
    \ea \label{ex:zi91:44a} 
        \gll tezi dve knigi na Ivanka \\
        these two books of Ivanka \\
        \glt 
    \ex \label{ex:zi91:44b}
        \gll tezi dve Ivankini knigi \\
        these two Ivanka's books \\
        \glt
    \z
\z

\noindent In \REF{ex:zi91:44a}, the possessor is expressed by a prepositional phrase, whereas \REF{ex:zi91:44b} shows a possessive adjective derived by the suffix \textit{-in}. Its semantics is quite parallel to the one given in (30a$'$) for the genitive possession marker which applies to Bulgarian \textit{na} `of' as well. Whereas \textit{na} `of' is applicable to any DP, the possessive suffixes \textit{-in} and \textit{-ov} are highly restricted.\footnote{For the possessive suffixes \textit{-in}, \textit{-ov} see \citet[153f.]{Tilkov83Gramatika-na-bulgarskata}.} One peculiarity consists in the requirement of a definite (“known”) argument of the possession relation. Possibly, the semantics of the derivational affix must take this fact into consideration (see \citealt{Zimmermann90Pranominale-Argument}). As to the syntax of possessive adjectives, one thing is remarkable. Whereas possessive pronouns occur in SpecN (see \REF{ex:zi91:28}), it is evident from \REF{ex:zi91:44b}) that the possessive adjectives do not. They behave like other adjectival modifiers in staying at home, in N$'$ after QP.

Let us sum up. Bulgarian, like Hungarian, Italian, and German, clearly demonstrates that there is no complementary distribution of determiners and possessors and that there is reference type variation of the noun phrases with possessors. Bulgarian, like Italian, shows that possessor phrases can be located as right neighbours of the determiners, in SpecN, provided they do not belong to the possessive adjectives in N$'$. A common property of Bulgarian and Italian, German and many other languages concerns the placement of possessor expressions in accordance with the universal \REF{ex:zi91:15}. Possessive pronouns and possessive adjectives occur on the lefthand side of the nominal head, whereas genitive and prepositional subject phrases normally are located on its right. Furthermore, Bulgarian, like Italian, German and other languages has simple (non-tautological) signalization of the reference type of the container noun phrase. In Scandinavian languages (see \citealt{Hellan86The-headedness}, \citealt{Delsing88The-Scandinavian, Delsing90A-DP-analysis}, and \citealt{Vater89Determinantien-in-DP}) and in Albanian (see \citealt{Buchholz87Albanische-grammatik}) one finds various manifestations of tautological definiteness signalization. In German, English and Russian, on the other hand, we have no definiteness signalization at all in noun phrases with prenominal possessive pronouns or with prenominal possessive genitives, provided there is no demonstrative pronoun in D. How can our assumptions on the syntactic structure of DPs with possessors be reconciled with all these facts?

\subsection{} \label{sec:zi91:4.3}

First, very shortly, I will turn to Albanian.\footnote{I am very grateful to Oda Buchholz for her patience in explaining me the very complicated facts of Albanian noun phrase syntax and morphology.} Two phenomena in the structure of Albanian noun phrases are especially relevant for our considerations.

Possessive pronouns as well as possessive genitive phrases are located after the nominal head.\footnote{I ignore the highly restricted prenominal use of possessive pronouns (see \citealt[284f., 411f.]{Buchholz87Albanische-grammatik}).} As a rule, the possessive pronouns precede other constituents of NP, whereas prepositional phrases and the genitive possessor figure at the right periphery of NP. I will assume that the various orderings of postnominal constituents of NP obey filters regulating the linearization of dependants according to their syntactic category and to their status as to focus and background.\footnote{For Russian see \citet{Huste89Zur-Syntax, Huste89Zur-Topologie} and \citet{Bischof91Sachverhaltsbezeichnungen-des}. For German see \citet{Freytag90Die-syntaktische, Freytag91Sachverhaltsbezeichnungen-des}.} I cannot go into the very complicated details here. In any case, Albanian does not follow the universal \REF{ex:zi91:15}. Nouns govern rightwards, and phrases agreeing with the nominal head -- except for QPs -- are placed on its right, as are genitive and prepositional phrases. Nevertheless, I assume the DS \REF{ex:zi91:11} for Albanian noun phrases.

\ea \label{ex:zi91:45}
    \gll \minsp{(} të) gjithë këta libra-t tanë të rinj \\
    {} \textsc{am} all these books-the our \textsc{am} new \\
    \glt 

\ex \label{ex:zi91:46}
    \gll \minsp{(} të) gjithë këta libra-t e rinj të Besim-i-t \\
    {} \textsc{am} all these books-the \textsc{am} new \textsc{am} Besim-\textsc{gen}-the \\
    \glt
\z

\noindent The totality pronoun in \REF{ex:zi91:45}, \REF{ex:zi91:46} is adjoined to D, with a trace in QP. The demonstrative pronoun is the D head of DP with tautological definiteness marking on the nominal head. The postnominal adjective phrase and the possessor, the possessive pronoun in \REF{ex:zi91:45} and the genitive phrase in \REF{ex:zi91:46}, are adjuncts of N$'$, at least in DS. The formatives \textit{të} and \textit{e} in \REF{ex:zi91:45} and \REF{ex:zi91:46} are attribute markers, whose precise distributional properties need not concern us here. I assume that they are adjuncts to the respective XPs, as shown in \REF{ex:zi91:47} for the genitive possessor in \REF{ex:zi91:46}.

\ea \label{ex:zi91:47} $[$\textsubscript{DP} të [\textsubscript{DP} Besimit]]
\z

\noindent As in Hungarian, Italian and Bulgarian, the occurrence of a possessor expression in Albanian noun phrases is compatible with different reference types of the container DP. Whereas in \REF{ex:zi91:45}, \REF{ex:zi91:46} and \REF{ex:zi91:49} the DPs are definite, the DP in \REF{ex:zi91:48} is indefinite.

\ea \label{ex:zi91:48}
    \ea \label{ex:zi91:48a}
        \gll një çantë e tij e bukur prej lëkure \\
        a bag \textsc{am} his/her \textsc{am} nice of leather \\
        \glt 
    \ex \label{ex:zi91:48b}
        \gll një çantë e bukur prej lëkure e Besim-i-t \\
        a bag \textsc{am} nice of leather \textsc{am} Besim-\textsc{gen}-the \\
        \glt
    \z
\ex \label{ex:zi91:49}
    \ea \label{ex:zi91:49a}
        \gll $\emptyset \;$ çant-a e tij e bukur prej lëkure \\
        {} bag-\textsc{nom}.the \textsc{am} his/her \textsc{am} nice of leather \\
        \glt 
    \ex \label{ex:zi91:49b}
        \gll $\emptyset \;$ çant-a e bukur prej lëkure e Besim-i-t \\
        {} bag-\textsc{nom}.the \textsc{am} nice of leather \textsc{am} Besim-\textsc{gen}-the \\
        \glt
    \z
\z

\noindent I regard \textit{një} `a' / `one' as bifunctional, like German \textit{ein}: as numeral it is Q, as determiner it is D. $\emptyset$ indicates the silent definite determiner, which is legitimized by \citeauthor{Emonds85Unified-theory}' (\citeyear{Emonds85Unified-theory, Emonds87Invisible-category}) empty head principle \REF{ex:zi91:50}.

\ea \label{ex:zi91:50} A closed category B with positively specified features C$_{i}$ may remain empty throughout a syntactic derivation if the features C (save possibly B itself) are all alternatively realized in a phrasal sister of B \citep[615]{Emonds87Invisible-category}.
\z

\noindent Precisely this is the case in \REF{ex:zi91:49}, where the feature definite of the phonologically empty determiner is signalized on the head noun. Interestingly enough, this assistant definiteness marking is on the adjective when it precedes the noun as in \REF{ex:zi91:51}. In such very marked cases of constituent order the adjective phrase is emphasized.

\ea \label{ex:zi91:51}
    \gll e bukur-a çantë \\
    \textsc{am} nice-the bag \\
    \glt
\z

\noindent As to the semantics of the Albanian possessive expressions, nothing new must be added to the representations in (30a$'$), (33$'$), which are valid for Albanian, too.

To conclude, Albanian delivers much empirical evidence for the correctness of the assumption that possessors originate in NP. Furthermore, \citeauthor{Emonds87Invisible-category}' (\citeyear{Emonds87Invisible-category}) 
empty head principle is strongly confirmed by the facts from Albanian. Evidently, the universal \REF{ex:zi91:15} must be parametrized. It is not valid for Albanian. With these insights, let us turn to German, English and Russian.

%section 5
\section{} \label{sec:zi91:5}

The plan of this section is as follows. In \ref{sec:zi91:5.1}. I will characterize the SS positions for prenominal subject expressions more concretely. Special attention will be directed to $+$poss-marked phrases. In \ref{sec:zi91:5.2}, the morpho-syntactic behavior of possessive pronouns is discussed. \ref{sec:zi91:5.3} concerns the relation of prenominal and postnominal subject phrases. In \ref{sec:zi91:5.4}, the subject expressions are inspected with respect to their possible functional characterization as argument-adjuncts. All this takes into consideration data from German, English and Russian.

\subsection{} \label{sec:zi91:5.1}

In view of the facts of the languages considered so far, there are three different prenominal positions subject expressions can occupy in SS. Leaving aside the CNP layer of Hungarian noun phrases, I will reckon with the positions indicated in \REF{ex:zi91:52}.

\ea \label{ex:zi91:52} $[$\textsubscript{DP} a [\textsubscript{D$'$} D [\textsubscript{NP} b [\textsubscript{N$'$} ... (QP) ... [\textsubscript{N$'$} c ... N ...]]]]]
\z

\noindent Position a can be occupied by German prepositional phrases including possessive $von$-phrases, as in \REF{ex:zi91:1b}. Position b can be occupied by all other prenominal subject expressions, except for the Slavonic possessive adjectives occurring in position c, as in \REF{ex:zi91:44b}. This amounts to saying that prenominal possessive pronouns as in \REF{ex:zi91:2c} -- even combined with a dative possessor as in (3) --, prenominal genitive subjects as in \REF{ex:zi91:2b} and the Italian pronominal \textit{di}-phrase as in \REF{ex:zi91:24} figure in b, the SpecN position. The same is true for English and Russian prenominal subjects.

Essential to my analysis is the assumption that all phrases capable of figuring in SpecN -- adjacent to D -- are marked by the case feature $+$poss and with this case-marking can be licensed only in the b position of \REF{ex:zi91:52}. A characteristic structural property of SpecX is the possibility of its being governed from outside (see \citealt{Chomsky86Barriers, Chomsky86Knowledge-of-language}). A possible governor for SpecN is D. So, I assume that D entities which have licensing capacity are equipped with the licensing feature <+poss>. In contrast to \citet{Fukui86A-theory}, \citet{Abney86Functional-elements, Abney87The-English} and many others I assume that D governs to the right, where is the position of its complement, NP, and the SpecN within. This means that D cannot license phrases in SpecD, on its lefthand side. Phrases occurring there do not need licensing in this position. And exactly this is the case with prepositional phrases as in \REF{ex:zi91:2b}. On the other hand, $+$poss-marked constituents cannot go to SpecD. They need licensing, and this takes place in SpecN.

Thus, checking (licensing) of structural case in DP in many languages operates uniformly rightwards and under the condition of adjacency of the pertinent governor, N or D, and the case-marked phrase, DP or PP as the Italian \textit{di}-phrase.

Both features, the case-marking feature $+$poss and the case-checking feature, originate in the lexicon. Possessive pronouns, the English phrasal affix \textit{-s} and optionally the German genitive affix or the corresponding preposition like the Italian \textit{di} are marked by the case feature $+$poss. Determiners which are compatible with subject expressions in SpecN bear the licensing feature <+poss>. By this marking of lexical entries there is a considerable span for language-specific variation and idiosyncratic restrictions.

Thus, for instance, Russian genitive phrases are not marked by $+$poss. At least in Standard Russian, they do not figure in SpecN and are located postnominally, in accordance with the universal \REF{ex:zi91:15}.

Likewise, determiners in one language and in different languages can differ in having or not having the licensing feature <+poss>. German, English and Russian have a silent determiner with <+poss> to license subject expressions in SpecN.\footnote{For empty functional heads see \citet{Zimmermann88Wohin-mit-Affixen, Zimmermann90Zur-Legitimierung}.} In contrast to Italian and Bulgarian, the German and English definite article is not compatible with subjects in SpecN and therefore lacks the feature <+poss>. The German and Russian demonstrative pronouns can license $+$poss-marked phrases in Spec. All these peculiarities have their basis in the pertinent lexical entries, which project their structural information into the structure of phrases.

\subsection{} \label{sec:zi91:5.2}

As to the syntactic categorization of possessive pronouns, I follow \citet{Olsen89Das-Possessivum} in considering them intransitive determiners. I classify them by the features $+$D $-$K $+$V $+$N $-$Q $+$poss $+$strong inflection. The inflectional affix adds the agreement features of case, number and gender. In contrast to \citet{Olsen89Das-Possessivum} and \citet{Bhatt90Die-syntaktische} and as a consequence of my analysis and of my general assumptions on morphology (see section \ref{sec:zi91:2}), I do not separate the inflectional affix from the stem of possessive pronouns. The stem and the affix come into syntactic structure from the lexicon as a syntactic unit which takes part in the agreement chain spreading from D to N in the container DP. (For an account of agreement in noun phrases see \citealt{Olsen88Agreement-und-Flexion, Olsen89AGReement-in}.)

For the semantics of possessive pronouns see (33$'$) above. As to possessive pronouns of third person, one peculiarity of colloquial German must be taken into account, i.e. the possibility of their combination with a dative phrase as in \REF{ex:zi91:3}. I follow \citet{Bhatt90Die-syntaktische} in considering the dative possessor as belonging to the DP constituted by the possessive pronoun as head of the construction. Semantically, the dative phrase can be conceived of as an explication of the meaning of the pronoun constituting a semantic parameter without this explication. The semantic form \REF{ex:zi91:54a} for \textit{sein-} `his' shows the facultative character of this explication. \REF{ex:zi91:54} gives the ingredients of the resulting SF \REF{ex:zi91:55} for \REF{ex:zi91:53}. \REF{ex:zi91:54e} shows the semantics of the silent determiner.

\largerpage

\ea \label{ex:zi91:53}
    \gll Peter sein Hund\\
    Peter his dog\\
    
\ex \label{ex:zi91:54}
    \ea \label{ex:zi91:54a} \textit{sein-} : $( {\widehat{P}}_{1})_{\alpha} \;\: \widehat{x} \; ([{P}_{1} \; [\widehat{y} \; )_{\alpha} \;\: [x \; R \; y] \; (]])_{\alpha}$
    \ex \label{ex:zi91:54b} \textit{Peter}: ${\widehat{P}_{1}} \; [P_{1} \; \textsc{peter}]$
    \ex \label{ex:zi91:54c} \textit{Hund}: $\widehat{x} \; [\textsc{hund} \; x]$
    \ex \label{ex:zi91:54d} modification template: $\widehat{Q}_{2} \; \widehat{Q}_{1} \; \widehat{x} \; [[Q_{1} \; x]:[Q_{2} \; x ]]$
    \ex \label{ex:zi91:54e} $\emptyset$ : $\widehat{P}_{2} \; \widehat{P}_{3} \; [\textsc{def} \; x \; [P_{2} \; x] \wedge [P_{3} \; x]]$ 
    \z
\ex \label{ex:zi91:55} $\widehat{P}_{3} \; [\textsc{def} \; x \; [[\textsc{hund} \; x]:[x \; R \; \textsc{peter}]] \wedge [P_{3} \; x]]$
\z

\noindent As to the regularities of strong vs. weak inflection of adjectival entities discussed in \citet{Olsen88Agreement-und-Flexion, Olsen89AGReement-in} and in \citet{Delsing88The-Scandinavian, Delsing90A-DP-analysis}, I assume the following. Adjectives in German inflect strongly in the default case. If in an agreement chain they are headed by a DP or a QP with their head inflecting strongly they inflect weakly. This is required by the filter \REF{ex:zi91:56}.

\ea \label{ex:zi91:56} 
\begin{align*} 
\left[ \begin{array}{c}
     +N \\
     +V
\end{array} \right]^{i} 
\rightarrow 
\left\{ \begin{array}{ll}
    \mbox{--strong inflection} \;\; / & 
        \left[ \begin{array}{l}
            \alpha D \\
            -\alpha Q \\
            +N \\
            +V \\
            \mbox{+strong inflection}
        \end{array} \right]^{i} \\
    & \\
    \mbox{+strong inflection} & \mbox{elsewhere}
\end{array} 
[_{N'} \; ... \; \mbox{---} \; ...] \right. 
\end{align*}
\z

\noindent The superscript relates the phrases involved in the argument chain. The filter applies to determiners including possessive pronouns and to quantifiers inflecting strongly. All these entities require weak inflection of the $+$N $+$V heads following them in the agreement chain, as shown in \REF{ex:zi91:57}.

\ea \label{ex:zi91:57} 
    \gll \minsp{(} wegen) dieses / des / eines / meines schönen großen \minsp{(} Gartens)\\
    {} {because of} this {} the {} a {} my beautiful large {} garden.\textsc{gen}\\ 
\z

\noindent This proposal amounts to saying that the switching entities need not necessarily occur in the DP layer as the analysis by \citet{Olsen88Agreement-und-Flexion, Olsen89AGReement-in}, \citet{Bhatt90Die-syntaktische} and \citet{Delsing88The-Scandinavian, Delsing90A-DP-analysis} suggests.

Not all possessive pronouns agree with the nominal head of the container noun phrase. There are frozen genitival pronouns with the case feature $+$poss. German \textit{wessen} ‘whose' (\textsc{int}.\textsc{poss}.\textsc{gen}), \textit{dessen} ‘his'/‘its' (\textsc{dem}.\textsc{poss}.\textsc{gen}.\textsc{sg}.\textsc{m/n})  / \textit{deren} ‘her'/‘their' (\textsc{dem}.\textsc{poss}.\textsc{gen}.\textsc{sg}.\textsc{f}/\textsc{pl}.\textsc{m/f/n}), Russian \textit{ego} `his'/
`its', \textit{ee} `her', \textit{ix} `their' belong to this class of lexical entries. They occur in SpecN, as do all other $+$poss-marked constituents.

\subsection{} \label{sec:zi91:5.3}

Let us now turn to the question where the various subject expressions are at home in DS and how the prenominal and postnominal phrases are interrelated. I assume that the subject of noun phrases in DS is a sister of N or of N$'$ depending on its function as argument of N or as modifier. This is shown in \REF{ex:zi91:58}.\footnote{The comma in \REF{ex:zi91:58} indicates that the argument a and the modifier b here are considered neutral as to their position on the right or the left of their co-constituent.}

\ea \label{ex:zi91:58} $[$\textsubscript{DP} [\textsubscript{D$'$} D [\textsubscript{NP} [\textsubscript{N$'$} [\textsubscript{N$'$} N, a], b]]]]
\z

\noindent Compare \REF{ex:zi91:17} and \REF{ex:zi91:28}, where the rightmost trace $t_{i}$ indicates the pertinent subject position in DS.\footnote{Observe that I regard the Hungarian possessor phrases as arguments of the head noun. The derivational possessive suffix converts absolute nouns into relational ones.} $+$poss-marked phrases cannot stay in one of the positions in \REF{ex:zi91:58}. They move to SpecN to be licensed by D under government and in adjacency to D. The trace left behind is theta-governed for an argument and antecedent-governed for the modifier. In view of the fact that arguments and modifiers can go to SpecN this position cannot be an $A$-position. It is an $\overline{A}$-position, as is SpecD. SpecD can be occupied by prepositional phrases as in \REF{ex:zi91:1b}. In this case, there will be two traces of the subject phrase moved to SpecD. One will be in the DS position a or b in \REF{ex:zi91:58}, the other will be in adjunct position to NP. This trace neutralizes the barrierhood of NP, as is the case with VP when constituents of VP leave this domain (see \citealt{Chomsky86Barriers, Chomsky89Some-notes}).\footnote{The structure given in \REF{ex:zi91:17a} deviates from \posscitet{Szabolosi87Functional-categories} analysis. The trace of the dative possessor should be in adjunct position to DP and not in SpecD. A further trace must be assumed in adjunct position to NP in my analysis of the Hungarian noun phrases with possessors.} Thus the phrase in SpecD antecedent-governs the trace in the adjunct position to NP, and this trace antecedent-governs the trace in the DS position of the moved constituent. All this is shown in \REF{ex:zi91:59}.

\ea \label{ex:zi91:59} 
    \ea \label{ex:zi91:59a} $[$\textsubscript{DP} [\textsubscript{D$'$} D [\textsubscript{NP} $\left\{ \begin{array}{c} \text{a}_{i} \\ \text{b}_{i} \end{array} \right\}$ [\textsubscript{N$'$} [\textsubscript{N$'$} N, $t_i$], $t_i$]]]]
    \ex \label{ex:zi91:59b} $[$\textsubscript{DP} $\left\{ \begin{array}{c} \text{a}_{i} \\ \text{b}_{i} \end{array} \right\}$ [\textsubscript{D$'$} D [\textsubscript{NP} $t_i$ [\textsubscript{NP} [\textsubscript{N$'$} [\textsubscript{N$'$} N, $t_i$], $t_i$]]]]]
    \z
\z

\noindent a$_i$ represents a moved argument of N, b$_i$ represents a modifier. The structures with b$_i$ are the SS representations of the examples in \REF{ex:zi91:1}--\REF{ex:zi91:3}, as concerns the prenominal subjects in interrelation with their DS position.

These assumptions are valid for corresponding English and Russian noun phra\-ses with subject expressions, too. I illustrate this by \REF{ex:zi91:60} and \REF{ex:zi91:61}.

\ea \label{ex:zi91:60}
    \ea \label{ex:zi91:60a} $[$\textsubscript{DP} [\textsubscript{D$'$} the [\textsubscript{NP} [\textsubscript{N$'$} [\textsubscript{N$'$} horses] [\textsubscript{PP} of [\textsubscript{DP} the queen]]]]]]
    \ex \label{ex:zi91:60b} 
    $[$\textsubscript{DP} [\textsubscript{D$'$} $\emptyset$ [\textsubscript{NP} $\left\{ \begin{array}{l} [_{\text{DP}_i} \; \text{the queen's}] \\\relax [_{\text{DP}_{i}} \; \text{her}] \end{array} \right\}$ [\textsubscript{N$'$} [\textsubscript{N$'$} horses] $t_i$]]]]
    \z
\ex \label{ex:zi91:61}
    \ea \label{ex:zi91:61a} $[_{\text{DP}} \; [\textsubscript{D$'$} \; \begin{array}{c} \emptyset \\ \text{(the)} \end{array} \; [_{NP} \; [_{N'} \; \begin{array}{c} \text{druz'ja} \\ \text{friends} \end{array} [_{\text{DP}} \; \begin{array}{c} \text{Petra} \\ \text{Peter.\textsc{gen}} \end{array}]]]]]$
    \ex \label{ex:zi91:61b} $[_{\text{DP}} \; [\textsubscript{D$'$} \; \begin{array}{c} \emptyset \\ \text{(the)} \end{array} \; [_{NP} \; [_{\text{DP}_{i}} \; \begin{array}{c} \text{ego} \\ \text{his} \end{array}] [\textsubscript{N$'$} \; \begin{array}{c} \text{druz'ja} \\ \text{friends} \end{array} \; t_{i}]]]]$
    \z
\z

\noindent According to my analysis, there is no subject expression in SpecD in English or in Russian. In these languages, prenominal subjects marked by the feature +poss figure in SpecN and are licensed by D bearing the licensing feature <+poss>. As mentioned above, German genitive phrases optionally can bear the case feature +poss and with this marking must go to Spec. English genitive phrases are always $+$poss-marked and therefore figure in SpecN, in SS.\footnote{Possibly, it is not quite correct to assume genitive phrases for English. The case system of Modern English is reduced to the features $\pm$governed (to differentiate \textit{he} vs. \textit{him}) and $\langle +$poss$\rangle$ (to differentiate \textit{he}, \textit{him} vs. \textit{his} and \textit{the queen} vs. \textit{the queen$'$s}). Another way of thinking would be to consider English \textit{of}-phrases as $-$poss-marked genitive phrases and the phrases affixed by \textit{'s} as $+$poss-marked genitive phrases.} In contrast to German and English, Russian genitive phrases are not marked by $+$poss. In accordance with the universal \REF{ex:zi91:15}, they remain in their postnominal position, as in \REF{ex:zi91:62a}. Only the genitival nonreflexive possessive pronouns of the third person as in \REF{ex:zi91:61b} and \REF{ex:zi91:62b} go to SpecN, like the possessive pronouns of the first and second person as in \REF{ex:zi91:62}. \REF{ex:zi91:62d} is colloquial Russian and does not obey principle \REF{ex:zi91:15}.

\ea \label{ex:zi91:62}
    \ea \label{ex:zi91:62a}
        \gll vse {\.e}ti pjatdesjat dve cennye knigi moego druga \\
        all these fifty two valuable books my.\textsc{gen} friend.\textsc{gen} \\
        \glt
    \ex \label{ex:zi91:62b}
        \gll vse {\.e}ti ego pjatdesjat dve cennye knigi \\
        all these his fifty two valuable books \\
        \glt 
    \ex \label{ex:zi91:62c}
        \gll vse {\.e}ti moi pjatdesjat dve cennye knigi \\
        all these my fifty two valuable books \\
        \glt 
    \ex \label{ex:zi91:62d}
        \gll vse {\.e}ti moego druga pjatdesjat dve cennye knigi \\
        all these my.\textsc{gen} friend.\textsc{gen} fifty two valuable books \\
        \glt
    \z
\z

\noindent The exceptional genitival possessive pronouns are \textit{ego} `his'/`its', \textit{eë} `her', \textit{ix} `their'. I assume that they constitute DPs which bear the feature +poss, like the possessive pronouns of the first and second person \textit{moj-} `my', \textit{tvoj-} `your', \textit{naš-} `our', \textit{vaš-} `your' and the reflexive possessive pronoun \textit{svoj-}, which all agree with the nominal head. The bifunctional relative vs. interrogative pronoun \textit{čij} `whose' also belongs to the group of agreeing +poss-marked subjects, whereas the genitival relative pronoun \textit{kotor+ogo/oj/yx} `of whom'/`of which' does not receive $+$poss and remains in its postnominal DS position. Nothing need be added to our system of assumptions. All that is necessary for a speaker of Russian is to know the peculiarities of the pertinent lexical items including affixes. So, he/she must know that the above nonreflexive possessive pronouns of the third person have a special feature, $+$poss, which forces them to appear in SpecN. On the other hand, he/she must know that the homophonous nonreflexive personal pronouns of the third person do not have the feature $+$poss. They stay at home, in postnominal position. In contact with prepositions a bridging \textit{n} will be inserted, which is absent in the case of the corresponding possessive pronouns. Compare:

\ea \label{ex:zi91:63}
    \ea \label{ex:zi91:63a} 
        \gll dlja nego \\
        for him.\textsc{gen} \\
        \glt
    \ex \label{ex:zi91:63b}
        \gll dlja ego brata \\ 
        for his brother.\textsc{gen} \\
        \glt
    \z
\z

\noindent Now, one characteristic of the movement of the subject phrase to prenominal positions, SpecN or SpecD, deserves special consideration. In accordance with principle \REF{ex:zi91:13}, only the most prominent argument of N can be realized externally to N$'$. Therefore, the DPs in \REF{ex:zi91:64} are correct, while those in \REF{ex:zi91:65} are not.

\ea \label{ex:zi91:64}
    \ea \label{ex:zi91:64a} 
    \gll Peters / seine Untersuchung\\
    Peter's {} his examination\\
    
    \ex \label{ex:zi91:64b} 
    \gll von Peter / ihm die Untersuchung\\
    of Peter {} him the examination\\
    
    \ex \label{ex:zi91:64c} 
    \gll Peters / seine Untersuchung des Vorfalls\\
        Peter's {} his examination the.\textsc{gen} incident.\textsc{gen}\\
        
    \ex \label{ex:zi91:64d} 
    \gll Peters / seine Untersuchung durch den Chefarzt\\
    Peter's {} his examination by the chief-physician\\
    \z
\ex \label{ex:zi91:65} 
    \gll \minsp{*} des Vorfalls Untersuchung Peters / von Peter\\
    {} the.\textsc{gen} incident.\textsc{gen} examination Peter's {} of Peter\\
\z

\noindent The most prominent argument of a lexical category X\textsuperscript{0} is the argument correlated with the nonreferential argument slot $\widehat{x}_{i} \; (i \leq 2)$ in the argument structure of X\textsuperscript{0} which has the most narrow scope over the SF of X\textsuperscript{0}. This is indicated in \REF{ex:zi91:66}.

\ea \label{ex:zi91:66} $\widehat{x}_{n} \; ... \; \widehat{x}_{2} \; \widehat{x}_{1} \; [...]$ \\ where $\widehat{x}_{1}$ is the referential argument slot of verbs and nouns and the external argument slot of adjectives and prepositions, respectively.
\z

\noindent Thus, in \REF{ex:zi91:64}, the most prominent argument of the deverbal noun, which is correlated with $\widehat{x}_{2}$ in the argument structure of \textit{Untersuchung} `examination', is realized externally to N$'$. In \REF{ex:zi91:65}, a less prominent argument, correlated with $\widehat{x}_{3}$ in the argument structure of the lexical head, is realized externally to N$'$. Therefore, \REF{ex:zi91:65} does not obey the principle \REF{ex:zi91:13} and is deviant.

\citet{Bhatt89Parallels-in, Bhatt90Die-syntaktische, Bhatt90Kasuszuweisung-in} treats deviant cases as in \REF{ex:zi91:65} on the basis of her assumption that there is a special kind of empty category, $e_{i}$ instead of $t_{i}$, in the DS position of the prenominal argument, which prevents the postnominal designated argument in SpecN from getting case.\footnote{For the characteristics of verbs with a designated argument see \citet{Haider87Deutsche-Syntax}.} In \REF{ex:zi91:67}, I confront Bhatt's analysis \REF{ex:zi91:67a} of \REF{ex:zi91:65} with that of mine \REF{ex:zi91:67b}.

\ea \label{ex:zi91:67}
    \ea \label{ex:zi91:67a} $[$\textsubscript{DP} [\textsubscript{DP$_i$} des Vorfalls] [\textsubscript{D$'$} $\emptyset$ [\textsubscript{NP} [\textsubscript{N$'$} Untersuchung $e_i$] [\textsubscript{DP} Peters]]]]
    \ex \label{ex:zi91:67b} $[$\textsubscript{DP} [\textsubscript{D$'$} $\emptyset$ [\textsubscript{NP} [\textsubscript{DP$_i$} des Vorfalls] [\textsubscript{N$'$} Untersuchung [\textsubscript{DP} Peters] $t_i$]]]]
    \z
\z

\noindent Without considering the analyses by \citet{Williams81Argument-structure}, \citet{di1987definition}, \citet{Bierwisch89Event-nominalizations:} and others concerning the internalization of the external argument of verbs and adjectives in nominalizations, Bhatt assumes that nominalizations have designated or external and internal arguments as do their derivational bases and that the designated argument in DS is in SpecX (X = V, N), as in \REF{ex:zi91:67a}. Now, the designated argument in \REF{ex:zi91:67a} cannot get structural case because case assignment in German requires adjacency of the case assigner and the case receiver which is not the case in \REF{ex:zi91:67a}. $e_{i}$ intervenes between \textit{Untersuchung} and \textit{Peters}. Therefore, the case filter \REF{ex:zi91:12} is not obeyed and the DP in \REF{ex:zi91:65} is ungrammatical. Evidently, Bhatt's system of assumptions works. But it has its essential base in the postulation of $e_{i}$ in contrast to $t_{i}$. And this needs independent legitimization. It seems that it cannot be given.

My proposal follows the analyses of nominalizations given by \citet{Williams81Argument-structure}, \citet{di1987definition}, \citet{Bierwisch89Event-nominalizations:} and resides in the assumption that all arguments of nouns have their DS position in N$'$, as sisters of N.\footnote{See also \citet{Huste89Zur-Syntax, Huste89Zur-Topologie}, \citet{Bischof91Sachverhaltsbezeichnungen-des} and \citet{Freytag90Die-syntaktische, Freytag91Sachverhaltsbezeichnungen-des}.} Furthermore, my analysis exploits principle \REF{ex:zi91:13}, which is well motivated independently, in explaining why DPs as in \REF{ex:zi91:65} are deviant.

\subsection{} \label{sec:zi91:5.4}

Finally, I will consider \posscitet{Grimshaw88Adjuncts-and} 
conception of argument-adjuncts, where again prominence relations between arguments play a decisive part. \citet{Grimshaw88Adjuncts-and} assumes that the argument slot for the most prominent argument of verbs is blocked by passivization and by nominalization.\footnote{\posscitet{Grimshaw88Adjuncts-and}  assumptions also apply to DPs with a deadjectival noun as lexical head.} This argument can be integrated into the structure of sentences or noun phrases as a modifier. By its being correlated to and licensed by the corresponding argument position in the SF of the pertinent lexical head it is conceived of as argument-adjunct.

In my system of assumptions, this would amount to the following. In DS, the most prominent argument of a noun is not a sister of N, but occurs in modifier position, as a sister of N$'$ (see \REF{ex:zi91:58}). From this position it can go to SpecN or to SpecD. Semantically, argument-adjuncts would be interpreted like modifiers and only secondarily be correlated to an argument position (an unbound variable) in the SF of the head noun. As a consequence of this, the possession markers and the possessive pronouns and adjectives always would deliver the modifier meaning, as in the representations \REF{ex:zi91:30a} and \REF{ex:zi91:33}. A special rule of semantic interpretation would have to be added, though, in order to correlate the modifier meaning of the argument-adjunct to the corresponding unspecified argument position of the lexical head. I propose \REF{ex:zi91:69}, which will apply to the SF of \textit{il mio amico} ‘my friend’ (lit. ‘the my friend’) and of the corresponding expressions in German, English, Russian and other languages as is shown in \REF{ex:zi91:70}. The semantics of the head noun is indicated in \REF{ex:zi91:68}. It makes clear that there is no argument slot for specifying the argument position $y$.

\ea \label{ex:zi91:68} \textit{amico}: $\widehat{x} \; [x \; \textsc{friend} \; y]$
\ex \label{ex:zi91:69} $[... \; x \; ... \; y \; ...]:[x \; R \; a] \; \equiv \; [... \; x \; ... \; a \; ...]$ \\
condition: The constant $a$ is compatible with the semantic role of $y$.
\ex \label{ex:zi91:70} $\widehat{P_{3}} \; [\textsc{def} \; x \; [[x \; \textsc{friend} \; y]:[x \; R \; \textsc{speaker}]] \wedge [P_{3} \; x]] \equiv \widehat{P_{3}} \; [\textsc{def} \; x \; [x \; \textsc{friend} \; \textsc{speaker}]\wedge[P_{3} \; x]]$
\z

\noindent By \REF{ex:zi91:69}, the unspecified argument position $y$ in the SF \REF{ex:zi91:68} of \textit{amico} ‘friend’ is substituted by the constant \textsc{speaker} of the modifier meaning $[x \; R \; \textsc{speaker}]$ with the anonymous relation $R$, which becomes superfluous by this substitution. Thus, the semantic parameters $y$ and $R$ in \REF{ex:zi91:70} become specified by \textsc{speaker} and \textsc{friend}, respectively.

In connection with this interpretation of \citeauthor{Grimshaw88Adjuncts-and}'s argument-adjunct, it seems tempting and, possibly, more adequate to have the subject expressions of noun phrases in SpecN already in DS. In accordance with the universal \REF{ex:zi91:15}, possessive pronouns agreeing with the head noun would be placed as left daughters of NP whereas genitival and prepositional subject phrases would be right daughters of NP. Furthermore, $+$poss-marked phrases would have to be left daughters of NP to become licensed by D, in adjacency to it. Thus, the functional characterization of “subject” -- as given in \REF{ex:zi91:4} -- would be valid in DS and to a great extent also in SS. Only German prepositional subject phrases can move to SpecD. Semantically, all subject expressions of noun phrases would be interpreted as modifiers. Those of them which correspond to an argument position in the SF of the head noun would be argument-adjuncts and subject to rule \REF{ex:zi91:69}. The only precondition of this very general solution would be the blocking of the argument slot for the most prominent argument of relational nouns as \textit{Freund} `friend' or \textit{Untersuchung} `examination'.

It seems reasonable to reckon with the subject in SpecN. SpecX in general provides a position for one constituent XP. And as in sentences, there is only one subject (possessor) in noun phrases. As to scope relations, the subject expressions in SpecN would c-command all modifiers in N$'$ -- including QP (see \REF{ex:zi91:11}) -- and the complements of N. At least for subjects in modifier function, this seems desirable. Thus, I regard the subject (possessor) expressions in \REF{ex:zi91:71} and \REF{ex:zi91:72} as the most prominent modifiers c-commanding all others.

\ea \label{ex:zi91:71}
    \ea \label{ex:zi91:71a} 
    \gll seine zwei neuen Hüte aus Paris\\
    his two new hats from Paris\\
    \ex \label{ex:zi91:71b} 
    \gll die zwei neuen Hüte aus Paris von ihm\\
    the two new hats from Paris of him\\
    \z
\ex \label{ex:zi91:72}
    \ea \label{ex:zi91:72a} 
    \gll Peters zwei neue Hüte aus Paris\\
    Peter's two new hats from Paris\\
    
    \ex \label{ex:zi91:72b} 
    \gll die zwei neuen Hüte $t_{i}$ Peters [\textsubscript{PP$_i$} aus Paris]\\
   the two new hats {} Peter's {} from Paris\\
    \z
\z

\noindent In view of the fact that German postnominal genitive phrases which are structurally case-marked must be adjacent to their case assigner N in SS intervening prepositional phrases must move rightward, as in \REF{ex:zi91:72b}. They become adjuncts of NP. The same is true of complements of N, as in \REF{ex:zi91:73}.

\ea \label{ex:zi91:73}
    \ea \label{ex:zi91:73a} 
     \gll die finanzielle Abhängigkeit $t_{i}$ Peters [\textsubscript{PP$_i$} von den Eltern]\\
    the financial dependence {} Peter's {} of the parents\\
    \glt ‘Peter's financial dependence on his parents’
    \ex \label{ex:zi91:73b} 
    \gll la dépendence $t_{i}$ financière de Pierre [\textsubscript{XP$_i$} des parents]\\
    the dependence {} financial of Pierre {} of.the parents\\
    \glt ‘Pierre's financial dependence on his parents’
    \z
\z

\noindent As the French example in \REF{ex:zi91:73b} shows, the adjacency condition for structural case assignment needs parametrization as to possible intervening adjectival modifiers.\footnote{For Russian, which does not require adjacency of N and the postnominal genitival subject, see \citet{Huste89Zur-Syntax, Huste89Zur-Topologie} and \citealt{Bischof91Sachverhaltsbezeichnungen-des}.} In any case, \REF{ex:zi91:71}--\REF{ex:zi91:73} make clear that it is not unconceivable to assume that the subject of noun phrases has its place in SpecN and that PPs intervening between it and the nominal head are extraposed. This would be a generalization of \citeauthor{Bhatt89Parallels-in}'s (\citeyear{Bhatt89Parallels-in}, \citeyear{Bhatt90Die-syntaktische}, \citeyear{Bhatt90Kasuszuweisung-in}) analysis of noun phrases with the designated argument of N being placed in SpecN to the subject of noun phrases, irrespective of its being correlated with the most prominent unspecified argument position of N or its functioning as a modifier.\footnote{A comparison of \posscitet{Bhatt89Parallels-in,Bhatt90Die-syntaktische,Bhatt90Kasuszuweisung-in} analysis of noun phrases and of mine is given in \citet{Zimmermann91Die-Syntax-Substantivgruppe}.} So, instead of \REF{ex:zi91:11}, the DS of noun phrases would be \REF{ex:zi91:74}, where a indicates the possible subject positions.

\ea \label{ex:zi91:74} $[$\textsubscript{DP} [\textsubscript{D$'$} D [[\textsubscript{NP} a [\textsubscript{N$'$} ... (QP) ... N ...]\;a]]]
\z

\noindent It goes without saying that this proposal residing in \citeauthor{Grimshaw88Adjuncts-and}'s (\citeyear{Grimshaw88Adjuncts-and}) conception of argument-adjuncts needs further empirical support and theoretical research, especially as to the regularities of case assignment (or: case checking) and to their interplay with other principles of linearization (see \citealt{Huste89Zur-Syntax, Huste89Zur-Topologie}, \citealt{Bischof91Sachverhaltsbezeichnungen-des}  and \citealt{Freytag90Die-syntaktische}, \citealt{Freytag91Sachverhaltsbezeichnungen-des}).

%section 6
\section{} \label{sec:zi91:6}

I will sum up. The generalization of $\overline{X}$-syntax to functional categories as D and C gives new insights into the structure of sentences and noun phrases and into the syntactic and semantic parallelism of these referring syntagmas. C and D specify the reference type of sentences and noun phrases, respectively, and as projecting categories bring in the specifier positions SpecC and SpecD.

As to the various subject expressions, I have tried to give an account of their semantics and of all syntactic positions they typically occupy in the structure of DPs. This account takes into consideration the facts of different languages and is open to refinements, empirically and theoretically. My analysis confirms the insight that D and the subject of noun phrases are not in complementary distribution. D serves the specification of the reference type of the pertinent DP while subject phrases -- DPs, PPs, APs -- function as arguments of relational nouns or as modifiers. Accordingly, the theta-role of the subject has its basis in the semantics of the lexical head of the noun phrase or in the SF of the possession marker, which possessive pronouns and the Slavonic possessive adjectives have incorporated in their semantics.

According to my analysis, the SpecD position in noun phrases can be occupied by prepositional subject expressions but not by a $+$poss-marked subject. Phrases in SpecD, like phrases in SpecC, the prefield position (= \textit{Vorfeld}), are not assigned a theta-role by D. Functional categories by definition are void of descriptive content and consequently unable to assign a theta-role.


\largerpage
In view of the linear word order totality pronoun -- determiner subject expression -- numerals and other quantifying expressions -- adjectival modifiers~-- head noun in many languages, I have assumed a second prenominal position for subject expressions, besides the one in SpecD, namely in SpecN, adjacent to D. All phrases capable of occurring in this position are marked by the case feature $+$poss. They are licensed in SpecN by D entities with the licensing feature $\langle +$poss$\rangle$, under government. In contrast to many analyses, I assume that D governs rightwards.

Besides possessive pronouns and genitival or prepositional subject phrases, Slavonic languages have possessive adjectives derived from nouns by suffixes. They occur prenominally, after the numerals. I assume that they are adjuncts to N$'$.

As to postnominal positions of the subject, I have discussed two possibilities with different semantic and syntactic consequences. The first proposal differentiates syntactically and semantically between two positions: subjects with argument function are right sisters of N, subjects with modifier function are right adjuncts to N$'$, in DS. German PPs can move to SpecD, $+$poss-marked phrases must move to SpecN to get licensed. The second proposal envisages Spec as the characteristic position for prenominal or postnominal subject expressions. Again, PPs can move to SpecD, in German. $+$poss-marked constituents occupy the SpecN position adjacent to D, this time without being moved. The second solution is a simpler one with respect to semantics, too. Subject phrases in SpecN or with a trace in SpecN are interpreted as modifiers and treated as argument-adjuncts in case they are correlated to an unspecified argument position in the SF of the head noun.

Inner- and interlinguistic variation in the realization of subject expressions in noun phrases is determined by parameter setting as to general principles of syntactic structure and to lexical properties. It has been shown that the case filter \REF{ex:zi91:12}, the prominence requirement \REF{ex:zi91:13} for the external realization of an argument, the direction of government \REF{ex:zi91:14}, the preference location of genitive phrases and of phrases agreeing with the nominal head \REF{ex:zi91:15} and last but not least the syntactic categorization of subject expressions as DP, PP or AP and their being marked by the case feature $\langle +$poss$\rangle$ constitute a system of interrelated conditions determining the specific realization of the subject in noun phrases.

Further research must answer the questions why Standard Russian ideally obeys the implicational universal \REF{ex:zi91:15}, but Albanian does not, which phrases in which languages can occupy SpecD and in which cases noun phrases with a subject can vary with respect to their reference type. I have given some warning against the preconception that noun phrases with prenominal subject expressions and a silent determiner are always definite.

\section*{Abbreviations}
\begin{tabularx}{.5\textwidth}{@{}lQ}
\textsc{1}&first person\\
\textsc{2}&second person\\
\textsc{3}&third person\\
\textsc{acc}&accusative\\
\textsc{am}&attributive marker\\
\textsc{dat}&dative\\
\textsc{dem}&demonstrative\\
\textsc{f}&feminine\\
\textsc{gen}&genitive\\
\end{tabularx}%
\begin{tabularx}{.5\textwidth}{lQ@{}}
\textsc{int}&interrogative\\
\textsc{m}&masculine\\
\textsc{n}&neuter\\
\textsc{nom}&nominative\\
\textsc{pl}&plural\\
\textsc{poss}&possessive\\
\textsc{poss.ref}&possessive reflexive\\
\textsc{sg}&singular\\
&\\ % this dummy row achieves correct vertical alignment of both tables
\end{tabularx}

\section*{Editors' note}

 Glosses not occurring in the original text were added by the guest editors.

\printbibliography[heading=subbibliography,notkeyword=this]

\end{document}
