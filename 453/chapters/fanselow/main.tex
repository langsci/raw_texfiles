\documentclass[output=paper,colorlinks,citecolor=brown]{langscibook}
\ChapterDOI{10.5281/zenodo.15471449}
\author{Gisbert Fanselow\affiliation{University of Potsdam}\orcid{}}
\title{Remarks on the distribution of wh-scope marking}
\abstract{The present chapter is concerned with ``wh-scope marking'' -- a phenomenon that occurs in complex structures. The wh-phrase, instead of moving overtly to its scope position, remains within the complement clause. A wh-pronoun that shows up in the matrix clause takes over the function of signalling scope. The goal of my contribution is to complete and systematize empirical facts as well as probe ways to explain the cross-linguistic distribution of the construction. The typological distinction between SVO and SOV languages might play a role. On the other hand, the linguistic facts that are presented in this chapter suggest that an areal factor plays a crucial role. These considerations delineate a direction of research that appears worthwhile pursuing.

\keywords{wh-scope marking, wh-movement, wh-phrase, OV vs. VO, typology}
}



\begin{document}
\begin{otherlanguage}{english}
\shorttitlerunninghead{Remarks on the distribution of wh-scope marking}
\maketitle

% %%%%%%%%%%%%%%%%%%%%%%%%%%%%%%%%%%%%%%%%%%%%%%%%%%%%%%%%%%%%%%%
% %%%%%%%%%%%%%%%%%%%%%%%%%%%%%%%%%%%%%%%%%%%%%%%%%%%%%%%%%%%%%%%

\section{Introduction}

German is among the languages that offer various ways of asking for a constituent of an embedded clause, as illustrated in \REF{ex:fanselow:1a}--\REF{ex:fanselow:1d}.

\ea\label{ex:fanselow:1}
\ea     Long movement: \newline
\gll  Wen denkst du, dass sie angerufen hat?  \\ 
     who.\textsc{acc} think.\textsc{2sg} you that she call.\textsc{ptcp} have.\textsc{3sg} \\  
\label{ex:fanselow:1a}
\ex Extraction from V2-complement: \newline
\gll  Wen denkst du hat sie angerufen? \\
    who.\textsc{acc} think.\textsc{2sg} you have.\textsc{3sg} she call.\textsc{ptcp} \\   
\label{ex:fanselow:1b}
\ex Wh-doubling: \newline
\gll  Wen denkst du wen sie angerufen hat? \\
    who.\textsc{acc} think.\textsc{2sg} you who.\textsc{acc} she call.\textsc{ptcp} have.\textsc{3sg}   \\   
\label{ex:fanselow:1c}
\ex Wh-scope marking: \newline
\gll  Was denkst du wen sie angerufen hat? \\
    what think.\textsc{2sg} you who.\textsc{acc} she call.\textsc{ptcp} have.\textsc{3sg}   \\  \label{ex:fanselow:1d}
\z
\glt ‘Who do you think that she has called?' \hfill (German)
\z

\noindent Wh-phrases can move to their scope position out of clauses with \REF{ex:fanselow:1a} and without \REF{ex:fanselow:1b} a complementizer. In a ‘wh-doubling' construction, one or more copies of the wh-phrase in the scope position appear in lower positions  \REF{ex:fanselow:1c} as well.\footnote{The case with more copies is shown in the following German example sentence:
\ea 
\gll  Wen denkst du, wen Maria meint, wen Paul behauptet, wen der Chef angerufen hat?  \\
    who.\textsc{acc} think.\textsc{2sg} you who.\textsc{acc} Maria mean.\textsc{3sg} who.\textsc{acc} Paul claim.\textsc{3sg} who.\textsc{acc} the boss call.\textsc{ptcp} have.\textsc{3sg} \\
    \glt    ‘Who do you think Mary believes that Paul claims the boss has called?' \hfill (German)\label{ex:fanselow:1fn}
    \z
}  In a further construction \REF{ex:fanselow:1d}, a neuter wh-pronoun can be inserted into the position corresponding to the scope of the ‘real' wh-phrase, which has moved to the left periphery of the embedded clause.

Other languages show further alternatives to standard long wh-movement. In Russian, long wh-movement out of finite complement clauses is often dispreferred. Instead, one can observe “partial” movement to the edge of the complement CP as in \REF{ex:fanselow:2}, cf. e.g., \citet{Gelderen2001} -- the wh-phrase has been displaced, yet not reached its scope position.

\ea 
\gll  Ty	dumaeš’	kogo	ja		videla?  \\
    you think.\textsc{2sg} who.\textsc{acc} I see.\textit{l}-\textsc{ptcp}.\textsc{sg}.\textsc{f}  \\
    \glt    ‘Who do you think that I saw?' \hfill (Russian)
    \label{ex:fanselow:2}
    \z

\noindent Partial movement also applies in constructions such as \REF{ex:fanselow:3} from Wolof \citep{Torrence2013}, in which the complement CP having the wh-phrase at its left periphery itself undergoes movement to the SpecCP position of the matrix clause, i.e., to the position corresponding to scope of the real wh-phrase. The construction is called “clausal pied piping”.

\ea 
\gll  Lan		l-a-ñu	jënd	l-a		Bintë	foog?  \\
    what \textsc{expl}-\textsc{cop}-\textsc{3pl} buy \textsc{expl}-\textsc{cop} Binta think  \\
    \glt    ‘What does Binta think that they bought?' \hfill (Wolof)
    \label{ex:fanselow:3}
    \z

\noindent Wh-phrases in embedded clauses with matrix scope may also simply remain \textit{in situ} (as in Mandarin Chinese), or the wh-phrase at the left periphery of the matrix clause may be linked to a resumptive pronoun. The options (possibly) involving partial movement of the wh-phrase have been discussed and compared in \citet{Fanselow2017a}.  

The focus of the present chapter is on the fourth construction \REF{ex:fanselow:1d}, in which the “real” wh-phrase \textit{wen} is placed into the left periphery of the complement clause, while its “semantic scope” is the matrix clause, and is marked there by placing \textit{was}, the neuter wh-pronoun, at the left periphery of the matrix clause. The construction \REF{ex:fanselow:1d} is sometimes called “partial movement”, based on the insight that the “real” wh-phrase has only moved to the left periphery of the embedded clause in \REF{ex:fanselow:1d} and not to its scope position, but the term would be a misnomer for a Hindi counterpart such as \REF{ex:fanselow:4}, in which \textit{kis-ko} `who' appears to occupy its base position in the embedded clause. Furthermore, movement of the wh-phrase to a position between the original and the scope position occurs in \REF{ex:fanselow:2} and \REF{ex:fanselow:3} as well.

\ea 
\gll  Siitaa-ne 	kyaa 	socaa 	ki 	ravii-ne 	kis-ko 	dekhaa?  \\
    Sita-\textsc{erg} what thought that Ravi-\textsc{erg} who saw  \\
    \glt    ‘Who did Sita think that Ravi saw?' \hfill (Hindi)
    \label{ex:fanselow:4}
    \z

\noindent Using the label “wh-scope marking” for \REF{ex:fanselow:1d} and \REF{ex:fanselow:4} (a term introduced by \citealt{Riemsdijk1983}) seems more appropriate, because \textit{was} `what' and \textit{kyaa} `what' occupy positions in which they can mark the scope of the “real” wh-phrase. There are two notable aspects of the construction. First, the “real” wh-phrase is part of an embedded clause that could function as an indirect question as well (the “correlate” of \textit{was}/\textit{kyaa} has the form of an embedded question), and scope marking is done by a wh-phrase itself.\footnote{This would require an appropriate matrix predicate, though. Consider the following German example:

\ea 
\gll  Du fragst / \minsp{*} denkst, wen sie angerufen hat.  \\
    you ask.\textsc{2sg} {} {} think.\textsc{2sg} who.\textsc{acc} she call.\textsc{ptcp} have.\textsc{3sg}  \\
    \glt    ‘You ask who she has called.' \hfill (German)
    \z

} Normally, this wh-phrase is the neuter wh-pronoun (as in German and Hindi), but other languages such as Polish and Warlpiri may employ other wh-pronouns (see below), if these can be used in questions demanding an answer with propositional content. These criteria distinguish wh-scope marking construction as in \REF{ex:fanselow:1d} and \REF{ex:fanselow:4} from constructions in which a particle marks the scope of an \textit{in-situ} wh-phrase. A more precise label for \REF{ex:fanselow:1d} and \REF{ex:fanselow:4} would thus be “wh-scope marking by a further wh-phrase”, but we will stick to the shorter version in this chapter because it has been used in much of the literature.

Several quite different syntactic and semantic analyses for wh-scope marking have been proposed in the literature, many of which are represented in \citet{Lutz-Muller-etal2000}, cf. also \citet{Fanselow2017a} for a recent overview. Here, we do not wish to take up again the question of what is the best analysis of \REF{ex:fanselow:1d}. Rather, the chapter will be concerned with an issue that has not really been addressed so far: what is the cross-linguistic distribution of \REF{ex:fanselow:1d}, and are there ways of understanding this distribution? Our discussion suggests that wh-scope marking constructions primarily occur in languages in which long wh-movement is difficult for various reasons (possibly related to the SVO/SOV distinction), but there also seems to be a strong areal factor: wh-scope marking is attested in Central and Eastern Europe and in Indo-Aryan languages, but appears to be quite rare (or even non-existent) in other parts of the world, e.g. in the area of the Niger Congo language family. For reasons discussed below, our conclusions are somewhat tentative where they go beyond Europe and South Asia -- so our chapter describes a research agenda more than presenting indisputable results.

In what follows, I examine the presence of wh-scope marking from a typological perspective by elaborating on data from the following languages: Germanic (\sectref{section:germanic}), Basque, Celtic, Romance, Latin, Albanian, Modern Greek (\sectref{section:romance+}), Slavic, Baltic, Romani and Uralic (\sectref{section:slavic+}), languages of Asia (\sectref{languages:of:asia}), Mauritius Creole, Louisiana Creole, Warlpiri, Niger-Congo languages, Nilo-Saharan languages, Khoisan langua\-ges, and languages of the Americas (\sectref{further:languages}). Finally, in \sectref{concluding:remarks} I summarize the main findings and address some open questions.

\section{Germanic} \label{section:germanic}

The literature on wh-scope marking has a focus on German, Hindi, and Hungarian. The point of departure of our survey are the Germanic languages.

The West-Germanic languages do not behave in a uniform way with respect to wh-scope marking. Dutch, Frisian, German allow the construction (e.g., \citealt{Hiemstra1986}), and so does Afrikaans \REF{ex:fanselow:5}, while English does not.\footnote{For the list of linguists and native speakers who helped me with object language data, see the end of this chapter.
} The division with respect to wh-doubling follows the same lines. 

\ea \label{ex:fanselow:5}
\gll  Wat 	het 	jy 	nou 	weer 	gesê 	wat 	het 	Sarie 	gedog 	met 	wie 	gaan 	Jan 	trou?  \\
    what 	have 	you 	now 	again 	said	what 	has 	Sarie 	thought 	with 	who 	will 	Jan 	marry \\
    \glt    ‘Who have you said again that Sarie thought who Jan will marry?' \\\hfill (Afrikaans)
    \z

\noindent Based on \posscitet{Haider2012} insights concerning the clustering of grammatical properties in the Germanic language family, one wonders whether the usual suspect could be made responsible for the difference concerning wh-scope marking as well: the West-Germanic languages with wh-scope marking are OV languages, those without come with VO order. 

Verb complement order could influence the formation of long distance dependencies in various ways. Note that complement clauses typically appear to the right of the verb even in the Germanic OV languages, i.e., they undergo extraposition. If extraposed clauses are (not too strong) islands for wh-movement, standard long movement as in \REF{ex:fanselow:1a} would therefore be less acceptable in the OV  than in the VO languages. This could create the need for alternative strategies, such as wh-scope marking, or wh-doubling (alternatively, questioning a complement clause constituent could be blocked altogether, as in Iron Ossetic, as David Erschler kindly pointed out to me).\footnote{That right-peripheral complement clauses are not placed on the “correct” side of the verb in OV languages has first been utilized in an account of wh-scope marking by \citet{Mahajan1990}.
}

Yiddish appears to not allow wh-scope marking either, in line with its VO character.

\ea[*]{
\gll  Vos 	meynstu 		mit 	vemen 	hot 	er 	gered?  \\
     what	{think you}		with	who		has	he	spoken  \\
    \glt    Intended: ‘Who do you think he has spoken to?' \hfill (Yiddish)}
    \label{ex:fanselow:6}
    \z

\noindent Of course, an interpretation of the absence of wh-scope marking in Yiddish may be difficult because of its close ties with Polish, a language also often taken to not allow wh-scope marking, i.e., the absence of wh-scope marking in Yiddish might be considered to be a result of language contact. Indeed, when one zooms into varieties of German, one can see a plausible impact of language contact on wh-scope marking. In their analysis of Hunsrück German as spoken in the USA, \citet{Hopp-Putnam-etal2019} found that wh-scope marking does not appear grammatical to its speakers, while the doubling construction exemplified in \REF{ex:fanselow:1c} is accepted. The absence of \REF{ex:fanselow:1d} is likely a consequence of contact with English (though one wonders why \REF{ex:fanselow:1c} has not disappeared either, and why counterparts to \REF{ex:fanselow:1d} seem to be grammatical in Brazilian Hunsrück German, in spite of their contact with Portuguese). Mocheno (or Fersentalerisch), a variety of German spoken in Italy also seems to possess wh-scope marking, yet the construction comes with an ironic undertone there.

Of course, languages can lose a construction in contact with others, only if they had possessed the construction at some time before the contact. For German, wh-scope marking seems to be a relatively young construction.\footnote{I am grateful to Svetlana Petrova for helping me with the historical perspective.
} While standard long wh-movement has been possible already in Old High German (750--1050), wh-scope marking seems to be attested first in \textit{Prose Lancelot} of the 13th century (cf. \REF{ex:fanselow:7}; \citealt[81]{Axel-Tober2012}), and this example even allows an alternative interpretation as an instance of the wh-doubling construction \REF{ex:fanselow:7}, in which a wh-phrase is realized at more than one location in the clause -- a construction occurring quite frequently at that time.

\ea \label{ex:fanselow:7}
\gll  Sagent	mir 	was 	irselb 	wollent 	was 	ich 	herumb 	thun  \\
    tell	me	what	you	want	what	I	therefore	do \\
    \glt    ‘Tell me what you want me to do therefore.' \hfill (Old High German)
    \z

\noindent  We therefore would not have any evidence for the claim that the construction was widespread in German any time before the 13th century, and even for that time, the evidence is ambiguous. According to \citet{Reis2000b}, one cannot identify a point of time before the 17th century for which one could safely say that wh-scope marking was established at that point. In other words, it is far from obvious that West Germanic allowed wh-scope marking at a time before Yiddish got into close contact with Slavic, and could have lost it in this context. 

Present-day North-Germanic languages disallow wh-scope marking -- in line with their status as SVO languages. However, \citet{Hakanson2004} pointed out that \citet{Falk-Torp1900} gave examples for wh-scope marking for Old Icelandic/Old Norwegian. The verb-complement order for these stages of Old Scandinavian cannot be identified unambiguously, because OV and VO serializations co-occur. \citet{Sigurdsson1988} and \citet{Haugen2000} argue for a VO analysis with the change from OV to VO still being in progress. In that respect, the presence of a wh-scope marking construction in some Old Scandinavian languages could be considered a remnant from an OV period.


\section{Basque, Celtic, Romance, Latin, Albanian, Modern Greek} \label{section:romance+}

Turning to the languages of Western and Southern Europe, we find no real evidence for wh-scope marking. The construction \REF{ex:fanselow:8} is forbidden in Basque, which employs clausal pied-piping of the kind exemplified in \REF{ex:fanselow:3} for Wolof, and normal long distance wh-movement instead.

\ea[*]{
\gll Zer	 uste 	duzu 	\minsp{[} nor 	maite 	du-ela 	Klarak]? \\
     what 	think  \textsc{aux}.\textsc{2sg} {} who 	love \textsc{aux}-\textsc{comp} Klara.\textsc{erg}  \\
    \glt    Intended: ‘Who do you think Klara loves?' \hfill (Basque)}
    \label{ex:fanselow:8}
    \z

\noindent Given that Basque is usually analyzed as an SOV language, Basque shows that OV order does not force wh-scope marking into existence. With clausal pied-piping, we observe at least another non-standard question formation process in Basque in an OV context. We also fail to find wh-scope marking in Celtic Breton.

The Romance languages all come with SVO order, and lack wh-scope marking. This also holds for Romanian, a language in contact with languages possessing wh-scope marking constructions, cf. \REF{ex:fanselow:9}:

\ea[*]{
\gll Ce 	crezi 	tu 	{pe 	cine} 	iubește 	el?  \\
     what 	think 	you  	who.\textsc{acc} 	loves 	he   \\
    \glt    Intended: ‘Who do you think he loves?' \hfill (Romanian)}
    \label{ex:fanselow:9}
    \z

\noindent No variety of Ladin has wh-scope marking in spite of the close contact with German -- perhaps because they all have SVO order.


Surprisingly, we find wh-scope marking in Romance child language (not unlike what was observed for English). E.g., \citet{Jakubowicz-Strik2008} report that sentences such as \REF{ex:fanselow:10}  are produced by French children in an experimental situation.

\ea[*]{
\gll{Qu’est-ce que} 	Billy 	a 	dit 	qui 	boit 	{de l’eau}? \\
    what	Billy	has	said	who	drinks	water   \\
    \glt    ‘Who has Bill said drinks water?' \hfill (French)}
    \label{ex:fanselow:10}
    \z

\noindent The occurrence of \REF{ex:fanselow:10} suggests that wh-scope marking is not strictly incompatible with SVO order (as is already evident from Old Icelandic/Norwegian), but that it is merely a dispreferred option. The dispreference may be due to the fact that right-peripheral complement CPs can occupy a base position in VO languages, so that extraction out of them is not impeded by the extraposed status a complement clause has in an SOV language. There is thus simply no need for an alternative way of asking for complement clause constituents. However, when there are other impediments for long movement, such as processing-related ones in child language, wh-scope marking may be employed in SVO languages, too.

In his discussion of the semantics of wh-scope marking, \citet{Staudacher2000} points to the existence of constructions such as \REF{ex:fanselow:11} in Latin, which could be candidates for wh-scope marking in this language.

\ea \label{ex:fanselow:11}
\gll  Quid 	enim 	censemus 	superiorem 	illum 	Dionysium 	quo 	cruciatu timoris 	angi 		solitum? \\
    what 	indeed 	think.\textsc{1pl} 	older 	that 	Dionysius  which.\textsc{abl} 	torture.\textsc{abl} 	
	fear.\textsc{gen} 	{cause\_distress.\textsc{pass}.\textsc{prs}.\textsc{inf}} {used to.\textsc{acc}.\textsc{m}.\textsc{sg}}
  \\
    \glt    ‘By which torture of fear do we think the older Dionysius used to be choked?' (Cicero, \textit{De Officiis} 2,25) \hfill (Latin)
    \z


\noindent Given that Latin is usually assumed to have SOV base order, the presence of a wh-scope marking construction would not be too unexpected. However, Latin is a language with a rich set of examples for long distance wh-movement even leaving islands, so that the functional motivation of a wh-scope marking construction appears to be non-existent. A closer look at \REF{ex:fanselow:11} (and the sentence following it) casts some doubt on an analysis of \REF{ex:fanselow:11} as a wh-scope marking construction in a narrow sense, however.\footnote{I am indebted to Peter Staudacher for explaining this to me, and to Lieven Danckaert for discussing \REF{ex:fanselow:11} with me; cf. also \textcite[157]{Danckaert2012}.} Recall that a crucial property of wh-scope marking lies in the fact that the clause linked to the scope marking wh-phrase must be a syntactically legal indirect question. \REF{ex:fanselow:11} does not conform to this characteristic of wh-scope marking: indirect questions must appear with subjunctive inflection in Latin, there are no indirect wh-questions formed with an infinitive complement as we have it in \REF{ex:fanselow:11}. Thus, while \REF{ex:fanselow:11} is a very interesting construction by itself, it is not clear if it constitutes a clear example of wh-scope marking in a narrow sense.

Albanian and Modern Greek lack wh-scope marking as well. 

\section{Slavic, Baltic, Romani and Uralic} \label{section:slavic+}

While there is no wh-scope marking to the West and South of the West Germanic language area in Europe, the situation is quite different in the East. The presence of wh-scope marking contrasts with the marginal role it plays in the syntactic analysis of languages from Eastern/Central Europe.\footnote{\citet[Section 3.3]{Meyer2004} is an exception. In his monograph, he discusses wh-scope constructions with \textit{jak} ‘how’ in Polish resp. \textit{kak} ‘how’ in Russian and a construction in which wh-phrases take long scope. These, however, differ from wh-scope marking in that they do not have an overt wh-pronoun in the matrix clause.
}

Sorbian is the only Slavic language with SOV basic order, and allows wh-scope marking.\footnote{Upper Sorbian examples are used here for illustration.
} Crucially, wh-scope marking can be readily constructed with third person subjects \REF{ex:fanselow:12a}, and can be embedded \REF{ex:fanselow:12b}, so that its status as an integrated syntactic object is beyond any doubt. Nevertheless, some speakers prefer a verb second version of \REF{ex:fanselow:12a}, as in \REF{ex:fanselow:12c}.

\ea\label{ex:fanselow:12}
\ea     
\gll    Što 	měnjachu 	starši 	tehdom, 	koho 	wučer 	(po)chłostać 	chcyše? \\
        what	think.3\textsc{pl}.\textsc{pst}	 parents	then	who.\textsc{acc}	teacher	punish.\textsc{inf}	want.3\textsc{sg}.\textsc{pst}  	  \\
\glt ‘Who did the parents think then the teacher wanted to punish?' 
\label{ex:fanselow:12a}
\ex     
\gll Wón 	mi 	nochcyše 	rjec, 	što	sej 	ty 		mysliš, što 	\minsp{(} my) 	za 	swjedźeń hišće		trjebamy. \\
    he 	me.\textsc{dat} 	\textsc{neg}:want.3\textsc{sg}.\textsc{pst} 	say.\textsc{inf} 	what 	\textsc{refl} 	you 	think.2\textsc{sg} 	what 	{} we 	for 	celebration still 		need.1\textsc{pl} 	  \\
\glt ‘He did not want to tell me what you think we still need for the celebration.'
\label{ex:fanselow:12b}
\ex     
\gll Što 	starši 	tehdom 	měnjachu, 	koho 	chcyše 	wučer 	(po)chłostać? \\
    what 	parents 	then 	think.3\textsc{pl}.\textsc{pst} 	who.\textsc{acc}	 	want.\textsc{sg}.\textsc{pst} 	teacher 	punish.\textsc{inf}  	  \\
\glt ‘Who did the parents think then the teacher wanted to punish?'\\\hfill (Upper Sorbian)
\label{ex:fanselow:12c}
\z\z


\noindent Silesian (\textit{ślōnskŏ gŏdka}) is a further Slavic variety which was in close contact with German. The data from three informants, which Jolanta Tambor kindly collected for me, suggest that wh-scope marking is an option in this language:\footnote{The Silesian informants used different orthography and different lexemes in translating our target sentences. We refrain from attempting “standardization” here.
} 

\ea\label{ex:fanselow:15}
\ea     
\gll Prezydynt 	sce 	wiedzieć, 	co 	medykuje 	Hana, 	kiej 	jutro 	bydzie loło. \\
    president 	want.3\textsc{sg} 	know.\textsc{inf} 	what 	think.3\textsc{sg} 	Hana 	when 	tomorrow	will.3\textsc{sg} rain.\textit{l}-\textsc{ptcp}.\textsc{sg}.\textsc{n}  \\ 
\glt ‘The president wants to know when Hana thinks that it will rain tomorrow.'
\label{ex:fanselow:15a}
\ex     
\gll Niy 	łobchodzi 	mie, 	co 	ty 	medykujesz, 	kiej 	bydzie 	loło. \\
    \textsc{neg} 	matter.3\textsc{sg} me.\textsc{acc} what you think.2\textsc{sg} when 	will.3\textsc{sg} rain.\textit{l}-\textsc{ptcp}.\textsc{sg}.\textsc{n}  \\ 
\glt ‘I do not care when you think it will rain.'
\label{ex:fanselow:15b}
\ex     
\gll A 	co 	ty 	forsztelujesz, 	kedy 	bydzie 	loło? \\
    and 	what 	you	think.2\textsc{sg} 	when will.3\textsc{sg} rain.\textit{l}-\textsc{ptcp}.\textsc{sg}.\textsc{n}   \\ 
\glt ‘And when do you think will it rain?'\hfill (Silesian)
\label{ex:fanselow:15c}
\z\z


\noindent The discussion of Polish and Russian in the literature on wh-scope marking has focussed on a slightly different construction, in which scope marking is expressed by the wh-pronoun \textit{jak} ‘how' and not by \textit{co} ‘what', as illustrated in \REF{ex:fanselow:16a} for Polish. \citet{Lubanska2004} has argued convincingly that the construction in \REF{ex:fanselow:16a} does not exemplify wh-scope marking, however. First, the construction is not really acceptable with third person subjects \REF{ex:fanselow:16b}, and it cannot be embedded \REF{ex:fanselow:16c}.\footnote{I should like to point out that judgments vary among speakers. For some, example \REF{ex:fanselow:16b} seems to be out but improves if the matrix predicate is inflected for present tense. Example \REF{ex:fanselow:16c}, on the other hand, is judged by some as completely fine. This does not speak against the analysis of the \textit{jak}-phrase as an integrated parenthetical construction, though.
} Therefore, \REF{ex:fanselow:16a} does not involve a subordinate clause linked to \textit{jak}, but it is rather an integrated parenthetical construction. Cf. \citet{Korotkova2012} for related conclusions concerning Russian.


\ea\label{ex:fanselow:16}
\ea     
\gll Jak	myślisz,	kiedy 	ona	przyjdzie? \\
    how think.2\textsc{sg} when she come.3\textsc{sg} \\
\glt ‘When do you think she will come?'
\label{ex:fanselow:16a}
\ex[?/*]{     
\gll Jak	Janek	myślał,	kogo	Maria	kocha? \\
     how Janek think.\textit{l}-\textsc{ptcp}.\textsc{sg}.\textsc{m} who.\textsc{acc} Maria love.3\textsc{sg} \\
\glt Intended: ‘Who did Janek think Maria loves?' }
\label{ex:fanselow:16b}
\ex[*]{     
\gll Zastanawiam	 	się,	jak	myślisz,		kogo	Janek	kocha.	\\
    wonder.1\textsc{sg} \textsc{refl} how think.2\textsc{sg} who.\textsc{acc} Janek love.3\textsc{sg} \\
\glt Intended: ‘I wonder who you think Janek loves.'\hfill (Polish)}
\label{ex:fanselow:16c}
\z\z

\noindent Czech allows structures similar to \REF{ex:fanselow:16a} with \textit{co} ‘what' \REF{ex:fanselow:19a},\footnote{However, \citet[195]{Meyer2004}, reporting his corpus search, points out that this type of structure is rare (“not productive”) in Czech.
} but again, the construction cannot be embedded \REF{ex:fanselow:19b}, which suggests it needs to be analyzed as an integrated parenthetical as well, rather than as wh-scope marking.

\ea\label{ex:fanselow:19}
\ea     
\gll Co 	si 	           myslíš, 	koho 	učitel 	potrestá? \\
    what 	\textsc{refl} 	think.2\textsc{sg} 	who.\textsc{acc} 	teacher	punish.3\textsc{sg} \\
\glt ‘What do you think, who will the teacher punish?'
\label{ex:fanselow:19a}
\ex[*]{      
\gll Je 	mi 	jedno, 	                              co 	si 	myslíš, 	koho 	učitel 	potrestá. \\
     be.3\textsc{sg} 	me.\textsc{dat} 	one 	what 	\textsc{refl} 	think.2\textsc{sg} 	who.\textsc{acc} 	teacher 	punish.3\textsc{sg} \\
\glt Intended: ‘It doesn’t matter to me who you think the teacher will punish.' \hfill (Czech) }
\label{ex:fanselow:19b}
\z\z

\noindent Such data suggest that Czech, Polish, and Russian lack wh-scope marking altogether (perhaps as a consequence of their status as SVO languages).\footnote{\citet{Smiecinska2011}, however, points to the existence of yet another construction in colloquial Polish. This will be neglected here.
} This may suggest that the status of a structure involving a wh-clause with a wh-correlate in the main clause is a function of the type of the wh-correlate. A hypothesis could be: If it is the neuter wh-pronoun, the constellation can be interpreted as wh-scope marking, but if it is the interrogative word \textit{how}, it is a parenthetical construction. \citet{Fanselow2017a} also discusses such a connection. The postulation of such a correlation, however, already appears problematic within the Slavic languages, because in Silesian the construction with \textit{jak} also shows the typical wh-scope marking properties such as embeddability \REF{ex:fanselow:21}. Below, we will observe the same general pattern for Udmurt.

\ea \label{ex:fanselow:21}
\gll  Niy! 	Prezydynt 	chce 		wiedzieć, 	jak 	myśli 	Joanna, 	kedy 	jutro 	bydzie 	padać. \\
    no 	president 	want.3\textsc{sg} 	know.\textsc{inf} 	how 	think.3\textsc{sg} 	Joanna 	when 	tomorrow 	will.3\textsc{sg} 	rain.\textsc{inf} \\
    \glt    ‘No! The president wants to know when Joanna thinks it will rain tomorrow.' \hfill (Silesian)
    \z

\noindent To sum up, at least Silesian and Sorbian are languages with wh-scope marking, quite independently of the order of the verb and the object (SVO in Silesian, SOV in Sorbian). Above, we discussed whether verb-complement order predicts the availability of wh-scope marking, and Silesian does not conform to this prediction. However, if we want to uphold such a connection in the light of the West Slavic evidence, we could follow \citet{Haider2012} in working with a tripartite categorization of verb-object order (OV, VO, and undetermined) and classify the Slavic languages as undetermined with respect to object verb order. The “ban” against wh-scope marking would then have to be restricted to languages with strict VO order (English and Swedish, but not Polish and Russian). However, it is less obvious why such a correlation should hold, because extraposition of a right-peripheral complement clause would neither be necessary in VO languages nor in those with undetermined verb complement order. It is thus unclear why complement clauses should be more island-like in languages with undetermined verb complement order.

The South Slavic languages (all showing SVO/undetermined verb complement order) show a clearer picture with respect to wh-scope marking -- the construction is possible in the whole subfamily. Thus, the pertinent constructions can be embedded in Slovenian \REF{ex:fanselow:22}, in Serbian \REF{ex:fanselow:23}, and in Macedonian \REF{ex:fanselow:24}, and it occurs in Bulgarian \REF{ex:fanselow:25}.

\ea \label{ex:fanselow:22}
\gll  Učitelj 	je 	hotel 	vedeti, 	kaj 	so 	starši 	verjeli, 	koga 	da 	je Peter 	videl.\\
     teacher 	\textsc{aux} 	want.\textit{l}-\textsc{ptcp}.\textsc{sg}.\textsc{m} 	know.\textsc{inf} 	what 	\textsc{aux} 	parents 	believe.\textit{l}-\textsc{ptcp}.\textsc{pl} 	who.\textsc{acc} 	that	\textsc{aux}  Peter 	see.\textit{l}-\textsc{ptcp}.\textsc{sg}.\textsc{m} \\
    \glt    ‘The teacher wanted to know who the parents believed that Peter saw.' \\\hfill (Slovenian)
    \z

\ea\label{ex:fanselow:23}
\ea     
\gll Šta 	misliš, 	ko 	je 	stigao?  \\
     what 	think.2\textsc{sg} 	who 	\textsc{aux} 	arrive.\textit{l}-\textsc{ptcp}.\textsc{sg}.\textsc{m} \\
\glt ‘Who do you think arrived?'
\label{ex:fanselow:23a}
\ex     
\gll Šta 	kaže 	nastavnik, 	ko 	je 	došao? \\
    what 	say.3\textsc{sg} 	teacher 	who 	\textsc{aux} 	come.\textit{l}-\textsc{ptcp}.\textsc{sg}.\textsc{m} \\
\glt ‘Who did the teacher say came?'
\label{ex:fanselow:23b}
\ex     
\gll Pitam 	se, 	šta 	misli, 	ko 	je 	došao. \\
ask.1\textsc{sg} 	\textsc{refl} 	what 	think.3\textsc{sg} 	who 	\textsc{aux} 	come.\textit{l}-\textsc{ptcp}.\textsc{sg}.\textsc{m} \\
\glt ‘I wonder who he thinks came.'\hfill (Serbian)
\z\z

\ea\label{ex:fanselow:24}
\ea     
\gll Što 	misliš, 		koj 	\minsp{\{} stigna / stignal\}? \\
what 	think.2\textsc{sg}  	who 	{} arrived.\textsc{aor} {} arrive.\textit{l}-\textsc{ptcp}.\textsc{sg}.\textsc{m}  \\
\glt ‘Who do you think arrived?'
\label{ex:fanselow:24a}
\ex     
\gll  Što 	veli 	nastavnikot, 	koj 	došol? \\
what 	say.3\textsc{sg} 	teacher 	who 	come.\textit{l}-\textsc{ptcp}.\textsc{sg}.\textsc{m} \\
\glt ‘What does the teacher say who came?'
\label{ex:fanselow:24b}
\ex     
\gll Se 	prašuvam, 	što 	misli, 	koj 	došol. \\
\textsc{refl}	ask.1\textsc{sg} 	what 	think.3\textsc{sg} who 	come.\textit{l}-\textsc{ptcp}.\textsc{sg}.\textsc{m} \\
\glt ‘I wonder who he thinks came.'\hfill (Macedonian)\label{ex:fanselow:24c}
\z\z

\ea \label{ex:fanselow:25}
\gll  \minsp{\{} Kak / kakvo\}  	misli  	Ivan, 	na 	kogo 	se  	obažda 	Stojan? \\
    {} how {} what 	think.3\textsc{sg} 	Ivan	on 	whom 	\textsc{refl} 	call.3\textsc{sg}	Stojan \\
    \glt    ‘Who does Ivan think that Stojan is calling?' \hfill (Bulgarian)
    \z

\noindent Wh-scope marking turns out to be a well-attested construction in the Slavic languages, then. In addition, quite a number of the further genetically less related languages spoken in the “Slavic language area” possess the construction as well. The SVO language Romani (Central/North Western Indo-Aryan) was one of the first languages for which wh-scope marking was analyzed in great detail in the seminal work of \citet{McDaniel1989}. \citet{Horvath1997} is an in-depth study of the construction in Hungarian.

The two Baltic languages have wh-scope marking as well. \REF{ex:fanselow:26a} exemplifies the construction in the form of an embedded complement clause in Lithuanian. The construction has no counterpart with long distance wh-movement here. However, long distance wh-movement is not generally excluded in Lithuanian \REF{ex:fanselow:26c}. Embedded wh-scope marking in Latvian is illustrated in \REF{ex:fanselow:28}.


\largerpage[2]
\ea\label{ex:fanselow:26}
\ea
\gll Man	įdom-u, 	kaip  	tu  	manai  	kas  	išėjo. \\
    me.\textsc{dat} interesting-\textsc{nom} what 	you think who.\textsc{nom} leave.3\textsc{sg}.\textsc{pst} \\
\glt ‘I wonder what you think who left.'
\label{ex:fanselow:26a}
\ex[*]{
\gll  Man  	įdom-u, 	kas  	tu  	manai,  	kad 	išėjo.  \\
    me.\textsc{dat} interesting-\textsc{nom} who.\textsc{nom} you think that leave.3\textsc{sg}.\textsc{pst} \\
\glt Intended: ‘I wonder who you think left.'}
\label{ex:fanselow:26b}
\ex
\gll       Kas  	tu  	manai,  	kad 	išėjo? \\
        who.\textsc{nom} you.\textsc{nom}  	think 	that 	leave.3\textsc{sg}.\textsc{pst} \\
\glt ‘Who do you think left?'\hfill (Lithuanian)
\label{ex:fanselow:26c}
\z

\ex\label{ex:fanselow:28}
\gll  Prezidents 	bija 	aizmirsis, 	ko 	ministrs 	domā, 	kādā 		veidā 	varētu  	
	cilvēkus 	nomierināt. \\
    president have.3\textsc{sg}.\textsc{pst} forgotten what minister believe.3\textsc{sg} {in which} way could people {be calmed down} \\
    \glt    ‘The president had forgotten in which way the minister believes people could be 	calmed down.' \hfill (Latvian)
\z\clearpage

\noindent Hungarian is not the only Uralic language with wh-scope marking. As \citet{Allkivi2018} points out, there is wh-scope marking in Estonian, while, again, long distance wh-movement is not freely permitted. Nele Ots kindly provided me with \REF{ex:fanselow:29} as an example of an embedded wh-scope marking construction in Estonian. Finnish, however, appears to lack such sentences, perhaps because it shows SVO rather than SOV basic order.

\ea \label{ex:fanselow:29}
\gll  Meid 	ei 	huvita, 	mis 	te 	arvate, 	kes 		seda 	tegi. \\
    us \textsc{neg} matter.3\textsc{sg} what you think who it do.3\textsc{sg}.\textsc{pst} \\
    \glt    ‘We do not care who you think did it.' \hfill (Estonian)
    \z

\noindent Udmurt exemplifies wh-scope marking with third person subjects both for \textit{kyzhy} ‘how' and \textit{mar} ‘what' as scope markers, as shown in \REF{ex:fanselow:30}, the latter “sounding a bit more Russian”. Long wh-movement is bad \REF{ex:fanselow:31} and wh-phrases sitting in complement clauses cannot take matrix scope by themselves \REF{ex:fanselow:32}. However, the wh-scope marking construction does not require that the wh-phrase be moved to the left periphery in the complement clause \REF{ex:fanselow:33a}. Still, the scope marking construction is unavailable when the embedded clause is typed as \{$-$wh\} by a complementizer \REF{ex:fanselow:33b}. Both \textit{mar} ‘what' and \textit{kyzhy} ‘how' allow the embedding of the scope marking construction (\ref{ex:fanselow:34}, \ref{ex:fanselow:35}).  Udmurt therefore corroborates the conclusion reached above for Silesian that the nature of the wh-pronoun is uncorrelated with the status of a constellation as wh-scope marking construction.

\ea\label{ex:fanselow:30}
\ea     
\gll  Kyzhy 	Italmas 	malpa  	kinjos 		gondyrez 	vijozy? \\
     how Italmas think.3\textsc{sg} who.\textsc{pl} bear.\textsc{acc} kill.3\textsc{pl} \\
\glt ‘Who all does Italmas think will kill the bear?'
\label{ex:fanselow:30a}
\ex     
\gll  Mar  	Italmas 	malpa  	kinjos 		gondyrez 	vijozy?  \\
      what Italmas think.3\textsc{sg} who.\textsc{pl} bear.\textsc{acc} kill.3\textsc{pl}   \\
\glt ‘Who all does Italmas think will kill the bear?'\hfill (Udmurt)
\label{ex:fanselow:30b}
\z\z

\ea[*]{
\gll Kinjos  		Italmas 	malpa  	gondyrez 	vijozy?\\
    who.\textsc{pl} Italmas think.3\textsc{sg} bear.\textsc{acc} kill.3\textsc{pl} \\
    \glt  Intended: ‘Who all does Italmas think will kill the bear?' \hfill (Udmurt)}
    \label{ex:fanselow:31}
    \z

 \ea[*]{  
\gll  Italmas 	malpa  	kinjos	 	gondyrez 	vijozy?\\
     Italmas think.3\textsc{sg} who.\textsc{pl} bear.\textsc{acc} kill.3\textsc{pl} \\
    \glt  Intended: ‘Who all does Italmas think that will kill the bear?' \hfill (Udmurt)}
    \label{ex:fanselow:32}
    \z   

\ea\label{ex:fanselow:33}
\ea     
\gll   Mar  	Italmas 	malpa  	gondyrez 	kinjos 		vijozy? \\
     what Italmas think.3\textsc{sg} bear.\textsc{acc} who.\textsc{pl} kill.3\textsc{pl} \\
\glt ‘Who all does Italmas think will kill the bear?'
\label{ex:fanselow:33a}
\ex[*]{     
\gll Mar  	Italmas 	malpa  	kinjos 	gondyrez 	vijozy 		schuysa? \\
        what Italmas think.3\textsc{sg} who.\textsc{pl}  bear.\textsc{acc} kill.3\textsc{pl} that  \\
\glt Intended: ‘Who all does Italmas think will kill the bear?'\hfill (Udmurt)}
\label{ex:fanselow:33b}
\z\z

\ea\label{ex:fanselow:34}
\ea     
\gll  Tyneshtyd 	juashko 	mar  	Italmas 	malpa  	kine 	pejschurasjos 	vijozy. \\
      you.\textsc{abl} ask.1\textsc{sg} what Italmas think.3\textsc{sg} who.\textsc{acc} hunters kill.3\textsc{pl}   \\
\glt ‘I ask you, who Italmas thinks that the hunters will kill.'
\label{ex:fanselow:34a}
\ex     
\gll    Tyneshtyd 	juashko 	kyzhy  	Italmas 	malpa  	schuysa   	kine 	pejschurasjos 	vijozy. \\
         you.\textsc{abl} ask.1\textsc{sg} how Italmas think.3\textsc{sg} that who.\textsc{acc} hunters kill.3\textsc{pl}  \\
\glt ‘I ask you, who Italmas thinks that the hunters will kill.'\hfill (Udmurt)
\label{ex:fanselow:34b}
\z\z

\ea \label{ex:fanselow:35}
\gll  Tchatchabej 	pajme 	\minsp{\{} mar / kyzhy\}  	Italmas 	malpa	kine 		pejschurasjos   vijozy. \\
    Chachabej wonder.3\textsc{sg} {} what {} how Italmas think.3\textsc{sg} who.\textsc{acc} hunters kill.3\textsc{pl}   \\
    \glt    ‘Chachabej wonders, who Italmas thinks that the hunters will kill.' \\\hfill (Udmurt)
    \z

\noindent Moving to the South, we find wh-scope marking in Georgian, too (\citealt[150ff.]{Borise2019}).

\ea \label{ex:fanselow:36}
\gll  Ra 	tkv-a 	Nino-m, 	 \minsp{(} rom) 	vi-s 	unda 	v-u-q’ur-o-t? \\
    what.\textsc{nom} say-\textsc{aor}.3\textsc{sg} Nino-\textsc{erg} {} \textsc{comp} who-\textsc{dat} \textsc{mod} 1-\textsc{ver}-watch-\textsc{opt}.1-\textsc{pl}   \\
    \glt    ‘Whom did Nino say that we must watch?' \hfill (Georgian)
    \z

\noindent When we look at Europe, we thus observe a clear and (given the previous focus of the discussion in the literature on German and Hungarian) partially surprising areal pattern for the distribution of wh-scope marking: the construction occurs in Central and Eastern Europe, mainly in the broader Slavic linguistic area (there, it is also found in the non-Slavic languages) and in the West Germanic OV languages. In contrast, the construction is absent in Western, Northern and Southern Europe.

There is also a second, more structural, factor that comes into play when one considers the wh-scope marking area: the languages that lack the construction (Finnish, Yiddish, Czech) have normal SVO  rather than SOV order, while it is present in all SOV languages we have evidence for (German, Dutch, Frisian, Sorbian, Estonian, Hungarian -- if it is really SOV --, Udmurt and Georgian). 

\section{Languages of Asia } \label{languages:of:asia}

Obviously, the question arises whether the presence of the wh-scope marking construction follows the two trends (areal concentration and verb-complement order) in other parts of the world as well. Our evidence for non-European languages is much more sparse, yet we will try to identify some patterns on the basis of whatever little data we have. 

It will not surprise anyone familiar with the literature on wh-scope marking that the construction can be found in Indo-Aryan languages -- but of course, as mentioned above, we are confronted here with structures in which the embedded wh-phrase is not moved to the left edge of the sentence, as shown in Hindi \REF{ex:fanselow:4}, repeated here as \REF{ex:fanselow:37} for convenience. However, the complement clause is still a legal embedded question, hence the classification of \REF{ex:fanselow:40} as a wh-scope marking construction conforms with our criteria.

\ea 
\gll  Siitaa-ne 	kyaa 	socaa 	ki 	ravii-ne 	kis-ko 	dekhaa?  \\
    Sita-\textsc{erg} what thought that Ravi-\textsc{erg} who saw  \\
    \glt    ‘Who did Sita think that Ravi saw?'\hfill (Hindi)
    \label{ex:fanselow:37}
    \z

\noindent Kashmiri only marginally allows long distance wh-movement, and “com\-pen\-sates” for this difficulty by employing wh-scope marking and wh-doubling, as shown in \REF{ex:fanselow:38}--\REF{ex:fanselow:39} taken from \citet{Antnonenko2010}, but cf. \citet{Wali-Koul1997}.

\ea 
\gll  Tse 	KYAA	chu-y 	baasaan 	me 	chi 	soochaan 	raj-an 	kyaa 	dyut 	Mohn-as?  \\
    you 	what 	\textsc{aux} 	believe 	I 	\textsc{aux} 	think 	Raj-\textsc{erg} 	what 	gave 	Mohan-\textsc{dat} \\
    \glt    ‘What do you believe I think Raj gave to Mohan?'\hfill (Kashmiri)
    \label{ex:fanselow:38}
    \z

    \ea 
\gll  Tse 	kemis 	chu-y 	baasan 	me 	chu 	soochaan 	Raaj 	kemis 	dihey 	kitaab?  \\
    you 	who 	\textsc{aux} 	believe 	I 	\textsc{aux} 	think 		Raj 	who 	give 	book \\
    \glt    ‘Whom do you believe I think Raj gave a book to?'\hfill (Kashmiri)
    \label{ex:fanselow:39}
    \z

\noindent A construction similar to \REF{ex:fanselow:37}--\REF{ex:fanselow:38} can also be found in Bangla \REF{ex:fanselow:40} (taken from \citealt{Bayer1996}), but the language also takes recourse to clausal pied-piping as a means of coping with the difficulties of scope taking of wh-phrases in OV languages with right-peripheral complement clauses (\citealt{Simpson-Bhattacharya2003}).

\ea 
\gll  Tumi 	ki 	bhebe-cho 	ke 	baaRi 	kore-che?  \\
        you	what	think		who	house	built \\
    \glt    ‘Who do you think built a house?' \hfill (Bangla)\label{ex:fanselow:40}
    \z

\noindent On the other hand, not all Indo-Aryan languages tolerate wh-scope marking: Marathi lacks wh-scope marking and utilizes clausal pied-piping instead (\citealt[247]{Dhongde-Wali2009}), and the construction is also absent in Assamese.

The Indo-Iranian languages show a similar mixed picture. Persian disallows constructions such as \REF{ex:fanselow:41}. A wh-phrase in an embedded clause can take matrix scope without the use of any scope marking device. In contrast, as already remarked above, there is no way of questioning elements of a lower clause in Ossetic (David Erschler, p.c.). However, \citet{Karimi-Taleghani2007} report wh-scope marking for Dari, a version of Farsi spoken in Afghanistan.

\ea[*]{ 
\gll  Chi 	fekr 	mi-kon-i 	ke 	Ali 	ki-ro 		doost-dare?  \\
     what 	thought 	\textsc{dur}-do-2\textsc{sg} 	that 	Ali 	who-\textsc{acc} 	like-has.3\textsc{sg} \\
    \glt    Intended: ‘Who do you think that Ali liked?' \hfill (Persian)}
    \label{ex:fanselow:41}
    \z

\noindent While Kurdish must mark the scope of wh-phrases \textit{in situ}, full wh-pronouns can never function as such elements (\citealt{Hamid2019}). Furthermore, the wh-scope marking is missing in Turkish. In other words, the European wh-scope marking area does not appear to be strongly connected to the Indo-Aryan one. Likewise, the Indo-Aryan area does not extend to the South of India, since Dravidian languages such as Malayalam employ clausal pied-piping (\citealt{Aravind2018}) rather than wh-scope marking. Telugu uses prosody as a means of scope marking for wh-phrases in embedded clauses (\citealt{Giblin-Steddy2014}). If the languages we have considered are representative, the Indo-Aryan language area constitutes a second “wh-scope marking hotspot”. The languages all show OV base order, so that no new insights concerning the role played by head-complement order can be gained. 

Mandarin Chinese, Vietnamese and Khmer do not have wh-scope marking, while the construction is possible in Japanese (\citealt{Fujiwara2021}), with the complement CP fronted and the wh-pronoun occupying the object position of the matrix clause. The construction co-exists with other means of indicating the scope of a wh-phrase sitting in a complement clause. Korean appears to lack a similar construction:

\ea \label{ex:fanselow:42}
\ea[??]{
\gll   John-un  	\minsp{[} nwu-ka	senke-eyse 	iki-l-ci]	mwe-la-ko 	malha-ess-ni? \\
      John-\textsc{top} 	{} who-\textsc{nom} 	election-at  	win-\textsc{fut}-\textsc{q}  	what-be-\textsc{comp} 	say-\textsc{pst}-\textsc{q}\\
    \glt    Intended: ‘Who did John say would win the election?' }
    \label{ex:fanselow:42a}
    \ex[??]{ 
\gll   \minsp{[} Nwu-ka  	senke-eyse  	iki-l-ci]  	ettehkey 	sayngkakha-ni? \\
     {} who-\textsc{nom} 	election-at   	win-\textsc{fut}-\textsc{q}  	how  	  	think-\textsc{q}\\
    \glt    Intended: ‘Who do you think will win the election?' \hfill (Korean)}
    \label{ex:fanselow:42b}
    \z \z

\noindent Wh-phrases can undergo partial movement of the sort exemplified in \REF{ex:fanselow:2} in Malay (the wh-phrase moves to the specifier position of the embedded clause but takes matrix scope), a construction for which \citet{Cole-Hermon1998} postulate an “emp\-ty” wh-pronoun sitting in the matrix clause, so that the abstract analysis they propose is identical to the one for wh-scope marking constructions. Obviously, the Malay constellation does NOT meet our criteria for wh-scope marking (the abstract element sitting in the matrix clause is not a wh-pronoun) and can therefore not be considered a further “hit” in our search for the construction.

Syrian Arabic conflicts with our expectations in two ways (cf. \citealt{Sulaiman2016}); wh-scope marking as in \REF{ex:fanselow:43} (\citealt[141]{Sulaiman2016}) co-occurs with unproblematic long distance wh-movement in a VSO/SVO language. Wh-scope marking can also be identified in Iraqi Arabic (\citealt{Wahba1992}).

\ea 
\gll  Ma tʔul-i-l-u	šw	ʔal-et-l-el	mama	min	jayeh	la-ʕanna	bukra.  \\
       \textsc{neg} say-2\textsc{sg}.\textsc{f}-to-3\textsc{sg}.\textsc{m}	what	said.3\textsc{sg}.\textsc{f}.\textsc{sbj}-to-2\textsc{sg}.\textsc{f}.\textsc{obj}	mom	who	coming	to-ours   tomorrow  \\
    \glt    ‘Don’t tell him what mom told you about who is visiting us tomorrow.' \\\hfill (Syrian Arabic)\label{ex:fanselow:43}
    \z

\section{Further languages} \label{further:languages}

Detailed information on the occurrence of wh-scope marking in other regions of the world is rarely available. At least two creole languages, Mauritius Creole \REF{ex:fanselow:44} (cf. \citealt[82]{Adone-Vainikka1999}) and Louisiana Creole (\citealt{Brandt2020}) have wh-scope marking constructions. With their SVO normal order, they also do not fit well into a model that restricts wh-scope marking primarily to SOV languages.

\ea 
\gll    Ki 	    Zan ti 		       krwar 		ar 	    kinsala Mari ti 	pe 	koze?   \\
        what	Zan \textsc{tns}	believe 	with 	whom  Mari \textsc{tns} \textsc{asp}	talks.3\textsc{sg}  \\
    \glt    ‘Who does Zan believe Mari is talking with?' \hfill (Mauritius Creole)\label{ex:fanselow:44}
    \z

\noindent Warlpiri is the only Australian language for which there is a detailed analysis of its wh-scope marking construction (\citealt{Legate2011}). Warlpiri forms the construction with the counterpart to \textit{how}, partly because the use of the counterpart to \textit{what} is very restricted. Warlpiri is an SOV language in which complement clauses are islands for movement.

The grammar of question formation in the languages of sub-Saharan Africa is comparatively well-studied. This is especially true for the Niger-Congo language family. Partial wh-movement is widespread, i.e., we find constructions like \REF{ex:fanselow:2}, in which a question phrase is placed in the left periphery of the complement clause, even if it has matrix clause scope: Awing (\citealt{Fominyam2021}), Buli (\citealt{Ferreira-Ko2000}), Dagbani (\citealt{Issah2013}), Dangme (\citealt{Caesar2016}), Dholou (\citeauthor{Schardl2013} \citeyear{Schardl2012}, \citeyear{Schardl2013}), Gichuka (\citealt{Muriungi-Mutegi-etal2014}), Ibibio (\citealt{Doherty2016}), Ikalanga (\citealt{Letsholo2006}), Kikuyu (\citealt{Sabel2000}), Kitharaka (\citealt{Muriungi2005}), Lubukusu (\citealt{Wasike2006}), Moro (\citealt{Rohde2006}), Shona (\citealt{Zentz2016}), several Tano languages (\citealt{Kandybowicz-Torrence2015}, \citealt{Torrence-Kandybowicz2013}), and Zulu (\citealt{Sabel-Zeller2006}). In contrast, the occurrence of wh-scope marking constructions is never reported in these analyses of interrogative sentence formations (nor in others). Nor have I found wh-scope marking described in grammars for Nilo-Saharan and Khoisan languages. In other words, sub-Saharan Africa is apparently free of wh-scope marking, or at least it is relatively rare. Again, areal observations and syntactic ones (the SVO pattern is predominant) coincide.

Languages of the Americas use wh-scope marking (Passamaquoddy, \citealt{Bruening2006}), clausal pied piping (Karitiana, \citealt{Vivanco2019}), or simple partial movement (Ancash Quechua, \citealt[487]{Muller-Sternefeld1996}) as alternatives to long movement. Unfortunately, the issue of long movement and its alternatives is mostly not addressed in grammatical descriptions of interrogative sentence formation in the languages of the Americas, so without additional empirical research, statements about the distribution of wh-scope marking in the Americas are not possible.

\section{Concluding remarks} \label{concluding:remarks}

For the European languages we could identify a rather clear picture of the wh-scope marking construction. It is present in Central, Eastern and Southeastern Europe, although languages with VO order seem to have the construction less frequently. However, it is difficult to determine the origin of the construction, partly because it is more likely to belong to the spoken than to the written language, and therefore it is less likely to be found in the written records.

For an evaluation of the data situation in languages outside Europe, we have mostly tried to draw on the existing literature. This not only means that we have been able to work with evidence for only a few languages -- often this evidence is inconclusive. After all, the clearest test for the presence of wh-scope marking in a language are constructions in which the interrogative clause is embedded, so that a parenthetical analysis can be excluded. In contrast, one often only finds examples of matrix sentences in the literature in which the subject of the sentence is a 2nd person pronoun. In this respect, our small survey might even have overestimated the frequency of occurrence of wh-scope marking.

Nevertheless, the lack of mention of wh-scope marking in the very intensive discussion of interrogative sentence formation in the Niger-Congo languages seems to indicate the existence of a linguistic area in which the construction does not occur or only very rarely occurs. The area would have in common with Western and Southern Europe the characteristic that SVO languages dominate in it.

Thus, our chapter shows that it is worthwhile to investigate in more detail what connections exist between the position of verb and object and the occurrence and concrete choice of constructions that are alternative to the long movement in interrogative clauses.

\section*{Abbreviations}
\begin{multicols}{3}
\begin{tabbing}
MMM \= first persion\kill
1 \> first person \\
2 \> second person \\
3 \> third person \\
\textsc{abl} \> ablative \\
\textsc{acc} \> accusative \\
\textsc{aor} \> aorist \\
\textsc{asp} \> aspect marker \\
\textsc{aux} \> auxiliary \\
\textsc{comp} \> complementizer \\
\textsc{cop} \> copula \\
\textsc{dat} \> dative \\
\textsc{dur} \> durative \\
\textsc{erg} \> ergative \\
\textsc{expl} \> expletive \\
\textsc{f} \> feminine \\
\textsc{fut} \> future \\
\textsc{gen} \> genitive \\
\textsc{inf} \> infinitive \\
\textit{l}-\textsc{ptcp} \> \textit{l}-participle \\
\textsc{m} \> masculine \\
\textsc{mod} \> modal \\
\textsc{n} \> neuter \\
\textsc{neg} \> negation \\
\textsc{nom} \> nominative \\
\textsc{obj} \> object \\
\textsc{opt} \> optative \\
\textsc{pass} \> passive \\
\textsc{pl} \> plural \\
\textsc{prs} \> present \\
\textsc{pst} \> past \\
\textsc{ptcp} \> participle perfect \\
\textsc{q} \> question particle \\
\textsc{refl} \> reflexive \\
\textsc{sbj} \> subject \\
\textsc{sg} \> singular \\
\textsc{tns} \> tense marker \\
\textsc{top} \> topic \\
\textsc{ver} \> version marker 
\end{tabbing}
\end{multicols}

% ---------------------------------

\section*{Acknowledgements}

The preoccupation with sentences introduced by the Russian wh-phrase \textit{kak} ‘how' respectively its equivalent in other languages is one of the overlaps in the academic writings of Ilse Zimmermann and myself, even if the individual issues dealt with are very different. The choice of the topic for my contribution also reflects the wish to include Sorbian in the discussion, because Ilse had her first experience as a hitchhiker in the Sorbian language area -- at an age of 90! What an admirable woman. Ilse accompanied me as a motherly friend for almost 40 years in my professional career -- for that I am very grateful. And also for the nice hours we spent together in her garden.

I would like to express my gratitude to the following linguists and native speakers for discussing and/or providing example sentences: Johanita Kirsten, Jean Marie Potgieter (Afrikaans), Elton Prifti (Albanian), Shakuntala Mahanta (Assamese), Itziar Laka (Basque), Cleo V. Altenhofen (Brazilian Hunsrück German), Snejana Iovtcheva (Bulgarian), Mélanie Jouitteau (Celtic Breton), Radek Šimík (Czech), Nele Ots (Estonian), Dara Jokilehto (Finnish), Shin-sook Kim (Korean), Jan Casalicchio (Ladin), Evija Baša (Latvian), Milena Šereikaitė (Lithuanian), Branimir Stanković (Macedonian), Rosmin Mathew (Malayalam), Federica Cognola (Mocheno), Stavros Skopeteas (Modern Greek), David Erschler (Ossetic), Fereshteh Modarresi (Persian), Emil Ionescu (Romanian), Boban Arsenijević (Serbian), Jolanta Tambor (Silesian), Lanko Marušič (Slovenian), Jaklin Kornfilt (Turkish), Andreas Schmidt, Svetlana Edygarova (Udmurt), Božena Braumanowa, Franciska Grajcarekec, Sonja Wölke from Domowina (Upper Sorbian), Lea Schäfer, Moshe Taube (Yiddish).


The research for this chaper was funded by the Deutsche Forschungsgemeinschaft (DFG, German Research Foundation) — Project-ID 317633480-SFB 1287.
\section*{Editors' note}
The present version of the chapter is the result of a revision by Łukasz Jędrzejowski and Uwe Junghanns. Due to severe illness, the author, unfortunately, was not able to revise his contribution himself, and asked us to revise the original submission. We readily complied with his request, and also added the abstract. We heartily thank Roland Meyer and Catherine Rudin for discussing the Slavic data with us. All remaining mistakes and errors are our responsibility.
% %%%%%%%%%%%%%%%%%%%%%%%%%%%%%%%%%%%%%%%%%%%%%%%%%%%%%%%%%%%%%%%
% %%%%%%%%%%%%%%%%%%%%%%%%%%%%%%%%%%%%%%%%%%%%%%%%%%%%%%%%%%%%%%%

\DeclareRobustCommand{\VAN}[3]{#3}
\printbibliography[heading=subbibliography,notkeyword=this]

\end{otherlanguage}
\end{document}
