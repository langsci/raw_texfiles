\documentclass[output=paper, colorlinks, citecolor=brown, booklanguage=german]{langscibook} 
\ChapterDOI{10.5281/zenodo.15471431}

\author{Ilse Zimmermann\affiliation{Zentrum für Allgemeine Sprachwissenschaft (ZAS), Berlin}}
\title{\textit{So} und \textit{wie} in satzadverbiellen Phrasen}

\abstract{Im Rahmen neuerer Entwicklungen der von \citet{chomsky1992MinimalistProgramLinguisticTheory, chomsky1995MinimalistProgram} geprägten Syn\-tax\-theorie und ihrer Ergänzung durch die von \citet{bierwisch1996lexicalinformationminimalist, bierwisch1988on-the-grammar-of-local-prepositions,  bierwisch1989eventnominalizations} und \citet{lang1987semantikderdimensionsauszeichnungraumlicherobjekte, lang1990sprachkenntnis, lang1994semantische} vertretene Unterscheidung zwischen der Semantischen Form als grammatisch determinierter Bedeutungsstruktur sprachlicher Ausdrücke und konzeptuellen Repräsentationen der Welterfahrung von Sprechern und Hörern werden satzadverbielle Phrasen mit den sie einleitenden Formativen \textit{so} und \textit{wie} untersucht. Dabei geht es um den syntaktischen und semantischen Status dieser Konstruktionen und um die Rolle von \textit{so} und \textit{wie} in ihnen. Es werden Hauptsätze, Nebensätze, Satzparenthesen und entsprechende elliptische Ausdrücke betrachtet. Angelpunkt der Analyse sind die beiden Hypothesen, daß \textit{wie} in den zur Debatte stehenden Konstruktionen einen Relativsatz einleitet und daß \textit{so} bzw. sein stummes Pendant den entscheidenden Faktor für den spezifischen Beitrag der satzadverbiellen Phrasen zur Bedeutung der Gesamtkonstruktion darstellt.}

\IfFileExists{../localcommands.tex}{%hack to check whether this is being compiled as part of a collection or standalone
    \addbibresource{../localbibliography.bib}
    \usepackage{langsci-optional}
\usepackage{langsci-gb4e}
\usepackage{langsci-lgr}

\usepackage{listings}
\lstset{basicstyle=\ttfamily,tabsize=2,breaklines=true}

%added by author
% \usepackage{tipa}
\usepackage{multirow}
\graphicspath{{figures/}}
\usepackage{langsci-branding}

    
\newcommand{\sent}{\enumsentence}
\newcommand{\sents}{\eenumsentence}
\let\citeasnoun\citet

\renewcommand{\lsCoverTitleFont}[1]{\sffamily\addfontfeatures{Scale=MatchUppercase}\fontsize{44pt}{16mm}\selectfont #1}
  
    \togglepaper[23]
}{}
% % % https://www.overleaf.com/project/5faa78eaf7428e3e7bee4ea0

\begin{document}
\begin{otherlanguage}{german}
\maketitle

% The part below is from the generic LangSci Press template for papers in edited volumes. Delete it when you're ready to go.

%LINE FOR THE FOOTNOTE FROM THE HEADING\footnote{\label{fn0}I presented a shorter version of this paper at the second meeting of the Slavic Linguistics Society in Berlin in August 2007 and profited from the discussion. For contemporary Russian, I have consulted Elena Gorichneva, Nikolai Grettschak, Wladimir Klimonow, and Faina Pimenova. For Slovenian, I am indebted to Boštjan Dvořák. For discussion on German and for help in various respects I am grateful to Brigitta Haftka. Barbara Jane Pheby and Jean Pheby have helped me with the English translation of the examples. To Uwe Junghanns and a \textit{JSL} reviewer I owe many valuable suggestion}

% section 1
\section{Aufgabenstellung} \label{sec:zi97:1}

Im Mittelpunkt der Betrachtung stehen die pronominalen Ausdrücke \textit{so} und \textit{wie} in satzadverbiellen Phrasen wie in \REF{ex:zi97:1}--\REF{ex:zi97:2} und ihren Verkürzungen wie in \REF{ex:zi97:3}--\REF{ex:zi97:4}. Es geht um die Syntax und Semantik dieser Konstruktionen im Verhältnis zu Satzadverbien wie in \REF{ex:zi97:5}--\REF{ex:zi97:6} und zu satzadverbiellen Präpositionalphrasen wie in \REF{ex:zi97:7}--\REF{ex:zi97:8}.

\ea \label{ex:zi97:1} Das Konsulat ist, (so) wie es scheint, geschlossen.
\ex \label{ex:zi97:2} Das Konsulat, so scheint es, ist geschlossen.
\ex \label{ex:zi97:3} Das Konsulat wird, (so) wie erwartet, wieder geöffnet.
\ex \label{ex:zi97:4} Das Konsulat wird, so die Brüsseler Nachrichten, wieder geöffnet. 
\ex \label{ex:zi97:5} Scheinbar ist der Konsul abgereist.
\ex \label{ex:zi97:6} Erwartungsgemäß blieb der Konsul dem Empfang fern.
\ex \label{ex:zi97:7} Allem Anschein nach ist der Konsul abgereist.
\ex \label{ex:zi97:8} Das Konsulat wird den Brüsseler Nachrichten zufolge wieder geöffnet.
\z

\noindent Die zu untersuchenden Phrasen haben wie alle Satzadverbiale im übergeordneten Satz keinen Satzgliedstatus als Argument, Modifikator oder Prädikativ. Sie sind Zusätze. Was ihre spezielle Funktion ist, wird näher zu bestimmen sein.

Die Analyse muß sich der Frage stellen, inwieweit satzadverbielle Phrasen mit \textit{so} und \textit{wie} mit Vergleichskonstruktionen wie \REF{ex:zi97:9}--\REF{ex:zi97:10} zu tun haben. Dabei ist auch die Mehrdeutigkeit von Satzverknüpfungen wie in \REF{ex:zi97:11}--\REF{ex:zi97:12} zu berücksichtigen. Es sollte als Vorzug gelten, wenn \textit{so} und \textit{wie} in den verschiedenen Konstruktionen eine möglichst einheitliche Behandlung erfahren.

\ea \label{ex:zi97:9} Hans ist so zuverlässig, wie es sein Vater war.
\ex \label{ex:zi97:10} Ingrid argumentiert (so) wie ein Meister.
\ex \label{ex:zi97:11} Der Bote hat dem Konsul die Einladung übergeben, (so) wie wir es verabredet hatten.
\ex \label{ex:zi97:12} Der Institutsdirektor tanzt mit der Magnifica, (so) wie es sich gehört.
\z

\noindent In den Rahmen unserer Thematik gehört diejenige Interpretation von \REF{ex:zi97:11} und \REF{ex:zi97:12}, bei der der mit (\textit{so}) \textit{wie} eingeleitete Satz nicht eine Modalangabe bildet, sondern satzadverbielle Funktion hat.

Eine zentrale Frage wird sein, was syntaktisch und semantisch die Bezugsdomäne für die mit \textit{so} und \textit{wie} gebildeten satzadverbiellen Phrasen ist. Ein anderer Problemkreis der Untersuchung ist die interne Strukturierung der mit \textit{so} und \textit{wie} konstituierten Einheiten. Es ist vor allem zu klären, worauf sich \textit{wie} bzw. \textit{so} beziehen und wie sie syntaktisch und semantisch mit dem lexikalischen Kopf des satzadverbiellen Satzes bzw. seiner Verkürzung zusammenhängen und in welchem Sinn gegebenenfalls Ellipsen vorliegen.

Nicht zu übersehen ist auch das Auftreten satzbezüglicher Pronomen wie \textit{es} und \textit{das} als mögliche Bezugsgrößen für \textit{so} und \textit{wie}. Es stellt sich die Frage, wo möglicherweise Small-clause-Konstruktionen vorliegen, mit \textit{so} bzw. \textit{wie} resp. dessen Spur als Prädikatausdruck und \textit{es} oder \textit{das} als Argumentausdruck.

Insgesamt versteht sich die Studie als vorläufiger Schlußakt von mehreren Teiluntersuchungen zur Syntax und Semantik von \textit{so} und vor allem von \textit{wie} (s. \citealt{zimmermann1987zursyntaxvonkomparationskonstruktionen, zimmermann1991diesubordinierendekonjunktionwie, zimmermann1992derskopusvonmodifikatoren, zimmermann1995bausteine}).\footnote{Vorfassungen der vorliegenden Untersuchung habe ich auf dem 2. und 4. Treffen des Netzwerks ``Sprache und Pragmatik'' in Rendsburg, 26.--30.9.1994 bzw. 7.--11.10.1996, auf dem 7. Wuppertaler Linguistischen Kolloquium ``Koordination, Subordination und andere Formen der Satzverknüpfung'', 25.-26.11.1994 und im KIT-Kolloquium der Technischen Universität Berlin, Interdisziplinäres Forschungsprojekt Kognition und Kontext am 8.11.1996 vorgestellt. Ich danke den Teilnehmern für stimulierende Diskussion. Besonderer Dank gebührt Margareta Brandt für sehr anregenden Austausch und große Geduld. Zu danken habe ich auch Rainer Bäuerle, Johannes Dölling, Brigitta Haftka, Barbara Kraft, Renate Pasch, Christopher Piñón, Marga Reis, Inger Rosengren, Kerstin Schwabe, Anita Steube und Edeltraud Winkler für hilfreiche Diskussion und einschlägige Hinweise.\label{ft:1}}

\section{Syntax}\label{sec:zi97:2}
\subsection{Annahmen zur obersten Strukturdomäne von Sätzen}\label{sec:zi97:2.1}

Satzadverbielle Phrasen mit \textit{so} und \textit{wie} als Einleitung sind von Hauptsätzen und von adverbiellen sowie nichtadverbiellen Nebensätzen zu unterscheiden.

Adverbielle Nebensätze sind PPs. Ihr lexikalischer Kopf P ist ein zweistelliger Prädikatausdruck, der an seine beiden Argumente Thetarollen vergibt. Die temporale adverbielle Konjunktion \textit{wie} ist ein solches P (s. \citealt{steube1980temporalebedeutungimdeutschen}).

\ea \label{ex:zi97:13} Wie Hans ins Zimmer kam, schlief das Kind.
\z

\noindent Hauptsätze und nichtadverbielle Nebensätze sind CPs. Ihr funktionaler Kopf C leistet die Bindung des referentiellen Arguments des lexikalischen Kopfs V von CP und vergibt keine Thetarollen. Neben den Konjunktionen \textit{daß} und \textit{ob} ist auch \textit{wie} eine Instanz von C (s. \citealt{zimmermann1991diesubordinierendekonjunktionwie}).

\ea \label{ex:zi97:14} Hans hörte, wie das Kind weinte.
\z

\noindent Der Satztyp von CP wird durch morphosyntaktische Merkmale in C bzw. in SpecC charakterisiert (s. \citealt{brandt1989satzmodusmodalitatundperformativitat}, \citealt{brandt1992satztyp}, \citealt{zimmermann90}, \citealt{zimmermann1991diesubordinierendekonjunktionwie, zimmermann93, zimmermann94}).

Wie alle lexikalischen und funktionalen Kategorien verstehe ich C als Menge spezifizierter morphosyntaktischer Merkmale. Das Merkmal $+$C kennzeichnet Nebensätze, während $-$C Hauptsätze charakterisiert. Das Merkmal $+$fin gilt im Kontext von $-$C als stark und attrahiert das finite Verb, so daß sich die V1- bzw. V2-Sätze des Deutschen ergeben. Sätze mit $+$C haben Verbendstellung.

Relativsätze sind CPs mit +C als funktionalem Kopf, ergänzt durch $-$w $-$imp für deklarativen Satzmodus und mit der Charakterisierung $-$w $-$def $+$rel in SpecC, wo sich das Relativpronomen bzw. das relativische Pronominaladverb befindet.

Die mit \textit{wie} eingeleiteten satzadverbiellen Sätze bzw. ihre Verkürzungen sind solche Relativsatzkonstruktionen.\footnote{Sprachen wie das Schwedische und Bulgarische bieten besondere Evidenz für diese Annahme. Das Schwedische hat das für Relativsätze typische Formativ \textit{som} als Einleitung bestimmter satzadverbieller Sätze. Auch ein dem Relativsatz vorangehendes Korrelat ist möglich.
\ea 
\gll \minsp{(} Så) som jag ser det, är han sjuk. \\
{} so wie ich sehen.\textsc{1sg.präs} das ist er krank\\
\glt `(So) wie ich das sehe, ist er krank.'
\z

\noindent Das Bulgarische kennzeichnet Relativpronomen durch das enklitische Formativ \textit{to} und hebt sie damit von Interrogativ- und Indefinitpronomen der \textit{k}-Reihe ab. Auch hier ist ein vorangestelltes Korrelat möglich.
\ea 
\gll \minsp{(} Taka) kakto znaeš, toj e bolen. \\
{} so wie wissen.\textsc{2sg.präs} er ist krank\\
\glt `(So) wie du weißt, ist er krank.'
\z

\noindent Aufklärung über diese Regularitäten verdanke ich Margareta Brandt und Ivanka Petkova Schick. Siehe auch \citet{brandt1997derredesituierendewiesatz}%Brandt  (in diesem Heft)
.
} \textit{Wie} mit den genannten Kennzeichnungen befindet sich in SpecC, C ist phonologisch leer. \textit{Wie} und \textit{so} sowie sein stummes Pendant werden als Adjektivadverbien analysiert, so daß \textit{so} zusammen mit dem durch \textit{wie} eingeleiteten Relativsatz eine Adverbphrase bildet (s. Abbildung \ref{tree:SatzadverbielleNebensätze}). Ich nenne diese Phrasen satzadverbielle Nebensätze. 


\begin{figure}
    \begin{forest}
      %delay={where n children=0{tier=word, if={instr("P",content("!u"))}{roof}{}}{}}, 
      for tree= {minimum width=4em}
        [(AdvP)
            [Adv
                [{$\left\{ \begin{array}{c} so \\ \varnothing \end{array} \right\}$}, tier=word, no edge]
            ]
            [CP 
                [AdvP$_{i}$
                    [\emph{wie}, tier=word, no edge]
                ]
                [C'
                    [C 
                        [{$\varnothing$}, tier=word]]
                    [VP, edge=dashed
                         [{\hspace{2em}$t_i$ \hspace{2em}}V, roof, tier=word]
                     ]
                ]
            ]
        ]
    \end{forest}
    \caption{Satzadverbielle Nebensätze}
    \label{tree:SatzadverbielleNebensätze}
\end{figure}

\iffalse 

\begin{figure}
    \begin{forest}
        [(AdvP), for tree= {minimum width=4em}
            [Adv[$\left\{ \begin{array}{c} so \\ \varnothing \end{array} \right\}$, tier=word, no edge]]
            [CP[AdvP_{i}[wie, tier=word, no edge]]
                [C'
                    [C[\varnothing, tier=word]]
                    [VP, edge=dashed
                            [\hspace{2em}t_{i}{\hspace{2em}, tier=word]
                            [V, tier=word]
                    
                    ]
                ]
                
            ]
        ]
    \end{forest}
    \caption{Satzadverbielle Nebensätze}
    \label{tree:SatzadverbielleNebensätzeALT}
\end{figure}

\fi 

Der mit \textit{wie} eingeleitete Relativsatz ist das Komplement des Korrelats \textit{so}. Diese Konstellation entspricht der Konfiguration von D(eterminierer) und CP als Komplement, die \citet{kayne1994antisymmetry} für die Einbettung von Relativsätzen annimmt. Inwiefern \textit{so} artikelartige Funktion hat, wird bei seiner semantischen Charakterisierung deutlich werden. Fehlt das Korrelat \textit{so}, fungiert der \textit{wie}-Satz als ein freier Relativsatz, der sich an ein phonologisch stummes Adjektivadverb angliedert (vgl. \citealt{steube91, steube92}), bzw. als CP in appositiver Funktion.

Den satzadverbiellen Nebensätzen vergleichbare Hauptsätze mit dem Adjektivadverb \textit{so} im Vorfeld, die wie im Beispiel \REF{ex:zi97:2} parenthetisch verwendet werden können, sind CPs mit V2-Stellung und deklarativem Satzmodus:% (s. \REF{tree:SatzadverbielleHauptsätze}).

\begin{figure}
    \begin{forest}
        %delay={where n children=0{tier=word, if={instr("P",content("!u"))}{roof}{}}{}},
        [CP, for tree={minimum width=4em}
            [AdvP$_{i}$
                [\emph{so}, tier=word, no edge]
            ]
            [C'
                [C
                    [V$_{j}$, tier=word]
                    [C
                        [$\varnothing$, tier=word]
                    ]
                ]
                [VP, edge=dashed
                    [{\hspace{2em}$t_i$ \hspace{2em}}$t_j$, roof, tier=word]
                ]
            ]
        ]
    \end{forest}
%                    [\uline{\hspace{2em}}$t_i$\uline{\hspace{2em}}$t_j$, tier=word]
    \caption{Satzadverbielle Hauptsätze}
    \label{tree:SatzadverbielleHauptsätze}
\end{figure}

\iffalse 

\begin{figure}
    \begin{forest}
        [CP, for tree={minimum width=4em}
            [AdvP_{i}[\emph{so}, tier=word, no edge]]
            [C'
                [C[V_{j}, tier=word]
                    [C[\varnothing, tier=word]]
                ]
                [VP, edge=dashed
                    [\hspace{2em}t_{i} \hspace{2em}, tier=word]
                    [t_{j}, tier=word]
                ]
            ]
        
        ]
    \end{forest}
    \caption{Satzadverbielle Hauptsätze}
    \label{tree:SatzadverbielleHauptsätzeALT}
\end{figure}
\fi 

% \ea \label{tree:Satzadverbielle}
%     \ea \label{tree:SatzadverbielleNebensätze} Satzadverbielle Nebensätze
%         \begin{forest}
%             [(AdvP), for tree= {minimum width=4em}
%                 [Adv[$\left\{ \begin{array}{c} so \\ \varnothing \end{array} \right\}$, tier=word, no edge]]
%                 [CP[AdvP_{i}[wie, tier=word, no edge]]
%                     [C'
%                         [C[\varnothing, tier=word]]
%                         [VP, edge=dashed
%                                 [\uline{\hspace{2em}}t_{i}\uline{\hspace{2em}}, tier=word]
%                                 [V, tier=word]
                        
%                         ]
%                     ]
                    
%                 ]
%             ]
%         \end{forest}
%     \ex \label{tree:SatzadverbielleHauptsätze} Satzadverbielle Hauptsätze
%         \begin{forest}
%             [CP, for tree={minimum width=4em}
%                 [AdvP_{i}[so, tier=word, no edge]]
%                 [C'
%                     [C[V_{j}, tier=word]
%                         [C[\varnothing, tier=word]]
%                     ]
%                     [VP, edge=dashed
%                         [\uline{\hspace{2em}}t_{i}\uline{\hspace{2em}}, tier=word]
%                         [t_{j}, tier=word]
%                     ]
%                 ]
            
%             ]
%         \end{forest}
%     \z
% \z

\subsection{Annahmen zur Kategorisierung lexikalischer Kategorien} \label{sec:zi97:2.2}

Es ist zu beweifeln, daß zwei syntaktische Merkmale zur Differenzierung lexikalischer Kategorien ausreichen (s. \citealt{zimmermann85}, \citealt{zimmermann87}, \citealt{zimmermann1987zursyntaxvonkomparationskonstruktionen}, \citealt{zimmermann88b}, \citealt{zimmermann88a}, \citealt{zimmermann94}). Ich plädiere für das V und N ergänzende Merkmal Adv mit folgender (hier nicht vollständig dargebotener) Werteverteilung:

\begin{table}[H]
\caption{Merkmale}
\label{tab:zi97:1}
\begin{tabular}{cccl}
\lsptoprule
    V & N & Adv & \\\midrule
    $+$ & $-$ & $-$ & Verben \\
    $-$ & $+$ & $-$ & Substantive \\
    $+$ & $+$ & $-$ & Adjektive \\
    $+$ & $+$ & $+$ & Adjektivadverbien \\
    $-$ & $-$ & $+$ & Adverbien, adverbielle Präpositionen und Konjunktionen \\\lspbottomrule
\end{tabular}
\end{table}

\iffalse
\ea \begin{tabular}{c c c l}
    V & N & Adv & \\
    + & - & - & Verben \\
    - & + & - & Substantive \\
    + & + & - & Adjektive \\
    + & + & + & Adjektivadverbien \\
    - & - & + & Adverbien, adverbielle Präpositionen und Konjunktionen \\
    \end{tabular}
\z
\fi 

%\noindent 
\textit{So} und \textit{wie} rechne ich, wie schon gesagt, den Adjektivadverbien zu, für die charakteristisch ist, daß sie nicht adnominal auftreten können, genau wie \textit{anders}. Sie sind die adverbiellen Pendants zu \textit{solcher} und \textit{welcher}.

Während $+$N $+$V $+$Adv syntaktische Merkmale von \textit{so} und \textit{wie} als lexikalische Kategorien sind und mit der semantischen Eigenschaft von Adverbien korrespondieren, Kopf einstelliger Prädikatausdrücke zu sein, entsprechen die morphosyntaktischen Kennzeichnungen $-$w $-$def $+$rel für \textit{wie} und $-$w $+$def $-$rel für \textit{so} bestimmten Operatoreigenschaften dieser Formative, was unten noch genauer zur Sprache kommen wird.\footnote{Ich vernachlässige hier das ungelöste Problem, wie Attributiva tantum wie \textit{solcher}, \textit{welcher}, Adverbia tantum wie \textit{sehr}, \textit{gern}, Prädikativa tantum wie \textit{entzwei}, \textit{wert} und Prädikatausdrücke wie \textit{so}, \textit{wie}, \textit{anders}, die adverbiell und prädikativ verwendet werden können, von Adjektiven wie \textit{gut}, \textit{wahrscheinlich}, die attributiv, prädikativ und adverbiell auftreten, kategoriell zu unterscheiden sind. Möglicherweise ist mit morphosyntaktischen Merkmalen zu rechnen, die die Verteilung regeln. \textit{So}, \textit{wie}, \textit{anders} könnten dann als Adjektive angesehen werden, für die die attributive Verwendung ausgeschlossen ist. Zur Charakterisierung von \textit{so} und \textit{wie} vgl. \citet[564--566, 742f]{1896deutsches-worterbuch}.} 

Diese Analyse steht zu den von \citet{pittner93, pittner95} zu den satzeinleitenden Formativen \textit{so} und \textit{wie} gemachten Feststellungen im Gegensatz. Bei Pittner wird das Zusammenvorkommen von \textit{so} und \textit{wie} nicht in Betracht gezogen und die Möglichkeit, \textit{wie} als relativischen pronominalen Ausdruck aufzufassen, (folglich) nicht erwogen. \citet[314ff.]{pittner93} sagt ausdrücklich, daß \textit{wie} im Gegensatz zu \textit{was} an der Satzspitze kein Relativum sei und daß \textit{so} und \textit{wie} keine Proformen mit Satzgliedfunktion seien. \textit{Wie} sei eine subordinierende Partikel und bedinge Endstellung des finiten Verbs, \textit{so} sei ein ``vielfältig verwendbares textkonnektierendes Element'' \citep[306, 311, 316]{pittner93}.

Es erscheint geboten, die Analyse von satzadverbiellen Phrasen mit \textit{so} und \textit{wie} an Untersuchungen zu Vergleichskonstruktionen zu orientieren, wo \textit{so} und \textit{wie} als systematisch aufeinander bezogene und teilweise miteinander kookkurrierende pronominale Ausdrücke mit bestimmtem Satzgliedwert angesehen werden (s. \citealt{bierwisch1987semantikgraduierung}, \citealt{zimmermann1987zursyntaxvonkomparationskonstruktionen}, \citealt{zimmermann1992derskopusvonmodifikatoren}, \citealt{zimmermann1995bausteine}).

\subsection{Die Position der satzadverbiellen Phrasen mit \textit{so} und \textit{wie} in der übergeordneten Konstruktion}\label{sec:zi97:2.3}

Wie Satzadverbiale generell können auch die hier untersuchten Phrasen mit satzadverbieller Funktion in sehr verschiedenen Positionen der übergeordneten Konstruktion auftreten. Sie können ihr voran- oder nachgestellt sein oder in sie an verschiedenen Stellen eingebaut sein. Dabei ist zwischen phonologischer Integrierung und phonologischer Selbständigkeit der Satzadverbialphrase zu unterscheiden. Insbesondere als Parenthesen und bei Nachstellung können die satzadverbiellen Fügungen selbständige Informationseinheiten, mit eigener Fokus\hyp Hintergrund\hyp Gliederung und entsprechender prosodischer Autonomie, bilden (s. \citealt{brandt1990weiterfuhrendenebensatze, brandt1994subordinationundparenthese, brandt1997zurpragmatiksatzadverbiellerwiephrasen}).

Im Folgenden werde ich die parenthetischen Hauptsätze mit \textit{so} an der Spitze vernachlässigen und die möglichen Positionen satzadverbieller Nebensätze ohne Berücksichtigung ihrer Fokus-Hintergrund-Gliederung betrachten.

Satzadverbielle Nebensätze haben in der sie einbettenden Konstruktion keine syntaktische Funktion als Argument, Prädikativ oder Modifikator (s. \citealt{sommerfeldt83}). Sie sind Zusätze, die satzbezüglichen, mit \textit{was} bzw. seinen Suppletivformen eingeleiteten appositiven Relativsätzen sehr ähnlich sind, sich von diesen jedoch semantisch und syntaktisch dadurch unterscheiden, daß sie, wie auch \citet[26, 31]{hetland1992satzadverbienimfokus} feststellt, Satzadverbien sowie satzadverbiellen PPs und genitivischen DPs vergleichbar sind (vgl. die Beispiele \REF{ex:zi97:1}, \REF{ex:zi97:2}, \REF{ex:zi97:5}--\REF{ex:zi97:8}. Wie sie können satzadverbielle Nebensätze im Vorfeld, nicht aber im Vorvorfeld, links versetzt mit einem Resumptivum im Vorfeld auftreten (s. \REF{ex:zi97:17}--\REF{ex:zi97:18}).\footnote{Von Fällen wie \REF{ex:zi97:18} sind Vorkommen von Satzadverbialia im Vorvorfeld als Linksversetzung ohne Resumptivum im Vorfeld oder als Signal der Turnübernahme (s. dazu \citealt{kraft1996adverbialezwischensatzen}) zu unterscheiden. Vgl.: \ea Tatsächlich, ich habe gewonnen. \ex (So) wie ich es vermutet habe, ich habe gewonnen. \ex A: Sind deine Nachbarn verreist? \\ B: Vermutlich, ich sehe abends nie Licht bei ihnen. \z}

\ea[]{(So) wie du weißt, bin ich mit deiner Arbeit meistens sehr zufrieden. \label{ex:zi97:17}}
\ex[*]{(So) wie du weißt, so bin ich mit deiner Arbeit meistens sehr zufrieden. \label{ex:zi97:18}}
\z

\noindent Genau wie Satzadverbiale beziehen sich satzadverbielle Sätze nicht auf den Satzmodus der einbettenden Konstruktion, der in C verankert ist. Ihre semantische Wirkungsdomäne liegt unterhalb von C. Das gilt auch in einbettenden adverbiellen und nichtadverbiellen Nebensätzen. Das bedeutet, daß die Stellung satzadverbieller Phrasen im Vorfeld wie in \REF{ex:zi97:17} oder im Nachfeld des einbettenden Satzes wie in \REF{ex:zi97:19}--\REF{ex:zi97:22} keine den semantischen Skopus der Phrasen anzeigende Position sein kann.

\ea Hat der Bote dem Konsul die Einladung übergeben, (so) wie wir es vereinbart hatten? \label{ex:zi97:19}
\ex Übergeben Sie dem Konsul die Einladung, (so) wie wir es vereinbart haben. \label{ex:zi97:20}
\ex Es ist völlig offen, ob das Problem bald gelöst werden wird, wie wir es erhoffen. \label{ex:zi97:21}
\ex Als ich über den Rasen lief, wie es viele praktizieren, riefen mich drei Grüne zur Ordnung. \label{ex:zi97:22}
\z

\noindent Im Inneren der einbettenden Konstruktion können Satzadverbiale und satzadverbielle Nebensätze unmittelbar nach C, aber auch nach einem oder mehreren Satzgliedern rechts von C okkurrieren.\footnote{Die Zeichen ``/'', ``*/'' und ``?/'' markieren in den Beispielen \REF{ex:zi97:23}--\REF{ex:zi97:26} mögliche, unmögliche bzw. fragwürdige Positionen für die jeweilige Satzadverbialphrase im Inneren der einbettenden Konstruktion.}

\ea Ich bin, (so) wie du weißt, mit deiner Arbeit / meistens / (nicht) sehr zufrieden. \label{ex:zi97:23}
\ex der, (so) wie du weißt, mit deiner Arbeit / meistens / (nicht) sehr zufriedene Direktor \label{ex:zi97:24}
\z

\noindent Das Beispiel \REF{ex:zi97:24} deutet an, daß satzadverbielle Phrasen auch in satzartigen Modifikatoren auftreten können. Die Beispiele \REF{ex:zi97:25} und \REF{ex:zi97:26} zeigen, daß die Stellung der Satzadverbphrase relativ zu den links von ihr plazierten Konstituenten nicht beliebig sein kann.

\ea Die Erde steht */ noch nicht / vor ihrem Ende, wie Skeptiker es prophezeien. \label{ex:zi97:25}
\ex Peter bekam ?/ gestern / plötzlich / einen heftigen Hustenanfall, wie es leider dann und wann vorkommt. \label{ex:zi97:26}
\z

\noindent \textit{Noch nicht} in \REF{ex:zi97:25} und \textit{gestern} in \REF{ex:zi97:26} sind nicht im semantischen Geltungsbereich des satzadverbiellen Nebensatzes,\footnote{Eine bezüglich der Plazierung der satzadverbiellen Phrase verunglückte, irreführende Mitteilung ist folgender Hörbeleg vom 6.9.1995: \ea Die Serben haben sich aus diesem Gebiet wie vorgesehen immer noch nicht zurückgezogen. \z \textit{Wie vorgesehen} müßte hinter \textit{immer noch nicht} stehen.} das kann in \REF{ex:zi97:26} auch für \textit{plötzlich} gelten. Dagegen liegen in \REF{ex:zi97:23} alle Satzglieder des die Satzadverbialphrase einbettenden Satzes in deren semantischem Skopus, unabhängig von ihrer Position. Auch die Satznegation befindet sich im Skopus des Satzadverbials, was in \REF{ex:zi97:23} und \REF{ex:zi97:24} angedeutet ist. (Zur kontrastiven Negation siehe die Diskussion im Abschnitt \ref{sec:zi97:4}.)

Was ist nun aus all dem zu schließen? Gibt es eine Basisposition der satzadverbiellen Phrase und wo wäre sie? \citet{hetland1992satzadverbienimfokus} verneint diese Frage. In \citet{brandt1989satzmodusmodalitatundperformativitat} und in \citet{brandt1992satztyp} haben wir angenommen, daß Satzadverbiale als Adjunkte von VP basisgeneriert werden und in dieser Position Operatorstatus haben. Ich halte für denkbar, daß die Satzadverbialphrase nicht VP, sondern der maximalen Projektion einer funktionalen Kategorie oberhalb von VP adjungiert ist, wie z.B. \citet{haftka94} annimmt. Ich vernachlässige in dieser Arbeit weitgehend das Problem, welche funktionalen Strukturdomänen gegebenenfalls zwischen CP und VP liegen. Alle Konstituenten rechts vom Satzadverbial befinden sich in seinem Skopus. Wie weit das auch für diejenigen Phrasen, die aus informationsstrukturellen Gründen nach links bewegt wurden und in der syntaktischen Oberflächenstruktur links vom Satzadverbial figurieren, gilt, bedarf besonderer Untersuchung. Operatorausdrücke in SpecC und C selbst sowie auch adverbielle Konjunktionen wie in \REF{ex:zi97:22} liegen jenseits der Reichweite des Satzadverbials.

Ich will davon ausgehen, daß diese Analyse im Prinzip richtig ist und auch auf die hier untersuchten satzadverbiellen Nebensätze und ihre Verkürzungen zutrifft. Die anhand der Beispiele \REF{ex:zi97:25} und \REF{ex:zi97:26} demonstrierten Skopusverhältnisse machen es allerdings erforderlich, vorzusehen, daß in der Basisstruktur zwischen C und der Satzadverbialphrase Konstituenten auftreten können, die nicht in den Skopus des Satzadverbials eingehen.

Figuriert die satzadverbielle Phrase in SpecC wie in \REF{ex:zi97:5}--\REF{ex:zi97:7} und \REF{ex:zi97:17}, handelt es sich um eine abgeleitete und semantisch nicht wirksame Position. Das könnte auch bei Nachstellung gelten. Allerdings ist in dieser Position -- mindestens für die satzadverbiellen Sätze -- auch mit appositiver Funktion des Nebensatzes zu rechnen, derart daß er eine ganz parallele Interpretation erhält wie ein nachgestellter satzadverbieller Hauptsatz mit \textit{so} an der Spitze. Vgl.:

\ea Die Philharmoniker sind auf Tournee, (so) wie Karl meint. \label{ex:zi97:27}
\ex Die Philharmoniker sind auf Tournee, so meint Karl. \label{ex:zi97:28}
\z

\noindent Für appositive Konstruktionen wie \REF{ex:zi97:27} nehme ich an, daß die satzadverbielle Phrase in der Nachstellung basisgeneriert ist und ein Adjunkt von CP ist. Abweichend von \citet{kayne1994antisymmetry}, dessen Syntaxtheorie Rechtsadjunktion ausschließt, rechne ich mit ihr, und zwar für alle der einbettenden Konstruktion nachgestellten satzadverbiellen Nebensätze (s. \REF{ex:zi97:11}, \REF{ex:zi97:12}, \REF{ex:zi97:19}--\REF{ex:zi97:22}, \REF{ex:zi97:25}--\REF{ex:zi97:27}).

Satzadverbielle Nebensätze (XP) mit \textit{so} bzw. \textit{wie} als Einleitung haben in der sichtbaren syntaktischen Struktur also folgende Stellung:

\ea 
    \ea $[$... C ... [\textsubscript{FP} XP [\textsubscript{FP} ... VP ]]] \hfill Adjunkt von FP\label{ex:zi97:29a}
    \ex $[$XP \; C ... VP] \hfill SpecC\label{ex:zi97:29b}
    \ex $[$[ ... C ... VP ] XP] \hfill Adjunkt von CP bzw. PP\label{ex:zi97:29c}
    \z
\z

\noindent In \REF{ex:zi97:29a} nimmt XP seine Basisposition in der Funktion eines Operators ein. \REF{ex:zi97:29b} und \REF{ex:zi97:29c} zeigen mögliche abgeleitete Positionen von XP. \REF{ex:zi97:29c} kann auch als Struktur gelten, in der XP als basisgenerierte Apposition fungiert.

\subsection{Elliptische satzadverbielle Phrasen} \label{sec:zi97:2.4}

Satzadverbielle Sätze können wie in \REF{ex:zi97:3} und \REF{ex:zi97:4} verkürzt werden, indem das Vollverb bzw. der gesamte Hilfsverbkomplex und das pronominale Subjekt bzw. Objekt \textit{es} oder \textit{das} unausgedrückt bleiben. Ich nehme an, daß es sich um reguläre Satzstrukturen handelt, die den in den Abbildungen \ref{tree:SatzadverbielleNebensätze} und \ref{tree:SatzadverbielleHauptsätze} angegebenen Konstituentenkonstellationen für satzadverbielle Sätze entsprechen.

In der Position des Vollverbs figuriert eine leere Kategorie, die in der Semantischen Form eine sehr allgemeine Bedeutung in Form einer Prädikatvariablen als Parameter zugeschrieben bekommt, der in der konzeptuellen Struktur kontext- und situationsabhängig spezifiziert wird, ganz analog, wie \citet{1994syntax-und-semantik-situativer-ellipsen} das für situative Ellipsen vorsieht. Entsprechendes ist für das ausgelassene Pronomen anzunehmen. Auch die Kopula als Verbmacher ist durch eine phonologisch stumme Instanz mit entsprechender Bedeutungscharakterisierung repräsentiert. Inwieweit in diesen elliptischen Ausdrücken auch Hilfsverben durch stumme Pendants vertreten sind, soll hier unentschieden bleiben.

Es wird also nicht angenommen, daß bei elliptischen satzadverbiellen Phrasen irgendwelche Konstituenten getilgt werden. Die betreffenden elliptischen Ausdrücke sind semantisch vage. Eine Ausfüllung der Lücken würde zu einer semantischen Übercharakterisierung und durch die erforderliche Elidierung zu einer unnatürlichen Komplizierung führen.

Es erscheint auch nicht angemessen, solche Ellipsen als lückenhafte Ausdrücke ganz ohne eine syntaktische Position für das Verb und für das Pronomen zu erzeugen. Es läßt sich leicht zeigen, daß so einzelne Konstituenten syntaktisch und semantisch ohne Zusammenhang blieben, wie z.B. in \REF{ex:zi97:30}--\REF{ex:zi97:33}.

\ea (So) wie seit langem allgemein bekannt, ist Rauchen ungesund. \label{ex:zi97:30}
\ex Wie schon vorauszusehen, werden wir eine reiche Obsternte haben. \label{ex:zi97:31}
\ex Mitten in der Opernaufführung prasselte in der Arena, wie von vielen befürchtet, ein heftiger Gewitterregen auf die Zuschauer nieder. \label{ex:zi97:32}
\ex Die Konfliktparteien werden nur unter UNO-Mitwirkung an den Runden Tisch zu bekommen sein, so letzte Woche der türkische Botschafter zu dieser Krisensituation. \label{ex:zi97:33}
\z

\noindent So produktiv und formelhaft diese Verkürzungen satzadverbieller Sätze auch sein mögen, ist nicht abzusehen, wie bestimmte Satzglieder, darunter auch \textit{so} und \textit{wie}, ohne Verb als dem lexikalischen Kopf von Sätzen semantisch und syntaktisch zu einem Konstruktionsganzen mit den erforderlichen funktionalen Strukturdomänen zu integrieren wären.

Ich plädiere also für eine weitgehend parallele Behandlung vollständiger satzadverbieller Sätze und ihrer elliptischen Verkürzungen. Die genauen Bedingungen für die Weglassungen bedürfen einer separaten Untersuchung.

\subsection{\textit{So} und \textit{wie} als Prädikatausdrücke} \label{sec:zi97:2.5}

Meine syntaktische und semantische Analyse von \textit{so} und \textit{wie} in satzadverbiellen Phrasen geht davon aus, daß es sich bei diesen pronominalen Lexemen um Prädikatausdrücke handelt, wenn sie nicht wie \textit{wie} in \REF{ex:zi97:13} und \REF{ex:zi97:14} Konjunktionen sind. Nicht nur in der Funktion als Prädikativ wie in \REF{ex:zi97:34} und \REF{ex:zi97:35} oder als Modifikator wie in \REF{ex:zi97:36}--\REF{ex:zi97:38},\footnote{Zur Analyse von \textit{so} und \textit{wie} als Prädikatausdrücke in nichtsatzadverbiellen Konstruktionen s. \citet{zimmermann1987zursyntaxvonkomparationskonstruktionen, zimmermann1991diesubordinierendekonjunktionwie, zimmermann1992derskopusvonmodifikatoren, zimmermann1995bausteine}.} sondern auch in allen satzadverbiellen Konstruktionen haben \textit{so} und \textit{wie} Prädikatstatus.

\ea Ingrid ist so, wie sie ist. \label{ex:zi97:34}
\ex Das ist so, wie es ist, und nicht anders. \label{ex:zi97:35}
\ex Die Leiter ist gerade so lang, wie der Baum hoch ist. \label{ex:zi97:36}
\ex So wie die meisten leben, werde ich mich nicht einrichten. \label{ex:zi97:37}
\ex Keiner würde so handeln, wie Hubert es getan hat. \label{ex:zi97:38}
\z

\noindent Zwei Besonderheiten heben die satzadverbiellen Konstruktionen mit \textit{so} und \textit{wie} von diesen Beispielen ab. Erstens: Das Korrelat \textit{so} kann nicht von dem mit \textit{wie} eingeleiteten Relativsatz getrennt werden. Der satzadverbielle Zusatz wird offenbar der einbettenden Konstruktion immer als Einheit einverleibt. Zweitens: Die satzadverbielle Phrase hat in der übergeordneten syntaktischen Fügung keine Satzgliedfunktion als Argument, Prädikativ oder Modifikator.

Es fragt sich nun unausweichlich, worauf sich satzadverbielle Phrasen als Ganzes beziehen und wie \textit{so} und \textit{wie} in sie integriert sind.

Für \textit{so} ist zu berücksichtigen, daß es in satzadverbiellen Hauptsätzen auftritt (s. Abbildung \ref{tree:SatzadverbielleHauptsätze}) bzw. als Korrelat fungiert, das den mit \textit{wie} eingeleiteten Relativsatz als Komplement hat (s. Abbildung \ref{tree:SatzadverbielleNebensätze}). Im ersten Fall ist \textit{so} Prädikatausdruck, im zweiten Fall hat es eine Argumentstelle für den Relativsatz.

\textit{Wie} ist ein relativischer Operator und befindet sich in SpecC. Seine Spur hat wie \textit{so} ohne Relativsatz den Status eines Prädikats.\footnote{Eine andere Möglichkeit, \textit{wie} zu analysieren, habe ich in \citet{zimmermann1995bausteine} diskutiert. Dieser Analyse zufolge würde \textit{wie} in situ als Prädikat interpretiert und für die relativische Funktion ein phonologisch stummer Operator in SpecC vorgesehen. Ich lasse hier offen, welche Analyse zu bevorzugen ist. Die Redeweise, daß \textit{wie} genau wie \textit{so} in den zur Rede stehenden Konstruktionen ein Prädikatausdruck sei, ist also in der hier gewählten Analysevariante immer auf seine Spur zu beziehen. \label{footnote8}}

\textit{So} und die Spur von \textit{wie} können sich auf ein Individuenargument bzw. auf ein propositionales Argument beziehen. Ersteres ist in \REF{ex:zi97:34} und auch in \REF{ex:zi97:36}--\REF{ex:zi97:38} der Fall, letzteres gilt für \REF{ex:zi97:35}, wo \textit{das} und \textit{es} pronominale Vertreter von Propositionen sind.

Es ist genau zu prüfen, welchen Status \textit{so} in satzadverbiellen Hauptsätzen bzw. die Spur von \textit{wie} in satzadverbiellen Relativsätzen hat, was in dem betreffenden Satz als ihr Argument gilt. Hier kann zu diesem Problem nur Grundsätzliches gesagt werden, weil eine genaue Beantwortung dieser Frage eine semantische Analyse des jeweiligen lexikalischen Kopfs des satzadverbiellen Satzes voraussetzt.

Ich rechne mit zwei Fällen. Erstens können \textit{so} bzw. die Spur von \textit{wie} auch in satzadverbiellen Phrasen auf ein Individuenargument bezogen werden. Das ist in \REF{ex:zi97:39} und \REF{ex:zi97:40} der Fall. In \REF{ex:zi97:39} ist es das Aussehen der zur Rede stehenden Person, in \REF{ex:zi97:40} die benannte Situation.

\ea \label{ex:zi97:39} Der Theologe Schorlemmer könnte mein Onkel sein, so wie er aussieht. (Originaläußerung von Maike Schoorlemmer am 25. 7. 1996)
\ex \label{ex:zi97:40} (So) wie Spezialisten die Situation einschätzen, wird es bald zu Steuererhöhungen kommen.
\z

\noindent Im Folgenden lasse ich solche Satzadverbiale vorerst beiseite. Im Abschnitt \ref{sec:zi97:4} beleuchte ich ihre Semantik.

Zweitens können \textit{so} bzw. die Spur von \textit{wie} auf eine pronominal ausgedrückte oder eine entsprechende phonologisch stumme propositionale Entität Bezug nehmen, was für die Mehrzahl der angeführten Beispiele satzadverbieller Phrasen zutrifft und für diese als typisch gelten kann.

\textit{So} und die Spur von \textit{wie} werden in satzadverbiellen Sätzen im Unterschied zum Haupt- und Nebensatz im Beispiel \REF{ex:zi97:35} ohne Vermittlung durch die Kopula auf ihr propositionales Argument bezogen, das entweder pronominal ausgedrückt ist wie in \REF{ex:zi97:1}, \REF{ex:zi97:2}, \REF{ex:zi97:11}, \REF{ex:zi97:12}, \REF{ex:zi97:19}--\REF{ex:zi97:22}, \REF{ex:zi97:25}, \REF{ex:zi97:26} oder stumm bleibt wie in \REF{ex:zi97:17}, \REF{ex:zi97:23}, \REF{ex:zi97:24}, \REF{ex:zi97:27}, \REF{ex:zi97:28}. In elliptischen Satzadverbialen wie in \REF{ex:zi97:3}, \REF{ex:zi97:4} und \REF{ex:zi97:30}--\REF{ex:zi97:33} ist das propositionale Argument systematisch abwesend. Man könnte sagen, daß \textit{so} bzw. die Spur von \textit{wie} in den hier zur Debatte stehenden satzadverbiellen Phrasen zusammen mit ihrem propositionalen Argument semantisch eine Small-clause-Einheit bilden, die beispielsweise für den satzadverbiellen Nachsatz in \REF{ex:zi97:28} unter Einschaltung der Kopula als \textit{Karl meint, es ist so} paraphrasierbar ist.

Was nun den Geltungsbereich der satzadverbiellen Nebensätze und ihrer Verkürzungen angeht, ist wichtig, daß in diesen Konstruktionen die propositionale Argumentstelle von \textit{so} oder seinem stummen Pendant (s. Abbildung \ref{tree:SatzadverbielleNebensätze}) noch offen ist, so daß dadurch für die gesamte satzadverbielle Phrase in der einbettenden Konstruktion eine propositionale Bezugsgröße gesucht werden kann. Sie ist syntaktisch durch die in der Basisstruktur gegebene rechte Kokonstituente des Satzadverbials repräsentiert oder im Satz links von einem appositiv nachgestellten satzadverbiellen Ausdruck zu finden.

In gewisser Übereinstimmung mit der verbreiteten Feststellung, daß es sich bei Satzadverbialen um etwas Zusätzliches, gegenüber der eigentlichen Mitteilung Zweitrangiges, mit propositionalen Einstellungen Zusammenhängendes handelt,\footnote{Siehe dazu vor allem \citet{lang1979zum-status-der-satzadverbiale, lang1983einstellungsausdrucke}, \citet{lang1976recension} und \citet{helbig1990lexikon}.} könnte man sagen, daß Satzadverbiale mit \textit{so} und \textit{wie} der eigentlichen Mitteilung beigefügte Prädikationen beinhalten, die mit der Geltung ihres propositionalen Arguments zu tun haben und in diesem Sinn Metaprädikate sind. Die Spezifik sämtlicher Satzadverbiale, von Satzadverbien wie \textit{wahrscheinlich}, \textit{glücklicherweise}, Präpositionalphrasen wie \textit{nach meiner Meinung}, Genitivphrasen wie \textit{meines Wissens} und von satzadverbiellen Phrasen mit \textit{so} und \textit{wie}, besteht darin, daß sie semantisch als Operatorausdrücke fungieren. Die mit \textit{so} und \textit{wie} eingeleiteten satzadverbiellen Phrasen können auch appositiv verwendet werden. Die folgenden Darlegungen zur Semantik der Satzadverbiale werden das Gesagte verdeutlichen.

\section{Semantik} \label{sec:zi97:3}

\subsection{Satzadverbiale in Operatorfunktion} \label{sec:zi97:3.1}

Satzadverbiale kommen in Sätzen und satzartigen Modifikatoren vor. Sie betreffen den Geltungsgrad des in ihrem Skopus liegenden propositionalen Inhalts oder sie setzen diesen Inhalt als gegeben voraus und kommentieren ihn auf bestimmte Weise (s. dazu \citealt{brandt1997derredesituierendewiesatz}%Brandt in diesem Heft
). Im Einklang mit \citet{brandt1989satzmodusmodalitatundperformativitat} und \citet{brandt1992satztyp} nehme ich an, daß das Satzadverbial in Operatorfunktion die Verbbedeutung der einbettenden Konstruktion, mit allen Argumenten und Modifikatoren, und gegebenenfalls die Satznegation in seinen Skopus nimmt, sofern die betreffenden Satzglieder nicht in Operatorfunktion in SpecC stehen oder links vom Satzadverbial basisgeneriert sind, wie in \REF{ex:zi97:25} und \REF{ex:zi97:26} die Adverbiale \textit{noch nicht} und \textit{gestern}. Bei der semantischen Amalgamierung hat die erweiterte Projektion FP der Verbphrase noch eine offene Argumentstelle, nämlich für das referentielle Argument des Verbs, das in C durch den Satzmodusoperator gebunden wird. Das bedeutet, daß sich das Satzadverbial mit seiner offenen propositionalen Argumentstelle durch Funktionale Komposition mit der FP-Bedeutung verbindet. \REF{ex:zi97:41} verdeutlicht das in verallgemeinerter Form. \REF{ex:zi97:42} ist ein Beispiel.\footnote{In \REF{ex:zi97:42} ist wie in allen folgenden Bedeutungsrepräsentationen das Tempus vernachlässigt.}\textsuperscript{,}\footnote{Anmerkung der Herausgeber: die Autorin verwendet das Hochkomma nach objektsprachlichen Ausdrücken zur Kennzeichnung von Denotationen.} Eine eventuell in Betracht zu ziehende Alternative zu \REF{ex:zi97:41} kommt im Abschnitt \ref{sec:zi97:4} zur Sprache.

\newpage
\ea Die Amalgamierung von Satzadverbial und FP-Bedeutung \\
$\lambda p [...p...](\lambda e [...[e \; \cnst{inst} [...]]...]) = \newline\lambda e [...[e \; \cnst{inst} [...]]...]$ \label{ex:zi97:41}
\ex Peter schläft vermutlich.' = \\
$[\exists e [e' \; \cnst{inst} [x \; \textsc{vermuten} [e \; \cnst{inst} [\textsc{schlafen} \; \textsc{peter} ]]]]] = \newline \lambda Q [\exists e [Q e]] (\lambda p [e' \; \cnst{inst} [x \; \textsc{vermuten} \; p]] (\lambda e [e \; \cnst{inst} [\textsc{schlafen} \; \textsc{peter}]]))$ \label{ex:zi97:42}
\z

\noindent Was also den Skopus des Satzadverbials bildet, ist die Instantiierung der durch die einbettende Konstruktion ausgedrückten Proposition. Die Bindung des referentiellen Arguments $e$ und damit die Referenztypspezifizierung des einbettenden Satzes erfolgt in C, im Beispiel \REF{ex:zi97:42} durch den unmarkierten deklarativen Satzmodus. Der durch das deverbale Satzadverb \textit{vermutlich} ins Spiel kommende Sachverhalt des Vermutens bleibt bezüglich Tempus und Satzmodus und auch bezüglich des vermutenden Subjekts unspezifiziert.\footnote{Ich lasse offen, ob es angebracht wäre, in \REF{ex:zi97:42} und \REF{ex:zi97:43} die in der Bedeutungscharakterisierung des Satzadverbials bzw. des Adjektivs vorkommenden freien Variablen existenziell zu binden. Möglicherweise findet das -- per Default -- beim Übergang der Semantischen Form in die konzeptuelle Struktur statt.} Satzadverbien haben keine Potenz, bezüglich Tempus und Satzmodus charakterisiert zu werden. Durch das unspezifizierte Subjekt propositionaler Einstellungen wie bei \textit{vermutlich} ist es immer möglich, den jeweiligen Sprecher in den Kreis der kontextuell und situativ in Frage kommenden Einstellungsträger einzubeziehen.

Die in \REF{ex:zi97:42} verdeutlichten gegenseitigen Skopusverhältnisse von Satzmodus und Satzadverbial entsprechen den in Substantivgruppen mit Adjektiven wie \textit{vermeintlich} geltenden Konfigurationen. Der Artikel bindet das referentielle Argument des Substantivs und nimmt das wie Satzadverbiale als propositionaler Operator fungierende Adjektiv in seinen Skopus. \REF{ex:zi97:43} illustriert das.

\ea \label{ex:zi97:43} der vermeintliche Dom' = \\
$\iota x [e' \; \cnst{inst} [y \; \textsc{meinen}  \; [\textsc{dom} \; x]]] = \newline \lambda P [\iota x [P \; x]](\lambda p [e' \; \cnst{inst}  \; [y \; \textsc{meinen} \; p]] (\lambda x [\textsc{dom} \; x]))$
\z 

\noindent Nicht nur die Parallelität der Konstituentenkonstellationen in CPs und DPs als referierenden Syntagmen, sondern auch Sätze mit interrogativischem bzw. imperativischem Satzmodus wie in \REF{ex:zi97:19} und \REF{ex:zi97:20} lassen die Annahme als angemessen erscheinen, daß die Satzadverbiale den Satzmodus der einbettenden Konstruktion jenseits ihres Skopus lassen. Die für Satzadverbiale angenommene Basisposition als FP-Adjunkt (s. \REF{ex:zi97:29a}) trägt dem Rechnung.

Ganz analog wie \textit{vermutlich} in \REF{ex:zi97:42} sollen auch die satzadverbiellen Phrasen mit \textit{so} und \textit{wie} an der Spitze und ihre Verkürzungen behandelt werden. \REF{ex:zi97:44} deutet das an.

\ea \label{ex:zi97:44} Peter schläft, (so) wie wir (es) vermuteten.' = \\
$[\exists e [... [\exists e' [e' \; \cnst{inst} [\textsc{wir} \; \textsc{vermuten} ... q]]] %\newline
... [e \; \cnst{inst} [\textsc{schlafen} \; \textsc{peter}]] ... ]] = %\newline 
\lambda Q [\exists e [Q e]] (\lambda p [... [\exists e' [e' \cnst{inst} [\textsc{wir} \; \textsc{vermuten} ... q ]]]...p...]\newline(\lambda e [e \; \cnst{inst} [\textsc{schlafen} \; \textsc{peter}]]))$
\z 

\noindent Die durch `...' markierten Stellen in der Bedeutung der Satzadverbialphrase betreffen die semantischen Komponenten, die \textit{so} und \textit{wie} in die satzadverbielle Phrase einbringen. Wir wenden uns ihnen in den folgenden Abschnitten zu. Die Variable $q$ repräsentiert in \REF{ex:zi97:44} die Bedeutung des
Pronomens \textit{es} bzw. seines stummen Pendants. Offenbar muß dieses Pronomen zu seiner Bezugsdomäne im Hauptsatz in Koreferenzbeziehung gesetzt werden.

Die hier skizzierte Funktion der Satzadverbiale als Operatoren und ihr Verhältnis zum Satzmodus gelten auch, wenn das Satzadverbial in SpecC der einbettenden Konstruktion figuriert oder ihr nachgestellt ist (s. \REF{ex:zi97:29b} bzw. \REF{ex:zi97:29c}). Diese Positionen werden als mögliche abgeleitete Stellungen des Satzadverbials angesehen. Eine nachgestellte satzadverbielle Phrase kann aber auch als Apposition betrachtet werden.

\subsection{Satzadverbielle Phrasen als Appositionen} \label{sec:zi97:3.2}

Satzadverbielle Nebensätze wie in \REF{ex:zi97:27} werden als mehrdeutig angesehen. Entweder ist die satzadverbielle Phrase in Nachstellung zu dem einbettenden Satz basisgeneriert oder aus ihrer Basisposition als FP-Adjunkt gegebenenfalls dorthin bewegt worden. Die für Satzadverbiale typische Operatorfunktion hat sie nur im letzteren Fall.

Als Apposition erhält die satzadverbielle Phrase mit \textit{so} und \textit{wie} eine Interpretation, die dem entsprechenden Hauptsatz mit \textit{so} wie in \REF{ex:zi97:28} vergleichbar ist. Entsprechend diesen Annahmen wird neben der in \REF{ex:zi97:44} angegebenen Interpretation der satzadverbiellen Phrase \textit{(so) wie wir (es) vermuteten}, die ihrer Operatorfunktion Rechnung trägt, auf der Basis der gleichen SF für das Satzadverbial folgende appositive Funktion vorgesehen:

\ea \label{ex:zi97:45} Peter schläft, (so) wie wir (es) vermuteten.' = \\
$[\exists e [e \; \cnst{inst} [\textsc{schlafen} \; \textsc{peter} ]]][...[\exists e' [e' \; \cnst{inst} [\textsc{wir} \; %\newline 
\textsc{vermuten} ... q]]] ...p... ] = %\newline 
[\exists e [e \; \cnst{inst} [\textsc{schlafen} \; \textsc{peter}]]] \lambda Q [Q\;p] \newline(\lambda p [... [\exists e' [e'\; \cnst{inst} [\textsc{wir} \; \textsc{vermuten} ... q]]] ...p...])$
\z

\noindent Hier stehen der Hauptsatz und die satzadverbielle Phrase semantisch nebeneinander, wie ich das auch für den Hauptsatz und den appositiven Satz in \REF{ex:zi97:46} mit dem Relativpronomen \textit{was} an der Spitze annehme.

\ea \label{ex:zi97:46} Peter schläft, was wir vermuteten.' = \\
$[\exists e [e \; \cnst{inst} [\textsc{schlafen} \; \textsc{peter} ]]][\exists e' [e' \; \cnst{inst} [\textsc{wir} \; %\newline 
\textsc{vermuten} \; p]]] = \newline [\exists e [e \; \cnst{inst} [\textsc{schlafen} \; \textsc{peter}]]] \lambda Q [Q\;p] (\lambda p [\exists e' %\newline 
[e' \cnst{inst} [\textsc{wir} \; \textsc{vermuten} \; p]]])$
\z

\noindent In die Interpretation der appositiven Zusätze in \REF{ex:zi97:45} und \REF{ex:zi97:46} ist im Vergleich mit \REF{ex:zi97:44} ein Template, $\lambda Q [Q\;p]$, eingeschaltet, dessen Wirkungsweise darin besteht, die Argumentstelle $\lambda p$ des appositiven Prädikatausdrucks zu blockieren (zu appositiven Relativsätzen s. \citealt{zimmermann1992derskopusvonmodifikatoren}). Dadurch kommt es zu einer semantischen Verselbständigung der appositiven Konstruktion. Ob satzadverbielle Nebensätze mit \textit{so} und \textit{wie} auch in anderen Positionen außer in der Nachstellung appositiv zu interpretieren sind, indem auf ihre SF das erwähnte Template angewendet wird, will ich als Möglichkeit offen lassen. Als Ergebnis dieser Operation entstünden parenthetische satzadverbielle Phrasen.

Die propositionalen Variablen $p$ und $q$ in \REF{ex:zi97:45} und \REF{ex:zi97:46} müssen zu ihrer Bezugsdomäne im Hauptsatz und gegebenenfalls zueinander in Koreferenzbeziehung gesetzt werden.

Es ist nun zu klären, worin sich die appositiven Ergänzungen in \REF{ex:zi97:45} und \REF{ex:zi97:46} unterscheiden, d.h. wie die in \REF{ex:zi97:45} und auch in \REF{ex:zi97:44} durch `...' gekennzeichneten semantischen Komponenten aussehen. Es ist in \REF{ex:zi97:45} und \REF{ex:zi97:46} schon deutlich, daß sich appositive \textit{(so) wie}-Sätze von appositiven \textit{was}-Sätzen nicht nur formativisch, sondern auch semantisch durch größere Komplexität unterscheiden. Das liegt an \textit{so} und \textit{wie}, deren semantische Spezifik im folgenden Abschnitt zur Sprache kommen wird.

Zuvor ist noch eine nicht unwesentliche Ergänzung zu der bisher vorgestellten Analyse von satzadverbiellen Phrasen mit \textit{so} und \textit{wie} nötig. Wie Abbildung \ref{tree:SatzadverbielleNebensätze} vorsieht, kann der mit \textit{wie} eingeleitete Relativsatz auch ohne ein explizites oder phonologisch stummes Korrelat auftreten. Wenn man auf diesen korrelatlosen Relativsatz ein Template anwendet, das den durch \textit{wie} repräsentierten Operator blockiert, erhält man eine semantische Repräsentation, die \REF{ex:zi97:27} und \REF{ex:zi97:28} mit dem satzadverbiellen Nachsatz \textit{wie Karl meint} bzw. \textit{so meint Karl} absolut bedeutungsgleich macht. Mehr noch: Es ist in der Bedeutungsrepräsentation von \textit{so} zu berücksichtigen, daß satzadverbielle Phrasen wie in \REF{ex:zi97:27} mit und ohne \textit{so} die gleiche Bedeutung haben können. Es ist also zu unterscheiden zwischen Fällen wie in \REF{ex:zi97:27} oder \REF{ex:zi97:1}, wo \textit{so} als bedeutungsleerer Aufhänger für den mit \textit{wie} eingeleiteten Relativsatz fungiert, und Fällen wie in \REF{ex:zi97:45} oder in \REF{ex:zi97:11} und \REF{ex:zi97:12}, wo \textit{so} zusammen mit dem Relativsatz wie in Vergleichskonstruktionen fungiert (vgl. \REF{ex:zi97:9}--\REF{ex:zi97:10}). Kennzeichnend für dieses \textit{so} ist, daß es dual zu \textit{anders} ist und mit Adverbien wie \textit{ganz}, \textit{genau}, \textit{beinahe} usw. verbunden werden kann.

\subsection{Zur Semantik von \textit{so} und \textit{wie}} \label{sec:zi97:3.3}

Es wird hier davon ausgegangen, daß \textit{so} und \textit{wie} bzw. die Spur von \textit{wie} in sehr vielen satzadverbiellen Phrasen propositionsbezügliche Prädikatausdrücke sind und daß für \textit{so} mehrere Bedeutungsvarianten vorzusehen sind.

Folgende für unser Thema relevanten Bedeutungsanteile sind wirksam:

\ea \label{ex:zi97:47} 
    \ea so' = \label{ex:zi97:47a}
        \ea $(\lambda W) \lambda x [\forall P [W \; P] \rightarrow [P \;x ]]$ \label{ex:zi97:47ai}
        \ex $\lambda W [W\; P]$ \label{ex:zi97:47aii}
        \ex $P$ \label{ex:zi97:47aiii}
        \z
    \ex die Spur von \textit{wie} = \newline $Q$ \label{ex:zi97:47b}
    \ex wie' = \newline $\lambda p \; \lambda Q [p]$ \label{ex:zi97:47c}
    \ex ander(s)' = \newline $(\lambda W) \lambda x [ \sim[\forall P [W \; P] \rightarrow [P \; x]]]$ \label{ex:zi97:47d}
    \ex es', das' = \newline $x$ \label{ex:zi97:47e} \\
    \vspace{5pt}
    \hspace*{-20pt} mit $x \in \alpha$; $P,Q \in S/\alpha$; $W \in S/(S/\alpha)$; $\alpha \in \{ S,N \}$
    \vspace{5pt}
    \ex Lambdaabstraktion = \newline $\lambda p \; \lambda x [p]$ \label{ex:zi97:47f}
    \ex Argumentstellenunterdrückung = \newline $\lambda P [P \; x]$ \label{ex:zi97:47g} \\
    \vspace{5pt}
    \hspace*{-20pt}mit $x \in \{S, N, S/N, S/S, ...\}$
    \z
\z

\noindent Die Analyse von \textit{so} in (\ref{ex:zi97:47a}.i) und von \textit{anders} in \REF{ex:zi97:47d} ist der von \citet[95, 173, 191ff]{bierwisch1987semantikgraduierung} für diese Formative in Vergleichskonstruktionen angenommenen Bedeutung analog. Beide Prädikatwörter können einen Relativsatz als Komplement haben. Im Fall von \textit{anders} würde er durch das Formativ \textit{als} eingeleitet werden (s. dazu auch \citealt{zimmermann1987zursyntaxvonkomparationskonstruktionen}). Im Fall von \textit{so} wird er durch \textit{wie} eingeleitet, das hier als Lambdaabstraktor für Prädikate interpretiert ist (vgl. Anmerkung \ref{footnote8}). \textit{So}, die Spur von \textit{wie} und \textit{anders} sind auf Individuen bzw. Propositionen bezügliche Prädikate. In den hier betrachteten satzadverbiellen Phrasen sind sie auf \textit{das} bzw. \textit{es} oder auf ihr stummes Pendant als pronominale Stellvertreter von Propositionen bezogen. Diese Pronomen können auch durch einen Nebensatz spezifiziert werden (s. \citealt{zimmermann93}). \REF{ex:zi97:47f} und \REF{ex:zi97:47g} beinhalten zum semantischen System gehörende Templates. (\ref{ex:zi97:47a}.ii) und \REF{ex:zi97:47c} sind durch \textit{so} bzw. \textit{wie} ausgedrückte Spezialfälle der Templates \REF{ex:zi97:47g} bzw. \REF{ex:zi97:47f}. (\ref{ex:zi97:47a}.i) gilt auch für das stumme Pendant von \textit{so}.

Ich gehe nun davon aus, daß die in \REF{ex:zi97:47} angegebenen Bedeutungen nicht nur für prädikative Konstruktionen wie in \REF{ex:zi97:35} zutreffen, sondern für die betreffenden Konstituenten auch in satzadverbiellen Phrasen mitwirken.

Es ist offensichtlich, daß die Bedeutung von \textit{so} und der Spur von \textit{wie} den besonderen über die Geltung der Bezugsproposition reflektierenden Aspekt der mit \textit{so} und \textit{wie} gebildeten satzadverbiellen Phrasen ausmacht, so daß diese Satzadverbiale zur Klasse bestimmter die Gültigkeit von Propositionen betreffender Metaprädikate gehören. Es ist auch verständlich, daß diese satzadverbiellen Phrasen nur mit solchen lexikalischen Köpfen verträglich sind, die -- erstens -- ein propositionales Argument haben und -- zweitens -- mit diesem reflektierenden Gestus kompatibel sind. Faktive Prädikatausdrücke wie \textit{bedauern}, \textit{sich freuen} usw. sind das nicht, weil die Geltung ihres propositionalen Arguments nicht problematisiert werden kann, sondern als gegeben vorausgesetzt wird. Deshalb haben nicht-epistemische evaluative Satzadverbiale wie \textit{leider}, \textit{erfreulicherweise} usw. keine entsprechenden satzadverbiellen Sätze mit \textit{so} und \textit{wie} als Einleitung (s. \citealt{brandt1997derredesituierendewiesatz}%Brandt in diesem Heft
). Nur kognitive und starke Evidenz betreffende Prädikate wie \textit{wissen}, \textit{bekannt}, \textit{offensichtlich} können als lexikalischer Kopf satzadverbieller Phrasen mit \textit{so} und \textit{wie} fungieren. Es ist eine lohnende Aufgabe, den diesbezüglichen Fügungspotenzen der mit Satzadverbien korrespondierenden Verben und Adjektive nachzugehen. Die Grammatiken, Wortbildungslehren, Handbücher und berühmten Wörterbücher schweigen zu diesem interessanten Phänomen.

Die hier vorgestellte Analyse von \textit{so} und \textit{wie} in satzadverbiellen Phrasen impliziert, daß wir es in dem mit \textit{wie} eingeleiteten Relativsatz bzw. in satzadverbiellen Hauptsätzen mit \textit{so} semantisch jeweils mit einer Small\hyp clause\hyp Konstruktion zu tun haben. Und zwar wird die propositionale Argumentstelle $p$ des betreffenden lexikalischen Kopfs des satzadverbiellen Satzes um eine Prädikatvariable $P$ zu $[ P p ]$ angereichert und die Argumentstruktur des betreffenden lexikalischen Kopfs um eine Argumentstelle für dieses Prädikat erweitert. Genau diese Stelle wird dann durch die SF der Spur von \textit{wie} (s. \REF{ex:zi97:47b}) oder des wie in \REF{ex:zi97:28} ohne Relativsatz auftretenden pronominalen Adverbs \textit{so} (s. \ref{ex:zi97:47a}.i bzw. \ref{ex:zi97:47a}.iii) gesättigt.

Dabei ist eine weitere Erscheinung zu beachten. Das propositionale Argument, auf das sich diese Metaprädikate beziehen, kann lexemabhängig pronominal ausgedrückt werden bzw. muß stumm bleiben. Ich deute das mit \textit{meinen} vs. \textit{vermuten} an, für die ich folgende Argumentstruktur und nicht weiter dekomponierte SF annehme:\footnote{Zu komplexen Lexikoneinträgen mit Booleschen Konditionen für die Konstruktionseigenschaften der jeweiligen Lexeme siehe \citet{bierwisch1996lexicalinformationminimalist}.}

\ea \label{ex:zi97:48} meinen' = \\
$(\lambda P)_\alpha \; (\lambda p)_{-\alpha} \; \lambda x \; \lambda e \; [e \; \cnst{inst} [x \; \textsc{meinen} [(P)_\alpha \; p]]]$

\ex \label{ex:zi97:49} vermuten' = \\
$(\lambda P)_\alpha \; (\lambda p) \; \lambda x \; \lambda e \; [e \; \cnst{inst} [x \; \textsc{vermuten} [(P)_\alpha \; p]]]$
\z

\noindent Die komplizierten Einzelheiten, die mit den Bedingungen für die Weglassung pronominaler Ausdrücke, besonders propositionaler, zusammenhängen, kann ich hier nicht verfolgen.\footnote{Vgl. in diesem Zusammenhang auch die Analyse von \citet{reis95, reis96a, reis96b} zu
Verben in V1-Parenthesen und Verben, die unselbständige V2-Sätze erlauben, wo jeweils die propositionale Argumentstelle blockiert ist.}

Ich gebe nun für die in \REF{ex:zi97:44}, \REF{ex:zi97:27} und \REF{ex:zi97:28} enthaltenen satzadverbiellen Phrasen die aus \REF{ex:zi97:47}, \REF{ex:zi97:48} und \REF{ex:zi97:49} resultierenden semantischen Repräsentationen an.

\ea \label{ex:zi97:50} (so) wie wir (es) vermuteten' = \\
$\lambda p [\forall P [\exists e' [e' \; \cnst{inst} [\textsc{wir} \; \textsc{vermuten} [P \; q]]]] \rightarrow [P \; p]] =$ \newline
so' \; (wie' (Satzmodus (vermuten' (Spur von \textit{wie}) (es')$_{\alpha}$ (wir')))) = \newline
$\lambda W \; \lambda p [\forall P [W \; P] \rightarrow [P \; p]] (\lambda p \; \lambda Q [p] (\lambda S [\exists e' [S \; e']] \newline (\lambda Q \; (\lambda q)_{\alpha} \; \lambda x \; \lambda e [e \; \cnst{inst} [x \; \textsc{vermuten} [Q \; q]]] (Q)(q)_{\alpha} \; (\textsc{wir}))))$
\z

\noindent Dieses auf propositionale Entitäten beziehbare komplexe Prädikat kann nun, wie in \REF{ex:zi97:44} und \REF{ex:zi97:45} angegeben, als satzadverbieller Operator bzw. als appositive Ergänzung in die Bedeutungsstruktur der einbettenden Konstruktion einbezogen werden. Bei appositiver Verwendung wird die Argumentstelle $\lambda p$ durch das Template \REF{ex:zi97:47b} blockiert.

Satzadverbielle Zusätze wie \textit{so meint Karl} in \REF{ex:zi97:28} und \textit{(so) wie Karl meint} in \REF{ex:zi97:27} haben folgende semantische Repräsentationen:

\ea \label{ex:zi97:51} so meint Karl' = \\
$[\exists e [e \; \cnst{inst} [\textsc{karl} \; \textsc{meinen} \; [P \; p]]]] =$ \newline 
Satzmodus (meinen' (so') (Karl')) = \newline
$\lambda Q [\exists e [Q\; e]] (\lambda P \; \lambda x \; \lambda e \; [e \; \cnst{inst} [x \; \textsc{meinen} [P \; p]]] (P) (\textsc{karl}))$
\z

\noindent Hier ist \textit{so} auf das propositionale Argument von \textit{meinen} bezogen und ohne abhängigen Relativsatz verwendet. Das propositionale Argument $p$ von \textit{meinen} ist in der satzadverbiellen Phrase \textit{so meint Karl} nicht pronominal ausgedrückt. Die Argumentstruktur dieses Verbs sieht das vor (s. \REF{ex:zi97:48}).

\REF{ex:zi97:52} zeigt die mit \REF{ex:zi97:51} bedeutungsgleiche Repräsentation eines appositiven \textit{wie}-Satzes ohne Korrelatbezug.

\ea \label{ex:zi97:52} wie Karl meint' = \\
$[\exists e [e \; \cnst{inst} [\textsc{karl} \; \textsc{meinen} [Q \; p]]]] =$ \newline
Template \REF{ex:zi97:47g} (wie' (Satzmodus (meinen' (Spur von \textit{wie}) (Karl')))) = \newline
$\lambda P [P \; y] (\lambda p \; \lambda Q [p] (\lambda S [\exists e [S \; e]] (\lambda P \; \lambda x \; \lambda e %\newline
[e \; \cnst{inst} [x \; \textsc{meinen} [P \; p]]] (Q) (\textsc{karl}))))$
\z

\noindent Auch \REF{ex:zi97:53} ist mit \REF{ex:zi97:51} bedeutungsgleich.

\ea \label{ex:zi97:53} so wie Karl meint' = \\
$[\exists e [e \; \cnst{inst} [\textsc{karl} \; \textsc{meinen} [P \; p]]]] =$ \newline
so' in (\ref{ex:zi97:47a}.ii) (wie' (Satzmodus (meinen' (Spur von \textit{wie}) (Karl')))) = \newline
$\lambda W [W \; P] (\lambda p \; \lambda Q [p] (\lambda S [\exists e [S \; e]] (\lambda P \; \lambda x \; \lambda e %\newline
[e \; \cnst{inst} [x \; \textsc{meinen} [P \; p]]] \newline (Q) (\textsc{karl}))))$
\z

\noindent Hier wird durch die für das Korrelat \textit{so} in (\ref{ex:zi97:47a}.ii) vorgesehene semantische Operation die durch \textit{wie} eingebrachte Argumentstelle $\lambda Q$ des Relativsatzes blockiert.

Sollen die in \REF{ex:zi97:51} und \REF{ex:zi97:52} in ihrer appositiven Funktion repräsentierten satzadverbiellen Phrasen \textit{wie Karl meint} bzw. \textit{so wie Karl meint} als Operatoren fungieren, müssen sie durch das Template \REF{ex:zi97:47f} $\lambda p$ als Argumentstelle erhalten. Dann können diese satzadverbiellen Phrasen nach dem Schema \REF{ex:zi97:41} als Operatorausdrücke in die SF des einbettenden Satzes integriert werden.

In \REF{ex:zi97:54} fasse ich die in diesem Abschnitt auf ihre Bausteine zurückgeführten satzadverbiellen Phrasen mit \textit{so} und \textit{wie} in verallgemeinerter Form zusammen. Die in \REF{ex:zi97:47} angeführten Bedeutungsanteile liefern also folgende Typen satzadverbieller Phrasen mit \textit{so} und \textit{wie}:\footnote{In \REF{ex:zi97:54c} und \REF{ex:zi97:54d} deutet das als fakultativ gekennzeichnete Adverb \textit{genau} an, daß \textit{so} und sein stummes Pendant in Vergleichskonstruktionen durch eine solche Angabe ergänzt werden können. Dabei fällt auf, daß diese Ergänzung nicht überall möglich ist. Vgl.: 
\ea Peter schläft, genau(so) wie wir es vermutet haben.
\ex *Peter schläft, genau(so) wie es scheint.
\ex *Die Philharmoniker sind auf Tournee, genau(so) wie Karl meint.
\z
Ich nehme an, daß die SF von \textit{so} in (\ref{ex:zi97:47a}.i) für die Integration der SF dieser Adverbien durch eine fakultative Argumentstelle und eine entsprechende Erweiterung der Prädikat-Argument-Struktur anzureichern ist. Für \textit{so} in (\ref{ex:zi97:47a}.ii) und (\ref{ex:zi97:47a}.iii) ist diese Möglichkeit nicht gegeben.}

\ea \label{ex:zi97:54}
    \ea \label{ex:zi97:54a} $[\exists e [e \; \cnst{inst} [... [P \; p]]]]$ \\
    so scheint es
    \ex \label{ex:zi97:54b} $(\lambda p)[\exists e [e \; \cnst{inst} [... [P \; p]]]]$ \\
    wie es scheint \\ 
    so wie es scheint
    \ex \label{ex:zi97:54c} $[\exists e [e \; \cnst{inst} [... [\forall P [W \; P] \rightarrow [P \; p ]]]]]$ \\
    (genau)so hatten wir es verabredet
    \ex \label{ex:zi97:54d} $(\lambda p) [\forall p [\exists e [e \; \cnst{inst} [... [P \; q]]]] \rightarrow [P \; p]]$ \\
    (genau)(so) wie wir es verabredet hatten
    \z
\z

\section{Ausblick} \label{sec:zi97:4}

Die für satzadverbielle Phrasen mit \textit{so} und \textit{wie} gegebene Analyse erfaßt eine große Zahl von systematisch bildbaren Konstruktionen, setzt diese zu satzadverbiellen Adjektivadverbphrasen, PPs und genitivischen DPs in paradigmatische Beziehung und behandelt \textit{so} und \textit{wie} -- sofern (\ref{ex:zi97:47a}.i) wirksam wird -- parallel zu Vergleichskonstruktionen ohne satzadverbiellen Status. Sie wirft aber auch Probleme auf, von denen einige zur Sprache kommen sollen.

Erstens fragt man sich, wieviele Bedeutungsvarianten für \textit{so} vorzusehen sind und worin sie übereinstimmen. Mindestens kann festgehalten werden, daß \textit{so} und die Spur von \textit{wie} Prädikatstatus haben und in den in der vorstehenden Analyse betrachteten satzadverbiellen Phrasen Bezug auf propositionale Entitäten haben. Es gibt jedoch auch Fälle wie \REF{ex:zi97:55} oder die Beispiele \REF{ex:zi97:39} und \REF{ex:zi97:40}, in denen im Relativsatz nicht auf eine propositionale Einheit, sondern auf ein Individuum Bezug genommen wird.

\ea \label{ex:zi97:55} So wie ich mir Peter vorstelle, müßtet ihr miteinander gut auskommen.
\z

\noindent Das macht eine gegenüber (\ref{ex:zi97:47a}.i) modifizierte Variante für die Bedeutung von \textit{so} erforderlich, nämlich \REF{ex:zi97:56}.

\ea \label{ex:zi97:56} $\lambda W \; \lambda p [[\forall P [W \; P]] \rightarrow p]$
\z

\noindent Auf diese Weise erhielte die satzadverbielle Phrase im Beispiel \REF{ex:zi97:40} folgende semantische Struktur:

\ea (so) wie Spezialisten die Situation einschätzen' $=$ \\
$\lambda p [[\forall P [\exists e' [e' \; \cnst{inst} [\text{Spezialisten'} \; \textsc{einschätzen} %\newline
[P \; \iota y [\textsc{situation} \; y]]]]]] \rightarrow p]$
\z

\noindent Das heißt, als Ganzes haben auch diese satzadverbiellen Phrasen mit \textit{so} und \textit{wie} -- wie generell Satzadverbiale -- eine propositionale Argumentstelle, $\lambda p$. Die propositionale Bezugsgröße $p$ liegt jedoch nicht im Skopus des Prädikats $P$, sondern dieses ist auf ein im Relativsatz genanntes Individuenargument zu beziehen. Entsprechend gehören $P$ und die Spur von \textit{wie} hier dem semantischen Typ S/N an. Wiederum besteht die Notwendigkeit einer genauen Analyse der Prädikatwörter, die diese Konstruktionsmöglichkeit haben. \textit{Sehen, verstehen, einschätzen, beurteilen, aussehen, sich anfühlen, riechen, sich anhören} gehören hierher.

Zweitens kommt die Frage auf, ob nicht auch für propositionsbezügliche Satzadverbiale mit \textit{so} und \textit{wie} von der SF \REF{ex:zi97:56} für \textit{so} auszugehen ist, derart daß $P$ und die Spur von \textit{wie} dem semantischen Typ S/$\alpha$ (mit $\alpha \in \{S,N\}$) angehören müßten. Allerdings scheint gerade die Anwesenheit von $P$ in der Prämisse und in der Konklusion von \textit{so} (s. \ref{ex:zi97:47a}.i) das zu sein, was den Vergleich der Beurteilung von Sachlagen in Konstruktionen mit satzadverbiellen Phrasen mit \textit{so} und \textit{wie} ausmacht. Ich erläutere das anhand von Beispielen, die \citet{pittner93, pittner95} diskutiert hat.

\ea \label{ex:zi97:58}
    \ea \label{ex:zi97:58a} Wir werden die Steuern nicht erhöhen, wie Kohl sagte.
    \ex \label{ex:zi97:58b} Wir werden die Steuern nicht erhöhen, so sagte Kohl.
    \ex \label{ex:zi97:58c} Kohl / Kohls Rede zufolge werden wir die Steuern nicht erhöhen.
    \z
\z

\noindent \REF{ex:zi97:58a} hat zwei Interpretationsmöglichkeiten. Die eine ist \REF{ex:zi97:58b} und \REF{ex:zi97:58c} analog, wo der Sprecher die Behauptung, daß wir die Steuern nicht erhöhen werden, Kohl zuschreibt. In dieser Interpretation ist der \textit{wie}-Satz meiner Analyse zufolge appositiv und ohne Korrelatbezug verwendet (s. \REF{ex:zi97:52}). Die andere Interpretation von \REF{ex:zi97:58a}, die für \REF{ex:zi97:58b} und \REF{ex:zi97:58c} ausscheidet, liegt darin, daß der Sprecher und Kohl in der Beurteilung der ins Auge gefaßten Sachlage einer möglichen Steuerhöhung nicht übereinstimmen. Aus dieser durch die akzentuierte Negation ausgedrückten Nichtübereinstimmung läßt sich dann ableiten, daß -- wie Pittner feststellt -- Kohl nicht zu der mit \textit{wir} bezeichneten Personenmenge gehört und daß die Negation nicht zum Inhalt von Kohls Äußerung zu rechnen ist. Für solche Fälle nehme ich an, daß satzadverbielle Phrasen mit \textit{so} bzw. mit dessen stummem Pendant und \textit{wie} die Negation nicht in ihrem Skopus haben. Auf der Basis von (\ref{ex:zi97:47a}.i) ergäbe sich dann für das Satzadverbial in \REF{ex:zi97:58a} zusammen mit der Negation die semantische Repräsentation \REF{ex:zi97:59}.\footnote{`...' steht in \REF{ex:zi97:59} abkürzend für die Repräsentation der Bezugsdomäne des Satzadverbials. Es ist die SF der VP des Hauptsatzes von \REF{ex:zi97:58a} ohne Satzmodus und Negation; $p$ und $q$ sind in \REF{ex:zi97:59} als referenzidentisch anzusehen.}

\ea \label{ex:zi97:59} nicht so wie Kohl sagt ...' = \\
$\lambda e [\sim [\forall P [\exists e' [e' \; \cnst{inst} [\textsc{kohl} \; \text{sagen'} [P \; q]]]] \rightarrow %\newline 
[P \; ...]]] = \newline
\lambda e [\exists P [\exists e' [e' \; \cnst{inst} [\textsc{kohl} \; \text{sagen'} [P \; q]]]] \wedge %\newline
[\sim [P \; ...]]]$
\z

\noindent Wenn man sich vorstellt, daß die Prädikatvariable $P$ durch das Metaprädikat \textsc{gültig} belegt wird, dürfte \REF{ex:zi97:59} nicht als abwegig gelten.

Satzadverbiale wie \textit{Kohls Rede zufolge} oder \textit{so sagte Kohl} beinhalten einen solchen Vergleich, wie er für satzadverbielle Phrasen mit \textit{so} bzw. seinem stummen Pendant und \textit{wie} typisch ist, nicht. Das berücksichtigt die Repräsentation in \REF{ex:zi97:51}.

Drittens fragt sich, wie satzadverbielle Präpositionalphrasen und besonders die Präposition in ihnen semantisch zu repräsentieren wären und inwieweit sie mit den satzadverbiellen Phrasen mit \textit{so} und \textit{wie} korrespondieren. Auch Wortbildungen sind zu berücksichtigen. Vgl.:

\ea \label{ex:zi97:60}
    \ea (so) wie Karl meint \label{ex:zi97:60a}
    \ex nach Karls Meinung \label{ex:zi97:60b}
    \z
\ex \label{ex:zi97:61}
    \ea (so) wie wir (es) erwarteten
    \ex gemäß unserer Erwartung
    \ex erwartungsgemäß
    \z
\z

\noindent Es ist zu prüfen, ob die SF von \textit{nach}, \textit{gemäß}, \textit{zufolge} usw. so strukturiert ist, daß -- wie in \REF{ex:zi97:56} für \textit{so} angenommen -- eine zweistellige Relation mit $p$ als Folgerung vorliegt. \REF{ex:zi97:62} deutet das an.

\ea \label{ex:zi97:62} nach', gemäß', zufolge' = \\
$\lambda x \lambda p [[... \; x \; ...] \rightarrow p]$
\z

Die Argumentstelle $\lambda x$ wäre dann durch die SF von DPs wie \textit{Karls Meinung, unsere Erwartung, eine Mitteilung in der New York Times} oder einfach \textit{die New York Times, Karl} zu spezifizieren. Dabei müßten bei der bloßen Erwähnung von Informationsträgern wie \textit{Karl} oder \textit{die New York Times} durch ein geeignetes Template bestimmte semantische Anreicherungen vorgenommen werden. Vor allem aber ist zu klären, wie die in \REF{ex:zi97:62} offen gelassenen, durch `...' gekennzeichneten Bedeutungsanteile zu charakterisieren wären. Möglicherweise ist statt \REF{ex:zi97:62} besser \REF{ex:zi97:63} anzunehmen.

\ea \label{ex:zi97:63} $\lambda x \lambda p \; [p \; \textsc{korrespondieren mit} \; x]$
\z

\noindent Dann müssen im konzeptuellen System gegebenenfalls Äquivalenzbeziehungen zwischen synonymen Ausdrucksvarianten satzadverbieller Zusätze hergestellt werden.

Viertens: Alle Satzadverbiale haben gemeinsam, nicht zu denjenigen Konstituenten des einbettenden Satzes zu gehören, die dessen Referenten charakterisieren. Darin stimmen die bisher betrachteten Satzadverbiale mit solchen Satzergänzungen wie in \REF{ex:zi97:64} und \REF{ex:zi97:65} überein.

\ea \label{ex:zi97:64} %I m\;  B ü r o 
\textit{Im Büro} ist Renate zuverlässig.
\ex \label{ex:zi97:65} %A l s\;  S e k r e t ä r i n 
\textit{Als Sekretärin} ist Renate zuverlässig.
\z

\noindent Es ist ohne weiteres möglich, zusätzlich zu den durch Kursivsetzung %Sperrung 
gekennzeichneten Phrasen Satzadverbiale wie \textit{wahrscheinlich}, \textit{vermutlich}, \textit{bekanntlich}, \textit{(so) wie allgemein bekannt} oder \textit{(so) wie Karl meint} zu verwenden. Vgl. auch \REF{ex:zi97::66} und \REF{ex:zi97::67}, wo zwei satzadverbielle Phrasen zusammen auftreten:

\ea\label{ex:zi97::66} So wie es aussieht, wird es vermutlich gleich regnen.
\ex\label{ex:zi97::67} So wie meine Nachbarin sagt, kommt die Postbotin bestimmt noch.
\z 

\noindent Es fragt sich angesichts all dieser Zusätze, in welcher paradigmatischen und syntagmatischen Beziehung sie zueinander stehen.\footnote{Zu den Skopusverhältnissen mehrerer Satzadverbiale siehe \citet{bartsch1972adverbialsemantik}, \citet{lang1979zum-status-der-satzadverbiale, lang1983einstellungsausdrucke}, \citet{hetland1992satzadverbienimfokus}.} Dabei muß die Analyse in Betracht gezogen werden, die \citet{maienborn1996lokalesatzadverbiale,maienborn1996situation} für Lokalangaben wie in \REF{ex:zi97:64} gegeben hat. Maienborn nimmt an, daß solche Adverbiale sich semantisch auf eine Bezugssituation beziehen, die den Referenzrahmen für die VP-Bedeutung bildet und Tempus- und Modusspezifizierungen zu integrieren gestattet. Es ist zu klären, wo in diesem System die verschiedenen Satzadverbialtypen ihren Platz haben und wie die mit \textit{so} und \textit{wie} eingeleiteten Phrasen unterzubringen wären.

Fünftens: Ein schwerwiegendes Problem ist, wie sich die satzadverbiellen Phrasen bezüglich der Wahrheitsbedingungen für die jeweilige Konstruktion verhalten. Ich illustriere die Problematik anhand von \REF{ex:zi97:68}.

\ea \label{ex:zi97:68} 
   \ea \label{ex:zi97:68a} Schulzes sind vermutlich / wie ich vermute / so meint Peter verreist.
   \ex \label{ex:zi97:68b} Das glaube ich nicht.
   \z
\z

\noindent Ganz gleich, ob als Operatorausdruck, als Parenthese oder als Apposition verwendet, die satzadverbielle Phrase ist nicht Gegenstand der in \REF{ex:zi97:68b} ausgedrückten Stellungnahme zum Zutreffen der in \REF{ex:zi97:68a} enthaltenen Mitteilung, daß Schulzes verreist sind. Wie und auf welcher Ebene das zu erfassen ist, ist unklar. Haben wir in den Satzadverbialia vielleicht semantisch nicht-integrierbare subsidiäre Information über die die Hauptmitteilung begleitende Einstellung vor uns? Ist gegebenenfalls nur mit dem in \REF{ex:zi97:45} angegebenen Schema für die appositive Verknüpfung der SF der einbettenden Konstruktion und der satzadverbiellen Phrase zu rechnen? Oder ist statt \REF{ex:zi97:41} besser (39') als Amalgamierungsschema für die SF der Satzadverbiale mit Operatorfunktion und der SF ihrer Bezugskonstituente vorzusehen, derart daß der mit A indizierte Teil der resultierenden SF als Präsupposition gilt?  (40') wäre dann die revidierte Fassung von \REF{ex:zi97:42}.

%exp{ex:nouns}\label{ex:zi97::nounsprime} .... \REF{ex:zi97::nounsprime}

\begin{exe}
\exp{ex:zi97:41}{Die Amalgamierung von Satzadverbial und FP-Bedeutung \\
$\lambda p \; [... \; p \; ...]_{A} \; : \; [p] (\lambda e \; [ ... [e \; \cnst{inst} \; [...]]...])$\label{ex:zi97:41'}}
\exp{ex:zi97:42}{\label{ex:zi97:42'} Peter schläft vermutlich.' $=$ \\
$[\exists e \; [e' \; \cnst{inst} [x \; \textsc{vermuten} \; [e \; \cnst{inst} \; [\textsc{schlafen} \; %\newline 
\textsc{peter}]]]]\; : \newline \; [e \; \cnst{inst} \; [\textsc{schlafen} \; \textsc{peter}]]] \; =$ \newline Satzmodus (vermutlich' (schlafen' (Peter'))) $=$ \newline $\lambda Q [\exists e [ Q\; e ]] (\lambda p [e'\ \cnst{inst} [ x \; \textsc{vermuten} \; p ]] : \newline [p] (\lambda y \lambda e [e \; \cnst{inst} [\textsc{schlafen} \; y]] (\textsc{peter})))$}
\end{exe}

\noindent Ganz analog wären satzadverbielle Phrasen mit \textit{so} und \textit{wie} zu integrieren. Ich betrachte diese Alternative zu dem Schema \REF{ex:zi97:41} nicht als abwegig, überlasse aber eine Entscheidung der weiteren Forschung zur Semantik der funktionalen Strukturdomänen von Sätzen.

\largerpage[2]
Sechstens: Dennoch halte ich die für \textit{so} und \textit{wie} und die interne Organisation der satzadverbiellen Phrasen mit diesen Formativen gegebene Analyse für triftig. Wie oben schon gesagt, ist es ein Desiderat, die Prädikatwörter näher zu untersuchen, die in satzadverbiellen Phrasen mit \textit{so} bzw. seinem stummen Pendant und mit \textit{wie} als Einleitung verträglich sind (s. die offene Liste solcher Lexeme im Anhang).\footnote{Siehe auch \citet{brandt1997derredesituierendewiesatz}, \citet{sitta70}, \citet{eggers1972partikelwie} und \citet{cinque1989embedded, cinque1990two}. Vgl. auch die für V1-Parenthesen wie in \textit{Wo glaubt Maria läuft die Ausstellung?} und für abhängige V2-Sätze wie in \textit{Maria glaubt, die Ausstellung läuft in Köln.} tauglichen Prädikatausdrücke. Siehe dazu \citet{reis95, reis96a, reis96b}. Offenbar gibt es einen großen Überschneidungsbereich dieser Prädikatausdrücke mit denen, die satzadverbielle Phrasen mit \textit{so} und \textit{wie} zulassen. Es existieren aber auch signifikante Unterschiede. Beispielsweise sind Gewißheitsprädikate wie \textit{überzeugt, sicher, feststehen} und Präferenzprädikate wie \textit{vorziehen, besser, das beste} weder mit \textit{so} und \textit{wie} vereinbar noch in V1-Parenthesen möglich, wohl aber mit einem abhängigen V2-Satz. Normausdrücke mit \textit{üblich, Regel, sich gehören} wiederum sind mit \textit{so} und \textit{wie} kombinierbar, treten aber nicht in V1-Parenthesen oder mit einem V2-Satz auf.} \citet{cinque1989embedded, cinque1990two} kommt zu der wesentlichen, nicht nur fürs Italienische geltenden Generalisierung, daß die Satzeinleitung \textit{come} ,wie' in satzadverbiellen Phrasen nur mit solchen Verben, Adjektiven und Substantiven verträglich ist, die eine Argumentstelle für eine CP haben, und zwar in der Po\-si\-tion des direkten Objekts, wenn es sich um transitive Verben wie \textit{sperare} ,hoffen', \textit{dire} ,sagen' usw. handelt, oder in der Po\-si\-tion des Subjekts von ergativen Prädikatwörtern wie \textit{succedere} ,vorkommen', \textit{prevedibile} ,vorhersehbar' usw. Neben dieser Bedingung ist allerdings zu beachten, daß Faktivität der Prädikatwörter die Bildung satzadverbieller Phrasen mit \textit{so} und \textit{wie} bzw. mit \textit{come} beeinflußt. Nur kognitive faktive Prädikate wie \textit{sapere} ,wissen', \textit{noto} ,bekannt' u.a. erlauben die Konstruktion. Ferner gilt, wie die Beispiele \REF{ex:zi97:39}, \REF{ex:zi97:40} und \REF{ex:zi97:55} zeigen, nicht für alle satzadverbiellen Phrasen mit \textit{so} und \textit{wie} die Forderung, daß der lexikalische Kopf der Konstruktion eine Argumentstelle für eine CP haben müsse. Eine genaue Inspektion der mit \textit{so} und \textit{wie} bzw. mit \textit{come} verträglichen Prädikatwörter ist nötig. Auch ein Sprachenvergleich ist lohnend.\footnote{Siehe dazu auch \citet{brandt1997derredesituierendewiesatz}.}

Siebentens: Ein augenfälliges offenes Problem ist die Bestimmung der Bezugsdomäne bzw. des Variationsbereichs der in den angeführten semantischen Repräsentationen in Gestalt freier Variable figurierenden Parameter. Wie wird garantiert, daß \textit{so} einerseits einen Variationsspielraum über Gütigkeitsprädikate wie \textit{gültig}, \textit{zutreffen} etc. hat, mit \textit{anders} und \textit{ähnlich} korrespondiert und bei Individuenbezug auf belangvolle Eigenschaften verweist? Wie wird verdeutlicht, daß nicht spezifizierte Einstellungsträger z.B. bei \textit{vermutlich, (so) wie bekannt, (so) wie erwartet} etc. den jeweiligen Sprecher einschließen? Auf welcher Ebene erfolgt die Identischsetzung referenzidentischer Variablen? Welche Präferenzen und Beschränkungen gibt es für die Belegung von Variablen? Offenbar spielen auch selektionelle Zusammenhänge eine Rolle. Beispielsweise korrespondiert mit der Eigenschaft, daß \textit{verabreden} keinen Interrogativ- oder Imperativsatz einbettet, die Tatsache, daß in den Beispielen \REF{ex:zi97:19}-\REF{ex:zi97:21} Interrogativität bzw. Imperativität des einbettenden Satzes jenseits der semantischen Reichweite der satzadverbiellen Ergänzung liegen. Ebenso kann ein passender Bezug für eine satzadverbielle Phrase wie \textit{so findet Peter, (so) wie Peter findet} nur eine relevante Information beinhaltende Proposition sein. Vgl.:

\ea \label{ex:zi97:69} Das Kleid ist zu teuer, so wie Peter findet.
\ex[*] {Das Kleid kostet 80 DM, so wie Peter findet.} \label{ex:zi97:70}
\z

\noindent Es gehört -- wie \citet{reis1993wer-findet-wird-suchen} entdeckt hat -- zu den selektionellen Verwendungsbedingungen von \textit{finden}, daß sein propositionales Argument nichttriviale Information enthält.

\largerpage
Obwohl \textit{so}, \textit{wie}, \textit{das} und \textit{es} als pronominale Ausdrücke für mein Thema konstitutive Bausteine sind, habe ich zu Referenzeigenschaften von Pronomina nichts Neues gesagt. Hauptthesen dieser Arbeit sind, daß wir \textit{wie} in den hier untersuchten satzadverbiellen Phrasen als relativisches Adjektivadverb anzusehen haben und daß \textit{so} ein multivalenter Aufhänger für den mit \textit{wie} eingeleiteten Relativsatz ist.

\section*{Anhang} \label{anhang}

Offene Liste der mit \textit{so} und \textit{wie} in satzadverbiellen Phrasen verträglichen lexikalischen Köpfe.

\begin{longtable}{llll}
    sagen & meinen & hören & wollen \\
    schreiben & finden & lesen & wünschen \\
    mitteilen & denken & erfahren & erwarten \\
    bekanntgeben & glauben & sich zeigen & (er)hoffen \\
    feststellen & annehmen & sich offenbaren & ersehnen \\
    behaupten & vermuten & sich herausstellen & fordern \\
    erklären & mutmaßen & sich abzeichnen & verlangen \\
    erwähnen & scheinen & deutlich werden & raten \\
    angeben & vorkommen & ersichtlich & empfehlen \\
    andeuten & aussehen & verlauten & vorschlagen \\
    bemerken & sich vorstellen & hervorgehen & planen \\
    melden & einsehen & heißen & vereinbaren \\
    berichten & verstehen & {} & verabreden \\
    darstellen & begreifen & sich ereignen & befürchten \\
    schildern & beurteilen & sich zutragen & {} \\
    beschreiben & einschätzen & geschehen & {} \\
%\end{tabular}
%\begin{tabular}{llll}
    formulieren & sehen & {} & {} \\
    sich ausdrücken & wissen & tun & {} \\
    argumentieren & bekannt & praktizieren & {} \\
    kommentieren & erwiesen & verfahren & {} \\
    erläutern & offensichtlich & vorgehen & {} \\
    ergänzen & offenkundig & {} & {} \\
    hinzufügen & {} & {} & {} \\
    hervorheben & üblich & {} & {} \\
    unterstreichen & Usus & {} & {} \\
    betonen & Regel & {} & {} \\
    antworten & Norm & {} & {} \\
    einräumen & Mode & {} & {} \\
    stottern & Sitte & {} & {} \\
    zuflüstern & sich gehören & {} & {} \\
    orakeln & sich geziemen & {} & {} \\
    prophezeien & sich gebühren & {} & {} \\
    weissagen & vorsehen & {} & {} \\
    voraussagen & vorschreiben & {} & {} \\
\end{longtable}

\section*{Anmerkungen der Herausgeber}
Bei Beispielen aus anderen Sprachen als dem Deutschen haben wir Glossen und Übersetzungen ergänzt, wo sie fehlten.

\section*{Abkürzungen}
\begin{tabularx}{.5\textwidth}{@{}lQ}
\textsc{1}&erste Person\\
\textsc{2}&zweite Person\\
\end{tabularx}%
\begin{tabularx}{.5\textwidth}{lQ@{}}
\textsc{präs}&Präsens\\
\textsc{sg}&Singular\\
%&\\ % this dummy row achieves correct vertical alignment of both tables
\end{tabularx}

\printbibliography[heading=subbibliography,notkeyword=this]
\end{otherlanguage}

\end{document}
