\documentclass[output=paper,colorlinks,citecolor=brown, booklanguage=german]{langscibook}
\ChapterDOI{10.5281/zenodo.15471427}
% \bibliography{localbibliography}

\author{Ilse Zimmermann\affiliation{Zentralinstitut für Sprachwissenschaft der Akademie der Wissenschaften der DDR, Berlin}}
% replace the above with you and your coauthors
% orcid doesn't appear printed; it's metainformation used for later indexing

%%% uncomment the following line if you are a single author or all authors have the same affiliation
% \SetupAffiliations{mark style=none}

%% in case the running head with authors exceeds one line (which is the case in this example document), use one of the following methods to turn it into a single line; otherwise comment the line below out with % and ignore it
%\lehead{Šimík, Gehrke, Lenertová, Meyer, Szucsich \& Zaleska}
% \lehead{Radek Šimík et al.}

\title{Wohin mit den Affixen?}
% replace the above with your paper title
%%% provide a shorter version of your title in case it doesn't fit a single line in the running head
% in this form: \title[short title]{full title}
\abstract{\noabstract}

% \usepackage{langsci-optional}
\usepackage{langsci-gb4e}
\usepackage{langsci-lgr}

\usepackage{listings}
\lstset{basicstyle=\ttfamily,tabsize=2,breaklines=true}

%added by author
% \usepackage{tipa}
\usepackage{multirow}
\graphicspath{{figures/}}
\usepackage{langsci-branding}

% 
\newcommand{\sent}{\enumsentence}
\newcommand{\sents}{\eenumsentence}
\let\citeasnoun\citet

\renewcommand{\lsCoverTitleFont}[1]{\sffamily\addfontfeatures{Scale=MatchUppercase}\fontsize{44pt}{16mm}\selectfont #1}
  

% \togglepaper[42]
% the chapter number will be provided by volume editors; for now keep this way

\begin{document}
\begin{otherlanguage}{german}
\maketitle

% Just comment out the input below when you're ready to go.
%For a start: Do not forget to give your Overleaf project (this paper) a recognizable name. This one could be called, for instance, Simik et al: OSL template. You can change the name of the project by hovering over the gray title at the top of this page and clicking on the pencil icon.

\section{Introduction}\label{sim:sec:intro}

Language Science Press is a project run for linguists, but also by linguists. You are part of that and we rely on your collaboration to get at the desired result. Publishing with LangSci Press might mean a bit more work for the author (and for the volume editor), esp. for the less experienced ones, but it also gives you much more control of the process and it is rewarding to see the quality result.

Please follow the instructions below closely, it will save the volume editors, the series editors, and you alike a lot of time.

\sloppy This stylesheet is a further specification of three more general sources: (i) the Leipzig glossing rules \citep{leipzig-glossing-rules}, (ii) the generic style rules for linguistics (\url{https://www.eva.mpg.de/fileadmin/content_files/staff/haspelmt/pdf/GenericStyleRules.pdf}), and (iii) the Language Science Press guidelines \citep{Nordhoff.Muller2021}.\footnote{Notice the way in-text numbered lists should be written -- using small Roman numbers enclosed in brackets.} It is advisable to go through these before you start writing. Most of the general rules are not repeated here.\footnote{Do not worry about the colors of references and links. They are there to make the editorial process easier and will disappear prior to official publication.}

Please spend some time reading through these and the more general instructions. Your 30 minutes on this is likely to save you and us hours of additional work. Do not hesitate to contact the editors if you have any questions.

\section{Illustrating OSL commands and conventions}\label{sim:sec:osl-comm}

Below I illustrate the use of a number of commands defined in langsci-osl.tex (see the styles folder).

\subsection{Typesetting semantics}\label{sim:sec:sem}

See below for some examples of how to typeset semantic formulas. The examples also show the use of the sib-command (= ``semantic interpretation brackets''). Notice also the the use of the dummy curly brackets in \REF{sim:ex:quant}. They ensure that the spacing around the equation symbol is correct. 

\ea \ea \sib{dog}$^g=\textsc{dog}=\lambda x[\textsc{dog}(x)]$\label{sim:ex:dog}
\ex \sib{Some dog bit every boy}${}=\exists x[\textsc{dog}(x)\wedge\forall y[\textsc{boy}(y)\rightarrow \textsc{bit}(x,y)]]$\label{sim:ex:quant}
\z\z

\noindent Use noindent after example environments (but not after floats like tables or figures).

And here's a macro for semantic type brackets: The expression \textit{dog} is of type $\stb{e,t}$. Don't forget to place the whole type formula into a math-environment. An example of a more complex type, such as the one of \textit{every}: $\stb{s,\stb{\stb{e,t},\stb{e,t}}}$. You can of course also use the type in a subscript: dog$_{\stb{e,t}}$

We distinguish between metalinguistic constants that are translations of object language, which are typeset using small caps, see \REF{sim:ex:dog}, and logical constants. See the contrast in \REF{sim:ex:speaker}, where \textsc{speaker} (= serif) in \REF{sim:ex:speaker-a} is the denotation of the word \textit{speaker}, and \cnst{speaker} (= sans-serif) in \REF{sim:ex:speaker-b} is the function that maps the context $c$ to the speaker in that context.\footnote{Notice that both types of small caps are automatically turned into text-style, even if used in a math-environment. This enables you to use math throughout.}$^,$\footnote{Notice also that the notation entails the ``direct translation'' system from natural language to metalanguage, as entertained e.g. in \citet{Heim.Kratzer1998}. Feel free to devise your own notation when relying on the ``indirect translation'' system (see, e.g., \citealt{Coppock.Champollion2022}).}

\ea\label{sim:ex:speaker}
\ea \sib{The speaker is drunk}$^{g,c}=\textsc{drunk}\big(\iota x\,\textsc{speaker}(x)\big)$\label{sim:ex:speaker-a}
\ex \sib{I am drunk}$^{g,c}=\textsc{drunk}\big(\cnst{speaker}(c)\big)$\label{sim:ex:speaker-b}
\z\z

\noindent Notice that with more complex formulas, you can use bigger brackets indicating scope, cf. $($ vs. $\big($, as used in \REF{sim:ex:speaker}. Also notice the use of backslash plus comma, which produces additional space in math-environment.

\subsection{Examples and the minsp command}

Try to keep examples simple. But if you need to pack more information into an example or include more alternatives, you can resort to various brackets or slashes. For that, you will find the minsp-command useful. It works as follows:

\ea\label{sim:ex:german-verbs}\gll Hans \minsp{\{} schläft / schlief / \minsp{*} schlafen\}.\\
Hans {} sleeps {} slept {} {} sleep.\textsc{inf}\\
\glt `Hans \{sleeps / slept\}.'
\z

\noindent If you use the command, glosses will be aligned with the corresponding object language elements correctly. Notice also that brackets etc. do not receive their own gloss. Simply use closed curly brackets as the placeholder.

The minsp-command is not needed for grammaticality judgments used for the whole sentence. For that, use the native langsci-gb4e method instead, as illustrated below:

\ea[*]{\gll Das sein ungrammatisch.\\
that be.\textsc{inf} ungrammatical\\
\glt Intended: `This is ungrammatical.'\hfill (German)\label{sim:ex:ungram}}
\z

\noindent Also notice that translations should never be ungrammatical. If the original is ungrammatical, provide the intended interpretation in idiomatic English.

If you want to indicate the language and/or the source of the example, place this on the right margin of the translation line. Schematic information about relevant linguistic properties of the examples should be placed on the line of the example, as indicated below.

\ea\label{sim:ex:bailyn}\gll \minsp{[} Ėtu knigu] čitaet Ivan \minsp{(} často).\\
{} this book.{\ACC} read.{\PRS.3\SG} Ivan.{\NOM} {} often\\\hfill O-V-S-Adv
\glt `Ivan reads this book (often).'\hfill (Russian; \citealt[4]{Bailyn2004})
\z

\noindent Finally, notice that you can use the gloss macros for typing grammatical glosses, defined in langsci-lgr.sty. Place curly brackets around them.

\subsection{Citation commands and macros}

You can make your life easier if you use the following citation commands and macros (see code):

\begin{itemize}
    \item \citealt{Bailyn2004}: no brackets
    \item \citet{Bailyn2004}: year in brackets
    \item \citep{Bailyn2004}: everything in brackets
    \item \citepossalt{Bailyn2004}: possessive
    \item \citeposst{Bailyn2004}: possessive with year in brackets
\end{itemize}

\section{Trees}\label{s:tree}

Use the forest package for trees and place trees in a figure environment. \figref{sim:fig:CP} shows a simple example.\footnote{See \citet{VandenWyngaerd2017} for a simple and useful quickstart guide for the forest package.} Notice that figure (and table) environments are so-called floating environments. {\LaTeX} determines the position of the figure/table on the page, so it can appear elsewhere than where it appears in the code. This is not a bug, it is a property. Also for this reason, do not refer to figures/tables by using phrases like ``the table below''. Always use tabref/figref. If your terminal nodes represent object language, then these should essentially correspond to glosses, not to the original. For this reason, we recommend including an explicit example which corresponds to the tree, in this particular case \REF{sim:ex:czech-for-tree}.

\ea\label{sim:ex:czech-for-tree}\gll Co se řidič snažil dělat?\\
what {\REFL} driver try.{\PTCP.\SG.\MASC} do.{\INF}\\
\glt `What did the driver try to do?'
\z

\begin{figure}[ht]
% the [ht] option means that you prefer the placement of the figure HERE (=h) and if HERE is not possible, you prefer the TOP (=t) of a page
% \centering
    \begin{forest}
    for tree={s sep=1cm, inner sep=0, l=0}
    [CP
        [DP
            [what, roof, name=what]
        ]
        [C$'$
            [C
                [\textsc{refl}]
            ]
            [TP
                [DP
                    [driver, roof]
                ]
                [T$'$
                    [T [{[past]}]]
                    [VP
                        [V
                            [tried]
                        ]
                        [VP, s sep=2.2cm
                            [V
                                [do.\textsc{inf}]
                            ]
                            [t\textsubscript{what}, name=trace-what]
                        ]
                    ]
                ]
            ]
        ]
    ]
    \draw[->,overlay] (trace-what) to[out=south west, in=south, looseness=1.1] (what);
    % the overlay option avoids making the bounding box of the tree too large
    % the looseness option defines the looseness of the arrow (default = 1)
    \end{forest}
    \vspace{3ex} % extra vspace is added here because the arrow goes too deep to the caption; avoid such manual tweaking as much as possible; here it's necessary
    \caption{Proposed syntactic representation of \REF{sim:ex:czech-for-tree}}
    \label{sim:fig:CP}
\end{figure}

Do not use noindent after figures or tables (as you do after examples). Cases like these (where the noindent ends up missing) will be handled by the editors prior to publication.

\section{Italics, boldface, small caps, underlining, quotes}

See \citet{Nordhoff.Muller2021} for that. In short:

\begin{itemize}
    \item No boldface anywhere.
    \item No underlining anywhere (unless for very specific and well-defined technical notation; consult with editors).
    \item Small caps used for (i) introducing terms that are important for the paper (small-cap the term just ones, at a place where it is characterized/defined); (ii) metalinguistic translations of object-language expressions in semantic formulas (see \sectref{sim:sec:sem}); (iii) selected technical notions.
    \item Italics for object-language within text; exceptionally for emphasis/contrast.
    \item Single quotes: for translations/interpretations
    \item Double quotes: scare quotes; quotations of chunks of text.
\end{itemize}

\section{Cross-referencing}

Label examples, sections, tables, figures, possibly footnotes (by using the label macro). The name of the label is up to you, but it is good practice to follow this template: article-code:reference-type:unique-label. E.g. sim:ex:german would be a proper name for a reference within this paper (sim = short for the author(s); ex = example reference; german = unique name of that example).

\section{Syntactic notation}

Syntactic categories (N, D, V, etc.) are written with initial capital letters. This also holds for categories named with multiple letters, e.g. Foc, Top, Num, etc. Stick to this convention also when coming up with ad hoc categories, e.g. Cl (for clitic or classifier).

An exception from this rule are ``little'' categories, which are written with italics: \textit{v}, \textit{n}, \textit{v}P, etc.

Bar-levels must be typeset with bars/primes, not with an apostrophe. An easy way to do that is to use mathmode for the bar: C$'$, Foc$'$, etc.

Specifiers should be written this way: SpecCP, Spec\textit{v}P.

Features should be surrounded by square brackets, e.g., [past]. If you use plus and minus, be sure that these actually are plus and minus, and not e.g. a hyphen. Mathmode can help with that: [$+$sg], [$-$sg], [$\pm$sg]. See \sectref{sim:sec:hyphens-etc} for related information.

\section{Footnotes}

Absolutely avoid long footnotes. A footnote should not be longer than, say, {20\%} of the page. If you feel like you need a long footnote, make an explicit digression in the main body of the text.

Footnotes should always be placed at the end of whole sentences. Formulate the footnote in such a way that this is possible. Footnotes should always go after punctuation marks, never before. Do not place footnotes after individual words. Do not place footnotes in examples, tables, etc. If you have an urge to do that, place the footnote to the text that explains the example, table, etc.

Footnotes should always be formulated as full, self-standing sentences.

\section{Tables}

For your tables use the table plus tabularx environments. The tabularx environment lets you (and requires you in fact) to specify the width of the table and defines the X column (left-alignment) and the Y column (right-alignment). All X/Y columns will have the same width and together they will fill out the width of the rest of the table -- counting out all non-X/Y columns.

Always include a meaningful caption. The caption is designed to appear on top of the table, no matter where you place it in the code. Do not try to tweak with this. Tables are delimited with lsptoprule at the top and lspbottomrule at the bottom. The header is delimited from the rest with midrule. Vertical lines in tables are banned. An example is provided in \tabref{sim:tab:frequencies}. See \citet{Nordhoff.Muller2021} for more information. If you are typesetting a very complex table or your table is too large to fit the page, do not hesitate to ask the editors for help.

\begin{table}
\caption{Frequencies of word classes}
\label{sim:tab:frequencies}
 \begin{tabularx}{.77\textwidth}{lYYYY} %.77 indicates that the table will take up 77% of the textwidth
  \lsptoprule
            & nouns & verbs  & adjectives & adverbs\\
  \midrule
  absolute  &   12  &    34  &    23      & 13\\
  relative  &   3.1 &   8.9  &    5.7     & 3.2\\
  \lspbottomrule
 \end{tabularx}
\end{table}

\section{Figures}

Figures must have a good quality. If you use pictorial figures, consult the editors early on to see if the quality and format of your figure is sufficient. If you use simple barplots, you can use the barplot environment (defined in langsci-osl.sty). See \figref{sim:fig:barplot} for an example. The barplot environment has 5 arguments: 1. x-axis desription, 2. y-axis description, 3. width (relative to textwidth), 4. x-tick descriptions, 5. x-ticks plus y-values.

\begin{figure}
    \centering
    \barplot{Type of meal}{Times selected}{0.6}{Bread,Soup,Pizza}%
    {
    (Bread,61)
    (Soup,12)
    (Pizza,8)
    }
    \caption{A barplot example}
    \label{sim:fig:barplot}
\end{figure}

The barplot environment builds on the tikzpicture plus axis environments of the pgfplots package. It can be customized in various ways. \figref{sim:fig:complex-barplot} shows a more complex example.

\begin{figure}
  \begin{tikzpicture}
    \begin{axis}[
	xlabel={Level of \textsc{uniq/max}},  
	ylabel={Proportion of $\textsf{subj}\prec\textsf{pred}$}, 
	axis lines*=left, 
        width  = .6\textwidth,
	height = 5cm,
    	nodes near coords, 
    % 	nodes near coords style={text=black},
    	every node near coord/.append style={font=\tiny},
        nodes near coords align={vertical},
	ymin=0,
	ymax=1,
	ytick distance=.2,
	xtick=data,
	ylabel near ticks,
	x tick label style={font=\sffamily},
	ybar=5pt,
	legend pos=outer north east,
	enlarge x limits=0.3,
	symbolic x coords={+u/m, \textminus u/m},
	]
	\addplot[fill=red!30,draw=none] coordinates {
	    (+u/m,0.91)
        (\textminus u/m,0.84)
	};
	\addplot[fill=red,draw=none] coordinates {
	    (+u/m,0.80)
        (\textminus u/m,0.87)
	};
	\legend{\textsf{sg}, \textsf{pl}}
    \end{axis} 
  \end{tikzpicture} 
    \caption{Results divided by \textsc{number}}
    \label{sim:fig:complex-barplot}
\end{figure}

\section{Hyphens, dashes, minuses, math/logical operators}\label{sim:sec:hyphens-etc}

Be careful to distinguish between hyphens (-), dashes (--), and the minus sign ($-$). For in-text appositions, use only en-dashes -- as done here -- with spaces around. Do not use em-dashes (---). Using mathmode is a reliable way of getting the minus sign.

All equations (and typically also semantic formulas, see \sectref{sim:sec:sem}) should be typeset using mathmode. Notice that mathmode not only gets the math signs ``right'', but also has a dedicated spacing. For that reason, never write things like p$<$0.05, p $<$ 0.05, or p$<0.05$, but rather $p<0.05$. In case you need a two-place math or logical operator (like $\wedge$) but for some reason do not have one of the arguments represented overtly, you can use a ``dummy'' argument (curly brackets) to simulate the presence of the other one. Notice the difference between $\wedge p$ and ${}\wedge p$.

In case you need to use normal text within mathmode, use the text command. Here is an example: $\text{frequency}=.8$. This way, you get the math spacing right.

\section{Abbreviations}

The final abbreviations section should include all glosses. It should not include other ad hoc abbreviations (those should be defined upon first use) and also not standard abbreviations like NP, VP, etc.


\section{Bibliography}

Place your bibliography into localbibliography.bib. Important: Only place there the entries which you actually cite! You can make use of our OSL bibliography, which we keep clean and tidy and update it after the publication of each new volume. Contact the editors of your volume if you do not have the bib file yet. If you find the entry you need, just copy-paste it in your localbibliography.bib. The bibliography also shows many good examples of what a good bibliographic entry should look like.

See \citet{Nordhoff.Muller2021} for general information on bibliography. Some important things to keep in mind:

\begin{itemize}
    \item Journals should be cited as they are officially called (notice the difference between and, \&, capitalization, etc.).
    \item Journal publications should always include the volume number, the issue number (field ``number''), and DOI or stable URL (see below on that).
    \item Papers in collections or proceedings must include the editors of the volume (field ``editor''), the place of publication (field ``address'') and publisher.
    \item The proceedings number is part of the title of the proceedings. Do not place it into the ``volume'' field. The ``volume'' field with book/proceedings publications is reserved for the volume of that single book (e.g. NELS 40 proceedings might have vol. 1 and vol. 2).
    \item Avoid citing manuscripts as much as possible. If you need to cite them, try to provide a stable URL.
    \item Avoid citing presentations or talks. If you absolutely must cite them, be careful not to refer the reader to them by using ``see...''. The reader can't see them.
    \item If you cite a manuscript, presentation, or some other hard-to-define source, use the either the ``misc'' or ``unpublished'' entry type. The former is appropriate if the text cited corresponds to a book (the title will be printed in italics); the latter is appropriate if the text cited corresponds to an article or presentation (the title will be printed normally). Within both entries, use the ``howpublished'' field for any relevant information (such as ``Manuscript, University of \dots''). And the ``url'' field for the URL.
\end{itemize}

We require the authors to provide DOIs or URLs wherever possible, though not without limitations. The following rules apply:

\begin{itemize}
    \item If the publication has a DOI, use that. Use the ``doi'' field and write just the DOI, not the whole URL.
    \item If the publication has no DOI, but it has a stable URL (as e.g. JSTOR, but possibly also lingbuzz), use that. Place it in the ``url'' field, using the full address (https: etc.).
    \item Never use DOI and URL at the same time.
    \item If the official publication has no official DOI or stable URL (related to the official publication), do not replace these with other links. Do not refer to published works with lingbuzz links, for instance, as these typically lead to the unpublished (preprint) version. (There are exceptions where lingbuzz or semanticsarchive are the official publication venue, in which case these can of course be used.) Never use URLs leading to personal websites.
    \item If a paper has no DOI/URL, but the book does, do not use the book URL. Just use nothing.
\end{itemize}
\noindent 
Im Rahmen der Rektions- und Bindungstheorie soll anhand des Deutschen der Frage nachgegangen werden, was auf den einzelnen Repräsentationsebenen sprachlicher Ausdrücke die Bezugsdomäne von Affixen ist und wieweit man ohne Affixbewegungen auskommt.\footnote{Siehe \citet{Chomsky1981,Chomsky1982,Chomsky1986}.} Im Mittelpunkt der Betrachtung werden das Supinum und das Partizip des ersten Status stehen, also infinite Verbformen mit dem Suffix \textit{-en} bzw. \textit{-end}. Dabei sollen die Grundpositionen und das Hauptanliegen meiner Diskussionsbeiträge auf der Rundtisch-Veranstaltung ,,Der Beitrag der Wortstruktur-Theorien zur Wortbildungsforschung'', nämlich die spezifische Operationsweise von Affixen im Zusammenwirken von Syntax, Morphologie und Semantik aufzudecken, verdeutlicht werden.

In drei Hinsichten gehen die folgenden Analysevorschläge über die von Chomsky und vielen anderen entwickelten Gramatikmodellvorstellungen hinaus. Erstens wird mit der von Bierwisch entworfenen Repräsentationsebene der Semantischen Form (SF) gerechnet,\footnote{Siehe \citet{Bierwisch1982,Bierwisch1986,Bierwisch1987b,Bierwisch1987a,Bierwisch-Drucka}.} die zwischen der Ebene der Logischen Form (LF) und dem Interpretationsbereich sprachlicher Ausdrücke, der Konzeptuellen Struktur, vermittelt.\footnote{Zu dem hier vorausgesetzten Grammatikmodell siehe \citet{Zimmermann1987b,Zimmermann1987a}.} Zweitens wird ein reicheres syntaktisches Kategorieninventar angenommen, als das im Rahmen der $\bar{\text{X}}$-Theorie allgemein üblich ist.\footnote{Siehe \citet{Zimmermann1985,Zimmermann1987a,Zimmermann1988-druck,Zimmermann1987d}.} Drittens wird -- mindestens aus heuristischen Gründen -- versucht, Wortbildung und Wortformenbildung lexikalistisch, ohne Inanspruchnahme syntakti\-scher Transformationsregeln, die zwischen der D-Struktur und der S-Struktur bzw. zwischen dieser und der O-Struktur vermitteln, zu behandeln.\footnote{Zu wortstrukturellen Paradoxa und zu Möglichkeiten ihrer Überwindung vgl. \citet{Pesetsky1985}, \citet{Bierwisch-Druckb}, \citet{Zimmermann1988}.}

Ausgabe des Lexikons sind die Wortformen. Sie genügen den Wortstrukturvorschriften und bestimmen mit ihren syntaktischen und semantischen Fügungseigenschaften die Wortgruppenstruktur. Neben der syntaktischen Kategorisierung ist dafür die Argumentstruktur der betreffenden lexikalischen Einheiten bzw. deren Status als Operatorausdruck von besonderer Wichtigkeit. Zu den Operatorausdrücken gehören vor allem die Artikel, subordinierende Konjunktionen, Relativ- und Fragepronomen. Sie haben die semantische Funktion, Variable zu binden. Dagegen sind Verben, Substantive, Adjektive, adverbielle Präpositionen und Konjunktionen und viele Adverbien n-stellige Prädikatausdrücke, die mit ihren syntaktischen Ergänzungen (Valenzpartnern, Argumenten) semantisch komplexe Einheiten bilden oder der modifikatorischen Spezifizierung anderer Ausdrücke dienen können.\footnote{Zu etwas generelleren Vorstellungen der Komposition von Bedeutungen sprachlicher Ausdrücke siehe \citet{Fanselow1985,Fanselow1986b,Fanselow1986a,Fanselow1988}.}

Das SF-Schema von Verben hat nach \citet{Bierwisch1987d} folgende Form:\footnote{Anmerkung der Herausgeber: Die Notation $\hat{x}$  ist äquivalent zu $\lambda x$.}


\ea\label{ex:zi88:1} $\hat{x}_n \; \dots \; \hat{x}_1 \; \hat{e} \,\, (\hat{t} \; [t = Te]:) [e\; \cnst{inst} \; [\dots ]]$ \newline
mit $t,e \in N, \,T \in N/N, \, = \, \, \in (S/N)/N, \cnst{inst} \in (S/N)/S, \, : \, \in (\alpha/\alpha)/\beta$
\z

\noindent Dabei repräsentieren die Lambdaabstraktoren die semantischen Leerstellen (,,$\Theta$-Rollen'') und die Lambdaabstraktorenfolge die semantische Argumentstruktur (das ,,$\Theta$-Raster'') des Verbs. $\hat{x}_n \dots \hat{x}_2$ sind die Leerstellen für die internen Argumente, $\hat{x}_1$ ist die Leerstelle für das externe Argument, $\hat{e}$ ist die referentielle Argumentstelle, und $\hat{t}$ ist die für Verben charakteristische Bezugsstelle für Tempusspezifizierungen. Diese Leerstelle und der betreffende Teil der Prädikat-Argument-Struktur (PAS) ist abwesend, wenn der betreffende Verbstamm ohne Tempusspezifizierung bleibt.\footnote{Mit der An- bzw. Abwesenheit dieser SF-Komponente von Verben korrespondiert deren Klassifizierung als tempusspezifizierbare Einheiten mittels des morpho-syntaktischen Merkmals $\alpha$TS (siehe unten).} Die auf das Lambdaabstraktorenpräfix folgende PAS liest sich so: Das Zeitintervall $t$ ist gleich dem Zeitintervall des Sachverhalts $e$, derart daß $e$ eine Instanz der Proposition [\dots] ist.

Die Abfolge der Lambdaabstraktoren ist nicht beliebig. Der Platz einer Leerstelle $\hat{y}$ in der semantischen Argumentstruktur der SF einer lexikalischen Einheit entspricht nach \citet{Bierwisch1987b,Bierwisch1987c} dem absoluten Rang $\text{AR}(y)$, der sich aus dem Minimum der Ränge der Vorkommen der durch $\hat{y}$ gebundenen Variablen $y$ in der PAS der betreffenden SF ergibt. Die Ränge terminaler Einheiten in der PAS der SF lexikalischer Einheiten errechnen sich wie folgt:

\ea\label{ex:zi88:2} $R(A) = R(B) + 1$, gdw. $A$ Funktor oder Konnektiv und $B$ Argument bzw. Konnektivpartner ist.
\z 

\noindent Für zwei Leerstellen $\hat{y}_i$ und $\hat{y}_j$ in der semantischen Argumentstruktur gilt dann:

\ea\label{ex:zi88:3} $\hat{y}_i \gg \hat{y}_j$, gdw. $\text{AR}(y_i) > \text{AR}(y_j)$
\z 

\noindent Diesen Annahmen zufolge beschränken sich die semantische Argumentstruktur (das ,,$\Theta$-Raster'') und die PAS der SF einer lexikalischen Einheit gegenseitig, was auch für alle durch Flexion, Derivation und Komposition bewirkten Modifikationen der SF der betroffenen Einheit gilt. 

Die in (\ref{ex:zi88:4a}) angegebene SF für das auch kausativ verwendbare inchoative Verb \textit{schmelzen} veranschaulicht die in \REF{ex:zi88:1}--\REF{ex:zi88:3} enthaltenen Festlegungen. Die Ziffern geben den Rang der betreffenden terminalen Einheiten an. Un\-ter\-strei\-chung kennzeichnet die absoluten Ränge der durch Lambdaabstraktoren gebundenen Variablen. (\ref{ex:zi88:4b}) zeigt die morpho-syntaktische Kategorisierung der Verbverwendungen in Abhängigkeit von der An- bzw. Abwesenheit bestimmter SF-Komponenten.

\ea\label{ex:zi88:4}
    \ea\label{ex:zi88:4a} $\hat{x}_2 \; (_\alpha \; \hat{x}_1) \; \hat{e} \; (_\beta \; \hat{t} \; [\underset{\uline{0}}{t} \; \underset{3}{=} \; \underset{2}{T} \underset{\uline{1}}{e}] \; \underset{19}{:}) \; [\underset{4}{e} \; \underset{18}{\cnst{INST}} \; (_\alpha \; [\underset{\uline{5}}{x_1} \; \underset{7}{\textsc{tun}} \; \underset{6}{x_3}] \newline \underset{17}{:} \; [[\underset{8}{x_3} \; \underset{10}{\textsc{bewirken}} \; \underset{9}{x_4}] \; \underset{16}{:} \; [\underset{11}{x_4} \; \underset{15}{\cnst{INST}}) \; [\underset{14}{\textsc{werden}} \newline [\underset{13}{\textsc{flüssig}} \; \underset{\uline{12}}{x_2}]](_\alpha]])]$ \newline mit \textsc{tun}, \textsc{bewirken} $\in \; (S/N)/N$, \textsc{werden} $\in \; S/S$, \newline \textsc{flüssig} $\in \; S/N$
    \ex\label{ex:zi88:4b} $+\text{V} \; \alpha \text{DA} \; \beta \text{TS}$
\z\z

\noindent Das Beispiel (\ref{ex:zi88:4}) ist in mehrerer Hinsicht aufschlußreich. Es zeigt die einzig zulässige Abfolge der Lambdaabstraktoren. Ferner macht es Zusammenhänge zwi\-schen der morpho-syntaktischen Kategorisierung der betreffenden Lexikoneinheit und der Zusammensetzung seiner SF deutlich. Bei Anwesenheit von $\hat{x}_1$ und der entsprechenden Agentivität und Kausativität beinhaltenden SF-Kom\-po\-nen\-ten hat man es mit einem transitiven Verb mit designiertem externen Argument, kategorial als $+$DA gekennzeichnet, zu tun.\footnote{Zum Begriff des designierten Arguments siehe \citet{Haider1984,Haider1986b}.} Bei Abwesenheit von $\hat{x}_1$ und der entsprechenden SF-Teile ergibt sich ein intransitives Verb mit nicht-designiertem externen Argument, angezeigt durch $-$DA. $+$DA-Verben selegieren im Perfekt und Plusquamperfekt das Hilfsverb \textit{haben}, $-$DA-Verben verbinden sich mit \textit{sein}. Nur $-$DA-Verben erlauben die Bildung eines attributiv verwendbaren Partizips im dritten Status, vgl.:

\ea\label{ex:zi88:5}
    \ea\label{ex:zi88:5a} Gleisarbeiter \textit{haben} die dicke Eisschicht auf der Weiche mit Gasbrennern geschmolzen ($+$DA).
    \ex\label{ex:zi88:5b} Die dicke Eisschicht auf der Weiche \textit{ist} allmählich geschmolzen ($-$DA).
    \ex\label{ex:zi88:5c} die allmählich \textit{geschmolzene} ($-$DA) dicke Eisschicht auf der Weiche
    \ex\label{ex:zi88:5d} die von Gleisarbeitern mit Gasbrennern \textit{geschmolzene} ($-$DA) dicke Eisschicht auf der Weiche
\z\z 

\noindent (\ref{ex:zi88:5d}) deutet an, daß auch mit dem Passiv korrelierende Partizipien des dritten Status $-$DA-Einheiten sind und somit attributiv verwendet werden können.\footnote{Siehe \citet{Zimmermann1988-druck}.}

Unter dem Gesichtswinkel der Wortbildung zeigt (\ref{ex:zi88:4}) die lexikalische Ver\-wandt\-schaft inchoativer und entsprechender, hier völlig gleichlautender kausativer Verben. Um es klar zu sagen: Eine Dekausativierung ist nicht vorgesehen und auch nicht zulässig. Sie wäre mit einer Reduzierung nicht nur der semantischen Argumentstruktur, sondern auch der PAS verbunden. Und das ist prinzipiell nicht möglich. Wortbildungen und Wortformenbildungen können mit Anreicherungen der PAS und entsprechend der Argumentstruktur der betroffenen lexikalischen Einheit verbunden sein, nicht aber mit Reduzierungen der PAS. Reduzierungen können nur das Lambdaabstraktorenpräfix betreffen. Bei\-spiels\-wei\-se ist für Nominalisierungen typisch, daß die Leerstellen der verbalen oder adjektivischen Derivationsbasis weglaßbar sind, wodurch es zu so\-ge\-nann\-ten impliziten, eben mitverstandenen, aber nicht näher spezifizierten Argumenten kommt, die in der PAS als ungebundene Variable figurieren. Vgl.:

\ea\label{ex:zi88:6} Das Schmelzen (der dicken Eisschicht auf der Weiche) nahm mehrere Stunden in Anspruch.
\z 


\largerpage
\noindent Hier werden kontextabhängig das ,,Objekt'' und das ,,Subjekt'' des Schmelzens mitverstanden, unabhängig von der Anwesenheit entsprechender syntaktischer Ergänzungen.

Bezüglich möglicher Tempusspezifizierungen enthält die SF in (\ref{ex:zi88:4a}) nach dem Schema (\ref{ex:zi88:1}) die betreffende semantische Leerstelle, $\hat{t}$, und die dazugehörige PAS-Komponente. Das Merkmal $+$TS gemäß (\ref{ex:zi88:4b}) kennzeichnet die lexikalische Einheit als temporal spezifizierbar, d.h. als mit Tempusmorphemen kombinierbar. $-$TS-Markierung entspricht der Abwesenheit der betreffenden SF-Komponenten. Wiederum wäre eine Reduzierung der SF um die entsprechenden PAS-Teile nicht möglich. Nur die Leerstelle $\hat{t}$ könnte beseitigt werden. Das ist charakteristisch für infinite Satzeinbettungen, in denen eine temporale Einordnung des durch die betreffende Konstruktion bezeichneten Sachverhalts mitverstanden wird.\footnote{Ebenda.}

\largerpage
Im Folgenden sollen nun das Supinum und das Partizip im ersten Status näher betrachtet werden. Das Supinum bilden nach \citet{Bech1955} infinite adverbale Verbformen, das Partizip entsprechende adnominale Verbformen. Für die Kategorisierung infiniter deutscher Verbformen soll folgende Merkmalspezifizierung gelten:\footnote{Zu den Einzelheiten siehe \citet{Zimmermann1988-druck,Zimmermann1987d}. Durch die Hinzunahme der morpho-syntaktischen Merkmale $\alpha$TS und $\beta 1S$ ergeben sich hier gegenüber \citet{Zimmermann1988-druck} bestimmte Verfeinerungen. Eine Modifikation der Merkmalsverteilungen betrifft den Infinitiv. Er ist als $+$V$-$N$-$A$-$Adv-Einheit aufzufassen. Der Infinitiv mit \textit{zu} muß bei attributiver Verwendung (siehe die Kolumne 5 in Tabelle \ref{tab:zi88:1}) wie das Partizip I ein \textit{d} als Auslaut des Infinitivformativs haben. Es soll angenommen werden, daß dies durch Epenthese von \textit{d} zwischen dem Infinitivformativ und dem adjektivischen Flexionsaffix zu regeln ist. Das Merkmal $\alpha$TS in Tabelle \ref{tab:zi88:1} ist \citet{Bierwisch1987d} entlehnt.}

\begin{table}
\begin{tabular}{lccccccccccc}
\lsptoprule
    &1  &2  &3  &4  &5  &6  &7  &8  &9  &10 &11 \\
\midrule
V   & $+$ & $+$ & $+$ & $+$ & $+$ & $+$ & $+$ & $+$ & $+$ & $+$  & $+$  \\
N   & $-$ & $-$ & $-$ & $+$ & $-$ & $-$ & $-$ & $-$ & $-$ & $-$  & $-$  \\
A   & $-$ & $-$ & $-$ & $-$ & $+$ & $+$ & $+$ & $+$ & $-$ & $+$  & $+$  \\
Adv & $-$ & $-$ & $-$ & $+$ & $-$ & $+$ & $-$ & $+$ & $-$ & $-$  & $+$  \\
TS  & $+$ & $-$ & $-$ & $-$ & $-$ & $-$ & $-$ & $-$ & $+$ & $-$  & $-$  \\
1S  & $+$ & $+$ & $-$ & $-$ & $-$ & $-$ & $+$ & $+$ & $-$ & $-$  & $-$  \\
2S  & $-$ & $-$ & $+$ & $+$ & $+$ & $+$ & $-$ & $-$ & $-$ & $-$  & $-$ \\
 \lspbottomrule
\end{tabular}
\caption{Merkmale infiniter deutscher Verbformen}
\label{tab:zi88:1}
\end{table}

\iffalse
\ea\label{ex:zi88:7}
    \begin{tabularx}{.85\textwidth}{|XXXXXXXXXXXX|}
    \hline 
    & 1 & 2 & 3 & 4 & 5 & 6 & 7 & 8 & 9 & 10 & 11 \\
    \hline 
V   & + & + & + & + & + & + & + & + & + & +  & +  \\
N   & - & - & - & + & - & - & - & - & - & -  & -  \\
A   & - & - & - & - & + & + & + & + & - & +  & +  \\
Adv & - & - & - & + & - & + & - & + & - & -  & +  \\
TS  & + & - & - & - & - & - & - & - & + & -  & -  \\
1S  & + & + & - & - & - & - & + & + & - & -  & -  \\
2S  & - & - & + & + & + & + & - & - & - & -  & - \\
    \hline 
    \end{tabularx}
\z 

\fi 

\begin{enumerate}
    \item Infinitiv im analytischen Futur
    \item Infinitiveinbettung ohne \textit{zu}
     \item Infinitiveinbettung mit \textit{zu}
     \item Infinitiveinbettung mit \textit{zu} prädikativ
     \item Infinitiveinbettung mit \textit{zu} attributiv
     \item Infinitiveinbettung mit \textit{zu} adverbiell
     \item Partizip I ohne \textit{zu} attributiv
     \item Partizip I ohne \textit{zu} adverbiell
     \item Partizip II in analytischen Verbformen
     \item Partizip II attributiv   
     \item Partizip II adverbiell
\end{enumerate}

\noindent Dieser Merkmalspezifizierung zufolge sind \citeauthor{Bech1955}s Supinum und Partizip durch $+$V$-$A bzw. $+$V$+$A erfaßt. (Zur adverbiellen Verwendung von Partizipien siehe unten.) In den folgenden Lexikonrepräsentationen für die Affixe \textit{-en} und \textit{-end} werden nur positiv spezifizierte Merkmale angeführt, die nach Tabelle \ref{tab:zi88:1} negativ spezifizierten sind als redundante Information automatisch zu ergänzen.

Das Infinitivformativ \textit{-en} ist im Lexikon folgendermaßen repräsentiert:

\ea\label{ex:zi88:8}
    \ea\label{ex:zi88:8a} /-en/
    \ex\label{ex:zi88:8b} $+$V $\alpha$TS$+$1S
    \ex\label{ex:zi88:8c} [$+$V$+$TS \_\_ ]
    \ex\label{ex:zi88:8d} $\hat{P} \; [P \; (_{-\alpha} \; t)] \;  \;  \; \text{mit} \; P \; \in \; \text{S/N}$
\z\z 

\noindent (\ref{ex:zi88:8a}) steht stellvertretend für die phonologische Struktur des Infinitivformativs. (\ref{ex:zi88:8b}) zeigt die von der SF (\ref{ex:zi88:8d}) abhängige morpho-syntaktische Kategorisierung des Morphems, die sich nach den Regeln der Wortstrukturierung auf die ab\-ge\-lei\-te\-te Einheit vererbt.\footnote{Zu den Kategorisierungsvorschriften von Wortstruktureinheiten siehe \citet{Lieber1980,Lieber1981,Lieber1983}. Vgl. auch \citet{Zimmermann1987c}.} (\ref{ex:zi88:8c}) verlangt als wortstrukturellen Partner eine mit $+$TS gekennzeichnete Verbform. (\ref{ex:zi88:8d}) ist bei $+$TS-Kennzeichnung eine iden\-ti\-sche Abbildung eines einstelligen Prädikats auf ein einstelliges Prädikat. Bei $-$TS-Kennzeichnung wird das einstellige Prädikat auf ein Argument $t$ bezogen, wodurch -- mittels Lambdakonversion -- eine Proposition $[\dots t \dots]$ vom Typ $S$ entsteht. Das heißt, daß die betreffende Argumentstelle des Prädikats, $\hat{t}$, absorbiert wird, ohne weitere inhaltliche Spezifizierung. Das ist der Vorgang der Argumentstellenreduktion oder -blockierung.\footnote{Generalisierend ausgedrückt, funktioniert Argumentstellenunterdrückung wie folgt.
    \ea $[[\hat{y} \; \dots \; [\dots \; y \; \dots]](y) \; \dots] \; \equiv \; [\dots \; [\dots \; y \; \dots] \; \dots \; ]$\z 
    \noindent Dabei absorbiert $y$ die Leerstelle $\hat{y}$, und zwar durch Lambdakonversion (siehe (\ref{ex:zi88:9})).
    }

Was also durch das Infinitivformativ absorbiert werden kann, ist die Leerstelle $\hat{t}$ für den Tempusbezug. Diese ist aber ohne weitere Festlegungen für Lambdakonversion nicht unmittelbar zugängig. Denn funktionale Applikation erfolgt schritt\-wei\-se von außen nach innen, angewendet auf entsprechende Argumente. (\ref{ex:zi88:9}) ist die allgemeine Regel für funktionale Applikation.

\ea\label{ex:zi88:9} $[\hat{y} \; [\dots \; y \; \dots]] \; (a) \; \equiv \; [\dots \; a \; \dots \; ]$ \newline mit \; $[\hat{y} \; [\dots \; y \; \dots]] \; \in \; \alpha/\beta, \; a \; \in \; \beta$
\z 

\noindent Angewendet auf das Beispiel in (\ref{ex:zi88:4a}), könnte $\hat{t}$ gemäß (\ref{ex:zi88:8d}) durch $t$ erst absorbiert werden, nachdem die Leerstellen $\hat{x}_2$, $\hat{x}_1$ (letzere falls vorhanden) und $\hat{e}$ gesättigt, absorbiert oder anderweitig beseitigt wurden. Nun ist in der Kategorialgrammatik neben funktionaler Applikation auch funktionale Komposition möglich, deren Wesen darin besteht, zwei Funktionen zu einer komplexen Funktion zu vereinen.\footnote{Zum Mechanismus der funktionalen Komposition und zu seiner Anwendung in der Syntax siehe \citet{Ades.Steedman1982} und \citet{Steedman1985}.} Das erscheint für die Verknüpfung von Affixbedeutungen mit der SF ihrer Ableitungsbasis das geeignete Verfahren zu sein,\footnote{Siehe \citet{Moortgat1984,Moortgat1987} und \citet{Bierwisch1987b,Bierwisch1987d}.} bei dem seman\-ti\-sche Leerstellen der Ableitungsbasis gewissermaßen übergangen werden können, wenn es um den Anschluß der SF des Affixes, das den Hauptfunktor bildet, an die SF der Ableitungsbasis als dem Nebenfunktor geht. Die ausgeblendeten Leerstellen der Ableitungsbasis gehen als Erbgut an das semantische Amalgam der Funktoren. In ihrer generalisierten Form funktioniert die funktionale Komposition folgendermaßen:

\ea\label{ex:zi88:10} $P \; (Q) \; \equiv \; \hat{y}_n \; \dots \; \hat{y}_1 \; [P \; (Q \; (y_n) \; \dots \; (y_1))] \newline \text{mit} \; P \; \in \; \alpha/\beta, \; Q \; \in \; (\dots(\beta/\gamma_1 \; \dots)/\gamma_n, \; y_i \; \in \; \gamma_i$
\z 

\noindent Die $\hat{y}_i$ sind die von dem Nebenfunktor $Q$ an die komponierte Funktion $P(Q)$ vererbten Leerstellen. Sind keine Leerstellen zu vererben, findet funktionale Appli\-kation (siehe (\ref{ex:zi88:9})) statt, die demnach ein Spezialfall der funktionale Komposition (\ref{ex:zi88:10}) ist.

Angewendet auf die SF des inchoativen Verbs \textit{schmelzen} aus (\ref{ex:zi88:4a}) und die in (\ref{ex:zi88:8d}) für das Infinitivmorphem angegebene Funktion ergeben sich gemäß (\ref{ex:zi88:9}) und (\ref{ex:zi88:10}) die funktionalen Amalgame (\ref{ex:zi88:11a}) und (\ref{ex:zi88:11b}), je nach der Spezifizierung von $\alpha$TS in (\ref{ex:zi88:8}).

\newpage
\ea\label{ex:zi88:11}
    \ea\label{ex:zi88:11a} $\hat{P} \; [P \; t] \; (\hat{x}_2 \; \hat{e} \; \hat{t} \; [t=Te]:[e \; \cnst{INST} \; [\textsc{werden} \; [\textsc{flüssig} \; x_2]]]) \; \equiv \newline
    %
    \hat{x}_2 \; \hat{e} \; [[\hat{P} \; [P \; t]] \; ([\hat{x}_2 \; [\hat{e} \; [\hat{t} \; [t=Te]: \newline
    %
    [e \; \cnst{INST} \; [\textsc{werden} \; [\textsc{flüssig} \; x_2]]]]]] \; (x_2)(e))] \; \equiv \newline
    %
    \hat{x}_2 \; \hat{e} \; [[\hat{P} \; [P \; t]] \; (\hat{t} \; [t = Te] : [e \; \cnst{INST} \; [\textsc{werden} \; [\textsc{flüssig} \; x_2]]])] \; \equiv \newline
    %
    \hat{x}_2 \; \hat{e} \; [[\hat{t} \; [t=Te] : [e \; \cnst{INST} \; [\textsc{werden} \; [\textsc{flüssig} \; x_2]]]](t)] \; \equiv \newline
    %
    \hat{x}_2 \; \hat{e} \; [t=Te]:[e \; \cnst{INST} \; [\textsc{werden} \; [\textsc{flüssig} \; x_2]]] \; $
    
    \ex\label{ex:zi88:11b} $\hat{P} \; [P] \; (\hat{x}_2 \; \hat{e} \; \hat{t} \; [t=Te] : [e \; \cnst{INST} \; [\textsc{werden} \; [\textsc{flüssig} \; x_2]]]) \; \equiv \newline
    %
    \hat{x}_2 \; \hat{e} \; [[\hat{P}[P]] \; ([\hat{x}_2 \; [\hat{e} \; [\hat{t}[t=Te] : [e \; \cnst{INST} \newline
    %
    [\textsc{werden} \; [\textsc{flüssig} \; x_2]]]]]] \; (x_2)\; (e))] \; \equiv \newline
    %
    \hat{x}_2 \; \hat{e} \; [[\hat{P} \; [P]] \; (\hat{t} \; [t = Te] : [e \; \cnst{INST} \; [\textsc{werden} \; [\textsc{flüssig} \; x_2]]])] \; \equiv \newline
    %
    \hat{x}_2 \; \hat{e} \; \hat{t} \; [t=Te] : [e \; \cnst{INST} \; [\textsc{werden} \; [\textsc{flüssig} \; x_2]]]$
\z\z 

\noindent Die in (\ref{ex:zi88:11a}) für den Infinitiv von \textit{schmelzen} resultierende SF ist ein zweistelliges Prädikat, bei dem die Leerstelle $\hat{t}$ für die Tempusspezifizierung absorbiert ist. Das Argument in der PAS geht als freier Parameter in die konzeptuelle Interpretation der Infinitivkonstruktionen ein, z.B. in (\ref{ex:zi88:12}).

\ea\label{ex:zi88:12} Wir sahen die dicke Eisschicht auf der Weiche schmelzen.
\z 

\noindent Die in (\ref{ex:zi88:11b}) resultierende SF weist den Infinitiv von \textit{schmelzen} als dreistelliges Prädikat aus, mit einer Leerstelle $\hat{t}$ für Tempusspezifizierung. Letztere erfolgt durch die in (\ref{ex:zi88:13d}) angeführte SF für die durch das Hilfsverb \textit{werd}- transportierte Futurbedeutung.\footnote{In (\ref{ex:zi88:13b}) sind per Konvention folgende unmarkierten, negativ spezifizierten syntaktischen und morpho-syntaktischen Merkmale zu ergänzen: $-$N$-$A$-$Adv$-$TS$-$1S$-$2S, $-$1Pers$-$2Pers$-$Plur. Mit dem Merkmal $+$Fin korrespondiert generell die Kennzeichnung der semantischen Leerstelle für das externe Argument der betreffenden Verbform mit Kongruenzmerkmalen, die ihrerseits den Nominativ des externen Arguments legitimieren. Die aus \textit{wird} und dem inchoativen Verb \textit{schmelzen} kombinierte Einheit würde demzufolge $\hat{x}_2$ in der SF (\ref{ex:zi88:14}) mit den für \textit{wird} geltenden Kongruenzmerkmalen $-$1Pers$-$2Pers$-$Plur korrelieren. Wie diese Adressenzuweisung für semantische Leerstellen erfolgt, ist eine separat zu behandelnde Frage. Sie kann hier vernachlässigt werden.}

\ea\label{ex:zi88:13}
    \ea\label{ex:zi88:13a} /wird/
    \ex\label{ex:zi88:13b} $+$V$+$Fin
    \ex\label{ex:zi88:13c} [$+$V$+$TS$+$1S \_\_ ]
    \ex\label{ex:zi88:13d} $\hat{P} \; [P \;[t:t \; \textsc{nach} \; t_0]]$ \; \; mit \textsc{nach} $\in \text{(S/N)/N}$
\z\z 

\noindent Aus (\ref{ex:zi88:11b}) und (\ref{ex:zi88:13b}) ergibt sich nach (\ref{ex:zi88:9}) und (\ref{ex:zi88:10}) die SF (\ref{ex:zi88:14}) für die finite analytische Verbform \textit{schmelzen wird}.

\ea\label{ex:zi88:14} $\hat{x}_2 \; \hat{e} \; [[t : [t \; \textsc{nach} \; t_0]] = Te] : [e \; \cnst{INST} \; [\textsc{werden} \; [\textsc{flüssig} \; x_2]]]$
\z 

\noindent Gemäß der hier vorausgesetzten lexikalistischen Behandlung von Wortbildung und Wortformenbildung verbinden sich Wortstämme und Affixe nach den Regeln der Wortsyntax zu komplexen morphologischen Einheiten. Ganz analog verknüpfen sich Wörter nach den Regeln der Wortgruppensyntax zu komplexen syntaktischen Einheiten. Darunter gibt es Verbindungen von Wörtern auf der untersten syntaktischen Projektionsstufe X\textsuperscript{0}. Das trifft für alle analytischen Verbformen zu. Vgl. für das hier betrachtete Beispiel die in (\ref{ex:zi88:15}) angegebene D-Struktur-Repräsentation:

\ea\label{ex:zi88:15} $\Bigl[_{+\textrm{V}+\textrm{Fin}^0}\Bigl[_{+\textrm{V}+\textrm{TS}+\textrm{1S}^0}\Bigl[_{+\textrm{V}+\textrm{TS}} \textrm{schmelz}\Bigr] \newline \newline
\Bigl[_{+\textrm{V}+\textrm{TS}+\textrm{1S}} \textrm{en}\Bigr]\Bigr]\Bigl[_{+\textrm{V}+\textrm{Fin}^0} \textrm{wird}\Bigl]\Bigl]$
\z 

\noindent In solche Verbkomplexe gehen zusammen mit Vollverben nicht nur Hilfsverben, sondern auch Modalverben, Phasenverben und vermutlich auch Sub\-jekt\-he\-bungsverben wie \textit{scheinen}, \textit{drohen}, \textit{pflegen} ein. Die Schwesterkonstituenten dieser Verben sind immer infinite Verbformen im ersten, zweiten oder dritten Status des \citeauthor{Bech1955}schen Supinums.\footnote{Wie (\ref{ex:zi88:15}) zeigt, werden solche Verbkomplexe hier als $+$V\textsuperscript{0}-Einheiten angesehen. Bei \citet{Haider1986b} haben sie die Kategorisierung VK (,,Verbkomplex''). Bei \citet{Steedman1985} sind es komplexe syntaktische Funktionen, entstanden durch funktionale Komposition.} Semantisch gesehen, liegt bei solchen Verbkomplexen paarweise immer Komposition von Funktionen vor.

Wichtig ist nun zu erkennen, daß das Infinitivformativ oder das die Futurbedeutung transportierende Hilfsverb \textit{werd}- wie auch alle anderen Tempus und Modus spezifizierenden Formative nicht oberhalb der Projektionsstufe X\textsuperscript{0} fi\-gu\-rie\-ren, also einen relativ beschränkten ,,Skopus'' haben, der \textbf{nicht} mit ihrer semantischen Operationsdomäne übereinstimmt. Diese sind ein- bzw. zweistellige Prädikate mit einer Leerstelle, $\hat{t}$, für Tempusspezifizierung und ggf. einer wei\-te\-ren Leerstelle, $\hat{e}$, für Modusspezifizierungen bezüglich der Tatsachengeltung des durch die betreffende PAS charakterisierten Sachverhalts. Funktionale Komposition ermöglicht, $n$-stellige Prädikate als $n-i$-stellige Prädikate zu behandeln, so als wären die $n-i + 1$ Leerstellen schon durch die SF von Argumentausdrücken spezifiziert, wenn die SF eines Affixes oder eines affixähnlichen Worts als Haupt\-funktor hinzutritt. Deshalb sieht es so aus, als müßten Tempus- und Modusmorpheme in der hierarchischen Struktur von Sätzen relativ weit oben fi\-gu\-rie\-ren. Und tatsächlich haben das die meisten Modelle der generativen Trans\-for\-ma\-tions\-gram\-ma\-tik seit Chomskys ,Syntactic Structures' (\citeyear{Chomsky1957}) bis heute -- auch für das Deutsche -- als unausweichlich angesehen.\footnote{\citet{Haider1986b,Haider1986a} verankert die Infl(ection)-Merkmale fürs Deutsche in der D-Struktur im Komplementierer Comp (hier: $+$Spez, siehe unten) und entwickelt eine Theorie der Merkmalverdrängung und -- komplementär dazu -- der Verbanhebung, damit die betreffenden Merkmale zu ihrem eigentlichen Träger, dem Verb, gelangen.} Nicht zuletzt haben sich solche Vorstellungen auch deshalb so lange halten können, weil man mit Syntax zu einem nicht geringen Teil das Fehlen einer Semantikkomponente in der Grammatik zu kompensieren versuchte. Jedenfalls eröffnen sich mit der Annahme der Repräsentationsebene der Semantischen Form neue Sehweisen zum Verhältnis von Syntax, Morphologie und Semantik und -- wie es scheint -- gute Aussichten für ein lexikalistisches Konzept von Wortbildung und Wortformenbildung.\footnote{\label{fn:zi88:19} In einem solchen Klärungsprozeß ist es normal und wahrscheinlich, daß in der Theoriebildung Halbheiten und Inkonsequenzen vorkommen. Möglicherweise hat \citet{Jackendoff1987} recht, wenn er die Kontroll- und Bindungstheorie der semantischen Komponente der Grammatik zurechnet.}

Für die folgenden Darlegungen zu den Konstruktionseigenschaften von Partizipien des ersten Status ist es erforderlich, in Ergänzung zu Tabelle \ref{tab:zi88:1} einige Grund\-an\-nah\-men über die morpho-syntaktische Kategorisierung und die Wortgruppenstrukturierung zu verdeutlichen. Neben den überkommenen syntaktischen Wortklassenmerkmalen $\alpha$V und $\beta$N wird mit $\gamma$A, $\delta$Adv, $\epsilon$Spez, $\zeta$Q und $\eta$K gerechnet.\footnote{Siehe \citet{Zimmermann1988-druck,Zimmermann1987d}.} Alle übrigen Merkmale sind morpho-syntaktische Differenzierungen. In den Repräsentationen erscheinen nur positiv spezifizierte Merkmale, alle anderen für die jeweilige Wortklasse charakteristischen differentiellen Merkmale gelten unmarkierterweise als negativ spezifiziert. Es ist von alters her üblich, zwischen lexikalischen, offenen und funktionalen (oder: gram\-ma\-ti\-schen), re\-la\-tiv geschlossenen Wortklassen zu unterscheiden.\footnote{Siehe \citet{Emonds1985,Emonds1987,Chomsky1986,Fukui1986}.} Verben ($+$V), Substantive ($+$N$\alpha$A), Adjektive ($+$A, \dots). Adverbien ($+$A$+$Adv) gehören zur ersten Gruppe, Präpositionen und adverbielle Konjunktionen ($+$Adv), Artikel ($+$Spez, $+$N, $+$A), nicht-adverbielle Konjunktionen ($+$Spez, \dots), koordinierende Konjunktionen ($+$K), Quantorenausdrücke ($+$Q, \dots) gehören zur zweiten Gruppe.

\largerpage
In diesem System syntaktischer Merkmale lassen sich viele syn\-tak\-ti\-sche, morphologische und semantische Generalisierungen ausdrücken.\footnote{Siehe \citet{Zimmermann1985,Zimmermann1987a,Zimmermann1988-druck,Zimmermann1987d}.} Be\-kann\-ter\-ma\-ßen konzentrieren sich Wortbildungen auf die lexikalischen Wort\-klas\-sen. Attributiv verwendete Adjektive ($+$V$+$N$+$A$-$Adv) und Partizipien ($+$V$-$N$+$A$-$Adv) teilen wesentliche syntaktische, morphologische und se\-man\-ti\-sche Eigenschaften. Für alle mit $+$V gekennzeichneten Einheiten gilt, daß sie die semantische Leerstelle $\hat{e}$ mit der entsprechenden PAS-Komponente $[e\; \cnst{INST}[ \dots]]$ (vgl. (\ref{ex:zi88:1})) aufweisen, also eine referentielle Argumentstelle haben.

Für den Syntagmenaufbau und die SF von Sätzen und satzartiger Konstruktionen ist nun wichtig, daß sie maximale Projektionen von lexikalischen\linebreak $+$V-Einheiten sind und daß sie genau wie Substantivgruppen referierende Ausdrücke darstellen. Das heißt, daß diese Konstruktionen der Kategorie $X^m$ einen referentiellen Spezifikator Spez\textsuperscript{0} enthalten, der gegebenenfalls auch phonologisch leer sein kann. Seine semantische Funktion besteht darin, die referentielle Argumentstelle seiner Schwesterkonstituente $X^{m-1}$ durch einen referentiell spezifizierenden Operator zu substituieren.\footnote{Zur Substitution der referentiellen Leerstelle einer lexikalischen Einheit durch einen referentiellen Operator siehe Anm. \ref{fn:zi88:34}.} Entsprechend soll für referierende Syntagmen folgende syntaktische Strukturvorschrift gelten:\footnote{Die Strukturvorschrift (\ref{ex:zi88:16}) ist eine Knotenzulässigkeitsbedingung. Siehe dazu \citet{McCawley1968}. Die für X zugelassenen Merkmalspezifizierungen erfassen mit $+$N$+$A Substantivgruppen mit adjektivisch flektierendem Konstruktionskern wie z.B. \textit{Kranke}, mit $+$N$-$A normale Substantivgruppen mit substantivisch flektierendem Kernwort, mit $+$V$+$N$+$A$+$Adv Adjektivgruppen und mit $+$V$-$N$+$A$+$Adv Partizipialgruppen in adverbieller Funktion, mit $+$V$+$N$+$A$-$Adv Adjektivgruppen und mit $+$V$-$N$+$A$-$Adv Partizipialgruppen in attributiver Funktion sowie mit $+$V$-$N$-$A$-$Adv Infinitivgruppen im ersten bzw. zweiten Status und Sätze mit finitem Verb als Konstruktionskern. Zu den Einzelheiten der syntaktischen Kategorisierung siehe Tabelle \ref{tab:zi88:1} und \citet{Zimmermann1988-druck}.}

\ea\label{ex:zi88:16} $\langle$ X\textsuperscript{m} ; (Y) Spez\textsuperscript{0} X\textsuperscript{m-1} $\rangle$ \newline 
für X $= +$N$\alpha$A bzw. $+$V$\alpha$N$\beta$A$\gamma$Adv
\z 
%\ea\label{ex:zi88:16} $< X^m ; (Y) \; \textrm{Spez}^0 \; X^{m-1}> \newline 
%
%\textrm{für} \; X=+\textrm{N}\alpha\textrm{A} \; \textrm{bzw.} \; +\textrm{V}\alpha \textrm{N}\beta \textrm{A}\gamma \textrm{Adv}$
%\z 

\noindent Damit ist festgelegt, daß alle substantivischen und alle verbalen Konstruktionen einen Spezifikator enthalten. Y markiert das sogenannte Vorfeld in $+$V-Syn\-tag\-men und den Platz für Possessivpronomen bzw. genitivische Substantivgruppen in $+$N-Syntagmen. Spez\textsuperscript{0} ist eine mögliche Landestelle für $+$V\textsuperscript{0}, wodurch Verberst- bzw. Verbzweitstellung entsteht (siehe \citealt{Haider1986b,Haider1986a}). Anders als bei \citet{Fukui1986} wird hier also angenommen, daß verbale und substantivische maximale Projektionen referentiell abgeschlossen sind. Das soll universell gelten. Substantivgruppen und Sätze (einschließlich finiter Satzeinbettungen) unterscheiden sich in einer Sprache bzw. in verschiedenen Sprachen nicht durch referentielle Abgeschlossenheit bzw. Nichtabgeschlossenheit, sondern durch ein verschieden reichhaltiges Inventar von Spezifikatorausdrücken.\footnote{Vgl. im Deutschen das Auftreten des unbestimmten Artikels wie in \textit{ein Kind} vs. \textit{Kinder}, \textit{Geld} oder die unterschiedliche Verfügbarkeit von Spezifikatorausdrücken in germanischen bzw. slawischen Sprachen, z.B. \textit{ein Buch}, \textit{das Buch}, \textit{dieses Buch} vs. \textit{kniga} (‘ein Buch’), \textit{kniga} (‘das Buch’), \textit{ėta kniga} (‘dieses Buch’) im Rus\-si\-schen.}

Spez\textsuperscript{0} ist den funktionalen Kategorien zuzurechnen, zu deren Charakteristika gehört, daß sie phonologisch leer bleiben können (siehe \citealt{Emonds1985,Emonds1987}). Generell gilt, daß phonologisch leere Konstituenten der Legitimierung bedürfen (siehe \citealt{Chomsky1981,Chomsky1982,Chomsky1986}). Als sehr allgemeines Prinzip könnte man annehmen, daß leere Konstituenten spätestens in der LF-Repräsentation eine semantische Interpretation -- und sei es als Variable -- erhalten müssen (siehe \citealt{Schwabe1987}, vgl. \citealt{Bierwisch1987a} und \citealt{Zimmermann1987a}). Für phonologisch leeres Spez\textsuperscript{0} soll $\varepsilon x_i \; (\in N/S)$ bzw. $\exists x_i  \; (\in \; S/S)$ 
als SF-Zuschreibung gelten, sowohl für $+$V-Syn\-tag\-men als auch für $+$N-Syntagmen. Man kann diese semantische Charakterisierung als den Default-Wert für referentielle Spezifizierung ansehen. Er trifft im Deutschen für folgende $+$V-Konstruktionen zu: für alle infiniten Satzeinbettungen, ein\-schließ\-lich attributiv bzw. adverbiell verwendeter Adjektivgruppen, für Relativsätze, für w-Wort-Fragesätze, für Deklarativsätze, für asyndetische Satzeinbettungen und für adverbielle Nebensätze, die nach \citet{Jackendoff1974,Jackendoff1977} und \citet{Steube1987-druck,Steube1987} aus adverbiellen Präpositionen (Konjunktionen) und Komplementsätzen aufgebaut sind. Durch diese SF-Zuschreibung für einen phonologisch leeren Spezifikator Spez\textsuperscript{0} werden die genannten Konstruktionen aus semantisch einstelligen Prädikaten vom semantischen Typ S/N mit einer re\-fe\-ren\-tiel\-len Argumentstelle zu referentiell abgeschlossenen Einheiten vom se\-man\-ti\-schen Typ N bzw.~S.

Für Relativsätze und attributive Adjektiv- und Partizipialgruppen soll $\exists x_i$ als Spezifikatorbedeutung gelten, so daß diese Konstruktionen mit dem Relativpronomen bzw. dessen phonologisch leerer Entsprechung an der Konstruktionsspitze dem semantischen Typ S/N angehören. Siehe das folgende SF-Schema für diese Syntagmen und die entsprechende LF-Repräsentation in (\ref{ex:zi88:18}):

\ea\label{ex:zi88:17} $\hat{x}_j \; [\exists \; x_i \; [t=Tx_i] : [x_i \; \cnst{INST} \; [\dots \; x_j \; \dots \; ]]]$
\z 

\ea\label{ex:zi88:18} $[_{+\textrm{V}^m} \; + \textrm{N}_{j}^{m} \; \textrm{Spez}^0 \; [_{+\textrm{V}^{m-1}} \; \dots \; [_{+\textrm{N}_{j}^{m}} \; e] \; \dots \; ]]$
\z 

\noindent Mit diesen durchaus nicht trivialen Voraussetzungen können nun Konstruktionen betrachtet werden, deren lexikalischer Kopf X\textsuperscript{0} ein Partizip im ersten Status ist. Dabei ist zu unterscheiden zwischen adjektivischen Syntagmen, die in Genus, Kasus, Numerus mit dem attributiv modifizierten Bezugsnomen kongruieren, und modifikatorischen Syntagmen, die sich ohne Kongruenz in die übergeordnete Konstruktion einordnen. Vgl.:

\ea\label{ex:zi88:19}
    \ea\label{ex:zi88:19a} Unser schon mehrere Jahre in Moskau lebender Bekannter kennt die dortige Kulturszene bestens.
    \ex\label{ex:zi88:19b} Unser Bekannter, schon mehrere Jahre in Moskau lebend, kennt die dortige Kulturszene bestens.
    \ex\label{ex:zi88:19c} Schon mehrere Jahre in Moskau lebend, kennt unser Bekannter die dortige Kulturszene bestens.
\z\z 

\noindent Das Partizip in (\ref{ex:zi88:19a}) hat die Kategorisierung $+$V$+$A$+$1S$+$Mask und redundanterweise $-$N$-$Adv$-$TS$-$2S$-$Fem$-$Plur$-$R$-$O$-$G.\footnote{$\alpha$R(egiert), $\beta$O(blique), $\gamma$G(enitiv) sind Kasusmerkmale, die strukturelle von nichtstrukturellen Kasus zu unterscheiden und Synkretismen zu erfassen gestatten. Zu diesen Merkmalen vgl. \citet{Bierwisch1967}.} Dem Partizip in (\ref{ex:zi88:19b}) fehlen die Kongruenzmerkmale $+$Mask$-$Fem$-$Plur$-$R$-$O$-$G, ansonsten -- so soll angenommen werden -- ist es wie das Partizip in (\ref{ex:zi88:19a}) kategorisiert, also adjektivisch. Für das Partizip in (\ref{ex:zi88:19c}) soll die Kategorisierung $+$V$+$A$+$Adv$+$1S gelten, also eine Einordnung in die Klasse adverbieller Ausdrücke erfolgen, womit auch eine Bedeutungsdifferenzierung gegenüber dem adjektivischen Partizip korrespondiert (siehe unten). Genau wie bei Adjektiven wird hier also mit einer bestimmten syntaktischen Polyfunktionalität des Partizips gerechnet, nämlich mit seiner Rolle als attributiver bzw. adverbieller Modifikator (vgl: \REF{ex:zi88:19a}, \REF{ex:zi88:19b} vs. \REF{ex:zi88:19c}).\footnote{Siehe \citet{Zimmermann1985,Zimmermann1987c,Zimmermann1987a,Zimmermann1988-druck}.} Im ersten Fall wird die SF der Partizipialkonstruktion mittels des Konnektivs ,,:" (vgl. (\ref{ex:zi88:1})) mit der SF des Bezugsnomens verknüpft, im zweiten Fall mit der SF des übergeordneten Satzes, in beiden Fällen im Skopus der referentiellen Leerstelle des Modifikanden.\footnote{\label{fn:zi88:28} Das folgende SF-Schema für Modifikation sieht eine Unifizierung der externen Argumentstelle des Modifikanden vor und gleichzeitig die modifikatorische Verknüpfung der SF von Modifikand- und Modifikatorausdruck:
    \ea $\hat{x}_i \; [_{\textrm{A}} \; \dots \; x_i \; \dots], \; \hat{x}_j \; [_{\textrm{B}} \; \dots \; x_j \; \dots \; ] \; \Rightarrow \; \hat{x}_i \; [_{\textrm{A}} \; \dots \; x_i \; \dots] \; : \; [_{\textrm{B}} \; \dots \; x_i \; \dots \; ]$\z}

Der Lexikoneintrag für das Partizip des ersten Status sieht folgendermaßen aus:

\ea\label{ex:zi88:20}
    \ea\label{ex:zi88:20a} /-end/
    \ex\label{ex:zi88:20b} $+$V$+$A$\alpha$Adv$+$1S
    \ex\label{ex:zi88:20c} $[+\textrm{V}+\textrm{TS} \_\_ ]$
    \ex\label{ex:zi88:20d} $\hat{P} [P \; t]$
\z\z 

\noindent Es zeigt sich -- abgesehen von der phonologischen Form -- sehr große Ähnlichkeit mit dem Infinitivformativ (vgl. (\ref{ex:zi88:8})). Das ist kein Zufall. Infinite Formen haben keine Leerstelle für Tempusspezifizierung. Die betreffende Argumentstelle der Ableitungsbasis wird durch die SF des Suffixes absorbiert (beim Infinitiv, sofern er nicht mit \textit{werd}- verknüpft wird). Diese Absorption bedeutet, daß $t$ in der PAS des Partizips eine ungebundene Variable ist und als solche für die kontextabhängige konzeptuelle Interpretation offen ist. Und das ist gerade erwünscht, um den durch die Partizipialkonstruktion bezeichneten Sachverhalt in Zusammenhänge der übergeordneten Konstruktion temporal einzuordnen, wie in (\ref{ex:zi88:19}). Eine weitere Gemeinsamkeit der SF des Partizipformativs mit der SF des Infinitivformativs besteht darin, daß die SF mit der SF der Ableitungsbasis durch funktionale Komposition verbunden wird. Auch hier lassen sich bis in jüngste Zeit Analysen verfolgen, die für das Partizipformativ in der DS einen in den oberen Etagen der Satzstruktur figurierenden Platz vorsehen, der der semantischen Bezugsdomäne des Partizipmorphems in der Syntax Rechnung tragen soll, eine Hypothek, die mit transformationeller Affixwanderung oder Verbanhebung bezahlt werden muss.\footnote{Siehe beispielsweise \citet{Toman1986,Toman1988}.} In der hier verfolgten nicht-transformationellen Behandlung von Wortbildung und Wortformenbildung entfällt diese Unnatürlichkeit. 

\largerpage
Nach der Funktionenkombination der SF des Verbstamms und der SF (\ref{ex:zi88:20d}) nach (\ref{ex:zi88:10}) geht das Verb in Partizipialform als semantisch \textit{n}$-$1-stelliger Prä\-di\-kat\-aus\-druck in die Wortgruppenstruktur ein, in der Valenzpartner dieser Einheit, ggf. adverbielle Modifikatoren und Satzadverbiale und schließlich der Spezifikator auftreten. Nur bezüglich des externen Arguments des Partizips als einer infiniten Verbableitung besteht die Beschränkung, daß es nicht lexikalisch realisiert sein kann. Das verbietet die Kasustheorie.\footnote{Siehe \citet{Chomsky1981,Chomsky1982,Chomsky1986}.} Es soll deshalb -- ganz pa\-rallel zu attributiv verwendeten Adjektivgruppen -- angenommen werden, daß das externe Argument der Partizipialgruppe durch ein phonologisch leeres Relativpronomen repräsentiert ist, das spätestens in der LF-Struktur in die Y-Position von (\ref{ex:zi88:16}) gewandert sein muß und in seiner ursprünglichen Position eine Spur hinterläßt, die als Variable zu interpretieren ist.\footnote{Vgl. die ganz parallelen Analysen von \citet{Fanselow1986b,Fanselow1986a} und \citet{Zimmermann1988-druck}.} Somit ergäbe sich für das attributiv verwendete Partizip der Wortgruppe \textit{die schmelzende Eisschicht} die in (\ref{ex:zi88:21a}) angeführte LF-Struktur mit der SF (\ref{ex:zi88:21b}) (vgl. (\ref{ex:zi88:17}), (\ref{ex:zi88:18})).\footnote{Die Annahmen zur syntaktischen Konfigurationsbildung teile ich im wesentlichen mit \citet{Haftka1988-drucka,Haftka1988-druckb,Haftka1987}. Das Merkmal $+$Fem in (\ref{ex:zi88:21a}) ist durch folgende negativ spezifizierten Kongruenzmerkmale zu ergänzen: $-$Mask$-$Plur$-$R$-$O$-$G. Als SF des Relativpronomens wird $\hat{x}_i \; (\in S/S)$ angenommen.}

\ea\label{ex:zi88:21}
    \ea\label{ex:zi88:21a} 
        $\Bigl[_{+\textrm{V}+\textrm{A}+\textrm{1S}+\textrm{Fem}^4} \Bigl[_{+\textrm{N}^m} \varnothing\Bigr]_i\; \Bigl[_{+\textrm{Spez}} \varnothing\Bigr] \Bigl[_{+\textrm{V}+\textrm{A}+\textrm{1S}+\textrm{Fem}^3}$ \newline \newline
        $\Bigl[_{+\textrm{V}+\textrm{A}+\textrm{1S}+\textrm{Fem}^2} \Bigl[_{+\textrm{N}^m} e\Bigr]_i\; \Bigl[_{+\textrm{V}+\textrm{A}+\textrm{1S}+\textrm{Fem}^1} \Bigl[_{+\textrm{V}+\textrm{A}+\textrm{1S}+\textrm{Fem}^0}$ \newline
        
        $\Bigl[_{+\textrm{V}+\textrm{A}+\textrm{1S}} \Bigl[_{+\textrm{V}+\textrm{TS}} \textrm{schmelz}\Bigr]\Bigr]\Bigl[_{+\textrm{V}+\textrm{A}+\textrm{1S}} \textrm{end}\Bigr]\Bigr]\Bigl[_{+\textrm{Fem}} \textrm{e}\Bigr]\Bigr]\Bigr]\Bigr]\Bigr]\Bigr]$ \newline
    
    \ex\label{ex:zi88:21b} $\hat{x}_i \; [\exists  e \; [t=Te] : [e \; \cnst{INST} \; [\textsc{werden} \; [\textsc{flüssig} \; x_i]]]]$
\z\z 

\noindent In der SF repräsentiert $\hat{x}_i$ die Bedeutung des an die Konstruktionsspitze permutierten phonologisch leeren Relativpronomens. $\exists e$ ist die Bedeutung des Spezifikators $+$Spez\textsuperscript{0},  $x_i$ ist die Bedeutung der Spur $[_{+\textrm{N}^m}\; e]$ in (\ref{ex:zi88:21a}), durch funktionale Applikation in die betreffende Position der PAS des Partizips gelangt. Die SF (\ref{ex:zi88:21b}) verknüpft sich mit der SF des Bezugsnomens \textit{Eisschicht} modifikatorisch,\footnote{Siehe das Modifikationsschema in Anm. \ref{fn:zi88:28}.} die SF des bestimmten Artikels \textit{die} bindet die dem Modifikator und dem Modifikanden gemeinsame Variable $x_i$.\footnote{\label{fn:zi88:34} Es soll gelten, daß Spezifikator-Bedeutungen durch Substitution der Leerstelle für das referentielle Argument substantivischer oder verbaler Prädikatausdrücke mit deren SF nach folgendem Schema verknüpft werden: \ea $\exists x_i, \hat{x}_j [\dots x_j \dots ] => \exists x_i [\dots x_i \dots]$\z (analog für $\varepsilon x_i$ und andere Spezifikator-Bedeutungen).} Ganz analog ergäbe sich die SF eines attributiv verwendeten Adjektivs, z.B. \textit{dicke}, bezogen auf \textit{Eisschicht}, das als lexikalischer Kopf die Kategorisierung $+$V$+$N$+$A$+$Fem hätte.

Wie kommt nun die semantische Differenzierung für die adverbielle Verwendungsweise des Partizips (vgl. (\ref{ex:zi88:19c})), d.h. bei positiver Spezifizierung des Merkmals $\alpha$Adv des Partizipformativs in (\ref{ex:zi88:20b}) zustande? Als Ausgangspunkt ist wichtig festzuhalten, daß in (\ref{ex:zi88:20d}) der gegenüber attributiver vs. adverbieller Verwendung des Partizips invariante SF-Teil des Morphems \textit{-end} angegeben ist. Diese Charakterisierung ist für die attributive Funktion des Partizips ausreichend. Was kommt bei adverbieller Verwendung hinzu? Warum kommt etwas hinzu? Und wo ist der betreffende Bedeutungsanteil einzusiedeln? Handelt es sich um einen Bedeutungsaspekt, der zur SF der betreffenden Konstruktion gehört, oder handelt es sich um eine Bedeutungsspezifizierung, die nur auf der konzeptuellen Ebene zum Tragen kommt?

Wie ein Vergleich von (\ref{ex:zi88:19b}) und (\ref{ex:zi88:19c}) zeigt, besteht in den lexikalischen, einschließlich der morphologischen Ausdrucksmittel kein Unterschied dieser Konstruktionen. Nur die syntaktische Zuordnung der Wortgruppe mit dem Partizip als Kern ist verschieden. Das wäre genau so, würde das Partizip durch ein Adjektiv, z.B. \textit{wohnhaft} ersetzt. Mit der unterschiedlichen syntaktischen Einbettung der partizipialen oder adjektivischen Wortgruppe geht ein Bedeutungsunterschied Hand in Hand. Eben attributive vs. adverbielle Modifikation mit verschiedenen Modifikanden. Bei attributiver Modifikation ist das Bezugsnomen ein Substantiv, bei adverbieller Modifikation ist die Bezugsgröße der übergeordnete Satz, genauer: die SF des übergeordneten Satzes auf der Projektionsstufe $+$V$^2$, d.h. derjenigen Projektionsstufe, auf der die Spezifikatorbedeutung und Satzadverbiale noch unwirksam sind und modifikatorische Erweiterung der betreffenden SF stattfinden kann, bezogen auf die referentielle Argumentstelle $\hat{e}$ des Modifikanden. Präpositionalgruppen, einschließlich adverbieller Nebensätze hätten auf dieser syntaktischen Hierarchiestufe des übergeordneten Satzes ebenfalls modifikatorische Funktion. All diese Modifikatoren teilen das syntaktische Merkmal $+$Adv. Auch adverbielle Substantivgruppen wie \textit{jedes Jahr} oder \textit{mehrere Jahre} in (\ref{ex:zi88:19}) haben es. Hinter dieser syntaktischen Generalisierung steckt auch eine semantische, nämlich daß $+$Adv-Einheiten typischerweise adverbielle Modifikatoren sind, während $+$A-Einheiten typische attributive Modifikatoren sind.\footnote{Siehe \citet{Zimmermann1988-druck,Zimmermann1987d}.}

Lexikalische $-$V$+$Adv-Einheiten sind Präpositionen und adverbielle Konjunktionen.\footnote{Vgl. \citet{Steinitz.Lang1969} und \citet{Jackendoff1974,Jackendoff1977}.} Semantisch sind diese nach \citet{Bierwisch-Drucka} zweistellige Prädikate. Beispielsweise sieht die SF der temporalen Präposition \textit{nach}, wie \citet{Steube1987-druck,Steube1987} vorschlägt, folgendermaßen aus:

\ea\label{ex:zi88:22} $\hat{x}_2 \; \hat{x}_1 \; [Tx_1 \; \textsc{nach} \; Tx_2] \quad \textrm{mit}\;  \textsc{nach} \in \textrm{(S/N)/N}$
\z 

\noindent $\hat{x}_2$, die interne Argumentstelle dieser Präposition, wird durch die SF einer passenden Substantivgruppe wie \textit{das Essen} oder durch die SF des pronominalen Korrelatausdrucks \textit{da(r)}- spezifiziert, d.h. durch funktionale Applikation (siehe (\ref{ex:zi88:9})) beseitigt. $\hat{x}_1$, die Leerstelle für das externe Argument, kann durch modifikatorische Kombination mit der SF eines Modifikanden beseitigt werden.\footnote{Wenn eine adverbielle Präpositionalgruppe oder ein Adverb prädikativ verwendet wird, wird ihre Leerstelle für das externe Argument durch die SF des ,,Subjekts'' der betreffenden kopulahaltigen Konstruktion gesättigt (siehe \citealt{Bierwisch-Drucka}).}

Für die $+$V$+$A$+$Adv-Einheiten, also Partizipialkonstruktionen wie in (\ref{ex:zi88:19c}) und entsprechende Adjektivgruppen (z.B. mit \textit{wohnhaft} statt \textit{lebend} in (\ref{ex:zi88:19c})) ist die Beziehung zum Modifikanden semantisch unspezifiziert. Diesem Faktum soll durch folgende SF-Repräsentation Rechnung getragen werden, die auf allgemeinste Weise mit der SF von Präpositionen, d.h. $+$Adv-Formativen, korrespondiert:

\ea\label{ex:zi88:23} $\hat{x} \; \hat{z} \; [z \; R \; x] \quad (R \; \in \; \textrm{(S/N)/N})$
\z 

\noindent Es handelt sich um eine inhaltlich unbestimmte zweistellige Relation, als deren internes Argument die SF der Partizipial- oder auch Adjektivgruppe figurieren könnte und deren externe Argumentstelle bei modifikatorischem Anschluß der SF der adverbiellen Konstruktion an die SF des Matrixsatzes mit deren referentieller Argumentstelle unifiziert wird. Was die SF in (\ref{ex:zi88:23}) von der lexikalischer Präpositionen unterscheidet, ist die Prädikatvariable $R$, durch die -- wie erforderlich -- die Beziehung zwischen $z$ und $x$ auf der SF-Ebene unspezifiziert bleibt, so daß die ungebundene Variable R als freier Parameter in die konzeptuelle Interpretation der betreffenden Konstruktion eingeht. Damit wird der ,,semantischen Weiträumigkeit'' der hier betrachteten Konstruktionen Rechnung getragen.\footnote{Der Terminus ist \citet[90]{Ruzicka1980} entlehnt. Siehe dazu \citet{Zimmermann1981}.}

Wo aber figuriert die mit den $+$V$+$A$+$Adv-Einheiten verbundene SF (\ref{ex:zi88:23})? Es wäre denkbar, die SF (\ref{ex:zi88:23}) als die Default-In\-ter\-pre\-ta\-tion des Merkmals $+$Adv der adverbiell verwendeten Partizipial- bzw. Adjektivgruppe anzusehen und diese SF-Anreicherung durch eine kategorienabhängige semantische Regel zu bewerkstelligen. Eine andere Möglichkeit wäre, die betreffende Wortgruppe als Schwe\-ster\-kon\-sti\-tu\-en\-te einer phonologisch leeren Präposition anzusehen und dieser die SF (\ref{ex:zi88:23}) zuzuschreiben, auch als Default-Interpretation.\footnote{\citet{Emonds1985,Emonds1987} mißt den Präpositionen als Konstruktionsbausteinen einen hohen Stellenwert bei. Als funktionale Kategorien können sie gemäß einem besonderen Prinzip für unsichtbare Kategorien auch phonologisch leer bleiben, u.a. bei adverbiell verwendeten Substantivgruppen wie \textit{letzte Woche}, \textit{Montag} usw. Vgl. dazu \citet{Steinitz.Lang1969,Bresnan.Grimshaw1987} und \citet{Larson1985,Larson1987}.} Eine Entscheidung zwi\-schen diesen Alternativen ist nicht leicht. Sie hängt auch von der Angemessenheit des hier zugrunde gelegten Systems syntaktischer Kategorisierung und ihrer Korrelation mit SF-Komponenten zusammen. In jedem Fall haben adverbiell verwendete Partizipial- und Adjektivgruppen auf der Basis der SF (\ref{ex:zi88:23}) und der eben gemachten Annahmen zur Interpretation des phonologisch leeren Spezifikators die folgende verallgemeinerte SF:

\ea\label{ex:zi88:24} $\hat{z} \; [z \; R \; \varepsilon x_i \; [t = \; Tx_i] : [x_i \; \cnst{INST} \; [\dots]]]$
\z 

\noindent Zu den adverbiell verwendeten Partizipial- und Adjektivgruppen ist noch eine Ergänzung nötig. Es handelt sich wie bei den entsprechenden attributiv verwendeten Gruppen um infinite Konstruktionen, die kein lexikalisches externes Argument dulden, da ihm kein Nominativ zugewiesen werden kann. Es wird allgemein angenommen, daß solche adverbiellen Konstruktionen wie in (\ref{ex:zi88:19c}) oder in (\ref{ex:zi88:25}) ein implizites externes Argument haben, das in der übergeordneten Satzstruktur einen Antezedenten (Kontrolleur) haben kann, der aber seinerseits implizit bleiben kann wie in (\ref{ex:zi88:26}).

\ea\label{ex:zi88:25} Schmelzend und sich in immer größere Rinnsale verwandelnd, zogen sich die Schneemassen von den Bergen zurück.
\z 

\ea\label{ex:zi88:26} Es wurde schweigend demonstriert.
\z 

\noindent Wie auch immer eine angemessene Kontrolltheorie aussehen mag,\footnote{Vgl. Anm. \ref{fn:zi88:19}.} soll hier angenommen werden, daß die semantische Leerstelle für das externe Argument in der SF von adverbiell verwendeten Partizipien und Adjektiven durch die SF von PRO, dem phonologisch leeren externen Argument (,,Subjekt'') von der be\-tref\-fen\-den Konstruktionen, spezifiziert wird. Für die SF von PRO soll (\ref{ex:zi88:27}) gelten.

\ea\label{ex:zi88:27} $[\varepsilon x_i \; [P \; x_i]] \quad \textrm{mit} \; P \in \textrm{S/N}$
\z 

\noindent Bei Kontrolle dieser Einheit durch einen Antezedenten könnte eine SF-Regel vorsehen, $x_i$ mit dem Antezedenten zu koindizieren und P und $\varepsilon x_i$ zu tilgen. Fehlt ein Antezedent, könnte (\ref{ex:zi88:27}) das liefern, was allgemein mit ,,arbiträrer Referenz'' bezeichnet wird. Demzufolge hätte \textit{schmelzend} in (\ref{ex:zi88:25}) die folgende SF (vgl. (\ref{ex:zi88:24})):

\ea\label{ex:zi88:28} $\hat{z} \; [z \; R \; \varepsilon x_i \; [t = Tx_i] : [x_1 \; \cnst{INST} \; [\textsc{werden} \; [\textsc{flüssig} \; x_2]]]]$
\z 

\noindent $x_2$ ist durch die erwähnte SF-Regel mit seinem Antezedenten, hier \textit{die Schneemassen}, koindiziert und nach (\ref{ex:zi88:27}) aus $[\varepsilon x_2 [P x_2]]$ hervorgegangen. Alles übrige ergibt sich aus den Regeln (\ref{ex:zi88:9}) und (\ref{ex:zi88:10}) und den in (\ref{ex:zi88:4a}) und (\ref{ex:zi88:23}) repräsentierten SF-Anteilen der betrachteten adverbiell verwendeten Partizipialgruppe sowie der Default-Interpretation für den phonologisch leeren Spezifikator.

Es ist bemerkenswert, daß die hier für die adverbielle Verwendungsweise von Partizipial- und Adjektivgruppen angenommene SF (\ref{ex:zi88:23}) und die Default-In\-ter\-pre\-ta\-tion des phonologisch leeren Spezifikators Fanselows Regeln der Hinzunahme einer geeigneten (hier einer völlig unspezifizierten) se\-man\-ti\-schen Relation und der Existenzqualifizierung (hier des referentiellen Arguments verbaler Konstruktionen) entsprechen.\footnote{Siehe \citet{Fanselow1985,Fanselow1986b,Fanselow1986a,Fanselow1988}. Es muß allerdings beachtet werden, daß Fanselow die semantischen Interpretationsprinzipien nicht als zur Grammatik gehörig ansieht. Dieser Standpunkt wird hier ohne weitere Diskussion einfach ignoriert. Von Interesse sind die Prinzipien selbst mit ihren Bezügen auf syntaktische Kategorien und mit ihrer Geltung sowohl für die Wortsyntax wie auch für die Wortgruppensyntax.} Es sind kategorienbezogene SF-Anreicherungen durch sehr allgemeine SF-Anteile, die die SF infiniter Satzeinbettungen in semantische Parallelität zu entsprechenden finiten Satzeinbettungen bringen, in denen die betreffenden Bedeutungskomponenten insbesondere in der Konjunktion \textit{daß} als Spezifikatorausdruck und in adverbiellen Präpositionen (Konjunktionen) als spezielleren Relationsausdrücken ihre for\-ma\-ti\-vi\-schen Träger haben.\footnote{Zur SF von \textit{daß} und temporaler Präpositionen und Konjunktionen siehe \citet{Steube1987-druck,Steube1987}.} Dasselbe gilt auch für das Verhältnis des in attributiven Partizipial- und Adjektivgruppen angenommenen phonologisch leeren Relativpronomens zu lexikalischen Relativpronomen im Nominativ wie auch für die komplementäre Distribution des in adverbiell verwendeten Partizipial- und Adjektivgruppen vorausgesetzten phonologisch leeren PRO-Subjekts und entsprechender lexikalisch ausgefüllter Substantivgruppen im Nominativ.

Freilich ist es trotz der entsprechenden Prinzipien von \citet{Chomsky1981,Chomsky1982,Chomsky1986} und \citet{Emonds1985,Emonds1987} eine weiterer Klärung bedürftige Frage, in welchen Fällen paradigmatischer und syntagmatischer Gegebenheiten pho\-no\-lo\-gisch leere Konstituenten zuzulassen sind. Dabei muß der relativen Autonomie der Semantik gegenüber der Syntax Rechnung getragen werden. Bezüglich der SF-Anreicherung (\ref{ex:zi88:23}) ist zu prüfen, für welche anderen Kon\-struk\-tions\-ty\-pen sie gegebenenfalls auch zutrifft und wie ihre Konstruktionsabhängigkeit entsprechend generalisierend zu charakterisieren wäre. Modifikatorische Genitivphrasen beispielsweise kämen für eine Erweiterung des Geltungsbereichs der SF-Kom\-po\-nen\-te (\ref{ex:zi88:23}) in Betracht.

Entsprechend der Problemstellung dieser Arbeit wurde an den produktiven Suffixen \textit{-en} und \textit{-end} versucht zu zeigen, daß die mit ihnen gebildeten de\-ri\-vier\-ten Wörter mit modifizierter SF und veränderter morpho-syntaktischer Ka\-te\-go\-ri\-sie\-rung als nichttransformationelle Produkte des Lexikons angesehen werden können. Das war möglich, weil bestimmte SF-Komponenten der mit diesen Derivaten als Kern gebildeten Konstruktionen von der SF der betrachteten Suffixe separiert und in allgemeinere Zusammenhänge der semantischen Interpretation referierender und modifikatorischer Ausdrücke einbezogen wurden.

Keine der hier gemachten Annahmen gilt ausschließlich für die Behandlung des Infinitiv- bzw. Partizipformativs. Vielmehr nutzen die unterbreiteten Ana\-ly\-se\-vor\-schlä\-ge Theorieansätze, die für viele andere Erscheinungen der Laut-Be\-deu\-tungs-Zuordnung deutscher Satzkonstruktionen Gültigkeit haben. Mehr noch: Es wurde versucht, für die einzelnen Lösungen streng im Rahmen des vorgegebenen Faktenbereichs des Deutschen zu argumentieren. Es scheint jedoch nicht verfehlt zu behaupten, daß die Darlegungen auch auf vergleichbare Konstruktionen englischer und russischer Infinitiv- und Partizipialkonstruktionen -- mit nur geringfügigen sprachspezifisch bedingten Verfeinerungen -- angewendet werden können.\footnote{Für slawische Sprachen siehe \citet{Ruzicka1978,Ruzicka1980,Ruzicka1982,Ruzicka1986,Ruzicka1987}.}

%\section*{Abbreviations}
%\begin{tabularx}{.5\textwidth}{@{}lQ@{}}
%AR  & Absoluter Rang \\
%D-Struktur & ---\\
%DS  & \\
%LF &  Logische Form\\
%PAS & Prädikat-Argument-Struktur\\
%SF & Semantische Form \\
%\end{tabularx}%
%\begin{tabularx}{.5\textwidth}{@{}lQ@{}}
%S-structure & --- \\
%O-Struktur & --- \\
%VK  & Verbkomplex\\
%\end{tabularx}

%\section*{Acknowledgments}
%Place your acknowledgements here and funding information here.

\printbibliography[heading=subbibliography,notkeyword=this]
\end{otherlanguage}
\end{document}
