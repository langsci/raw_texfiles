\documentclass[output=paper,colorlinks,citecolor=brown]{langscibook}
\ChapterDOI{10.5281/zenodo.15471437}
\author{Ilse Zimmermann\affiliation{Centre for General Linguistics (ZAS), Berlin}}
\title{Approaching the morphosyntax and semantics of mood}
\abstract{\noabstract}


\IfFileExists{../localcommands.tex}{%hack to check whether this is being compiled as part of a collection or standalone
   \usepackage{langsci-optional}
\usepackage{langsci-gb4e}
\usepackage{langsci-lgr}

\usepackage{listings}
\lstset{basicstyle=\ttfamily,tabsize=2,breaklines=true}

%added by author
% \usepackage{tipa}
\usepackage{multirow}
\graphicspath{{figures/}}
\usepackage{langsci-branding}

   
\newcommand{\sent}{\enumsentence}
\newcommand{\sents}{\eenumsentence}
\let\citeasnoun\citet

\renewcommand{\lsCoverTitleFont}[1]{\sffamily\addfontfeatures{Scale=MatchUppercase}\fontsize{44pt}{16mm}\selectfont #1}
  
   %% hyphenation points for line breaks
%% Normally, automatic hyphenation in LaTeX is very good
%% If a word is mis-hyphenated, add it to this file
%%
%% add information to TeX file before \begin{document} with:
%% %% hyphenation points for line breaks
%% Normally, automatic hyphenation in LaTeX is very good
%% If a word is mis-hyphenated, add it to this file
%%
%% add information to TeX file before \begin{document} with:
%% %% hyphenation points for line breaks
%% Normally, automatic hyphenation in LaTeX is very good
%% If a word is mis-hyphenated, add it to this file
%%
%% add information to TeX file before \begin{document} with:
%% \include{localhyphenation}
\hyphenation{
affri-ca-te
affri-ca-tes
an-no-tated
com-ple-ments
com-po-si-tio-na-li-ty
non-com-po-si-tio-na-li-ty
Gon-zá-lez
out-side
Ri-chárd
se-man-tics
STREU-SLE
Tie-de-mann
}
\hyphenation{
affri-ca-te
affri-ca-tes
an-no-tated
com-ple-ments
com-po-si-tio-na-li-ty
non-com-po-si-tio-na-li-ty
Gon-zá-lez
out-side
Ri-chárd
se-man-tics
STREU-SLE
Tie-de-mann
}
\hyphenation{
affri-ca-te
affri-ca-tes
an-no-tated
com-ple-ments
com-po-si-tio-na-li-ty
non-com-po-si-tio-na-li-ty
Gon-zá-lez
out-side
Ri-chárd
se-man-tics
STREU-SLE
Tie-de-mann
}
    \bibliography{localbibliography}
    \togglepaper[23]
}{}

\begin{document}
\begin{otherlanguage}{english}
\shorttitlerunninghead{Approaching the morphosyntax and semantics of mood}
\maketitle

% %%%%%%%%%%%%%%%%%%%%%%%%%%%%%%%%%%%%%%%%%%%%%%%%%%%%%%%%%%%%%%%
% %%%%%%%%%%%%%%%%%%%%%%%%%%%%%%%%%%%%%%%%%%%%%%%%%%%%%%%%%%%%%%%

\section{The issue}

Within a minimalist framework of sound-meaning correlation, in this chapter I articulate my hypotheses on the difference between sentence mood and verbal mood. Sentence mood relates sentence types to illocutionary types of root sentences, while verbal mood relates the propositional content of root and embedded clauses to worlds. I focus on declarative speech acts in the indicative and the subjunctive verbal moods in Russian. I am especially interested in the paradigmatic and syntagmatic distribution of the subjunctive marker and in its invariant, grammatically determined meaning. I show how verbal mood and tense are interrelated and pay some attention to conditional modifiers.

In the sound-meaning correlation of utterances, we are accustomed to reckoning with reference to eventualities, to their participants $x$ and to time spans $t$. In view of much of the work on event semantics, one could content oneself with extensional semantics. Nevertheless, we have to ask ourselves how intensional factors of meaning come into play (cf. \citeauthor{Montague1970a} \citeyear{Montague1970a}, \citeyear{Montague1970b}, \citeyear{Montague1970c}, \citeyear{Montague1973} [all reprinted in \citealt{Thomason1974}]; \citealt{Heim-Kratzer1998}; \citealt{Fintel-Heim2011}; \citealt{Stechow2012}, and many others).

I would like to ask the question ``Where are the worlds?'' (cf. \citeauthor{Zimmermann2017} \citeyear{Zimmermann2010}, \citeyear{Zimmermann2013}, \citeyear{Zimmermann2015a}, \citeyear{Zimmermann2017}). Clearly people’s worlds are connected with their mental states. But how is this fact reflected in the structure of linguistic utterances? By what means of expression do we refer to worlds -- that is, to people’s mental states?

Within a minimalist framework of sound-meaning correlation, I concentrate on mood in its relation to worlds and time, and I argue for differentiating between sentence mood and verbal mood. The analysis here offers some improvements on my earlier work (\citeauthor{Zimmermann2009} \citeyear{Zimmermann2009}, \citeyear{Zimmermann2010}, \citeyear{Zimmermann2013}).

The Russian system of tense and mood markers and the meaning associated with them deserve special attention, in view of their peculiarities in contrast with other Slavic and non-Slavic languages. I start with declarative root sentences with the indicative and subjunctive moods, as shown in \REF{ex:16:1} and \REF{ex:16:2}, respectively.

\ea\label{ex:16:1}
\gll	V poslednie gody v Potsdame vosstanavl-iva-l=sja gorodskoj dvorec.\\
during last years in Potsdam reerect-\textsc{ipfv}-\textsc{ptcp}.\textsc{pst}[\textsc{m}.\textsc{sg}]=\textsc{refl} town castle \\
\glt `During the last few years in Potsdam the town castle was reerected.'
\z

\ea\label{ex:16:2}
\gll Pri GDR by\textsubscript{\textalpha} gorodskoj dvorec ne vosstanavl-iva-l=sja by{\textsubscript–}\textsubscript{\textalpha}. \\
during GDR would town castle not reerect-\textsc{ipfv}-\textsc{ptcp}.\textsc{pst}[\textsc{m}.\textsc{sg}]=\textsc{refl} would \\
\glt `During the GDR the town castle would not have been reerected.'
\z

\noindent While the preterite indicative in \REF{ex:16:1} is expressed by the inflectional affix -\textit{l}, the subjunctive in \REF{ex:16:2} is composed of this suffix and the clitic particle \textit{by}. I will show how the lexical entries and syntactic configurations determine the modal and temporal interpretation of these sentences.

\section{Basic assumptions}

My considerations are built on a conception of minimalism and on the differentiation between grammatically determined semantic form (SF) and conceptual structure (CS) (\citeauthor{Bierwisch1983} \citeyear{Bierwisch1983}, \citeyear{Bierwisch1985}, \citeyear{Bierwisch2007}; \citealt{Bierwisch-Lang1987}; \citealt{Dolling1997}; \citealt{Lang-Maienborn2011}).

In the correlation between sound and meaning in linguistic expressions, the lexicon plays a central role. Every lexical entry contains a phonetic characterization (except for zero morphemes), a morphosyntactic categorization, and the SF of the pertinent lexical item. I show that the association of certain grammatical formatives with their SF is delayed and conveyed only in the functional domains in the left periphery of the clause.

Syntactic representations are of a purely syntactic nature. For clauses, I assume the following hierarchical domains.\footnote{Cf. \posscitet{Ambar2016} split categories of the left periphery:

\begin{exe}
\ex XP* EvaluativeP AssertiveP XP* FinP TP … (with XP* for dislocated DPs)
\end{exe}

\noindent \textcite{Ambar2016} assumes that EvalP and AssertP are pragmatic categories that allow for semantic decomposition of modality and constrain the influence of pragmatic factors on the semantic interpretation of clauses.
} 

\begin{exe}
\ex (ForceP) CP MoodP TP PolP \textit{v}P *VP
\end{exe}

\noindent As I show later in the chapter, functional domains can be fused. ForceP delivers various illocutionary types of sentences. It is absent in embedded clauses. CP characterizes the various sentence types (\citealt{Brandt-Reis-etal1992}; \citeauthor{Zimmermann2009} \citeyear{Zimmermann2009}, \citeyear{Zimmermann2010}, \citeyear{Zimmermann2013}, \citeyear{Zimmermann2015a}, \citeyear{Zimmermann2017}). In generative grammar, it is not unusual to assume a functional projection MoodP (cf. \citealt{Lohnstein2000}; \citeauthor{Giannakidou2009} \citeyear{Giannakidou2009}, \citeyear{Giannakidou2011}, \citeyear{Giannakidou2011a}, \citeyear{Giannakidou2014}, \citeyear{Giannakidou2016}). I show which kind of meaning is contributed by this functional phrase. TP relates the topic (reference) time \textit{t} to the utterance time $t^0$. In PolP, the decision between affirmation and negation takes place. As for aspect (\citealt{Klein1994}), I assume that it is delivered by the verb.\footnote{Russian deverbal nominalizations like \textit{razrabatyvanie} vs. \textit{razrabotka} `elaboration' are derived lexically and based on the corresponding verb stem plus aspectual markers and their semantics. Therefore, I depart from \textcite{Swart2016}, who assumes the existence of aspect phrases below the tense phrase.
}
The main verb of clauses is the head of VP. Its arguments are merged in \textit{v}P and/or VP. In general, I assume that the syntactic domains \textit{v}P and VP serve to describe a situation with its participants and modifiers, while the functional domains ForceP, CP, MoodP, and TP relate \textit{v}Ps to discourse or to matrix clauses.\footnote{In Slavic grammatical tradition, the factors represented in these functional domains are characterized as components of ``predikativnost'" (`predicativity'), cf. \textcite[177ff.]{Pitsch2014}.
}

As regards morphology, I adhere to a conception according to which the lexicon brings in fully derived and inflected word forms (\citealt{Wunderlich1997}). Thus the finite verb in \REF{ex:16:1} is represented in the lexicon with its word structure \REF{wordstructure}, morphosyntactic categorization \REF{morphosyntax}, and semantics \REF{semantics}.\footnote{For reasons of explicitness, redundant morphosyntactic features are not omitted. Elementary semantic types are $\langle e \rangle$ for various sorts of individuals, $\langle t \rangle$ for propositions, and $\langle s \rangle$ for worlds. Argument positions can be associated with morphosyntactic conditions on the pertinent argument expression.
}

\begin{exe}
\ex	\label{ex:16:4}
\begin{xlist}
	\ex	\mbox{[[[[voz[ stanavl']] iva] l] sja]} \label{wordstructure} 
	\ex	{+}V--N--pf+part+\textit{l}-part+pret--imp--subj--fin+max \label{morphosyntax}
	\ex	$\lambda y_{\langle–\text{neut}–\text{fem}–\text{pl}\rangle} \lambda t \lambda e [[\tau(e) \supseteq t ] \wedge [e\,\,\cnst{inst}\, [\textsc{reerect}\, y\, x]]]$ \newline
		with $\tau \in \langle e, e\rangle, \, \, \supseteq, \textsc{reerect} \in \langle e, \langle e,t\rangle\rangle, \, \, \cnst{inst} \in \langle t, \langle e,t \rangle \rangle$     \label{semantics}
\end{xlist}
\end{exe}

\noindent The clitic \textit{sja} marks the passive voice of the imperfective verb and blocks its external argument position.\footnote{Passivization as an operation on the argument structure of the verb takes place in the lexicon. Other functions of the formative \textit{sja} are left out of consideration here. Cf. \textcite{Fehrmann-Junghanns-etal2010}.
} Imperfectivity of the verb in \REF{ex:16:1} and \REF{ex:16:2} is expressed by the suffix -\textit{iva}. The suffix -\textit{l} expresses preterite tense and \mbox{indicative --} that is, nonimperative and nonsubjunctive mood.\footnote{Morphosyntactic features are chosen in correspondence with their phonetic realization. In Russian, the subjunctive and the imperative are signaled by special overt morphemes. This is not the case with respect to the indicative. Likewise, the neuter and the feminine gender have explicit markers, in contrast to the masculine.
} The word-structure feature $+$max characterizes the verb form in \REF{wordstructure} as capable of being merged in syntax. The SF of the verb consists of the argument structure (thematic grid) and the predicate-argument structure. The argument position $\lambda y$ is associated with selectional agreement requirements $-$neut, $-$fem, and $-$pl for the pertinent argument expression, \textit{gorodskoj dvorec} `the town castle'. The person features $\alpha$I, $\beta$II associated with the external argument position of a verb would characterize it as a finite verb form \citep[129]{Pitsch2014}. Correspondingly, the verb form in \REF{ex:16:4} is categorized as $-$fin. The \textit{l}-participle does not inflect for person, like participles in general.

As in \citeauthor{Pitsch2014} (\citeyear{Pitsch2013}, \citeyear{Pitsch2014}), inflectional affixes are considered to be formal reflexes of their meaning represented by functional (zero) heads.\footnote{On the split between semantics and morphology, cf. \textcite[section 8]{Stechow2012}.
} These heads \mbox{c-command} the verb form and select certain of its morpho-syntactic features \citep[cf.][]{Sternefeld2006}.\footnote{My treatment of selection departs from the common system of feature checking in generative syntax (cf. \citealt[section 8]{Stechow2012} and \citealt{Ambar2016}). It deserves additional study to compare the different approaches to selection, especially with regard to minimality.
} While \REF{ex:16:5} represents the lexical entry of the suffix -\textit{l} in the word structure \REF{ex:16:4}, the functional zero heads T$^0$ in \REF{ex:16:6} and Mood\textsuperscript{0} in \REF{ex:16:8} deliver the respective temporal and modal meaning components.\footnote{I do not use terms like PAST in syntax. Instead the involved constituents have corresponding features like $+$pret, and the functional head is phonetically zero.
}
 

\section{Tense morphology and tense meaning}

Russian has a nonhomogeneous tense system. While the preterite is expressed by an ancient Slavic participle form, the \textit{l}-participle, which agrees with the subject in gender and number and is based on the infinitival stem, nonpreterite verb forms agree with the subject in person and number and are based on the present stem. Thus the suffix -\textit{l} in \REF{ex:16:1} and \REF{ex:16:2} has the following lexical representation:

\begin{exe}
\ex	\label{ex:16:5}
\begin{xlist}
	\ex  /-l/
    \ex $+$part$+$\textit{l}$-$part$+$pret$-$imp($-$subj)\textsubscript{$\beta$}$-$fin$\alpha$max \label{takjest}
    \ex $\lambda P_{\langle +\text{V}–\text{N}–\text{pres} \, \text{stem–max}\rangle} \, \lambda x_{(<–\text{neut}–\text{fem}–\text{pl}>)\alpha} \, \lambda t\lambda e \, [P \, x \, t \, e]$ \newline
    with $P \in \langle e, \langle e, \langle e,t \rangle\rangle\rangle$ \label{truskawka}
\end{xlist}
\end{exe}

\noindent The suffix -\textit{l} selects the infinitival verb stem and adds the features {$+$part}, {$+$\textit{l}-part}, {$+$pret}, {$-$imp}, and optionally {$-$subj} and {$-$fin}.\footnote{I do not agree with \posscitet{Ambar2016} analysis of the Russian subjunctive. She assumes that in Russian subjunctive clauses the particle \textit{by} is combined with the past indicative form of the verb. The past verb form selected by the subjunctive particle \textit{by} is left unspecified with respect to mood, in my system (see \REF{takjest}, \REF{ex:16:6c}, and \REF{nozyczki23}--\REF{universum}).
} The word-structure feature $\alpha$max co-varies with the presence of masculine agreement features in the external argument position. In \REF{ex:16:1} and \REF{ex:16:2}, the \textit{l}-participle agrees with the masculine \mbox{($-$neut$-$fem$-$pl)} subject and counts as a {$+$max} verb form. The semantic representation of the \textit{l}-participle in \REF{truskawka} amounts to an empty function, at the level of word structure.  Its temporal semantic contribution is delivered by a zero formative in T\textsuperscript{0}.\footnote{In current generative syntax, the feature {$+$pret} in \REF{takjest} would be u(ninterpretable)past, which is checked by the feature i(nterpretable)past in T\textsuperscript{0}, instead of the selection feature {$+$pret} in \REF{ex:16:6c}. \label{footnote11}
}

\begin{exe}
\ex	\label{ex:16:6}
\begin{xlist}
	\ex /$\emptyset$/
    \ex +T
    \ex $\lambda P _{\langle +\text{pret}–\text{imp}(–\text{subj})\alpha \rangle} \, \lambda t\lambda e \, [([t < t^0] \, \wedge \, )_{\alpha}[P\,t\,e]] $ \newline
    with $P \in \langle e \langle et \rangle\rangle$ \label{ex:16:6c}
\end{xlist}
\end{exe}

\noindent The functional zero head T\textsuperscript{0} selects a preterite complement and relates its topic time $t$ to the utterance time $t^0$. This relation characterizes the topic time as being before the utterance time, but only in the indicative ({$-$imperative}, {$-$subj}). In the subjunctive ({$-$imperative}, {$+$subj}), this relation is absent (see \sectref{sec:zi16:subjunctive}). 

\section{The analysis of mood}

In the analysis of mood, the left periphery of clauses with its functional domains TP, MoodP, CP, and ForceP is important.

\subsection{Verbal mood: The indicative}

Verbal mood relates propositions to mental models (\citealt{Lohnstein2000}) by reference to worlds $w$ and/or situations $\sigma$ and binds the referential argument $e$ of verbs. In situation semantics, \citeauthor{Kratzer1989} (\citeyear{Kratzer1989}, \citeyear{Kratzer2004}, \citeyear{Kratzer2011}) regards worlds as maximal situations, $\sigma \le w$, and propositions of type $\langle s, t \rangle$ as sets of possible situations. In this sense, one could understand intensionalization of propositions of type $\langle t \rangle$ as in \REF{wieza1} or \REF{wieza2} (see fn \REF{kremn}).

\begin{exe}
\ex	\label{ex:16:7}
\begin{xlist}
	\ex   $^{\wedge}p=\lambda w \, [\ldots w \ldots]$ \label{wieza1}
    \ex   $^{\wedge}p=\lambda \sigma \, [\ldots \sigma \ldots]$ \label{wieza2}
\end{xlist}
\end{exe}

\noindent In the following, I take \REF{wieza1} for granted. In Russian, verbal mood is realized as indicative, subjunctive, or imperative. Their respective SFs are brought in by the functional zero head Mood\textsuperscript{0}. In \REF{ex:16:8}, the indicative verb form is associated with its meaning.

\begin{exe}
\ex	\label{ex:16:8}
\begin{xlist}
	\ex /$\emptyset$/
    \ex +Mood
    \ex $\lambda P_{\langle –\text{imp}–\text{subj} \rangle} \, (\lambda w) \, \exists e \, [[P \, t \, e \,](w)]$ \newline
    with $w \in \langle s \rangle, P \in \langle e, \langle e, t \rangle\rangle $ \label{ex:16:8c}
\end{xlist}
\end{exe}

\noindent The functional zero head Mood\textsuperscript{0} selects an indicative complement, blocks its topic-time argument position,\footnote{Unbound variables can be activated by lambda abstraction in SF, or they are specified, co-indexed, or existentially bound in CS. In indicative root clauses like \REF{ex:16:1}, $t$ is a parameter and gets existentially bound in CS.
} binds the referential eventuality argument $e$, and optionally relates the pertinent proposition to worlds. The latter operation takes place in cases of intensionalization. Since in example \REF{ex:16:1} MoodP is the complement of the declarative operator, which selects {} $^{\wedge}p$ (see \sectref{sentencemood}), MoodP must deliver a world-related complement.

\subsection{Sentence mood} \label{sentencemood}

While \textcite{Brandt-Reis-etal1992} and \citeauthor{Reis1997} (\citeyear{Reis1997}, \citeyear{Reis1999}) regard CP as the highest functional projection for root and embedded clauses, I have deviated from this conception since 2004 with my contribution ``Satzmodus'' (published in \citeyear{Zimmermann2009}). I accept Krifka's (\citeyear{Krifka2001}, \citeyear{Krifka2004}, \citeyear{Krifka2013}) assumption of illocutionary type operators for root clauses converting propositions of type $\langle s, t \rangle$ into illocutionary types. This takes place in ForceP, as shown in \REF{ex:16:9}.

\begin{exe}
\ex	\label{ex:16:9}
\begin{xlist}
	\ex /$\emptyset$/
    \ex +Force 
    \ex $\lambda ^{\wedge}p  $ [\textsc{declar/quest/exclam/dir} $ \, ^{\wedge}p ]$ \newline
    with \textsc{declar, quest, exclam, dir} $\in \langle\langle s, t \rangle, a\rangle $
\end{xlist}
\end{exe}

\noindent Example  \REF{ex:16:9} contains lexical entries for $+$Force zero morphemes with their respective SF. The corresponding illocutionary-type operators are parameters that vary according to the social-cultural and linguistic context. They allow one to derive the commitments and modal mental states (see \REF{ex:16:11}--\REF{kukulka}) connected with the pertinent speech act (for parameters in SF and their specification, see \citealt{Dolling1997}). As a rule, Force\textsuperscript{0} and C\textsuperscript{0} are fused in root clauses.

Example \REF{ex:16:10} is the lexical entry of the fused functional heads Force\textsuperscript{0} and C\textsuperscript{0} in declarative root clauses. Syntactically, it represents declarative root clauses as the unmarked sentence type by the feature $-$wh. Semantically, it selects a nonimperative (indicative or subjunctive) intensional propositional argument and combines it with the declarative illocutionary type.\footnote{Corresponding unmarked complement clauses are introduced by the complementizer \textit{čto} `that'. It has the lexical entry \REF{stiftr}. Semantically, it is an empty function (\citeauthor{Zimmermann2015a} \citeyear{Zimmermann2015a}, \citeyear{Zimmermann2016}).

\begin{exe}
\ex	\label{stiftr}
\begin{xlist}
	\ex /čto/
    \ex --Force+C--wh 
    \ex $\lambda p_{\langle –\text{imp} \rangle} [p] $ \newline
    with $p \in \{t, \langle s, t \rangle \} $
\end{xlist}
\end{exe}


} The choice between indicative and subjunctive is free (unselected) in declarative and interrogative root clauses.

\begin{exe}
\ex	\label{ex:16:10}
\begin{xlist}
	\ex /$\emptyset$/
    \ex  $+$Force$+$C$-$wh
    \ex  $\lambda \, ^{\wedge} p_{\langle–\text{imp} \rangle} \, [\textsc{declar} \, ^{\wedge} p ] $
\end{xlist}
\end{exe}

\noindent For my considerations of root declarative clauses, the following meaning postulates (MPs) are important:

\begin{exe}
\ex MP 1: $\forall\, ^{\wedge} p \, [[\textsc{declar} \, ^{\wedge} p] \to \forall w \, [[w \in \, ^{\wedge}p \cap \text{CG}]  \to [w \in M_{\textsc{volit}} \, sp]]] $ \label{ex:16:11}
\end{exe}

\noindent For all assertions, it follows that the speaker wants the pertinent proposition to be in the common ground.\footnote{As for the common ground, see \REF{ex:16:19c} and footnote \ref{fn26}.
} The same is true with questions. But assertions and questions differ with respect to the following MP:

\begin{exe}
\ex MP 2: $\forall \, ^{\wedge} p \, [[\textsc{declar} \, ^{\wedge}p] \to    \forall w \, [[w \in M_{EP} \, sp] \to [w \in \, ^{\wedge}p]]] $ \label{ex:16:12}
\end{exe}

\noindent MP 2 in \REF{ex:16:12} relates declarative root sentences to the epistemic mental model of the speaker $M_{EP} \,  sp$. As will become clear, indicative declarative root clauses open a veridical modal space with respect to the mental state of the speaker (cf. \citealt{Giannakidou2014}, \citeyear{Giannakidou2016}).

I believe that these characterizations coincide with Truckenbrodt’s (\citeyear{Truckenbrodt2006}, \citeyear{Truckenbrodt2006a}) semantic interpretation of the syntactic feature $+$C, $-$wh, but with one difference from my system of assumptions. Truckenbrodt does not assume illocutio\-nary-type operators. For me, there is an important difference between a small universal set of mental models and their specification in the SF of language-specific lexical entries. Therefore, I assume that the following MP applies to veridical verbs like \textit{assume} and \textit{believe}, which describe particular modal mental states:

\begin{exe}
\ex MP 3: \\$\forall  \, ^{\wedge} p \, \forall x \, [[\, \textsc{assume}/\textsc{believe}/ \ldots \} \, ^{\wedge}p \, x] \to \forall w [[w \in M_{\textsc{ep}} \, x] \to [w \in \, ^{\wedge}p]]]       $ \label{kukulka}
\end{exe}

\noindent With these components of morphosyntactic expressions and empty functional categories, the semantic structure of \REF{ex:16:1}, whose syntax is represented in \REF{ex:16:14},\footnote{In root clauses, movements to ForceP/CP as in \REF{ex:16:14} and \REF{ex:16:16} take place in PF in order to mark the beginning of a clause and are not visible in LF.} will be \REF{ex:16:15}.

\begin{exe}
\exr{ex:16:1}[]{
\gll	V poslednie gody v Potsdame vosstanavl-iva-l=sja gorodskoj dvorec.\\
during last years in Potsdam reerect-\textsc{ipfv}-\textsc{ptcp}.\textsc{pst}[\textsc{m}.\textsc{sg}]=\textsc{refl} town castle \\
\glt `During the last few years in Potsdam the town castle was reerected.'}
\end{exe}


\begin{exe}
\ex {} [\textsubscript{ForceP/CP} V poslednie gody$_i$ [\textsubscript{ForceP/CP} $\emptyset$ [\textsubscript{MoodP} $t_i$ [\textsubscript{MoodP} $\emptyset$ \newline [\textsubscript{TP} $t_i$ [\textsubscript{TP} $\emptyset$ [\textsubscript{PolP} $\emptyset$ [\textsubscript{VP} v Potsdame [\textsubscript{VP} [\textsubscript{V} vosstanavlivalsja] \newline [\textsubscript{DP} $\emptyset$ [\textsubscript{NP} gorodskoj dvorec]]]]]]]]]]] \label{ex:16:14}
\end{exe}

\begin{exe}
\ex $\textsc{declar} \, \lambda w \, \exists e \, [[[t \, < t^0] \, \land \, [[\exists!y \, [[\textsc{town castle} \, y] \, \land [[\tau (e) \supseteq t] \, \land $
    \newline
    $ [e \, \cnst{inst} \, [\textsc{reerect} \, y \, x]]]] \, \land [\textsc{loc}(e)  \subset \textsc{potsdam}]] \, \land \, \exists z \, [[\textsc{last years} \, z] \, \land $
    \newline
    $[t \, \subseteq \, z]]]] \, w]$\footnote{The unbound variables $t$ and $x$ in \REF{ex:16:15} are parameters and will be existentially bound in CS. As regards the modifiers and the subject in \REF{ex:16:1}, I assume that they are merged and interpreted as adjuncts of TP and VP and in the complement position of V, respectively.
    }
\label{ex:16:15}
\end{exe}

\subsection{Verbal mood: The subjunctive} \label{sec:zi16:subjunctive}

Now let us turn to the subjunctive (cf. \citealt{Quer1998}; \citealt{Hacquard2006}; \citeauthor{Giannakidou2011} \citeyear{Giannakidou2009}, \citeyear{Giannakidou2011}, \citeyear{Giannakidou2011a}, \citeyear{Giannakidou2014}, \citeyear{Giannakidou2016};  \citealt{Fintel-Heim2011}) and to the morphosyntactic and semantic representation of sentence \REF{ex:16:2}, with its syntactic structure \REF{ex:16:16}.\footnote{\textcite{Isacenko1962}, \textcite{Barnetova-Belicova-Krizkova-etal1979}, and \textcite{Svedova1980}, describe the Russian subjunctive in detail. Concerning generative approaches to mood, see \textcite{Ambar2016}.
} 

\begin{exe}
\exr{ex:16:2}[]{
\gll Pri GDR by\textsubscript{\textalpha} gorodskoj dvorec ne vosstanavl-iva-l=sja by{\textsubscript–}\textsubscript{\textalpha}. \\
during GDR would town castle not reerect-\textsc{ipfv}-\textsc{ptcp}.\textsc{pst}[\textsc{m}.\textsc{sg}]=\textsc{refl} would \\
\glt `During the GDR the town castle would not have been reerected.'}
\end{exe}

\begin{exe}
\ex {} [\textsubscript{ForceP/CP} Pri GDR$_i$ [\textsubscript{ForceP/CP} $\emptyset$ [\textsubscript{MoodP} $t_i$ [\textsubscript{MoodP} by$_\alpha$ $\emptyset$ [\textsubscript{TP} $\emptyset$ \newline [\textsubscript{PolP} [\textsubscript{DP} $\emptyset$ [\textsubscript{NP} gorodskoj dvorec]]$_j$ [\textsubscript{PolP} ne \newline [\textsubscript{VP} [\textsubscript{V'}  [\textsubscript{V}[\textsubscript{V} vosstanavlivalsja] by$_{–\alpha}$] $t_j$ ]]]]]]]]]
\label{ex:16:16}
\end{exe}

\noindent In Russian, the subjunctive is expressed by the preterite form of the verb or the infinitive\footnote{Infinitival control constructions with the subjunctive marker \textit{by} are not considered here.
} and the enclitic particle \textit{by}. As shown in \REF{ex:16:16}, the particle is adjacent to the verb or to the first constituent of the clause.\footnote{In colloquial Russian, the particle \textit{by} can occur twice, after the verb and on the left periphery of the pertinent clause.
} In complement or conditional clauses, it is cliticized to the respective complementizer \textit{čto} `that' or \textit{esli} `if'. I assume that \textit{by} is merged in V and can move to Mood.\footnote{\textcite{Migdalski2006} and \textcite{Tomaszewicz2012} assume for Polish that \textit{by} is merged as the head of a special modal functional phrase between TP and NegP (my PolP) and moved to the left periphery. In \textcite{Ambar2016}, the Russian subjunctive marker \textit{by} is merged as the head of EvalP and the complementizer \textit{čto} is raised from FinP to this position.
} Its lexical entry is given in \REF{ex:16:17}.
\largerpage

\begin{exe}
\ex	\label{ex:16:17}
\begin{xlist}
	\ex /by/, V\_, C\_, XP\_
    \ex $+$subj \label{nozyczki23}
    \ex $\lambda P_{\langle \{+\text{pret}/–\text{part}\}–\text{fin}+\text{max} \rangle} \, [P] $ \label{universum}
\end{xlist}
\end{exe}

\noindent The particle selects a maximal nonfinite preterite or nonparticipial (infinitival) verb form and adds the feature $+$subj to it.\footnote{Observe that the selected verb form is unspecified for mood (cf. \REF{ex:16:6c}).} Its subjunctive semantics is delivered by the functional head Mood\textsuperscript{0} in \REF{ex:16:18}. The meaning in \REF{universum} is an empty function.\footnote{As with tense (see footnote \ref{footnote11}), the feature $+$subj in \REF{nozyczki23} could be u(ninterpretable) subj, which is checked by the feature i(nterpretable) subj in Mood, instead of the selection feature $+$subj in \REF{ex:16:18c}.
}

\begin{exe}
\ex	\label{ex:16:18}
\begin{xlist}
\ex /$\emptyset$/
    \ex $+$Mood
    \ex $\lambda P_{\langle +\text{subj} \rangle} \, \lambda w \, \exists e \, [[\sim \negthinspace \negthinspace   [w \, R_\text{conf} \, w_u]]  :  [[P \, t \, e]w]]  $  \newline
    with $w, \, w_{u} \, \in \langle s \rangle, \, P \in \langle e, \langle e,t \rangle \rangle, R_\text{conf} \in \langle s, \langle s, t \rangle \rangle $ 
    \label{ex:16:18c}
\end{xlist}
\end{exe}

\noindent The semantic contribution of the subjunctive is comparable to the indicative and adds the restriction that the worlds $w$ do not conform to the world of the modal subject $u$.\footnote{The relation $R_\text{conf}$ is comparable to the accessibility relation between worlds described by \citeauthor{Kratzer1991} (\citeyear{Kratzer1991}, \citeyear{Kratzer1991b}). Taking into account developments in situation semantics, \REF{ex:16:8c} and \REF{ex:16:18c}, equivalently, could be represented as \REF{ex:16:1infn23} and \REF{ex:16:2infn23}, respectively.

\begin{exe}
\ex    $\lambda P (\lambda \sigma)_{\alpha} \exists e \, [([e \leq \sigma] \, \wedge)_{\alpha} \, [P \, t \, e]]  $ \label{ex:16:1infn23}
\ex    $ \lambda P \lambda \sigma \exists e \, [[[\sim \negthinspace \negthinspace[\sigma \, R_\text{conf} \, \sigma_{u}]] \, \wedge [e \leq \sigma]] \, \wedge  [P \, t \, e]] \newline   $ 
with $ P \, \in \langle e, \langle e,t \rangle \rangle, \sigma, \sigma_{u} \, \in \langle s \rangle, \leq \in \, \alpha, \alpha \in \, \{ \langle s, \langle e,t \rangle, \langle s, \langle s, t \rangle \rangle \}   $ \label{ex:16:2infn23}
\end{exe} \label{kremn}

} This characteristic restriction of the subjunctive corresponds with semantic properties of prospective predicate expressions like \textit{trebovat'} `require' and \textit{dlja togo} `in order to', which embed clauses with \textit{by}.\footnote{The treatment of mood selection in complement clauses demands a thorough examination of the logical properties of the matrix predicates (see \citeauthor{Schwabe-Fittler2014} \citeyear{Schwabe-Fittler2014}, \citeyear{Schwabe-Fittler2014a}). The subjunctive typically occurs also in case of dependence on negated nonfactive verbs. \textcite{Dahl1971} observed a striking parallelism between definiteness and unspecificity of noun phrases and between presupposed and subjunctive clauses. This parallelism is relevant for indicative and subjunctive in relative clauses, too.} The modal subject $u$ will be specified as the speaker in the context of the declarative sentence mood.

As the equivalences in \REF{ex:16:19} show, I regard the proposition $\lambda w \, [[\sim \negthinspace \negthinspace [w \, R_{conf} \, w_{u}]] : [w \in \, ^{\wedge}p]$ as tantamount to $\lambda w \sim \negthinspace  \forall w' \, [[[w' \, \in \, M_{EP} \, u] \, \wedge \, [w' \leq w]] \, \to [w' \in \, ^{\wedge}p]]$-- that is, it is similar to what \textcite{Giannakidou2016} characterizes as nonveridicality, in contrast to $ \lambda w \, \forall w' \, [[[ w' \, \in \, M_{EP} \, u] \, \wedge \, [w' \leq w]] \, \to [w' \in \, ^{\wedge}p]]$ (cf. the meaning postulates in \REF{ex:16:12} and \REF{kukulka}). Moreover, with $u$=speaker, we are dealing here with a negated presupposition.\footnote{It seems conceivable to interpret the subjunctive dependent on emotive predicates in some Romance languages as a negated presupposition -- that is, the speaker is not certain about the truth of the embedded proposition (cf. the assumptions of \citealt{Giannakidou2016}).
}

\begin{exe}
\ex	\label{ex:16:19}
\begin{xlist}
\ex $\forall w \, \exists ^{\wedge}p \, [[[\sim \negthinspace \negthinspace [w \, R_\text{conf} \, w_{u}]] : [w \in \, ^{\wedge}p]] \leftrightarrow \, \sim \negthinspace  \forall w' \, [[[w' \in M_{EP} \, u] \, \wedge \newline [w' \leq w]] \to [w' \in \, ^{\wedge}p]]]       $
\ex $ \forall w \, \exists ^{\wedge}p \, \forall w' \, [[[[w' \in M_{EP} \, u] \wedge [w' \leq w]] \to [w' \in \, ^{\wedge}p]] \leftrightarrow [[w  \in M_{EP} \, u] \wedge [w \in \, ^{\wedge}p]]]    $
\ex $\forall w  \,\exists \, ^{\wedge}p \, \, Gen\,x \, [[[[\textsc{pers} \, x] \wedge [w \in M_{EP} \, x]]  \wedge [w \in \, ^{\wedge}p]] \leftrightarrow [w \in \text{CG}]]$\footnote{Example \REF{ex:16:19c} concerns the common ground (CG). Only if a world $w$ with $[w \in \, ^{\wedge}p]$ belongs to the epistemic mental model of the speaker and the hearer (and possibly others) does it belong to the CG. For different assumptions, cf. \citeauthor{Portner2007} (\citeyear{Portner2007}, \citeyear{Portner2009}); \textcite{Zanuttini-Pak-etal2012}; and \textcite{Ambar2016}.\label{fn26}}\label{ex:16:19c}
\end{xlist}
\end{exe}

\noindent Thus, in contrast to \textcite{Wiltschko2016} and \textcite{Chrisodoulou-Wiltschko2012}, I assume that functional categories do have content and that the semantic characterization of the subjunctive in \REF{ex:16:18c} is the SF pendant of the syntactic feature [$-$coin(cident)] in Infl\textsuperscript{0} in the analysis of the subjunctive by the authors.\footnote{Observe that the indicative (see \REF{ex:16:8}) by itself is not related to a mental model, but only via the MPs 2 and 3 in \REF{ex:16:12} and \REF{kukulka}, which are based on the realization of Force\textsuperscript{0} as DECLAR or on corresponding matrix predicates, respectively. This shows the dependency of mood interpretation on syntactically higher factors. With regard to these interpretations, I fully agree with the findings of \textcite{Wiltschko2016} and \textcite{Chrisodoulou-Wiltschko2012}. Our disagreement concerns the division of labor between morphosyntax and semantics.\label{fn27}}

Now, the preterite form of the selected verb with the subjunctive marker \textit{by} deserves special attention. The preterite functional zero head T\textsuperscript{0} varies in its meaning contribution depending on the value of the feature $\alpha$subj (cf. \REF{ex:16:6c}). The temporal characterization is absent in the subjunctive. This phenomenon is called ``fake preterite" and corresponds to the infinitive, lacking temporal specification.\footnote{Characteristically, embedded infinitival control constructions can be marked by the subjunctive formative \textit{by}, depending on the embedding predicate.
} Thus the preterite verb form \textit{vosstanavlivalsja} in its temporal meaning in \REF{ex:16:1} is characterized by \REF{ex:16:6c} as referring to a time span before the utterance time. This is not the case for this verb form in \REF{ex:16:2}, where it is accompanied by the subjunctive marker \textit{by}. In \REF{ex:16:2} the relation of the topic time $t$ to the utterance time is unspecified in SF and resides in the knowledge of the speaker, including the linguistic context. If examples  \REF{ex:16:1} and \REF{ex:16:2} are considered as a coherent text, the temporal interpretation of the topic time in \REF{ex:16:2}  in CS is inherited from  \REF{ex:16:1}, where $t$ is expressed as being before the utterance time. If \REF{ex:16:2} is considered independently of \REF{ex:16:1}, $t$ is existentially bound, leaving the relation to $t^0$ unspecified in CS -- that is, whether $t$ is to be interpreted as being before $t^0$ or if it is not.\footnote{There is no tense agreement in Russian subjunctive clauses comparable to that in Romance languages.
}  

Thus, the SF of the inner MoodP in \REF{ex:16:16} will be \REF{ex:16:20}.

\begin{exe}
\ex  \label{ex:16:20}
$\lambda w \, \exists e \, [[\sim \negthinspace R_\text{conf} \, w_{u}]] : \exists ! y \, [[[\textsc{town castle} \, y] \, \wedge \sim \negthinspace  [[\tau (e) \, \supseteq t] \newline  
\wedge [e \, \cnst{inst} \, [\textsc{reerect} \, y \, x]]]]w]]  $
\end{exe}

\noindent The semantic contribution of the presence of the subjunctive particle \textit{by}  characterizes the worlds $w$ to which the proposition expressed by the MoodP applies as not conforming with the world of the modal subject $u$.

Observe that in \REF{ex:16:20} there are two different occurrences of negation. The first one is brought in by Pol\textsuperscript{0} and negates the propositional kernel of the clause. The second one is delivered by the restriction connected with the subjunctive mood (cf. \REF{ex:16:18c}). It is relevant that the subjunctive meaning represented by the empty functional category Mood\textsuperscript{0} takes scope over the so-called sentence negation in Pol\textsuperscript{0} (cf. \REF{ex:16:16}).

In a sense, one could regard the zero heads T\textsuperscript{0} and Mood\textsuperscript{0} together as a covert auxiliary verb that conveys discourse-oriented meaning components of the clause. I leave open the question whether  T\textsuperscript{0} and Mood\textsuperscript{0} are fused in syntax, as assumed by \textcite{Pitsch2014}.

\subsection{Conditional modifiers}

First, the conditional adverbial modifier \textit{pri GDR} in example \REF{ex:16:2} will be integrated into the sentence structure. Conditional modifiers are merged and interpreted as adjuncts of MoodP (see the trace $t_{i}$ in \REF{ex:16:16}). The conditional meaning of the preposition \textit{pri} is represented in \REF{ex:16:21}.\footnote{Other meanings of the preposition \textit{pri} are ignored here.
} The semantic template \REF{ex:16:22}, which relates DP meanings to worlds (cf. \citealt{Schwarz2012}), applies to the proper name \textit{GDR} with the resultant representation \REF{ex:16:23}.


\begin{exe}
\ex	\label{ex:16:21}
\begin{xlist}
	\ex /pri/
    \ex $-$V$-$N
    \ex $\lambda p \, \lambda q \, \lambda w \, \forall w' \, [[[p \, w'] \, \to [q \, w']]w]   $ \label{ex:16:21c}
\end{xlist}
\end{exe}

\begin{exe}
\ex  \label{ex:16:22}
$\lambda u \, \lambda w \, \exists z \, [[z \le w] \, \wedge [z=u]]  $
\newline
with $z, \, u \in \langle e \rangle $
\end{exe}

\begin{exe}
\ex  \label{ex:16:23}
$\lambda w \, \exists z \, [[z \le w] \, \wedge [z=GDR]] $
\end{exe}

\noindent With the semantic components of Force\textsuperscript{0}/C\textsuperscript{0} in \REF{ex:16:10}, of \textit{pri} in \REF{ex:16:21}, and of its complement in \REF{ex:16:23}, the SF of sentence \REF{ex:16:2} is \REF{ex:16:24}.

\begin{exe}
\ex  $\textsc{declar} \, \lambda w \, \forall w' \, \exists z \, [[[[z \le w'] %\wedge [z=GDR]]
\newline
\to \exists e \, [[\sim \negthinspace [w'R_\text{conf}w_{u}]] : \exists ! y \, [[[\textsc{town castle} \, y]  \, \wedge \sim \negthinspace [[\tau (e)  \, \supseteq t] \, \wedge  
\newline
[e \, \cnst{inst} \, [\textsc{reerect} \, y \, x]]]]w']]]w]$ \label{ex:16:24}
\end{exe}

\noindent The implication in the scope of the declarative sentence-type operator can be true if both the antecedent and the consequent are not true in the speaker's epistemic mental model. In the SF \REF{ex:16:24}, the nonconformity of $w'$ and $w_{u}$ is not represented for the antecedent. But since -- according to the knowledge of the speaker -- the \textit{GDR} does not belong to the world $w_{u}$ in the given temporal context and, consequently, $w'$ does not conform to $w_{u}$, the implication in \REF{ex:16:24} is true.

In contrast to example \REF{ex:16:2} with the conditional PP \textit{pri GDR}, the corresponding conditional clause in \REF{ex:16:25} with the conjunction \textit{esli} `if' expresses the restriction that the \textit{GDR} does not exist in the world of the speaker.\footnote{On the syntax of conditional clauses see \textcite{Bhatt-Pancheva2006} and \textcite{Tomaszewicz2012}; on the semantics cf. \citeauthor{Kratzer1991} (\citeyear{Kratzer1991}, \citeyear{Kratzer1991b}).
}

\ea\label{ex:16:25}
\gll Esli by GDR suščestvova-l-a, gorodskoj dvorec ne vosstanavl-iva-l=sja by.  \\
if would GDR exist.\textsc{ipfv}-\textsc{ptcp.pst}-\textsc{f.sg} town castle not reerect-\textsc{ipfv}-\textsc{ptcp.pst}[\textsc{m.sg}]=\textsc{refl} would  \\
\glt `If the GDR had existed, the town castle would not have been reerected.'
\z

\noindent The meaning of the conjunction \textit{esli} corresponds to the meaning of \textit{pri} in \REF{ex:16:21c}. Thus the SF of the conditional clause in \REF{ex:16:25} will be \REF{ex:16:26}.

\begin{exe}
\ex $\lambda q \, \lambda w \, \forall w' \, \exists e' \, [[[\sim \negthinspace \negthinspace [w' \, R_\text{conf} \, w_{u}]] : [[[\tau (e') \supseteq t] \, \wedge 
\newline
[e' \cnst{inst} \, [\textsc{ex:16:ist} \, GDR]]]w']] \to [qw']]w]  $   \label{ex:16:26}
\end{exe}

\noindent Examples \REF{ex:16:27}–\REF{ex:16:29} illustrate the possibility of variation in the expression of Russian conditional constructions (\citealt[vol. 2, 104ff]{Svedova1980}).

\ea\label{ex:16:27}
\gll Esli by syn uči-l=sja, mat' by ne ogorča-l-a=s'.  \\
  if would son learn.\textsc{ipfv}-\textsc{ptcp.pst}[\textsc{m.sg}]=\textsc{refl} mother would not worry.\textsc{ipfv}-\textsc{ptcp.pst}-\textsc{f.sg}=\textsc{refl} \\
\glt `If the son would \{learn/have learned\}, the mother would not \{worry/have worried\}.'
\z

\ea\label{ex:16:28}
\gll Uči-l=sja by syn, mat' by ne ogorča-l-a=s'. \\
 learn.\textsc{ipfv}-\textsc{ptcp.pst}[\textsc{m.sg}]=\textsc{refl} would son mother would not worry.\textsc{ipfv}-\textsc{ptcp.pst}-\textsc{f.sg}=\textsc{refl}  \\
\z

\ea\label{ex:16:29}
\gll Uč-i=s' (by) syn, mat' by ne ogorča-l-a=s'. \\
 learn.\textsc{ipfv}-\textsc{imp.2sg}=\textsc{refl} would son mother would not worry.\textsc{ipfv}-\textsc{ptcp.pst}-\textsc{f.sg}=\textsc{refl} \\
\z

\noindent In \REF{ex:16:28}, the subjunctive verb seems to substitute for the conditional conjunction.\footnote{Cf. the corresponding substitution of the conditional conjunction in German:

\begin{exe}
\ex 
\gll \minsp{\{\{} Würde der Sohn lernen / lernte der Sohn\}, \minsp{\{} würde die Mutter nicht besorgt sein / wäre die Mutter nicht besorgt\}. / Hätte der Sohn gelernt, wäre die Mutter nicht besorgt gewesen.\} \\
{} would.3\textsc{sg} the son learn.\textsc{infv} {} learn.3\textsc{sg}.\textsc{pst} the son {} would.3\textsc{sg} the mother \textsc{neg} worried be.\textsc{infv} {} were.3\textsc{sg} the mother \textsc{neg} worried {} had.3\textsc{sg} the son learn.\textsc{ptcp.pst} were.3\textsc{sg} the mother \textsc{neg} worried be.\textsc{ptcp.pst}    \\
\glt ‘Learned the son, would the mother not be worried. / Had the son learned, would the mother not have been worried.’
\end{exe}

} In the colloquial variant \REF{ex:16:29}, the substituting entity is the second-person singular form of the imperative,\footnote{In Russian, the second person singular form of the imperative can be used with explicit singular and plural subjects in conditional constructions like \REF{ex:16:29} and in cases like in \REF{ex:16:infn33}.

\ea\label{ex:16:infn33}
\gll Vse šli guljat', a my uč-i=s'.  \\
  all went walk and we learn.\textsc{ipfv}-\textsc{imp.2.sg}=\textsc{refl} \\
\glt `All took a walk, and we must learn.'
\z
} with or without the subsequent subjunctive particle \textit{by}.

There is a close semantic relationship between the subjunctive and the imperative that allows for their mutual substitution in many cases (see \citeauthor{Zimmermann2009} \citeyear{Zimmermann2009}, \citeyear{Zimmermann2017}; \citealt{Dvorak-Zimmermann2007}). The temporal unspecificity of the subjunctive and its characteristic restriction are valid for the imperative too.\footnote{\textcite{Kaufmann2012} analyzes imperatives as not referring to a time span before the utterance time. She characterizes this restriction as a presupposition.
}


While in \REF{ex:16:28} and \REF{ex:16:29} there is a conditional zero conjunction, to which the raised verb is adjoined, \REF{ex:16:31} and \REF{ex:16:32} syntactically are not sentences  with a conditional clause, in contrast to \REF{ex:16:30}. Nevertheless, the referential parallelism of the subject in \REF{ex:16:31} and \REF{ex:16:32} and the anaphoric pronoun \textit{{\.e}to} ‘this' in \REF{ex:16:30}, as well as its relation to the proposition of the antecedent, deserve special attention.

\ea\label{ex:16:30}
\gll Esli by Sergej menja poseti-l, {\.e}to menja ob-radova-l-o by. \\
if would Sergej me visit.\textsc{pfv}-\textsc{ptcp.pst}[\textsc{m.sg}] this me \textsc{pfv}-please-\textsc{ptcp.pst}-\textsc{n.sg} would   \\
\glt `If Sergej would \{visit/have visited\} me, it would \{please/have pleased\} me.'
\z

\ea\label{ex:16:31}
\gll Poseščenie Sergeja menja ob-radova-l-o by.  \\
 %a/the.
 visit Sergej.\textsc{gen} %of.Sergej 
 me \textsc{pfv}-please-\textsc{ptcp.pst}-\textsc{n.sg} would  \\
\glt `\{A/The\} visit from Sergej would \{please/have pleased\} me.'
\z

\ea\label{ex:16:32}
\gll Kniga menja ob-radova-l-a by. \\
   %a/the.
   book me \textsc{pfv}-please-\textsc{ptcp.pst}-\textsc{f.sg} would \\
\glt `\{A/The\} book would \{please/have pleased\} me.'
\z

\noindent The semantic interpretation of the subject phrases in \REF{ex:16:31} and \REF{ex:16:32} involves reference to possible worlds (situations) that must be related to the worlds (situations) the respective sentences refer to. Moreover, the far-reaching semantic synonymy of constructions like \REF{ex:16:30} and \REF{ex:16:31} with a nominalization must be taken into account.

First, I propose the application of the template \REF{ex:16:22} and of a conditional template, \REF{ex:16:33}, to the noun phrases in \REF{ex:16:31} and \REF{ex:16:32}. The resultant semantic representation for example \REF{ex:16:32} is \REF{ex:16:34}.

\begin{exe}
\ex  \label{ex:16:33}
$\lambda p \, \lambda q \, \lambda w \, \forall w' [[[p \, w' ] \to [q \, w']]w] $
\end{exe}


\begin{exe}
\ex $ \textsc{declar} \, \lambda w \,  \forall w' \, \exists x_{i} \, [[[[\textsc{book} \, x_{i}] \, \wedge \, \exists z \, [[z \le w'] \, [z=x_{i}]]  
\newline
\to \, [[\sim \negthinspace \negthinspace [w' \, R_\text{conf} \, w_{u}]] : \newline
\exists e \, [[\tau (e) \, \subseteq t] \, \wedge \, [e \, \cnst{inst} \, [\textsc{please} \, sp \, x_{i}]]]w']]]w] $
\newline
with $\textsc{book} \, \in \langle e,t \rangle, \, \textsc{please} \, \in \langle \alpha \rangle, \alpha \, \in \, \{e, t\}   $
\label{ex:16:34}
\end{exe}

\noindent A tacit assumption in this analysis is the movement of the subject phrase of \REF{ex:16:32} to MoodP, and then to ForceP/CP (see \REF{ex:16:35}). The last occurrence of its trace $t_i$ is semantically interpreted as $x_i$.

\begin{exe}
\ex {} [\textsubscript{Force/CP} [\textsubscript{DP} $\emptyset$ kniga]$_i$ $\emptyset$ [\textsubscript{MoodP} $t_i$ [\textsubscript{MoodP} $\emptyset$ [\textsubscript{TP} $\emptyset$ [\textsubscript{PolP} $\emptyset$ [\textsubscript{\textit{v}P} $t_i$ menja obradovala by]]]]]]  \label{ex:16:35}
\end{exe}

\noindent The syntactic derivation and semantic interpretation of \REF{ex:16:31} proceed analogously. Thus the semantic representation of \REF{ex:16:31} will be \REF{ex:16:36}.

\begin{exe}
\ex $\textsc{declar} \, \lambda w \, \forall w' \, \exists x_{i} \, [[x_{i} \, \cnst{inst} \, [\textsc{visit} \, sp \, \textsc{Sergej}]] \, \wedge \, \exists z \, [[z \le w'] 
\newline
\wedge \, [z=x_{i}]] \, \to \, \exists e \, [[\sim \negthinspace \negthinspace [w' \, R_\text{conf} \, w_{u}]] : [[\tau (e)  \subseteq t] \, \wedge \, \newline
[e \, \cnst{inst} \, [\textsc{please} \, sp \, x_{i}]]]w']]w]  $
\newline
with $ \textsc{visit} \, \in \, \langle e,  \langle e,t \rangle \rangle $ \label{ex:16:36}
\end{exe}

\noindent It must be mentioned that DPs in general and as such the nominalization in \REF{ex:16:31}, \textit{poseščenie Sergeja} `Sergej's visit', are not specified for tense or for the reference to mental models. Therefore the antecedent of the implication in \REF{ex:16:34} and \REF{ex:16:36} lacks the modal specification $\sim \negthinspace \negthinspace [w' \, R_\text{conf} \, w_{sp}]$, which can only be added in CS. Thus the implication in the scope of the declarative operator is true and is in accordance with MP 2 in \REF{ex:16:12}.

\begin{exe}
\exr{ex:16:12}[]{
MP 2: $\forall \,  ^{\wedge} p \, [[\textsc{declar} \, ^{\wedge}p] \to    \forall w \, [[w \in M_{EP} \, sp] \to [w \in \, ^{\wedge}p]]] $}
\end{exe}

\noindent In \REF{ex:16:30}, both the antecedent and the consequent have the modal specification $\sim \negthinspace \negthinspace [w' \, R_\text{conf} \, w_{sp}]$, which is expressed by the subjunctive, represented on the level of SF.

Now, we must ask how the pronominal subject of \REF{ex:16:30} is related to the conditional clause, which is merged and interpreted as adjunct of MoodP and moved to ForceP/CP.\footnote{Cf. \textcite{Schwabe2016} for German.
} As in \textcite{Zimmermann2016} for German \textit{es} `it' , I propose the SF \REF{ex:16:37} for the Russian anaphoric pronoun \textit{{\.e}to} `this'.

\begin{exe}
\ex $\lambda P \, \exists x \, [[x = y] \, \wedge \, [P \, x]]$
\newline
with $P \in \, \langle \alpha,t \rangle, \alpha \, \in \, \{e,t, …\}  $\label{ex:16:37}
\end{exe}

\noindent In CS, the parameter $y$ in this generalized quantifier is co-indexed with a coreferential proposition in the conditional antecedent in \REF{ex:16:30}. This is shown in \REF{ex:16:38}.

\begin{exe}
\ex $\textsc{declar} \, \lambda w \, \forall w' \, \exists e' \, [[[\sim \negthinspace \negthinspace [w' \, R_\text{conf} \, w_{u}]] : [[[\tau (e') \subseteq t'] \, \wedge \, [e' \, \cnst{inst}
\newline
[\textsc{visit} \, sp \, Sergej]]]_{i} \, w']] \to \exists e \, [[\sim \negthinspace \negthinspace [w' \, R_\text{conf} \, w_{u}]] : [[[\tau (e) \subseteq t] \, \wedge
\newline
[e \, \cnst{inst} \, [\textsc{please} \, sp \, x_{i}]]] \, w']]w] $  \label{ex:16:38}
\end{exe}

\subsection{The flexible temporal interpretation of subjunctive clauses}

Finally, I will make some observations on the flexible temporal interpretation of subjunctive clauses. Without context, the examples \REF{ex:16:39} and \REF{ex:16:40} with the subjunctive do not have any specified temporal relation to the utterance time $t\textsuperscript{0}$. I assume that the topic-time argument $t$, unbound in SF, is existentially quantified in CS or specified in dependence on the context, as in \REF{ex:16:2}, on the basis of \REF{ex:16:1}. The clauses in \REF{ex:16:39} and \REF{ex:16:40} can refer to the past or to the non-past. The first one has to do with a counterfactual interpretation of the subjunctive, and the second one with the so-called conditional.\footnote{\textcite{Giannakidou2014} regards the past (in contrast to the future) as a veridical domain. Counterfactual propositions typically have to do with a presupposed -- i.e., veridical -- proposition in the past.
}

\ea\label{ex:16:39}
\gll Boris kupi-l by mašinu. \\
   Boris buy.\textsc{pfv}-\textsc{ptcp.pst}[\textsc{m.sg}] would %a/the.
   car  \\
\glt `Boris would \{buy/have bought\} \{a/the\} car.'
\z

\ea\label{ex:16:40}
\gll Ja izvini-l-a=s' by. \\
   I apologize.\textsc{pfv}-\textsc{ptcp.pst}-\textsc{f.sg}=\textsc{refl} would \\
\glt `I would \{apologize/have apologized\}.'
\z

\noindent In contrast to German, which marks counterfactivity with subjunctive pluperfect verb forms, Russian counterfactive interpretations reside exclusively in the context.

\begin{exe}
\ex \label{ex:16:41}
\begin{xlist}
 \exi{A:}{
\gll Počemu Nina ne \minsp{(} na-\minsp{)$_{\alpha}$} pisa-l-a emu? \\
why Nina not {} \textsc{pfv}- write.\textsc{ipfv}-\textsc{ptcp.pst.ind}-\textsc{f.sg} him \\
\glt `\{Why did Nina not write him/Why has Nina not written him\}?’
}
\exi{B:}{
\gll Esli by Nina zna-l-a ego adres, ona by emu \minsp{(} na\minsp{-)$_{\alpha}$} pisa-l-a.\\
if would Nina know.\textsc{ipfv}-\textsc{ptcp.pst}-\textsc{f.sg} his address she would him {} \textsc{pfv}- write.\textsc{ipfv}-\textsc{ptcp.pst.}-\textsc{f.sg} \\
\glt `If Nina would have known his address, she would have written him.’
}
\end{xlist}
\end{exe}

\begin{exe}
\ex \label{ex:16:42}
\begin{xlist}
 \exi{A:}{
\gll Počemu Nina ne  \minsp{(} na\minsp{-)$_{\alpha}$} piš-et emu? \\
why Nina not {} \textsc{pfv}- write.\textsc{ipfv}-\textsc{prs.ind.3.sg} him \\
\glt ‘Why \{does/will\} Nina not write him?’
}
\exi{B:}{
\gll Esli by Nina zna-l-a ego adres, ona by emu \minsp{(} na\minsp{-)$_{\alpha}$} pisa-l-a.\\
if would Nina know.\textsc{ipfv}-\textsc{ptcp.pst}-\textsc{f.sg} his address she would him {} \textsc{pfv}- write.\textsc{ipfv}-\textsc{ptcp.pst.}-\textsc{f.sg} \\
\glt ‘If Nina knew his address, she would write him.’
}
\end{xlist}
\end{exe}

\noindent From the questions, speaker B knows that Nina has not written/did not write him, in case \REF{ex:16:41}, and that she does not/will not write him, in case \REF{ex:16:42}.


\section{Summing up}

In this chapter I have shown how the morphosyntactic and semantic components of root and conditional clauses determine their semantic form (SF), which is enriched by meaning postulates and on the level of conceptual structure (CS).

Root clauses are characterized by illocutionary-type operators in Force, fused with C. The reference of root and embedded clauses and of noun phrases to worlds (situations) is anchored in the functional categories Mood and D, respectively.

Concerning verbal mood, I have concentrated on the question of what subjunctive is in Russian. I believe that the subjunctive always expresses the restriction that the pertinent world does not conform to the world of the respective modal subject $u$. It can be shown that this is valid for all occurrences of the subjunctive, in both root and embedded clauses. Furthermore, subjunctive clauses are temporally underspecified. Their topic time $t$ is not related to the utterance time $t^0$, in SF. The subjunctive shares these two semantic properties with the imperative. Therefore subjunctive verb forms and the imperative can occur in complementary distribution in many cases. 

Conditional modifiers, PPs or clauses, are considered to be adjuncts to MoodP. They describe circumstances that restrict the set of worlds to which the respective matrix clause refers. I have proposed two templates that accommodate the subject phrase of certain nonembedded clauses to this function.

In the sound-meaning correlation the differentiation between SF (as 
the grammatically determined meaning of the overt and covert morphosyntactic components of clauses) and CS (as a representation of nonlinguistic knowledge) proved essential. In particular, the treatment of the 
topic-time argument as a parameter in SF explains the flexibility of the temporal interpretation of Russian 
subjunctive clauses.

The treatment of the functional categories T and Mood as zero categories with the meaning contribution of selected verb forms opens a window for interesting comparative analyses of languages and language types. Whereas the selected morphosyntactic formatives and their categorizations may vary considerably, the categorial and semantic contribution of the selecting functional categories tends to be universal.

\section*{Abbreviations}

\begin{tabularx}{.45\textwidth}{@{}lQ}
%1 & first person \\
2 & second person \\
3 & third person \\
\textsc{f} & feminine \\
\textsc{imp} & imperative mood \\
\textsc{ind} & indicative mood \\
\textsc{infv} & infinitive \\
\textsc{ipfv} & imperfective aspect \\
\textsc{m} & masculine \\
\end{tabularx}
\begin{tabularx}{.45\textwidth}{lQ@{}}
\textsc{n} & neuter \\
\textsc{neg} & negation \\
\textsc{pfv} & perfective aspect \\
\textsc{prs} & present tense \\
\textsc{pst} & past tense \\
\textsc{ptcp} & participle \\
\textsc{refl} & reflexive \\
\textsc{sg} & singular \\
%\textsc{cg} & common ground \\
%\textsc{cs} & conceptual structure \\
%\textsc{m}\_{EP} & epistemic mental model \\
%\textsc{sf} & semantic form \\
\end{tabularx}

% ---------------------------------

\section*{Acknowledgements}

I would like to thank the organizers of the TRAIT workshop and its participants for the opportunity to present some ideas about the morphosyntax and semantics of time and mood. I profited from discussion of these topics at the ZAS in Berlin; at the Seminar für Slavische Philologie, University of Göttingen; and at the Institute of Slavic Languages and Literatures, University of Potsdam. For critical remarks, I am indebted to two anonymous reviewers, and especially to Anastasia Giannkidou and Hagen Pitsch. For consultation concerning the examples and their interpretation, I thank Anatoli Strigin. \\

\noindent ``Approaching the Morphosyntax and Semantics of Mood'' by Ilse Zimmerman. Reprinted with permission from \emph{Mood, Aspect, Modality Revisited: New Answers to Old Questions}, edited by Joanna Błaszczak, Anastasia Giannakidou, Dorota Klimek-Jankowska, and Krzysztof Migdalski. Published by The University of Chi\-ca\-go Press. © 2016 by The University of Chicago. All rights reserved.

\section*{Editors' note}
The present chapter has been republished with minor correction and type changes made by editors. We thank Anastasia Giannakidou and Jason Merchant who helped us to obtain a copyright reprint permission. 

% %%%%%%%%%%%%%%%%%%%%%%%%%%%%%%%%%%%%%%%%%%%%%%%%%%%%%%%%%%%%%%%
% %%%%%%%%%%%%%%%%%%%%%%%%%%%%%%%%%%%%%%%%%%%%%%%%%%%%%%%%%%%%%%%

\sloppy
\printbibliography[heading=subbibliography,notkeyword=this]

\end{otherlanguage}
\end{document}
