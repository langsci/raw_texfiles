\documentclass[output=paper, colorlinks, citecolor=brown, booklanguage=german]{langscibook} 
\ChapterDOI{10.5281/zenodo.15471439}

\author{Ilse Zimmermann\affiliation{Zentrum für Allgemeine Sprachwissenschaft (ZAS), Berlin}}
\title{Das Korrelat in temporalen Nebensätzen}  
\abstract{The present contribution concerns the German correlate \textit{es} and its suppletive forms, in comparison with Russian \textit{to} in various case forms, as they show up in temporal adverbial clauses. The morphosyntactic and semantic structures of temporal connectives and the relation of temporal clauses to prepositional phrases with proforms and to DPs with relative clauses will be
considered.

It is argued that temporal clauses are modifiers of type $\langle it \rangle$ and that many of them contain a relative clause as modifier of a suitable head noun. The correlate functions as a cataphoric entity and is characterized as a definite non-deictic determiner with an additional position for an explicative modifier.
}


\begin{document}
\begin{otherlanguage}{german}
\maketitle

% The part below is from the generic LangSci Press template for papers in edited volumes. Delete it when you're ready to go.

% section 1
\section{Einleitung} \label{sec:18:1}

Die Themenbereiche dieses Beitrags sind temporale Adverbialbestimmungen und Pronomen. Bezüglich der Adverbiale ist zu klären, wie ihre interne Struktur aus\-sieht und worauf sie sich beziehen.

% example 1
\ea \label{ex:18:1} Hier hat es im vorigen Jahr mehrmals länger nicht geregnet.
\z

% example 2
\ea \label{ex:18:2} Seitdem du im Ausland bist, hat es hier mehrmals länger nicht geregnet.
\z

\noindent Spricht etwas dagegen, die Zeitangabe \textit{im vorigen Jahr}, die Frequenzangabe \textit{mehr\-mals} und die Durationsangabe \textit{länger} jeweils mit der Topikzeit $t$ des einbettenden Satzes zu verbinden? Und Temporalsätze wie in \REF{ex:18:2} auch? Diese Fragen sind keinesfalls trivial, zumal man in Sätzen mit der Ereigniszeit $\tau(e)$, der Topikzeit $t$, der Sprechaktzeit $t^0$ und der Evaluationszeit $t^{ev}$ zu rechnen hat (\citealt{Reichenbach1947}; \citealt{Klein1994} und viele andere).

Bezüglich der Pronomen interessiert vor allem, worauf das in \textit{seitdem} eingebaute Korrelat referiert und wie es sich zu im Deutschen gleichlautenden definiten Determinierern verhält. Das Russische bietet eine aufschlussreiche Ver\-gleichs\-ba\-sis.

Das Korrelat \textit{es} und seine Suppletivformen \textit{dessen}, \textit{dem} und \textit{da}(\textit{r}) im Deutschen und \textit{to} in verschiedenen Kasus im Russischen ist mit Komplementsätzen von Verben und Adjektiven assoziiert \citep{Schwabe-Frey-etal2016,Schwabe2013,Sudhoff2016,Zimmermann1993,Zimmermann2015,Zimmermann2016} und in Adverbialsätzen zu finden. Dem Korrelat folgen, wie sich zeigen wird, eine fakultative NP und ein attributiver Nebensatz, mit denen es zusammen von einem Regens abhängig ist. In vielen Adverbialsätzen ist letzteres eine Präposition.

Hier werden deutsche und russische temporale Adverbialsätze sprach\-ver\-glei\-chend betrachtet, in denen mindestens in einer der beiden Sprachen ein Korrelat auftritt bzw. auftreten kann.

% example 3
\ea \label{ex:18:3}
	\ea{ \label{ex:18:3a} Nachdem Peter genesen war, fuhr er ans Meer.
    }
	\ex{ \label{ex:18:3b}
    \gll Posle togo kak Pëtr vyzdorovel, on poexal na more.\\
         nach das.\textsc{gen} wie Peter genesen.\textsc{prät} er fahren.\textsc{prät} auf Meer\\
    \glt     ‘Nachdem Peter genesen war, fuhr er ans Meer.’
    }
	\z
\z

% example 4
\ea \label{ex:18:4}
	\ea{ \label{ex:18:4a} Seit(dem) ich Peter kenne, raucht er.
    }
	\ex{ \label{ex:18:4b}
    \gll S \minsp{\{} togo vremeni / tex роr\} kak ja znaju Petra, on kurit.\\
         seit {} jene.\textsc{gen}.\textsc{sg} Zeit {} jene.\textsc{gen}.\textsc{pl}  Zeiten wie ich kennen.\textsc{präs} Peter er rauchen.\textsc{präs}\\
         \glt     ‘Seitdem ich Peter kenne, raucht er.’
    }
	\z
\z 

% example 5
\ea \label{ex:18:5}
	\ea{ \label{ex:18:5a} (Während der Zeit,) \{als / wie\} Peter studierte, unterstützten ihn die Eltern.
    }
	\ex{ \label{ex:18:5b}
    \gll \minsp{(} V to vremja,) \minsp{\{} kogda / kak\} Pëtr učilsja, roditeli podderživali ego.\\
         {} in jene.\textsc{akk} Zeit {} als {} wie Peter studieren.\textsc{prät} Eltern unterstützen.\textsc{prät} ihn\\
        \glt     ‘Solange Peter studierte, unterstützten die Eltern ihn.’
    }
	\z
\z 

\noindent Die den Temporalsatz einleitenden Konnektive sind morphosyntaktisch und semantisch unterschiedlich transparent. Nach der Präposition zeigt sich in \REF{ex:18:3}--\REF{ex:18:5} ein Korrelat, und zwar ein definiter nicht-deiktischer Determinierer. Ihm folgen als nominaler Kopf einer DP ein in \REF{ex:18:5} overtes Substantiv und ein mit \textit{als} oder \textit{wie} bzw. mit \textit{kogda} oder \textit{kak} beginnender Relativsatz. In \REF{ex:18:3a} und \REF{ex:18:4a} ist in dem Nebensatz von solchen Komponenten nichts sichtbar. Und in \REF{ex:18:5} kann die ein\-lei\-ten\-de PP entfallen, und es fragt sich, wie der verbleibende Relativsatz strukturiert ist.

All diese Eigenarten spielen in der folgenden Analyse eine konstitutive Rolle. Folgende Fragen sollen beantwortet werden:

\begin{itemize} 
\item Worauf bezieht sich der Temporalsatz als Modifikator?
\item Was hat er mit dem Tempus des Matrix- und des Nebensatzes zu tun? 
\item Wie ist die temporale Satzeinleitung morphosyntaktisch und semantisch
aufgebaut? 
\item Was ist die Rolle des Korrelats in Temporal- und in Komplementsätzen?
\item Wie werden mit dem Korrelat korrespondierende Proformen behandelt? 
\item Wodurch wird ein Nebensatz zu einem Attributsatz?
\end{itemize}

\noindent Die folgenden Darlegungen behandeln in Abschnitt \ref{sec:18:2} grammatiktheoretische Voraussetzungen, in  Abschnitt \ref{sec:18:3} Vorschläge zur Analyse von Temporaladverbialen, in Abschnitt \ref{subsec:18:3.1} Präpositionalphrasen als Modifikatoren, in Abschnitt \ref{subsec:18:3.2} mit \textit{seitdem} bzw. \textit{seit} eingeleitete Temporalsätze, in Abschnitt \ref{subsec:18:3.3} den mit \textit{s togo vremeni kak} eingeleiteten Temporalsatz, in Abschnitt \ref{subsec:18:3.4} einen mit \textit{kogda} bzw. \textit{als} oder \textit{wie} eingeleiteten Temporalsatz und in Abschnitt  \ref{subsec:18:3.5} temporale Adverbiale mit Proformen. Zusammenfassung und Ausblick bilden den letzten Abschnitt dieser Arbeit.

% section 2
\section{Grammatiktheoretische Voraussetzungen} \label{sec:18:2}

\noindent Es wird von einem minimalistisch verstandenen Modell der Laut-Bedeu\-tungs\-zuord\-nung ausgegangen, in dem das Lexikon eine zentrale Rolle spielt.

Es gibt für jede Einheit die phonetische Form, die morphosyntaktische Kategorisierung und -- sofern vorhanden -- die Bedeutung an. Ich verfolge ein le\-xi\-ka\-li\-sti\-sches Morphologiekonzept, dem zufolge derivierte und flektierte Wortformen Produkte des Lexikons sind \citep{Wunderlich1997}. Modus-, Tempus- und Aspekt\-semantik ist von den morphologischen Markern separiert und wird durch funktionale Zerokategorien eingebracht \citep{Zeijlstra2004,Gronn-Stechow2010,Gronn-Stechow2012,Gronn-Stechow2013a,Stechow-Gronn2013,Pitsch2014,Pitsch2014a,Zimmermann1990,Zimmermann2013,Zimmermann2015a}.

Zu einem minimalistischen Konzept sprachlicher Bedeutungen gehört auch die wesentliche Frage, wie sich Weltkenntnis der Kommunikationspartner und grammatisch determinierte Bedeutungen sprachlicher Einheiten zueinander verhalten. In dieser Hinsicht teile ich die Unterscheidung von Semantischer Form und Konzeptueller Struktur, wie sie seit Jahren von \citet{Bierwisch-Lang1987,Dolling1997,Bierwisch2007,Lang-Maienborn2011} und vielen Anhängern dieser Konzeption vertreten wird. Und nicht zuletzt mache ich von semantischen Anpassungen (type shifts) bei der Amalgamierung der Bedeutung von Struktureinheiten Gebrauch \citep{Partee1987}. Diese Operationen sind als semantische Nothelfer anzusehen, die bei der semantischen Interpretation zur Verfügung stehen.

In den semantischen Strukturen der wort- und phrasenstrukturellen Komponenten figurieren als Variable $x$, $y$, $z$ für Individuen (Typ $\langle e \rangle$), $e$ für Eventualitäten (Typ $\langle e \rangle$), $t$ für Zeitintervalle (Typ $\langle i \rangle$), ferner $p$ für Propositionen vom Typ $\langle t \rangle$, $w$ für Welten vom Typ $\langle s \rangle$ und Variable für Prädikate, generalisierte Quantoren und intensionalisierte Propositionen, die entsprechend komplexe semantische Typen haben. Es wird sich zeigen, wo diese Variablen mit ihren verschiedenen Bezügen ihren Ursprung haben. Besonderes Augenmerk gilt dem Problem, wo der Bezug auf Zeitintervalle, Eventualitäten (Situationen) und Welten verankert ist, was auch für die Verankerung der verschiedenen Adverbialbestimmungen relevant ist. Ungebundene Variablen bieten bei der semantischen Amalgamierung von Konstituentenbedeutungen die Möglichkeit der Aktivierung durch Lambdaabstraktion. Andernfalls sind es Parameter, die in der konzeptuellen Struktur passend spezifiziert bzw. existenzquantifiziert werden.

Für Verben ist charakteristisch, neben den Partizipanten eine Argumentstelle $e$ für Eventualitäten zu haben. Sie ist das referentielle Argument von Verben. Mit \citet{Bierwisch-Lang1987} nehme ich an, dass der Operator INST das Argument $e$ als Instantiierung einer Proposition charakterisiert.

% example 6
% \ea \label{ex:18:6} \ldots\,$\lambda$e [e INST [\,\ldots\,]]\\
% INST $\in \langle$t $\langle$et$\rangle \rangle$
% \z

\ea\label{ex:18:6} $\dots\,\lambda e[e\,\cnstx{inst}\,[\dots]]$\\
$\cnstx{inst}\in \langle t \langle et \rangle\rangle$
\z


\noindent Mit folgenden Strukturdomänen wird hier in der Satzsyntax gerechnet \citep{Zimmermann2016}:

% example 7
\ea \label{ex:18:7} (PP) CP ModP TP AspP VP
\z

\noindent Es wird sich zeigen, dass zahlreiche adverbielle Nebensätze wie auch andere Mo\-di\-fi\-ka\-to\-ren als PPs zu analysieren sind. CP gilt für Hauptsätze, Komplementsätze und Relativsätze. In ModP erfolgt die Bindung des referentiellen Arguments des Verbs sowie die mögliche Intensionalisierung der Proposition durch den Bezug auf Welten $w$ vom Typ $\langle s \rangle$ \citep{Zimmermann2009,Zimmermann2015a}. TP liefert die Tempusspe\-zi\-fi\-zie\-rung des finiten Verbs. Es handelt sich bei ModP und TP (und im Russischen auch bei AspP) um die semantische Interpretation der modalen, temporalen und gegebenenfalls auch aspektuellen morphologischen Merkmale des Verbs. In AspP wird ein Topikzeitargument $t$ eingeführt und eine Aspektrelation zwischen der Ereigniszeit $\tau(e)$ und der Topikzeit $t$ spezifiziert.\footnote{\label{fn:18:1}Wo die funktionale Domäne PolP für die Unterscheidung von Affirmation und Negation zu platzieren ist, lasse ich hier offen. Das Negationsformativ \textit{nicht} bzw. \textit{ne} wird im Deutschen an das Hauptverb und im Rus\-si\-schen an das jeweils finite Verb komplexer Verbformen adjungiert.}

DPs haben die in \REF{ex:18:8} angegebene Struktur und können, wie sich zeigen wird, bis auf ihren funktionalen Kopf D reduziert werden.

% example 8
\ea \label{ex:18:8} ([\textsubscript{DP})\textsubscript{$\alpha$} [\textsubscript{DP} D (XP)](YP])\textsubscript{$\alpha$}
\z

\noindent Es wird deutlich werden, dass D die Position des Korrelats ist. Der explikative Modifizierer ist YP. Ihm kann XP als nominaler Kern der DP vorausgehen.

% section 3
\section{Vorschläge zur Analyse von Temporaladverbialen} \label{sec:18:3}

\noindent Die folgende Analyse fußt auf Anregungen der Arbeiten von \citet{Beaver-Condoravdi2003,Gronn2015,Gronn-Stechow2010,Gronn-Stechow2012,Paslawska-Stechow2003,Penka-Stechow2008,Tatevosov2015} und \citet{Gronn-Stechow2013a,Stechow-Gronn2013}. Hier sollen der kategorielle Status, die Modifikatorfunktion und die interne morphosyntaktische Struktur von Temporalsätzen und ihre kompositionelle Bedeutung genauer betrachtet werden. Besondere Aufmerksamkeit gilt dem Auftreten und der Rolle des Korrelats.

% sub-section 3.1
\subsection{Präpositionalphrasen als Modifikatoren} \label{subsec:18:3.1}

In der Mehrheit der Fälle sind Bedeutung tragende Präpositionen relationale Köpfe von Modifikatoren. Nach außen konstituieren die durch sie gebildeten PPs einstellige Prädikate, nach innen haben sie eine DP als Komplement \citep{Penka-Stechow2008}:

% example 9
\ea \label{ex:18:9}
	\ea{ \label{ex:18:9a} nach dem Konzert
    }
	\ex{ \label{ex:18:9b}
    \gll posle koncerta\\
         nach Konzert.\textsc{gen} \\
         \glt ‘nach dem Konzert’
    }
	\z
\z 

% example 10
\ea \label{ex:18:10}
	\ea{ \label{ex:18:10a} vor dem Mittagessen
    }
	\ex{ \label{ex:18:10b}
    \gll pered obedom\\
         vor Mittagessen.\textsc{instr}\\
         \glt ‘vor dem Mittagessen’
    }
	\z
\z 

\noindent Bei nicht wenigen Einleitungen adverbieller, nicht nur temporaler Nebensätze bleibt dieser Befund bestehen:

% example 11
\ea \label{ex:18:11}
	\ea{ \label{ex:18:11a} nachdem wir uns getroffen hatten
    }
	\ex{ \label{ex:18:11b}
    \gll posle togo, kak my vstretilis’\\
         nach das.\textsc{gen} wie wir treffen.\textsc{prät}.\textsc{refl}\\
          \glt ‘nachdem wir uns getroffen hatten’
    }
	\z
\z 

% example 12
\ea \label{ex:18:12}
	\ea{ \label{ex:18:12a} bevor wir Mittag gegessen hatten
    }
	\ex{ \label{ex:18:12b}
    \gll pered tem, kak my poobedali\\
         vor das.\textsc{instr} wie wir Mittag.essen.\textsc{prät}\\
         \glt ‘bevor wir Mittag gegessen hatten’
    }
	\z
\z 

% example 13
\ea \label{ex:18:13}
	\ea{ \label{ex:18:13a} damit sich die Kinder normal entwickeln
    }
	\ex{ \label{ex:18:13b}
    \gll dlja togo, čtoby deti normal’no razvivalis’\\
         für das.\textsc{gen} dass.\textsc{subj} Kinder normal entwickeln.\textsc{prät}.\textsc{refl}\\
         \glt ‘damit sich die Kinder normal entwickeln’
    }
	\z
\z

\noindent In \REF{ex:18:11}, \REF{ex:18:12b} und \REF{ex:18:13} tritt an der Spitze der von der Präposition abhängigen DP ein kataphorischer Determinierer auf. Ein nominaler Kopf wie in \REF{ex:18:9} und \REF{ex:18:10} fehlt. Er kann in einigen Fällen passgerecht eingefügt werden, wie in \REF{ex:18:14} (vgl. auch \REF{ex:18:4b}, \REF{ex:18:5} und \REF{ex:18:24}):

% example 14
\ea \label{ex:18:14}
	\ea{ \label{ex:18:14a} mit dem Ziel, dass sich die Kinder normal entwickeln
    }
	\ex{ \label{ex:18:14b}
    \gll s toj cel’ju, čtoby deti normal’no razvivalis’\\
         mit jenes.\textsc{instr} Ziel dass.\textsc{subj} Kinder normal entwickeln.\textsc{prät}.\textsc{refl}\\
         \glt ‘damit sich die Kinder normal entwickeln’
    }
	\z
\z

\noindent In den Finalsätzen \REF{ex:18:13b} und \REF{ex:18:14} ist die attributive Proposition durch die unmarkierte Konjunktion \textit{dass} bzw. \textit{čto} eingeleitet. Im Russischen tritt noch -- ent\-spre\-chend der prospektiven Bedeutung des Finalsatzes -- die Subjunktivpartikel \textit{by} hinzu. In den Temporalsätzen erscheint in den russischen Beispielen \REF{ex:18:3b}, \REF{ex:18:4b} und \REF{ex:18:11b} das Formativ \textit{kak} ‘wie' oder wie in \REF{ex:18:5b} auch \textit{kogda} ‘als'. In den deutschen Temporalsätzen treten das mit den russischen \textit{k}-Wörtern vergleichbare \textit{w}-Element \textit{wie} oder wie in \REF{ex:18:5a} \textit{als} auf.\footnote{\label{fn:18:2}Besonders in süddeutschen Mundarten kann auch in temporalen Nebensätzen der Komplementierer \textit{dass} die eingebettete CP wie in \REF{ex:18:1fn2} einleiten (vgl. \REF{ex:18:2}).

% example (i) in footnote 2
\ea \label{ex:18:1fn2} Seitdem dass du im Ausland bist, hat es hier mehrmals länger nicht geregnet.
\z

} Trotz dieser Unterschiede in den Ausdrucksmitteln gehe ich davon aus, dass auf den kataphorischen De\-ter\-mi\-nie\-rer und gegebenenfalls ein Nomen ein Attributsatz folgt.

In Fällen wie \REF{ex:18:14} und auch \REF{ex:18:24} unten entsteht der Attributsatz durch ein besonderes Template, das auf den Konjunktionalsatz angewendet wird und sich inhaltlich -- auf das Nomen \textit{Ziel} bezogen -- als `das darin besteht' umschreiben lässt (siehe \REF{ex:18:49} im Abschn. \ref{sec:18:4} und \citealt{Zimmermann2015}). In den temporalen Adverbialsätzen mit komplexen Einleitungen ist der abhängige Satz ein Relativsatz vom Typ $\langle it \rangle$. Wie dieser Typ entsteht, ist im Deutschen wie in \REF{ex:18:3a}, \REF{ex:18:4a}, \REF{ex:18:11a} und auch \REF{ex:18:12a} oft verborgen, anders als im Russischen mit der Einleitung \textit{kak} oder \textit{kogda}.

Wir wenden uns nun der näheren Analyse der Beispiele in \REF{ex:18:4} zu.

% sub-section 3.2
\subsection{Mit \textit{seitdem} bzw. \textit{seit} eingeleitete Temporalsätze} \label{subsec:18:3.2}

Was in dem Beispielsatz \REF{ex:18:4a} ausgedrückt ist, hat eine Bedeutung, vermittelt durch die Lexikoneinträge \REF{ex:18:16}—\REF{ex:18:23} und die Syntax \REF{ex:18:15}. Unausgedrücktes kann hinzukommen, durch Zeroformative und/oder Templates.

% \noindent (4) a. Seit(dem) ich Peter kenne, raucht er.

\begin{exe}
\exr{ex:18:4a}[]{Seit(dem) ich Peter kenne, raucht er.}
\end{exe}

% example 15
\ea \label{ex:18:15} [\textsubscript{CP} [\textsubscript{PP} [\textsubscript{P} seit][\textsubscript{DP} [\textsubscript{DP} [\textsubscript{D} \{dem / $\varnothing$\}]][\textsubscript{CP} [\textsubscript{PP} $\varnothing$]\textsubscript{i} [\textsubscript{C'} [\textsubscript{C} $\varnothing$][\textsubscript{ModP} [\textsubscript{Mod} $\varnothing$] \newline
[[\textsubscript{TP} [\textsubscript{T} $\varnothing$][\textsubscript{AspP} [\textsubscript{AspP} [\textsubscript{Asp} $\varnothing$][\textsubscript{VP} ich Peter kenne]] t\textsubscript{i}]]]]]]]\textsubscript{k} \newline
[[\textsubscript{C'} [\textsubscript{C} raucht\textsubscript{j} $\varnothing$]][\textsubscript{ModP} [\textsubscript{Mod} $\varnothing$][\textsubscript{TP} [\textsubscript{T} $\varnothing$][\textsubscript{AspP} [\textsubscript{AspP} [\textsubscript{Asp} $\varnothing$][\textsubscript{VP} er t\textsubscript{j}]] t\textsubscript{k}]]]]]
\z


\noindent Der temporale Modifikator \textit{seit}(\textit{dem}) \textit{ich Peter kenne} figuriert auf der Strukturebe\-ne der Phonetischen Form PF in SpecCP und das finite Verb in C des Hauptsatzes. Die Ausgangsstellung des temporalen Modifikators ist die Adjunktposition von AspP. Hier findet er seinen Modifikanden, die durch Asp eingeführte Topikzeit $t$ des Hauptsatzes. Auch der Nebensatz hat eine PP als Modifikator von AspP, die im Deutschen in vielen Fällen nicht overt ist und den russischen \textit{k}-PPs \textit{kak} bzw. \textit{kogda} entspricht \citep{Gronn-Stechow2012}.

Die für die hier betrachteten Sätze spezifischen funktionalen und le\-xi\-ka\-li\-schen Köpfe haben die Lexikoneinträge \REF{ex:18:16}--\REF{ex:18:23}, jeweils mit der phonologischen Cha\-rak\-te\-ri\-sie\-rung in (a), der morphosyntaktische Kategorisierung in (b), der Semantischen Form in (c) und anschließenden Erläuterungen zu den Angaben.

Begonnen wird in \REF{ex:18:16} und \REF{ex:18:17} mit der Analyse des Komplementierers C in Haupt- und Nebensatz.

% example 16
\ea \label{ex:18:16}
	\ea{ \label{ex:18:16a} $\varnothing$
    }
	\ex{ \label{ex:18:16b} $+$C$+$force$-$Q$-$DIR$+$EF
    }
    \ex{ \label{ex:18:16c} $\lambda p [\cnstx{decl}\,\lambda w [[p] w]]$\\
    $\cnstx{decl} \in\langle\langle st \rangle a\rangle$ 
    }
	\z
\z

\noindent Die funktionale Kategorie C hat in \REF{ex:18:16} das Hauptsätze kennzeichnende Merkmal +force und -Q(uestion) und -DIR(ective) für den illokutiven Satztyp De\-kla\-ra\-tiv\-satz sowie das für die Besetzung der SpecC-Position relevante Merkmal +EF (edge feature). Dieses Merkmal lizensiert auch die V2-Stellung des finiten Verbs. \REF{ex:18:16c} gibt die Bedeutung dieses C an. Sein Komplement ist eine intensionalisierte Proposition, die auf deklarative illokutive Handlungen vom Typ $\langle a\rangle$ abgebildet wird \citep[vgl.][]{Krifka2001}. Für sie ist typisch, dass behauptet wird, dass es eine Welt \textit{w} gibt, in der die betreffende Proposition wahr ist.

% example 17
\ea \label{ex:18:17}
	\ea{ \label{ex:18:17a} \{dass / $\varnothing_\alpha$\}
    }
	\ex{ \label{ex:18:17b} $+$C$-$force($\beta$EF)\textsubscript{$\alpha$}
    }
    \ex{ \label{ex:18:17c} $\lambda p [p]$\\
    $p \in \{t, st\}$
    }
	\z
\z

\noindent C als unmarkierter Komplementierer von Nebensätzen ist in \REF{ex:18:17} repräsentiert. Der Doubly filled Comp-filter schließt fürs Hochdeutsche die Besetzung von SpecC aus (vgl. Fn. \ref{fn:18:2}). Wenn die SpecCP-Position besetzt ist, bleibt C stumm. Dieses C bettet Satzkonstruktionen vom Typ $\langle t \rangle$ bzw. $\langle st\rangle$ ein.\footnote{\label{fn:18:3}In \citet{Zimmermann2015} habe ich zu der Diskussion Stellung genommen, ob es überhaupt propositionale Satzeinbettungen gibt oder ob es sich bei allen Satzeinbettungen um Relativsätze handelt, wie \citet{Moulton2014,Moulton2015} meint (siehe auch \citealt{Capoingro-Polinsky2011} und \citealt{Kratzer2006,Kratzer2011,Kratzer2015}).} Semantisch handelt es sich bei diesem C um eine identische Funktion.

In \REF{ex:18:18}—\REF{ex:18:20} folgen die Lexikoneinträge für die projizierenden Kategorien des Modus, Tempus und Aspekts.

\largerpage[2]
% example 18
\ea \label{ex:18:18}
	\ea{ \label{ex:18:18a} $\varnothing$
    }
	\ex{ \label{ex:18:18b} $+$MOD$-$imp$-$subj
    }
    \ex{ \label{ex:18:18c} $\lambda P (\lambda w)_\alpha \exists e [P\, e](w)_\alpha$
    }
	\z
\z

\noindent Die funktionale Kategorie MOD bindet das referentielle Argument $e$ von Verben, interpretiert die Modusmerkmale imp(erativ) und subj(unktiv) des Verbs und nimmt gegebenenfalls auf Welten $w$ vom Typ $\langle s\rangle$ Bezug \citep{Zimmermann2009,Zimmermann2013,Zimmermann2015a,Zimmermann2016}. In \REF{ex:18:18} handelt es sich um den unmarkierten Indikativ.

% example 19
\ea \label{ex:18:19}
	\ea{ \label{ex:18:19a} $\varnothing$
    }
	\ex{ \label{ex:18:19b} $+$T$-$prät
    }
    \ex{ \label{ex:18:19c} $\lambda P \exists t [[t \supset t^0] \wedge [P\, t]]$
    }
	\z
\z

\noindent Die funktionale Kategorie $T$ interpretiert die Tempusmerkmale des Verbs, in \REF{ex:18:19} das Präsens. Das Topikargument $t$ wird zur Sprechaktzeit $t^0$ in Beziehung gesetzt und wie bei \citeauthor{Gronn-Stechow2010} existentiell gebunden.

% example 20
\ea \label{ex:18:20}
	\ea{ \label{ex:18:20a} $\varnothing$
    }
	\ex{ \label{ex:18:20b} $+$ASP($\alpha$pf)
    }
    \ex{ \label{ex:18:20c} $\lambda P \lambda t [[t\,\cnstx{R}\,\tau(e)] \wedge [P\, e]]$ \newline
    $\cnstx{R} \in \{=,\supseteq_\alpha,\subseteq_{-{\alpha}}\}, \tau \in \langle ei \rangle, P \in \langle et \rangle$
    }
	\z
\z

\noindent Neu in meinen Arbeiten ist die funktionale Strukturdomäne AspP. Der Kopf Asp bringt das Topikzeitargument $t$ ein und setzt es in Beziehung zur Ereigniszeit von $e$.\footnote{\label{fn:18:4} \citeauthor{Gronn-Stechow2010} führen das Topikzeitargument $t$ bereits in der Verbsemantik ein. Das Eventualitätsargument $e$ bleibt außer Betracht.} Im Deutschen ist diese Relation die Identität, im Russischen eine Ent\-hal\-ten\-seins- oder Identitätsbeziehung, je nach der morphologischen Kategorisierung des Verbs. Das deutsche Verb hat keine Aspektkennzeichnung.\footnote{\label{fn:18:5}Auch deverbale Nominalisierungen haben keine Aspektmerkmale \citep{Tatevosov2015}. Entsprechend ist in der DP-Syntax keine AspP zugegen.} Das deutsche Perfekt, Plus\-quam\-per\-fekt und zweite Futur gelten nicht als Aspektrealisierungen.

In \REF{ex:18:21} sind die als adverbielle PP kategorisierte lexikalische Einheit \textit{wie} und
ihr Zeroäquivalent angegeben.\footnote{\label{fn:18:6}+max in \REF{ex:18:21b} ist ein wortstrukturelles Merkmal, das die betreffende Entität als maximale Projektion kennzeichnet.}

% example 21
\ea \label{ex:18:21}
	\ea{ \label{ex:18:21a} \{wie\textsubscript{$\alpha$} / $\varnothing$\}
    }
	\ex{ \label{ex:18:21b} $-$V$-$N$+$ADV($+$EF)\textsubscript{$\alpha$}$+$max
    }
    \ex{ \label{ex:18:21c} $\lambda t' [t'' \supseteq t']$
    }
	\z
\z

\noindent Zur Bedeutung in \REF{ex:18:21c} sei angemerkt, dass sie die Defaultspezifizierung einer Prädikatvariablen, sagen wir von Q (oder äquivalent hier von $\lambda x [Q x]$ mit $x  \in \{e, i,\dots\}$), ist \citep{Zimmermann1995}. Es handelt sich im Kern um einen komparativen Funktor, der zwei Entitäten in eine Identitäts- oder Enthaltenseinsbeziehung setzt. Das Merkmal +EF in \REF{ex:18:21b} ist typisch für \textit{w}- bzw. \textit{k}-Formative, wenn sie in der PF an der linken Satzperipherie figurieren wie in Temporalsätzen.

In \REF{ex:18:22} wird -- stellvertretend für Präpositionen, die in temporalen Kon\-nek\-ti\-ven wie \textit{nachdem}, \textit{seitdem}, \textit{bevor} auftreten, -- die lexikalische Einheit \textit{seit} ana\-ly\-siert.

% example 22
\ea \label{ex:18:22}
	\ea{ \label{ex:18:22a} seit
    }
	\ex{ \label{ex:18:22b} $-$V$-$N$+$ADV
    }
    \ex{ \label{ex:18:22c} $\lambda t'' \lambda t [[\cnstx{init}\,t] \supseteq [\cnstx{init}\,t'']]$\\
    $\cnstx{init} \in \langle ii \rangle$
    }
	\z
\z

\noindent Semantisch setzt \textit{seit} die Anfangsstücke von zwei Zeitintervallen in Beziehung, derart dass $\cnstx{init}\, t$ des Matrixsatzes identisch mit $\cnstx{init}\, t''$ des Modifikators ist oder dieses enthält.

In \REF{ex:18:23} folgt das Kernstück dieser Arbeit, der Lexikoneintrag für den da\-ti\-vi\-schen definiten Determinierer \textit{dem}, wie er in den temporalen Konjunktionen \textit{nachdem} und \textit{seitdem} vorkommt bzw. in dem Konnektiv \textit{seit} phonetisch leer ist. Er gehört zu den Suppletivformen des Korrelats \textit{es}, das ich im Zusammenhang mit Komplementsätzen von Verben untersucht habe \citep{Zimmermann2016}. \REF{ex:18:23} gilt analog auch für das russische Korrelat \textit{to} in den verschiedenen Kasus. Das Russische ist bekanntlich eine artikellose Sprache. Das meint jedoch lediglich das Fehlen eines nicht-korrelativen Definitheitsoperators und eines unspezifischen Indefinitheitsoperators.

% example 23
\ea \label{ex:18:23}
	\ea{ \label{ex:18:23a} \{dem\textsubscript{$\alpha$} / $\varnothing$\}
    }
	\ex{ \label{ex:18:23b} $+$D$+$def$\beta$deikt($+$reg$+$obl$-$fem$-$pl)\textsubscript{$\alpha$}
    }
    \ex{ \label{ex:18:23c} ($\lambda P_1)(\lambda Q)_{-\beta} \lambda P_2 \exists !t [([)_\gamma [P_1\, t] (\wedge [Q\, t])_\gamma] \wedge [P_2\, t]]$ \newline
    $Q, P_1, P_2 \in \langle\delta t \rangle, \delta \in \{e, i, t, st, \dots\}$
    }
	\z
\z

\noindent In \REF{ex:18:23a} ist der funktionale Kopf D einer definiten DP im Dativ bzw. seine Zeroentsprechung angedeutet. \REF{ex:18:23b} gibt die morphosyntaktischen Merkmale des Formativs an. \REF{ex:18:23c} repräsentiert dessen Bedeutung. In jedem Fall resultiert aus der Anwendung dieses komplexen Operators auf seine Kokonstituenten in DP (siehe \REF{ex:18:8}) ein generalisierter Quantor, $\lambda P_2 \exists !t [[\dots t \dots]] \wedge [P_2\, t]]$. Als Ergänzung zur Bedeutung des definiten Artikels mit $P_1$ als Restriktor und $P_2$ als Nukleus weisen pronominale Einheiten wie in \REF{ex:18:23} fakultativ $Q$ als $P_1$ modifizierendes Prädikat auf, so nehme ich an.\footnote{\label{fn:18:7}Die Annahme der zusätzlichen Prädikatvariablen $Q$ in der Bedeutung des Korrelats ist hier neu, in Ergänzung zu der Analyse des Korrelats in \citet{Zimmermann2016}.} Für das Korrelat ist die Unspezifiziertheit des Modifikanden P\textsubscript{1} charakteristisch, aber nicht zwingend, was das Beispiel \REF{ex:18:24} verdeutlicht.

% example 24
\ea \label{ex:18:24} Ich kann [\textsubscript{DP} [\textsubscript{DP} dem ([\textsubscript{NP} Urteil])], dass Peter faul ist,] nicht widersprechen.
\z

\noindent Hier spezifiziert die NP \textit{Urteil} den Restriktor $P_1$ des Determinierers, und der ex\-pli\-ka\-ti\-ve Nebensatz den Modifikator Q. Fehlt der Restriktor in der Syntax, geht $P_1$ als durch den Nebensatz spezifizierter Parameter in die konzeptuelle Interpretation ein.

Bleibt auch $Q$ unspezifiziert, hat man es mit dem deiktischen Determinierer wie in \textit{seitdem}, \textit{währenddessen}, \textit{davor}, \textit{danach} zu tun (von der Form in \REF{ex:18:23a} und von den in \REF{ex:18:23b} fürs Deutsche angegebenen Flexionsmerkmalen von \textit{dem} abgesehen). Das Demonstrativpronomen \textit{dies-} entspricht dieser Bedeutung (siehe auch Abschn. \ref{subsec:18:3.5}).

Außer den in \REF{ex:18:16}--\REF{ex:18:23} angeführten Lexikoneinheiten spielen in der semantischen Interpretation hier zunächst drei type shifts eine Rolle: die Argumentstrukturanpassung \REF{ex:18:25}, die Modifikatorintegrierung \REF{ex:18:26} \citep{Zimmermann1992} und die Lambdaabstraktion \REF{ex:18:27}. Im Gegensatz zu \citeauthor{Gronn-Stechow2010} nehme ich keine Verankerung der Lambdaabstraktion in der Syntax vor. Sie ist eine semantische Operation.

% example 25
\ea \label{ex:18:25} TS\textsubscript{ASA}: $\lambda P \lambda \wp \lambda x_{n-1} \dots \lambda x_1\, [ \wp \lambda x_n [P\, x_n \dots x_1]]$
\newline
$P \in \langle\alpha \langle \dots t \rangle\rangle, \wp \in \langle\langle\alpha t \rangle t \rangle, \alpha \in \{e,i\}$
\z

\noindent Die Argumentstrukturanpassung ist erforderlich, um einen generalisierten Quantor vom Typ $\langle\langle et \rangle t \rangle$ oder $\langle\langle it \rangle t \rangle$ für eine Argumentstelle vom Typ $\langle e \rangle$ bzw. $\langle i \rangle$ einsetzbar zu machen.

% example 26
\ea \label{ex:18:26} TS\textsubscript{MOD}: $\lambda Q_2 \lambda Q_1 \lambda x [[Q_1\, x] \wedge [Q_2\, x]]$ \newline
$Q_2, Q_1 \in \langle\alpha t \rangle, \alpha \in \{e,i\}$
\z

\noindent Das Modifikationstemplate unifiziert zwei Prädikate und wird zuerst auf den Modifikator und dann auf den Modifikanden angewendet.

% example 27
\ea \label{ex:18:27} TS\textsubscript{LA}: $\lambda p \lambda X\, [p]$\\
$X$ von beliebigem Typ
\z

\noindent Die Lambdaabstraktion ermöglicht, eine Variable zunächst ungebunden zu lassen und bei Bedarf später zu aktivieren. Davon profitiert meine Behandlung von Satzkonstruktionen mit \textit{w}-Einheiten wesentlich.

Mit diesen Bauelementen entsteht der temporale Nebensatz des Beispiels \REF{ex:18:4a}, der die syntaktische Struktur \REF{ex:18:28} und die semantische Interpretation \REF{ex:18:29} hat.

% example 28
\ea \label{ex:18:28} [\textsubscript{PP} [\textsubscript{P} seit][\textsubscript{DP} [\textsubscript{DP} [\textsubscript{D} \{dem / $\varnothing$ \}]][\textsubscript{CP} [\textsubscript{C} $\varnothing$][\textsubscript{ModP} [\textsubscript{Mod} $\varnothing$][\textsubscript{TP} [\textsubscript{T} $\varnothing$]\newline[\textsubscript{AspP} [\textsubscript{AspP} [\textsubscript{Asp} $\varnothing$][\textsubscript{VP} ich Peter kenne]][\textsubscript{PP} $\varnothing$]]]]]]]
\z

\noindent Die Zeroformative in D, C, Mod, T, Asp und P in der phonetisch leeren PP transportieren die ihnen durch das Lexikon zugewiesenen Bedeutungen. Die in \REF{ex:18:29} versammelten Konstituentenbedeutungen ergeben schließlich die in \REF{ex:18:29g} re\-prä\-sen\-tier\-te SF des Nebensatzes von \REF{ex:18:4a}.

% example 29
\ea \label{ex:18:29}
	\ea{ \label{ex:18:29a}
	$\parallel$[\textsubscript{VP} ich Peter kenne]$\parallel$ $=$
    \newline
	$\lambda e' [e'\, \cnstx{inst}\, [\textsc{know}\, \textsc{peter}\, sp]]$
    }
	\ex{ \label{ex:18:29b}
	$\parallel$[\textsubscript{AspP} $\varnothing$\textsubscript{Asp} \REF{ex:18:29a}]$\parallel$ $=$
    \newline
	$\lambda t' [[t' = \tau(e')] \wedge [e'\, \cnstx{inst}\, [\textsc{know}\,\textsc{peter}\, sp]]]$
    }
    \ex{ \label{ex:18:29c}
    $\parallel$[\textsubscript{AspP} TS\textsubscript{MOD} ($\parallel$[\textsubscript{PP} $\varnothing$]$\parallel$) \REF{ex:18:29b}]$\parallel$ $=$
    \newline
    $\lambda Q_2 \lambda Q_1 \lambda t' [[Q_1 t'] \wedge [Q_2 t']] (\lambda t' [t'' \supseteq t']) (\lambda t' [[t' = \tau (e')] \wedge \newline [e'\, \cnstx{inst}\, [\textsc{know}\, \textsc{peter}\, sp]]]) \equiv \lambda t' [[[t' = \tau(e')] \wedge \newline [e'\, \cnstx{inst}\, [\textsc{know}\, \textsc{peter}\,sp]]] \wedge [t'' \supseteq t']]$
    }
    \ex{ \label{ex:18:29d}
    $\parallel$[\textsubscript{ModP} $\varnothing$\textsubscript{Mod} (TS\textsubscript{LA} ($\parallel$[\textsubscript{TP} $\varnothing$\textsubscript{T} \REF{ex:18:29c})]$\parallel$))]$\parallel$ $=$
    \newline
    $\lambda P \exists e' [P\, e'] (\lambda p \lambda e' [p] (\lambda P \exists t' [[t' \supset t^0] \wedge [P\,  t']] (\lambda t' [[[t' = \tau(e')] \wedge \newline [e'\, \cnstx{inst}\, [\textsc{know}\, \textsc{peter}\, sp]]] \wedge [t'' \supseteq t']]))) \equiv \newline \exists e' \exists t' [[t' \supset t^0] \wedge [[[t' = \tau(e')] \wedge [e'\, \cnstx{inst}\, [\textsc{know}\, \textsc{peter}\, sp]]] \wedge \newline [t'' \supseteq t']]]$
    }
    \ex{ \label{ex:18:29e}
    $\parallel$[\textsubscript{CP} TS\textsubscript{LA} ($\parallel$[\textsubscript{C}$\varnothing$\textsubscript{C} \REF{ex:18:29d})]$\parallel$)]$\parallel$ $=$
    \newline
    $\lambda p \lambda t'' [p] (\lambda X [X] (\exists e' \exists t' [[t' \supset t^0] \wedge [[[t' = \tau(e')] \wedge \newline [e'\, \cnstx{inst}\, [\textsc{know}\, \textsc{peter}\, sp]]] \wedge [t'' \supseteq t']]])) \equiv$\\
    $\lambda t'' \exists e' \exists t' [[t' \supset t^0] \wedge [[[t' = \tau(e')] \wedge [e'\, \cnstx{inst}\, [\textsc{know}\, \textsc{peter}\, sp]]] \wedge \newline [t'' \supseteq t']]]$
    }
    \ex{ \label{ex:18:29f}
    $\parallel$[\textsubscript{DP} \{dem / $\varnothing$\} \REF{ex:18:29e}]$\parallel$ $=$
    \newline
    $\lambda Q \lambda P_2 \exists !t'' [[[P_1\,  t''] \wedge [Q\,  t'']] \wedge [P_2\,  t'']] (\lambda t'' \exists e' \exists t' [[t' \supset t^0] \wedge \newline [[[t' = \tau(e')] \wedge [e'\, \cnstx{inst}\, [\textsc{know}\, \textsc{peter}\,sp]]] \wedge [t'' \supseteq t']]]) \equiv$\\
    $\lambda P_2 \exists !t'' [[P_1\,  t''] \wedge \exists e' \exists t' [[t' \supset t^0] \wedge [[[t' = \tau (e')] \wedge \newline [e'\, \cnstx{inst}\, [\textsc{know}\, \textsc{peter}\,sp]]] \wedge [t'' \supseteq t']]] \wedge [P_2\, t'']]$
    }
    \newpage
    \ex{ \label{ex:18:29g}
    $\parallel$[\textsubscript{PP} seit \REF{ex:18:29f}]$\parallel$ $=$
    \newline
    $\parallel$[\textsubscript{PP} TS\textsubscript{ASA} ($\parallel$[\textsubscript{P} seit]$\parallel$) \REF{ex:18:29f}]$\parallel$)]$\parallel$\\
    $\lambda P \lambda \wp \lambda t [\wp \lambda t'' [P\,  t''\,  t]] (\lambda t'' \lambda t [\cnstx{init}\, t \supseteq \cnstx{init}\, t'']) (\lambda P_2 \exists !t'' [[[P_1\,  t''] \wedge \newline \exists e' \exists t' [[t' \supset t^0] \wedge [[t' = \tau (e')] \wedge [e'\, \cnstx{inst}\, [\textsc{know}\, \textsc{peter}\,sp]]] \wedge \newline [t'' \supseteq t']]]] \wedge [P_2\,  t'']]) \equiv$\\
    $\lambda t \exists !t''[[[P_1\,  t''] \wedge \exists e' \exists t' [[t' \supset t^0] \wedge [[[t' = \tau (e')] \wedge \newline [e'\, \cnstx{inst}\, [\textsc{know}\, \textsc{peter}\, sp]]] \wedge [t'' \supseteq t']]]] \wedge [\cnstx{init}\, t \supseteq \cnstx{init}\, t'']]$
    }
\z
\z

\noindent Die AspP des Hauptsatzes von \REF{ex:18:4a} ist syntaktisch genauso strukturiert wie die des Nebensatzes. Ihre semantische Interpretation ist in \REF{ex:18:30} repräsentiert.

% example 30
\ea \label{ex:18:30} $\parallel$[\textsubscript{AspP} $\varnothing$\textsubscript{Asp} ($\parallel$[\textsubscript{VP} er raucht]$\parallel$)]$\parallel$ $=$ \newline
$\lambda t [[t = \tau (e)] \wedge [e \,\cnstx{inst}\, [\textsc{smoker}\, x]]]$
\z

\noindent Der type shift \REF{ex:18:26}, TS\textsubscript{MOD}, verknüpft den Modifikator \REF{ex:18:29g} mit dem Modifikanden \REF{ex:18:30} zu einem komplexen Prädikat, das zusammen mit der Bedeutung von T und MOD und dem Illokutionsoperator in C die semantische Struktur \REF{ex:18:31} von Beispiel \REF{ex:18:4a} mit der syntaktischen Struktur \REF{ex:18:15} liefert.

% % example 31
% \ea \label{ex:18:31} \sib{[\textsubscript{CP} $\varnothing$\textsubscript{$+$C$+$force} (\sib{[\textsubscript{ModP} $\varnothing$\textsubscript{Mod} (TS\textsubscript{LA} (\sib{[\textsubscript{TP} $\varnothing$\textsubscript{T} (\sib{AspP})]}))]})]} $=$\\
% $\cnstx{decl}\,\lambda w \exists e \exists t [[[t \supset t^0] \wedge [[[t = \tau (e)] \wedge [e \,\cnstx{inst}\, [\textsc{smoker}\, x]]] \,\mathbf{\wedge}\, \exists !t'' [[[P_1 t''] \wedge \exists e' \exists t' [[t' \supset t^0] \wedge [[[t' = \tau (e')] \wedge [e'\,\cnstx{inst}\, [\textsc{know}\,\textsc{peter}\, sp]]] \wedge [t'' \supseteq t']]]] \wedge [\cnstx{init}\, t \supseteq \cnstx{init}\, t'']]]] w]$
% \z

% example 31
\ea \label{ex:18:31} $\parallel$[\textsubscript{CP} $\varnothing$\textsubscript{$+$C$+$force} ($\parallel$[\textsubscript{ModP} $\varnothing$\textsubscript{Mod} (TS\textsubscript{LA} ($\parallel$[\textsubscript{TP} $\varnothing$\textsubscript{T} ($\parallel$AspP$\parallel$)]$\parallel$))]$\parallel$)]$\parallel$ $=$
\newline
$\cnstx{decl}\,\lambda w \exists e \exists t [[[t \supset t^0] \wedge [[[t = \tau (e)] \wedge [e \,\cnstx{inst}\, [\textsc{smoker}\, x]]]$ \fbox{$\wedge$} 
\newline
$\exists !t'' [[[P_1\,  t''] \wedge \exists e' \exists t' [[t' \supset t^0] \wedge [[[t' = \tau(e')] \, \wedge$
\newline 
$[e'\,\cnstx{inst}\, [\textsc{know}\,\textsc{peter}\, sp]]] \wedge [t'' \supseteq t']]]] \wedge [\cnstx{init}\, t \supseteq \cnstx{init}\, t'']]]] w]$
\z

\noindent Die Bedeutungen des Hauptsatzes und des adverbiellen Nebensatzes sind durch den mit einem Kasten markierten Konnektor $\wedge$ verknüpft. Der Modifikator und der Modifikand haben nach der Prädikatunifizierung das Topikzeitargument $t$ des Hauptsatzverbs gemeinsam. Es wird in T bezüglich der Äußerungszeit $t^0$ temporal eingeordnet.

In der Verknüpfung der Nebensatz-CP mit dem Determinierer \textit{dem} spezifiziert die CP-Bedeutung den Modifikator $Q$ des unspezifizierten Kopfes P\textsubscript{1} der DP (siehe \REF{ex:18:23c}). Im russischen Pendant \REF{ex:18:4b} von \REF{ex:18:4a} hat das Nomen \textit{vremja} `Zeit' die nominale Kopffunktion. Die CP-Bedeutung des Adverbialsatzes ist Attribut und drückt wie im Deutschen ein einstelliges Prädikat aus, das durch den type shift \REF{ex:18:27}, TS\textsubscript{LA}, entsteht.

% sub-section 3.3
\subsection{Der mit \textit{s togo vremeni kak} eingeleitete Temporalsatz} \label{subsec:18:3.3}

\begin{exe}
\exr{ex:18:4b}[]{
\gll S \minsp{\{} togo vremeni / tex роr\} kak ja znaju Petra, on kurit.\\
         seit {} jene.\textsc{gen}.\textsc{sg} Zeit {} jene.\textsc{gen}.\textsc{pl}  Zeiten wie ich kennen.\textsc{präs} Peter er rauchen.\textsc{präs}\\
         \glt     ‘Seitdem ich Peter kenne, raucht er.’}
\end{exe}

%\textbf{EXAMPLE 4, repeated\\
%\\
% example 4
%(4) b. S togo vremeni kak ja znaju Petra, on kurit\\
%seit DEF Zeit wie ich kennen.PRÄS Peter er %rauchen.PRÄS}\\

%\vspace{\baselineskip}

\noindent Im Unterschied zum deutschen Beispiel \REF{ex:18:4a} sind hier der nominale Kopf \textit{vremja} `Zeit' und das overte, mit \textit{w}-Elementen vergleichbare \textit{k}-Formativ \textit{kak} ‘wie' mit ihrer jeweiligen syntaktischen Funktion und Bedeutung zu berücksichtigen. Die syntaktische Struktur ist in \REF{ex:18:32} repräsentiert, die Lexikoneinträge für die beiden zusätzlichen Ausdrücke in \REF{ex:18:33}—\REF{ex:18:34} und die semantische Interpretation von \REF{ex:18:4b} in \REF{ex:18:35}.

% example 32
\ea \label{ex:18:32} [\textsubscript{CP} [\textsubscript{PP} [\textsubscript{P} s][\textsubscript{DP} [\textsubscript{DP} [\textsubscript{D} togo][\textsubscript{NP} vremeni]][\textsubscript{CP} [\textsubscript{PP} kak]\textsubscript{i} [\textsubscript{C'} [\textsubscript{C} $\varnothing$][\textsubscript{ModP}
\newline
[\textsubscript{Mod} $\varnothing$][\textsubscript{TP} [\textsubscript{T} $\varnothing$][\textsubscript{AspP} [\textsubscript{AspP} [\textsubscript{Asp} $\varnothing$][\textsubscript{VP} ja [\textsubscript{V'} [\textsubscript{V} znaju] Petra]]] t\textsubscript{i}]]]]]]\textsubscript{j} 
\newline
[[\textsubscript{C'} [\textsubscript{C} $\varnothing$]][\textsubscript{ModP} [\textsubscript{Mod} $\varnothing$][\textsubscript{TP} [\textsubscript{T} $\varnothing$][\textsubscript{AspP} [\textsubscript{AspP} [\textsubscript{Asp} $\varnothing$][\textsubscript{VP} on [\textsubscript{V} kurit]]] t\textsubscript{j}]]]
\z

\noindent Hier erscheint also die Nebensatz-CP als Attribut zu der DP mit dem Nomen \textit{vremeni} in NP. An der Spitze dieser CP figuriert das als PP kategorisierte \textit{k}-Element \textit{kak}.\footnote{\label{fn:18:8}Während im Russischen die nebensatzinterne PP \textit{kak} in PF an der CP-Spitze figuriert, unterbleibt im Deutschen die Bewegung der entsprechenden phonetisch leeren PP (vgl. \REF{ex:18:28}).} Es ist nicht Komplementierer, sondern ein Modifikator von AspP. Durch ein EF-Merkmal wird er in PF nach SpecCP transportiert, wo er für die semantische Interpretation nicht sichtbar ist.

% example 33
\ea \label{ex:18:33}
	\ea{ \label{ex:18:33a} vremeni
    }
	\ex{ \label{ex:18:33b} $-$V$+$N$+$gen$-$pl
    }
    \ex{ \label{ex:18:33c} $\lambda t [\textsc{time}\, t]$\\
    $\textsc{time}\, \in \langle it \rangle$
    }
	\z
\z

% example 34
\ea \label{ex:18:34}
	\ea{ \label{ex:18:34a} kak
    }
	\ex{ \label{ex:18:34b} $-$V$-$N$+$ADV$+$EF$+$max
    }
    \ex{ \label{ex:18:34c} $\lambda t' [t'' \supseteq t']$ ($=$ \REF{ex:18:21c})
    }
	\z
\z

% example 35
\ea \label{ex:18:35}
	\ea{ \label{ex:18:35a} $\parallel$[\textsubscript{PP} [\textsubscript{P} s][\textsubscript{DP} [\textsubscript{DP} [\textsubscript{D} togo][\textsubscript{NP} vremeni]] CP]]$\parallel$ $=$
    \newline
	$\lambda t \exists !t''[[[\textsc{time}\, t''] \wedge \exists e' \exists t' [[t' \supset t^0] \wedge [[[t' \subseteq \tau (e')] \wedge $
     \newline 
    $[e' \,\cnstx{inst}\, [\textsc{know}\, \textsc{peter}\, sp]]] \wedge [t'' \supseteq t']]]] \wedge [\cnstx{init}\, t \supseteq \,\cnstx{init}\, t'']]$
    }
\ex{ \label{ex:18:35b} $\parallel$[\textsubscript{AspP} $\varnothing$\textsubscript{Asp} [\textsubscript{VP} on\textsubscript{i} kurit]]$\parallel$ $=$
\newline
	$\lambda t [[t \subseteq \tau (e)] \wedge [e \,\cnstx{inst}\, [\textsc{smoker}\, x]]]$
    }
\ex{ \label{ex:18:35c} $\parallel$[\textsubscript{AspP} TS\textsubscript{MOD} \REF{ex:18:35a} \REF{ex:18:35b}]$\parallel$ $=$
\newline
    $\lambda Q_2 \lambda Q_1 \lambda t [[Q_1\,  t] \wedge [Q_2\,  t]]$ \REF{ex:18:35a} \REF{ex:18:35b}  $\equiv$
    \newline
    $\lambda t [[[t \subseteq \tau(e)] \wedge [e \,\cnstx{inst}\, [\textsc{smoker}\, x]]] \wedge \exists !t'' [[[\textsc{time}\, t''] \wedge \newline \exists e'\exists t' [[t' \supset t^0] \wedge [[[t' \subseteq \tau (e')] \wedge [e' \,\cnstx{inst}\, [\textsc{know}\, \textsc{peter}\, sp]]] \wedge \newline [[t'' \supseteq t']]] [\cnstx{init}\, t \supseteq \,\cnstx{init}\, t'']]]]$
    }
    \ex{ \label{ex:18:35d} $\parallel$[\textsubscript{CP} $\varnothing$\textsubscript{+C+force} [\textsubscript{ModP} $\varnothing$\textsubscript{Mod} TS\textsubscript{LA} ([\textsubscript{TP} $\varnothing$\textsubscript{T}\REF{ex:18:35c}])]]$\parallel$ $=$
    \newline
    $\cnstx{decl}\,\lambda w \exists e \exists t [[[t \supset t^0] \wedge [[[t \subseteq \tau (e)] \wedge [e \,\cnstx{inst}\, [\textsc{smoker}\, x ]]] \wedge \newline \exists !t'' [[[\cnstx{\textbf{time}}\, t''] \wedge \exists e' \exists t' [[t' \supset t^0] \wedge [[[t' \subseteq \tau (e')] \wedge \newline [e' \,\cnstx{inst}\,
    [[\textsc{know}\, \textsc{peter}\, sp]]] \wedge [t'' \supseteq t']]]] \wedge [\cnstx{init}\, t \supseteq \,\cnstx{init}\, t'']]]]w]$
    }
	\z
\z

\noindent In der semantischen Struktur \REF{ex:18:35d} des russischen Satzgefüges \REF{ex:18:4b} ist durch Fettdruck hervorgehoben, dass im Vergleich mit der Bedeutung von \REF{ex:18:4a} als Spe\-zi\-fi\-zie\-rung von $P_1$ die Komponente mit der Bedeutung des nominalen Kopfs, $\textsc{time}$, auftritt. Bei dem Vergleich der syntaktischen Struktur für \REF{ex:18:4a} und \REF{ex:18:4b} fällt auf, dass nur in \REF{ex:18:32} für das russische Beispiel \REF{ex:18:4b} eine NP als Komplement von D vorhanden ist. In \REF{ex:18:15} für \REF{ex:18:4a} fehlt sie und $P_1$ des Determinierers bleibt unspezifiziert und geht als Parameter in die konzeptuelle Interpretation des Beispiels ein (siehe \REF{ex:18:29g}).

% sub-section 3.4
\subsection{Ein mit \textit{kogda} bzw. \textit{als} oder \textit{wie} eingeleiteter Temporalsatz} \label{subsec:18:3.4}

\noindent In \citet{zi18:Zimmermann2000} sind die Adverbien \textit{kogda} `wann', \textit{togda} `damals', \textit{vsegda} `immer', \textit{inogda} `manchmal' als semantisch transparente, auf der Wurzel\linebreak \textit{-gda} `\textsc{time}' basierende Adverbiale analysiert, die verschiedene Ope\-ra\-tor\-aus\-drücke enthalten. Entsprechend ist die Bedeutungskomponente $\textsc{time}$ auch für die Satz\-ein\-lei\-tung \textit{kogda} wie in \REF{ex:18:36} und \REF{ex:18:37} konstitutiv, anders als bei der Satzeinleitung  \textit{kak}.

% example 36
\ea \label{ex:18:36}
    \gll Kogda ja obedala, Pëtr spal.\\
    während ich Mittag.aß Peter schlief\\
    \glt ‘Während ich zu Mittag aß, schlief Peter.’
\z

\noindent Hier ist \textit{kogda} Relativadverb und korrespondiert semantisch mit der komplexen Satzeinleitung in \REF{ex:18:37} und auch mit dem gleichlautenden Frageadverb \textit{kogda}.

% example 37
\ea \label{ex:18:37}
    \gll V to vremja \minsp{\{} kak / kogda\} ja obedala, Pëtr spal.\\
    in der Zeit {} wie {} als ich Mittag.aß Peter schlief\\
    \glt ‘Während ich zu Mittag aß, schlief Peter.’
\z

\noindent Anders als in den Arbeiten von \citeauthor{Gronn-Stechow2010} 
sieht meine Analyse für die Satzeinleitungen \textit{kak} und \textit{kogda} jeweils spezifische Bedeutungen vor, und zwar \REF{ex:18:34} für \textit{kak} und \REF{ex:18:38} für \textit{kogda}. Sie unterscheiden sich minimal durch die Anwesenheit des Sortenprädikats \textsc{time} in der Bedeutung von \textit{kogda}, ganz ähnlich wie \textsc{person} bei \textit{kto} `wer'.\footnote{\label{fn:18:9}Der Doppelpunkt in der semantischen Struktur \REF{ex:18:38c} kennzeichnet die Komponente $[\textsc{time}\, t'']$ als Präsupposition.}

% example 38
\ea \label{ex:18:38}
	\ea{ \label{ex:18:38a} kogda
    }
	\ex{ \label{ex:18:38b} $-$V$-$N$+$ADV$+$EF$+$max
    }
    \ex{ \label{ex:18:38c} $\lambda t' [[$\textsc{time}$\, t''] : [t'' \supseteq t']]$
    }
	\z
\z

\noindent Wie bei \textit{kak} ist die Variable $t''$ hier ungebunden, und das Merkmal +EF steuert die Bewegung des Adverbials an die Satzspitze, nach SpecCP (vgl. \REF{ex:18:32} und \REF{ex:18:40}).

\largerpage
Zusammen mit der Präteritalbedeutung \REF{ex:18:39} für die beteiligten Verben im Bei\-spiel \REF{ex:18:36} ergibt sich dessen semantische Interpretation \REF{ex:18:41} mit der syn\-tak\-ti\-schen Struktur \REF{ex:18:40}. Die Präteritalbedeutung wird morphologisch durch das verbale Suffix -\textit{l} signalisiert und in der funktionalen Domäne TP semantisch interpretiert. Die syntaktische Struktur des temporalen Nebensatzes in \REF{ex:18:40} zeigt, dass es sich hier nicht um eine PP handelt wie in den bisher betrachteten Temporalsätzen, sondern um eine CP mit einem \textit{k}-Adverb an der Satzspitze.

% example 39
\ea \label{ex:18:39}
	\ea{ \label{ex:18:39a} $\varnothing$
    }
	\ex{ \label{ex:18:39b} $+$T$+$prät
    }
    \ex{ \label{ex:18:39c} $\lambda P \exists t' [[t' < t^0] \wedge [P\, t']]$
    }
	\z
\z

% example 40
\ea \label{ex:18:40} [\textsubscript{CP} [\textsubscript{CP} [\textsubscript{PP} kogda]\textsubscript{i} [\textsubscript{C'} [\textsubscript{C} $\varnothing$][\textsubscript{ModP} [\textsubscript{Mod} $\varnothing$][\textsubscript{TP} [\textsubscript{T} $\varnothing$][\textsubscript{AspP} [\textsubscript{AspP} [\textsubscript{Asp} $\varnothing$][\textsubscript{VP} ja [\textsubscript{V} obedala]]] t\textsubscript{i}]]]]]\textsubscript{j}
\newline
[\textsubscript{C'} [\textsubscript{C} $\varnothing$][\textsubscript{ModP} [\textsubscript{Mod} $\varnothing$][\textsubscript{TP} [\textsubscript{T} $\varnothing$][\textsubscript{AspP} [\textsubscript{AspP} [\textsubscript{Asp} $\varnothing$][\textsubscript{VP} Pëtr [\textsubscript{V} spal]]] t\textsubscript{j}]]]]]
\z

% example 41
\ea \label{ex:18:41}
	\ea{ \label{ex:18:41a} $\parallel$[\textsubscript{AspP} TS\textsubscript{MOD} ($\parallel$[\textsubscript{PP} kogda]$\parallel$) ($\parallel$[\textsubscript{AspP} [\textsubscript{Asp} $\varnothing$][\textsubscript{VP} ja [\textsubscript{V} obedala]]]]$\parallel$)]$\parallel$
	$=$ \newline $\lambda Q_2 \lambda Q_1 \lambda t' [[Q_1 t'] \wedge  [Q_2\, t']] (\lambda t'[[\textsc{time}\, t''] : [t'' \supseteq t']]) 
 \newline
 (\lambda t' [[t' \subseteq \tau (e')] \wedge  [e' \,\cnstx{inst}\, [\textsc{take}\, \textsc{dinner}\, sp]]]) \equiv$
 \newline
	$\lambda t' [[[t' \subseteq \tau (e')] \wedge [e' \,\cnstx{inst}\, [\textsc{take}\,\textsc{dinner}\, sp]]] \wedge [[\textsc{time}\, t''] : $
    \newline 
    $[t'' \supseteq t']]]$
    }
	\ex{ \label{ex:18:41b} $\parallel$[\textsubscript{CP} TS\textsubscript{LA} ($\parallel$[\textsubscript{C'} [\textsubscript{C} $\varnothing$]($\parallel$[\textsubscript{ModP} [\textsubscript{Mod} $\varnothing$](TS\textsubscript{LA} ($\parallel$[\textsubscript{TP}[\textsubscript{T} $\varnothing$]\REF{ex:18:41a}]$\parallel$))]$\parallel$)]$\parallel$)]$\parallel$ \newline
	$ = \lambda p \lambda t'' [p] (\exists e' \exists t' [[t' < t^0] \wedge [[[t' \subseteq \tau (e')] \wedge $
    \newline 
 $[e' \,\cnstx{inst}\, [\textsc{take}\, \textsc{dinner}\, sp]]] \wedge [[\textsc{time}\, t''] : [t'' \supseteq t']]]]) \equiv $
 \newline
 $ \lambda t'' \exists e' \exists t' [[t' < t^0] \wedge [[[t' \subseteq \tau (e')] \wedge [e' \,\cnstx{inst}\, [\textsc{take}\,\textsc{dinner}\,	sp]]] \wedge $
 \newline 
 $[[\textsc{time}\, t''] : [t'' \supseteq t']]]]$
    }
    \ex{ \label{ex:18:41c} $\parallel$[\textsubscript{AspP} TS\textsubscript{MOD} \REF{ex:18:41b} ($\parallel$[\textsubscript{AspP} [\textsubscript{Asp} $\varnothing$][\textsubscript{VP} Pëtr [\textsubscript{V} spal]]]$\parallel$)]$\parallel$ $=$
    \newline
    $\lambda Q_2 \lambda Q_1 \lambda t [[Q_1\, t] \wedge [Q_2\, t]] (\lambda t''
     \exists t' [[t' < t^0] \wedge $
     \newline
     $ \exists e' \exists t' [[t' \subseteq \tau (e')] \wedge [e' \,\cnstx{inst}\, [\textsc{take}\,\textsc{dinner}\, sp]]] \wedge $
     \newline
     $ [[\textsc{time}\, t''] : [t'' \supseteq t']]]) (\lambda t [[t \subseteq \tau (e)] \wedge [e \,\cnstx{inst}\, [\textsc{sleep}\,\textsc{peter}]]]) \equiv$
     \newline
    $\lambda t [[[t \subseteq \tau (e) \wedge [e \,\cnstx{inst}\, [\textsc{sleep}\,\textsc{peter}]]] \wedge \exists e' \exists t' [[t' < t^0] \wedge $
    \newline
    $ [[[t' \subseteq \tau (e')] \wedge [e' \,\cnstx{inst}\, [\textsc{take}\,\textsc{dinner}\, sp]]] \wedge [[\textsc{time}\, t] : [t \supseteq t']]]]]$
    }
    \ex{ \label{ex:18:41d} $\parallel$[\textsubscript{CP} $\varnothing$\textsubscript{$+$C$+$force} ($\parallel$[\textsubscript{ModP} $\varnothing$\textsubscript{Mod}] (TS\textsubscript{LA}($\parallel$[\textsubscript{TP} $\varnothing$\textsubscript{T} \REF{ex:18:41c}]$\parallel$)])$\parallel$)]$\parallel$ $=$
    \newline
    $\cnstx{decl}\,\lambda w \exists e \exists t [[[t < t^0] \wedge [[t \subseteq \tau (e)] \wedge \newline [e \,\cnstx{inst}\, [\textsc{sleep}\,\textsc{peter}]]]] \wedge \exists e' \exists t' [[t' < t^0] \wedge [[t' \subseteq \tau (e')] \wedge \newline [e' \,\cnstx{inst}\, [\textsc{take}\,\textsc{dinner}\, sp]] \wedge [[\textsc{time}\, t] : [t \supseteq t']]]] w]$
    }
	\z
\z

\noindent Entsprechend dieser Analyse erweist sich das als PP kategorisierte Formativ \textit{kog\-da} nur scheinbar als temporale Konjunktion. Es ist hier als adverbieller Modifikator im Nebensatz behandelt und wandert in PF -- geleitet durch die Kennzeichnung +EF -- nach SpecCP. Der type shift Lambdaabstraktion macht in \REF{ex:18:41b} den Nebensatz zum passenden temporalen Modifikator des Hauptsatzes.\footnote{\label{fn:18:10}Ausdrücklich ist hier zu betonen, dass das benutzte Merkmal +EF als verkürzendes Provisorium angesehen wird. Es ist vermutlich durch geeignete Merkmale für die multifunktionalen \textit{w}- bzw. \textit{k}-Formative zu ersetzen (siehe \citealt{zi18:Zimmermann2000}). In der vorliegenden Arbeit sind diese Einheiten Relativadverbien, deren Kategorisierung mit einem entsprechenden Merkmal in C, z.B. +rel, korrespondiert und die Bewegung nach SpecCP verlangt, die jedoch nicht in allen Sprachtypen zu beobachten ist.}

Bei komplexen Konnektiven wie \textit{v to vremja kogda} oder gleichbedeutendem \textit{togda kogda} besteht Äquivalenz mit den einfachen Konnektiven \textit{kak} und \textit{kogda} an der Spitze der $P_1$ modifizierenden CP. \REF{ex:18:42} und \REF{ex:18:43} zeigen die syntaktische Struktur bzw. die semantische Interpretation (vgl. \REF{ex:18:35d}).

% example 42
\ea \label{ex:18:42} [\textsubscript{CP} [\textsubscript{PP} \{v to vremja / togda\}[\textsubscript{CP} [\textsubscript{PP} kogda]\textsubscript{i} [\textsubscript{C'} [\textsubscript{C} $\varnothing$][\textsubscript{ModP} [\textsubscript{Mod} $\varnothing$]
\newline
[\textsubscript{TP} [\textsubscript{T} $\varnothing$][\textsubscript{AspP} [\textsubscript{AspP} [\textsubscript{Asp} $\varnothing$][\textsubscript{VP} ja [\textsubscript{V} obedala]]] t\textsubscript{i}]]]]]]\textsubscript{j} \newline
[\textsubscript{C'} [\textsubscript{C} $\varnothing$]][\textsubscript{ModP} [\textsubscript{Mod} $\varnothing$][\textsubscript{TP} [\textsubscript{T} $\varnothing$][\textsubscript{AspP} [\textsubscript{AspP} [\textsubscript{Asp} $\varnothing$][\textsubscript{VP} Pëtr [\textsubscript{V} spal]]] t\textsubscript{j}]]]]]
\z

% example 43
\ea \label{ex:18:43} $\cnstx{decl}\,\lambda w \exists e \exists t [[[t < t^0] \wedge [[t \subseteq \tau (e)] \wedge [e \,\cnstx{inst}\, [\textsc{sleep}\,\textsc{peter}]]]] \wedge \newline \exists !t''' [[\cnstx{\textbf{time}}\, \textbf{\textit{t}}'''] \wedge \exists e' \exists t' [[[t' < t^0] \wedge [[t' \subseteq \tau (e')] \wedge [e' \,\cnstx{inst}\, [\textsc{take}\,\textsc{dinner}\, sp]] \wedge \stkout{[[\textsc{time}\, t''] :\, }[t'' \supseteq t']\stkout{]}] \wedge [t \supseteq t''']]]]]w]$
\z

\noindent Der durchgestrichene Teil in der Bedeutungsstruktur \REF{ex:18:43} entspricht der ge\-gen\-über \textit{kak} präsupponierten zusätzlichen Bedeutungskomponente von \textit{kogda} und erweist sich hier als redundant und somit als reduzierbare Komponente der SF. Solche Doppelungen wie in \REF{ex:18:43} findet man in vielen Adverbialsatztypen (\citealt[B 5.5.2]{Pasch-Brausse-etal2003}).

Anschließend an die Betrachtung der mit dem Relativadverb \textit{kogda} bzw. komplexer mit \{\textit{v to vremja / togda}\} \textit{kogda} beginnenden russischen Temporalsätze sollen für das Deutsche Adverbialsätze beleuchtet werden, die mit dem Konnektiv \textit{als} bzw. \textit{wie} eingeleitet sind. Für beide Formative gilt die in \REF{ex:18:21c} angegebene Bedeutung, $\lambda t' [t'' \supseteq t']$. Das temporale Konnektiv \textit{als} unterliegt noch der Bedingung, dass das Tempus des betreffenden Satzes präterital ist.

% example 44
\ea \label{ex:18:44} \{Als / Wie\} ich Mittag aß, schlief Peter.
\z

\noindent Die syntaktische und semantische Struktur dieses Satzgefüges sind analog den für das parallele russische Beispiel in \REF{ex:18:40} und \REF{ex:18:41} angegebenen. Nur das an \textit{kogda} gebundene Sortenprädikat fehlt. Das entspricht auch der Mul\-ti\-funk\-tio\-na\-li\-tät der Formative \textit{als} und \textit{wie}. Mit ihrer äußerst allgemeinen Relation $\supseteq$ treten sie typischerweise in Komparationskonstruktionen und anderen Typen von Vergleichen auf. In temporalen Nebensätzen sind sie auf Zeitintervalle bezogen.

% sub-section 3.5
\subsection{Temporale Adverbiale mit Proformen} \label{subsec:18:3.5}

Korrelate sind DPs mit einem Attributsatz, auf den das Korrelat kataphorisch verweist. Fehlt dieser Attributsatz, tritt anstelle des Korrelats ein anaphorisches Pronomen auf. Im Deutschen sind kataphorische und anaphorische De\-ter\-mi\-nie\-rer meistens homophon, während im Russischen ganz systematisch der Kontrast \textit{to} vs. \textit{ėto} existiert. Vgl. \textit{seitdem} vs. \textit{s ėtogo vremeni}, \textit{danach} vs. \textit{posle ėtogo} usw.

Worauf referiert das Pronomen in solchen Adverbialen? Vgl. \REF{ex:18:3} und \REF{ex:18:45}:

\begin{exe}
\exr{ex:18:3}[]{
\ea Nachdem Peter genesen war, fuhr er ans Meer.
\ex
\gll Posle togo kak Pëtr vyzdorovel, on poexal na more.\\
nach das.\textsc{gen} wie Peter genesen.\textsc{prät} er fahren.\textsc{prät} auf Meer\\
\glt    ‘Nachdem Peter genesen war, fuhr er ans Meer.’
\z
}
\end{exe}


%% example 3
%	\ex[]{(3) \hspace{0,4cm}a. Nachdem Peter genesen war, fuhr er ans Meer.
%    }
%	\ex[]{
%    \gll \hspace{0,9cm}b. Posle togo kak Pëtr vyzdorovel, on poexal na more.\\
%         {} nach das.\textsc{gen} wie Peter genesen.\textsc{prät} er fahren.\textsc{prät} ans Meer\\
%         \glt ‘Nachdem Peter genesen war, fuhr er ans Meer.’
%    }


% example 45
\ea \label{ex:18:45}
	\ea \label{ex:18:45a} Peter war genesen. Danach fuhr er ans Meer.
	\ex \label{ex:18:45b}
    \gll Pëtr vyzdorovel. Posle {\.e}togo on poexal na more.\\
         Peter genesen.\textsc{prät} nach dies.\textsc{gen} er fahren.\textsc{prät} auf Meer\\
         \glt ‘Peter war genesen. Danach fuhr er ans Meer.’
\z
\z

\noindent In Sätzen wie \REF{ex:18:45} ist das Formativ \textit{da} bzw. \textit{ėtogo} kein Korrelat, sondern eine deiktische DP, die auf ein Zeitintervall referiert, in dem die im vorerwähnten Satz identifizierte Situation $e$ stattfand. \REF{ex:18:46} ist der betreffende Le\-xi\-kon\-eint\-rag
(vgl. \REF{ex:18:23}).

% example 46
\ea \label{ex:18:46}
	\ea{ \label{ex:18:46a} \{dem / {\.e}togo\}
    }
	\ex{ \label{ex:18:46b} $+$D$+$def$+$deikt\{$+$reg$+$obl$-$fem$-$pl / $+$gen$-$fem$-$pl\}
    }
    \ex{ \label{ex:18:46c} ($\lambda P_1) \lambda P_2 \exists !t [[[P_1\, t] \wedge [Q\, t]] \wedge [P_2\, t]]$\\
    $Q, P_1, P_2 \in \langle \alpha t \rangle, \alpha \in \{e, i, t, st, \dots\}$
    }
	\z
\z

\noindent Charakteristisch für den in \REF{ex:18:46} repräsentierten deiktischen Determinierer ist, dass der Restriktor $P_1$ durch ein geeignetes Nomen spezifiziert werden kann, während der Modifikator $Q$ von $P_1$ als Parameter in die kontextuell determinierte konzeptuelle Interpretation eingeht.

% section 4
\section{Zusammenfassung und Ausblick} \label{sec:18:4}

\noindent Temporale Nebensätze sind Modifikatoren vom semantischen Typ $\langle it \rangle$. Sie schlie\-ßen sich per Prädikatunifizierung an das Topikzeitargument $t$ des Matrixsatzes an. Ihre Satzeinleitung ist unterschiedlich komplex. Präpositionen und Determinierer wie in \textit{nachdem} und \textit{seit}(\textit{dem}) sind morphosyntaktisch und semantisch transparent und korrespondieren auch mit entsprechenden anaphorischen Ausdrücken wie \textit{danach} und \textit{seitdem} (vgl. auch \textit{währenddessen}, \textit{dabei}, \textit{dazwischen}, \textit{davor}).

Semantisch unterscheiden sich die beteiligten temporalen Konnektive bezüg\-lich der Positionen oder Strecken, die sie für die betreffenden Intervalle oder Momente auf der Zeitachse festlegen. \citet{Beaver-Condoravdi2003} und mit ihnen \citet{Gronn-Stechow2010,Gronn-Stechow2012,Gronn-Stechow2013a} ordnen dem definiten Determinierer in Temporalsätzen die Komponente EARLIEST zu. Ich habe in meiner Analyse wie \citet{Krifka2010} darauf verzichtet und mich an die in anderen Zusammenhängen \citep{Zimmermann1993,Zimmermann1995,zi18:Zimmermann2000,Zimmermann2016} angenommenen Bedeutungen für den Determinierer und die \textit{w}- bzw. \textit{k}-Einheiten gehalten. Eine semantische Komponente wie \textsc{earliest} könnte man allenfalls der Bedeutung temporaler Präpositionen wie \textit{seit} zurechnen, nämlich als den frühesten Zeitpunkt des zur Rede stehenden Zeitintervalls $t'$. Meine Analyse rechnet stattdessen mit den in Beziehung gesetzten Anfangsintervallen $\cnstx{init}\, t'$ für den temporalen Nebensatz und $\cnstx{init}\, t$ für den Matrixsatz.

Neu in dieser Arbeit ist die weitreichende Annahme einer fakultativen Prä\-di\-kat\-stel\-le $Q$ in der Bedeutung des Determinierers (siehe \REF{ex:18:23c}). Sie unterscheidet hier deiktische anaphorische Determinierer von nichtdeiktischen kataphorischen und einfachen Determinierern, die im Russischen fehlen. Bleibt $Q$ unspezifiziert, hat man es mit einem Parameter zu tun, der für die Bedeutung von Demonstrativpronomen und -adverbien konstitutiv ist. Wird $Q$ durch einen Attributsatz spezifiziert, hat der betreffende Determinierer kataphorische Funktion. Weitere willkommene Konsequenzen von $Q$ in der Bedeutung von De\-ter\-mi\-nie\-rern bleiben hier unerwähnt.

Der kataphorische Determinierer ist in Adverbialsätzen ein Definitheitsope\-ra\-tor. Er kann mit einem Zeitabschnitte bezeichnenden Nomen und einem restriktiven Attributsatz auftreten. Oder der nominale Kopf der DP bleibt un\-spe\-zi\-fi\-ziert. Zusammen ergeben der Determinierer und sein Komplement einen ge\-ne\-ra\-li\-sier\-ten Quantor, der in üblicher Weise mit einem Regens kombiniert wird, das den Kasus der DP determiniert. Semantisch spielt eine entscheidende Rolle, von welchem Typ die durch den Determinierer gebundenen Einheiten sind. Bei temporalen Adverbialen hat man ein Zeitargument vom Typ $\langle it \rangle$ vor sich.

Generell nehme ich für das Korrelat an, dass es sich auf attributive Satzkonstruktionen folgender Typen beziehen kann: in Temporalsätzen auf CPs vom semantischen Typ $\langle it \rangle$, in Komplementsätzen intensionaler Prädikate auf eine CP vom Typ $\langle st,t \rangle$, in Komplementsätzen extensionaler Prädikate auf eine CP vom Typ $\langle t,t \rangle$. Und als Komplemente oder Attribute abstrakter Nomen wie \textit{Idee}, \textit{Vermutung}, \textit{Überzeugung}, \textit{Ziel} sowie bei allen faktiven Verben haben die Y spezifizierenden CPs den Typ $\langle et \rangle$.

Sofern die betreffenden eingebetteten CPs wie in \REF{ex:18:14} und \REF{ex:18:24} nicht selbst den entsprechenden Typ besitzen, sind type shifts wie \REF{ex:18:48} und \REF{ex:18:49} erforderlich \citep{Zimmermann2015,Zimmermann2016}:

% example 47
\ea \label{ex:18:47} ($\lambda P_1)_\alpha \lambda Q \lambda P_2 \exists !x\, [[[P_1\, x] \wedge [Q\, x]] \wedge [P_2\, x]] (\dots)_\alpha$(TS ($\parallel$CP$\parallel$)) $=$
\newline
$\lambda P_2 \exists !x\, [[[\dots \,x] \wedge [$(TS ($\parallel$CP$\parallel$$))\, x]] \wedge [P_2\, x]]$
\z

% example 48
\ea \label{ex:18:48} TS\textsubscript{PM1}: $\lambda p \lambda q [p = q]$\\
$p, q \in \{t, st\}$
\z

% example 49
\ea \label{ex:18:49} TS\textsubscript{PM2}: $\lambda p \lambda x [\cnstx{content}\, p\, x]$\\
$\cnstx{content}\, \in \langle\langle st \rangle\langle et \rangle\rangle$
\z

\noindent Von dem Prädikatmacher \REF{ex:18:48} habe ich in \citet{Zimmermann2016} Gebrauch ge\-macht, von dem in \REF{ex:18:49} in \citet{Zimmermann2015}.

In Satzeinbettungen mit Korrelat spezifizieren also attributive Nebensätze verschiedener Typen die durch das Korrelat eingebrachte Modifikatorstelle Y.

Insgesamt verfolgt meine Analyse ein minimalistisches Grammatikkonzept, mit einer strengen Trennung syntaktischer und semantischer Strukturen. Ope\-ra\-to\-ren treten nur in der Semantik auf.

Bezüglich des Zusammenspiels von Morphologie und Semantik folge ich\linebreak \citet{Zeijlstra2004,Pitsch2014,Pitsch2014a,Gronn-Stechow2010,Gronn-Stechow2012,Gronn-Stechow2013a,Stechow-Gronn2013}. Die Interpretation der Tempus, Aspekt und Modus (ТАМ) betreffenden Morphologie und gegebenenfalls auch der Negation erfolgt verzögert, und zwar erst in funktionalen Strukturbereichen. Die beteiligten Merkmalpaare sind als iF in den interpretierenden Domänen bzw. als uF in der mor\-pho\-syn\-tak\-ti\-schen Kategorisierung der Wortformen zu verstehen \citep{Zimmermann1990,Zimmermann2013,Zimmermann2015a,Zimmermann2015,Zimmermann2016}.\footnote{\label{fn:18:11}Beispielsweise ist das Merkmal +prät einer deutschen oder russischen Präteritalverbform als uninterpretiertes, lizensierungsbedürftiges Merkmal zu verstehen, während das entsprechende Merkmal in T als die Präteritalform des Verbs lizensierend und sie semantisch interpretierend gilt.}

Die hier benutzten semantisch unsichtbaren Bewegungen von Konstituenten mit der Kennzeichnung +EF bedürfen einer separaten Rechtfertigung (siehe Fn. \ref{fn:18:10}). Sie sind für viele Ausdrucksbesonderheiten verschiedener Sprachen verantwortlich.

Der Vergleich des Deutschen und des Russischen hat sich an vielen Stellen als erhellend erwiesen. Es interessierte mich vor allem das Auftreten von Korrelatformen in adverbiellen Nebensätzen, von denen ich mich auf die temporalen beschränkt habe.

Selbstredend gilt vieles meiner Analysevorschläge zum Korrelat auch für nichttemporale Adverbialsätze, von denen ich nur die Finalkonstruktionen in \REF{ex:18:13} und \REF{ex:18:14} erwähnt habe. Also sehe ich diesen Beitrag als Aufforderung an mich selbst und an andere zum Weitermachen und Verbessern an.

% Just uncomment the input below when you're ready to go.

%\input{ex:18:ample-osl.tex}

\section*{Abkürzungen}

\begin{tabularx}{.5\textwidth}{@{}lX@{}}
\textsc{akk}&Akkusativ\\
\textsc{gen}&Genitiv\\
\textsc{instr}&Instrumental\\
\textsc{pl}&Plural\\
\textsc{präs}&{Präsens}\\
\end{tabularx}%
\begin{tabularx}{.5\textwidth}{@{}lX@{}}
\textsc{prät}&{Präteritum}\\
\textsc{refl}&Reflexiv\\
\textsc{sg}&Singular\\
\textsc{subj}&Subjunktiv\\
&\\
\end{tabularx}

\section*{Danksagung}
Zu Dank für Anregungen und Kritik bin ich Kollegen vom ZAS in Berlin und vom Institut für Slawistik der HU verpflichtet, wo ich Ergebnisse dieser Arbeit vorgestellt habe. Dankenswerte kollegiale Hilfe wurde mir durch detaillierte Er\-gän\-zungs- und Änderungsvorschläge von zwei Gutachtern der LB zuteil.

\vspace{5mm} %5mm vertical space

\noindent Ich widme diese Arbeit Arnim von Stechow mit großem Respekt.

\section*{Anmerkungen der Herausgeber}
Beispiele aus anderen Sprachen als dem Deutschen haben wir glossiert und übersetzt. In einigen Beispielen haben wir Glossen an die \textit{Leipzig Glossing Rules} an\-ge\-passt. 

\sloppy
\printbibliography[heading=subbibliography,notkeyword=this]

\end{otherlanguage}
\end{document}
