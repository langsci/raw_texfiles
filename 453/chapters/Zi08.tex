\documentclass[output=paper,colorlinks,citecolor=brown]{langscibook}
\ChapterDOI{10.5281/zenodo.15471435}

\author{Ilse Zimmermann\affiliation{Centre for General Linguistics (ZAS), Berlin}}
\title{On the syntax and semantics of \textit{kakoj} and \textit{čto za} in Russian}  
\abstract{This contribution deals with the attributive pronouns \textit{kakoj} and \textit{čto za} in interrogative and exclamative sentences of Russian. It is an investigation into the polyfunctionality of these expressions, their integration into the DP structure, and their interplay with sentence mood. The morphosyntactic and semantic properties of these lexical items will be considered within the framework of Chomsky's Minimalist Program, taking into account their semantic form and conceptual structure.}


% \IfFileExists{../localcommands.tex}{%hack to check whether this is being compiled as part of a collection or standalone
%   % add all extra packages you need to load to this file

\usepackage{tabularx,multicol}
\usepackage{url}
\urlstyle{same}

\usepackage{listings}
\lstset{basicstyle=\ttfamily,tabsize=2,breaklines=true}

\usepackage{langsci-basic}
\usepackage{langsci-optional}
\usepackage{langsci-lgr}
\usepackage{langsci-osl}
% \usepackage{./langsci/styles/langsci-lgr}
% \usepackage{./langsci/styles/langsci-osl}
% \usepackage{langsci-gb4e}

\usepackage{tikz}
\usetikzlibrary{patterns,calc}
\pgfdeclarepatternformonly{south east lines}{\pgfqpoint{-0pt}{-0pt}}{\pgfqpoint{3pt}{3pt}}{\pgfqpoint{3pt}{3pt}}{
    \pgfsetlinewidth{0.6pt}
    \pgfpathmoveto{\pgfqpoint{0pt}{3pt}}
    \pgfpathlineto{\pgfqpoint{3pt}{0pt}}
    \pgfpathmoveto{\pgfqpoint{.2pt}{-.2pt}}
    \pgfpathlineto{\pgfqpoint{-.2pt}{.2pt}}
    \pgfpathmoveto{\pgfqpoint{3.2pt}{2.8pt}}
    \pgfpathlineto{\pgfqpoint{2.8pt}{3.2pt}}
    \pgfusepath{stroke}}
    
\usepackage{stmaryrd}
\usepackage{wasysym}
\usepackage{multirow}
\usepackage{caption}
\usepackage{subcaption}
\usepackage{mathrsfs}
\usepackage{qtree}

\usepackage{linguex}


%   %pminos do not split footnotes
% \interfootnotelinepenalty=10000 %Footnote in Laporte chapters has to be split SN


%\DeclareIndexNameFormat{default}{%
%\nameparts{#1}%
%\usebibmacro{index:name}%
%{\index[names]}%
%{\namepartfamily}%
%{\namepartgiveni}%
% {}% L1
% {}% L2
%{\namepartprefix}% generates spurious space L3
%{\namepartsuffix}% generates spurious space L4
%}

%  {\DeclareIndexNameFormat{default}{%
%     \usebibmacro{index:name}{\index[names]}{#1}{#3}{#5}{#7}}}

%\DeclareIndexNameFormat{default}{%
%  \usebibmacro{index:name}{\sindex[nom]}{#1}{#3}{#5}{#7}}

%\DeclareIndexNameFormat{default}{%
%  \usebibmacro{index:name}{\sindex[person]}{#1}{#3}{#5}{#7}}
%\DeclareIndexNameFormat{default}{%
%\nameparts{#1} \usebibmacro{index:name}{\sindex[person]]}{\namepartfamily}{‌​\namepartgiven}{\nam‌​epartprefix}{\namepa‌​rtsuffix}}

%\newcommand{\smiley}{:)}

%\renewbibmacro*{index:name}[5]{%
%\usebibmacro{index:entry}{#1}%
%{\iffieldundef{usera}{}{\thefield{usera}\actualoperator}\mkbibindexname{#2}{#3}{#4}{#5}}}

% \newcommand{\noop}[1]{}

%remove for final
%\overfullrule=1mm

\newcommand{\tobi}[2]}}
\renewcommand{\S}[1]{\tobi{#1}{\textsc{*}}}

% this volume references
% puts: [this volume]
% already defined: \citetv
%\newcommand{\citepv}[1]{(\citeauthor{#1} \citeyear*{#1} [this volume])}
\newcommand{\citealtv}[1]{\citeauthor{#1} \citeyear*{#1} [this volume]}

%parentheses around example number
\newcommand{\pref}[1]{(\ref{#1})}

% in-text examples

\newcommand{\lnex}[1]{\textit{#1}} %target lang word
\newcommand{\lnlit}[1]{(lit.: `#1')} %literal reading
\newcommand{\lnlat}[1]{(#1)} % latinization
\newcommand{\lntrans}[1]{`#1'} %translation
\newcommand{\lnexl}[2]%
{\lnex{#1}{} \lnlat{#2}} % ex with latinization
\newcommand{\lnexlat}[3]{\lnex{#1}{} \lnlat{#2}{} \lntrans{#3}} % ex with latinization and tranl.

%ch01
\newcommand{\co}[1]{\mbox{\textbf{#1}}}

%ch09

\newcommand{\cyrbulg}[1]{\begin{otherlanguage*}{bulgarian}#1\end{otherlanguage*}}


%ch10
\newcommand{\nlp}{{\small NLP}}
\newcommand{\mwe}{{\small MWE}}
\newcommand{\rae}{{\small RAE}}
\newcommand{\lvc}{{\small LVC}}
\newcommand{\pos}{{\small P}o{\small S}}
%\newcommand{\todo}[1]{ \textcolor{red}{#1} }

%\renewcommand{\labelenumi}{\theenumi}
%\ainamefmt{{vv}{ll}{, ff}{, jj}} % fullname

\newcommand{\biberror}[1]{{\color{red}#1}}

\newcommand{\osenovaitem}{--~} 
% \togglepaper[23]
% }{}https://www.overleaf.com/project/5faa78eaf7428e3e7bee4ea0

\begin{document}
\maketitle

% The part below is from the generic LangSci Press template for papers in edited volumes. Delete it when you're ready to go.

%LINE FOR THE FOOTNOTE FROM THE HEADING\footnote{\label{fn:zi08:0}I presented a shorter version of this paper at the second meeting of the Slavic Linguistics Society in Berlin in August 2007 and profited from the discussion. For contemporary Russian, I have consulted Elena Gorichneva, Nikolai Grettschak, Wladimir Klimonow, and Faina Pimenova. For Slovenian, I am indebted to Boštjan Dvořák. For discussion on German and for help in various respects I am grateful to Brigitta Haftka. Barbara Jane Pheby and Jean Pheby have helped me with the English translation of the examples. To Uwe Junghanns and a \textit{JSL} reviewer I owe many valuable suggestion}

% section 1
\section{Introduction} \label{sec:zi08:1}
I examine Russian interrogative and exclamative sentences with the attributive wh-pronouns \textit{kakoj} and \textit{čto za} within the framework of the Minimalist Program \citep{zi08:Chomsky1995,Chomsky2004}. As in my treatment of sentence mood \citep{Zimmermann2009}, I assume that the differentiation of sentence types with illocutionary force is anchored syntactically and semantically in the functional head C. Phrases containing \textit{kakoj} and \textit{čto za} are attracted to SpecCP. I will argue that the meaning of these pronouns is integrated into the semantic form of the corresponding utterances in situ and that it consists of a parameter (variable) which will be bound by the semantics of С. I will discuss how \textit{kakoj} and \textit{čto za} are built into the syntax of the DP structure and how they relate to determiners. I will account for the fact that it is possible to answer questions like \textit{Kakuju/Čto za knigu on čitaet}? `Which/What kind of book is he reading?' not only by specific predicate expressions like \textit{interesnuju} `an interesting one', \textit{о revoljucii} `one about the revolution', etc., but also by terms like \textit{istoričeskij roman} `a historical novel', \textit{Vojna i mir} `\textit{War and Peace}', etc. I hope to shed light on the syntactic and semantic analysis of \textit{čto za}, a borrowed expression, with respect to compositionality and the external behavior of the container DP. I will also address the stylistic distribution of expressions with \textit{kakoj} vs. \textit{čto za}.


% section 2
\section{The data} \label{sec:zi08:2}

Contemporary Russian interrogative and exclamative sentences are characterized by special intonation and word order and/or lexical means, which contribute by their semantics to the respective sentence type. The pronominal entities \textit{kakoj} and \textit{čto za} serve this aim. They occur as components of noun phrases at the left periphery of exclamative or interrogative sentences.

% example 1
\ea \label{ex:zi08:1}
    \gll Kakoj prekrasnyj den' segodnja!\\
    what nice day today\\
    \glt `What a nice day it is today!'
\z

% example 2
\ea \label{ex:zi08:2}
    \gll Čto za jubilej zdes' otmečajut?\\
    what for anniversary here is-being-celebrated\\
    \glt `What anniversary is being celebrated here?'
\z

\noindent Russian allows the separation of the pronominal \textit{kakoj} and the pronominal component \textit{čto} of \textit{čto za} from their DP.

% example 3
\ea \label{ex:zi08:3}
    \gll Kakaja ėto reka?\\
    which this river\\
    \glt `What river is this?'
\z

% example 4
\ea \label{ex:zi08:4}
    \gll Čto ėto za reka?\\
    what this for river\\
    \glt `What river is this?'
\z

\noindent The integration of \textit{kakoj} and \textit{čto za} into the DP structure must account for this possibility.

\textit{Čto za} seems to prefer negative qualifications of the pertinent referent, where\-as \textit{kakoj} is neutral in this regard.{\footnote{\label{fn:zi08:1}For sentences \REF{ex:zi08:5}--\REF{ex:zi08:6} and \REF{ex:zi08:39} I thank E. Gorishneva (p.c.).}}

% example 5
\ea \label{ex:zi08:5}
    \gll Čto ona za ženščina! Nastojaščaja zmeja! / \minsp{??} Umnica, krasavica!\\
    what she for woman real snake {} {} clever-girl beauty\\
    \glt `What a woman she is! A real snake!/A clever girl, a beauty!'
\z

% example 6
\ea \label{ex:zi08:6}
    Kakaja ona ženščina! Umnica, krasavica! /  Nastojaščaja zmeja!
\z

\noindent But \REF{ex:zi08:7}--\REF{ex:zi08:9} show that \textit{čto za} is compatible with positive characterizations of the referent as well.

% example 7
\ea \label{ex:zi08:7}
    \gll Kakoj / čto za čudesnyj čelovek moj drug!\\
    what {} what for wonderful person my friend\\
    \glt `What a wonderful person my friend is!'
    
\z

% example 8
\ea \label{ex:zi08:8}
    \gll Čto za prelest' ėti skazki!\\
    what for treasure these fairy-tales\\
    \glt `What a treasure these fairy-tales are!' \xspace\hfill(Google)
\z

% example 9
\ea \label{ex:zi08:9}
    \gll Čto ona za ženščina! Čto za aktrisa! No kak ona živet!\\
    what she for woman what for actress but how she lives\\
    \glt `What a woman she is! What an actress! But the way she lives!'
    \xspace\hfill(Google)
\z

\noindent In contrast to \textit{kakoj}, the composed expression \textit{čto za}, which is borrowed from Germanic languages and is also used in other Slavic languages, is restricted to colloquial speech. This stylistic restriction does not apply to German \textit{was für ein}. 

Furthermore, \textit{čto za} in Russian — in contrast to German, Bulgarian, and Slovenian\footnote{\label{fn:zi08:2}Compare:

% example (i) in footnote 2
\ea \label{ex:zi08:fn2i}
    \gll Na što za čovek Mitko pomaga!\\
    in what for person Mitko is-helping\\
    \glt `What kind of person Mitko is helping!'
    \xspace\hfill(Bulgarian)
\z

% example (ii) in footnote 2
\ea \label{ex:zi08:fn2ii}
    \gll Na kaj za enem uradu dela?\\
    in what for an office works\\
    \glt `For which office does he work?'
    \xspace\hfill(Slovenian)
\z

} — does not co-occur with prepositions.

% example 10
\ea \label{ex:zi08:10}
    \gll S kakimi / \minsp{*} S čto za ljud'mi {} Maša obščaetsja?\\
    with what {} {} with what for people {} Masha has-contact\\
    \glt `What kind of people does Masha have contact with?'
\z

% example 11
\ea \label{ex:zi08:11}
    \gll Za kakogo / \minsp{*} Za čto za čeloveka Maša vyšla zamuž?\\
    for what {} {} for what for man Masha went for-husband\\
    \glt `What kind of man did Masha marry?'
\z

\noindent Moreover, the overwhelming majority of examples with \textit{čto za} I have seen are in nominative or accusative noun phrases. Oblique cases are exceptions, at least in Russian. Here are some examples:\footnote{\label{fn:zi08:3}Example \REF{ex:zi08:12} is the only example from 97 pages of Google citations in which \textit{čto za} shows up in a noun phrase with an oblique case.}

% example 12
\ea \label{ex:zi08:12}
    \gll Čto za erundoj ej moročili golovu!\\
    what for nonsense.\textsc{inst} her.\textsc{dat} turned head\\
    \glt `What nonsense they filled her head with!'
    \xspace\hfill(Google)%\footnote{\label{fn:zi08:3}This is the only example from 97 pages of Google citations in which \textit{čto za} shows up in a noun phrase with an oblique case.}
\z

% example 13
\ea \label{ex:zi08:13}
    \gll Čto za čeloveku on doveril svoju naxodku?\\
    what for person.\textsc{dat} he trusted-with his finding\\
    \glt `What kind of person did he trust with his finding?'
    \xspace\hfill\citep{Israeli2006}
\z

% example 14
\ea \label{ex:zi08:14}
    \gll Čto za celej Boris xočet dostič'?\\
    what for aims.\textsc{gen} Boris wants to-achieve\\
    \glt `What goals does Boris want to achieve?'
\z

\noindent These co-occurrence restrictions must somehow be accounted for in a descriptively adequate grammar of Russian. For the present I cannot offer any explanation.

Whereas \textit{kakoj} morphosyntactically behaves like an adjective, \textit{čto za} does not agree with the head noun in gender, number, or case (see the examples above). It is an uninflected composite expression. Its components also behave strangely. \textit{Za} does not function as a preposition: it does not assign case. (At least not in Russian; in Czech \textit{za} can assign the accusative to the head noun, as in \textit{Co je to za knihu/kniha?} `What kind of book.\textsc{acc}/book.\textsc{nom} is this?'.) Furthermore, in Russian the wh-word \textit{čto} of the composed expression \textit{čto za} does not combine with specificity markers like \textit{kое-}, \textit{-to}, or \textit{-libo}.

% example 15
\ea \label{ex:zi08:15}
    \gll Ja tebe koe-čto / {} koe-kakoj podarok / \minsp{*} koe-čto za podarok prinesla.\\
    I you something {} {} some present {} {} some-what for present have-brought\\
    \glt `I have brought you something (nice)/some (nice) present.'
\z

\noindent Both \textit{kakoj} and \textit{čto za} function as modifiers which introduce a semantic parameter, i.e., an unspecified predicate. This predicate refers to properties or gradations. In addition to the examples \REF{ex:zi08:1} and \REF{ex:zi08:7} with adjectival modifiers, the following exclamation shows that nouns like \textit{sčast'e} `luck', \textit{radost'} `joy', \textit{čudo} `miracle', \textit{razočarovanie} `disappointment', \textit{erunda} `nonsense', \textit{durak} `idiot', and \textit{krasavica} `beauty' also express gradable qualifications.

% example 16
\ea \label{ex:zi08:16}
    \gll Kakое / {} Čto za \minsp{(} bol'šoe) udovol'stvie bylo naše putešestvie!\\
    what {} {} what for {} great pleasure was our journey\\
    \glt `What a (great) pleasure our journey was!'
\z

\noindent In many cases (compare \REF{ex:zi08:2}--\REF{ex:zi08:4}), \textit{kakoj} and \textit{čto za} seem to function like determiners which convert a predicate of type $\langle e,t \rangle$ into a term of type $\langle e \rangle$ or $\langle \langle e,t \rangle ,t \rangle $.\footnote{\label{fn:zi08:4}The determiner function of \textit{kakoj} prevails when the particles \textit{koe-}, \textit{-to}, or \textit{-libo} are added (see \citealt{Kagan2007}).}

In the analysis below I will try to account for these semantic peculiarities of \textit{kakoj} and \textit{čto za}.

But first, a short comparison with German might be useful. I will concentrate on the use of \textit{welch} (\textit{ein}), \textit{welcher}, and \textit{was für ein} (see \citealt{Engel1988} and \citealt{Bhatt1990}). 

\textit{Welch} (\textit{ein}) is restricted to exclamative sentences. \textit{Welch} is the uninflected part of the composite pronoun. \textit{Ein} can be omitted when it is not inflected.

% example 17
\ea \label{ex:zi08:17}
    \gll Welch (ein) Vergnügen / Welch einen Spaß wir bei Dieter hatten!\\
    which a pleasure {} which a fun we at Dieter had\\
    \glt `What pleasure/What fun we had at Dieter's!'
\z

\noindent Plural DPs with \textit{welch} have a zero article and occur only with inflected attributes.

% example 18
\ea \label{ex:zi08:18}
    \gll Welch \minsp{*(} schöne) Blumen er bekommen hat!\\
    which {} beautiful flowers he received has\\
    \glt `What nice flowers he got!'
\z

\noindent \textit{Welcher} is a selective determiner and refers to alternative individuals of a given set, whereas \textit{was für ein} refers to properties.

% example 19
\ea \label{ex:zi08:19}
    \gll Welcher Koffer gehört Ihnen?\\
    which suitcase belong you.\textsc{dat}\\
    \glt `Which suitcase belongs to you?'
\z

% example 20
\ea \label{ex:zi08:20}
    \gll Was für einen Koffer haben Sie verloren?\\
    what for a suitcase have you lost\\
    \glt `What kind of suitcase did you lose?'
\z

\noindent Admittedly, the lines of differentiation here are fuzzy, but I will not go into details.

Like Russian \textit{kakoj} and \textit{čto za}, German \textit{welcher} and \textit{was für ein} occur in interrogative and exclamative sentences.

Like the \textit{čto} of the composite expression \textit{čto za}, the corresponding German pronominal component \textit{was} of \textit{was für ein} can be separated from the DP.

% example 21
\ea \label{ex:zi08:21}
    \gll Was möchten Sie für einen Koffer kaufen?\\
    what like you for a suitcase buy\\
    \glt `What kind of suitcase would you like to buy?'
\z

\noindent In contrast to Russian \textit{čto za}, German \textit{was für ein} can be used in DPs of all cases, with or without a preposition.

% example 22
\ea \label{ex:zi08:22}
    \gll Mit was für einem Buch könnte ich dich erfreuen?\\
    with what for a book could I you please\\
    \glt `What kind of book would please you?'
\z

\noindent German requires that the indefinite article within the composed pronoun \textit{was für ein} agree with the noun in gender, number, and case. Russian does not have explicit articles. When the noun phrase is absent, the plural form of the German indefinite article \textit{ein} is suppleted by \textit{welche}.

% example 23
\ea \label{ex:zi08:23}
    \gll Was für Bücher / was für welche bevorzugen Sie?\\
    what for books {} what for which prefer you\\
    \glt `What kind of books/which ones do you prefer?'
\z

\noindent In contrast to \textit{was für ein}/\textit{was für welche}, \textit{welcher} can be used as an indefinite determiner.

% example 24
\ea \label{ex:zi08:24}
    \gll Wir brauchen zum Gartenfest viele Stühle. Können Sie welche /\hspace{.7cm} \minsp{*} was für welche mitbringen?\\
    we need for garden-party many chairs can you some {} %*
    {} what for some bring-with\\
    \glt `We need a lot of chairs for the garden party. Can you bring some with you?'
\z

\noindent In view of these Russian and German data, the following analysis is proposed to account for the polyfunctionality of \textit{kakoj} and \textit{čto za}, for the case restrictions of \textit{čto za}, and for the separability of \textit{kakoj} and \textit{čto} from their DP.

% section 3
\section{The analysis} \label{sec:zi08:3}

% sub-section 3.1
\subsection{The left periphery} \label{subsec3.1}

Russian phrases with \textit{kakoj} and \textit{čto za} or the pronouns \textit{kakoj} or \textit{čto} alone appear at the left periphery of interrogative, exclamative, and relative clauses. This is also true in many other languages. In line with my morphosyntactic analysis of Russian wh-pronouns and adverbials like \textit{kto} `who' or \textit{kogda} `when' \citep{zi08:Zimmermann2000} and my treatment of sentence mood \citep{Zimmermann2009}, I assume that wh-phrases are attracted to the left periphery of their clauses by a $+$wh-feature of the functional category С at the top of root and embedded clauses, which are differentiated by the feature [$\pm$force].\footnote{Editors' note: \citet{Zimmermann2009} was referred to as Zimmermann (forthcoming) in the original text.} Exclamative sentences have the feature [$+$exclam], whereas interrogative sentences are characterized by [$+$quest]. These features correspond to the semantic interpretation of the zero complementizer C.

% example 25
\ea \label{ex:zi08:25} $/\varnothing/$;\,$+$C, $\alpha$force, $+$wh, $+$exclam; $\lambda P\,(\textsc{exclam})_{\alpha}\, \exists ! X [P\, X]\,\textsc{exclam} \in\,\langle t,a \rangle , P \in\,\langle \beta,t \rangle , \beta \in\,\{e, t, \langle e,t \rangle \}$
\z

% example 26
\ea \label{ex:zi08:26} $/\varnothing/$;\,+C, $\alpha$force, +wh, +quest; $\lambda P\,(\textsc{quest})_{\alpha}\, ? X [P\, X]\,\textsc{quest} \in\,\langle t,a \rangle , P \in\,\langle \beta,t \rangle , \beta \in\,\{e, t, \langle e,t \rangle \}$
\z

\noindent It is important that the Semantic Form (SF) of the complement of С constitutes an open proposition with an unbound variable which is only bound by the semantics of C. In exclamative $+$wh-sentences this variable is bound by the definite existential operator $\exists !X$ corresponding to the iota-operator \citep{zi08:Zybatow1990,Rosengren1992,Brandt-Reis-etal1992}. In interrogative $+$wh-sentences the binder of the variable is the question operator $?X$ characterizing the openness of the respective proposition (cf. \citealt{Brandt-Reis-etal1992}). In addition, root sentences are mapped by the sentence-mood operator \textsc{exclam} or \textsc{quest} to speech act types (cf. \citealt{zi08:Krifka2001}).

In this respect, my analysis of С departs from \citet{Brandt-Reis-etal1992}, \citet{zi08:Reis1999}, \citet{dAvis2002}, \citet{Zanuttini-Portner2003}, and \citet{Abels2007}, who do not assume any element in syntax responsible for introducing illocutionary force characterizations. According to \citet{Zimmermann2009}, every root clause adds to the common ground a proposition and a set of communicative commitments associated with the respective force type.\footnote{\label{fn:zi08:5}Whereas interrogative clauses denote a set of alternatives specifying the variable bound by the question operator, exclamative clauses widen the domain of quantification for the wh-operator. This aspect of meaning derives from the exclamative-force operator or from the corresponding embedding predicate. (For the concept of widening and the relation of exclamatives to scales of expectedness, see \citealt[49ff.]{Zanuttini-Portner2003}) Likewise the factive presupposition which is associated with exclamative clauses is not represented in syntax -- as \citet{Zanuttini-Portner2003} propose -- but derives from the embedded predicate or from the exclamative force operator of root clauses.}

The treatment of sentence mood as in \REF{ex:zi08:25} and \REF{ex:zi08:26} implies that there is no need for the movement of wh-phrases to the left periphery at the level of Logical Form (LF). The appearance of wh-phrases at the left periphery is a surface matter. The $+$wh-phrases move cyclically, from syntactic phase to syntactic phase (see \citealt{zi08:Chomsky1995,Chomsky2004}). They are attracted to SpecCP by the feature $+$wh in С (see \REF{ex:zi08:25}--\REF{ex:zi08:26}).

Semantically,  wh-phrases are interpreted in situ. Their semantics introduce the unbound variable as a parameter which is specified by appropriate answers to the question involved or by the conceptual representation of individual terms or specific predicates on the level of Conceptual Structure (CS), depending on the shared knowledge of hearers and speakers and/or the situational context. This assumption corresponds to the analysis of exclamative sentences by \citet{zi08:Zybatow1990} and is in accord with the treatment of the Dutch \textit{was} (\textit{voor}) (\textit{een}) constructions by \citet{Bennis1995} and \citet{Bennis-Corver-etal1998}.

% sub-section 3.2
\subsection{On the syntax and semantics of \textit{kakoj}} \label{subsec3.2}

The syntactic structure of phrases containing \textit{kakoj} must account for its semantic functions, for the separability of the pronoun from its noun phrase, for its agreement in gender, number, and case with the head noun, and for its movement to the specifier position of С (SpecCP). Semantically, the analysis must guarantee the introduction of a predicate or individual variable. 

As for syntax, I assume the following structures:

% example 27
\ea \label{ex:zi08:27} [\textsubscript{NP} [\textsubscript{AP}\,\textit{kakoj}] NP]
\z

% example 28
\ea \label{ex:zi08:28} [\textsubscript{DP} [\textsubscript{D}\,$\varnothing$][\textsubscript{NP} [\textsubscript{AP}\,\textit{kakoj}] NP]]
\z

\noindent In \REF{ex:zi08:27} \textit{kakoj} is represented as adjunct to NP, and in \REF{ex:zi08:28} the DP layer with a zero head in D is added. 

\textit{Kakoj} as a lexical entry is characterized by the following morphosyntactic and semantic features:

% example 29
\ea \label{ex:zi08:29} /kakoj/; $+$N$+$V, $+$wh; $(\lambda Q) \lambda x [Q x]$
\z

\noindent I consider \textit{kakoj} an adjective phrase which is characterized as a $+$wh-entity. Semantically, it has a dual character. The predicate variable $Q$ can function as a parametrical modifier of the noun phrase or it can be specified by the predicate expressed by the head NP. These two possibilities are illustrated in \REF{ex:zi08:31} and \REF{ex:zi08:32}.

% example 30
\ea \label{ex:zi08:30} \cnst{MOD} = $\lambda Q_1 \lambda Q_2 \lambda x [[Q_1\, x] \wedge [Q_2 x]]$
\z

% example 31
\ea \label{ex:zi08:31} \sib{\textit{kakaja kniga}} $=$ \cnst{MOD} ($\lambda x [\textsc{book}\, x])(\lambda x [Q\, x])$\\
$\equiv \lambda x [[\textsc{book}\, x] \wedge [Q\, x]]$ 
\z

\noindent The semantic template  \cnst{MOD} in \REF{ex:zi08:30} (see \citealt{Zimmermann1992}) combines the modifier with the meaning of the head noun. As for agreement, I assume that an adjectival modifier like \textit{kakoj} and what it modifies must agree in case, number, and gender. This is to say that the specified morphosyntactic features associated with the external argument of the respective predicates must coincide. 

In \REF{ex:zi08:32} the semantic contribution of \textit{kakoj} is an identity function.

% example 32
\ea \label{ex:zi08:32} \sib{\textit{kakaja kniga}} $= \lambda Q \lambda x [Q\, x](\lambda x [\textsc{book}\, x])$\\
$\equiv \lambda x [\textsc{book}\, x]$
\z

\noindent In combination with the semantics of the determiner we obtain \REF{ex:zi08:33} and \REF{ex:zi08:34}. In \REF{ex:zi08:33}, D is categorized by $+$Det $-$wh, while in \REF{ex:zi08:34} by $+$Det $+$wh.

% example 33
\ea \label{ex:zi08:33} \sib{$\varnothing$}\,(\sib{NP}) $= \lambda P_1 \lambda P_2 \exists x [[P_1 x] \wedge [P_2\, x]](\lambda x [[\textsc{book}\, x] \wedge [Q\, x]])$\\
$\equiv \lambda P_2 \exists x [[[\textsc{book}\, x] \wedge [Q\, x]] \wedge [P_2 x]]$
\z

% example 34
\ea \label{ex:zi08:34} \sib{$\varnothing$}\,(\sib{NP}) $= \lambda P_1 \lambda P_2 [[P_1\, x] \wedge [P_2\, x]](\lambda x [\textsc{book}\, x])$\\
$\equiv \lambda P_2 [[\textsc{book}\, x] \wedge [P_2\, x]]$
\z

\noindent In the last case the predicate variable $Q$ is absent and \textit{kakaja kniga} is interpreted like \textit{kto} (\citealt{zi08:Zimmermann2000}; see also \citealt[145]{Paul1958}).

% example 35
\ea \label{ex:zi08:35} \sib{\textit{kto}} $= \lambda P_2 [[\textsc{animate}\, x] \wedge [P_2\, x]]$
\z

\noindent Thus the semantics of \textit{kakoj} as given in \REF{ex:zi08:29} brings in the parameter $Q$ or enables the NP to get its external argument $\lambda$x blocked by a $+$wh-determiner. The free variables, $Q$ and $x$, are bound by the semantics of С depending on the respective clause type, as in \REF{ex:zi08:36}--\REF{ex:zi08:39}.

% example 36
\ea \label{ex:zi08:36}
    \gll Kakie knigi vy predpočitaete, populjarnye ili xudožestvennye?\\
    what books you prefer popular or art\\
    \glt `What kind of books do you prefer, popular or literary?'
\z

% example 37
\ea \label{ex:zi08:37}
    \gll Kakie reki vpadajut v Volgu?\\
    which rivers flow into Volga\\
    \glt `Which rivers flow into the Volga?'
\z

% example 38
\ea \label{ex:zi08:38}
    \gll Kakie u vas est' interesnye knigi!\\
    what by you are interesting books\\
    \glt `What interesting books you have!'
\z

% example 39
\ea \label{ex:zi08:39}
    \gll Kakim prekrasnym pedagogom ty stal!\\
    what good teacher you have-become\\
    \glt `What a good teacher you have become!'
\z

\noindent In the last example, the phrase with \textit{kakoj} is a predicate expression, an NP without the DP layer. Moreover, \textit{kakoj} here refers to a high degree of the property expressed by the adjective \textit{prekrasnyj}. The same is true in the example \REF{ex:zi08:40}, from \citet{zi08:Zybatow1990}.

% example 40
\ea \label{ex:zi08:40}
    \gll Kakoj krepkij čaj vy svarili!\\
    what strong tea you have-made\\
    \glt `What a strong tea you have made!'
\z

\noindent In these cases, another template, \cnst{MOD'}, is activated in order to combine the parameter Q introduced by \textit{kakoj} with the semantics of its container NP. This template integrates modifiers like \textit{očen'} `very' into the semantics of their gradable modifiees, as in [\textsc{dp} $\varnothing$ [\textsubscript{NP} [\textsubscript{AP} \textit{očen' krepkij}][\textsubscript{NP} \textit{čaj}]]].

% example 41
\ea \label{ex:zi08:41} \cnst{MOD'} $= \lambda Q_1 \lambda Q_2 \lambda x \exists d [[Q_1 d\, x] \wedge [Q_2\, d]]$
\z

\noindent In \REF{ex:zi08:39} and \REF{ex:zi08:40} the pronominal adjective \textit{kakoj} does not figure within the adjective phrase together with the adjective semantically modified by it. For cases like \REF{ex:zi08:40} I assume the syntactic configuration [\textsubscript{DP} $\varnothing$ [\textsubscript{NP} [\textsubscript{AP} \textit{kakoj}][\textsubscript{NP} [\textsubscript{AP} \textit{krepkij}][\textsubscript{NP} \textit{čaj}]]]]. The semantic amalgamation for the complex NP proceeds as follows:

% example 42
\ea \label{ex:zi08:42} \sib{\textit{kakoj krepkij čaj}}\\
$=$ (\cnst{MOD} ($\lambda x  [\textsc{tea}\, x])(\cnst{MOD'} (\lambda d \lambda x [\textsc{strong}\, d\, x])))(\lambda x [Q\, x])$\\ 
\hspace{41pt} $=$ \sib{\textit{čaj}} \hspace{55pt} $=$ \sib{\textit{krepkij}} \hspace{35pt} $=$ \sib{\textit{kakoj}}\\ 
$\equiv \lambda Q_2 \lambda x [[\textsc{tea}\, x] \wedge \exists d [[\textsc{strong}\, d\, x] \wedge [Q_2\, d]]](\lambda x [Q\, x])$\\ 
$\equiv \lambda x [[\textsc{tea}\, x] \wedge \exists d [[\textsc{strong}\, d\, x] \wedge [Q\, d]]]$
\z

\noindent The enriched argument structure of the AP \textit{krepkij} is inherited by the modified NP \textit{krepkij čaj} so that \textit{kakoj} as a predicate expression can come into play at this rather remote level of syntactic structure.

Thus \textit{kakoj} in its function as modifier gets a syntactic and semantic analysis which applies uniformly to cases like \REF{ex:zi08:36} and \REF{ex:zi08:40}. The templates in \REF{ex:zi08:30} and \REF{ex:zi08:41} are necessary for independent reasons. I think it is worthwhile to reduce syntax and to exploit the capacities of semantics.

% sub-section 3.3
\subsection{Consequences for the analysis of the colloquial expression \textit{čto za}} \label{subsec3.3}

As mentioned in section \ref{sec:zi08:2}, \textit{za} does not function as a preposition. It is a semantically empty annex to NP. In contrast to German, Dutch, and Slovenian, the Russian pronominal expression \textit{čto za} lacks an explicit determiner. I propose an analysis which is quite similar to the one given in the preceding section for \textit{kakoj}.

% example 43
\ea \label{ex:zi08:43} [\textsubscript{DP} [\textsubscript{D} $\varnothing$][\textsubscript{NP} [\textsubscript{NP} \textit{čto}][\textsubscript{NP} \textit{za} ([\textsubscript{NP} AP)[\textsubscript{NP} \textit{kniga}](])]]]
\z

\noindent The semantic burden of the composite expression \textit{čto za} is assigned to \textit{čto}.

% example 44
\ea \label{ex:zi08:44}
    \sib{\textit{čto}} $= (\lambda Q)_\alpha \lambda x [Q x](\lambda x \exists y [[\cnst{kind}\, x\, y] \wedge [Q y]])_\beta$\\
    with $\beta = + \rightarrow \alpha = +$\\
	\ea{ \label{ex:zi08:44a}
    $= \lambda x \exists y [[\textsc{kind}\, x\, y] \wedge [Q y]]$\\
    }
	\ex{ \label{ex:zi08:44b}
    $= \lambda x [Q x]$\\
    }
    \ex{ \label{ex:zi08:44c}
    $= \lambda Q \lambda x [Q x]$
    }
	\z
\z

\noindent The only semantic difference between \textit{čto za} and \textit{kakoj} consists in the facultative presence of the reference to kinds in the case of \textit{čto za}.\footnote{\label{fn:zi08:6}Cf. \citeauthor{Pafel1991} (\citeyear{Pafel1991,Pafel1996b,Pafel1996a}) for German \textit{was für} (\textit{ein}). In contrast to his analysis, I regard the reference to kinds as one possible interpretation and as a modifier to the head NP.}

% example 45
\ea \label{ex:zi08:45}
    \gll Čto za knigu ty čitaeš?\\
         what for book you read\\
    \glt `Which/what kind of book are you reading?'
	\ea{ \label{ex:zi08:45a}
    \gll Ljubovnyj roman.\\
         love story\\
    \glt `A love story.'
    }
	\ex{ \label{ex:zi08:45b}
    \gll О revoljucii.\\
         about revolution\\
    \glt `One about the revolution.'
    }
    \ex{ \label{ex:zi08:45c}
    \gll \textit{Vojna} \textit{i} \textit{mir.}\\
         war and peace\\
    \glt `\textit{War and peace.}'
    }
	\z
\z

\noindent The three possible answers to the question in \REF{ex:zi08:45} correspond to the three meaning variants of \textit{čto za}: (a) reference to kinds; (b) specification of $Q$ as predicate variable in the modifier $[Q x]$ (cf. \REF{ex:zi08:33}); and (c) specification of $x$ as individual variable in $\lambda P_2 \,[[\textsc{book}\, x] \wedge [P_2 \, x]]$ (cf. \REF{ex:zi08:34}).

The lexical entry for \textit{čto} in \textit{čto za} has to account for its peculiar combinability: the co-occurrence with \textit{za}, the lack of occurrence within preposition phrases, and the lack of enrichment by specificity markers like \textit{koe-}, \textit{-to}, \textit{-libo}.\footnote{\label{fn:zi08:7}In contrast to \textit{čto} in \textit{čto za}, the pronoun \textit{kakoj} — like other wh-pronouns and adverbs — can be enriched by specificity markers. They characterize conventional implicatures concerning various types of specificity or non-specificity (see \citealt{Kagan2007}). The wh-pronouns and adverbs by themselves are neutral in this respect \citep{zi08:Zimmermann2000}. I assume that the particles \textit{koe-}, \textit{-to}, \textit{-libo}, \textit{ne-}, \textit{ni-} have their base position in D and combine with the right wh-neighbor and also with a preceding preposition in PF. Thus \textit{ni s kem} `with nobody', \textit{koe s kem} `with a certain person', \textit{dlja kakogo-to studenta} `for some student' will have the base structures \REF{ex:zi08:fn8i} and \REF{ex:zi08:fn8ii}, respectively:

% example (i) in footnote 8
\ea \label{ex:zi08:fn8i} [\textsubscript{PP} [\textsubscript{P} \textit{s}][\textsubscript{DP} [\textsubscript{D} \textit{ni/koe}][\textsubscript{NP} \textit{kem}]]]
\z

% example (ii) in footnote 8
\ea \label{ex:zi08:fn8ii} [\textsubscript{PP} [\textsubscript{P} \textit{dlja}][\textsubscript{DP} [\textsubscript{D} \textit{to}][\textsubscript{NP} [\textsubscript{AP} \textit{kakogo}][\textsubscript{NP} \textit{studenta}]]]]
\z

\noindent Editors' note: A later version of \textcite{Kagan2007} has been published as \textcite{Kagan2011}.
} Furthermore, \textit{čto} in \textit{čto za} is uninflectible and does not show any agreement with the head NP, in contrast to the AP \textit{kakoj}. Like \textit{kakoj}, \textit{čto} can leave its container phrase and appear at the left periphery of the sentence. The peculiarities have to be worked out.

% section 4
\section{Conclusion} \label{sec:zi08:4}

The inspection of the morphosyntactic and semantic properties of the Russian pronouns \textit{kakoj} and \textit{čto za} has brought me to important conclusions on the division of labor between syntax and semantics in the mutual correlation of sound and meaning. I have utilized current work within the framework of Minimalism and taken note of the differentiation between the Semantic Form of linguistic expressions, which is grammatically determined, and their Conceptual Structure, which has to do with world knowledge (see \citealt{zi08:Bierwisch-Lang1987}, \citealt{zi08:Bierwisch2007}). I have taken templates like \cnst{MOD} and \cnst{MOD'} (see \REF{ex:zi08:30} and \REF{ex:zi08:41}) to be considered as universal components of the semantic interpretation. But discussion of their status in comparison with corresponding functional zero heads in the syntactic representation goes beyond the limits of this paper.

My analysis of Russian wh-pronouns and adverbs \citep{zi08:Zimmermann2000} has shown how phrases with \textit{kakoj} and \textit{čto za} get an unbound variable in their base position. I believe the multifunctionality of wh-words can best be captured by taking into account the various syntactic domains of referential specification (binding) of entities in situ, in the middle field, or in the CP domain (cf. \citealt{Postma1994,Postma1995}). I have shown the interplay of unbound variables of various semantic types with the semantics of sentence mood (see also \citealt{Zimmermann2009}). The existence of semantic parameters (variables) introduced by the semantics of \textit{kakoj} and \textit{čto za} gives broad room for the conceptual interpretation of the relevant clauses.

% example 46
\ea \label{ex:zi08:46}
    \gll Kakaja ona ženščina!\\
    what she woman\\
    \glt `What a woman she is!'
\z

\noindent This exclamation, in the context of a volume honoring an outstanding linguist, is immediately understood as high estimation.


I hope that my treatment of the Russian pronouns \textit{kakoj} and \textit{čto za} and of the correlation of form and meaning can be applied to the corresponding expressions of other languages as well.

% Just uncomment the input below when you're ready to go.

%For a start: Do not forget to give your Overleaf project (this paper) a recognizable name. This one could be called, for instance, Simik et al: OSL template. You can change the name of the project by hovering over the gray title at the top of this page and clicking on the pencil icon.

\section{Introduction}\label{sim:sec:intro}

Language Science Press is a project run for linguists, but also by linguists. You are part of that and we rely on your collaboration to get at the desired result. Publishing with LangSci Press might mean a bit more work for the author (and for the volume editor), esp. for the less experienced ones, but it also gives you much more control of the process and it is rewarding to see the quality result.

Please follow the instructions below closely, it will save the volume editors, the series editors, and you alike a lot of time.

\sloppy This stylesheet is a further specification of three more general sources: (i) the Leipzig glossing rules \citep{leipzig-glossing-rules}, (ii) the generic style rules for linguistics (\url{https://www.eva.mpg.de/fileadmin/content_files/staff/haspelmt/pdf/GenericStyleRules.pdf}), and (iii) the Language Science Press guidelines \citep{Nordhoff.Muller2021}.\footnote{Notice the way in-text numbered lists should be written -- using small Roman numbers enclosed in brackets.} It is advisable to go through these before you start writing. Most of the general rules are not repeated here.\footnote{Do not worry about the colors of references and links. They are there to make the editorial process easier and will disappear prior to official publication.}

Please spend some time reading through these and the more general instructions. Your 30 minutes on this is likely to save you and us hours of additional work. Do not hesitate to contact the editors if you have any questions.

\section{Illustrating OSL commands and conventions}\label{sim:sec:osl-comm}

Below I illustrate the use of a number of commands defined in langsci-osl.tex (see the styles folder).

\subsection{Typesetting semantics}\label{sim:sec:sem}

See below for some examples of how to typeset semantic formulas. The examples also show the use of the sib-command (= ``semantic interpretation brackets''). Notice also the the use of the dummy curly brackets in \REF{sim:ex:quant}. They ensure that the spacing around the equation symbol is correct. 

\ea \ea \sib{dog}$^g=\textsc{dog}=\lambda x[\textsc{dog}(x)]$\label{sim:ex:dog}
\ex \sib{Some dog bit every boy}${}=\exists x[\textsc{dog}(x)\wedge\forall y[\textsc{boy}(y)\rightarrow \textsc{bit}(x,y)]]$\label{sim:ex:quant}
\z\z

\noindent Use noindent after example environments (but not after floats like tables or figures).

And here's a macro for semantic type brackets: The expression \textit{dog} is of type $\stb{e,t}$. Don't forget to place the whole type formula into a math-environment. An example of a more complex type, such as the one of \textit{every}: $\stb{s,\stb{\stb{e,t},\stb{e,t}}}$. You can of course also use the type in a subscript: dog$_{\stb{e,t}}$

We distinguish between metalinguistic constants that are translations of object language, which are typeset using small caps, see \REF{sim:ex:dog}, and logical constants. See the contrast in \REF{sim:ex:speaker}, where \textsc{speaker} (= serif) in \REF{sim:ex:speaker-a} is the denotation of the word \textit{speaker}, and \cnst{speaker} (= sans-serif) in \REF{sim:ex:speaker-b} is the function that maps the context $c$ to the speaker in that context.\footnote{Notice that both types of small caps are automatically turned into text-style, even if used in a math-environment. This enables you to use math throughout.}$^,$\footnote{Notice also that the notation entails the ``direct translation'' system from natural language to metalanguage, as entertained e.g. in \citet{Heim.Kratzer1998}. Feel free to devise your own notation when relying on the ``indirect translation'' system (see, e.g., \citealt{Coppock.Champollion2022}).}

\ea\label{sim:ex:speaker}
\ea \sib{The speaker is drunk}$^{g,c}=\textsc{drunk}\big(\iota x\,\textsc{speaker}(x)\big)$\label{sim:ex:speaker-a}
\ex \sib{I am drunk}$^{g,c}=\textsc{drunk}\big(\cnst{speaker}(c)\big)$\label{sim:ex:speaker-b}
\z\z

\noindent Notice that with more complex formulas, you can use bigger brackets indicating scope, cf. $($ vs. $\big($, as used in \REF{sim:ex:speaker}. Also notice the use of backslash plus comma, which produces additional space in math-environment.

\subsection{Examples and the minsp command}

Try to keep examples simple. But if you need to pack more information into an example or include more alternatives, you can resort to various brackets or slashes. For that, you will find the minsp-command useful. It works as follows:

\ea\label{sim:ex:german-verbs}\gll Hans \minsp{\{} schläft / schlief / \minsp{*} schlafen\}.\\
Hans {} sleeps {} slept {} {} sleep.\textsc{inf}\\
\glt `Hans \{sleeps / slept\}.'
\z

\noindent If you use the command, glosses will be aligned with the corresponding object language elements correctly. Notice also that brackets etc. do not receive their own gloss. Simply use closed curly brackets as the placeholder.

The minsp-command is not needed for grammaticality judgments used for the whole sentence. For that, use the native langsci-gb4e method instead, as illustrated below:

\ea[*]{\gll Das sein ungrammatisch.\\
that be.\textsc{inf} ungrammatical\\
\glt Intended: `This is ungrammatical.'\hfill (German)\label{sim:ex:ungram}}
\z

\noindent Also notice that translations should never be ungrammatical. If the original is ungrammatical, provide the intended interpretation in idiomatic English.

If you want to indicate the language and/or the source of the example, place this on the right margin of the translation line. Schematic information about relevant linguistic properties of the examples should be placed on the line of the example, as indicated below.

\ea\label{sim:ex:bailyn}\gll \minsp{[} Ėtu knigu] čitaet Ivan \minsp{(} často).\\
{} this book.{\ACC} read.{\PRS.3\SG} Ivan.{\NOM} {} often\\\hfill O-V-S-Adv
\glt `Ivan reads this book (often).'\hfill (Russian; \citealt[4]{Bailyn2004})
\z

\noindent Finally, notice that you can use the gloss macros for typing grammatical glosses, defined in langsci-lgr.sty. Place curly brackets around them.

\subsection{Citation commands and macros}

You can make your life easier if you use the following citation commands and macros (see code):

\begin{itemize}
    \item \citealt{Bailyn2004}: no brackets
    \item \citet{Bailyn2004}: year in brackets
    \item \citep{Bailyn2004}: everything in brackets
    \item \citepossalt{Bailyn2004}: possessive
    \item \citeposst{Bailyn2004}: possessive with year in brackets
\end{itemize}

\section{Trees}\label{s:tree}

Use the forest package for trees and place trees in a figure environment. \figref{sim:fig:CP} shows a simple example.\footnote{See \citet{VandenWyngaerd2017} for a simple and useful quickstart guide for the forest package.} Notice that figure (and table) environments are so-called floating environments. {\LaTeX} determines the position of the figure/table on the page, so it can appear elsewhere than where it appears in the code. This is not a bug, it is a property. Also for this reason, do not refer to figures/tables by using phrases like ``the table below''. Always use tabref/figref. If your terminal nodes represent object language, then these should essentially correspond to glosses, not to the original. For this reason, we recommend including an explicit example which corresponds to the tree, in this particular case \REF{sim:ex:czech-for-tree}.

\ea\label{sim:ex:czech-for-tree}\gll Co se řidič snažil dělat?\\
what {\REFL} driver try.{\PTCP.\SG.\MASC} do.{\INF}\\
\glt `What did the driver try to do?'
\z

\begin{figure}[ht]
% the [ht] option means that you prefer the placement of the figure HERE (=h) and if HERE is not possible, you prefer the TOP (=t) of a page
% \centering
    \begin{forest}
    for tree={s sep=1cm, inner sep=0, l=0}
    [CP
        [DP
            [what, roof, name=what]
        ]
        [C$'$
            [C
                [\textsc{refl}]
            ]
            [TP
                [DP
                    [driver, roof]
                ]
                [T$'$
                    [T [{[past]}]]
                    [VP
                        [V
                            [tried]
                        ]
                        [VP, s sep=2.2cm
                            [V
                                [do.\textsc{inf}]
                            ]
                            [t\textsubscript{what}, name=trace-what]
                        ]
                    ]
                ]
            ]
        ]
    ]
    \draw[->,overlay] (trace-what) to[out=south west, in=south, looseness=1.1] (what);
    % the overlay option avoids making the bounding box of the tree too large
    % the looseness option defines the looseness of the arrow (default = 1)
    \end{forest}
    \vspace{3ex} % extra vspace is added here because the arrow goes too deep to the caption; avoid such manual tweaking as much as possible; here it's necessary
    \caption{Proposed syntactic representation of \REF{sim:ex:czech-for-tree}}
    \label{sim:fig:CP}
\end{figure}

Do not use noindent after figures or tables (as you do after examples). Cases like these (where the noindent ends up missing) will be handled by the editors prior to publication.

\section{Italics, boldface, small caps, underlining, quotes}

See \citet{Nordhoff.Muller2021} for that. In short:

\begin{itemize}
    \item No boldface anywhere.
    \item No underlining anywhere (unless for very specific and well-defined technical notation; consult with editors).
    \item Small caps used for (i) introducing terms that are important for the paper (small-cap the term just ones, at a place where it is characterized/defined); (ii) metalinguistic translations of object-language expressions in semantic formulas (see \sectref{sim:sec:sem}); (iii) selected technical notions.
    \item Italics for object-language within text; exceptionally for emphasis/contrast.
    \item Single quotes: for translations/interpretations
    \item Double quotes: scare quotes; quotations of chunks of text.
\end{itemize}

\section{Cross-referencing}

Label examples, sections, tables, figures, possibly footnotes (by using the label macro). The name of the label is up to you, but it is good practice to follow this template: article-code:reference-type:unique-label. E.g. sim:ex:german would be a proper name for a reference within this paper (sim = short for the author(s); ex = example reference; german = unique name of that example).

\section{Syntactic notation}

Syntactic categories (N, D, V, etc.) are written with initial capital letters. This also holds for categories named with multiple letters, e.g. Foc, Top, Num, etc. Stick to this convention also when coming up with ad hoc categories, e.g. Cl (for clitic or classifier).

An exception from this rule are ``little'' categories, which are written with italics: \textit{v}, \textit{n}, \textit{v}P, etc.

Bar-levels must be typeset with bars/primes, not with an apostrophe. An easy way to do that is to use mathmode for the bar: C$'$, Foc$'$, etc.

Specifiers should be written this way: SpecCP, Spec\textit{v}P.

Features should be surrounded by square brackets, e.g., [past]. If you use plus and minus, be sure that these actually are plus and minus, and not e.g. a hyphen. Mathmode can help with that: [$+$sg], [$-$sg], [$\pm$sg]. See \sectref{sim:sec:hyphens-etc} for related information.

\section{Footnotes}

Absolutely avoid long footnotes. A footnote should not be longer than, say, {20\%} of the page. If you feel like you need a long footnote, make an explicit digression in the main body of the text.

Footnotes should always be placed at the end of whole sentences. Formulate the footnote in such a way that this is possible. Footnotes should always go after punctuation marks, never before. Do not place footnotes after individual words. Do not place footnotes in examples, tables, etc. If you have an urge to do that, place the footnote to the text that explains the example, table, etc.

Footnotes should always be formulated as full, self-standing sentences.

\section{Tables}

For your tables use the table plus tabularx environments. The tabularx environment lets you (and requires you in fact) to specify the width of the table and defines the X column (left-alignment) and the Y column (right-alignment). All X/Y columns will have the same width and together they will fill out the width of the rest of the table -- counting out all non-X/Y columns.

Always include a meaningful caption. The caption is designed to appear on top of the table, no matter where you place it in the code. Do not try to tweak with this. Tables are delimited with lsptoprule at the top and lspbottomrule at the bottom. The header is delimited from the rest with midrule. Vertical lines in tables are banned. An example is provided in \tabref{sim:tab:frequencies}. See \citet{Nordhoff.Muller2021} for more information. If you are typesetting a very complex table or your table is too large to fit the page, do not hesitate to ask the editors for help.

\begin{table}
\caption{Frequencies of word classes}
\label{sim:tab:frequencies}
 \begin{tabularx}{.77\textwidth}{lYYYY} %.77 indicates that the table will take up 77% of the textwidth
  \lsptoprule
            & nouns & verbs  & adjectives & adverbs\\
  \midrule
  absolute  &   12  &    34  &    23      & 13\\
  relative  &   3.1 &   8.9  &    5.7     & 3.2\\
  \lspbottomrule
 \end{tabularx}
\end{table}

\section{Figures}

Figures must have a good quality. If you use pictorial figures, consult the editors early on to see if the quality and format of your figure is sufficient. If you use simple barplots, you can use the barplot environment (defined in langsci-osl.sty). See \figref{sim:fig:barplot} for an example. The barplot environment has 5 arguments: 1. x-axis desription, 2. y-axis description, 3. width (relative to textwidth), 4. x-tick descriptions, 5. x-ticks plus y-values.

\begin{figure}
    \centering
    \barplot{Type of meal}{Times selected}{0.6}{Bread,Soup,Pizza}%
    {
    (Bread,61)
    (Soup,12)
    (Pizza,8)
    }
    \caption{A barplot example}
    \label{sim:fig:barplot}
\end{figure}

The barplot environment builds on the tikzpicture plus axis environments of the pgfplots package. It can be customized in various ways. \figref{sim:fig:complex-barplot} shows a more complex example.

\begin{figure}
  \begin{tikzpicture}
    \begin{axis}[
	xlabel={Level of \textsc{uniq/max}},  
	ylabel={Proportion of $\textsf{subj}\prec\textsf{pred}$}, 
	axis lines*=left, 
        width  = .6\textwidth,
	height = 5cm,
    	nodes near coords, 
    % 	nodes near coords style={text=black},
    	every node near coord/.append style={font=\tiny},
        nodes near coords align={vertical},
	ymin=0,
	ymax=1,
	ytick distance=.2,
	xtick=data,
	ylabel near ticks,
	x tick label style={font=\sffamily},
	ybar=5pt,
	legend pos=outer north east,
	enlarge x limits=0.3,
	symbolic x coords={+u/m, \textminus u/m},
	]
	\addplot[fill=red!30,draw=none] coordinates {
	    (+u/m,0.91)
        (\textminus u/m,0.84)
	};
	\addplot[fill=red,draw=none] coordinates {
	    (+u/m,0.80)
        (\textminus u/m,0.87)
	};
	\legend{\textsf{sg}, \textsf{pl}}
    \end{axis} 
  \end{tikzpicture} 
    \caption{Results divided by \textsc{number}}
    \label{sim:fig:complex-barplot}
\end{figure}

\section{Hyphens, dashes, minuses, math/logical operators}\label{sim:sec:hyphens-etc}

Be careful to distinguish between hyphens (-), dashes (--), and the minus sign ($-$). For in-text appositions, use only en-dashes -- as done here -- with spaces around. Do not use em-dashes (---). Using mathmode is a reliable way of getting the minus sign.

All equations (and typically also semantic formulas, see \sectref{sim:sec:sem}) should be typeset using mathmode. Notice that mathmode not only gets the math signs ``right'', but also has a dedicated spacing. For that reason, never write things like p$<$0.05, p $<$ 0.05, or p$<0.05$, but rather $p<0.05$. In case you need a two-place math or logical operator (like $\wedge$) but for some reason do not have one of the arguments represented overtly, you can use a ``dummy'' argument (curly brackets) to simulate the presence of the other one. Notice the difference between $\wedge p$ and ${}\wedge p$.

In case you need to use normal text within mathmode, use the text command. Here is an example: $\text{frequency}=.8$. This way, you get the math spacing right.

\section{Abbreviations}

The final abbreviations section should include all glosses. It should not include other ad hoc abbreviations (those should be defined upon first use) and also not standard abbreviations like NP, VP, etc.


\section{Bibliography}

Place your bibliography into localbibliography.bib. Important: Only place there the entries which you actually cite! You can make use of our OSL bibliography, which we keep clean and tidy and update it after the publication of each new volume. Contact the editors of your volume if you do not have the bib file yet. If you find the entry you need, just copy-paste it in your localbibliography.bib. The bibliography also shows many good examples of what a good bibliographic entry should look like.

See \citet{Nordhoff.Muller2021} for general information on bibliography. Some important things to keep in mind:

\begin{itemize}
    \item Journals should be cited as they are officially called (notice the difference between and, \&, capitalization, etc.).
    \item Journal publications should always include the volume number, the issue number (field ``number''), and DOI or stable URL (see below on that).
    \item Papers in collections or proceedings must include the editors of the volume (field ``editor''), the place of publication (field ``address'') and publisher.
    \item The proceedings number is part of the title of the proceedings. Do not place it into the ``volume'' field. The ``volume'' field with book/proceedings publications is reserved for the volume of that single book (e.g. NELS 40 proceedings might have vol. 1 and vol. 2).
    \item Avoid citing manuscripts as much as possible. If you need to cite them, try to provide a stable URL.
    \item Avoid citing presentations or talks. If you absolutely must cite them, be careful not to refer the reader to them by using ``see...''. The reader can't see them.
    \item If you cite a manuscript, presentation, or some other hard-to-define source, use the either the ``misc'' or ``unpublished'' entry type. The former is appropriate if the text cited corresponds to a book (the title will be printed in italics); the latter is appropriate if the text cited corresponds to an article or presentation (the title will be printed normally). Within both entries, use the ``howpublished'' field for any relevant information (such as ``Manuscript, University of \dots''). And the ``url'' field for the URL.
\end{itemize}

We require the authors to provide DOIs or URLs wherever possible, though not without limitations. The following rules apply:

\begin{itemize}
    \item If the publication has a DOI, use that. Use the ``doi'' field and write just the DOI, not the whole URL.
    \item If the publication has no DOI, but it has a stable URL (as e.g. JSTOR, but possibly also lingbuzz), use that. Place it in the ``url'' field, using the full address (https: etc.).
    \item Never use DOI and URL at the same time.
    \item If the official publication has no official DOI or stable URL (related to the official publication), do not replace these with other links. Do not refer to published works with lingbuzz links, for instance, as these typically lead to the unpublished (preprint) version. (There are exceptions where lingbuzz or semanticsarchive are the official publication venue, in which case these can of course be used.) Never use URLs leading to personal websites.
    \item If a paper has no DOI/URL, but the book does, do not use the book URL. Just use nothing.
\end{itemize}

\section*{Abbreviations}

\begin{tabularx}{.5\textwidth}{@{}lX@{}}
% \textsc{acc}&accusative\\
\textsc{dat}&{dative}\\
\textsc{gen}&{genitive}\\
\end{tabularx}%
\begin{tabularx}{.5\textwidth}{@{}lX@{}}
\textsc{inst}&{instrumental}\\
% \textsc{nom}&nominativ\\
&\\ % this dummy row achieves correct vertical alignment of both tables
\end{tabularx}

\section*{Acknowledgements}
I presented a shorter version of this paper at the second meeting of the Slavic Linguistics Society in Berlin in August 2007 and profited from the discussion. For contemporary Russian, I have consulted Elena Gorishneva, Nikolai Grettschak, Wladimir Klimonow, and Faina Pimenova. For Slovenian, I am indebted to Boštjan Dvořák. For discussion on German and for help in various respects I am grateful to Brigitta Haftka. Barbara Jane Pheby and Jean Pheby have helped me with the English translation of the examples. To Uwe Junghanns and a \textit{JSL} reviewer I owe many valuable suggestions.

\sloppy
\printbibliography[heading=subbibliography,notkeyword=this]

\end{document}
