\documentclass[output=paper,colorlinks,citecolor=brown]{langscibook}
\ChapterDOI{10.5281/zenodo.15471447}
% \bibliography{localbibliography}

\author{Boban Arsenijević\affiliation{University of Graz}\orcid{0000-0002-1124-6319}}
% replace the above with you and your coauthors
% rules for affiliation: If there's an official English version, use that (find out on the official website of the university); if not, use the original
% orcid doesn't appear printed; it's metainformation used for later indexing

%%% uncomment the following line if you are a single author or all authors have the same affiliation
% \SetupAffiliations{mark style=none}

%% in case the running head with authors exceeds one line (which is the case in this example document), use one of the following methods to turn it into a single line; otherwise comment the line below out with % and ignore it
%\lehead{Šimík, Gehrke, Lenertová, Meyer, Szucsich \& Zaleska}
% \lehead{Radek Šimík et al.}

\title[Strengthening to reciprocity]{Strengthening to reciprocity: The syntax and semantics of \textsc{one$+$other} expressions}
% replace the above with your paper title
%%% provide a shorter version of your title in case it doesn't fit a single line in the running head
% in this form: \title[short title]{full title}
\abstract{I argue on the basis of data from Serbo-Croatian that reciprocity expressions combining the equivalents of the words \textit{one} and \textit{other} as two separate constituents bear a non-reciprocal compositional core meaning, as formulated by \citeposst{v10} and \citeposst{z14} work on Spanish, Russian and German data. The additional components of the two analyses postulated to capture reciprocity -- cumulation and type shift, respectively -- are not part of the expression's semantics. Rather, reciprocity emerges from pragmatics, in the competition of these expressions with closest competitors whose meanings do not even include reciprocal interpretations as special cases. Both a syntactic and a semantic analysis is provided.
\keywords{reciprocity, Serbo-Croatian, one-other expression, pragmatic competition}
}

%\abstract{Abstract goes here and should not have more than 150 words.



\begin{document}
\maketitle

% Just comment out the input below when you're ready to go.
%For a start: Do not forget to give your Overleaf project (this paper) a recognizable name. This one could be called, for instance, Simik et al: OSL template. You can change the name of the project by hovering over the gray title at the top of this page and clicking on the pencil icon.

\section{Introduction}\label{sim:sec:intro}

Language Science Press is a project run for linguists, but also by linguists. You are part of that and we rely on your collaboration to get at the desired result. Publishing with LangSci Press might mean a bit more work for the author (and for the volume editor), esp. for the less experienced ones, but it also gives you much more control of the process and it is rewarding to see the quality result.

Please follow the instructions below closely, it will save the volume editors, the series editors, and you alike a lot of time.

\sloppy This stylesheet is a further specification of three more general sources: (i) the Leipzig glossing rules \citep{leipzig-glossing-rules}, (ii) the generic style rules for linguistics (\url{https://www.eva.mpg.de/fileadmin/content_files/staff/haspelmt/pdf/GenericStyleRules.pdf}), and (iii) the Language Science Press guidelines \citep{Nordhoff.Muller2021}.\footnote{Notice the way in-text numbered lists should be written -- using small Roman numbers enclosed in brackets.} It is advisable to go through these before you start writing. Most of the general rules are not repeated here.\footnote{Do not worry about the colors of references and links. They are there to make the editorial process easier and will disappear prior to official publication.}

Please spend some time reading through these and the more general instructions. Your 30 minutes on this is likely to save you and us hours of additional work. Do not hesitate to contact the editors if you have any questions.

\section{Illustrating OSL commands and conventions}\label{sim:sec:osl-comm}

Below I illustrate the use of a number of commands defined in langsci-osl.tex (see the styles folder).

\subsection{Typesetting semantics}\label{sim:sec:sem}

See below for some examples of how to typeset semantic formulas. The examples also show the use of the sib-command (= ``semantic interpretation brackets''). Notice also the the use of the dummy curly brackets in \REF{sim:ex:quant}. They ensure that the spacing around the equation symbol is correct. 

\ea \ea \sib{dog}$^g=\textsc{dog}=\lambda x[\textsc{dog}(x)]$\label{sim:ex:dog}
\ex \sib{Some dog bit every boy}${}=\exists x[\textsc{dog}(x)\wedge\forall y[\textsc{boy}(y)\rightarrow \textsc{bit}(x,y)]]$\label{sim:ex:quant}
\z\z

\noindent Use noindent after example environments (but not after floats like tables or figures).

And here's a macro for semantic type brackets: The expression \textit{dog} is of type $\stb{e,t}$. Don't forget to place the whole type formula into a math-environment. An example of a more complex type, such as the one of \textit{every}: $\stb{s,\stb{\stb{e,t},\stb{e,t}}}$. You can of course also use the type in a subscript: dog$_{\stb{e,t}}$

We distinguish between metalinguistic constants that are translations of object language, which are typeset using small caps, see \REF{sim:ex:dog}, and logical constants. See the contrast in \REF{sim:ex:speaker}, where \textsc{speaker} (= serif) in \REF{sim:ex:speaker-a} is the denotation of the word \textit{speaker}, and \cnst{speaker} (= sans-serif) in \REF{sim:ex:speaker-b} is the function that maps the context $c$ to the speaker in that context.\footnote{Notice that both types of small caps are automatically turned into text-style, even if used in a math-environment. This enables you to use math throughout.}$^,$\footnote{Notice also that the notation entails the ``direct translation'' system from natural language to metalanguage, as entertained e.g. in \citet{Heim.Kratzer1998}. Feel free to devise your own notation when relying on the ``indirect translation'' system (see, e.g., \citealt{Coppock.Champollion2022}).}

\ea\label{sim:ex:speaker}
\ea \sib{The speaker is drunk}$^{g,c}=\textsc{drunk}\big(\iota x\,\textsc{speaker}(x)\big)$\label{sim:ex:speaker-a}
\ex \sib{I am drunk}$^{g,c}=\textsc{drunk}\big(\cnst{speaker}(c)\big)$\label{sim:ex:speaker-b}
\z\z

\noindent Notice that with more complex formulas, you can use bigger brackets indicating scope, cf. $($ vs. $\big($, as used in \REF{sim:ex:speaker}. Also notice the use of backslash plus comma, which produces additional space in math-environment.

\subsection{Examples and the minsp command}

Try to keep examples simple. But if you need to pack more information into an example or include more alternatives, you can resort to various brackets or slashes. For that, you will find the minsp-command useful. It works as follows:

\ea\label{sim:ex:german-verbs}\gll Hans \minsp{\{} schläft / schlief / \minsp{*} schlafen\}.\\
Hans {} sleeps {} slept {} {} sleep.\textsc{inf}\\
\glt `Hans \{sleeps / slept\}.'
\z

\noindent If you use the command, glosses will be aligned with the corresponding object language elements correctly. Notice also that brackets etc. do not receive their own gloss. Simply use closed curly brackets as the placeholder.

The minsp-command is not needed for grammaticality judgments used for the whole sentence. For that, use the native langsci-gb4e method instead, as illustrated below:

\ea[*]{\gll Das sein ungrammatisch.\\
that be.\textsc{inf} ungrammatical\\
\glt Intended: `This is ungrammatical.'\hfill (German)\label{sim:ex:ungram}}
\z

\noindent Also notice that translations should never be ungrammatical. If the original is ungrammatical, provide the intended interpretation in idiomatic English.

If you want to indicate the language and/or the source of the example, place this on the right margin of the translation line. Schematic information about relevant linguistic properties of the examples should be placed on the line of the example, as indicated below.

\ea\label{sim:ex:bailyn}\gll \minsp{[} Ėtu knigu] čitaet Ivan \minsp{(} často).\\
{} this book.{\ACC} read.{\PRS.3\SG} Ivan.{\NOM} {} often\\\hfill O-V-S-Adv
\glt `Ivan reads this book (often).'\hfill (Russian; \citealt[4]{Bailyn2004})
\z

\noindent Finally, notice that you can use the gloss macros for typing grammatical glosses, defined in langsci-lgr.sty. Place curly brackets around them.

\subsection{Citation commands and macros}

You can make your life easier if you use the following citation commands and macros (see code):

\begin{itemize}
    \item \citealt{Bailyn2004}: no brackets
    \item \citet{Bailyn2004}: year in brackets
    \item \citep{Bailyn2004}: everything in brackets
    \item \citepossalt{Bailyn2004}: possessive
    \item \citeposst{Bailyn2004}: possessive with year in brackets
\end{itemize}

\section{Trees}\label{s:tree}

Use the forest package for trees and place trees in a figure environment. \figref{sim:fig:CP} shows a simple example.\footnote{See \citet{VandenWyngaerd2017} for a simple and useful quickstart guide for the forest package.} Notice that figure (and table) environments are so-called floating environments. {\LaTeX} determines the position of the figure/table on the page, so it can appear elsewhere than where it appears in the code. This is not a bug, it is a property. Also for this reason, do not refer to figures/tables by using phrases like ``the table below''. Always use tabref/figref. If your terminal nodes represent object language, then these should essentially correspond to glosses, not to the original. For this reason, we recommend including an explicit example which corresponds to the tree, in this particular case \REF{sim:ex:czech-for-tree}.

\ea\label{sim:ex:czech-for-tree}\gll Co se řidič snažil dělat?\\
what {\REFL} driver try.{\PTCP.\SG.\MASC} do.{\INF}\\
\glt `What did the driver try to do?'
\z

\begin{figure}[ht]
% the [ht] option means that you prefer the placement of the figure HERE (=h) and if HERE is not possible, you prefer the TOP (=t) of a page
% \centering
    \begin{forest}
    for tree={s sep=1cm, inner sep=0, l=0}
    [CP
        [DP
            [what, roof, name=what]
        ]
        [C$'$
            [C
                [\textsc{refl}]
            ]
            [TP
                [DP
                    [driver, roof]
                ]
                [T$'$
                    [T [{[past]}]]
                    [VP
                        [V
                            [tried]
                        ]
                        [VP, s sep=2.2cm
                            [V
                                [do.\textsc{inf}]
                            ]
                            [t\textsubscript{what}, name=trace-what]
                        ]
                    ]
                ]
            ]
        ]
    ]
    \draw[->,overlay] (trace-what) to[out=south west, in=south, looseness=1.1] (what);
    % the overlay option avoids making the bounding box of the tree too large
    % the looseness option defines the looseness of the arrow (default = 1)
    \end{forest}
    \vspace{3ex} % extra vspace is added here because the arrow goes too deep to the caption; avoid such manual tweaking as much as possible; here it's necessary
    \caption{Proposed syntactic representation of \REF{sim:ex:czech-for-tree}}
    \label{sim:fig:CP}
\end{figure}

Do not use noindent after figures or tables (as you do after examples). Cases like these (where the noindent ends up missing) will be handled by the editors prior to publication.

\section{Italics, boldface, small caps, underlining, quotes}

See \citet{Nordhoff.Muller2021} for that. In short:

\begin{itemize}
    \item No boldface anywhere.
    \item No underlining anywhere (unless for very specific and well-defined technical notation; consult with editors).
    \item Small caps used for (i) introducing terms that are important for the paper (small-cap the term just ones, at a place where it is characterized/defined); (ii) metalinguistic translations of object-language expressions in semantic formulas (see \sectref{sim:sec:sem}); (iii) selected technical notions.
    \item Italics for object-language within text; exceptionally for emphasis/contrast.
    \item Single quotes: for translations/interpretations
    \item Double quotes: scare quotes; quotations of chunks of text.
\end{itemize}

\section{Cross-referencing}

Label examples, sections, tables, figures, possibly footnotes (by using the label macro). The name of the label is up to you, but it is good practice to follow this template: article-code:reference-type:unique-label. E.g. sim:ex:german would be a proper name for a reference within this paper (sim = short for the author(s); ex = example reference; german = unique name of that example).

\section{Syntactic notation}

Syntactic categories (N, D, V, etc.) are written with initial capital letters. This also holds for categories named with multiple letters, e.g. Foc, Top, Num, etc. Stick to this convention also when coming up with ad hoc categories, e.g. Cl (for clitic or classifier).

An exception from this rule are ``little'' categories, which are written with italics: \textit{v}, \textit{n}, \textit{v}P, etc.

Bar-levels must be typeset with bars/primes, not with an apostrophe. An easy way to do that is to use mathmode for the bar: C$'$, Foc$'$, etc.

Specifiers should be written this way: SpecCP, Spec\textit{v}P.

Features should be surrounded by square brackets, e.g., [past]. If you use plus and minus, be sure that these actually are plus and minus, and not e.g. a hyphen. Mathmode can help with that: [$+$sg], [$-$sg], [$\pm$sg]. See \sectref{sim:sec:hyphens-etc} for related information.

\section{Footnotes}

Absolutely avoid long footnotes. A footnote should not be longer than, say, {20\%} of the page. If you feel like you need a long footnote, make an explicit digression in the main body of the text.

Footnotes should always be placed at the end of whole sentences. Formulate the footnote in such a way that this is possible. Footnotes should always go after punctuation marks, never before. Do not place footnotes after individual words. Do not place footnotes in examples, tables, etc. If you have an urge to do that, place the footnote to the text that explains the example, table, etc.

Footnotes should always be formulated as full, self-standing sentences.

\section{Tables}

For your tables use the table plus tabularx environments. The tabularx environment lets you (and requires you in fact) to specify the width of the table and defines the X column (left-alignment) and the Y column (right-alignment). All X/Y columns will have the same width and together they will fill out the width of the rest of the table -- counting out all non-X/Y columns.

Always include a meaningful caption. The caption is designed to appear on top of the table, no matter where you place it in the code. Do not try to tweak with this. Tables are delimited with lsptoprule at the top and lspbottomrule at the bottom. The header is delimited from the rest with midrule. Vertical lines in tables are banned. An example is provided in \tabref{sim:tab:frequencies}. See \citet{Nordhoff.Muller2021} for more information. If you are typesetting a very complex table or your table is too large to fit the page, do not hesitate to ask the editors for help.

\begin{table}
\caption{Frequencies of word classes}
\label{sim:tab:frequencies}
 \begin{tabularx}{.77\textwidth}{lYYYY} %.77 indicates that the table will take up 77% of the textwidth
  \lsptoprule
            & nouns & verbs  & adjectives & adverbs\\
  \midrule
  absolute  &   12  &    34  &    23      & 13\\
  relative  &   3.1 &   8.9  &    5.7     & 3.2\\
  \lspbottomrule
 \end{tabularx}
\end{table}

\section{Figures}

Figures must have a good quality. If you use pictorial figures, consult the editors early on to see if the quality and format of your figure is sufficient. If you use simple barplots, you can use the barplot environment (defined in langsci-osl.sty). See \figref{sim:fig:barplot} for an example. The barplot environment has 5 arguments: 1. x-axis desription, 2. y-axis description, 3. width (relative to textwidth), 4. x-tick descriptions, 5. x-ticks plus y-values.

\begin{figure}
    \centering
    \barplot{Type of meal}{Times selected}{0.6}{Bread,Soup,Pizza}%
    {
    (Bread,61)
    (Soup,12)
    (Pizza,8)
    }
    \caption{A barplot example}
    \label{sim:fig:barplot}
\end{figure}

The barplot environment builds on the tikzpicture plus axis environments of the pgfplots package. It can be customized in various ways. \figref{sim:fig:complex-barplot} shows a more complex example.

\begin{figure}
  \begin{tikzpicture}
    \begin{axis}[
	xlabel={Level of \textsc{uniq/max}},  
	ylabel={Proportion of $\textsf{subj}\prec\textsf{pred}$}, 
	axis lines*=left, 
        width  = .6\textwidth,
	height = 5cm,
    	nodes near coords, 
    % 	nodes near coords style={text=black},
    	every node near coord/.append style={font=\tiny},
        nodes near coords align={vertical},
	ymin=0,
	ymax=1,
	ytick distance=.2,
	xtick=data,
	ylabel near ticks,
	x tick label style={font=\sffamily},
	ybar=5pt,
	legend pos=outer north east,
	enlarge x limits=0.3,
	symbolic x coords={+u/m, \textminus u/m},
	]
	\addplot[fill=red!30,draw=none] coordinates {
	    (+u/m,0.91)
        (\textminus u/m,0.84)
	};
	\addplot[fill=red,draw=none] coordinates {
	    (+u/m,0.80)
        (\textminus u/m,0.87)
	};
	\legend{\textsf{sg}, \textsf{pl}}
    \end{axis} 
  \end{tikzpicture} 
    \caption{Results divided by \textsc{number}}
    \label{sim:fig:complex-barplot}
\end{figure}

\section{Hyphens, dashes, minuses, math/logical operators}\label{sim:sec:hyphens-etc}

Be careful to distinguish between hyphens (-), dashes (--), and the minus sign ($-$). For in-text appositions, use only en-dashes -- as done here -- with spaces around. Do not use em-dashes (---). Using mathmode is a reliable way of getting the minus sign.

All equations (and typically also semantic formulas, see \sectref{sim:sec:sem}) should be typeset using mathmode. Notice that mathmode not only gets the math signs ``right'', but also has a dedicated spacing. For that reason, never write things like p$<$0.05, p $<$ 0.05, or p$<0.05$, but rather $p<0.05$. In case you need a two-place math or logical operator (like $\wedge$) but for some reason do not have one of the arguments represented overtly, you can use a ``dummy'' argument (curly brackets) to simulate the presence of the other one. Notice the difference between $\wedge p$ and ${}\wedge p$.

In case you need to use normal text within mathmode, use the text command. Here is an example: $\text{frequency}=.8$. This way, you get the math spacing right.

\section{Abbreviations}

The final abbreviations section should include all glosses. It should not include other ad hoc abbreviations (those should be defined upon first use) and also not standard abbreviations like NP, VP, etc.


\section{Bibliography}

Place your bibliography into localbibliography.bib. Important: Only place there the entries which you actually cite! You can make use of our OSL bibliography, which we keep clean and tidy and update it after the publication of each new volume. Contact the editors of your volume if you do not have the bib file yet. If you find the entry you need, just copy-paste it in your localbibliography.bib. The bibliography also shows many good examples of what a good bibliographic entry should look like.

See \citet{Nordhoff.Muller2021} for general information on bibliography. Some important things to keep in mind:

\begin{itemize}
    \item Journals should be cited as they are officially called (notice the difference between and, \&, capitalization, etc.).
    \item Journal publications should always include the volume number, the issue number (field ``number''), and DOI or stable URL (see below on that).
    \item Papers in collections or proceedings must include the editors of the volume (field ``editor''), the place of publication (field ``address'') and publisher.
    \item The proceedings number is part of the title of the proceedings. Do not place it into the ``volume'' field. The ``volume'' field with book/proceedings publications is reserved for the volume of that single book (e.g. NELS 40 proceedings might have vol. 1 and vol. 2).
    \item Avoid citing manuscripts as much as possible. If you need to cite them, try to provide a stable URL.
    \item Avoid citing presentations or talks. If you absolutely must cite them, be careful not to refer the reader to them by using ``see...''. The reader can't see them.
    \item If you cite a manuscript, presentation, or some other hard-to-define source, use the either the ``misc'' or ``unpublished'' entry type. The former is appropriate if the text cited corresponds to a book (the title will be printed in italics); the latter is appropriate if the text cited corresponds to an article or presentation (the title will be printed normally). Within both entries, use the ``howpublished'' field for any relevant information (such as ``Manuscript, University of \dots''). And the ``url'' field for the URL.
\end{itemize}

We require the authors to provide DOIs or URLs wherever possible, though not without limitations. The following rules apply:

\begin{itemize}
    \item If the publication has a DOI, use that. Use the ``doi'' field and write just the DOI, not the whole URL.
    \item If the publication has no DOI, but it has a stable URL (as e.g. JSTOR, but possibly also lingbuzz), use that. Place it in the ``url'' field, using the full address (https: etc.).
    \item Never use DOI and URL at the same time.
    \item If the official publication has no official DOI or stable URL (related to the official publication), do not replace these with other links. Do not refer to published works with lingbuzz links, for instance, as these typically lead to the unpublished (preprint) version. (There are exceptions where lingbuzz or semanticsarchive are the official publication venue, in which case these can of course be used.) Never use URLs leading to personal websites.
    \item If a paper has no DOI/URL, but the book does, do not use the book URL. Just use nothing.
\end{itemize}

\section{Introduction}\label{sec:arsenijevic:intro}

Certain languages, including Serbo-Croatian (SC) as exemplified in \REF{ex:arsenijevic:Jarci}, but also German, Russian, and Spanish, display a strategy for marking reciprocity that involves the combination of the counterparts of the English words \textit{one} and \textit{other} as two separate constituents (\textsc{one$+$other} expressions in further text).

\ea\label{ex:arsenijevic:Jarci}
\gll Jarci su se sudarali \textit{jedan} s \textit{drugim}. \\
capricorn.\textsc{pl} \textsc{aux.pl} \textsc{refl} collided.\textsc{pl} 
one.\textsc{nom} with other.\textsc{inst}\\
\glt`The capricorns were colliding with each other.' \hfill (SC) 
\z

\noindent \citet{v10} and \citet{z14} provide compositional analyses of the meaning of expressions of this type, which hold in non-reciprocal as well as in reciprocal situations. In order to eliminate the non-reciprocal meanings, both postulate additional operations that enforce reciprocity. I argue in this paper that the non-reciprocal analyses correctly capture the narrow meaning of the expressions, and that reciprocal interpretation emerges from pragmatic strengthening.

I provide evidence from SC for the non-reciprocal interpretation of the \textsc{one$+$other} expression, and for the emergence of the reciprocal interpretation through a competition with a range of alternatives in which the \textsc{one$+$other} expression is substituted by a minimal pair sharing its surface pattern but with different lexical items. Based on the comparison of the SC \textsc{one$+$other} expression with its minimal pairs, I further refine Vicente's and Zimmermann's analyses. In particular, I point out the role of the non-specific indefinite nature of the two components of this expression, and of their being referentially restricted by the same antecedent, in addition to their competition with their semantically similar alternatives. I also sketch a syntactic analysis, and discuss the roles of grammatical number and aspect in their interpretation.

The paper is organized as follows. \sectref{sec:arsenijevic:2} presents the expression under discussion and the two analyses I depart from, \sectref{sec:arsenijevic:3} argues for a weaker meaning of the \textsc{one$+$other} expression and its pragmatic strengthening to various degrees of reciprocal strength, and \sectref{sec:arsenijevic:4} overviews its possible alternatives in SC and their relevant properties. \sectref{sec:arsenijevic:5} sketches a syntactic analysis. In \sectref{sec:arsenijevic:6}, I discuss the central aspects of the semantic interaction of this expression with grammatical number and aspect, which are both richly morphologically marked in SC and reveal important aspects of the meaning of the expression. \sectref{sec:arsenijevic:concl} concludes.


\section{\citet{v10} and \citet{z14} on the reciprocity marker of the type \textsc{one$+$other}}\label{sec:arsenijevic:2}

Reciprocity expressions, such as English \textit{each} \textit{other} or \textit{one} \textit{another} and their counterparts in other languages are characterized by a great semantic and syntactic variation in the range of available marking strategies such as reflexives or reciprocal adverbs (\textit{mutually}, \textit{reciprocally}), as well as in the interpretation between different degrees of strength (according to Dalrymple et al: strong, intermediate, one-way weak, intermediate alternative reciprocity, and the inclusive alternative ordering) and special reciprocal meanings (\citealt{dkmp94}, \citealt{n07}). The dichotomy is thus not a simple one, i.e. between the situations in which members of a particular plurality of football players scream  \REF{ex:arsenijevic:Football-a} and those in which each member of the plurality of football players screams at, and is screamed at by, each other member \REF{ex:arsenijevic:Football-b}.

\ea\label{ex:arsenijevic:Football}
\ea[]{The football players were screaming.}\label{ex:arsenijevic:Football-a}
\ex[]{The football players were screaming at each other.}\label{ex:arsenijevic:Football-b}
\z\z

\noindent A range of pragmatically salient in-between scenarios are available for \REF{ex:arsenijevic:Football-b}, such as each member participating at least once as the one who screams and at least once as the one who is screamed at.

I focus on the type of reciprocal construction which involves a reciprocity marker composed of two expressions: the first of which shares the semantics of the English word \textit{one}, and the second of the English word \textit{other} (henceforth the \textsc{one$+$other} expression), available in a selection of Germanic, Romance, and Slav\-ic languages, and previously analyzed by \citet{v10} and \citet{z14}. This construction is exemplified in \REF{ex:arsenijevic:Nachbarn1-D}--\REF{ex:arsenijevic:Nachbarn1-E} (\citeauthor{z14}'s examples (1)--(3)), for German, Russian and Spanish, respectively. \citeauthor{z14} remarks that the German example is rather marked, due to the availability of a related construction with the reciprocity marker \textit{einander} `one another'.

\ea\label{ex:arsenijevic:Nachbarn1-D}
\gll Die Nachbarn helfen \textit{die} \textit{einen} \textit{den} \textit{anderen}. \\
 the neighbor.\textsc{pl} help.\textsc{pl} the.\textsc{pl} one.\textsc{pl} the.\textsc{dat.\textsc{pl}} other.\textsc{pl}\\
\glt`The neighbors help each other.' \hfill (German)
\z

\ea\label{ex:arsenijevic:Nachbarn1-R}
\gll Sosedi pomogajut \textit{odni} \textit{drugim}.\\
neighbor.\textsc{pl} help.\textsc{pl} one.\textsc{pl} other.\textsc{dat.\textsc{pl}}\\
\glt`The neighbors help each other.' \hfill (Russian)
\z

\ea\label{ex:arsenijevic:Nachbarn1-E}
\gll Los vecinos se ayudan \textit{los} \textit{unos} \textit{a} \textit{los} \textit{otros}.\\
 the neighbor.\textsc{pl} \textsc{refl} help the.\textsc{pl} one.\textsc{pl} to the.\textsc{pl} other.\textsc{pl}\\
\glt`The neighbors help each other.' \hfill (Spanish)
\z

\noindent \citet{v10} and \citet{z14}, in two closely related analyses, aim to identify the compositional semantics of these expressions. 

While \citet{b01}, in her discussion of the English reciprocal \textit{each} \textit{other}, argues that its components \textit{each} and \textit{other} are (relational) definites, Vicente treats the two components of the \textsc{one$+$other} expression as indefinites, one of which (that expressed by a counterpart of \textit{one}, henceforth the \textsc{one}-component) freely ranges over the plurality introduced by the antecedent, while the other (expressed by the counterpart of \textit{other}, henceforth the \textsc{other}-component) is relational, i.e. additionally specified as non-overlapping with the \textsc{one}-component. As this derives a single non-reciprocal event interpretation whereby two mutually disjoint indefinites from the same plurality introduce the respective event participants, \citet{v10} employs \citeposst{s98} and \citeposst{b01} cumulative operator (coded as ** in \REF{ex:arsenijevic:Hombres} below) to achieve the reciprocal interpretation.\footnote{The formula is slightly modified, without affecting the analysis.}

\ea\label{ex:arsenijevic:Hombres}
\ea
\gll Los hombres conocen los unos a los otros\\
The men know the one.\textsc{pl} to the others\\
\glt `The men know each other.' \hfill (Spanish)\\
\ex
Let $S$ be the set \{the men\}\\
**$[\exists x, \exists y [\textsc{know}([y \in B \subset S], [[x \in A \subset S] \wedge [A \cap B = \varnothing])]]$ 

A cumulation of events in which a non-specific member of some given plurality S knows another non-specific member of the same plurality, with which it does not overlap.
\z
\z

%\ea\label{ex:arsenijevic:Hombres}
%\gll Los hombres conocen los unos a los otros\\
%The men know the one.\textsc{pl} to the others\\
%\glt `The men know each other.' \hfill \textsc{Spanish}\\
%Let $S$ be the set \{the men\}\\
%**$[\exists x, \exists y [\textsc{know}([y \in B \subset S], [[x \in A \subset S] \wedge [A \cap B = \varnothing])]]$ \footnote{The formula is adopted with slight modification, without affecting the analysis.} \\ 
%
 %A cumulation of events in which a non-specific member of some given plurality S knows another non-specific member of the same plurality, with which it does not overlap.
% 
%\z

\noindent The definition of the cumulation operation ** from \citet[][304]{s98} is provided in \REF{ex:arsenijevic:Sternefeld} (his (5)).

\ea\label{ex:arsenijevic:Sternefeld}
For any two-place relation $R$, let $\text{**}R$ be the smallest relation such that\\
 \ea $R \subseteq \text{**}R$, and\\
 \ex if $\langle a, b\rangle \in \text{**}R$ and $\langle c, d \rangle \in \text{**}R$, then $\langle a \oplus c, b \oplus d\rangle \in \text{**}R$. 
\z
\z

\noindent Crucially, the interpretation is assumed to take the maximal result of the cumulation operation, i.e. the maximal set of possible events (i.e. relations) satisfying the description. Due to the indefinite nature of the event participants in each event, the same entity ends up occurring both as the `knower' and as the `knowee' in some events, yielding the reciprocal semantics. I argue in this paper that the semantics achieved through the cumulation, and its maximality, are both possible, but not obligatory for the interpretation of this expression. 

On \citeposst{z14} analysis, the \textsc{one$+$other} expression is a DP projected by the \textsc{other}-component, which takes another DP projected by the \textsc{one}-component as its specifier, as in \REF{ex:arsenijevic:DP}.

\ea\label{ex:arsenijevic:DP}
[DP [DP die einen] [D$'$ den anderen]]
\z

\noindent Case is assigned to the big DP, as it occurs in the corresponding argument position -- the indirect object in the examples in \REF{ex:arsenijevic:Nachbarn1-D}--\REF{ex:arsenijevic:Nachbarn1-E} -- which is why in these examples it bears the dative case. The DP in its specifier is assigned case (nominative in all the examples discussed so far) through agreement with the antecedent.

As formally presented in \REF{ex:arsenijevic:Definite} (the formula given is mildly modified compared to the origial), the meaning of the bigger DP is composed from the two DPs, where the semantics of the \textsc{other}-component, specifying distinctness, plays a crucial role. Similar to \citeposst{b01} analysis, the \textsc{one}-component is taken to be a specific individual which is part of the plural introduced by the antecedent (`the neighbors' in \REF{ex:arsenijevic:Nachbarn1-D}--\REF{ex:arsenijevic:Nachbarn1-E}). The definite article (which is plausibly silent in Russian, i.e. only present in the form of a corresponding feature) marks uniqueness. In combination, the \textsc{one}-component denotes a unique individual from the plurality denoted by the related argument -- in the present case a unique neighbor. The \textsc{other}-component ranges over the difference between the restrictor plurality and the referent of the \textsc{one}-component, in the example used for illustration, all the neighbors except for the one referred to by the \textsc{one}-component.

\ea\label{ex:arsenijevic:Definite}
\ea the definite article: $\lambda P_2 \lambda P_1 \exists ! x \, [P_1 \, x] \wedge [P_2 \, x]$\\
\ex the \textsc{one}-item: $\lambda x \, x \in Z$\\
\ex the \textsc{other}-item: $\lambda y \exists x [x \neq y] \wedge x \in Z \wedge y \in Z$
\z \z

\noindent Similar to \citeposst{v10} %Vicente's 
analysis, the resulting interpretation is non-reciprocal. One specific subplurality of the antecedent is associated with one thematic role, and another, different one with the other, i.e. one specific subgroup of the neighbors helps another, disjoint, specific subgroup. To secure a reciprocal interpretation, \citet{z14} %Zimmermann 
introduces a type-shift operation formalized as in \REF{ex:arsenijevic:Definite2}, which directly assigns reciprocity.

\ea\label{ex:arsenijevic:Definite2}
Type-shift into the reciprocal interpretation: $\lambda x \lambda y \lambda R \, \cnst{rec}\, [R \, y \, x]$
\z

\noindent On both views, the \textsc{one$+$other} expression is closely related to the sentences with items \textit{one} and \textit{other} (i.e. their equivalents) realized within lexically headed NPs in the respective positions, as in \REF{ex:arsenijevic:Einen}--\REF{ex:arsenijevic:Unos}, in which no reciprocal interpretation is available.

\ea\label{ex:arsenijevic:Einen}
\gll Die einen Nachbarn helfen den anderen \minsp{(} Nachbarn).\\
the one.\textsc{pl} neighbor.\textsc{pl} help.\textsc{pl} the.\textsc{dat.\textsc{pl}} other.\textsc{pl} {} neighbor.\textsc{pl}\\
\glt `One group of neighbors helps the other (group of neighbors).' \hfill (German)
\z

\ea\label{ex:arsenijevic:Odni}
\gll Odni sosedi pomogajut drugim \minsp{(} sosedami).\\
one.\textsc{pl} neighbor.\textsc{pl} help.\textsc{pl} other.\textsc{dat.\textsc{pl}} {} neighbor.\textsc{dat.\textsc{pl}}\\
\glt `One group of neighbors helps the other (group of neighbors).' \hfill (Russian)
\z

\ea\label{ex:arsenijevic:Unos}
\gll Los unos vecinos ayudan a los otros \minsp{(} vecinos).\\
the one.\textsc{pl} neighbor.\textsc{pl} help to the.\textsc{pl} other.\textsc{pl} {} neighbor.\textsc{pl}\\

\glt `One group of neighbors helps the other (group of neighbors).' \hfill (Spanish)
\z 

\noindent Both analyses rely on the disjoint reference of the components of the \textsc{one$+$other} expression, which share a plural restriction introduced by the antecedent. \citet{v10} %Vicente 
includes in his analysis a plurality of events, each member of which independently selects a pair of participants for two fixed participant roles from a fixed restrictor set, while \citet{z14} %Zimmermann 
focuses on the description of a single (sub)event and introduces reciprocity via a type-shift. It is striking that both analyses compositionally derive interpretations which are not necessarily reciprocal, but include reciprocity as a special case, and both therefore need to restrict the derived interpretation to reciprocity by stipulating an additional operation (maximal cumulation for \citeauthor{v10}, type shift for \citeauthor{z14}%Zimmermann
), which either has a non-compositional origin, or somehow comes from the syntactic structure of the expression. As the expressions in \REF{ex:arsenijevic:Einen}--\REF{ex:arsenijevic:Unos} do not trigger reciprocal
interpretations, in the latter case, this component must be tied to the structural derivation of the \textsc{one$+$other} expression in \REF{ex:arsenijevic:Nachbarn1-D}--\REF{ex:arsenijevic:Nachbarn1-E} from those figuring in \REF{ex:arsenijevic:Einen}--\REF{ex:arsenijevic:Unos}.

In the rest of the paper, I argue that \citeauthor{v10}'s %Vicente's 
and \citeauthor{z14}'s %Zimmermann's 
compositional analyses are correct, and that no additional operations should be postulated in the narrow semantics, as reciprocity is not compositionally contributed by the \textsc{one$+$other} expression. Reciprocity is imposed in pragmatics, through a strengthening resulting from the competition of the sentences containing the \textsc{one$+$other} expression with two relevant sets of alternatives. I further tackle the question whether the \textsc{one$+$other} expression, i.e. its components, are specific or non-specific, opting, with \citeauthor{v10}%Vicente
, for the latter option. On this ground, I outline the syntactic structure underlying these expressions, as well as the pragmatic competition driving the reciprocal strengthening -- which is the question that I tackle first.

\section{Reciprocity as pragmatic strengthening}\label{sec:arsenijevic:3}

One of the striking properties of the \textsc{one$+$other} expression that both \citet{z14} and \citet{v10} %Zimmermann and Vicente 
aim to account for is that, as they put it, it can be quantified. As \citeauthor{v10} %Vicente 
shows for Spanish, and \citeauthor{z14} %Zimmermann 
additionally for Russian and German, the \textsc{one$+$other} expression can occur with quantifiers which give it different reciprocal strengths. \citeauthor{v10}'s data from Spanish are given in \REF{ex:arsenijevic:Todos-b}--\REF{ex:arsenijevic:Todos-c} (his (50) and (52)).

\ea\label{ex:arsenijevic:Todos}
\ea[] {\gll Los hombres conocen los unos a los otros\\
The man.\textsc{pl} know the one.\textsc{pl} to the other.\textsc{pl}\\
\glt `The men know each other.'}\label{ex:arsenijevic:Todos-a}

\ex[]{\gll Los hombres conocen cada uno a todos los otros.\\
 The man.\textsc{pl} know each one to all the other.\textsc{pl}\\
 \glt `Each of the men knows each of the others.'}\label{ex:arsenijevic:Todos-b}

\ex[]{\gll Los hombres conocen cada uno a \minsp{(} alguno) otro.\\
 The man.\textsc{pl} know each one to {} some other\\
 \glt `Each of the men knows some other man.' \hfill (Spanish)}\label{ex:arsenijevic:Todos-c}
 \z \z

\noindent I argue rather that the expression under discussion illustrated in \REF{ex:arsenijevic:Todos-a} (denoting reciprocity of any degree of strength, underspecified) stands in competition with those in \REF{ex:arsenijevic:Todos-b} (denoting strong reciprocity) and \REF{ex:arsenijevic:Todos-c} (illustrating one-way weak reciprocity), as well as with the expressions of the type in \REF{ex:arsenijevic:Unos}, and that these two competitions crucially determine its overall interpretation. Let me first discuss the latter competition, the one which strengthens the meaning of the \textsc{one$+$other} expression to reciprocity, and then return to the former, which works in the opposite direction.

The two competing types of expressions are given in \REF{ex:arsenijevic:Otros}, from Spanish.

\ea\label{ex:arsenijevic:Otros}
\ea[] {\gll Los unos vecinos ayudan a los otros.\\
 the one.\textsc{pl} neighbor.\textsc{pl} help to the.\textsc{pl} other.\textsc{pl} \\
\glt `One group of neighbors helps the other.'}\label{ex:arsenijevic:Otros-a}

\ex[]{\gll Los vecinos se ayudan los unos a los otros.\\
 the neighbor.\textsc{pl} \textsc{refl} help the.\textsc{pl} one.\textsc{pl} to the.\textsc{pl} other.\textsc{pl}\\
\glt `The neighbors help each other.' \hfill (Spanish)}\label{ex:arsenijevic:Otros-b}
 \z \z

\noindent On \citeauthor{v10}'s and \citeauthor{z14}'s analysis, their compositional meanings differ only in one restriction. While in \REF{ex:arsenijevic:Otros-a} the two groups of neighbors are referentially independent of each other, in \REF{ex:arsenijevic:Otros-b} they are restricted to being parts of the plurality denoted by the antecedent, in this case the subject DP \textit{los vecinos} `the neighbors'. Taking \citeauthor{v10}'s view, an additional difference is that the \textsc{one$+$other} expression \textit{los unos a los otros} `ones to others' in \REF{ex:arsenijevic:Otros-b} is indefinite, while the corresponding expressions \textit{los unos vicinos} `the one group of neighbors' and \textit{los otros vicinos} `the other group of neighbors' in \REF{ex:arsenijevic:Otros-a} are both definite. As a result, \REF{ex:arsenijevic:Otros-a} means that there were one or more events such that the one group of neighbors helped the other, and \REF{ex:arsenijevic:Otros-b} that there were one or more events such that one subgroup of the neighbors helped a disjoint one. \citeauthor{v10} further applies max\-i\-mal cumulation to impose those pluralities of events which effectively imply reciprocity (e.g., if all the possible pairs of subgroups of the neighbors engage in helping events, strong reciprocity obtains).

I argue that the maximal cumulation is not part of the semantics of sentences involving the \textsc{one$+$other} expression. Rather, the non-specific nature of the \textit{one+ other} expression and the shared restriction of its two components make this expression more marked, which pragmatically promotes its strongest interpretations -- which are the reciprocal ones. Crucial factors are antipresupposition (\citealt{h91}, \citealt{p06}): the fact that the more marked expression is uttered, and the salience of the reciprocal interpretation: it is a prominent interpretation which fits a large number of frequently obtaining contexts. This leads to a pragmatic strengthening of the type of sentences in \REF{ex:arsenijevic:Otros-b} to reciprocity. In \sectref{sec:arsenijevic:4}, \sectref{sec:arsenijevic:5}, and \sectref{sec:arsenijevic:6}, %sections 4, 5 and 6
I discuss the roles of non-specificity, of the shared restriction, of grammatical number and of grammatical aspect in the susceptibility of the \textsc{one$+$other} expression to strengthening to reciprocity.

First, I have to show that the \textsc{one$+$other} expression indeed allows for non-reciprocal and non-plural interpretations, and that it is not referentially specific. To test this view, I designed a questionnaire in SC. Apart from the practical reasons, SC is selected because it displays a broad range of different alternatives for the \textsc{one$+$other} expression (see \sectref{sec:arsenijevic:4}), thus giving ample room to test the competition hypothesis.

The questionnaire consisted of two types of questions. One asked for the truth of a sentence involving the \textsc{one$+$other} expression in a non-reciprocal, non-specific context, with a forced choice between the answers \textit{true} and \textit{false}, as illustrated in \REF{ex:arsenijevic:Context1}. The other asked for a numerical answer to a question involving the expression \textsc{one$+$other} for a context that yields different answers for the reciprocal and for the more general interpretation, with a forced choice between three answers: the number of events of the kind described by the verb irrespective whether reciprocal or not, counting reciprocal events as one event, the number of events of the kind described by the verb irrespective whether reciprocal or not, counting each reciprocal event as two events, only the number of reciprocal events, and the cumulative number of reciprocal events (i.e. counting also any pair of independent events which are reciprocal with respect to each other; in the example in \REF{ex:arsenijevic:Context2}, this is 2, as Matija stepped on Luka's foot twice, and Luka stepped on Matija's foot twice). After filling in the questionnaire the informants were asked if they had any comments.

\eanoraggedright
\textit{Context:} One monk has hit another monk once. Never before had a monk of that monastery hit another monk. The bishop says: `I don't know, and I don't want to know which monk hit which other monk. I just want to stress one thing.'\medskip
\begin{xlist}
\exi{}[]{\gll Ovo je prvi put da monasi ovog manastira udare jedan drugog.\\
 this is first time that monk.\textsc{pl} this.\textsc{gen} monastery.\textsc{gen} hit one other\\
\glt `This is the first time that a monk of this monastery hits another one.'\medskip}
\end{xlist}
For the given context, the last sentence the bishop uttered is

\begin{itemize}
\item false
\item true
\end{itemize}\label{ex:arsenijevic:Context1}
 \z 

\eanoraggedright\label{ex:arsenijevic:Context2} \textit{Context:} Matija and Luka are playing a game where each player is supposed to step on the other player's foot and to avoid being stepped on. They played 5 rounds. In rounds 1, 2 and 5 Luka stepped on Matija's foot, in round 4 Matija stepped on Luka's foot, and in round 3 both Matija stepped on Luka's foot and Luka stepped on Matija's foot.\medskip

\begin{xlist}
\exi{}[]{\gll Koliko puta su dečaci zgazili jedan drugog?\\
 {how many} times \textsc{aux} boy.\textsc{pl} {stepped on} one other\\
\glt `How many times did the boys step on each other's foot?'}
\begin{itemize}
\item 6
\item 5
\item 1
\item 2
\end{itemize}
\end{xlist}
 \z

\noindent The questionnaire was administered to 58 native speakers. The prediction of the present analysis for the first task was that a significant number of participants will judge the sentences as true, thus confirming that they can be used for non-reciprocal single instantiation events, and with non-specific reference. The prediction was confirmed: 81\% of answers judged the sentences to be true in the given context, while 19\% judged them to be false. Some of the informants commented that they only receive the reciprocal interpretation (the other monk hit back), hence at least some rejections were because the speakers could only interpret the sentences reciprocally. This does not contradict the proposed analysis, as there is variation in the ability to impose or suppress the contribution of pragmatics (\citealt{i18}, see also \citealt{ma13} for SC). The hypothesis is confirmed by answers which had the non-reciprocal single event interpretation (and which made up a significant majority).

The prediction for the second task was that a significant number of participants would choose the numbers that require the inclusion of non-reciprocal events (i.e. 5 or 6 for the example in \REF{ex:arsenijevic:Context2}). Contra the report in \citet{d08}, a vast majority of speakers accepted 6 as the answer (86\% of the answers, this number was among the ones selected), i.e. its counterparts in other questions, and a few of them (also) accept (the type of answer represented by) the
numerosity 5 (around 12\% of answers). Both these answers count non-reciprocal events in. Only 5\% of times the answers of the type specifying the number of narrowly reciprocal event instances were accepted (1 in the given example), and not a single time those corresponding to a cumulation of reciprocal events (2 in the given example). 

Note that the prediction of the proposed view is not that no one will have a strict reciprocal interpretation, as it is a possible, and moreover strong and salient interpretation. The distinctive prediction was that a significant number of speakers will accept the non-reciprocal interpretation. This is clearly confirmed.

In language use, non-reciprocal interpretations are relatively rare, but not unusual for this type of expressions. In a Google search, among the first 100 hits for the SC \textit{jedni drugima} `ones to others', 4 were of the type in \REF{ex:arsenijevic:Ribari}, where again it is possible that only a small number of the fishermen ever have excess catch, and that none of those ever happen to receive excess catch from other fishermen, for the sentence to be true -- hence a non-reciprocal scenario.

\ea\label{ex:arsenijevic:Ribari} \gll Ribari višak ulova daju jedni drugima, lovci ga ostave u šumi.\\
 fisherman.\textsc{pl} excess catch give.\textsc{pl} one.\textsc{pl} other.\textsc{dat.\textsc{pl}} hunters it leave in wood\\
 \glt `Fishermen give their excess catch to other fishermen, hunters leave it in the wood.'
 \z

\largerpage
\noindent In summary, the \textsc{one$+$other} expression in SC does not necessarily impose a reciprocal interpretation, or even a plural event interpretation. Moreover, the first task confirmed that the \textsc{one$+$other} expression is neither definite nor specific. 

If reciprocity emerges through competition, the availability of non-reciprocal interpretations is expected to be higher in languages that have stronger minimal alternatives for expressing reciprocity, i.e. where there are productive competitors with the \textsc{one$+$other} expression along the other dimension, introduced at the beginning of this section. This is the topic of the next section.

\section{Alternatives}\label{sec:arsenijevic:4}
\begin{sloppypar}
In this section, I examine the possible (quantified) alternatives for the \textsc{one$+$other} expression, sharing exactly the same syntactic structure and position, but contributing different semantics. \citet{v10} and \citet{z14} report two such alternatives, which derive the strong and the weak reciprocal interpretation, respectively. In combination with the fact that in Spanish the universal quantifier is added on top of the components \textsc{one} and \textsc{other}, this suggests that this position is universally filled with \textsc{one$+$other} as a reciprocal marker, and that this marker can be strengthened by quantification to narrow the reciprocal interpretation to a particular degree of strength. Serbo-Croatian is similar to Russian and Spanish in allowing quantifiers to occur in place of the \textsc{one$+$other} construction (see \citeauthor{m14} \citeyear{m14}, \citeyear{m16} for a detailed discussion). However, when it does, it does not preserve the \textsc{one}- and \textsc{other}-component. 
\end{sloppypar}

The view advocated in this paper takes reciprocity to be pragmatically derived, from the less restricted narrow semantics of the expression. While on the view on which \textsc{one$+$other} expressions have reciprocal semantics, it is expected that this expression takes only those combinations of quantifiers that yield the salient degrees of strength of reciprocity (weak and strong reciprocity in particular), the present view makes no such prediction. In SC, a much broader range of combinations are available than those two which derive the weak and strong reciprocity, discussed by \citet{v10} and \citet{z14} and illustrated in \REF{ex:arsenijevic:Todos-b}--\REF{ex:arsenijevic:Todos-c}. Consider \REF{ex:arsenijevic:Šahisti1}, which does not nearly exhaust the possible combinations.

\ea\label{ex:arsenijevic:Šahisti1}
\ea {\gll Šahisti su igrali jedan protiv drugog.\\
 chess\_player.\textsc{pl} \textsc{aux} played one against other\\
 \glt `The chess players played against each other.' / \\
 `The chess players who played -- played against other chess players.'\\}\label{ex:arsenijevic:Šahisti1-a}
\ex {\gll Šahisti su igrali svaki protiv svakog.\\
 chess\_player.\textsc{pl} \textsc{aux} played each against each\\
 \glt `The chess players played against each other.' (necessarily strongly reciprocal)\\}\label{ex:arsenijevic:Šahisti1-b}
\ex {\gll Šahisti su igrali svaki protiv drugog.\\
 chess\_player.\textsc{pl} \textsc{aux} played each against other\\
 \glt `The chess players played each against a different one.' \\}\label{ex:arsenijevic:Šahisti1-c}
\ex {\gll Šahisti su igrali jedan protiv svih.\\
 chess\_player.\textsc{pl} \textsc{aux} played one against all\\
 \glt `The chess players played so that one always played against all the others.'\\}\label{ex:arsenijevic:Šahisti1-d}
\ex {\gll Šahisti su igrali svaki protiv ponekog.\\
 chess\_player.\textsc{pl} \textsc{aux} played each against some.\textsc{distr}\\
 \glt `The chess players played so that each played against some (one or more).'\\}\label{ex:arsenijevic:Šahisti1-e}
\ex {\gll Šahisti su igrali jedan protiv dvojice.\\
 chess\_player.\textsc{pl} \textsc{aux} played one against two\\
 \glt `The chess players played so that one always played against two others.'\\}\label{ex:arsenijevic:Šahisti1-f}
\ex {\gll Šahisti su igrali dvoje protiv dvoje.\\
 chess\_player.\textsc{pl} \textsc{aux} played two against two\\
 \glt `The chess players played so that it was always two playing against two.'}\label{ex:arsenijevic:Šahisti1-g}
\z \z

\noindent The availability of so many alternatives, some of which can only have the reciprocal meaning, like \REF{ex:arsenijevic:Šahisti1-b}, should weaken the reciprocal interpretation of the \textsc{one$+$other} expression. How come antipresupposition does not deprive the \textit{one+ other} expression of its reciprocal interpretation, i.e. why does it nevertheless fi\-gure as the standard expression of reciprocity to the extent that in some languages other alternatives are not available? The answer is that the \textsc{one$+$other} expression is the only one among the alternatives which restricts its two components to refer within the same plurality -- the one denoted by the antecedent. For all other combinations, this is possible as an accidental scenario, but other scenarios are possible too, and can be explicitly expressed, as shown in \REF{ex:arsenijevic:Šahisti2}.

\ea\label{ex:arsenijevic:Šahisti2}
\ea {\gll \minsp{*} Šahisti su igrali jedan protiv drugog posetioca.\\
 {} chess\_player.\textsc{pl} \textsc{aux} played one against other visitor\\
 \glt Intended: `The chess players played such that a chess player played against a visitor.'\\}\label{ex:arsenijevic:Šahisti2-a}
\ex {\gll Šahisti su igrali svaki protiv svakog posetioca.\\
 chess\_player.\textsc{pl} \textsc{aux} played each against each visitor\\
 \glt `The chess players played such that each played against each visitor.' \\} \label{ex:arsenijevic:Šahisti2-b}
\ex {\gll Šahisti su igrali svaki protiv drugog posetioca.\\
 chess\_player.\textsc{pl} \textsc{aux} played each against other visitor\\
 \glt `The chess players played each against a different visitor.' \\} \label{ex:arsenijevic:Šahisti2-c}
\ex {\gll Šahisti su igrali jedan protiv svih posetilaca.\\
 chess\_players.\textsc{pl} \textsc{aux} played one against all visitor.\textsc{pl}\\
 \glt `The chess players played so that one always played against all the visitors.'\\} \label{ex:arsenijevic:Šahisti2-d}
\ex {\gll Šahisti su igrali svaki protiv ponekog posetioca.\\
 chess\_player.\textsc{pl} \textsc{aux} played each against some.\textsc{distr} visitor\\
 \glt `The chess players played so that each played against some visitors.'\\} \label{ex:arsenijevic:Šahisti2-e}
\ex {\gll Šahisti su igrali jedan protiv dvojice posetilaca.\\
 chess\_player.\textsc{pl} \textsc{aux} played one against two visitor.\textsc{pl}\\
 \glt `The chess players played so that one always played against two visitors.'\\} \label{ex:arsenijevic:Šahisti2-f}
\ex {\gll Šahisti su igrali dvoje protiv dvoje posetilaca.\\
 chess\_player.\textsc{pl} \textsc{aux} played two against two visitor.\textsc{pl}\\
 \glt `The chess players played against visitors two against two.'} \label{ex:arsenijevic:Šahisti2-g}
 \z \z

\noindent With predicates which are not inherently reciprocal, the meaning emerging with different restrictions is non-reciprocal too.

\ea\label{ex:arsenijevic:Lekari}
\ea {\gll Lekari su operisali jedan drugog \minsp{(*} pacijenta).\\
 doctor.\textsc{pl} \textsc{aux} operated one other {} patient\\
 \glt `The doctors operated on each other.' \\}\label{ex:arsenijevic:Lekari-a}
\ex {\gll Lekari su operisali svaki svakog pacijenta.\\
 doctor.\textsc{pl} \textsc{aux} operated every every patient\\
 \glt `The doctors operated so that each operated on each patient.'\\}\label{ex:arsenijevic:Lekari-b}
\ex {\gll Lekari su operisali svaki drugog pacijenta.\\
 doctor.\textsc{pl} \textsc{aux} operated each other patient\\
 \glt `The doctors each operated on a different patient.'}\label{ex:arsenijevic:Lekari-c}
 \z \z
 
\noindent The contrast between \REF{ex:arsenijevic:Lekari-a} and \REF{ex:arsenijevic:Lekari-c} confirms that while the dependent semantics of the component \textsc{other} prominent in both \citeposst{z14} and \citeposst{v10} account probably plays an important role in introducing a bias for reciprocity, it is the non-specific indefinite semantics of the component \textsc{one} that eliminates other readings. The only way for the latter to distribute over a plurality is if the distinctness interpretation of the \textsc{other}-component takes the \textsc{one}-component rather than an independent discourse referent as its other argument. And this requires that the two components of the \textsc{one$+$other} expression share a common restriction.

It is hence the requirement for a shared restriction that promotes the reciprocal interpretation: on a plural event interpretation, the two participant roles freely selecting referents from the same plurality are more likely to result in a reciprocal scenario than if they receive referents from two independent restrictions. A pragmatic strengthening of the former (a maximized plurality of events) matches the reciprocal interpretation, while an analogous strengthening of the latter may include reciprocal segments only if the two restrictions have a non-empty intersection (in the example in \REF{ex:arsenijevic:Lekari} that requires a context in which some doctors are also patients). Hence, the competition of the \textsc{one$+$other} expression with its quantified alternatives goes in both directions: it weakens (when they denote particular degrees of strength of reciprocity), but also strengthens its reciprocal interpretation (since only in the \textsc{one$+$other} expression the two components are bound to share their restriction). I get back to the issue of shared and different restrictions in \sectref{sec:arsenijevic:5}, regarding the syntax of the construction.

\largerpage
There is one more type of expression in SC that can shed additional light on the construction under discussion. In SC, not only functional items -- quantifiers, numbers and proforms -- are allowed to occur in the place of \textsc{one} and \textsc{other}, but also full nouns. Consider \REF{ex:arsenijevic:Klizači1}, where as long as the verb is reciprocal, nouns can be used to restrict the reference of each of the two participants in a plurality of events.

\ea\label{ex:arsenijevic:Klizači1}
\ea {\gll Klizači su se sudarali devojka sa dečakom, starac sa mladićem, profesionalac sa početnikom.\\
 skater.\textsc{pl} \textsc{aux} \textsc{refl} collided girl with boy old\_man with young\_man pro with beginner\\
 \glt `The skaters collided so that a girl would collide with a boy, an old man with a young man, a pro with a beginner.'}\label{ex:arsenijevic:Klizači1-a}
\ex {\gll Potkazivali su otac sina, sestra brata, učitelj učenika.\\
 snitched \textsc{aux} father son.\textsc{acc} sister brother.\textsc{acc} teacher student.\textsc{acc}\\
 \glt `Fathers snitched on sons, sisters on brothers, teachers on students.'}\label{ex:arsenijevic:Klizači1-b}
\ex {\gll Figure su poređane manja ispred veće.\\
 figure.\textsc{pl} are arranged smaller in\_front bigger.\textsc{gen}\\
 \glt `The figures are arranged so that in each pair, the smaller one stands in front of the bigger one.'}\label{ex:arsenijevic:Klizači1-c}
\ex {\gll Izbeglice su prodavale frizer makaze, šnajder igle, kuvar šerpe.\\
 refugee.\textsc{pl} \textsc{aux} sold hair-dresser scissors tailor needles cook pots\\
 \glt `The refugees were selling stuff so that the hairdresser was selling his scissors, the tailor his needles, the cook his pots.'}\label{ex:arsenijevic:Klizači1-d}
\z \z 

\noindent Just like in the \textsc{one$+$other} expression, the girl and the boy, the old man and the young man, the pro and the beginner are all recruited from the same plurality of skaters in \REF{ex:arsenijevic:Klizači1-a}, denoted by the clausal subject. However, at least \REF{ex:arsenijevic:Klizači1-b}, \REF{ex:arsenijevic:Klizači1-d}, the two sentences with non-reciprocal verbal predicates, do not match any type or strength of reciprocity -- they are genuinely non-reciprocal. \REF{ex:arsenijevic:Klizači1-b} does not imply in any way that sons snitched on fathers, nor does \REF{ex:arsenijevic:Klizači1-d} that scissors were selling hairdressers. Also note that
even though it does not have an independent plural subject, \REF{ex:arsenijevic:Klizači1-b} can be shown to involve a plural \textit{pro} subject, and hence is the same type of construction as sentences with the \textsc{one$+$other} expression. The verb shows plural agreement although the nominative nouns are all singular, and the sentence cannot be paraphrased as \REF{ex:arsenijevic:Otac1-a}. Moreover, an explicit subject can be added to the sentence from which the participants in each of the plurality of events are recruited, as in \REF{ex:arsenijevic:Otac1-b}.

\ea\label{ex:arsenijevic:Otac1}
\ea {\gll Otac, sestra i učitelj su potkazivali sina, brata i učenika.\\
 father sister and teacher \textsc{aux} snitched.\textsc{pl} son.\textsc{acc} brother.\textsc{acc} and student.\textsc{acc}\\
 \glt `The father, the sister and the teacher snitched on the son, the brother and the student.'}\label{ex:arsenijevic:Otac1-a}
\ex {\gll Građani su potkazivali otac sina, sestra brata, učitelj učenika.\\
 citizen.\textsc{pl} \textsc{aux} snitched.\textsc{pl} father son.\textsc{acc} sister brother.\textsc{acc} teacher student.\textsc{acc}\\
 \glt `The citizens snitched on each other so that fathers snitched on sons, sisters on brothers, teachers on students.'}\label{ex:arsenijevic:Otac1-b}
\z \z

\noindent Observe further that both verbs \textit{potkazivati} `snitch on someone' and \textit{prodavati} `sell' in principle allow for a reciprocal relation, although the latter with serious pragmatic obstacles. Hence the non-reciprocity of the interpretation must be coming from the asymmetric relations expressed by the pairs of nouns used (\textit{father}-\textit{son}, \textit{sister}-\textit{brother}, etc.). Indeed, were one of the pairs symmetric, e.g. if \REF{ex:arsenijevic:Otac1-b} included the pair \textit{brother}-\textit{brother}, this would introduce a reciprocal sub-meaning to the aggregate semantics.

Sentences with a \textit{noun}+\textit{noun} expression in place of the expression \textsc{one$+$other} show the requirement that there cannot be only one pair of nouns. Two is a minimum, and three or more yield the best stylistic results. The reason is again plausibly in the economy: the meaning expressed by only one pair of nouns is equivalent to the meaning of a simpler sentence in which the two nouns are realized in the respective argument positions, as in \REF{ex:arsenijevic:Otac2} (but see the next section for an additional argument from the syntactic mechanism behind the shared restriction).

\ea\label{ex:arsenijevic:Otac2}
\gll Očevi su potkazivali sinove.\\
 father.\textsc{pl} \textsc{aux} snitched son.\textsc{pl}.\textsc{acc}\\
 \glt `Fathers snitched on sons.'
 \z

\noindent We may conclude that \citeposst{v10} and \citeposst{z14} semantic analyses can be simplified: the component that imposes reciprocity (in \citeauthor{z14}'s analysis the type-shift operation, in \citeauthor{v10}'s the maximality of cumulation of events) should be relegated to pragmatics.

\section{Syntax}\label{sec:arsenijevic:5}

\citet{z14} postulates an underlying structure for the \textsc{one$+$other} reciprocals where both its components are base-generated with a noun, as in \REF{ex:arsenijevic:Slikari1-a}, and then a series of syntactic operations derive the surface form with a noun only in the antecedent, i.e. the subject in \REF{ex:arsenijevic:Slikari1-b}.

\ea\label{ex:arsenijevic:Slikari1}
\ea {\gll Slikari su kritikovali \minsp{$[$} jedan slikar drugog slikara$]$ \\
 painter.\textsc{pl} \textsc{aux} criticized {} one painter other.\textsc{acc} painter.\textsc{acc} \\
\glt `The painters were criticizing one another.'} \label{ex:arsenijevic:Slikari1-a}
 
\ex {\gll Slikari su kritikovali jedan drugog.\\
 painter.\textsc{pl} \textsc{aux} criticized one other.\textsc{acc}\\
\glt `The painters were criticizing one another.'}\label{ex:arsenijevic:Slikari1-b}
\z \z

\noindent As her focus is on the semantics, the syntactic analysis is only partly laid out, and not each step in the derivation is fully specified. I propose the following syntactic analysis, much in the spirit of \citeauthor{z14}'s view, but with some further technical elaboration, crucially aimed to represent the sharing of the same restriction by the two components of the \textsc{one$+$other} expression as discussed in \sectref{sec:arsenijevic:4}%section 4
. 

Building on \citet{a06} in assuming that PartP is one of the universal projections of the nominal structure, which may result in a partitive marking in the presence of numerals and certain quantifiers, but is otherwise left without it, I argue that the surface subject starts out as a PartP of the higher nominal (corresponding to the \textsc{one}-component), as illustrated in Figure \ref{fig:VP}. 

\begin{figure}
    \caption{The base-generated VP `criticizes (one of the painters, other of the painters'}
    \label{fig:VP}
\begin{forest}
    [VP
        [DP
            [one]
            [PartP
                [part]
                [DP [painters$_i$]]
            ]
        ]
        [V$'$
            [criticize]
            [DP
                [other]
                [PartP
                    [part]
                    [DP [painters$_i$]]
                ]
            ]
        ]
    ]
\end{forest}
\end{figure}

As argued above, semantic well-formedness conditioned the two expressions from which the \textsc{one$+$other} expression derives to share the same restriction. Syntactically, this means that they need to have identical PartPs. The further %Further 
syntactic derivation goes as follows. The PartP from the \textsc{other}-expression elides, as in Figure \ref{fig:ellipsis}. 

\begin{figure}
    \caption{After ellipsis}
    \label{fig:ellipsis}
\begin{forest}
    [VP
        [DP
            [one]
            [PartP
                [part]
                [DP [\sout{painters}$_i$]]
            ]
        ]
        [V$'$
            [criticize]
            [DP
                [other]
                [PartP
                    [part]
                    [DP [\sout{painters}$_i$]]
                ]
            ]
        ]
    ]
\end{forest}
\end{figure}

The remaining overt PartP then moves higher up in the structure to its argument position or to the position of the topic. In the given example, this is the subject position, as in Figure \ref{fig:mvt}. 

\begin{figure}
    \caption{After the movement of the DP from the partitive complement to the specifier of IP}
    \label{fig:mvt}
\begin{forest}
    [IP
        [DP [painters$_i$]]
        [I$'$
            [criticize$_j$]
            [VP
                [DP
                    [one]
                    [PartP
                        [part]
                        [DP [\sout{painters}$_i$]]
                    ]
                ]
                [V$'$
                    [\sout{criticize}$_j$]
                    [DP
                        [other]
                        [PartP
                            [part]
                            [DP [\sout{painters}$_i$]]
                        ]
                    ]
                ]
            ]
        ]
    ]
\end{forest}

\end{figure}

The \textsc{one}-component, or in the alternative expression another quantifier or even a noun, remains in the lower position, which plausibly matches the position of a floating quantifier.

%\ea\label{ex:arsenijevic:Trees}
%\ea{The base generated VP ``criticizes (one of the painters, other of the painters)''} \label{ex:arsenijevic:Trees-a}
%\includegraphics[scale=0.5]{tree1}
%
%I also 'wrote' these trees in latex, they are in the file trees.tex, do these pictures can be replaced by them if
%
%\ex{After ellipsis} \label{ex:arsenijevic:Trees-b}
%\includegraphics[scale=0.5]{tree2}
%
%\ex{After the movement of the DP from the partitive complement to the specifier of IP} \label{ex:arsenijevic:Trees-c}
%\includegraphics[scale=0.5]{tree3}
%\z \z


It is the functional load of this configuration that makes it frequently generated, and gives it a somewhat idiomatic status. The degree of idiomatization is expected to vary across languages, and to correspond to a lower productivity of the configuration, i.e. with a fewer number of combinations of expressions that can occur in it. This further increases the pragmatic strengthening, making the reciprocal interpretation more likely or even the only option. Full idiomatization leads to reciprocity markers like \textit{one another} or \textit{each other} in English. 

Moreover, while it is not a condition for the derivation of the \textsc{one$+$other} expression, or for the reciprocal interpretation (see the discussion in \sectref{sec:arsenijevic:4}), the type of interpretation involving a plurality of events (whether for a single VP, or for a reduced conjunction of more than one) fits a greater number of naturally occurring contexts, and hence is more prominent than the single event reading.

The two relevant DPs, corresponding to the \textsc{one}- and \textsc{other}-components, can involve quantifiers or nouns, thus deriving examples as in \REF{ex:arsenijevic:Šahisti1} and in \REF{ex:arsenijevic:Klizači1}, and the partitive complement can also be pronominal, as in \REF{ex:arsenijevic:Klizači1-b}, where it is a \textit{pro}, or in \REF{Oni}, where it is an overt pronoun.

\ea\label{Oni}
\ea{\gll Oni su potkazivali jedan drugog.\\
 they \textsc{aux} snitched.\textsc{pl} one other.\textsc{acc} \\
\glt `They snitched on each other.'} \label{ex:arsenijevic:Oni-a}

 \ex{\gll Oni su potkazivali otac sina, sestra brata, učitelj učenika.\\
 they \textsc{aux} snitched.\textsc{pl} father son.\textsc{acc} sister brother.\textsc{acc} teacher student.\textsc{acc}\\
\glt `Fathers snitched on sons, sisters on brothers, teachers on students.'} \label{ex:arsenijevic:Oni-b}
\z \z

\noindent It is a conjunction of VPs in combination with ellipsis that leads to the structures as in \REF{ex:arsenijevic:Oni-b}, which explains their strict requirement of a plural VP. This operation is not reserved for expressions with lexical nouns, as illustrated in \REF{Šahisti3}.

\ea\label{Šahisti3} \gll Šahisti su igrali svaki protiv svakog, jedan protiv svih i jedan protiv dvojice.\\
 chess\_player.\textsc{pl} \textsc{aux} played each against each.\textsc{gen} one against all.\textsc{gen} and one against two.\textsc{gen}\\
\glt `The chess players played each against each, one against all and one against two.'
 \z

\noindent The interpretation is that there was a plurality of events in which a group of chess-players played, and which included all the combinations of one playing one and some or all combinations of one playing all, i.e. of one playing two other players.

The availability of the reciprocal interpretation is conditioned by the identity of the two partitive complements. The proposed analysis predicts that they may as well be different, but in that case a reciprocal interpretation does not obtain, as the participants in the plurality of events are not recruited from the same set. This is confirmed in SC:

\ea\label{Vojnici3} \gll Vojnici su nosili svaki \minsp{(} po) 6 bombi.\\
 soldier.\textsc{pl} \textsc{aux} carried each {} \textsc{distr} 6 bomb.\textsc{gen.pl}\\
 \glt `The soldiers were carrying 6 bombs each.'
\z

\noindent As discussed above, the combination of the nonspecific indefinite nature of the expression \textsc{one$+$other} and the type of relational meaning of the \textsc{other}-com\-po\-nent make it underivable by the structure in \REF{Vojnici3}.

While \citet{v10} and \citet{z14} explain the plural requirement for the antecedent of the \textsc{one$+$other} expression by the reciprocal interpretation, the present view cannot resort to this explanation because it derives reciprocity in pragmatics. The requirement is explained by the count partitive relation (\textit{one of} DP, \textit{other of} DP), which can only be established with a plurality (cf. *\textit{one of the boy}, *\textit{each of the girl},\textit{ }*\textit{two of the book}). The role of number is not exhausted by this restriction -- it also matters what number is marked on the \textsc{one}- and \textsc{other}-components for the interpretation of the expression and availability and type of available reciprocity. This issue is discussed in \sectref{sec:arsenijevic:6}.

\section{The role of number and grammatical aspect}\label{sec:arsenijevic:6}

In SC, each component of the \textsc{one$+$other} expression may bear plural as well as singular number. Out of the four possible combinations of number on its two components, only one is not acceptable: that where the \textsc{one}-component is plural and the \textsc{other}-component is singular, as illustrated in \REF{Susedi3}.

\ea\label{Susedi3} 
\ea{\gll Susedi pomažu \textit{jedni} \textit{drugima}.\\
 neighbor.\textsc{pl} help.\textsc{pl} one.\textsc{pl} other.\textsc{dat.\textsc{pl}} \\
 \glt `Groups of neighbors help each other.' /\\
 `The groups of neighbors that help -- help other groups of neighbors.'}\label{Susedi3-a}

\ex{\gll Susedi pomažu \textit{jedan} textit{drugom}.\\
 neighbor.\textsc{pl} help.\textsc{pl} one.\textsc{sg} other.\textsc{sg.dat}\\
 \glt `The neighbors help each other.' /\\
 `The neighbors are taking part in events of helping other neighbors.'}\label{Susedi3-b} 

\ex{\gll Susedi pomažu \textit{jedan} \textit{drugima}.\\
 neighbor.\textsc{pl} help.\textsc{pl} one.\textsc{sg} other.\textsc{dat.\textsc{pl}} \\
 \glt `The neighbors help each other so that one neighbor helps the others.'}\label{Susedi3-c} 

\ex{\gll \minsp{??/\#} Susedi pomažu \textit{jedni} \textit{drugom}.\\
 {} neighbor.\textsc{pl} help.\textsc{pl} one.\textsc{pl} other.\textsc{sg.dat}\\ 
 \glt Intended: `The neighbors help each other.'} \label{Susedi3-d} 
\z \z 

\noindent The source of degradation in \REF{Susedi3-d} probably has to do with the markedness relation between the plural and the singular: \textit{jedni} `one.\textsc{pl}' presupposes a division of the set into pluralities, but the singularity of \textit{drugom} `other.\textsc{sg.dat}' requires that each division is into one plurality and one singular member. Not only are such contexts unrealistic. There also is degradation coming from the fact that the first argument imposes a division into pluralities (the partitive complement is a set of pluralities), which makes it unlikely that the rest of the divisioned set is a singularity (as opposed to vice versa: when it is divided into singularities, the rest after taking one is likely to be a plurality itself). The availability of more appropriate alternatives such as \textit{svi jednom} `all to one' or one where the remaining single member receives a description (\textit{svi onome kome je potrebno} `all to the one who needs it') further contributes to the degradation.

This shows clearly that the number morphology on the \textsc{one$+$other} expression and its alternatives has a semantic effect. With the plural, as in \REF{Susedi3-a}, groups of neighbors are helping other groups of neighbors. With the singular, as in \REF{Susedi3-b}, only situations where a single neighbor is helping a single other neighbor are allowed. In the latter case, the numerosity of the antecedent does not have to be two -- greater plurals are possible too, as long as each (sub)event is assigned a pair of its members. 

However, when the antecedent is a plurality consisting of two units of coun\-ting, as in \REF{Jovan}, the two-membered plurality disallows the formation of nonidentical subpluralities, and hence in each situation, both the agent and the benefactor are restricted to being singulars. In examples of this type, as expected, plural marking on the \textsc{one$+$other} expression is impossible.

\ea\label{Jovan} \gll Jovan i Petar pomažu \minsp{\{} jedan drugom / \minsp{*} jedni drugima / {\minsp{*} jedan} drugima\}.\\
 J and P help.\textsc{pl} {} one.\textsc{sg} other.\textsc{sg.dat} {} {} one.\textsc{pl} other.\textsc{dat.\textsc{pl}} {} {{} one.\textsc{sg}} other.\textsc{dat.\textsc{pl}} \\
 \glt `Jovan and Petar help each other.'
 \z

\noindent The proposed analysis follows \citet{z14} in allowing the expression \textsc{one$+$other} to be used with single events, and goes even further in allowing it for single non-reciprocal events. \sectref{sec:arsenijevic:3} shows empirically that this is correct in SC. In the SC verbal domain, the role of the nominal grammatical number is played by the grammatical aspect.

SC verbs are marked for grammatical aspect -- a verb can be perfective, imperfective, or biaspectual. Imperfective interpretations come in two main flavors: one in which the reference time falls inside the described eventuality, and another in which the reference time falls inside a plurality of consecutive iterations of the eventuality, as in \REF{Jovan1} (e.g. \citealt{a06}, \citealt{j07}). Perfective verbs describe single instances of the described event within a reference time, as in \REF{Jovan1-b}, and may derive plural interpretations only by distributing over plural reference times (\citealt{a06}). Such distribution takes place in \REF{Jovan1-c}, where the source of the plurality of events is the plurality of reference times, and each reference time contains only one instantiation of the event.

\ea\label{Jovan1} 
\ea{ \gll Jovan je kupovao lampe.\\
 J \textsc{aux} bought.\textsc{ipfv} lamp.\textsc{pl}\\
\glt `Jovan was buying lamps.' \\
(one event of buying a plurality of lamps or a plurality of events of buying one or more lamps, both with a viewpoint within)}\label{Jovan1-a}

\ex{ \gll Jovan je kupio lampe.\\
 J \textsc{aux} bought.\textsc{pfv} lamp.\textsc{pl}\\
\glt `Jovan bought lamps.'\\
(one event of buying a plurality of lamps seen from after its temporal trace)}\label{Jovan1-b} 

\ex{ \gll Jovan je kupio lampe mnogo puta.\\
 J \textsc{aux} bought.\textsc{pfv} lamp.\textsc{pl} many times\\
\glt `Jovan bought lamps many times.' \\
(a series of reference times, each with one instance of the event of buying a plurality of lamps seen from after its temporal trace)}\label{Jovan1-c}
\z \z

\noindent Since reciprocity necessarily involves a plurality of (sub)events, for a single reference time, a telic VP with a perfective verb and a singular \textsc{one$+$other} expression may refer to a single non-reciprocal event, or to a single reciprocal event. Hence the sentence in \REF{Drugovi-a} can in principle mean that one of the friends forgave the other, or, more prominently due to antipresupposition (see \sectref{sec:arsenijevic:3}), that they forgave each other. The one in \REF{Drugovi-b} refers to an ongoing event of a friend forgiving the other friend or each other, viewed from within the temporal interval of the event, or to a plurality of such events, viewed from within the plurality. In both cases, the prominence of the reciprocal interpretations is even higher than in the perfective example.

\ea\label{Drugovi} 
\ea{ \gll Drugovi su \textit{oprostili} jedan drugome.\\
 friend.\textsc{pl} \textsc{aux} forgiven.\textsc{pfv.pl} one.\textsc{pl} other.\textsc{dat.\textsc{pl}} \\
 \glt `The friends forgave one to the other / each other.'}\label{Drugovi-a}

\ex{ \gll Drugovi su \textit{opraštali} jedan drugome.\\
 friend.\textsc{pl} \textsc{aux} forgiven.\textsc{ipfv.pl} one.\textsc{pl} other.\textsc{dat.\textsc{pl}} \\
 \glt `The neighbors were forgiving one to the other / each other.'}\label{Drugovi-b}
 \z \z

\noindent It now becomes clear why in questions of the first type in the questionnaire from \sectref{sec:arsenijevic:3}, the sentences with the \textsc{one$+$other} expression all had an all-singular \textsc{one$+$other} expression and a perfective verb.

The role of aspect too becomes more obvious in examples with conjoined or numeral antecedents. The imperfective aspect of the verb enables example \REF{Umetnici3-a} to receive a partitioned reciprocity interpretation (a plurality of subevents, characterized by involving each a subset of the artists as the agent and a subset of the scientists as the theme or vice versa). With a perfective verb, as in \REF{Umetnici3-b}, this reading is difficult to get, because it only allows for one event of praising per reference time -- and the partitioned reciprocity is based on a plurality of events. The most prominent interpretation is that the entire group of artists praised the entire group of scientists and vice versa, in one praising event.

\ea\label{Umetnici3} 
\ea{ \gll Umetnici i naučnici su hvalili jedni druge.\\
 artist.\textsc{pl} and scientists \textsc{aux} praised.\textsc{ipfv.pl} one.\textsc{pl} other.\textsc{acc.\textsc{pl}} \\
 \glt `The artists and the scientists were praising each other.'}\label{Umetnici3-a}

\ex{ \gll Umetnici i naučnici su pohvalili jedni druge.\\
 artist.\textsc{pl} and scientists \textsc{aux} praised.\textsc{pfv.pl} one.\textsc{pl} other.\textsc{acc.\textsc{pl}} \\
 \glt `The artists and the scientists praised each other.'}\label{Umetnici3-b}
 \z \z

\noindent Similarly, the noun$+$noun variant of the expression is degraded with perfective verbs, and derives an unlikely interpretation, where in \REF{Klizači3} there is a single colliding event such that a girl collides with a boy, an old man with a young man, and a pro with a beginner, one snitching event where a father snitches on his son, a sister on her brother and a teacher on his student and one selling event where a hairdresser sells scissors, a tailor needles and a cook his pots.

\ea\label{Klizači3} 
\ea{ \gll \minsp{?} Klizači su se sudarili devojka sa dečakom, starac sa mladićem, profesionalac sa početnikom.\\
{} skater.\textsc{pl} \textsc{aux} \textsc{refl} collided.\textsc{pfv.pl} girl with boy old\_man with young\_man pro with beginner\\
 \glt `The skaters collided so that a girl would collide with a boy, an old man with a young man, a pro with a beginner.'}\label{Klizači3-a}

\ex{ \gll Potkazali su otac sina, sestra brata, učitelj učenika.\\
 snitched.\textsc{pfv.pl} \textsc{aux} father son.\textsc{acc} sister brother.\textsc{acc} teacher student.\textsc{acc}\\
 \glt `Fathers snitched on sons, sisters on brothers, teachers on students.'}\label{klizači3-b}

\ex{ \gll \minsp{?} Izbeglice su prodale frizer makaze, šnajder igle, kuvar šerpe.\\
 {} refugee.\textsc{pl} \textsc{aux} sold.\textsc{pfv.pl} hair-dresser scissor.\textsc{pl} tailor needle.\textsc{pl} cook pot.\textsc{pl}\\
 \glt `The refugees were selling stuff so that the hair-dresser was selling his scissors, the tailor his needles, the cook his pots.'}\label{Klizači3-c}
\z \z 

\noindent Aspect and number show interaction which is clearest when the antecedent is a numeral-noun expression or a conjunction of singular nominals. The \textit{one+ other} expression with both plural components in \REF{Vidovnjaci3-a} gets degraded with a perfective verb (unless there is a presupposed division of the psychics in two groups, e.g. that it is presupposed that two of the psychics are good guys and three are bad guys), but is well-formed with an imperfective in \REF{Vidovnjaci3-b} (the interpretation is that attacking events took place between different divisions of the five psychics). The reason is again that the imperfective verb introduces a plurality of events, and hence also a plurality of divisions of the antecedent, while for a small plural, whose members are well individuated in the context, the perfective verb implies only one division, and requires that it be disambiguated.

\ea\label{Vidovnjaci3} 
\ea{ \gll Pet vidovnjaka su napali \minsp{\{} jedan drugog / \minsp{??} jedni druge / \minsp{*} jedan druge\}.\\
 five psychic.\textsc{pl} \textsc{aux} attacked.\textsc{pfv.pl} {} one.\textsc{sg} other.\textsc{acc.sg} {} {} one.\textsc{pl} other.\textsc{acc.\textsc{pl}} {} {} one.\textsc{sg} other.\textsc{acc.\textsc{pl}} \\
 \glt `Five psychics attacked each other.'\\}\label{Vidovnjaci3-a}

\ex{ \gll Pet vidovnjaka su napadali \minsp{\{} jedan drugog / jedni druge / \minsp{*} jedan druge\}.\\
 five psychic.\textsc{pl} \textsc{aux} attacked.\textsc{ipfv.pl} {} one.\textsc{sg} other.\textsc{acc.sg} {} one.\textsc{pl} other.\textsc{acc.\textsc{pl}} {} {} one.\textsc{sg} other.\textsc{acc.\textsc{pl}} \\
 \glt `Five psychics were attacking each other.'}\label{Vidovnjaci3-b}
\z \z 

\noindent I leave the precise modeling of these interactions for future research.

\section{Conclusion}\label{sec:arsenijevic:concl}

The paper builds on the work of \citet{v10} and \citet{z14}, who offer compositional analyses for the type of reciprocity expressions combining near semantic equivalents of the English words \textit{one} and \textit{other} in German, Russian, Spanish and a number of other languages. Both their analyses crucially involve a pair of non-overlapping participants, both contained within a plurality denoted by the antecedent of the expression. \citeauthor{v10}'s analysis treats them as indefinites and additionally postulates a cumulation operation, yielding a plurality of events satisfying the derived restriction. \citeauthor{z14}'s analysis takes them to be specific, and specifies the meaning with a single event. Both accounts compositionally derive meanings which are not reciprocal, i.e. where reciprocity is merely a special case, and in order to restrict the meaning to this special case, both postulate additional operations -- maximal cumulation in \citeauthor{v10}, and type shift in \citeauthor{z14}. I argued that this additional operation (the maximal cumulation or the type shift operation) is not part of the narrow semantics of the expression. Rather, the narrow semantics of the respective expressions is approximately the one they derive without it, and the reciprocal interpretation is then reached through pragmatic strengthening induced by a set of competing alternatives. 

Empirical support for this view is provided from SC, where the expression can have a non-reciprocal, as well as a single event interpretation. It is further argued that the reciprocal interpretation emerges from the non-specific indefinite semantics of both components of the expression in combination with the relational meaning of the component \textsc{other}, which forces the two components to range over pairs within a single restriction. This is reflected in the proposed syntactic analysis, where these two components start out with identical partitive complements, with the one in the \textsc{other}-component getting elided, and the one in the \textsc{one}-component raising to the position of the respective argument (in all the examples in this paper -- the subject, although this is not necessarily the target position), thus yielding the surface order. The fact that the \textsc{one$+$other} expression is only interpretable when the two expressions share their partitive base, as opposed to the pragmatic competitors of this expression, which also allow different partitive complements, provides it with the status of the default marker of reciprocity, even compared with expressions that, when reciprocally interpreted, force stronger reciprocal meanings. Evidence from SC, including sentences with different restrictions for the two components 
of the construction, support this analysis. The view that the expression is not necessarily plural, but a plurality of events is pragmatically a more prominent interpretation, is further supported by the roles grammatical number and grammatical aspect play in its interpretation in SC.

\section*{Abbreviations}

\begin{multicols}{3}
\begin{tabbing}
MMM \= accusative\kill
\textsc{acc} \> accusative\\
\textsc{aux} \> auxiliary\\
\textsc{dat} \> dative\\
\textsc{distr} \> distributive\\
\textsc{gen} \> genitive\\
\textsc{inst} \> instrumental\\
\textsc{ipfv} \> imperfective\\
\textsc{nom} \> nominative\\
\textsc{pl} \> plural\\
\textsc{pfv} \> perfective\\
\textsc{refl} \> reflexive\\
\textsc{sg} \> singular
\end{tabbing}
\end{multicols}

\section*{Acknowledgments}
Getting to know Ilse Zimmermann -- her human warmth, her linguistic curiosity and devotion, and the strength of her will -- is the greatest thing that I got from the time I spent in Berlin. I am grateful to the editors of this volume for giving me a chance for another journey through one of her favorite landscapes. 

%Place your acknowledgements here and funding information here.

\printbibliography[heading=subbibliography,notkeyword=this]

\end{document}
