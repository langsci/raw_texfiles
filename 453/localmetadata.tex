\title{Syntax, semantics, and the lexicon}
\subtitle{Papers by and in honor of Ilse Zimmermann} 
\author{Łukasz Jędrzejowski and Uwe Junghanns and Kerstin Schwabe  and Carla Umbach} 

\BackBody{This book is dedicated to Ilse Zimmermann, who was a pioneer of Generative Grammar in Germany and made important contributions to the analysis of German and Slavic languages. It contains original articles by Zimmermann as well as newly written papers inspired by her work. Zimmermann's original articles cover a wide range of topics over a long period of research – the earliest dating from 1983 – and they make it clear that issues that are highly topical today have long been the subject of linguistic research. The newly written papers are closely related to Zimmermann's topics ranging from DP structure, verbal inflection and reciprocity to the modification of causative verbs, all from a Slavic perspective. They are rounded off by a contribution highlighting the leading role played by the Strukturelle Grammatik research group, of which Zimmermann was a member, in the development of linguistics in Germany.\bigskip\\
\begin{otherlanguage}{german}
\noindent
Dieses Buch ist Ilse Zimmermann gewidmet, die eine Pionierin der Generativen Grammatik in Deutschland war und wichtige Beiträge zur Analyse des Deutschen und der slawischen Sprachen geleistet hat. Es enthält sowohl Originalartikel von Zimmermann als auch neu verfasste Beiträge, die durch ihre Arbeit inspiriert wurden. Zimmermanns Originalartikel decken ein breites Spektrum von Themen in einem langen Forschungszeitraum ab – der früheste datiert aus dem Jahr 1983 – und sie machen deutlich, dass Fragen, die heute hochaktuell sind, schon lange Gegenstand der linguistischen Forschung sind. Die neu verfassten Kapitel stehen in engem Zusammenhang mit Zimmermanns Themen. Sie reichen von der DP-Struktur über verbale Flexion und Reziprozität bis hin zur Modifikation kausativer Verben, alles aus slawischer Perspektive. Abgerundet werden sie durch einen Beitrag, der die führende Rolle der Forschungsgruppe Strukturelle Grammatik, deren Mitglied Zimmermann war, für die Entwicklung der Linguistik in Deutschland hervorhebt.
\end{otherlanguage}} 

\renewcommand{\lsISBNdigital}{978-3-96110-491-8}
\renewcommand{\lsISBNhardcover}{978-3-98554-120-1}
\BookDOI{10.5281/zenodo.14093374}
% \typesetter{}
\proofreader{Amir Ghorbanpour,
Andreas Hölzl,
Carla Umbach,
Elliott Pearl,
Georgios Vardakis,
Jean Nitzke,
Jeroen van de Weijer,
Katja Politt,
Ljudmila Geist,
Ludger Paschen,
Patricia Cabredo Hofherr,
Rebecca Madlener}
\lsCoverTitleSizes{48pt}{16mm}% Font setting for the title page


\renewcommand{\lsSeries}{osl}
\renewcommand{\lsSeriesNumber}{9}
\renewcommand{\lsID}{453}
