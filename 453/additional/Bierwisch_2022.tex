\documentclass[output=paper,colorlinks,citecolor=brown]{langscibook}
\bibliography{localbibliography}

\author{Manfred Bierwisch\affiliation{Leibniz-ZAS Berlin}\orcid{}}
\SetupAffiliations{mark style=none}

\title[Strukturelle Grammatik: Wie es anfing und was daraus wurde]{Strukturelle Grammatik: Wie es anfing und was daraus wurde\\ \medskip {\LARGE Für Ilse Zimmermann}}
\abstract{This article outlines the scientific and institutional context in which llse Zimmermann conducted her research and became a distinguished member of the linguistic community. The article is not meant to be an academic biography -- it is a `workshop visit'. The reader will be guided through the scientific development of the Structural Grammar Research Group established at the linguistic department of the German Academy of Sciences (East Berlin) at the end of the 1950s, which Ilse Zimmermann joined in the mid-1960s.

The Structural Grammar Research Group (Arbeitsstelle Strukturelle Grammatik, ASG) was at its beginning strongly influenced by Roman Jakobson's structuralism, and later adopted Chomsky's generative grammar. One outstanding result of the early period was the insight that the basic position of the German verb is sentence-final. This raised questions about the difference between verbs and nouns, more precisely, nominalizations, which then became the focus of Ilse Zimmermann's work. A structure of lexical entities was suggested that takes into account the parallels between sentences and nominalizations including semantic features referring to basic cognitive properties while showing the asymmetry characteristics known from structuralism.

While the ASG's research attracted increasing attention in the linguistic field, the situation in the Academy became politically tight, eventually leading to the liquidation of the group in the beginning of the 1970s. It was re-established ten years later under the label of cognitive linguistics, with Ilse Zimmermann back on board. The focus of second the  group was on dimensional adjectives and included syntactic, semantic and also cognitive aspects. Open questions from the earlier period were revisited and considered from a more general point of view. 

After the Fall of the Wall the group was finally transformed into a Max Planck research group where its research agenda finally came to an end, or rather, was taken up in various other linguistic projects, from different perspectives.
%\keywords{here, come, your, 4 to 6 keywords or keyphrases}
}

% add all extra packages you need to load to this file

\usepackage{tabularx,multicol}
\usepackage{url}
\urlstyle{same}

\usepackage{listings}
\lstset{basicstyle=\ttfamily,tabsize=2,breaklines=true}

\usepackage{langsci-basic}
\usepackage{langsci-optional}
\usepackage{langsci-lgr}
\usepackage{langsci-osl}
% \usepackage{./langsci/styles/langsci-lgr}
% \usepackage{./langsci/styles/langsci-osl}
% \usepackage{langsci-gb4e}

\usepackage{tikz}
\usetikzlibrary{patterns,calc}
\pgfdeclarepatternformonly{south east lines}{\pgfqpoint{-0pt}{-0pt}}{\pgfqpoint{3pt}{3pt}}{\pgfqpoint{3pt}{3pt}}{
    \pgfsetlinewidth{0.6pt}
    \pgfpathmoveto{\pgfqpoint{0pt}{3pt}}
    \pgfpathlineto{\pgfqpoint{3pt}{0pt}}
    \pgfpathmoveto{\pgfqpoint{.2pt}{-.2pt}}
    \pgfpathlineto{\pgfqpoint{-.2pt}{.2pt}}
    \pgfpathmoveto{\pgfqpoint{3.2pt}{2.8pt}}
    \pgfpathlineto{\pgfqpoint{2.8pt}{3.2pt}}
    \pgfusepath{stroke}}
    
\usepackage{stmaryrd}
\usepackage{wasysym}
\usepackage{multirow}
\usepackage{caption}
\usepackage{subcaption}
\usepackage{mathrsfs}
\usepackage{qtree}

\usepackage{linguex}


%pminos do not split footnotes
% \interfootnotelinepenalty=10000 %Footnote in Laporte chapters has to be split SN


%\DeclareIndexNameFormat{default}{%
%\nameparts{#1}%
%\usebibmacro{index:name}%
%{\index[names]}%
%{\namepartfamily}%
%{\namepartgiveni}%
% {}% L1
% {}% L2
%{\namepartprefix}% generates spurious space L3
%{\namepartsuffix}% generates spurious space L4
%}

%  {\DeclareIndexNameFormat{default}{%
%     \usebibmacro{index:name}{\index[names]}{#1}{#3}{#5}{#7}}}

%\DeclareIndexNameFormat{default}{%
%  \usebibmacro{index:name}{\sindex[nom]}{#1}{#3}{#5}{#7}}

%\DeclareIndexNameFormat{default}{%
%  \usebibmacro{index:name}{\sindex[person]}{#1}{#3}{#5}{#7}}
%\DeclareIndexNameFormat{default}{%
%\nameparts{#1} \usebibmacro{index:name}{\sindex[person]]}{\namepartfamily}{‌​\namepartgiven}{\nam‌​epartprefix}{\namepa‌​rtsuffix}}

%\newcommand{\smiley}{:)}

%\renewbibmacro*{index:name}[5]{%
%\usebibmacro{index:entry}{#1}%
%{\iffieldundef{usera}{}{\thefield{usera}\actualoperator}\mkbibindexname{#2}{#3}{#4}{#5}}}

% \newcommand{\noop}[1]{}

%remove for final
%\overfullrule=1mm

\newcommand{\tobi}[2]}}
\renewcommand{\S}[1]{\tobi{#1}{\textsc{*}}}

% this volume references
% puts: [this volume]
% already defined: \citetv
%\newcommand{\citepv}[1]{(\citeauthor{#1} \citeyear*{#1} [this volume])}
\newcommand{\citealtv}[1]{\citeauthor{#1} \citeyear*{#1} [this volume]}

%parentheses around example number
\newcommand{\pref}[1]{(\ref{#1})}

% in-text examples

\newcommand{\lnex}[1]{\textit{#1}} %target lang word
\newcommand{\lnlit}[1]{(lit.: `#1')} %literal reading
\newcommand{\lnlat}[1]{(#1)} % latinization
\newcommand{\lntrans}[1]{`#1'} %translation
\newcommand{\lnexl}[2]%
{\lnex{#1}{} \lnlat{#2}} % ex with latinization
\newcommand{\lnexlat}[3]{\lnex{#1}{} \lnlat{#2}{} \lntrans{#3}} % ex with latinization and tranl.

%ch01
\newcommand{\co}[1]{\mbox{\textbf{#1}}}

%ch09

\newcommand{\cyrbulg}[1]{\begin{otherlanguage*}{bulgarian}#1\end{otherlanguage*}}


%ch10
\newcommand{\nlp}{{\small NLP}}
\newcommand{\mwe}{{\small MWE}}
\newcommand{\rae}{{\small RAE}}
\newcommand{\lvc}{{\small LVC}}
\newcommand{\pos}{{\small P}o{\small S}}
%\newcommand{\todo}[1]{ \textcolor{red}{#1} }

%\renewcommand{\labelenumi}{\theenumi}
%\ainamefmt{{vv}{ll}{, ff}{, jj}} % fullname

\newcommand{\biberror}[1]{{\color{red}#1}}

\newcommand{\osenovaitem}{--~}

\togglepaper[42]
% the chapter number will be provided by volume editors; for now keep this way

\begin{document}
\begin{otherlanguage}{german}
 \setlocalecaption{german}{chapter}{Kapitel} 
 \setlocalecaption{german}{table}{Tabelle}
\maketitle

% %%%%%%%%%%%%%%%%%%%%%%%%%%%%%%%%%%%%%%%%%%%%%%%%%%%%%%%%%%%%%%%
% %%%%%%%%%%%%%%%%%%%%%%%%%%%%%%%%%%%%%%%%%%%%%%%%%%%%%%%%%%%%%%%

\bigskip

\noindent Auf den folgenden Seiten will ich aus meiner Sicht und Erfahrung den inhaltlichen und institutionellen Rahmen skizzieren, in dem Ilse Zimmermann die Entwicklung machen und mitgestalten konnte, deren Ergebnisse sie zu einem Mitglied der \textit{Linguistic Community} gemacht haben, in deren Befunde Ilses Arbeiten eingegangen sind. Das kann und soll nicht auf eine wissenschaftliche Biographie hinauslaufen, die aber ohne das, wovon diese Rahmenerzählung handelt, ganz gewiss nicht möglich wäre. Es geht eher um eine Werkstattbegehung.

In gewissem Sinn begann es mit Roman Jakobson. Denn dass es in den 1950er Jahren in der Deutschen Akademie der Wissenschaften zu Berlin eine Arbeitsstelle für Grammatik gab, in der es um die neuere Linguistik gehen sollte, das hatte Wolfgang Steinitz bewirkt, der persönlich und fachlich mit Jakobson befreundet war -- noch aus der sogenannten Weimarer Zeit und dann aus der Emigration in Schweden. Als Steinitz nach dem Ende der Katastrophe nach Berlin zurückkehrte und eine einflussreiche Position an der Akademie erlangte, da war Jakobson, der inzwischen in den USA lebte, bei Reisen durch Europa nicht nur wiederholt sein Gast, sondern mit ihm auch ein Stück leibhaftiger Strukturalismus, den es in Deutschland kaum gab. Wolfgang Steinitz hatte im Institut für deutsche Sprache und Literatur an der Akademie die Einrichtung einer Abteilung für Gegenwartssprache erreicht, in der auch eine moderne Grammatik entstehen sollte. Die Arbeiten von Jakobson zur Kasustheorie und zur Sprachstruktur überhaupt wurden für die fünf Absolventen, die die Grammatik-Arbeitsstelle bilden sollten und von denen ich einer war, zu einer Art Grundorientierung. Die hatte in der (ost)deutschen akademischen Landschaft zunächst keinen rechten Platz, und so musste sich Steinitz mit meinem Doktorvater in einem Kompromiss über meine Arbeit zum deutschen Verbalsystem einigen. Ohne diesen Spielraum wäre der Plan einer Grammatik mit aktuellem Anspruch schon im Ansatz gescheitert.

\section{}\label{sec:1}

Die Jakobsonsche Grundorientierung wurde dann sehr bald erweitert um die Arbeiten von Louis Hjelmslev (\citeyear{Hjelmslev1943,Hjelmslev1961}) und dem an ihm orientierten Gunnar Bech, dessen \textit{Studien über das deutsche Verbum infinitum} (\citeyear{Bech1955}/\citeyear{Bech1957}) ein exemplarischer Vorstoß zur strukturellen Analyse der deutschen Syntax waren. Bechs Theorie der infiniten Verbkomplexe und der Rektion, der sie unterliegen, war de facto ein Vorgriff auf Strukturvorstellungen, die erst mehr als 25 Jahre später ihren Platz in der generativen Syntaxtheorie im Rahmen von Chomskys \textit{Lectures on Government and Binding} \citep{Chomsky1981} finden sollten. Bech erläutert an Beispielen wie \REF{ex:1}, dass die Glieder einer Verbkette nicht einfach nebeneinander stehen, sondern verknüpft sind durch eine Operation, die er \glqq Multiplikation\grqq{} nennt, wobei der Multiplikand in der Regel dem Multiplikator vorausgeht. So ist in diesem Beispiel das Infinitum \textit{angerufen} mit dem Verb \textit{haben} multipliziert zu dem Produkt \textit{angerufen haben}, dann das in diesem Produkt regierende Verb \textit{haben} mit dem Infinitum \textit{behaupten} zu dem Produkt \textit{angerufen zu haben behaupten} usw., wobei jeweils von Bech systematisch charakterisierte Rektionsbedingungen wirksam werden.

\ea
\ldots weil er sie sonst angerufen zu haben behaupten können würde
\label{ex:1}
\z

\noindent Dieses an arithmetischer Strukturbildung orientierte Verständnis der Syntax des Deutschen stellte eine ganz neue Perspektive von großer Suggestivität und Aufschlusskraft für die Behandlung grammatischer Probleme dar. Wir haben sie nicht nur fasziniert zur Kenntnis genommen, sondern sie uns als Blickweise zu eigen gemacht und die damit verbundenen Kategorien in unser Begriffsrepertoire übernommen, auch wenn sie zunächst syntaktische Strukturen eher ordneten als erklärten. Denn bei aller Strenge und Eleganz war Bechs Analyse strikt auf die Infinit-Komplexe des Deutschen und ihre morphologisch-syntaktischen Eigenschaften bezogen und erfasste nur indirekt die formale Strukturerzeugung als generelles Prinzip der Sprache.

Die fundamentale Rolle genau dieses Prinzips hatte ungefähr zur gleichen Zeit auf ganz anderem Weg Noam Chomsky in einem tausendseitigen Manuskript zum Thema gemacht. Ein Extrakt von 116 Seiten mit dem Titel \textit{Syntactic Structures} \citep{Chomsky1957} aus dem zunächst unpublizierbaren Werk \textit{The Logical Structure of Linguistic Theory} stellte sich als entscheidender Anstoß für eine neue Etappe der Linguistik heraus, an der die von Steinitz initiierte \glqq Arbeitsstelle Strukturelle Grammatik\grqq, kurz ASG, sich entschlossen und zum deutlich bekundeten Misstrauen der etablierten Germanistik beteiligte. Chomskys Grammatiktheorie sah zwei Komponenten der syntaktischen Strukturbildung vor: die hie\-rarchische Gliederung mit ihren selektiven und morphologischen Abhängigkeiten wurde durch die Phrasenstruktur-Komponente bestimmt, davon ausgehende Modifikationen der linearen Ordnung durch die Transformations-Komponente. Diese Konzeption erlaubte einen neuartigen Zugang zur deutschen Satzstruktur, was besonders faszinierend an den Eigentümlichkeiten der Syntax des Verbs deutlich wurde. Die besteht ja insbesondere darin, dass das Verb auf eigenartige Weise an ganz verschiedenen Stellen im Satz auftaucht, mal als Einheit, mal verteilt, wie in den Beispielen in \REF{ex:2} und \REF{ex:3} zu sehen ist, wobei in \REF{ex:3} nur das Präteritum durch das Perfekt ersetzt ist, und (a) und (b) sich jeweils durch die Hauptsatzkonjunktion \textit{denn} statt der Nebensatzkonjunktion \textit{weil} unterscheiden, während (c) ohne Konjunktion ein Fragesatz sein kann:

\ea
\ea \ldots denn der Vortrag \textit{fing} etwas verspätet \textit{an} \label{ex:2a}
\ex \ldots weil der Vortrag etwas verspätet \textit{anfing} \label{ex:2b}
\ex \ldots \textit{fing} der Vortrag etwas verspätet \textit{an} \label{ex:2c}
\z
\label{ex:2}
\z

\ea 
\ea \ldots denn der Vortrag \textit{hat} etwas verspätet \textit{angefangen} \label{ex:3a}
\ex \ldots weil der Vortrag etwas verspätet \textit{angefangen hat} \label{ex:3b}
\ex \ldots \textit{hat} der Vortrag etwas verspätet \textit{angefangen} \label{ex:3c}
\z
\label{ex:3}
\z

\noindent Im Rahmen von Chomskys Theorie sind die verschiedenen Verbstellungen dann eine Sache der Transformationskomponente, der in den drei Versionen in \REF{ex:2} und \REF{ex:3} jeweils die gleiche Phrasenstruktur zugrunde liegt. Was in Bechs ingenieuser Analyse wie eine Unordentlichkeit, ein Störfaktor wirkt und keinen rechten Platz findet, erweist sich nun als zentraler Ordnungsfaktor: die Stellung des Finitums. Zwei entscheidende Einsichten machen diese Analyse möglich: Erstens, die Grundform der deutschen Satzstruktur ist die Nebensatzstellung (b) mit dem Verb am Ende, eine Idee, die fast gleichzeitig auch Jean Fourquet (\citeyear{Fourquet1959}) und Emmon Bach (\citeyear{Bach1962}) vertraten. Zweitens, die Stellungsvarianten betreffen nicht das ganze Verb, sondern nur den Teil mit den Finit-Merkmalen, also in \REF{ex:2} nur \textit{fing}, nicht \textit{anfing}, in \REF{ex:3} nur \textit{hat}, nicht \textit{angefangen hat}. Für diesen finiten Teil gibt es drei syntaktisch bedingte Positionen: Erst-, Zweit- und Endstellung, ganz unabhängig von der Verbbedeutung, denn weder \textit{an} noch \textit{fing} geben in \REF{ex:2} Teile der Bedeutung von \textit{anfangen} wieder. Die Nebensatzstellung als Grundform und die Verbstellungsregel, die den Satztyp festlegt, als Kernstück der Satzstruktur des Deutschen waren ein entscheidender Befund mit beträchtlichen Konsequenzen.

So macht z.\,B. die Verbstellung klar, dass die syntaktische Struktur autonom ist und keine einfache Parallele in der Bedeutungsstruktur hat. Das gilt bis in die Wortstruktur hinein, wie das ganz unterschiedliche Verhalten von Präfixen wie dem betonten, trennbaren \textit{an} und dem unbetont-untrennbaren \textit{be} in den praktisch bedeutungsgleichen Wörtern \textit{anfangen} und \textit{beginnen} in \REF{ex:4a} und \REF{ex:4b} zeigt:

\ea
\ea Der Vortrag \textit{fängt} erst später \textit{an}. \label{ex:4a}
\ex Der Vortrag \textit{beginnt} erst später. \label{ex:4b}
\z
\label{ex:4}
\z

\noindent Das wiederum hat weitreichende Folgen für die Beziehung zwischen Lautstruktur und Bedeutung und damit dann auch für den Prozess des Sprachverstehens. Denn was es mit der Bedeutung von \textit{fängt} als Teil von \textit{anfängt} auf sich hat, das entscheidet sich erst am Satzende. Strukturen wie \REF{ex:5} zeigen am Kontrast von \textit{abbrechen} und \textit{ausbrechen}, dass dazwischen beliebig erweiterbare Strukturen liegen können:

\ea \textit{brachen} die Streitigkeiten erst nach der lange angekündigten Intervention der Veranstalter des Wettstreits \textit{ab}/\textit{aus}
\label{ex:5}
\z

\noindent Dabei gilt für Verbstellung in \REF{ex:5} genau wie in \REF{ex:2} und \REF{ex:3} -- wenn man die Bedingungen richtig formuliert -- die gleiche einfache Verb-Regel, da sie nur das Finitum betrifft, dessen Position zunächst immer das Satzende ist. Das war das Aha-Erlebnis einer langen Diskussion, die ich mit Karl Erich Heidolph hatte und die die Essenz der Studie \textit{Grammatik des deutschen Verbs} \citep{Bierwisch1963} ausmacht, mit der die ASG eine Adresse wurde. In diesen Rahmen fügen sich übrigens auch die Verbketten der Bech'schen Analyse ein, da die gerade keine finiten Formen enthalten. Erste Erkundungen, was in diesem Rahmen für Verben und Adjektive gemeinsam gilt (oder nicht), waren denn bald in Motschs \textit{Syntax des deutschen Adjektivs} (\citeyear{Motsch1964}) zu finden.

\section{}\label{sec:2}

Bis dahin hatte sich die ASG mit administrativer Rückendeckung durch Steinitz im Wesentlichen selbst organisiert. Eine reguläre Leitung erhielt sie 1962 mit Alexander Isačenko, den Steinitz für Berlin interessieren konnte und der mit seiner russischen und allgemein-linguistischen Kompetenz der ASG neue Interessenten gewann. Dazu gehörte Ilse Zimmermann, die gerne, wenn auch nicht ganz ohne Reibung, von russischer auf deutsche und allgemeine Syntax umsattelte. Nach einem Einstieg über die Parallelitäten, die die Adjektivsyntax mit dem Verb teilt \citep{Zimmermann1967}, war ihre Energie lange auf die Parallelität und Differenz von Verb und Nomen, genauer auf die nominalisierten Verben gerichtet, die sich schon in einfachen Fällen wie \REF{ex:6} zeigt:

\ea
\ea Peter hat das Bild beschrieben
\ex Peters Beschreibung des Bildes
\z
\label{ex:6}
\z

\noindent Die analogen Beziehungen zwischen den Referenten von \textit{Peter}, \textit{das Bild} und dem Stamm von \textit{beschreiben} sind offensichtlich, sie werden aber nicht nur auf scheinbar einfache Weise verschieden realisiert -- Subjekt und direktes Objekt des Verbs werden zu Genitiv-Ergänzungen der Nominalisierung --, sie unterliegen auch anderen Konstruktionsbedingungen, wie man an den Unterschieden in \REF{ex:7} und \REF{ex:8} sieht (wo der Genitiv \textit{Peters} durch die gleichbedeutende, aber hier akzeptablere \textit{von}-Phrase ersetzt ist): die Umstellungsmöglichkeiten der Verbergänzungen gelten nicht für die Ergänzungen des Nomens. Während in \REF{ex:7} zwar der Fokus wechselt, die Beziehungen ansonsten aber gleich bleiben, ändert sich in \REF{ex:8} die ganze Konstruktion: \REF{ex:8a} ist kaum noch grammatisch, und in \REF{ex:8b} wie in \REF{ex:8c} -- falls das noch als akzeptabel gilt -- wird das Bild zwar Peter als Besitzer oder als Dargestellter zugeordnet, aber nicht, wie in \REF{ex:7}, als von ihm beschrieben.

\ea
\ea das Bild hat Peter beschrieben \label{ex:7a}
\ex beschrieben hat Peter das Bild \label{ex:7b}
\ex das Bild beschrieben hat Peter \label{ex:7c}
\z
\label{ex:7}
\z

\ea
\ea des Bildes Beschreibung von Peter \label{ex:8a} 
\ex die Beschreibung des Bildes von Peter \label{ex:8b}
\ex des Bildes von Peter Beschreibung \label{ex:8c}
\z
\label{ex:8}
\z

\noindent Auch ohne auf die Regeln und Beschränkungen der unterschiedlichen Beziehungen in \REF{ex:6} bis \REF{ex:8} einzugehen, lässt sich feststellen, dass sich die Argumente des Verbs bei der Nominalisierung in Genitiv- oder Präpositionalattribute verwandeln und zudem in der Nominalgruppe ganz andere Anordnungsmöglichkeiten gelten als im Satz.

Den Charakter dieser unterschiedlichen Organisation zu verstehen, war in der Folge eine zentrale Fragestellung der Linguistik. Auf das dafür entscheidende Problem, die Entsprechung und die Differenz zwischen Laut und Bedeutung, war die inzwischen von Chomsky entwickelte Theorie der syntaktischen Tiefenstruktur eine faszinierende Antwort. Nach dieser Theorie ist die syntaktische Grundstruktur -- im Deutschen also im Wesentlichen die der Nebensatzform -- zum einen die Basis für die semantische Integration der Lexeme zur Satzbedeutung, zum andern Grundlage für die Transformationen, die die syntaktische Struktur der Laut- oder Schriftform ergeben.

Mit dieser Vorstellung, die programmatisch die \glqq Standard-Theorie\grqq{} hieß und von Chomsky in \textit{Aspects of the Theory of Syntax} (\citeyear{Chomsky1965}) klassisch formuliert worden war, wurde unter anderem die Struktur der lexikalischen Einheiten ein zentrales Thema, in dem alle Bereiche der Linguistik sich trafen. Die Frage nach ihrer Lautstruktur war durch Jakobsons Phonologie, in der Segmente aus phonetischen Merkmalen bestanden, die einem universellen Repertoire entstammen, im Grundsatz beantwortet, sofern man entscheidende Ergänzungen durch \citet{ChomskyHalle1968} in \textit{The Sound Pattern of English} hinzunimmt, die die kombinatorischen Prinzipen der Lautstruktur auf eine neue Grundlage stellten. Das Problem war nun, wie parallel oder kontrastierend dazu semantische Strukturen vorzustellen wären. Ein Zwischenschritt, der aber in Wahrheit eher ein zusätzliches Problem als eine Antworthilfe ergab, war die Analyse morphologischer Kategorien -- Tempus, Numerus, Kasus etc. -- als Komplexe morphologischer Merkmale, parallel zur Auflösung der phonologischen Segmente in Bündel phonetischer Merkmale (was wiederum bereits auf \citealt{Jakobson1932} zurückging). Damit war klar, dass das fundamentale Prinzip der asymmetrischen Markiertheit auch die anderen Aspekte der Sprachstruktur dominiert: So wie unmarkierte Segmente, z.\,B. die stimmlosen Verschlusslaute /p, t, k/, kombinatorisch weniger beschränkt sind als ihre markierten Partner /b, d, g/, so sind Kategorien wie Singular oder Präsens unmarkiert gegenüber ihrem Gegenteil Plural und Präteritum. Und wie die phonetischen Merkmale, so sind auch grammatische Merkmale wie [plural], [präteritum] etc. in einem generellen Inventar zu erfassen, das der Asymmetrie der Merkmalswerte unterliegt. Für die lautliche und die morpho-syntaktische Komponente lexikalischer Einheiten waren damit entsprechende Merkmalkomplexe anzunehmen. Es lag nahe, das auch auf die Bedeutungsseite auszudehnen.

Ein aufschlussreicher Vorstoß in diese Richtung -- angeregt übrigens u.\,a. durch eine faszinierende Studie von Gunnar Bech %\citeauthor{Bech1951} 
über die semantische Entwicklung der Modalverben \citep{Bech1951} %(\citeyear{Bech1951})
-- war meine Analyse der Dimensions-Adjektive in \textit{Some Semantic Universals in German Adjectivals}  \citep{Bierwisch1967}%(\citeyear{Bierwisch1967})
. Ausgehend von Paaren wie in \REF{ex:9} waren da zunächst zwei semantische Komponenten zu unterscheiden: zum einen wird die paarweise gemeinsame Dimension durch Merkmale wie [maximal] für \textit{lang/kurz}, [sekundär] für \textit{breit/schmal}, [vertikal] für \textit{hoch/niedrig} etc. markiert, zum andern der jeweilige Kontrast durch ein Merkmal für die Ausdehnung über oder unter der Norm:

\ea
\glll Dimension: Länge Breite Höhe Dicke Tiefe Größe Geschwindigkeit \\
{(a) [positiv]} \textit{lang} \textit{breit} \textit{hoch} \textit{dick} \textit{tief} \textit{groß} \textit{schnell} \\
{(b) [negativ]} \textit{kurz} \textit{schmal} \textit{niedrig} \textit{dünn} \textit{flach} \textit{klein} \textit{langsam} \\ 
\label{ex:9}
\z

\noindent Dabei ist das Merkmal [positiv]/[negativ] in seinen Kombinationsmöglichkeiten wieder eindeutig asymmetrisch: so wie die Markiertheit stimmhafter Verschlusslaute verhindert, dass sie im Auslaut vorkommen können (das \textit{b} in \textit{Lob} wird systematisch \textit{p} gesprochen), so können die [negativ]-Adjektive nicht normal mit Maßangaben verbunden werden: \textit{der Film ist 10 Minuten kurz} wird verstanden im Sinn von \textit{10 Minuten lang}, also \textit{kurz}. Andere Kombinationen sind noch mehr eingeschränkt: \textit{drei mal so lang} ist eine simple Multiplikation, \textit{drei mal so kurz} hat gar keine klare Bedeutung (\textit{ein Drittel so lang?} \textit{drei mal so weit unterm Durchschnitt?} Unklar!)

Mit Merkmalsanalysen dieser Art war zudem die spannende und bis heute offene Frage verbunden, ob die Semantik ebenso wie die Phonologie auf einem universellen, natürlich viel größeren (und sehr anders organisierten), jedenfalls aber in der kognitiven Grundausstattung des Menschen angelegten Merkmals\-repertoire beruht, aus dem verschiedene Sprachen unterschiedlich auswählen und komplexe Bedeutungen aufbauen. Wie die Raumorientierung sind z.\,B. auch die Farben oder die Verwandtschaftsbeziehungen in verschiedenen Sprachen zwar nicht gleich, aber doch mit gleichen Merkmalen charakterisiert. Die Frage, ob das für beliebige Bereiche gilt und damit auch für die Bedeutungsstruktur beliebiger lexikalischer Einheiten, ist umstritten. Das betrifft nicht zuletzt die Frage, wie sich Syntax und Semantik in komplexen Strukturen aufeinander beziehen, ob und wie also die Syntax von der Semantik abhängt oder die Semantik von der Syntax.

\section{}\label{sec:3}

Nach dem Aufbruch, den die Standard-Theorie mit sich brachte, war die Frage nach dem Verhältnis von Lautstruktur und Semantik allenthalben präsent und führte -- stimuliert u.\,a. durch Chomskys \textit{Remarks on Nominalization} \citep{Chomsky1970} -- zu recht unterschiedlichen Analysevorschlägen, die allerdings das erwähnte Problem eines universellen Inventars semantischer Merkmale vorerst nicht auf der Agenda hatten. Als zentrale Grundlage für die Beziehung zwischen Laut und Bedeutung in beliebig komplexen Ausdrücken war eher die Struktur lexikalischer Einheiten zu klären, die ja nicht nur die phonetische und semantische, sondern auch die grammatische Form der Lexeme festhalten muss, die ihr kombinatorisches Potential bestimmt. Diese grammatische Information war in zwei verschiedene Komponenten aufzuteilen, die zunächst als Kategorisierung und Subkategorisierung der Einheiten betrachtet worden waren, sich nun aber als zwei ganz verschiedene Aspekte herausstellten: nämlich außer der Kategorisierung einer Einheit durch syntaktische und morphologische Merkmale (Verb, Nomen etc. sowie Numerus, Kasus, usw.), die generell Bedingungen für grammatische Kombinationen sind, sodann das sogenannte \glqq Theta-Raster\grqq, das aus einer oder mehreren Theta-Rollen besteht, die festlegen, welche Ergänzungen mit welchen grammatischen Eigenschaften eine Einheit verlangt oder erlaubt. Dabei sind die Theta-Rollen einer lexikalischen Einheit die Schnittstelle zwischen Semantik und Syntax: sie identifizieren einerseits im Kontext die Träger der verlangten oder zulässigen Bedeutungsmomente, denen durch die Theta-Rollen entsprechende Beziehungen zugeordnet werden, die aber andererseits durch die grammatischen Merkmale markiert sind, die dabei gelten müssen. Tabelle \ref{tab:entry} ist ein stark vereinfachtes Beispiel dafür, wie wir uns die genannten Faktoren damals zurecht gelegt haben, später nachlesbar u.\,a. in \citet{Bierwisch1988,Bierwisch1990}:

%\ea
%\glllll {\phantom{cc}/ ähn--lich /} {\phantom{ccc}[$+$N,$+$V]} {\phantom{ccccci}$(\lambda y) \lambda x$} {\phantom{cc}$[x [ \cnst{similis}\, y ]]$} \\
%{} {} {\footnotesize \phantom{cccccci}[\textsc{dat}]} \\
%{\footnotesize Phonetische Form}  {\footnotesize \phantom{c}Kategorisierung} {\footnotesize \phantom{cc}Argument-Struktur} {\footnotesize \phantom{c}Semantische Form} \\
%{} {\small \phantom{cccccc}Grammatische} {\small Form} {} \\
%\phantom{cccccci}PF \phantom{ccccccccccccc}GF {} \phantom{cccccci}SF {} \\
%\label{ex:10}
%\z

\begin{table}[ht]
    \centering
    \begin{tabular}{ccccc}
       /ähn--lich/  &  [$+$N,$+$V] & ($\lambda y$) & $\lambda x$ & $[x [ \cnst{similis}\, y ]]$ \\
        &  & {\small [\textsc{dat}]} & & \\
       Phonetische Form & Kategorisierung & \multicolumn{2}{c}{Argument-Struktur} & Semantische Form \\
       & \multicolumn{3}{c}{Grammatische Form} & \\
       PF & \multicolumn{3}{c}{GF} & SF \\
    \end{tabular}
    \caption{Struktur lexikalischer Einheiten}
    \label{tab:entry}
\end{table}

Der Eintrag deutet unvollständig an, was das mentale Lexikon für das einfache, aber relationale Adjektiv \textit{ähnlich} enthalten muss: Die Theta-Rollen $\lambda x$ %\textlambda x 
für das Thema und $\lambda y$ %\textlambda y 
für die Dativ-markierte Vergleichsgröße sind durch die mit $\cnst{similis}$ abgekürzte semantische Relation verbunden; in einem Satz wie \textit{Hans ist ihr ähnlich} sind diese Positionen beide korrekt besetzt. Die Vergleichsgröße kann aber fehlen, etwa in \textit{das ist ein ähnlicher Fall}, diese Position ist daher als Möglichkeit eingeklammert, während die Position für das Thema unentbehrlich, aber ohne Kasusbedingung ist, weil sie mit verschiedenen Kasus realisiert werden kann, etwa in \textit{wegen ähnlicher Fälle} oder \textit{in ähnlichem Sinn}. Was der Eintrag in Tabelle \ref{tab:entry} illustrieren soll -- die Struktur lexikalischer Einheiten -- war nun insbesondere von Interesse in Bezug auf ihre Funktion bei den bereits erwähnten Nominalisierungen in \REF{ex:7}, \REF{ex:8} und nun besonders in \REF{ex:11} und \REF{ex:12}:

\ea 
\ea Die Kinder schlafen unruhig. 
\ex der unruhige Schlaf der Kinder
\z
\label{ex:11}
\z

\ea 
\ea Eva hat das Bild erneut restauriert.
\ex Evas erneute Restaurierung des Bildes
\z
\label{ex:12}
\z

\noindent Dabei ist zunächst festzuhalten, dass zwischen Nomen und Verb in Paaren wie \textit{Keim/keimen}, \textit{Blut/bluten}, \textit{Schlaf/schlafen}, \textit{Buch/buchen}, \textit{Spiel/spielen} und zahllosen anderen offenkundig sehr verschiedene, in aller Regel lexikalisch fixierte Relationen möglich sind. Für alle gilt dabei die Grundbedingung, dass Nomina als Kopf einer Nominalphrase (wie \textit{ein Buch}, \textit{das Blut}, \textit{der Schlaf}, \textit{die Spiele} etc.) auf Dinge, Substanzen oder Zustände sowie Prozesse und Ereignisse referieren können, die Referenz des Kopfs einer verbalen Konstruktion dagegen immer ein Ereignis, einen Zustand oder Prozess verlangt: Das Nomen in \textit{der Druck ist neu} z.\,B. kann sich auf einen Zustand beziehen, auf einen Vorgang, aber auch einen Gegenstand, der beim Drucken entsteht, das Verb in \textit{er druckt es} dagegen nur auf den Vorgang. Die Einsicht in diese elementare, aber keineswegs triviale Asymmetrie hat mich in \textit{Event-Nominalizations} \citep{Bierwisch1990} zu der Annahme geführt, dass in der Semantischen Form eines Verbs als Rahmen des Ganzen immer eine Position für eine Ereignisinstanz enthalten sein muss, die nicht durch ein Argument des Verbs (Subjekt oder Objekt) zu besetzen ist, aber durch Adverbiale z.\,B. qualifiziert oder begrenzt werden kann. Das mentale Lexikon muss dann z.\,B. für \textit{Buch/buchen} Einträge wie in \REF{ex:13} enthalten, wobei \textsc{book} den Merkmalskomplex andeutet, der $x$ als einen entsprechenden Informationsträger charakterisiert. Dieser Komplex wird im verbalen Lexem \REF{ex:13b} zur Anforderung an den durch $y\, \textsc{cause}\, [ z\, \textsc{in}\, x ]$ markierten Vorgang, denn dieser Vorgang verlangt, dass sein Akteur $y$ die Aufnahme des Objekts $z$ in das Buch $x$ bewirkt; und dieser Vorgang wird mit allen daran Beteiligten durch die Relation \textsc{inst} auf die Ereignis-Instanz $e$, das Referendum des Verbs, bezogen:

\ea
\ea /buch/ [$+$N,$-$V,$-$Mask\dots] $\lambda x [ \textsc{book}\, x ]$
\label{ex:13a}
\ex /buch/ [$-$N,$+$V,\dots]\phantom{rask\ldots]} $\lambda z \lambda y \lambda e \big\lbrack e\, \textsc{inst}\, [ y\, \textsc{cause}\, [ z\, \textsc{in}\, x ] : \textsc{book}\, x ]\big\rbrack$
\label{ex:13b}
\z
\label{ex:13}
\z 

\noindent Entsprechend verhalten sich die beiden Theta-Raster zueinander: Die re\-fe\-ren\-tiel\-le Position $\lambda x$ des Nomens in \REF{ex:13a}, auf die sich auch Modifikatoren wie in \textit{dicke Bücher} oder \textit{ein Buch über Plato} beziehen, die kommt in der Argumentstruktur des Verbs gar nicht vor (wohl aber in der Semantik die syntaktisch unzugängliche Argumentstelle x der Bedeutung von \textit{Buch}); die Verbalisierung bringt aber außer der Referenz auf das Ereignis $e$, die durch die Referenzposition $\lambda e$ vermittelt wird, noch den Akteur y und das Thema z durch die Subjekt- und Objektposition $\lambda y$ und $\lambda z$ ins Spiel.

Zwei Bemerkungen noch zu diesem komplexen Netz von Bezügen: Zum einen manifestiert sich die oben genannte Asymmetrie in der Referenz von Verb und Nomen in der letzten, der referentiellen Position im Theta-Raster: das Argument $\lambda x$ von \textsc{book} bezieht sich auf ein (konkretes oder abstraktes) Objekt, das Argument $\lambda e$ von \textsc{inst} aber verlangt eine andere Art von Entität, die durch das bestimmt ist, was $y$ veranlassen kann. Zum anderen unterliegen alle Positionen des Theta-Rasters generellen Bedingungen, die im Wesentlichen von der Kategorisierung abhängen und z.\,B. in \REF{ex:13b} automatisch die Kasusanforderung für Objekt und Subjekt festlegen oder auch die bedingte Weglassbarkeit des Objekts, die in Fällen wie \textit{sie hat schon gebucht} auftritt. Und damit sind wir bei den regulären Aspekten, um die es bei der Nominalisierungsanalyse primär geht.

Wenn \REF{ex:13b} das Lexem für den Verbstamm von \textit{buchen} ist, der u.\,a. die Basis für die zugehörigen Flexionsformen darstellt, dann entsteht durch Verknüpfung des Stamms mit dem Nominalisierungssuffix \textit{-ung} das Nomen in \REF{ex:14}, wobei das Suffix zunächst nichts anderes bewirkt, als die Ersetzung der verbalen Kategorisierung in \REF{ex:13b} durch die nominale in \REF{ex:14}, unter unveränderter Beibehaltung von PF und SF sowie im Prinzip auch der Argumentstruktur:

\ea /buch--ung/ [$+$N,$-$V,$+$Fem\dots] $\lambda z \lambda y \lambda e \big\lbrack e\, \textsc{inst}\, [y\, \textsc{cause}\, [z\, \textsc{in}\, x] : \textsc{book}\, x ]\big\rbrack$
\label{ex:14}
\z

\noindent Die Umkategorisierung hat allerdings zwei entscheidende generelle Konsequenzen: Erstens und vor allem wird die verbale Ereignisreferenz, auf die sich Tempus und Modus beziehen, ersetzt durch die nominale Referenz, die allerdings, wie oben gesagt, Ereignisse als Option einschließt und den Bedingungen nominaler Definitheit unterliegt. Damit wird aus \textit{Eva hat gebucht} zum Beispiel \textit{die Buchung Evas}. Zweitens werden damit zugleich die automatischen Kasusanforderungen des Verbs ersetzt durch die (ebenfalls automatischen) Kasusbedingungen des Nomens, das anstelle von Subjekt und Objekt Kasusergänzungen nur im Genitiv zulässt, die den -- hier nicht ausgeführten -- Bedingungen adnominaler Adjunkte unterliegen, wie man z.\,B. in \REF{ex:15} sieht:

\ea 
\ea Rainer bucht planmäßig eine Reise nach Rom.
\ex Rainers planmäßige Buchung einer Reise nach Rom
\z
\label{ex:15}
\z

\noindent Die hier bloß angedeuteten Effekte sind an das Suffix \textit{–ung} gebunden, aber nicht darauf beschränkt, sie gelten gleichermaßen für die regulären Aspekte der viel begrenzteren Ereignisnominalisierungen mit \textit{-t} in \textit{fahren/Fahrt,} mit \textit{–e} in \textit{ansagen/Ansage}, mit Null-Suffix und Vokalwechsel in \textit{springen/Sprung} und mit verschiedenen anderen Möglichkeiten. Zu den damit nicht erfassten Problemen gehört die für die Suffigierung typische Idiosynkrasie der Selektion der Formative, also: \textit{absprechen/Absprache}, \textit{ansprechen/Ansprache} aber \textit{besprechen/Besprechung}, \textit{versprechen/Versprechung} oder \textit{reiten/Ritt}, aber \textit{streiten/Streit} und zahllose andere Fälle. Ob und welche Bedingungen da identifizierbar sind, ist erst mal offen.

Mit dieser Art von Linguistik, ihren Problemen, den Methoden ihrer Klärung und den erzielten Einsichten, für die wir seit Beginn der 1960er Jahre zunehmendes Interesse auch außerhalb der Landesgrenzen fanden, waren wir in den 1970er Jahren im eigenen Hause in ernste Schwierigkeiten gekommen. Da ging es nicht mehr um philologische Vorurteile, sondern um politische Borniertheit und Missgunst. Die später als Stagnation bezeichnete lange letzte Phase im Ostblock brachte bedrohliche Restriktionen mit sich. Die \glqq Strukturelle Grammatik\grqq{} wurde zerschlagen, Steinitz war als Vizepräsident längst ins Abseits gedrängt und hatte einen inkompetenten Nachfolger; erst dessen Nachfolger Werner Kalweit, ein kluger Ökonom, respektierte wieder die Rolle internationaler Kontakte in der Wissenschaft und machte die Weiterführung ernsthafter Linguistik möglich -- unter anderem, indem er 1980 die Einrichtung der \glqq Arbeitsstelle Kognitive Linguistik\grqq{} ermöglichte.

\section{}\label{sec:4}

Ewald Lang und natürlich Ilse Zimmermann waren für die semantische und die syntaktische Seite unseres Gemeinschaftsunternehmens \citep{BierwischLang1987} dabei, und was wir uns zur Bekräftigung der Vorgeschichte mit der weiteren Arbeit zunächst vornahmen, waren die offenen Fragen rund um die Dimensionsadjektive in \REF{ex:9}. Ganz unerledigt war da der grammatische Rahmen, in den diese Adjektive gehören und die Spezifik, die sie dabei annehmen. Beides wird in Konstruktionen wie \REF{ex:16} und \REF{ex:17} sichtbar:

\ea
\ea Die Straße war breiter, als wir vermutet hatten. \label{ex:16a}
\ex Die Straße war zu schmal, um zu wenden. \label{ex:16b}
\z
\label{ex:16}
\z 

\ea
\ea Die Straße war so schmal, wie wir befürchtet hatten. \label{ex:17a}
\ex Die Straße war breit genug für den Wagen. \label{ex:17b}
\z
\label{ex:17}
\z

\noindent Die Komparativ- und Äquativ-Konstruktionen \textit{breiter als} und \textit{so breit wie} sowie die auf einen Sollwert bezogenen Kombinationen \textit{zu breit} und \textit{breit genug} mit den jeweiligen Gegenstücken \textit{schmaler als} etc. sind charakteristische Möglichkeiten für alle graduierbaren Adjektive; die Hinzufügung von Maßangaben dagegen ist begrenzt auf Dimensionsadjektive, sie ist asymmetrisch auf die Antonyme in (\ref{ex:9}a) und (\ref{ex:9}b) verteilt, und sie gilt auch für Konstruktionen wie die in \REF{ex:16}, also etwa \textit{3m breiter} oder \textit{2m zu schmal}, aber sie ist ausgeschlossen in \REF{ex:17}: \textit{3m so schmal} oder \textit{20m breit genug} sind defekte Kombinationen. In der Bilanz unserer Arbeit in dem Sammelband über Dimensionsadjektive \citep{BierwischLang1987} hat Ilse Zimmermann den syntaktischen Rahmen resümiert und Ewald Lang die Analyse der Merkmale für die Dimensionscharakterisierung auf den Status quo gebracht, einschließlich der Rätsel, die mit dem elementaren Paar \textit{groß/klein} verbunden sind, das ein bis drei Dimensionen haben kann in \textit{große Entfernung}, \textit{kleiner Platz}, \textit{großer Ball} oder auch gar keine, wie in \textit{eine große Überraschung} oder \textit{eine kleine Freude}. Dabei hat eine spezielle Studie von Karin Goede \citep{Goede1987} den interessanten Befund ergeben, dass der Erwerb nicht als schematische Kenntnisdifferenzierung vor sich geht, sondern stufenweise Systematisierung enthalten kann: Die zunächst globale Interpretation von \textit{groß} wird auf die einfachere maximale Dimension reduziert, ehe die komplexe Produktbildung erfasst wird.

Ganz neu anzugehen war die oben durch das Merkmal [positiv/negativ] angedeutete Antonymie der Paare in \REF{ex:9}. Sie unterliegt, wie vermerkt, dem allgemeinen Prinzip der Markiertheits-Asymmetrie, die hier aber sehr spezifische Aspekte aufweist. Das hängt zunächst zusammen mit dem Bezug auf das, was im jeweiligen Kontext der Normal- oder Erwartungswert für die Dimension ist: \textit{ein langer Weg} dehnt sich über das Übliche aus, \textit{ein kurzer Weg} bleibt darunter. Der Betrag über oder unter dem Normalwert ist sehr vom Zusammenhang abhängig, ebenso der Normalbereich N\textsubscript{C} selbst. Diese Unschärfe und der Bezug zum Normalbereich verschwindet sofort, wenn eine Maßangabe auftritt. Deren Anfang ist keineswegs N\textsubscript{C}, sondern 0. Diese Dispensierung von N\textsubscript{C} ist ganz normal bei \textit{lang}, aber nicht beim negativen Antonym: \textit{Die Leiter ist 2m lang} ist völlig in Ordnung, \textit{Die Leiter ist 2m kurz} dagegen irgendwie defekt. Allerdings verschwindet diese Asymmetrie im Komparativ wieder: \textit{Die Leiter ist 20cm kürzer} ist genau so normal wie \textit{Die Leiter ist 20cm länger}, und zwar offenbar, weil der Bezug auf N\textsubscript{C} ersetzt wird durch die (mitverstandene) Vergleichsgröße des Komparativs, etwa in \textit{Die Leiter ist 20cm kürzer (als der Tisch)}. Und das gilt dann auch für die Kombination mit \textit{zu}, bei der N\textsubscript{C} offensichtlich suspendiert wird zugunsten des (mit-zu-verstehenden) Sollwerts in Konstruktionen wie \textit{Die Leiter ist 20cm zu kurz (dafür)}. Im Äquativ, der generell keine Maßangaben zulässt, verschwindet die Suspendierung von N\textsubscript{C} wieder, aber interessanterweise nur bei den negativ polaren Adjektiven: In Konstruktionen wie \textit{Die Bank ist (genau) so kurz wie der Tisch} sind die verglichenen Objekte beide kurz, während in \textit{Die Bank ist (genau) so lang wie der Tisch} Bank und Tisch auch ziemlich kurz sein können, wenn die Bank so lang wie der Tisch ist. Anders gesagt: gleich lange Dinge können kurz sein, aber gleich kurze Dinge sind nicht lang. Diese ungleiche Ersetzung von N\textsubscript{C} setzt sich fort in der dazu dualen Konstruktion mit \textit{genug}, die wie der Äquativ auch keine Maßangaben erlaubt: \textit{Die Bank ist zwar kurz, aber doch lang genug dafür}.

Die relativ extensive Literatur über die Vergleichskonstruktionen, um die es hier geht, in der allerdings viele der angesprochenen Fakten gar nicht beachtet werden, geht von der Grundannahme aus, dass ein Adjektiv wie \textit{hoch} oder \textit{lang} an einem Objekt oder Sachverhalt zwei Dinge spezifiziert: eine Dimension, markiert durch Merkmale der Art, wie Ewald Lang sie aufgearbeitet hat, und einen damit gegebenen (meist unscharfen) Wert in dieser Dimension. In bestimmten Konstellationen kann dieser Wert aber durch Maßangaben spezifiziert werden. Merkwürdigerweise kommen fast alle Ansätze dabei nicht mit der Markiertheitsasymmetrie zurecht und ignorieren zudem genuine Fakten wie die unterschiedlichen Bedingungen der Maßphrasen.

Der entscheidende Punkt unserer Analyse war demgegenüber die Annahme, dass Dimensionsadjektive außer den Merkmalen, die die jeweilige Dimension bestimmen, hier abgekürzt durch die Variable $D$, nicht zwei Komponenten -- eine Dimension und einen so oder so bestimmten Wert darin --, sondern drei Größen miteinander verbinden: a) den Dimensionswert $D(x)$ eines Objektes $x$, der durch eine Operation $Q$ auf eine Skala abgebildet wird, b) einen Vergleichs- oder Grundbetrag $v$, für den es einen vom jeweiligen Kontext $C$ abhängigen Normbereich N\textsubscript{C} gibt, und c) einen Differenzbetrag $d$, durch den sich $QD(x)$ von $v$ unterscheidet. Für die positiven Adjektive wird $d$ mit $v$ additiv verknüpft, für die negativen wird $d$ von $v$ subtrahiert. Dimensionsadjektive haben damit folgendes Grundschema, das analog für alle Antonymen-Paare gilt, hier angegeben für das Beispiel \textit{lang/kurz} für die maximale Dimension von $x$:

\ea 
\ea /lang/ \,\, [$+$V,$+$N,\dots] \,\, $(\lambda d) \lambda x \,\,\,\, \big\lbrack [ Q\, [ \textsc{max}\, x ]] = v + d ] \big\rbrack$
\label{ex:18a}
\ex /kurz/ \,\, [$+$V,$+$N,\dots] \,\, $(\lambda d) \lambda x \,\,\,\, \big\lbrack [ Q\, [ \textsc{max}\, x ]] = v - d ] \big\rbrack$
\label{ex:18b}
\z
\label{ex:18}
\z

\noindent Die Variablen in diesen Repräsentationen verlangen einige Kommentare. Die durch $\lambda x$  gebundene Variable $x$ ist eine reguläre Argumentposition, durch die das Adjektiv Prädikat eines Objekts oder Ereignisses $x$ wird. Die nur fakultative Position $\lambda d$ bindet die Variable $d$, die einen sehr speziellen Status hat: Zum einen kann $d$ nur Werte auf einer angemessenen Gradskala annehmen, was bei $\lambda d$ als Selektionsbedingung anzugeben wäre; zum andern nimmt $d$, wenn $\lambda d$ syntaktisch unbesetzt bleibt, einen Default-Wert an, was am ehesten der Belegung durch einen kontextuell plausiblen Skalenwert entspricht. Einen noch spezielleren Status hat die Variable $v$, die zunächst gar keinen syntaktisch bestimmbaren Wert hat, sondern einfach den kontextbedingten Wert N\textsubscript{C} der Dimension $D(x)$ ausmacht. (Ob N\textsubscript{C} als obere bzw. untere Grenze des jeweiligen Normalbereichs oder einfach als unscharfer Normwert anzusehen ist, kann hier offen bleiben.) N\textsubscript{C} ist nun der Wert für $v$, solange nichts dagegen spricht, was aber dann der Fall ist, wenn $d$ durch eine Maßeinheit belegt ist, denn die kann nicht mit einem unscharfen Betrag verknüpft werden. Dann wird $v$ mit dem Wert 0 zum Skalenanfang, der den Grund-Ankerpunkt für Maße bildet. So ergibt sich die Interpretation für \textit{ein langer Weg} vs. \textit{ein 10m langer Weg}. Der Wechsel von N\textsubscript{C} zu 0 ist aber so bei \textit{kurz} nicht möglich, da Subtraktion von 0 keinen Wert auf der Skala ergibt. Was der Satz \textit{Der Weg ist 10m kurz} als Kompromiss zwischen zwei unverträglichen Optionen bedeutet, ist am ehesten zu umschreiben mit \textit{Der Weg ist 10m lang, und das ist kurz}. Dass N\textsubscript{C} als Anker für Maßangaben ausscheidet, kann aber auch anders bedingt sein, wie der Vergleich von \textit{5m kurz} mit der Sollwertkonstruktion \textit{5m zu kurz} zeigt. Der kontextbedingte Sollwert S\textsubscript{C} -- wie auch immer er zustande kommt -- ist jedenfalls ein ebenso deutlicher Anker für Maßangaben wie der Wert 0, und zwar für \textit{lang} ebenso wie für \textit{kurz}.

Weitere Versionen des Spiels mit dem Normbezug sind beim Komparativ und seinem Gegenstück, dem Äquativ, zu beobachten: Im Komparativ entfällt sowohl bei \textit{länger} wie bei \textit{kürzer} der Bezug zur Norm; der Grund: die Belegung für $v$ wird hier durch die Vergleichsphrase geliefert (auch wenn sie -- wie in \textit{Der Weg ist 10m kürzer} -- gar nicht realisiert wird, sondern implizit als Anker für mögliche Maßangaben wirkt). Das alles folgt aus der Analyse in \REF{ex:18}, falls das Komparativmorphem /-er/ (mit einigen Ergänzungen) daraus die Repräsentationen in \REF{ex:19} erzeugt:

\ea
\ea\gll 
{/läng -- er/} \,\, [$+$V,$+$N,\dots] \,\, ($\lambda d$) ($\lambda z$) {$\lambda x$} \,\, {$\big\lbrack [ Q\, [\textsc{max}\, x ]] = z + d ] \big\rbrack$} \\
{} {} {} {} {} \phantom{(}{\small als} {} {} {} {} \\ 
\ex\gll 
{/kürz -- er/} \,\, [$+$V,$+$N,\dots] \,\, ($\lambda d$) ($\lambda z$) {$\lambda x$} \,\, {$\big\lbrack [ Q\, [\textsc{max}\, x ]] = z - d ] \big\rbrack$} \\
{} {} {} {} {} \phantom{(}{\small als} {} {} {} {} \\ 
\z
\label{ex:19}
\z 

\noindent Die nötigen Ergänzungen betreffen, außer der Lautstruktur, vor allem drei Punkte: Erstens wird im Komparativ die Vergleichsgröße $v$ (und damit der Platz für N\textsubscript{C}) durch die Variable $z$ besetzt, die durch die fakultative Argumentposition $\lambda z$ syntaktisch zugänglich ist; das ist formal nicht trivial und hier einfach in die Struktur eingetragen. Zweitens ist die Angabe \glq als\grq{} für die Selektionsbedingungen der Vergleichsphrase eine Abkürzung, die hier nicht aufzulösen ist. Und drittens müssen in diesem Zusammenhang sehr komplizierte Alternativen berücksichtigt werden, um Fälle wie \textit{30cm kürzer als man dachte} und weitere komplexere Möglichkeiten zu erfassen. Es ist aber klar, wieso $d$ im Komparativ \textit{kürzer} ganz normal numerische Gradwerte haben kann, die für den Positiv \textit{kurz} problematisch sind: sie werden weder von 0 noch von N\textsubscript{C} subtrahiert. Anders im Äquativ. Da besagt \textit{Eva ist so groß wie Hans} nur die gleiche Größe beider, \textit{Eva ist so klein wie Hans} aber zusätzlich, dass beide klein sind. Dafür sind nun keine Maßangaben möglich, weil die Variable d durch den \textit{wie}-Vergleich belegt ist. Mit den schon bei \REF{ex:19} erwähnten Ergänzungen folgt das aus den Repräsentationen in \REF{ex:20}:

\ea
\ea\gll {/so lang/} \,\, [$+$V,$+$N,\dots] \,\, {($\lambda d$)} {$\lambda x$} \,\, {$\big\lbrack [ Q\, [ \textsc{max}\, x ] ] = [ v + d ] \big\rbrack$} \\
{} {} {} {} \phantom{(}{\small wie} {} {} {} \\ 
\ex\gll {/so kurz/} \,\, [$+$V,$+$N,\dots] \,\, {($\lambda d$)} {$\lambda x$} \,\, {$\big\lbrack [ Q\, [ \textsc{max}\, x ] ] = [ v – d ] \big\rbrack$} \\
{} {} {} {} \phantom{(}{\small wie} {} {} {} \\ 
\z
\label{ex:20}
\z 

\noindent Mit den nötigen formalen Ergänzungen folgen damit alle die erwähnten Asymmetrien aus dem Plus/Minus im Innenleben der Dimensionsadjektive. Zu den bis heute nicht gelösten Rätseln gehört der ambivalente Status von \textit{10m kurz}, der abweichend ist, aber doch regulär bestimmt ist und in Fällen wie \textit{Sie ist 17 Jahre jung} bewusst in Anspruch genommen wird. Und bei einer Grad-Größe ohne Maßangabe wie in \textit{so kurz wie ein Blitz} entsteht der Eindruck von Abweichung auch gar nicht. Noch rätselhafter ist der Status von \textit{doppelt so kurz}, von dem man nicht weiß, wie er sich zu \textit{halb so lang} verhält und ob er überhaupt eine Bedeutung hat.

Ein zentrales Rätsel, das (auch) mit dem verwickelten Netz der besprochenen Kombinationen rund um die Dimensionsadjektive verbunden ist, wird deutlich durch die Feststellung, dass mit den verschiedenartigen Konstruktionen sehr verschiedene, aber ziemlich eindeutige Intuitionen über ihre Akzeptabilität und ihre Bedeutung verbunden sind, obwohl da praktisch keine Lern- und kaum Gebrauchserfahrungen mit verbunden sein können. Sprachliches Wissen, also die Intuition über die subtilen Eigenschaften sprachlicher Ausdrücke, ist nicht einfach das Resultat von Erfahrung. Das gehört zum Horizont und den Grundlagen der Linguistik, die eben darum eine so faszinierende kognitive Wissenschaft ist.

\section{}\label{sec:5}

Die von Steinitz initiierte und von Kalweit wiederbelebte Arbeitsstelle Strukturelle Grammatik kam damit zu einem thematischen Einschnitt, der für mich ganz amtlich ein Jahr in Holland als Gast und Auswärtiges Mitglied am Max-Planck-Institut für Psycholinguistik in Nijmegen ergab. Das Ende der DDR, zu dem die in dieser Zeit alles verändernde Perestroika führte, überstand die Arbeitsgruppe mit Erleichterung und Gewinn -- nunmehr als Max-Planck-Gruppe des Instituts in Nijmegen, aber an der Humboldt-Universität in Berlin. Ewald Lang, der uns kurz vor dem Ende der Teilung nach Wuppertal verloren gegangen war, leitete nun das neu gegründete Zentrum für Allgemeine Sprachwissenschaft (ZAS) nebenan, aber Ilse Zimmermann war weiter dabei, pensioniert, doch sehr präsent. Als wir uns wieder den Problemen der Argumentstruktur zuwenden konnten, mahnte sie %(\citeyear{Zimmermann1991a}) 
die Klärung von Desideraten der Nominalisierung an \citep{Zimmermann1991a}, die in der oben skizzierten Form enthalten waren: Das anscheinend erfolgreich etablierte Schema, das aus dem Verb \textit{buchen} in \REF{ex:13b} das Nomen \textit{Buchung} in \REF{ex:14} erzeugt, verdeckt mindestens zwei unerledigte Punkte. Zum einen verlangt die Umwandlung eines Satzes wie \REF{ex:21a} in der nominalen Form \REF{ex:21b} außer dem besprochenen Genitiv für Subjekt und Objekt die (nicht besprochene) Suspendierung der Tempus-information, die beim Verb mit der Ereignis-Instanz $e$ verbunden ist, beim Nomen stattdessen aber zur Definitheit der Phrase führt:

\ea
\ea 
Peter bucht eine Reise. \label{ex:21a}
\ex Peters Buchung einer Reise \label{ex:21b}
\z
\label{ex:21}
\z

\noindent Die Wichtigkeit des zweiten unerledigten Punkts hat \citet{Zimmermann1991a} deutlich gemacht: der Genitiv \textit{Peters} in \REF{ex:21b} besetzt nämlich in Wahrheit zwei Argumentstellen des Verbs -- die Subjekt- und die Ereignisposition -- denn mit der Suspendierung des Tempus entsteht zugleich die erwähnte Definitheitsfestlegung für den Objektbezug der NP. Dass das Nomen \textit{Peters} in \REF{ex:21b} sowohl die Aktor- wie die Artikel-Position von \textit{Buchung} besetzt, wird sofort deutlich, wenn der Genitiv nicht vor dem Nomen steht: anstelle von \textit{Peters Buchung} ergibt sich dann äquivalent \textit{die Buchung Peters}, was klar definit ist gegenüber indefinitem \textit{eine Buchung Peters}. Die Definitheit, die mit dem adnominalen Genitiv wie mit dem Possessivpronomen verbunden ist, hatte \citet{Szabolcsi1983} allgemein der Possessor-Relation zugeordnet, sie betrifft aber adnominale Genitive generell, unabhängig von der Relation, die sie realisieren: in \textit{Peters Besuch} kann \textit{Peter} sowohl Agens wie Betroffener des Vorgangs sein, in \textit{Peters Buch} sowohl Autor wie Besitzer. Die Bedingungen und die formale Realisierung dieser Integration von referentieller Definitheit und sogenannter Possessor-Relation stellen ein offenbar zentrales Moment der Sprachstruktur dar: Possessivpronomina, die diese Integration morphologisch realisieren, sind ein typologisch verbreitetes Phänomen, das der Erklärung bedarf, nicht der bloßen Registrierung.

Eine ganz andere Variante der Nominalisierungsproblematik hatte Ilse Zimmermann %\citet{Zimmermann1988} 
im Verhältnis Adjektiv/Nomen anhand der einfachen Umkategorisierung von Adjektiven in Konstruktionen wie \textit{Das Sichere ist nicht sicher} untersucht \citep{Zimmermann1988}. Ein Adjektiv wie \textit{ähnlich} in Tabelle \ref{tab:entry} oder \textit{lang} bzw. \textit{kurz} in \REF{ex:18} wird dabei durch bloßen Wechsel der syntaktischen Kategorisierung (mit hier nicht zu verhandelnden Flexionseffekten) ganz direkt zum nominalen Kopf, und zwar kraft der Argumentposition, durch die das Adjektiv sonst als Modifikator oder Prädikativ fungiert: \textit{der Lange} meint die Person oder Sache $x$, die durch einen entsprechenden Wert für die maximale Dimension zu identifizieren ist. Im Lexikoneintrag \REF{ex:18a} ändert sich dabei nur die syntaktische Kategorisierung [$+$V,$+$N] zu [$-$V,$+$N] für Nomina (jeweils mit dem Genus des suspendierten Nomens).

Diese rein kategoriale Nominalisierung von Adjektiven, um die es im einschlägigen Artikel von Zimmermann geht, gilt vermöge der Partizip-Morphologie auch für Konstruktionen wie Schönbergs \textit{Ein Über\-lebender von Warschau} oder Brechts \textit{An die Nachgeborenen}. Daneben gibt es jedoch eine morpho\-logische Ad\-jek\-tiv-No\-mi\-na\-li\-sierung mit dem Suffix \textit{-e} plus Umlaut, die in \textit{Länge}, \textit{Höhe}, \textit{Tiefe}, \textit{Güte}, etc. auftritt. Hier ist jedoch lexikalische Idiosynkrasie im Spiel, die Fälle wie z.\,B. \textit{*Schmäle}, \textit{*Kleine}, \textit{*Dünne} als Nomina für \textit{schmal}, \textit{klein}, \textit{dünn} ausschließt und \textit{Fläche} von \textit{flach} separiert. Wo aber das Nomen -- wie bei \textit{lang} und \textit{kurz} -- regulär bildbar ist, hat das Suffix einen festgelegten Effekt: der Skalenwert $d$, den \textit{lang} dem Objekt $x$ zuweist, ist bei \textit{Länge} nun die Größe, auf die das so erzeugte Nomen referiert und die ihrerseits durch ein Adjektiv modifiziert werden kann z.\,B in \textit{die große Länge}. Der Träger dieser Eigenschaft erscheint nun als adnominaler Genitiv etwa in \textit{die Länge der Straße}. Formal werden also durch das Nominalisierungs\-suffix \textit{-e} z.\,B. die Adjektive in \REF{ex:18} in die Nomina in \REF{ex:22} überführt:

\ea
\ea /läng -- e / \,\, [$-$V,$+$N,$+$Fem,\dots] \,\, $(\lambda x) \lambda d \,\,\,\, \big\lbrack [ Q\, [ \textsc{max}\, x ]] = v + d ]\big\rbrack$ 
\ex /kürz -- e/ \,\,\, [$-$V,$+$N,$+$Fem,\dots] \,\, $(\lambda x) \lambda d \,\,\,\, \big\lbrack [ Q\, [ \textsc{max}\, x ]] = v - d ]\big\rbrack$
\z
\label{ex:22}
\z

\noindent Das Suffix \textit{-e} bewirkt den Umlaut des Stammvokals und die Kategorisierung als feminines Nomen, vor allem aber eine Änderung der Argumentstruktur, die Semantik dagegen bleibt unverändert. Nach der in \REF{ex:22} angenommenen Version wird das Objekt $x$, um dessen Dimensionen es geht, zum fakultativen (automatisch genitivischen) Komplement, während das, worauf referiert wird, nun mit der Grad-Differenz $d$ identifiziert wird. Für diese keineswegs offensichtliche Annahme gibt es mehrere plausible Gründe. Für \textit{die Länge des Weges} referiert $d$ entweder kontrastiv auf den überdurchschnittlichen Wegteil oder neutral auf den gesamten in Rede stehenden Weg; formal ist das nicht unterscheidbar. Kommt jedoch eine Maßphrase wie in \textit{die Länge von 10m} hinzu, dann scheidet wie beim Adjektiv die kontrastive Lesung aus. Für \textit{die Kürze des Weges} dagegen ist die Ambiguität ausgeschlossen (auch das wie beim Adjektiv): $d$ referiert hier klar auf den Abstand von der Norm. Kommt eine Maßangabe hinzu, wie in \textit{die Kürze von 10m}, entsteht die vom Adjektiv bekannte Ambivalenz: es geht um die Länge von 10m, die zugleich als kurz gilt.

Das alles ergibt sich ohne Zusatzannahmen aus dem, was das Suffix \textit{-e} zu \REF{ex:22} beisteuert. Dennoch ist fraglich, ob das die ganze Geschichte ist. Denn zum einen nominalisiert das Suffix \textit{-e} auch Adjektive wie \textit{treu}, \textit{gut}, \textit{still}, \textit{hart}, und andere, für die eine Skala mit Normalwert und Differenz dazu keinen Sinn macht, sodass Referenz auf anderes als die Skalengröße $d$ möglich sein muss -- etwa auf den Zustand, in den die relevante Eigenschaft versetzt, wie \textit{Treue} oder \textit{Stille} etc. Das sähe probeweise aus wie \REF{ex:23}, wo \textsc{treu} die Bedingungen andeutet, die $x$ erfüllen muss, um treu zu sein, und $y$ im oben angenommenen Sinn eine Instanz von deren Realisierung ist:

\ea /treu -- e/ \,\, [$-$V,$+$N,$+$Fem,\dots] \,\, $(\lambda x) \lambda y \,\,\,\, [ y\, \textsc{inst}\, [\textsc{treu}\, x ]]$
\label{ex:23}
\z 

\noindent Das könnte dann überdies auch den Nominalisierungseffekt von \textit{Länge} in der kontrastiven Version wiedergeben, die dem Sinn von \textsuperscript{?}\textit{Langheit} entspräche.

Zum anderen können auch bei kanonischen Dimensionen wie \textit{breit/schmal}, \textit{hoch/niedrig} oder \textit{groß/klein} andere Suffixe ins Spiel kommen wie in \textit{Schmalheit} als Antonym zu \textit{Breite}, \textit{Flachheit} zu \textit{Tiefe}, \textit{Niedrigkeit} zu \textit{Höhe}; aber obwohl damit genau die entsprechenden Dimensionswerte besetzt sind (und \textit{–heit/-keit} als Varianten zu \textit{–e} fungieren), sind etwa \textit{Kleinheit} oder \textit{Niedrigkeit} nur schwer auf diesen Wert einzuschränken, was bei \textit{Dummheit}, \textit{Wildheit}, \textit{Güte} oder \textit{Zartheit} ohnehin unmöglich ist. Kurz, mit dem Suffix \textit{-e} und seinen Varianten kommen die Idiosynkrasien morphologisch-lexikalischer Optionen ins Spiel, und damit ein Geflecht von Wechselbezügen, das Ilse Zimmermann wiederholt ins Auge gefasst hat, das aber natürlich keiner zu Ende bringen kann.

Die Arbeitsgruppe Strukturelle Grammatik dagegen war als akademischer Rahmen mit ihrer letzten Metamorphose bei der Max-Planck-Gesellschaft an ein gedeihliches Ende gelangt. Ihre Themen und Ziele leben natürlich in verschiedener Weise weiter, die Verbstellung in der deutschen Syntax, mit der sie angefangen hat, hat soeben neue Varianten in der Reihe \textit{Studia Grammatica} gefunden.

% %%%%%%%%%%%%%%%%%%%%%%%%%%%%%%%%%%%%%%%%%%%%%%%%%%%%%%%%%%%%%%%
% %%%%%%%%%%%%%%%%%%%%%%%%%%%%%%%%%%%%%%%%%%%%%%%%%%%%%%%%%%%%%%%

\sloppy
\printbibliography[heading=subbibliography,notkeyword=this]

\begin{figure}
%    \hbox{}\hfill
    \includegraphics[height=5.5cm]{figures/1.vortrag ohne.jpg}%
%    \hfill%
%    \includegraphics[height=5.5cm]{figures/2.Baum.jpg}%
%    \hfill\hbox{}
\end{figure}


\end{otherlanguage}
\end{document}