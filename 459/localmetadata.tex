\author{Jesse {Wichers Schreur}}
\title{Intensive language contact in the Caucasus}
\subtitle{The case of Tsova-Tush}
\ISBNdigital{978-3-96110-510-6}
\ISBNhardcover{978-3-98554-139-3}
\BookDOI{10.5281/zenodo.15275286}
% \typesetter{}
\proofreader{Amir Ghorbanpour,
David Carrasco Coquillat,
Giorgia Troiani,
Jeroen van de Weijer,
Kate Bellamy,
Laurentia Schreiber,
Ludger Paschen,
Mary Ann Walter,
Nicoletta Romeo,
Rodolfo Basile,
Silvie Strauß,
Steven Kaye,
Tom Bossuyt}
% \lsCoverTitleSizes{51.5pt}{17pt}

\BackBody{Tsova-Tush is an East Caucasian language spoken in one single village in Eastern Georgia by approximately 300 speakers. Since its early description, scholars have been intrigued by the high degree of linguistic influence from the Georgian language. This book has a threefold goal: (1) To contribute to the overall description of the Tsova-Tush language, by filling gaps in the previous literature in absence of a reference grammar. (2) To contrast Tsova-Tush constructions with functionally equivalent constructions in Chechen and Ingush, its closest relatives, and with Georgian, the language of wider communication which all Tsova-Tush speakers speak as a second language, in order to form hypotheses concerning which Tsova-Tush construction is inherited, and which has arisen under influence of Georgian. (3) To provide the most probable diachronic scenario of language contact, by looking at historical Tsova-Tush language data, as well as at its historical sociolinguistics.


This book provides a basic description of Tsova-Tush, in particular in the domain of spatial cases (which exhibit a two-slot system similar to Daghestanian languages), TAME categories (indentifying a Iamitive and a Past Subjunctive developing indirect evidential semantics), complex verbs, and subordination and clause-chaining (which in Tsova-Tush is finite).


In terms of language contact, this book concludes that (1) Tsova-Tush conforms to most established borrowing hierarchies and theories surrounding intensity of contact, except for the borrowing of a verbal inflection marker in a remarkably early stage of contact; (2) The Georgian influence that Tsova-Tush shows in sources from the 1850 suggest that a notable increase in bilingualism occured already at a point where there was little institutional or numeral dominance of surrounding the Georgian-language population. A change in ethnic self-identification can be the underlying factor for the early instances of contact-induced change.}

\renewcommand{\lsID}{459}
\renewcommand{\lsSeries}{loc} 
\renewcommand{\lsSeriesNumber}{5} 
