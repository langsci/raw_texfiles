\chapter{Introduction} \label{introduction}
\section{About this work}


The Caucasus is a mountainous region between the Black Sea and the Caspian Sea. It is famous for its linguistic diversity, with close to 65 languages of six different language families spoken in an area approximately the size of Germany. Many languages in the Caucasus are spoken by fewer than 5000 speakers in only one or a handful of villages. Many of these “small” languages, especially those in Daghestan, have been known to be relatively stable in the last centuries in terms of their numbers of speakers (\cite[523]{daniel2021franca}). Other “small” languages, especially those spoken in Georgia and Azerbaijan, have been characterised by heavy language contact or language shift (or sometimes both), such as Khinalug (\cite{rindpawlow2023khinaluginfluence}), Kryz (\cite[]{authier2010kryzcopy}), Udi (\cite[]{gippert08}) and Tsova-Tush (\cite{WS}).

Tsova-Tush is an East Caucasian language spoken in one single village in Eastern Georgia by approximately 300 speakers. Since its early description, scholars have been intrigued by the high degree of linguistic influence from the Georgian language (\cites[4]{schiefner56}).

\begin{quote}
	Der fremde Einfluss erstreckt sich namentlich bei dem Nomen sehr weit; bei der Wortbildung spielen fremde Ableitungssilben eine grosse Rolle und in der Syntax ist so manche Erscheinung eben nur durch den Einfluss des Georgischen zu erklären.\medskip\\\relax
	[The foreign influence extends far, especially on the Noun; derivational suffixes play a large part, and in the domain of syntax, some phenomena can only be explained by Georgian influence.] 
\end{quote}

The present work has a threefold goal: (1) to contribute to the overall description of the Tsova-Tush language, by filling gaps in the previous literature in absence of a reference grammar. (2) To contrast Tsova-Tush constructions with functionally equivalent constructions in Chechen and Ingush, its closest relatives, and with Georgian, the language of wider communication that all Tsova-Tush speakers speak as a second language, in order to form hypotheses concerning which Tsova-Tush constructions are inherited, and which have arisen under the influence of Georgian. (3) To provide the most probable diachronic scenario of language contact, by looking at historical Tsova-Tush language data, as well as at its historical sociolinguistics. 

This introductory chapter will provide a general overview of Tsova-Tush (\sectref{genremarks}), describing its nomenclature, its place in the East Caucasian language family and its degree of dialectal variation. A typological outline is given in \sectref{outline}, after which a short history of investigations into Tsova-Tush is provided. Subsequently a description of Tusheti (\sectref{tusheti}), a short history of the Tsova-Tush (\sectref{history}), a basic description of Tsova\hyp Tush cultural practices (\sectref{sociology}) and an outline of the sociolinguistics of Tsova\hyp Tush (\sectref{socioling}) will be presented. \sectref{introgeorgian} will describe the basics of Georgian grammar. In \sectref{study}, the present study will be outlined methodologically, presenting the relevant topics in the language contact literature, as well as the main data and methods used in this work.

In the subsequent chapters, various topics of Tsova-Tush grammar will be discussed. In each chapter, a detailed description of Tsova-Tush contructions pertaining to a given grammatical domain will be provided, as well as a comparison with functionally equivalent constructions in Chechen and Ingush (the closest relatives of Tsova-Tush), as well as in Georgian. At the end of each chapter, it will be established which aspects of Tsova-Tush grammar can be best explained by contact with Georgian. The chapters are centered around several classical domains of grammatical description: phonology (\chapref{phon}), nominal inflection and the noun phrase (\chapref{nounphrase}), verbal inflection (\chapref{inflection}), verbal derivation and valency (\chapref{derivation}), and clause combining (\chapref{subordination}). In \chapref{conclusion}, all instances of Georgian influence on Tsova-Tush are summarised and classified chronologically and in terms of the sociolinguistic scenario that most plausibly led to the borrowing of Georgian constructions and features into Tsova-Tush.

\section{The Tsova-Tush language and its speakers}

\subsection{General remarks} \label{genremarks}
Tsova-Tush is the native language of approximately 300 adults of the Tsova-Tush people, the vast majority of whom live in the village of Zemo Alvani in the eastern part of Georgia, just south of the Great Caucasus mountain range.

\subsubsection{Language name}

When speaking Tsova-Tush, speakers usually use the term \textit{bac} (\textit{bacav} `a Tsova-Tush man', \textit{bac-bi} Bats-\textsc{pl}, `the Tsova-Tush') when referring to their people, and call their language \textit{bacbur mot't'} `Bats.\textsc{adjz} language'. The most common occurance of the language name is in the phrase `in Tsova-Tush', which can be \textit{bejcbrat}, but more often is simply \textit{vej\u{g}eš} `in our (language)'. The root \textit{bac} has formed the basis for several derivations that are used as the language name in English (\textit{Batsbi}), French (\textit{Bats}), German (\textit{Batsisch}) and Russian (Бацбийский). However, when speaking Georgian or any other language, the Tsova-Tush refer to themselves or their language as `Tsova-Tush' (\foreignlanguage{georgian}{წოვათუშური ენა} \textit{c'ovatušuri ena} `the Tsova-Tush language'; \foreignlanguage{georgian}{წოვათუში}  \textit{c'ovatuši} `a Tsova-Tush person'), which has been the preferred designation by most contemporary scholars (e.g. \cite{holiskygagua,hauk}). A minority of Tsova-Tush, spearheaded by native-speaker linguists, prefer the term ``the Tush language'', which could potentially lead to confusion with the Tush dialect of the Georgian language.\footnote{According to Abram Shavkhelishvili (text E013 in the corpus (see \sectref{sources} for abbreviations)), the Ossetic name for the Tsova-Tush (or perhaps for the Tush in general) is \textit{guda}. The Tsezic names for the Tush are Bezhta \textit{iⁿq} (Majid Khalilov, p.c.), Hinuq \textit{eq}, Tsez \textit{a\textsuperscript{ʕ}q}, Asakh Tsez \textit{ha\textsuperscript{ʕ}q}. Chechen and Ingush use the designation \textit{bacoj} for the Tsova-Tush.}

\subsubsection{Tsova-Tush and the East Caucasian familiy}	
Tsova-Tush belongs to the Nakh branch of the East Caucasian family. The other two members of the Nakh branch, Chechen and Ingush, show a high degree of structural and phonological overlap but are more closely related to each other than to Tsova-Tush, see Figure \ref{nakhtree}. Chechen and Ingush taken together are sometimes referred to as Vainakh. Ingush is spoken by the Ingush people, numbering approximately 300,000, most of whom live in the Republic of Ingushetia within the Russian Federation. Chechen dialects are spoken by the vast majority of the approximately one million Chechens, who live in Chechnya (the Chechen Republic, also part of the Russian Federation). In Chechnya, Standard Chechen, based on the lowland dialects, is an official language besides Russian. Chechen is also spoken by the Akki Chechens in Daghestan and by the Kisti Chechens living in the Pankisi valley in Georgia (see Figure \ref{map-tusheti}), a half-hour drive from the Tsova-Tush village of Zemo Alvani.


\begin{figure}
% % % 	\includegraphics{figures/Proto-Nakh.png}
\begin{forest} for tree = {forked edge}
[Proto-Nakh
  [Tsova-Tush,tier=language]
  [,shape=coordinate
    [Ingush,tier=language]
    [Chechen dialects,tier=language]
  ]
]
\end{forest}
\caption{Nakh family tree}
\label{nakhtree}
\end{figure}

The Nakh branch is often assumed to be the first to split off from Proto-East-Caucasian, since the Nakh languages are typologically quite distinct from Daghestanian languages (hence the label ``Nakh-Daghestanian''), with their high number of vowel phonemes and the absence of a system of spatial cases as in Daghestanian languages (but in fact, see \sectref{spacase} for a Daghestanian-style system of spatial cases in Tsova-Tush). However, at this point it is not clear whether the distinctive features of Nakh are innovations, in which case they could have arisen at any point in the family tree. The only possible evidence for the early split between Nakh and the ancestor of all other East Caucasian languages, would be non-trivial innovations reconstructed for the ancestor of all Daghestanian languages. Since reconstructions of Proto-East-Caucasian are very scarce, only one instance of such a Daghestanian innovation to the exclusion of Nakh has been proposed, a (relatively trivial) phonological change from PEC \textit{*st} to Proto-Daghestanian \textit{*c} (\cites[]{nichols03cc}). Hence, a cautious researcher is, for the time being, left with a rather shallow tree, that looks approximately like the one presented in Figure \ref{tree}.

\begin{figure}
% % % 	\includegraphics[width=\textwidth]{figures/Proto-East-Caucasian2.png}
\begin{forest} for tree = {forked edge}
	[Proto-East-Caucasian
	  [Nakh]
	  [Tsezic]
	  [Avar-Andic]
	  [Lak]
	  [Dargic]
	  [Khinalug]
	  [Lezgic]
	]
\end{forest}
\caption{Shallow East Caucasian tree}
\label{tree}
\end{figure}

\subsubsection{Dialectal variation}

Variation in Tsova-Tush is mostly observed between different generations of speakers. See \sectref{sources} for historical linguistic changes in the last 200 years that characterise the speech of different generational groups. Tiny differences are reported in the speech of inhabitants of different neighbourhoods in Zemo Alvani (\cite{gagua1977encyc}), corresponding to different villages in Tsovata (see \sectref{tusheti}).


\subsubsection{Typological outline} \label{outline}

Tsova-Tush has a relatively large consonant system with a five-way contrast in obstruents (voiced, aspirated, long aspirated, ejective, long ejective), which includes uvulars and pharyngeals. Pharyngealisation (phonetically, probably epiglottalisation)  is a phonation property of syllables, audible mostly on the vowel but traditionally analysed as consonant phonemes. Phonemically, the vowel system is small, with several phonological processes creating diphthongs, long vowels and nasal vowels. In contrast to the other Nakh languages, there is no centralisation of short unaccented vowels. 
There is a system of morphophonological alternations between the presence and absence of a vowel, resulting from various deletion processes.

The numeral system is decimal for the first two decades and vigesimal thereafter (although numerals higher than 1000 are usually borrowed from Georgian, see \sectref{numerals}). From cardinals, a rich set of derivatives such as ordinals, distributives and multiplicatives are formed.

Tsova-Tush features both head-marking and dependent-marking. Nouns and pronouns distinguish four or five basic grammatical cases, whose endings are monoexponential, and 24 spatial cases, most of which are cumulative (i.e. they consist of two suffixes). Most nouns have a distinct form, called Oblique, to which non-Nominative cases are added. Cases follow one of several lexically determined plural suffixes. There are four gender agreement markers which distinguish, depending on how one counts them, up to eight genders. Gender agreement is a partial category: only about 30\% of the verb roots and 10 adjective roots take gender agreement. 

Verbs distinguish a large number of TAME forms, which blend aspect and evidentiality with pure tense. In addition to these, there are a handful of suppletive verb stems marking argument plurality, and several, no longer productive means of indicating pluractionality, both partial categories (the distinction is expressed only in a limited set of verbs). For most tense-aspect forms Tsova-Tush distinguishes a corresponding indirect evidential form by means of suffixation and periphrastic constructions.

Two converbs are derived from verbs: anterior and simultaneous, used in temporal adverbial clauses and clause chaining. There is a rich system of finite subordination with subordinating conjunctions as the main device for complementation and adjunct subordination. Relativisation uses deletion, and non-finite relative clauses precede the noun, while finite relative clauses follow it. Clause chaining is achieved by using specialised sequential verb forms, which are finite.

There are derivations that change argument structure (detransitivising, transitivising, causativising) but no inflectional or syntactically-based passivisation.



\subsubsection{Previous scholarship}


Tsova-Tush is significant for having received the first grammatical description within the East Caucasian language family. The priest and native speaker Iob Tsiskarishvili produced a manuscript of a grammar, a glossary and a New Testament translation (\cite{tsiskarovgloss,tsiskarovgram,tsiskarovbible}). The material was taken to Saint Petersburg by Marie-Félicité Brosset (\cite{brosset}) with the purpose of publishing it. Although the manuscripts themselves were never published, they formed the basis of a grammar written by Anton Schiefner (\cite{schiefner54,schiefner56,schiefner59}). This grammar is greatly expanded from Tsiskarishvili's original, and Schiefner enlisted the help of Tsiskarishvili's brother, the native speaker Giorgi Tsiskarishvili, for elicitation. 

Apart from a very brief field recording by Adolf Dirr (\cite{dirr-tt}), no known investigation into Tsova-Tush was made from the 1850s until the early Soviet era, when two scholars independently wrote new descriptions of the language. In Tbilisi, Rusudan Gagua started her description of Tsova-Tush with a dissertation on the case system (\cite{gagua43diss}), after which she continued publishing throughout the Soviet era (\cite{gagua48erg,gagua52,gagua56vowel,gagua62,gagua83noun,gagua87,holiskygagua}). Meanwhile in Moscow, the Chechen Yunus Desheriev, head of the Department of Caucasian languages (which had recently moved from Saint Petersburg/Leningrad to Moscow) wrote an influential grammar (\cite{desheriev53}). In Tbilisi, Gagua was joined by Latavra Sanikidze (\cite{sanikidze66redupl,sanikidze76decl,sanikidze84q}) and native speaker Kote Chrelashvili (\cite{chrelashvili67,chrelashvili82,chrelashvili84,chrelashvili87,chrelashvili90}). Both scholars produced short grammars in the Georgian grammatical tradition: \textcite{chrelashvili02,chrelashvili07,sanikidze10}. 

In the United States, Dee-Ann Holisky started working on Tsova-Tush in the early eighties (\cite{holisky84,holisky85,holisky87}), connecting Tsova-Tush data with theoretical frameworks. Her work culminated in the much-cited grammar sketch, written with Rusudan Gagua (\cite{holiskygagua}) and was continued by Alice Harris, who introduced Tsova-Tush to a wider audience with her work on exuberant exponence (\cite{harris08,harris09,harris11,harris13metath,harrissamuel}). Together with Bryn Hauk (\cite{hauk}), she wrote a new grammar sketch (\cite{haukharris}). 

Meanwhile in Telavi, the regional capital near Zemo Alvani, a team of native speaker linguists at Telavi State University recently created a number of monographs on various  subdomains of Tsova-Tush grammar (\cite{mikeladze08interf,mikeladze11,mikeladze13verb,gigashvili16cs,bertlani12phon}), in addition to their collections of texts and dictionary volumes. 

Four dictionaries have been published to date: \textcite{kadkad84}, an impressive, 1000-page handwritten Tsova-Tush-Georgian-Russian dictionary of high quality\footnote{This dictionary was started and collected by father and son Davit (1861--1937) and Niko (1895--1976) Kadagidze and edited by Rusudan Gagua with the help of Latavra Sanikidze, Izolda Jolbordi, Eva Usharauli-Kadagidze, and Elene Kadagidze. It was digitised by Rusudan Papiashvili and Jost Gippert, and is available at \url{http://titus.fkidg1.uni-frankfurt.de/texte/etce/cauc/batsbi/tt_dict/tt_dilex.htm}}; \textcite{faehnrich01dict}, a more concise, Tsova-Tush-German dictionary based on \textcite{kadkad84}; a newer, four-volume dictionary by the Telavi team (\cite{teldict1,teldict2,teldict3,telavi2019tsovatushdict4}). A new Tsova-Tush-Georgian dictionary has recently been compiled at the Chikobava Institute of Linguistics by Diana Kakashvili, based on \textcite{kadkad84}, but containing several more lemmas and some additional grammatical information on nouns (\cite{kakashvili2022dict,kakashvili19aboutdict}).



\subsection{Tusheti and Eastern Georgia} \label{tusheti}

Tusheti, the ancestral home of the Tush, is a small mountainous region of approximately 1200 km\textsuperscript{2} bordered by Khevsureti in the West, Chechnya in the north, Daghestan in the east and the plains of Kakheti in the south. Its inhabited areas lie at an elevation of approximately 2000 metres, while its peaks are at 3000--4250 meters. The Tush are traditionally divided into four \textit{temis} (singular \textit{temi}, meaning both `group of villages' and `clan'), corresponding to the major valleys of Tusheti (see Figure \ref{map-tusheti}). Members of three \textit{temis}, the Pirikiti (`other side') Tush, the Chaghma (`downhill') Tush and the Gometsari Tush traditionally speak the Tush dialect of the Georgian language.\footnote{Since the Chaghma valley is now the most populous, and many villages in the Gometsari and Pirikiti valleys are abandoned, the term `Chaghma' is often used for all Tush other than the Tsova-Tush. In 1886, when the Russian imperial census was conducted (after the migration to the plains, but before the construction of Zemo Alvani, see below at the end of \sectref{history}), the Tsova community had 1533 members, Pirikiti 1296, Gometsari 1358, Chagma 1420 (\cite{statistics1893}).} Only the Tsova \textit{temi} additionally speaks the Nakh language Tsova-Tush, and is thereby perhaps the only group identifying as ethnically Georgian that speaks a language that is not Kartvelian.



\begin{figure}
	\includegraphics[width=\textwidth]{figures/Tusheti4.png}
	\caption{Map of Tusheti, key locations highlighted in green}
	\label{map-tusheti}
\end{figure}

Tusheti and Khevsureti\footnote{As well as the region known as Khevi, west of Khevsureti.} lie directly south of the topographical ridge, the line connecting the highest peaks of the Great Caucasus range, which forms the border between Georgia and the Russian Federation. It is important to note, however, that both regions lie north of the Caucasian watershed, and rivers (and therefore footpaths) connect Tusheti to Daghestan, and Khevsureti to Chechnya and Ingushetia.

There are five main entrances into Tusheti:
\begin{enumerate}
	\item Abano Pass, crossed by the current jeep road connecting Tusheti to the Kakhetian plains. 
	
	\item Sakorne pass, connecting the Alazani basin (that runs southward via Tbatana and the Pankisi valley into the Kakhetian plains and into Azarbaijan) with the Tushetis Alazani basin (running eastward into the Gometsari valley and the Chagma basin, where it joins the Pirikitis Alazani to form the Andi Koysu). The area around the source of both the Alazani and the Tushetis Alazani is called Alaznistavi (Georgian \textit{alaznistavi} `beginning of Alazani', Tsova-Tush \textit{ʕambarča}). The Sakorne pass is the one the Tsova-Tush traditionally crossed to travel between the Tsovata valley and Alvani.
	
	\item Borbalo pass, connecting the Alaznistavi area with Pshavi and lower Khevsureti to the west.
	
	\item Atsunta pass, connecting the Pirikiti valley with upper Khevsureti to the west.
	
	\item Samaghele pass, on a path parallel to the Andi Koysu, east of Omalo, crossing the national border at Intsukhi (Georgian \textit{inc'uxi/ibc'oxi}). This pass connects the Chaghma basin to Daghestan in the East, and is closed for civilians and guarded by the military.
	
\end{enumerate}

\subsection{History of the Tsova-Tush people} \label{history}

\begin{figure}
	\includegraphics[width=\textwidth]{figures/indurta-photo.jpg}
	\caption{Ruined village of Indurta with its towers still standing. Photo by Diana Kakashvili}
	\label{photo-indurta}
\end{figure}


The ethnogenesis of the Tsova-Tush people and the settling of the Tsovata valley is a hotly debated topic. The scarcity of historic sources from before the 19th century combined with sometimes strong ethno-nationalist tendencies of researchers from the area give rise to conflicting perspectives and a tense scholarly debate. Since the present work is about language contact, and different possible historical scenarios could have given rise to different language contact situations, the topic needs to be addressed here.

Three possible historical scenarios are represented in the historical literature. All three will be discussed in more detail below.

\begin{enumerate}
	\item The ancestors of the Tsova-Tush are a Nakh tribe who migrated from the northern slopes of the Caucasus mountains to Tusheti (inhabited by Georgians) after inter-tribal strife. 
	\item The regions Khevsureti and Tusheti (both north of the Caucasian drainage divide) were first populated by Nakh tribes, which then shifted to Georgian ethnic self-identification and (all but the Tsova-Tush) to the Georgian language.
	\item The Tsova-Tush have been ethnically Georgian since time immemorial, and Nakh features in the Tsova-Tush language are due to contact with Nakh peoples throughout history.
\end{enumerate}

According to \textcite{nichols11,nichols2004alpine}, Proto-Nakh was spoken in the highlands between the upper Alazani river (present-day Tusheti), the upper Argun river (present-day Chechnya and Khevsureti), and the upper Assa river (present-day Ingushetia and Khevsureti). This area, at altitudes of around 2000 meters, is where these major rivers have their sources to the south of the crest, which then flow north through passes, making it natural for one ethnic group to occupy both the north and south slopes. Furthermore, it has long been assumed that the Khevsur and the Tush peoples were originally Nakh, based on similarities in both material and intangible culture, such as the construction of towers, the animal sacrifice at stone shrines, and the worship of a number of deities (later adopted into the local version of orthodox Christianity) (\cite{klaproth,zisserman,Tsiskarov1846newspaper,guldenstad1}). Many 19th century sources mention the diverse linguistic situation in Tusheti, with both Georgian-speaking villages, as well as Nakh-speaking\footnote{The term “Kist” was used by Georgian scholars of this period to refer to Nakh in general, as opposed to the contemporary Kisti Chechens in the Pankisi gorge.} ones in Tsovata, but also in the Pirikiti valley (\cite{zisserman, eliosidze} as cited in \cite{gigashvili2014migration}), which is now completely Georgian-speaking. Additionally, most toponyms in Tusheti have a clear Nakh origin, although their precise etymology is often unknown. Furthermore, the Tush and Khevsur dialects of Georgian betray a clear Nakh substrate in their phonology and lexicon (\cites[]{uturgaidze1966mountainous}). According to Georgian semi-legendary historiography (9th century \textit{Georgian Chronicles} compiled by Leonti Mroveli), king Saurmag, successor to, and son-in-law of legendary king Parnavaz, invited the Dzurdzuks (i.e. Nakh tribes) to inhabit the southern slopes of the Caucasus in the 2nd century BC (\cites[24]{rayfield2012history}). Whether this account is accurate or not, it is clear that the areas now known as Tusheti, Khevsureti (and possibly also Khevi) have been inhabited by Nakh tribes for many centuries, although the date of their settlement, as well as of its subsequent Georgianisation, remains unclear. It is tempting to view the Tsova-Tush as the last Nakh-speaking remnant of this migration, the only group not to have switched to Georgian under influence of Georgian-speaking tribes that have moved up from the south along the Aragvi, Iori, and Alazani rivers.


The clаim that the ancestors of the Tsova-Tush migrated to Tusheti from Nakh-speaking areas elsewhere (\cites[]{dirr1909names,elanidze1988tusheti}) has in the past been supported by some Tsova-Tush authors (\cite{bukurauli1897moambe,Tsiskarov1846newspaper}). Its main argument centers around a toponymic parallel. The archaic place name for the Tsovata valley is Tsova-Tush \textit{vabu}, and its inhabitants were called \textit{vab-bi} (remembered by older speakers\footnote{Recorded in E288 and MM407 (for abbreviations of subcorpora, see \sectref{sources}).} and attested in poems in \textcite{tsiskarovgloss} and \textcite{schiefner54}). This can be connected with Chechen \textit{vabo} (\cite{desheriev1963comparative}), a historical region of the central Caucasus (\cite{suleymanov1976topo}). \textcite{ellis1788memoir} places the Vabbi (spelled Wabi) near the source of the Terek river, in the present-day Georgian region of Khevi (that is to say, on the southern slopes, but north of the drainage divide). Whatever be its exact location, this region is identified as the homeland of the Feappii (Ingush \textit{fæppī}), a historical Nakh clan federation (\cites[82--83]{wixman1984peoples}, \cite{nichols11}). The Feappii are closely related to the origins of the Ingush, who before the 18th century seem to have consisted of two clan confederations, the Ghalghaai (the modern self-designation for all contemporary Ingush), who lived near the confluence of the Assa and Ghuloi-khi rivers, and the Feappii, who apparently lived higher up and on the southern slopes (\cites[]{nichols11}). Time estimates for this proposed migration of the \textit{vabbi}/\textit{fæppī} to the Tsovata valley range from the 8th/9th century (\cite[17]{makalatia1933tusheti}) to the second half of the 17th century (\cite{elanidze1988tusheti}). Another possibility is that such a migration did not take place, and that the homeland of the Vabbi/Feappii in fact included the Tsovata valley itself.

More contemporary Georgian scholars have compared the modern Tsova-Tush, other Tush, and Chechens, and have come to the conclusion that the Tsova-Tush are virtually indistinguishable from other Tush clans, as seen from an anthropological perspective (\cites[]{sharashidze1960anthro}). These scholars sharply contrast the Tsova-Tush with (Kisti) Chechens, mainly based on religious practices, which leads them to the conclusion that the Tsova-Tush cannot have anything to do with Nakh peoples, and must have been ethnically and linguistically Georgian since time immemorial (e.g. \cite{itonishvili2012tush}). In this view, Nakh features and words in Tsova-Tush are explained as contact phenomena. This sentiment is in line with many Tsova-Tush themselves, who strongly self-identify as ethnically Georgian, and emphasise their difference to Nakh people (see below, \sectref{attitude}).

Although much remains unclear regarding the origins of the Tsova-Tush, one can hypothesise the following: (1) Tusheti (as well as Khevsureti and Khevi) must have been Nakh speaking before the advent of Georgian tribes from the southern lowlands; (2) The Tsova-Tush (\textit{vab-bi}) can be connected to the Feappii of Ingush legendary history. This does not necessarily imply that the Tsova-Tush language is genealogically more closely related to Ingush than to other Nakh varieties, or that the Tsova-Tush must have migrated from the northern slopes (since the Feappii are said to have inhabited the southern slopes as well and it is unknown whether the Nakh variety they spoke actually resembled contemporary Ingush more closely than any other Nakh variety).

The Christianisation of Tusheti was completed in the 9th century.\footnote{For a description of pre-Christian practices, some of which survive to this day, see e.g. 
	\textcite{charachidze19688paienne,tuite2004lightning}.} In 1575, the Tsova-Tush, together with other Tush tribes, sought the protection of King Levan of Kakheti against the Daghestanians. They were allowed to pasture their flocks in the Alvani area (\cites[288--289]{allen1970history}). From that time onward the Tsova-Tush accepted the kings of Kakheti as their patron (\cites[]{desheriev53}). Almost a century later, the Tush (both Tsova-Tush and the other \textit{temis}) affirmed their claim over Alvani by coming to the aid of the princes of Kakheti and Aragvi. In 1659, under the leadership of legendary hero Zezva Gaprindauli, they helped temporarily defeat the Persians at the fortresses of Bakhtrioni and Alaverdi.

\begin{figure}
	\includegraphics[width=\textwidth]{figures/Tsovata1.png}
	\caption{Map of the Tsovata valley}
	\label{map-tsovata}
\end{figure}

The Tsovata valley (see Figure \ref{map-tsovata}) is a valley of approximately 25 km\textsuperscript{2} around the stream Tsovatistskali (\textit{c'ovatisc'q'ali}), a tributary of the Tushetis Alazani. Although we do not know when exactly it was first settled, we know it contained four villages: Sagirta, Etelta, Mozarta, Indurta (see Figure \ref{map-tsovata}). The village of Shavtskala (Georgian \textit{šavc'q'ala} `blackwater'), located just outside of the Tsovata valley, was founded at a later (but otherwise unknown) date. The distance between each of the settlements in Tsovata was between 500m and 1 km. These settlements were, according to legend, established by Tsova-Tush brothers from the village of Tsaro, 4--5 km from these villages, at the entrance of the Tsovata valley. Tsaro is therefore considered the legendary homeland of the Tsova-Tush forefathers (\cite{desheriev53}). In fact, the root lying at the basis of Tsova, \textit{c'o} \footnote{The \textit{-va} of \textit{c'o-va} being an individualising suffix, compare \textit{mastxo-v(a)} `enemy', \textit{donlo-v(a)} `rider'.} is straightforwardly connected to Tsaro (\textit{c'ar}).\footnote{The two forms show an archaic Nominative-Oblique alternation, compare \textit{šo} `year (\textsc{nom})', \textit{šar-} `year (\textsc{obl})'; \textit{\u{g}o} `clan, family (\textsc{nom})', \textit{\u{g}ar-} `clan, family (\textsc{obl})'; \textit{pħu} `dog (\textsc{nom})', \textit{pħar-} `dog (\textsc{obl})'.} Towards the end of the 18th century, houses were built in Tbatana, the Pankisi valley, and the Alvani region. From that time, almost all families went to these lower places for the winter.


Up until the 1820s the Tsova-Tush lived in Tsovata semi-permanently, practicing vertical transhumance (see \sectref{sociology}), with winter pastures in the Alvani area and intermediate pastures and some accommodation in Tbatana. In the 1820s, a large avalanche or landslide (\cite{makalatia1933tusheti}) or flood (\cite{itonishvili2012tush}) destroyed the village of Etelta (\cite{makalatia1933tusheti}) or Indurta (\cite{itonishvili2012tush}) or Sagirta (\cite{desheriev53}). The aforementioned sources also mention the already difficult living conditions in Tsovata, with little food, harsh winters, and frequent raids by Daghestanians. Subsequently, in the 1820s and 30s, most families abandoned Tsovata and settled near their winter pastures in the Alvani region, and moved to Gurgalchala, Pkhakalkura and Tsistolkura, hamlets that had been founded in the 18th century (\cites[]{faehnrich01dict}). Tsovata remains empty to this day, the ruined villages (see Figure \ref{photo-indurta}) are only used as shelters by sheepherders in summer.


Around 1925, the modern villages of Zemo Alvani and Kvemo Alvani were built on a grid template under the supervision of Tsova-Tush Davit Abashidze, an early Soviet surveyor. Zemo Alvani has nine streets that are three kilometres long and run east-west, crossed by seven “connector streets” (with no houses on them) at appr. 500 meter intervals, see Figure \ref{photo-alvani}. Small channels/gulleys run along all streets for irrigation. According to the 2014 census, Zemo Alvani had 3306 inhabitants, an estimated two-thirds of whom are Tsova-Tush, with the remainder consisting of other Tush, who also make up the entire population of Kvemo Alvani and Laliskuri (see Figure \ref{map-tusheti}).

\begin{figure}
	\includegraphics[width=\textwidth]{figures/Zemo-Alvani-Photo.jpg}
	\caption{Photo of Zemo Alvani with the Alazani river to its left, and the Gombori hills in the background. The foothills of the Greater Caucasus can be seen on the right-hand side. Photo by Pete Mozay.}
	\label{photo-alvani}
\end{figure}

\subsection{Basic sociological and anthropological facts} \label{sociology}

A starting point for the anthropological study of the Tush is \textcite{makalatia1933tusheti}. The Tsova-Tush, as well as other Tush, display a combination of customs connected to traditional mountain life, and customs they have in common with the other Georgians living on the Kakhetian plains. 

Traditional vertical transhumance is still practiced by some Tush. Adult men take their flocks of sheep, numbering several hundred heads, from their summer pastures in Tsovata to their winter pastures in Alvani and Shiraki (see Figure \ref{map-shiraki}) each year. The alpine meadows of Tbatana function as a middle station. Women, children and the elderly are not involved in the practice of transhumance, and stay in Alvani (previously, in the high villages of Tsovata). This type of vertical transhumance has been prevalent in all of the central and eastern Caucasus for centuries (\cites[57]{wixman1980ethnic}). These days, the majority of men, too, lead a sedentary lifestyle, working in urban centers like Akhmeta, Telavi or Tbilisi, or working as seasonal workers in Grozny, Vladikavkaz, Moscow or other cities in the Russian Federation, or in Western Europe. Many other families lead a subsistence lifestyle with a small number of cattle and a vegetable garden. The traditional diet consisted of mutton, cheese and bread, including traditional dishes such as \textit{k'ot'or}, a cheese-filled flatbread baked in a skillet, and khinkali, mutton-filled dumplings. Nowadays, families have adopted a diet typical of the plains, involving products such as eggplant, bell pepper, cornflour and many fruits. Even though few native plant names remain in Tsova-Tush and most have been replaced by Georgian, Tsova-Tush plant use in fact shows clear traces of a (previous) pastoralist lifestyle (\cite[393]{pieronietal20}).

\begin{figure}
	\includegraphics[width=\textwidth]{figures/Shiraki1.png}
	\caption{Map of Kakheti, key locations highlighted in green}
	\label{map-shiraki}
\end{figure}

The Tush traditionally brewed beer, although many Tush now own small-scale or medium-size vineyards. Beer was drunk (rarely libated) in many rituals of purification, remembrance of the deceased, and initiation into adulthood. The Tsova-Tush celebrate several holidays, the most important being Dadaloba (\textit{dad(a)} `father', \textit{-loba} is an abstract noun suffix). Dadaloba takes place in Tsovata, and involves ritualistic toasting with beer at a stone shrine, a horse race, the remembrance of `extinct' clans, the sacrificial slaughter of animals, and the preparation of khinkali. Another holiday, shared by all Tush, celebrates the above-mentioned victory by Zezva Gaprindauli, and is called Zezvaoba. Taking place in Kvemo Alvani, it involves the most important horse race of the season, as well as boxing and wrestling matches, trading of local handicrafts such as carpets and felt clothing, and singing and dancing. Since their migration to the plains, the Tsova-Tush observe Georgian orthodox Christian holidays (most prominently Easter) and their appurtenant traditions. Zemo Alvani features two big churches with accompanying graveyards, as well as two smaller churches. Most Tush do not go to mass.\footnote{As is the case for many communities in Kakheti.}

Traditional Tsova-Tush society was clan-based. The Tsova-Tush \textit{temi} is divided into several smaller clans (Tsova-Tush \textit{gor} from Georgian \textit{gvari}, having partially replaced older \textit{k'ur} `smoke' > `hearth' > `family'), originally centered around one defence tower in Tsovata. Marriage was strictly within the \textit{temi} but exogamous in terms of sub-clans. Nowadays, the majority of Tsova-Tush marry outside the \textit{temi} with other Tush, or with Kakhetians, Ossetians, Kisti or others. This change in marriage practice is assumed to have taken place shortly after the Tsova-Tush migration from the mountains (although direct sociolinguistic evidence is lacking), and has contributed greatly to language shift, for which see below, \sectref{shift}.



\subsection{Sociolinguistics} \label{socioling}

\subsubsection{Basics}


Tsova-Tush has no recognised official status in Georgia. According to Article 8 of the Georgian Constitution, the only official language in the country is the Georgian language, alongside the Abkhaz language in the occupied/secessionist territory of Abkhazia. Although minor policies towards the integration of some minorities have sporadically been implemented, such as the creation of minority language textbooks written in Georgian, and Georgian language textbooks written in minority languages, this has been done only for the minority languages with the largest numbers of speakers: Abkhaz, Armenian, Azeri, Megrelian, Ossetian, and Svan. Smaller, one-village languages such as Tsova-Tush, Udi, Tat, Bezhta, and Hunzib (all spoken in the eastern region of Kakheti) receive no governmental support.\footnote{This is not to say that the bigger languages do receive a lot of support. Most support is for the Kartvelian languages Svan and Megrelian (since these speakers identify as ethnic Georgians), and even for those languages, support is usually professed, not practised. Kisti children receive Kisti Chechen classes, and the Avar, Hunzib and Bezhta children receive Avar classes from teachers paid by the Georgian state. This is because the Chechens and Avars are large, recognised Caucasian ethnic groups, whereas the Tsova-Tush are classified as Georgian, and thus receive only Georgian education.} One has to note that whenever non-governmental organisations run projects with the goal of strengthening minority rights, these projects select minorities based on ethnic self-identification, and therefore often do not include the Tsova-Tush, who have been self-identifying as ethnically Georgian for at least a century.

According to a questionnaire carried out in 2020 by Rezo Orbetishvili, my main collaborator in Zemo Alvani, among the 2700 Tsova-Tush living in the village, only 275 claim to speak Tsova-Tush, and an additional 783 report to be passively or partially proficient in the language.

\subsubsection{Multilingualism and language attitudes} \label{attitude}

All Tsova-Tush speakers are bilingual in Georgian. Most speakers report having learned Tsova-Tush and Georgian simultaneously as a child. The Georgian language is used by the Tsova-Tush in all aspects of life, from formal domains, such as education, governance, media, to informal domains, such as talking to neighbours and family. Only when all participants of a given conversation are fluent Tsova-Tush speakers, which is very rare (considering there are 275 of them), some Tsova-Tush might be used, although the use of Georgian is very common here as well (Rezo Orbetishvili, p.c.). 

Nowadays, the Tsova-Tush speak a regional variety of Standard Georgian that betrays their Tush heritage and their Kakhetian environment on a phonetic and lexical level. Although the contemporary state of the Georgian dialects is poorly investigated, it is clear that the vast majority of speakers no longer preserve the phonological, morphosyntactic, and most lexical peculiarities of the dialects as they are described in the middle of the 20th century (see e.g. \textcite{khubutuia1969tushlexicon,shanidze1957mountaindialects,uturgaidze1966mountainous,uturgaidze60} for Tush, and \textcite{jorbenadze1989geodialect,kartulidialekt} for an extensive bibliography of all Georgian dialects). Even though dialect characteristics seem to be decreasing, this work adduces evidence of a much stronger influence of the Tush dialect in the past, see Sections~\ref{loanwordsphon} and~\ref{loanwordsconcl}. As this work deals with the morphosyntactic influence of Georgian on Tsova-Tush, and as Standard Georgian and contemporary Tush and Kakhetian Georgian do not differ significantly in that domain, Standard Georgian will serve as a point of comparison (although occasionally, the Tush dialect is compared against directly).


Many Tsova-Tush speakers age 40 and above are also proficient in Russian. Although during the Soviet era, Russian was considered by many Georgians not to be a truly ‘foreign’ language, but rather a sort of second native language (\cites[]{blauvelt2013russian}), this statement does not fully apply to rural Kakheti. There were very few Russian schools and most local Georgians only received Russian classes as a foreign language, except those that received university education. Men who served in the Soviet Army know Russian, women mostly do not (Diana Kakashvili, p.c.). Nowadays, although attitudes towards the Russian language are relatively neutral, it is used in restricted situations only, such as in communication with tourists or Azerbaijani nationals.

Attitudes of Tsova-Tush speakers towards their language have not been systematically investigated and can be said to be complicated. On the one hand, the gradual loss of Tsova-Tush is generally lamented, and many Tsova-Tush, whether they are fluent speakers or not, welcome supporting initiatives such as the creation of learning materials and Tsova-Tush language classes being included in the curriculum. On the other hand, as stated elsewhere, it is crucial for the Tsova-Tush to be considered ethnically Georgian, and not Nakh. Language plays an important role in the construction of the Georgian ethnic and national identity, and for some Tsova-Tush, their ancestral language might represent something of an obstacle towards full acceptance into the larger Georgian ethnic identity. Many Georgians from other regions who have not visited Alvani confuse the Tush with non-Georgian Muslims, and the loss of the their language might just make it easier for the Tsova-Tush to counteract these stereotypes. It should be noted, however, that most Georgians are unaware of the existence of the Tsova-Tush language.


\subsubsection{Language shift} \label{shift}



The traditional multilingual situation of the Tsova-Tush was probably what \textcite{nichols2013vertical} refers to as ``asymmetrical vertical multilingualism''. This means that highlanders who were speakers of rather small languages acquired larger languages spoken further down the mountains and in the lowlands in order to be able to communicate in the market places located within these larger communities. For the Tsova-Tush, this meant that before the 1830s, only a relatively small number of men were fluent in Georgian. In fact, some contemporary speakers report that among the generation of their grandparents (born approximately between 1880 and 1920), there were still many monolingual Tsova-Tush speakers. It is, however, unclear how prevalent bilingualism really was at this period, let alone before. \textcite[109]{gigashvilietal2020migration} suggest that, prior to the 1820s, there was only ``weakly developed individual bilingualism'' among the Tsova-Tush. \textcite{shanidze1970tush}, on the other hand, suggests that a shift to Georgian as the dominant language began as early as the 18th century.

Estimates of numbers of speakers have dropped from 3000 (\cites[]{holiskygagua}) to 1000 (\cites[]{gippert08}), to 500 (\cites[]{unesco,}, \cites[]{koryakov2002atlas}, Rieks Smeets p.c.). As mentioned, based on the survey by Rezo Orbetishvili (p.c.) the number of fluent speakers is now (in 2023) estimated to be around 300 speakers, with another 750 speakers having some (passive) proficiency. \textcite{gigashvili2014migration} have carried out their own survey, concluding that the number of speakers below 40 is 160, with no speaker but one scoring ``very well'' on their fluency test. Furthermore, the language is hardly, if at all, transmitted to the generation under the age of 25. Out of the approximately 180 school children (age 6--18) in Zemo Alvani, only 4\% claim to be able to speak Tsova-Tush, while 60\% claim their parents speak it in and around the house. Education is conducted in Georgian exclusively. Informal teaching sessions are being organised with children age 6--10, but learning materials are lacking, and the language of instruction is Georgian.  

All in all, Tsova-Tush is classified as severely endangered (\cites[]{unesco}) due to its low number of speakers and its lack of transmission to new generations. Therefore, unless serious revitalisation efforts are initiated now, Tsova-Tush is estimated to go extinct within the next decades.

\section{The Georgian language} \label{introgeorgian}

Georgian is a Kartvelian language spoken by approximately 4 million speakers, mostly in Georgia (where it is the national language), in neighbouring Turkey, Azerbaijan, Iran and by diaspora communities in Russia, Ukraine, Armenia, the USA, Germany and elsewhere. It is one of very few languages of the Caucasus that have a historical writing tradition, and Georgian inscriptions have been found as early as the 5th c. AD. 

In terms of phonology, Georgian features a three-way distinction in consonants (voiced, aspirated, ejective), and is famous for its large consonant clusters with up to 8 segments in onset position. The vowel system consists of the vowels /a, e, i, o, u/.  

Georgian features both head-marking and dependent-marking. Nouns distinguish 7 core cases and an additional 9 secondary cases that can also be analysed as postpositions. Core grammatical cases are the Nominative, Dative and Ergative, clearly distinguished in nouns, but not distinguished in pronouns. All nouns inflect according to the same declension pattern (with minor morphophonological variation), and Georgian does not have grammatical gender. 

Georgian verbal morphology is agglutinative in principle, although TAME and person inflection involves so-called distributed exponence, in which the marking of grammatical meaning is distributed across multiple morphs, each of which contribute to that meaning, but without a defined meaning itself. Verbs distinguish 11 synthetic TAME forms, grouped into 3 so-called series, based on shared morphological principles, and the argument encoding for transitive and active intransitive verbs (Nominative subject in the ``Present'' series, Ergative subject in the ``Aorist'' series, Dative subject in the ``Perfect'' series).  The Georgian verb can mark up to 3 arguments, which involves a complex hierarchy to decide which marker is overtly expressed. A special set of markers (pre-radical vowels) are used for valency operations, but each is also used to signal tense-aspect forms.

Georgian features mostly finite subordination (see \chapref{subordination}) using relative pronouns, subordinating conjunctions and a general subordinator used in relative, complement and adjunct clauses. 

Descriptions of Georgian in this work are based on grammars such as \textcite{hewitt95}, \textcite{vogt}, and \textcite{shanidze53sapudz}, and on the more detailed references cited in the individual sections.
\pagebreak
\section{The present study} \label{study}

This work has a threefold goal: 

\begin{enumerate} 
	\item To contribute to the overall description of the Tsova-Tush language, by filling gaps in the previous literature, most prominently, the system of spatial cases (\sectref{spacase}), the verbal forms that have indirect evidential semantics (\sectref{evid}), and the domain of clause combining (\sectref{subordination}); 
	\item To contrast Tsova-Tush constructions with functionally equivalent constructions in two groups of languages: (a) Chechen and Ingush, its closest relatives, and (b) Georgian, the language of wider communication, which all Tsova-Tush speakers speak as a second language. This is done in order to form hypotheses concerning which Tsova-Tush construction is inherited, and which has arisen under influence of Georgian; 
	\item To provide the most probable diachronic scenario of language contact, by looking at historical Tsova-Tush language data, as well as at its historical sociolinguistics. 
\end{enumerate}


Additionally, the adaptation of Georgian lexical material into the Tsova-Tush grammatical system is described. See \sectref{loanwordsphon} for phonological adaptation, \sectref{morphadapt} for the morphological adaptation of nouns, and \sectref{loanverb} for the morphosyntactic adaptation of verbs. See \sectref{loanwordsconcl} for different semantic domains within the borrowed lexicon.

In \sectref{theory}, relevant topics in the language contact theory will be discussed. \sectref{question} contains the main research questions, and \sectref{methods} is devoted to the main methods and the Tsova-Tush language data that forms the basis for this work.

\subsection{Language contact} \label{theory}

\subsubsection{Introduction to the field}

The field of language contact studies can be divided broadly into three main areas of research: (1) the psycholinguistics of multilingual individuals, (2) the sociolinguistics of multilingual communities, and (3) studies into the results of language contact, i.e. contact\hyp induced language change. The term “language contact”, therefore can be viewed as shorthand for a range of different complex psycholinguistic and sociolinguistic scenarios (\cites[25]{pakendorf2007sakha}). This characterisation of language contact is by no means meant as a definition. For a more in-depth and theoretical treatment, see the sources in the following sections, where a brief overview of relevant topics is given.


\subsubsection{Structural dimensions}

In research on language contact, most scholars distinguish between two main types of contact-induced change (i.e. structural outcomes of language contact): the transfer of linguistic form (i.e. concrete phonological shapes) and the transfer of structural and semantic patterns (i.e. restructuring, replication or calques of form-meaning pairs) (see \cite{weinreich1953contact,haugen1953bilingual,heath1984contactchange,matrassakel2007pattern}). This distinction received various names (e.g. ``direct diffusion'' vs. ``indirect diffusion'' (\cite[119]{heath1978arnhemland}), ``global copying'' vs. ``selective copying'' (\cite{johanson1999codecopying,johanson2002framework})), but recently the opposition has become known as pattern borrowing (PAT) vs. matter borrowing (MAT) (\cite{matrassakel2007pattern}, adopted by many authors, such as \cite{gardani2020matpatmorph,arkadiev2018preverbs,wiemerwaelchli2023intro}). \textcite{sakel2007matpat} gives a definition: 

\begin{quote}
	We speak of MAT-borrowing when morphological material and its phonological shape from one language is replicated in another language. PAT describes the case where only the patterns of the other language are replicated, while the form itself is not borrowed. In many cases of MAT-borrowing, also the function of the borrowed element is taken over, that is MAT and PAT are combined.
\end{quote}

Another important topic of research into the structural aspects of language contact is the relative likelihood of borrowing of different types of forms and categories, that is, the borrowability of items and constructions (see \cite{haspelmathtadmor09wold,moravcsik1978universalscontact,thomasonkaufman1988,vanhoutmuysken1944borrow,field2002borrow,matras2011universals}). For example, it seems that nouns are more likely to be borrowed than non-nouns and function words\footnote{This, of course, does not mean that it is “difficult” to borrow function words, just that nouns are even more likely to be borrowed.}, free morphemes more likely than bound morphemes, derivational morphology more than inflectional morphology, agglutinating affixes more than fusional affixes, superlatives more than comparatives, clause linking more than word morphology, among many other hierarchies.

Note that the above generalisations have to be viewed as tendencies, as no linguistic feature is entirely “borrowing\hyp proof” (\cite[2]{aikhenvald2006contactintro}). In fact, this observation has led some scholars to the suggestion that “the attempt to develop any universal hierarchy of borrowing should perhaps be abandoned” (\cite{curnow2001borrowed}). However, even though it has been established that everying can be borrowed, it remains clear that not everything is borrowed equally frequently: some grammatical and other features are particularly open to~-- and others are more resistant to~-- borrowing. \figref{owens-elements-table}, adapted from \textcite{owens1996sprachkontakt} reflects which parts of the language are more likely to be shared with its relatives, and which are easily attributable to language contact.

\begin{figure}
		\begin{tabular}{ll}
			
			\multicolumn{2}{r}{\textbf{More similar to genealogical relatives}} \\
			
			Form-function pairings in inflectional morphology & \multirow{5}{4em}{\includegraphics[scale=0.1]{figures/arrow2.png}} \\ 
			Core lexicon &  \\
			Syntactic constructions &  \\
			Discourse structure &  \\
			Structure of idioms &  \\
            \multicolumn{2}{r}{\textbf{More similar to neighbouring languages}}  \\
			
			
		\end{tabular}
	\caption{Genealogical versus contact-induced elements in a language}
	\label{owens-elements-table}
\end{figure}



Tsova-Tush is no exception this generalisation. It will become clear (see \sectref{summary}) that the contact-induced change in Tsova-Tush is restricted to clausal syntax, calques in morphology, and the borrowing of non-core vocabulary, with only a single instance of a borrowed inflectional morpheme (see \sectref{suffixpl}).


\subsubsection{Social dimensions}

Groups of people may be exposed to, and/or use more than one language. Certain particularities of different types of community multilingualism can have consequences for (1) the sociolinguistic situation (i.e. the number and type of people speaking a given language, and in which circumstances), and (2) the structure and make-up of the language itself.

The process of abandonment of one language for another by a community or parts of a community is referred to as language shift. \textcite[20]{clyne2003contact}, however, distinguishes several uses of the phrase language shift, whether it refers to individuals or communities, and whether it is a shift in (1) main language, (2) dominant language, (3) the language use in one or several specific social domains (e.g. home, work, school), or (4) particular language skills (reading, writing, speaking, listening). 

The notion of social prestige is often used in research concerning language shift and contact-induced language change (e.g. \cite{haugen1966dialect}). The prestige factor of a given language (e.g. a rich literary heritage, high degree of language planning, potential international standing) or of its speakers (political, cultural and/or economic elite) motivates speakers to switch to speaking this language (additionally or exclusively), or adopt speech patterns or words into their own language. 

However, the notion of prestige does not provide a direct explanation for the borrowing of word-forms or construction to replace forms or construction that already existed in the language prior to contact. It also does not explain why some categories are more likely to be borrowed than others. Therefore, prestige can sometimes be better understood as a licence to employ forms or constructions from a language that has e.g.  more institutional support or a wider community of speakers (\cite[19]{matras2012activity}). Instead, recent research has focussed more on decisions and communicative acts by individual speakers against the background of social prestige, rather than centering a theory of language contact around social factors such as prestige.

Sociolinguistic aspects such as language attitudes, prestige, political dominance, and language planning have long been viewed as the prime predictors of the degree of contact-induced change (see e.g. \cite{kiparsky1938comment,coteanu1957contact}, cited in \cite{thomasonkaufman1988}).

\textcite[66]{thomasonkaufman1988} mention three concrete measurable aspects of the so-called intensity of contact between two populations:

\begin{enumerate}
	\item Relative population size. The smaller the community using the affected language is compared to the community using the donor language, the higher the intensity of contact.
	\item Length of contact. Borrowing that involves extensive structural change often have a history of several hundred years of intimate contact (\cite[41]{thomasonkaufman1988}). On the other hand, as stated already, language shift may take as little as a generation.
	\item Degree of bilingualism. Although lexical borrowing frequently takes place without widespread bilingualism, extensive structural borrowing requires extensive bilingualism over a considerable period of time (although counterexamples exist (\cite[346]{thomasonkaufman1988})). 
\end{enumerate}

Using the criteria above, but without giving explicit benchmarks for any of them, \textcite{thomasonkaufman1988} have devised five categories of intensity of contact.

\pagebreak

\begin{enumerate}
	\item \emph{Casual contact.} Only non-basic vocabulary is borrowed.
	\item \emph{Slightly more intense contact.} Besides the above, also conjunctions and adverbs are borrowed. Phonologically, new phonemes can be adopted, but only in loanwords. Syntactically, new functions of existing constructions are borrowed, or new orderings that cause no typological change.
	\item \emph{More intense contact.} Besides the above, also adpositions, derivational affixes, personal and demonstrative pronouns, low numerals are borrowed more likely than in category 2. Phonologically, stress rules and prosody are borrowed. Syntactically, minor orderings such as noun-adposition can be affected.
	\item \emph{Strong cultural pressure.} Besides the above, phonological rules can be borrowed, as well as the additional or loss of entire contrastive features. Extensive word order changes occur and inflectional categories and affixes are borrowed.
	\item \emph{Very strong cultural pressure.} Significant typological change: new morphophonological rules, phonetic changes, word structure rules (such as prefixation vs. suffixation, flexional vs. agglutinative), alignment systems, agreement rules, and bound pronominal elements can all be borrowed.
\end{enumerate}


As stated above, prestige as a technical concept has been widely used to explain the degree of contact-induced change in a given speech community. In its most extreme form, the claim is that ``nothing can be borrowed from a language which is not regarded as prestigious by speakers of the borrowing language'' (\cite{moravcsik1975verbborrow}). However, there are clear exceptions: communities that are shifting to another language often take phonological features and/or some lexical items into the target language, resulting in influence from a less prestigious language into a more prestigious one (\cite[37]{thomasonkaufman1988}).

More exceptions are those of dialect borrowings and so-called small-scale multilingualism. These communities are variously described as practicing ``egalitarian multilingualism'' (\cite{francois2012egalitarianmulti}) or ``balanced multilingualism'' (\cite{aikhenvald2006contactintro}). When attempting to generalise over these types of community, it is noted that (1) multilingual language use is not primarily motivated by power relations or prestige (this does not mean that these societies are necessarily egalitarian or traditional). (2) These communities seemed to have  remained at the margin of those processes that create officially monolingual societies with enforced standard language cultures (\cite[46--47]{luepke2016smallscale}, see also \cite{pakendorfetal2021smallscale}). This must have been the situation in the Caucasus before the advent of regional lingua francas (see \cite{daniel2021franca}) and later national languages like Georgian, Azeri and Russian.

\subsubsection{Psychological dimensions}


Code-switching, or the alternation between two (or more) languages within a single discourse, sentence, or constituent, is observed in the speech of many multilinguals (\cite[583]{poplack1980codeswitch}). These multilinguals, even those who acquired their languages from birth (like the Tsova-Tush), have to master the rules on appropriate, context-bound selection of one form or construction over another as part of a process of socialisation (\cite{lanza1997infantmixing}). Some environments, language pairs, or social networks allow for a greater flexibility of choices (\cite{grosjean2001bilingualmode,grosjean2008bilinguals}), and hence, code-switching is more likely to occur.

It is known for decades now that multilingual speakers do not ``switch off'' one of their languages in communication, but that their full linguistic repertoire is available to them at all times (see e.g. \cite{bialystok2012bilingualbrain}). Hence `contact' is by some researchers seen as speakers' creative negotiation of this repertoire along with the pragmatic and sociolinguistic conventions that govern their distribution (\cite{matras2012activity}).

Single-word code-switches (i.e. insertions) have long been recognised as a source of loanwords (see e.g. \cite{gardnerchloros19887cs}). The distinction between single code-switches and borrowings is 
conceptually transparent: single-word insertions are part of the synchronic, discourse-based behaviour of multilinguals, whereas borrowings are instances of language change, fully entrenched in the entire speech community. Most often, there exists a lack of direct data on the spread of each Georgian lexical item in the Tsova-Tush community, when they engage in monolingual Tsova-Tush practices. Therefore, inclusion of lexical items in a dictionary will be used as a proxy: in this work, Georgian lexemes that occur in the Tsova-Tush dictionary (\cite{kadkad84}) are considered to be borrowings, and other Georgian
material to be code-switching (see \cite{deucharetal2018welshbilingual} for a methodological precedent; see also
\cite{forker19}).

Pattern borrowing, too, can be seen as a result of the above-mentioned creative negotiation of the multilingual speaker's repertoire (\cite{matras2012activity}). All that is needed for an individual's spontaneous innovation to become an instance of contact-induced change, is (1) the ability and willingness of a community of interlocutors to understand and accept the meaning of the new construction and subsequently (2) a relatively low degree of normative control on the language that is exercised in this community.


\subsubsection{Previous work on Tsova-Tush-Georgian language contact} \label{previouscontact}

Outcomes of contact in the Caucasus have not yet been studied on a systematic basis, and have only attracted considerable (albeit selective) attention in recent decades (\cite[61]{dobrushinaetal2020overview}). For a discussion of contact phenomena in Adyghe (West Caucasian), see \textcite{hoehlig1997adyghecontact}. There is some research on language contact in East Caucasian, but many languages of the area have not been documented thoroughly enough to identify most contact-induced changes. Some recent studies include \textcite{johanson2006caucasus,dobrushina2017contactvolitional,authier2010kryzcopy,rindpawlow2023khinaluginfluence} on Turkic influence in East Caucasian.  For other studies, see also \textcite{belyaev2019contactossetic} on Ossetic, \textcite{forker19} on Hinuq and \textcite{maisak2019udicontact,maisak2019agulcontact,maisak2016lezgiccontact} on Lezgic languages. An important case study is \textcite{kojima19} on Georgian influence in Tsova-Tush resulting in the development of person marking, which is discussed in-depth in \sectref{person}. \textcite{khalilov04contact} discusses Georgian loanwords in Tsezic and other Daghestanian languages.

Already in the 19th century, the Tsova-Tush scholar Ivane Bukurauli laments the idea that his native language ``becomes more and more mutilated as it is being influenced by the Georgian language and at a certain point in time will perish completely''  (\cite[43]{bukurauli1897moambe}).\footnote{\begin{otherlanguage}{georgian}წოვები ლაპარაკობენ დამახინჯებულ ღლიღურს ენაზედ და ქართული ენის გავლენის ქვეშ ყოფნის გამო უფრო მახინჯდება და მოვა დრო, რომ სრულიად ბოლო მოეღება.\end{otherlanguage}} %not rendering yet [JWS]

Even though linguistic influence of Georgian is mentioned in every description of Tsova-Tush, a structured overview of contact phenomena in Tsova-Tush is lacking. Most scholarly effort has instead been devoted to:

\begin{itemize}
	\item Describing the lexical and phonological similarities between the Nakh languages and the Northeast Georgian dialects such as Tush, Khevsur, Pshav and Mokheve (e.g. \cite{kakashvilietym,uturgaidze1966mountainous,chincharauli2003parallels}). This is generally seen as Nakh substrate influence on these dialects.
	\item Characterising the number and type of Georgian loanwords in Tsova-Tush (throughout \cite{schiefner56} and \cite{desheriev53}). For an overview, see \textcite{faehnrich1998loanwords,WS,gippert08}. See Sections \ref{loanwordsphon} for phonological adaptation, \ref{morphadapt} for the morphological adaptation of nouns, and \ref{loanverb} for the morphosyntactic adaptation of verbs. See \sectref{loanwordsconcl} for different semantic domains within the borrowed lexicon.
	\item Describing the changing sociolinguistic situation of the Tsova-Tush during the last two centuries (\cite{gigashvili16cs,gigashvili2014bi,gigashvili2014migration}).
\end{itemize} 

Regarding the third topic, two scholars have independently devised a periodisation of recent sociolinguistic changes of the Tsova-Tush language. Firstly, \textcite[5--20]{desheriev53} uses a model describing language shift that mirrors Marxist social theory.\footnote{Based on the famous linguistic article by \textcite{stalin}, which itself is perhaps ghost-written, or at least heavily inspired by Arnold Chikobava (\cite{medvedev2003stalin}).}
See Table \ref{Desheriev-phases} for the general framework adopted by Desheriev.

\begin{table}
		\small
		\begin{tabularx}{\textwidth}{l QQQ}
        \lsptoprule
			& {1} & {2} & {3} \\
			\midrule
			Multilingualism &	Monolingualism & Bilingualism &	Monolingualism \\\addlinespace
			Lexicon & Some borrowings pertaining to hitherto unfamiliar concepts  &	Intense borrowing, replacement, mixed multi-word expressions & Some words of the former language might be retained  \\\addlinespace
			Phonology & Borrowing of some sounds & Greater influence & \\\addlinespace
			Morphology &	Borrowing is rare & Borrowed bound morphemes used with native roots & \\\addlinespace
			Syntax &	None &	Some pattern borrowing	& \\
            \lspbottomrule
		\end{tabularx}
	\caption{Desheriev's phases of language contact}
	\label{Desheriev-phases}
\end{table}

Applied to the Tsova-Tush situation, Desheriev arrives at the periodisation as seen in Table \ref{Desheriev-phases-TT}, dividing recent Tsova-Tush history into three periods, mainly based on the rate of multilingualism of its speakers. 


\begin{table}
		\small
		\begin{tabularx}{\textwidth}{l QQQ}
        \lsptoprule
			& 1 & 2 & 3 \\
			\midrule
			Multilingualism	& Tsova-Tush 					& Tsova-Tush+Georgian 				& Georgian \\\addlinespace
			Lexicon 		& Some borrowings pertaining to hitherto unfamiliar concepts& Intense borrowing, replacement, mixed multi-word expressions& Some Tsova-Tush relics  \\\addlinespace
			Phonology 		& Borrowing of some sounds 		& Greater influence 				& \\\addlinespace
			Morphology 		& Borrowing is rare 			& \textit{-ur}, \textsc{nom.pl}-morphemes, verb adaptation  & \\\addlinespace
			Syntax 			& None 							& Finite relative clauses	& \\
			\lspbottomrule
		\end{tabularx}
	\caption{Desheriev's phases of Tsova-Tush-Georgian language contact}
	\label{Desheriev-phases-TT}
\end{table}

The application of the top-down model of Table \ref{Desheriev-phases} has certain disadvantages. Firstly, it uses the multilingualism status of the Tsova-Tush people as the deciding factor in demarcating the three periods. However, we must assume a gradual shift from Tsova-Tush to Georgian in the past 200 years. If a sharp increase in the number of Georgian bilinguals or a sharp decrease in the number of Tsova-Tush speakers exists, Desheriev does not identify them in absolute terms. Secondly, Desheriev does not correlate these three stages with specific outcomes of contact-induced language change. Most instances of contact-induced change take place in the second period, where Tsova-Tush-Georgian bilingualism is prevalent, with no further chronology. Instances of contact-induced change taking place in the first period are described in vague and general terms, and no concrete examples (except for lexical items) are mentioned. A third problem, for our purposes, is the fact that no dates have been attached to the three periods, so that no correlations between specific histocial processes and outcomes of contact-induced change can be drawn.


Another periodisation attempt, this time with absolute dates, is by \textcite[14--15]{mikeladze08interf}, see Table \ref{mikeladze-phases}. 
This periodisation has the advantage of being more concrete, since it provides more or less absolute dates that demarcate the different phases. Additionally, it notes the different varieties of Georgian that the Tsova-Tush must have been in contact with. The first period is characterised by life in the Tsovata valley (see \sectref{history}), where the Tsova-Tush were mostly in contact with Tush Georgian. After migration downhill to the Alvani region, they came into closer contact with the Kakhetian dialect of Georgian, and to some extent also with the standard language. In the Soviet era, the standard language was propagated more intensively, and Russian was introduced (although it had a marginal status in Kakheti). After the collapse of the Soviet Union, the standard language played an even bigger role, decreasing the use of regional dialects.

\begin{table}
	\small
	\begin{tabular}{lllll}
    \lsptoprule
		& 1 & 2 & 3 & 4 \\
		\midrule
		Dates & --1820 & 1820--1920 & 1920--1990 & 1990-- \\
		Government & Tsarist & Tsarist & Soviet & Post-Soviet \\
		Settlement & Tsovata valley & Alvani region & Alvani & Alvani \\
		Conctact      & Tush    Geo. & Kakhetian Geo. & Standard Geo. & Standard Geo. \\
		\quad variety &              & (+standard)    & (+Russian) &  \\
		\lspbottomrule
	\end{tabular}
	\caption{Mikeladze's phases of Tsova-Tush-Georgian language contact. Geo.: Georgian.}
	\label{mikeladze-phases}
\end{table}




Mikeladze's 4 phases provide a useful description of the changing sociolinguistic situation of the past 200 years, especially in regards to the changing contact varieties, the intensity of pressure from the prestige variety, and the settlement history (see also \cite{gigashvili2014migration}). No correlations to linguistic outcomes of the given language contact situation are proposed.

These outcomes (i.e. proposed instances of contact-induced change), however, have been noted already since the earliest descriptions of Tsova-Tush. Both \textcite[4]{schiefner56} and \textcite[12--14]{desheriev53} have mentioned the following contact phenomena, even if only in passing. It is interesting to note that \textcite[12--14]{desheriev53} characterises the amount of structural influence from Georgian as relatively little, while \textcite[4]{schiefner56} calls the influence, especially in the nominal domain, extensive. The following phenomena were observed by Desheriev and Schiefner:

\begin{itemize}
	\item A large amount of loanwords (see \textcite{WS,gippert08} and \sectref{loanwordsconcl})
	\item Borrowing of the derivational suffix \textit{-ur} (see \sectref{adjectives})
	\item Borrowing of a plural/polite verbal suffix \textit{-t} (see \sectref{suffixpl})
	\item Development of subject cross-reference markers on the verb (see \sectref{person})
	\item The use of postpositions/adverbs as verbal prefixes\footnote{This phenomenon falls beyond the scope of this work.}
	\item The use of conjunctions and finite subordination (see \sectref{subordination})
\end{itemize}

The fact that these phenomena are actually structural copies of Georgian is by no means proven. In the present work, I critically assess these phenomena and adduce several more. I then provide the most probable scenario of contact for each of them, and establish correlations between the sociolinguistic and historical factors on the one hand, and specific outcomes of contact-induced change on the other.

\subsection{Main research questions}	\label{question}

In other words, besides trying to contribute to a more thorough description of the Tsova-Tush language, the main aims of this work are to answer the following two questions. 

\begin{enumerate}
	\item What Tsova-Tush constructions have likely arisen due to contact with Georgian?
	\item How can these influences be explained by the changing sociolinguistic situation of Tsova-Tush in the past 200 years?
\end{enumerate}



\subsection{Data and methods} \label{methods}

\subsubsection{Theoretical framework}

The theoretical framework employed in this work can be best described with the expression ``Basic Linguistic Theory'', following \textcite{dixon1997blt,dixon2010blt1}. Basic Linguistic Theory is the framework that underlies most descriptive and typological work of the last 50 years, but, being an informal theory, this fact is not always made explicit in these works (\cite{dryer2006blt}). The usage of the term ``Basic Linguistic Theory'' stems from the observations that (1) there is no such thing as atheoretical or theory-neutral description, since it is impossible to describe language without making some theoretical assumptions; and (2) the extent to which most descriptive work shares the same theoretical assumptions is striking, especially when one considers how much such work has in common in its assumptions compared to other theoretical frameworks. It is important to note, however, that theoretical notions used in description should be kept distinct from explanatory theories that try to answer the question why a given language is the way it is. In this work, these explanations will be couched in general theories about grammaticalisation and language change (following \cite{dryer2006blt}) and, more specifically, will be compared against observations in the domain of contact-induced language change (see Secion \ref{theory}).

\subsubsection{Methodology}

Contact-induced change has often been invoked as the explanation for a situation where two languages feature a similar word or structure that cannot be the result of common inheritance. To warrant such an explanation, however, at least two things need to be proven (see \cite{poplacklevey2010contact}).

\begin{enumerate}
	\item Actual language change occured, and the observed phenomenon is not simply an instance of synchronic variation in the language.
	\item The change is contact-induced, and not a result of language-internal developments.
\end{enumerate}

As to the first condition, variation is a necessary condition for change, but it is not the same. Variation is prevalent when the speaker group is bilingual and/or the speakers are residents of minority-language communities in intense contact with a majority language (\cites[394]{poplacklevey2010contact}, and see \sectref{theory}). However, it cannot on face value be excluded that a suspected outcome of contact-induced change wasn't already present in the language before contact occurred. Thus, the first step in establishing the existence of change is comparison over time. Three comparison points could potentially be found to compare a language variety against, in order to test if the feature in question is actually the result of change, or simply an instance of previously existing variation. 

\begin{enumerate}
	\item The language variety used by an older generation of speakers. Good examples of this include \textcite{otheguyetal2007contactspanish} and \textcite{saad2020phd}. Unfortunately, because virtually all Tsova-Tush speakers are between 60 and 90 years of age, no attempt at discerning inter-generational differences in speech was made.
	
	\item Previous stages of the same language variety. Tsova-Tush language data is available from the middle of the 19th century and the middle of the 20th century, allowing us to identify differences and inferring change between these different stages. See \sectref{sources} for a characterisation of these different stages of attestation. Also, even if earlier data of a given construction is lacking, it can sometimes be ascertained using internal reconstruction (see e.g. \cite[234]{traskshistorical}, \cite[243]{josephjanda2003historical})
	
	\item Comparison with related varieties. If a given feature in a language variety is absent in its immediate sister languages, and most other languages of the family, this feature can be hypothesised to be an innovation. Therefore, in this work, Tsova-Tush constructions are sometimes compared against functionally equivalent constructions in Chechen and Ingush, especially when historical Tsova-Tush-internal evidence is inconclusive. This is operationalised with the historical-comparative method (see e.g. \cite[191]{traskshistorical}, \cite[199]{josephjanda2003historical})
\end{enumerate}

Speakers of Tsova-Tush self-report to have Tsova-Tush and Georgian as (nearly) simultaneously acquired L1s. Therefore, in contrast to ethnic minority speech communities where the investigated minority language is clearly an L2 for many speakers (see \cite[305]{poplack1997convergence}), Tsova-Tush speakers have fully acquired Tsova-Tush as one of their first languages. Even if certain aspects of Tsova-Tush grammar could be attributed to attrition due to underuse, the corpus-based nature of the present research ensures that potential attrition features are presented only when they are shared by a majority of speakers, and which can therefore be characterised as language change at a community level.

Once  a given linguistic feature has been established as an innovation, it is compared against a structurally equivalent system in Georgian, the presumed donor language. Only when the constructions in the two languages are sufficiently parallel, we are able to claim that this similarity of constructions is the outcome of contact. In other words (\cite[400--401]{poplacklevey2010contact}): 

\begin{quote}
	If the hierarchy of constraints conditioning the variable occurrence of a candidate for change in a contact variety is the same as that of its pre-contact precursor, while differing from that of its presumed source, no structural change has taken place. If it features a constraint hierarchy different from those of both its pre-contact precursor and the presumed source, we can infer that change has occurred, but not one that is contact-induced. Only when a candidate for change in a contact variety features a constraint hierarchy different from that of its pre-contact precursor, but parallel to that of its presumed source, can we conclude in favour of contact-induced change.
\end{quote}

In this work, the presumed donor language is always Georgian. Although Tsova-Tush contains many loanwords that ultimately originate from Russian, Turkic or Iranian, these (with very few exceptions) are also found in Georgian, whether they occur in standard Georgian or in eastern dialects. Turkic and Iranian lexical items that are not found in Georgian are not treated in this work for two reasons. First of all, they are very few. The dictionary of \textcite{kadkad84} only contains 4 nouns of Turkic origin that do not occur in Georgian: \textit{doxt'ur} `doctor' (but also found in Megrelian, a sister language of Georgian), \textit{tep} `hill', \textit{som} `som (Turkic currency)' and \textit{zoran} `strong, brave'. This last word is also found indirectly in Chechen \textit{zørtala} `dense, strongly built, stocky' and may be of Iranian origin. Secondly, lexical borrowing is not the focus of this work. For a brief discussion on Russian loanwords, all of which, again, are found in colloquial Georgian, see \textcite{WS}. In sum, we only have verified historical evidence for contact with Georgian, and the vast majority of Iranian, Turkic and Russian lexical items can also be explained through Georgian contact. There is no evidence for long-lasting and/or intense contact between Tsova-Tush and Turkic or Iranian peoples that would result in the type of morposyntactic influence discussed in this work.



To summarise, this work follows the workflow below, paraphrased and adapted from \textcite[93--94]{thomason2001contacthandbook}:

\begin{enumerate}
	\item Situate the proposed change with respect to its host linguistic system. Tsova-Tush linguistic facts are described in similar fashion to a sketch grammar, in order to show how a given construction is embedded in the larger grammatical system.
	
	\item Identify a presumed source of the change.  As a first step, the donor language in this work is always presumed to be Georgian. The specific variety (Old Georgian, Tush Georgian, or the modern standard language) can sometimes be compared against directly, or has to be deduced indirectly through periodisation (see \sectref{previouscontact}).
	
	\item Locate structural features shared by the source and recipient languages. Tsova-Tush and Georgian constructions are compared in depth.
	
	\item Prove that the proposed interference features were not present in the pre-contact variety. This is usually achieved by comparing contemporary Tsova-Tush against historical sources from the mid-19th and mid-20th century. Additionally, Tsova-Tush constructions are compared against Chechen and Ingush equivalents.
	
	\item Prove that the proposed interference features were present in the source variety prior to contact. Occasionally, historical stages of Georgian will be compared, if a given construction has changed in the history of the language. 
	
	
\end{enumerate}

Another issue is the type of data that is used to investigate contact-induced changes. Since manifestations of change can be rare in running discourse, much published evidence comes from experimental or quasi-experimental data (\cites[396]{poplacklevey2010contact}), such as elicitation, reaction tests, or forced interviews. Since this type of atypical situations may result in atypical behaviour, the present work makes use of corpus data only. The corpus is sufficiently large (see \sectref{sources} below), and only potentially contact-induced constructions that have a sufficient corpus frequency have been presented here.

\subsubsection{The Tsova-Tush corpus} \label{sources}

The Tsova-Tush language data in this work are from a corpus containing approximately 250,000 tokens. It consists of a number of subcorpora, shown in \tabref{sources-table1}. 

\begin{table}
		\begin{tabularx}{\textwidth}{llQ}
        \lsptoprule
			Abbr. & Reference & \\
			\midrule
			IT & \cite{tsiskarovgloss,tsiskarovbible} & New Testament translation\footnote{The transliteration and translation of this manuscript is still in its early stages.} and 9 songs \\
			AS & \cite{schiefner54,schiefner56, schiefner59} & 7 Bible excerpts, 2 folk tales and a poem \\
			YD & \cite{desheriev53} & 18 sample texts from grammar \\
			KK & \cite{kadkad84} & $\pm$ 10.000 sample sentences from dictionary\footnote{Digitised by Rusudan Papiashvili and Jost Gippert, and available at \url{http://titus.fkidg1.uni-frankfurt.de/texte/etce/cauc/batsbi/tt_dict/tt_dilex.htm}.} \\
			KC & \cite{chrelashvili02} & 12 sample texts from grammar \\
			E & described in: \cite{gippert08} & Large documentation project ECLING\footnote{Endangered Caucasian Languages in Georgia, available at \url{https://archive.mpi.nl/tla/islandora/object/tla\%3A1839\_00\_0000\_0000\_0008\_24AD\_F}.} \\
			EK & \cite{kadagidze09} & Text collection \\
			LS & \cite{sanikidze10} & 15 sample texts from grammar \\
			MM1 & \cite{MM1} & Text collection \\
			MM2 & \cite{MM2} & Text collection \\
			MM3 & \cite{AM} & Poetry collection \\
			MM4 & \cite{MM4} & Text collection \\
			LJ & \cite{jamarishvili} & Text collection \\
			BH & described in: \cite{hauk} & Bryn Hauk's documentation\footnote{Available at \url{https://scholarspace.manoa.hawaii.edu/handle/10125/42581}.} \\
			WS & (not published) & This author's documentation (\sectref{fieldwork}) \\
        \lspbottomrule
		\end{tabularx}
	\caption{Tsova-Tush sources (chronologically)}
	\label{sources-table1}
\end{table}

Most examples in this work are from subcorpus E, which contains the most free spoken material. Each example is tagged with a letter combination indicating a subcorpus, as seen in Table \ref{sources-table1}, followed by a number combination indicating text, paragraph, sentence. Hence, EK002-5.4 refers to the fourth sentence of the fifth paragraph of the second text in \textcite{kadagidze09}. English translations of E and BH  are taken over directly, unless specified otherwise. Other English translations are produced by myself, translated from the Tsova-Tush, taking the Georgian (KK, KC, EK, LS, MM, LJ), Russian (IT, YD) or German (AS) translations into account. Third persons are translated with `she', `he', or `they' according to context, except in KK, which consists of separate sample sentences, where 3rd persons are translated with `s/he' or `they'.

Other Tsova-Tush examples are attributed directly to their source, be it example sentences from grammar sketches and articles, or from separately published single texts such as \textcite{holiskykadagidze} and \textcite{kojima09}. The oldest audio recording is \textcite{dirr-tt}.

Based on phonological features (for which see \sectref{processes}), the subcorpora can be divided into three periods: 

\begin{enumerate}
	\item The middle of the 19th century (IT, AS). Many final vowels are not yet reduced or apocopated, final \textit{-ħ} is not yet deleted, umlaut causes diphthongs.
	\item The middle of the 20th century (YD, KK). Final vowels are written as reduced, final \textit{-ħ} is written as optional (they are written in brackets), umlaut can cause diphthongs or monophthongs.
	\item The 21st century (E, EK, LS, LJ, KC, MM, BH, WS). Final vowels are apocopated (although sometimes written as reduced vowels orthographically), final \textit{-ħ} is deleted, umlaut causes mostly monophthongs.
\end{enumerate}

Unless otherwise specified, all Standard Modern Georgian examples are taken from the Georgian National Corpus (GNC)\footnote{\url{http://gnc.gov.ge}}. The author is indicated with most examples. Georgian examples without reference to an author are taken from the Georgian Reference corpus (GRC), also located in the GNC. Old Georgian, Chechen and Ingush examples are taken from the academic literature, and are cited as such. This work contains several examples of the Tush dialect of Georgian. They are cited similarly to Tsova-Tush examples, with two-letter tags for each source (see Table \ref{sources-table2}). 

\begin{table}
		\begin{tabularx}{\textwidth}{llQ}
        \lsptoprule
			Abbr. & Reference & \\\midrule
			TU & in: \cite{uturgaidze60} & Text collection \\
			IG & in: \cite{kartulidialekt} & Text collection \\
			GC & (online) & Texts on GDC\footnote{Georgian Dialect Corpus: \url{http://corpora.co}} collected by Giorgi Tsotsanidze in the 1980s \\
			TT & (online) & Texts on GDC collected in the 2000s \\
		\lspbottomrule
		\end{tabularx}
	\caption{Tush Georgian sources}
	\label{sources-table2}
\end{table}

Although not always referred to explicitly, this work makes use of dictionaries of Modern Georgian (\cite{rayfield06dict}), Old Georgian (\cites[]{abuladze1973dict}), Tush Georgian (\cites[]{tsotsanidze02tushdict}), Chechen (\cite{matsiev61dict,nichols04chechendict}), and Ingush (\cite{nichols04ingushdict}). 


\subsubsection{Fieldwork} \label{fieldwork}

In 2017, 2018 and 2019, I conducted fieldwork in Zemo Alvani, residing in the village for six weeks every summer. I organised elicitation sessions with a group of 5 collaborators, age 55--65. Furthermore, I recorded mainly dialogues of male participants, in an effort to complement the ECLING corpus (subcorpus E, see \sectref{sources} above), which consists mostly of monologues, where female speakers are overrepresented. All recordings were transcribed and translated into Georgian with the help of my main collaborator, Rezo Orbetishvili. The texts are glossed and translated into English, and will be published online, along with the rest of the corpus. For the recordings, a ZOOM H4NSP 
handheld audio recorder was used, and the recordings were processed using ELAN\footnote{\url{https://archive.mpi.nl/tla/elan/download}} and FLEx.\footnote{\url{https://software.sil.org/fieldworks/download/}}
In all, 3 hours of dialogue were recorded (2500 tokens), and additionally, 5 short texts were translated from Russian or Georgian (another 2500 tokens).

\subsubsection{Conventions} \label{conventions}

All glossing is mine. The transcription of Georgian, Ingush and Chechen from secondary sources is adapted for consistency.
All examples of languages other than Tsova-Tush are preceded by the name of the language variety. Hence, all examples without a language indicator are Tsova-Tush.
In running text, labels that are used for specific Tsova-Tush form-function pairings (i.e. language-particular categories such as tense-aspect forms and case forms) are capitalised, following the convention of \textcite{comrie76}. Phonological processes are also capitalised.
The practical transcription used in this work is a broad phonetic transcription (i.e. expresses surface forms), and is used throughout this work. Tsova-Tush data in numbered examples is given in its surface form (with all phonetic processes already applied), whereas Tsova-Tush data in tables represents underlying morphological forms. Data in running text can be either surface form or underlying morphology, which will be indicated in each case.

In glossed examples, round brackets are used for glosses of morphemes that have disappeared through regular phonological processes. The exception is the verbal root \textit{i} `do', which when deleted is expressed as “∅” in examples. A period is used to separate several abbreviations that are rendering a single object-language morpheme. If English lacks a single-word gloss for a lexical root, an underscore is used to connect elements of a multi-word expression.

When citing a lexical Tsova-Tush verb, the Verbal Noun in \textit{-ar} is used as citation form, as per tradition.





