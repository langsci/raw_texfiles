\chapter{Nominal inflection and the noun phrase}\label{nounphrase}

\section{Introduction}

In this chapter, structural aspects of the Tsova-Tush noun phrase will be discussed. In \sectref{nouns}, the inflectional categories of nouns will be discussed, with special attention given to a novel analysis of the system of spatial cases. In the same section, a first attempt at classifying nouns into declension classes is given. In \sectref{modifiers}, the inflection of the most common types of modifiers is presented, and \sectref{agreement} shows along which parameters these modifiers agree with the head noun. \sectref{complexnp} shows a number of ways in which noun phrases can be combined to form other noun phrases (including a preliminary analysis of caseless modifying nouns in \sectref{barenoun}), whereas in \sectref{headlessnp}, attention will be devoted to pronoun phrases and other noun phrases without a noun.


In terms of contact-induced change, this chapter shows that Georgian influenced Tsova-Tush noun phrases considerably: (1) The Essive cases often have lative semantics (\sectref{spacase}), (2) The comparative and superlative constructions are borrowed or calqued from Georgian (\sectref{adjectives}), (3) Tsova-Tush numerals higher than one hundred are usually borrowed from Georgian (\sectref{numerals}), (4) the distal demonstrative is used as a deictically neutral third person personal pronoun (\sectref{dempro}), and (5) Tsova-Tush uses negative pronouns as opposed to constructions with indefinite pronouns and clausal negation (\sectref{negpro}).


Furthermore, \sectref{decladapt} discusses the two different layers of loanwords as seen by their morphological adaptation, \sectref{genderadapt} presents data on the gender assignment of loanwords, and \sectref{adjadapt} discusses the morphological adaptation of adjectives.





\section{Simple noun phrases} \label{simplenp}

Tsova-Tush noun phrases are headed by nouns or pronouns. For headless noun phrases (featuring only modifiers), see \sectref{headlessnp}. However, core arguments do not need to be expressed overtly.\is{Zero arguments} This fact is expected, as this feature is found in almost all languages of the Caucasus (\cite[9]{polinskyintro}), including in Georgian (\cite[43--36]{tuite98}). Tsova-Tush features an ergative alignment system\footnote{The alignment system of Tsova-Tush is not dependent on tense-aspect or any other feature that would cause a split alignment system, such as the one found in Georgian.\is{Ergative alignment} It does, however, allow for intransitive arguments in the Ergative under specific conditions, for which see \sectref{valency}. } (see \sectref{valency}).
A transitive verb requires its subject to be in the Ergative case, but in Example (\ref{simplenp-ex18a}), it is left out completely. Similarly, in (\ref{simplenp-ex18b}) and (\ref{simplenp-ex18c}), the object, which would otherwise have been in the Nominative case, is not expressed.\is{Ergative case}

\begin{exe}
	\ex\label{simplenp-ex18}
	\begin{xlist}

			\ex\label{simplenp-ex18a}
			\gll  kuirc'l-e-x ʕajrtvaⁿ veⁿ maɬ-eⁿ. \\
			wedding-{\Obl}-{\Cont} plenty wine drink.{\Pfv}-{\Aor} \\
			\trans `At the wedding, [they] drank a lot of wine.' 
			\hfill (KK038-3018)

		

			\ex\label{simplenp-ex18b}
			\gll magram šeron=aɁ d-i-eⁿ o cok'l-e-v. \\
			but {\Refl}={\Emph} {\D}-do-{\Aor} {\Dist} fox-{\Obl}-{\Erg} \\
			\trans `But the fox did [it] himself.'
			\hfill (E159-29)

		

			\ex\label{simplenp-ex18c}
			\gll t'atbu-v daxsna-d-∅-or mšobl-i-v. \\
			silver.{\Obl}-{\Ins} save-{\D}-{\Tr}-{{\Imprf}} parent-{\Pl}-{\Erg} \\ 
			\trans `The parents saved [them] with money.'
			\hfill (MM116-2.3)

		
	\end{xlist}
\end{exe}

It is very common for a noun phrase to consist of only one noun, as in (\ref{simplenp-ex1a}), where both agent and object are expressed by a single noun each, as well as in (\ref{simplenp-ex1b}), which contains an experiencer and an object.

\begin{exe}
	\ex\label{simplenp-ex1} \begin{xlist}
		

			\ex\label{simplenp-ex1a}
			\gll pst'uin-čo-v šur agrilbad-j-i-eⁿ. \\
			woman-{\Obl}-{\Erg} milk cool-{\J}-{\Tr}-{\Aor} \\
			\trans `The/a woman cooled the/some/∅ milk.'
			\hfill (KK001-3018)

		

			\ex\label{simplenp-ex1b}
			\gll badr-e-n nan j-ag-iⁿ.  \\
			child-{\Obl}-{\Dat} mother {\F}.{\Sg}-see-{\Aor} \\
			\trans `The/a child saw a/the/its mother.'
			\hfill (KK001-0017)

		
	\end{xlist}
\end{exe}

As can be seen from Example (\ref{simplenp-ex1}), Tsova-Tush features no dedicated class of articles, i.e. words that appear regularly in combination with nouns to show the definiteness, specificity, anaphoricity, case, gender or number of the noun they modify (\cite[152]{dryer07}).  The Tsova-Tush demonstrative, however, can be used with an anaphoric or definite function, while the numeral `one' can be used as an indefinite article to introduce new information. See \sectref{demonstratives} for demonstratives and \sectref{numerals} for numerals.\is{Article}



In default word order, demonstratives, quantifiers, participial relative clauses, Genitives, adjectives and numerals all precede the head noun. Relative clauses using a relative pronoun or the general subordinating conjunction \textit{me}, as well as quantified nouns in the Genitive follow the head noun. For a more detailed discussion of all relative clauses, see \sectref{relative}.\is{Word order}

\section{Nouns} 
\subsection{Introduction} \label{nouns}

Nouns inflect for case and number, with regular case morphology, but unpredictable plural formation. Many, but not all nouns feature an opposition between a Nominative and an Oblique stem, to which all case inflections attach (see \sectref{Oblique}). 

The plural is marked by one of several plural suffixes. The productive suffix \textit{-i} is found, besides the non-productive and less frequent \textit{-ar}, \textit{-š}, \textit{-bi}, \textit{-mi}, \textit{-ni}, \textit{-er} (<~\textit{-ar-i}), \textit{-iš} (<~\textit{-i-š}), \textit{-arč} (<~\textit{-ar-š}), and \textit{-erč} (<~\textit{ar-i-š}).\is{Plural formation}

\subsection{Grammatical cases} \label{corecase}

Tsova-Tush features a set of five grammatical cases and 24 spatial cases. The boundaries between  these two sets are fuzzy and somewhat arbitrary. The distinction made in this work is based on several criteria: grammatical cases are syntactically mostly used as verbal arguments and modifiers, they do not participate in a two-slot system morphologically and phonologically consist of a single consonant. Spatial cases, on the other hand, mostly convey spatial information, syntactically form adjunct phrases, mostly participate in a two-slot system and are phonologically mostly syllabic.

Table \ref{table-corecase} shows the core grammatical case markers which will be discussed briefly.

\begin{table}[h]
	\begin{tabular}{ll}
		\lsptoprule
		Case & Marker\\
		\midrule
		Nominative\footnote{The label Nominative is preferred over Absolutive in most Caucasological works.} & ∅  \\
		Ergative & \textit{-v, -s}  \\
		Genitive\footnote{Word-final Genitive \textit{-n} is dropped, nasalising the preceding vowel, see \sectref{processes}.} & \textit{-n}    \\
		Dative\footnote{Dative \textit{-n} is historically from \textit{-ni}, and does not drop in auslaut, see below in this section.} & \textit{-n}    \\
		Instrumental\footnote{See Section `Ergative' below in this section for a discussion on whether this is a separate case.} & \textit{-v}    \\
		\lspbottomrule
	\end{tabular}
	\caption{Tsova-Tush grammatial cases}
	\label{table-corecase}
\end{table}

The Nominative case\is{Nominative case} is used to signify the intransitive subject (\ref{simplenp-ex21a}), transitive object (\ref{simplenp-ex21b}) and nominal predicate (\ref{simplenp-ex21c}). Since the Nominative case involves zero marking, in this work, nouns in this case are glossed without the label \textsc{nom}. Many phonological processes can obscure the relation between the surface form of the Nominative noun (i.e. the bare stem) and its underlying phonological form, for which see \sectref{processes}.\pagebreak

\begin{exe}
	\ex\label{simplenp-ex21}
	\begin{xlist}
		

			\ex\label{simplenp-ex21a}
			\gll buħ-loħ \textbf{duq} \textbf{nax} d-av-iⁿ. \\
			war-{\Interess} \textbf{many} \textbf{people} {\D}-die-{\Aor} \\
			\trans `In the war, many people died.'
			\hfill (KK001-0032)

		

			\ex\label{simplenp-ex21b}
			\gll bac-bi-v \textbf{zorajš\u{\i}} \textbf{duq} \textbf{ougr-i} j-ak'-or. \\
			Tsova\_Tush-{\Pl}-{\Erg} \textbf{very} \textbf{many} \textbf{brick-{\Pl}} {\J}-bake-{{\Imprf}} \\
			\trans `The Tsova-Tush used to bake very many bricks.'
			\hfill (KK001-0021)

		

			\ex\label{simplenp-ex21c}
			\gll qa badr-e-ⁿ \textbf{ajtaɁ\u{o}} v-a-s\u{o}. \\
			three.{\Obl} child-{\Obl}-{\Gen} \textbf{godparent} {\M}.{\Sg}-be-{\Fsg}.{\Nom} \\
			\trans `I am a godfather to three children.' 
			\hfill (KK001-0061)

		
		
	\end{xlist}
\end{exe}



The Ergative case\is{Ergative case} is used for the subject of transitive verbs (\ref{simplenp-ex22a}, \ref{simplenp-ex22b}).\footnote{
	Additionally, with first and second person pronouns, the Ergative can be used with a particular set of intransitive verbs that allow both Nominative and Ergative subjects and with some intransitive verbs that require their first or second person subject to be in the Ergative (see \sectref{valency} and \cite{holisky87}).} The \textit{-s} suffix is used only in the singular for most nouns (and demonstratives) referring to humans, including proper nouns and personified animals (\cite{haukharris}). The \textit{-v} suffix is used for all other singular nouns, all plurals, and all demonstratives referencing non-human referents. This represents a fairly common type of Differential Subject Marking (see \cite{aikhenwalddixon2001dsm,hoopswart2008dsm}).\is{Humanness}\is{Differential Argument Marking}

\begin{exe}
	\ex\label{simplenp-ex22}
	\begin{xlist}
		

			\ex\label{simplenp-ex22a}
			\gll \textbf{bac-bi-v} zorajš\u{\i} duq ougr-i j-ak'-or. \\
			\textbf{Tsova\_Tush-{\Pl}-{\Erg}} very many brick-{\Pl} {\J}-bake-{\Imprf} \\
			\trans `The Tsova-Tush used to bake very many bricks.'
			\hfill (KK001-0021)

		

			\ex\label{simplenp-ex22b}
			\gll \textbf{ajkm-e-s} ab-i d-aɬ-i\textsuperscript{n} soⁿ. \\
			\textbf{doctor-{\Obl}-{\Erg}} pill-{\Pl} {\B}.{\Pl}-give-{\Aor} {\Fsg}.{\Dat} \\
			\trans `The doctor gave me pills.'
			\hfill (KK001-0002)

		
	\end{xlist}
\end{exe}

Note that the Instrumental case is also marked by \textit{-v}\is{Instrumental case} (see below in this section) (\cite{gagua48erg}). Since nouns that receive \textit{-s} in the Ergative all refer to humans, and hence are not attested with Instrumental case marking, another way of analysing this part of Tsova-Tush case morphology would be to divide nouns into two categories: (1) a small class consisting of most singular nouns referring to humans, which receive \textit{-s} in the Ergative, and which do not occur in Instrumental, and (2) all other nouns (including the plurals of nouns in category 1), which can be marked by \textit{-v}, indicating a single Ergative/Instrumental case.

The Genitive case\is{Genitive case}\is{Possession} is mostly used to express possession (\ref{simplenp-ex23a}), materials (\ref{simplenp-ex23b}) and part-whole relationships (\ref{simplenp-ex23c}). It is formed by adding the suffix \textit{-n} to the Oblique stem. Through regular phonological rules, in word-final position, the suffix is realised as nasalisation on the preceding vowel. See \sectref{Genitive} for a full discussion on Genitive modification within the noun phrase.

\begin{exe}
	\ex\label{simplenp-ex23}
	\begin{xlist}
		
		   
			\ex\label{simplenp-ex23a}
			\gll j-ax-eⁿ o \textbf{ħaš-e-ⁿ} matt-e. \\
			{\F}.{\Sg}-go-{\Aor} {\Dist} \textbf{guest-{\Obl}-{\Gen}} bed-{\Obl}({\Ess}) \\
			\trans `She went to that guest's bed.'
			\hfill (E179-92)
		   
		
		   
			\ex\label{simplenp-ex23b}
			\gll gam xeⁿ exk'-uin\u{\i} \textbf{ʕajħk'-e-ⁿ} gerc' d-a. \\
			chisel wood dig-{\Ptcp}.{\Npst} \textbf{iron-{\Obl}-{\Gen}} tool {\D}-be \\
			\trans `A chisel is an iron woodcarving tool.'
			\hfill (KK003-0677)
		
		
		
			\ex\label{simplenp-ex23c}
			\gll c'q'e lec'a-d-∅-oš \textbf{gutn-e-ⁿ} c'ʕop' st'en-ax-čo-x uill-d-is-eⁿ. \\
			once plough-{\D}-{\Tr}-{\Simul} \textbf{plough-{\Obl}-{\Gen}} tip what.{\Obl}-{\Indf}-{\Obl}-{\Cont} knock-{\D}-{\Lv}-{\Aor} \\
			\trans `While ploughing one day, the tip of the plough bumped into something.'
			\hfill (WS001-10.4)
		
		
	\end{xlist}
\end{exe}

The Dative case\is{Dative case} is used for the indirect object (\ref{simplenp-ex24a}) of a verb and for the subject of experiential verbs\is{Experiential verbs} (\ref{simplenp-ex24b}). Many postpositions govern the Dative case in nouns (\ref{simplenp-ex24c}).\is{Postpositions} The Dative case is formed by suffixing \textit{-n} to the root. Preceding vowels, however, are not nasalised by the Dative ending when it is in final position.\is{Nasalisation}\is{Umlaut} This can be explained historically, since it can be concluded clearly from older texts that the Dative ending must have been \textit{-ni} underlyingly. The \textit{i} was apocopated regularly (see \sectref{processes}), which caused umlaut of the preceding vowel (see Example (\ref{simplenp-ex24a}), where the form \textit{oquin\u{\i}} is still underlyingly \textit{oqu-ni}). However, the vowel of the ending \textit{-ni} never surfaces in contemporary Tsova-Tush, and some nouns do not undergo Umlaut in the dative (Example (\ref{simplenp-ex24d})). Therefore, the Dative case marker poses a problem for synchronic phonological analysis. In one analysis, there are two Dative case markers, \textit{-ni}, which triggers Umlaut, and \textit{-na} (or potentially \textit{-ne}), which does not, the choice of which is determined lexically. The final vowels never show up in surface forms. Alternatively, the Dative case ending is a uniform \textit{-n}, which defies the Tsova-Tush Nasalisation rule.

\begin{exe}
	\ex\label{simplenp-ex24}
	\begin{xlist}
		
		
		
		   
			\ex\label{simplenp-ex24a}
			\gll as laum-reⁿ v-eɁ-čeħ doliⁿ \textbf{oqui-n\u{\i}} šariⁿ k'ex j-aɬ-j-al-iⁿ. \\
			{\Fsg}.{\Erg} mountain-{\Abl} {\M}.{\Sg}-come-{\Ante} after \textbf{{\Dist}.{\Obl}-{\Dat}} {\Refl}.{\Poss} saddle\_tree {\J}-give-{\J}-{\Intr}-{\Aor} \\
			\trans `After I came back from the mountains, I gave him/her his/her own saddle tree.'
			\hfill (KK001-0114)
		
		
		   
			\ex\label{simplenp-ex24d}
			\gll q'ovel \textbf{načxo-n} cħa bui, ax k'il xruišoⁿ j-ec' ču j-oxk'-aⁿ. \\
			every \textbf{cheese-{\Dat}} one fist half kilo rock\_salt {\J}-need({\Npst}) in {\J}-put.{\Pfv}.{\Pl}-{\Inf} \\
			\trans `For every cheese, you have to put in a handful to half a kilo of rock salt.'
			\hfill (E005-34)
		
		
		
		   
			\ex\label{simplenp-ex24b}
			\gll \textbf{natesv-i-n=a} xaɁ-iⁿ, \textbf{mezobl-i-n=a} xaɁ-iⁿ me išt'-išt' d-a-r=en. \\
			\textbf{relative-{\Pl}-{\Dat}={\Add}} know.{\Pfv}-{\Aor} \textbf{neighbour-{\Pl}-{\Dat}={\Add}} know.{\Pfv}-{\Aor} {\Subord} so.{\Prox}-so.{\Prox} {\D}-be-{{\Imprf}}={\Quot} \\
			\trans `Both the relatives and the neighbours found out that it was like this.' \\
			\hfill (E255-70)
		
		
		   
			\ex\label{simplenp-ex24c}
			\gll \textbf{do-i-n}=mak=aɁ xabž-en-etx wumaɁ nax. \\
			\textbf{horse.{\Obl}-{\Pl}-{\Dat}}=on={\Emph} sit\_down-{\Aor}-{\Fpl}.{\Erg} all people \\
			\trans `We all mounted the horses.'
			\hfill (EK019-5.15)
		
		
	\end{xlist}
\end{exe}

The Instrumental case is mostly used for instruments (\ref{simplenp-ex27a}) and means of transportation (\ref{simplenp-ex27b}). 

\begin{exe}
	\ex\label{simplenp-ex27}
	\begin{xlist}
		
		\ex\label{simplenp-ex27a}
		\gll \textbf{cark'-i-v} daħ tet'-oš, daħ j-aɬ-eⁿ o ħun-a-x. \\
		\textbf{tooth-{\Pl}-{\Ins}} {\Pv} cut.{\Ipfv}-{\Simul} {\Pv} {\F}.{\Sg}-go\_out-{\Aor} {\Dist} forest-{\Obl}.{\Pl}-{\Cont} \\
		\trans `Biting with her teeth, she escaped those woods.'
		\hfill (E179-114)
		
		\ex\label{simplenp-ex27b}
		\gll uis-ren=a t'q'oɁ, \textbf{mankan-e-v} v-ex-n-as. \\
		there-{\Abl}={\Add} again, \textbf{car-{\Obl}-{\Ins}} {\M}.{\Sg}-go-{\Aor}-{\Fsg}.{\Erg} \\
		\trans `And from there again, I went by car.'
		\hfill (E275-17)
		
	\end{xlist}
\end{exe}



\subsection{Spatial cases} \label{spacase}\is{Ablative case}\is{Essive case}\is{Locative case}\is{Spatial cases}

As shown in Table \ref{table-spacase}, Tsova-Tush has 24 spatial cases, 20 of which are combinations of one of 4 possible locational suffixes (zero for neutral, \textit{-go} \textsc{ad} `at, near', \textit{-lo} \textsc{inter} `between, among, in', \textit{-i} \textsc{in} `in'), and one of 5 movement suffixes (zero for Lative, \textit{-ħ} Essive, \textit{-ren} Ablative, \textit{-\u{g}} Translative, \textit{-mcin} Terminative).
Additionally, an \textsc{apud}-series (`by, near') of case suffixes and a Contact case are found, which do not participate in the two-slot system of the other spatial cases. While not observing this morphological principle, the cases can nonetheless be considered spatial cases along phonological, semantic and syntactic criteria (see \sectref{corecase}).\footnote{The Contact case is special: on a semantic and syntactic level, it can mark both non-canonical objects (like grammatical cases) and spatial adjuncts (like spatial cases), it does not participate in a two-slot system (just as grammatical cases) morphologically, and, also like grammatical cases, is phonologically non-syllabic and mono-consonantal.}


\begin{table}
	\begin{tabular}{llllll}
    \lsptoprule
		& Lative &  {Essive} & {Ablative} & {Translative} &  {Terminative} \\
		\midrule
		Neutral & (\textit{-∅}) &  \textit{-ħ} &  \textit{-ren} & \textit{-\u{g}} & \textit{-mcin} \\
		
		\textsc{{ad}} &  \textit{-go}  & \textit{-go-ħ} &  \textit{-go-ren} & \textit{-go-\u{g}} &  \textit{-go-mcin} \\
		
		\textsc{{inter}} &  \textit{-lo} &  \textit{-lo-ħ} &  \textit{-lo-ren} & \textit{-lo-\u{g}} & \textit{-lo-mcin} \\
		
		\textsc{{in}} &  \textit{-i} &  \textit{-i-ħ}  & \textit{-i-ren} & \textit{-i-\u{g}} & \textit{-i-mcin} \\
		
		\midrule
		
		\textsc{{apud}}  &  \textit{-goħi} &  \textit{-cin}  & \textit{-xin} & & \\
		
		\textsc{{cont}} & \multicolumn{2}{c}{\textit{-x}} & & & \\
		\lspbottomrule
	\end{tabular}
	\caption{Tsova-Tush spatial cases}
	\label{table-spacase}
\end{table}


Many of these spatial cases (\textit{-ren, -xin, -mcin}) were analysed as postpositions\is{Postpositions} in earlier descriptions (\cite[168--170]{holiskygagua}), and it is important to distinguish between the two. In this work, postpositions are defined as words that follow nouns, with the noun governed by the postposition being in either the Dative case or a spatial case. That is, postpositions can only follow nouns bearing case inflection. The two morphs that cause the boundary between spatial cases and postpositions to be somewhat fuzzy are \textit{maka} `on' and \textit{k'ik'el} `under'. These elements can be used both as locational suffixes (since they can be added directly to the stem), but are also found as postpositions, used with nouns in the Dative case, as seen in Example (\ref{simplenp-ex48}). Note that \textit{mak} can also be used as a preverb. In what follows, reference is made to the Super-lative, Superessive, Superablative, Sublative, Subessive cases, but it should be kept in mind that these are recent or ongoing grammaticalisations.

\begin{exe}
	\ex\label{simplenp-ex48}
	\begin{xlist}
		
		
			\ex\label{simplenp-ex48a}
			\gll dapx-ire-č \textbf{don-e-n=mak} xaɁ-eⁿ. \\
			sweat.{\Obl}-{\Adjz}-{\Obl} \textbf{horse-{\Obl}-{\Dat}=on} sit-{\Aor} \\
			\trans `S/he sat on a sweaty horse.'
			\hfill (E193-4)
		
		
		
			\ex\label{simplenp-ex48b}
			\gll ši-l\u{g}e-č \textbf{don-mak} ʕe-v-a\u{g}-in st'ak' t'q'uiħ t'ot'-i axk'-in v-a. \\
			two-{\Ord}-{\Obl} \textbf{horse-{\Superlat}} sit\_down-{\M}.{\Sg}-{\Lv}-{\Ptcp}.{\Aor} man behind hand-{\Pl} tie-{\Ptcp}.{\Pst} {\M}.{\Sg}-be \\
			\trans `The man that mounted the second horse has his hands tied behind.' \\
			\hfill (E145-36)
		
		
	\end{xlist}
	
\end{exe}


The Tsova-Tush system of spatial cases in many ways resembles that of Daghestanian languages. In many of these languages, the same two suffixal slots can be identified: The first slot specifies the location of an item with respect to a reference point, while in the second slot, ``movement'' suffixes indicate the type of direction (or lack thereof). This can potentially lead to many combined spatial case suffixes, for example 42 in Inkhokwari\il{Khwarshi} (\cite{khalilova09}), or 112 in Tsez\il{Tsez} (\cite[103]{comriepolinsky98}). In some Daghestanian languages, these spatial case suffixes can in principle be combined with any noun stem, although most languages show preferences for some case suffixes to combine with a semantically defined subset of nouns, such as in Sanzhi Dargwa\il{Dargwa}, which distinguishes between animate nouns (usually marked with the \textsc{ad}-series) and inanimate nouns (usually marked with the \textsc{loc}-series) (\cite[64]{forker20}). In Tsova-Tush, the lexical semantics of a noun determine the locational suffix (\textsc{ad, inter, in}) used. The \textsc{ad}-suffixes are used primarily with singular animate nouns, \textsc{inter}-suffixes combine only with nouns denoting liquids, masses and collections, as well as with all plural nouns, and \textsc{in}-suffixes are used exclusively with certain toponyms, nouns denoting rooms, and the noun \textit{kalak} `city'.\footnote{This noun, a borrowing from Georgian, is most often used to refer to the city of Tbilisi.} Nouns that do not fit in any of these three categories use the ``neutral'' location, whereas the \textsc{apud}-series of case suffixes can combine with nouns from any of the above categories.\is{Differential Argument Marking}


In the following section, the Lative cases (the Allative, Interlative, Illative and Super-lative) will be discussed first, after which the Essive cases (the Essive, Adessive, Interessive, Inessive and Superessive), the Ablative cases (the Ablative, Adablative, Interablative, Superablative and Elative), the Translative cases (Translative, Adtranslative, Intertranslative and Intranslative), and the Terminative cases (the Terminative, Adterminative and Interterminative) will be discussed. The \textsc{apud}-series and the Contact case will then be discussed separately, after which special attention is devoted to the ``neutral'' Lative. It is once again important to note that regular phonological processes (see \sectref{processes}) apply to these endings, such that final \textit{-o} is reduced to \textit{-\u{o}} [w] or is apocopated completely, final \textit{-i} is apocopated, triggering i-umlaut on the preceding vowel, and final \textit{-ħ} is dropped, preventing the preceding vowel from apocopating.\footnote{In the older subcorpora AS and KK, final \textit{-ħ} is retained. In AS, most final vowels are retained.}

The Lative cases (Allative (\ref{simplenp-ex30a})\footnote{Original orthography of (\ref{simplenp-ex30a}): Samairleni baxke Jesuigo.}, Interlative (\ref{simplenp-ex30b}), Illative (\ref{simplenp-ex30c}) and Super-lative (\ref{simplenp-ex30d})) describe movement towards a reference point.\is{Lative case}\is{Allative case}\is{Interlative case}\is{Illative case}\is{Super-lative case}

\begin{exe}
	\ex\label{simplenp-ex30}
	\begin{xlist}
		
			\ex\label{simplenp-ex30a}
			\gll samair-len-i b-axk'-eⁿ \textbf{jesui-go}. \\
			Samaria-{\Adjz}-{\Pl} {\M}.{\Pl}-come.{\Pl}-{\Aor} \textbf{Jesus-{\All}} \\
			\trans `The Samarians came to Jesus.'\hfill\hbox{(AS001-1.30)}
		
		
		
			\ex\label{simplenp-ex30b}
			\gll bader \textbf{lav-e-l\u{o}} d-ax-iti-en=\u{e}, daħa pšel-d-i-eⁿ. \\
			child \textbf{snow-{\Obl}-{\Interlat}} {\D}-go-{\Caus}-{\Aor}=and {\Pv} cold-{\D}-{\Tr}-{\Aor} \\
			\trans `They let the child go into the snow, and let it get cold.'
			\hfill (KK004-1115)
		
		
		
			\ex\label{simplenp-ex30c}
			\gll xk'olix tab-ajrč\u{\i} \textbf{kalajk-\u{\i}} qel-\u{o} daħ d-oxk'-aⁿ. \\
			in\_summer gelded\_ram \textbf{city-{\Ill}} bring-{\Npst} {\Pv} {\D}-sell.{\Pfv}-{\Inf} \\
			\trans `In the summer, one brings the gelded rams to Tbilisi to sell them.' \\
			\hfill (KK004-1115)
		
		
		
			\ex\label{simplenp-ex30d}
			\gll joħ ču mič-x=ak' xeɁ-mak'-er, cħan šu-ciⁿ \textbf{keč-mak} beden=a?. \\
			girl {\Pv} where-{\Cont}={\Indf} sit\_down-{\Pot}-{\Imprf} {\Ptcl} 2{\Pl}-{\Apudess} \textbf{back\_of\_neck-{\Superlat}} except={\Emph} \\
			\trans `Where could the girl sit down, if not on the back of your necks?' \\
			\hfill (E058-21)
		
		
		
	\end{xlist}
\end{exe}

Additionally, the Allative case is used to signify addressees of speech verbs (\ref{simplenp-ex31}).\is{Addressee}

\begin{exe}
	\ex\label{simplenp-ex31}
	\gll \textbf{mgzavr-i-g} d-u\u{g}-a lat-er ``qor-i''. \\
	\textbf{traveller-{\Pl}-{\All}} {\D}-cry-{\Inf} do.{\Hab}-{\Imprf} apple-{\Pl} \\
	\trans `He used to shout at travellers: ``Apples!{''}'
	\hfill (E031-4)
\end{exe}

The Essive cases (Essive (\ref{simplenp-ex32a}), Adessive (\ref{simplenp-ex32b}), Interessive (\ref{simplenp-ex32c}) and Inessive (\ref{simplenp-ex32d})) describe a location at a certain reference point. The Essive cases end in \textit{-ħ}, which is deleted word-finally (see \sectref{processes}). The ``neutral'' Essive consists of only  \textit{-ħ} and when it is deleted, the gloss {\Ess} is written in parentheses attached to the previous morpheme, as per the conventions outlined in \sectref{conventions}.\is{Essive case}\is{Adessive case}\is{Interessive case}\is{Inessive case}

\begin{exe}
	\ex\label{simplenp-ex32}
	\begin{xlist}

			\ex\label{simplenp-ex32a}
			\gll ča-i-ⁿ ditx qexk'-d-∅-ora-l\u{o} d-aqqui-č\u{o} \textbf{herc'ni-ħ}.\\
			bear-{\Obl}-{\Gen} meat prepare-{\D}-{\Tr}-{\Imprf}-{\Sbjv} {\D}-big-{\Obl} \textbf{pot.{\Obl}-{\Ess}}\\
			\trans `Apparently they used to prepare bear meat in big pots.'
			\hfill (KK033-5408)
		
		
		
			\ex\label{simplenp-ex32b}
			\gll k'nat-i že-g\u{o} b-ax-en\u{e}, \textbf{že-go} b-ʕiɁ-ene šine-q batta-ħ, inc ma že-greⁿ ču b-a\u{g}-\u{o}.\\
			boy-{\Pl} sheep-{\All} {\M}.{\Pl}-go-{\Seq}, \textbf{sheep-{\Adess}} {\M}.{\Pl}-stay.{\Pfv}-{\Seq} two.{\Obl}-{\Approx} month.{\Obl}-{\Ess} now but sheep-{\Adabl} {\Pv} {\M}.{\Pl}-come-{\Npst}\\
			\trans `The boys went to the sheep, stayed with the sheep for about two months, but now they are coming (back) from the sheep.'
			\hfill (KK017-3195)
		
		
		
			\ex\label{simplenp-ex32c}
			\gll magram \textbf{bac-bi-lo} is d-aqqoⁿ ʕep d-a - co šeʒleba-l-a vir-e-n=mak st'ak' xaɁ-aⁿ.\\
			but \textbf{Tsova\_Tush-{\Pl}-{\Interess}} {\Med} {\D}-big shame {\D}-be {} {\Neg} can-{\Intr}-{\Npst} donkey-{\Obl}-{\Dat}=on man sit-{\Inf}\\
			\trans `But among the Tsova-Tush, that is a big sin: A man cannot sit on a donkey's back.'
			\hfill (E010-42)
		
		
		
			\ex\label{simplenp-ex32d}
			\gll \textbf{širk-i} magram deniɁ q'aħeⁿ dažar d-a.\\
			\textbf{Shiraki-{\Iness}} but generally bitter grass {\D}-be\\
			\trans `In Shiraki, however, the grass is generally bitter.'
			\hfill (E043-53)
		
		
	\end{xlist}
\end{exe}

Although a Superessive case is attested (\ref{simplenp-ex49a}), the meaning of a location on top of a reference point is usually conveyed by the postposition \textit{mak} following a noun in the Dative case (Example (\ref{simplenp-ex49b})).\is{Superessive case}

\begin{exe}
	\ex\label{simplenp-ex49}
	\begin{xlist}
		
		\ex\label{simplenp-ex49a}
		\gll lom-re-č \textbf{naq'-maka} vai-ⁿ dui teɬ-iš-x d-a. \\
		mountain-{\Adjz}-{\Obl} \textbf{road.{\Obl}-{\Superess}} {\Fpl}.{\Incl}-{\Gen} horse.{\Pl} be\_better-{\Simul}-{\Cmp} {\B}.{\Pl}-be \\
		\trans `On a mountain path, our horses are better.'
		\hfill (E043-109)
		
		\ex\label{simplenp-ex49b}
		\gll cħa d-aqqoⁿ zoraⁿ ħac'uk' d-a, \textbf{k'mat'u-n=mak} teg-o-d-∅ beⁿ. \\
		one {\D}-big brave bird {\D}-be \textbf{rock.{\Obl}-{\Dat}=on} make.{\Ipfv}-{\Npst}-{\D}-{\Tr} nest \\
		\trans `It's a big brave bird, it makes its nests on a rock.'
		\hfill (E042-273)
		
	\end{xlist}
\end{exe}

Additionally, the Adessive is used with all types of nouns in predicative possessive constructions, (\ref{simplenp-ex33}).\footnote{The Genitive case is never used for this type of construction, in contrast to Chechen and Ingush.}\is{Possession}



	\begin{exe}
		\ex\label{simplenp-ex33}
		\gll \textbf{qoqb-e-goħ} j-axxeⁿ, q'arc'eⁿ mu\u{g} j-a. \\
		\textbf{pheasant-{\Obl}-{\Adess}} {\J}-long colourful tail {\J}-be \\
		\trans `A pheasant has a long colourful tail.'
		\hfill (KK033-3195)
	\end{exe}


This case is also used with human nouns and with pronouns to signify a type of affectedness of a referent to a given event, as in  (\ref{simplenp-ex34}). In these cases, the Adessive marks a person that is emotionally invested in the action or event, but does not participate directly, also known as a benefactive, or malefactive\footnote{Sometimes called an ethical dative in European languages.}.\is{Benefactive}\is{Malefactive}\is{Dative case!ethical}

\begin{exe}
	\ex\label{simplenp-ex34}
	\begin{xlist}
		
		
			\ex\label{simplenp-ex34a}
			\gll cħajnčone met'r\u{o} daħ č'ʕa\u{g}-d-∅-oger \textbf{txo-go}. \\
			almost metro {\Pv} close-{\D}-{\Tr}-{\Iam}.{\Imprf} \textbf{{\Fpl}-{\Adess}} \\
			\trans `The metro almost closed on us.'
			\hfill (MM107-2.16)
		
		
		
		
		
			\ex\label{simplenp-ex34b}
			\gll j-aq'i-č maqo-v \textbf{bečv-e-go} cark' j-ʕog-iⁿ. \\
			{\J}-dry-{\Obl} bread-{\Erg} \textbf{poor\_thing-{\Obl}-{\Adess}} tooth {\J}-break.{\Pfv}-{\Aor} \\
			\trans `The dry bread broke the poor thing's tooth.'
			\hfill (WS001-2.12)
		
		
		
			\ex\label{simplenp-ex34c}
			\gll as is joħ šarn j-ik'-o-s \textbf{šu-goħ}=en. \\
			{\Fsg}.{\Erg} {\Med} girl away {\F}.{\Sg}-take.{\Anim}-{\Npst}-{\Fsg}.{\Erg} \textbf{2{\Pl}-{\Adess}}={\Quot} \\
			\trans `I will take that girl away from you.' \footnote{The malefactive \textit{šugoħ} is translated as `from you' by the compilers of corpus E.}
			\hfill (E153-31)
		
		
		
		
	\end{xlist}
\end{exe}



The Ablative cases (Ablative (\ref{simplenp-ex35a}), Adablative (\ref{simplenp-ex35b}), Interablative (\ref{simplenp-ex35c}), Elative (\ref{simplenp-ex35d}) and Superablative (\ref{simplenp-ex35e})\footnote{Original orthography of (\ref{simplenp-ex35e}):  Jeso K`rist aḥwose gornakma\'{k}re, mi\.{c}ḥe \'{t}ec̣die K`ristanul duila.}) are used to signal movement away from a reference point.\is{Ablative case}\is{Adablative case}\is{Interablative case}\is{Elative case}\is{Superablative case}

\begin{exe}
	\ex\label{simplenp-ex35}
	\begin{xlist}
		
			\ex\label{simplenp-ex35a}
			\gll as	\textbf{laum-reⁿ}	v-eɁ-čeħ doliⁿ oqui-n\u{\i}	šariⁿ	k'ex	j-aɬ-j-al-iⁿ.    \\
			{\Fsg}.{\Erg}	\textbf{mountain-{\Abl}}	{\M}.{\Sg}-come-{\Ante}	after	{\Dist}.{\Obl}-{\Dat}	{\Poss}.{\Refl}	saddle\_tree	{\J}-give-{\J}-{\Intr}-{\Aor}   \\
			\trans `After I came back from the mountains, s/he gave him/her a saddle tree.’
			\hfill (KK001-0114)
		
		
		
			\ex\label{simplenp-ex35b}
			\gll k'nat-i že-g\u{o} b-ax-en\u{e}, že-go b-ʕiɁ-en\u{e} šine-q batta-ħ, inc ma \textbf{že-greⁿ} ču b-a\u{g}-\u{o}. \\
			boy-{\Pl} sheep-{\All} {\M}.{\Pl}-go-{\Aor}.{\Seq}, sheep-{\Adess} {\M}.{\Pl}-stay.{\Pfv}-{\Aor}.{\Seq} two.{\Obl}-{\Approx} month.{\Obl}-{\Ess} now but \textbf{sheep-{\Adabl}} {\Pv} {\M}.{\Pl}-come-{\Npst}      \\
			\trans `The boys went to the sheep, stayed with the sheep for about two months, but now they are coming (back) from the sheep.'
			\hfill (KK017-3195)
		
		
		   
			\ex\label{simplenp-ex35c}
			\gll \textbf{ħun-lreⁿ} pšaj-š\u{\i} alzn-e-l\u{o} ix-\u{o}. \\
			\textbf{forest-{\Interabl}} tributary-{\Pl} Alazani-{\Obl}-{\Interlat} go.{\Ipfv}-{\Npst}    \\
			\trans `The tributaries flow out of the forest into the Alazani.'
			\hfill (KK022-3871)
		
		
		
			\ex\label{simplenp-ex35d}
			\gll \textbf{aln-ireⁿ} cħa so v-is-e-s\u{o}. \\
			\textbf{Alvani-{\Elat}} one {\Fsg}.{\Nom} {\M}.{\Sg}-remain-{\Aor}-{\Fsg}.{\Nom} \\
			\trans `I was the only one remaining from Alvani.'
			\hfill (E094-14) \\
		
		
		
			\ex\label{simplenp-ex35e}
			
			\gll jeso krist' aħ v-oss-eⁿ \textbf{gornak'-makreⁿ}, mič-ħ-e tec'-d-i-eⁿ krist'anul d-∅-uila. \\
			Jesus Christ down {\M}.{\Sg}-go\_down-{\Aor} \textbf{hill-{\Superabl}} where-{\Ess}-{\Rel} teach-{\D}-{\Tr}-{\Aor} Christian {\D}-do-{\Nmlz} \\
			\trans `Jesus Christ came down from (on top of the) hill where he taught Christian deeds.'
			\hfill (AS002-1.1)
		
		
	\end{xlist}
\end{exe}

The Translative case is used with nouns that express the result of a change of identity (\ref{simplenp-ex29a}, \ref{simplenp-ex29b}) and with nouns that signify a capacity in which the subject performs an action (\ref{simplenp-ex29c}). These constructions are known as depictive (\cite{himmelmannschultze}) or functive (\cite{creissels14functive}) phrases.\is{Translative case}\is{Depictives}\is{Functive}


\begin{exe}
	\ex\label{simplenp-ex29}
	\begin{xlist}
		
		
			\ex\label{simplenp-ex29a}
			\gll magram o nan t'q'oɁ \textbf{qa-e-\u{g}} j-erc'-iⁿ. \\
			but {\Dist} mother again \textbf{pig-{\Obl}-{\Trans}} {\F}.{\Sg}-turn-{\Aor} \\
			\trans `But that mother turned into a pig again.'
			\hfill (E179-113)
		
		
		\ex\label{simplenp-ex29b}
		\gll mor-i \textbf{picr-i-\u{g}=a\u{e}}, \textbf{svet'-i-\u{g}=a\u{e}} daħ j-arl-iⁿ. \\
		log-{\Pl} \textbf{plank-{\Pl}-{\Trans}={\Add}} \textbf{pole-{\Pl}-{\Trans}={\Add}} {\Pv} {\J}-cut-{\Aor} \\
		\trans `They cut the logs into planks and poles.'
		\hfill (KK004-1115)
		
		
			\ex\label{simplenp-ex29c}
			\gll so oqui-ciⁿ d-ʕivɁ šar-e v-a-ra-s \textbf{moǯamajgr-e-\u{g}}. \\
			{\Fsg}.{\Nom} {\Dist}.{\Obl}-{\Apudess} {\D}-four year-{\Obl}({\Ess}) {\M}.{\Sg}-be-{{\Imprf}}-{\Fsg}.{\Nom} \textbf{hired\_labourer-{\Obl}-{\Trans}} \\
			\trans `I was with him for four years as a hired labourer.'
			\hfill (E115-28)
		
		
	\end{xlist}
\end{exe}

The Translative case is also used with the postpositions \textit{aħ}, `down' \textit{daħ} `away', and \textit{ħal} `down' to convey the meanings `towards', `from' or `through', as in Example (\ref{simplenp-ex51a}). This same meaning is the only attested use of the Adtranslative (\ref{complexnp-ex51b}), Intertranslative (\ref{simplenp-ex51c}) and Intranslative (\ref{simplenp-ex51d}).\is{Adtranslative case}\is{Intertranslative case}\is{Intranslative case}

\begin{exe}
	\ex\label{simplenp-ex51}
	\begin{xlist}
		
		
			\ex\label{simplenp-ex51a}
			\gll qor \textbf{juq'-e-\u{g}=daħ} daħ tit'-n-as. \\
			apple \textbf{middle-{\Obl}-{\Trans}=through} {\Pv} cut.{\Pfv}-{\Aor}-{\Fsg}.{\Erg} \\
			\trans `I cut the apple through the middle.'
			\hfill (KK004-1115)
		
		
		
			\ex\label{complexnp-ex51b}
			\gll \textbf{so-go\u{g}=\={a}} b-a-r o niq' ħal b-aɁ-uin. \\
			\textbf{{\Fsg}-{\Adtrans}=down} {\B}.{\Sg}-be-{\Imprf} {\Dist} road {\Pv} {\B}.{\Sg}-come.{\Pfv}-{\Ptcp}.{\Npst} \\
			\trans `That road was directed down towards me.'
			\hfill (E145-20)
		
		
		
			\ex\label{simplenp-ex51c}
			\gll mak qaxk'-uš do-i-n o \textbf{lav-e-lo\u{g}=da} daħ co d-et'-mak'-\u{e}, ču ploba-l-a.  \\
			on\_top hang.{\Pl}-{\Simul} horse.{\Obl}-{\Pl}-{\Dat} {\Dist} \textbf{snow-{\Obl}-{\Intertrans}=through}  {\Pv} {\Neg} {\D}-run-{\Pot}-{\Npst} {\Pv} sink-{\Intr}-{\Npst} \\
			\trans `When they are loaded, the horses cannot run through that snow, they sink.'
			\hfill (EK005-15.1)
		
		
		
			\ex\label{simplenp-ex51d}
			\gll \textbf{t'batn-i\u{g}=\={a}} aħo v-ik'-nor, buis\u{u}.   \\
			\textbf{Tbatana-{\Intrans}=down} down {\M}.{\Sg}-take.{\Anim}-{\Nw}.{\Rem} at\_night \\
			\trans `They (apparently) took him down to Tbatana at night.'
			\hfill (E146-28)
		
		
	\end{xlist}
\end{exe}


The Terminative cases (Terminative, Adterminative, Interterminative, Interminative) signify a motion up until a reference point. With these cases, the picture is less clear-cut than with the other spatial cases. Firstly, the Interterminative case suffix \textit{-lomcin} is also used as a postposition meaning `until', which is cliticised to adverbs and even to verb stems, where it forms temporal adjunct clauses (see \sectref{temp}). Furthermore, besides the Interminative case (which is indeed restricted to the same subset of nouns as the other \textsc{in}-cases, see Example (\ref{simplenp-ex47e})) the Terminative cases do not adhere exactly to the semantic rules laid out at the beginning of this section. That is, the Interterminative is not used with nouns denoting liquids, masses, groups, or with plural nouns, but is used exclusively with nouns denoting points in time and periods (see Example \ref{simplenp-ex47a}). Personal pronouns and nouns referring to humans expectedly take the Adterminative case (\ref{simplenp-ex47b}), and all other nouns can take either the Terminative or the Adterminative case (\ref{simplenp-ex47c}, \ref{simplenp-ex47d}). A noun in the Superterminative case has not been attested.\is{Terminative case}\is{Postpositions}


\begin{exe}
	\ex\label{simplenp-ex47}
	\begin{xlist}
		
			\ex\label{simplenp-ex47a}
			\gll goneb sa\u{g} j-a-r ai t'q'ujsinlu-č pxi-it't' \textbf{c'ut-e-lomciⁿ}. \\
			mind fit {\J}-be-{\Imprf} {\Deict} last-{\Obl} five-ten \textbf{minute-{\Obl}-{\Interterm}} \\
			\trans `Her mind was clear up until her last fifteen minutes.'
			\hfill (E224-26)
		
		
		
			\ex\label{simplenp-ex47b}
			\gll ħatx=da seⁿ voħ v-aq-iⁿ ħal, qeⁿ \textbf{so-gomciⁿ} v-eɁ-eⁿ.\\
			in\_front=from {\Fsg}.{\Gen} boy {\M}.{\Sg}-take-{\Aor} up then \textbf{{\Fsg}-{\Adterm}} {\M}.{\Sg}-come-{\Aor}\\
			\trans `First he lifted my boy up, then he came up to me.'
			\hfill (E060-18)
		
		
		
			\ex\label{simplenp-ex47c}
			\gll ħal lil-n-atx o sak'ist'o \textbf{ʒir-e-mciⁿ}. \\
			up walk.{\Ipfv}-{\Aor}-{\Fpl}.{\Erg} {\Dist} Sakisto \textbf{base-{\Obl}-{\Term}} \\
			\trans `We walked up to the foot of that Sakisto.'
			\hfill (E197-16)
		
		
		
			\ex\label{simplenp-ex47d}
			\gll \textbf{juq'-gomciⁿ} ʕarč'iⁿ k'oc'l-i j-at'-er. \\
			\textbf{middle-{\Adterm}} black braid-{\Pl} {\J}-lie\_around-{\Imprf} \\
			\trans `Black braids were hanging up to her waist.'
			\hfill (KK011-2121)
		
		
		
			\ex\label{simplenp-ex47e}
			\gll o-bi \textbf{t'batn-imciⁿ} b-ax-en\u{o} b-ux=aɁ b-erc'-iⁿ. \\
			{\Dist}-{\Pl} \textbf{Tbatana-{\Interm}} {\M}.{\Pl}-go.{\Pfv}-{\Ptcp}.{\Aor} {\M}.{\Pl}-back={\Emph} {\M}.{\Pl}-turn-{\Aor} \\
			\trans `Having gone up until Tbatana, they turned back.'
			\hfill (\cite[72]{desheriev53})
		
		
	\end{xlist}
\end{exe}



The Apudlative case is used to signal a general movement towards a reference point (\ref{simplenp-ex36}), whereas the Apudablative case signifies a general movement away from a reference point, as in (\ref{simplenp-ex45}).\is{Apudlative case}\is{Apudablative case}\is{Directional case}

\begin{exe}
	
		\ex\label{simplenp-ex36}
		\gll nips \textbf{sinatl-e-guiħ} v-ax-eⁿ. \\
		straight \textbf{light-{\Obl}-{\Apudlat}} {\M}.{\Sg}-go-{\Aor} \\
		\trans `He went directly towards the light.'
		\hfill (WS001-12.30)
	
\end{exe}



\begin{exe}
	
		\ex\label{simplenp-ex45}
		\gll \textbf{kuirc'l-e-xiⁿ} d-a\u{g}-o-tx aħo=e. \\
		\textbf{wedding-{\Obl}-{\Apudabl}} {\D}-come-{\Npst}-{\Fpl} down=and \\
		\trans `We are coming down from the wedding.'
		\hfill (E157-9)
	
\end{exe}


The Apudessive case (often called comitative case) is used to express the meaning `with, near, alongside' and is often found with nouns denoting humans (\ref{simplenp-ex26a}), although not necessarily (\ref{simplenp-ex26b}). It is formed by suffixing \textit{-cin} to the Oblique stem (surfacing as \textit{-ciⁿ} in word-final position).\is{Apudessive case}\is{Comitative case}

\begin{exe}
	\ex\label{simplenp-ex26}
	\begin{xlist}
		
		
			\ex\label{simplenp-ex26a}
			\gll \textbf{badr-i-ciⁿ} osi lejp'c'-ra-ħ=e. \\
			\textbf{child-{\Pl}-{\Apudess}} there play.{\Ipfv}-{\Imprf}-{\Ssg}=and \\
			\trans `You used to play there with the children.'
			\hfill (E130-8)
		
		
		
			\ex\label{simplenp-ex26b}
			\gll at-c'iⁿ bo duq-čui-š-v \textbf{maq-ciⁿ} čamli-š leħ-\u{o}. \\
			become\_soft-{\Priv} garlic many-{\Obl}-{\Pl}-{\Erg} \textbf{bread-{\Apudess}} tasty-{\Adv} eat\_alongside-{\Npst} \\
			\trans `Many peoply like eating raw garlic with their bread.' (Lit.: Many people tastily put uncrushed garlic alongside bread.')
			\hfill (KK001-0057)
		
		
	\end{xlist}
\end{exe}


The Tsova-Tush Contact case is formed with the suffix \textit{-x} and specifies the point of contact of a given action, such as `hit' (\ref{simplenp-ex25a}). It is also used as the complement of some verbs, such as `fear' (\ref{simplenp-ex25b}) and `ask', and as the standard of comparison (\ref{simplenp-ex25c}). A verbal noun in the Contact case can be used to form causal clauses (see \sectref{caus}).\is{Contact case}\is{Comparison}

\begin{exe}
	\ex\label{simplenp-ex25}
	\begin{xlist}
		
		
			\ex\label{simplenp-ex25a}
			\gll vorɬ-eɁ ħar-e-ⁿ qer dev-i-n \textbf{korti-x}.  \\
			seven-{\Incl} mill-{\Obl}-{\Gen} stone demon-{\Pl}-{\Dat} \textbf{head.{\Obl}-{\Cont}}\\
			\gll b-iš-b-i-eⁿ. \\
			{\B}.{\Sg}-strike-{\B}.{\Sg}-{\Tr}-{\Aor} \\
			\trans `He hit the demons on their head with all seven millstones.' (Lit.: `He hit all seven millstones onto the demon's head.')
			\hfill (WS001-13.4)
		
		
		   
			\ex\label{simplenp-ex25b}
			\gll sicx-xorš-ale-č\u{o} \textbf{hav-e-x} qerɬ-i-s\u{o}. \\
			heat-malaria-{\Adjz}-{\Obl} \textbf{climate-{\Obl}-{\Cont}} fear-{\Npst}-{\Fsg}.{\Nom} \\
			\trans `I am afraid of a malaria climate.'
			\hfill (KK035-5515)
		
		
		   
			\ex\label{simplenp-ex25c}    
			\gll mit'\u{o} \textbf{k'ot'-e-x} čaq hallu-vx v-a. \\
			Mito \textbf{Kote-{\Obl}-{\Cont}} far quiet-{\Cmp} {\M}.{\Sg}-be \\
			\trans `Mito is far more peaceful than Kote.'
			\hfill (KK035-5520)
		
		
		
	\end{xlist}
\end{exe}


If we consider Table \ref{table-spacase} again, we can clearly see that there is no dedicated form to signal the ``neutral'' Lative. That is, there is no Lative case suffix for nouns that do not belong to one of the three semantic subsets described above (animates, liquids/masses, empty spaces). Tsova-Tush fills this apparent lacuna in different ways, using three distinct strategies.  The first strategy uses a zero-marked Oblique nominal stem with lative semantics, which morphologically would be the expected outcome, since the exponents of both the Lative and the ``neutral'' series of spatial markers are zero. See Example (\ref{simplenp-ex37}) for this strategy, which can be considered archaic, only attested with very few nouns (e.g. \textit{lamu} `mountain(s)', \textit{bare} `valley', \textit{c'eni} `house'). Not only is a zero ending the expected form from a structuralist point of view, it is also typologically very common to find a zero-ending Lative for words that occur most frequently in spatial cases (\cite{haspeltmath2019dpm}).\is{Lative case}

Note that the Essive cases, too, often do not carry an overt marker. This, however, is a phonological process, rather than a morphological fact. Compare the Nominative, Essive and Lative cases of the noun \textit{bar} `valley' in Table \ref{table-essivelative}. See Sections \ref{sources} and \ref{processes} for the diachronic aspects of the phonological processes.


\begin{table}
	\begin{tabular}{lllll}
\lsptoprule
			& {Morphemes}	& 				& \multicolumn{2}{c}{{Surface form}}  \\
			&						&				& {Pre-1980} 	& {Contemporary} \\
                    \midrule
Nominative  & \textit{bar}-∅		& \rightarrow 	& \textit{bar}		 	& \textit{bar} \\
			& valley-{\Nom} 		& 				& 						& \\
			& 						& 				& 						& \\
Lative		& \textit{bar-e}-∅		& \rightarrow 	& \textit{bar\u{e}}		& \textit{bar} \\
			& valley-{\Obl}-{\Lat}	& 				& 						& \\ 
			& 						& 				& 						& \\
Essive	    & \textit{bar-e-ħ}		& \rightarrow 	& \textit{bareħ} 		& \textit{bare} \\
			& valley-{\Obl}-{\Ess}	& 				& 						& \\ 
	\lspbottomrule
    \end{tabular}
	\caption{Similar case markers}
	\label{table-essivelative}
\end{table}

\begin{exe}
	\ex\label{simplenp-ex37}
	\begin{xlist}
		
		\ex\label{simplenp-ex37a}
		\gll \textbf{laum\u{u}} d-ot'-uš k'ak'al en-e-ħ daħ tiv-n-atx. \\
		\textbf{mountain.{\Lat}} {\D}-go-{\Simul} walnut shadow-{\Obl}-{\Ess} {\Pv} rest-{\Aor}-{\Fpl}.{\Erg} \\
		\trans `When we were going to the mountains, we rested in the shadow of a walnut tree.'
		\hfill (KK005-1273)
		
		\ex\label{simplenp-ex37b}
		\gll lamu-ħ lac'-aⁿ v-ol-in\u{o} ʕuv \textbf{bar-\u{e}} aħ v-ik'-eⁿ. \\
		mountain-{\Ess} hurt-{\Inf} {\M}.{\Sg}-begin-{\Ptcp}.{\Aor} shepherd \textbf{valley-{\Obl}({\Lat})} down {\M}.{\Sg}-take-{\Aor} \\
		\trans `They took a shepherd that had fallen ill in the mountain down to the valley.'
		\hfill (KK001-0359)
		
	\end{xlist}
\end{exe}


Another strategy is using the Allative case from the \textsc{ad}-series in the ``neutral'' series, as in (\ref{simplenp-ex38}).\is{Allative case}

\begin{exe}
	\ex\label{simplenp-ex38}
	\begin{xlist}
		
		\ex\label{simplenp-ex38a}
		\gll lap' \textbf{k'edl-e-g\u{o}} daħ ott-b-i-eⁿ. \\
		ladder \textbf{wall-{\Obl}-{\All}} {\Pv} place-{\B}.{\Sg}-{\Tr}-{\Aor} \\
		\trans `They put a ladder up against the wall.'
		\hfill (KK004-1115)
		
		\ex\label{simplenp-ex38b}
		\gll \textbf{q'ajrps-e-g\u{o}} dok' ep'c'-\u{o} seⁿ. \\
		\textbf{watermelon-{\Obl}-{\All}} heart reach-{\Npst} {\Fsg}.{\Gen} \\
		\trans `I'm yearning for watermelon.' (Lit. `My heart is reaching towards watermelon.')
		\hfill (KK004-1155)
		
		\ex\label{simplenp-ex38c}
		\gll \textbf{ak'ošk'-i-g} j-ex-n-as. \\
		\textbf{small\_window-{\Pl}-{\All}} {\F}.{\Sg}-go.{\Pfv}-{\Aor}-{\Fsg}.{\Erg} \\
		\trans `I went up to the small windows.'
		\hfill (E098-7)
		
	\end{xlist}
\end{exe}

The third and most common strategy, however, is using the ``neutral'' Essive case \textit{-ħ}  as the ``neutral'' Lative, for which see Example (\ref{simplenp-ex39}).\is{Essive case}

\begin{exe}
	\ex\label{simplenp-ex39}
	\begin{xlist}
		
		\ex\label{simplenp-ex39a}
		\gll k'alinin-e-reⁿ d-a\u{g}-or \textbf{barnaul-e}. \\
		Kalinino-{\Obl}-{\Abl} {\D}-come-{\Imprf} \textbf{Barnaul-{\Obl}({\Ess})} \\
		\trans `They were coming to Barnaul from Kalinino.'
		\hfill (E275-42)
		
		\ex\label{simplenp-ex39b}
		\gll eq \textbf{koco-ħ} ʕarč'iⁿ veⁿ d-ajtt-\u{u}.\\
		{\Prox}.{\Obl} \textbf{wine\_jar-{\Ess}} black wine {\D}-be\_poured-{\Npst} \\
		\trans `The red wine is being poured into this wine jar.'
		\hfill (KK001-0052)
		
		\ex\label{simplenp-ex39c}
		\gll j-ax-eⁿ o ħaš-e-ⁿ \textbf{matt-e}. \\
		{\F}.{\Sg}-go.{\Pfv}-{\Aor} {\Dist} guest-{\Obl}-{\Gen} \textbf{bed-{\Obl}({\Ess})} \\
		\trans `She went to that guest's bed.'
		\hfill (E179-92)
		
	\end{xlist}
\end{exe}


Parallels to this third strategy as seen in Example (\ref{simplenp-ex39}), i.e. using the Essive case in a lative function, can be seen in the following examples, where other Essive cases are used in a lative function. In (\ref{simplenp-ex41a}), the Inessive \textit{t'batni} is used, as opposed to the Illative \textit{t'batin}, while in (\ref{simplenp-ex41b}), the Interessive \textit{xilo} is attested, where an Interlative \textit{xil\u{o}} is expected.\is{Locative case}

\begin{exe}
	\ex\label{simplenp-ex41}
	\begin{xlist}
		
		\ex\label{simplenp-ex41a}
		\gll c'q'e ič\u{g}ar gogl-e-s v-ik'-en-es \textbf{t'batn-i}. \\
		once Ichoant Gogale-{\Obl}-{\Erg} {\M}.{\Sg}-take.{\Anim}-{\Aor}-{\Fsg}.{\Erg} \textbf{Tbatana-{\Iness}} \\
		\trans `Once, Gogale Ichoant took me to Tbatana.'
		\hfill (E288-116)
		
		\ex\label{simplenp-ex41b}
		\gll karcx-olen=aɁ ču \textbf{xi-lo} v-oɬ-iⁿ. \\
		clothes-{\Adjz}={\Emph} in \textbf{water-{\Interess}} {\M}.{\Sg}-put-{\Aor} \\
		\trans `They put him into the water with his clothes on.'
		\hfill (E309-56)
		
	\end{xlist}
\end{exe}

This lative function of otherwise Essive cases accounts for 10 percent of all instances of the Interessive, Inessive and Superessive cases in the ECLING corpus, which represents the largest source of contemporary colloquial Tsova-Tush. This use is not observed in the subcorpora that feature earlier stages of Tsova-Tush (IT, AS and KK).\is{Georgian influence!Morphological}
In Georgian, the spatial cases (which are traditionally analysed as postpositions) show similarities with the apparent essive/lative merger in Tsova-Tush. In Georgian, there is no formal distinction between the essive and the lative function of the case endings \textit{-ši} `in', \textit{-tan} `near' and \textit{-ze} `on'. Compare Example (\ref{simplenp-ex42}), where the Georgian \textsc{in}-Locative (\ref{simplenp-ex42a}, \ref{simplenp-ex42b}) and \textsc{super}-Locative (\ref{simplenp-ex42c}, \ref{simplenp-ex42d}) cases show both an essive (\ref{simplenp-ex42a}, \ref{simplenp-ex42c}) and a lative (\ref{simplenp-ex42b}, \ref{simplenp-ex42d}) function.

\begin{exe}
	\ex\label{simplenp-ex42}
	Georgian
	\begin{xlist}
		
		
		\ex\label{simplenp-ex42a}
		\gll me-rv-e tavis \textbf{ba\u{g}-ši} q'vavil-eb-s rc'q'avs. \\
		{\Ord}-eight-{\Ord} {\Refl}.{\Poss} \textbf{garden-{\In}} flower-{\Pl}-{\Dat} s/he\_waters\_sth \\
		\trans `The eighth one waters the flowers in his/her own garden.' \\
		\hfill (GNC: E. Akhvlediani)
		
		\ex\label{simplenp-ex42b}
		\gll čveulebriv sabavšvo \textbf{ba\u{g}-ši} miviq'vanet. \\
		as\_usual children's \textbf{garden-{\In}} we\_took\_sb \\
		\trans `As usual, we took him/her/them to the kindergarten.' \\
		\hfill (GNC: E. Akhvlediani)
		
		\ex\label{simplenp-ex42c}
		\gll \textbf{mic'a-ze} vart tu mic'-is kveš. \\
		\textbf{ground-{\Super}} we\_are or ground-{\Gen} under \\
		\trans `We are on the ground or under it.'
		\hfill (GNC: O. Chiladze)
		
		\ex\label{simplenp-ex42d}
		\gll \textbf{mic'a-ze} da-\u{g}vr-ili zet-i=vit tu mazut-i=vit. \\
		\textbf{ground-{\Super}} {\Pv}-pour-{\Ptcp}.{\Aor} oil-{\Nom}=like or petrol-{\Nom}=like \\
		\trans `spilled onto the ground like oil or petrol'
		\hfill (GNC: O. Chiladze)
		
	\end{xlist}
\end{exe}

Since essive cases with lative semantics have not been found in Chechen and Ingush, and since this feature is a recent development in Tsova-Tush, it can be best described as an instance of contact-induced change due to contact with Georgian.



\subsection{Oblique stems and declension classes} \label{Oblique}

Tsova-Tush features several declension classes, distinguished mainly by their formation of the Oblique stem. The Oblique stem is the stem to which all case endings other than the Nominative are attached. In some declension classes, such as those containing most nouns ending in \textit{-u} and \textit{-o}, there is no morphological distinction between Nominative and Oblique. In these instances, whatever formal difference is observed between the Nominative and Oblique stems (e.g. \textit{mouq\u{u}} and \textit{maqu-} in Table \ref{table-ostems}) is  caused purely by phonological processes, as presented in \sectref{processes}. The only morphologically irregular aspect of this paradigm is the formation of the Genitive and Dative cases, which are discussed in \sectref{corecase} above.\is{Declension}\is{Oblique stems}


\begin{table}
	\begin{tabular}{lllll}
    \lsptoprule
		& {\textit{o-}stems} & `chair' & {\textit{u-}stems} & `razor' \\
		& Morphemes & Surface & Morphemes & Surface \\
		\midrule
		Nominative & \textit{čak'o} & \textit{čak'\u{o}} & \textit{maqu} & \textit{mouq\u{u}} \\
		
		Ergative/Instrumental & \textit{čak'o-v} & \textit{čak'ov} & \textit{maqu-v} & \textit{maquv} \\
		
		Genitive & \textit{čak'ui-n} & \textit{čak'uiⁿ} & \textit{maqui-n} & \textit{maquiⁿ}  \\
		
		Dative & \textit{čak'ui-n(i)} &  \textit{čak'uin} &  \textit{maqui-n(i)} & \textit{maquin} \\
		
		Spatial (Contact) & \textit{čak'o-x} & \textit{čak'ox} & \textit{maqu-x}& \textit{maqux} \\
        \lspbottomrule
	\end{tabular}
	\caption{Tsova-Tush \textit{o-} and \textit{u-}stems}
	\label{table-ostems}
\end{table}

Most nouns ending in a consonant form the Oblique stem by adding the vowel \textit{-e}, as in Table \ref{table-cstems}. Nouns historically ending in \textit{-e} also belong to this class, since \textit{-e} is dropped without a trace word-finally (although still written as \textit{-\u{e}} in some sources).

\begin{table}
	\fittable{%
	\begin{tabular}{lllll}
    \lsptoprule
		& {C-stems} & `rheumatism' & {\textit{e-}stems} & `calf' \\
		& Morphemes & Surface & Morphemes & Surface \\
		\midrule
		Nominative & \textit{an} & \textit{aⁿ} & \textit{as(e)} & \textit{as} \\
		
		Ergative/Instrumental & \textit{an-e-v} & \textit{anev} & \textit{as(-)e-v} & \textit{asev} \\
		
		Genitive & \textit{an-e-n} & \textit{aneⁿ} & \textit{as(-)e-n} & \textit{aseⁿ} \\
		
		Dative & \textit{an-e-n} & \textit{anen} & \textit{as(-)e-n} & \textit{asen} \\
		
		Spatial (Contact) & \textit{an-e-x} & \textit{anex} & \textit{as(-)e-x} & \textit{asex} \\
        \lspbottomrule
	\end{tabular}}
	\caption{Tsova-Tush C- and \textit{e-}stems}
	\label{table-cstems}
\end{table}

Some nouns display an inflectional pattern with a combination of C-stem and \textit{u}-stem behaviour: their Nominative ends in a consonant and the Oblique stem is formed by adding \textit{-u} (see Table \ref{table-custems}). 

\begin{table}
	\begin{tabular}{lllll}
    \lsptoprule
		& {C/\textit{u}-stems} & `silver' &  & `mountain' \\
		& Morphemes & Surface & Morphemes & Surface \\
		\midrule
		Nominative & \textit{t'ateb} & \textit{t'ateb} & \textit{lam} & \textit{lam} \\
		
		Ergative/Instrumental & \textit{t'atebu-v} & \textit{t'atbuv} & \textit{lamu-v} & \textit{lamuv} \\
		
		Genitive & \textit{t'atebu-n} & \textit{t'atbuⁿ} & \textit{lamu-n} & \textit{lamuⁿ} \\
		
		Dative & \textit{t'atebu-n} & \textit{t'atbun} & \textit{lamu-n} & \textit{lamun} \\
		
		Spatial (Contact) & \textit{t'atebu-x} & \textit{t'atbux} & \textit{lamu-x} & \textit{lamux} \\
        \lspbottomrule
	\end{tabular}
	\caption{Tsova-Tush C-/\textit{u-}stems}
	\label{table-custems}
\end{table}



Many inherited Tsova-Tush nouns have a declension paradigm that features ablaut of the second vowel of the root (see Table \ref{table-ablautstems1}), while others have a paradigm that features ablaut of both the first and second vowel of the root (see \tabref{table-ablautstems2})\is{Ablaut!Nominal}

\begin{table}
	\begin{tabular}{lllll}
    \lsptoprule
		&  & `bread' &  & `head' \\
		& Morphemes & Surface & Morphemes & Surface \\
		\midrule
		Nominative & \textit{maqi} & \textit{mejq} & \textit{korto} & \textit{kort\u{o}} \\
		
		Ergative/Instrumental & \textit{maqo-v} & \textit{maqov} & \textit{korti-v} & \textit{kortiv} \\
		
		Genitive & \textit{maqo-n} & \textit{maqoⁿ} & \textit{korti-n} & \textit{kortiⁿ} \\
		
		Dative & \textit{maqo-n} & \textit{maqon} & \textit{korti-n} & \textit{kortin} \\
		
		Spatial (Contact) & \textit{maqo-x} & \textit{maqox} & \textit{korti-x} & \textit{kortix} \\
        \lspbottomrule
	\end{tabular}
	\caption{Tsova-Tush ablauting declension 1}
	\label{table-ablautstems1}
\end{table}


\begin{table}
	\fittable{
	\begin{tabular}{lllll}
    \lsptoprule
		&  & `heart' &  & `sister-in-law' \\
		& Morphemes & Surface & Morphemes & Surface \\
		\midrule
		Nominative & \textit{dok'} & \textit{dok'} & \textit{nus} & \textit{nus} \\
		
		Ergative/Instrumental & \textit{dak'a-v} & \textit{dak'av} & \textit{nasi-s} & \textit{nasis} \\
		
		Genitive & \textit{dak'i-n} & \textit{dak'iⁿ} & \textit{nasi-n} & \textit{nasiⁿ} \\
		
		Dative & \textit{dak'a-n} & \textit{dak'an} & \textit{nasi-n} & \textit{nasin} \\
		
		Spatial (Contact) & \textit{dak'o-x} & \textit{dak'ox} & \textit{nasi-x} & \textit{nasix} \\
        \lspbottomrule
	\end{tabular}}
	\caption{Tsova-Tush ablauting declension 2}
	\label{table-ablautstems2}
\end{table}


The most common pattern is a Nominative in C\textsubscript{1}\textit{i/u/o}C\textsubscript{2}, Oblique in C\textsubscript{1}\textit{a}C\textsubscript{2}\textit{i/u/o}. For more examples of ablauting nouns (both with first vowels and second vowels) see \textcite[9]{haukharris}, \textcite[68]{desheriev53}, \textcite[107]{mikeladze11} for further synchronic description and see \textcite{schrijver21nakh} for their historical development.


\subsection{Morphological adaptation of borrowed nouns} \label{morphadapt}\is{Loanword adaptation!Nouns}

\subsubsection{Declension class} \label{decladapt} 

Georgian nouns end in either a vowel \textit{a, e, o, u}, or in a consonant, in which case the noun takes a Nominative case suffix \textit{-i} in Georgian. Examples include \textit{anteba} `inflammation', \textit{q'ava} `coffee', \textit{mepe} `king', \textit{cixe} `fortress', \textit{ǯado} `magic', \textit{q'aq'ačo} `poppy', \textit{alču} `knucklebone (game)', \textit{ru} `channel', \textit{bal-i} `cherry', \textit{pandur-i} `panduri (musical instrument)'.

Georgian nouns are borrowed into Tsova-Tush in their stem-form only, that is, consonant-ending nouns are borrowed without the Nominative marker \textit{-i}. Georgian nouns ending in \textit{u} or \textit{o} are inflected like Tsova-Tush regular \textit{u/o-}stem nouns, as seen in Table \ref{table-ostems}, without a morphological distinction between Nominative and Oblique stem (see Table \ref{table-ostemsgeo}).\footnote{Monosyllabic nouns in \textit{u} are written with an additional ending \textit{-v} in the Nominative in \textcite{kadkad84}. Since word-final \textit{-v} is pronounced as a bilabial [w], this can also be seen as a type of monosyllabic lengthening.} 

\begin{table}
	\begin{tabular}{lllll}
		\lsptoprule
		& {\textit{o-}stems}  & & {\textit{u}-stems}  &  \\
		from Georgian & \textit{ǯado}  & \textit{q'aq'ačo} & \textit{alču}  & \textit{ru}  \\
		              & `magic'        &  `poppy' & `knucklebone' & `channel'\\
		\midrule
		
		Nominative & \textit{ǯad\u{o}} & \textit{q'aq'ač\u{o}} & \textit{alč\u{u}} & \textit{ruv} \\
		
		Ergative/Instrumental & \textit{ǯadov} & \textit{q'aq'čov} & \textit{alčuv} & \textit{ruv} \\
		
		Genitive & \textit{ǯaduiⁿ} & \textit{q'aq'čuiⁿ} & \textit{alčuiⁿ} & \textit{ruiⁿ} \\
		
		Dative & \textit{ǯaduin} & \textit{q'aq'čuin} & \textit{alčuin} & \textit{ruin} \\
		
		Spatial (Contact) & \textit{ǯadox} & \textit{q'aq'čox} & \textit{alčux} & \textit{rux} \\
        \lspbottomrule
	\end{tabular}
	\caption{Georgian borrowings in \textit{-o} or \textit{-u}}
	\label{table-ostemsgeo}
\end{table}

Georgian nouns ending in a consonant are borrowed into the regular consonant\hyp ending declension (see Table \ref{table-cstems}). Since final \textit{-e} and \textit{-a} are regularly deleted word-finally (see \sectref{processes}), there is no formal distinction between \textit{a}/\textit{e}-stems and consonant stems (see Table \ref{table-cstemsgeo}).

\begin{table}
	\begin{tabular}{lllllll}
		\lsptoprule
	                  & {C-stems}  & & {\textit{e}-stems}  & {\textit{a}-stems} \\
		from Georgian & \textit{bal-i}  & \textit{pandur-i}  & \textit{cixe}  & \textit{q'ava}  \\
		              & `cherry' & `panduri' & `fortress' & `coffee'\\
		\midrule
		
		Nominative & \textit{bal} & \textit{pandur} & \textit{cix} & \textit{q'av} \\
		
		Ergative/Instrumental & \textit{balev} & \textit{pejndrev} & \textit{cixev} & \textit{q'avev} \\
		
		Genitive & \textit{baleⁿ} & \textit{pendreⁿ} & \textit{cixeⁿ} & \textit{q'aveⁿ} \\
		
		Dative & \textit{balen} & \textit{pendren} & \textit{cixen} & \textit{q'aven} \\
		
		Spatial (Contact) & \textit{balex} & \textit{pendrex} & \textit{cixex} & \textit{q'avex} \\
        \lspbottomrule
	\end{tabular}
	\caption{Georgian borrowings in a consonant, \textit{-e} or \textit{-a}}
	\label{table-cstemsgeo}
\end{table}

A minority of Georgian consonant-ending nouns are borrowed into the C/\textit{u-}declension (see Table \ref{table-custems}). This pattern is no longer productive, and later loans are borrowed into the consonant declension (Table \ref{table-cstemsgeo}). The fact that the borrowing of nouns into the C/\textit{u-}declension is no longer possible, combined with the fact that \textit{geps, gepsu-} is an old loan (cf. Old Georgian \textit{msgeps-i} `week', not found in the modern standard language, or any of the dialects) gives reason to assume that these C/\textit{u-}declension borrowings are older. 

\begin{table}
	\begin{tabular}{lllllll}
    \lsptoprule
		& {C/\textit{u}-stems}  & &  & \\
		from Georgian & \textit{angariš-i}  & \textit{msgeps-i} & \textit{saat-i}  & \textit{žang-i}  \\
		              &  `account, bill' &    `week' &  `hour' & `rust'  \\
		\midrule
		
		Nominative & \textit{angriš} & \textit{geps} & \textit{saat} & \textit{žang} \\
		
		Ergative/Instrumental & \textit{engršuv} & \textit{gepsuv} & \textit{satuv} & \textit{žanguv} \\
		
		Genitive & \textit{engršuⁿ} & \textit{gepsuⁿ} & \textit{satuⁿ} & \textit{žanguⁿ} \\
		
		Dative & \textit{engršun} & \textit{gepsun} & \textit{satun} & \textit{žangun} \\
		
		Spatial (Contact) & \textit{engršux} & \textit{gepsux} & \textit{satux} & \textit{žangux} \\
        \lspbottomrule
	\end{tabular}
	\caption{Georgian borrowings with rare \textit{u-}declension}
	\label{table-custemsgeo}
\end{table}

All Georgian loans take the plural marker \textit{-i}. Examples include \textit{moʒ\u{g}r-i} `priests' (Georgian \textit{moʒ\u{g}or-i}), \textit{souxr-i} `servants' (Georgian \textit{msaxur-i}), \textit{edgl-i} `places' (Georgian \textit{adgil-i}), \textit{ak'azm-i} `hand-looms' (Georgian \textit{ak'azma}), \textit{bat'-i} `geese' (Georgian \textit{bat'-i}).\is{Plural formation}\largerpage[-1]\pagebreak

Only six nouns\footnote{More might be found.} have been found that form their plurals along a different model. It concerns monosyllabic nouns with a root vowel \textit{i} or \textit{u}, that form their plural with root ablaut and a suffix \textit{-bi}, along the model of Tsova-Tush \textit{durk', dark'-bi} `bucket'; \textit{\u{g}rut', \u{g}rat'-bi} `hole'; \textit{č'uk', č'ak'-bi} `drinking horn'; \textit{k'uč', k'ač'-bi} `hill'. The 6 loans are \textit{ʒirk', ʒark'-bi} `tree stump' (Georgian \textit{ʒirk'v-i}); \textit{ʒir, ʒar-bi} `root, base' (Georgian \textit{ʒir-i}); \textit{čxir, čxar-bi} `stick' (Georgian \textit{čxir-i}); \textit{kud, kad-bi} `hat' (Georgian \textit{kud-i}); \textit{k'unʒ, k'anʒ-bi} `tree stump' (Georgian \textit{k'unʒ-i}); \textit{vir, var-bi} `donkey' (Georgian \textit{vir-i}). The fact that both `hat' and `donkey' are also borrowed in Vainakh (Chechen \textit{kuj, kujn-aš} (without ablaut) `hat' and \textit{vir, varr-aš} (with ablaut) `donkey') gives reason to assume that these are old borrowings, and (some of them) can perhaps even be dated to Proto-Nakh times.\largerpage[-1]

A small number of Georgian nominal borrowings ending in \textit{-a} receive the ending \textit{-aɁo} in Tsova-Tush. This has been identified as a diminutive suffix (\citealt{kadagidze87ao}), that is no longer productive. It can also be seen in the forms \textit{vaš-l-aɁ\u{o}} from \textit{vašo} `brother', \textit{jaš-l-aɁ\u{o}} from \textit{jašo} `sister', and \textit{šič-l-aɁ\u{o}} from \textit{šiče} `mother's sister's child', as well as from a small selection of Tsova-Tush vocabulary that has either no clear Georgian equivalent, or has noticeably innovated. See \tabref{table-diminutivett}.\is{Diminutive}

\begin{table}
\fittable{%
	\begin{tabular}{lll}
		\lsptoprule
		\textit{asaɁo} & `calf, heifer' & cf. \textit{ase} `calf'  \\
		\textit{ʒrolaɁo} & `metal tip, arrowhead' & \\
		\textit{ejtaɁo} & `godparent' & \\
		\textit{burdlaɁo} & `scarecrow' & cf. Georgian \textit{burdo} `straw'  \\
		\textit{gagaɁo} & `belly' & cf. \textit{gogo} `round' \\
		\textit{daxaɁo} & `snowcock' &  \\
		\textit{k'ok'amzik'aɁo} & `spinning top' &  \\
		\textit{k'ark'araɁo} & `jaw, chin' & cf. \textit{k'ark'arui} `beard' \\
		\textit{q'uq'naɁo} & `bugle, hunting horn' & cf. Pshav Georgian \textit{q'uq'uni} `sound' \\
		\textit{kejǯberaɁo} & `spider' & cf. Georgian \textit{kaǯi} `evil spirit', \textit{beri} `old' \\
		\textit{gugaɁo} & `puppet, doll' & \\
		\textit{ujrǯk'aɁo} & `inside of khinkali' &  \\
		\textit{cicaɁo} & `flying insect' & \\
		\lspbottomrule
	\end{tabular}}
	\caption{Tsova-Tush nouns with the \textit{-aɁo} diminutive suffix (adapted from \citealt[213]{kadagidze87ao})}
	\label{table-diminutivett}
\end{table}

\textcite[]{kadagidze87ao} identified 43 clear loanwords in \textit{-a} from Georgian receiving the ending \textit{-aɁo} in Tsova-Tush. The vast majority of these nouns can be said to be compatible with diminutive semantics, as presented in Table \ref{table-diminutivegeo1} (small plants (1a); small birds, young animals, insects (1b); terms of endearment for animals (1c); small household items (1d); derived diminutives (1e); other small objects (1f)). For a small number of nouns, the diminutive semantics is relatively opaque (see (2)). 

The fact that not all nouns show diminutive semantics leads \textcite{kadagidze87ao} to speculate whether this suffix was used productively to incorporate Georgian nouns ending in \textit{-a}, but this seems unwarranted, given that (a) a fair number of Georgian nouns that do not end in \textit{-a} also receive this suffix (see Table \ref{table-diminutivegeo1}(3)), and (b) most nouns in \textit{-a} are borrowed without any additional morphology, as shown in Table \ref{table-cstemsgeo}. Furthermore, the ending \textit{-aɁo} seems to replace the ancient Georgian diminutive suffix \textit{-ak'-i}\footnote{Which, one could speculate, could be the origin of the \textit{-aɁo} suffix, although more research is needed. \textit{-ak'-i} exists in Old Georgian, but is already being replaced with \textit{-aj} in the Old Georgian text corpus (Jost Gippert, p.c.). However, many relics remain in Modern Georgian, e.g.: \textit{bagir-i} `thick cable', \textit{bagir-ak'-i} `thin cable'; \textit{bazma} `oil lamp', \textit{bazm-ak'-i} `candlestick, lamp'; \textit{balta} `clasp, buckle', \textit{balt-ak'-i} `small clasp'; \textit{birtv-i} `ball; kernel, core', \textit{birtv-ak'-i} `kernel'; \textit{galia} `(bird)cage', \textit{gali-ak'-i} `small cage' (\textcite{rayfield06dict} contains many more pairs). Tush Georgian, which contains hundreds of nouns ending in \textit{-aɁ\u{i}} (\cite{tsotsanidze02tushdict}), could be the direct donor of this suffix, but the final vowel in the Tsova-Tush suffix \textit{-aɁo} would remain unexplained.} (see Table \ref{table-diminutivegeo1}(4)).
\largerpage[-2]\pagebreak

	\begin{longtable}{llll}
    \caption{Borrowed nouns with the \textit{-aɁo} diminutive suffix (adapted from \textcite{kadagidze87ao})} \label{table-diminutivegeo1}\\
    \lsptoprule
    	& {Noun} & & {from Georgian} \\
	\midrule
	\endfirsthead
	\midrule
    	& {Noun} & & {from Georgian} \\
	\midrule
	\endhead
        

       (1a) & \textit{bambaɁo} & `cotton' & \textit{bamba} \\
			& \textit{p'it'naɁo} & `mint' & \textit{p'it'na} \\
			& \textit{samq'uraɁo} & `clover' & \textit{samq'ura} \\
			& \textit{duidglaɁo} & `elder(berry)' & \textit{didgula} \\
			& \textit{bejq'laɁo} & `young tree' & \textit{baq'ila} \\
			& \textit{ʒirxvnaɁo} & `burdock' & \textit{ʒirxvena} \\

        \midrule
        
	(1b) & \textit{gogbič'aɁo} & `oriole' & \textit{bič'o-gogia} \\
			& \textit{t'uiraɁo} & `lark' & \textit{t'orola} \\
			& \textit{q'odlaɁo} & `woodpecker' & \textit{k'odala} \\
			& \textit{c'ic'k'naɁo} & `greenfinch' & \textit{c'ic'k'ana} \\
			
			& \textit{vejraɁo} & `hen not yet in lay' & \textit{varia} \\
			& \textit{guišaɁo} & `small dog' & \textit{gošia} \\
			& \textit{bejč'aɁo} & `young hare' & \textit{bač'ia} \\
			& \textit{č'ič'q'naɁo} & `young fish' & \textit{č'ič'q'ina} \\
			
			& \textit{ank'raɁo} & `grass-snake' & \textit{ank'ara} \\
			
			& \textit{bost'naɁo} & `mole cricket' & \textit{bost'ana} \\
			& \textit{k'ejlaɁo} & `locust' & \textit{k'alia} \\
			& \textit{c'urblaɁo} & `leech' & \textit{c'urbela} \\
			& \textit{napt'aɁo} &  `roach' & \textit{napot'a} \\
			& \textit{kočraɁo} & `moth' &  \textit{kočora} \\
			& \textit{enʒlaɁo} & `crocus' & \textit{enʒela} \\
			& \textit{q'urblaɁo} & `earwig' & \textit{q'urbela} \\

    		\midrule
            
			(1c) & \textit{lurǯaɁo} & `dark grey horse' & \textit{lurǯa}  \\
			& \textit{k'udaɁo} & `tail-less sheep' & \textit{k'uda} \\
			& \textit{q'uraɁo} & `one-eared, earless' & \textit{q'ura} \\
            
		\midrule
        
			(1d) & \textit{dumaɁo} & `sheep's fat tail' & \textit{duma} \\
			& \textit{čurčxlaɁo} & `churchkhela' & \textit{čurčxela} \\
			& \textit{tatraɁo} & `tatara'\footnote{Boiled grape juice thickened with wheat flour} & \textit{tatara} \\
			& \textit{t'apaɁo} & `frying pan' & \textit{tap'a} \\
			& \textit{q'alq'alaɁo} & `jug with narrow neck' & \textit{q'arq'ara} \\
			& \textit{bejdaɁo} & `bowl' & \textit{badia} \\
			& \textit{čarkaɁo} & `small wine jar' & \textit{čareka} \\
            
		\midrule
        
			(1e) & \textit{č'alaɁo} & `small riverside copse' & \textit{č'ala}\footnote{Also borrowed as \textit{č'al} `riverside copse'.} \\
			& \textit{k'uink'laɁo} & `yoke prop at front of ox-cart' & \textit{k'onk'ila}\footnote{Also borrowed as \textit{k'onk'il}.} \\
			& \textit{bibaɁo} & `midwife' & \textit{bebia} `grandmother' \\
            
			\midrule
            
			(1f)
			& \textit{k'altaɁo} & `(coat/dress) hem, skirt, tails' & \textit{k'alta} \\
			& \textit{xuxlaɁo} & `hut, hovel' & \textit{xuxula}  \\
			
			\midrule
            
			(2) & \textit{mušaɁo} & `worker' & \textit{muša} \\
			& \textit{čoqaɁo} & `outer tunic, Caucasian coat' & \textit{čoxa} \\
			& \textit{qeldaɁo} & `2-litre jar' & \textit{xelada} \\
			& \textit{birk'aɁo} & `carnival mummer' & \textit{berik'a} \\
			& \textit{k'op'laɁo} & `Kopala' (pagan god) & \textit{k'op'ala} \\


            		\midrule
                    
		(3) & \textit{st'ejplaɁo} & `carrot' & \textit{st'apilo} \\
		& \textit{dindglaɁo} & `beeswax' & \textit{dindgeli} \\
		& \textit{k'ač'k'ač'aɁo} & `magpie' & \textit{k'ač'k'ač'i} \\
		& \textit{mercxlaɁo} & `swallow' & \textit{mercxali} \\
		& \textit{bolk'aɁo} & `radish' & \textit{bolok'i} \\
		
		\midrule
        
		(4) & \textit{ešmaɁo} & `devil' & \textit{ešmak'i} \\
		& \textit{xoršaɁo} & `malaria' & \textit{xoršak'i} \\
		& \textit{sarsaraɁo} & `little bustard' & \textit{sarsarak'i} \\
        
        \lspbottomrule
	\end{longtable}

\subsubsection{Gender assignment} \label{genderadapt}\is{Loanword adaptation!Nouns}\is{Loanword adaptation!Gender}\is{Gender assignment}

\hspace*{\fill} \textit{This section is largely based on \textcite{WS}.}

\medskip

For gender agreement in the noun phrase and the labels used in this work, see \sectref{genderagree}, for verbs indexing gender, see \sectref{verbalgender}.




Before discussing the gender assignment (i.e. the rules that determine which noun goes into which gender category) of loanwords, a brief overview of gender assignment of native words is provided here.
Only genders M and F show clear semantic rules,\is{Humanness} in that only nouns denoting male rationals (i.e. humans, supernatural beings, personified animals) are assigned to gender M, and only those that denote female rationals are assigned to gender F. All other nouns are assigned to one of the three ``neuter'' genders. Semantic preferences do play a role in determining which noun goes where, as can be seen from Table \ref{table-genderassign}.

\begin{table}
\fittable{%
		\begin{tabular}{lll}
    \lsptoprule
			J & D & B \\
			\midrule
			most weather phenomena & & winds, storms \\
			fruit growing in bunches & & \\
			huge fruit & most plants and trees & \\
			weak, decrepit cattle & & \\
			most small animals, young & `horse' & most horses, `horse'\footnote{\textit{doⁿ}, the most frequent word for `horse' as well as words for `foal', `stallion', `mare' have gender B, while \textit{ulaq'}, another word for `horse' has gender D.} \\
			small stinging/biting insects & most other insects & \\
			building materials & gems and metals & most rocks, boulders, dirt \\
			holes & cracks & most scraps and splinters \\\addlinespace
			firearms & most bladed weapons & \\
			logs, blocks & paper items & \\
			animal straps, reins & horse gear & \\
			drums & & string instruments \\
			most jugs & most sacks, bags & chests, boxes \\
			most functional buildings & most houses and  & \\
		                              & \quad shelters & \\
			pastry & drinks & \\
			& headscarves & hats \\\addlinespace
			& & abstract nouns denoting \\
			& & \quad bad feelings\\
			human sounds & non-human sounds & most language utterances \\
			groups of people & rationals (unspecified  & \\
			                 & \quad gender) & \\
			administrative units & & \\
			most diseases & & \\
			& & month names \\
			\lspbottomrule
		\end{tabular}}
	\caption{Semantics of ``neuter'' genders (from \textcite[24]{WS})}
	\label{table-genderassign}
\end{table}

It has to be taken into account that this table shows only approximately 15\% of all nouns from \textcite{kadkad84}. Most semantic domains feature nouns in all three classes, such as birds, 30 of which have gender D, 4 have J, and 13 have B. Similarly, with nouns denoting sources or bodies of water, 8  have gender J, 6 have B, while the word \textit{xi} ‘water, source, river’ has gender D). Also, some nouns cannot be reliably grouped into a semantic class at all.


Besides semantic rules, some morphological rules, mostly derivational processes, can trigger a certain gender assigment of some nouns. Nouns denoting abstract concepts with the derivational suffix \textit{-ol} (usually deadjectival) are mostly assigned to the J gender. Examples include \textit{mossol} `badness, evil’ from \textit{mossiⁿ} `bad’, \textit{\u{g}elol}  `weakness, from \textit{\u{g}eliⁿ} `weak’. Deadjectival abstract nouns in \textit{-a} are assigned to the D gender, such as \textit{ʕejrč'na}  `blackness’, from \textit{ʕarč'iⁿ} `black’ and \textit{k'ʕuik'rna} `depth’ from \textit{k'ʕok'ruⁿ} `deep’. Furthermore, all verbal nouns (suffixed by \textit{-ar}) are assigned to the D gender. Interestingly, abstract nouns in \textit{-ob}, as well as nouns which feature the merged suffix \textit{-lob} are divided between genders D and J. 

Phonology plays a part in gender assignment, albeit a modest one (see \cite{bellamyWS}). In Tsova-Tush, nouns that have \textit{d, j} or \textit{b} as their first phone have a greater-than-chance probability of being assigned to the gender corresponding to these consonants. Out of all nouns in the  dictionary (\cite{kadkad84}), approximately 60--65\% of those starting in \textit{b} are in gender B, with similar percentages for \textit{d-}nouns in D gender and j\textit{}-nouns in J gender. This can be seen as a case of fossilised class marking in nouns, varyingly called overt inherent gender, head class, head gender, source gender and autogender (cf. \cite[147]{nichols11}). Alternatively, the phenomenon is analysed as a case of alliterative concord (\cite[117]{corbett91}), where the assignment of a noun to a gender is triggered by the phonology of the word. In Tsova-Tush, the two explanations can exist side by side: overt inherent gender can be used in diachronic context, namely the first consonant of a given noun can be described as a fossilised class marker (which can happen to be deverbal, as in Tsova-Tush \textit{daq'o} `corpse’ from a root \textit{*d-aq'} meaning `die’ (\cite[258]{nichols03cc})), while synchronically, the concept of alliterative concord can help us understand the principle according to which the first phone of a noun triggers an assignment rule in the lexicon of a given speaker.

Borrowed nouns generally show the same interplay of patterns with regards to gender assignment. All nouns denoting humans with a known sex are assigned to either M or F. All other nouns are divided between J, D and B (with very few exceptions). In the majority of cases, Georgian translation equivalents of native Tsova-Tush nouns (ending up as (near-)synonyms after having been borrowed) are assigned the same gender as the original word. Examples include \textit{k'ut} `muscle’ from Georgian \textit{k'unti}, which has gender D, just as the native Tsova-Tush \textit{k'art}; \textit{oǯax} `family’ (Georgian \textit{oǯaxi}) and \textit{kulpat} `family’ both are in gender J; and so are \textit{doħa} `spring’ and \textit{gazapxuila} `spring’, the latter being a borrowing from Georgian \textit{gazapxuli}. Semantically related words, often with a metonymic or hyponymic relation to a native Tsova-Tush word, are assigned similarly. For example, all borrowed nouns denoting anything made of paper are assigned to gender D, analogous to the native word \textit{žagno} ‘book, letter’ (cf. Chechen \textit{žajna} ‘book’). Similarly, most flowers are assigned gender D on the basis of the native noun \textit{bubuk'} `flower’. 

Phonological features of initial consonants also play a role here. 47\% of borrowed nouns starting with \textit{b-} are assigned to gender B, which is significant if one compares this with, for example, borrowed nouns in \textit{t'-} (18\% in B), \textit{k-} (22\%), or \textit{ǯ-} (4\%). Unfortunately, the same pattern cannot be seen in the J gender, since Georgian does not have a phoneme \textit{j-} /j/ - the closest phonetic alternative would be the sequence /i/ + vowel. However, only 6 nouns with this cluster are borrowed and from these, it can be seen that Tsova-Tush adapts this sequence as /ij/\,+ vowel, e.g. \textit{ijanvar} (B gender) from Georgian \textit{ianvari} `January’.

It is striking that the B gender receives significantly fewer borrowings than genders D and J. If we again base ourselves on \textcite{kadkad84}, genders D and J receive 518 and 505 borrowings respectively, while B receives only 236 borrowings. The relatively low number of loanwords might indicate that productive gender assignment to this class has stopped. Four arguments could support this claim: 

\begin{enumerate}
	\item Semantically, many loans in gender B denote archaic concepts. The words in class B denoting the most recent innovation in technology or culture are \textit{pit'oⁿ} `carriage’ from Georgian \textit{paet'oni}, and \textit{sap'irc'omal} `gunpowder’ from Georgian \textit{sap'irc'amali}.\footnote{The modern word for `gunpowder' is \textit{topis c'amali}. The word \textit{sap'irc'amali}, although not attested in Georgian directly, must have been an archaic (possibly dialectal) form, as seen by the attested derived \textit{sap'irisc'amle} `small case for gunpowder', which is also borrowed into Tsova-Tush.} All words denoting more recent technology, weapons, vehicles, documents, etc. are in genders D or J. 
	
	\item We find some cases of borrowing in class B that can alternatively demand agreement marking of another gender: \textit{purn} `bread oven’ (from Georgian \textit{purne}) is listed in \textcite{kadkad84} as having both J and B agreement. Similarly, both \textit{c'ic'k'naɁ\u{o}} `greenfinch’ (from Georgian \textit{c'ic'k'ana} assigned to B and D) and \textit{t'lap'\u{o}} `mud’ (from Georgian \textit{t'lap'o} assigned to B and J) show fluctuation in their agreement. A possible interpretation of this is that these nouns are “leaving” the archaic class B for a more productive one. 
	
	\item The few derivational processes that produce nouns (those discussed above) create nouns that are assigned to genders D and J exclusively. Stated more generally, no single noun in class B is the result of Tsova-Tush-internal derivation.
	
	\item Borrowed nouns that show a clear semantic analogy to native nouns in gender B often are still assigned to another gender, for example: \textit{švriv} `oats’ (D) (from Georgian \textit{švria}), but native \textit{cu} `oatmeal’ (B) (cf. Chechen \textit{sula} ‘oats’); \textit{bojt'i} `billy-goat’ (D) (from Georgian \textit{bot'i} `billy-goat’), but native \textit{bʕok'} `billy-goat’ (B); \textit{nezv} `ewe’ (D) (from Georgian \textit{nezvi} `ewe’), but native \textit{uistx} `ewe’ (B); \textit{ulaq'} `horse’ (J) (Georgian \textit{ulaq'i} `stallion'), but native \textit{doⁿ} `horse’ (B); \textit{k'enč'} `pebble’ (J) (from Georgian \textit{k'enč'i} `pebble’), but native \textit{k'urk'um} `pebble’ (B).  
	
\end{enumerate}


In contrast to the observation that the B gender seems to be a closed class, \textcite{bellamyWS} clearly demonstrate that the B gender is available to Tsova-Tush speakers in code-switched nominal constructions, presumed to be the source of many borrowings. Further research is necessary to reconcile this observation with the arguments enumerated above.

It has already been made clear that phonology plays only a modest role in guiding gender assignment, mostly due to the fact that semantic preferences disturb the clear phonological distribution, or vice versa, phonological constraints provide exceptions to otherwise clear semantic clusters. As an example, let us take Tsova-Tush bird names (Table \ref{table-birdass}). 

\begin{table}
	\begin{tabular}{lllll}
	\lsptoprule
	D & & & J & \\
	native & {borrowed} & & {native} & {borrowed} \\
    \midrule
	owl & eagle & rosy starling & crane & black vulture \\
	
	snowcock & goose & lark & & great bustard \\
	
	quail & nightingale\footnote{\cite{kadkad84} classifies \textit{bulbulaɁ\u{o}} as both D and B gender.} & chicken, hen & & kite \\
	
	white wagtail & golden oriole & corncrake & & \\
	
	stork & pelican & woodpecker & & \\
	
	bird & eagle-owl & falcon & & \\
	
	& turkey & blackbird & & \\
	& duck & starling & & \\
	& swallow & jay & & \\
	& sparrow-hawk & little bustard & & \\
	
	& merlin & greenfinch & & \\
	& black grouse & mistle-thrush & & \\\addlinespace
	
	& & young corvid & & \\
	& & baby owl & & \\
	& & turkey chick & & \\
	& & duckling & & \\
	\midrule
	B & \\
	native & {borrowed} \\
	\midrule
	raven  & crow \\
    pigeon & wood pigeon \\
           & turtle dove \\
    cuckoo & hoopoe \\
           & nightingale \\
           & siskin \\
           & greenfinch\footnote{\cite{kadkad84} classifies \textit{c'ic'k'naɁ\u{o}} as both D and B gender.} \\ 
           & pheasant \\
           & partridge \\
	\lspbottomrule
	\end{tabular}	
	\caption{Gender assignment of Tsova-Tush bird names}
	\label{table-birdass}
\end{table}

Tsova-Tush \textit{ħac'uk'} `bird' (cf. Chechen \textit{ħoza} `bird’, Ingush \textit{ħazalgʲ} `bird’) belongs to class D, as do the nouns denoting `owl’, `snowcock’, `quail’, `white wagtail’, and `stork’. On this basis, most borrowed bird names (24 in total) are assigned gender D. The four words denoting young birds can additionally be explained on the basis of Tsova-Tush \textit{bader} `child’ which is also in gender D. The exceptions in class J are easily accounted for: the borrowed nouns for `vulture’, `great bustard’ and `kite (bird of prey)’ are assigned gender J with  \textit{c'ʕ(v)eraⁿ} `crane’ as a basis. Size is clearly the semantic link here, although it has to be noted that `stork’ and `pelican’ are in the ``rest class'' D. As for birds in gender B, we find some indications that these borrowings might be old: \textit{q'oɁ} `crow’ from Georgian \textit{q'vavi} `crow’ shows a phonologically irregular correspondence of Georgian final \textit{-v} corresponding with Tsova-Tush \textit{-Ɂ}, which we do not find in more recent loans.\footnote{Cf. \textit{sov} `vulture’ from Georgian \textit{svavi}, but also \textit{niav} `breeze’ and \textit{tavtav} `ear of wheat’, from Georgan \textit{niavi} and \textit{tavtavi}. In theory, Georgian \textit{q'vav-i} could be explained as a loan from Proto-Nakh \textit{*q'VvaɁ}.} Besides `crow’, \textit{č'ilq'oɁ} `rook’ (from Georgian \textit{č'ilq'vav-i}) has also been assigned to B on the basis of native \textit{ħa\u{g}ar} `raven’ (cf. Chechen, Ingush \textit{ħa\u{g}ra}). Similarly, the borrowed `wood pigeon’, `turtle dove’ are in B on the basis of \textit{qauq\u{u}} `pigeon’ (cf. Chechen, Ingush \textit{qoqa} ‘pidgeon’). Six borrowed bird names in gender B remain after we substract the aforementioned four birds. One of them, \textit{bulbul} `nightingale’, can be explained by the fact that it starts with a \textit{b-}. This word could then have functioned as a ``Trojan horse'' (\cite{corbett91}), that is, it could have established, together with the corvids and the pigeons, a small semantic cluster on the basis of which other birds have been assigned to gender B, before its effects wore out and B gender assignment rules became unproductive. At this point, it has to be noted that in most other East Caucasian languages, all non-human animates (i.e. animals) are assigned gender B (\cite[156]{vandenberg2005intro}), a situation which can be assumed to go back to Proto-East-Caucasian.

It is also possible that the basis for a given semantic analogy was lost. All Tsova-Tush nouns denoting small stinging or biting insects (\textit{k'el} `wasp’, \textit{put'k'ar} `bee’, \textit{k'o\u{g}\u{o}} `mosquito’, \textit{t'k'ip'} `tick’) are in gender J, while all other borrowed nouns denoting insects are in class D. What is more, these four insects are all borrowed from Georgian, while the only native noun in gender J that comes close to being an insect is \textit{qaup'\u{u}} `worm’. The four Georgian nouns must have been assigned to gender J on the basis of one or more native Tsova-Tush synonyms having this gender. It is clear that these synonyms (before having been replaced by the Georgian items) belonged to gender J, as is shown by their Vainakh semantic equivalents Chechen \textit{čyrk} `mosquito’, Ingush \textit{kamār} `mosquito’ (from Russian \textit{komar}) and Chechen \textit{zʕuga} `wasp’, all of which have gender J (\cite{matsiev61dict}). 



\section{Modifiers} \label{modifiers}

\subsection{Demonstratives} \label{demonstratives}

Demonstratives can modify nouns,\footnote{And are thus sometimes called demonstrative adjectives in this function.} and involve a three-way contrast in terms of distance from the speaker: proximal \textit{e}, medial \textit{is} and distal \textit{o}.\footnote{A further proximal demonstrative \textit{i} and distal \textit{(h)as} are becoming obsolete.} The likeness of Tsova-Tush medial \textit{is} (from \textit{*icx}, and also found as Chechen \textit{is} and Ingush \textit{ɨz}) and the Georgian distal pronoun \textit{is} is coincidental. See \sectref{dempro} for demonstratives used as pronouns.\is{Medial demonstrative}\is{Proximal demonstrative}\is{Distal demonstrative}

\begin{exe}
	\ex\label{simplenp-ex3} \begin{xlist}
		\ex\label{simplenp-ex3a}
		\gll \textbf{e} doⁿ sačukar b-a. \\
		\textbf{{\Prox}} horse gift {\B}.{\Sg}-be \\
		\trans `this horse is a gift.'
		\hfill (BH023-16.1)
		
		\ex\label{simplenp-ex3b}
		\gll `` čangašvili v-ik'-a-t, '' aɬ-iⁿ...  \\
		{} Changashvili {\M}.{\Sg}-take.{\Anim}-{\Imp}-{\Pl} {} say-{\Aor}  \\
        
		\gll - `` soⁿ co v-ec' \textbf{is} st'ak' '' aɬ-iⁿ.\\
		{} {} {\Fsg}.{\Dat} {\Neg} {\M}.{\Sg}-want \textbf{{\Med}} man {} say-{\Aor} \\
		
        \trans `{``}Take Changashvili,'' he said... \\
		- ``I don't want this/that man [that you propose],'' he said.'
		\hfill (E167-64)
		
		\ex\label{simplenp-ex3c}
		\gll ču lat'-d-i-en-es \textbf{o} daš-ni. \\
		{\Pv} add-{\D}-{\Tr}-{\Aor}-{\Fsg}.{\Erg} \textbf{{\Dist}} word-{\Pl} \\
		\trans `I added those words.' 
		\hfill (MM202-1.57)
		
	\end{xlist}
\end{exe}

As can be seen from Example (\ref{simplenp-ex3b}), the medial demonstrative is often used to point to an entity near, or associated with the addressee. Hence, the second speaker in (\ref{simplenp-ex3b}) uses the medial demonstrative to refer to the man that was talked about by the first speaker.\is{Addressee}

Demonstratives can also be used anaphorically, as in  (\ref{simplenp-ex46}), where the first clause introduces the girl (\textit{joħ}), while the second clause refers back to her using the distal demonstrative \textit{o}. The same demonstrative is used for the fox, since it is the protagonist of the story, and is referred to many times. The same function could also be analysed as marking definiteness, as opposed to the indefinite function of the numeral \textit{cħa} `one' (see \sectref{numerals}).\is{Anaphora}\is{Definiteness}

\begin{exe}
	\ex\label{simplenp-ex46}
	
	\gll oqar-go cħa lamzur joħ j-a-nor=e, \textbf{o} \textbf{joħ} moc'onad-j-el-noer \textbf{o} \textbf{cok'l-e-n}. \\
	{\Dist}.{\Pl}.{\Obl}-{\Adess} one beautiful girl {\F}.{\Sg}-be-{\Nw}.{\Rem}=and {\Dist} girl like-{\F}.{\Sg}-{\Intr}-{\Nw}.{\Rem} {\Dist} fox-{\Obl}-{\Dat} \\
	\trans `They had (it is told) a beautiful daughter, and the fox liked (it is told) that girl.' 
	\hfill (E153-30)
	
\end{exe}

Demonstratives show rudimentary case agreement with the nouns they modify, see \sectref{agreement}. Demonstratives can be used as pronouns in noun-less noun phrases, see \sectref{headlessnp}. See \sectref{dempro} for a comparison with Georgian and Vainakh demonstratives.


\subsection{Numerals} \label{numerals}
Numerals precede the head noun. Table \ref{tablenumerals} shows the formation of cardinal numerals (\cite{haukharris}).\is{Cardinal numerals}


\begin{table}
	\begin{tabular}{r@{~}l r@{~}l r@{~}l r@{~}l}
		\lsptoprule
		1 & cħa & 11 & cħait't'  & 21 & t'q'a cħa  & 40 & šauzt'q'(a) \\
		
		2 & ši & 12 & šiit't' & 22 & t'q'a ši & 50 & šauzt'q'a it't' \\
		
		3 & qo & 13 & qoit't' & 23 & t'q'a qo & 60 & qouzt'q'(a) \\
		
		4 & d-ʕivɁ & 14 & d-ʕevɁet't' & 24 & t'q'a  d-ʕivɁ & 70 & qouzt'q'a it't' \\ 
		
		5 & pxi & 15 & pxiit't' & \multicolumn{1}{c}{$\vdots$} & & 80 &  d-ʕevɁuzt'q'(a) \\
		
		6 & jetx & 16 & jetxet't' & 30 & t'q'a it't' & 90 & d-ʕevɁuzt'q'a it't' \\
		
		7 & vorɬ & 17 &  vorɬet't' & 31 & t'q'a cħait't' & 100 & pxauzt'q'(a) \\
		
		8 & barɬ & 18 &  barɬet't' & 32 & t'q'a šiit't' & 101 & pxauzt'q'a cħa \\
		
		9 & iss & 19 & t'q'exc' & 33 & t'q'a qoit't' & 102 & pxauzt'q'a ši \\
		
		10 & it't' & 20 & t'q'a  & 34 & t'q'a d-ʕevɁet't' & 1000 & atas \\
		\lspbottomrule
	\end{tabular}
	\caption{Tsova-Tush cardinal numerals}
	\label{tablenumerals}
\end{table}

As is clear from Table \ref{tablenumerals}, and as is found in most East Caucasian and Kartvelian languages, Tsova-Tush features a vigesimal system, i.e. numbers above twenty are constructed from the largest possible factor of twenty, followed by a numeral `one' to `nineteen'. Hence, \textit{qouzt'q'a šiit't'} `72' is constructed from 3 x 20 + 12. The numerals \textit{šauzt'q'a} `fourty', \textit{qouzt'q'a} `sixty', \textit{d-ʕevɁuzt'q'a} `eighty' and \textit{pxauzt'q'a} `hundred' show regular vowel apocope, except when another numeral follows, which enables the possibility of analysing these numerals as compound words (apocope would only occur word-finally). The vigesimal system in principle continues after one hundred, to four hundred and beyond (\citealt{kadkad84} list \textit{t'q'auztq'} `four hundred' ($20 \times 20$) and \textit{t'q'auzt'q'a pxauzt'q'} `five hundred' ($20 \times 20 + 5 \times 20$)), although in contemporary Tsova-Tush, numbers after one hundred are most often borrowed from Georgian. Hence, when referring to a year, most Tsova-Tush speakers indicate the century in Georgian, and the lower number (1 to 99) in Tsova-Tush (see \cite[173]{gippert08}). See Example (\ref{simplenp-ex43}), where `1900' is referred to with the Georgian \textit{at-as cxra-as} `ten-hundred nine-hundred'.\is{Georgian influence!Lexical}

\begin{exe}
	\ex\label{simplenp-ex43}
	\gll j-a-s dabadbad-j-\={e}l-n {\{ \textbf{atas}} {\textbf{cxraas} \}} t'q'a pxiit't' šar-e. \\
	{\F}.{\Sg}-be-{\Fsg}.{\Nom} be\_born-{\F}.{\Sg}-{\Intr}-{\Ptcp}.{\Aor} \textbf{thousand} \textbf{nine\_hundred} twenty fifteen year-{\Obl}({\Ess}) \\
	\trans `I was born in the year 1935.'
	\hfill (E208-2)
\end{exe}

The numeral \textit{cħa} `one' can be used as an indefinite article. See Example (\ref{simplenp-ex2a}), where \textit{cħa} is used to indicate that the noun phrase is indefinite, and (\ref{simplenp-ex2b}), where the numeral is used to introduce a character in the story that was not previously mentioned.\is{Article}\is{Definiteness}\largerpage

\begin{exe}
	\ex\label{simplenp-ex2}
	\begin{xlist}
		
		\ex\label{simplenp-ex2a}
		\gll \textbf{cħa} k'ac'k'oⁿ ʕexk'-e-ⁿ wun-ax d-a, ab-aⁿ j-a oqui-ⁿ sakm. \\
		\textbf{one} small iron-{\Obl}-{\Gen} what-{\Indf} {\D}-be sew-{\Inf} {\J}-be {\Dist}.{\Obl}-{\Gen} task \\
		\trans `It is a small iron thing, its duty is to sew.'
		\hfill (E042-178)
		
		
		\ex\label{simplenp-ex2b}
		\gll oqar-go \textbf{cħa} lamzur joħ j-a-nor=e, o joħ moc'onad-j-el-noer o cok'l-e-n. \\
		{\Dist}.{\Pl}.{\Obl}-{\Adess} \textbf{one} beautiful girl {\F}.{\Sg}-be-{\Nw}.{\Rem}=and {\Dist} girl like-{\F}.{\Sg}-{\Intr}-{\Nw}.{\Rem} {\Dist} fox-{\Obl}-{\Dat} \\
		\trans `They had (it is told) a beautiful daughter, and the fox liked (it is told) that girl.' 
		\hfill (E153-30)
		
	\end{xlist}
\end{exe}

This use of the numeral `one' is attested in Chechen, Ingush and in Georgian, but recent descriptions are lacking (but see \citealt{lomtatidze1962one}).

See \sectref{dempro} for the anaphoric use of demonstratives in Example (\ref{simplenp-ex2b}). The numerals `one', `two', and `three' show rudimentary case agreement with the nouns they modify, while the numeral `four' (and its derivatives) show gender agreement (see \sectref{agreement}). Numerals can be used as heads in noun-less noun phrases (see \sectref{subst}).

Besides cardinal numerals (see Example (\ref{simplenp-ex4a})), Tsova-Tush features various derivational suffixes to form other types of numerals which can be used to modify nouns.\footnote{Another suffix \textit{-c'} is used to form multiplicatives, which are used adverbially.
}
See Table \ref{tabledernumerals} for these derivations, compared against their counterparts in Georgian (\cite[265]{aronson91}) and Vainakh (\cite[197]{nichols11}, \cite[53]{nichols94Che}). Note that the distributive is formed by reduplicating the first consonant of the numeral. Hence, we find \textit{cħa-c} `one each' (Example (\ref{simplenp-ex4f})). For the numeral 3, the distributive \textit{qo-q} `three each' (Example (\ref{simplenp-ex4e})) is homophonous with the approximative \textit{qo-q} `about three'.\is{Restrictive numerals}\is{Ordinal numerals}\is{Collective numerals}\is{Distributive numerals}\is{Approximative numerals}

\begin{table}
	\tabcolsep=0.75\tabcolsep
	\begin{tabular}{llllll}
    \lsptoprule
		Type & Example & Translation & Suffix  & Georgian & Vainakh \\\midrule
		Ordinal & (\ref{simplenp-ex4b}) & `Xth' & \textit{-l(o)\u{g}en} & \textit{me- -e} &\textit{-l(a)\u{g}(a)} \\
		Restrictive & (\ref{simplenp-ex4c}) & `only X' & \textit{-k'} & & \\
		Collective & (\ref{simplenp-ex4d}) & `all X' & \textit{-k'eɁ} & \textit{-ve} & \textit{-qqie, -Ɂie} \\
		Distributive & (\ref{simplenp-ex4e},\ref{simplenp-ex4f}) & `X each' & reduplication & redupl. & redupl. \\
		Approximative & (\ref{simplenp-ex4g}) & `about X' & \textit{-q} & \textit{-iode} & \\
		\lspbottomrule
	\end{tabular}
	\caption{Tsova-Tush derived numerals}
	\label{tabledernumerals}
\end{table}



\begin{exe}
	\ex\label{simplenp-ex4} \begin{xlist}
		
		\ex\label{simplenp-ex4a}
		\gll \textbf{jetx} badr-aⁿ badr-i d-a so-go. \\
		\textbf{six} child-{\Gen}.{\Pl} child-{\Nom}.{\Pl} {\D}-be {\Fsg}-{\Adess} \\
		\trans `I have six grandchildren.'
		\hfill (E208-79)
		
		\ex\label{simplenp-ex4b}
		\gll mais šari-ⁿ \textbf{pxi-l\u{g}eⁿ} butt b-a. \\
		May year.{\Obl}-{\Gen} \textbf{five-{\Ord}} month {\B}.{\Sg}-be \\
		\trans `May is the fifth month of the year.'
		\hfill (KK013-2495)
		
		\ex\label{simplenp-ex4c}
		\gll \textbf{ši-k'} cark' j-a-ħer babui-goħ, qeⁿ wux č'ir b-a-r. \\
		\textbf{two-{\Rstrct}} tooth {\J}-be-{\Cond}.{\Aor} grandfather-{\Adess} then what.{\Nom} trouble {\B}.{\Sg}-be-{\Imprf} \\
		\trans `If grandfather had only two teeth, then what would be the problem?' \\
		\hfill (KK026-4424)
		
		\ex\label{simplenp-ex4d}
		\gll \textbf{ši-k'eɁ} kok' ħal b-ak'-iⁿ o seⁿ biʒ-i-go.    \\
		\textbf{two-{\Incl}} leg {\Pv} {\B}.{\Sg}-burn-{\Aor} {\Dist} {\Fsg}.{\Gen} uncle-{\Obl}-{\Adess} \\
		\trans `Both my uncle's legs were burnt.'
		\hfill (E151-37)
		
		\ex\label{simplenp-ex4e}
		\gll \textbf{qo-q} manat qajč-\u{u} šuⁿ. \\
		\textbf{three-{\Distr}} rouble fall\_to 2{\Pl}.{\Dat}  \\
		\trans `You guys are allotted three roubles each.'
		\hfill (KK033-5454)
		
		\ex\label{simplenp-ex4f}
		\gll bacbur do-i-v \textbf{cħa-c} baq'\u{o} d-∅-or. \\
		Tsova\_Tush horse-{\Pl}-{\Erg} \textbf{one-{\Distr}} foal {\D}-give\_birth-{\Imprf} \\
		\trans `Tsova-Tush horses used to give birth to one foal each.'
		\hfill (E008-16)
		
		\ex\label{simplenp-ex4g}
		\gll pxauzt'q' šar-e-x ħatx\u{e} \textbf{ši-q} žagn qet-in\u{\i} st'ak' bedeⁿ co v-a-r vaj-loħ.  \\
		hundred year-{\Obl}-{\Cont} ago \textbf{two-{\Approx}} book know-{\Ptcp}.{\Npst} man except {\Neg} {\M}.{\Sg}-be-{\Imprf} {\Fpl}.{\Incl}-{\Interess}    \\
		\trans `A hundred years ago, there weren't any among us who were knowledgeable of books, but about two.'
		\hfill (KK017-3192)
		
	\end{xlist}
\end{exe}


The plural form of a noun is used when its referent is plural, unless it is quantified by a number or other quantity expression, in which case it is in the singular (see Example (\ref{simplenp-ex4})). Verbal agreement is with the gender of the head noun, not with the referent: in constructions that feature a verb which inflects for gender (\ref{simplenp-ex5}), the quantified noun requires singular agreement on the verb.\is{Quantification}

\begin{exe}
	\ex\label{simplenp-ex5}
	\gll šar-go ši joħ \textbf{j}-a-nor \textbf{j}-al-in. \\
	{\Refl}-{\Adess} two girl \textbf{{\F}.{\Sg}}-be-{\Nw}.{\Rem} \textbf{{\F}.{\Sg}}-loose-{\Ptcp}.{\Aor}   \\
	\trans `She herself had apparently lost two girls.'
	\hfill (E29-13)
\end{exe}


The use of a plural noun modified by a numeral (and subsequent plural gender agreement on the verb) is, however, attested with animates. (\ref{simplenp-ex6}). 

\begin{exe}
	\ex\label{simplenp-ex6}
	\gll pxi \textbf{jažar} \textbf{d}-a-ra-tx\u{o}.  \\
	five \textbf{sister.{\Pl}} \textbf{{\F}.{\Pl}}-be-{\Aor}-{\Fpl}.{\Excl} \\
	\trans `We were five sisters.'
	\hfill (E170-14)
\end{exe}



\subsection{Quantifiers} \label{quantifiers}\is{Quantifiers}

Tsova-Tush features the following set of quantifiers, used in the same way as numerals, i.e. they  modify nouns and cannot co-occur with numerals in the same noun phrase. Additionally, they show a similar agreement pattern to cardinal numerals higher than `3' (see \sectref{agreement}), that is, they do not agree in case or number (and only \textit{d-aniɁ} agrees in gender). They include
\textit{q'ovel} `every, all' (\ref{simplenp-ex7a}), borrowed from Georgian \textit{q'oveli} `every, any, each, all';
\textit{duq} `much, many' (\ref{simplenp-ex7b});
\textit{meɬmi, meɬax} `several' (\ref{simplenp-ex7c});
\textit{meⁿ} `some' (\ref{simplenp-ex7d});
\textit{d-aniɁ}  `all' (\ref{simplenp-ex7e});
\textit{wumaɁ} `all' (\ref{simplenp-ex7f}).



\begin{exe}
	\ex\label{simplenp-ex7}
	\begin{xlist}
		
		\ex\label{simplenp-ex7a}
		\gll \textbf{q'ovel} j-aqqoⁿ sakm k'ack'ui-čo-rna j-ebl-l-a! \\
		\textbf{every} {\J}-big work small-{\Obl}-{\Abl} {\J}-begin-{\Intr}-{\Npst} \\
		\trans `Every big work starts from small ones.'
		\hfill (MM218-1.38)
		
		\ex\label{simplenp-ex7b}
		\gll bac-bi-go zorejš \textbf{duq} moʒ\u{g}vr-i b-a-r. \\
		Tsova\_Tush-{\Pl}-{\Adess} very \textbf{many} priest-{\Pl} {\M}.{\Pl}-be-{{\Imprf}} \\
		\trans `The Tsova-Tush had very many priests.'
		\hfill (E014-34)
		
		\ex\label{simplenp-ex7c}
		\gll daħ d-ebl-o-č\u{o} deni ʕurdeⁿ kikoɁ \textbf{meɬmi} st'ak' saplav axk'-aⁿ \u{g}-o. \\
		{\Pv} {\D}-put-{\Ptcp}.{\Npst}-{\Obl} day.{\Obl}({\Ess}) in\_the\_morning early \textbf{several} man grave dig-{\Inf} go-{\Npst} \\
		\trans `On the day of the burial, very early, several men go to dig the grave.' \\
		\hfill (EK023-2.9)
		
		\ex\label{simplenp-ex7d}
		\gll \textbf{meⁿ} majq\u{\i} arl-c'iⁿ j-is-eⁿ txo-goħ. \\
		\textbf{some} bread smash-{\Priv}.{\Adjz} {\J}-stay-{\Aor} {\Fpl}.{\Excl}-{\Adess} \\
		\trans `We had some whole pieces of bread left.'
		\hfill (KK001-0168)
		
		
		
		\ex\label{simplenp-ex7e}
		\gll \textbf{d-aniɁ} survil-i ħal srulba-d-∅-or badr-i-n. \\
		\textbf{{\D}-all} wish-{\Pl} {\Pv} fulfill-{\D}-{\Tr}-{{\Imprf}} child-{\Pl}-{\Dat} \\
		\trans `He fulfilled all wishes for the children.'
		\hfill (Е038-37.2)
		
		\ex\label{simplenp-ex7f}
		\gll do-i-n=mak=aɁ xabž-ene-tx \textbf{wumaɁ} nax. \\
		horse-{\Pl}-{\Dat}=on={\Emph} sit\_down-{\Aor}.{\Seq}-{\Fpl}.{\Excl}.{\Erg} \textbf{all} people \\
		\trans `We all mounted the horses.' (Lit. `All people we mounted the horses.') \\
		\hfill (EK019-5.15)
		
	\end{xlist}
\end{exe}

It has to be noted that \textit{wumaɁ} `all' and \textit{d-aniɁ} `all' are in fact rarely used as modifiers within the noun phrase. They are more commonly found as pronouns (\ref{simplenp-ex8}) or adverbs (\ref{simplenp-ex9}).\footnote{Perhaps best described as floating quantifiers (\cite{gerdts1987quantfloat,miyagawa2006quantfloat}).}\is{Floating quantifiers}

\begin{exe}
	\ex\label{simplenp-ex8}
	\begin{xlist}
		\ex\label{simplenp-ex8a}
		\gll nipsi-č\u{o} st'ak'-o-x \textbf{wumaɁ} teš-\u{e}. \\
		straight-{\Obl} man-{\Obl}-{\Cont} \textbf{all} believe-{\Npst} \\
		\trans `Everyone believes a righteous man.'
		\hfill (KK014-3008)
		
		\ex\label{simplenp-ex8b}
		\gll haɁ, \textbf{d-aniɁ} teg-d-∅-or! \\
		yes \textbf{{\D}-all} do.{\Ipfv}-{\D}-{\Tr}-{{\Imprf}} \\
		\trans `Yes, s/he did everything!'
		\hfill (E038-31)
		
	\end{xlist}
\end{exe}

\begin{exe}
	\ex\label{simplenp-ex9}
	\begin{xlist}
		\ex\label{simplenp-ex9a}
		\gll e žagnu-j \textbf{wumaɁ} seⁿ d-a. \\
		{\Prox} book-{\Pl} \textbf{all} {\Fsg}.{\Gen} {\D}-be \\
		\trans `These books are all mine.'
		\hfill (KK006-1426)
		
		\ex\label{simplenp-ex9b}
		\gll e pešk'r-i \textbf{d-aniɁ} ʕer-d-av-iⁿ vaš-bi-g\u{o} ħeps-uš. \\
		{\Prox} youngster-{\Pl} \textbf{{\D}-all} be\_spoiled-{\D}-{\Lv}-{\Aor} {\Recp}-{\Pl}-{\All} look-{\Simul} \\
		\trans `These boys are all good for nothing, mimicking each other.' \\
		\hfill (KK006-1397)
		
	\end{xlist}
\end{exe}

No quantifiers, except perhaps \textit{duq}, show any case or number agreement whatsoever, and thus behave like numerals higher than `3'. Only the quantifier \textit{d-aniɁ}, (originally \textit{d-a-ni-Ɂ} `\textsc{d}-be-\textsc{ptcp.npst-emph}')  agrees in gender with its head.




\subsection{Adjectives} \label{adjectives}

Most adjectives inherited from Proto-Nakh end in any of the five vowels plus a nasal stop \textit{n} (which drops and nasalises the preceding vowel in word-final position), such as \textit{at't'aⁿ} `easy', \textit{\u{g}azeⁿ} `good', \textit{arliⁿ} `left', \textit{komoⁿ} `male (animal)', \textit{laxuⁿ} `low'. The ending \textit{-Vⁿ} bears a striking resemblance to the Genitive of nouns (see \sectref{Genitive}).\is{Genitive case}

Tsova-Tush exhibits many suffixes that derive adjectives from nouns, such as \textit{-(a)ren, -\u{g}en, -lon, -lun, -lin}, e.g.  \textit{pst'uin\u{g}eⁿ} `female', from \textit{pst'uin\u{o}} `woman'.\is{Derivation!Adjectivising}\is{Derivation!Denominal}

Ten adjectives show gender agreement, 2 show number agreement, while all show rudimentary case agreement. See \sectref{agreement}.

Nakh languages feature comparative and superlative constructions using affixes on the adjective (\cite[452]{desheriev1963comparative}), in contrast to Daghestanian languages, that usually only mark the standard of comparison (the noun against which something is compared, see e.g. Lezgian (\cite[432]{haspelmath1993lezgian})). Comparative adjectives can be formed synthetically with the suffix \textit{-(iv)xu} (\ref{simplenp-ex13}). As can be seen from Example (\ref{simplenp-ex13b}), the Tsova-Tush contact case (\textit{-x}) is used to signal the standard of comparison.\is{Comparison}\is{Comparative adjectives}\is{Superlative adjectives}

\begin{exe}
	\ex\label{simplenp-ex13}
	\begin{xlist}
		
		\ex\label{simplenp-ex13a}
		\gll bulaq' k'ʕok'ar-b-i-n-as me \textbf{\u{g}az-ivx} xi d-aɬ-u-l\u{o}. \\
		well deepen-{\B}.{\Sg}-{\Tr}-{\Aor}-{\Fsg}.{\Erg} {\Subord} \textbf{good}-{\Cmp} water {\D}-go\_out-{\Npst}-{\Sbjv} \\
		\trans `I deepened the well, so that better water would come out.' \\
		\hfill (KK011-2183)
		
		\ex\label{simplenp-ex13b}
		\gll alazaⁿ cer-e-ħ bat'-a-x \textbf{j-aqqou-vx} savt-i ʕa-j-axk'-er. \\
		Alazani edge-{\Obl}-{\Ess} goose-{\Obl}.{\Pl}-{\Cont} \textbf{{\J}-big-{\Cmp}} great\_bustard-{\Pl} sit-{\J}-{\Lv}.{\Pl}-{{\Imprf}} \\
		\trans `On the banks of the Alazani were sitting great bustards bigger than geese.'
		\hfill (KK019-3292)
		
	\end{xlist}
\end{exe}

The superlative is formed by adding \textit{-č'} to the comparative (Example (\ref{simplenp-ex14a})) or the positive stem (\ref{simplenp-ex14b}).

\begin{exe}
	\ex\label{simplenp-ex14}
	\begin{xlist}
		
		\ex\label{simplenp-ex14a}
		\gll nin\u{o} \textbf{k'ac'k'o-xuč'} badr-e-x elan\u{e} c'-e.\\
		Nino \textbf{small-{\Superl}} child-{\Obl}-{\Cont} Elane be\_called-{\Npst} \\
		\trans `Nino's youngest child is called Elane.'
		\hfill (KK011-2017)
		
		\ex\label{simplenp-ex14b}
		\gll doš \textbf{d-aqqon-č'} \textbf{gerc'} d-a admian-goħ. \\
		word \textbf{{\D}-big-{\Superl}} \textbf{weapon} {\D}-be person-{\Adess} \\
		\trans `The word is man's biggest weapon.'
		\hfill (KK004-1191)
		
	\end{xlist}
\end{exe}

Additionally,  the meaning `slightly' can be added to an adjective by means of the enclitic \textit{=k'aɁ} (\ref{simplenp-ex15}). 


\begin{exe}
	\ex\label{simplenp-ex15}
	\begin{xlist}
		
		\ex\label{simplenp-ex15a}
		\gll \textbf{k'ʕok'ruⁿ=k'aɁ} j-a-ħer e orm\u{o}, duq k'or bak'-b-i-en-b-a-ra-s. \\
		\textbf{deep=slightly} {\J}-be-{\Cond}.{\Aor} {\Prox} hole much charcoal extract-{\B}.{\Sg}-{\Tr}-{\Ptcp}.{\Aor}-{\B}.{\Sg}-be-{{\Imprf}}-{\Fsg}.{\Nom} \\
		\trans `If the hole had been a little deep(er), I would have extracted more charcoal.'
		\hfill (KK011-2106)
		
		\ex\label{simplenp-ex15b}
		\gll \textbf{ubral=k'aɁ} araq' d-aɬ-eⁿ soⁿ.\\
		\textbf{simple=slightly} vodka {\D}-go\_out-{\Aor} {\Fsg}.{\Dat} \\
		\trans `I ended up with slightly ordinary vodka.'
		\hfill (E209-15)
		
	\end{xlist}
\end{exe}


However, a common alternative to the synthetic forms above are analytically formed comparatives using \textit{upr(o)}, borrowed from Georgian \textit{upro} `more' (\ref{simplenp-ex16}).\is{Georgian influence!Morphological}

\begin{exe}
	\ex\label{simplenp-ex16}
	\begin{xlist}
		
		\ex\label{simplenp-ex16a}
		\gll \textbf{upr} \textbf{duq} maq-iš c'era-d-i-en d-a-r. \\
		\textbf{more} \textbf{many} song-{\Pl} write-{\B}.{\Pl}-{\Tr}-{\Ptcp}.{\Aor} {\B}.{\Pl}-be-{{\Imprf}} \\
		\trans `S/he could have written more songs.'
		\hfill (E171-11)
		
		\ex\label{simplenp-ex16b}
		\gll soⁿ c'ona-l-a \textbf{upr} \textbf{mac'riⁿ} mač'ar. \\
		{\Fsg}.{\Dat} like-{\Intr}-{\Npst} \textbf{more} \textbf{sweet} wine. \\
		\trans `I like sweeter wine.'
		\hfill (BH024-31.1)
		
		
		\ex\label{simplenp-ex16c}
		\gll lamu \textbf{upr} \textbf{\u{g}azeⁿ} dažar d-a. \\
		mountain({\Ess}) \textbf{more} \textbf{good} grass {\D}-be \\
		\trans `In the mountains, there is better grass.'
		\hfill (E043-53)
		
	\end{xlist}
\end{exe}

Analytically formed superlatives using inherited \textit{ħamaxeɁ} `most' (\ref{simplenp-ex17}) are more common than the synthetic forms in (\ref{simplenp-ex14}). This form is grammaticalised from \textit{ħama-x=eɁ} `than all' (< `who\textsc{.obl-con=emph}'). 

\begin{exe}
	\ex\label{simplenp-ex17}
	\begin{xlist}
		
		\ex\label{simplenp-ex17a}
		\gll kalk-i \textbf{ħamaxeɁ} \textbf{laq-ivx} adgil. \\
		city-{\Iness} \textbf{most} \textbf{high-{\Cmp}} place \\
		\trans `The highest position in the city'
		\hfill (E171-15)
		
		\ex\label{simplenp-ex17b}
		\gll ča\u{g}marto-ⁿ \textbf{ħamaxeɁ} \textbf{b-aqqu-x\u{u}} pħe omlo j-a. \\
		Chaghma\_Tush-{\Gen} \textbf{most} \textbf{{\B}.{\Sg}-big-{\Cmp}} village Omalo {\J}-be \\
		\trans `Omalo is the biggest village of the Chaghma-Tush.'
		\hfill (KK027-4537)
		
		\ex\label{simplenp-ex17c}
		\gll uqar gor-e-lo, \textbf{ħamaxeɁ} \textbf{d-aqqoⁿ} gor d-a o cisk'rob-iⁿ. \\
		{\Dist}.{\Pl}.{\Obl} clan-{\Obl}-{\Interess} \textbf{most} \textbf{{\D}-big} clan {\D}-be {\Dist} Tsiskarishvili-{\Gen}.{\Pl} \\
		\trans `Among their clan(s), the biggest clan is that of the Tsiskarishvilis.' \\
		\hfill (E287-10)
		
		\ex\label{simplenp-ex17d}
		\gll k'att-e-ħ aln-iħ \textbf{ħamaxeɁ} \textbf{zoraⁿ} sicx j-a. \\
		July-{\Obl}-{\Ess} Alvani-{\Iness} \textbf{most} \textbf{severe} heat {\J}-be \\
		\trans `In Alvani, the most severe heat is in July.'
		\hfill (KK011-1934)
		
	\end{xlist}
\end{exe}


Modern standard Georgian forms its comparatives (\ref{simplenp-ex44a}) and superlatives (\ref{simplenp-ex44b}) analytically (\cite[236]{vogt}). Old Georgian, however, used circumfixes of the form \textit{xu- -o/-e(is)} to form comparatives (\cite{gippert2000compar}), such as \textit{xu-did-ejs-i} `bigger'. These forms have shifted in function and are now used as a secondary, archaic superlative in Modern Georgian (cf. \textit{u-did-es-i} `very big, biggest', from \textit{did-i} `big').\il{Old Georgian}

\begin{exe}
	\ex\label{simplenp-ex44}
	Modern Georgian
	\begin{xlist}
		
		\ex\label{simplenp-ex44a}
		\gll somx-isgan me-or-e, \textbf{upro} \textbf{did-i} saxl-i iq'ida. \\
		Armenian-{\Abl} {\Ord}-two-{\Ord} \textbf{more} \textbf{big-{\Agr}} house-{\Nom} s/he\_bought\_it \\
		\trans `S/he bought a second, bigger house from an Armenian.' \\
		\hfill (GNC: Ch. Amirejebi)
		
		\ex\label{simplenp-ex44b}
		\gll am čven-s kveq'ana-ši k'anon-i \textbf{q'vela-ze} \textbf{iap-i} sakonel-i=a. \\
		{\Prox}.{\Obl} {\Fpl}.{\Poss}-{\Obl} country-{\In} law-{\Nom} \textbf{all-{\Super}} \textbf{cheap-{\Agr}} cattle-{\Nom}={\Cop} \\
		\trans `In this country of mine, the law is the cheapest cattle.' \\
		\hfill (GNC: Ch. Amirejebi)
		
	\end{xlist}
\end{exe}

Compare Table \ref{compsuperl}, where the Tsova-Tush degrees of comparison are shown besides the Ingush (\cite[219]{nichols11}), Chechen (\cite[30]{nichols94Che}) and Georgian (\cite[236]{vogt}). As for the comparative, it is clear that the borrowed analytic construction with \textit{upro} `more' is competing with the original Nakh comparative marker \textit{-xu} in Tsova-Tush. As for the superlative, it is at this point uncertain whether the superlative suffix \textit{-č'} is the original Nakh formant. Since Chechen and Ingush display analytic constructions as well (but different ones to Tsova-Tush), it is also uncertain whether the superlative was originally formed analytically or synthetically. It is clear, however, that the specific Tsova-Tush superlative marker \textit{ħamaxeɁ} is a calque from Georgian \textit{q'velaze}: both mean `than all'. Note that in all four languages, the superlative can be formed from  both a positive and a comparative stem. Compare Examples (\ref{simplenp-ex17a}) and (\ref{simplenp-ex17b}), where the superlative marker is used in combination with an adjective already marked with the comparative, whereas in Examples (\ref{simplenp-ex17c}) and (\ref{simplenp-ex17d}), the superlative is used in combination with an uninflected adjective.\largerpage

% % % % \begin{table}
% % % % 		\begin{tabular}{llllll}
% % % %     \lsptoprule
% % % % 			& {Ingush} & {Chechen} & \multicolumn{2}{c}{{Tsova-Tush}} & {Georgian} \\
% % % % 			\midrule
% % % % 			
% % % % 			`big' & d-oaqqa & d-oqqa & \multicolumn{2}{c}{d-aqqoⁿ} & did-i \\
% % % % 			
% % % % 			`bigger' & d-oaqqa-\u{g} & d-oqqa-x & d-aqqu-x & upr d-aqqoⁿ & upro did-i \\
% % % % 			
% % % % 			`biggest' & eggara d-oaqqa & uggar d-oqqa & d-aqqon-č' & ħama-x=eɁ d-aqqoⁿ & q'vela-ze did-i \\
% % % % 			
% % % % 			& eggara d-oaqqa\u{g} & uggar d-oqqax & d-aqqo-xuč' & ħama-x=eɁ d-aqqux & q'vela-ze upro did-i \\
% % % % 			\lspbottomrule
% % % % 			
% % % % 		\end{tabular}
% % % % 	\caption{Synthetic and analytic degrees of comparison in Nakh and Georgian}
% % % % 	\label{compsuperl}
% % % % \end{table}

\begin{table}
\small
\begin{tabular}{lllll}
	\lsptoprule
	           & `big' & `bigger' & \multicolumn{2}{c}{`biggest'}\\\midrule
	Ingush     & d-oaqqa & d-oaqqa-\u{g} & eggara d-oaqqa & eggara d-oaqqa\u{g} \\
	Chechen    & d-oqqa  & d-oqqa-x      & uggar d-oqqa &  uggar d-oqqax\\
	           &         & d-aqqu-x      & d-aqqon-č' & d-aqqo-xuč'\\
	Tsova-Tush & d-aqqoⁿ & upr d-aqqoⁿ   & ħama-x=eɁ d-aqqoⁿ & ħama-x=eɁ d-aqqux\\
	Georgian   & did-i   & upro did-i    & q'vela-ze did-i & q'vela-ze upro did-i \\
	\lspbottomrule
\end{tabular}
\caption{Synthetic and analytic degrees of comparison in Nakh and Georgian}
\label{compsuperl}
\end{table}

The analytic superlative construction is not found in the oldest Tsova-Tush sources, but is attested from the mid-20th century onward (see Examples (\ref{agreement-ex04})). The analytic comparative form with borrowed \textit{upro} is only found in the more contemporary sources. Sources from before 1985 use the synthetic comparative \textit{-xu}.

\begin{exe}
	\ex\label{agreement-ex04}
	\begin{xlist}
		
		
			\ex\label{agreement-ex04a}
			\gll k'ui-š q'arc'eⁿ tut tut-i-loħ \textbf{ħamaxeɁ} čamliⁿ j-a.  \\
			white-{\Adv} variegated mulberry mulberry-{\Pl}-{\Interlat} \textbf{{\Superl}} tasty {\J}-be \\
			\trans `The white mulberry is the tastiest among the mulberries.' \\
			\hfill (KK008-1711)
		
		
		
			\ex\label{agreement-ex04b}
			\gll ħamaxeɁ sabu-xǔ xalx bze-n č'irba-l-a. \\
			{\Superl}	exceeding-{\Cmp} people chaff-{\Dat} need-{\Intr}-{\Npst}\\
			\trans `Most people need chaff.'
			\hfill (YD009-13.1)
		
		
	\end{xlist}
\end{exe}

The data is best explained by assuming a borrowing scenario where Tsova-Tush calqued the analytic superlative construction and directly borrowed the comparative marker \textit{upro} from Georgian. Note that these events likely happened at different stages: based on our textual material, the calquing of the superlative construction happened sometime between 1850 and 1950, whereas the borrowing of comparative \textit{upro} happened after 1950.

\subsection{Borrowed modifiers} \label{adjadapt}\is{Loanword adaptation!Adjectives}

Adjectives are freely borrowed from Georgian. The dictionary of \textcite[]{kadkad84} features 149 Georgian and 95 native underived adjectives. A representative sample (WOLD\footnote{\url{https://wold.clld.org/}}, \cite{haspelmathtadmor09wold}), however, shows only 13 borrowed adjectives versus 122 native adjectives (many of which are derived from verbs or nouns, see \textcite{WS}). The stark difference between these two samples can be attributed to language use: the WOLD sample aims to capture all frequently used concepts expressed by adjectives, while the dictionary aims to be exhaustive and lists many more Georgian adjectives as a result.

Simple, underived Georgian adjectives are borrowed similarly to Georgian nouns, i.e. in the stem form, without the nominative marker \textit{-i}, for example \textit{blant'} `sticky' (Georgian \textit{blant'-i}). If the Georgian adjective ends in a vowel, Apocope applies as it does to native forms (see \sectref{processes}), such as in \textit{ašk'ar} `obvious' (Georgian \textit{ašk'ara}), \textit{laq'} `rotten, spoiled' (Georgian \textit{laq'e}), \textit{toxl\u{o}} `soft-boiled' (Georgian \textit{toxlo}).



Borrowed adjectives that contain Georgian derivational morphology can be adopted wholesale, or the derivational affixes can be replaced by native morphology. Thus, we find privative adjectives containing the Georgian circumfix \textit{u- -o}: \textit{ubarak\u{o}} `fruitless, unrewarding' (Georgian \textit{u-barak-o}), \textit{uk'anon\u{o}} `lawless' (Georgian \textit{u-k'anon-o}) next to forms with the native privative marker \textit{-c'in} : \textit{gunb-e-c'iⁿ} `having a bad mood' (Georgian \textit{u-guneb-o}), \textit{meml-e-c'iⁿ} `landless' (Georgian \textit{u-mamul-o}). The same can be observed for the Georgian denominal suffix \textit{-ian} `having X'. We find wholesale borrowings (e.g. \textit{nemseⁿ} `honest, conscientious' (Georgian \textit{namus-ian-i}), \textit{k'uizeⁿ} `hunchbacked' (Georgian \textit{k'uz-ian-i})), as well as replacement of the Georgian suffix by native \textit{-aren} (e.g. \textit{čedn-areⁿ} `fern-covered' (Georgian \textit{čadun-ian-i}), \textit{k'uižr-aleⁿ} (with R-dissimilation, see \sectref{rdissim}) `callous, with callus' (Georgian \textit{k'užr(-eb)-ian-i}).

Several adjectives borrowed from Georgian have received an ending \textit{-оn}. They are \textit{xširoⁿ/qširoⁿ} `frequent' (Northeastern Georgian \textit{qšir-i}, Standard Georgian \textit{xšir-i}), \textit{xabroⁿ} `greedy, stingy' (Georgian \textit{xarb-i}), \textit{ʒviroⁿ} `expensive' (Georgian \textit{ʒvir-i}), \textit{parsoⁿ} `useful, proper' (Georgian \textit{parsag-i, parsak'a}). Many of these adjectives betray a greater age than the endingless borrowings described above: \textit{qširoⁿ} has an initial consonant \textit{q} no longer found in Modern Standard Georgian; \textit{xabroⁿ} shows metathesis which is absent in more recent loans (cf. \textit{darbaz} `hall' from Georgian \textit{darbaz-i}); and \textit{parsoⁿ} seems to be borrowed from an unattested Georgian form \textit{*pars-}, which formed the basis of the Georgian formation \textit{pars-ak'-a} `good, useful'.\footnote{Compare e.g. Old Georgian \textit{dan-ak'-i}  and \textit{dana}, both meaning `knife'.} It is therefore assumed that these adjectives were borrowed at an earlier stage. After having received their \textit{-on} ending, they were fully integrated into the agreement system, receiving the ending \textit{-čo} when the head noun is in a non-Nominative case, see Example (\ref{simplenp-ex54}).

\begin{exe}
	\ex\label{simplenp-ex54}
	\gll \textbf{xabro-č\u{o}} st'ak'o-n ħaš co d-ec'-\u{e}. \\
	\textbf{stingy-{\Obl}} man.{\Obl}-{\Dat} guest {\Neg} {\D}-like-{\Npst}\\
	\trans `A stingy man does not like guests.'
	\hfill (KK032-5081)
\end{exe}


The Georgian adjectivising suffix \textit{-ur} seems to have been borrowed into Tsova-Tush, but its productivity is limited. It is found in borrowed adjectives (\textit{st'erilur} `sterile' (Georgian \textit{st'erilur-i}), \textit{kimiur} `chemical' (Georgian \textit{kimiur-i}) and many others) and the suffix is often found deriving adjectives from ethnonyms. Many of these are borrowed from Georgian entirely (e.g. \textit{lek'ur} `Daghestani' (Georgian \textit{lek'-ur-i}), \textit{kist'ur} `Kisti' (Georgian \textit{kist'-ur-i})), but some are based on native Tsova-Tush stems, such as \textit{pxe-ur} `Khevsur'\footnote{Potentially a borrowing of Georgian \textit{pxouri} `Pshav-Khevsur'.} (cf. Georgian \textit{xevsur-ul-i}), \textit{k'ox-ur} `Georgian' (cf. Georgian \textit{k'ax-ur-i} `Kakhetian'), \textit{nemc-ur} `German' (from Russian \textit{nemec}, cf. Georgian \textit{german-ul-i}), and crucially \textit{bacb-ur} `Tsova-Tush'. Thus, one can conclude that the suffix \textit{-ur} has very limited productivity, deriving adjectives from ethnonyms that are not already borrowed from Georgian, of which the four adjectives above might be the only examples.\footnote{One exception is Tsova-Tush \textit{lamzur} `beautiful' (cf. Georgian \textit{lamaz-i}), which must be an old borrowing from a Northeastern Georgian dialect, although it is absent from the standard dictionaries and corpora of Tush, Khevsur, Pshav and Mokhevi dialects.} Adjectives containing the suffix \textit{-ur} do not agree in case, see \sectref{caseagree}.

Besides adjectives, only one other modifier is borrowed: \textit{q'ovel} `every' (Georgian \textit{q'ovel-i}), see \sectref{quantifiers}.


\section{Agreement} \label{agreement}

Tsova-Tush shows gender, case and (albeit marginal) number agreement of modifiers with their head. The exact same patterns (gender agreement of a limited number of adjectives and of the numeral `four'; a distinction between Nominative and Oblique forms of the adjective, demonstrative and of the numerals `one', `two' and `three'; and the adjectives `big' and `small' showing number agreement) are found in the other Nakh languages Ingush and Chechen. Georgian, on the other hand, does not have grammatical gender, does not show number agreement on modifiers, and features an entirely different system of case agreement (see for example \cite[45--60]{hewitt95}).


\subsection{Gender agreement} \label{genderagree}\is{Gender agreement}\is{Noun class}\is{Agreement!Gender}\is{Gender agreement!on adjectives}

In Tsova-Tush, 11 underived adjectives, one underived numeral, and one underived quantifier inflect for gender.\footnote{Besides one preverb, and approximately a third of all underived verbs.
} These words are listed in Table \ref{simplenp-table1}, where \textit{d-} stands for any gender prefix. Additionally, all participles that are derived from verbs that inflect for gender, keep this agreement pattern as participles. 

\begin{table}
	\begin{tabular}{ll}
		\lsptoprule
		\textit{d-aviⁿ} & `light' \\
		\textit{d-aseⁿ} & `empty' \\
		\textit{d-apxeⁿ} & `hot' \\
		\textit{d-aq'iⁿ} & `dry' \\
		\textit{d-acuⁿ} & `short' \\
		\textit{d-ac'iⁿ} & `heavy' \\
		\textit{d-arst'iⁿ} & `fat' \\
		\textit{d-axxeⁿ} & `long' \\
		\textit{d-aqqoⁿ} & `big' \\
		\textit{d-ut'q'iⁿ} & `thin' \\
		\textit{d-uq'iⁿ} & `thick, dense, frequent' \\\addlinespace
		\textit{d-ʕivɁ} & `four' \\\addlinespace
		\textit{d-aniɁ} & `all, everyone' \\
		\lspbottomrule
	\end{tabular}
	\caption{Underived modifiers inflecting for gender}
	\label{simplenp-table1}
\end{table}

All adjectives are inherited from Proto-Nakh, as indicated by their Vainakh cognates,\footnote{See Chechen \textit{d-ajn} `light',  \textit{d-\={e}sa} `empty', \textit{d-ouxa} `warm', \textit{d-eq'a} `dry', \textit{d-\={o}ca} `short',  \textit{d-eza} `heavy', \textit{d-erstana} `fat', \textit{d-\={e}xa} `long', \textit{d-oqqa} `big', \textit{d-ut'q'a} `thin', \textit{d-uq'a} `thick, dense'. Chechen also features  \textit{d-erzina} `naked' (compare Tsova-Tush \textit{d-arc'in\u{o}} `id.'), \textit{d-\={o}raxa} `cheap', \textit{d-yzna} `full' (compare Tsova-Tush \textit{d-uc'in\u{o}}). The Tsova-Tush forms are Past Participles.} and are usually reconstructed for Proto-East-Caucasian, including the prefixal gender slot. The numeral `four' is also reconstructed for Proto-East-Caucasian as containing a gender slot, which some languages have retained, and some have fossilised. \textit{d-aniɁ} `all' seems to be derived from the Present Participle of the verb `be' \textit{d-ani} (\textit{dejn} after Apocope and Umlaut (see \sectref{processes})).\largerpage


All the items in Table \ref{simplenp-table1} take one of four gender prefixes to agree with their head noun in one of five genders. Three of these genders (M, F, B) require different prefixes for singular and plural, which leads some scholars to describe this type of agreement as gender-number agreement. The five genders and the prefixes they require on the agreement target are listed in Table \ref{simplenp-table2}, along with the corresponding glosses used in this work. In much work on Nakh languages, apart from M for masculine and F for feminine, the labels for the genders are based on their marker in the singular. This contrasts with the tradition of labelling gender in Daghestanian languages, which uses Roman numerals (where the same Roman numeral can refer to a different set of agreement affixes in different languages).

 Some 30 nouns show one of three agreement patterns that are different from any of these five genders (see \cite[163]{holiskygagua}, \cite[170--172]{corbett91}).\is{Inquorate gender} Alternatively, these 3 agreement patterns can be said to constitute three more genders, which enables the claim that Tsova-Tush has in fact 8 genders (see e.g. \cite{chrelashvili67,haukharris}). Deciding which of these analyses (5 or 8 genders) is more appropriate falls beyond the scope of this work.


For gender assignment, i.e. the rules that dictate which nouns belong to which gender, and for the adaptation of Georgian loanwords using these rules, see \textcite{WS} and \sectref{genderadapt}. 

\begin{table}
	\begin{tabular}{ccccc}
    \lsptoprule
		Gender & {{\Sg}} & {{\Pl}} & \multicolumn{2}{c}{{Glosses}}  \\
		\midrule
		
		M & \textit{v-} & \textit{b-} & \textsc{m.sg} & \textsc{m.pl} \\
		F & \textit{j-} & \textit{d-} & \textsc{f.sg} & \textsc{f.pl}\\
		B & \textit{b-} & \textit{d-} & \textsc{b.sg} & \textsc{b.pl}\\
		D & \textit{d-} & \textit{d-} & \multicolumn{2}{c}{\textsc{d}} \\
		J & \textit{j-} & \textit{j-} & \multicolumn{2}{c}{\textsc{j}} \\
		
		\midrule
		Bb & \textit{b-} & \textit{b-} & \textsc{b} & \textsc{b} \\
		
		Bj & \textit{b-} & \textit{j-} & \textsc{b} & \textsc{j} \\
		
		Dj & \textit{d-} & \textit{j-} & \textsc{d} & \textsc{j} \\
		\lspbottomrule
		
	\end{tabular}
	\caption{Tsova-Tush genders}
	\label{simplenp-table2}
\end{table}

Example (\ref{agreement-ex03}) shows gender agreement on adjectives (\ref{agreement-ex03a}, \ref{agreement-ex03b}), the numeral `four' (\ref{agreement-ex03c}), and the quantifier \textit{d-aniɁ} `all' (\ref{agreement-ex03d}).\largerpage[2]

\begin{exe}
	\ex\label{agreement-ex03}
	\begin{xlist}
		
		
			\ex\label{agreement-ex03a}
			\gll {{\normalfont[}cħa} \textbf{b-aqqoⁿ} {pst'u{\normalfont]}} b-a-r. \\
			one \textbf{{\B}.{\Sg}-big} ox(B) {\B}.{\Sg}-be-{\Imprf} \\
			\trans `There was one big ox.'
			\hfill (E153-23)
		
		
		
			\ex\label{agreement-ex03b}
			\gll k'eč' lark'-ar {{\normalfont[}žorejš} \textbf{j-ac'iⁿ} {sakm{\normalfont]}} j-a.\\
			wool shear-{\Vn} very \textbf{{\J}-heavy} business(J) {\J}-be \\
			\trans `Sheep-shearing is very hard work.'
			\hfill (E006-87)
		   
		
		
			\ex\label{agreement-ex03c}
			\gll {{\normalfont[}\textbf{d-ʕivɁ}} {bader{\normalfont]}} ħal d-aq-d-i-en-es. \\
			\textbf{{\D}-four} child(D) {\Pv} {\D}-raise-{\D}-{\Tr}-{\Aor}-{\Fsg}.{\Erg} \\
			\trans `I have raised four children.'
			\hfill (E130-32)
		   
		
		
			\ex\label{agreement-ex03d}
			\gll {{\normalfont[}\textbf{b-aniɁ}} {žab{\normalfont]}} =e, cħa doⁿ, t'q'a qo sa d-ik'-eⁿ txo-go. \\
			\textbf{{\B}.{\Sg}-all} cattle(B) =and one horse twenty three soul {\D}-take.{\Anim}-{\Aor} {\Fpl}.{\Excl}-{\Adess} \\
			\trans `All the cattle and one horse, twenty three heads they took from us.' \\
			\hfill (E041-42)
		
		
		
		
	\end{xlist}
\end{exe}

\subsection{Case agreement} \label{caseagree}\is{Agreement!Case}

Tsova-Tush shows rudimentary case agreement of adjectives (see Examples (\ref{agreement-ex01a}, \ref{agreement-ex01b})), ordinal numerals (\ref{agreement-ex01c}, \ref{agreement-ex01d})\footnote{Original orthography of (\ref{agreement-ex01d}): e \.{s}il\.{g}e\.{c}o deniḥ wee o\.{s}ti saqdri.} and participles (see \sectref{relative}) with their head noun. These modifiers appear in their bare form when modifying nouns in the Nominative (Examples (\ref{agreement-ex01a}, \ref{agreement-ex01c})), but receive the suffix \textit{-čo} when they modify a noun in any other case (\ref{agreement-ex01b}, \ref{agreement-ex01d}). Note that the final \textit{-n} in adjectives drops before \textit{č}, not nasalising the preceding vowel.\is{Nominative case}\is{Oblique stems}

\begin{exe}
	\ex\label{agreement-ex01}
	\begin{xlist}
		
		\ex\label{agreement-ex01a}
		\gll k'mat'-xiⁿ \textbf{pšeliⁿ} xi žura-l-ar.\\
		cliff-{\Apudabl} \textbf{cold} water seep-{\Intr}-{\Imprf}\\
		\trans `Cold water was seeping from the cliff.'
		\hfill (KK017-3211)
		
		\ex\label{agreement-ex01b}
		\gll ħal d-aq-o-etx o k'arak, \textbf{pšeli-č} xi-l ču d-oɬ-o-tx.\\
		up {\D}-take-{\Npst}-{\Fpl}.{\Erg} {\Dist} butter \textbf{cold-{\Obl}} water-{\Interlat} in d-put-{\Npst}-{\Fpl}\\
		\trans `We take the butter out [and] put it in cold water.'
		\hfill (E019-80)
		
		\ex\label{agreement-ex01c}
		\gll giorgi-s krist'ad-v-i-eⁿ seⁿ \textbf{ši-l\u{g}eⁿ} voħ.\\
		Giorgi-{\Erg} baptise-{\M}.{\Sg}-{\Tr}-{\Aor} {\Fsg}.{\Gen} \textbf{two-{\Ord}} boy\\
		\trans `Giorgi baptised my second son.'
		\hfill (E220-10)
		
		\ex\label{agreement-ex01d}
		\gll je \textbf{ši-l\u{g}e-čo} deni-ħ v-eɁ-eⁿ ošt'iɁ saq'dr-i.\\
		and \textbf{two-{\Ord}-{\Obl}} day.{\Obl}-{\Ess} {\M}.{\Sg}-come.{\Ipfv}-{\Aor} again temple-{\Ill}\\
		\trans `And on the second day, he came to the temple again.'
		\hfill (AS003-1.2)  
%large space between \gll and \trans in pdf		
	\end{xlist}
\end{exe}


Simple, underived borrowed adjectives do not agree in case like native adjectives. Thus, in Example (\ref{simplenp-ex53}), the adjective \textit{ubral\u{o}} `simple' (from Georgian \textit{ubralo}) does not receive the suffix \textit{-čo}, even though the head noun is in a non-Nominative case.

\begin{exe}
	\ex\label{simplenp-ex53}
	\gll \textbf{ubral\u{o}} st'eme-n=mak=daħ dok' co d-ec'-\u{e} ħoⁿ lek-d-al-aⁿ. \\
	\textbf{simple} thing.{\Obl}-{\Dat}=on=from heart {\Neg} {\D}-must-{\Npst} {\Ssg}.{\Dat} throw.{\Ipfv}-{\D}-{\Intr}-{\Inf} \\
	\trans `You should not be heartbroken due to simple things.'
	\hfill (KK021-3663)
\end{exe}

Adjectives formed with the derivational morpheme \textit{-ur} (which is of Georgian origin, see \sectref{adjadapt}) also do not add the Oblique marker \textit{-čo} when the head noun is non-Nominative (\cite[13]{desheriev53}), as in (\ref{agreement-ex02b}, \ref{agreement-ex02d}).\footnote{Exception to this rule is \textit{lamzur} `beautiful', which does take the Oblique marker. The ending of \textit{lamzur}, however, is synchronically not a derivational suffix.}

\begin{exe}
	\ex\label{agreement-ex02}
	\begin{xlist}
		
		\ex\label{agreement-ex02a}
		\gll oqriⁿ badr-i-n \textbf{bacbur} mot't' co qet. \\
		{\Dist}.{\Pl}.{\Gen} child-{\Pl}-{\Dat} \textbf{Tsova\_Tush} language {\Neg} know \\
		\trans `Their children don't know Tsova-Tush.'
		\hfill (E118-15)
		
		\ex\label{agreement-ex02b}
		\gll \textbf{bacbur} že-v cħa čuix d-∅-o. \\
		\textbf{Tsova\_Tush} sheep-{\Erg} one lamb {\D}-give\_birth-{\Npst} \\
		\trans `Tsova-Tush sheep give birth to (only) one lamb.'
		\hfill (E002-27)
		
		\ex\label{agreement-ex02c}
		\gll daħ d-ot'-e\textsuperscript{n} \textbf{k'oxur} mat't'a-n=mak.  \\
		{\Pv} {\D}-go.{\Ipfv}-{\Aor} \textbf{Georgian} language.{\Obl}-{\Dat}=on \\
		\trans `They changed over to the Georgian language.'
		\hfill (E019-149)
		
		\ex\label{agreement-ex02d}
		\gll \textbf{k'oxur} mot't' uišt'=aj=c\u{\i} qet šun=a. \\
		\textbf{Georgian} language so.{\Dist}={\Add}={\Aff} know({\Npst}) 2{\Pl}.{\Dat}={\Emph} \\
		\trans `We know the Georgian language just as well.'
		\hfill (MM371-1.5)
		
	\end{xlist}
\end{exe}

Demonstrative adjectives, too, inflect for case. The Nominative and Oblique stems are shown in Table \ref{Obliquedem}, although Nominative demonstratives are sometimes also found with non-Nominative nouns.

\begin{table}
	\begin{tabular}{llll}
    \lsptoprule
		& {Proximal} & {Medial} & {Distal} \\ 
		\midrule
		Nominative & e & is & o \\
		Oblique & eq & icx & oq \\
		\lspbottomrule
	\end{tabular}
	\caption{Nominative and Oblique demonstrative adjectives}
	\label{Obliquedem}
\end{table}

Finally, the same distinction is found in the numerals `1', `2' and `3', for which see Table \ref{Obliquenum}.

\begin{table}
	\begin{tabular}{llll}
    \lsptoprule
		& `one' & `two' & `three' \\ 
		\midrule
		Nominative & cħa & ši & qo \\
		Oblique & cħani & šin & qa \\
        \lspbottomrule
	\end{tabular}
	\caption{Nominative and Oblique numerals}
	\label{Obliquenum}
\end{table}




No numeral higher than `3' and no quantifier agrees in case with its head noun. Personal and reflexive pronouns used as modifiers seem to make a Nominative-Oblique distinction as well, for which see Table \ref{tablepossadj} in \sectref{Genitive}.

\subsection{Number agreement}\is{Agreement!Number}

Only two adjectives agree in number with their head noun. These are \textit{k'ac'k'oⁿ} `small' (see Examples (\ref{simplenp-ex11a},\ref{simplenp-ex11b})) and \textit{d-aqqoⁿ} `big' (Examples (\ref{simplenp-ex11c},\ref{simplenp-ex11d})), which have plural forms \textit{k'ac'kaⁿ} and \textit{d-aqqaⁿ}, respectively. One possible historical explanation for the fact that only these two adjectives show this type of agreement can be found in the origin of `small'. This adjective, \textit{k'ac'k'oⁿ}, is in fact historically a derivation of \textit{k'aʒak'} `a little bit'.\footnote{Compare the variant form \textit{k'ic'k'oⁿ} `little', from the more common form of the word for `a little bit', \textit{k'aʒik'}.} The origin of the adjectival \textit{-aⁿ} suffix can be compared directly to the Genitive plural noun suffix \textit{-aⁿ}. The plural \textit{d-aqqaⁿ} `big' must then have arisen analogically to \textit{k'ac'k'aⁿ}.

\begin{exe}
	\ex\label{simplenp-ex11}
	\begin{xlist}
		
		\ex\label{simplenp-ex11a}
		\gll \textbf{k'ac'k'aⁿ} badr-i ma, tejlz-a-x ču xabž-d-i-noer. \\
		\textbf{little.{\Pl}} children-{\Pl} but, saddle\_bag-{\Obl}.{\Pl}-{\Cont} {\Pv} sit\_down-{\D}-{\Tr}-{\Nw}.{\Rem}.{\Seq} \\
		\trans `The little children, however, were apparently put in saddle-bags.' \\
		\hfill (MM122-2.13)
		
		\ex\label{simplenp-ex11b}
		\gll uiš zorajš\u{\i} v-ec'-er txo\textsuperscript{} \textbf{k'ac'k'aj-č\u{o}} badr-i-n ik'i, me ... \\
		so.{\Dist} very {\M}.{\Sg}-love-{{\Imprf}} {\Fpl}.{\Excl}.{\Dat} \textbf{little.{\Pl}-{\Obl}} child-{\Pl}-{\Dat} Iki {\Subord} [...] \\
		\trans `We little children loved Iki so much, that [...].'
		\hfill (MM208-1.9)
		
		\ex\label{simplenp-ex11c}
		\gll ai ošruⁿ \textbf{d-aqqaⁿ} pħarč d-a-r kikoɁ bac-bi-go. \\
		look such.{\Dist} \textbf{{\B}.{\Pl}-big.{\Pl}} dog.{\Pl} {\B}.{\Pl}-be-{{\Imprf}} early Tsova\_Tush-{\Pl}-{\Adess} \\
		\trans `Look, such big dogs the Tsova-Tush used to have in the old days.' \\
		\hfill (E011-17)
		
		\ex\label{simplenp-ex11d}
		\gll \textbf{b-aqqej-č\u{o}} važar-g\u{o} lajv-nor: ... \\
		\textbf{{\M}.{\Pl}-big.{\Pl}-{\Obl}} brother.{\Pl}-{\All} tell-{\Nw}.{\Rem} [...] \\
		\trans `He told his older brothers: [...].'
		\hfill (MM407-1.5)
		
	\end{xlist}
\end{exe}

No other modifier, including other adjectives, agrees in number with their head noun. The same form of the demonstrative (\ref{simplenp-ex12a}, \ref{simplenp-ex12b}) or adjective (\ref{simplenp-ex12c}, \ref{simplenp-ex12d}) is used, whether plural (\ref{simplenp-ex12a}, \ref{simplenp-ex12c}) or singular (\ref{simplenp-ex12b}, \ref{simplenp-ex12d}).

\begin{exe}
	\ex\label{simplenp-ex12}
	\begin{xlist}
		\ex\label{simplenp-ex12a}
		\gll txo-go \textbf{e} sxal-i wumaɁ d-a-r. \\
		{\Fpl}.{\Excl}-{\Adess} \textbf{{\Prox}} pear-{\Pl} all {\D}-be-{{\Imprf}} \\
		\trans `We had all these pears.'
		\hfill (E015-24)
		
		\ex\label{simplenp-ex12b}
		\gll \textbf{e} qor mac'riⁿ b-a-r=e. \\
		\textbf{{\Prox}} apple sweet {\B}.{\Sg}-be-{{\Imprf}}=and \\
		\trans `This apple is sweet.'
		\hfill (E015-52)
		
		\ex\label{simplenp-ex12c}
		\gll c'egeⁿ \textbf{lamzur} qor-i d-a xabž-d-i-en\u{o}. \\
		red \textbf{beautiful} apple-{\Pl} {\B}.{\Pl}-be sit\_down-{\B}.{\Pl}-{\Tr}-{\Ptcp}.{\Aor} \\
		\trans `Beautiful red apples had been put (here: skewered).'
		\hfill (EK021-3.7)
		
		\ex\label{simplenp-ex12d}
		\gll daxaɁ\u{o} lamu-ⁿ \textbf{lamzur} ħac'uk' d-a. \\
		snowcock mountain-{\Gen} \textbf{beautiful} bird {\D}-be \\
		\trans `A snowcock is a beautiful mountain bird.'
		\hfill (KK004-1099)
		
	\end{xlist}
\end{exe}




\section{Complex noun phrases} \label{complexnp}

In Tsova-Tush, nouns can be modified by other noun phrases by means of the Genitive, as well as juxtaposition of uninflected nouns.


\subsection{Genitive modification} \label{Genitive}

In Genitive constructions, the modifier (being the dependent) is marked with the Genitive case, while the head noun shows no marking to indicate the Genitive relation. See \sectref{nouns} for the formation of the Genitive. Modifying nouns in the Genitive generally precede the head noun, as in (\ref{complexnp-ex01}). No formal distinction is made between possession (\ref{complexnp-ex01a}, \ref{complexnp-ex01b}), materials (\ref{complexnp-ex01c}), kinship (\ref{complexnp-ex01d}) or part-whole relations (\ref{complexnp-ex01e}).\is{Genitive case}

\begin{exe}
	\ex\label{complexnp-ex01}
	\begin{xlist}
		
		\ex\label{complexnp-ex01a}
		\gll šaroⁿ dak'lav, moħ d-aqqoⁿ d-ec'-er xiɬ-a\textsuperscript{} seⁿ \textbf{babui-ⁿ} pardul. \\
		{\Refl}.{\Emph} think.{\Pfv} how {\D}-big {\D}-must-{\Imprf} be.{\Pfv}-{\Inf} {\Fsg}.{\Gen} \textbf{grandfather-{\Gen}} barn \\
		\trans `Imagine, how big my grandfather's barn must have been.' \\
		\hfill (WS001-11.11)
		
		
		\ex\label{complexnp-ex01b}
		\gll c'ʕerkoⁿ \textbf{st'ak'i-ⁿ} tataⁿ xac'-eⁿ soⁿ. \\
		suddenly \textbf{man.{\Obl}-{\Gen}} voice hear-{\Aor} {\Fsg}.{\Dat} \\
		\trans `Suddenly, I heard a man's voice.'
		\hfill (E060-16)
		
		\ex\label{complexnp-ex01c}
		\gll is cent'r-e \textbf{ʕaixk'-e-ⁿ} st'olba amarto-l-aera-l\u{o} laq-iš. \\
		there.{\Med} center-{\Obl}({\Ess}) \textbf{iron-{\Obl}-{\Gen}} pole erect-{\Intr}-{\Imprf}-{\Sbjv} high-{\Adv} \\
		\trans `There in the center, an iron pole was erected up high.'
		\hfill (E115-46)
		
		\ex\label{complexnp-ex01d}
		\gll qa \textbf{badr-e-ⁿ} ajtaɁ\u{o} v-a-s\u{o}.  \\
		three.{\Obl} child-{\Obl}-{\Gen} godparent {\M}.{\Sg}-be-{\Fsg}.{\Nom} \\
		\trans `I am a godfather to three children.'
		\hfill (E001-61)
		
		\ex\label{complexnp-ex01e}
		\gll \textbf{gutn-e-ⁿ} c'ʕop' st'en-ax-čo-x uill-d-is-eⁿ. \\
		plow-{\Obl}-{\Gen} tip what.{\Obl}-{\Indf}-{\Obl}-{\Cont} stick-{\D}-stay-{\Aor} \\
		\trans `The tip of the plow got stuck in something.'
		\hfill (WS001-10.4)
		
		
	\end{xlist}
\end{exe}



Genitives of personal, reflexive and demonstrative pronouns also behave in the same way syntactically, namely, they precede the head noun. \textcite{haukharris} claim that the personal and reflexive pronouns in the Genitive show case agreement, where the nasalisation is dropped when the head noun is in a non-Nominative case, see Table \ref{tablepossadj}. Although many counterexamples have been found, this claim seems to hold statistically, as in Examples (\ref{complexnp-ex02a}--\ref{complexnp-ex02d}). Genitives of demonstrative pronouns make no such distinction (\ref{complexnp-ex02e}--\ref{complexnp-ex02f}).\is{Agreement!Case}


\begin{table}
	\begin{tabular}{llllllll}
    \lsptoprule
		& \multicolumn{3}{c}{{Demonstrative}} & \multicolumn{2}{c}{{Personal}} & \multicolumn{2}{c}{{Reflexive}}  \\\cmidrule(lr){2-4}\cmidrule(lr){5-6}\cmidrule(lr){7-8} 
		
		& \textsc{{prox}} & \textsc{{med}} & \textsc{{dist}} &  \textsc{{nom}} & \textsc{{obl}} & \textsc{{nom}} & \textsc{{obl}} \\
		
		\midrule
		
		{\Fsg} & & & & seⁿ & se & saiⁿ & sai, se \\
		
		{\Ssg} & & & & ħeⁿ & ħe & ħaiⁿ & ħai, ħe \\
		
		{\Tsg} & equiⁿ & icxuiⁿ & oquiⁿ &  &  & šariⁿ & šer \\
		
		\midrule
		
		{\Fpl}.{\Incl} & & & & vaiⁿ & ve & vaiⁿ & ve \\
		
		{\Fpl}.{\Excl} & & & & txeⁿ & txe & txaiⁿ & txe \\
		
		2{\Pl} & & & &  šuⁿ & šu & šuiⁿ & šui \\
		
		3{\Pl} & eqriⁿ & icxriⁿ & oqriⁿ & & & šuiⁿ & šui \\
		\lspbottomrule
	\end{tabular}
	\caption{Tsova-Tush pronouns in the Genitive case}
	\label{tablepossadj}
\end{table}

\begin{exe}
	\ex\label{complexnp-ex02}
	\begin{xlist}
		
		\ex\label{complexnp-ex02a}
		\gll macn-e \textbf{ħeⁿ} herc'\u{o} so-ciⁿ d-a-r=ici, daħ čul-d-al-in=\u{e}. \\
		when-{\Rel} \textbf{{\Ssg}.{\Gen}} pot {\Fsg}-{\Apudess} {\D}-be-{\Imprf}={\Subord}, {\Pv} get\_pregnant-{\D}-{\Intr}-{\Aor}=and \\
		\trans `When your pot was with me, it got pregnant.'
		\hfill (EK057-4.1)
		
		\ex\label{complexnp-ex02b}
		\gll \textbf{ħe} vaš-e-ⁿ korc'il macaⁿ d-a=jn\u{o}. \\
		\textbf{{\Ssg}.{\Gen}.{\Obl}} brother-{\Obl}-{\Gen} wedding when {\D}-be={\Quot} \\
		\trans `{``}When is your brother's wedding?{''}'
		\hfill (EK059-2.4)
		
		\ex\label{complexnp-ex02c}
		\gll o \textbf{saiⁿ} doⁿ ior\u{g} doⁿ b-a-r. \\
		{\Dist} \textbf{{\Fsg}.{\Refl}.{\Gen}} horse palfrey horse {\B}.{\Sg}-be-{\Imprf} \\
		\trans `My own horse was a palfrey.'
		\hfill (EK015-2.1)
		
		\ex\label{complexnp-ex02d}
		\gll as \textbf{saj} v-ec'-r-e-x=a ħerč-in=a. \\
		{\Fsg}.{\Erg} \textbf{{\Fsg}.{\Refl}.{\Gen}.{\Obl}} {\M}.{\Sg}-love-{\Vn}-{\Obl}-{\Cont}={\Emph} embrace-{\Aor}={\Emph} \\
		\trans `I embraced my lover.'
		\hfill (EK028-6.1)
		
		\ex\label{complexnp-ex02e}
		\gll \textbf{equi-ⁿ} dip'lom daco-d-i-n-e-s. \\
		\textbf{{\Med}.{\Obl}-{\Gen}} diploma defend-{\D}-{\Tr}-{\Aor}-{\Seq}-{\Fsg}.{\Erg} \\
		\trans Lit.:`I defended the diploma in it, and...' 
		\hfill (E096-10)
		
		\ex\label{complexnp-ex02f}
		\gll \textbf{equi-ⁿ} dad-e-n qet-ra-lo=e. \\
		\textbf{{\Prox}.{\Obl}-{\Gen}} father-{\Obl}-{\Dat} know-{\Imprf}-{\Sbjv}=and \\
		\trans `His father knew it.'
		\hfill (E206-93)
		
	\end{xlist}
	
	
\end{exe}

If a Genitive noun modifies a noun phrase that names a quantity, measure or container, the Genitive modifier follows this noun, as in (\ref{complexnp-03}a,b). The Genitive modifier in these instances is the controller in terms of gender, since agreement on the verb is with the gender of the Genitive noun, not of the noun that signifies the measure or container. See Example (\ref{complexnp-03a}), where the noun phrase has assumed gender J (indicated in the verbal agreement slot) from the Genitive of \textit{šur} `milk', while \textit{herc'\u{o}} is gender D. Similarly, in (\ref{complexnp-03b}), the noun phrase \textit{ši butt vadeⁿ} takes on gender J from \textit{vad} `time (limit)' and not from \textit{butt} `month', which is gender {\B}.

\begin{exe}
	\ex\label{complexnp-03}
	\begin{xlist}
		
		\ex\label{complexnp-03a}
		\gll {\normalfont[} cħa herc'\u{o} {\textbf{šur-e-ⁿ} {\normalfont]}} daħ maxk'-\textbf{j}-i-n-atx. \\
		{} one pot(D) \textbf{milk(J)-{\Obl}-{\Gen}} {\Pv} pour-\textbf{J}-{\Tr}-{\Aor}-{\Fpl}.{\Erg} \\
		\trans `We poured one pot of milk [out into sth].'
		\hfill (KK004-1115)
		
		\ex\label{complexnp-03b}
		\gll t'ateb d-ux d-erc'-d-∅-aⁿ {\normalfont[} ši butt {\textbf{vad-e-ⁿ} {\normalfont]}}  \textbf{j}-ajɬ-n-as. \\
		money {\D}-back {\D}-turn-{\D}-{\Tr}-{\Inf} {} two month(B) \textbf{time(J)-{\Obl}-{\Gen}} \textbf{J}-give-{\Aor}-{\Fsg}.{\Erg} \\
		\trans `I gave [sb] two months time to return the money.'
		\hfill (KK006-1370)
		
	\end{xlist}
\end{exe}



\subsection{Modification using uninflected nouns} \label{barenoun}\is{Oblique stems}\is{Bare noun modification}\is{Genitive case!Zero marked}

In Tsova-Tush, nouns that are not inflected for case can be used to modify other nouns. These modifying nouns always precede the head noun. Several examples can be seen in (\ref{complexnp-ex04}).

\begin{exe}
	\ex\label{complexnp-ex04}
	\begin{xlist}
		
		\ex\label{complexnp-ex04a}
		\gll \textbf{alazaⁿ} cer-e-ħ bat'-a-x j-aqqou-vx savt-i ʕa-j-axk'-er. \\
		\textbf{Alazani} edge-{\Obl}-{\Ess} goose-{\Obl}.{\Pl}-{\Cont} {\J}-big-{\Cmp} great\_bustard-{\Pl} sit-{\J}-{\Lv}.{\Pl}-{\Imprf}    \\
		\trans `On the banks of the Alazani are sitting great bustards bigger than geese.'
		\hfill (KK019-3292)
		
		\ex\label{complexnp-ex04b}
		\gll aklm-a-ⁿ j-aqqoⁿ karvaⁿ j-ot'-ur \textbf{širvaⁿ} duz-e-\u{g}=aħ\u{o}.    \\
		camel-{\Obl}.{\Pl}-{\Gen} {\J}-big caravan {\J}-go-{\Imprf} \textbf{Shirvan} grassland-{\Obl}-{\Trans}=through        \\
		\trans `A big camel caravan was going through the Shirvan steppe.'
		\hfill (KK023-3948)
		
		\ex\label{complexnp-ex04c}
		\gll  \textbf{k'ak'al} nač'uč'-i-loħ t'um lex-o-s. \\
		\textbf{walnut} shell-{\Pl}-{\Interess} kernel search.{\Ipfv}-{\Npst}-{\Fsg}.{\Nom}       \\
		\trans `I am looking for a kernel among the walnut shells.'
		\hfill (KK014-2981)
		
		\ex\label{complexnp-ex04d}
		\gll  \textbf{ba\u{g}} bolo-ħ psa b-iv-ur so-goħ. \\
		\textbf{garden} end-{\Ess} barley {\B}.{\Sg}-sow-{\Imprf} {\Fsg}-{\Adess}      \\
		\trans `At the end of my garden, barley was being sown.'
		\hfill (KK002-0550)
		
		
	\end{xlist}
\end{exe}

These modifying nouns all appear in the Oblique stem, but without case morphology, as exemplified by the examples in (\ref{complexnp-ex05}). It seems fitting, therefore, to analyse all examples in (\ref{complexnp-ex04}) as instances of Oblique stems, too. These nouns simply have an Oblique suffix \textit{-e} which is deleted word-finally, resulting in a situation where the Nominative and the caseless Oblique stem are formally identical.

\begin{exe}
	\ex\label{complexnp-ex05}
	\begin{xlist}
		
		\ex\label{complexnp-ex05a}
		\gll \textbf{att} leš-e-n=mak sov-i ʕe-j-axk'-er.  \\
		\textbf{cow.{\Obl}} corpse-{\Obl}-{\Dat}=on vulture-{\Pl} sit-{\J}-{\Lv}.{\Pl}-{\Imprf}  \\
		\trans `Vultures were sitting on top of the cow cadaver.'
		\hfill (KK038-5715) 
		
		\ex\label{complexnp-ex05b}
		\gll zoraj-č\u{o} \textbf{st'ajk'\u{\i}} tur-e-v zoraj-š\u{\i} tet'-\u{o}. \\
		brave-{\Obl} \textbf{man.{\Obl}} sword-{\Obl}-{\Erg} powerful-{\Adv} cut.{\Ipfv}-{\Npst} \\
		\trans `A brave man's sword cuts strongly.'
		\hfill (KK007-1479)
		
	\end{xlist}
\end{exe}    

The constructions seen in Examples (\ref{complexnp-ex04}) and (\ref{complexnp-ex05}) show superficial similarities with noun compounding, which is rarely found in Tsova-Tush. Example (\ref{complexnp-ex06}), however, shows that other modifiers can appear between the head and the bare noun modifier. More research is needed to explain which types of nouns can be used as uninflected nominal modifiers, and what the functional difference is to Genitive modifiers. 

\begin{exe}
	\ex\label{complexnp-ex06}
	\gll  \textbf{nin\u{o}} k'ac'k'o-xu-č' badr-e-x elan\u{e} c'-e. \\
	\textbf{Nino} small-{\Cmp}-{\Superl} child-{\Obl}-{\Cont} Elane be\_called-{\Npst}  \\
	\trans `Nino's youngest child is called Elane.'
	\hfill (KK011-2017)
\end{exe}




\subsection{Conjoined nouns} \label{conjoined}

A noun phrase may also consist of conjoined nouns. The additive particle \textit{=ae}/\textit{=a} is suffixed to each noun.\is{Conjunction}\is{Additive}\is{Conjoined nouns}

\begin{exe}
	\ex\label{complexnp-ex07}
	\begin{xlist}
		
		\ex\label{complexnp-ex07a}
		\gll mor-i \textbf{picr-i-\u{g}=a\u{e}}, \textbf{svet'-i-\u{g}=a\u{e}} daħ j-arl-iⁿ. \\
		log-{\Pl} \textbf{plank-{\Pl}-{\Trans}={\Add}} \textbf{pole-{\Pl}-{\Trans}={\Add}} {\Pv} {\J}-cut-{\Aor} \\
		\trans `They cut the logs into planks and poles.'
		\hfill (KK004-1115)
		
		\ex\label{complexp-ex07b}
		\gll \textbf{as-i=a\u{e}}, \textbf{du-j=a\u{e}} xi=mak daħ d-ik'-a-l\u{e}. \\
		\textbf{calf-{\Pl}={\Add}} \textbf{horse.{\Obl}-{\Pl}={\Add}} water=on {\Pv} {\D}-take.{\Anim}-{\Imp}-{\Hort} \\
		\trans `Please take the calves and the horses for water.'
		\hfill (KK004-1115)
		
		\ex\label{complexnp-ex07c}
		\gll k'edlo-x taru-j let-d-∅-o=e, mak ma \textbf{herc'-elč=a\u{e}} qen-i \textbf{bung-i=a\u{e}} d-exk'-\u{o}. \\
		wall-{\Cont} shelf-{\Pl} assemble-{\D}-{\Tr}-{\Npst}=and on {\Contr} \textbf{pot-{\Pl}={\Add}} other-{\Pl} \textbf{dish-{\Pl}={\Add}} {\D}-put-{\Npst} \\
		\trans `They nail the shelves to the wall and put the pots and the other dishes on top.'
		\hfill (KK008-1545)
		
	\end{xlist}
\end{exe}

When all conjoined nouns have the masculine gender, agreement with the (gender-inflecting) verb is with the masculine plural. See Example (\ref{complexnp-ex08a}), where the verb \textit{teg-b-i-r-as} has masculine plural gender, referring to the horse seller and the horse buyer. If one or more of the conjoined nouns have a non-masculine gender, the verb shows agreement with a default gender D, as in (\ref{complexnp-ex08b}--\ref{complexnp-ex08c}), even if all conjoined nouns belong to gender J (\ref{complexnp-ex08d}). When two or more nouns of the feminine gender are conjoined, the agreement marker \textit{d-} is used too, which can be interpreted as the feminine plural marker, or the default D gender (\cite[191]{holiskygagua}).\is{Agreement!Gender}

\begin{exe}
	\ex\label{complexnp-ex08}
	\begin{xlist}
		\ex\label{complexnp-ex08a}
		\gll doⁿ \textbf{b-exk'-ujn=a\u{e}}, doⁿ \textbf{ev-b-}∅\textbf{-ujn=a\u{e}} maxi-x teg-\textbf{b}-i-r-as. \\
		horse \textbf{{\B}.{\Sg}-sell-{\Ptcp}.{\Npst}={\Add}} horse \textbf{buy-{\B}.{\Sg}-{\Tr}-{\Ptcp}-{\Aor}={\Add}} price.{\Obl}-{\Cont} settle\_with-\textbf{{\M}.{\Pl}}-{\Tr}-{\Imprf}-{\Fsg}.{\Erg} \\
		\trans `I settled on a price with the horse seller and the horse buyer.' \\
		\hfill (KK008-1573)
		
		\ex\label{complexnp-ex08b}
		\gll qeⁿ, moħ-e \textbf{ǯog=a\u{e}} \textbf{nax=a\u{e}} \textbf{d}-ag-in=c, {\normalfont[...].} \\
		then, how-{\Rel} \textbf{herd({\J})={\Add}} \textbf{people({\D})={\Add}} \textbf{{\D}}-see-{\Aor}={\Subord} \\
		\trans `Then, as it saw the herd and the people, [...].'
		\hfill (MM411-1.24)
		
		\ex\label{complexnp-ex08c}
		\gll taⁿ ma \textbf{adam=a\u{e}}, \textbf{ev=a\u{e}}, ev-e-ⁿ \textbf{vašo=a\u{e}}, le\u{o}, badrulob-e=doliⁿ cħan-\u{g} \textbf{d}-aq-l-ar. \\
		alongside {\Contr} \textbf{Adam({\M})={\Add}} \textbf{Eva({\F})={\Add}} Eva-{\Obl}-{\Gen} \textbf{brother({\M})={\Add}} Leo childhood-{\Obl}=after one.{\Obl}-{\Trans} \textbf{{\D}}-grow\_up-{\Intr}-{\Imprf} \\
		\trans `Meanwhile, Adam, Eva, and Eva's brother Leo grew up together.' \\
		\hfill (MM414-1.13)
		
		\ex\label{complexnp-ex08d}
		\gll o \textbf{gutan=a\u{e}}, \textbf{ħara=a\u{e}} ħen=i \textbf{d}-ejɬ-nor? \\
		{\Dist} \textbf{plough({\J})={\Add}} \textbf{mill({\J})={\Add}} {\Ssg}.{\Gen}={\Q} \textbf{{\D}}-appear-{\Nw}.{\Rem} \\
		\trans `That plough and that mill, are they yours?' [Lit. `do they apparently appear to be yours?'] 
		\hfill (MM406-1.27)
		
		
		
	\end{xlist}
\end{exe}




\section{Other noun phrases} \label{headlessnp}

Noun phrases can also be headed by personal, demonstrative, or reflexive pronouns. Additionally, noun phrases can appear without a head, where the case marking attaches directly to the Oblique stem of the adjective, participle or numeral.


\subsection{Personal pronouns} \label{perspro}\is{Pronouns!Personal}

Apart from the Ergative, personal pronouns take the same case endings as nouns. They can take all grammatical cases, except the Instrumental, and can take all spatial cases except the \textsc{in}-series. The \textsc{ad}-series is the most common spatial case with pronouns. Additionally, plural pronouns can take the \textsc{inter}-series of spatial cases, with the meaning `among'. See \sectref{nouns} for the functional meaning of each case form. In contrast to the nominal domain, the Dative case \textit{-n} does nasalise the preceding vowel in all personal pronouns except the 1st person plural inclusive. Table \ref{tableperspro} shows the personal pronouns in their underlying morphological form. Of the spatial cases, only the most frequently attested are given.

\begin{table}
	\begin{tabular}{llllll}
    \lsptoprule
		& {1\textsc{sg}} & {2\textsc{sg}} & {1\textsc{pl.excl}} & {1\textsc{pl.incl}} & {2\textsc{pl}} \\
		\midrule
		{Nominative} & so & ħo & txo & vej & šu \\
		{Ergative} & as & aħ & atx & vej & ejš \\
		{Genitive} & sen & ħen & txen & vej-n & šun \\
		{Dative} & son & ħon & txon & vej-n(i) & šun \\
		
		
		\midrule
		
		{Allative} & so-go & ħo-go & txo-go & vej-go & šu-go \\
		{Adessive} & so-go-ħ & ħo-go-ħ & txo-go-ħ & vej-go-ħ & šu-go-ħ \\
		{Adablative} & so-go-ren & ħo-go-ren & txo-go-ren & vej-go-ren  & šu-go-ren \\
		{Interlative} & & & txo-lo & vej-lo  & šu-lo \\
		{Interessive} & & & txo-lo-ħ & vej-lo-ħ & šu-lo-ħ \\
		{Interablative} & & & txo-lo-ren & vej-lo-ren & šu-lo-ren \\
		{Contact} & so-x & ħo-x & txo-x & vej-x & šu-x \\
		{Translative} & so-\u{g} & ħo-\u{g} & txo-\u{g} & vej-\u{g} & šu-\u{g} \\
		{Apudessive} & so-cin & ħo-cin & txo-cin & vej-cin & šu-cin \\
		\lspbottomrule
	\end{tabular}
	\caption{Inflection of the personal pronouns}
	\label{tableperspro}
\end{table}


\subsection{Demonstrative pronouns} \label{dempro}\is{Pronouns!Demonstrative}

Demonstratives used as pronouns can take all case forms. Demonstratives in the singular referring to humans take the Ergative ending \textit{-s}, those referring to non-humans take \textit{-v}. Plural demonstratives in the Ergative are equivalent to the Oblique plural stem.\is{Medial demonstrative}\is{Proximal demonstrative}\is{Distal demonstrative}


\begin{table}
	\begin{tabular}{lllllll}
		\lsptoprule
		& {\textsc{prox.sg}} & {\textsc{med.sg}} & {\textsc{dist.sg}} & {\textsc{prox.pl}} & {\textsc{med.pl}} & {\textsc{dist.pl}}  \\
		\midrule
		{Nominative} & e & is & o & ebi & ipsi & obi \\
		{Ergative} & equ-s, -v & icxu-s, -v & oqu-s, -v & eqar & icxar & oqar \\
		{Contact} & equ-x & icxu-x & oqu-x & eqar-x & icxar-x & oqar-x \\
		{Genitive} & equi-n & icxui-n & oqui-n & eqrin & icxrin & oqrin \\
		{Other cases} & equi- & icxui- & oqui- & eqar- & icxar- & oqar- \\
		\lspbottomrule
	\end{tabular}
	\caption{Inflection of the demonstrative pronouns}
	\label{tabledempro}
\end{table}

The Tsova-Tush distal demonstrative pronoun can function as a deictically neutral third person personal pronoun, as in (\ref{headless-ex01}a,b).

\begin{exe}
	\ex\label{headless-ex01}
	\begin{xlist}
		
		
			\ex\label{headless-ex01a}
			\gll qeⁿ saxiɬ-uš mič-ax j-ax-eⁿ \textbf{o}. \\
			then dawn-{\Simul} where-{\Indf} {\F}.{\Sg}-go.{\Pfv}-{\Aor} \textbf{{\Dist}} \\
			\trans `Then, at daybreak, she went out somewhere.'
			\hfill (E179-100)
		
		
		
			\ex\label{headless-ex01b}
			\gll  \textbf{o} ma j-ux j-erc'-iⁿ šer maxk'ar-n daħ j-ax-en=e. \\
			\textbf{{\Dist}} {\Contr} {\F}.{\Sg}-back {\F}.{\Sg}-turn-{\Aor} {\Refl}.{\Poss} girl.{\Pl}-{\Dat} {\Pv} {\F}.{\Sg}-go-{\Aor}=and \\
			\trans `She, however, returned to her girls.'
			\hfill (E179-124.1)
		
	\end{xlist}
\end{exe}

Modern Georgian shows a similar three-way deictic distinction in demonstratives:\is{Georgian influence!Morphological} we find proximal \textit{es}, medial \textit{eg} and distal \textit{is} (for a detailed account, see \cite[]{martirosov64pronoun}). The same configuration is already attested in Old Georgian, where we find the pronouns \textit{ese, ege, igi}, respectively (\textit{isi} being an alternative variant to \textit{igi} found from the 10th century onward (\cite{gippertOGeo})). Whereas in Old Georgian,\il{Old Georgian} any of the three demonstrative could function as a 3rd person pronoun (\cite{gippertOGeo}), in Modern Georgian, the distal demonstrative \textit{is} is used as the default 3rd person pronoun (the medial and proximal are only used when multiple 3rd person entities need to be distinguished deictically) (\cite[599]{faehnrich12}).
See Example (\ref{headless-ex03}).

\begin{exe}
	\ex\label{headless-ex03}
	Standard Modern Georgian
    
	\gll  magram gana parnaoz-i ak-amde tvaltmakcobda? ara, \textbf{is} ak-amde ibrʒoda. \\
	but {\Q} Parnaoz-{\Nom} here-{\Term} s/he\_was\_deceptive no, \textbf{{\Dist}} here-{\Term} s/he\_battled \\
	\trans `But was Parnaoz really deceitful until now? No, he fought before.' \\
	\hfill (GNC: O. Chiladze)
\end{exe}

The other Nakh languages, Chechen and Ingush, also feature a three-way deictic distinction in demonstratives. In these languages, however, the demonstrative pronoun \textit{iz}/\textit{ɨz}, which is cognate with the Tsova-Tush medial, functions as a personal pronoun \parencites[174, 179]{nichols11}[34]{nichols94Che}, see Example (\ref{headless-ex02}).

\begin{exe}
	
	\ex\label{headless-ex02}
	Ingush
    
	\gll Qoana ħo v-ol-ča qɨ juxa v-oa\u{g}-arg v-\={a}-c \textbf{ɨz}. \\
	tomorrow {\Ssg} {\M}.{\Sg}-be.{\Ptcp}-{\Obl} any\_more again {\M}.{\Sg}-come-{\Fut} {\M}.{\Sg}-be-{\Neg} \textbf{{\Tsg}} \\
	\trans `He won't come back to your place again tomorrow.'
	\hfill (\cite{dumezil36})
	
\end{exe}

It is therefore likely that Tsova-Tush developed the third person pronoun function of the distal demonstrative under the influence of Georgian. This must have happened after Georgian itself grammaticalised the distal demonstrative as a default 3rd person pronoun, which presumably happened in the modern Georgian period. In all Tsova-Tush sources, including the ones from the 19th century, the distal demonstrative is used as a 3rd person pronoun.

Demonstrative pronouns can also be used in a resumptive function, as in (\ref{headless-ex04}), where the relative clause between square brackets is resumed by the distal demonstrative \textit{o}. See \sectref{relative} for relative clauses.\is{Resumptive}

\begin{exe}
	\ex\label{headless-ex04}
	\gll {{\normalfont[} ħan-e} d-ʕev-in=c seⁿ {mamal {\normalfont]}} \textbf{o} l-iba-t son=en. \\
	who.{\Erg}-{\Rel} {\D}-kill-{\Aor}={\Subord} {\Fsg}.{\Gen} rooster \textbf{{\Dist}} give-{\Imp}-{\Pl} {\Fsg}.{\Dat}={\Quot} \\
	\trans `{``}Who(ever) killed my rooster, let them give \textbf{it} to me.{''}'
	\hfill (E153-22)
\end{exe}

\subsection{Negative pronouns} \label{negpro}\is{Pronouns!Negative}

Noun phrases can also be headed by negative pronouns, as described by \textcite[202]{ankernegation,mikeladze11}. The negative pronouns are historically derived from negated indefinite pronouns: \textit{comena} `nobody' (< \textit{co+men+a} {\Neg}+ `somebody' + {\Emph}), Oblique form \textit{coħan}<\textsc{case}>\textit{a}, and \textit{com} `nothing' (< \textit{co+wum} {\Neg} + `something'), Oblique form \textit{cost'en}<\textsc{case}>\textit{a}. 

The most common way of forming a clause with a negative pronoun is in an otherwise positive clause, as in (\ref{headless-ex07}a,b).

\begin{exe}
	\ex\label{headless-ex07}
	\begin{xlist}
		
		\ex\label{headless-ex07a}
		\gll isev \textbf{comena} d-ec-er d-aɁ-it-aⁿ oqu-s. \\
		there.{\Lat} \textbf{nobody} {\D}-must-{\Imprf} {\D}-come-{\Caus}-{\Inf} {\Dist}.{\Obl}-{\Erg} \\
		\trans `She wasn't allowed to let anybody enter there.'
		\hfill (E181-143)
		
		\ex\label{headless-ex07b}
		\gll \textbf{coħan<n>a} xajc’-nor txe pħarči-ⁿ ax-ar. \\
		\textbf{nobody.{\Obl}<{\Dat}>} hear-{\Nw}.{\Rem} {\Fpl}.{\Gen}.{\Obl} dog.{\Pl}-{\Gen} bark-{\Vn} \\
		\trans `Nobody heard our dog’s barking.’
		\hfill (\cite[202]{mikeladze11}, taken from \textcite{ankernegation}, glossing is mine)
	\end{xlist}
\end{exe}


According to \textcite[202]{mikeladze11}, the construction in (\ref{headless-ex07}a,b) represents a structural copy from Georgian, see e.g. Example (\ref{headless-ex08}).\is{Georgian influence!Morphological}

\begin{exe}
	\ex\label{headless-ex08}
	Modern Georgian
    
	\gll mis-i xma \textbf{aravis} gaegona. \\
	{\Tsg}.{\Poss}-{\Agr} voice \textbf{nobody} s/he\_heard\_it \\
	\trans `No one heard her voice.'
	\hfill (GNC: Ch. Amirejibi)
\end{exe}

In Georgian, a clausal negator can also occur in combination with a negative pronoun (see Example (\ref{headless-ex09})), a construction which is also copied in Tsova-Tush, see (\ref{headless-ex010}).

\begin{exe}
	\ex\label{headless-ex09}
	Modern Georgian 
    
	\gll  maga-s \textbf{aravin} \textbf{ar} ipikrebda. \\
	{\Med}.{\Obl}-{\Dat} \textbf{nobody} \textbf{{\Neg}} s/he\_would\_think\_it\\
	\trans `Nobody would think that.'
	\hfill (GNC: I. Chavchavadze)
\end{exe}


\begin{exe}
	\ex\label{headless-ex010}
	\begin{xlist}
		
		\ex\label{headless-ex010a}
		\gll oq deni=doliⁿ o st'ak' din \textbf{coħan<n>a} \textbf{co} v-ag-in-v-a. \\
		{\Dist}.{\Obl} day.{\Obl}=after {\Dist} man alive \textbf{nobody<{\Dat}>} \textbf{{\Neg}} {\M}.{\Sg}-see-{\Ptcp}.{\Pst}-{\M}.{\Sg}-be \\
		\trans `From that day on nobody saw that man alive.'
		\hfill (E113-119)
		
		\ex\label{headless-ex010b}
		\gll \textbf{coħan<n>a} \textbf{co} xajc’-nor txe pħarči-ⁿ ax-ar. \\
		\textbf{nobody.{\Obl}<{\Dat}>} \textbf{{\Neg}} hear-{\Nw}.{\Rem} {\Fpl}.{\Gen}.{\Obl} dog.{\Pl}-{\Gen} bark-{\Vn} \\
		\trans `Nobody heard our dog’s barking.’
		\hfill (\cite[202]{mikeladze11}, taken from \textcite{ankernegation}, glossing is mine)
		
	\end{xlist}
\end{exe}


According to \textcite[202]{mikeladze11}, the original Tsova-Tush pattern consists of an indefinite pronoun plus clausal negation, see (\ref{headless-ex11}a,b), as the same construction is also found in the other Nakh languages, such as Ingush (\ref{headless-ex12}).

\begin{exe}
	\ex\label{headless-ex11}
	\begin{xlist}
		
		\ex\label{headless-ex11a}
		\gll \textbf{co} xaɁ-it-r-as \textbf{ħame-g}=ē... \\
		\textbf{{\Neg}} hear-{\Caus}-{\Imprf}-{\Fsg}.{\Erg} \textbf{somebody-{\All}}=and \\
		\trans `I didn't inform anybody.'
		\hfill (E169-12)
		
		\ex\label{headless-ex11b}
		\gll  \textbf{ħama-s} \textbf{co} aɬ-in-d-a - ``...'' \\
		\textbf{somebody-{\Erg}} \textbf{{\Neg}} say.{\Pfv}-{\Ptcp}.{\Pst}-{\D}-be \\
		\trans `Nobody said: ``...{''}'
		\hfill (E307-40)
		
		
		\ex\label{headless-ex11c}
		\gll \textbf{ħame-n} \textbf{co} xajc'-nor txe pħarči-ⁿ ax-ar. \\
		\textbf{somebody-{\Dat}} \textbf{{\Neg}} hear-{\Nw}.{\Rem} {\Fpl}.{\Gen}.{\Obl} dog.{\Pl}-{\Gen} bark-{\Vn} \\
		\trans `Nobody heard our dog’s barking.’
		\hfill (\cite[202]{mikeladze11}, taken from \textcite{ankernegation}, glossing is mine)
		
	\end{xlist}
\end{exe}


\begin{exe}
	\ex\label{headless-ex12}
	Ingush
    
	\gll taxan balxa \textbf{sag} \textbf{ħa-v-e-nʒar}. \\
	today work.{\Lat} \textbf{person} \textbf{{\Pv}-{\M}.{\Sg}-come-{\Neg}.{\Pst}} \\
	\trans `Nobody came to work today.'
	\hfill (\cite[696]{nichols11})
\end{exe}

Clausal negation in Tsova-Tush remains the same throughout the historical stages: the negator \textit{co} is positioned directly before the finite verb (\textit{ma} for imperatives).
In the 19th-century Tsova-Tush sources, none of the three negative constructions (negative pronoun, negative pronoun with clausal negation, indefinite pronoun with clausal negation) are found, probably due to the small corpus sizes. 
In the 20th-century sources (KK, YD), we find 50 instances of a construction with a negative pronoun as in Example (\ref{headless-ex07}), and 8 instances of a negative pronoun with an additional clausal negator \textit{co}. No examples of the clausal negator with an indefinite pronoun, like in Example (\ref{headless-ex11}), were found. In contemporary Tsova-Tush, we also find constructions with a negative pronoun, approximately 50\% with clausal negation, 50\% without. Only extremely rarely do we find the supposedly `old' construction with an indefinite pronoun and clausal negation: in fact, the examples in (\ref{headless-ex11}) might be the only ones. Still, based on our evidence from the other Nakh languages, this construction can indeed be considered the original one.



\subsection{Headless noun phrases} \label{subst}\is{Headless noun phrases}\is{Substantivisation}

Noun phrases can appear without a head noun, where any case marking attaches directly to the Oblique stem of the modifier. This Oblique stem is formed by suffixing the Oblique marker \textit{-čo} to the base stem. The Nominative plural is formed with the productive marker \textit{-i}, whereas the Oblique plural is marked by \textit{-ši}, where the \textit{i} is dropped, triggering umlaut on the preceding vowel, see Table \ref{tableheadless}. Note that the Dative does not trigger umlaut as it does on nouns (see \sectref{corecase}).
At least two nouns, \textit{pst'uin\u{o}} `woman' (Oblique stem \textit{pst'uinčo-}) and \textit{želreⁿ} `sheep' (Oblique stem \textit{želrečo-}) form their Oblique singular cases using the suffix {-čo}.\footnote{Compare \textit{pst'u} `wife' and \textit{že} `sheep' (collective), whence \textit{že-lo-ren} `from among the sheep'.} 

\begin{table}
	\begin{tabular}{lll}
		\lsptoprule
		& \textsc{sg} & \textsc{pl}\\\midrule
		Nominative & -∅ & -i  \\
		Ergative & -čo-v & -čui-š-v  \\
		Dative & -čo-n & -čui-š-n \\
		Other cases & -čo- & -čui-š- \\
		\lspbottomrule
	\end{tabular}
	\caption{Inflection of substantivised modifiers}
	\label{tableheadless}
\end{table}



Modifiers that can appear without a head include all those that, when used attributively, display a nominative-Oblique distinction using the Oblique marker \textit{-čo} (see \sectref{caseagree}), i.e. adjectives (\ref{headless-ex05a}), ordinal numerals (\ref{headless-ex05b})\footnote{Original orthography of (\ref{headless-ex05b}): la\'{t}i ixa cḥa \.{s}il\.{g}e\.{c}ox wac̣u\.{s}.} and participles (\ref{headless-ex05c}). See \sectref{participial} for noun phrases with participial relative clauses.

\begin{exe}
	\ex\label{headless-ex05}
	\begin{xlist}
		
		\ex\label{headless-ex05a}
		\gll \u{g}osxet-iⁿ zorajš \textbf{q'ajn-čo-n}. \\
		be\_happy-{\Aor} very \textbf{old-{\Obl}-{\Dat}} \\
		\trans `The old [person] was very happy.'
		\hfill (EK053-2.7)
		
		\ex\label{headless-ex05b}
		\gll latt-iⁿ ix-aⁿ cħa \textbf{ši-l\u{g}e-čo-x} v-ac'-uš. \\
		stand-{\Aor} go-{\Inf} one \textbf{two-{\Ord}-{\Obl}-{\Cont}} {\M}.{\Sg}-follow-{\Simul} \\
		\trans `They stood up to go, the one following the other.'

		\hfill (AS003-1.10)
		
		\ex\label{headless-ex05c}
		\gll oqui-ⁿ j-aqqui-č\u{o} važk'acb-e-ⁿ ambuj=a j-eɁ-eⁿ b-ux \textbf{b-erc'-in-čui-š-v}. \\
		{\Dist}.{\Obl}-{\Gen} {\J}-big-{\Obl} bravery-{\Obl}-{\Gen} story={\Add} {\J}-bring-{\Aor} {\M}.{\Pl}-back \textbf{{\M}.{\Pl}-turn-{\Ptcp}.{\Aor}-{\Obl}-{\Pl}-{\Erg}} \\
		\trans `Those that returned brought the story of his great bravery.' \\
		\hfill  (MM117-2.20)   
		
	\end{xlist}
\end{exe}

Furthermore, some modifiers that do not feature the Oblique suffix \textit{-čo} when used attributively, do so when substantivised, such as the quantifier \textit{duq} (\ref{headless-ex06a}--\ref{headless-ex06b}), and Genitive personal pronouns (\ref{headless-ex06c}--\ref{headless-ex06d}).



\begin{exe}
	\ex\label{headless-ex06}
	\begin{xlist}
		
		\ex\label{headless-ex06a}
		\gll dasavlet-e-ħ=a \textbf{duq-i} b-a-r. \\
		west-{\Obl}-{\Ess}={\Add} \textbf{many-{\Pl}} {\M}.{\Pl}-be-{\Imprf} \\
		\trans `In the west, there were many of them, too.'
		\hfill (E146-37)
		
		\ex\label{headless-ex06b}
		\gll at-c'iⁿ bo \textbf{duq-čuj-š-v} maq-ciⁿ čamli-š leħ-\u{o}. \\
		pound-{\Priv} garlic \textbf{many-{\Obl}-{\Pl}-{\Erg}} bread-{\Apudess} tasty-{\Adv} bring-{\Npst} \\
		\trans `Many [people] put raw garlic with their bread for the taste.' \\
		\hfill (KK001-0057)
		
		\ex\label{headless-ex06c}
		\gll du-j xaxk'-in\u{}e \textbf{txe-čo-n} ħatx\u{e} co b-aɬ-mak'-iⁿ. \\
		horse.{\Obl}-{\Pl} gallop-{\Aor}.{\Seq} \textbf{{\Fpl}.{\Gen}-{\Obl}-{\Dat}} in\_front {\Neg} {\B}.{\Sg}-go\_out-{\Pot}-{\Aor} \\
		\trans `The horses were galloping, and ours couldn't go in front.'  \\
		\hfill (KK008-1735)
		
		\ex\label{headless-ex06d}
		\gll bak' ħal b-ust'-b-al-in\u{e}, \textbf{šu-čo-x} ešuš-x=k'aɁ b-aɬ-eⁿ. \\
		paddock {\Pv} {\B}.{\Sg}-measure-{\B}.{\Sg}-{\Intr}-{\Aor}.{\Seq} \textbf{{\Spl}.{\Gen}-{\Obl}-{\Cont}} less-{\Cmp}=a\_little {\B}.{\Sg}-go\_out-{\Aor} \\
		\trans `The paddock was measured, and it turned out [to be] a little smaller than yours.'
		\hfill (KK036-5563)
		
	\end{xlist}
\end{exe}

No productive way of substantivising Tsova-Tush nouns in the Genitive or in a spatial case have been attested. Tsova-Tush features many nominalising derivational suffixes, which fall beyond the scope of this work.




\section{Summary}

In terms of basic description, this chapter has provided new insight into the following domains:

\begin{enumerate}
	\item Most spatial cases (all except the \textsc{apud}-series) involve a two-slot system, very similar to a typical Daghestanian system of spatial cases, see \sectref{spacase}.
	
	\item Although many details remain to be investigated, an attempt at distinguishing different nominal declension classes has been made, see \sectref{Oblique}.
	
	\item Although much remains unclear, a modification construction is investigated where an endingless noun in the Oblique form can modify another noun, see \sectref{barenoun}.
	
\end{enumerate}


In terms of structural language contact, this chapter has shown the following parallels between Tsova-Tush and Georgian, which are most likely to be attributed to influence of the latter on the former. Note that in the case of the first three features, the native Tsova-Tush equivalent is maintained alongside the contact-induced innovation.

\begin{enumerate}
	\item In a significant number of instances of nouns in the Essive, Interessive, Inessive and Superessive, these Essive cases are used with a lative function. This use is not attested in other Nakh languages, but is regular in Georgian, where the lative-essive distinction is not overtly expressed on nouns at all, see \sectref{spacase}
	
	\item  Alongside its original synthetic variant, Tsova-Tush features an analytic construction to express the comparative of adjectives. Not only is this analytic construction absent in Ingush and Chechen and obligatory in Georgian, the exact comparative morph \textit{upro} is borrowed from Georgian into Tsova-Tush. Furthermore, although the superlative construction in Chechen and Ingush is also analytic, the specific morph used to express the superlative in Tsova-Tush, \textit{ħamaxeɁ}, is a calque from Georgian \textit{q'velaze}, see \sectref{adjectives}.
	
	
	\item Tsova-Tush numerals higher than one hundred are usually borrowed from Georgian, see \sectref{numerals}.
	
	
	\item As opposed to the medial demonstrative in Chechen and Ingush, the Tsova-Tush distal demonstrative \textit{o} is used as a deictically neutral third person personal pronoun. This is exactly parallel to its Georgian counterpart \textit{is}, see \sectref{dempro}.
	
	\item Phrases with a negative pronoun are formed in the same way as their Georgian counterparts, as opposed to the original construction (found in the other Nakh languages), consisting of an indefinite pronoun and clausal negation, see \sectref{negpro}.
\end{enumerate}

In terms of loanword adaptation, this chapter has shown that: 

\begin{enumerate}
	\item Georgian nouns are borrowed into the consonant declension class, except words ending in \textit{-u} or \textit{-o}, which are borrowed into the \textit{u}-class and \textit{o}-class respectively, see \sectref{decladapt}.
	
	\item The gender of loanwords is determined by the same interplay of phonological and semantic criteria that assign gender to native words, see \sectref{genderadapt} and \cite{WS}.
	
	\item Adjectives are borrowed frequently, and do not show case agreement as native adjectives do, see \sectref{adjadapt}.
	
	\item A set of older nominal loans are borrowed into the C/\textit{u}-declension (\sectref{decladapt}), and a set of older borrowed adjectives receive a Tsova-Tush adjective ending \textit{-on}, integrating them more closely into the native grammar, as they do receive the Oblique marker \textit{-čo} to signal case agreement (\sectref{adjadapt}).
\end{enumerate}
