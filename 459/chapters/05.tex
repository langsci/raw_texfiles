\chapter{Valency, valency derivation, and complex verbs} \label{derivation}


\section{Introduction}

In this chapter, several topics surrounding verbal morphosyntax and derivation are discussed. In \sectref{valency}, all basic valency patterns in Tsova-Tush are presented, with a more in-depth discussion of intransitive subjects in the Ergative in \sectref{intrerg}. \sectref{lightverbs} discusses several types of complex verbs, in particular constructions involving a light verb (Sections \ref{lvconstr} and \ref{lightverbsp}). Valency derivation (that makes use of exactly those light verbs or other suffixes historically derived from verbs) is discussed in \sectref{verbderivproper}. 

In terms of contact-induced change, two main topics are addressed. For the first, the aforementioned occurrence of Ergative subjects with intransitive verbs, it will be concluded that a Georgian influence is not warranted as an explanation of this phenomenon (\sectref{intr-erg-geo}). The second topic concerns the incorporation of verbs borrowed from Georgian. The general pattern is described in \sectref{verbloanbasic}, while the incorporation of a special type of Georgian intransitive verb (with transitive morphology and Ergative case marking in the Aorist) into Tsova-Tush is discussed separately, in \sectref{loanverbmed}.




\section{Valency and alignment} \label{valency}\is{Valency}\is{Alignment}

A Tsova-Tush verb usually governs only one valency pattern\is{Labile verbs} (i.e. is not labile (\cite[193]{holiskygagua})), which is stable in all tense-aspect forms. The most common case forms for argument marking are the Ergative, the Nominative and the Dative (\cite{chrelashvili87}), although the Allative and Contact cases can be used for Oblique objects. Thus, Tsova-Tush case marking of arguments is largely similar to other East Caucasian languages (\cite{forker17ergativity}), the exception being Ergative marking on certain intransitive subjects. Note that the following overview considers simple, underived verbs only. For derived and otherwise complex verbs, see Sections  \ref{lightverbs} and \ref{verbderivproper}.


\subsection{Single-argument verbs} \label{1arg}

\subsubsection{Nominative argument}

Most monovalent verbs expressing states and changes of state take a subject in the Nominative\footnote{The label Nominative is preferred over Absolutive in most Caucasological works.}, see Example (\ref{verbderiv-ex01}). This construction represents the basic intransitive clause type.\is{Monovalent verbs}\is{Intransitive verbs}

\begin{exe}
	\ex\label{verbderiv-ex01}
	\begin{xlist}
		
		
			\ex\label{verbderiv-ex01a}
			\gll t'q'uiħ qejc' j-acuⁿ mu\u{g}. \\
			behind hang({\Npst}) {\J}-short tail.\textbf{{\Nom}} \\
			\trans `A short tail hangs behind.'
			\hfill (E064-23)
		
		
		
			\ex\label{verbderiv-ex01b}
			\gll ošt'iɁ laum\u{u} d-ax-c'iⁿ d-is-eⁿ.  \\
			again mountain.{\Lat} {\D}-go-{\Priv}.{\Adjz}.\textbf{{\Nom}} {\D}-stay-{\Aor} \\
			\trans `Again, those that didn't go to the mountains stayed behind.' \\
			\hfill (KK015-3120)
		
		
		
			\ex\label{verbderiv-ex01c}
			\gll že daħ=i d-av-iⁿ? \\
			sheep.\textbf{{\Nom}} {\Pv}={\Q} {\D}-be\_killed-{\Aor} \\
			\trans `Were the sheep slaughtered?'
			\hfill (EK009-7.1)
		
		
	\end{xlist}
\end{exe}



Two-argument verbs that most often occur with an Ergative agent and a Nominative patient (see \ref{2arg}) can occur without an agent to express impersonal semantics, see Example (\ref{verbderiv-ex21}). See \textcite{creissels14labile,forker17ergativity} for this type of P-lability in other East Caucasian languages.\is{Impersonal construction}\largerpage

\begin{exe}
	\ex\label{verbderiv-ex21}
	\begin{xlist}
		
		
			\ex\label{verbderiv-ex21a}
			\gll ǯange-č\u{o} nek'-e-v ditx tit'-eⁿ.\\
			rusty-{\Obl} knife-{\Obl}-{\Ins} meat.\textbf{{\Nom}} cut.{\Pfv}-{\Aor}\\
			\trans `They cut the meat with a rusty knife.' or: `The meat was cut with a rusty knife.' 
			\hfill (KK033-5485)
		
		
		
			\ex\label{verbderiv-ex21b}
			\gll matx-mak j-arž-j-∅-aⁿ nʕajɁ qeħ-iⁿ barg.\\
			sun-{\Superlat} {\J}-unfold-{\J}-{\Tr}-{\Inf} outside carry.{\Ipfv}-{\Aor} luggage.\textbf{{\Nom}}\\
			\trans `They took the luggage outside to splay it out in the sun.' or: `The luggage was taken outside [...]'
			\hfill (KK002-0432)
		
		
		
	\end{xlist}
\end{exe}



\subsubsection{Ergative argument} \label{intrerg}

As shown by \textcite{holisky87} (but see already \cite[73]{schiefner59,holisky84}), intransitive subjects of the first and second person can be marked by the Ergative case. This occurs in one of the following conditions:\is{Ergative case!On intransitive subject}\is{Volition}

\begin{enumerate}
\item The intransitive verb belongs to a class that allows variable marking in first and second person subjects, where an Ergative subject signals a greater amount of volition and/or agency than a Nominative subject. Depending on the semantics of the verb, the volitional (Ergative) or the less-volitional (Nominative) subject is more common, or both are attested in equal measure. These verbs include: changes of state, verbs of falling, rolling, slipping, locative statives, some verbs of motion (mostly \textsc{erg}), some verbs of communication (mostly \textsc{erg}), and others. In Example (\ref{verbderiv-ex02a}), the verb \textit{virc'nas} (\textit{v-erc'-in-as}) features a gender marker that refers to the Ergative argument. That is, in this example, the intransitive verb \textit{d-erc'-} `turn' regularly agrees in gender with its intransitive subject, but because of the variability in case marking that the verb \textit{d-erc'-} `turn' allows (a Nominative argument would have expressed a less volitional subject), the 1st person subject of this verb is marked by the Ergative case. This results in an exception to the general pattern in East Caucasian languages that gender marking on the verb exclusively cross-references the Nominative argument in the clause. 

\begin{exe}
	
	\ex\label{verbderiv-ex02a}
	\gll oqar-n \textbf{v-irc'-n-as} v-ux=ajn\u{o}. \\
	{\Dist}.{\Pl}.{\Obl}-{\Dat} \textbf{{\M}.{\Sg}-turn-{\Aor}-\textbf{{\Fsg}.}{\Erg}} {\M}.{\Sg}-back={\Quot} \\
	\trans `I returned for them.' (i.e. `I chose to return for them.')
	\hfill (EK005-4.7)
\end{exe}

\item Alternatively, the intransitive verb belongs to a class that \textit{requires} its subject to be marked by the Ergative. These verbs include all verbs of motion or communication (other than those that show variable marking), as well as approximately 28 other verbs. See Example (\ref{verbderiv-ex02b}). Note that in this example, too, the gender marker cross-references an argument coded with the Ergative because of the restrictions that the verb \textit{d-ot'-} imposes on its intransitive subject in first or second person.

\begin{exe}
	\ex\label{verbderiv-ex02b}
	\gll inc širk-ileⁿ v-a\u{g}-o-s, \textbf{v-uit'-as} tinit lam-na-x.\\
	now Shiraki-{\Elat} {\M}.{\Sg}-come-{\Npst}-{\Fsg}.{\Erg} \textbf{{\M}.{\Sg}-go({\Npst})-{\Fsg}.\textbf{{\Erg}}} Tianeti.{\Ill} mountain-{\Obl}.{\Pl}-{\Cont}\\
	\trans `Now I am coming from Shiraki, and I'm going to the Tianeti mountains.' \hfill\hbox{(E040-20)}
\end{exe}


\item Additionally, \textcite[]{holisky87} has identified 6 verbs that show morphological similarities with transitive verbs and require Ergative subjects in all three persons. However, they exclusively occur without a direct object, and are therefore treated syntactically as intransitives by Holisky (see Example (\ref{verbderiv-ex02c})). 

\begin{exe}
	
	\ex\label{verbderiv-ex02c}
	\gll šin šar-e k'olekt'iv-e \textbf{mušeba(d)-d-i-n-as} [...] osi-ħ=a moc'inav j-a-ra-s. \\
	two.{\Obl} year-{\Obl}({\Ess}) collective-{\Obl}({\Ess}) \textbf{work-{\D}-{\Tr}-{\Aor}-{\Fsg}.\textbf{{\Erg}}} {} there-{\Ess}={\Add} leader {\F}.{\Sg}-be-{\Imprf}-{\Fsg}.{\Nom} \\
	\trans `I worked in a collective for two years, I was a leader there too.'
	\hfill (E116-17)
\end{exe}

In (\ref{verbderiv-ex02c}), a different type of gender cross-referencing is observed. The verb \textit{mušebadinas} `I worked' occurs with a gender marker \textit{d-}, whereas the first-person subject is feminine, as seen by the subsequent verb \textit{jaras} `I was'. Hence, if the verb \textit{mušebad-d-ar} patterned like the verbs in Examples (\ref{verbderiv-ex02a}--\ref{verbderiv-ex02b}), we would expect a form \textit{mušebadjinas}, with a feminine marker \textit{j-}. Instead, we find a gender marker \textit{d-} not referring to any object. Thus, this verb (and the 5 other verbs identified by \cite{holisky87}) shows regular transitive morphology (\textit{mušebadinas} carries a transitive morpheme \textit{-i-}\footnote{See \sectref{loanverb} for the adaptation of Georgian verbs such as this one.}), whilst never being attested with an object. See \sectref{loanverb} for a more in-depth treatment of this phenomenon, which occurs with certain verbs borrowed from Georgian.
\end{enumerate}

Consider again the 6 verbs of which \textcite{holisky87} claims they have an intransitive Ergative subject in all three persons, see Table  \ref{verbderiv-table-ergativeholisky}.

\begin{table}
	\begin{tabular}{ll}
		\lsptoprule
		\textit{mušeba(d)-d-ar} & `work' \\
		\textit{gamarǯba(d)-d-ar} & `be victorious' \\
		\textit{naidreba(d)-d-ar} & `hunt' \\ 
		\textit{toxar/tepxar} & `hit, beat' \\
		\textit{ħač'q'ar/ħeč'q'ar} & `pinch (of shoes)' \\
		\textit{curi ħaqar} &  `swim' \\
		\lspbottomrule
	\end{tabular}
	\caption{Intransitive verbs with a single argument in the Ergative case in all persons according to \textcites[188--189]{holisky84}}
	\label{verbderiv-table-ergativeholisky}
\end{table}

Even though \textcite[188--189]{holisky84} groups these verbs together, we in fact see a striking variety of valency patterns. Out of these 6, \textit{mušeba(d)-d-ar} `work' and  \textit{gamarǯba(d)-d-ar} `be victorious', both borrowed from Georgian, indeed have only a single Ergative argument in all persons. Similarly, the verb \textit{naidreba(d)-d-ar} `hunt' is also borrowed from Georgian along with its valency construction: an Ergative subject and an Oblique (\textsc{super}-locative) object. The verb \textit{toxar/tepxar} `hit, beat' is in fact not a single-argument, but a three-argument verb (with the instrument/body part in the Nominative and the goal/point of contact in the Contact case, see \ref{3arg}). \textit{ħač'q'ar/ħeč'q'ar} `pinch (of shoes)' is a regular two-argument verb with the meanings `seize, grab, catch; press, wring out', but occurs without an object with the meaning `pinch (of shoes)'. This leaves \textit{curi ħaqar} `swim', which consists of the two-argument verb \textit{ħaqar} (which in isolation means `smear, wipe'), and the element \textit{curi}\footnote{Borrowed from Georgian \textit{cur-} `swim'. The ending \textit{-i} needs further investigation.}, which has no meaning in isolation, but functions as a direct object in the construction \textit{curi ħaqar} `swim'. Hence, all real single-argument verbs with an Ergative subject in all persons are borrowed from Georgian, and will be discussed in \sectref{loanverb}.

In sum, we can identify at least 4 verb types, categorised by a decreasing level of prototypical intransitivity, seen in \tabref{verbderiv-table1}. All these verbs demand a single argument, but in terms of gender cross-referencing, all but type IV verbs agree in gender with their subject. In terms of the  case that this subject is required to have, only type I is prototypically intransitive; all other types allow variable marking or require the Ergative.


\begin{table}
	\small
	\tabcolsep=.66\tabcolsep
	\begin{tabular}{llllll}
    \lsptoprule
		& \multicolumn{3}{c}{{Prototypically intransitive in terms of}} & & \\\cmidrule(lr){2-4}
		& {Arg.} & {Gender}  & {Case} & Number\footnote{\parencite{holisky87}} & {Examples} \\
		\midrule
		I & yes (1) & yes (agrees with S)& yes (\textsc{nom})  & 31\,+\,prod. & \textit{d-aq'-d-alar} `dry up' \\
		II & yes (1) & yes (agrees with S) & no (variable)  & 124 & \textit{d-erc'ar} `turn' \\
		III & yes (1) & yes (agrees with S) & no (\textsc{erg})  & 78 & \textit{d-ot'ar} `go' \\
		IV & yes (1) & no (fixed {\D}-gender) & no (\textsc{erg})  & prod. (GE) & \textit{mušeba(d)-d-ar} `work' \\
		\lspbottomrule
	\end{tabular}
	\caption{Tsova-Tush intransitive verb categories. Arg: Arguments, prod.: productive.}
	\label{verbderiv-table1}
\end{table}


No alignment pattern with a single ergative argument is found in Chechen or Ingush. See \sectref{intr-erg-geo} for a comparison with Georgian.


\subsection{Two-argument verbs} \label{2arg}\is{Transitive verbs}\is{Bivalent verbs}

\subsubsection{Ergative + Nominative}

Most two-argument verbs require their subject to be in the Ergative, and the direct object in the Nominative, see Example (\ref{verbderiv-ex03}). This construction represents the basic transitive clause type.

\begin{exe}
	\ex\label{verbderiv-ex03}
	\begin{xlist}
		
		
			\ex\label{verbderiv-ex03a}
			\gll kikoɁ txo-goħ lek'-i-v že d-ik'-er. \\
			early {\Fpl}-{\Adess} Daghestanian-{\Pl}-\textbf{{\Erg}} sheep.\textbf{{\Nom}} {\D}-take.{\Anim}-{\Imprf}\\
			\trans `In the olden days, Daghestanians used to take our sheep.' \\
			\hfill (KK009-1789)
		
		
		
			\ex\label{verbderiv-ex03b}
			\gll co d-aq'-or oqu-s o-bi ħal\u{o}.  \\
			{\Neg} {\D}-eat.{\Ipfv}-{\Imprf} {\Dist}.{\Obl}-\textbf{{\Erg}} {\Dist}-{\Pl}.\textbf{{\Nom}} {\Pv} \\
			\trans `She didn't eat those things.'
			\hfill (E179-82)
		
		
	\end{xlist}
\end{exe}


\subsubsection{Nominative + Oblique}

Many verbs show a valency pattern with a Nominative subject and an object in an Oblique case, such as Contact case (\ref{verbderiv-ex04a}, \ref{verbderiv-ex04b}), Dative (\ref{verbderiv-ex04c}, \ref{verbderiv-ex04d}) or Allative (\ref{verbderiv-ex05}). These verbs are analysed as two-place intransitives with an (indirect) object in an Oblique case (\cite[193]{holiskygagua}), i.e. they represent an extended intransitive clause type (\cites[]{dixonaikh2000intro}).

\begin{exe}
	\ex\label{verbderiv-ex04}
	\begin{xlist}
		
		
			\ex\label{verbderiv-ex04a}
			\gll cxovrb-e-x=a deniɁ qečnerat lat-e-s=a.  \\
			life-{\Obl}-\textbf{{\Cont}}={\Emph} fully differently fight-{\Npst}-{\Fsg}.\textbf{{\Nom}}={\Emph} \\
			\trans `I fight life [i.e. I struggle with life] in a completely different way.' \\
			\hfill (E046-12)
		
		
		
			\ex\label{verbderiv-ex04b}
			\gll o šer korti-x=a dak'liv. \\
			{\Dist}.\textbf{{\Nom}} {\Refl}.{\Poss} head.{\Obl}-\textbf{{\Cont}}={\Add} think.{\Ipfv}({\Npst}) \\
			\trans `He thinks about himself as well.' 
			\hfill (E002-47)
		
		
		
			\ex\label{verbderiv-ex04c}
			\gll kist' qet-iⁿ oqui-n\u{\i}, psarl-xan-e. \\
			Kist.\textbf{{\Nom}} attack-{\Aor} {\Dist}.{\Obl}-\textbf{{\Dat}} evening-time-{\Obl}({\Ess}) \\
			\trans `The Kist attacked him in the evening.'
			\hfill (EK006-2.4)
		
		
		
			\ex\label{verbderiv-ex04d}
			\gll qeⁿ axmit'-reⁿ gamgebel lat'-eⁿ txoⁿ zorejš, \\
			then Akhmeta-{\Elat} leader.\textbf{{\Nom}} help.{\Pfv}-{\Aor} {\Fpl}.\textbf{{\Dat}} very \\
			\trans `Then the leader from Akhmeta helped us a lot.'
			\hfill (E041-76)
		
		
		
	\end{xlist}
\end{exe}


\subsubsection{Ergative + Oblique}

Some extended intransitive verbs, similar to some regular intransitive verbs, can also appear with an Ergative subject, although relatively few such valency patterns with an Ergative subject and an Oblique object are found (\cite[195]{holiskygagua}, who don't give examples). One example is the verb \textit{ħač'ar/ħeč'ar} `look', which, besides a Lative (usually Allative) object, requires an Ergative subject in the 1st or 2nd person (see Example (\ref{verbderiv-ex05a})), similar to monovalent verbs of type III as seen in \sectref{1arg}. In the 3rd person, a Nominative subject is required (\ref{verbderiv-ex05b}).

\begin{exe}
	\ex\label{verbderiv-ex05}
	\begin{xlist}
		
		
			\ex\label{verbderiv-ex05a}
			\gll saj badr-e-g=saⁿ ħič'-as ħo-g\u{o}. \\
			{\Fsg}.{\Poss}.{\Emph}.{\Obl} child-{\Obl}-{\All}=like look.{\Ipfv}({\Npst})-{\Fsg}.\textbf{{\Erg}} {\Ssg}-\textbf{{\All}} \\
			\trans `I look at you like [I look] at my own child.'
			\hfill (MM116 2.18)
		
		
		
			\ex\label{verbderiv-ex05b}
			\gll o joħ oqui-g ħič'. \\
			{\Dist} girl.\textbf{{\Nom}} {\Dist}.{\Obl}-\textbf{{\All}} look.{\Ipfv}({\Npst})\\
			\trans `That girl looks at him.'
			\hfill (E182-99)
		
		
	\end{xlist}
\end{exe}

\subsubsection{Dative + Nominative}

Many experiential verbs show a valency pattern with a Dative subject/experiencer and a Nominative object/stimulus. Examples include the verbs `see' (\ref{verbderiv-ex07a}), `want' (\ref{verbderiv-ex07b}), `find' (\ref{verbderiv-ex07c}) and `know' (\ref{verbderiv-ex07d}).


\begin{exe}
	\ex\label{verbderiv-ex07}
	\begin{xlist}
		
		
			\ex\label{verbderiv-ex07a}
			\gll važa ħeⁿ arc'iv txa g-u soⁿ. \\
			Vazha {\Ssg}.{\Gen} eagle.\textbf{{\Nom}} today see.{\Pfv}-{\Npst} {\Fsg}.\textbf{{\Dat}}\\
			\trans `Vazha, I see your eagle now.'
			\hfill (E207-8)
		
		
		
			\ex\label{verbderiv-ex07b}
			\gll aznaurob leɁ soⁿ. \\
			nobility.\textbf{{\Nom}} want.{\Ipfv}({\Npst}) {\Fsg}.\textbf{{\Dat}} \\
			\trans `I want nobility.'
			\hfill (EK053-3.14)
		
		
		
			\ex\label{verbderiv-ex07c}
			\gll oqu-s d-av-d-i-en\u{o} žagn\u{o} soⁿ ħal xet-iⁿ.  \\
			{\Dist}.{\Obl}-{\Erg} {\D}-loose-{\D}-{\Tr}-{\Ptcp}.{\Pst} book.\textbf{{\Nom}} {\Fsg}.\textbf{{\Dat}} {\Pv} find-{\Aor} \\
			\trans `I found the book that s/he lost.'
			\hfill (KK001-0036)
		
		
		
			\ex\label{verbderiv-ex07d}
			\gll ħan-n qet šun bacbur?  \\
			who.{\Obl}-\textbf{{\Dat}} know({\Npst}) {\Spl}.{\Dat} Tsova\_Tush.\textbf{{\Nom}} \\
			\trans `Who amongst you knows Tsova-Tush?
			\hfill (BH064-38.1)
		
		
	\end{xlist}
\end{exe}



\subsection{Three-argument verbs} \label{3arg}\is{Ditransitive verbs}

\subsubsection{Ergative + Nominative + Oblique}

Most three-place verbs feature a subject in the Ergative case, a direct object in the Nominative case, and an indirect object in an Oblique case. These verbs represent the extended transitive clause type (\cites{dixonaikh2000intro,forker17ergativity}). An indirect object that refers to a recipient is usually marked by the Dative case, such as with the verb `give' in Example (\ref{verbderiv-ex06a}).\footnote{Nakh languages do not feature an alienability distinction in recipients (see e.g. \cites[592]{nichols11}), like most Daghestanian languages do.}
With speech verbs such as `tell' (Example (\ref{verbderiv-ex06b})), addressees are marked by the Allative case. The verb `ask, demand' requires its indirect object to be in the Contact case (\ref{verbderiv-ex06c}). With verbs of shooting, hitting, etc., the instrument (be it an implement or a body part) is in the Nominative, and the goal (i.e. the person or object being hit or shot) is in the Contact case (Example \ref{verbderiv-ex06d}). This argument mapping for such verbs (which typically include `hit', `shoot', `touch'), i.e. the instrument of the action being expressed as a direct object, and the undergoer appearing in the dative or a spatial case, is found in most, if not all, Kartvelian and East Caucasian languages (\cites[58--59]{klimov78struktur}).\is{Addressee}\is{Dative case}\is{Allative case}\is{Contact case}\is{Recipient}



\begin{exe}
	\ex\label{verbderiv-ex06}
	\begin{xlist}
		
		
			\ex\label{verbderiv-ex06a}
			\gll busu-busu ma d-al-in-čo-ⁿ dad-i-v nax-n majq\u{\i} teɬ-\u{o}.  \\
			night-night however {\D}-die-{\Ptcp}.{\Pst}-{\Obl}-{\Gen} patron-{\Pl}-\textbf{{\Erg}} people-\textbf{{\Dat}} bread.\textbf{{\Nom}} give.{\Ipfv}-{\Npst}\\
			\trans `But at night, the family of the deceased give bread to the people.' \\
			\hfill (EK023-2.7)
		
		
		
			\ex\label{verbderiv-ex06b}
			\gll jaša-s jažar-g\u{o}, važar-g\u{o} pal-i d-epc-or. \\
			sister-\textbf{{\Erg}} sister.{\Pl}-\textbf{{\All}} brother.{\Pl}-\textbf{{\All}} tale-{\Pl}.\textbf{{\Nom}} {\B}.{\Pl}-tell.{\Ipfv}-{\Imprf} \\
			\trans `The sister was telling stories to her siblings.'
			\hfill (KK010-1873)
		
		
		
			\ex\label{verbderiv-ex06c}
			\gll ħan-e b-ex-in-c\u{\i} so-x doⁿ, oqui-n b-ajɬ-n-as. \\
			who.\textbf{{\Erg}}-{\Rel} {\B}.{\Sg}-ask-{\Aor}-{\Subord} {\Fsg}-\textbf{{\Cont}} horse.\textbf{{\Nom}} {\Dist}.{\Obl}-{\Dat} {\B}.{\Sg}-give.{\Pfv}-{\Aor}-{\Fsg}.{\Erg} \\
			\trans `Who(ever) asked a horse from me, I gave [it] to him/her.' \\
			\hfill (KK013-2697)
		
		
		
			\ex\label{verbderiv-ex06d}
			\gll top=a tox-iⁿ oqu-s gak'o-x.  \\
			gun.\textbf{{\Nom}}={\Emph} hit.{\Pfv}-{\Aor} {\Dist}.{\Obl}-\textbf{{\Erg}} stomach-\textbf{{\Cont}} \\
			\trans `S/he shot [him/her] in the stomach.' (Lit. `S/he shot the gun into the stomach.')
			\hfill (EK059-2.9)
		
		
	\end{xlist}
\end{exe}


\subsection{Comparison with Georgian} \label{intr-erg-geo}

\subsubsection{Introduction to Georgian valency and alignment} \label{geoalign}\is{Alignment!Georgian}

The Georgian language is famous for its intricate system of grammatical case marking (see e.g. \cites[]{harris1982unaccusative,harris1981syntax}). The main grammatical cases, Nominative, Ergative and Dative, are all used as subjects, depending on the verb class and the TAM form of the verb. Example (\ref{verbderiv-ex22}) illustrates this phenomenon. 

\begin{exe}
	\ex\label{verbderiv-ex22}
	Modern Georgian
	\begin{xlist}
		
		
			\ex\label{verbderiv-ex22a}
			\gll  glex-\textbf{i} tesavs simind-\textbf{s}. \\
			peasant-\textbf{{\Nom}} s/he\_sows\_it  corn-\textbf{{\Dat}} \\
			\trans `The peasant is sowing corn.'
			\hfill
		
		
		
			\ex\label{verbderiv-ex22b}
			\gll  glex-\textbf{ma} datesa simind-\textbf{i}. \\
			peasant-\textbf{{\Erg}} s/he\_sowed\_it corn-\textbf{{\Nom}} \\
			\trans `The peasant sowed corn.'
			\hfill 
		
		
		
			\ex\label{verbderiv-ex22c}
			\gll glex-\textbf{s} dautesavs simind-\textbf{i}. \\
			peasant-\textbf{{\Dat}} s/he\_has\_sown\_it corn-\textbf{{\Nom}}\\
			\trans `The peasant has sown corn.'
			\hfill (All from \cites[1]{harris1981syntax})
		
		
	\end{xlist}
\end{exe}


Georgian has 4 verb classes. The morphological criteria for distinguishing these classes are summed up in \tabref{verbderiv-table-georgianverbclasses} (after \cites[260]{harris1981syntax}).

\begin{table}
	\begin{tabular}{lcccccc}
    \lsptoprule
		
		& Class 1 & Class 2 & Class 3 & Class 4 \\
		\midrule
		\textsc{fut/aor} formed with & \textsc{pv} & \textsc{pv} / \textit{e-} & \textit{i-} (\textit{-eb}) &  \textit{e-} \\
		
		Subject marker in \textsc{fut}, \textsc{3sg/3pl} & \textit{-s/-en} & \textit{-a/-an} & \textit{-s/-en} & \textit{-a}/various \\
		
		Subject marker in \textsc{aor}, \textsc{3pl} & \textit{-es} & \textit{-nen} & \textit{-es} & as \textsc{sg} \\
		\lspbottomrule
	\end{tabular}
	\caption{Georgian verb classes (\textsc{pv} = preverb)}
	\label{verbderiv-table-georgianverbclasses}
\end{table}

Roughly\footnote{But see \textcites[]{harris1981syntax} for exceptions.}, Class 1 consists of transitive verbs, Class 2 of so-called “inactive” intransitive verbs, Class 3 of so-called `active', agentive, atelic intransitive verbs (see \cites[]{holisky1981medial}), and Class 4 of verbs of cognition, emotion and possession. \tabref{verbderiv-table-georgianverbclassexamples} gives several examples per class (taken from \cites[261--267]{harris1981syntax}). Since in this work (following \cites[]{harris1981syntax}, but differing slightly from traditional Georgian grammatical description) the verb classes are defined purely on morphological grounds (\tabref{verbderiv-table-georgianverbclasses}), some exceptions do exist, such as intransitive verbs `yawn',  `cough' in Class 1 and transitives like `answer somebody' in Class 3.

\begin{table}
	\begin{tabular}{llll}
		\lsptoprule
		Class 1 & {Class 2} & {Class 3} & {Class 4} \\
		\midrule
		`heat sth' & `be' & `dance' & `love' \\
		`bake sth' & `fall' & `play' & `like' \\
		`rip sth' & `remain' & `quarrel' & `forget' \\
		`break sth off' & `be spread out' & `fight' & `have' \\
		`bend sth' & `be locked' & `cry' &  `can' \\
		`write sth' & `become white' & `talk' & `find sth difficult' \\
		`wash sth' & `become king' & `run' & `be afraid of sth' \\
		`sow sth' & `begin' & `roll' & `be hurt by sth' \\
		`yawn' &  `grow up' & `turn' & `be hungry' \\
		`cough' & `begin to play' & `answer sb' & `feel sleepy' \\
		\lspbottomrule
		
	\end{tabular}
	\caption{Examples of Georgian verb classes (Here and in the following tables, `sb' is `somebody' and `sth' is `something'.)}
	\label{verbderiv-table-georgianverbclassexamples}
\end{table}

It is important to note that verbs of different classes can be formed from the same root. The root \textit{-c'q'-}, for example, can appear as a Class 1 verb \textit{daic'q'ebs} `s/he/it will begin something (transitive)', and as a Class 2 verb \textit{daic'q'eba} `s/he/it will begin (intransitive)'. These verbs clearly have a derivational relationship, although it is not always straightforward to know which form is derived from which. Additionally, Class 1 verbs and Class 2 verbs can be derived from almost any adjective and many nouns.

Georgian Series, i.e. groups of related TAM forms, are defined by several criteria. Morphologically, Series I  is characterised by a so-called “thematic suffix”.\footnote{A handful of “root verbs”, verbs without a thematic suffix in Series I, also exists.} These suffixes (\textit{-eb, -en, -ev, -av, -i, -ob, -am}) are attached directly to the verb root, and do not appear in Series II. More clearly, however, the Series are shown by different alignment patterns (shown in \tabref{verbderiv-table-georgianalignment}), where Class 1 verbs take a Nominative Subject and a Dative object in Series I (Present, Imperfect, Future, Conditional and two Subjunctives), an Ergative subject and a Nominative object in Series II (Aorist and Optative), and a Dative subject and Nominative object in Series III (Perfect, Pluperfect and Perfect Subjunctive).

\begin{table}
	\begin{tabular}{lcccccc}
		\lsptoprule
		& \multicolumn{2}{c}{{Class 1}} & {Class 2} & {Class 3} & \multicolumn{2}{c}{{Class 4}}\\\cmidrule(lr){2-3}\cmidrule(lr){4-5}\cmidrule(lr){6-7}
		& S & O & S & S & S & O \\
		\midrule
		TAM series I & \textsc{nom}  & \textsc{dat} & \textsc{nom} & \textsc{nom} & \textsc{dat} & \textsc{nom} \\
		TAM series II & \textsc{erg} & \textsc{nom} & \textsc{nom} & \textsc{erg} & \textsc{dat} & \textsc{nom} \\
		TAM series III & \textsc{dat} & \textsc{nom} & \textsc{nom} & \textsc{dat} & \textsc{dat} & \textsc{nom} \\
		\lspbottomrule
	\end{tabular}
	\caption{Georgian alignment patterns}
	\label{verbderiv-table-georgianalignment}
\end{table}

Valency can be altered on Georgian verbs by adding a vowel (\textit{a, e, i, u}) directly before the verb root. These vowels can form indirect reflexives and various applicatives. Additionally, Georgian can form morphological causatives, all of which belong to Class 1. None of these will be discussed further here (see \cites[118--162]{vogt}, \cites[]{gerardin2022valencegeo}, \cites[170--204]{hewitt95} for a detailed description).


If we compare Tables \ref{verbderiv-table-georgianverbclasses}, \ref{verbderiv-table-georgianverbclassexamples} and \ref{verbderiv-table-georgianalignment}, we see that Class 3 verbs share many properties with Class 1 verbs, in terms of morphology (same person marking for subjects), semantics (agentive) and alignment pattern (a three-way split ergativity, contrary to Class 2 and 4). 

Furthermore, if we look at the alignment patterns for Series II, we find a similar system to that of Tsova-Tush verbs. That is, in both languages we find a basic transitive pattern with Ergative subjects and Nominative objects (similar to Georgian Class 1, see also \sectref{2arg}), a basic intransitive pattern with Nominative subjects (similar to Georgian Class 2) as well as a class of intransitive verbs with Ergative subjects (similar to Georgian Class 3, see \sectref{1arg}), and a class of experience verbs with Dative subjects and Nominative objects (Class 4, see \sectref{2arg}). Where verb types of Class 1, 2 and 4 are very common in almost all languages of the Caucasus, including in Chechen and Ingush, Class 3 verbs are relatively rare, and deserve a closer look.

\subsubsection{A closer comparison}\is{Medial verbs}


As seen in Sections \ref{ergintrans} and \ref{intr-erg-geo}, both Georgian and Tsova-Tush feature a class of single-argument verbs that allow (or require) an Ergative subject. However, one needs to look at what verbs actually belong to this class in both Tsova-Tush and Georgian. Below, the individual verbs in both languages are compared, relying heavily on work by Dee Ann Holisky, who has researched these types of verbs in both Georgian (\cites[]{holisky1981medial}) and Tsova-Tush (\cites[]{holisky87}), from which the data below is taken. All Tsova-Tush non-borrowed verbs without a preverb that Holisky classifies as having a usual or required Ergative subject will be discussed. I will discuss groups of verbs based on semantic criteria, starting with verbs of motion, and adding the Georgian translation given by \textcites[]{kadkad84} in the Future tense.



\begin{table}
	\small
	\tabcolsep=.75\tabcolsep
	\begin{tabular}{lllll}
    \lsptoprule
		Tsova-Tush & {S-Marking} & {English} & {Georgian} & {S-Marking} \\
		\midrule
		\textit{d-aržar} & \textsc{erg(nom)} &  `spread oneself out'  &  \textit{gaišleba} & \textsc{nom} \\
		\textit{d-eħ-d-alar}& \textsc{erg(nom)} & `sneak' &  \textit{miep'areba} & \textsc{nom} \\
		\textit{tatt-d-alar/tett-d-alar} & \textsc{erg(nom)} & `inch forward'  & \textit{miic'eva} & \textsc{nom} \\
		\textit{k'arčar/k'erčar} & \textsc{erg(nom)} & `roll around' &  \textit{gagordeba} & \textsc{nom} \\
		\textit{lak-d-alar/lek-d-alar} & \textsc{erg(nom)} & `rush' &  \textit{movardeba} & \textsc{nom} \\
		\textit{d-oc'-d-alar/d-ebc'd-alar} & \textsc{erg(nom)} & `follow' & \textit{aedevneba} & \textsc{nom} \\
		\textit{d-aɬar/ixar} & \textsc{erg} & `go'  & \textit{ava, miva} & \textsc{nom} \\ 
		\textit{d-at'ar/it'ar} & \textsc{erg} & `run'  & \textit{gaikceva} & \textsc{nom} \\
		\textit{d-axar/d-ot'ar}& \textsc{erg} & `leave, go' &  \textit{c'ava} & \textsc{nom} \\
		\textit{d-aɁar/d-a\u{g}ar} & \textsc{erg} & `come' &  \textit{mova} & \textsc{nom} \\
		\textit{egar} & \textsc{erg} & `enter, mix'  &  \textit{šeereva} & \textsc{nom} \\
		\textit{eqqar/letxar} & \textsc{erg} & `jump' &  \textbf{\textit{xt'is}} & \textsc{erg} \\
		\textit{taqar/teqar} & \textsc{erg} & `crawl' &  \textit{gaetreva} & \textsc{nom} \\
		\textit{lalar/lelar} & \textsc{erg} & `walk' &  \textit{dadis} & \textsc{nom} \\
		\textit{ottar/ettar} & \textsc{erg} & `stand up' &  \textit{dadgeba} & \textsc{nom} \\
		\textit{d-oɬar/deplar}& \textsc{erg} & `crawl, squeeze' & \textit{šeʒvreba} & \textsc{nom} \\
		\textit{qačar}& \textsc{erg} & `reach, arrive' &  \textbf{\textit{mia\u{g}c'evs}} & \textsc{erg}\footnote{labile verb}\\
		\textit{ħet'ar} & \textsc{erg} & `run'  & \textit{gaikceva} & \textsc{nom} \\
		\lspbottomrule
	\end{tabular}
	\caption{Tsova-Tush motion verbs and their Georgian counterparts.
		\textsc{erg(nom)} = “Ergative is the norm, Nominative is possible, but unusual or rare”, according to \textcites[]{holisky87}.}
	\label{verbderiv-table-motion}
\end{table}

Judging from \tabref{verbderiv-table-motion}, Tsova-Tush single-argument motion verbs with Ergative subjects mostly correspond to Georgian Class 2 verbs (which have a Nominative subject, as seen in the rightmost column), with two exceptions (bolded in the table), \textit{xt'is} `jump' and \textit{mia\u{g}c'evs} `reach, arrive' (the latter also occurs with an object). However, \textcites[111]{holisky1981medial} cites many Georgian Class 3 verbs of motion. They include: \textit{goravs} `roll', \textit{curavs} `swim, slide', \textit{seirnobs} `walk, stroll', \textit{cocavs} `crawl, climb', \textit{srialebs} `slide, slip, slither', \textit{kris} `rush'.

Turning to verbs of sound and expression, it is clear from \tabref{verbderiv-table-sound} that most Tsova-Tush verbs of this type correspond to Georgian verbs that have s single argument in the Ergative. Exceptions are \textit{čumdeba} `fall silent', which is a Class 2 verb, and \textit{iʒaxebs} `call' and \textit{ambobs} `talk, say', which are Class 1 verbs (and thus are transitive). 

\begin{table}
\begin{tabular}{lllll}
\lsptoprule
	Tsova-Tush & {S-Marking} & {English} & {Georgian} & {S-Marking} \\
	\midrule
	\textit{d-adar} & \textsc{erg} & `swear' & \textit{ipicebs} & \textsc{erg} \\
	\textit{d-atxar} & \textsc{erg} & `cry' & \textit{t'iris} & \textsc{erg} \\
	\textit{at'ar} & \textsc{erg} & `fall silent'  & \textit{čumdeba} &  \textsc{nom} \\
	\textit{axar} & \textsc{erg} & `bark' & \textit{q'eps} & \textsc{erg} \\
	\textit{d-elar} & \textsc{erg} & `laugh' & \textit{icinis} & \textsc{erg} \\
	\textit{d-ekar/qekar} & \textsc{erg} & `call' & \textit{iʒaxebs} & \textsc{erg}\footnote{transitive verb} \\
	\textit{tarsar/tersar} & \textsc{erg} & `neigh' & \textit{ič'ixvinebs} & \textsc{erg} \\
	\textit{lavar/levar} & \textsc{erg} & `talk, say' & \textit{ambobs} & \textsc{erg}\textsuperscript{\textit{a}} \\
	\textit{sart'ar/sert'ar} & \textsc{erg} & `curse' & \textit{c'q'evlis} & \textsc{erg} \\
	\textit{du\u{g}ar} & \textsc{erg} & `scream, cry' & \textit{q'viris} & \textsc{erg} \\
	\textit{kat'ar} & \textsc{erg} & `complain' & \textit{čivis} & \textsc{erg} \\
	\lspbottomrule
\end{tabular}
\caption{Tsova-Tush expression verbs and their Georgian counterparts}
\label{verbderiv-table-sound}
\end{table}




\begin{table}
\small
\tabcolsep=.75\tabcolsep
\begin{tabular}{lllll}
\lsptoprule
	Tsova-Tush & {S-Marking} & {English} & {Georgian} & {S-Marking} \\
	\midrule
	\textit{d-exk'-d-alar/axk'-d-alar} & {\Erg} ({\Nom}) & `get stuck' & \textit{miebmeba} & {\Nom} \\
	\textit{d-itt-d-alar}  & {\Erg} ({\Nom}) & `take a bath' & \textit{ibanavebs} & {\Erg} \\
	\textit{d-il-d-alar}  & {\Erg} ({\Nom}) & `get washed' & \textit{ibanavebs} & {\Erg} \\
	\textit{lat'ar} & {\Erg} ({\Nom})  & `add to' & \textit{imat'ebs} & {\Erg}  \\
	\textit{d-ol-d-alar/d-ebl-d-alar}  & {\Erg} ({\Nom}) & `begin' & \textit{daic'q'eba} & {\Nom} \\
	\textit{kott-d-alar}  & {\Erg} ({\Nom}) & `be worried' & \textit{šec'uxeba} & {\Nom} \\
	\textit{c'am-d-alar/c'em-d-alar}  & {\Erg} ({\Nom}) & `get clean' & \textit{daic'mindeba} & {\Nom} \\
	\textit{ħarčar/ħerčar}  & {\Erg} ({\Nom}) & `surround' & \textit{moexveva} & {\Nom} \\ 
	\textit{cer-d-aɬar} & {\Erg}  & `finish'  & \textit{gatavdeba} & {\Nom} \\
	\textit{d-arc'-d-alar} & {\Erg} & `get undressed' & \textit{gat'it'vldeba} & {\Nom} \\
	\textit{d-axar}  & {\Erg} & `live'  & \textit{icxovrebs} & {\Erg} \\
	\textit{d-aqar}  & {\Erg} & `suck, nurse'  & \textit{mosc'evs} & {\Erg}\footnote{transitive verb} \\
	\textit{dak'-d-aɬar}  & {\Erg} & `figure out' & \textit{mixvdeba} & {\Nom} \\
	\textit{dak'lavar}  & {\Erg} & `think'  & \textit{ipikrebs} & {\Erg}\footnote{labile verb} \\
	\textit{tešar}  & {\Erg} & `believe'  & \textit{eǯereba} & {\Nom} \\
	\textit{txil-d-alar}  & {\Erg} & `be careful'  & \textit{gaprtxildeba} & {\Nom} \\
	\textit{latar/letar}  & {\Erg} & `fight' & \textit{eč'ideaveba} & {\Nom} \\
	\textit{lap'c'ar}  & {\Erg} & `play'  & \textit{itamašebs} & {\Erg} \\
	\textit{lat'ar/let'ar}  & {\Erg} & `help' & \textit{miešveleba} & {\Nom} \\
	\textit{qap't'ar/qep't'ar}  & {\Erg} & `reach' & \textit{misc'vdeba} & {\Nom} \\
	\textit{ħač'ar/ħapsar}  & {\Erg} & `look (\textsc{sg/pl})' & \textit{šexedavs} & {\Erg}\textsuperscript{\textit{a}}  \\
	\lspbottomrule
\end{tabular}
	\caption{Other Tsova-Tush verbs and their Georgian counterparts (\textsc{erg(nom)} = `Ergative is the norm, nominative is possible, but unusual or rare', according to \textcites[]{holisky87})}
	\label{verbderiv-table-other}
\end{table}

As seen from \tabref{verbderiv-table-other}, from the 21 intransitive Tsova-Tush verbs that have no semantics related to motion or expression, but do require or prefer an Ergative subject, only five correspond to a Georgian Class 3 verb: \textit{ibanavebs} `wash' (the translation of both Tsova-Tush \textit{d-itt-d-alar} and \textit{d-il-d-alar}), \textit{imat'ebs} `add', \textit{icxovrebs} `live', \textit{ipikrebs} `think' (which can occur with an object) and \textit{itamašebs} `play'. Other than these Class 3 and two Class 1 verbs (\textit{mosc'evs} `suck', \textit{šexedavs} `look'), most Tsova-Tush verbs in \tabref{verbderiv-table-other} correspond to Georgian Class 2 verbs, i.e. verbs with Nominative subject marking throughout the TAM system.

In trying to answer the question whether the creation of a class of Tsova-Tush intransitive verbs with (possible or obligatory) Ergative subjects was influenced by Georgian at all, using the data in Tables~\ref{verbderiv-table-motion}--\ref{verbderiv-table-other}, I draw the conservative conclusion that Georgian influence is not warranted as an explanation of this type of verb. Only verbs of utterance show a clear correlation between Tsova-Tush and Georgian case marking, whereas most Tsova-Tush verbs of motion and verbs with other semantics correspond to Georgian Class 2 verbs, i.e. Georgian verbs with a Nominative subject in all TAM forms.

Additionally, there are three crucial differences between the sets of verbs in Tsova-Tush and Georgian.
\begin{enumerate}
	\item Tsova-Tush allows both Nominative and Ergative subjects for some verbs, whereas Georgian requires Ergative subjects for all Class 3 verbs.
    
	\item Tsova-Tush only requires/allows 1st and 2nd person subjects to be in the Ergative for this type of verb. In Georgian, all three persons are required to be in the Ergative, although 1st and 2nd person pronouns do not distinguish Nominative, Ergative or Dative case. Thus, Ergative case marking on intransitive subjects can only be observed with 1st and 2nd person in Tsova-Tush, and only with 3rd person in Georgian.
    
	\item In Georgian, single-argument verbs require an Ergative subject when they belong to Class 3, that is, when they exhibit the morphological features described above in \sectref{geoalign}. In other words, the Georgian rule is based on morphosyntax.\footnote{Although, in fact, some monovalent Class 1 verbs exist, such as \textit{daiʒinebs} `sleep', \textit{gaiğviʒebs} `wake up', \textit{daisvenebs} `take a rest'. I thank an anonymous reviewer for this important qualification.} In Tsova-Tush, single-argument verbs require a 1st or 2nd person Ergative subject if that subject is stereotypically agentive (that is, when the subject is acting in control and voluntarily, see \sectref{intrerg} above), which is a rule based directly on semantics (\cites[]{holisky87}).
\end{enumerate}

This means that the origin of Tsova-Tush variable case marking on intransitive subjects remains an open question. In some languages like Udi\il{Udi} (\cites[]{harris10unacc}) or Ingush, intransitive subjects marked with the Ergative case can relatively clearly be attributed to a historical process of noun incorporation. See for example Ingush \textit{nabj-u} `s/he sleeps', historically from \textit{d-u} `s/he does' with gender marker \textit{j-} referencing the incorporated noun \textit{nab} `sleep' (\cites{nichols08ingushcase,forker17ergativity}). In Tsova-Tush, however, this pathway is unlikely. From the verbs in Tables~\ref{verbderiv-table-motion}--\ref{verbderiv-table-other}, only three are noun-incorporating: \textit{cer-d-aɬar} `finish' from `boundary' + `go out', \textit{dak'-d-aɬar} `figure out', from `heart' + `go out', and \textit{dak'-lavar} `think', from `heart' + `speak'. Even with these complex verbs (see \sectref{lightverbs}), the original verb is an intransitive verb of motion or expression, not a transitive one.\is{Noun incorporation}

Since Ergative marking on subjects of monovalent verbs often occurs through elision of frequently occurring objects (such as `eat (food)'), one could imagine such a type of A-lability to be the basis of Tsova-Tush verbs of expression, with the elision of objects (the expression or utterance) causing erstwhile transitive verbs to become intransitive with retention of the Ergative marking. However, A-lability is shown to be very rare in languages of the Caucasus. 



Many Georgian class 3 verbs are themselves also borrowed into Tsova-Tush, for which see \sectref{loanverbmed}.

\section{Complex verbs} \label{lightverbs}

Besides simple verbs, Tsova-Tush features many complex verbs, consisting of at least two recognisable parts.
Before we can shed some light on Tsova-Tush verbal derivation, it is helpful to gain a general understanding of complex verbs in Tsova-Tush. In this work, I distinguish four types of complex verbs: (1) compound verbs containing a light verb (\sectref{lightverbsp}, of which some contain a fossilised gender marker, see \sectref{lightverbfoss}), (2) Dvandva compound verbs, (3) reduplicating verbs and (4) verbs containing the suffixes \textit{-d-i} or \textit{-d-al}. Before investigating complex verbs themselves, it is useful to take a look at a specific type of complex predicate, the light verb construction.


\subsection{Light verb construction} \label{lvconstr}\is{Light verbs}


Light verb constructions are idiomatic expressions consisting of a verb and a nominal object. They are verbal constructions, where the combined meaning is more than the sum of its parts. Most nominal and verbal components can be found independently as well, for example \textit{bak b-aɬar} `come to an agreement' (`mouth' + `give'), \textit{bek'i b-ar} `joke' (`joke' + `do') \textit{bexk' b-aqar} `accuse' (`fault' + `take'), \textit{botx b-ar} `work' (`work' + `do'), \textit{gon j-aɬar} `be startled' (`mind' + `rush'), \textit{mʕaɁ\u{o} j-aɬar} `butt' (`horn' + `give'). Others, however, are not found independently, such as \textit{baram b-aqar} `go through, make path' (\textit{baram} + `take'), \textit{bad jo\u{g}ar} `trap with net' (`net' + \textit{d-o\u{g}ar}\footnote{\textit{d-o\u{g}ar} is additionaly only found in \textit{k'ur bo\u{g}ar} `blacken with smoke' (from  `smoke') and \textit{c'e jo\u{g}ar} `name, nominate' (from `name').}), where \textit{baram} and \textit{d-o\u{g}ar} do not have an independent meaning. 

The fact that these constructions are syntactically equivalent to any other transitive predicative construction is shown by (1) the fact that the negative particle \textit{co}, interrogatives and preverbs can be inserted between the nominal and the verbal element (Example (\ref{verbderiv-ex18a})), (2) the possibility of changing the OV order (Example (\ref{verbderiv-ex18b})), and (3) the fact that the gender marker on the verb cross-references the nominal element of the construction, not another argument.\is{Cross-referencing!Gender}



\begin{exe}
	\ex\label{verbderiv-ex18}
	\begin{xlist}
		
		
			\ex\label{verbderiv-ex18a}
			\gll  as k'nat-a-x \textbf{bexk'}=a moħ \textbf{b-aq-o-s}? \\
			{\Fsg}.{\Erg} boy-{\Obl}.{\Pl}-{\Cont} \textbf{fault({\B})}={\Emph} how \textbf{{\B}.{\Sg}-take-{\Npst}-{\Fsg}.{\Erg}} \\
			\trans `How can (lit. do) I reproach the boys?'
			\hfill (E032-3)
		
		
		
			\ex\label{verbderiv-ex18b}
			\gll at't'a-v qe-č\u{o} as-e-n \textbf{teɬ-\u{o}} \textbf{mʕaɁ\u{o}}.  \\
			cow.{\Obl}-{\Erg} other-{\Obl} calf-{\Obl}-{\Dat} \textbf{give.{\Ipfv}-{\Npst}} \textbf{horn} \\
			\trans `A cow head-butts an unknown calf.'
			\hfill (MM422-1.1)
		
		
		
	\end{xlist}
\end{exe}

The verbal element of this type of predicative construction can be classified as so-called light verbs\is{Light verbs} (in the sense of \cites[117--118]{jespersen54lightverb}). A Tsova-Tush light verb is a verb that is semantically bleached when combined with another element in a predicative construction (or in a compound, see \sectref{lightverbsp}). In fact, the term ``light'' here refers to the fact that these verbs, in their combination with certain lexical material, are referentially extremely broad in terms of their semantics. For a list of the most common light verbs, see \tabref{verbderiv-table-lightverbs}.


Tsova-Tush light verbs are distinct from auxiliary verbs.\is{Auxiliary verbs} Auxiliary verbs, often with modal semantics, occur in combination with an Infinitive form of the main verb (as in Example (\ref{verbderiv-ex31}), and have defective inflection (i.e. they do not occur in all tense-aspect forms (see \cite{holisky94}).


\begin{exe}
	\ex\label{verbderiv-ex31}
	\begin{xlist}
		
		
			\ex\label{verbderiv-ex31a}
			\gll lamu že aħ \textbf{d-ec'} d-ett-aⁿ. \\
			mountain.{\Ess} sheep {\Pv} \textbf{{\D}-must} {\D}-milk-{\Inf} \\
			\trans `In the mountains, the sheep must be milked.'
			\hfill (E002-31)
		
		
		
			\ex\label{verbderiv-ex31b}
			\gll  kikoɁ bac-bi širik ix-aⁿ \textbf{lat-er} že-v=aɁ. \\
			early Tsova\_Tush-{\Pl} Shiraki.{\Ill} go.{\Ipfv}-{\Inf} \textbf{{\Hab}-{\Imprf}} sheep-{\Ins}={\Emph} \\
			\trans `Earlier, the Tsova-Tush would go to Shiraki with their sheep.' \\
			\hfill (E058-33)
		
		
	\end{xlist}
\end{exe}


\subsection{Complex verbs containing a light verb} 

The same set of light verbs as discussed above in \sectref{lvconstr} can be used to form compound verbs. These verbs consist of a first element (usually non-verbal) bearing the lexical semantic information, and a light verb, to which all verbal inflection is attached.

\subsubsection{Basic pattern} \label{lightverbsp}\is{Compounding!Verbal}

In \tabref{verbderiv-table-lightverbs}, the most common compounds containing a light verb are listed. Notice that some lexical components (\textit{bʕar}, \textit{ʕa}, \textit{\u{g}os}) do not have an independent meaning.

\begin{table}
	\small
	\begin{tabular}{l@{~}l l@{~}l l@{~}l}
		\lsptoprule
		\multicolumn{2}{l}{{Light verb}} & \multicolumn{2}{l}{{Example}} & \multicolumn{2}{l}{Lexical component} \\
        \midrule
        
		\textit{d-ot'ar} & `go' & \textit{aq'r-d-ot'ar} & `bend down' & \textit{aq'r} `supine' \\
		
		\textit{d-(i)-ar} & `do' & \textit{bʕar-d-ar} & `meet' & \textit{bʕar}\footnote{But compare \textit{bʕark'} `eye'.} \\
		
		\textit{d-isar} & `stay' & \textit{bad-d-isar} & `be left as orphan' & \textit{bad\u{o}} `orphan' \\
		
		\textit{d-axar} & `go' & \textit{bʕar-d-axar} & `meet' & \textit{bʕar} \\
		
		\textit{d-aɬar} & `appear, go out' & \textit{dak'-d-aɬar} & `realise' & \textit{dak'-} `heart' \\
		\textit{d-aqar} & `take' & \textit{dak'-d-aqar} & `bring to mind' & \textit{dak'-} `heart' \\
		
		\textit{d-a\u{g}ar} & `come' & \textit{ʕa-d-a\u{g}ar} & `sit'\footnote{Note that despite the light verb being a verb of motion, the compound does not mean `(go and) sit down', but stative `sit'.} & \textit{ʕa} \\
		
		\textit{aɬar} & `say' & \textit{auħaɬar}  & `cough' & \textit{auħ} (onomat.) \\
		
		\textit{xiɬar} & `be' & \textit{a\u{g}azxiɬar} &	`be used' & \textit{a\u{g}az} `suitable' \\
		
		\textit{ixar} & `go' & \textit{zaq'q'ixar} & `crack' & \textit{zaq'q'} `crack (\textsc{n})' \\
		
		& & \textit{ʕepixar} & `be ashamed' & \textit{ʕep} `shame' \\
		
		\textit{tasar} & `drop, leave' & \textit{gontasar} & `bring to senses' & \textit{gon} `mind' (GE) \\
		
		\textit{lavar} & `speak' & \textit{dak'lavar} & `think' & \textit{dak'-} `heart' \\
		
		\textit{xetar} & `find, deem, think' & \textit{bek'xetar}	& `be surprised' & \textit{bek'} `wonder' \\
		& & \textit{\u{g}osxetar} & `be glad' & \textit{\u{g}os} \\
		& & \textit{č'irxetar} & `be lazy' & \textit{č'ir} `trouble' (GE) \\
		\lspbottomrule
	\end{tabular}
	\caption{Common light verb compounds. (\textsc{n}) = Noun; (GE) = borrowed from Georgian)}
	\label{verbderiv-table-lightverbs}
\end{table}


All verbs containing a light verb can be shown to be single words, evidenced by the fact that negative particles, preverbs and interrogatives come before the entire compound, not immediately before the light verb, (see Example (\ref{verbderiv-ex20})). Furthermore, the gender marker on the light verb, if present, cross-references an external argument, not the lexical component itself, as in (\ref{verbderiv-ex20a}) where the gender marker \textit{v-} cross-references the Nominative subject, which is masculine, not the lexical component of the verb \textit{dak'}, which is D gender. 

\begin{exe}
	\ex\label{verbderiv-ex20}
	\begin{xlist}
		
		
			\ex\label{verbderiv-ex20a}
			\gll moħ co dak'-v-aɬ-en-v-a-ra-s txa=lomciⁿ?  \\
			how {\Neg} heart-{\M}.{\Sg}-{\Lv}-{\Ptcp}.{\Pst}-{\M}.{\Sg}-be-{\Imprf}-{\Fsg}.{\Nom} today=until \\
			\trans `How have I never remembered it until today?'
			\hfill (MM417-1.46)
		
		
		
			\ex\label{verbderiv-ex20b}
			\gll co bʕarix-or soⁿ. \\
			{\Neg} meet.{\Ipfv}-{\Imprf} {\Fsg}.{\Dat} \\
			\trans `I didn't meet her.'
			\hfill (E255-24)
		
		
	\end{xlist}
\end{exe}


\subsubsection{With a fossilised gender marker} \label{lightverbfoss}

In a minority of compound verbs containing a light verb, the gender marker is fossilised. These compounds have also fully fused and are single words, as shown by the placement of the negative particle, interrogative or the preverb before the noun-verb complex (see Example (\ref{verbderiv-ex19})). However, if the verbal part of the compound verb had a gender marker, this prefix is now frozen and unchanging, resulting in a compound where the noun or onomatopeia in question is completely incorporated. As such, in Example (\ref{verbderiv-ex19}), the original construction \textit{k'eč j-aq-o} `boiling({\J}) {\J}-take-{\Npst}' is now fused into a single intransitive verb, where the gender marker \textit{j-} does not reference any external argument anymore.  Examples are presented in \tabref{verbderiv-table-unilightverbs}. 



	\begin{exe}
		\ex\label{verbderiv-ex19}
		\gll moħ-e k'i ħal \textbf{k'eč'jaq}-uj-c oqu-s mak k'alt' j-ebl-\u{o}.  \\
		how-{\Rel} {\Contr} {\Pv} \textbf{come\_to\_boil}-{\Npst}-{\Subord} {\Dist}-{\Erg} on.{\Pv} cottage\_cheese {\J}-begin.{\Ipfv}-{\Npst} \\
		\trans `When it begins to boil, upon its surface the curds appear.'
		\hfill (E007-49)
	\end{exe}




\begin{table}
	\small
	\begin{tabular}{ll ll ll}
		\lsptoprule
		\multicolumn{2}{l}{{Light verb}} & \multicolumn{2}{l}{{Example}} & \multicolumn{2}{l}{Lexical component} \\
		\midrule
		\textit{d-(i)ar} & `do' & \textit{aⁿhdar} & `moan, groan' & (onomat.) \\
				
		                &        & \textit{buudar}	& `buzz, hum, drone' & (onomat.) \\
		                &        & \textit{t'laq'dar} & `rumble, growl' & (onomat.) \\
		                &        & \textit{buħdar} & `throw a fight' & `fight' \\
		                &        & \textit{bʕark'bar} & `put the evil eye' & `eye' \\
		                &        & \textit{dadoldar} & `protect, patronise' & `ownership, propriety' \\
		                &        & \textit{tešombar} & `give assent, agree' & `believe, faith' \\
		\textit{d-aɁar} & `come' & \textit{xsnildaɁar} & `break Lent' & `Lent' (GE) \\
		\textit{d-aɬar} & `give' & \textit{bakbaɬar} & `come to an agreement' & `heart' \\
		\textit{d-aqar} & `take' & \textit{k'eč'jaqar} & `bring to a boil' & \textit{k'eč'} \\
		\lspbottomrule
	\end{tabular}
	\caption{Fossilised light verb compounds}
	\label{verbderiv-table-unilightverbs}
\end{table}

Note that the distinction between the complex verb containing a light verb in \tabref{verbderiv-table-lightverbs} and those in \tabref{verbderiv-table-unilightverbs} is only relevant for verbs that have gender marking. Only when a gender marker is present, and this gender marker is not fossilised, but is cross-referencing an external argument, can we speak of a synchronically complex verb. This type, exemplified in (\ref{verbderiv-ex20a}), can be characterised as having a bipartite stem: a verb with inflection splitting the stem in two. The East Caucasian family is known for the widespread occurrence of bipartite verbal stems (\cite{nichols03bipartite}), and these, although more frequent in other members of the family, can be found in Tsova-Tush as well.\is{Bipartite stems}
\pagebreak

\subsection{Dvandva compound verbs}\is{Compounding!Verbal}

Dvandva compound verbs are verbs that consist of two verbal stems, whose combined meaning is the sum of the semantics of both stems. These verbs can be analysed as verbal \textit{dvandva} (i.e. copulative/coordinating) compounds (for an analogue in Greek, see \cite{nicholasjosephgreek}), and possibly go back to earlier serial verb constructions. Examples include \textit{at'xalar} `die out, fade away' (\textit{at'ar} `become silent', \textit{xalar} `fade'), \textit{d-uit'-d-a\u{g}ar} `go back and forth' (\textit{d-ot'ar} `go', \textit{d-a\u{g}ar} `come'), \textit{qall-maɬar} `eat and drink' (\textit{qallar} `eat', \textit{maɬar} `drink'), see Example (\ref{verbderiv-ex16}).


	\begin{exe}
		\ex\label{verbderiv-ex16}
		\gll  \textbf{qall-maɬ}-en\u{e}, supr daħ d-ox-d-i-en\u{e}. \\
		\textbf{eat.{\Pfv}-drink.{\Pfv}}-{\Aor}.{\Seq} feast {\Pv} {\D}-destroy-{\D}-{\Tr}-{\Aor}.{\Seq} \\
		\trans `They ate and drank and messed up the banquet.'
		\hfill (MM335-1.7)
	\end{exe}



\subsection{Reduplicating verbs}\is{Reduplication}

Reduplicating verbs consist of a verbal stem and a second element that is identical to this stem but for the first segment, which is replaced by another consonant, for example \textit{tak'-sak'ar} `patch up' (from \textit{tak'ar} `sew'), \textit{t'at'-šat'-d-alar} `become a little moist' (from \textit{t'at'-d-alar} `become moist'), \textit{d-opx-sopx-d-ar} `adorn, decorate' (from \textit{d-opx-d-ar} `dress') and \textit{kak'-lak'-d-ar} `stir, mix up' (from \textit{kak'-d-ar} `mix'). Since the second parts of these verbs (\textit{sak'-, šat'-, sopx-, lak'-}) do not exist as independent verbs, they are synchronically classed as reduplicating.


	\begin{exe}
		\ex\label{verbderiv-ex17}
		\gll lebiv \textbf{kak'-lak'-b-}∅\textbf{-eba-t}, coħ\u{e} bux-e-x ču lajt-b-is-\u{u}. \\
		beans \textbf{stir-{\Redupl}-{\B}.{\Sg}-{\Tr}-{\Imp}-{\Pl}} if\_not base-{\Obl}-{\Cont} {\Pv} fix-{\B}.{\Sg}-{\Lv}-{\Npst}\\
		\trans `Stir the beans around, otherwise they will stick to the bottom.' \\
		\hfill (KK023-3932)
	\end{exe}



\subsection{Suffixes \textit{-d-i, -d-al}} \label{didal}

Superficially similar to compounds containing a light verb are derived verbs containing the suffixes \textit{-d-i} or \textit{d-al}, where \textit{-d-} is any gender marker. Although very similar to the light verb compounds seen in \sectref{lightverbsp}, the elements \textit{-d-i} and \textit{-d-al} are in this work analysed as synchronically suffixes (\cites[185]{holiskygagua,harris09}), for two reasons: 

\begin{enumerate}
\item An independent verb \textit{d-alar} does not exist.\footnote{The verb \textit{d-alar} `die', Imperfect \textit{lar} is semantically too far removed, and is not likely to be cognate; the verb \textit{d-alar}, Imperfect \textit{d-alir} `be held' is part of another inflection class, and therefore also not likely to be cognate.} The suffix \textit{-d-i} is formally identical to the light verb \textit{d-i} (Verbal Noun \textit{d-ar}) that also exists as an independent verb `do'. However, since the suffixes \textit{-d-i} and \textit{-d-al} derive pairs of verbs (see below in \sectref{deadjverb}), I assume the same level of grammaticalisation for both suffixes.

\item Light verb compounds are unproductive, whereas verb pairs containing the suffixes \textit{-d-i, -d-al} are to a large extent productive (see below for deadjectival verbs), and can even be derived from light verb compounds (e.g. \textit{gontas-d-alar} `come to one's senses', from \textit{gon(-)tasar} `bring to one's senses', from \textit{gon} `intellect'\,+ \textit{tas-} `drop, leave, throw').
\end{enumerate}

\subsubsection{Deadjectival verbs} \label{deadjverb}\is{Derivation!Verbalising}\is{Derivation!Deadjectival}

Both transitive and intransitive verbs can be derived from all native Tsova-Tush adjectives, see \tabref{verbderiv-table4}. All adjectives lose their ending \textit{-Vn} if they have it, and some require an additional suffix \textit{-ar}. Note that the absence of the ending \textit{-Vn} is not an instance of phonological reduction, but a historical morphological alternation: the loss of high vowels \textit{i, u} does not trigger umlaut on the preceding vowel. All formations that add \textit{-ar} are listed under (a) in \tabref{verbderiv-table4}, whereas all other deadjectival verbal derivation behaves as in the examples under (b). The origin of the morph \textit{-ar} is unknown.\footnote{It could be connected to the denominal adjectivising suffix \textit{-aren} (e.g. \textit{lavareⁿ} `snowy' from \textit{lav} `snow'), but if it were related, the same suffix would be expected on the adjective, that is, one would expect *\textit{\u{g}azareⁿ} `good', \textit{\u{g}azar-d-ar} `make good'. Instead, we find \textit{\u{g}azeⁿ} `good', without the suffix.}

\begin{table}
	\footnotesize
	\begin{subtable}{\textwidth}
	\caption{}
	\begin{tabular}{ *3{l@{~}l} }
        \lsptoprule
		\multicolumn{2}{l}{Intransitive} & \multicolumn{2}{l}{Transitive} & \multicolumn{2}{l}{Adjective} \\
		\midrule
		\textit{baxar-d-alar} & `grow rich' & \textit{baxar-d-ar} & `enrich' & \textit{bax} & `rich' \\
		\textit{duqar-d-alar} & `increase' & \textit{duqar-d-ar} & `increase' & \textit{duq} & `many' \\
		\textit{tišar-d-alar} & `age' & \textit{tišar-d-ar} & `age'  & \textit{tišin} & `old' \\
		\textit{k'ac'k'ar-d-alar} & `become smaller' & \textit{k'ac'k'ar-d-ar} & `make smaller' & \textit{k'ac'k'on} & `small' \\
		\textit{mosar-d-alar} & `go bad' & \textit{mosar-d-ar} & `make bad' & \textit{mossin} & `bad' \\
		\textit{učar-d-alar} & `get dark' &\textit{učar-d-ar} & `darken' & \textit{učin} & `dark' \\
		\textit{\u{g}azar-d-alar} & `become good' &\textit{\u{g}azar-d-ar} & `make good' & \textit{\u{g}azen} & `good' \\
		\textit{c'inar-d-alar} & `become new' & \textit{c'inar-d-ar} & `renew' & \textit{c'in} & `new' \\
		\textit{vadar-d-alar} & `be angry' & \textit{vadar-d-ar} & `make angry' &  \textit{vadon} & `harmful, bad'\\
		\lspbottomrule
	\end{tabular}
	\end{subtable}\medskip\\
	\begin{subtable}{\textwidth}
	\centering
	\caption{}
	\begin{tabular}{ *3{l@{~}l} }
		\lsptoprule
		\multicolumn{2}{l}{Intransitive} & \multicolumn{2}{l}{Transitive} & \multicolumn{2}{l}{Adjective} \\
		\midrule
		\textit{ap-d-alar} & `become green' & \textit{ap-d-ar} & `make green' & \textit{apen} & `green' \\
		\textit{d-aq'-d-alar} & `dry out' & \textit{d-aq'-d-ar} & `dry' & \textit{d-aq'in} & `dry' \\
		\textit{d-aq-d-alar} & `grow' & \textit{d-aq-d-ar} & `raise' & \textit{d-aqqon} & `big' \\
		\textit{k'ap'rš-d-alar} & `become yellow' & \textit{k'ap'rš-d-ar} & `make yellow' & \textit{k'ap'raš} & `yellow' \\
		\textit{k'ʕav-d-alar} & `become lame' & \textit{k'ʕav-d-ar} & `make lame' & \textit{k'ʕaven} & `lame' \\
		\lspbottomrule
	\end{tabular}
	\end{subtable}
	\caption{Tsova-Tush equipollent derivation from adjectives}
	\label{verbderiv-table4}
\end{table}


\subsubsection{Other complex verbs in \textit{-d-i, -d-al}}


The suffixes \textit{-d-i} and \textit{-d-al}, as well as the light verb \textit{d-isar} `stay', can be used in valency derivations based on verbal stems, see Sections \ref{intr} and \ref{tr}. Productivity is relative: \textit{d-isar} is only found with 8 verb stems (see \tabref{verbderiv-table6}), and although \textit{-d-i} and \textit{-d-al} are used to incorporate all verbs borrowed from Georgian (see \sectref{loanverb}), they cannot be used to derive new verbs from any native verbal stem. As can be seen from \sectref{tr}, more transitive verbs are derived from intransitive ones than vice versa, as is typical for western East Caucasian languages (\cite{nichols13}).


\section{Valency derivation} \label{verbderivproper}
\subsection{Introduction} \label{valintro}

East Caucasian languages are known to have little to no grammatical voice systems (e.g. \cites[]{comrie2000tsezval}), and Tsova-Tush is no different in this respect. It does, however, have lexical derivations that change valency. One, described in \sectref{intr}, is the suffix \textit{-d-al}. This suffix detransitivises transitive verbs,\is{Derivation!Deverbal}\is{Derivation!Verbalising} producing mostly anticausative\is{Anticausative} verbs. The second derivation, described in \sectref{tr}, is the suffix \textit{-d-i} (which is formally identical to the light verb \textit{d-ar} (see \sectref{lvconstr} above), and the verb \textit{d-ar} `do'). \textit{-d-i} is a transitivising suffix, producing causative or other transitive verbs.  Since both suffixes are also used to derive verbs from nominal, adjectival and borrowed verbal stems that do not show any valency, \textit{-d-al} receives the basic label \textsc{intr} in this work, and \textit{-d-i} the basic label \textsc{tr}. Both suffixes produce lexical derivations, and are by no means productive: not every underived intransitive verb has a derived transitive counterpart, or vice versa.


Clearly, the verbs containing the valency-changing suffixes \textit{-d-i} or \textit{-d-al} bear a strong resemblance to the light verb compounds of the type \textit{bʕar-d-ar} `meet', \textit{dak'-lavar} `think' described in \sectref{lightverbsp}. Nevertheless, although the Tsova-Tush suffixes \textit{-d-al} and \textit{-d-i} must have undoubtedly grammaticalised from the same type of light verb construction, they are synchronically derivational suffixes. This analysis, as mentioned above in \sectref{didal}, is supported by the following two facts:

\begin{enumerate}
\item An independent verb \textit{d-alar} does not exist. The independent existence of a light verb is a defining characteristic of light verbs, at least by authors such as \textcite[106]{wohlgemut09loanverbtyp}.\footnote{In descriptions of East Caucasian languages, however, the term light verb is often used for the second part of a verbal compound, which is clearly verbal in morphology and origin, but does not exist as an independent lexical item.} The suffix \textit{-d-i} is formally identical to the light verb \textit{d-i} (Verbal Noun \textit{d-ar}) that also exists as an independent verb `do'. However, since the suffixes \textit{-d-i} and \textit{-d-al} derive pairs of verbs (see \sectref{deadjverb} for deadjectival verbs, and \sectref{equi} for deverbal verbs), I assume the same level of grammaticalisation for both suffixes.

\item Light verb compounds are unproductive, whereas verb pairs containing the suffixes \textit{-d-i, -d-al} are to a large extent productive (see \sectref{deadjverb} for deadjectival verbs), and can even be derived from light verb compounds (e.g. \textit{gontas-d-alar} `come to one's senses', from \textit{gon(-)tasar} `bring to one's senses', from \textit{gon} `intellect' + \textit{tas-} `drop, leave, throw').
\end{enumerate}

In addition to these two basic constructions, Tsova-Tush features a highly productive causative suffix \textit{-it}, which can combine with all verbs, described in \sectref{it-caus}. Due to its complete productivity, it could be considered inflectional rather than derivational. Regardless, it is a primary way of changing valency, and thus it is described here.

Although not strictly changing valency, the productive suffix \textit{-mak'}, creating Potential verbs, is described in \sectref{potential}.


\subsection{Detransitive} \label{intr}\is{Intransitive verbs}\is{Anticausative}

Suffixing \textit{-d-al} to the verbal stem creates single-argument (usually anticausative) verbs. Once added to inherently two-argument verbs, the Ergative argument is dropped. Gender agreement is with the single Nominative argument. \tabref{verbderiv-table2} illustrates several of the 19 verb pairs in \textcite{kadkad84} that consist of an intransitive verb derived from a transitive verb. Two verbs (\textit{d-ott-d-alar} `get rough, agitated' and \textit{ħot't'-d-alar} `approach') do not have a transitive counterpart. The suffix \textit{-d-al} is rarely added to already intransitive verbs, and cannot be suffixed to transitive verbs which carry the suffix \textit{-d-i} (see below in \sectref{tr}).


\begin{table}
	\tabcolsep=.75\tabcolsep
	\begin{tabular}{llll}
		\lsptoprule
		\multicolumn{2}{l}{Transitive} & \multicolumn{2}{l}{Intransitive} \\
		\midrule
		\textit{atar} & `pound, thresh' & \textit{at-d-alar} & `become soft' \\
		\textit{gontasar} & `bring sb to one's senses' & \textit{gontas-d-alar} & `come to one's senses' \\
		\textit{d-aɬar} & `give' & \textit{d-aɬ-d-alar} & `be given to sb' \\
		\textit{et'ar} & `spread out' & \textit{et'-d-alar} & `be spread out' \\
		\textit{q'(ʕ)egar} & `break sth'  & \textit{q'(ʕ)eg-d-alar} & `break'  \\
		
		\lspbottomrule
	\end{tabular}
	\caption{Tsova-Tush detransitive derivation}
	\label{verbderiv-table2}
\end{table}

Examples (\ref{verbderiv-ex08}) and (\ref{verbderiv-ex09}) show how the transitive verbs \textit{d-ʕogar}/\textit{q'egar} `break' and \textit{d-ottar}/\textit{d-ettar} `pour' serve as the basis for detransitive derivation using the suffix \textit{-d-al}. The transitive verbs show subjects in the Ergative and objects in the Nominative, whereas the derived verbs allow only one Nominative argument.

\begin{exe}
	\ex\label{verbderiv-ex08}
	\begin{xlist}
		
		
			\ex\label{verbderiv-ex8a}
			\gll cħajn mercxlaɁo-s p'ʕaⁿ b-ʕog-iⁿ. \\
			one.{\Obl} swallow-{\Erg} wing.{\Nom} {\B}.{\Sg}-break.{\Pfv}-{\Aor} \\
			\trans `One swallow broke its wing.'
			\hfill (E024-2)
		
		
		
			\ex\label{verbderiv-ex08b}
			\gll urm-e-goħ ši-k'eɁ borbol b-ʕog-\textbf{b-al}-iⁿ. \\
			cart-{\Obl}-{\Adess} two-{\Incl} wheel.{\Nom} {\B}.{\Sg}-break.{\Pfv}-\textbf{{\B}.{\Sg}-{\Intr}}-{\Aor} \\
			\trans `Both wheels of the cart broke.' 
			\hfill (KK002-0554)
		
		
	\end{xlist}
\end{exe}

\begin{exe}
	\ex\label{verbderiv-ex09}
	\begin{xlist}
		
		
			\ex\label{verbderiv-ex09a}
			\gll ču ši-qo t'apaɁ\u{o} kir-e-ⁿ ču=a b-ett-o-s. \\
			in two-three pan.{\Nom} whey-{\Obl}-{\Gen} in={\Emph} {\B}.{\Sg}-pour.{\Ipfv}-{\Npst}-{\Fsg}.{\Erg}\\
			\trans `I pour two or three pans of whey inside.'
			\hfill (EK045-28.2)
		
		
		
			\ex\label{verbderiv-ex09b}
			\gll veⁿ kat-ba-x d-ott-d-\textbf{al}-iⁿ.  \\
			wine.{\Nom} wineskin-{\Obl}.{\Pl}-{\Cont} {\D}-pour.{\Pfv}-{\D}-\textbf{{\Intr}}-{\Aor} \\
			\trans `The wine was poured into wineskins.'
			\hfill (KK015-3051)
		
		
	\end{xlist}
\end{exe}

Some archaic derivation is found with the light verb \textit{d-isar} `stay', see \tabref{verbderiv-table6}, which lists all attested constructions. The umlaut that accompanies the derivation is historical and not the result of a synchronic phonological process.\is{Light verbs}


\begin{table}
\tabcolsep=.75\tabcolsep
\begin{tabular}{llll}
	\lsptoprule
	\multicolumn{2}{l}{Derived verb} & \multicolumn{2}{l}{Derivational base}  \\
	\midrule
	\textit{tuiħ-d-isar} & `go to sleep' & \textit{toħar} & `sleep' (intr) \\
	\textit{qejc'-d-isar} & `grip, clutch' & \textit{qac'ar} & `hang' (stat) \\
	\textit{ejp'q'-d-isar} & `sink, get bogged down' & \textit{ap'q'ar} & `stick, drive into' (tr) \\
	\textit{uill-d-isar} & `touch'; `pester' & \textit{ollar} & `lay, put' (tr) \\
	\textit{uixk'-d-isar} & `touch'; `pester' (pl) & \textit{oxk'ar} & `lay, put' (tr pl) \\
	\textit{ħejč'-d-isar} & `stare' & \textit{ħač'ar} & `look' (exp) \\
	\textit{d-ʕip'-d-isar} & `get stuck, remain' & \textit{d-ʕep'-d-ar} & `lock up' (tr) \\
	\textit{lejtt-d-isar} & `fall (from old age/fatigue) & \textit{lattar} & `stand' (intr) \\
	\lspbottomrule
\end{tabular}
\caption{Unproductive valency pairs}
\label{verbderiv-table6}
\end{table}

\subsection{Transitive} \label{tr}\is{Transitive verbs}
Suffixing \textit{-d-i} to the verbal stem creates two-argument verbs (see \cites[130]{desheriev53}{chrelashvili90}). If added to inherently single-argument verbs, an Ergative argument is added. \tabref{verbderiv-table3} illustrates some of the 70 verb pairs found in \textcite{kadkad84} that consist of a transitive verb derived from an intransitive verb. 


Note that \textit{d-i-} is also a free verb `do, make'. The stem \textit{-i-} is elided in the Present (and all forms historically derived from it) and the Verbal Noun/citation form \textit{d-ar}. The suffix \textit{-d-i} cannot be added to intransitive verbs that have the suffix \textit{-d-al} (see \sectref{intr} above).

\vfill
\begin{table}[H]
	\begin{tabular}{llll}
		\lsptoprule
		\multicolumn{2}{l}{Intransitive} & \multicolumn{2}{l}{Transitive} \\
		\midrule
		\textit{d-elar} & `laugh' &	\textit{d-el-d-ar} &  `make laugh' \\
		\textit{qexk'ar} & `(come to) boil' & \textit{qexk'-d-ar} & `(bring to) boil' \\
		\textit{lepsar} & `be/get dry' & \textit{leps-d-ar} & `dry' \\
		\textit{qerɬar} & `fear' & \textit{qerɬ-d-ar} & `frighten' \\
		\textit{d-erc'ar} & `turn' & \textit{d-erc'-d-ar} & `turn sb/sth' \\
		\textit{d-ebžar} & `fall' & \textit{d-ebž-d-ar} & `cause to fall' \\
		\lspbottomrule
	\end{tabular}
	\caption{Tsova-Tush transitive derivation}
	\label{verbderiv-table3}
\end{table}
\vfill
\pagebreak

Examples (\ref{verbderiv-ex10}) and (\ref{verbderiv-ex11}) show transitive verbs in \textit{d-i-} derived from intransitive verbs \textit{d-ožar}/\textit{d-ebžar} `fall' and \textit{lapsar}/\textit{lepsar} `dry'. The intransitive verbs show a single Nominative subject, whereas with derived transitive verbs, subjects are in the Ergative and objects in Nominative.

\begin{exe}
	\ex\label{verbderiv-ex10}
	\begin{xlist}
		
		
			\ex\label{verbderiv-ex10a}
			\gll t'ariel c'ʕerkoⁿ aħo v-ož-eⁿ soⁿ ħatx. \\
			Tariel.{\Nom} suddenly down {\M}.{\Sg}-fall.{\Pfv}-{\Aor} {\Fsg}.{\Dat} in\_front \\
			\trans `Tariel suddenly fell down in front of me.'
			\hfill (E041-22)
		
		
		
			\ex\label{verbderiv-ex10b}
			\gll dasa-g\u{o} b-ax-en-č\u{o} nax-v d-aqqaⁿ k'ajrcxl-i d-ebž-\textbf{d-i}-eⁿ. \\
			firewood.{\Obl}-{\All} {\M}.{\Pl}-go-{\Ptcp}.{\Pst}-{\Obl} people-{\Erg} {\D}-big.{\Pl} hornbeam-{\Pl}.{\Nom} {\D}-fall.{\Ipfv}-\textbf{{\D}-{\Tr}}-{\Aor} \\
			\trans `The people that went [to look] for firewood felled big hornbeams.' \\
			\hfill (KK005-1239)
		
		
	\end{xlist}
\end{exe}

\begin{exe}
	\ex\label{verbderiv-ex11}
	\begin{xlist}
		
		
			\ex\label{verbderiv-ex11a}
			\gll din\u{\i} čuv daħ laps-iⁿ seⁿ txa-bus. \\
			whole innards.{\Nom} {\Pv} dry.{\Pfv}-{\Aor} {\Fsg}.{\Gen} today-night \\
			\trans `All of the [animal] intestines dried out last night [to my benefit]. \\
			\hfill (EK008-19.1)
		
		
		
			\ex\label{verbderiv-ex11b}
			\gll matxo-v kajrcx\u{\i} ħal laps-\textbf{d-i}-eⁿ. \\
			sun-{\Erg} clothes.{\Nom} {\Pv} dry.{\Pfv}-\textbf{{\D}-{\Tr}}-{\Aor}\\
			\trans `The sun dried the clothes.'
			\hfill (KK036-5563)
		
		
	\end{xlist}
\end{exe}



A small number of verb pairs exist with a transitive verb derived from another transitive verb, all of which are given in \tabref{verbderiv-table7}. Each derived verb has idiosyncratic argument marking and functions range considerably (causative, applicative, intensive), which will not be discussed further here.


\begin{table}
	\begin{tabularx}{\textwidth}{ *2{l>{\hangindent=1em}Q} }
		\lsptoprule
		\multicolumn{2}{l}{Transitive} & \multicolumn{2}{l}{Derived trans.} \\
		\midrule
		\textit{d-ekar} &	`call, summon' &	\textit{d-ek-d-ar} &	`make sb call; plead' \\
		\textit{toxar} &	`hit, strike' &	\textit{tox-d-ar} &	`shake' \\
		\textit{lacar} &	`grab, hold' &	\textit{lac-d-ar} &	`get hold of, arrest' \\
		\textit{lac'ar} &	`hurt' &	\textit{lac'-d-ar} &	`hurt sth' \\
		\textit{ollar} &	`lay, put' &	\textit{oll-d-ar} &	`put on, place on' \\
		\textit{ħaqar} &	`smear, sweep' &	\textit{ħaq-d-ar} &	`smear, knead, rub' \\
		\textit{ħot't'ar} &	`drive, stick, put, force into' &	\textit{ħot't'-d-ar} &	`drive, stick, put, force into' \\
		\textit{maɬar} &	`drink' &	\textit{maɬ-d-ar} &	`make sb drink' \\
		\lspbottomrule
	\end{tabularx}
	\caption{Tsova-Tush transitive derivation from a transitive base}
	\label{verbderiv-table7}
\end{table}




\subsection{Equipollent} \label{equi}\is{Equipollent derivation}\is{Stative verbs}

Some verbs serve as the basis for both transitive and detransitive derivation. All of these verbs, many of which are stative, are given in \tabref{verbderiv-table8}.

\begin{table}
	\small
	\tabcolsep=.8\tabcolsep
	\begin{tabular}{ *3{ l@{~} l} }
		\lsptoprule
		\multicolumn{2}{l}{Intransitive} & \multicolumn{2}{l}{Transitive} & \multicolumn{2}{l}{Source} \\
		\midrule
		\textit{ak'-d-al-ar} & `catch fire' & \textit{ak'-d-ar} & `set afire' & \textit{ak'ar} & `burn' (stat) \\
		\textit{d-arž-d-alar} &	`open up' &	\textit{d-arž-d-ar} &	`open up' &	\textit{d-aržar} &	`be spread' (stat) \\
		\textit{d-apx-d-alar} &	`get undressed' &	\textit{d-apx-d-ar} &	`undress' &	\textit{d-apxar} &	`wear' (stat) \\
		\textit{tag-d-alar} & `be done' &	\textit{tag-d-ar} &	`do' & \textit{tagar} & `suit' (stat) \\
		\textit{tarɬ-d-alar} &	`adapt' & \textit{tarɬ-d-ar} &	`liken' & \textit{tarɬar} &	`be similar to' (stat) \\
		\textit{qac'-d-al-ar} & `hang' & \textit{qac'-d-ar} & `hang (up)' & \textit{qac'ar} & `hang' (stat) \\
		\midrule
		
		\textit{d-oss-d-alar} & `go down' & \textit{d-oss-d-ar} & `bring down' & \textit{d-ossar} & `descend' (intr) \\
		\textit{ħarč-d-alar} & `be wrapped' & \textit{ħarč-d-ar} & `wrap' & \textit{ħarčar} & `embrace' (intr) \\
		\textit{ʕam-d-alar} & `study' & \textit{ʕam-d-ar} & `study sth'& \textit{ʕamar} & `get used to' (intr) \\
		
		\midrule
		
		\textit{d-opx-d-alar} & `get dressed' & \textit{d-opx-d-ar} & `dress' & \textit{d-opxar} & `wear' (tr) \\
		\textit{xarc-d-alar} & `change' & \textit{xarc-d-ar} & `change' & \textit{xarcar} & `exchange' (tr) \\
		\textit{xac'-d-alar} & `be heard' & \textit{xac'-d-ar} & `mention' & \textit{xac'ar} & `hear' (exp) \\
		\textit{qoc'-d-alar} & `hang' & \textit{qoc'-d-ar} & `hang up' & \textit{qoc'ar} & `load' (tr) \\
		\textit{ʕop-d-alar} & `hide' & \textit{ʕop-d-ar} & `cover' & \textit{ʕopar} & `cover' (tr) \\
		
		\lspbottomrule
	\end{tabular}
\caption{Tsova-Tush equipollent derivation from verbs}
\label{verbderiv-table8}
\end{table}







Verbs borrowed from Georgian also make use of equipollent derivation using both derivational suffixes, see \sectref{loanverb}.

\largerpage[-1]\pagebreak


\subsection{Causative} \label{it-caus}\is{Causative}

The suffix \textit{-it} can attach to any verbal stem to form a causative (see \cite{gagua87}). The suffix is completely productive and derives a verb meaning `make sb X, force sb to X, let sb X, give possibility to X'. The suffix is historically related to the verb \textit{d-itar} `let, leave', but this verb is not simply added to the verbal stem, as is e.g. \textit{d-i-} `do', as explained above in \sectref{tr}. The gender marker of the original verb \textit{d-itar} is not kept in the suffix, and the suffix can be added to derived verbs with the suffixes \textit{d-i-} or (albeit rarely) \textit{d-al-}.\footnote{One can hypothesise that either (1) the verb \textit{d-itar} grammaticalised differently and to a greater extent than the other derivational suffixes;  (2) the suffix \textit{-it} is in fact not derived from \textit{d-itar} `leave, let', but the verb \textit{d-itar} is underlyingly \textit{d-i-it-ar} `\textsc{d}-do-\textsc{caus-vn}' and is itself a derived verb; or (3) the underived verb \textit{d-itar} `let' was first suffixed to underived verbs only, which gave rise to the opposition ``stem-\textit{d-∅-ar} (derived transitive verb)'' vs. ``stem-\textit{d-it-ar} (derived causative verb)''. Verb-\textit{d-it-ar} was subsequently re-analysed as verb-\textit{d-∅-it-ar}, isolating the morpheme \textit{-it} which could then be applied directly to underived verbs too.}
Several examples are given in \tabref{verbderiv-table5}.

\begin{table}
	\begin{tabular}{llll}
		\lsptoprule
		Base verb & & Causative & \\
        \midrule
		\textit{tit'ar} (tr) & `cut' & \textit{tit'itar} & `make/let sb cut' \\
		\textit{d-eq'ar} (tr) & `divide' & \textit{d-eq'itar} & `make/let sb divide' \\
		\textit{xit'-d-ar} (tr) & `break sth' & \textit{xit'-d-itar} & `make/let sb break sth' \\
		\textit{ʕam-d-ar} (tr) & `learn' & \textit{ʕam-d-itar} & `teach' \\
		\textit{d-aɬar} (intr) & `go out' & \textit{d-aɬitar} & `release' \\
		\textit{xiɬar} (intr) & `be, become' & \textit{xiɬitar} & `make/let sth be' \\
		\textit{d-eħ-d-alar} (intr) & `sneak' & \textit{d-eħ-d-alitar} & `make/let sb sneak' \\
		\lspbottomrule
	\end{tabular}
	\caption{Tsova-Tush causative derivation}
	\label{verbderiv-table5}
\end{table}





By adding the suffix \textit{-it} to an intransitive verb, the original subject of the intransitive verb becomes an object, and a new agent subject in the Ergative is added (see Example \ref{verbderiv-ex13}). 


\begin{exe}
	\ex\label{verbderiv-ex13}
	\begin{xlist}
		
		
			\ex\label{verbderiv-ex13a}
			\gll bʕa nax ix-\u{o} ʕurdeⁿ latin\u{o} psarlo-mciⁿ. \\
			alltogether people.{\Nom} go.{\Ipfv}-{\Npst} morning starting evening-{\Term} \\
			\trans `People are always going around from morning to night.'
			\hfill (EK023-2.6)
		
			\ex\label{verbderiv-ex13b}
			\gll macme nʕeiɁ ix-\textbf{it}-o-tx že, čuxu-i=a nʕeiɁ xec-o-tx. \\
			when.{\Rel} out go.{\Ipfv}-\textbf{{\Caus}}-{\Npst}-{\Fpl}.{\Erg} sheep.{\Nom} lamb-{\Pl}={\Add} out release-{\Npst}-{\Fpl}.{\Erg}\\
			\trans `When we let the sheep go out, we release the lambs too.'
			\hfill (E043-141)
		
		
	\end{xlist}
\end{exe}

%pproofreadingstop

This derivational pattern can be described as canonically causative according to criteria set up by \textcite[240]{dixon2012blt3}. At this point, the exact semantic distinctions between the causative in \textit{-it} and the transitivising suffix \textit{-d-i} (which is syntactically exactly parallel and can also have causative semantics) have not been studied, but it seems clear that \textit{-it} is extremely productive, and \textit{-d-i} occurs on a limited number of morphological bases.

When the suffix \textit{-it} is added to a transitive verb, the original Ergative subject becomes the causee, which in Tsova-Tush is expressed with the Allative case. Morover, a causer in the Ergative case is added (Example (\ref{verbderiv-ex12})).

\begin{exe}
	\ex\label{verbderiv-ex12}
	\begin{xlist}
		
		
			\ex\label{verbderiv-ex12a}
			\gll \u{g}azi-š gag-j-∅-o-s ħo. \\
			good-{\Adv} look\_after-{\F}.{\Sg}-{\Tr}-{\Npst}-{\Fsg}.{\Erg} {\Ssg}.{\Nom}\\
			\trans `I will take good care of you.'
			\hfill (EK056-2.36)
		
		
		
			\ex\label{verbderiv-ex12b}
			\gll doⁿ naq'bist'-e-g\u{o} gag-b-∅-\textbf{it}-o-s.  \\
			horse.{\Nom} friend-{\Obl}-{\All} look\_after-{\B}.{\Sg}-{\Tr}-\textbf{{\Caus}}-{\Npst}-{\Fsg}.{\Erg} \\
			\trans `I let a friend look after [my] horse.'
			\hfill (KK003-0662)
		
		
		
		
	\end{xlist}
\end{exe}

\subsection{Other valency alternations}

Only two labile verbs have been found: \textit{axk'ar} `be tied up' (stative), `tie up' (tr) and \textit{d-ivar} `sow, plant', `be sown, planted'. Two known verb pairs show a valency relation by suppletion: \textit{d-alar} `die', \textit{ʕavar} `kill' and \textit{qallar} `eat', \textit{teɬar} `feed' (also `give' (\textsc{ipfv})).


Two verbs are derived from the experiential verb \textit{d-agar/guar} `see': \textit{gu-d-aqar} `show' and \textit{gu-d-aɬar} `appear', which are derived using the verbs \textit{d-aqar} `take' and \textit{d-aɬar} `go out'. Various complex verbs consisting of a (synchronically) non-verbal root plus a light verb (often a verb of motion) are discussed in \sectref{lightverbs}.



\subsection{Modal derivation} \label{potential}\is{Potential}\is{Modality}\largerpage

The suffix \textit{-mak'} is added to any verbal stem to derive verbs with the meaning `be able to', see Example (\ref{verbderiv-ex14}). Derived verbs of this type are labelled Potential\footnote{Not to be confused with the term “Potentialis” from classical languages, denoting a modal form that indicates possibility\slash probability of an event.} (compare e.g. \cite{comrieetal2015valencybezhta}), and demand subjects in the Dative case. 

\begin{exe}
	\ex\label{verbderiv-ex14}
	\begin{xlist}
		
		
			\ex\label{verbderiv-ex14a}
			\gll txoⁿ psarluj-n xi co d-oɁ-\textbf{mak'}-in=e. \\
			{\Fpl}.{\Dat} evening-{\Dat} water {\Neg} {\D}-bring-\textbf{{\Pot}}-{\Aor}=and \\
			\trans `We couldn't bring water tonight.'
			\hfill (E153-63)
		
		
		
			\ex\label{verbderiv-ex14b}
			\gll mak qaxk'-uš do-i-n o lav-e-lo\u{g}=da daħ co d-et'-\textbf{mak'}-\u{e}, ču ploba-l-a.  \\
			on\_top hang.{\Pl}-{\Simul} horse.{\Obl}-{\Pl}-{\Dat} {\Dist} snow-{\Obl}-{\Intertrans}=through  {\Pv} {\Neg} {\D}-run-\textbf{{\Pot}}-{\Npst} {\Pv} sink-{\Intr}-{\Npst} \\
			\trans `When they are loaded, the horses cannot run through that snow, they sink.'
			\hfill (EK005-15.1)
		
		
		
		
		
			\ex\label{verbderiv-ex14c}
			\gll oqui-n magram ħatteɁ d-abc'-\textbf{mak'}-iⁿ, me ešm-i lev-d-∅-or. \\
			{\Dist}-{\Dat} but immediately {\D}-recognise-\textbf{{\Pot}}-{\Aor} {\Subord} devil-{\Pl} be.{\Nw}-{\D}-{\Tr}-{\Imprf} \\
			\trans `He could see immediately that they must be devils.'
			\hfill (\cite{kojima09})
		
		
	\end{xlist}
\end{exe}

Morover, \textit{mak'ar} is also a freestanding verb meaning `be able'.


	\begin{exe}
		\ex\label{verbderiv-ex15}
		\gll so \textbf{mak'}-eš v-a, soⁿ wumaɁ \textbf{mak'}-\u{e} : že-x v-ac'-ar, že d-ett-ar, že lerk'-ar. \\
		{\Fsg}.{\Nom} \textbf{can}-{\Simul} {\M}.{\Sg}-be {\Fsg}.{\Dat} all \textbf{can}-{\Npst} {} sheep-{\Cont} {\M}.{\Sg}-follow.{\Ipfv}-{\Vn} sheep {\D}-milk-{\Vn} sheep shear.{\Ipfv}-{\Vn}\\
		\trans `I am able! I can [do it] all: herding sheep, milking sheep, shearing sheep. \\
		\hfill (KK013-2505)
	\end{exe}





\section{Morphosyntax of borrowed verbs} \label{loanverb}\is{Loanword adaptation!Verbs}



Even though Tsova-Tush does not need overt morphology to signal transitive or intransitive verbs (see \sectref{valency}), it obligatorily uses the intransitive \textit{-d-al} and transitive \textit{-d-i} suffixes to incorporate Georgian borrowed verbs (\cites{harris08nativize}). The dictionary of \textcite[]{kadkad84} features approximately 300 native Tsova-Tush verb roots, 1,500 derived roots from native elements (using all of the derivation and compounding methods found in this chapter, as well as with preverbs), and approximately 500 borrowed verbs. Additionally, since all Tsova-Tush speakers are fluent in Georgian, they are able to insert single verbs as code-switches (see \sectref{theory} for the distinction between code-switching and borrowing, and see \textcite[]{ritchiebhatia99cs} and \textcite[]{bandiraodendikken14cs} for similar code-switching constructions).\footnote{The mechanism used for code-switching is predestined to be used productively and thus lead to the conventionalisation of forms that are accommodated in such a way. The distinction between code-switching and loan words lies in the conventionalisation on a community level, something that has not been systematically investigated in this work.} In Tsova-Tush, these single-word insertions morphosyntactically behave identically to borrowing. See \tabref{table-verbborrow1} for several examples.

\begin{table}
	\begin{tabular}{lll}
		\lsptoprule
		Verb & & from Georgian \\
        \midrule
		\textit{ak'lebad-d-ar}	& `devastate, tear down' & \textit{ak'leba} \\
		\textit{garidbad-d-alar}	& `move away, leave' &	\textit{garideba} \\ 
		\textit{dabzarod-d-alar}	& `burst, crack' &	\textit{dabzarva} \\
		\textit{toxnad-d-ar}	& `hoe' &	\textit{toxna} \\
		\textit{nelbad-d-ar}	& `digest, stomach' &	\textit{neleba} \\ 
		\textit{pxek'ad-d-ar}	& `scrape, plane, grate' &	\textit{pxek'a} \\ 
		\textit{ǯ\u{g}abnad-d-ar}	& `scrawl, scribble' &	\textit{ǯ\u{g}abna} \\
		\lspbottomrule
	\end{tabular}
	\caption{Georgian verbal borrowings}
	\label{table-verbborrow1}
\end{table}

\subsection{Basic pattern} \label{verbloanbasic}

As can be seen from \tabref{table-verbborrow1}, verbal borrowings consist of two parts: a Georgian part ending in \textit{-d} and one of two Tsova-Tush valency suffixes \textit{-d-i} (Verbal Noun \textit{d-ar}, see \sectref{tr}) or \textit{-d-al}. The Georgian part will be discussed first.\is{Verbal Noun!Georgian}

In Georgian, Verbal Nouns are used as action nouns and as non-finite complements to several modal, phasal, and other auxiliary verbs. Most Verbal Nouns consist of several morphs: a preverb, a verb stem, a thematic marker (\textsc{tm}), and a Verbal Noun suffix \textit{-a}, see \tabref{table-georgianvn}. Some verbs do not require a preverb, and some verbs do not require a thematic marker. 

\begin{table}
	\begin{tabular}{lllll}
    \lsptoprule
		Preverb & {Stem} & {-\textsc{tm}} & {\textsc{-vn}} & \\
        \midrule
		\textit{a-}	& \textit{k'l}  & \textit{-eb} & \textit{-a} & \textit{ak'leba} `devastate, tear down' ({\Pfv}) \\
		\textit{ga-}	& \textit{rid}  & \textit{-eb} & \textit{-a} &	\textit{garideba} `move away, leave' ({\Pfv})\\ 
		\textit{da-}	& \textit{bzar} & \textit{-v} & \textit{-a} &	\textit{dabzarva}  `burst, crack' ({\Pfv})\\
		& \textit{toxn} & & \textit{-a} &	\textit{toxna} `hoe'  ({\Ipfv})\\
		& \textit{nel} & \textit{-eb} & \textit{-a} &	\textit{neleba} `digest, stomach'  ({\Ipfv})\\ 
		& \textit{mq'n} & \textit{-ob} & \textit{-a} & \textit{mq'noba} `graft, bud, implant' ({\Ipfv})\\
		\textit{da-} & \textit{rb} & \textit{-ev} & \textit{-a} & \textit{darbeva}	`ravage' ({\Pfv}) \\
		& \textit{ǯ\u{g}abn}	& & \textit{-a} &	\textit{ǯ\u{g}abna} `scrawl, scribble'  ({\Ipfv})\\
        \lspbottomrule
	\end{tabular}
	\caption{Georgian Verbal Nouns}
	\label{table-georgianvn}
\end{table}

As can be seen from \tabref{table-georgianvn}, the presence of a Georgian preverb marks perfectivity. See Example (\ref{verbderiv-ex25}), where in (\ref{verbderiv-ex25a}) \textit{bzarva} `crack' is an imperfective verb denoting a prolonged event, whereas in (\ref{verbderiv-ex25b}), \textit{dabzarva} `crack' denotes an immediate event. Sometimes, the presence or absence of a preverb marks an (added) distinction in lexical semantics: \textit{msaxur-eb-a} `service, employ', \textit{da-msaxur-eb-a} `earn, achieve'; \textit{t'ar-eb-a} `carry sth, bear; let sth through, pass', \textit{ga-t'ar-eb-a} `let sth through, pass'. The presence or absence of a Thematic Marker is determined lexically (see e.g. \textcite{hewitt95} for an overview).

\begin{exe}
	\ex\label{verbderiv-ex25}
	Modern Georgian 
	\begin{xlist}
		
		
			\ex\label{verbderiv-ex25a}
			\gll  da mxolod p'irvel-i ic'q'ebs \textbf{bzar-v-a-s}. \\
			and only first-{\Nom} s/he\_started \textbf{crack-{\Tm}-{\Vn}-{\Dat}} \\
			\trans `And only the first one started to crack.'
			\hfill (GNC: I. Brodski)
		
			\ex\label{verbderiv-ex25b}
			\gll  ``dalas-s'' ar uč'irs met'ok-is k'edl-is \textbf{da-bzar-v-a}. \\
			Dallas-{\Dat} {\Neg} s/he\_has\_problem opponent-{\Gen} wall-{\Gen} \textbf{{\Pv}-crack-{\Tm}-{\Vn}}\\
			\trans `Dallas has no problem in cracking the opponent's wall.'
			\hfill (GRC)
		
		
	\end{xlist}
\end{exe}

As can be seen from Example (\ref{verbderiv-ex25a}), Verbal Nouns inflect like regular nouns and thus bear case marking. In Old Georgian, Verbal Nouns very often appear in the Adverbial case indicating an aim or purpose, approximately like an infinitive in many Indo-European languages (\cites{gippertOGeo,kobaidzevamling}). See Example (\ref{verbderiv-ex26}), with Verbal Nouns indicating an adjunct purpose clause in (\ref{verbderiv-ex26a}) and a complement clause to a phasal verb in (\ref{verbderiv-ex26b}).\is{Adverbial case}\is{Infinitive}\is{Purpose clauses}



\begin{exe}
	\ex\label{verbderiv-ex26}
	Old Georgian
	\begin{xlist}
		
		
			\ex\label{verbderiv-ex26a}
			\gll  uzaxebdes xuro-j igi mč’edel-sa mas \textbf{č’ed-ad}, \textbf{k'wer-v-ad}, uro-jsa \textbf{cem-ad}, \textbf{da-mšč’wal-v-ad}, \textbf{a\u{g}-mart-eb-ad}, \textbf{da-dgm-ad}, da \textbf{da-mt’k’ic-eb-ad}, rajta ara šeirq’ios. \\
			s/he\_should\_call architect-{\Nom} {\Dist} smith-{\Dat} {\Def}.{\Dat} \textbf{forge-{\Vn}.{\Advb}} \textbf{pound-{\Tm}-{\Vn}.{\Advb}} hammer-{\Gen} \textbf{give-{\Vn}.{\Advb}} \textbf{{\Pv}-nail-{\Tm}-{\Vn}.{\Advb}} \textbf{{\Pv}-erect-{\Tm}-{\Vn}.{\Advb}} \textbf{{\Pv}-mount-{\Vn}.{\Advb}} and \textbf{{\Pv}-fix-{\Tm}-{\Vn}.{\Advb}} so\_that {\Neg} s/he\_would\_move \\
			\trans `The architect should call the smith to forge, pound, hammer, nail, erect, mount, and fix it so that it will not be moved.’  \\
			\hfill (Is. 41.7, from \cites[]{gippertOGeo})
		
		
		
			\ex\label{verbderiv-ex26b}
			\gll  vic'q'o \textbf{txr-ob-ad} tkwen-da, saq'warel-no, k'ar...\\
			I\_will\_begin \textbf{tell-{\Tm}-{\Vn}.{\Advb}} {\Spl}-{\Advb} beloved-{\Pl}.{\Voc} door \\
			\trans `I will begin to tell you all, my beloved, of the Door.' \\
			\hfill (Ioane Sabanisdze, `Life of St. Abo', dated at 786)
		
		
	\end{xlist}
\end{exe}

In Old Georgian\il{Old Georgian}, Verbal Nouns in the Adverbial case were frequently combined with the copula or verbs like \textit{-c-} ‘give’ to denote a possibility, necessity or causation. Examples include \textit{da-dgm-ad iq’o} ({\Pv}-put-{\Vn}.{\Advb} it\_was)  `it (crown) was to be put on'; \textit{da-dgm-ad sca} ({\Pv}-put-{\Vn}.{\Advb} he\_gave) `he caused them to mount it (coffin)' (\cites{gippertOGeo}). These light verb-like constructions were also used to derive verbs from adjectival and nominal stems, such as \textit{uzrunvel-q’opa} `warrant’ (`carefree-do’), \textit{tavisupal-kmna} `liberate’ (`free-make’), \textit{natl-is-cema} `baptise’ (`light-{\Gen}-give’), \textit{natl-is-\u{g}eba} `be baptised’ (`light-{\Gen}-take’). As noted by \textcite{gippertOGeo}, these constructions were, at least in the earliest period, not completely univerbised, with both parts inflecting independently in accordance with their syntactic status. Unlike Tsova-Tush, Old Georgian did not form complex verbs of the template verb-{\Vn}-{\Advb}+`do'. This type of light verb construction is also common in Tsova-Tush, but it has to be noted that most Caucasian languages (of multiple families) make use of light verbs and light verb-like constructions (\cites{nichols03bipartite}; for a case study of Tsez, see \cites[]{comrie2000tsezval}). For an Ingush parallel to Tsova-Tush verb incorporation, see \textit{diktovat' du} `dictate', from Russian \textit{diktovat'} `id.' (\cites[337]{nichols11}).

Constructions with a Verbal Noun in the Adverbial case, as seen in Example (\ref{verbderiv-ex26}), are frequently attested in Old Georgian, but are absent in Standard Modern Georgian. They are found occasionally in Tush Georgian, see the first verb in Example (\ref{verbderiv-ex29}). Modern Georgian uses a Future Participle in these constructions, which is also the default strategy in Tush Georgian, see the second verb in Example (\ref{verbderiv-ex29}).\il{Tush Georgian}

	\begin{exe}
		\ex\label{verbderiv-ex29}
		Tush Georgian
        
		\gll  mo-di, šen c'a-di q'oran-o, \textbf{c'a-ğ-eb-a-d} ambav-isa=o, gada-di buxurta-šia \textbf{sa-tkm-el-ad} samʒimr-isa=o. \\
		hither-go.{\Imp} {\Ssg} away-go.{\Imp} raven-{\Voc} \textbf{away-take-{\Tm}-{\Vn}-{\Advb}} story-{\Gen}={\Quot} across-go.{\Imp} Bukhurta-{\In} \textbf{{\Ptcp}.{\Fut}-say-{\Ptcp}.{\Fut}-{\Advb}} condolences-{\Gen}={\Quot}\\
		\trans `Alright, go forth, raven, to take away this story. Go over to Bukhurta\footnote{A village in the Gometsari valley in Tusheti.} to say [my] condolences.'
		\hfill (GA021-1.40)
	\end{exe}


It is clearly this form, i.e. the Georgian Verbal Noun in the Adverbial case, that constituted the input in Tsova-Tush borrowed verb constructions. This can be compared directly to other languages that borrow a verbal noun form (also called masdar), as summarised by \textcite[84]{wohlgemut09loanverbtyp}. See for instance Example (\ref{verbderiv-ex30}) from\il{Bezhta} Bezhta (\cites[101]{khalilov04contact}), where we find a similar construction with a Georgian Verbal Noun (from Georgian \textit{ga-carc-v-a/ga-ʒarc-v-a} `rob') and a light verb of the borrowing language.


	\begin{exe}
		\ex\label{verbderiv-ex30}
		Bezhta
        
		\gll  gacarsa b-ow-al \\
		defeudalise {\B}.{\Sg}-do-{\Inf} \\
		\trans `defeudalise'
		\hfill (\cites[101]{khalilov04contact}, as cited in \cites[85]{wohlgemut09loanverbtyp})
	\end{exe}


However, the Tsova-Tush construction shows a clear difference. In languages that use a verbal noun as the input form for the verbal borrowing, that input form often needs to be more `noun-like' to allow borrowing in the first place. However, the Georgian inflected Verbal Noun is morphosyntactically in fact not noun-like, but is more `adverb-like', as it is used in not just complement clauses, but mainly adjunct clauses such as in Example (\ref{verbderiv-ex26a}). This makes compatibility with Wohlgemuth's typology of input forms less straightforward. 

In addition, the Georgian Verbal Noun in the Adverbial case is functionally equivalent to an infinitive in many European languages (\cites{gippertOGeo}), and cross-linguistically, infinitives are often used as the input form for borrowed verb constructions (\cites[80]{wohlgemut09loanverbtyp}), see again Ingush \textit{diktovat' du} `dictate', from Russian \textit{diktovat'} `id.' (\cites[337]{nichols11}).  However, in these examples, the infinitive is a citation form in the donor language, which is not the case for the Georgian Verbal Noun in the Adverbial case, which is much more marginal than the citation form in the Nominative case.

As explained in \sectref{rootperf}, if a Georgian Verbal Noun frequently occurs both with and without a preverb, Tsova-Tush borrows both forms, in order to make a distinction in perfectivity (e.g. to be able to distinguish the Present and the Future, see \sectref{ind}). Note that the Tsova-Tush part of the verbal complex (i.e., the suffix \textit{-d-i} or \textit{-d-al}) is not marked for perfectivity.\is{Aspect}

\begin{table}
	\begin{tabular}{lll}
    \lsptoprule
		Perfective & Imperfective & \\
        \midrule
		\textit{agrilbad-d-ar} & \textit{grilbad-d-ar} & `cool sth' \\
		\textit{dat'anǯod-d-ar} &  \textit{t'anǯod-d-ar} & `torture' \\
		\textit{gamarglod-d-ar} & \textit{marglod-d-ar} & `weed, hoe' \\
		\textit{šerisxod-d-ar} & \textit{risxod-d-ar} & `invoke wrath on' \\
		\textit{moc'amlod-d-ar} &  \textit{c'amlod-d-ar} &  `apply poison/pesticide on' \\
        \lspbottomrule
	\end{tabular}
	\caption{Aspect pairs of verbal loans from Georgian}
	\label{table-georgianborrowaspect}
\end{table}


One problem arises when combining (1) the above analysis of the Georgian Verbal Noun with the Adverbial case ending as the input form with (2) the perfectivity distinction expressed by preverbs. The construction with the Georgian Verbal Noun in the Adverbial case is most clearly observed in Old Georgian; however, Old Georgian did not yet express perfectivity by way of preverbs (\cite{gippertOGeo}). Hence, the two aspects of the loanword strategy (preverbs and Adverbial case) cannot have been established simultaneously in contact with Old Georgian. Two possible solutions can be imagined, although it seems hard to prove any of them at this point. 

\begin{enumerate}
	\item Both aspects of the loanword strategy were established simultaneously in contact with a Georgian variety that featured both these aspects. As seen in Example (\ref{verbderiv-ex29}), Tush Georgian preserves, albeit marginally, the Verbal Noun in Adverbial case construction, and is also employing preverbs to distinguish perfective from imperfective verb forms. Middle Georgian, too, features both aspects, as can be seen in Example (\ref{verbderiv-ex32}).
	
	\is{Middle Georgian}
	
		\begin{exe}
			\ex\label{verbderiv-ex32}
			Middle Georgian
            
			\gll   vubrʒane or-ta=ve amat \textbf{gan-mzad-eb-ad} čem-twis samgzavro-jsa nuzl-isa. \\
			I\_ordered\_it two-{\Obl}.{\Pl}={\Emph} {\Prox}.{\Obl}.{\Pl} \textbf{{\Pv}-prepare-{\Tm}-{\Vn}.{\Advb}} {\Fsg}-{\Ben} travel-{\Gen} provisions-{\Gen}	\\
			\trans `I ordered them both to prepare travel provisions for me.' \\
			\hfill (G. Avalishvili, Mgzavroba 1.1.18.11 (dated 1820))
		\end{exe}
	
	
	\item  Alternatively, the two aspects of the loanword strategy were established at different times. First, the Georgian Verbal Noun in Adverbial case was established as the input form for borrowed verbs into Tsova-Tush. Potentially, this could have occurred during the Old Georgian stage. This form was then established as the loan verb adaptation strategy in Tsova-Tush, even when Old (or Middle) Georgian eventually lost this construction. A very close parallel are verbal forms containing the suffix \textit{-miš} in Turkic verbal borrowings in Iranian languages (\cites[]{ido06}, as cited in \cites[112]{wohlgemut09loanverbtyp}). According to \citeauthor{ido06}, the \textit{-miš} form is not productive anymore at least in some of the modern varieties of the donor languages (Uzbek, Uyghur), so verbs from these languages must either have been borrowed centuries ago, or the suffix itself got borrowed and became an integrated and productive part of a separate loan verb accommodation pattern in some of the recipient languages (like Tajik, Sarikoli) (\cites[112]{wohlgemut09loanverbtyp}). Since in Tsova-Tush, verbal borrowing is ongoing due to frequent code-switching, the second of Ido's explanations must be true for Tsova-Tush: the morph \textit{-ad}, no longer productive in Modern Standard Georgian, has become a productive part of a specialised loan verb accommodation strategy. The part of the borrowing strategy that involves perfectivising preverbs was then established independently at a later stage.
	
\end{enumerate}

It is difficult to say whether the two aspects of the borrowing strategy were established simultaneously or independently, but at any rate, the morph \textit{-ad}, only found in Old Georgian, Middle Georgian and (marginally) Tush Georgian, has become a productive part of a specialised loan verb accommodation strategy, as evidenced by the fact that even verbal borrowings from Standard Modern Georgian, which does not feature the \textit{-ad} construction itself, receive the suffix.

Georgian Verbal Nouns do not mark diathesis, e.g. \textit{sargebloba} `take advantage of sth; advantage sth'; \textit{k'valipicireba} `qualify sth; be qualified'; \textit{k'nineba} `turn tiny; diminish sth'. Due to this fact, combined with the relatively rigid distinction in Tsova-Tush between transitive and intransitive verbs (see \sectref{valency}), Tsova-Tush needs to make this distinction in the suffixal part of the verbal complex. Hence, for the the majority of verbs (except for Georgian Class 3 verbs, see \sectref{loanverbmed} below), Tsova-Tush is able to derive both a transitive and an intransive verb from the same Georgian base, see \tabref{table-georgianborrowtrans}.\is{Transitive verbs}\is{Intransitive verbs}

\begin{table}
	\begin{tabular}{llll}
    \lsptoprule
		Transitive & & Intransitive & \\
        \midrule
		\textit{agrilbad-d-ar} & `cool sth' & \textit{agrilbad-d-alar} & `cool down' \\
		\textit{garidbad-d-ar} & `remove' & \textit{garidbad-d-alar} & `leave' \\
		\textit{datxvenad-d-ar} & `startle' & \textit{datxvenad-d-alar} & `be startled' \\
		\textit{k'virbad-d-ar} & `surprise' & \textit{k'virbad-d-alar} & `be surprised' \\
		\textit{ridbad-d-ar} &`keep sth away from' &
		\textit{ridbad-d-alar} & `avoid' \\
		\lspbottomrule
	\end{tabular}
	\caption{Transitivity pairs of verbal loans from Georgian}
	\label{table-georgianborrowtrans}
\end{table}

As with all verbs with a suffix \textit{-d-i} or \textit{-d-al}, the gender marker cross-references the object of a transitive clause or the subject of an intransitive clause (most often in the Nominative case), see Example (\ref{verbderiv-ex27}). If the gender marker is \textit{d-}, the final \textit{-d} of the Georgian Adverbial ending \textit{-ad} is dropped, as in (\ref{verbderiv-ex27b}).

\begin{exe}
	\ex\label{verbderiv-ex27}
	\begin{xlist}
		
		
			\ex\label{verbderiv-ex27a}
			\gll isi daħ=a damtavrbad-\textbf{b}-al-iⁿ is \textbf{pal}. \\
			there.{\Med}.{\Ess} {\Pv}={\Add} finish.{\Pfv}-\textbf{{\B}.{\Sg}}-{\Intr}-{\Aor} {\Med} \textbf{tale({\B})} \\
			\trans `That fairy tale ended there.'
			\hfill  (E155-8)
		
		
		
			\ex\label{verbderiv-ex27b}
			\gll \textbf{inst'it'ut'} damtavrba-\textbf{d}-i-n-as daħ. \\
			\textbf{institute({\D})} finish.{\Pfv}-\textbf{{\D}}-{\Tr}-{\Aor}-{\Fsg}.{\Erg} {\Pv} \\
			\trans `I graduated from the institute.'
			\hfill (E045-31)
		
		
	\end{xlist}
\end{exe}

When trying to situate the Tsova-Tush loan verb adaptation strategy in a cross-linguistic perspective, some intricacies arise related to the analysis and definition of the light verbs/suffixes \textit{d-i} and \textit{d-al}, as already mentioned in \sectref{valintro}. Superficially, they bear a strong resemblance to the light verb compounds of the type \textit{bʕar-d-ar} `meet', \textit{dak'-lavar} `think' described in \sectref{lightverbsp}. In fact, the light verb construction as a strategy for loan verb adaptation (see \cite[102]{wohlgemut09loanverbtyp}) is widespread in the Caucasus, see the examples from Ingush and Bezhta (Example (\ref{verbderiv-ex30})) above. Nevertheless, although the Tsova-Tush suffixes \textit{-d-al} and \textit{-d-i} must have undoubtedly grammaticalised from the same type of light verb construction, they are synchronically derivational suffixes. This analysis, as mentioned above in \sectref{didal}, is supported by the following two facts:

\begin{enumerate}
\item An independent verb \textit{d-alar} does not exist. The independent existence of a light verb is a defining characteristic of light verbs, at least according to \textcite[106]{wohlgemut09loanverbtyp}. The suffix \textit{-d-i} is formally identical to the light verb \textit{d-i} (Verbal Noun \textit{d-ar}) that also exists as an independent verb `do'. However, since the suffixes \textit{-d-i} and \textit{-d-al} derive pairs of verbs (see e.g. \sectref{deadjverb}), I assume the same level of grammaticalisation for both suffixes.

\item Light verb compounds are unproductive, whereas verb pairs containing the suffixes \textit{-d-i, -d-al} were to some extent productive, and can even be derived from light verb compounds (e.g. \textit{gontas-d-alar} `come to one's senses', from \textit{gon(-)tasar} `bring to one's  senses', from \textit{gon} `intellect' + \textit{tas-} `drop, leave, throw').
\end{enumerate}

Therefore, rather than as a light verb construction, the Tsova-Tush strategy for adapting loan verbs can be classified as \citeauthor{wohlgemut09loanverbtyp}'s Indirect Insertion (\citeyear[94]{wohlgemut09loanverbtyp}); borrowed verbs require overt affixation of some kind. Wohlgemuth distinguishes between constructions that use a verbalising suffix for loan verb adaptation and a causative suffix. In Tsova-Tush, the suffixes \textit{-d-i} and \textit{-d-al} are verbalising suffixes (they derive verbs from adjectives, see \sectref{deadjverb}), as well as causative suffixes (at least, the suffix \textit{-d-i} is), since both suffixes are used in valency derivation (see \sectref{verbderivproper}). Hence, in terms of adaptation strategy, Wohlgemuth's distinction is not straightforwardly applicable to Tsova-Tush.



\subsection{Georgian Class 3 verbs} \label{loanverbmed}\is{Medial verbs}

As mentioned in \sectref{intr-erg-geo}, Georgian features a class of active intransitive verbs, called Class 3 verbs or medial verbs (\cites{holisky1981medial}). They are characterised (1) morphologically, by forming their perfective stem without a preverb (\cites{harris1981syntax}); (2) semantically, by containing mostly atelic verbs (\cites{gerardin2022valencegeo})
; (3) syntactically, by demanding an Ergative subject in the 2nd TAM Series (primarily in the Aorist). This class is opposed to Class 2 verbs, which are also intransitive, but are usually telic, have a Nominative subject in the Aorist, and show an intransitive inflectional pattern.

It is important vis-à-vis the grammar of Tsova-Tush that, in short, Georgian medial verbs are morphologically similar to Georgian transitive verbs, but only occur with a single argument, which is marked by the Ergative in the Aorist. Some of these medial verbs are borrowed into Tsova-Tush, see \tabref{table-georgianborrowmed}. Note that there is often not an exact formal match between the Tsova-Tush verb and the corresponding Georgian Verbal Noun. This is due to the fact that many verbs were not borrowed from Standard Georgian but from Northeastern Georgian dialects (mostly Tush Georgian), which can have different Thematic Markers to the same verb stems (\cite{uturgaidze60,kartulidialekt}).


\begin{table}
	\begin{tabular}{lll}
    \lsptoprule
		Tsova-Tush & English & from Georgian \\
        \midrule
		\textit{bardod-d-ar} & `snow heavily' & \textit{bardna}  \\
		\textit{nejdrebad-d-ar} & `hunt' & \textit{nadiroba} \\
		\textit{xnešad-d-ar} & `sigh' & \textit{xvneša} \\
		\textit{mušebad-d-ar} & `work' & \textit{mušaoba} \\
		\textit{gamarǯbad-d-ar} & `be victorious' & \textit{gamarǯveba} \\
    \lspbottomrule
	\end{tabular}
	\caption{Borrowed Georgian Class 3 verbs}
	\label{table-georgianborrowmed}
\end{table}

As briefly seen in Example (\ref{verbderiv-ex02c}) and mentioned in \sectref{intrerg}, these verbs are incorporated differently to the basic transitive and intransitive verbs described in \sectref{verbloanbasic}. First of all, even though they are monovalent verbs, they are incorporated with the transitivising suffix \textit{-d-i} (Verbal Noun \textit{-d-ar}), as can be seen in \tabref{table-georgianborrowmed}.

Secondly, congruent with their transitive morphology, these verbs demand a subject in the Ergative case in all persons, see Example (\ref{verbderiv-ex28}). 

\begin{exe}
\ex\label{verbderiv-ex28}
\begin{xlist}
		\ex\label{verbderiv-ex28a}
		\gll gamarǯba-d-i-r-\textbf{as}=ajn\u{o}! \\
		be\_victorious-{\D}-{\Tr}-{\Rem}-\textbf{{\Fsg}.{\Erg}}={\Quot} \\
		\trans `I had been victorious, he said.'
		\hfill (MM404-1.74)
	
		\ex\label{verbderiv-ex28b}
		\gll zoovet'inst'it'-e mušeba-d-∅-o \textbf{mar-v}=aj \textbf{pst'uin-v}=aj. \\
		veterinary\_institute-{\Obl}({\Ess}) work-{\D}-{\Tr}-{\Npst} \textbf{husband-{\Erg}}={\Add} \textbf{woman-{\Erg}}={\Add} \\
		\trans `Both husband and wife work at the veterinary institute.' 
		\hfill (E116-55)
\end{xlist}
\end{exe}

Thirdly, as can be seen from the examples under (\ref{verbderiv-ex28}), the gender marker is always \textit{d-}, corresponding with the default gender D (\sectref{verbalgender}). This gender marker is a necessary part of the transitive light verb \textit{d-ar}, the verb that is the source of the transitivising suffix \textit{-d-i}. Hence, a gender marker is obligatory to incorporate these verbs from Georgian, but in fact it does not cross-reference an argument in the clause, since these verbs are monovalent.\is{Cross-referencing!Gender}

It has to be emphasised that this class of borrowed Georgian verbs does not mirror any native Tsova-Tush verb class. In other words, the Georgian Class 3 verbs borrowed into Georgian constitute a new morphosyntactic pattern: they are verbs that are morphologically transitive (using the transitive light verb \textit{d-i}) but syntactically intransitive (since they are monovalent). This new class is in contrast with the existing class of intransitive verbs that demand or allow Ergative subject marking in the 1st or 2nd person (see \sectref{intrerg}). The difference between the two construction types is summarised in \tabref{table-ergintrans}.

\begin{table}[hp]
		\begin{tabularx}{\textwidth}{QQ}
        \lsptoprule
			Native verbs with variable or obligatory Ergative subjects & Borrowed Georgian class 3 verbs \\\midrule
			Some verbs allow both Ergative and Nominative subjects & All verbs require Ergative subjects \\\addlinespace
			Ergative subjects only for 1st and 2nd person & Ergative subject for all persons \\\addlinespace
			(Possible) Ergative case controlled by semantics of volition & Ergative case controlled by belonging to a morphological class \\\addlinespace
			Gender marker cross-references Ergative subject & Default gender marker \textit{d-}, not cross-referencing anything \\
		\lspbottomrule
		\end{tabularx}
	\caption{Two types of intransitive verbs with Ergative subjects}
	\label{table-ergintrans}
\end{table}


\begin{table}[hp]
	\small
	\begin{tabular}{llll}
    \lsptoprule
                       &           &           & Derived/compounded\\
		\multicolumn{2}{l}{{Type}} & {Example} & {from} \\
        \midrule
		\multicolumn{2}{l}{Light verb constructions} & \textit{bexk' b-aq-ar} `reproach' & `fault' + `take' \\
		\multicolumn{2}{l}{Light verb compounds}  & & \\
		& Default & \textit{dak'-d-aɬ-ar} `realise' & `heart' + `appear' \\
		& With fossilised gender  & \textit{k'eč'(-)j(-)aq-ar}  & `boiling' + `take' \\
		& \quad marker            & \quad  `come to a boil'  & \\
		\multicolumn{2}{l}{Dvandva compounds} & \textit{at'-xal-ar} `die out, fade' & `become silent' + `fade' \\
		\multicolumn{2}{l}{Reduplicating verbs} & \textit{tak'-sak'-ar} `patch up' & `sew' + \textit{sak'} \\
		\multicolumn{2}{l}{With suffixes \textit{-d-i, -d-al}} \\
		& Deadjectival \textit{-d-i} & \textit{ap-d-ar} `make green' & \textit{apen} `green' \\
		& Deadjectival \textit{-d-al} & \textit{ap-d-al-ar} `become green' & \textit{apen} `green' \\
		& Deverbal \textit{-d-i} & \textit{d-opx-d-ar} `dress sb' & \textit{d-opx-ar} `put on, wear' \\
		& Deverbal \textit{-d-al} & \textit{d-opx-d-al-ar} `get dressed' & \textit{d-opx-ar} `put on, wear' \\
		\multicolumn{2}{l}{Other deverbal suffixes} \\
		& Causative \textit{-it} & \textit{d-aɬ-it-ar} `release' & \textit{d-aɬ-ar} `go out' \\
		& Potential \textit{-mak'} & \textit{d-et'-mak'-ar} `able to run' & \textit{d-et'-ar} `run' \\
		
		\lspbottomrule
	\end{tabular}
	\caption{A typology of Tsova-Tush complex verbs}
	\label{table-complexverbtyp}
\end{table}

\pagebreak
\section{Summary}


In terms of basic description, this chapter has provided new insight into the following domains:

\begin{enumerate}
	\item Contrary to \textcite{holisky87}, Tsova-Tush does not feature a class of intransitive verbs that demand an Ergative argument in all 3 persons (other than borrowed Georgian Class 3 verbs).
	
	\item A typology of complex verbs has been outlined in \sectref{lightverbs}, which is summarised in \tabref{table-complexverbtyp}.
	
	
\end{enumerate}


In terms of structural language contact, this chapter has shown the following parallels between Tsova-Tush and Georgian.

\begin{enumerate}
	\item The variable marking of 1st and 2nd person subjects of some Tsova-Tush intransitive verbs has likely not arisen due to contact with Georgian.
	
	\item Due to the borrowing of a number of Georgian Class 3 verbs, Tsova-Tush now has a new class of intransitive verbs, characterised by transitive morphology (the light verb \textit{d-i}), a default gender marker \textit{d-} (not cross-referencing any argument), atelic semantics, and an Ergative subject in all persons.
	
\end{enumerate}


In terms of loan word adaptation, this chapter has shown that: 

\begin{enumerate}
	\item Verbs borrowed from Georgian are adapted using the suffixes \textit{-d-i} for transitive and \textit{-d-al} for intransitive verbs, to which all inflectional material is added. Thus, considerable integrational effort (in the terminology of \cite[135]{wohlgemut09loanverbtyp}) has been undertaken to integrate borrowed verbs. Root perfectivity (as defined in \sectref{rootperf}) is indicated by Georgian preverbs borrowed along with the verb root.
	
	\item Georgian Class 3 verbs (monovalent verbs characterised morphologically by forming their perfective stem without a preverb, semantically by containing mostly atelic verbs, and syntactically by demanding an Ergative subject in the 2nd TAM Series) are adapted as monovalent Tsova-Tush verbs with transitive morphosyntax.
	
	\item The basic input form of these borrowings is modelled on the Old Georgian Verbal Noun in the Adverbial case, not found in Modern Georgian. A very close parallel are verbal forms containing the suffix \textit{-miš} in Turkic verbal borrowings in Iranian languages (\cites[]{ido06}, as cited in \cites[112]{wohlgemut09loanverbtyp}). According to \citeauthor{ido06}, the \textit{-miš} form is no longer productive at least in some of the modern varieties of the donor languages (Uzbek, Uyghur). Therefore, verbs from these languages must either have been borrowed centuries ago, or the suffix itself got borrowed and became an integrated and productive part of a separate loan verb accommodation pattern in some of the recipient languages (like Tajik, Sarikoli) (\cites[112]{wohlgemut09loanverbtyp}). Since in Tsova-Tush, verbal borrowing is ongoing due to frequent code-switching, the second of Ido's explanations must be true for Tsova-Tush: the morph \textit{-ad}, no longer productive in Modern Standard Georgian, has become a productive part of a specialised loan verb accommodation strategy.
	
	
\end{enumerate}
