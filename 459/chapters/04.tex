\chapter{Verbal inflection} \label{inflection}

\section{Introduction}

In this chapter, the inflectional categories of Tsova-Tush verbs will be discussed. Tsova-Tush finite verb forms inflect for gender, person, number, tense, aspect, mood and evidentiality.\footnote{In this work, voice and valency operations are considered to be derivational (see \sectref{derivation}).} All verbs inflect for TAME and person, but only different subsets of verbs allow gender and number. Additionally, due to the fact that some verbs lack the distinction between a Perfective and an Imperfective stem (see \sectref{rootperf}), these verbs thereby also lack certain tense and aspect distinctions. The categories mentioned above can be expressed by various morphological means. Gender is expressed by prefixes (\sectref{verbalgender}), the aspect of the root is expressed by means of infixing and ablaut (\sectref{rootperf}), while TAME categories (\sectref{TAME}) and person marking (\sectref{person}) are all marked by suffixes. Number (more specifically, argument plurality), on the other hand, is marked (1) indirectly through gender marking, but only for the M, F and B genders; (2) indirectly through person marking, but only in the 1st and 2nd person; (3) directly through stem suppletion (see \sectref{verbalnumber}), but only in a handful of verbs; and (4) directly through a dedicated plural suffix, but only in imperative moods (see \sectref{suffixpl}). Although preverbs often occur before the verb root, they are not considered affixes since the negative particle \textit{co} can intervene, and since preverbs can sometimes take sentence-final position. 


Two features are discussed in the context of contact-induced language change: the development of subject agreement suffixes (\sectref{person}), and the borrowing of a suffix \textit{-t} signalling the plurality of a speech act participant subject (\sectref{suffixpl}).

Table \ref{slots} gives all verbal affix slots of the Tsova-Tush simplex finite verb (for complex verbs, see \sectref{valintro}).
\pagebreak
\begin{table}
	\tabcolsep=.555\tabcolsep
	\begin{tabular}{cccccccc}
	\lsptoprule
		$-2$ & $-1$ & 0 & 1 & 2 & 3 & 4 & 5\\
		Preverb= & Gender- & Stem & -TAME & \textsc{-pl} & \textsc{-subord/aff} & -Subj.Pers. & =Obj.Pers \\
		& & & & & \textsc{-q} & & \\
		& & & & & \textsc{-seq} & & \\
	\lspbottomrule
	\end{tabular}
	\caption{Verb slots}
	\label{slots}
\end{table}

\begin{description}[font=\normalfont]
\item[$-2$:] A preverb can be cliticised to the root to signal direction, aspect, lexical semantics or telicity (\cite[184]{holiskygagua}).\footnote{An investigation into the precise function of the preverbs, and their comparison with Georgian and Vainakh preverbs, remains to be carried out.}\is{Preverb}
\item[$-1$:] Four possible gender prefixes can refer to 10 possible gender-number values and agree with the Nominative argument of a two-place verb, or with the sole argument of a one-place verb (be it in the Nominative or Ergative case). See \sectref{verbalgender}.
\item[0:] The stem\footnote{In Caucasian linguistics, the term “root” is often used to refer to the monoconsonantal East Caucasian verb root. Hence, in this work, the term “stem” is used. (The Tsova-Tush inherited verbal stems often consist of a root plus a grammaticalised Proto-East-Caucasian preverb.)} is used to mark a two-way aspect distinction (used to distinguish the Present from the Future, or single-action from pluractional verbs) in approximately 160 verbs (\sectref{rootperf}). In 15 verbs, stem suppletion marks the plurality of one of the verbal arguments (\sectref{verbalnumber}).\is{Verb root}\is{Verb stem}
\item[1:] Tsova-Tush distinguishes three tense values (past, present, future), two aspect values (perfective vs. imperfective), five moods (Indicative/neutral, Subjunctive, Imperative, Conditional, Iamitive) and two evidential values (Witnessed/neutral vs. Non-Witnessed). See \sectref{TAME}.\is{Tense}\is{Aspect}\is{Mood}
\item[2:] In some verb forms, a suffix \textit{-t} can mark argument plurality (\sectref{suffixpl}).
\item[3:] The suffix \textit{-ici} can be used to signal that the verb is part of a subordinate clause (\sectref{subordination}), and is also, though rarely, used in main clauses to mark the Affirmative.\footnote{Which is hitherto poorly understood and will not be pursued in this work.} The suffix \textit{-i} is a question suffix.\footnote{This is most often treated as a clitic, since it attaches to any part of speech. It will not be discussed further in this work.} The suffix \textit{-e} is a Sequential suffix, discussed in \sectref{coord}.
\item[4:] Speech act participant (i.e. first or second person) subject indices are suffixed to the Tsova-Tush verb and are marked for case (Nominative or Ergative), see \sectref{person}.
\item[5:] Speech act participant object pronouns can be cliticised to the verb form and are marked for case (Nominative, Dative, Contact or Allative).
\end{description}


\section{Gender} \label{verbalgender}

Tsova-Tush, like its immediate Nakh relatives Chechen and Ingush, is a language with grammatical gender, sometimes called noun class (\cite{gagua52}). All East Caucasian languages except for three Lezgic languages (Udi, Lezgian and Aghul) have gender, while the feature is completely absent in Georgian and its Kartvelian sister languages. A gender is defined as a category of nouns, whose membership is reflected by agreement markers attached to other sentence elements, called agreement targets (\cite[146--147]{corbett91}). In Tsova-Tush, approximately one third of all underived simple verbs  inflect for gender.\footnote{One preverb, one underived numeral, one underived quantifier, and 10 underived adjectives also inflect for gender (see \sectref{agreement}).}\is{Gender agreement!on verbs}\is{Cross-referencing!Gender} 
Similarly to some adjectives (see \sectref{adjectives}), these verbs take one of four gender prefixes to agree with a noun phrase in the same clause. For verbs, this noun phrase is the subject of an intransitive verb, or the object of a transitive verb, typically marked by the Nominative case (but see the discussion on page \pageref{ergintrans} for exceptions). Three of these genders (M, F, B) require different prefixes for singular and plural, which leads some scholars to describe this type of agreement as gender-number agreement. In much work on Nakh languages, apart from M for masculine and F for feminine, the labels for the genders are based on their marker in the singular. This contrasts with the tradition of labeling gender in Daghestanian languages, which uses Roman numerals (where the same Roman numeral can refer to a different set of agreement affixes in different languages). The five genders and the prefixes they require on the agreement target are listed in Table \ref{verbalgender-table1}, along with the corresponding glosses used in this work. Some 30 nouns show one of three agreement patterns that differ from those of any of these five genders (see \cites[163]{holiskygagua}[170--172]{corbett91}{WS}).

\begin{table}
	\begin{floatrow}
	\ttabbox{%
	\begin{tabular}{lllll}
    \lsptoprule
		Gender & {\textsc{sg}} & {\textsc{pl}} & \multicolumn{2}{c}{{Glossing}}  \\\midrule
		M & \textit{v-} & \textit{b-} & \textsc{m.sg} & \textsc{m.pl} \\
		F & \textit{j-} & \textit{d-} & \textsc{f.sg} & \textsc{f.pl}\\
		B & \textit{b-} & \textit{d-} & \textsc{b.sg} & \textsc{b.pl}\\
		D & \textit{d-} & \textit{d-} & \multicolumn{2}{c}{\textsc{d}} \\
		J & \textit{j-} & \textit{j-} & \multicolumn{2}{c}{\textsc{j}} \\
	\lspbottomrule
	\end{tabular}}
	{\caption{Tsova-Tush genders}\label{verbalgender-table1}}
	
	\ttabbox{\begin{tabular}{llll}
		\lsptoprule
		\textit{d-aɬar} & `give' & \textit{aɬar} & `say' \\
		\textit{d-ettar} & `pour' & \textit{ettar} & `stand, stay'  \\
		\textit{d-oc'ar} & `tie' & \textit{oc'ar} & `pull, move' \\
		\lspbottomrule
	\end{tabular}}
	{\caption{Tsova-Tush verbs with and without gender marking}\label{verbalgender-table2}}
	\end{floatrow}
\end{table}

Of all underived verbs, only those that begin with a vowel or \textit{ʕ} can agree in gender, but not every verb that meets these conditions shows gender marking (\cites{haukharris}[278]{harris09}). It is impossible to predict which verbs feature gender marking, as can be seen from Table \ref{verbalgender-table2}, taken from \textcite{haukharris}.

In Example (\ref{verbalgender-ex01}), the verb forms \textit{bar} `was’ and \textit{baɬiⁿ} `gave away’ agree in gender with the Nominative noun \textit{pst'u} `ox', which is gender B, and therefore show a \textit{b-} in their designated prefix slots. Note that \textit{pst'u} `ox' functions as the subject of the intransitive verb `be' and as the object of the transitive verb `give'.

\begin{exe}
	\ex\label{verbalgender-ex01}
	\gll cħa b-aqqoⁿ pst'u \textbf{b}-a-r, daħ=a \textbf{b}-aɬ-iⁿ. \\
	one	{\B}.{\Sg}-big ox(B) \textbf{{\B}.{\Sg}}-be-{\Imprf} away={\Add} \textbf{{\B}.{\Sg}}-give.{\Pfv}-{\Aor} \\
	\trans ‘There was one big ox and they gave it away.’
	\hfill (E153-23)
\end{exe}

Tsova-Tush verb forms can contain multiple gender markers, all cross\hyp referencing the same argument, due to (1) the fact that evidentiality is marked by a grammaticalised auxiliary verb \textit{d-a} `be' (see \sectref{evid}), (2) the fact that Tsova-Tush features valency derivation by light verbs inflecting for gender (see \sectref{derivation}), and (3) the fact that Tsova-Tush features some verbal compounds consisting of two gender-inflecting verbs (see Example (\ref{verbalgender-ex04})\footnote{Original orthography of (\ref{verbalgender-ex04}): osiḥ dawdakdie \.{s}ari \'{k}oneb, waxe\.{s} moi\.{s}.}). This phenomenon, called multiple or exuberant exponence, is described extensively by \textcite{harris08,harris09} and \textcite{harrissamuel}.\is{Multiple exponence}



	\begin{exe}
		\ex\label{verbalgender-ex04}
		\gll osi-ħ \textbf{d-av-d-ak'-d-i-eⁿ} šariⁿ koneb, v-ax-eš moiš. \\
		there-{\Ess} \textbf{{\D}-lose-{\D}-burn-{\D}-{\Tr}-{\Aor}} {\Refl}.{\Poss} property(D) {\M}.{\Sg}-live-{\Simul} badly \\
		\trans `There, he squandered his fortune, living badly.'
		\hfill (AS006-1.3)
	\end{exe}




Example (\ref{verbalgender-ex02}) shows how the verb \textit{d-axar} `leave' interacts with nouns having different genders.

\begin{exe}
	\ex\label{verbalgender-ex02}
	\begin{xlist}
		
		\ex
		\gll vaš\u{o} v-ax-eⁿ / važar b-ax-eⁿ \\
		brother.{\Nom} {\M}.{\Sg}-leave-{\Aor} {} brother.{\Nom}.{\Pl} {\M}.{\Pl}-leave-{\Aor} \\
		
		\ex
		\gll jaš\u{o} j-ax-eⁿ / jažar d-ax-eⁿ \\
		sister.{\Nom} {\F}.{\Sg}-leave-{\Aor} {} sister.{\Nom}.{\Pl} {\F}.{\Pl}-leave-{\Aor} \\
		
		\ex
		\gll pħu b-ax-eⁿ / pħarč d-ax-eⁿ \\
		dog.{\Nom} {\B}.{\Sg}-leave-{\Aor} {} dog.{\Pl}.{\Nom} {\B}.{\Pl}-leave-{\Aor} \\
		
		\ex 
		\gll bader d-ax-eⁿ / badr-i d-ax-eⁿ \\
		child.{\Nom} {\D}-leave-{\Aor} {} child-{\Pl}.{\Nom} {\D}-leave-{\Aor} \\
		
		\ex
		\gll sov j-ax-eⁿ / sov-i j-ax-eⁿ \\
		vulture.{\Nom} {\J}-leave-{\Aor} {} vulture-{\Pl}.{\Nom} {\J}-leave-{\Aor} \\
		\trans `The brother(s)/sister(s)/dog(s)/child/children/vulture(s) left. \\
		\hfill (adapted from \textcite[177]{holiskygagua})
	\end{xlist}
\end{exe}

For gender assignment, i.e. the rules that dictate which nouns belong to which gender, and for the adaptation of Georgian loanwords using these rules, see \textcite{WS} and \sectref{genderadapt}. 

\label{ergintrans}
As mentioned, the gender-inflecting verbs mark the gender of the noun that functions as the subject of an intransitive verb or the object of a transitive verb, which, Tsova-Tush being a language with ergative alignment, is usually marked by the Nominative case. However, as shown by \textcite{holisky87}, intransitive subjects of the first and second person can be marked by the Ergative case. This occurs in one of the following conditions: 
\begin{enumerate}
	\item The intransitive verb belongs to a class that allows variable marking in first and second person subjects, where an Ergative subject signals a greater amount of volition and agency than a Nominative subject. Depending on the semantics of the verb, the volitional (Ergative) or the less volitional (Nominative) subject is more common, or both are attested in equal measure. These verbs include: some verbs of communication (mostly \textsc{erg}); some verbs of motion (mostly \textsc{erg}); verbs of falling, rolling, slipping; changes of state; locative statives, and others.\footnote{The full list can be found as an appendix to \textcites{holisky87}.} See Example (\ref{verbalgender-ex03a}), where the verb \textit{d-erc'ar} `turn' takes an Ergative first person singular subject. Or:
	\item The intransitive verb belongs to a class that requires its subject to be marked by the Ergative. These verbs include all verbs of motion or communication (other than those that show variable marking), as well as approximately 28 other verbs. See Example (\ref{verbalgender-ex03b}). 
\end{enumerate}

Additionally, Holisky has identified 6  verbs that show morphological similarities with transitive verbs and require Ergative subjects in all three persons. They exclusively occur without a direct object, and are therefore  treated syntactically as intransitives by Holisky (see Example (\ref{verbalgender-ex03c}), and see \sectref{1arg}).\is{Intransitive verbs}\is{Monovalent verbs}

Thus, these two categories of verbs allow (in the case of variable subject marking) or require (in the case of obligatory Ergative subject marking) gender marking to agree with a verbal argument marked with the Ergative case.

\begin{exe}
	\ex\label{verbalgender-ex03}
	\begin{xlist}
		
		\ex\label{verbalgender-ex03a}
		\gll oqar-n \textbf{v-irc'-n-as} v-ux=ajn\u{o}. \\
		{\Dist}.{\Pl}.{\Obl}-{\Dat} \textbf{{\M}.{\Sg}-turn-{\Aor}-{\Fsg}.{\Erg}} {\M}.{\Sg}-back={\Quot} \\
		\trans `I returned for them.' (i.e. `I chose to return for them.')
		\hfill (EK005-4.7)
		
		
		\ex\label{verbalgender-ex03b}
		\gll inc širk-ileⁿ v-a\u{g}-o-s, \textbf{v-uit'-as} tinit lam-na-x. \\
		now Shiraki-{\Elat} {\M}.{\Sg}-come-{\Npst}-{\Fsg}.{\Erg} \textbf{{\M}.{\Sg}-go({\Npst})-{\Fsg}.{\Erg}} Tianet.{\Ill} mountain-{\Obl}.{\Pl}-{\Cont} \\
		\trans `Now I am coming from Shiraki, and I'm going to the Tianeti mountains.'
		\hfill (E040-20)
		
		\ex\label{verbalgender-ex03c}
		\gll šin šar-e k'olekt'iv-e \textbf{mušeba(d)-d-i-n-as} [...] osi-ħ=a moc'inav j-a-ra-s. \\
		two.{\Obl} year-{\Obl}({\Ess}) collective-{\Obl}({\Ess}) \textbf{work-{\D}-{\Tr}-{\Aor}-{\Fsg}.{\Erg}} {} there-{\Ess}={\Add} leader {\F}.{\Sg}-be-{\Imprf}-{\Fsg}.{\Nom} \\
		\trans `I worked in a collective for two years, I was a leader there too.' \\
		\hfill (E116-17)
		
	\end{xlist}
\end{exe}\is{Gender agreement!Default}

In Tsova-Tush, the D gender is used in cases where so-called neutral gender agreement is required (see \cites[205]{corbett91}). That is, when an agreement target is `forced' to agree with an item that lacks gender, or that is unspecified in terms of gender, a default gender D is used. This is seen in three scenarios:\largerpage

\begin{enumerate}
	\item The speaker does not know the gender of a human referent, or chooses to leave it unspecified. In Example (\ref{verbalgender-ex05a}), the headless modifier \textit{d-aqq-čo-v} `big' refers to any member of the Tsova-Tush clan, regardless of gender, and therefore marks the neutral gender D. The verb \textit{do} `calls' agrees in gender with the Nominative argument \textit{ganq'op} `relative', which is a human noun, which would trigger M or F agreement. The speaker, however, chooses to leave the gender unspecified, triggering D agreement.
	
	\item A word agrees with two or more nouns that have different genders (\ref{verbalgender-ex05b}); or
	
	\item Agreement is with a clause (\ref{verbalgender-ex05c}).\footnote{Original orthography of (\ref{verbalgender-ex05c}): Dagi cruen, me cḥain uirwas dakardie itt baḥ o\'{k}rui.}
\end{enumerate}

\begin{exe}
	\ex\label{verbalgender-ex05}
	\begin{xlist}
		
		\ex\label{verbalgender-ex05a}
		\gll bac-bi-v cħajn\u{\i} gor-le-čui-š-v - st'ak'o-v, pst'uin-čo-v, \textbf{d}-aqq-čo-v, k'ac'k'-čo-v vašba-x ganq'op \textbf{d}-∅-o. \\
		Tsova\_Tush-{\Pl}-{\Erg} one.{\Obl} clan-{\Adjz}-{\Obl}-{\Pl}-{\Erg} {} man.{\Obl}-{\Erg} woman-{\Obl}-{\Erg}  \textbf{{\D}}-big-{\Obl}-{\Erg} small-{\Obl}-{\Erg} {\Recp}-{\Cont} relative({\M}/{\F}) \textbf{{\D}}-call-{\Npst} \\
		\trans `The Tsova-Tush, [being all] of one clan, [whether they be] man, woman, old, young, call each other relatives.'
		\hfill (KK003-0693)
		
		\ex\label{verbalgender-ex05b}
		\gll taⁿ ma \textbf{adam=a\u{e}}, \textbf{ev=a\u{e}}, ev-e-ⁿ \textbf{vašo=a\u{e}}, le\u{o}, badrulob-e=doliⁿ cħan-\u{g} \textbf{d}-aq-l-ar. \\
		alongside {\Contr} \textbf{Adam({\M})={\Add}} \textbf{Eva({\F})={\Add}} Eva-{\Obl}-{\Gen} \textbf{brother({\M})={\Add}} Leo childhood-{\Obl}({\Ess})=after one.{\Obl}-{\Trans} \textbf{{\D}}-grow\_up-{\Intr}-{\Imprf} \\
		\trans `Meanwhile, Adam, Eva, and Eva's brother Leo grew up together.' \\
		\hfill (MM414-1.13)
		
		
		\ex\label{verbalgender-ex05c}
		\gll \textbf{d}-ag-iⁿ	cru-e-n,	{{\normalfont[} me}	cħajn	uirv-a-s	dak'ar-d-i-eⁿ	it't'	baħ	{okrui-ⁿ {\normalfont]}}.  \\
		\textbf{{\D}}-see-{\Aor}	rogue-{\Obl}-{\Dat}	{\Subord}	one.{\Obl}	Jew-{\Obl}-{\Erg}	count-{\D}-{\Tr}-{\Aor}	ten	100\_pieces	gold-{\Gen}    \\
		\trans `The rogue saw that a Jewish man was counting a thousand gold pieces.’	
		\hfill (AS008-11.3)
		
	\end{xlist}
\end{exe}


\section{Aspect of the verbal stem} \label{rootperf}

Approximately 160 verb stems (one third of verbal stems, excluding derivations and loans) appear in two different variants, traditionally labeled as Perfective and Imperfective stems (\cites[149]{desheriev53}{gagua62}[179]{holiskygagua}). Imperfective stems are generally derived historically from Perfective ones, although no productive process is observed. The distinction is made through various morphological means:\is{Aspect}\is{Pluractionality}\is{Ablaut!Verbal}\is{Infix}

\begin{enumerate}
	\item Ablaut: The vowel in the Perfective stem changes to \textit{-e-} to form the Imperfective stem.
	\begin{itemize}
		\item 78 verbs: \textit{a - e}
		\subitem \textit{maɬar} (\textsc{pfv}) – \textit{meɬar} (\textsc{ipfv}) `drink’
		\item 36 verbs: \textit{o - e}
		\subitem \textit{xop't'ar} (\textsc{pfv}) – \textit{xep't'ar} (\textsc{ipfv}) `suck, slurp’
		\item 4 verbs: \textit{i - e}
		\subitem \textit{lič'ar} (\textsc{pfv}) – \textit{leč'ar} (\textsc{ipfv}) `peel, scrape'
		\item 17 verbs: vowel (14 of which are \textit{o}) - \textit{e}. Additionally, an infix \textit{-b-} is added. The \textit{-b-} infix only occurs before \textit{c', s, ž, x, l}\footnote{Several other verbs (with a greater variety of root consonants) show a petrified infix \textit{-b-} which has been spread to the Perfective stem, leaving only ablaut as a distinctive marker, e.g. \textit{xop't'ar} (\textsc{pfv}) / \textit{xep't'ar} (\textsc{ipfv}) `suck, slurp'.} and assimilates in voice and airstream mechanism to the following consonant.
		\subitem \textit{tox-d-ar} (\textsc{pfv}) – \textit{tepx-d-ar} (\textsc{ipfv}) `strike’
		\subitem \textit{d-ožar} (\textsc{pfv}) – \textit{d-ebž-ar} (\textsc{ipfv}) `fall’
		\subitem \textit{d-oc'ar} (\textsc{pfv}) – \textit{d-ep'c'-ar} (\textsc{ipfv}) `tie up’
		\item 9 verbs: \textit{a - e}. Additionally, the Imperfective stem is marked by a gender prefix (the reverse pattern compared to the verbs under point 2 below).
		\subitem \textit{aħar} (\textsc{pfv}) – \textit{d-eħar} (\textsc{ipfv}) `steal’
		\subitem \textit{ak'ar} (\textsc{pfv}) – \textit{d-ek'ar} (\textsc{ipfv}) `fall’
		\subitem \textit{aqar} (\textsc{pfv}) – \textit{d-eqar} (\textsc{ipfv}) `pay for’
		\subitem \textit{aq'ar} (\textsc{pfv}) – \textit{d-eq'ar} (\textsc{ipfv}) `divide’
		\subitem \textit{axk'ar} (\textsc{pfv}) – \textit{d-exk'ar} (\textsc{ipfv}) `tie up’
	\end{itemize}
	\item Gender marking: The presence of a gender prefix marks the Perfective stem (the reverse pattern from the 9 verbs presented directly above).
	\begin{itemize}
		\item 1 verb: Presence of gender marking marks Perfective. Additionally, the Imperfective is marked by an ablaut grade \textit{-e-}.
		\subitem \textit{d-at'ar} (\textsc{pfv}) - \textit{et'ar} (\textsc{ipfv}) `crack, break'
		\item 2 verbs: Presence of gender marking marks Perfective. Additionally, the Perfective is marked by an ablaut grade \textit{-e-} (the reverse of the general pattern shown under point 1).
		\subitem \textit{d-ʕep'-d-ar} (\textsc{pfv}) - \textit{ʕap'-d-ar} (\textsc{ipfv}) `lock up'
		\subitem \textit{d-ʕevar} (\textsc{pfv}) - \textit{ʕavar} (\textsc{ipfv}) `kill'
		\item 2 verbs: Presence of gender marking marks Perfective, replacing the initial consonant of the Perfective. 
		\subitem \textit{d-erc'ar} (\textsc{pfv}) - \textit{ħerc'ar} (\textsc{ipfv}) `turn, return'
		\subitem \textit{d-ekar} (\textsc{pfv}) - \textit{qekar} (\textsc{ipfv}) `call'
		\item 2 verbs: Presence of gender marking marks Perfective, replacing the initial consonant of the Perfective. Additionally, the Imperfective root is marked by an \textit{-e-} vowel.
		\subitem \textit{d-ʕogar} (\textsc{pfv}) - \textit{q'ʕegar} (\textsc{ipfv}) `break'
		\subitem \textit{d-aɬar} (\textsc{pfv}) - \textit{teɬar} (\textsc{ipfv}) `give'
		
	\end{itemize}
	\item Other pairs
	\begin{itemize}
		\item 3 verbs: Irregular pairs
		\subitem \textit{d-uq'ar} (\textsc{pfv}) - \textit{ap'q'ar} (\textsc{ipfv}) `drive into, stick into'
		\subitem \textit{d-agar} (\textsc{pfv}) - \textit{guar} (\textsc{ipfv}) `see'
		\subitem \textit{gu-d-aqar} (\textsc{pfv}) - \textit{gu-d-axar} (\textsc{ipfv}) `show'
		\item 8 verbs: Suppletion
        \subitem \textit{aɬar} (\textsc{pfv}) - \textit{lev-d-ar} (\textsc{ipfv}) `say'
		\subitem \textit{d-isar} (\textsc{pfv}) - \textit{meq'ar} (\textsc{ipfv}) `stay'
		\subitem \textit{d-ʕaɁar} (\textsc{pfv}) - \textit{ak'-d-ar} (\textsc{ipfv}) `light, kindle'
		\subitem \textit{qallar} (\textsc{pfv}) - \textit{d-aq'ar} (\textsc{ipfv}) `eat'
		\subitem \textit{d-ettar} (\textsc{pfv}) - \textit{d-iš-d-ar} (\textsc{ipfv}) `hit, strike'
		\subitem \textit{d-axar} (\textsc{pfv}) - \textit{d-ot'ar} (\textsc{ipfv}) `go'
		\subitem \textit{d-aɁar} (\textsc{pfv}) - \textit{d-a\u{g}ar} (\textsc{ipfv}) `come'
		\subitem \textit{eqqar} (\textsc{pfv}) - \textit{letxar} (\textsc{ipfv}) `jump, leap, dance' 
		
		
	\end{itemize}
\end{enumerate}

Verbs borrowed from Georgian can distinguish between Perfective and Imperfective stems using an inflectional pattern (the affixation of a preverb) that has been borrowed along with the lexical item (see \sectref{loanverb} for more information on the adaptation of Georgian verbs). Which Georgian preverb is chosen is determined lexically. Hence, a fourth way of marking perfectivity can be identified. For more detail, see \sectref{loanverb}.\is{Preverb}
\begin{enumerate}
	\setcounter{enumi}{3}
	\item Preverbs (productive): Both Perfective and Imperfective members of a Georgian verb pair are borrowed.
	\subitem  \textit{agrilbad-d-ar} (\textsc{pfv}) - \textit{grilbad-d-ar} (\textsc{ipfv}) `cool'
	\subitem \textit{dat'anǯod-d-ar} (\textsc{pfv}) - \textit{t'anǯod-d-ar} (\textsc{ipfv}) `torture'
	\subitem \textit{gamarglod-d-ar} (\textsc{pfv}) - \textit{marglod-d-ar} (\textsc{ipfv}) `weed, hoe'
	\subitem \textit{šerisxod-d-ar} (\textsc{pfv}) - \textit{risxod-d-ar} (\textsc{ipfv}) `invoke wrath on'
	\subitem \textit{moc'amlod-d-ar} (\textsc{pfv}) - \textit{c'amlod-d-ar} (\textsc{ipfv}) `apply poison/pesticide on'
\end{enumerate}

In some instances, the use of an additional Tsova-Tush preverb can be observed, such as in \textit{daħ daxat'od-d-ar} (\textsc{pfv}) vs. \textit{xat'od-d-ar} (\textsc{ipfv}) `paint’. This use has been claimed to spread to native Tsova-Tush verbs
(\cite{tetradze}, who cites a single example: \textit{ču ottar} (\textsc{pfv}) vs. \textit{ettar} (\textsc{ipfv}) `stand’). However, a purely locational function of the preverb cannot be excluded. Furthermore, the verbs that can occur with a Tsova-Tush preverb can do so with both the Perfective or Imperfective verb stem. The exact function of these preverbs (in terms of aspect, telicity, spatial direction, and lexical semantics) needs to be investigated separately, which falls beyond the scope of this work.

The distinction between the two verb stems serves multiple functions. The Present form of the verb is based on the Imperfective verb stem, whereas the Future is based on the Perfective stem (see \sectref{TAME} for the inflection of these forms). In the Aorist (the primary past tense), however, the perfectivity distinction is primarily one of pluractionality (\cite{holiskygagua,holisky85}, contrary to \cites{gagua62}[52]{schiefner59}); Aorist forms based on the Perfective stem denote single-action events, whereas those based on the Imperfective stem refer to multiple actions. See \sectref{TAME} for usage examples. Analyses of Chechen (\cite{yu2003pluract}) and Ingush (\cite[313--318]{nichols11}) provide comparative evidence for this verbal category. It is important to note that only approximately 160 verb stems, which constitute around a third of all underived verbs, make a Perfective-Imperfective distinction. This means that for the majority of verbs, pluractional semantics in the past, or the difference between present and future tense, is not expressed overtly.

\pagebreak
\section{Argument plurality} \label{verbalnumber}\is{Cross-referencing!Number}
The category of plurality can be indicated indirectly by gender marking (in the M, F and B genders, see \sectref{verbalgender}), by person marking (in the 1st and 2nd person, see \sectref{person}), and directly by the suffix \textit{-t} (in the imperative moods, see \sectref{imp}). Additionally, at least 15 verbs (counting valency derivations as different verbs, but not historically derived imperfectives) show a number distinction using stem suppletion (see \cites[150]{desheriev53}[178]{holiskygagua}). As is clear from Table \ref{verbalnumber-table1} (1), one recognisable pattern consists of replacing the root consonant with \textit{-xk'-} (clearly an ancient root meaning `move (\textsc{pl})'). Other replacements of the root consonant (2), as well as ablaut (3) and suppletion (4), are observed.

\begin{table}
	\fittable{\begin{tabular}{l@{~}lllll}
    \lsptoprule
		& \multicolumn{2}{@{~}c}{{Singular}} & \multicolumn{2}{c}{{Plural}}  &   \\\cmidrule(lr){2-3}\cmidrule(lr){4-5}
		& \textsc{pfv} & {\textsc{ipfv}} & {\textsc{pfv}} & {\textsc{ipfv}} & \\
		\midrule
		(1) & \multicolumn{4}{@{~}l}{{root \textit{xk'}}} \\
		& d-illar & d-eblar & d-ixk'ar & d-exk'ar & `put down, lay down' \\
		& d-ollar & d-eblar & d-oxk'ar & d-exk'ar & `put, put inside'\\
		& ollar & eblar & oxk'ar & exk'ar & `put, pour, skewer' \\
		& oll-d-ar & ebl-d-ar & oxk'-d-ar & exk'-d-ar & `put on, set' \\
		& tillar & teblar & tixk'ar & texk'ar & `stoke up (firewood)' \\
		& lallar & lellar & laxk'ar & lexk'ar & `drive, herd'\footnote{Only found in \textcite[53]{schiefner56} and \textcite[133]{desheriev53}, singular not known by contemporary speakers.} \\
		& qollar & qeblar & qoxk'ar & qexk'ar & `hang' (\textsc{intr}) \\
		& qoll-d-ar & qebl-d-ar & qoxk'-d-ar & qexk'-d-ar & `hang' (\textsc{tr}) \\
		& qoc'ar & qep'c'ar & qoxk'ar & qexk'ar  & `be hung up, be loaded' \\
		& qoc'-d-ar & qep'c'-d-ar & qoxk'-d-ar & qexk'-d-ar  & `pack, load, hang up' \\
		& d-aɁar & d-a\u{g}ar\footnote{also \textit{d-eɁar}} & d-axk'ar & d-exk'ar & `come' \\
		\midrule
		(2) & \multicolumn{4}{@{~}l}{{other roots}} \\
		& xaɁar & xeɁar & xabžar & xebžar & `sit down' \\
		& xoɁ-d-ar & xeɁ-d-ar & xobžar & xebžar & `squeeze in, find room' \\ 
		& ħač'ar & ħeč'ar & ħapsar & ħepsar &  `look to, look after' \\
		\midrule
		(3) & \multicolumn{4}{@{~}l}{{ablaut}}\\
		& qosar & qepsar & qasar & qepsar & `throw, shoot, hurl' \\
		\midrule
		(4) & \multicolumn{4}{@{~}l}{{suppletion}} \\
		& eqqar & & letxar & & `jump'\footnote{At this point it is not clear whether the pair \textit{eqqar : letxar} exhibits a distinction in perfectivity (i.e. event plurality) or argument plurality, or both.}
		\\
		\lspbottomrule
	\end{tabular}}
	\caption{Tsova-Tush verbs with suppletive plural stems}
	\label{verbalnumber-table1}
\end{table}

Examples (\ref{verbflex-ex01}) and (\ref{verbflex-ex45}) illustrate the fact that different verbal stems are used depending on the number of the Nominative argument. Example (\ref{verbflex-ex01}) shows a Nominative object `claw(s)', which triggers two distinct suppletive verbs `put'.\footnote{Note that \textit{mʕajr\u{\i}} `claw' is one of a small number of nouns that trigger an agreement pattern that does not fit any of the 5 genders: \textit{b-} in singular, \textit{j-} in plural.} Similarly, in (\ref{verbflex-ex45}), the verb `come' shows a plural and a singular form, depending on the Nominative subject.

\begin{exe}
	\ex\label{verbflex-ex01}
	\begin{xlist}
		
		
			\ex\label{verbflex-ex01a}
			\gll k'ot'i-v daxk'i-n mʕajr\u{\i} \textbf{oll-b-i-eⁿ}. \\
			cat.{\Obl}-{\Erg} mouse.{\Obl}-{\Dat} claw \textbf{put.{\Pfv}.{\Sg}-{\B}.{\Sg}-{\Tr}-{\Aor}} \\
			\trans `The cat puts its claw on the mouse.'
			\hfill (KK013-2885)
		
		
		
			\ex\label{verbflex-ex01b}
			\gll k'ot'i-v daxk'i-n mʕar-elč \textbf{oxk'-j-i-eⁿ}.    \\
			cat.{\Obl}-{\Erg} mouse.{\Obl}-{\Dat} claw-{\Pl} \textbf{put.{\Pfv}.{\Pl}-{\J}-{\Tr}-{\Aor}} \\
			\trans `The cat puts its claws on the mouse.'
			\hfill (KK013-2885)
		
	\end{xlist}
\end{exe}

\begin{exe}
	\ex\label{verbflex-ex45}
	\begin{xlist}
		
		
			\ex\label{verbderiv-ex45a}
			\gll mercxlaɁu-i so \textbf{d-axk'-eⁿ} j-apxe-č\u{o} kveq'n-a-xiⁿ.  \\
			swallow-{\Pl} hither \textbf{{\D}-come.{\Pfv}.{\Pl}-{\Aor}} {\J}-warm-{\Obl} country-{\Obl}.{\Pl}-{\Apudabl} \\
			\trans `The swallows are coming [back] from warm countries.'
			\hfill (E024-6)
		
		
		
			\ex\label{verbflex-ex45b}
			\gll qeⁿ laqši-\u{g}=da cħa kuir \textbf{d-a\u{g}-o}=e.  \\
			then high\_up-{\Trans}=from one hawk \textbf{{\D}-come.{\Ipfv}.{\Sg}-{\Npst}}=and \\
			\trans `Then a hawk comes down from high up.'
			\hfill (E179-36)
		
	\end{xlist}
\end{exe}



\pagebreak

\section{Synthetic TAME forms} \label{TAME}
The subsequent sections will give a form-to-function description of all finite synthetic (i.e. single-word) forms of Tsova-Tush verbs (see also \cites{chrelashvili84}[179--182]{holiskygagua}). The following subsections will be divided by mood: Indicative (\sectref{ind}), Imperative (\sectref{imp}), Subjunctive (\sectref{subj}) and Iamitive (\sectref{cont}). \sectref{evid} will provide a description of synthetic evidential forms. All Tsova-Tush finite verb forms are summarised in Appendix~\ref{appendix:a}, in \tabref{allverbflex}.


\subsection{Indicative} \label{ind}\is{Indicative}
Tsova-Tush exhibits four primary tenses, which are formed synthetically, are non-modal and are neutral in terms of evidentiality. They are the Non-Past (some verbs distinguish between a Present and a Future), the Imperfect, the Aorist, and the Remote Past, see Table \ref{table-indverbflex}.

\begin{table}
	\begin{tabular}{llll}
		\lsptoprule
		
		& {Morphemes} & {Glossing} & {Surface form} \\
		\midrule
		
		Non-Past (Present) & \textit{tet'-o} & cut.\textsc{ipfv-npst} & \textit{tet'\u{o}} \\
		Non-Past (Future) & \textit{tit'-o} & cut.\textsc{pfv-npst} & \textit{tit'\u{o}} \\
		
		Imperfect & \textit{tet'-ora} & cut.\textsc{ipfv-imprf} & \textit{tet'or} \\
		& \textit{tit'-ora} & cut.\textsc{pfv-imprf} & \textit{tit'or} \\
		
		Aorist  & \textit{tet'-in} & cut.\textsc{ipfv-aor} & \textit{tet'iⁿ} \\
		& \textit{tit'-en} & cut.\textsc{pfv-aor} & \textit{tit'eⁿ} \\
		
		Remote Past  & \textit{tet'-ira} & cut.\textsc{ipfv-rem} & \textit{tet'ir} \\
		& \textit{tit'-era} & cut.\textsc{pfv-rem} & \textit{tit'er} \\
		\lspbottomrule
	\end{tabular}
	\caption{Tsova-Tush basic indicative finite verb forms}
	\label{table-indverbflex}
\end{table}



\subsubsection{Present}\is{Non-Past tense}\is{Present tense}
\largerpage

The Tsova-Tush Present is formed by adding one of the five vowels to an Imperfective verbal stem (the vowel \textit{a} is only used for the verb `be' and the light verb which derives intransitives (see \sectref{intr})). Which verb chooses which vowel is determined lexically. The vowels are glossed as \textsc{npst} (Non-Past), since they are used to form the Future as well. Table \ref{TAME-table1} shows the formation of the Present (3rd person forms). Since the Non-Past vowels occur at the end of the word in 3rd person forms, regular vowel apocope and other phonological processes occur, for which see \sectref{processes} and \textcite{outtier99}.

\begin{table}
	\begin{tabular}{lll}
		\lsptoprule
		Morphemes & {Glossing} & {Surface form} \\
		\midrule
		\textit{d-∅-a} & \textsc{d}-be.\textsc{ipfv-npst}\footnote{Throughout this work, the glossing \textit{d-a} `\textsc{d}-be' is used.} & \textit{da}  \\
		\textit{lel-e} & walk.\textsc{ipfv-npst} & \textit{lel} \\
		\textit{lev-i} & speak.\textsc{ipfv-npst} & \textit{liv} \\
		\textit{ix-o} & pass.\textsc{ipfv-npst} & \textit{ix\u{o}} \\
		\textit{d-aɬ-u} & \textsc{d}-appear.\textsc{ipfv-npst} & \textit{dejɬ\u{u}} \\
		\lspbottomrule
	\end{tabular}
	\caption{Tsova-Tush Present formation}
	\label{TAME-table1}
\end{table}

The Tsova-Tush Present can be used to signal both durative and punctual events, as well as refer to general truths, see Example (\ref{verbflex-ex02}).\footnote{For Example (\ref{verbflex-ex02}), note that a noun referring to a human with unspecified gender receives gender D.} In order to convey a habitual meaning, Tsova-Tush uses an auxiliary verb \textit{latar} with an infinitive, not a simple Present. Note that many verbs do not distinguish between a Perfective and an Imperfective stem (see \sectref{rootperf}). For these verbs, the Present is homophonous with the Future.\is{Durative}\is{Punctual}\is{General truth}\is{Habitual}



\begin{exe}
	\ex\label{verbflex-ex02}
	
	
		\gll bned ix-oš lac'mar aħ \textbf{d-ebž-\u{e}}. \\
		epilepsy come.{\Ipfv}-{\Simul} patient down \textbf{{\D}-fall.{\Ipfv}-{\Npst}} \\
		\trans `A patient is falling down as an epileptic fit comes to him/her.' (durative) or:  \\ `As an epileptic fit comes to him/her, the patient falls down.' (punctual) or: \\ `When an epileptic fit comes to him/her, a patient falls down.' (general truth)
		\hfill (KK002-0540)
	
	
\end{exe}

A minority of verbs lose their gender marker and first vowel in the Present/Future (and TAME forms based on this stem). Examples are \textit{d-alar} `give', \textsc{prs/fut} \textit{lo}; \textit{d-alar} `die', \textsc{prs/fut} \textit{la}; \textit{d-agar} `see', \textsc{prs/fut} \textit{gu}; \textit{d-axar} `go.{\Pfv}', \textsc{fut} (suppletive) \textit{\u{g}o}. All verbs that have the detransitive derivational suffix \textit{-d-al} (see \sectref{intr}) have a Present/Future in \textit{la}, e.g. \textit{d-epx-d-alar} `heat up ({\Intr}.{\Ipfv})', {\Prs} \textit{depxla}.

\subsubsection{Future}\is{Future tense}
The Tsova-Tush Future is formed by adding one of four vowels (\textit{e, i, o, u}) to a Perfective verbal stem. Table \ref{TAME-table2} shows several examples of the Future (3rd person forms). 

\begin{table}
	\begin{tabular}{lll}
		\lsptoprule
		Morphemes & {Glossing} & {Surface form} \\\midrule
		\textit{sart'-e} & curse.\textsc{pfv-npst} & \textit{sart'} \\
		\textit{aɬ-o} & speak.\textsc{pfv-npst} & \textit{aɬ\u{o}} \\
		\textit{xiɬ-u} & be.\textsc{pfv-npst} & \textit{xiɬ\u{u}} \\
		\lspbottomrule
	\end{tabular}
	\caption{Tsova-Tush Future formation}
	\label{TAME-table2}
\end{table}

Many verbs do not distinguish between a Perfective and an Imperfective stem (see \sectref{rootperf}). For these verbs, the Present is homophonous with the Future (see Example (\ref{verbflex-ex03})).  Example (\ref{verbflex-ex04}) illustrates the use of the Tsova-Tush Future.

\begin{exe}
	\ex\label{verbflex-ex03}
	\gll tung xi-x \textbf{j-uic'-\u{u}.}  \\
	flagon water-{\Cont} {\J}-\textbf{be\_filled-{\Npst}} \\
	\trans `The flagon will be filled with water.'  Or:  `The flagon is being filled with water.'
	\hfill (KK021-3735)

	\ex\label{verbflex-ex04}
	\gll ʕunal e \textbf{xiɬ-\u{u}} ħo-goħ=ejn. \\
	salary {\Prox} \textbf{be.{\Pfv}-{\Npst}} {\Ssg}-{\Adess}={\Quot} \\
	\trans `This is the salary you will have (they said).'
	\hfill (E113-20)
\end{exe}

\subsubsection{Imperfect}\is{Imperfect past tense}

The Tsova-Tush Imperfect has historically been formed by adding a suffix \textit{-ra} to the stem with a Non-Past vowel. From a synchronic point of view, the Imperfect suffixes are thus \textit{-ara, -era, -ira, -ora, -ura}. Table \ref{TAME-table3} shows the formation of the Imperfect for selected verbs. 

\begin{table}
	\begin{tabular}{lll}
		\lsptoprule
		Morphemes & {Glossing} & {Surface form} \\\midrule
		
		\textit{d-∅-ara}\footnote{Throughout this work, the conventional glossing \textit{d-a-ra} `\textsc{d}-be.\textsc{ipfv-imprf}' is used.} & \textsc{d}-be.\textsc{ipfv-imprf} & \textit{dar} \\
		\textit{lel-era} & walk.\textsc{ipfv-imprf} & \textit{leler} \\
		\textit{lev-ira} & speak.\textsc{ipfv-imprf} & \textit{levir} \\
		\textit{ix-ora} & pass.\textsc{ipfv-imprf} & \textit{ixor} \\
		\textit{d-aɬ-ura} & \textsc{d}-appear.\textsc{ipfv-imprf} & \textit{daɬur} \\\midrule
		\textit{sart'-era} & curse.\textsc{pfv-imprf} & \textit{sart'er} \\
		\textit{aɬ-ora} & speak.\textsc{pfv-imprf} & \textit{aɬor} \\
		\textit{xiɬ-ura} & be.\textsc{pfv-imprf} & \textit{xiɬur} \\
		\lspbottomrule
	\end{tabular}
	\caption{Tsova-Tush Imperfect formation}
	\label{TAME-table3}
\end{table}


The Tsova-Tush Imperfect can be formed from both Imperfective stems (above the line in Table \ref{TAME-table3}) and Perfective stems (below the line), and is used to convey events in the past tense with durative aspect.\is{Durative} This aspectual difference between Imperfective and Perfective past tense forms has been defined in terms of\is{Pluractionality} pluractionality (\cite[179]{holiskygagua}). See Example (\ref{verbflex-ex46}), featuring an Imperfective verb stem \textit{d-aq'-} `eat', which is used to refer to multiple events of eating. Compare this with Example (\ref{verbflex-ex47}), which uses a verb root \textit{qall-} `eat', the Perfective counterpart to Imperfective \textit{d-aq'-}. The verb forms in (\ref{verbflex-ex47}), however, are still inflected for the Imperfect TAME form. We must therefore interpret these clauses as having past tense, durative aspect, but referencing a single action. Research into the exact parameters of pluractionality in Tsova-Tush, such as combinability with certain adverbs, remains outstanding.

\begin{exe}
	\ex\label{verbflex-ex46}
	\begin{xlist}
		
		
			\ex\label{verbflex-ex46a}
			\gll babo-s lebiv k'os-ren=daħ \textbf{b-aq'-or}. \\
			grandfather-{\Erg} beans large\_bowl-{\Abl}=from \textbf{{\B}.{\Sg}-eat.{\Ipfv}-{\Imprf}} \\
			\trans `Grandpa was eating beans out of a big bowl (multiple times).' \\
			\hfill (KK011-2115)
		
		
		
			\ex\label{verbflex-ex46b}
			\gll oqar \textbf{d-aq'-or}=a meɬ-or, duqeč xan-e d-ax-er. \\
			{\Dist}.{\Pl}.{\Erg} \textbf{{\D}-eat.{\Ipfv}-{\Imprf}}={\Add} drink.{\Ipfv}-{\Imprf} long.{\Obl} time-{\Obl}({\Ess}) {\D}-live-{\Imprf}  \\
			\trans `And they ate and drank and lived for a long time.' [i.e. `happily ever after']
			\hfill (WS001-14.10)
		
		
	\end{xlist}
\end{exe}

\begin{exe}
	\ex\label{verbflex-ex47}
	\begin{xlist}
		
		
			\ex\label{verbflex-ex47a}
			\gll xan-c'-ol-i co j-a-ħer, cħan-\u{g} k'ac'k'oⁿ majq\u{i} \textbf{qall-or} vaj. \\
			time-{\Priv}-{\Abstr}-{\Pl} {\Neg} {\J}-be-{\Cond}.{\Pst} one.{\Obl}-{\Trans} small bread \textbf{eat.{\Pfv}-{\Imprf}} {\Fpl}.{\Incl} \\
			\trans `If we didn't have such a lack of time, we could be eating a bit of bread together.' [i.e. for a while, but in one sitting]
			\hfill (KK032-5111)
		
		
		
			\ex\label{verbflex-ex47b}
			\gll daħ d-ik'-oer čujx\u{i}, ħalo=o \textbf{qall-or}. \\
			{\Pv} {\D}-take({\Anim})-{\Imprf}.{\Seq} lamb {\Pv}={\Add} \textbf{eat.{\Pfv}-{\Imprf}} \\
			\trans `It [a bear] took the lamb and ate it.' [i.e. for a while, but in one sitting]
			\hfill (MM215-1.14)
		
		
	\end{xlist}
\end{exe}




\subsubsection{Aorist}\is{Aorist}\is{Preterite}\is{Simple past tense}

The Tsova-Tush Aorist (also called the preterite, or simple past) is formed by suffixing \textit{-en} or \textit{-in} to the verbal root.\footnote{As discussed in \sectref{processes}, word-final nasal stops drop, nasalising the preceding vowel.} Which of the two suffixes is taken by a given verbal root is determined lexically. See Table \ref{TAME-table5} for the Aorist formation of selected verbs. There is no strict correlation between the choice of the Non-Past suffix and the Aorist suffix.

\begin{table}
\begin{tabular}{llll}
	\lsptoprule
	Morphemes & {Glossing} & {Surface form} & Compare \textsc{npst}-vowel\\
	\midrule
	\textit{lel-in} & walk.\textsc{ipfv-aor} & \textit{leliⁿ} & \textit{-e} \\
	\textit{lev-in} & speak.\textsc{ipfv-aor} & \textit{leviⁿ} & \textit{-i} \\
	\textit{ix-en} & pass.\textsc{ipfv-aor} & \textit{ixeⁿ} & \textit{-o} \\
	\textit{d-aɬ-en} & \textsc{d}-appear.\textsc{ipfv-aor} & \textit{daɬeⁿ} & \textit{-u} \\
	\midrule
	\textit{sart'-in} & curse.\textsc{pfv-aor} & \textit{sart'iⁿ} & \textit{-e} \\
	\textit{aɬ-in} & speak.\textsc{pfv-aor} & \textit{aɬiⁿ} & \textit{-o} \\
	\textit{xiɬ-en} & be.\textsc{pfv-aor} & \textit{xiɬeⁿ} & \textit{-u} \\
	\lspbottomrule
\end{tabular}
\caption{Tsova-Tush Aorist formation}
\label{TAME-table5}
\end{table}

The Aorist expresses a default, non-durative past tense. An Aorist formed from a Perfective stem expresses a single-action event (Example (\ref{verbflex-ex09a})), whereas one formed from an Imperfective root is pluractional (\ref{verbflex-ex09b}).

\begin{exe}
	\ex\label{verbflex-ex09}
	\begin{xlist}
		
		
			\ex\label{verbflex-ex09a}
			\gll qen=geɁ vedr \textbf{d-aɬ-iⁿ}, aħ d-ax-en, oqu-s xi maɬ-en=e. \\
			then={\Emph} bucket \textbf{{\D}-give.{\Pfv}-{\Aor}} down {\D}-go.{\Pfv}-{\Aor}({\Seq}) {\Dist}.{\Obl}-{\Erg} water drink.{\Pfv}-{\Aor}=and \\
			\trans `Then they gave him a bucket, he went down and drank some water, and [...].'
			\hfill (E153-73)
		
		
		
		
		
			\ex\label{verbflex-ex09b}
			\gll oqu-s ču v-ik'-e=s\u{o}, majq\u{\i} \textbf{teɬ-iⁿ} soⁿ, ħal v-ik'-en=s\u{o} dukn-i, navt so ec-iⁿ soⁿ. \\
			{\Dist}.{\Obl}-{\Erg} {\Pv} {\M}.{\Sg}-bring-{\Aor}={\Fsg}.{\Nom} bread \textbf{give.{\Ipfv}-{\Aor}} {\Fsg}.{\Dat} {\Pv} {\M}.{\Sg}-bring-{\Aor}={\Fsg}.{\Nom} inn-{\Ill} petrol {\Pv} buy.{\Pfv}-{\Aor} {\Fsg}.{\Dat} \\
			\trans `He took me along, fed me (repeatedly), brought me to an inn, and bought me some petrol.'
			\hfill (EK058 3.15)
		
		
	\end{xlist}
\end{exe}



\subsubsection{Remote Past}\is{Remote Past tense}

The Tsova-Tush Remote Past has historically been formed by adding a suffix \textit{-ra} to the Aorist stem. From a synchronic point of view, the Remote Past suffixes are \textit{-era, -ira}. The full use of the Remote Past is hitherto poorly understood, but the examples under  (\ref{verbflex-ex10}) suggest a past-before-past tense. 

\begin{exe}
	\ex\label{verbflex-ex10}
	\begin{xlist}
		
		
			\ex\label{verbflex-ex10a}
			\gll qeⁿ moħ-e dada-s \textbf{aɬ-ir}, b-axk'-eⁿ qo vašo lax-aⁿ cħan-\u{g} o k'roč', gan\u{g}aɁ co xet-iⁿ.\\
			then how-{\Rel} father.{\Obl}-{\Erg} \textbf{say.{\Pfv}-{\Rem}} {\M}.{\Pl}-come.{\Pfv}.{\Pl}-{\Aor} three brother find.{\Pfv}-{\Inf} one.{\Obl}-{\Trans} {\Dist} chest indeed {\Neg} find-{\Aor}\\
			\trans `Then, as their father had told them, the three brothers went together to look for the chest, and indeed they didn't find it.'
			\hfill (AS009-1.7)
		
		
		
		
		
			\ex\label{verbflex-ex10b}
			\gll gviaⁿ j-ox-j-i-er, macn-e c'in šenbad-j-∅-or. \\
			late {\J}-destroy-{\J}-{\Tr}-{\Rem} when-{\Rel} new build-{\J}-{\Tr}-{\Imprf}\\
			\trans `Later they (had) destroyed it, when they were building the new [school].'
			\hfill (E090-8)
		
		
	\end{xlist}
\end{exe}



\subsection{Imperative} \label{imp}\is{Imperative mood}

Tsova-Tush features various verb forms related to commands, suggestions, exhortations and wishes. The Hortative is always inclusive plural (although a distinction between two or more participants is made). An exclusive plural hortative is not attested, whereas a singular hortative or voluntative meaning is expressed by the Non-Past Subjunctive. A Simple Imperative, Polite Imperative and an Optative are found, but no dedicated jussive. The Imperative verb forms are characterised by allowing the plural suffix \textit{-t} (see \sectref{suffixpl}). Note that the Perfective/Imperfective aspect of the verbal root has been poorly investigated in non-indicative forms. Therefore, the exact semantics (pluractional or otherwise) of this aspectual distinction is unknown.

\begin{table}[h]
\begin{tabular}{llll}
\lsptoprule
	& Morphemes & {Glossing} & {Surface form} \\\midrule
	\multicolumn{4}{l}{Hortative} \\
	  & \textit{tet'-o(-t) vej} & cut.\textsc{ipfv-npst(-pl) 1pl.incl} & \textit{tet'\u{o} vej / tet'ot vej} \\
	  & \textit{tit'-o(-t) vej} & cut.\textsc{pfv-npst(-pl) 1pl.incl} & \textit{tit'\u{o} vej / tit'ot vej} \\
\midrule
	\multicolumn{4}{l}{Simple Imperative} \\
	 & \textit{tet'-a(-t)} & cut.\textsc{ipfv-imp(-pl)} & \textit{tet' / tet'at} \\
	 & \textit{tit'-a(-t)} & cut.\textsc{pfv-imp(-pl)} & \textit{tit' / tit'at} \\
	\multicolumn{4}{l}{Polite Imperative} \\
	 & \textit{tet'-a-le(-t)} & cut.\textsc{ipfv-imp-pol(-pl)} & \textit{tet'al / tet'alet} \\
	 & \textit{tit'-a-le(-t)} & cut.\textsc{pfv-imp-pol(-pl)} & \textit{tit'al / tit'alet} \\
	\multicolumn{4}{l}{Optative} \\
	 & \textit{tet'-a-la(-t)} & cut.\textsc{ipfv-imp-opt(-pl)} & \textit{tet'al / tet'alat} \\
	 & \textit{tit'-a-la(-t)} & cut.\textsc{pfv-imp-opt(-pl)} & \textit{tit'al / tit'alat} \\
\lspbottomrule
\end{tabular}
\caption{Tsova-Tush imperative verb forms}
\label{table-impverbflex}
\end{table}



\subsubsection{Hortative}\is{Hortative}\is{Imperative mood!Hortative}

The Hortative is formed by a verb in the Present or Future, and postposing the inclusive 1st person plural pronoun. This construction results in the meaning `Let's verb (you and me)', see Example (\ref{verbflex-ex05a}). By adding a plural marker \textit{-t} to the verb stem, more than two people are implied. The Hortative can be formed from both Imperfective (\ref{verbflex-ex05b}) and Perfective (\ref{verbflex-ex05c}) stems. Note that \textit{\u{go}} `go' is a verb that loses its gender marker and first vowel in the Present/Future, as described above, under the Present tense.

\begin{exe}
	\ex\label{verbflex-ex05}
	\begin{xlist}
		
		\ex\label{verbflex-ex05a}
		\gll gana čav, \textbf{\u{g}-o} \textbf{vaj}, mič-e b-ixk'-n-a don. \\
		\textsc{interjection}	 Chava, \textbf{go.{\Pfv}-{\Npst}} \textbf{{\Fpl}.{\Incl}} where-{\Rel}({\Ess}) {\B}.{\Sg}-tie.{\Ipfv}-{\Aor}-{\Ssg}.{\Erg} horse \\
		\trans `Alright Chava, let's go to where you tied your horse.'
		\hfill (E022-9)
		
		\ex\label{verbflex-ex05b}
		\gll o-bi co d-a\u{g}-o-ħe, ħa \textbf{ix-o-t} \textbf{vaj}. \\
		{\Dist}-{\Pl} {\Neg} {\D}-come.{\Ipfv}-{\Npst}-{\Cond} then \textbf{go.{\Ipfv}-{\Npst}-{\Pl}} \textbf{{\Fpl}.{\Incl}} \\
		\trans `When they're not coming, then let's leave.'
		\hfill (KK036-5534)
		
		\ex\label{verbflex-ex05c}
		\gll gari inc ču \textbf{xabž-u-t} \textbf{vej} - mejq \textbf{qaɬ-o-t} \textbf{vej}=en. \\
		\textsc{interjection} now {\Pv} \textbf{sit\_down.{\Pfv}.{\Pl}-{\Npst}-{\Pl}} \textbf{{\Fpl}.{\Incl}} {} bread \textbf{eat.{\Pfv}-{\Npst}-{\Pl}} \textbf{{\Fpl}.{\Incl}={\Quot}} \\
		\trans `Well, let's sit down now and eat some bread (they said).'
		\hfill (E179-23)
		
	\end{xlist}
\end{exe}

\subsubsection{Simple Imperative}\is{Imperative mood!Simple Imperative}

The Tsova-Tush Simple Imperative is formed by suffixing \textit{-a} to a verbal root. Since \textit{-a} will undergo apocope without a trace when it is the last segment of a word (see Example (\ref{verbflex-ex06a})\footnote{Original orthography of (\ref{verbflex-ex06a}): \'{t}it x̣el, aḽ, ḥain bstu ja ḥo?}), the \textit{-a} vowel will only surface when it is followed by other phonetic material in the same word, such as a plural marker (Example (\ref{verbflex-ex06b})), a politeness marker, or an enclitic. The Imperative can be formed from both Imperfective and Perfective stems, and is used to express requests, demands, and orders.

\begin{exe}
	\ex\label{verbflex-ex06}
	\begin{xlist}
		
		\ex\label{verbflex-ex06a}
		\gll \textbf{tit'} qel, \textbf{aɬ}, ħain pst'u j-a ħo? \\
		\textbf{cut.{\Pfv}({\Imp})} custom \textbf{say.{\Pfv}({\Imp})} who.{\Gen} wife {\F}.{\Sg}-be {\Ssg} \\
		\trans `Decide, tell us, whose wife are you?'
		\hfill (AS008-10.7)
		
		\ex\label{verbflex-ex06b}
		\gll mott \textbf{ot'-a-t} ču, equj-n bed \textbf{ot'-a-t} aɬ-in=e. \\
		bed \textbf{spread.{\Pfv}-{\Imp}-{\Pl}} {\Pv} {\Prox}.{\Obl}-{\Dat} separately \textbf{spread.{\Pfv}-{\Imp}-{\Pl}} say.{\Pfv}-{\Aor}=and \\
		\trans `Make your beds, make (a bed) separately for her, they said.'
		\hfill (E179-86)
		
	\end{xlist}
\end{exe}

A minority of Perfective verbs form the Imperative using a vowel + \textit{ba}. This vowel may or may not be the same as the vowel used to form the Present/Future. Table \ref{TAME-table4} gives examples of Imperative formation of Imperfective stems, of Perfective stems, and of Perfective stems irregularly using the \textit{-ba} suffix.

\begin{table}
\begin{tabular}{llll}
\lsptoprule
	Morphemes & Glossing & Surface form & Compare \textsc{npst}-vowel\\
\midrule
	\textit{lel-a} & walk.\textsc{ipfv-imp} & \textit{lel} & \textit{-e} \\
	\textit{lev-a} & speak.\textsc{ipfv-imp} & \textit{lev} & \textit{-i}\\
	\textit{ix-a} & pass.\textsc{ipfv-imp} & \textit{ix} & \textit{-o}\\
	\textit{d-aɬ-a} & \textsc{d}-appear.\textsc{ipfv-imp} & \textit{daɬ} & \textit{-u} \\
\midrule
	\textit{sart'-a} & curse.\textsc{pfv-imp} & \textit{sart'} & \textit{-e} \\
	\textit{aɬ-a} & speak.\textsc{pfv-imp} &  \textit{aɬ} & \textit{-o} \\
	\textit{xiɬ-a} & be.\textsc{pfv-imp} & \textit{xiɬ} & \textit{-u }\\
\midrule
	\textit{d-i-eba} & \textsc{d}-do\textsc{.pfv-imp} & \textit{deb}/\textit{dib} & \textit{-o} \\
	\textit{l-iba}\footnote{Note that \textit{d-aɬar / lo} `give', \textit{\u{g}o} `go' and \textit{d-aħar / ħo} `bring, take' are verbs that lose their gender marker and first vowel in the Present, Future and Imperfect.} & give.\textsc{pfv-imp} & \textit{lib} & \textit{-o} \\
	\textit{\u{g}-oba} & go.\textsc{pfv-imp} & \textit{\u{g}ob} & \textit{-o} \\
	\textit{ħ-oba} & bring.\textsc{pfv-imp} & \textit{ħob} & \textit{-o} \\ 
\lspbottomrule
\end{tabular}
\caption{Tsova-Tush Imperative formation}
\label{TAME-table4}
\end{table}

As seen in Example (\ref{verbflex-ex06b}), the Imperative can receive a plural marker \textit{-t}, as is true for the Polite Imperative and the Optative.

\subsubsection{Polite Imperative}\is{Imperative mood!Polite Imperative}
\largerpage[2]

The Polite Imperative is formed by adding the suffix \textit{-le} to an Imperative stem, and is used to express polite requests. The suffix \textit{-t} is added to indicate a plural addressee.

\begin{exe}
	\ex\label{verbflex-ex07}
	\begin{xlist}
		
		\ex\label{verbflex-ex07a}
		\gll gariel cħa c'in ambui \textbf{j-epc-a-l} equi-g=en. \\
		{\Hort} one new story \textbf{{\J}-tell.{\Ipfv}-{\Imp}-{\Pol}} {\Prox}.{\Obl}-{\All}={\Quot} \\
		\trans `Come on, please tell him one new story.'
		\hfill (E182-225)
		
		\ex\label{verbflex-ex07b}
		\gll  buis\u{u} \textbf{gag-d-}∅\textbf{-eb-le-t} so=en, ču d-ik'-eⁿ. \\
		at\_night \textbf{care\_for-{\D}-{\Tr}-{\Imp}-{\Pol}-{\Pl}} {\Fsg}.{\Nom}={\Quot} in {\D}-take.{\Anim}-{\Aor} \\
		\trans `“Please host me overnight,” [said the fox], and they took him in.' \\
		\hfill (E153-25)
		
	\end{xlist}
\end{exe}

\subsubsection{Optative}\is{Imperative mood!Optative}\is{Optative}

The Optative is formed by adding the suffix \textit{-la} to the Imperative stem, and is used for wishes, curses and blessings (\cite[181--182]{holiskygagua}). It seems to be available for all persons. Examples include the first person (inclusive) in Example (\ref{verbflex-ex08a}), second person (\ref{verbflex-ex08b}) and the third person (\ref{verbflex-ex08c}). Note that the Optative and the Polite Imperative are homophonous in the singular.




\begin{exe}
	\ex\label{verbflex-ex08}
	\begin{xlist}
		
			\ex\label{verbflex-ex08a}
	\gll šar-e samotx-i \textbf{d-}∅\textbf{-eba-la-t} ve=en. \\
	year-{\Obl}({\Ess}) paradise-{\Pl} \textbf{{\D}-do.{\Pfv}-{\Imp}-{\Opt}-{\Pl}} {\Fpl}.{\Incl}={\Quot} \\
	\trans `This year, let's perform “Paradises” (they said).'
	\hfill (E225-17)
		

		
		\ex\label{verbflex-ex08b}
		\gll ğaz-iš \textbf{xiɬ-a-la-ħo}=e duq-xan-e \textbf{j-ax-a-la-ħ}. \\
		good-{\Adv} \textbf{be.{\Pfv}-{\Imp}-{\Opt}-{\Ssg}.{\Nom}}=and much-time-{\Obl}({\Ess}) \textbf{{\F}.{\Sg}-live-{\Imp}-{\Opt}-{\Ssg}.{\Nom}}\\
		\trans `May you be well and live a long time.'
		\hfill (\cite[182]{holiskygagua})
		
		\ex\label{verbflex-ex08c}
\gll ħo-x bedeⁿ mama \textbf{tox-a-l} soⁿ. \\
{\Ssg}-{\Cont} except nobody.{\Proh} \textbf{hit.{\Pfv}-{\Imp}-{\Opt}} {\Fsg}.{\Dat} \\
\trans `May nobody hit me except for you.'
\hfill (E249-71)		



		
	\end{xlist}
\end{exe}


\subsection{Subjunctive} \label{subj}\is{Subjunctive}

Tsova-Tush features two Subjunctive TAME forms, i.e. forms with the primary function of finite subordination, formed by the suffix \textit{-lo}. The Tsova-Tush Subjunctive is analysed as finite since (1) it combines with person indexing markers, (2) it can be used in matrix clauses, (3) subordinate clauses that feature Subjunctive verbs are introduced by conjunctions.
Finite subordination is rare in East Caucasian (most subordination is non-finite), but subjunctive forms do occur in Chechen (\cite{komen}) and Ingush (\cites[289]{nichols11}), which both feature suffixes containing an \textit{-l-}, although there is no complete cognacy.\footnote{The Ingush Subjunctive is in \textit{-algʲa}, and the Chechen in \textit{-ila}.} Georgian also features several Subjunctive forms, the one functionally closest to the Tsova-Tush being called 2nd Subjunctive, or Optative in most grammars. An in-depth comparison between the use of the several subjunctives in Tsova-Tush, Georgian, Chechen and Ingush falls beyond the scope of this work, but see \sectref{purp} for a side-by-side comparison of Georgian and Tsova-Tush purpose clauses, both featuring a form labelled Subjunctive.


\begin{table}
	\begin{tabular}{llll}
		
		\lsptoprule
		& Morphemes & Glossing & Surface form \\\midrule
		
		
		Non-Past Subjunctive & \textit{tet'-o-lo} & cut.\textsc{ipfv-npst-sbjv} & \textit{tet'ol\u{o}} \\
		& \textit{tit'-o-lo} & cut.\textsc{pfv-npst-sbjv} & \textit{tit'ol\u{o}} \\
		
		Past Subjunctive & \textit{tet'-ora-lo} & cut.\textsc{ipfv-imprf-sbjv} & \textit{tet'ral\u{o}} \\
		& \textit{tit'-ora-lo} & cut.\textsc{pfv-imprf-sbjv} & \textit{tit'ral\u{o}} \\
		
		\lspbottomrule
		
	\end{tabular}
	\caption{Tsova-Tush Subjunctive verb forms}
	\label{table-subjverbflex}
\end{table}


\subsubsection{Non-Past Subjunctive}\is{Dubitative}

The Present Subjunctive is formed by adding the suffix \textit{-lo} to the Present, while the Future Subjunctive is formed by adding the same suffix to the Future. In matrix clauses, both forms are used to express uncertainty: in interrogative sentences, they express a meaning  `I wonder' (\ref{verbflex-ex11a}, \ref{verbflex-ex11b}), often with an indefinite particle \textit{=ak'} elsewhere in the clause. In declarative sentences, however, Non-Past Subjunctive forms convey a sense of longing: `if only' (\ref{verbflex-ex11c}).\footnote{In Georgian, both the meaning `I wonder' and `if only' are captured with the particle \textit{net'a(vi)}.}

\begin{exe}
	\ex\label{verbflex-ex11}
	\begin{xlist}
		
		\ex\label{verbflex-ex11a}
		\gll e qer=ak' ħan-n \textbf{b-a-l} b-iɬ-eno? \\
		{\Prox} stone={\Indf} who.{\Obl}-{\Dat} \textbf{{\B}.{\Sg}-be-{\Sbjv}} {\B}.{\Sg}-put.{\Pfv}-{\Ptcp}.{\Pst} \\
		\trans `(I wonder) for whom this stone was laid down?'
		\hfill (E058-13)
		
		\ex\label{verbflex-ex11b}
		\gll so co j-aɬ-mak'=e, wux=k' \textbf{d-}∅\textbf{-o-lo-s}. \\
		hither {\Neg} {\F}.{\Sg}-go\_out-{\Pot}({\Npst})=and what={\Indf} \textbf{{\D}-do.{\Pfv}-{\Npst}-{\Sbjv}-{\Fsg}.{\Erg}} \\
		\trans `She cannot come out, what(ever) should I do?'
		\hfill (E179-35)
		
		\ex\label{verbflex-ex11c}
		\gll ħal k'i \textbf{aɬ-lo-s} ħo-g=uin.  \\
		{\Pv} {\Contr} \textbf{say.{\Pfv}({\Npst})-{\Sbjv}-{\Fsg}.{\Erg}} {\Ssg}-{\All}={\Quot}  \\
		\trans `If only I could say it to you (s/he said).'
		\hfill (E175-33)
		
	\end{xlist}
\end{exe}

In subordinate clauses with the conjunction \textit{me}, Non-Past Subjunctive forms are used to form purpose clauses, as discussed in \sectref{purp}. See Example (\ref{verbflex-ex12}), which illustrates the Subjunctive in a subordinate clause.\is{Purpose clauses}


	\begin{exe}
		\ex\label{verbflex-ex12}
		\gll  čuxu-j šuiⁿ nan-i-goreⁿ čaq d-ec xiɬ-aⁿ, me vašbaⁿ daħ \textbf{d-ic-d-}∅\textbf{-o-l\u{o}}. \\
		lamb-{\Pl} {\Refl}.{\Poss}.{\Pl} mother-{\Pl}-{\Adabl} far {\D}-must be.{\Pfv}-{\Inf} {\Subord} {\Recp} {\Pv} \textbf{{\D}-forget-{\D}-{\Tr}-{\Npst}-{\Sbjv}} \\
		\trans `The lambs must be apart from their mothers so that they forget each other.'
		\hfill (E002-35)
	\end{exe}


Note that, as with imperatives, further investigations into the Perfective/Imperfective aspect of the verbal stem with subjunctive forms are beyond the scope of this work. Therefore, the exact semantics (pluractional or otherwise) of this aspectual distinction is unknown. 




\subsubsection{Past Subjunctive}

The Past Subjunctive is formed by adding the suffix \textit{-lo} to the Imperfect, and presents the same semantics as the Present/Future Subjunctive: (1) a dubitative in question sentences (\ref{verbflex-ex13a}), (2) a potential in declarative sentences (\ref{verbflex-ex13b}), and (3) a purposive in subordinate clauses with the conjunction \textit{me} (\ref{verbflex-ex13c}). 

\begin{exe}
	\ex\label{verbflex-ex13}
	\begin{xlist}
		
		\ex\label{verbflex-ex13a}
		\gll ipsi cħa-\u{g}=i \textbf{b-a-ra-l}?  \\
		{\Med}.{\Pl} one-{\Trans}={\Q} \textbf{{\M}.{\Pl}-be-{\Imprf}-{\Sbjv}} \\
		\trans `I wonder whether they were together?'
		\hfill (E242-25)
		
		\ex\label{verbflex-ex13b}
		\gll madel moɁ \textbf{b-a-ra-l} c'q'e t'q'o=a, t'batina bħark'-e-n j-ag-an=a. \\
		blessing just \textbf{{\B}.{\Sg}-be-{\Imprf}-{\Sbjv}} once again={\Emph} Tbatana eye-{\Obl}-{\Dat} {\J}-see-{\Inf}={\Emph}\\
		\trans `It would be a real blessing to see Tbatana once again.'
		\hfill (E058-27)
		
		\ex\label{verbflex-ex13c}
		\gll  nažt'r-e-n gargax qel-or xolme me, buisev daħ co \textbf{aħ-ra-l\u{o}} q'ač\u{g}u-i-v. \\
		shack-{\Obl}-{\Dat} near bring-{\Imprf} {\Hab} {\Subord} at\_night {\Pv} {\Neg} \textbf{steal.{\Ipfv}-{\Imprf}-{\Sbjv}} brigand-{\Pl}-{\Erg} \\
		\trans `They used to bring them near their houses, so that the bandits would not steal them.'
		\hfill (E008-17)
		
	\end{xlist}
\end{exe}

However, the most common use of the Past Subjunctive is to express an unwitnessed past tense, as discussed in \sectref{evid}.


\subsection{Conditional} \label{cond}\is{Conditional clauses}

Tsova-Tush features four Conditional verb forms, which can be either Past or Non-Past, and either Witnessed or Non-Witnessed. The Non-Witnessed Remote Conditional is the primary means to express a counterfactual conditional clause. Within Tsova-Tush, the  Conditional verb forms are best analysed as being finite, as (1) they co-occur with the morph \textit{-ra}, which, outside of the Conditional, only combines with finite forms, and (2) they can co-occur with person marking, which none of the non-finite forms (converbs, participles, Verbal Noun, Infinitive, all described in \sectref{nonfinite}) can. Thus, morphologically, these verb forms behave like other finite forms presented in this chapter, but syntactically, they are converbs, as they are the head of an adjunct clause that does not contain a conjunction.

\begin{table}
	\begin{tabular}{llll}
		\lsptoprule
		& {Morphemes} & {Glossing} & {Surface form} \\
		\midrule
		\multicolumn{4}{l}{Non-Past Conditional}\\\
		& \textit{tet'-o-ħe} & cut.\textsc{ipfv-npst-cond} & \textit{tet'oħ} \\
		& \textit{tit'-o-ħe} & cut.\textsc{pfv-npst-cond} & \textit{tit'oħ} \\
		
		\multicolumn{4}{l}{Past Conditional}\\
		& \textit{tet'-oħera} & cut.\textsc{ipfv-cond.pst} & \textit{tet'oħer} \\
		& \textit{tit'-oħera} & cut.\textsc{pfv-cond.pst} & \textit{tit'oħer} \\
		
		\multicolumn{4}{l}{Non-Witnessed Aorist Cond.}\\
		& \textit{tet'-ino-ħe} & cut.\textsc{ipfv-nw.aor-cond} & \textit{tit'noħ} \\
		& \textit{tit'-eno-ħe} & cut.\textsc{pfv-nw.aor-cond} & \textit{tit'noħ} \\
		
		\multicolumn{4}{l}{Non-Witnessed Remote Cond.}\\
		& \textit{tet'-inoħera} & cut.\textsc{ipfv-nw.rem.cond} & \textit{tit'noħer} \\
		& \textit{tit'-enoħera} & cut.\textsc{pfv-nw.rem.cond} & \textit{tit'noħer} \\
		\lspbottomrule
	\end{tabular}
	\caption{Tsova-Tush Conditional verb forms}
	\label{table-condverbflex}
\end{table}

\subsubsection{Non-Past Conditional}

The Non-Past Conditional is formed by adding the suffix \textit{-ħe} to the Present/Future stem.\footnote{Original orthography of (\ref{verbflex-ex14b}): Халахетинойахь с\~{e} йаш\u{o}, алъал окхуиго хала ма хетал.} It is used to form conditional adjunct clauses. For a full discussion of conditional clauses, see \sectref{condclause}.

\begin{exe}
	\ex\label{verbflex-ex14}
	\begin{xlist}
		
		\ex\label{verbflex-ex14a}
		\gll o-bi co \textbf{d-a\u{g}-o-ħe}, ħa ix-o-t vaj. \\
		{\Dist}-{\Pl} {\Neg} \textbf{{\D}-come.{\Ipfv}-{\Npst}-{\Cond}} then go.{\Ipfv}-{\Npst}-{\Pl} {\Fpl}.{\Incl} \\
		\trans `If they're not coming, let's go.'
		\hfill (KK036-5534)
		
		\ex\label{verbflex-ex14b}
		\gll xala-xet-ino \textbf{j-a-ħ} seⁿ jaš\u{o}, aɬ-a-l oqui-go xala ma xet-a-l. \\
		harmful-deem-{\Ptcp}.{\Pst} \textbf{{\F}.{\Sg}-be-{\Cond}} {\Fsg}.{\Gen} sister speak.{\Pfv}-{\Imp}-{\Pol} {\Dist}.{\Obl}-{\All} harmful {\Proh} deem-{\Imp}-{\Pol} \\
		\trans `If your sister is offended, please tell her not to be offended.' \\
		\hfill (YD001-49.1)
		
	\end{xlist}
\end{exe}

\subsubsection{Past Conditional}

The Past Conditional is historically formed by adding the suffix \textit{-ra} to the Non-Past Conditional. In this work, the entire morpheme complex \textit{-V-ħe-ra} will be glossed \textsc{cond.pst}. The Past Conditional is used in conditional clauses that are shifted to the past tense, such as in (\ref{verbflex-ex15})\footnote{Original orthography of (\ref{verbflex-ex15a}): Б\'{у}итıбагıчуишв дахьохьер бéд\~{e} стıéин (вух) барле х\'{и}лъ\u{у} цо хеэр.}, and hence acquire temporal, rather than conditional semantics (and can be translated with `whenever').



\begin{exe}
	\ex\label{verbflex-ex15}
	\begin{xlist}
		
		\ex\label{verbflex-ex15a}
		\gll b-uit'-b-a\u{g}-ču-iš-v \textbf{d-aħ-oħer} bedeⁿ, pst'e-i-n wux bar-leⁿ xiɬ-\u{u} co xeɁ-er. \\
		{\M}.{\Pl}-go.{\Ipfv}-{\M}.{\Pl}-come.{\Ipfv}({\Ptcp}.{\Npst})-{\Obl}-{\Pl}-{\Erg} \textbf{{\D}-bring.{\Pfv}-{\Cond}.{\Pst}} except woman-{\Pl}-{\Dat} what valley-{\Abl} be.{\Pfv}-{\Npst} {\Neg} know-{\Imprf} \\
		\trans `Except if people going back and forth would bring [news], the women didn't know what was happening in the valley.'
		\hfill (YD002-4.1)
		
		\ex\label{verbflex-ex15b}
		\gll bader balk'on-makreⁿ aħ\u{o} \textbf{teps-l-aħer}, eħat=a co šeʒleba-l-ar dada-s o ħal\u{o} ec-ar. \\
		child balcony-{\Superabl} down \textbf{fall.{\Ipfv}-{\Intr}-{\Cond}.{\Pst}} then={\Emph} {\Neg} be\_able-{\Intr}-{\Imprf} father-{\Erg} {\Dist} up take.{\Pfv}-{\Vn} \\
		\trans `Whenever a child was falling off a balcony, the father wasn't able to catch it.'
		\hfill (MM226-1.2)
		
	\end{xlist}
\end{exe}

\subsubsection{Non-Witnessed Aorist Conditional}

The Non-Witnessed Aorist Conditional is formed by adding the conditional suffix \textit{-ħe} to the Non-Witnessed Aorist (for which see \sectref{evid}). Its function is straightforward: to express a conditional clause, where the speaker of the utterance has gained knowledge of the event only indirectly. See Example (\ref{verbflex-ex16}).\largerpage


\begin{exe}
	\ex\label{verbflex-ex16}
	\begin{xlist}
		
		\ex\label{verbflex-ex16a}
		\gll oqu-x at't'aj-š \textbf{tiš-no-ħe-ħ} me herc'ni-v bader d-∅-or, oqu-x wux=ajno=g co teš-e-ħ me ħeⁿ herc'\u{o} bader d-∅-oš daħ l-ar.  \\
		{\Dist}.{\Obl}-{\Cont} easy-{\Adv} \textbf{believe-{\Nw}.{\Aor}-{\Cond}-{\Ssg}} {\Subord} pot.{\Obl}-{\Erg} child {\D}-make-{\Imprf} {\Dist}.{\Obl}-{\Cont} what={\Ben}={\Indf} {\Neg} believe-{\Npst}-{\Ssg} {\Subord} {\Ssg}.{\Gen} pot child {\D}-make-{\Simul} {\Pv} die-{\Imprf} \\
		\trans `If you (apparently) believe [so] easily that a pot has given birth to a child, why wouldn't you believe that your pot died while giving birth?'
		\hfill (EK057-9.1)
		
		\ex\label{verbflex-ex16b}
		\gll  \textbf{xit-no-ħ} mk'itxvel-e-n, ħan-x=a j-∅-o-s ambui, liv-as. \\
		\textbf{find-{\Nw}.{\Aor}-{\Cond}} reader-{\Obl}-{\Dat} who.{\Obl}-{\Cont}={\Emph} {\J}-do.{\Pfv}-{\Npst}-{\Fsg}.{\Erg} story say.{\Ipfv}-{\Fsg}.{\Erg} \\
		\trans `If the reader (apparently) finds out who I'm talking about, I tell them.' \\
		\hfill (MM412-1.136)
		
		\ex\label{verbflex-ex16c}
		\gll baq'eɁ j-aqqoⁿ gamocd čabarbad-j-i-eⁿ k'acobrioba-s st'alinizm-e (\textbf{čabarbad-j-i-no-ħ}!). \\
		truly {\J}-big exam pass-{\J}-{\Tr}-{\Aor} mankind-{\Erg} Stalinism-{\Obl}({\Ess}) \textbf{pass-{\J}-{\Tr}-{\Nw}.{\Aor}-{\Cond}}\\
		\trans `Mankind has passed a truly big exam in Stalinism (if indeed it turns out it did pass it!).'
		\hfill (MM415-1.104)
		
	\end{xlist}
\end{exe}

\subsubsection{Non-Witnessed Remote Conditional}

The Non-Witnessed Remote Conditional is formed by adding the suffix \textit{-ra} to the Non-Witnessed Aorist Conditional. It is used to form counterfactual conditional clauses, see Example (\ref{verbflex-ex17}). The verbal form in the matrix clause is a periphrastic tense, as discussed in \sectref{periph}.


\begin{exe}
	\ex\label{verbflex-ex17}
	\begin{xlist}
		
		\ex\label{verbflex-ex17a}
		\gll  o kotm-i=a, gagn-i=a co \textbf{xiɬ-noħer} txo-go, albat macu-x d-av-in-d-a-ra-tx. \\
		{\Dist} chicken-{\Pl}={\Add} egg-{\Pl}={\Add} {\Neg} \textbf{be.{\Pfv}-{\Nw}.{\Rem}.{\Cond}} {\Fpl}-{\Adess} probably hunger.{\Obl}-{\Cont} {\D}-die-{\Ptcp}.{\Pst}-{\D}-be-{\Imprf}-{\Fpl} \\
		\trans `If we didn't have these chickens and eggs, we would probably die from starvation.'
		\hfill (E039-84)
		
		\ex\label{verbflex-ex17b}
		\gll  o din \textbf{v-is-noħer} so-g ħal ʕam-d-∅-it-en-d-a-r oqu-s. \\
		{\Dist} alive {\M}.{\Sg}-stay-{\Nw}.{\Rem}.{\Cond} {\Fsg}-{\All} {\Pv} learn-{\D}-{\Tr}-{\Caus}-{\Ptcp}.{\Pst}-{\D}-be-{\Imprf} {\Dist}.{\Obl}-{\Erg}\\
		\trans `If he had stayed alive, he would have helped me study.'
		\hfill (E122-15)
		
	\end{xlist}
\end{exe}


\subsection{Iamitive} \label{cont}\is{Iamitive}

In Tsova-Tush, the verbal suffix \textit{-ge} expresses the notion of `already'. In this work, the term Iamitive will be used for these forms.\footnote{Thanks to Timur Maisak and Anastasia Panova for bringing this term to my attention.} This term has been introduced recently to describe a type of independent markers that were previously analysed as markers of the perfect tense-aspect, or as instances of a lexeme `already'. \textcites{olsson2013iam} argues that iamitive is a better fit for these markers since they indicate the notion of a ``new situation'' that holds after a transition (which iamitives have in common with `already'), but also denote the consequences that this situation has at reference time for the participants in the speech event (which iamitives have in common with the perfect). Additionally, these iamitives are said to have a high frequency in texts and to appear in natural development contexts, e.g., `it is already rotten' or `I am already married' (\cites{olsson2013iam,dahlwalchli2016iam}).\footnote{Although use of the term has been deemed by some to be superfluous (\cites[]{krajinovic2019iam}).} In this work, however, the term iamitive will be defined slightly differently. Here, the Tsova-Tush Iamitive has high semantic overlap with `already' in English and other languages, and little to no overlap with perfects (for the Tsova-Tush Perfect, see \sectref{periph}). The reason the term Iamitive is nevertheless used, and not the simple gloss `already', is the fact that \textit{-ge} is a bound suffix, not an independent word. The most frequent occurrence of the Tsova-Tush Iamitive is in negative sentences, where it denotes the semantics of `not anymore', a common type of polarity (\cites[]{lobner1989schon,auwera1993already}).

\begin{table}
	\begin{tabular}{llll}
		
		\lsptoprule
		& Morphemes & Glossing & Surface form \\
	\midrule
		
		
		Non-Past Iamitive & \textit{tet'-o-ge} & cut.\textsc{ipfv-npst-iam} & \textit{tet'og} \\
		& \textit{tit'-o-ge} & cut.\textsc{pfv-npst-iam} & \textit{tit'og} \\
		
		Past Iamitive & \textit{tet'-ogera} & cut.\textsc{ipfv-iam.pst} & \textit{tet'oger} \\
		& \textit{tit'-ogera} & cut.\textsc{pfv-iam.pst} & \textit{tit'oger} \\
		
		\lspbottomrule
		
	\end{tabular}
	\caption{Tsova-Tush Iamitive verb forms}
	\label{table-iamverbflex}
\end{table}

\subsubsection{Non-Past Iamitive}

The Non-Past Iamitive is formed by adding the suffix \textit{-ge} to the Non-Past. With the verb \textit{d-a} `be', a reduplicated form of the suffix, \textit{-gege}, is found most often. In interrogative sentences, it expresses uncertainty (Example (\ref{verbflex-ex18a})), similar to the Subjunctive. In declarative sentences it is used to convey a meaning `already' (Examples (\ref{verbflex-ex18a})\footnote{Original orthography of (\ref{verbflex-ex18a}): Мич гıогес, къаина ваггес\u{o}.}, (\ref{verbflex-ex18b})). However, its most common use is in negated contexts, where it expresses a meaning `not anymore' (\ref{verbflex-ex19}).\footnote{With Example  (\ref{verbflex-ex19}), note that \textit{k'aloš} `galosh' belongs to an inquorate gender that triggers \textit{b-} in both singular and plural, see \sectref{verbalgender}.}\is{Dubitative}

\begin{exe}
	\ex\label{verbflex-ex18}
	\begin{xlist}
		
		\ex\label{verbflex-ex18a}
		\gll mič \textbf{\u{g}-o-ge-s}, q'ain=a \textbf{v-a-gge-s\u{o}}. \\
		where \textbf{go.{\Pfv}-{\Npst}-{\Iam}-{\Fsg}.{\Erg}} old={\Emph} \textbf{{\M}.{\Sg}-be-{\Iam}-{\Fsg}.{\Nom}} \\
		\trans `Where should I go, I'm already [so] old.'
		\hfill (YD001-19.2)
		
		\ex\label{verbflex-ex18b}
		\gll so j-aqqoⁿ \textbf{j-a-g\u{e}}, čxindur=a\u{e} ħal d-∅-o-s, bak'-e-ħ jol=a\u{e} ħal j-i-n-as. \\
		{\Fsg}.{\Nom} {\F}.{\Sg}-big \textbf{{\F}.{\Sg}-be-{\Iam}} stocking={\Add} {\Pv} {\D}-make-{\Npst}-{\Fsg}.{\Erg} paddock-{\Obl}-{\Ess} hay={\Add} {\Pv} {\J}-do.{\Pfv}-{\Aor}-{\Fsg}.{\Erg} \\
		\trans `I am already old; I make socks, and I have done the hay in the field.' \\
		\hfill (KK036-5563)
		
	\end{xlist}
\end{exe}

\begin{exe}
	\ex\label{verbflex-ex19}
	\begin{xlist}
		
		\ex\label{verbflex-ex19a}
		\gll ħaps-eⁿ - tavtav ese co \textbf{b-a-g}. \\
		look.{\Pl}.{\Pfv}-{\Aor} {} wheat\_ear here({\Ess}) {\Neg} \textbf{{\B}.{\Sg}-be-{\Iam}} \\
		\trans `They looked around; the wheat ear is no longer here.'
		\hfill (E153-11)
		
		\ex\label{verbflex-ex19b}
		\gll k'aloš-i duq-čui-š-v co \textbf{b-epx-o-g\u{e}}. \\
		galosh-{\Pl} many-{\Obl}-{\Pl}-{\Erg} {\Neg} \textbf{{\B}-wear-{\Npst}-{\Iam}} \\
		\trans `Not many people wear galoshes anymore.'
		\hfill (KK011-1952)
		
	\end{xlist}
\end{exe}


\subsubsection{Past Iamitive}

The Past Iamitive is formed by adding the suffix \textit{-ra} to the Non-Past Iamitive. It is used in similar contexts to the Non-Past Iamitive, but with a shift to the past tense: like in the Non-Past, it can have a dubitative meaning in interrogative sentences (\ref{verbflex-ex20a}), a meaning `almost' (with or without an actual adverb meaning `almost') (\ref{verbflex-ex20b}, \ref{verbflex-ex20c}), and a meaning `not anymore' in negative sentences (\ref{verbflex-ex20d}).

\begin{exe}
	\ex\label{verbflex-ex20}
	\begin{xlist}
		
		
			\ex\label{verbflex-ex20a}
			\gll wux \textbf{d-}∅\textbf{-oger}, cok'l-e-n i joħ daħa j-aɬ-iⁿ. \\
			what \textbf{{\D}-do.{\Pfv}-{\Iam}.{\Pst}} fox-{\Obl}-{\Dat} {\Med} girl {\Pv} {\F}.{\Sg}-give.{\Pfv}-{\Aor} \\
			\trans `What were they supposed to do, so they gave that girl to the fox.' \\
			\hfill (E153-46)
		
		
		
			\ex\label{verbflex-ex20b}
			\gll cħajnčone korti-x dik' \textbf{d-iš-d-}∅\textbf{-oger}.\\
			almost head.{\Obl}-{\Cont} axe \textbf{{\D}-strike-{\D}-{\Tr}-{\Iam}.{\Pst}} \\
			\trans `S/he almost struck [somebody] on the head with an axe.'
			\hfill (EK053-4.11)
		
		
		
			\ex\label{verbflex-ex20c}
			\gll o xeⁿ ħal \textbf{dagleǯad-b-}∅\textbf{-oger}, nast' č'a\u{g}-v-al-iⁿ. \\
			{\Dist} tree {\Pv} \textbf{uproot.{\Pfv}-{\B}.{\Sg}-{\Tr}-{\Iam}.{\Pst}} with\_difficulty strong-{\M}.{\Sg}-{\Intr}-{\Aor} \\
			\trans `He almost uprooted the tree, he hardly stayed upright.'
			\hfill (E181-267)
		
		
		
			\ex\label{verbflex-ex20d}
			\gll doⁿ ħal co \textbf{g-uger}, dačo lark'-i g-ur. \\
			horse {\Pv} {\Neg} \textbf{be\_visible-{\Iam}.{\Pst}} only ear-{\Pl} be\_visible-{\Imprf} \\
			\trans `The horse was not visible anymore, only its ears were visible.' \\
			\hfill (EK058-3.8)
		
		
	\end{xlist}
\end{exe}

\subsubsection{Other Iamitives}

Various other formations with the Iamitive morpheme \textit{-ge} are found. These forms are rarely used, and their full functional scope is still poorly understood. Example (\ref{verbflex-ex21a}) shows a verb form with both Iamitive and Subjunctive morphemes, conveying the meaning `if only', which alternatively could have been expressed by a simplex Subjunctive form. Examples (\ref{verbflex-ex21b}) and (\ref{verbflex-ex21c}) show forms containing the morpheme \textit{-ge} with the meaning `not anymore', with a Non-Witnessed Aorist (\ref{verbflex-ex21b}) and a Non-Witnessed Remote Past (\ref{verbflex-ex21c}).

\begin{exe}
	\ex\label{verbflex-ex21}
	\begin{xlist}
		
		
			\ex\label{verbflex-ex21a}
			\gll so \textbf{v-ap'c'-ge-l} k'i ħoⁿ q'an-v-al-in. \\
			{\Fsg}.{\Nom} \textbf{{\M}.{\Sg}-recognise({\Npst})-{\Iam}-{\Sbjv}} {\Contr} {\Ssg}.{\Dat} old-{\M}.{\Sg}-{\Intr}-{\Ptcp}.{\Pst} \\
			\trans `If you could only recognise me, having gotten old.'
			\hfill (E239-8)
		
		
		
		
			\ex\label{verbflex-ex21b}
			\gll giorgi-ⁿ joħ co \textbf{j-abc'-mak'-in-g\u{e}} txoⁿ.  \\
			Giorgi-{\Gen} girl {\Neg} \textbf{{\F}.{\Sg}-discern-{\Pot}-{\Nw}.{\Aor}-{\Iam}} {\Fpl}.{\Dat} \\
			\trans `We (apparently) couldn't discern Giorgi's daughter anymore.' \\
			\hfill (KK001-0014)
		
		
		
			\ex\label{verbflex-ex21c}
			\gll cħa v-ejt'-noer šarn, t'q'uiħ=a co \textbf{v-oc'-v-ejl-noger}. \\
			one {\M}.{\Sg}-run-{\Nw}.{\Rem} away behind={\Emph} {\Neg} \textbf{{\M}.{\Sg}-follow-{\M}.{\Sg}-{\Intr}-{\Nw}.{\Rem}.{\Iam}} \\
			\trans `One (apparently) ran off, he (apparently) stopped following him.' \\
			\hfill (E144-10)
		
		
	\end{xlist}
\end{exe}

\section{Periphrastic TAME forms} \label{periph}\is{Periphrastic verb forms}

The four main periphrastic verb constructions in Tsova-Tush are in this work labelled Perfect, Pluperfect, Non-Witnessed Perfect, the Non-Witnessed Pluperfect. They consist of a Past Participle and an inflected form of \textit{d-a} `be'. In Tsova-Tush, the negative particle \textit{co} must immediately precede the finite verb. The negator \textit{co} very often stands before the entire verbal complex of the Participle + `be' (\cites{ankernegation}), even though most examples in this work, for instance (\ref{verbflex-ex28a}, \ref{verbflex-ex28c}, \ref{verbflex-ex30}, \ref{verbflex-ex31}), show a construction [`be' (intervening material) PARTICIPLE]. Therefore, I tentatively analyse these as periphrastic constructions undergoing a process of univerbation. See Table \ref{table-periverbflex} for the periphrastic verb forms, where surface forms are presented in their most advanced stage, i.e. with the order PARTICIPLE+`be', and univerbated (even though the reverse order is also found). The Non-Witnessed periphrastic forms are discussed in \sectref{evid}.


\begin{table}

		\begin{tabular}{llll}
			\lsptoprule
			& {Morphemes} & {Glossing} & {Surface form} \\
			\midrule
			\multicolumn{4}{l}{Perfect}\\
			& \textit{tet'-ino d-a} & cut.\textsc{ipfv-ptcp.pst d-}be & \textit{tet'inda} \\
			& \textit{tit'-eno d-a} & cut.\textsc{pfv-ptcp.pst d-}be & \textit{tit'enda} \\
			
			\multicolumn{4}{l}{Pluperfect}\\
			& \textit{tet'-ino d-a-ra} & cut.\textsc{ipfv-ptcp.pst d-}be-\textsc{imprf} & \textit{tet'indar} \\
			& \textit{tit'-eno d-a-ra} & cut.\textsc{pfv-ptcp.pst d-}be-\textsc{imprf} & \textit{tit'endar} \\
			
			\multicolumn{4}{l}{Non-Witnessed Perfect}\\
			& \textit{tet'-ino d-a-no} & cut.\textsc{ipfv-ptcp.pst d-}be-\textsc{nw.aor} & \textit{tet'indan\u{o}} \\
			& \textit{tit'-eno d-a-no} & cut.\textsc{pfv-ptcp.pst d-}be-\textsc{nw.aor} & \textit{tit'endan\u{o}} \\  
			
			\multicolumn{4}{l}{Non-Witnessed Pluperfect}\\
			& \textit{tet'-ino d-a-nora} & cut.\textsc{ipfv-ptcp.pst d-}be-\textsc{nw.rem} & \textit{tet'indanor} \\
			& \textit{tit'-eno d-a-nora} & cut.\textsc{pfv-ptcp.pst d-}be-\textsc{nw.rem} & \textit{tit'endanor} \\ 
			\lspbottomrule
		\end{tabular}

	\caption{Tsova-Tush periphrastic verb forms}
	\label{table-periverbflex}
\end{table}


\subsection{Perfect}\is{Perfect tense}\is{Resultative}

The Perfect is formed by a Past Participle together with a form of the verb `be' in the Present. The Tsova-Tush Perfect displays all the meanings typologically associated with perfects (\cites[24]{comrie85}{plungian16}{verhees19}), i.e. resultative/current relevance (\ref{verbflex-ex28a}), experiential (\ref{verbflex-ex28c}) and indirect (non-witnessed) evidentiality (\ref{verbflex-ex28d}).

\begin{exe}
	\ex\label{verbflex-ex28}
	\begin{xlist}
		
		
			\ex\label{verbflex-ex28a}
			\gll daħ \textbf{b-a} \textbf{b-al-in}, č'q'emp' b-a tit'-en. \\
			{\Pv} {\B}.{\Sg}-be {\B}.{\Sg}-die-{\Ptcp}.{\Pst} throat {\B}.{\Sg}-be cut.{\Pfv}-{\Ptcp}.{\Pst} \\
			\trans `It has died, it has its throat cut.'
			\hfill (E153-38)
		
		
		
			\ex\label{verbflex-ex28c}
			\gll ši baq' d-i-en vir co \textbf{xac'-d-al-in-d-a} bac-bi-go.  \\
			two foal {\D}-give\_birth-{\Ptcp}.{\Pst} donkey {\Neg} \textbf{hear-{\D}-{\Intr}-{\Ptcp}.{\Pst}-{\D}-be} Tsova\_Tush-{\Pl}-{\Adess} \\
			\trans `The Tsova-Tush have never heard of a donkey having given birth to two foals.'
			\hfill (E010 18)
		
		
		
			\ex\label{verbflex-ex28d}
			\gll tavtav ese co b-a-g, maml-e-v \textbf{b-a} \textbf{qall-in}=en. \\
			wheat\_ear here({\Ess}) {\Neg} {\B}.{\Sg}-be-{\Iam} rooster-{\Obl}-{\Erg} \textbf{{\B}.{\Sg}-be} \textbf{eat.{\Pfv}-{\Ptcp}.{\Pst}}={\Quot}\\
			\trans `The wheat ear is no longer there; the rooster must have eaten it.' \\
			\hfill (E153-11)
		
	\end{xlist}
\end{exe}



\subsection{Pluperfect}\is{Pluperfect}\is{Counterfactual conditional}

The Pluperfect is formed by a Past Participle together with a form of the verb `be' in the Imperfect. This tense-aspect form expresses counterfactual semantics, and is used most often in matrix clauses alongside a Non-Witnessed Aorist Conditional in the conditional clause, for which see Example (\ref{verbflex-ex29}).

\begin{exe}
	\ex\label{verbflex-ex29}
	\begin{xlist}
		
		
			\ex\label{verbflex-ex29a}
			\gll o kotm-i=a, gagn-i=a co xiɬ-noħer txo-go, albat macu-x \textbf{d-av-in-d-a-ra-tx}. \\
			{\Dist} chicken-{\Pl}={\Add} egg-{\Pl}={\Add} {\Neg} be.{\Pfv}-{\Nw}.{\Rem}.{\Cond} {\Fpl}-{\Adess} probably hunger-{\Cont} \textbf{{\D}-die-{\Ptcp}.{\Pst}-{\D}-be-{\Imprf}-{\Fpl}} \\
			\trans `If we didn't have chickens and eggs, we probably would have died from starvation.'
			\hfill (E039-84)
		
		
		
			\ex\label{verbflex-ex29b}
			\gll upr duq maq-iš \textbf{c'era-d-i-en-d-a-r}, upr \u{g}az-iš \textbf{c'era-d-i-en-d-a-r}. \\
			more many song.{\Obl}-{\Pl} \textbf{write-{\B}.{\Pl}-{\Tr}-{\Ptcp}.{\Npst}-{\B}.{\Pl}-be-{\Imprf}} more good-{\Adv} \textbf{write-{\B}.{\Pl}-{\Tr}-{\Ptcp}.{\Npst}-{\B}.{\Pl}-be-{\Imprf}} \\
			\trans `S/he could have written more songs, better ones.'
			\hfill (E171-11)
		
		
	\end{xlist}
\end{exe}

\is{Georgian influence!Morphological}It is important to note that in Georgian, a similarly constructed verb form, also called Pluperfect, is also used in counterfactual conditional clauses. For intransitive verbs, but not for other verbs, this verb form is formed historically by combining a past participle with a past form of the verb `be'. Synchronically, however, this is a single verb form, as the first person subject marking is prefixed to the beginning of the verb form (hence \textit{v-q'opiliq'avi}, where \textit{v-} is the first person marking, historically from \textit{q'opili} `been' + \textit{viq'avi} `I was'). In Georgian, however, this form is used in the subordinate conditional clause, while the matrix verb is expressed by a form that in Georgian linguistics is traditionally called the Conditional (see Example (\ref{verbflex-ex49})). Additionally, this way of forming a Georgian Pluperfect, namely using a Past Participle and a verb `be', is only used with intransitive verbs of Class II (see \sectref{geoalign}). Thus, the counterfactual conditional constructions are sufficiently different in Tsova-Tush and Georgian that it is unwarranted to assume any Georgian influence here.

	
	\begin{exe}
		\ex\label{verbflex-ex49}
		Standard Modern Georgian
        
		\gll ak rom \textbf{vq'opiliq'avi}, ver=c me vušvelidi. \\
		here {\Subord} \textbf{I\_had\_been} {\Neg}.{\Pot}={\Add} {\Fsg}  I\_would\_have\_helped. \\
		\trans `If I had been here, I wouldn't have been able to help either.' \\
		\hfill (GNC: M. Javakhishvili)
	\end{exe}



\section{Evidentiality} \label{evid}\is{Evidentiality}\is{Unwitnessed evidentiality}\is{Indirect evidentiality}\is{Non-witnessed evidentiality}

\subsection{Introduction}

Like in most East Caucasian languages, Tsova-Tush features verb forms that convey evidential semantics. As opposed to some languages, where several evidential values can be identified (such as visual, non-visual, inferential, or hearsay), East Caucasian languages typically feature a small number of evidential forms that (1) have the general, unspecified value `non-witnessed', (2) contrast with unmarked `witnessed' forms, and (3) usually consist of a non-finite form in combination with an auxiliary verb (\cite{verhees19,forker18}). Tsova-Tush features six distinct finite verb forms that convey  non-witnessed (i.e. indirect evidential) semantics. All Tsova-Tush finite indicative verb forms have a corresponding evidential form, all of which are listed in Table \ref{table-evid}. Additionally, the Past Subjunctive most often exhibits evidential semantics.


\begin{table}
	\begin{tabular}{lll}
		\lsptoprule
		& Neutral & Non-Witnessed \\
        \midrule
		Non-Past	& -V & -V-d-ano \\
		Imperfect & -Vra & -V-d-anora \\
		Aorist	& -in	& -ino \\
		Remote	& -ira	& -inora \\
		Perfect	& -ino d-a	& -ino d-a-no \\
		Pluperfect	& -ino d-a-ra	& -ino d-a-nora \\
		\lspbottomrule
	\end{tabular}
	\caption{Tsova-Tush Evidential suffixes. Some lexical verbs choose \textit{e} for every \textit{i} in this table.}
	\label{table-evid}
\end{table}

Note that, as with non-indicative forms, the Perfective/Imperfective aspect of the verbal stem has been poorly investigated in non-witnessed forms. Therefore, the exact semantics (pluractional or otherwise) of this aspectual distinction is unknown. Non-witnessed verbal forms are usually translated with `(apparently)' in brackets to clearly signal the indirect evidentiality, even though better English translations would be possible. 


\subsection{Formation of Evidential forms} \label{evidform}

\subsubsection{Non-Witnessed Non-Past}

The Non-Witnessed Non-Past is historically formed by a Non-Past verb form followed by the verb form \textit{d-a} `be' with a suffix \textit{-no}. The fact that this construction is now completely grammaticalised into a single word has been proven by \textcite{harris09}. This verb form principally conveys a present or future tense with non-witnessed evidentiality, as is illustrated in  Example (\ref{verbflex-ex25}).



\begin{exe}
	\ex\label{verbflex-ex25}
	\begin{xlist}
		
		
			\ex\label{verbflex-ex25a}
			\gll  ʕalin komoⁿ axr-i duq-ux \textbf{d-auv-d-an\u{o}} pst'ujn-č\u{o} axr-a-x. \\
			in\_winter male shearling\_lamb-{\Pl} many-{\Cmp} \textbf{{\D}-die-{\D}-{\Nw}.{\Npst}} female-{\Obl} shearling\_lamb-{\Obl}.{\Pl}-{\Cont} \\
			\trans `In the winter, more male shearling lambs (apparently) die than female ones.'
			\hfill (KK001-0333)
		
		
		
			\ex\label{verbflex-ex25b}
			\gll t'aranin \textbf{v-uit'-v-an\u{o}} mot'ocik'let'-e-v, t'q'uiħ ši st'ak' \textbf{ʕe-v-a\u{g}-v-an\u{o}}. \\
			Taranina \textbf{{\M}.{\Sg}-go.{\Ipfv}-{\M}.{\Sg}-{\Nw}.{\Npst}} motorcycle-{\Obl}-{\Ins} behind two man \textbf{sit-{\M}.{\Sg}-{\Lv}.{\Sg}-{\M}.{\Sg}-{\Nw}.{\Npst}} \\
			\trans `Taranina is driving a motorcycle and two men are sitting behind him.' [beginning of a joke]
			\hfill (EK065-1.1)
		
		
	\end{xlist}
\end{exe}

\is{Present tense}\is{Non-Past tense}In more contemporary Tsova-Tush, however, a non-witnessed past tense is the most common reading, see Example (\ref{verbflex-ex26}).

\begin{exe}
	\ex\label{verbflex-ex26}
	\begin{xlist}
		
		
			\ex\label{verbflex-ex26a}
			\gll ši saldat čuħ \textbf{v-iš-v-an\u{o}} kox-e. \\
			two soldier inside \textbf{{\M}.{\Sg}-lie\_down-{\M}.{\Sg}-{\Nw}.{\Npst}} hut-{\Obl}({\Ess}) \\
			\trans `Two soldiers (apparently) lay inside the hut.'
			\hfill (E147-139)
		
		
		
			\ex\label{verbflex-ex26b}
			\gll c'q'e cħajn ʕurdeⁿ rusudaⁿ dada-s wun-ax \textbf{tego-d-an\u{o}} osi k'rant'-mak. \\
			once one.{\Obl} in\_the\_morning Rusudan father.{\Obl}-{\Erg} what-{\Indf} \textbf{do.{\Ipfv}-{\D}-{\Nw}.{\Npst}} there({\Ess}) tap-{\Superlat} \\
			\trans `One morning, Rusudan's father was (apparently) doing something with the tap.'
			\hfill (MM107-3.4)
		
		
	\end{xlist}
\end{exe}



\subsubsection{Non-Witnessed Imperfect}

The Non-Witnessed Imperfect is historically formed by a Non-Past verb form followed by the verb form \textit{d-a} `be' with a suffix \textit{-nora}. This verb form conveys past tense, durative aspect and non-witnessed evidentiality, as is illustrated in  Example (\ref{verbflex-ex27}).\is{Imperfect past tense}


\begin{exe}
	\ex\label{verbflex-ex27}
	\begin{xlist}
		
		
			\ex\label{verbflex-ex27a}
			\gll c'q'e ši ʕu \textbf{v-uit'-v-anor}=e, pħe ču b-epl-iš cħana-n dak'-d-eɁ-nor, me [...]. \\
			once two shepherd \textbf{{\M}.{\Sg}-go.{\Imprf}-{\M}.{\Sg}-{\Nw}.{\Imprf}}=and village({\Lat}) in {\M}.{\Pl}-move.{\Ipfv}-{\Simul} one.{\Obl}-{\Dat} heart-{\D}-come-{\Nw}.{\Rem} {\Subord} [...]\\
			\trans `Once, two shepherds were (apparently) on the move, and when they entered a village, one remembered that [he had baptised a child there].
			\hfill (E058-35)
		
		
		
			\ex\label{verbflex-ex27b}
			\gll važar o k'urs-i-mak b-axk'-eⁿ - osi \textbf{ʕam-d-}∅\textbf{-o-d-anor}. \\
			brother.{\Pl} {\Dist} course-{\Pl}-{\Superlat} {\M}.{\Pl}-go.{\Pfv}.{\Pl}-{\Aor} {} there({\Ess}) \textbf{learn-{\D}-{\Tr}-{\Npst}-{\D}-{\Nw}.{\Imprf}} \\
			\trans `[Two] brothers went to those courses; they were (apparently) studying there.'
			\hfill (E092-18)
		
		
	\end{xlist}
\end{exe}


\subsubsection{Past Subjunctive}\is{Subjunctive}

As mentioned in \sectref{subj}, the Past Subjunctive, besides its actual subjunctive function, is mostly used as a Non-Witnessed Past, see Example (\ref{verbflex-ex22}). These forms are used as the primary means to express a past event with durative aspect that has not been witnessed directly by the speaker. 

\begin{exe}
	\ex\label{verbflex-ex22}
	\begin{xlist}
		
		
			\ex\label{verbflex-ex22a}
			\gll ai o q'arpuz, xil=a bazir nʕejɁ \textbf{qeħ-ra-l\u{o}}, kalik=a daħ \textbf{qeħ-ra-l\u{o}}, uis daħ \textbf{d-exk'-ra-l\u{o}}, t'atb-i \textbf{teg-d-}∅\textbf{-ora-l\u{o}}. \\
			{\Deict} {\Dist} watermelon fruit={\Add} bazaar.{\Ill} out \textbf{carry.{\Ipfv}-{\Imprf}-{\Sbjv}} city.{\Ill}={\Add} {\Pv} \textbf{carry.{\Ipfv}-{\Imprf}-{\Sbjv}} there {\Pv} \textbf{{\D}-sell.{\Ipfv}-{\Imprf}-{\Sbjv}} money-{\Pl} \textbf{make.{\Imprf}-{\D}-{\Tr}-{\Imprf}-{\Sbjv}} \\
			\trans `Those watermelons, that fruit they (apparently) took outside to the markets, too, they brought it to Tbilisi, there they sold it, they made money there.'
			\hfill (E014-15)
		
		
		
			\ex\label{verbflex-ex22b}
			\gll ʕurdeⁿ ħejps-če - o ž\u{g}ar=en karcxu-i ħal korti-x \textbf{d-at'-ra-l\u{o}}. \\
			in\_the\_morning look.{\Pl}.{\Pfv}-{\Ante} {} {\Dist} jingling={\Quot} clothes-{\Pl} {\Pv} head.{\Obl}-{\Cont} \textbf{{\D}-lie-{\Imprf}-{\Sbjv}} \\
			\trans `When they saw him in the morning, those jingling garments were (apparently) on his head.'
			\hfill (E156-13)
		
		
	\end{xlist}
\end{exe}

The Past Subjunctive seems to be used most often to express background durative past events in certain narratives. These narratives, such as folk tales or legends, are made up entirely of  non-witnessed verb forms  (\cite{wsverhees2024nakhevid}). See Example (\ref{verbflex-ex48}), where main events that drive the plot are represented by the Non-Witnessed Remote Past (for which see below on page \pageref{nw.rem}), and backgrounded information is given in the Past Subjunctive.


\begin{exe}
	\ex\label{verbflex-ex48}
	\begin{xlist}
		
		
			\ex\label{verbflex-ex48a}
			\gll cok'al \textbf{d-uit'-ra-l}, \textbf{d-uit'-ra-l}, \textbf{d-uit'-ra-lo}=e, naq'a tavtav xit-nor. \\
			fox \textbf{{\D}-go.{\Ipfv}-{\Imprf}-{\Sbjv}} \textbf{{\D}-go.{\Ipfv}-{\Imprf}-{\Sbjv}} \textbf{{\D}-go.{\Ipfv}-{\Imprf}-{\Sbjv}}=and road.{\Obl}({\Ess}) wheat\_ear find-{\Nw}.{\Rem}\\
			\trans `A fox was walking, walking, walking along, and on the road he found an ear (of wheat).'
			\hfill (E153-4)
		
		
		
			\ex\label{verbflex-ex48b}
			\gll ma c'ʕerkoⁿ daħ v-ic-v-ejl-nor wux c'-er o nejtlmam-e-x. nejtlmama-x xoxoba \textbf{c'-era-l}. \\
			but suddenly {\Pv} {\M}.{\Sg}-forget-{\M}.{\Sg}-{\Intr}-{\Nw}.{\Rem} what be\_called-{\Imprf} {\Dist} godfather-{\Obl}-{\Cont} godfather.{\Obl}-{\Cont} Khokhoba \textbf{be\_called-{\Imprf}-{\Sbjv}} \\
			\trans `But suddenly he forgot the godfather's name. The godfather's name was Khokhoba/``Pheasant''.'
			\hfill (E058-43,44)
		
		
	\end{xlist}
\end{exe}


\subsubsection{Non-Witnessed Aorist} \is{Aorist}

The Non-Witnessed Aorist has historically been formed by suffixing \textit{-no} to the Aorist stem. Synchronically, the suffixes are thus \textit{-ino} and \textit{-eno}. Note that this formation is homophonous with the Past Participle (see \sectref{participial}). For the verb `be', the suffix \textit{-no} is attached to the verbal root directly (see Example (\ref{verbflex-ex23b}). The Non-Witnessed Aorist is relatively rare, and seems to be almost completely replaced by other non-witnessed forms, possibly due to it being homophonous with the Past Participle.

\begin{exe}
	\ex\label{verbflex-ex23}
	\begin{xlist}
		
		
			\ex\label{verbflex-ex23a}
			\gll vʕalaɁ co \textbf{xarc-v-ajl-no-ħ}, uišt'n=eɁ v-a-ħ\u{o}. \\
			completely {\Neg} \textbf{change.{\Pfv}-{\M}.{\Sg}-{\Intr}-{\Nw}.{\Aor}-{\Ssg}} so.{\Dist}={\Emph} {\M}.{\Sg}-be-{\Ssg}.{\Nom} \\
			\trans `You (apparently) haven't changed at all, you are exactly the same.' \\
			\hfill (KK021-3686)
		
		
		
			\ex\label{verbflex-ex23b}
			\gll moliⁿ č'ʕa\u{g}oⁿ bacav \textbf{v-a-n} joħ. \\
			what\_kind strong Tsova\_Tush \textbf{{\M}.{\Sg}-be-{\Nw}.{\Aor}} girl \\
			\trans `What a strong Tsova-Tush he (apparently) was, woman.'
			\hfill (E173-82)
		
		
		
		
	\end{xlist}
\end{exe}



\subsubsection{Non-Witnessed Remote Past} \label{nw.rem}

The Non-Witnessed Remote Past is formed by suffixing \textit{-ra} to the Non-Witnessed Aorist. It is the primary TAME form used as a narrative tense in fairy tales, legends, or history, as illustrated in Example (\ref{verbflex-ex24}).

\begin{exe}
	\ex\label{verbflex-ex24}
	\begin{xlist}
		
		
			\ex\label{verbflex-ex24a}
			\gll lek'-i ušt' \textbf{b-av-b-i-nor} daħ me, vaš-bi-g\u{o} ħaps-aⁿ dro co \textbf{j-a-nor}. \\
			Daghestanian-{\Pl} so.{\Dist} \textbf{{\M}.{\Pl}-kill-{\M}.{\Pl}-{\Tr}-{\Nw}.{\Rem}} {\Pv} {\Subord} {\Recp}-{\Pl}-{\All} look.{\Pfv}.{\Pl}-{\Inf} time {\Neg} \textbf{{\J}-be-{\Nw}.{\Rem}} \\
			\trans `He killed the Daghestanians so [fast] that they didn't have time to look at each other.'
			\hfill (E144-9)
		
		
		
			\ex\label{verbflex-ex24b}
			\gll o joħ \textbf{moc'onad-j-el-nor} o cok'l-e-n, \textbf{dak'lev-nor} me, garat as is joħ šarn j-ik'-o-s šu-goħ=en. \\
			{\Dist} girl \textbf{like-{\F}.{\Sg}-{\Intr}-{\Nw}.{\Rem}} {\Dist} fox-{\Obl}-{\Dat} \textbf{think.{\Pfv}-{\Nw}.{\Rem}} {\Subord} wait {\Fsg}.{\Erg} {\Med} girl away {\F}.{\Sg}-take.{\Anim}.{\Pfv}-{\Npst}-{\Fsg}.{\Erg} {\Spl}-{\Adess}={\Quot} \\
			\trans `The fox liked that girl [and] he thought: ``Wait, I will take this girl away from you people.{''}'
			\hfill (E153-31)
		
		
	\end{xlist}
\end{exe}

\subsubsection{Non-Witnessed Perfect}

The Non-Witnessed Perfect is formed by a Past Participle together with a form of the verb `be' in the Non-Witnessed Aorist. It is used as a Perfect (i.e. resultative/current relevance or experiential), but with explicit non-witnessed evidentiality:

\begin{exe}
	\ex\label{verbflex-ex30}
	\begin{xlist}
		
		
			\ex\label{verbflex-ex30a}
			\gll duq že d-av-d-i-nor. e moq eħat \textbf{b-a-n} \textbf{aɬ-in}: \\
			many sheep {\D}-die-{\D}-{\Tr}-{\Nw}.{\Rem} {\Prox} song then \textbf{{\B}.{\Sg}-be-{\Nw}.{\Aor}} \textbf{say.{\Pfv}-{\Ptcp}.{\Pst} }\\
			\trans `Many sheep were (apparently) killed [by the frost]. This song was (apparently) composed in those days:' [The song itself follows] \\
			\hfill (E032-2)
		
		
		
			\ex\label{verbflex-ex30b}
			\gll xalx liv men, saneb loum-rena \textbf{d-a-n} aħ \textbf{d-eɁ-enu}=jn\u{o}. \\
			people say.{\Ipfv}({\Npst}) {\Subord} trinity mountain-{\Abl} \textbf{{\D}-be-{\Nw}.{\Aor}} down \textbf{{\D}-bring-{\Ptcp}.{\Pst}}={\Quot} \\
			\trans `People say that the trinity [icon] was (apparently) brought down from the mountains.'
			\hfill (E017-24)
		
		
	\end{xlist}
\end{exe}

\subsubsection{Non-Witnessed Pluperfect}

The Non-Witnessed Pluperfect is formed by a Past Participle together with a form of the verb `be' in the Non-Witnessed Remote past. It is used as a Pluperfect (i.e. a past-before-past), but with explicit non-witnessed evidentiality (see Example (\ref{verbflex-ex31})).


\begin{exe}
	\ex\label{verbflex-ex31}
	
	\gll išt'eⁿ ese \textbf{d-a-nor} ʕexk' \textbf{d-iš-d-i-en}. \\
	such.{\Prox} here({\Ess}) \textbf{{\D}-be-{\Nw}.{\Rem}} iron \textbf{{\D}-strike-{\D}-{\Tr}-{\Ptcp}.{\Pst}} \\
	\trans `Right here he must have been hit by iron.'
	\hfill (E041-66)
	
\end{exe}

\subsection{Georgian influence} \label{evidgeo}\is{Georgian influence!Morphological}

An in-depth analysis of the use of the different Tsova-Tush evidential forms will be provided in \textcite{wsverhees2024nakhevid}. The main conclusions of this article are:

\begin{enumerate}
	\item The origin of evidentiality as a category in Tsova-Tush is not likely to be due to Georgian influence. Evidentiality as a category is a very common feature of all languages of the Caucasus, from all families (\cite{friedman1996balkancaucasus,friedman2000evidential}). The use of a perfect-like form (past participle + `be') to express non-witnessed evidential semantics is common in Kartvelian as well as in several East Caucasian languages (\cite{forker18,verhees19}).
	\item The increase in use of the Tsova-Tush Perfect as the main Non-Witnessed evidential form can perhaps be attributed to Georgian influence. In East Georgian dialects (Kakhetian, Tush), as well as in Standard Georgian, the expression of non-witnessed evidentiality (and other marked status types, such as admirative and dubitative) is primarily through the use of the perfect (\cite{boeder2000evidentiality}). Actual proof of contact-induced change, however, is difficult here, since the other Nakh languages Chechen and Ingush feature similar verb formations with similar semantics (perfect, resultative, evidential).
\end{enumerate}

\section{Person marking} \label{person} \is{Cross-referencing!Person}

In Tsova-Tush, verbs show agreement with a first or second person subject, marked by a suffix corresponding to a personal pronoun (\cite{gagua52}) (see \sectref{perspro} for personal pronouns). The development of person marking has been described extensively by \textcite{kojima19}, on which large parts of this section are based. 

\subsection{Description} \label{persondescription}

Tsova-Tush indicates the clausal 1st or 2nd person subject by suffixing morphemes to the verb form, following TAME marking. The third person is unmarked (or marked by the absence of overt marking), nor is the 1st person plural inclusive marked on the verb. Since Tsova-Tush features a particular version of an ergative alignment system (see \sectref{valency}), markers that are formally identical to personal pronouns in the Ergative are suffixed to all transitive verbs. They are also suffixed to intransitive verbs that require an Ergative subject in 1st and 2nd person, and to intransitive verbs that allow 1st or 2nd person Ergative subjects when expressing a high degree of volition or control. See Table \ref{verbalperson-table1} for two verbs requiring Ergative subject marking in 1st and 2nd person. Here, the intransitive verb \textit{d-ot'ar} `go (\textsc{ipfv})' and the transitive verb \textit{aɬar} `say (\textsc{pfv})' are given in the three most frequently occuring tense-aspect forms, Non-Past (see \sectref{ind} for this tense label), Imperfect and Aorist. The verb \textit{d-ot'ar} `go' is given with a neutral gender marker \textit{d-} as a place holder. As can be seen from the table, the Non-Past vowel \textit{-u} is regularly syncopated in the penultimate syllable (see \sectref{processes}). This process, however, does not occur with verbs that have the Non-Past vowel \textit{-o}, such as \textit{aɬar} `say'.\footnote{The vowel sequence \textit{o-a} in \textit{aɬ-o-as} first contracted to \textit{o}, removing the condition for the syncope rule to apply. No such merger takes place with the vowels \textit{u-a} in \textit{d-ot'-u-as}, hence the vowel \textit{u} is regularly syncopated, triggering i-umlaut, see \sectref{processes}.} Note that the two verbs in Table \ref{verbalperson-table1} differ by the choice of Non-Past vowel, irrespective of the aspect of the verbal stem (see \sectref{ind}).




\begin{table}
	\begin{tabular}{l ll ll}
		\lsptoprule
	     & \multicolumn{2}{c}{`go.\textsc{ipfv}'}     &  \multicolumn{2}{c}{`say.\textsc{pfv}'}\\\cmidrule(lr){2-3}\cmidrule(lr){4-5}
		 & {Morphemes} & {Surface form}               & {Morphemes} & {Surface form} \\
		\midrule
		
		\multicolumn{5}{l}{\itshape Non-Past} \\
		\textsc{1sg} & \textit{d-ot'-u-as} & \textit{duit'as} & \textit{aɬ-o-as} & \textit{aɬos} \\
		\textsc{2sg} & \textit{d-ot'-u-aħ} & \textit{duit'a} & \textit{aɬ-o-aħ} & \textit{aɬo} \\
		\textsc{1pl(excl)} & \textit{d-ot'-u-atx} & \textit{duit'atx}  & \textit{aɬ-o-atx} & \textit{aɬotx} \\
		\textsc{2pl} & \textit{d-ot'-u-ejš} & \textit{duit'ejš} & \textit{aɬ-o-ejš} & \textit{aɬuiš} \\
		\textsc{3} & \textit{d-ot'-u} & \textit{duit'\u{u}}  & \textit{aɬ-o} & \textit{aɬ\u{o}} \\\addlinespace
		
		\multicolumn{5}{l}{\itshape Imperfect} \\
		\textsc{1sg} & \textit{d-ot'-ura-as} & \textit{duit'ras}  & \textit{aɬ-ora-as} & \textit{aɬras} \\
		\textsc{2sg} & \textit{d-ot'-ura-aħ} & \textit{duit'ra}  & \textit{aɬ-ora-aħ} & \textit{aɬra} \\
		\textsc{1pl(excl)} & \textit{d-ot'-ura-atx} & \textit{duit'ratx} & \textit{aɬ-ora-atx} & \textit{aɬratx}\\
		\textsc{2pl} & \textit{d-ot'-ura-ejš} & \textit{duit'rejš} & \textit{aɬ-ora-ejš} & \textit{aɬrejš} \\
		\textsc{3} & \textit{d-ot'-ura} & \textit{dot'ur}  & \textit{aɬ-ora} & \textit{aɬor} \\\addlinespace
		
		\multicolumn{5}{l}{\itshape Aorist} \\
		\textsc{1sg} & \textit{d-ot'-in-as} & \textit{duit'nas}  & \textit{aɬ-in-as} & \textit{ejɬnas}\\
		\textsc{2sg} & \textit{d-ot'-in-aħ} & \textit{duit'na}  & \textit{aɬ-in-aħ} & \textit{ejɬna} \\
		\textsc{1pl(excl)} & \textit{d-ot'-in-atx} & \textit{duit'natx} & \textit{aɬ-in-atx} & \textit{ejɬnatx} \\
		\textsc{2pl} & \textit{d-ot'-in-ejš} & \textit{duit'nejš}  & \textit{aɬ-in-ejš} & \textit{ejɬnejš} \\
		\textsc{3} & \textit{d-ot'-in} & \textit{dot'iⁿ}  & \textit{aɬ-in} & \textit{aɬiⁿ}\\
		\lspbottomrule
		
	\end{tabular}
	\caption{Tsova-Tush Ergative person marking}
	\label{verbalperson-table1}
\end{table}

Intransitive verbs that do not allow for Ergative subjects must and intransitive verbs that allow for variable marking may add a suffix identical to a personal pronoun in the Nominative to the verbal TAME inflected stem. See Table \ref{verbalperson-table2}, where the verb \textit{lattar} `stand' (which allows for variable marking) is given in the three most common tense-aspect forms. Note that final \textit{-\u{o}} and \textit{-\u{u}} (both phonetically [w]) can alternatively be dropped in rapid speech, erasing the overt distinction between Nominative and Ergative marking in the Imperfect. Compare for example \textit{lattras} (< \textit{latt-era-so}) and \textit{aɬras} (< \textit{aɬ-ora-as}). In contrast to the Ergative marking in Table \ref{verbalperson-table1}, verbs with Nominative person marking do not differ according to the choice of Non-Past vowel.

\begin{table}
	\begin{tabular}{lll}
		\lsptoprule
		 & \multicolumn{2}{c}{`stand'}\\\cmidrule(lr){2-3}
		 & Morphemes & Surface form \\
	\midrule
		
		\multicolumn{3}{l}{\emph{Present}} \\
		
		\textsc{1sg} & \textit{latt-e-so} & \textit{lattes\u{o}} \\
		
		\textsc{2sg} & \textit{latt-e-ħo} & \textit{latteħ\u{o}} \\
		
		\textsc{1pl(excl)} & \textit{latt-e-txo} & \textit{lattetx\u{o}} \\
		
		\textsc{2pl} & \textit{latt-e-šu} & \textit{lattiš\u{u}} \\
		
		\textsc{3} & \textit{latt-e} & \textit{latt} \\\addlinespace
		
		\multicolumn{3}{l}{\emph{Imperfect}} \\
		
		\textsc{1sg} & \textit{latt-era-so} & \textit{lattras\u{o}} \\
		
		\textsc{2sg} & \textit{latt-era-ħo} & \textit{lattraħ\u{o}} \\
		
		\textsc{1pl(excl)} & \textit{latt-era-txo} & \textit{lattratx\u{o}} \\
		
		\textsc{2pl} & \textit{latt-era-šu} & \textit{lattrejš\u{u}} \\
		
		\textsc{3} & \textit{latt-era} & \textit{latter} \\\addlinespace
		
		\multicolumn{3}{l}{\emph{Aorist}} \\
		
		\textsc{1sg} & \textit{latt-in-so} & \textit{lattiⁿs\u{o}} \\
		
		\textsc{2sg} & \textit{latt-in-ħo} & \textit{lattiⁿħ\u{o}} \\
		
		\textsc{1pl(excl)} & \textit{latt-in-txo} & \textit{lattiⁿtx\u{o}} \\
		
		\textsc{2pl} & \textit{latt-in-šu} & \textit{lattiⁿš\u{u}} \\
		
		\textsc{3} & \textit{latt-in} & \textit{lattiⁿ} \\
		\lspbottomrule
		
		
	\end{tabular}
	\caption{Tsova-Tush Nominative person marking}
	\label{verbalperson-table2}
\end{table}

The fact that these cross-reference markers are part of the verb form and no longer pronouns is clearly shown by \textcite{harris09}, and the fact that they are not clitic pronouns is shown by \textcite{harris11} and \textcite{kojima19}. In short:\is{Clitic pronouns}
\begin{itemize}
	\item The verb-pronoun complex is clearly one phonological word, as the phonological rules (Nasalisation, Apocope, Syncope) apply to the whole complex, not to the TAME form and the pronoun individually (\cite[281--284]{harris09}). Hence we find the surface form \textit{duit'nas} (and not \textit{dot'iⁿ as}) from \textit{d-ot'-in-as}, and \textit{lattes\u{o}} (and not \textit{latt so}) from \textit{latt-e-so}.
	
	\item The Ergative and Nominative agreement markers occur exclusively with verbs; they feature an arbitrary gap --- the first person inclusive does not agree; and verbs with person agreement are treated syntactically as single words, not affix groups. All these criteria indicate that the markers are affixes, not clitics. The only counterargument is the placement of the question clitic \textit{=i}, which occurs between the verbal form and the agreement marker (\cite[141--144]{harris11}).

	\item The agreement markers are obligatory when expressing the 1st or 2nd person, and can occur in combination with free pronouns (\cite[285]{kojima19}). Thus, \textit{so vas\u{o}} `I am' and \textit{as vuit'as} are grammatical, whereas *\textit{so va} and *\textit{as vuit'\u{u}} are not (anymore).
\end{itemize}

In Tsova-Tush, pronominal direct objects in the Nominative case, as well as Oblique (indirect) objects in the Dative, Allative or Contact case, can be cliticised to the verb form. These clitics do form a part of the phonological word, as is best exemplified by the Nominative object pronoun. In Example (\ref{verbflex-ex32b}), the vowel of the object pronoun \textit{so} `me' is reduced when cliticised, but the Non-Past vowel is not. However, the cliticisation is optional, as shown by Example (\ref{verbflex-ex32a}), and cliticised pronouns cannot occur in combination with free pronouns (\ref{verbflex-ex32c}).


\begin{exe}
	\ex\label{verbflex-ex32}
	\begin{xlist}
		
		
			\ex\label{verbflex-ex32a}
			\gll  oqu-s so v-ik'-\u{o}. \\
			{\Dist}.{\Obl}-{\Erg} {\Fsg}.{\Nom} {\M}.{\Sg}-take.{\Anim}-{\Npst} \\
			\trans `S/he takes me (somewhere).'
			\hfill (\cite[286]{kojima19})
		
		
		
			\ex\label{verbflex-ex32b}
			\gll oqu-s v-ik'-o=so. \\
			{\Dist}.{\Obl}-{\Erg} {\M}.{\Sg}-take.{\Anim}-{\Npst}={\Fsg}.{\Nom} \\
			\trans `S/he takes me (somewhere).'
			\hfill (\cite[286]{kojima19})
		
		
		
			\ex\label{verbflex-ex32c}
			\gll *oqu-s so v-ik’-o=so.  \\
			{\Dist}.{\Obl}-{\Erg} {\Fsg}.{\Nom} {\M}.{\Sg}-take.{\Anim}-{\Npst}={\Fsg}.{\Nom}\\
			\trans `S/he takes me (somewhere).'
			\hfill (\cite[287]{kojima19})
		
		
	\end{xlist}
\end{exe}


Example (\ref{verbflex-ex33}) shows pronouns in the Dative, Allative and Contact case cliticised to the verb form.\is{Pronouns!Personal}

\begin{exe}
	\ex\label{verbflex-ex33}
	\begin{xlist}
		
		
			\ex\label{verbflex-ex33a}
			\gll niq' co \textbf{b-aɬ-iⁿ=soⁿ}. \\
			road {\Neg} \textbf{{\B}.{\Sg}-give.{\Pfv}-{\Aor}={\Fsg}.{\Dat}} \\
			\trans `They didn't give me a road.'
			\hfill (E171-20)
		
		
		
		
			\ex\label{verbflex-ex33b}
			\gll uis c'era-l-a=en, \textbf{aɬ-iⁿ=so-g}. \\
			there({\Lat}) write-{\Intr}-{\Npst}={\Quot} \textbf{say.{\Pfv}-{\Aor}={\Fsg}-{\All}} \\
			\trans `{``}There it is written,'' he told me.'
			\hfill (E305-4)
		
		
		
			\ex\label{verbflex-ex33c}
			\gll inc o-bi d-a\u{g}-\u{o}, d-aq'-ar \textbf{d-ex=so-x}. \\
			now {\Dist}-{\Pl} {\D}-come-{\Npst} {\D}-eat-{\Vn} \textbf{{\D}-ask({\Npst})={\Fsg}-{\Cont}} \\
			\trans `They come and ask me for food.'
			\hfill (WS001-12.36)
		
	\end{xlist}
\end{exe}

\textcite{kojima19} argues that Nominative objects form a closer phonological bond with the verb form than objects in Oblique cases. When a Nominative object pronoun is cliticised to a verb form, the phonological rules (in this case apocope) do not apply on the last vowel of the TAME form, but instead apply to the object pronoun itself (see Example (\ref{verbflex-ex32b})), whereas these rules do apply in verb forms with an Oblique object (\ref{verbflex-ex33}): the Aorist ending \textit{-in} is realised as \textit{-iⁿ} in (\ref{verbflex-ex33a}) and (\ref{verbflex-ex33b}), and the Non-Past vowel \textit{-o} is deleted in (\ref{verbflex-ex33c}). However, in an alternative analysis, nasalisation also occurs word-internally (see \sectref{processes}), and the Non-Past vowel \textit{-o} has been elided as a result of syncope, since it would be in the penultimate syllable. Hence, the Oblique object pronouns can be said to be equal to the Nominative object pronouns in terms of the degree of cliticisation. That is, all object pronouns are part of the same phonological word as the verb itself, but none are affixes.

A cliticised direct or indirect object pronoun can be attached to a verb that already has subject person marking, although this is rare (see Example (\ref{verbflex-ex40})).



	\begin{exe}
		\ex\label{verbflex-ex40}
		\gll  as t'q'oɁ l-o-\textbf{s=ħoⁿ} krtam=en so-guiħ aɬ-iⁿ. \\
		{\Fsg}.{\Erg} again give.{\Pfv}-{\Npst}-\textbf{{\Fsg}.{\Erg}={\Ssg}.{\Dat}} bribe={\Quot} {\Fsg}-{\Apudlat} say.{\Pfv}-{\Aor} \\
		\trans `{``}I will give you another bribe, to my benefit,'' he said.'
		\hfill (WS001-11.14)
	\end{exe}


Lastly, Dative subjects of experiential verbs can also cliticise to the verb, and behave in the same way as Dative indirect objects: no additional free pronoun can be used in combination with the verb-pronoun complex (see Example (\ref{verbflex-ex44})).\is{Experiential verbs}


	\begin{exe}
		\ex\label{verbflex-ex44}
		\gll seⁿ bader, \textbf{g-u=soⁿ} me \u{g}azeⁿ st'ak' v-a-ħ. \\
		{\Fsg}.{\Gen} child \textbf{see-{\Npst}={\Fsg}.{\Dat}} {\Subord} good man {\M}.{\Sg}-be-{\Ssg}.{\Nom} \\
		\trans `My child, I see that you are a good man.'
		\hfill (WS001-10.11)
	\end{exe}



\subsection{Development}

In texts from the middle of the 19th century, represented by subcorpora IT and AS in this work, 1st and 2nd person subject pronouns can be cliticised to the verb (\cites[55--56]{schiefner56}[280--283]{kojima19}). The pronouns following the verb as clitics are optional, and do not occur in combination with free pronouns preceding the verb. In Example (\ref{verbflex-ex34}), which is one narrative sequence, we see (semantically) free variation between free pronouns preceding the verb (\ref{verbflex-ex34a})\footnote{Original orthography of (\ref{verbflex-ex34a}): so ma wain woitu maclex!}, (\ref{verbflex-ex34c})\footnote{Original orthography of (\ref{verbflex-ex34c}): as bie qa Daln ḥatxe.} and pronouns cliticised to the end of verbs (\ref{verbflex-ex34b})\footnote{Original orthography of (\ref{verbflex-ex34b}): x̣e\'{t}wes, \.{g}os sai dadego e aḽos ox̣ugo: << dad!}. No single example of a verb with both a free and a cliticised pronoun has been found in subcorpora IT or AS. Hence, at this stage, they are cliticised pronominal arguments, and not agreement markers (\cite[283]{kojima19}).\is{Georgian influence!Morphological}



\begin{exe}
	\ex\label{verbflex-ex34}
	\begin{xlist}
		
		
		\ex\label{verbflex-ex34a}
		\gll \textbf{so} ma v-ail-n \textbf{v-oit'-u} mac-l-e-x! \\
		\textbf{{\Fsg}.{\Nom}} but {\M}.{\Sg}-die-{\Ptcp}.{\Pst} \textbf{{\M}.{\Sg}-go-{\Npst}} hungry-{\Nmlz}-{\Obl}-{\Cont} \\
		\trans `But I am starving!'
		\hfill (AS006-1.6) \\
		
		\ex\label{verbflex-ex34b}
		\gll \textbf{qett-v-e-s,} \textbf{\u{g}-o-s} saiⁿ dad-e-go je \textbf{aɬ-o-s} oqu-go: dad!  \\
		\textbf{stand\_up-{\Npst}-{\Seq}-{\Fsg}.{\Erg}} \textbf{go.{\Pfv}-{\Npst}-{\Fsg}.{\Erg}} {\Fsg}.{\Gen} father-{\Obl}-{\All} and \textbf{say.{\Ipfv}-{\Npst}-{\Fsg}.{\Erg}} {\Dist}.{\Obl}-{\All} father \\
		\trans `I will stand up, go to my father and say to him: ``Father!{''}'
		\hfill (AS006-1.7) \\
		
		
		
		\ex\label{verbflex-ex34c}
		\gll \textbf{as} \textbf{b-i-eⁿ} q'a dal-n ħatxe. \\
		\textbf{{\Fsg}.{\Erg}} \textbf{{\B}.{\Sg}-do.{\Pfv}-{\Aor}} sin God-{\Dat} in\_front\_of \\
		\trans `{``}I have sinned before God.{''}'
		\hfill (AS006-1.8)
		
		
	\end{xlist}
\end{exe}

By the middle of the 20th century, the situation has changed. Texts from this period (subcorpora YD and KK) still show variation between verbs with (Example (\ref{verbflex-ex35b})\footnote{Original orthography of (\ref{verbflex-ex35b}): „Марши буисва хьон, Сандро!'' — ливас.}) and without (\ref{verbflex-ex35c})\footnote{Original orthography of (\ref{verbflex-ex35c}):     Атх\u{o} пхьор дакъи.} cliticised personal pronouns, but at this time, agreement marking in combination with a free pronoun is attested, albeit rare (\cites[84]{desheriev53}[284]{kojima19}). See Example (\ref{verbflex-ex35a})\footnote{Original orthography of (\ref{verbflex-ex35a}): Атхо навт\~{у} ламп дıьаи'натх\u{o}, стıoл мак хаибжнатх\u{o}.}. \textcite{chrelashvili82} considers personal marking optional but observes it is very prevalent in Tsova-Tush spoken by younger speakers. If a verb is inflected without person marking, a personal pronoun in the 1st or 2nd person is obligatory, whereas if person marking is used, the pronoun is optional. Hence, subject agreement marking in this period is possible, but not obligatory.

\begin{exe}
	\ex\label{verbflex-ex35}
	\begin{xlist}
		
		\ex\label{verbflex-ex35a}
		\gll \textbf{atx\u{o}} navtu-ⁿ lamp \textbf{d-ʕaiɁ-n-atx\u{o}}, st'ol-mak \textbf{xaibž-n-atx\u{o}}.  \\
		\textbf{{\Fpl}.{\Erg}} kerosene-{\Gen} lamp \textbf{{\D}-light.{\Pfv}-{\Aor}-{\Fpl}.{\Erg}} table-{\Superlat} \textbf{sit\_down.{\Pfv}.{\Pl}-{\Aor}-{\Fpl}.{\Erg}} \\
		\trans `We lit a kerosene lamp, and sat down at the table.'

		\hfill (YD005-27.1)
		
		
		\ex\label{verbflex-ex35b}
		\gll ``maršiⁿ buisv-a ħon, sandro!'' - \textbf{liv-as}. \\
		peaceful at\_night-{\Nmlz} {\Ssg}.{\Dat} Sandro {} \textbf{say.{\Ipfv}({\Npst})-{\Fsg}.{\Erg}} \\
		\trans `{``}Good night, Sandro!'' I said.'

		\hfill (YD005-31.2) 
		
				\ex\label{verbflex-ex35c}
		\gll \textbf{atx\u{o}} pħor \textbf{d-aq'-iⁿ}. \\
		\textbf{{\Fpl}.{\Erg}} dinner \textbf{{\D}-eat.{\Pfv}-{\Aor}} \\
		\trans `We had dinner.'
		\hfill (YD005-24.1)
		
		
		
	\end{xlist}
\end{exe}

As described above in \sectref{persondescription}, person agreement in 1st and 2nd person is now obligatory in contemporary Tsova-Tush (represented by subcorpora E, MM and recent fieldwork). A free personal pronoun can be present for emphasis\footnote{The exact nature of this emphasis (contrastive focus, introduction of theme, or other) has yet to be investigated more thoroughly.} (\ref{verbflex-ex36a}) or absent (\ref{verbflex-ex36b}).

\begin{exe}
	\ex\label{verbflex-ex36}
	\begin{xlist}
		
		
			\ex\label{verbflex-ex36a}
			\gll \textbf{as}=i \textbf{b-ʕiv-n-as}=en. \\
			\textbf{{\Fsg}.{\Erg}}={\Q} \textbf{{\B}.{\Sg}-kill-{\Aor}-{\Fsg}.{\Erg}}={\Quot} \\
			\trans `{``}Was I the one who killed it?'' / ``Did I kill it?{''}'
			\hfill (E091-45)
		
		
		
			\ex\label{verbflex-ex36b}
			\gll d-ʕivɁ šar-e \textbf{gag-b-i-n-as}. \\
			{\D}-four year-{\Obl}({\Ess}) \textbf{care\_for-{\B}.{\Sg}-{\Tr}-{\Aor}-{\Fsg}.{\Erg}} \\
			\trans `I have cared for it for four years.'
			\hfill (E027-19)
		
		
	\end{xlist}
\end{exe}

\subsection{Georgian, Vainakh and Daghestanian}

The Georgian verb can mark the grammatical subject, direct object and indirect object (\cite[244--247]{aronson91}), although indirect object agreement is often viewed as derivational (\cite{gerardin2022valencegeo}). See Table \ref{verbalperson-table3}, taken from \textcite[85]{vogt} below, for the subject and direct object agreement markers.

\begin{table}
	\begin{tabular}{cllll lll}
    \lsptoprule
		& & \multicolumn{6}{c}{{Object}} \\ 
		& & \textsc{1sg} & \textsc{2sg} & \textsc{3sg} & \textsc{1pl} & \textsc{2pl} & \textsc{}3pl \\\cmidrule(lr){3-5}\cmidrule(lr){6-8}
		
		\multirow{6}{*}{\STAB{\rotatebox[origin=c]{90}{{Subject}}}}
		&  \textsc{1sg} & & \textit{g-} & \textit{v-} & & \textit{g-} \hfill \textit{-t} & \textit{v-} \\
		
		& \textsc{2sg}  & \textit{m-} & & (\textit{h-}) & \textit{gv-} & & (\textit{h-}) \\
		
		& \textsc{3sg}  & \textit{m-} \hfill \textit{-s} & \textit{g-} \hfill \textit{-s} & \hfill \textit{-s} & \textit{gv-} \hfill \textit{-s} & \textit{g-} \hfill \textit{-s} & \hfill \textit{-s} \\\cmidrule(lr){3-5}\cmidrule(lr){6-8}
		
		& \textsc{1pl}  &  & \textit{g-} \hfill \textit{-t} & \textit{v-} \hfill \textit{-t} & & \textit{g-} \hfill \textit{-t} & \textit{v-} \hfill \textit{-t} \\
		
		& \textsc{2pl}  & \textit{m-} \hfill \textit{-t} & &(\textit{h-}) \hfill \textit{-t} & \textit{gv-} \hfill \textit{-t} & & (\textit{h-}) \hfill \textit{-t} \\
		
		& \textsc{}3pl & \textit{m-} \hfill \textit{-en} & \textit{g-} \hfill \textit{-en} & \hfill \textit{-en} & \textit{gv-} \hfill \textit{-en} & \textit{g-} \hfill \textit{-en} & \hfill \textit{-en} \\
        \lspbottomrule
	\end{tabular}
	\caption{Georgian person agreement affixes}
	\label{verbalperson-table3}
\end{table} 

Even though Georgian features a split ergative alignment system, person agreement marking follows an accusative pattern (\cite[267]{aronson91}). That is, the same agreement marker is used to cross-reference the subject, irrespective of the transitivity or tense-aspect form of the verb, see Example (\ref{verbflex-ex37}).

\begin{exe}
	\ex\label{verbflex-ex37}
	Modern Georgian
	\begin{xlist}
		
		
			\ex\label{verbflex-ex37a}
			\gll \textbf{v-}k'vdebi {} {//} {} mo-\textbf{v}-k'vdi \\ 
			{`I die'} {} {} {} {`I died'}\\
		
		
		
			\ex\label{verbflex-ex37b}
			\gll \textbf{v-}k'lav {} {//} {} mo-\textbf{v}-k'ali \\ 
			{`I kill'} {} {} {} {`I killed'} \\
			\hfill (adapted from \cite[291]{kojima19})
		
		
	\end{xlist}
\end{exe}

In several Chechen dialects, some verbs show person agreement using root ablaut. The same is true for one Ingush verb, see Example (\ref{verbflex-ex38}) (\cites[439]{nichols11}[103]{nichols94Ing}), which is the only verb in Ingush with person agreement. 

\begin{exe}
	\ex\label{verbflex-ex38}
	Ingush
	\begin{xlist}
		
		
			\ex\label{verbflex-ex38a}
			\gll \={a}z j\={a}x \\
			{\Fsg}.{\Erg} say({\Prs}) \\
			\trans `I say'
		
		
		
			\ex\label{verbflex-ex38b}
			\gll cuo joax \\
			{\Tsg}.{\Erg} say(3:{\Prs})    \\
			\trans `S/he says'
		
		
	\end{xlist}
\end{exe}

Some dialects of Chechen allow for cliticisation of personal pronouns to the finite verb form (\cites[210--213]{imnaishvili68}). This, however, does not equal person marking, since the cliticisation is optional, and not compatible with a corresponding free pronoun in the same clause (\cites[279]{kojima19}).

Although most East Caucasian languages do not feature person agreement marking at all, it is possible to find instances, or at least elements, of personal agreement in every subgroup of the family (\cite[131]{helmbrecht96}). These agreement patterns show considerable variety, which leads to the general assumption that the development of person marking is a secondary process which happened individually in the different languages. In some languages, person agreement shows a clear origin in free personal pronouns (besides Tsova-Tush also Tabasaran and Udi (\cites[143]{helmbrecht96})). See Example (\ref{verbflex-ex39}).\il{Udi}

	\begin{exe}
		\ex\label{verbflex-ex39}
		Udi (Vartashen)
        
		\gll \textbf{zu} bu-\textbf{zu} jaq' va do\u{g}rilu\u{g} va kar-x-esun. \\
		\textbf{{\Fsg}} be-\textbf{{\Fsg}} way and truth and live-{\Lv}-{\Vn}2 \\
		\trans `I am the way and the truth and the life.'
		\hfill (John 14:6, \cites{schulze11})
	\end{exe}




Instances of person marking in other languages (Lak, various Dargic languages, Akhwakh, Zaqatala Avar) betray fewer clues as to their origin. Interestingly, of languages with pronoun-based person marking, it has been claimed that this type of agreement has developed through language contact: Tabasaran in contact with Southern Dargwa, and (Old) Udi in contact with Old Armenian and/or Iranian languages (\cite{schulze11}). In contrast, in languages where person marking has an origin other than personal pronouns (Akhwakh (\cites{creissels08}), Dargic person-marking clitics (\cites{sumbatova11})), language contact is claimed to not be a viable explanation.

\subsection{Discussion}

Tsova-Tush has developed person marking on the verb, being the only Nakh language to have done so. Subject personal pronouns, whether they are Nominative or Ergative, have cliticised to the verb, and have become agreement markers over the course of the past 200 years. Nominative direct objects and Oblique direct and indirect objects may cliticise to the end of verb forms but have not become agreement markers: their cliticisation is optional and does not occur in combination with the same pronoun preceding the verb. It is extremely likely that the development of person agreement in Tsova-Tush is due to Georgian influence. What is expressed by a single verb form in Georgian is coded by 1st or 2nd person subject agreement marking in Tsova-Tush, with an optional direct or indirect object cliticised personal pronoun (\cites{kojima19}).

Recall that Tsova-Tush also features another cross-reference marking system: gender. As observed by \textcite{chrelashvili82}, if we compare Tsova-Tush verbs of the type \textit{d-ust'-o-s} `\textsc{d}-measure-\textsc{npst-1sg.erg} / I measure it' with Georgian \textit{m-xat'-av-s} `\textsc{o1sg}-paint-\textsc{npst-s3sg} / he paints me', we find identical structures: object-root-subject. However, this similarity might be accidental: in Georgian only the 3rd person subject cross-reference marker is suffixed (the 1st and 2nd person subject markers are prefixed), and in Tsova-Tush only 1st and 2nd person subject markers exist. 

With respect to a 3rd person cross-reference marker, \textcite[85]{desheriev53} explains the lack of such a suffix in Tsova-Tush by pointing to the fact that the 3rd person Nominative singular pronoun is \textit{o}, which would have (or has) assimilated with the vocalic ending of most verbs. However, this hypothesis does not explain why a 3rd person Ergative \textit{oqus} is not found either, nor is a 3rd person plural pronoun (\textsc{nom} \textit{obi}, \textsc{erg} \textit{oqar}). Instead, we can observe from other Caucasian languages, as well as from cross-linguistic data, that person agreement marking that exclusively marks speech act participants is not rare. See for instance Dargic languages that only mark 1st and 2nd person (Mehweb only marks 1st person) (\cites{magometov62}[138]{helmbrecht96}{sumbatova11}). This conforms perfectly to the established hierarchy of person marking 1>2>3 (\cites[152]{givon76}[162]{croft88} as cited in \cites[128]{helmbrecht96}). An explanation for this cross-linguistic observation has been stated in terms of frequency asymmetry: 3rd persons are (presumably) more frequent in discourse than speech-act participants (\cite{haspelmath2021assymmetry}). Hence, there is no need for an (ad hoc) language-internal explanation for the absence of a 3rd person cross-reference marker in Tsova-Tush, as it already conforms to typological tendencies.



\section{Suffixal number marking} \label{suffixpl}\is{Cross-referencing!Number}

As discussed in \sectref{imp}, a marker \textit{-t} is added to a tense-aspect form of the verb to indicate argument plurality in imperative TAME forms, including the Imperative, Polite Imperative, and Optative. Furthermore, the Hortative is formed by adding the same plural marker and the inclusive 1st person plural pronoun \textit{vej}, but this hortative construction plus \textit{-t} is only used when referring to more than two participants.

There are, however, more contexts in which the plural marker \textit{-t} occurs: in the indicative TAME forms when the subject is 1st person plural inclusive, see Example (\ref{verbflex-ex41}).


\begin{exe}
	\ex\label{verbflex-ex41}
	\begin{xlist}
		
		
			\ex\label{verbflex-ex41a}
			\gll išt' b-ec'-e-\textbf{t} \textbf{vej} mokcevad-b-al-aⁿ, daħ b-ec'-e-\textbf{t} \textbf{vej} gantavisuplebad-b-al-aⁿ.\\
			so.{\Prox} {\M}.{\Pl}-need-{\Npst}-\textbf{{\Pl}} \textbf{{\Fpl}.{\Incl}} behave.{\Pfv}-{\M}.{\Pl}-{\Intr}-{\Inf} {\Pv} {\M}.{\Pl}-need-{\Npst}-\textbf{{\Pl}} \textbf{{\Fpl}.{\Incl}} liberate.{\Pfv}-{\M}.{\Pl}-{\Intr}-{\Inf} \\
			\trans `This is how we must behave, we must free ourselves!'
			\hfill  (E146-41)
		
		
		
			\ex\label{verbflex-ex41b}
			\gll  joħ daħ j-aq-o-\textbf{t} \textbf{vaj} aɬ-iⁿ. \\
			girl {\Pv} {\F}.{\Sg}-take.{\Pfv}-{\Npst}-\textbf{{\Pl}} \textbf{{\Fpl}.{\Incl}} say.{\Pfv}-{\Aor} \\
			\trans `We will take your girl, they said.'
			\hfill (E153-53)
		
		
		
			\ex\label{verbflex-ex41c}
			\gll  qečnariⁿ iš xec'-če daħ at'ar-d-al-in-\textbf{t} \textbf{ve}. \\
			strange voice hear.{\Pfv}-{\Ante} {\Pv} become\_silent-{\D}-{\Intr}-{\Aor}-\textbf{{\Pl}} \textbf{{\Fpl}.{\Incl}} \\
			\trans `Upon hearing a strange sound, we fell silent.'
			\hfill (BH046-23.1)
		
		
	\end{xlist}
\end{exe}

Furthermore, the \textit{-t} suffix is attested when the verb involves the 1st person inclusive pronoun as a direct or indirect object, see Example (\ref{verbflex-ex42}).

\begin{exe}
	\ex\label{verbflex-ex42}
	\begin{xlist}
		
		
			\ex\label{verbflex-ex42a}
			\gll ve dad-i-v b-is-b-aq-in-\textbf{t} \textbf{ve} mastxo-xiⁿ. \\
			{\Fpl}.{\Incl}.{\Poss}.{\Obl} father-{\Pl}-{\Erg} {\M}.{\Pl}-rescue-{\M}.{\Pl}-{\Lv}-{\Aor}-\textbf{{\Pl}} \textbf{{\Fpl}.{\Incl}} enemy-{\Apudabl} \\
			\trans `Our fathers protected us from the enemy.'
			\hfill (BH021-3.1)
		
		
		
			\ex\label{verbflex-ex42b}
			\gll qeⁿ maia j-eɁ-en-\textbf{t} \textbf{ve-n}. \\
			then Maia {\F}.{\Sg}-come-{\Aor}-\textbf{{\Pl}} \textbf{{\Fpl}.{\Incl}-{\Dat}}\\
			\trans `Then Maia approached us.'
			\hfill (E171-47)
		
		
	\end{xlist}
\end{exe}



Note that forms inflected for 1st person plural exclusive (as seen in \sectref{person}) do not feature the plural marker. The distribution is therefore clear: \textit{-t} is used
\begin{enumerate}
	\item in plural Imperative TAME forms, which by default are 2nd person
	\item in verb forms where one of the arguments is a 1st person plural inclusive pronoun, except in the Hortative with exactly two participants.
\end{enumerate}

The suffix is already attested in the earliest sources (subcorpora AS and IT), with the same distribution, see Example (\ref{verbflex-ex43}).\footnote{Original orthography of (\ref{verbflex-ex43a}): la\'{t}e\'{t} wai daqan \'{k}ei\'{p}adbalane.}\footnote{Original orthography of (\ref{verbflex-ex43b}): \'{t}eḽ \.{g}o wai xato wai bstuin\.{c}ox.}

\begin{exe}
	\ex\label{verbflex-ex43}
	\begin{xlist}
		
		
		\ex\label{verbflex-ex43a}
		\gll lat-e-\textbf{t} \textbf{vai} d-aq'-an keipad-b-al-an=e. \\
		begin-{\Npst}-\textbf{{\Pl}} \textbf{{\Fpl}.{\Incl}} {\D}-eat.{\Pfv}-{\Inf} feast-{\M}.{\Pl}-{\Intr}-{\Inf}=and\\
		\trans `Let's start eating and feasting!'
		\hfill (AS006-1.14)
		
		
		
		\ex\label{verbflex-ex43b}
		\gll teɬ \u{g}-o vai xat't'-o vai bst'uin-čo-x.  \\
		prefer go-{\Npst} {\Fpl}.{\Incl} ask-{\Npst} {\Fpl}.{\Incl} woman-{\Obl}-{\Cont}\\
		\trans `Let us (two) rather go and ask the woman.'
		\hfill (AS008-10.4)
		
		
	\end{xlist}
\end{exe}


Chechen and Ingush do not feature the same or a similar suffixal number marker, but instead feature argument number marking by apophony of the verbal root vowel, similar to the Tsova-Tush system (see \sectref{verbalnumber}).\is{Georgian influence!Morphological}

As seen in Table \ref{verbalperson-table3}, Georgian also features a suffix \textit{-t}. This suffix is usually glossed as plural, but, as can be seen from the table, is only used in 1st and 2nd person forms.\footnote{However, Georgian does feature so-called ``inverse'' forms where the subject is cross-referenced by (direct or indirect) object agreement marking and vice versa. These include most experiential verbs, as well as transitive verbs in the Perfect and Pluperfect TAME forms. In these forms, the suffix \textit{-t} can refer to 3rd persons. Hence, in the form \textit{u-nd-a-t} \textsc{IO3}-want-\textsc{S3SG-{\Pl}} `they want it', the \textit{-t} indicates the plurality of the 3rd person subject (which is marked by the indirect object marker).}


It has to be noted that the suffix \textit{-t} in Georgian is used for all TAME forms. Georgian also does not feature a clusivity distinction in 1st person plural pronouns. However, both form and function of Tsova-Tush \textit{-t} and Georgian \textit{-t} are sufficiently similar to claim that this must be a borrowed morpheme. 




\section{Summary}

In terms of basic description, this chapter has provided new insight into the following domains:

\begin{enumerate}
	\item Tsova-Tush features a verbal category of Iamitive, since it uses a bound morpheme signifying the meaning `already' in positive clauses and `anymore' in negative clauses, see \sectref{cont}.
	
	\item Although more investigation is warranted, the Tsova-Tush Conditional is a morphologically finite form (since it combines with the morph \textit{-ra} and allows person marking) that is syntactically non-finite (i.e. it behaves as a converb), see \sectref{cond}.
	
	\item It was already known that the verb form in \textit{-ralo} indicated imperfective aspect and non-witnessed evidentiality (it is labeled Imperfect Reported in \textcite[180]{holiskygagua}). We know now that this form is in fact the Past Subjunctive (since it can exhibit subjunctive semantics in other contexts), which has taken on an additional function of background imperfective in narratives that are entirely non-witnessed, see \sectref{evidform}.
	
\end{enumerate}


In terms of structural language contact, this chapter has shown the following parallels between Tsova-Tush and Georgian, only some of which are likely to be attributable to the latter's influence on the former. 

\begin{enumerate}
	
	\item The Georgian influence on the domain of evidentiality falls beyond the scope of this work and will be further explored in \textcite{wsverhees2024nakhevid}. Preliminarily, we can say that the origin of evidentiality as a category in Tsova-Tush is not likely to be due to Georgian influence, but that the increase in use of the Tsova-Tush Perfect as the main Non-Witnessed evidential form can perhaps be attributed to contact with Georgian. 
	
	\item Even though both languages make use of a verb form labelled \textsc{pluperfect}, the counter-factual conditional constructions are sufficiently different in Tsova-Tush and Georgian that it is unwarranted to assume any Georgian influence here.
	
	\item Tsova-Tush has developed a system of subject person agreement marking, and it is thus able to express the same information as a single Georgian polypersonal finite verb form by combining gender marking with person marking.
\end{enumerate}

In terms of matter borrowing, this chapter has shown that: 

\begin{enumerate}
	
	\item Tsova-Tush has borrowed the Georgian strategy of distinguishing perfective and imperfective verb forms by means of a preverb, along with the borrowed verbs themselves, see \sectref{rootperf}.
	
	\item Tsova-Tush has borrowed the suffix \textit{-t} to convey plurality in imperatives and in forms involving a 1st person plural inclusive pronoun. This represents a rare case of a bound morpheme being borrowed, as opposed to the numerous accounts of pattern borrowing found in the rest of this work, see \sectref{suffixpl}.
\end{enumerate}
