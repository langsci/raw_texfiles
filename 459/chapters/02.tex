\chapter{Phonetics and phonology} \label{phon}

\section{Introduction}

In this chapter, the phoneme system of Tsova-Tush will be presented (\sectref{phonsystem}), including a novel description of the consonant \textit{w}. In Section \ref{processes}, the various phonological processes will be discussed. Georgian influence on Tsova-Tush phonology will be treated in Section \ref{georgianphon}, where it will be shown that (1) Tsova-Tush adopted the Georgian R-Dissimilation rule (which states that an /r/ in some suffixes changes to \textit{l} if the morphological base contains another /r/), and that (2) Tsova-Tush increased its number of possible consonant clusters due to the large amount of Georgian loanwords. Lastly, in Section \ref{loanwordsphon}, a brief overview of the phonological adaptation of Georgian loanwords into Tsova-Tush is given, which includes some remarks on historical aspects and different layers of loanwords. Note that, apart from \textcite{haukhakim}, no acoustic or other quantitative phonetic studies have been conducted for Tsova-Tush, and no audio files of laboratory recordings are available. Therefore, all descriptions in this section are based on observations by ear (from field recordings and elicitations) and on previous qualitative descriptions and analyses. 

\section{Phoneme system} \label{phonsystem}


\subsection{Consonants}
Table \ref{phon-table1} gives the consonant system of Tsova-Tush in а practical transcription system, common to Caucasian linguistics. Where the symbols do not coincide with IPA, the corresponding IPA symbols have been added in square brackets. For the voiced epiglottal fricative, shown in brackets in this table, see the discussion below in this section.



\begin{sidewaystable}
	\begin{tabular}{lllllllllll}
    \lsptoprule
		&  {Lab.}  & {Alv.}& & & {Post.}& {Vel.}& {Uv.} & & {Epi.}& {Glot.} \\
		\midrule
		Plosive & \textit{p} [pʰ] & \textit{t} [tʰ]&  \textit{tt} [tʰː] & \textit{c} [ʦ] & \textit{č} [ʧ] & \textit{k} [kʰ]& \textit{q} [qʰ]& \textit{qq} [qʰː] & & \textit{Ɂ} \\
		
		
		&   \textit{p'} & \textit{t'} & \textit{t't'} [t'ː]& \textit{c'} [ʦ'] & \textit{č'} [ʧ'] & \textit{k'} & \textit{q'} & \textit{q'q'} [q'ː] & &  \\
		
		
		&   \textit{b}  & \textit{d} & 
		& \textit{ʒ} [ʣ] & \textit{ǯ} [ʤ] & \textit{g} & &  & & \\
		
		
		Fricative    & & \textit{s} & \textit{ss} [sː] & & \textit{š} [ʃ]&  & \textit{x} [χ]& \textit{xx} [χː] & \textit{ħ} [ʜ] & \textit{h} \\
		
		&   & \textit{z} & & & \textit{ž} [ʒ]&  & \textit{\u{g}} [ʁ]&  & (\textit{ʕ} [ʢ])& \textit{w} [ɦʷ] \\
		
		Nasal  & \textit{m} & \textit{n} &  & &  &  &  & & & \\
		
		Lateral  &  & \textit{l} & \textit{ll} [ː]  & &  &  &  & & & \\
		&    & \textit{ɬ} [l̥]& & & & &  &  & &  \\
		
		Approx.   & \textit{v} [ʋ]& \textit{r} &  & &  \textit{j}  &  &  &  & \\
		\lspbottomrule
	\end{tabular}
	\caption{Tsova-Tush consonant system}
	\label{phon-table1}
\end{sidewaystable}


Tsova-Tush ejective stops\is{Ejective} are characterised by a shorter duration and creakier voice\is{Creaky voice} quality in the following
vowel, but not by closure duration\is{Closure duration} or the duration of
a preceding vowel. Long consonants\is{Long consonants}\is{Geminate consonants}\is{Fortis consonants} (\textit{tt, t't', qq, q'q', ss, xx, ll})  differ from their short counterparts only in total duration and closure duration (\cite{haukhakim}), and not by any other phonetic cue (such as a difference in Voice Onset Time or burst intensity, as in other Caucasian languages with so-called ``geminate\slash fortis\slash intensive'' consonants). They occur only at the alveolar and uvular place of articulation and are found only in non-initial position. They do not occur in mono\hyp morphemic consonant clusters\is{Consonant clusters} but when a cluster arises at a morpheme boundary, long consonants tend not to be shortened.
Some minimal pairs (from \cite{haukharris}) include \textit{qetar} `get up' - \textit{qettar} `know'; \textit{it'} `run (\textsc{npst})' - \textit{it't'} `ten'; \textit{eqar} `these (\textsc{erg})' - \textit{eqqar} `jump'; \textit{d-aq’-d-ar} `dry (v.)’ - \textit{d-aq’q'-d-ar} `examine, check'; \textit{is} `that (\textsc{med})' - \textit{iss} `nine'; \textit{d-axar} `go’ - \textit{d-axxar} `drown, choke'; \textit{qali\textsuperscript{n}} `triplet, threesome' - \textit{qalli\textsuperscript{n} } `ate (\textsc{pfv})'. An eighth long consonant \textit{šš} seems to be found only in one word (\textit{eššin\u{o}} ‘crazy’).




The voiceless lateral \textit{ɬ}\is{Lateral fricative} does not occur in initial position. Minimal pairs (\cite{haukharris}) contrasting \textit{ɬ} and \textit{l} are \textit{meɬa\textsuperscript{n}} `drink (\textsc{inf})' - \textit{mela\textsuperscript{n}} `ink', \textit{d-alar} `die' - \textit{d-aɬar} `go out, appear'. Tsova-Tush features another sound, the lenghtened voiced lateral approximant \textit{ll}. Whether this \textit{ll} and \textit{ɬ} are in free variation, or phonemically distinct, remains to be investigated.



Tsova-Tush \textit{r} is a single tap.\is{Rhotic consonant} It becomes voiceless when followed by the voiceless lateral approximant \textit{ɬ} (\textit{vorɬ} [vor̥l̥] `seven', \textit{marɬ\u{o}} [mar̥l̥w̥] ‘nose’).  Word-final \textit{v} is realised as a glide [w]. The sound \textit{h}, only found word-initially followed by a vowel, is rare. The sound \textit{w} (which can be analysed as [ɦʷ] or [w̤]) is found only in the words \textit{wum} `something', \textit{wun} `why', \textit{wux} `what', \textit{wumaɁ} `all' and is therefore lexically rare, but well attested in terms of corpus frequency (as these words are used often). It is distinct from \textit{v} (see the minimal pair \textit{wux} [ɦʷuχ] `what', \textit{vux} [ʋuχ] `backwards (\textsc{m.sg})'), and from \textit{h} (see the near-minimal pair \textit{wun} [ɦʷũ] `why' versus \textit{huv} [huw] `seed'). Its cognates in Chechen and Ingush are \textit{h} and \textit{f}, respectively (although these phonemes have a wider distribution in these languages).\is{Glottal fricatives}


Glottal \textit{Ɂ}\is{Glottal stop} occurs in all positions, but by tradition is not written word-initially in this work. Examples include \textit{e\textsuperscript{n}} [Ɂẽ] `shadow', \textit{d-aɁar} [daɁar] `come', \textit{q'oɁ} [q'oɁ], \textit{oɁ} [ɁoɁ] `grain, seed'. All words that seem to start in vowel actually start in a glottal stop (\cites[]{kojima11pharyng}). The sounds \textit{ħ} and \textit{ʕ} are pharyngeal or epiglottal (cognate sounds are generally described as epiglottal in closely related languages (\cite{nichols00pharyng}), although no direct acoustic evidence is available for Tsova-Tush).
Three pharyngeal/epiglottal sounds can be identified.\is{Pharyngeal fricatives}\is{Epiglottal fricatives} \textit{ħ} [ʜ] occurs word-initially (\textit{ħal\u{o}} `up', \textit{ħedar} `be late', \textit{ħič'\u{u}} `look (\textsc{npst})', \textit{ħo} `you (\textsc{nom.sg})', \textit{ħu\textsuperscript{n}} `forest'), in second position after voiceless non-ejective plosives (including affricates, excluding uvulars) (\textit{pħe} `village', \textit{tħak'} `footprint', \textit{cħa} `one', \textit{čħog} `type of fresh cheese', \textit{kħeki\textsuperscript{n}} `ready'), in intervocalic position (\textit{aħ\u{o}} `down', \textit{d-eħar} `steal'), and word-finally (\textit{duħ} `blade, tip', \textit{laħ} `snake'). The sound \textit{ʕ} [ʢ] exclusively occurs in the second position of a word-initial consonant cluster, containing a voiced or ejective (non-uvular) first segment (\textit{bʕark'} `eye', \textit{dʕeɁ} `liver', \textit{nʕan} `worm', \textit{vʕalaɁ} `(not) at all', \textit{jʕeɁ} `spindle', \textit{p'ʕa\textsuperscript{n}} `wing', \textit{t'ʕak'} `filth', \textit{c'ʕerko\textsuperscript{n}} `suddenly', \textit{č'ʕa\u{g}o\textsuperscript{n}} `strong', \textit{k'ʕavar} `lame sheep'). A third sound, sometimes described as an epiglottal stop, is found exclusively word-initially (\textit{ʕa} `winter', \textit{ʕep} `shame', \textit{ʕiri\textsuperscript{n}} `sharp', \textit{ʕop-d-ar} `cover (v.)', \textit{ʕurde\textsuperscript{n}} `in the morning'). However, this sound can be analysed (probably phonetically, but definitely phonologically) as a combination of \textit{Ɂ} + \textit{ʕ}. This gives us the following distribution: \textit{ʕ} as second consonant of a word, after voiced or ejective obstruent, or after a glottal stop; \textit{ħ} elsewhere. The two sounds are thus in complementary distribution and \textit{ħ} becomes voiced when the articulation of the preceding consonant involves ``movement of the glottis'' (\cite{kojima11pharyng}).\footnote{A feature [−idle glottis] is proposed by \textcite{faustetal2025ttidleglottis}.} In this work, [\textsc{h}] is written \textit{ħ}, \#C[ʢ]V is written \textit{\#CʕV}, \#[Ɂʢ]V is written \textit{\#ʕV}, and \#[Ɂ]V is written \textit{\#V}.


\subsection{Vowels}
Tsova-Tush features five vowels \textit{a, e, i, o, u}, and what seems like a marginal phoneme \textit{\={a}} [aː]. \textcite{kadkad84} contains six words (all verbs) with \textit{\={a}} (none with other long vowels) that cannot be explained by synchronic phonological processes. They are \textit{d-āɬar} `(intransitive verb used in multi-word expressions)' (cf. \textit{d-aɬar} `give'), \textit{d-āttar} `be poured' (cf. \textit{d-attar} `roast, bake'), \textit{d-āšar} `shave' (cf. \textit{d-ašar} `melt, thaw'), \textit{d-āxar} `live' (cf. \textit{d-axar} `go'), \textit{d-āxar} `extract' (cf. \textit{d-axar} `get drunk'), \textit{d-āxk'-d-ar} `bring along, drive' (cf. \textit{d-axk'-d-ar} `let rot'). Various processes create long vowels (due to Vowel Mergers and compensatory lengthening), diphthongs (due to Umlaut), and nasalised vowels (due to Nasalisation). Whatever their origin, long vowels are never written as such in practical transcription, but where necessary, phonetically long vowels are indicated by accompanying IPA transcription.\is{Long vowels}



\section{Phonological processes} \label{processes}
Tsova-Tush exhibits a relatively large number of phonological rules that obscure the relationship between phonetic surface form and underlying morphemes. This is contrasted with the situation in Georgian, where relatively little allomorphy or allophony is found. It is also contrasted with Chechen and Ingush, where several processes have taken place that are similar to those in Tsova-Tush, but the results of which have largely been phonemicised. For example, all Vainakh final unaccented vowels (only some of which triggered umlaut on the preceding vowel) have been reduced to \textit{a} [ɐ]. In words where the original (pre-umlaut) vowel never surfaces, the umlauted vowel is considered a phoneme. See e.g. Chechen, Ingush \textit{c'ena} [c'en(ɐ)] `clean' < Proto-Nakh \textit{*c'ani(n)}, as in Tsova-Tush \textit{c'\={e}n}, \textit{c'ejn\u{\i}} [c'eːn], underlyingly \textit{c'ani} `clean', still seen in the adverb \textit{c'ani-š} `cleanly'.


In tables in this work, morphemes will be cited with no phonological processes applied unless specified, while glossed examples and word forms will be provided in their surface form. 

\subsection{Nasalisation}\is{Nasalisation}

Word-final \textit{n} is deleted, nasalising the preceding vowel. See Table \ref{table-nasal} for examples. Remember that in this work, nasal vowels are written \textit{V\textsuperscript{n}}.


\begin{table}
	\begin{tabular}{lll}
		\lsptoprule
		\textit{gagan} & \textit{gaga\textsuperscript{n}} & `egg' \\
		\textit{bader-e-n} & \textit{badre\textsuperscript{n}} & `child (\textsc{gen})' \\
		\textit{aɬ-in} & \textit{aɬi\textsuperscript{n}} & `s/he said' \\
		\textit{d-aqqon} & \textit{daqqo\textsuperscript{n}} & `big' \\
		\textit{ʕa-lun} & \textit{ʕalu\textsuperscript{n}} & `wintery' \\
		\lspbottomrule
	\end{tabular}
	\caption{Tsova-Tush Nasalisation}
	\label{table-nasal}
\end{table}

Word-internal (but still syllable-final) Nasalisation is also observed, although its distribution is less transparent. Table \ref{table-nasal2} presents the most common sequences of Nasalisation on morpheme boundaries, as well as morpheme-internally (forms in brackets represent less frequent alternatives). 
As seen in the table, the sequence \textit{-i-en=so} (\textsc{-tr-aor=1sg.nom}) can be found as \textit{iens\u{o}} (in which Nasalisation has not applied), as \textit{ie\textsuperscript{n}s\u{o}} (with Nasalisation) and as \textit{ies\u{o}} (where the \textit{n} is lost without a trace). 
It is still unclear whether this instance of word-internal nasalisation has occurred because the Nasalisation rule was active before the grammaticalisation of the 1st person pronoun, or because Nasalisation occurs more readily before \textit{s} (no other instances of \textit{Vns} are attested). In subcorpus MM, the authors write any syllable-final nasal stop as nasalisation instead. It is unclear whether this represents actual phonetic observations or a spelling convention. Subcorpus E (consisting of audio recordings only) represents the most contemporary and least edited language and also features the largest amount of variation, which points to a system in the midst of change. The exact distribution of word-internal Nasalisation remains to be investigated.

\begin{table}
	\begin{tabular}{lllll}
    \lsptoprule
		& {KK} & {EK} & {E} & {MM} \\
        \midrule
		(1) & \textit{-ies\u{o}} & \textit{-ies\u{o}} (\textit{-ie\textsuperscript{n}s\u{o}, -iens\u{o}}) & \textit{-ies\u{o}} (\textit{-ie\textsuperscript{n}s\u{o}, iens\u{o}}) & \textit{-ie\textsuperscript{n}s\u{o}} \\
		
		(2) & \textit{-ali\textsuperscript{n}s\u{o}} (\textit{-alis\u{o}}) & \textit{-alis\u{o}} & \textit{-ali\textsuperscript{n}s\u{o}} & \textit{-ali\textsuperscript{n}s\u{o}} \\
		
		(3) & \textit{-inč\u{o}} & \textit{-inč\u{o}} & \textit{-inč\u{o}} (\textit{-i\textsuperscript{n}č\u{o}}) & \textit{-i\textsuperscript{n}č\u{o}} \\
		
		(4) & \textit{inc} & \textit{inc} & \textit{inc} & \textit{i\textsuperscript{n}c} \\
		\lspbottomrule
		
		
	\end{tabular}
	\caption{Tsova-Tush medial nasalisation. (1) \textit{-i-en=so} `\textsc{tr-aor=1sg.nom}'; (2) \textit{-al-in-so} `\textsc{intr-aor-1sg.nom}'; (3) \textit{-ino-čo}  `\textsc{ptcp.pst-obl}'; (4) any morpheme-internal nasal, like \textit{inc} `now'}
	\label{table-nasal2}
\end{table}



\subsection{Vowel mergers}\is{Vowel hiatus}

A vowel hiatus arising on a morpheme boundary is resolved by merging the two vowels. Since most stems end in a consonant and most suffixes start in one, few instances of mergers have been found, and more research into their precise distribution is necessary. Table \ref{phon-table2} shows the most common and most consistent vowel combinations and an illustrative example for each. \textit{ui} represents a monosyllabic diphthong, see \sectref{subsec:umlaut}.

\begin{table}
	\begin{tabular}{lll}
		\lsptoprule
		o + a & > oː & \textit{v-a\u{g}-o-as} > \textit{va\u{g}os} [ʋaʁoːs] `I will come' \\
		a + a & > aː & \textit{v-a\u{g}-ora-as} > \textit{va\u{g}ras} [ʋaʁraːs] `I was coming' \\
		
		o + i & > ui & \textit{do-i} > \textit{dui} `horses' \\
		u + i & > ui & \textit{ʕu-i} > \textit{ʕui} `shepherds' \\
		\lspbottomrule
	\end{tabular}
	\caption{Tsova-Tush vowel mergers}
	\label{phon-table2}
\end{table}


\subsection{Apocope}\is{Apocope}\is{Vowel deletion}\is{Compensatory lengthening}

In Tsova-Tush multi-syllabic words, word-final non-nasal vowels are, with very few exceptions\footnote{The plural marker \textit{-i} is never apocopated, nor is the nominalising suffix \textit{-a} or the relativising suffix \textit{-e}. This can be treated as an exception to a phonological rule, or (equally exceptionally) all three suffixes can be analysed as being underlyingly long (as is the case for Chechen and Ingush plural \textit{\={\i}}). The apocope rule is then extended to shorten final long vowels (of which the above-mentioned suffixes would be the only instances). Enclitic particles are also not reduced. A fuller investigation of phonological processes on the boundary of clitics and hosts in Tsova-Tush is needed.}, regularly deleted. Final \textit{a, e, i} are deleted fully, while final \textit{o} and \textit{u} are reduced to \textit{\u{o}, \u{u}} (both [w]). Table \ref{phon-table3} shows an example for each vowel. As seen from the table, Apocope causes compensatory lengthening on preceding vowels if these vowels end up in a monosyllable after apocope. This causes phonological oppositions like \textit{mott} [motʰː] `place, bed' and \textit{mott} [moːtʰː] `it seems'. It is yet unknown whether Apocope and vowel deletion in general triggers compensatory lengthening of vowels in multi\hyp syllabic Tsova-Tush words.

\begin{table}
	\begin{tabular}{lll}
		\lsptoprule
		\textit{nana} & \textit{nan} [naːn]& `mother (\textsc{nom})' \\
		\textit{mott-e} & \textit{mott} [moːtʰː] & `it seems' \\
		\textit{ditxi} & \textit{ditx} [diːtχ] & `meat (\textsc{nom})' \\
		\textit{nan-e-go} & \textit{naneg\u{o}} [nanegw] & `mother (\textsc{all})' \\
		\textit{xiɬ-u} & \textit{xiɬ\u{u}} [χiːl̥w̥] & `s/he/they will be' \\
		\lspbottomrule
	\end{tabular}
	\caption{Tsova-Tush Apocope}
	\label{phon-table3}
\end{table}

Compensatory lengthening is the underlying cause for all nouns and adjectives in \textcite{kadkad84} containing a long \textit{\={a}} vowel, see Table \ref{phon-table4}.

\begin{table}
	\begin{tabular}{lllll}
    \lsptoprule
		In Dict. & {Underlying} & {Phonetic} & \\
		\midrule
		\textit{\={a}g} & \textit{aga} & [aːg] & `grandmother' \\
		\textit{\={a}l\u{e}} & \textit{ale} & [aːl] & `lord' \\
		\textit{\={a}s\u{e}} & \textit{ase} & [aːs] & `calf' \\
		\textit{b\={a}l} & \textit{bala} & [baːl] & `misfortune' \\
		\textit{b\={a}x} & \textit{baxa} & [baːχ] & `rich' \\
		\textit{d\={a}q'} & \textit{daq'a} & [daːq'] & `share, part' \\
		\textit{m\={a}x} & \textit{maxa} & [maːχ] & `needle' \\
		\lspbottomrule
		
	\end{tabular}
	\caption{Tsova-Tush nouns with supposed long \textit{a}}
	\label{phon-table4}
\end{table}

In older works, now apocopated vowels \textit{e, i, o, u} were still written as reduced vowels $\langle$\u{e}, \u{\i}, \u{o}, \u{u}$\rangle$. These reduced vowels were pronounced as very short vowels or their non-syllabic equivalents (\cite{gagua56vowel}). In contemporary Tsova-Tush, this is only true for \textit{\u{o}} and \textit{\u{u}}, which are now pronounced [w]. However, in allegro speech, these two vowels too are deleted completely. The apocopated/reduced vowel \textit{a} was never written, and therefore presumably not pronounced since the middle of the 19th century.
Vowel apocope is also found in Tush Georgian (\cite{kartulidialekt}) and Chechen and Ingush (\cite{nichols11}). Apocopated \textit{i} and \textit{u} trigger Umlaut on the preceding vowel, see below.

The Apocope rule takes the output of the Nasalisation rule as input. That is, \textit{n} before word-final vowels is not deleted (e.g.  \textit{nan} (not \textit{*na\textsuperscript{n}}) `mother' from \textit{nana}, see \tabref{table-nasal3}a). Word-final nasal vowels are not deleted (see \tabref{table-nasal3}b).

\begin{table}
	\begin{subtable}[t]{.5\textwidth}
	\centering
	\caption{}
	\begin{tabular}{lll}
		\lsptoprule
		\textit{aɬ-ine} & \textit{aɬin} & `s/he said (\textsc{seq})' \\
		\textit{alon-i} & \textit{aluin} & `into Alvani' \\
		\textit{nana} & \textit{nan} & `mother' \\
		\lspbottomrule
	\end{tabular}
	\end{subtable}%
	\begin{subtable}[t]{.5\textwidth}
	\centering
	\caption{}
	\begin{tabular}{lll}
		\lsptoprule
		\textit{gagan} & \textit{gaga\textsuperscript{n}} & `egg' \\
		\textit{bader-e-n} & \textit{badre\textsuperscript{n}} & `child (\textsc{gen})' \\
		\textit{aɬ-in} & \textit{aɬi\textsuperscript{n}} & `s/he said' \\
		\textit{d-aqqon} & \textit{daqqo\textsuperscript{n}} & `big' \\
		\textit{ʕa-lun} & \textit{ʕalu\textsuperscript{n}} & `wintery' \\
		\lspbottomrule
	\end{tabular}
	\end{subtable}
	\caption{Tsova-Tush Nasalisation}
	\label{table-nasal3}
\end{table}


\subsection{Syncope}\is{Syncope}\is{Vowel deletion}

In words with more than two syllables, vowels in penultimate syllables are deleted, for example \textit{daq're\u{g}} `food (\textsc{trans})' from \textit{d-aq'-ar-e-\u{g}}; \textit{badrex} `child (\textsc{cont})' from \textit{bader-e-x}.
In words where both Syncope and Apocope can apply, Apocope is applied first: \textit{ak'id\u{o}} [ak'idw] `string of fruit' < Georgian \textit{ak'ido} (as opposed to *\textit{ejk'd\u{o}}, the expected form if Syncope would have operated first). This is especially clear in four-syllable words, such as \textit{tet'-ora-lo} `would cut'. First, Apocope applies, giving \textit{tet'oral\u{o}} [tet'oralw], after which the penultimate syllable is \textit{t'o} (instead of \textit{ra}) on which Syncope operates, yielding the surface form \textit{tet'ral\u{o}}.
Whether this type of vowel loss triggers compensatory lengthening is unknown at this point.


As already mentioned by \textcites[156]{holiskygagua}, Syncope does not occur if the result would be an ``impermissible consonant cluster''. Then, as now, it remains unclear what constitutes an impermissible cluster.\is{Consonant clusters}

\subsection{Umlaut}\label{subsec:umlaut}\is{Umlaut}\is{Vowel assimilation}

The vowels \textit{i} and \textit{u}, if (and only if) deleted by Apocope or Syncope, trigger Umlaut on the vowel in the preceding syllable. See Table \ref{table-umlaut2} for the umlauted version of each underlying vowel. Deleted \textit{i} triggers i-Umlaut, meaning the affected vowel becomes more like \textit{i} (i.e. more front and close), see Table \ref{table-umlaut1}a. Deleted \textit{u} also triggers i-Umlaut when the consonant(s) between the deleted \textit{u} and the vowel in the preceding syllable is coronal (i.e. (post)alveolar), see Table \ref{table-umlaut1}b. If the intervening consonant is not coronal (i.e. labial, velar, uvular or (epi)glottal), a deleted \textit{u} triggers u-Umlaut, meaning the affected vowel becomes more like \textit{u} (i.e. more close and rounded), see Table \ref{table-umlaut1}c. When a deleted \textit{u} is followed by a coronal consonant, it triggers i-Umlaut, regardless of the intervening consonant, see Table \ref{table-umlaut1}d.\footnote{Although exceptions can be found: \textit{agur-i} > \textit{ougri} `bricks'.}

\begin{table}
	\small
	\tabcolsep=.66\tabcolsep
	\begin{subtable}[t]{.45\textwidth}
	\centering
	\caption{}
	\begin{tabular}{lll}
		\lsptoprule
		\textit{ʕabik'-\={\i}} & \textit{ʕebk'i} & `spoons' \\
		
		\textit{lev-ira-as} & \textit{livras} & `I was saying' \\
		
		\textit{j-ot'-in-as} & \textit{juit'nas} & `I went' \\
		
		\textit{tuxi} & \textit{tuix} & `salt' \\
		\lspbottomrule
	\end{tabular}
	\end{subtable}%
	\begin{subtable}[t]{.55\textwidth}
	\centering
	\caption{}
	\begin{tabular}{lll}
		\lsptoprule
		\textit{ħac'uk'-\={\i}} & \textit{ħec'k'i} & `birds' \\
		
		\textit{qett-u} & \textit{qitt\u{u}} [qiːtːw] & `s/he stands up' \\
		
		\textit{j-ot'-u} & \textit{juit'\u{u}} & `she goes' \\
		
		\textit{busu} & \textit{buis\u{u}} & `at night' \\
		\lspbottomrule
	\end{tabular}
	\end{subtable}\medskip\\
	\begin{subtable}[t]{.55\textwidth}
	\centering
	\caption{}
	\begin{tabular}{lll}
		\lsptoprule
		\textit{gagu-a-x} & \textit{gougax} & `knees (\textsc{cont})' \\
		\textit{tabu} & \textit{toub\u{u}} & `gelded ram' \\
		\textit{dopxu} & \textit{dupx\u{u}} [duːpxw]& `clothes' \\
		\lspbottomrule
	\end{tabular}
	\end{subtable}%
	\begin{subtable}[t]{.45\textwidth}
	\centering
	\caption{}
	\begin{tabular}{lll}
		\lsptoprule
		\textit{k'ox-urat} & \textit{k'uixrat} & `in Georgian' \\
		\textit{gomur-\={\i}} & \textit{guimri} & `stables' \\
		\textit{apuš-\={\i}} & \textit{epši} & `lies' \\
		\lspbottomrule
	\end{tabular}
	\end{subtable}
	\caption{Tsova-Tush Umlaut}
	\label{table-umlaut1}
\end{table}

Similarly to deleted \textit{a, e, o}, deleted \textit{u} and \textit{i} trigger compensatory lengthening\is{Compensatory lengthening} if the outcome after deletion is a monosyllable. This results in some minimal pairs, like \textit{meq} `couch', \textit{meq} ([meːq] < \textit{maqi}) `bread'. This lenghtening is not reflected in the practical transcription.
The exact effect of Umlaut on vowel length in polysyllables is hitherto unknown: no phonetic research has been conducted, and existing sources (E, MM) mark vowel length erratically. Regardless of vowel length, Umlaut only occurs when the trigger is deleted, hence it can be called Compensatory Umlaut.

Table \ref{table-umlaut2} shows the outcome of the Umlaut process in attestations throughout the past two centuries. Stage 1 represents texts from the middle of the 19th century (IT, AS), stage 2 is around the middle of the 20th century, represented by subcorpora YD and KK, while stage 3 represents contemporary Tsova-Tush. Stage 4 represents contemporary Tsova-Tush in contexts where phonetic shortening applies, which seems to be in unstressed syllables, although a phonetic investigation is lacking here.

\begin{table}
	\begin{tabular}{llllll}
    \lsptoprule
		Underlying & \multicolumn{4}{c}{{Historical stages}} \\
		{vowel} & {1} & \textbf{2} & {3} & {4} \\
		\midrule
		{i-umlaut} & & \\
		\textit{a} & \textit{ai} & \textit{ai} & \textit{ej, e} [ej, eː] & (\textit{e}) \\
		\textit{e} & \textit{ei} & \textit{ei} & \textit{i} [iː] & (\textit{i}) \\
		\textit{o} & \textit{oi} & \textit{ui} & \textit{ui} & (\textit{y})\\
		\textit{u} & \textit{ui} & \textit{ui} & \textit{ui} & (\textit{y}) \\
		
		\midrule
		{u-umlaut} & & \\
		\textit{a} & \textit{au} & \textit{au} & \textit{ou} & (\textit{o})\\
		\textit{o} & \textit{ou} & \textit{ou} & \textit{u} [uː] & (\textit{u})\\
		
		\lspbottomrule
	\end{tabular}
	\caption{Tsova-Tush Umlaut historically}
	\label{table-umlaut2}
\end{table}

As is clear from Table \ref{table-umlaut2}, Umlaut started out as pure diphthongisation, and up until around the 1980s, this remained to be the case (although some diphthongs changed quality). Subcorpus KK (the dictionary by \cites[]{kadkad84}) mainly gives forms with diphthongs (stage 2), but gives some monophthongs (stage 3) as alternatives. Umlauted vowels in contemporary Tsova-Tush (stage 3) are a mix of diphthongs (\textit{ui, ou}) and monophthongs (\textit{e, i, u}). 


A small number of lexical items contain diphthongs or long vowels that, on a synchronic level, cannot be explained as the result of an Umlaut process, such as \textit{pst'uin\u{o}} `woman' and  \textit{ʕ\={e}rtva\textsuperscript{n}} (KK \textit{ʕairtva\textsuperscript{n}}) `plenty'.

The following overview presents the subsequent phonological processes for each deleted vowel.

\begin{itemize}
	\item \textit{a}: Compensatory Lengthening of vowel in preceding monosyllable; complete deletion
	\item \textit{e}: Compensatory Lengthening of vowel in preceding monosyllable; complete deletion
	\item \textit{i}: Compensatory Lengthening of vowel in preceding monosyllable; i-Umlaut of vowel in preceding syllable; complete deletion
	\item \textit{o}: Compensatory Lengthening of vowel in preceding monosyllable; if word-final, reduction to \textit{\u{o}} [w]; if word-internal, complete deletion.
	\item \textit{u}: Compensatory Lengthening of vowel in preceding monosyllable; i-Umlaut or u-Umlaut of vowel in preceding syllable. If word-final, reduction to \textit{\u{u}} [w]; if word-internal, complete deletion.
\end{itemize}



\subsection{Pharyngeal deletion}\is{Pharyngeal deletion}\is{Pharyngeal fricatives}\is{Epiglottal fricatives}

The pharyngeal/epiglottal fricative \textit{ħ} is deleted in polysyllabic words in word-final position. This rule is applied simultaneously with the Apocope rule: vowels before word-final \textit{ħ} are not apocopated (e.g.  \textit{xilo} (not \textit{*xil\u{o}}) `in the water' from \textit{xi-loħ}). That is, the Apocope rule is not applied to the output of Laryngeal Deletion. On the other hand, if \textit{ħ} becomes word-final after vowel deletions, it is not deleted (e.g.  \textit{vaħ} (not \textit{*va}) `if he is' from \textit{v-a-ħe}). That is, Laryngeal Deletion is not applied to the outcome of Apocope.

Due to this process, the Essive case marker \textit{-ħ} on nouns and the Ergative 2nd person singular cross-reference marker \textit{-aħ} often do not show up in surface forms, see Table \ref{table-laryng}. 


\begin{table}
	\begin{tabular}{llll}
		\lsptoprule
		\multicolumn{4}{l}{Polysyllabic:}\\
		\textit{gomur-e-ħ} & stable-\textsc{obl-ess} & \textit{guimre} & `in the stable' \\
		\textit{xi-loħ} & water-\textsc{interess} & \textit{xilo} & `in the water' \\
		\textit{kalak-iħ} & city-\textsc{iness} & \textit{kalki} & `in the city' \\
		
		\textit{tit'-en-aħ} & cut.\textsc{pfv-aor-2sg.erg} & \textit{tit'na} & `you have cut it' \\
		\textit{lev-i-aħ} & say.\textsc{ipfv-npst-2sg.erg} & \textit{liva} & `you say' \\
		\textit{v-a\u{g}-ora-aħ} & \textsc{m.sg}-come.\textsc{ipfv-imprf-2sg.erg} & \textit{va\u{g}ra} & `you were coming' \\\midrule
		
		\multicolumn{4}{l}{Monosyllabic:}\\
		\textit{pħe-ħ} & village-\textsc{ess} & \textit{pħeħ} & `in the village' \\
		\textit{laħ} & snake & \textit{laħ} & `snake' \\\midrule
		\multicolumn{4}{l}{Final vowel:}\\
		\textit{v-a-ħe} & {\M}.{Sg}-be-{\Cond} & \textit{vaħ} & `if he is' \\
		\lspbottomrule
	\end{tabular}
	\caption{Tsova-Tush Laryngeal Deletion}
	\label{table-laryng}
\end{table}

Final laryngeals are not deleted in the oldest subcorpora (IT, AS, YD). KK (\cite{kadkad84}) marks the Deletion as optional.




\section{Georgian influence} \label{georgianphon}\is{Georgian influence!Phonological}

\subsection{R-Dissimilation} \label{rdissim}\is{Rhotic consonant}\is{Ablative case}\is{Plural formation}
\largerpage

Another (morpho-)phonological process observed in Tsova-Tush has a clear parallel in Georgian. 
Two suffixes containing \textit{r}, the Ablative case suffix \textit{-ren} and the archaic plural suffix \textit{-erč} change their \textit{r} to \textit{l} when the root it attaches to contains an \textit{r}. See Table \ref{table-dissim} for examples.

\begin{table}
	\begin{tabular}{lllll}
    \lsptoprule
		\textit{-erč} & {\textsc{pl}} & {\textit{-ren}} & {\textsc{abl}} \\
		\midrule
		\textit{buin-erč} & `fists' & \textit{z\u{g}ven-e-re\textsuperscript{n}} & `from the attic' \\
		\textit{tab-erč} & `gelded rams' & \textit{sk'ol-re\textsuperscript{n}} & `from the school' \\
		\textit{maq-erč} & `songs' & \textit{naq'-re\textsuperscript{n}} & `from the road' \\
		\midrule
		
		\textit{t'ʕer-elč} & `stars' & \textit{kor-le\textsuperscript{n}} & `from the hand' \\
		\textit{herc'-elč} & `pots' & \textit{kuirt-le\textsuperscript{n}} & `from the head' \\
		\textit{mʕar-elč} & `nails' & \textit{nažt'r-e-le\textsuperscript{n}} & `from the stable' \\
		
		\lspbottomrule
	\end{tabular}
	\caption{Tsova-Tush R-Dissimilation}
	\label{table-dissim}
\end{table}

R-Dissimilation does not apply in compounds when only the first element in the compound contains an \textit{r} (e.g. \textit{kort-st'ejk'-re\textsuperscript{n}} `head-man-\textsc{abl}, `from the leader'), but it does when the \textit{r} is in the second element, e.g. \textit{pxak'al-q'ur-leⁿ} `rabbit-place-\textsc{abl}, from Pkhakalkure'. It also does not apply when there is an \textit{l} between the \textit{r} of the suffix and the \textit{r} of the root (\textit{jerusalim-re\textsuperscript{n}} `from Jerusalem'). The rule does apply regardless of intervening syllables (\textit{qer-ba-leⁿ} `from the stones', \textit{ru-e-leⁿ} `from the stream'), only intervening lexemes cause the non-application. The R-Dissimilation rule also does not apply to other affixes, such as the Verbal Noun suffix \textit{-ar} or the various verbal suffixes containing the morph \textit{-ra}, see \tabref{table-dissim2} (remember that word-final \textit{-ra} is reduced to \textit{-r}, as seen in the third column of \tabref{table-dissim2}. 

\begin{table}
	\begin{tabular}{lllll}
    \lsptoprule
		\textit{-ar} & {\textsc{vn}} & {\textit{-Vra}} & {\textsc{imprf}} \\
        \midrule
        
		\textit{dat'ar} & `run away' & \textit{dat'er} & `was running away' \\
		\textit{apšar} & `chew' & \textit{apšor} & `was chewing' \\
		\textit{mok'ecadalar} & `curl up' & \textit{mok'ecadalar} & `was curling up' \\
		\midrule
		
		\textit{xarcar} & `change' & \textit{xarcor} & `was changing' \\
		\textit{larɬar} & `count' & \textit{larɬer} & `was counting' \\
		\textit{t'urt'ladar} & `cover in dirt' & \textit{t'urt'lador} & `was covering in dirt' \\
		
		\lspbottomrule
	\end{tabular}
	\caption{Tsova-Tush absence of R-Dissimilation}
	\label{table-dissim2}
\end{table}


This same phenomenon is observed in several suffixes in Georgian, such as the adjectival affixes \textit{-ur, m- -ar, -ier}, see \tabref{table-dissim3} (examples from \cites[24]{shanidze53sapudz}).

\begin{table}
\fittable{%
	\begin{tabular}{lllllll}
    \lsptoprule
        \textit{-ur} & {\textsc{adj}} &
        {\textit{m- -ar}} & {\textsc{adj}} &
        {\textit{-ier}} & {\textsc{adj}} \\
        \midrule
        
		\textit{inglis-ur-i} & `English' & \textit{m-k'vd-ar-i} & `dead' & \textit{nič'-ier-i} & `gifted' \\
		\textit{k'ac-ur-i} & `manly' & \textit{m-dn-ar-i} & `molten' & \textit{gon-ier-i} & `intelligent' \\
		\textit{ʒa\u{g}l-ur-i} & `dog-like' & \textit{m-xm-ar-i} & `wilted' & \textit{sul-ier-i} & `animate' \\
		
		\midrule
		
		\textit{rus-ul-i} & `Russian' & \textit{m-q'r-al-i} & `stinking' & \textit{xorc-iel-i} & `mortal, carnal' \\
		\textit{imer-ul-i} & `Imeretian' & \textit{m-šr-al-i} & `dry' & \textit{carieli} & `empty' \\
		\textit{\u{g}or-ul-i} & `pig-like' & \textit{m-kr-al-i} & `faded' & \textit{gemrieli} & `tasty' \\
		
		\lspbottomrule
	\end{tabular}}
	\caption{Georgian R-Dissimilation}
	\label{table-dissim3}
\end{table}

In Georgian, exactly like in Tsova-Tush, this dissimilation does not occur in compounds (\textit{mk'erd-gan-ieri} `broad-chested' < `chest' + `width' + \textsc{adjz}),
or when an \textit{l} occurs between both \textit{r}'s in the same word (\textit{avst'rali-ur-i} `Australian') (\cites[24]{shanidze53sapudz}). A similar phonological process in Chechen or Ingush is not observed, which leads us to the hypothesis that the Tsova-Tush phonological rule is contact-induced, and has been introduced under influence of Georgian. Of course, dissimilation involving liquids has typological parallels (see \textcite{cohn1992dissimilation} for the same rule in Sundanese), and thus could have happened independently of Georgian. However, the fact that (1) not all suffixes that contain \textit{r} partake in the rule, making it morphophonological rather than purely phonological, and (2) this rule is not found elsewhere in the Caucasus, makes the coincidence too big to not warrant language contact as an explanation.

The Tsova-Tush rule is already observed in 19th century language material, such as in \textit{k'air-le\textsuperscript{n}} `from Cairo' (AS008-3.1), \textit{koirt-le\textsuperscript{n}} `from the head' (AS008-14.2).


\subsection{Phonotactics} \label{phonotact}\is{Consonant clusters}
Due to the introduction of numerous Georgian loanwords into Tsova-Tush, the set of consonant clusters that are allowed has expanded. I will focus on word-initial consonant clusters here. In inherited Nakh words in Tsova-Tush, just like in Chechen and Ingush, words can start with a single consonant, or a consonant plus an approximant or a laryngeal. Additionally, a word can start with a so-called harmonic cluster, i.e. a cluster of two consonants where the second consonant is further back in terms of place of articulation than the first, and both exhibit the same voicing and airstream mechanism (see \cites[86--87]{nichols11} and \cites[223]{aronson91} for harmonic clusters in Ingush and Georgian, respectively). Table \ref{table-nakhcluster} shows all basic types of initial consonant combinations. Not all cells are filled, since (1) uvulars and velars cannot be the first element of a harmonic cluster, (2) uvulars cannot combine with laryngeals, and (3) not every combination is attested, since Tsova-Tush only retains approximately 1200 Nakh roots.

\begin{sidewaystable}
	\begin{tabular}{llllllll}
		\lsptoprule
		\multicolumn{2}{c}{{Single}} & \multicolumn{2}{c}{{Harmonic}} & \multicolumn{2}{c}{{Approximant}} & \multicolumn{2}{c}{{Laryngeal}} \\
		\midrule
		\textit{bader} & `child' & \textit{bža\textsuperscript{n}} & `large cattle' & \textit{blu-} &  `mute' & \textit{bʕorc'} & `wolf' \\
		\textit{pešk'ar} & `kid' & \textit{pxi} & `five' & \textit{plok\u{o}} & `New Year ritual' & \textit{pħu} & `dog' \\
		\textit{p'ant'} & `crab apple' & & & & &  \textit{p'ʕa\textsuperscript{n}} & `wing' \\
		
		\midrule
		
		\textit{doš} & `word' & & & & & \textit{dʕeɁ} & `liver' \\
		\textit{tišar} & `sink' & \textit{txir} & `frost' & & & \textit{tħak'} & `footprint' \\
		\textit{t'iv} & `bridge' & \textit{t'q'a} & `twenty' & & & \textit{t'ʕir} & `star' \\
		
		\midrule
		
		\textit{cac} & `sieve' & & & \textit{cru} & `show-off'   & \textit{cħa} & `one'  \\
		
		\textit{c'a} & `house' & \textit{c'q'e} & `once' & \textit{c'rint'} & `infant'  & \textit{c'ʕop'} & `tip' \\
		
		\midrule
		
		\textit{ǯa\textsuperscript{n}} & `sandal' & & & & & & \\
		
		\textit{ča} & `bear' & \textit{čxot'} & `waterfall' & & &  \textit{čħog} & `fresh cheese'  \\
		
		\textit{č'ek'} & `green bean' & \textit{č'k'a} & `crowd' & & & \textit{č'ʕa\u{g}o\textsuperscript{n}}  & `strong'  \\
		
		\midrule
		
		\textit{gerc'} & `weapon' &  \\
		
		\textit{kotti\textsuperscript{n}} & `narrow' & & & & & \textit{kħeki\textsuperscript{n}} & `ready' \\
		
		\textit{k'uč'} & `hill top' & & & \textit{k'ramp'ul} &  `tusk' & \textit{k'ʕak'} & `heel' \\
		
		\midrule
		
		\textit{qer} & `stone'  \\
		
		\textit{q'ono\textsuperscript{n}} & `young' & & & \textit{q'lort'} & `gulp' &  \\
		
		\textit{(Ɂ)apuš} & `lie' & & & & & \textit{(Ɂ)ʕarč'i\textsuperscript{n}} & `black' \\ 
		\lspbottomrule
		
	\end{tabular}
	\caption{Tsova-Tush inherited word-initial consonants}
	\label{table-nakhcluster}
\end{sidewaystable}

Besides those in Table \ref{table-nakhcluster}, Tsova-Tush can form clusters with \textit{s} + ejective stop: \textit{sk'iv} `spark', \textit{st'ak'} `man' (for the development of this last cluster in the history of Nakh, see \cites[219--220]{nichols03cc}). Also found word-initially is the cluster \textit{xk'} : 
\textit{xk'or} `bladder', \textit{xk'e} `valley'. Only two words with an inherited word-initial three-consonant cluster are found: \textit{pst'u} `wife' (cf. Chechen \textit{stie} `wife', \textit{zuda} `wife'; Ingush \textit{sie-} `wife') and \textit{pst'u} `ox' (cf. Chechen \textit{stu} `ox'; Ingush \textit{ust} `ox').

Georgian loanwords exhibit a wider variety of initial consonant clusters, usually clusters with \textit{r} before a stop, or a harmonic cluster + approximant, e.g. \textit{grdeml} `anvil', \textit{t'q've} `prisoner', \textit{brʒandbad-d-ar} `order', \textit{k'vnet'ad-d-ar} `gnaw', \textit{rbevad-d-ar} `raid', \textit{txlešed-d-ar} `slap', \textit{rk'vevad-d-ar} `winnow', \textit{rc'me\textsuperscript{n}} `conviction'.

Additionally, Georgian loanwords have introduced word-initial \textit{r}, which is absent in inherited Nakh words. Examples include \textit{raxt'} `decorated bridle', \textit{recep't'} `recipe', \textit{riq'} `rocky river bank', \textit{roč'\u{o}} `black grouse', \textit{rus} `Russian'. Hence, since the adoption of these and similar loanwords, Tsova-Tush no longer features a constraint against word-initial \textit{r}, making its phonotactics closer to that of Georgian.\is{Rhotic consonant}



\section{Phonological adaptation of loanwords} \label{loanwordsphon}\is{Loanword adaptation!Phonological}


Tsova-Tush and Georgian have relatively similar phonological systems. More specifically, all Georgian phonemes exist in Tsova-Tush as well. However, we observe several phonological differences when we compare certain Tsova-Tush lexical items with their counterparts in Standard Modern Georgian. Note that word-final Georgian \textit{-i} is a Nominative ending, which is not borrowed into Tsova-Tush (see Section \ref{morphadapt}). 

\subsection{Tsova-Tush \textit{-o-}, Georgian \textit{-va-}}

The Georgian sequence -\textit{va}- corresponds with Tsova-Tush -\textit{o}-, such as \textit{gor} `family name', from Georgian \textit{gvar-i}; \textit{baro(d)} `dig with spade', from Georgian \textit{barva}; \textit{gol\u{o}} `drought', from Georgian \textit{gvalva}. At first glance, this looks like a case of monophthongisation in Tsova-Tush, but further Georgian evidence suggests otherwise. In Georgian, we find several doublets with a \textit{-va-}/\textit{-o-} alternation. Often, both are used in colloquial speech, and no geographical distribution is obvious (yet), such as in \textit{k'vank'ila} / \textit{k'onk'ila} `yoke prop on cart',
\textit{k'vac'axuri} / \textit{k'oc'axuri} `berberis',
\textit{\u{g}valo} / \textit{\u{g}olo} `wild sorrel',
\textit{xvadabuni} / \textit{xodabuni} `large clearing, arable tract',
\textit{dagvalvili} / \textit{dagoluli} `parched, withered',
\textit{lak'vara} / \textit{lak'ora} `groove in upper millstone',
\textit{mc'vadi} / \textit{mc'odi} `barbecued meat'.
Other word pairs do have a known dialectal distribution. Two clear patterns can be discerned: 

\begin{enumerate}
\item \textit{-o-} is used in Eastern Georgian (particularly the Northeastern dialects Khevsur, Tush, Pshav, Mokheve, Mtiulian), where standard Georgian has \mbox{\textit{-va-}:}
Standard Georgian \textit{k'vamli}  `smoke; household' ~ Khevsur, Tush, Pshav \textit{k'omli} `id.';
\textit{gvarvala} `vetch' (plant species, vicia)' ~ \textit{gorvela} `id.';
\textit{gvari} `last name' ~ Mtiulian \textit{gori} `id.';
\textit{alvani} `Alvani' ~ Tush \textit{aloni} `id.';
\textit{zvavi} `avalanche' ~ Mokheve, Pshav \textit{zovi} `id.';
\textit{zvarak'i} `sacrificial bullock' ~ Mtiulian \textit{zora} `id.';
\textit{tvali} `eye' ~ Mokheve \textit{toli} `id.';
\textit{savati} `great bustard' ~ Pshav \textit{saoti} `id.';
\textit{kvabi} `pot' ~ Khevsur \textit{kobi} `id.';
\textit{cvari} `drop of wine' ~ Kakhetian \textit{cori} `id.';
\textit{sxva} `other' ~ Mokheve \textit{cxo} `id.';
\textit{cxvari} `sheep' ~ Mokheve \textit{cxori} `id.';
\textit{xvavi} `heap of threshed corn stalks' ~ Ingilo, Kakhetian \textit{xovi} `id.';
\textit{ǯvaroba} `festival at pagan shrine' ~ Pshav \textit{ǯoroba}.\footnote{The only known counterexamples are Khevsur \textit{dvaleba} `lambing time', Standard \textit{doloba} `id.'; Tush \textit{ečva} `adze', Standard \textit{ečo} `id.'.}

\item Standard Georgian has \textit{-o-} where (Western Georgian) Imeretian  has \mbox{\textit{-va-},} such as Imeretian \textit{k'vaxunǯi} `mocassin', standard \textit{k'oxunǯi} `id.';
Lower Imeretian \textit{svali} `wedge', standard \textit{soli} `id.';
Imeretian \textit{xvamli} `pleiades', standard \textit{xomli} `id.'.
\end{enumerate}

Hence, it is clear that Tsova-Tush borrowed these words from a (North)east Georgian dialect (most likely Tush), which already had the \textit{o} before contact. Due to the continuous contact of Tsova-Tush speakers with Standard Georgian (remember that all are bilingual), words can in principle be borrowed again, replacing the first borrowing. Consequently, we find variants of the same item in the corpus, e.g. \textit{memcxor} (87 occurrences in my corpus) and \textit{memcxvar} (4 occurrences) ‘shepherd’ from Georgian \textit{memcxvare}. Knowing this, we can distinguish between different layers of loanwords. Words with \textit{-o-} where standard Georgian has \textit{-va-}, must have been borrowed when the main language of contact was Tush Georgian, whereas words with \textit{-va-} (such as \textit{zvar} `large vinyard' (Georgian \textit{zvari} `id.'), \textit{saʒ\u{g}var} `border' (Georgian \textit{saʒ\u{g}var-i} `id.')) must have been borrowed from Standard Georgian at a later stage.\il{Tush Georgian}

\subsection{Tsova-Tush \textit{-q-}, Georgian \textit{-x-}}

Tsova-Tush possesses some loanwords containing the consonant \textit{q}, which does not occur in standard Georgian. These words too must have been borrowed at the time when Tsova-Tush speakers were more closely in contact with speakers of Tush Georgian, which does have the sound \textit{q} (see Section \ref{previouscontact}). Hence we find Tsova-Tush \textit{qoqob} `pheasant', from Tush Georgian \textit{qoqobi} (cf. Standard Georgian \textit{xoxobi}); \textit{venaq} `vine' from Tush Georgian \textit{venaqi} (standard Georgian \textit{venaxi}); \textit{tiq} `clay' from Tush Georgian \textit{tiqa} (Standard Georgian \textit{tixa}). As with the \textit{\mbox{-va-}/\mbox{-o-}} alternation described above, these words can be borrowed several times, and more recently from Standard Georgian, which has \textit{-x-} in these instances. Thus, we find older \textit{qširoⁿ} `frequent' from Tush Georgian \textit{qširi}, and more recent \textit{xširoⁿ} from Standard Georgian \textit{xširi}.\il{Tush Georgian}

\subsection{Tsova-Tush cluster reduction}

Georgian \textit{m-} as the first segment of a word-initial consonant cluster is dropped: \textit{c'od} from Georgian \textit{mc'vad-i}; \textit{zitev} `dowry', from Georgian \textit{mzitev-i}; \textit{k'alav} `tinsmith' from Georgian \textit{mk'alav-i}. Other consonant clusters are usually retained, see Section \ref{phonotact}.\is{Consonant clusters}


\subsection{Other remarks}

Russian loans are relatively frequent and are presumed to have been borrowed with Georgian as an intermediate language (\cites[]{WS}). Thus, Russian voiceless consonants are adopted as ejectives in Georgian (and remain ejectives in Tsova-Tush), and all consonants lose any palatalisation, e.g. \textit{st'ak'a\textsuperscript{n}} `glass (cup)’ from Russian \textit{stakan}; \textit{ap'elsi\textsuperscript{n}} `orange' from Russian \textit{apʲelʲsin}; \textit{k'raot'} `bed' from Russian \textit{krovatʲ} [krɐvatʲ].\il{Russian}



Note that all loans undergo all phonological processes as described in Section~\ref{processes}. This means that all final and (under certain conditions) penultimate vowels are deleted, potentially triggering umlaut: \textit{mok'riv} `boxer', from Georgian \textit{mok'rive}, \textit{muine\textsuperscript{n}} `scabious, mangy' from Georgian \textit{munian-i} `id.' (where the \textit{-i-} triggered i-umlaut on the preceding syllable, after which the sequence \textit{ia} merged into \textit{-e-}); \textit{nejtlded} `godmother' from Georgian \textit{natlideda} `id.'.

\section{Summary}

In terms of basic description, this chapter has provided new insight into the following domains:

\begin{enumerate}
	\item A previously unknown phoneme \textit{w} ([ɦʷ] or [w̤]) has been identified.
\end{enumerate}


In terms of structural language contact, this chapter has shown the following parallels between Tsova-Tush and Georgian, which are most likely to be attributed to influence of the latter on the former language. 

\begin{enumerate}
	\item R-Dissimilation: Two suffixes containing \textit{r}, the Ablative case suffix \textit{-ren} and the archaic plural suffix \textit{-erč} change their \textit{r} to \textit{l} when the root it attaches to contains an \textit{r}. This feature was observed already in the oldest Tsova-Tush data.
	
\end{enumerate}

In terms of loanword adaptation, this chapter has shown that: 

\begin{enumerate}
	\item An older layer of loanwords can be identified containing the phonemes \textit{-o-} where standard Georgian has \textit{-va-} and \textit{-q-} where standard Georgian has \textit{-x-}. These sounds point to a Northeast Georgian donor dialect, most likely Tush Georgian.
	\item Georgian loanwords containing clusters with an initial \textit{m-} lose this sound in Tsova-Tush. Other clusters are preserved.
	
\end{enumerate}

