\chapter{Discussion} \label{conclusion}

\section{Summary of contact phenomena} \label{summaries}

The following sections will summarise all instances of contact-induced change discussed in this work. After a very brief section on phonology (\sectref{summphon}), matter borrowings will be discussed. Tsova-Tush features a large amount of Georgian loanwords, but since this is not the main topic of this work, what is discussed in \sectref{loanwordsconcl} is only the phonological and morphological adaptation of loanwords, and a chronology of borrowing we can establish. No affixes are borrowed into Tsova-Tush other than the plural marker \textit{-t} (\sectref{conclaffix}). \sectref{conclpattern} will summarise all pattern borrowings from Georgian into Tsova-Tush, both morphological (\sectref{conclmorph}) and syntactic (\sectref{conclsyntax}).

\subsection{Phonology} \label{summphon}\is{Georgian influence!Phonological}
As shown in \sectref{rdissim}, Tsova-Tush /r/ in certain suffixes dissimilates to /l/ when an /r/ occurs in the stem. The most likely explanation for this phenomenon is influence from Georgian. The outcome of this phonological development can already be observed in our earliest textual material.

\begin{table}
	\begin{tabular}{llccc}
    \lsptoprule
		&  & Mid-19th & Mid-20th &  \\
		Feature & Section & century & century & Present  \\
		\midrule
		\emph{Phonology}&  & & & \\
		R-Dissimilation & \ref{phon} & x & x & x \\
		\lspbottomrule
	\end{tabular}
	\caption{Phonological contact phenomena in different historical stages}
	\label{concl-table1}
\end{table}		

\subsection{Matter borrowing}
\subsubsection{Roots} \label{loanwordsconcl}\is{Georgian influence!Lexical}\is{Loanwords}

No extensive study on lexical borrowings into Tsova-Tush has been conducted to this date. \textcite{WS} discusses basic frequency of Georgian loanwords and lexical domains. An investigation into the historical development of loanword frequency has not been undertaken here for the following reason: only our corpus of contemporary Tsova-Tush (subcorpora E, MM, EK, LJ, WS (see \sectref{sources} for abbrevations) is sufficiently large and sufficiently diverse to investigate the frequency of occurrence of loanwords. For the older stages of Tsova-Tush, this is not possible: subcorpora AS and IT (mid-19th century) are too small and the wordlist in \textcite{tsiskarovgloss} contains many artificial loan translations that aren't found in naturalistic spoken data. The subcorpora YD (mid-20th century) is very small, and KK, although large, contains only deliberately devised sample sentences, produced by two or three speakers, which cannot be readily compared with naturalistic spoken data.


At this point, it is only possible to create a relative chronology of some groups of loanwords based on phonological (see \sectref{loanwordsphon}) and morphological (\sectref{morphadapt}) adaptation and on whether Georgian loanwords can be discerned in Chechen and Ingush as well. Hence we can divide lexical borrowings into three historical strata, without assigning absolute dating to these periods.

\begin{enumerate}
	
	\item An early group of loanwords contains Georgian loanwords that are found in all three Nakh languages. Of course it cannot be excluded that all three languages borrowed the same word from Georgian individually, but the splitting up of Proto-Nakh into its daughter languages has been understood to be fairly recent (\cite[1]{nichols03cc}), so the most efficient way of accounting for these shared loanwords is to assume one borrowing event, and subsequent splitting up of the Nakh languages. These words include Proto-Nakh \textit{*kotam} `chicken, hen' (Chechen/Ingush \textit{kuotam}, Tsova-Tush \textit{kotam}) from Georgian \textit{katam-i} `id.'; Proto-Nakh \textit{*kud} `hat' (Chechen/Ingush \textit{kuj}, Tsova-Tush \textit{kud}) from Georgian \textit{kud-i} `id.'; Proto-Nakh \textit{*vir} `donkey' (Chechen/Ingush/Tsova-Tush \textit{vir}) from Georgian \textit{vir-i} `id.', Proto-Nakh \textit{*xerx} `saw' (Chechen/Ingush \textit{xerx}) from Georgian \textit{xerx-i}, Proto-Nakh \textit{*gotan} `plough' (Chechen/Ingush \textit{guota}) from Georgian \textit{gutan-i}.\footnote{The vowel correspondence is irregular in this last word. Tsova-Tush \textit{gutaⁿ} represents a later (re-)borrowing.}
	
	\item In a later period, we find loanwords that show one or more of the following traits (see also \cite{faehnrich1998loanwords}):
	\begin{itemize}
		\item Words borrowed with the sound \textit{-o-} where Standard Georgian has \textit{-va-} (see \sectref{loanwordsphon}), which must have been borrowed from Tush Georgian. These include \textit{maq'ol} `blackberry', cf. Standard Georgian \textit{maq'val-i} `id.'; \textit{sov} `vulture', cf. Standard Georgian \textit{svav-i} `id.'; \textit{q'oil} `smallpox', cf. Standard Georgian \textit{q'vavil-i} `id.'.
		\item Words borrowed with the sound \textit{-q-}, which also must have been borrowed from Tush Georgian, such as \textit{qved} `wooden mallet', cf. Standard Georgian \textit{xveda} `id.'; \textit{qoqob} `pheasant', cf. Standard Georgian \textit{xoxob-i} `id.'.\il{Tush Georgian}
		\item Borrowed nouns that form their Oblique stem by suffixing \textit{-u} (\sectref{decladapt}).
		\item Borrowed nouns with the no longer productive diminutive suffix \textit{-aɁo} (\sectref{decladapt}).
		\item Borrowed adjectives that are incorporated into Tsova-Tush by adding a suffix \textit{-on} (\sectref{adjectives}), like \textit{ʒviroⁿ} `expensive', from Georgian \textit{ʒviri} `expensive', \textit{xabroⁿ} `stingy, greedy', from Georgian \textit{xarbi} `greedy, covetous', and \textit{qširoⁿ} `frequent', from Tush Georgian \textit{qširi} `frequent'.
	\end{itemize}
	
	\item The most recent layer of loanwords consists of words that do not exhibit any of the traits above. These are:
	
	\begin{itemize}
		\item Loanwords featuring the sequence  \textit{-va-} where Standard Georgian also has \textit{-va-}, borrowed more recently from Standard Georgian, such as \textit{ak'vaⁿ} `cradle' from Georgian \textit{ak'van-i} `id.', \textit{moʒ\u{g}var} `priest' from Georgian \textit{moʒ\u{g}var-i}  `id.', \textit{zvar} `large vinyard' from Georgian \textit{zvar-i}  `id.'.
		\item Words borrowed with the sound \textit{-x-}, where Tush Georgian has \textit{-q-}.
		\item Borrowed nouns that form their Oblique stem by suffixing \textit{-e} (\sectref{decladapt}).
		\item Borrowed adjectives that are not incorporated morphologically into Tsova-Tush, and appear in their bare stem, such as \textit{blant'} `sticky' (from Georgian \textit{blant'-i} `id.'), \textit{laq'} `rotten, bad' (from Georgian \textit{laq'e} `id.').
	\end{itemize}
	
\end{enumerate}

\textcite{gippert08} mentions several Tsova-Tush lexical items that seem like borrowings from Old or Middle Georgian. These include the word \textit{geps} (Oblique stem \textit{gepsu-}) `week', cf. Old Georgian \textit{msgeps-i} `id.' and \textit{k'atatv} `July', cf. Middle Georgian \textit{(m)k'atatve} `July'. Compare also the Tsova-Tush month names \textit{tibatv} `June', from Georgian \textit{tibatve} `June (lit. mowing month)'; \textit{enk'eneⁿ butt} `September', cf. Georgian \textit{enk'enistve} `September (lit. consecration month)'; \textit{\u{g}vinbeⁿ butt} `October', cf. Georgian \textit{\u{g}vinobistve} `October (lit. wine month)'. These archaic Georgian month names are all replaced by the pan-European words in the standard language, but survive in dialects from all over the Georgian language area (as evidenced by an extensive corpus search on the Georgian Dialect Corpus\footnote{\url{http://corpora.co/\#/corpus}}). 
In Tsova-Tush, all month names other than `June, July, September, October' are borrowed from Standard Modern Georgian (at least according to the dictionary of \textcite{kadkad84}). Thus, for the month names, a borrowing event at the Old Georgian stage is not necessary to assume. The word \textit{geps} `week' (Old Georgian \textit{msgeps-i}) is absent in all contemporary spoken varieties of Georgian and is also absent in Middle Georgian. Even though, strictly speaking, it is impossible to trace back the exact moment of borrowing into Tsova-Tush due to the lack of historic non-literary data, it is likely that this word was transmitted along with the Georgian christianisation attempts of the Nakh tribes (for Georgian religious loanwords in Vainakh, see \textcite{khalilov04contact}).





\subsubsection{Affixes} \label{conclaffix}\is{Georgian influence!Affix borrowing}
Only one bound affix was found that has been borrowed from Georgian. Tsova-Tush suffixes \textit{-t} to finite verbs in the following forms (see \sectref{suffixpl}): (1) imperative TAME forms, including the Simple Imperative, Polite Imperative, and Optative, (2) the Hortative is formed by adding the same plural marker and the inclusive 1st person plural pronoun \textit{vej}, but this form is only used when referring to more than two participants.
It is noteworthy that the only affix that is borrowed from Georgian into Tsova-Tush is a verbal inflectional affix, which is lowest in the borrowing hierarchy of \textcite[208]{matras2011universals}, see below in \sectref{conclusions}. This suffix is already found in the earliest Tsova-Tush sources (AS), in both the Simple Imperative (\textit{lakvibat} `throw him!' AS007-1.14) and the Hortative (\textit{latet vai} `let's begin' AS006-1.14).
In \sectref{adjadapt}, it is concluded that the suffix \textit{-ur} is not productive, and therefore not a borrowed derivational affix. Instead, it is only found on loanwords.

\begin{table}
	\begin{tabular}{llccc}
    \lsptoprule
		&  & {Mid-19th} & {Mid-20th} &  \\
		Feature & {Section} & {century} & {century} & {Present}  \\
		\midrule
		\emph{Affix borrowing}&  & & & \\
		Plural \textit{-t} & \ref{suffixpl} & x & x & x \\
		\lspbottomrule
	\end{tabular}
	\caption{Borrowing of affixes in different historical stages}
	\label{concl-table4}
\end{table}	


\subsection{Pattern borrowing} \label{conclpattern}
\subsubsection{Morphology} \label{conclmorph}
\begin{enumerate}
	\item In \sectref{spacase}, it became clear that in 10\% of all instances of Essive cases (Essive, Interessive, Inessive and Superessive), these cases are used to signal a lative function. It showed an exact functional parallel with Georgian, making contact-induced change the most likely explanation for this phenomenon. In Tsova-Tush, this merger is only observed in data from the past three decades, and as it is only 10\% of cases, this can be seen as a case of synchronic variation, rather than change.
	
	\item A similar case of contact-induced variation is shown in \sectref{adjectives}. Besides older, native synthetic expressions to form the degrees of comparison in adjectives, Tsova-Tush has borrowed analytic constructions from Georgian. The comparative is formed with a borrowed particle \textit{upro} `more' (Georgian \textit{upro} `more'), while the superlative is formed with a calqued expression \textit{ħama-x=eɁ} `than all' (compare Georgian \textit{q'vela-ze} `than all'). The analytic superlative construction is not found in the oldest Tsova-Tush sources, but is attested from the mid-20th century onward. The analytic comparative form with borrowed \textit{upro} is only found in the more contemporary sources. Sources from before 1985 use the synthetic comparative \textit{-xu}.
	
	\item As seen in \sectref{person}, Tsova-Tush person marking gradually developed over the course of the past two centuries. However, it can be said to constitute subject marking proper (and not cliticisation of pronouns) only in contemporary, 21st-century sources.
	
	\item In \sectref{relpro}, it was shown that Tsova-Tush created a series of relative pronouns on the model of Georgian ones, using an interrogative pronoun and suffixing a particle `and' \textit{menux-a} (non-Nominative \textit{menxu-čo-CASE-a}) `which', \textit{wun-e} `what', \textit{men-a} `who', compare Georgian \textit{romel-i-c} (non-Nominative \textit{romel-CASE-c}) `which', \textit{ra-c} `what', \textit{vin-c} `who'. These pronouns start to appear in sources from the mid-20th century onward, and are ubiquitous in the 21st century. Relative adverbs, however, such as \textit{moħ-e} `as, like', \textit{macn-e} `when', \textit{mičħ-e} `where' and \textit{meɬ-e} `as much as' are attested already in the 19th century (\sectref{adjunct}).
	
	\item As mentioned in \sectref{evidgeo}, it is at this point uncertain whether Tsova-Tush formed the periphrastic verb forms like the Perfect and Pluperfect on a Georgian model. This question will be addressed further in \textcite{wsverhees2024nakhevid}.
	
\end{enumerate}

Thus, we can summarise the instances of Tsova-Tush morphological pattern borrowing in \tabref{concl-table2}.

\begin{table}
	\begin{tabular}{llccc}
    \lsptoprule
		&  & {Mid-19th} & {Mid-20th} &  \\
		Feature & Section & {century} & {century} & {Present}  \\
		\midrule
		Lative-Essive merger 	& \ref{spacase}	& - & - & x \\
		Analytic comparative 	& \ref{adjectives} & - & - & x \\
		Analytic superlative 	& \ref{adjectives} & - & x & x \\
		Subject marking 		& \ref{person} & - & - & x \\
		Relative pronouns		& \ref{relpro} & - & x & x \\
		Relative  adverbs		& \ref{relpro} & x & x & x \\
		\lspbottomrule
	\end{tabular}
	\caption{Morphological pattern borrowing in different historical stages}
	\label{concl-table2}
\end{table}	


\subsubsection{Syntax} \label{conclsyntax}
In this work, syntactic pattern borrowings from Georgian have been investigated mainly in the domain of clause combining.

\begin{enumerate}
	\item As explained in \sectref{negpro}, Tsova-Tush now features constructions with a negative pronoun. These can occur with or without clausal negation, but both variants must be copied from Georgian as the original construction (marginally attested in Tsova-Tush, but the default in Ingush and Chechen) is formed by using an indefinite pronoun and clausal negation.
	
	\item In parallel with the formation of a relative pronoun itself, Tsova-Tush allows the formation of finite relative clauses with relative pronouns since the middle of the 20th century. In the 21st century, this strategy is the primary means to form relative clauses, see \sectref{relpro}. 
	
	\item The same can be said about the adjunct clauses with relative adverbs \textit{moħ-e} `as, like', \textit{macn-e} `when', \textit{mičħ-e} `where' and \textit{meɬ-e} `as much as', which are attested sparsely in the mid-20th century and commonly occur in the 21st century, see \sectref{adjunct}.
	
	\item In contemporary Tsova-Tush, we find finite relative clauses introduced by the general subordinating conjunction \textit{me}. These type of clauses appear only in the 21st-century sources, see \sectref{relme}.
	
	\item Complement clauses introduced by the general subordinating conjunction \textit{me} are found already in 19th-century sources. Examples mentioned in this work are: different-subject complement clauses to desiderative verbs (\sectref{desid}); complement clauses to verbs of cognition (\sectref{cognitive}) and perception (\ref{perception}).
	
	\item Adjunct clauses introduced by the general subordinating conjunction \textit{me} are found already in 19th-century sources. These in fact only concern purpose clauses (\sectref{purp}), which feature a subordinated verb in the Subjunctive. Georgian has more adjunct clause types which are formed using the general subordinator \textit{rom}, such as conditional or temporal clauses, but these have not been copied into Tsova-Tush.
	
	\item As described in \sectref{coord}, the Tsova-Tush conjunction \textit{je} is now used to coordinate some same-subject clauses, where previously this was done exclusively using verbs in the Sequential form. This more recent pattern has been observed in 21st-century Tsova-Tush mainly, with a handful of attestations in the 20th-century sources. It is absent from 19th century Tsova-Tush.
	
	
	
\end{enumerate}

Thus, we can summarise the instances of Tsova-Tush syntactic pattern borrowing in \tabref{concl-table3}.


\begin{table}
	\begin{tabularx}{\textwidth}{>{\hangindent=1em}Qlccc}
    \lsptoprule
		        &               & {Mid-19th} & {Mid-20th} &  \\
		Feature & {Section} & {century} & {century} & {Present}  \\
		\midrule
		
		Clauses with negative pronouns & \ref{negpro} & - & x & x \\
		Relative clauses with relative pronouns	& \ref{relpro} 	& -  & x  & x  \\
		Adjunct clauses with relative adverbs & \ref{adjunct} 	& x & x & x \\		
		Relative clauses with \textit{me} & \ref{relme}	& -  & -  & x  \\
		Complement clauses with \textit{me} & \ref{complement} & -	& x & x  \\
		
		Purpose clauses with \textit{me} & \ref{purp} & x & x & x \\
		Coordinating conjunction with same-subject clause & \ref{coord}	& - & x & x \\
		\lspbottomrule
		
	\end{tabularx}
	\caption{Syntactic pattern borrowing in different historical stages}
	\label{concl-table3}
\end{table}




\subsubsection{Functional extensions of existing forms}

Two existing Tsova-Tush form-function pairings have been extended to include an additional grammatical meaning, perhaps under the influence of Georgian. However, contact-induced change is difficult to prove here for different reasons.

\begin{enumerate}
	\item The distal demonstrative \textit{o} is used as the default 3rd person pronoun (where Tsova-Tush also has a proximal and a medial demonstrative), see \sectref{dempro}. This is observed in all stages of Tsova-Tush attestation. The same is true for Georgian: the distal demonstrative \textit{is} is used as the neutral 3rd person pronoun (Georgian also has a three-way deixis distinction in demonstratives). In Chechen and Ingush, however, it is the medial demonstrative (again out of three possible deixis values) that is used for the pronoun. It is safe to assume language contact as an explanation for the functional equivalence of the Tsova-Tush and Georgian forms; however, Daghestanian data is necessary to preclude a scenario where Chechen and Ingush were the ones who innovated.
	
	\item In both Tsova-Tush and Georgian, a verbal form called the Pluperfect, consisting of a past participle plus a past form of the verb `be', is used in counterfactual conditional clauses. In Georgian, however, this form is used in the subordinate conditional clause, while in Tsova-Tush it is used in the matrix clause (see \sectref{periph}). Furthermore, only Georgian intransitive verbs form the Pluperfect by combining a past participle and a past form of `be', whereas all Tsova-Tush verbs form the Pluperfect in such a way. Hence contact-induced change is at this point not a likely scenario for this particular usage of the Tsova-Tush Pluperfect.
\end{enumerate}


\subsection{Other typological similarities}

Other typological similarities that both Tsova-Tush and Georgian exhibit, but which are absent in Chechen and Ingush, were not investigated further, but deserve to be mentioned briefly.

\begin{itemize}
	\item In both Tsova-Tush and Georgian the meanings `read' and `ask' are colexified in the same verb: Tsova-Tush \textit{xat't'-ar}, Georgian \textit{k'itxv-a}. Compare Chechen/Ingush \textit{dieš-ar} `read', \textit{x\={a}tt-ar} `ask'.
	
	\item In Tsova-Tush, the word \textit{kort\u{o}} `head' can be used as an (Oblique) object in the meaning `self', see Example (\ref{concl-ex01}). 
	
	\begin{samepage}	
		\begin{exe}
			\ex\label{concl-ex01}
			\gll o šer korti-x=a dak'liv. \\
			{\Dist} {\Refl}.{\Gen}.{\Obl} head.{\Obl}-{\Cont}={\Add} think.{\Ipfv}({\Npst}) \\
			\trans `He thinks about himself as well.'
			\hfill (E002-47.1)
		\end{exe}
	\end{samepage}
	
	In Georgian, this is the default construction (\cite[84]{hewitt95}), but this use is not reported for Chechen or Ingush.
	
	\item In the lexical items `here' and `there', both Tsova-Tush and Georgian can express a lative meaning with (1) a zero ending, (2) an ending that in the nominal paradigm has the function of Instrumental. Compare Tsova-Tush \textit{uis} (< \textit{osi}) and \textit{osi-v} `thither', \textit{ese-v} `hither', \textit{tur-e-v} `with a sword'; Georgian \textit{ik} and \textit{ik-it} `thither', \textit{ak} and \textit{ake-t} `hither', \textit{xml-it} `with a sword'. This usage of the Instrumental case has not been observed in Chechen or Ingush.
	
	
	
\end{itemize}

\subsection{Summary} \label{summary}
Combining all tables presented in Sections~\ref{summphon}--\ref{conclsyntax}, we arrive at the following diachronic perspective of Tsova-Tush features that are a result of contact-induced change (see \tabref{concl-table}).


\begin{table}
	\begin{tabularx}{\textwidth}{>{\hangindent=1em}Q l *3{c}}
    \lsptoprule
		&  & {Mid-19th} & {Mid-20th} &  \\
		Feature & {Section} & {century} & {century} & {Present}  \\
		\midrule
		
		
		\emph{Phonology}& & & & \\
		
		R-Dissimilation & \ref{rdissim} & x & x & x \\

        \midrule
		\emph{Affix borrowing}&  & & & \\
		
		Plural \textit{-t} & \ref{suffixpl} & x & x & x \\
		\midrule
		\emph{Morphology (PAT)} &  & & & \\
		Lative-Essive merger 	& \ref{spacase}	& - & - & x \\
		Analytic comparative 	& \ref{adjectives} & - & - & x \\
		Analytic superlative 	& \ref{adjectives} & - & x & x \\
		Subject marking 		& \ref{person} & - & - & x \\
		Relative pronouns 		& \ref{relpro} & - & x & x \\
		Relative  adverbs		& \ref{relpro} & x & x & x \\
		\midrule
		\emph{Syntax (PAT)}&  & & & \\
		
		Clauses with negative pronouns & \ref{negpro} & - & x & x \\
		Relative clauses with relative pronouns	& \ref{relpro} 	& -  & x  & x  \\
		Adjunct clauses with relative adverbs & \ref{adjunct} 	& x & x & x \\		
		Relative clauses with \textit{me} & \ref{relme}	& -  & -  & x  \\
		Complement clauses with \textit{me} & \ref{complement} & -	& x & x  \\
		
		Purpose clauses with \textit{me} & \ref{purp} & x & x & x \\
		Coordinating conjunction with same-subject clause & \ref{coord}	& - & x & x \\
		\lspbottomrule
	\end{tabularx}
	\caption{Contact phenomena in different historical stages}
	\label{concl-table}
\end{table}

The most striking result is that phonological influence is present already in the mid-19th century, most syntactic influence starts in the middle of the 20th century, and the amount of morphological pattern borrowing increases sharply in the 21st century, from three to six patterns.

\section{Results}

If we combine the findings listed in \tabref{concl-table} with the sociolinguistic context of Tsova-Tush in the past 200 years (shown in \sectref{previouscontact}, based on \cite{mikeladze08interf}), we can draw up the following overview, presented in \tabref{concl-table-all}.

\begin{table}
	\small
	\begin{tabularx}{\textwidth}{>{\hangindent=1em}Qllll}
    \lsptoprule
		& 1 & 2 & 3 & 4 \\
		\midrule
		Dates & --1820 & 1820--1920 & 1920--1990 & 1990-- \\
		Government & Tsarist & Tsarist & Soviet & Post-Soviet \\
		Settlement & Tsovata valley & Alvani region & Alvani & Alvani \\
		Contact variety of Georgian	& Tush & 
				\multicolumn{1}{p{\widthof{(+Standard)}}}{Kakhetian (+Standard)} & 
				\multicolumn{1}{p{\widthof{(+Russian)}}}{Standard (+Russian)} & Standard \\
		\midrule
		           &   &        &        & E, MM, \\
		           &   &        &        & EK, LJ, \\
		Subcorpora & - & AS, IT & YD, KK & WS\\
		\midrule
		
		R-Dissimilation &  & x & x & x \\
		Plural \textit{-t} & & x & x & x \\
		Purpose clauses with \textit{me} &  & x & x & x \\
		Relative  adverbs		&  & x & x & x \\
		Adjunct clauses with relative adverbs &  	& x & x & x \\	
		
		Relative pronouns 		&  & - & x & x \\
		Relative clauses with relative pronouns	&  	& -  & x  & x  \\
		
		Clauses with negative pronouns &  & - & x & x \\
		Complement clauses with \textit{me} &  & -	& x & x  \\
		Coordinating conjunction  & 	& - & x & x \\	
		Analytic superlative 	&  & - & x & x \\
		
		Lative-Essive merger 	& 	& - & - & x \\
		Analytic comparative 	&  & - & - & x \\
		Subject marking 		& & - & - & x \\
		Relative clauses with \textit{me} & 	& -  & -  & x  \\
		\lspbottomrule
	\end{tabularx}
	\caption{Outcomes of Georgian-Tsova-Tush language contact by time period}
	\label{concl-table-all}
\end{table}


\section{Conclusions} \label{conclusions}

New descriptions of various facets of Tsova-Tush grammar were given in this work that are unrelated to language contact with Georgian.\footnote{More correctly: facets that at this point have not proven to be the result of language contact.} In \sectref{phon}, a previously unnoticed marginal phoneme \textit{w} ([ɦʷ] or [w̤]) is identified. In the nominal domain, it is recognised that most spatial cases involve a two-slot system, very similar to a typical Daghestanian system of spatial case (\sectref{spacase}), an attempt at distinguishing different nominal declension classes has been made (\sectref{Oblique}), and a modification construction is investigated where an endingless noun in the Oblique form can modify another noun (\sectref{barenoun}). In the verbal domain, a new category of Iamitive is identified, signifying the meaning `already' in positive clauses and `anymore' in negative clauses (\sectref{cont}), and a typology of complex verbs has been outlined in \sectref{lightverbs}. In the domain of clause combining, besides an extensive description of different subordination strategies in Tsova-Tush in \tabref{sub-table1}, this work provides a very preliminary analysis for Tsova-Tush coordination strategies with a basic distribution of the Sequential suffix \textit{-e} for same-subject clause coordination, and the coordinating conjunction \textit{je} for different-subject clause coordination.

Additionally, instances of contact-induced change have been identified and summarised above in \sectref{summaries}. In the sections below, these results will be put in the context of the general literature on language contact discussed in \sectref{theory}. The general conclusions on borrowability will be tested on the case of Tsova-Tush, as well as the theory of intensity of contact predicting certain structural outcomes. 

\subsection{Borrowability} \label{conclborrow}\is{Borrowability}

In \sectref{theory}, the issue of borrowability was discussed briefly. Several hierarchies have been established in the literature (e.g. \cite{moravcsik1978universalscontact,thomasonkaufman1988,vanhoutmuysken1944borrow,field2002borrow,matras2011universals}), such as arguably the most famous one: ``nouns are more likely to be borrowed than non-nouns and function words.'' These types of statements can be reformulated into verifiable or falsifiable claims in the following way: ``When non-nouns are borrowed into language A, nouns are also borrowed into language A, but the opposite is not necessarily true.'' Hence, the academic literature gives us testable claims that can be compared against the Tsova-Tush evidence for each historical period presented in \tabref{concl-table-all}.

\subsubsection{1820--1920}

In this period, we observe many loanwords (not investigated in this work), the plural suffix \textit{-t} and a phonological rule (R-dissimilation) that was borrowed. Additionally, we find pattern borrowing involving purpose clauses with the subordinator \textit{me}, as well as finite adverbial clauses with the relative adverbs \textit{moħe} `as, like', \textit{macne} `when' and \textit{mičħe} `where' calqued from Georgian. In this synchronic stage, we can try to test the following borrowing hierarchies (for each hierarchy, see \textcite{matras2011universals} and the literature cited there):\largerpage

\begin{description}[font=\normalfont]
	\item[\emph{Nouns > non-nouns, function words}] (\cite[208]{matras2011universals}): Even though this is not investigated in this work, we find many nouns, some verbs (\textit{c'olbala} `s/he suffers' AS002-1.7; \textit{c'eraddor} `s/he wrote' AS003-1.7), but no function words borrowed from Georgian in this period. Thus, if we do not differentiate between function words and other non-nouns, this hierarchy is not testable in Tsova-Tush. If we separate function words, however, we can confirm that at this stage, Tsova-Tush has borrowed nouns and verbs, but no function words, which is in accordance with this hierarchy.
	
	\item[\emph{Free morphemes > bound morphemes}]  (\cite[208]{matras2011universals}): Since at this stage Tsova-Tush borrowed both free morphemes (loanwords) and the bound morpheme \textit{-t}, this hierarchy is not testable in Tsova-Tush.
	
	\item[\emph{Derivational morphology > inflectional morphology}] (\cite[208]{matras2011universals}): The plural suffix \textit{-t} on imperative forms and on indicative forms of the first person inclusive is certainly an inflectional morpheme. In combination with the fact that Tsova-Tush did not borrow any derivational affix (or any other affix, for that matter) in this period or later, this leads us to the conclusion that the borrowing forms a counterexample to this hierarchy. One could argue that Tsova-Tush borrowed the \textit{-t} from Georgian contexts where it is a clitic on certain interjections, such as \textit{bodiši=t} `sorry', \textit{k'argi=t} `okay', \textit{salami=t} `hello', \textit{gamarǯoba=t} `hello'. Here, the plural suffix \textit{-t} found in verb forms (see \sectref{suffixpl})\footnote{Also used as a politeness marker.} has degrammaticalised into a clitic-like element attached to interjections (nouns and adjectives in origin). However, the use of the \textit{-t} in Tsova-Tush is completely parallel to its occurrence as a verbal suffix in Georgian, not as clitic to interjections. Also, the age of the usage of \textit{-t} with interjections in Georgian is not known, and it might be a recent innovation that occurred after borrowing into Tsova-Tush. Hence, at this point, the Tsova-Tush data form a counterexample to this proposed hierarchy.
	
	\item[\emph{Concessive, conditional, causal, purpose > other subordinators}] (\cite[220]{matras2011universals}): This hierarchy is not easily applied to the Tsova-Tush data. On the one hand, Tsova-Tush borrowed adjunct clause constructions at this stage, and no relative clause constructions, thus confirming this hierarchy. On the other hand, Tsova-Tush borrowed purpose clause constructions and those with the relative adverbs \textit{moħe} `as, like', \textit{macne} `when' and \textit{mičħe} `where' (i.e. manner, temporal, and locational clauses, respectively) and no concessive, conditional or causal clauses. Hence, it is prudent to use the Tsova-Tush data as neither an example in favour of, or a counterexample to this hierarchy. Note that the general subordinator \textit{me} is at this point exclusively used as purpose clause marker, and only later is used for other adjunct clauses and relative clauses. Thus, when applied only to the subordinator \textit{me}, at this historical stage, the hierarchy could be upheld.
	
	\item[\emph{Clause linking > word morphology}] (\cite[224]{matras2011universals}): In terms of pattern borrowing, it is clear that Tsova-Tush only copied constructions from Georgian in the domain of clause combining (at least at this historical stage). The only morphological borrowing that is observed is the calque of Georgian relative adverbs \textit{moħe} `as, like', \textit{macne} `when' and \textit{mičħe} `where', that make this type of finite subordination possible in the first place. If one is willing to view these word formations as part of the more general pattern borrowing of finite adjunct clauses, this hierarchy is confirmed by the Tsova-Tush data.
\end{description}

\subsubsection{1920--1990}

In addition to the material presented above, Tsova-Tush borrowed more constructions from Georgian in this period. The Georgian pattern of relative clauses with relative pronouns is borrowed, as well as clauses with negative pronouns. Complement clauses with the general subordinator \textit{me} are attested in this stage, as well as coordinating conjunctions with same-subject clauses. In terms of word morphology, the analytic superlative is calqued from Georgian. In this synchronic stage, we can test the following borrowing hierarchies:

\begin{description}
	% emph is not italics as above [JWS]
	\item [\emph{Higher numerals > lower numerals}] (\cite[213]{matras2011universals}): In this period, numerals higher than one hundred are all borrowed directly from Georgian (see \sectref{numerals}), confirming this hierarchy.
	\item [\emph{Superlative > comparative}] (\cite[220]{matras2011universals}): In this period, a superlative construction is calqued on the Georgian equivalent, whereas the comparative is not affected, hence confirming this hypothesis.
\end{description}



\subsubsection{1990--2020}

In this period, we witness the partial Lative-Essive merger, relative clauses with the subordinator \textit{me}, and the analytic comparative, of which the latter  is a case of matter borrowing. Additionally, we observe the true grammaticalisation and univerbation of subject pronouns into proper subject cross-reference markers.



\begin{description}
	\item[\emph{Clause linking > word morphology}] (\cite[224]{matras2011universals}): In terms of pattern borrowing, it is only in this period that we find morphological calquing besides the aforementioned relative pronouns and adverbs: the formation of subject cross-reference markers on verbs. 
\end{description}

Thus, Tsova-Tush largely follows the expected and established borrowing hierarchies with one clear exception and two constructions that can be called ambiguous. The borrowing of the plural marker \textit{-t} is a clear violation of the borrowing hierarchy ``derivational morphology > inflectional morphology'', since the only bound morpheme that Tsova-Tush borrowed is an inflectional one. Furthermore, it depends on one's analysis whether one should consider the formation of relative pronouns and adverbs as word morphology (and thus be a counterexample to the hierarchy ``clause linking > word morphology'') or view it as the borrowing of a larger syntactic construction of finite subordination. Lastly, concerning the hierarchy ``concessive, conditional, causal, purpose > other subordinators'', we have seen that Tsova-Tush borrows not only a finite purpose clause construction from Georgian, but also manner, temporal and locational adjunct clauses. 

\subsection{Intensity of contact}\is{Intensity of contact}

In terms of intensity of contact, it is clear that the sociolinguistic situation of Tsova-Tush changed throughout the last 200 years. In terms of the three factors outlined by \textcite[63--74]{thomasonkaufman1988} (relative population size, length of contact, and degree of bilingualism) the Tsova-Tush community underwent radical change. Before migrating to the Kakheti plain in the 1820s and 30s (see \sectref{history}), the relative size of the Tsova-Tush community was equal to each of the other Tush \textit{temis}: the Pirikiti, Chaghma, and Gometsari Tush (all between 1,300 and 1,600 individuals (\cite{statistics1893})). Thus, the Tsova-Tush were mostly in contact with the other Tush \textit{temis}, who together outnumbered them three to one. Only adult men took their flocks to the lowlands every winter, coming into (relatively superficial) contact with other Georgians like Kakhetians. After migrating, however, the Tsova-Tush found themselves on the Kakhetian plains among between 200,000 and 300,000 Kakhetians. The degree of bilingualism, although not directly investigated in this work, must have increased steadily, especially among adult women in the decades after migration, and especially with the advent of universal schooling, when children were becoming bilingual at an earlier age.

Thus, the degree of intensity of contact with the Georgian language increased sharply after the migration, but mapping the Tsova-Tush data onto the proposed levels of intense contact by \textcite{thomasonkaufman1988} is not straightforward, as can be seen in the overview below. Each level of intensity is discussed along with the predictions based on \textcite{thomasonkaufman1988} in quotes, and a discussion of the Tsova-Tush data.

\begin{description}
	\item[Casual contact:] ``Only non-basic vocabulary is borrowed.'' This must have been before the first written documents of Tsova-Tush in the mid-19th century, as we see a fair amount of pattern borrowing already there (see \ref{conclborrow}). At this stage, the language was probably (close to) Proto-Nakh, when non-basic vocabulary such as `chicken', `donkey', `hat', `saw', and `plough' were borrowed from Georgian.
	
	
	\item[Slightly more intense contact:] ``Besides the above, also conjunctions and adverbs are borrowed. Phonologically, new phonemes can be adopted, but only in loanwords. Syntactically, new functions of existing constructions are borrowed, or new orderings that cause no typological change.'' 
	
	\item[More intense contact:] ``Besides the above, also adpositions, derivational affixes, personal and demonstrative pronouns, low numerals are borrowed more likely than in category 2. Phonologically, stress rules and prosody are borrowed. Syntactically, minor orderings such as noun-adposition can be affected.'' 
	
	\item[Strong cultural pressure:] ``Besides the above, phonological rules can be borrowed, as well as the additional or loss of entire contrastive features. Extensive word order changes occur and inflectional categories and affixes are borrowed.'' 
	
	In the period \textbf{1820--1920}, we observe many loanwords, the plural suffix \textit{-t}, and a phonological rule (R-dissimilation) that was borrowed. Additionally, we find pattern borrowing involving purpose clauses with the subordinator \textit{me}, as well as finite adverbial clauses with the relative adverbs \textit{moħe} `as, like', \textit{macne} `when', and \textit{mičħe} `where' calqued from Georgian. 
	
	Hence, we see pattern borrowing that caused typological change (non-finite to finite subordination), borrowing of an inflectional affix, and borrowing of a morphophonological rule. However, no conjunctions, adpositions or pronouns were borrowed yet. In this work, prosodic features are not discussed, nor are word order changes. Still, it could be justified to classify the contact situation in this period as “strong cultural pressure”. Hence, the diagnostics that \citeauthor{thomasonkaufman1988} list do not fully apply to the Tsova-Tush data.
	
	\item [Very strong cultural pressure:] ``Significant typological change: new morphophonological rules, phonetic changes, word structure rules (such as prefixation vs. suffixation, flexional vs. agglutinative), alignment systems, agreement rules, and bound pronominal elements can all be borrowed.'' 
	
	The changes observed in the periods \textbf{1920--1990} (the Georgian pattern of relative clauses with relative pronouns, as well as clauses with negative pronouns; complement clauses with the general subordinator \textit{me}; coordinating conjunctions with same-subject clauses; the calqued analytic superlative) and \textbf{1990--2020} (partial Lative-Essive merger; relative clauses with the subordinator \textit{me}; the analytic comparative; the grammaticalisation and univerbation of subject pronouns into proper subject cross\hyp reference markers) together make up a significant addition to the instances of contact\hyp induced change. However, all these phenomena do not constitute changes in word structure rules, alignment systems or agreement rules, nor does do they involve borrowing of pronominal elements. Hence, this level of intensity as proposed by \textcite{thomasonkaufman1988} is not reached.
\end{description}

Thus, the change from \citeauthor{thomasonkaufman1988}'s “casual contact” to “strong cultural pressure” must have taken place before 1820, and in mountainous Tusheti. Clearly, more investigation is necessary on the bilingualism rate of the Tsova-Tush in the periods before 1820 (perhaps by utilising methods similar to those in \cite{dobrushina2013multipast}). Migrating to the Kakhetian plains intensified the contact situation heavily when looking at social and sociolinguistic indicators (relative population size, length of contact, and degree of bilingualism), but it does not constitute a new level of intensity in \citeauthor{thomasonkaufman1988}'s terms. Instead, it seems that language attitudes and ethnic self\hyp identification played a large part in explaining the cultural pressure before 1820. Most contact-induced changes occurred in this earlier period for which we do not have attestations, when the relative population size was more equal and the degree of bilingualism was relatively low, but ethnic self-identification shifted to Georgian.
