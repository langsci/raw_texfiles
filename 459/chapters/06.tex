\chapter{Clause combining} \label{subordination}

\section{Introduction}


In the following three sections, different strategies for forming subordinate clauses in Tsova-Tush will be described; relative clauses (\sectref{relative}), adjunct clauses (\sectref{adjunct}) and complement clauses (\sectref{complement}). Additionally, a first attempt at describing the system of coordination will be described in \sectref{coord}. After each construction type, a brief comparison will be made with Ingush and Chechen on the one hand, and Georgian (both the standard and the Tush variety) on the other, to better hypothesise about the archaic or innovative nature of the construction. Tsova-Tush makes use of both finite and non-finite strategies for subordination. It will be clear that Georgian has had a substantial influence on the syntax of Tsova-Tush in these domains, since Tsova-Tush developed an elaborate system of finite subordination that is absent in its sister languages. This profound influence of Georgian on Tsova-Tush in the domain of subordination constitutes a typological shift from non-finite subordination, as is ubiquitous for West Caucasian and East Caucasian languages, to finite subordination found in Kartvelian and Indo-European languages of the region. An important Trans-Caucasian parallel to the Tsova-Tush development is observed in Udi, a Lezgic language, having undergone a very similar shift under influence of Iranian languages and Armenian (see \cite{gippert2011udirelative,gippertschulze2023caucalbanian,schulzegippert2023caucalbanianudi}).

\section{Non-finite verb forms} \label{nonfinite}\is{Verbal Noun}\is{Participle}\is{Converb}\is{Derivation!Deverbal}\is{Derivation!Nominalising}\is{Derivation!Adjectivising}\is{Non-finite verb forms}

Non-finite subordination strategies use one of the following non-finite verb forms, listed in \tabref{table-subord1}. Tsova-Tush features two participles, a Non-Past Participle formed by suffixing \textit{-ni} to the Non-Past stem (ending in a lexically determined vowel, see \sectref{ind}) and a Past Participle, formed by suffixing \textit{-(n)o} to the Aorist stem. Participles are used mostly in participial relative clauses (\sectref{participial}) and the Past Participle is used in several periphrastic TAME paradigms (\sectref{periph}). Tsova-Tush features two converbs. I use \citeauthor{nedjalkov95converb}’s  (\citeyear{nedjalkov95converb}) definition, who refers to non-autonomous verb forms, different from infinitives, masdars/verbal nouns or participles, in that that they do not occur in complement clauses or in relative clauses). In comparison to Chechen, Ingush and other East Caucasian languages, Tsova-Tush has few converbs, which are general in semantics and only feature in temporal clauses. The Simultaneous Converb is based on the Non-Past stem, while the Anteceding Converb is based on the Aorist stem. Both are used in adjunct clauses (\sectref{adjunct}). The Tsova-Tush Verbal Noun in \textit{-ar} and the Infinitive in \textit{-an} are both used in complement clauses (\sectref{complement}), while the Infinitive is also used in purpose clauses (\sectref{purp}).\is{Infinitive}

Non-finite forms do not inflect for person, diathesis, evidentiality or mood, but do inflect for gender (for the one third of verb roots that have gender marking) and pluractionality (since all non-finite forms can be derived from both ``perfective'' and ``imperfective'' stems (see \sectref{rootperf})). Participles do inflect for tense and, similarly to adjectives (see \sectref{caseagree}), agree in case with their head noun.

\begin{table}
	\begin{tabular}{ll}
		\lsptoprule
		Non-Past Participle & \textit{-ani, -eni, -ini, -oni, -uni} \\
							& \textit{da\u{g}uin} (\textit{d-a\u{g}-oni}) `coming' \\
		Past Participle & \textit{-eno, -ino} \\
						& \textit{davin\u{o}} (\textit{d-av-ino}) `lost' \\
		Simultaneous Converb & \textit{-aš, -eš, -iš, -oš, -uš} \\
							 & \textit{st'exoš} (\textit{st'ex-oš}) `(while) waiting' \\
		Anteceding Converb & \textit{-ečeħ, -ičeħ} \\
						   & \textit{dejlče} (\textit{d-al-ičeħ} `(after) having died'  \\
		Verbal Noun & \textit{-ar} \\
				    & \textit{tit'ar} (\textit{tit'-ar}) `(the act of) cutting' \\
		Infinitive & \textit{-an} \\
		           & \textit{dagaⁿ} (\textit{d-ag-an}) `(to) see' \\
       \lspbottomrule
	\end{tabular}
	\caption{Tsova-Tush non-finite verb forms}
	\label{table-subord1}
\end{table}




\section{Relative clauses} \label{relative}\is{Relative clauses}\is{Georgian influence!Syntactic}

Tsova-Tush features both a gap strategy using participles, similar to other East Caucasian languages (\cite{nichols17rel,comrieforker17rel}), as well as finite  clauses with relative pronouns, similar to Kartvelian languages (\cite[190]{nichols17rel}, \cite[284--287]{aronson91}). Additionally, Tsova-Tush features a third type, using a general subordinator \textit{me}, copied from Georgian, as explained in \sectref{relme}.


\subsection{Participial clauses} \label{participial}

The first strategy for relativisation uses a participle showing case agreement (Nominative or Oblique) with the head noun in the matrix clause and gender agreement with the Nominative\footnote{Or sometimes Ergative, see \sectref{intrerg}.} argument in the relative clause. This type of relative clause precedes the head noun (\cites[205]{holiskygagua}[39--40]{haukharris}). In Example (\ref{part-ex1})\footnote{Original orthography of (\ref{part-ex1}): Isighaḥo baghuin Leki, ḥathx bḥo x̣ethbos \.{s}un le co?}, the participle bears no case marking, because the head it modifies is in the Nominative, while in (\ref{part-ex2}), the participle is marked Oblique, as the head it modifies bears the Dative case.\is{Agreement!Case}


	\begin{exe}
		\ex\label{part-ex1}
		\gll {{\normalfont [} isi-\u{g}}=aħo	{b-a\u{g}-\textbf{uin} {\normalfont]}}	lek'-i, ħatx	bʕo	qet-b-∅-o-s	šun	le	co?\\
		there-{\Trans}=down	{\M}.{\Pl}-come-\textbf{{\Ptcp}.{\Npst}}	Daghestanian-{\Pl} in\_front	army	attack.{\Ipfv}-{\B}.{\Sg}-{\Tr}-{\Npst}-{\Fsg}.{\Erg}	{\Spl}.{\Dat}	or	{\Neg} \\
		\trans `Daghestanians coming from down there, will I oppose your army or not?’	
		\hfill (AS010-8.1)
	\end{exe}



	\begin{exe}
		\ex\label{part-ex2}
		\gll wun-ak'	d-∅-o-lo-t	vaj	{{\normalfont[} ax-gomci\textsuperscript{n}}	{b-av-\textbf{in}-č\u{o} {\normalfont]}}	mat't'-a-n=a? \\
		what-{\Indf}	{\D}-do-{\Npst}-{\Sbjv}-{\Pl}	{\Fpl}.{\Incl}	half-{\Adterm} {\B}.{\Sg}-loose-\textbf{{\Ptcp}.{\Pst}}-{\Obl}	language-{\Obl}-{\Dat}={\Emph} \\
		\trans `What should we do for a partially lost language?’
		\hfill (MM128-11.1)
	\end{exe}



Arguments of participial verbs retain the case forms they would have in a matrix clause, as in Example (\ref{part-ex3}), where the subject of the verb \textit{at-} is Ergative (the same case as it would have in a matrix clause).


	\begin{exe}
		\ex\label{part-ex3}
		\gll {{\normalfont [} \textbf{as}}		{at-\textbf{in\u{o}} {\normalfont ]}}    tuix	mič	d-ax-e\textsuperscript{n}? \\
		\textbf{{\Fsg}.{\Erg}}	crush-\textbf{{\Ptcp}.{\Pst}}	salt	where	{\D}-go-{\Aor} \\
		\trans `Where did the salt that I have ground go?’
		\hfill (KK001-54.2)
	\end{exe}


Headless relative clauses can be formed using participles, with the case marking attaching directly to the participle, as in Example (\ref{part-ex4}). This is similar to other headless noun phrases as described in \sectref{subst}.\is{Headless relative clauses}\is{Relative clauses!Headless}


	\begin{exe}
		\ex\label{part-ex4}
		\gll {{\normalfont[} pex}	{d-ax-\textbf{en-čo-n} {\normalfont]}}	mak	\u{g}oč'	j-ett-or. \\
		near	{\D}-go-\textbf{{\Ptcp}.{\Pst}-{\Obl}-{\Dat}}	on	stick	{\J}-beat-{\Imprf} \\
		\trans `S/he was beating whatever came near (her/him) with a stick.’ 
		\hfill (\cite[205]{holiskygagua})
	\end{exe}


For the other Nakh languages, Chechen and Ingush, the strategy of forming relative clauses with participial verb forms is the most common (\cites[171--173]{desherieva}[587--600]{nichols11}{komen}). See Example (\ref{part-ex5}), where the Ingush nouns are modified with a pre-posed participial clause.

\begin{exe}
	\ex\label{part-ex5} 
    	Ingush
    \begin{xlist}
	
			\ex\label{part-ex5a}
			\gll {{\normalfont[} je}	hama	{d-\textbf{æ} {\normalfont]}}	sag  \\
			{\Prox}	thing	{\D}-do.\textbf{{\Ptcp}}	person	\\
			\trans `the person who did this’
			\hfill (\cite[590]{nichols11})
			
			\ex\label{part-ex5b}
			\gll {{\normalfont [} \=az}	kæxat	{j\=az-d-\textbf{æ} {\normalfont]}}	q'ɔlam \\
			{\Fsg}.{\Erg}	letter	write-D-\textbf{{\Tr}.{\Ptcp}}	pen	 \\
			\trans `the pen that I wrote the letter with’ 
			\hfill (\cite[591]{nichols11})
		
	\end{xlist}
\end{exe}

As in Tsova-Tush, participles in Ingush and Chechen agree with the head in either Nominative (unmarked) (\ref{part-ex6a}) or Oblique (\ref{part-ex6b}). The case marking of the arguments of the participial clause does not change compared to that of matrix clauses. Thus the subject of \textit{xuɁ-u} `know-{\Ptcp}' in Example (\ref{part-ex6a}) is in Dative, and the subject of \textit{j\={a}z-d-æ} `write-{\D}-{\Ptcp}' in Example (\ref{part-ex5b}) is in Ergative.

\begin{exe}
	\ex\label{part-ex6} 
    	Chechen
    \begin{xlist}
	
		
			\ex\label{part-ex6a}
			\gll {{\normalfont[} sajna}	{xuɁ-\textbf{u} {\normalfont]}}	dieš-naši	nīsa	sʕa-Ɂāl-a	læɁ-a	sūna. \\	
			{\Fsg}.{\Dat}	know-\textbf{{\Ptcp}.{\Prs}}	word-{\Pl}	right	{\Pv}-speak-{\Inf}	want-{\Prs}	{\Fsg}.{\Dat}	\\
			\trans `I want to pronounce the words that I know right.’ 
			\hfill (\cite{komen})
		
		
		
			\ex\label{part-ex6b}
			\gll {{\normalfont[} lyra}	{ħ\={ø}q-\textbf{u-ču} {\normalfont]}}   muox-uo	ditt-aš	uoram-aš-ca	sʕa-d-ōx-ura. \\	
			fiercely	blow-\textbf{{\Ptcp}.{\Prs}-{\Obl}}	wind-{\Erg}	tree-{\Pl}	root-{\Pl}-{\Ins}	{\Pv}-{\D}-extract-{\Imprf}	\\
			\trans `The fiercely blowing storm uprooted trees.’ 
			\hfill (\cite{komen})
		
	\end{xlist}
\end{exe}

Standard Modern Georgian features the same strategy as in Nakh (\cite[185--212]{hewitt87}). See Example (\ref{part-ex7}), where Georgian Past Participles in \textit{-ul} or \textit{-il} are preposed to create non-finite relative clauses. Agreement with the head noun is more extensive in Georgian than in Nakh\footnote{Modern Georgian adjectives have \textit{-i} if the head noun is in Nominative or Genitive, zero if the head noun is in Dative or Adverbial case, and adjectives copy the nominal case marker if the head noun is in Ergative or Vocative.}, and subject and object arguments of the participial clause are in the Genitive, as seen in \textit{p'lat'on-is} in Example (\ref{part-ex7b}).


\begin{exe}
	\ex\label{part-ex7} 
    Standard Modern Georgian
    \begin{xlist}
	
		
		
			\ex\label{part-ex7a}
			\gll daašora	mat	 tavis-i	siʒe	{{\normalfont[} m-is=gan}	{da-t'anǯ-\textbf{ul}-ma {\normalfont]}}	p'lat'on-ma.	\\
			s/he\_seperated\_it	{\Dem}.{\Pl}.{\Obl}	{\Poss}.{\Refl}-{\Agr}	brother\_in\_law	{\Dem}-{\Gen}={\Abl} {\Pv}-torment-\textbf{{\Ptcp}.{\Pst}}-{\Erg}	Plato-{\Erg} \\
			\trans `Plato, who had been tormented by his brother-in-law, got him away from them.’ 
			\hfill (\cite[187]{hewitt87})
		
		
		
			\ex\label{part-ex7b}
			\gll {{\normalfont[} p'lat'on-is}	{mo-txr-ob-\textbf{il}-ma {\normalfont ] }}	ambav-ma	bevr-i	sxva   tavgadasaval-i	gaaxsena.	\\
			Plato-{\Gen}	{\Pv}-tell-{\Tm}-\textbf{{\Ptcp}.{\Pst}}-{\Erg}	news-{\Erg}	much-{\Agr}	other	adventure-{\Nom}	it\_reminded\_of\_it	\\
			\trans `The news which Plato related reminded them of many other adventures.’ 
			\hfill (\cite[187]{hewitt87})
		
	\end{xlist}
\end{exe}

This strategy is already attested in Old Georgian, where modifiers such as relative clauses are are often postposed, as in Examples (\ref{part-ex8}).


	\begin{exe}
		\ex\label{part-ex8}
		Old Georgian
        
		\gll da	šeaginen	k'erp'-n-i	{{\normalfont[} vecxl-it}	mo-s-\textbf{il}-n-i	da 	okro-jt	{mo-s-\textbf{il}-n-i {\normalfont]}}. \\
		and	they\_cursed\_at	idols-{\Pl}-{\Nom}	silver-{\Ins}	{\Pv}-adorn-\textbf{{\Ptcp}.{\Pst}}-{\Pl}-{\Nom}	and gold-{\Ins}	{\Pv}-adorn-\textbf{{\Ptcp}.{\Pst}}-{\Pl}-{\Nom} \\
		\trans `And they were cursing at idols adorned with silver and at those adorned with gold.’ 
		\hfill (Oshki Bible: Isaiah 30.22)
	\end{exe}





\subsection{Clauses with a relative pronoun}\label{relpro}

In the second strategy, the Tsova-Tush head noun is followed by a finite relative clause,\is{Relative pronoun}\is{Pronouns!Relative} introduced by the relative pronoun \textit{menux-a} `which’. This relative pronoun consists of an interrogative pronoun followed by \textit{-a} or \textit{-e} (which finds its origin in the Additive\is{Additive} particle \textit{=a, =e}). Most often, a morpheme \textit{-(i)c(i)} is added to the verbal form in the relative clause (see Examples (\ref{pro-ex1}) and (\ref{pro-ex2})).


	\begin{exe}
		\ex\label{pro-ex1}
		\gll t'q'uiħsine	d-is-\u{u}	o-bi	{{\normalfont[} \textbf{menxu-čo-n-a}}	vašban	co	{d-ap'c'-mak'-in-\textbf{c\u{\i}} {\normalfont]}.} \\
		finally	{\D}-remain-{\Npst}	{\Dist}-{\Pl}	\textbf{which-{\Obl}-{\Dat}-{\Rel}}	{\Recp}	{\Neg}	{\D}-recognise-{\Pot}-{\Aor}-\textbf{{\Subord}}  \\
		\trans `At the end remain those who can’t recognise each other.’ \\
		\hfill (\cite[206]{holiskygagua})
	\end{exe}



	\begin{exe}
		\ex\label{pro-ex2}
		\gll načx teg-j-∅-an=en gamoq'opad-v-i-en-v-a st'ak', {{\normalfont[} \textbf{menxuj-čo-x-a}} mek'od {c'-e-\textbf{jc} {\normalfont]}.}\\
		cheese  do.{\Ipfv}-{\J}-{\Tr}-{\Inf}={\Ben} assign-{\M}.{\Sg}-{\Tr}-{\Ptcp}.{\Pst}-{\M}.{\Sg}-be man \textbf{which-{\Obl}-{\Cont}-{\Rel}} mekode be\_called-{\Npst}-\textbf{{\Subord}}\\
		\trans `For the manufacture of the cheese, a man is selected who is named a ``mekode''.
		\hfill (E005-4)
	\end{exe}


Examples (\ref{pro-ex3}) and (\ref{pro-ex4}) show how other relative pronouns based on the interrogatives `who' and `what' can be used to form headless relative clauses. A resumptive demonstrative is most often found in the matrix clause, as in (\ref{pro-ex3}), but is not obligatory (\ref{pro-ex4}).\is{Headless relative clauses}\is{Relative clauses!Headless}\is{Resumptive}


	\begin{exe}
		\ex\label{pro-ex3}
		\gll {{\normalfont[} \textbf{wun-e}}	{j-e-j\textbf{c} {\normalfont]}},	oh=a	nʕejɁ	qeħ bazir	daħ	j-oxk'-a\textsuperscript{n}. \\
		\textbf{what-{\Rel}}	{\J}-be-\textbf{{\Subord}}	{\Dist}={\Add}	out	take	bazaar.{\Ill}	{\Pv}	{\J}-sell-{\Inf} \\
		\trans `They also take what(ever) they have to the bazaar to sell.’
		\hfill (E006-85)
	\end{exe}



	\begin{exe}
		\ex\label{pro-ex4}
		\gll {{\normalfont[} \textbf{ħan-n-a}} {leɁ {\normalfont]}}, \u{g}-o, {{\normalfont[} \textbf{ħan-n-a}} co {leɁ {\normalfont]}}, co \u{g}-o-geg. \\
		\textbf{who.{\Obl}-{\Dat}-{\Rel}}	wish({\Npst})  go.{\Pfv}  \textbf{who.{\Obl}-{\Dat}-{\Rel}}	{\Neg} wish({\Npst})  {\Neg} go.{\Pfv}-{\Iam} \\
		\trans `Who wants to, will go, who doesn't want, will not go (anymore).’ \\
		\hfill (E226-44)
	\end{exe}


In Chechen and Ingush, these types of relative clauses seem to be absent, instead these language feature non-finite relativisation exclusively (\cites[587--600]{nichols11}{komen}), as is the case in the overwhelming majority of Daghestanian languages (\cites[147]{daniellander}[197]{hewitt04}).

In Standard Modern Georgian and in the Tush dialect,\il{Tush Georgian} we do find this type of relative clause, introduced by a relative pronoun that exactly mirrors the Tsova-Tush one.  To see this similarity, compare \tabref{adjunct-table3}, where the similar derivations in Tsova-Tush and Georgian are shown. To indicate the suppletion in the Tsova-Tush interrogatives ‘what’ and ‘who’, both the Nominative and an Oblique case are given.\is{Georgian influence!Morphological}

Compare the Georgian relative pronoun ‘which’ in (\ref{pro-ex5a}) and (\ref{pro-ex6}) with Tsova-Tush Example (\ref{pro-ex1}) and (\ref{pro-ex2}), and the Georgian relative pronoun ‘who’ (\ref{pro-ex5b}), with Tsova-Tush Examples (\ref{pro-ex3}) and (\ref{pro-ex4}), all three introducing a headless relative clause.

\begin{table}[H]
	\begin{tabular}{l@{~}lllll}
    \lsptoprule
		& & \multicolumn{2}{c}{{Tsova-Tush}} & \multicolumn{2}{c}{{Georgian}}\\\cmidrule(lr){3-4}\cmidrule(lr){5-6}
		& & {Interrogative} & {Relative} & {Interrogative} & {Relative} \\
		\midrule
		‘what’  & ({\Nom}) & \textit{wu\textsuperscript{n}} & \textit{wun-e} & \textit{ra} & \textit{ra-c} \\
		‘what’  & ({\Ins}) & \textit{st'e-v} & \textit{st'e-v-a} & \textit{r-it} & \textit{r-ita-c} \\
		‘who’   & ({\Nom}) & \textit{me\textsuperscript{n}} & \textit{men-e/a} & \textit{vin} & \textit{vin-c} \\
		‘who’   & ({\Dat}) & \textit{ħan-n} & \textit{ħan-n-a} & \textit{vi-s} & \textit{vi-sa-c} \\
		\multicolumn{2}{l}{‘which’} & \textit{menux} & \textit{menux-a} & \textit{romel-i} & \textit{romel-i-c} \\
		\midrule
		\multicolumn{2}{l}{(Additive particle)} & \multicolumn{2}{c}{\textit{=e, =a}} & \multicolumn{2}{c}{\textit{=c}}\\
        \lspbottomrule
	\end{tabular}
	\caption{Derived relative pronouns in Tsova-Tush and Georgian}
	\label{adjunct-table3}
\end{table}



\begin{exe}
	\ex\label{pro-ex5} 
    	Standard Modern Georgian
    \begin{xlist}
	
		
			\ex\label{pro-ex5a}
			\gll  xut-i	k'ac-i	gamočnda,	{{\normalfont[} \textbf{romel-i-c}}	{modiodnen {\normalfont]}}.	\\
			five-{\Nom}	man-{\Nom}	s/he\_appeared	\textbf{which-{\Nom}-{\Rel}}	they\_were\_coming \\	
			\trans `Five men appeared, who were approaching.’
			\hfill (\cite[186]{hewitt87})
		
		
		
			\ex\label{pro-ex5b}
			\gll cdebian	isini,	{{\normalfont[} \textbf{vin-c}}	t'q'via-c'aml-it	pikroben    glex-eb-is	{ga-čum-eba-s {\normalfont ]}}.    \\
			they\_err	{\Dist}.{\Pl}	\textbf{who-{\Rel}}	bullet-powder-{\Ins}	they\_think\_of\_sth peasant-{\Pl}-{\Gen}	{\Pv}-silence-{\Vn}-{\Dat}   \\
			\trans `They err who think to silence the peasants by bullets and gunpowder.’
			\hfill (\cite[186]{hewitt87})
		
	\end{xlist}
\end{exe}


	\begin{exe}
		\ex\label{pro-ex6}
		Tush Georgian
        
		\gll \={\i}Ɂs 	bič'-i=c 	\={\i}Ɂm 	otax-ši 	m\={\i}Ɂq'vanesa=v, {{\normalfont[} \textbf{romel-šia-c}} 	qelmc'ipe-m 	{utxra=v {\normalfont]}}.	\\
		{\Dist}.{\Nom}	boy-{\Nom}={\Add}	{\Dist}.{\Obl}	room-{\In}	they\_took\_sb={\Quot}	\textbf{which-{\In}-{\Rel}}	king-{\Erg}	s/he\_told\_sth\_to\_sb={\Quot}    \\
		\trans `They took that boy into that room too, in which the king told him […].’ \\
		\hfill (TU046-1.46)
	\end{exe}


The same type of construction is widespread in Old Georgian\il{Old Georgian}, where the interrogative pronoun \textit{romel-i} ‘which’ can function as a relative pronoun without the derivational marker \textit{-c} (see Example (\ref{pro-ex7})), which is obligatory in the modern language. It is, however, possible to add the morphemes \textit{-ca} ‘and’, \textit{-\u{g}a} ‘even’ to the interrogative/relative pronoun (\cite{gippertOGeo}). Additionally, third person pronouns can be attached. These personal pronouns are usually in the Nominative form, regardless of the case of the relative pronoun, as in \textit{igi} in Example (\ref{pro-ex7b}).


	\begin{exe}
		\ex\label{pro-ex7}
        	Old Georgian
        \begin{xlist}
		
            
			\ex\label{pro-ex7a}
			\gll msgavs 	arn 	per-i 	m-is-i 	iak'int-isa 	tual-isa, {{\normalfont[} \textbf{romel-i}} 	motetre 	{arn {\normalfont]}}. \\
			similar	it\_is	colour-{\Nom}	{\Dem}-{\Gen}-{\Nom}	hyacinth-{\Gen}	gem-{\Gen} \textbf{which-{\Nom}}	whitish	it\_is	\\
			\trans `It is similar in  colour as a hyacinth gem, which is whitish.’ \\
			\hfill (Cod.Satb., Epiph.Cypr. Gemm., B\_9, 157, 24 (88v, 61))
            
			\ex\label{pro-ex7b}
			\gll parisevel-man	man,	 {{\normalfont[} \textbf{ romel-man=ca=igi}}	xxada	{mas {\normalfont]}}	\\
			Pharisee-{\Erg}	{\Dem}.{\Erg}	\textbf{which-{\Erg}={\Add}={\Dist}.{\Nom}}	s/he\_summoned\_sb  {\Dem}.{\Dat}	\\
			\trans `the Pharisee who (had) summoned him’ \\
			\hfill (Lk. 7.39 W) (Gippert, forthc.)
		\end{xlist}
	\end{exe}




\subsection{Relative clauses with a subordinating conjunction}\label{relme}

A third strategy of Tsova-Tush relativisation involves a subordinating conjunction \textit{me} which introduces a finite relative clause. This \textit{me} is cognate with \textit{me\textsuperscript{n}}, the Nominative form of ‘who’ (in fact \textit{meⁿ} as a conjunction is found with some speakers), but it does not inflect: in Example (\ref{relme-ex1}), the relativised noun would have been an (intransitive) subject in the Nominative case, in (\ref{relme-ex2}) an agent in the Ergative, in (\ref{relme-ex3}) a possessor in the Adessive case, and in (\ref{relme-ex4}), an experiencer in the Dative case. Such a strategy is completely absent in the other Nakh languages (\cite{nichols11, komen}). The Ingush Emphatic particle \textit{mɨ} (see \textcite[726]{nichols11}) may or may not be cognate with Tsova-Tush \textit{me}. In any case it does not have a subordinating function in Ingush.


	\begin{exe}
		\ex\label{relme-ex1}
		\gll men-ak'	v-a-ra-l\u{o}	o	st'ak'	{{\normalfont[} \textbf{me}}	txoⁿ	ħatx	{uitt-ra-l\u{o} {\normalfont]}}. \\
		who-{\Indf}	{\M}.{\Sg}-be-{\Imprf}-{\Sbjv}	{\Dist}	man	\textbf{{\Subord}}	{\Fpl}.{\Dat}	in\_front	stand-{\Imprf}-{\Sbjv} \\
		\trans `Who would have been that man, who was apparently standing in front of us?’
		\hfill (EK001-2.10)
	\end{exe}



	\begin{exe}
		\ex\label{relme-ex2}
		\gll st'ak'o-v {{\normalfont[} \textbf{me}} \u{g}azi-š t'ot'-i {j-il-o {\normalfont]}} [...]   \\
		man.{\Obl}-{\Erg} \textbf{{\Subord}} good-{\Adv} hand-{\Pl} {\J}-wash-{\Npst} {} \\
		\trans `A man, who washes his hands well, [...]'
		\hfill (E147-167)
	\end{exe}



	\begin{exe}
		\ex\label{relme-ex3}
		\gll ma	oǯax	co	j-a-r	{{\normalfont[} \textbf{me}}	cħa	don	co	{b-a-ra-l\u{o} {\normalfont]}}.   \\
		but	family	{\Neg}	{\J}-be-{\Imprf}	\textbf{{\Subord}}	one	horse	{\Neg}	{\B}.{\Sg}-be-{\Imprf}-{\Sbjv}  \\
		\trans `But there wasn’t a family, that wouldn’t have [at least] one horse.’	 \\
		\hfill (E008-3)
	\end{exe}



	\begin{exe}
		\ex\label{relme-ex4}
		\gll comena	lex-mak'	txoⁿ	q'onoⁿ	pešk'r-i	{{\normalfont[} \textbf{me}}	bacbur	{qet-e-l\u{o} {\normalfont]}}.   \\
		no\_one	find-{\Pot}({\Npst})	{\Fpl}.{\Dat}	young	child-{\Pl}	\textbf{{\Subord}}	Tsova\_Tush	know-{\Npst}-{\Sbjv}  \\
		\trans `We can’t find any young people who know Tsova-Tush.’
		\hfill (BH076-152.1)
	\end{exe}


This type of relative clause can be found only in contemporary subcorpora containing 21st-century Tsova-Tush.
At this moment, it is unkown whether these type of relative clauses can occur without a head or not.

In Standard Modern Georgian, the same strategy is very widespread, and makes use of the conjunction \textit{ro(m)}, which developed from the Old Georgian relative/interrogative pronoun \textit{romel(i)} ‘which’. This strategy comes in three subtypes (\cite{harris1994rel}, whence the following examples are quoted). 

\begin{enumerate}[label=\Roman*.]
\item The postnominal gap strategy. It is considered gap strategy, because there is no nominal in the relative clause that is coreferential with the head noun (see \cite{harris1994rel}). A finite verb forms a relative clause after the head noun with the conjunction \textit{rom} in second position (Examples (\ref{relme-ex6}a,b)).

\begin{exe}
	\ex\label{relme-ex6}
    	Standard Modern Georgian
    \begin{xlist}
	
		
		
			\ex\label{relme-ex6a}
			\gll xalx-i	{{\normalfont[} k'ar-eb-tan}	axlos	\textbf{ro}	{idga {\normalfont]}},	aq'aq'anda.    \\
			people-{\Nom}	door-{\Pl}-{\Apud}	close	\textbf{{\Subord}}	s/he\_sat	s/he\_clapped \\	
			\trans `The people who sat close by the doors began to clap.’ \\
			\hfill(\cite[51]{vogt})
		
		
		
			\ex\label{relme-ex6b}
			\gll ert-i	mat-gan-i	{{\normalfont[} tma-ši}	\textbf{rom}	band-i	akvs	{čac'n-ul-i {\normalfont]}}.  \\
			one-{\Nom}	{\Tpl}-{\Abl}-{\Nom}	hair-{\In}	\textbf{{\Subord}}	band-{\Nom}	s/he\_has tie-{\Ptcp}.{\Pst}-{\Nom}	\\
			\trans `one of them who has a band tied in his hair’
			\hfill (\cite[51]{vogt})
		
	\end{xlist}
\end{exe}

\item The prenominal gap strategy, where a finite verb forms a relative clause before the head noun with the conjunction \textit{rom} in second posistion. A resumptive demonstrative is obligatory in the main clause (see Example (\ref{relme-ex7}).\is{Resumptive}


	\begin{exe}
		\ex\label{relme-ex7}
		Modern Georgian
        
		\gll {{\normalfont[} šen-gan}	\textbf{ro}	{mivi\u{g}eb {\normalfont]}}	\textbf{im}	pul-it	me  gadavixdi	val-s.   \\
		you-{\Abl}	\textbf{{\Subord}}	I\_receive\_it	\textbf{{\Dist}.{\Obl}}	money-{\Ins}	{\Fsg} I\_will\_pay\_it	debt-{\Dat}    \\
		\trans `I will pay off the debt with that money which I receive from you.’ \\
		\hfill (\cite[203]{tschenkeli})
	\end{exe}


\item The prenominal non-reduction strategy, where the head noun is integrated into the relative clause. It is called non-reduction, since there is a full noun phrase in the relative clause that is coreferential with the head noun (\cite{harris1994rel}). In the main clause, the noun can be resumed by a demonstrative (\ref{relme-ex8a}), but doesn’t have to. Alternatively, the entire nominal can be repeated (\ref{relme-ex8c}).

\begin{exe}
	\ex\label{relme-ex8}
    	Modern Georgian
    \begin{xlist}
	
		
			\ex\label{relme-ex8a}
			\gll minda,	{{\normalfont[} betania-ši}	\textbf{rom}	{k'olmeurnoba=a {\normalfont]}}	is	vnaxo.  \\
			I\_want\_it	Betania-{\In}	\textbf{{\Subord}}	collective={\Cop}	{\Dist}.{\Nom} I\_would\_see\_it   \\
			\trans `I want to see the collective farm that is in Betania.’ \\
			\hfill (\cite[51]{vogt})
		
		
		
		
			\ex\label{relme-ex8c}
			\gll {{\normalfont[} durmišxan-s}	alget-ze	\textbf{rom}	c'iskvil-i	{eč'ira {\normalfont]}}   is	c'iskvil-i.   \\
			Durmishkhan-{\Dat}	Alget-{\Super}	\textbf{{\Subord}}	mill-{\Nom}	s/he\_had\_it   {\Dist}.{\Nom}	mill-{\Nom}    \\
			\trans `… the mill which Durmishkhan had on [the river] Algeti’ \\
			\hfill (\cite[153]{ertelishvili})
		
	\end{xlist}
\end{exe}
\end{enumerate}

As discussed in \textcite{harris1994rel}, these strategies involving \textit{rom} are absent in Old Georgian and evolved around the Middle Georgian period. They are, however, observed in the Tush dialect (almost exclusively using the reduced form \textit{ro}).


\subsection{Summary}\label{relsum}

The strategies pertaining to relative clauses are summarised in \tabref{relme-table1}.

\begin{table}
	\begin{tabular}{lllll}
    \lsptoprule
		Chechen-Ingush  & {Tsova-Tush} & {Mod. Georgian} &  {Old Georgian} \\
		(and Daghestanian) & & {Tush Georgian} & \\
		\midrule
		Participle & Participle & Participle  & Participle \\
		& Pronoun & Pronoun  & Pronoun \\
		& Conjunction \textit{me} & Conjunction \textit{ro(m)} &  \\	
        \lspbottomrule
	\end{tabular}
	\caption{Relative clause types}
	\label{relme-table1}
\end{table}

All language varieties show a non-finite relativisation strategy involving a participle. It is likely that Tsova-Tush inherited this strategy from Proto-Nakh and Proto-East-Caucasian, since the entire family shows a clear preference for non-finite subordination (\cite[147]{daniellander}). Furthermore, the suffixes that derive participles are cognate across all Nakh languages, and the morphosyntax (agreement with the head noun and the marking of arguments in the relative clause) is the same. Exceptions to East Caucasian non-finite relative clauses are usually explained as structural copies from a dominant language, such as in Example (\ref{relme-ex9}), where Udi (East Caucasian > Lezgic) shows a subordinate clause with a relative pronoun, a construction very likely to have been borrowed from Georgian or Armenian.\il{Udi}


	\begin{exe}
		\ex\label{relme-ex9}
		Udi
        
		\gll čoban-ux	{{\normalfont[} \textbf{ma-t'-\u{g}-on-te}}	e<q'un>f-esai		biasun-un	{q'araul-ax {\normalfont]}}.   \\
		shepherd-{\Pl}	\textbf{which-{\Obl}-{\Pl}-{\Erg}-{\Rel}}	keep<{\Tpl}>-{\Imprf}		evening-{\Gen}	watch-{\Dat}2	\\
		\trans `… the shepherds who kept the evening watch.’ 
		\hfill (Luke 2:8. \cite{schulzeudigrammar})
	\end{exe}


Similar to Udi, the Tsova-Tush relative pronoun strategy shows clear parallels to Georgian. Although it is imaginable that Tsova-Tush copied this construction in the Old Georgian period, it is more likely to assume it did so after the Modern Georgian innovation of the form of the pronoun itself. While Old Georgian displays significant formal variation in the relative pronoun (often the bare interrogative \textit{romel-i} ‘which’, but also \textit{romel-i=ca, romel-i=\u{g}a, romel-i=igi, romel-i=ca=igi}, etc.), Modern Georgian settles on \textit{romel-i-c}, where the \textit{-c} is obligatory. It is exactly this form that is parallel to the Tsova-Tush relative pronoun \textit{menux-a}. The third strategy, which makes use of a general subordinating conjunction, is absent in Old Georgian, and only developed during the Middle Georgian period (\textit{romel}~> \textit{rome} (see Example (\ref{relme-ex10})) > \textit{rom} > \textit{ro}). 

	\begin{exe}
		\ex\label{relme-ex10}
		Middle Georgian
        
		\gll igi	sam-n-i	k'ac-n-i,	{{\normalfont[} \textbf{rome}}	lom-ta	zeda	{sxdes {\normalfont]}}.  \\
		{\Dist}.{\Nom}	three-{\Pl}-{\Nom}	man-{\Pl}-{\Nom}	\textbf{{\Subord}}	lion-{\Pl}.{\Obl}	on	they\_sit   \\	
		\trans `… those three men, that sit on lions.’ \\
		\hfill (Amirandar., 3, 303, 28. \cite[256]{dzidziguri})
	\end{exe}


It can therefore be hypothesised that Tsova-Tush copied the relative pronoun construction and the subordinator construction from Modern Georgian (whether Standard or Tush), where the form of the relative pronoun is settled on \textit{romel-i-c}, and the subordinator \textit{ro(m)} bears no resemblance anymore to its origin \textit{romel}.



\section{Adjunct clauses} \label{adjunct}\is{Adjunct clauses}\is{Adverbial clauses}\is{Georgian influence!Syntactic}

Tsova-Tush features both non-finite and finite adjunct clauses. Non-finite clauses make use of postpositions, of the Infinitive, of the Verbal Noun, and of the Simultaneous Converb \textit{-a/e/i/o/u-š} and the Anteceding Converb \textit{-ičeħ/-ečeħ}. Finite clauses  make use of the general subordinating conjunction \textit{me}, or of a series of relative adverbs. These are formed in the same way as the relative pronouns seen in \sectref{relpro} and are also calqued on the Georgian relative adverbs. See \tabref{adjunct-table4} for an overview.

\begin{table}
	\begin{tabular}{lllll}
    \lsptoprule
		&  \multicolumn{2}{c}{{Tsova-Tush}} & \multicolumn{2}{c}{{Georgian}}\\\cmidrule(lr){2-3}\cmidrule(lr){4-5}
		& {Interrogative} & {Relative} & {Interrogative} & {Relative} \\
		\midrule
		‘when’ & \textit{maca\textsuperscript{n}} & \textit{macn-e} & \textit{rodis} & \textit{rodesa-c} \\
		‘how’ & \textit{moħ} & \textit{moħ-e} & \textit{rogor} & \textit{rogor-c} \\
		‘how much’ & \textit{meɬ} & \textit{meɬ-e} & \textit{ramden-i} & \textit{ramden-i-c} \\
		‘where’ & \textit{mičeħ} & \textit{mičħ-e} & \textit{sad} & \textit{sada-c} \\
		\midrule
		(additive particle) & \multicolumn{2}{c}{\textit{=e, =a}} & \multicolumn{2}{c}{\textit{=c}}\\
    \lspbottomrule
	\end{tabular}
	\caption{Relative adverbs in Tsova-Tush and Georgian}
	\label{adjunct-table4}
\end{table}

Since multiple (finite and/or non-finite) strategies can be used for the same clause type, each of the following sections (temporal, purpose, conditional, causal and manner clauses) will contain all different strategies for that type.

\subsection{Temporal clauses}\label{temp}\is{Temporal clauses}\is{Converb}

Tsova-Tush features at least three distinct ways of forming temporal clauses: using converbs (I), nominal forms of the verb followed by a postposition (II), and conjunctions (III). It remains to be seen whether the general subordinating conjunction \textit{me} can be used in these types of adjunct clauses (IV). 

\begin{enumerate}[label=\Roman*.]
\item Temporal clauses can be expressed by way of specialised converbs, i.e. non-finite verbal forms functioning as adverbs. Tsova-Tush has two forms, a Simultaneous Converb \textit{-š} (\ref{temp-ex1}, \ref{temp-ex2}), and an Anteceding \textit{-čeħ} (\ref{temp-ex3}, \ref{temp-ex4}).\footnote{Original orthography of (\ref{temp-ex4}): O nḥaiwaiḽ\.{c}eḥ, wee ox̣un x̣ena mar quil.} These converbs often have the same subject as the matrix verb (\ref{temp-ex1}, \ref{temp-ex3}), but not always (\ref{temp-ex2}, \ref{temp-ex4}).

	\begin{exe}
		\ex\label{temp-ex1}
		\gll {{\normalfont[} cark'-i-v}	daħ	{tet'-\textbf{oš} {\normalfont]}},	daħ	j-aɬ-eⁿ	o	ħuna-x.    \\
		tooth-{\Pl}-{\Ins}	{\Pv}	cut-\textbf{{\Simul}}	{\Pv}	{\F}.{\Sg}-go\_out-{\Aor}	{\Dist}	forest.{\Obl}-{\Cont}  \\
		\trans `Biting with her teeth, she escaped from that forest.’
		\hfill (E179-114)

		\ex\label{temp-ex2}
		\gll qeⁿ	{{\normalfont[} sa-xiɬ-\textbf{uš} {\normalfont]}}	mič-ax	j-ax-eⁿ	o,	šarn	j-ax-e\textsuperscript{n}.  \\
		then	dawn-be.{\Pfv}-\textbf{{\Simul}}	whither-{\Indf}	{\F}.{\Sg}-go-{\Aor}	{\Dist}	away.{\Pv}	{\F}.{\Sg}-go-{\Aor} \\
		\trans `Then, as it was dawning, she went somewhere, she went away.’ \\
		\hfill (E179-101)

		\ex\label{temp-ex3}
		\gll {{\normalfont[} barnaul-i}	aħ	{v-ex-\textbf{če} {\normalfont]},}	c'ʕerkoⁿ	o	bʕar-d-ax-eⁿ	so\textsuperscript{n}.\\
		Barnaul-{\Iness}	down	{\M}.{\Sg}-go-\textbf{{\Ante}}	suddenly	{\Dist}	meet-D-{\Lv}-{\Aor}	{\Fsg}.{\Dat} \\
		\trans `And when I went to Barnaul, I suddenly met [them].’	
		\hfill (E275-43)

		\ex\label{temp-ex4}
		\gll {{\normalfont[} o}	nʕaiɁ {v-aiɬ-\textbf{čeħ} {\normalfont]}},	v-eɁ-eⁿ	oqu-n	qena	mar	q'uil.   \\
		{\Dist}	out {\M}.{\Sg}-go\_out-\textbf{{\Ante}}	{\M}.{\Sg}-come-{\Aor}	{\Dist}.{\Obl}-{\Dat}	other	husband	thief   \\
		\trans `When he had gone out, the other husband, the thief came to her.’ \\
		\hfill (AS008-1.10)
	\end{exe}


For the other Nakh languages, the use of converbs is the main strategy to convey temporal clauses (\cite{nichols11,nichols94Che}), using forms cognate to the Tsova-Tush ones, such as in (\ref{temp-ex5}), where Chechen uses the morpheme \textit{-ča} (cognate to Tsova-Tush \textit{-če(ħ)}) to form an anteceding temporal clause, and (\ref{temp-ex6}), where the Ingush morpheme \textit{-až} (cognate with Tsova-Tush \textit{-š}) is used to mark a simultaneous clause.


	\begin{exe}
		\ex\label{temp-ex5}
		Chechen
        
		\gll {{\normalfont[} šiena}	\={a}xča	{d-el-\textbf{ča} {\normalfont]}},	\={a}ra-v-\={e}l-ira	M\={u}s\={a}.    \\
		{\Refl}.{\Dat}	money.{\Nom}	{\D}-give-\textbf{{\Ante}}	out-{\M}.{\Sg}-go-{\Aor}	Musa	\\
		\trans `When (someone) gave him money, Musa went out.' \\
		\hfill (\cite[64]{nichols94Che})
		\ex\label{temp-ex6}
		Ingush
        
		\gll {{\normalfont[} šolla\u{g}}	nomer	{\={a}ra-j-oaqq-\textbf{až} {\normalfont]}},	so	sie	balxa	mɨ	ett-ar=\={\i}.    \\
		second	issue	out-{\J}-take-\textbf{{\Simul}}	{\Fsg}.{\Nom}	{\Fsg}.{\Refl}	hire.{\Adv}	{\Emph}	{\Lv}-{\Pst}={\Q}	\\
		\trans `When the second issue was coming out, I was already working there myself.’
		\hfill (\cite[602]{nichols11})
	\end{exe}


Besides these basic converbs, Chechen and Ingush feature a range of converbs for more specific temporal relations (Ingush has 13). For more details, see \textcite{nichols94Che,nichols11}, \textcite{good}. 


Modern standard Georgian shows one non-finite strategy for temporal adjunct clauses, by inflecting the verbal noun in the Dative of the Genitive (using so-called \textit{Suffixaufnahme} (i.e. case stacking), see \cite{boeder95}). This is illustrated in Example (\ref{temp-ex7}).\is{Suffixaufnahme}\is{Case stacking}


	\begin{exe}
		\ex\label{temp-ex7}
		Standard Modern Georgian
        
		\gll {{\normalfont[} aset-i}	gan-cxad-eb-is	{ga-k'et-eb-\textbf{isa-s} {\normalfont]}}	k'i  iuzer-ma	gaacnobieros.    \\
		such-{\Agr}	{\Pv}-announce-{\Tm}({\Vn})-{\Gen}	{\Pv}-make-{\Tm}\textbf{({\Vn})-{\Gen}-{\Dat}}	indeed	user-{\Erg}	s/he\_should\_realise	\\
		\trans `When making such an announcement, the user should realise…’ \\
		\hfill (\cite[55]{wier})
	\end{exe}


Thus, all language varieties make use of non-finite temporal clauses. In Chechen and Ingush, it is the default strategy, in Tsova-Tush it is one of many options, and in Georgian, it is a relatively marginal strategy.

\item To express the meanings ‘before, until’, or ‘after’, Tsova-Tush uses postpositions. \textit{doli\textsuperscript{n}}  ‘after’ requires an Anteceding Converb (Example (\ref{temp-ex8})), while \textit{-lomci\textsuperscript{n}} ‘until, before’ attaches directly to the verbal stem (Example (\ref{temp-ex9})).


	\begin{exe}
		\ex\label{temp-ex8}
		\gll {{\normalfont[} as}	laum-reⁿ	v-eɁ-\textbf{č'eħ} {\textbf{doli\textsuperscript{n}} {\normalfont]}} oqui-n\u{\i}	šariⁿ	k'ex	j-aɬ-j-al-i\textsuperscript{n}.    \\
		{\Fsg}.{\Erg}	mountain-{\Abl}	{\M}.{\Sg}-come-\textbf{{\Ante}}	\textbf{after}	{\Dist}.{\Obl}-{\Dat}	{\Poss}.{\Refl}	saddle\_tree	{\J}-give-J-{\Intr}-{\Aor}   \\
		\trans `After I came back from the mountains, s/he gave him/her a saddle tree.’
		\hfill (KK001-113.1)
	\end{exe}



	\begin{exe}
		\ex\label{temp-ex9}
		\gll {{\normalfont[} marxv}	{d-ast'-d-al\textbf{=lomci\textsuperscript{n}} {\normalfont]}}	ditxo-\u{g}eⁿ	dist'	com	ħ-or.   \\
		Lent	{\D}-undo-D-{\Intr}\textbf{=until}	meat-{\Adjz}	mouth({\Lat})	nothing	put-{\Imprf}   \\
		\trans `Until the breaking of the Lent, s/he wouldn’t put anything with meat in her mouth.’
		\hfill (KK013-1.179)
	\end{exe}


Chechen and Ingush  make use of the same strategy. Chechen uses the postpositions \textit{t'iaħa} ‘after’ (Example (\ref{temp-ex10a})) and \textit{ħalxa} ‘before’ (Example (\ref{temp-ex10b})). Compare Tsova-Tush cognates \textit{t'q'uiħ} ‘behind’ and \textit{ħatx} ‘in front of’, which are spatial postpositions, and never used with verbal nouns to form subordinate clauses). For similar constructions in Ingush, see \textcite[607]{nichols11}.\is{Postpositions}

\begin{exe}
	\ex\label{temp-ex10}
    	Chechen
    \begin{xlist}
	
		
			\ex\label{temp-ex10a}
			\gll {{\normalfont[} so}	c'a	v-eɁ-\textbf{ančul}    {\textbf{t'iaħa} {\normalfont]}},	s\={u}na	n\={a}b	q\={\i}t-ara.	\\
			{\Fsg}.{\Nom}	home	{\M}.{\Sg}-come-\textbf{{\Nmlz}}	\textbf{after}	{\Fsg}.{\Dat}	sleep	hit-{\Pst} \\	
			\trans `After I came home, I fell asleep.’ 
			\hfill (\cite{good})
		
		
		
			\ex\label{temp-ex10b}
			\gll {{\normalfont[} sunna}	g-\textbf{inčul} {\textbf{ħalxa} {\normalfont]}},	iza	aħmad-na	g-ira.  \\
			{\Fsg}.{\Dat}	see-\textbf{{\Nmlz}}	\textbf{before}	{\Tsg}	Ahmed-{\Dat}	see-{\Pst}	\\
			\trans `Before I saw him, Ahmed saw him.’
			\hfill (\cite{good})
		
	\end{xlist}
\end{exe}
\largerpage

Georgian too (both the modern standard variety and the Tush dialect) has the option to employ the non-finite strategy to express temporal clauses meaning `before' and `after'. In Example (\ref{temp-ex11}), Tush Georgian Verbal Nouns are used in combination with the Terminative case \textit{-amde}\footnote{In the Georgian grammatical tradition usually analysed as a clitic postposition.} to express the meaning `before', and the postposition \textit{šemdeg} `after'.\il{Tush Georgian}\is{Postposition}

\begin{exe}
	\ex\label{temp-ex11} 
    	Tush Georgian
        \begin{xlist}
	
		
			\ex\label{temp-ex11a}
			\gll zust'-ad	igr	mtavrdeba	{{\normalfont[} c'l-is}	{da-mtavr-\textbf{eb-amde} {\normalfont]}}.   \\
			exact-{\Advb}	so.{\Dist}	it\_finished	year-{\Gen}	{\Pv}-finish-{\Tm}\textbf{({\Vn})-{\Term}}  \\
			\trans `Exactly like that it finished before the end of the year.’ \\
			\hfill (GC052-1.44)
		
		
		
			\ex\label{temp-ex11b}
			\gll memre	ga\u{g}lec'aven	{{\normalfont[} ga-šr-\textbf{ob-is}} {\textbf{šemdeg} {\normalfont]}}. \\
			then	they\_thresh\_it	{\Pv}-dry-{\Tm}\textbf{({\Vn})-{\Gen}}	\textbf{after}   \\	
			\trans `Then, after drying it, they thresh it.’	
			\hfill (TT004-1.45)
		
	\end{xlist}
\end{exe}



\item In Tsova-Tush, it is also possible to form a finite temporal clause using a conjunction \textit{macne} ‘when’, which is formed in a similar way to relative pronouns, i.e. an interrogative pro-form \textit{macan} ‘when’ plus a relativiser \textit{-e} (which is originally the additive particle \textit{=e}). See Examples (\ref{temp-ex12}) and (\ref{temp-ex13}). Compare Georgian \textit{rodesa-c} ‘when-{\Rel} (Example (\ref{temp-ex14})), on the basis of which the Tsova-Tush \textit{macn-e} is calqued. Just as in relative clauses, the subordination suffix \textit{-(i)c(i)} occurs on the finite verb in this type of subordinate clause. The use of \textit{macne} is already observed in the oldest textual material (AS).\is{Conjunction}


	\begin{exe}
		\ex\label{temp-ex12}
		\gll {{\normalfont[} ħal}	\textbf{macn-e}	bʕark'-i	{q'at't'-in-c {\normalfont]}},	ʕurdna	j-a-nor.    \\
		{\Pv}	\textbf{when-{\Rel}}	eye-{\Pl}	open-{\Aor}-{\Subord}	morning	{\J}-be-{\Nw}.{\Rem}    \\
		\trans `When he opened his eyes, it was already morning.’
		\hfill (BH069-2.1)
	\end{exe}



	\begin{exe}
		\ex\label{temp-ex13}
		\gll {{\normalfont[} \textbf{macn-e}}	d-aq'-uin	men-ax	{d-a\u{g}-ui-c {\normalfont]}},	as	o	daħ	d-ʕev-o-s.    \\
		\textbf{when-{\Rel}}	{\D}-eat-{\Ptcp}.{\Npst}	who-{\Indf}	{\D}-come-{\Npst}-{\Subord}	{\Fsg}.{\Erg}	{\Dist}	{\Pv}	{\D}-kill-{\Npst}-{\Fsg}.{\Erg} \\
		\trans `When someone comes to eat [you], I will kill them.’
		\hfill (E181-209)
	\end{exe}


Modern Georgian has a subordinating conjunction that is structurally parallel to the Tsova-Tush one: \textit{rodesa-c} (from \textit{rodis(a)} + \textit{c} ‘when’ + a “relativiser” \textit{-c}, which is originally the additive particle \textit{=c}).


	\begin{exe}
		\ex\label{temp-ex14}
		Standard Modern Georgian
     
		\gll {{\normalfont[} \textbf{rodesac}} 	om-i	{damtavrdeba {\normalfont]}},	(mašin)	q'vela-n-i	bednier-eb-i	viknebit.	\\	
		\textbf{when({\Rel})}	war-{\Nom}	it\_will\_end	then	all-{\Pl}-{\Nom}	happy-{\Pl}-{\Nom}	we\_will\_be	\\
		\trans `When the war ends, (then) we shall all be happy.’ 
		\hfill (\cite[129]{hewitt87})
	\end{exe}


In Old Georgian, the equivalent for Modern Georgian \textit{rodesac} / \textit{roca}, was \textit{odes}, a related but structurally different conjunction, which also preceded the subordinate clause, as in (\ref{temp-ex15}).


	\begin{exe}
		\ex\label{temp-ex15}
		Old Georgian
        
		\gll da	{{\normalfont[} \textbf{odes}}	iq'o	{tw-isa=gan {\normalfont ]}},	hk'itxes... \\
		and	\textbf{when({\Rel})}	s/he\_was	{\Refl}-{\Gen}={\Abl}	they\_asked\_sb \\	
        \trans `And when he was alone, they asked him …’ 
		\hfill (Mk: 4.10, mss DE)
	\end{exe}


\is{Georgian influence!Lexical}Tsova-Tush speakers are also able to freely use the Georgian conjunction \textit{sanam} ‘before’,  as in (\ref{temp-ex16}). Compare the same construction in Standard Georgian (\ref{temp-ex17}).


	\begin{exe}
		\ex\label{temp-ex16}
		\gll {{\normalfont[} \textbf{sanam}}	seⁿ	mar	v-ux	{v-erc' {\normalfont]}},	as	ħo	šarn	v-ax-it-o-s.    \\
		\textbf{before}	{\Fsg}.{\Gen}	husband	{\M}.{\Sg}-back	{\M}.{\Sg}-return	{\Fsg}.{\Erg}	{\Ssg}	away	{\M}.{\Sg}-go-{\Caus}-{\Npst}-{\Fsg}.{\Erg}   \\
		\trans `I will let you go before my husband comes back.’
		\hfill (MM116-2.18)
	\end{exe}



	\begin{exe}
		\ex\label{temp-ex17}
		Modern Georgian
        
		\gll magram	{{\normalfont[} \textbf{sanam}}	ma\u{g}al	mta-ze	{avidoda {\normalfont]}},    sim\u{g}er-is	sit'q'v-eb-i	daavic'q'da.	\\
		but	\textbf{before}	high	mountain-on	s/he\_went\_up  song-{\Gen}	word-{\Pl}-{\Nom}	s/he\_forgot	\\
		\trans `But before he went up the mountains, he forgot the words to the song.’ 
		\hfill (GNC: E. Akhvlediani)
	\end{exe}


Chechen and Ingush do not form finite temporal clauses of this type.

\item Modern Georgian also features a subordination strategy using the conjunction \textit{rom} (as seen in relative relative clauses in \sectref{relative}), The conjunction can be used in first (\ref{temp-ex18a}) or second (\ref{temp-ex18b}) position. Because of the multifunctional nature of Georgian \textit{rom}, the temporal reading is not always the only one, see Example (\ref{temp-ex18b}).

\begin{exe}
	\ex\label{temp-ex18} 
    	Modern Georgian
    \begin{xlist}
	
		
			\ex\label{temp-ex18a}
			\gll axalgazrda	iq'o,	{{\normalfont[} \textbf{rom}}	gaatava	kutais-is	{gimnazia {\normalfont]}}. \\
			young	s/he\_was	\textbf{{\Subord}}	s/he\_finished\_it	Kutaisi-{\Gen}	high\_school    \\
			\trans `He was young, when he finished the Kutaisi high-school.’
			\hfill (\cite[131]{hewitt87})
		
		
		
			\ex\label{temp-ex18b}
			\gll {{\normalfont[} šen}	\textbf{rom}	aka	{xar {\normalfont]}},	ar	mešinia.    \\
			{\Ssg}	\textbf{{\Subord}}	here	you\_are	{\Neg}	I\_am\_afraid	\\
			\trans `When you are here, I am not afraid.’ (or ‘Because you are here, …’)
			\hfill (\cite[131]{hewitt87})
		
	\end{xlist}
\end{exe}

Tsova-Tush features a similar general subordinator \textit{me}, which, however, is not generally used used in temporal clauses found in the Tsova-Tush corpus. One instance of a mixed construction has been found (Example (\ref{temp-ex19})), using both the conjunction \textit{macne} and the conjunction \textit{me}, which is exceptional. 


	\begin{exe}
		\ex\label{temp-ex19}
		\gll eħat	cu=i 	teɬ-er	{{\normalfont[} \textbf{macne}}	\textbf{me}	atx=a	zurit'a	{ix-ora {\normalfont]}}. \\
		then	{\Neg}={\Q}	be\_better-{\Imprf}	\textbf{when({\Rel})}	\textbf{{\Subord}}  {\Fpl}.{\Excl}.{\Erg}={\Add}	Zurita({\Lat})	go.{\Ipfv}-{\Imprf}	\\
		\trans `Was it not better, when we were going to Zurita.’	
		\hfill (EK003-2.1)
	\end{exe}


Another example, (\ref{temp-ex20}), can be analysed as either a temporal clause, or a relative clause with the prenominal non-reduction strategy (where the head noun is modified by a demonstrative and is repeated in full in the relative clause).

\begin{exe}
	\ex\label{temp-ex20}
	\gll {{\normalfont[} duiħre-loⁿ} imp'erialist'ur msoplio buħ \textbf{me} {b-a=r {\normalfont]}}, o buħ-e-x lev-v-∅-or gac'vevad-v-i-en e.	\\
	first-{\Adjz} imperialist world war \textbf{{\Subord}} {\B}.{\Sg}-be-{\Imprf} {\Dist} war-{\Obl}-{\Cont} be.{\Nw}-{\M}.{\Sg}-{\Tr}-{\Imprf} invite-{\M}.{\Sg}-{\Tr}-{\Ptcp}.{\Pst} {\Prox} \\
	\\
		\trans `When it was the first World War, he was recruited for that war.' \\
		\hfill (E113-75)
	\end{exe}
\end{enumerate}


The strategies pertaining to temporal clauses are summarised in \tabref{temp-table1}.

\begin{table}
	\begin{tabular}{lllll}
    \lsptoprule
		Chechen-Ingush  & {Tsova-Tush} & {Mod. Georgian} &  {Old Georgian} \\
		& & {Tush Georgian} & \\
		\midrule
		Converb & Converb & (Inflected {\Vn}) &  \\
		{\Vn} + Postpos. & {\Vn} + Postpos. & {\Vn} + Postpos. &  \\
		& Conjunction & Conjunction   & Conjunction  \\	
		& & Conjunction \textit{ro(m)}  & \\
        \lspbottomrule
	\end{tabular}
	\caption{Temporal clause types}
	\label{temp-table1}
\end{table}



\subsection{Purpose clauses}\label{purp}\is{Purpose clauses}\is{Infinitive}

Tsova-Tush offers a non-finite, as well as a finite strategy to construct covalent purpose clauses. A non-finite construction involves the subordinated verb in the Infinitive from. The purpose clauses can be postposed, as in Example (\ref{purp-ex1}), or preposed, as in (\ref{purp-ex2}).


	\begin{exe}
		\ex\label{purp-ex1}
		\gll {{\normalfont[} wun-e}	{j-ej=c {\normalfont]}},	oh=a	nʕejɁ	qeħ	bazir	{{\normalfont[} daħ}	{j-oxk'-\textbf{a\textsuperscript{n}} {\normalfont]}}. \\
		what-{\Rel}	{\J}-be={\Subord}	{\Dist}={\Add}	out	take	bazaar.{\Ill}	{\Pv}	{\J}-sell-\textbf{{\Inf}}\\
		\trans `They also take what(ever) they have to the bazaar to sell.’
		\hfill (E006-85)
	\end{exe}



	\begin{exe}
		\ex\label{purp-ex2}
		\gll {{\normalfont[} bader}	šarn	{d-ik'-\textbf{a\textsuperscript{n}} {\normalfont]}}	d-eɁ-e\textsuperscript{n}.    \\
		child	away	{\D}-take-\textbf{{\Inf}}	{\D}-come-{\Aor}  \\
		\trans `She has come to take her child away.’	
		\hfill (E037-58)
	\end{exe}


The same strategy can be observed in Chechen and Ingush, with a cognate Infinitive suffix (Example (\ref{purp-ex3})).

\begin{exe}
	\ex\label{purp-ex3}
	Ingush
    
	\gll t'\={a}qqa	dʕ\={a}ra	ouša-næq'\={a}n	juxa	t'ɨ-b-æxk-\={a}b	{{\normalfont[} lat-\textbf{a} {\normalfont]}}. \\
	so	there	Ousha-neaqaan	back	at-{\Hpl}-come.{\Pl}-{\Nw}.{\Hpl}	fight-\textbf{{\Inf}}	\\
	\trans `Then the Ousha-neaq'aan clan returned to fight.’ 
	\hfill (\cite[288]{nichols11})
\end{exe}


Georgian forms these type of non-finite covalent purpose clauses with the subordinated verb in another non-finite form: the Future Participle in the Adverbial case (\ref{purp-ex4}). (Old Georgian uses the verbal noun in the Adverbial case, see \sectref{loanverb} on loan verb accommodation.)\is{Participle}

\begin{exe}
	\ex\label{purp-ex4}
    Modern Georgian
	 \begin{xlist}
		
		
			\ex\label{purp-ex4a}
			\gll {{\normalfont[} gamopen-is}	{\textbf{sa}-nax-\textbf{av-ad} {\normalfont]}}	movedi. \\
			exhibition-{\Gen}	\textbf{{\Ptcp}.{\Fut}}-see-\textbf{{\Ptcp}.{\Fut}-{\Advb}}	I\_came	\\
			\trans `I came to see the exhibition’ 
			\hfill (\cite[28]{hewitt87})
		
		
		
			\ex\label{purp-ex4b}
			\gll es	k'ac-i	{{\normalfont[} kal-is}	{mo-\textbf{sa}-k'l-\textbf{av-ad} {\normalfont]}}	gamoagzavnes.    \\
			{\Prox}.{\Nom}	man-{\Nom}	woman-{\Gen}	{\Pv}-\textbf{{\Ptcp}.{\Fut}}-kill-\textbf{{\Ptcp}.{\Fut}-{\Advb}}	they\_sent\_it	\\
			\trans `They sent this man here to kill the woman.’ 
			\hfill (\cite[28]{hewitt87})
		
	\end{xlist}
\end{exe}

Some Tsova-Tush purpose clauses are constructed with the general subordination conjunction \textit{me}, introducing a finite clause with the finite verb in the Subjunctive. This use of the conjunction \textit{me} is already attested in our oldest subcorpus from the 1850s, as can be seen from Example (\ref{purp-ex5}).\footnote{Original orthography of (\ref{purp-ex5}): ux dec̣es da, me xiḽul sogoḥ co x̣a\.{c}lain waxar?} All other uses of \textit{me} are younger, as will be discussed in \sectref{conclusions}. Example (\ref{purp-ex6}) gives another instance of \textit{me} introducing a purpose clause.


	\begin{exe}
		\ex\label{purp-ex5}
		\gll wux	d-ec'-e-s	d-∅-a\textsuperscript{n},	{{\normalfont[} \textbf{me}}	{xiɬ-u-\textbf{l}}	so-goħ	co-qač-l-ain	{v-ax-ar? {\normalfont]}}.    \\
		what	{\D}-must-{\Npst}-{\Fsg}.{\Erg}	{\D}-do-{\Inf}	\textbf{{\Subord}}	be.{\Pfv}-{\Npst}-\textbf{{\Sbjv}}	{\Fsg}-{\Adess}	{\Neg}-end-{\Intr}-{\Ptcp}.{\Npst}	{\M}.{\Sg}-live-{\Vn}    \\
		\trans `What must I do to obtain eternal life?’
		\hfill (AS005-1.2)
	\end{exe}



	\begin{exe}
		\ex\label{purp-ex6}
		\gll čuxu-i	šuiⁿ	nan-i-goreⁿ	čaq	d-ec'	xiɬ-a\textsuperscript{n},	{{\normalfont[} \textbf{me}}	vašbaⁿ	daħ {d-ic-d-∅-o-\textbf{l\u{o}} {\normalfont]}}.   \\
		lamb-{\Pl}	{\Refl}.{\Poss}	mother-{\Pl}-{\Adabl}	far	{\D}-must	be.{\Pfv}-{\Inf}	\textbf{{\Subord}}	{\Recp}	{\Pv}  {\D}-forget-{\D}-{\Tr}-{\Npst}-\textbf{{\Sbjv}}	    \\
		\trans `The lambs must be apart from their mothers so that they forget each other.’	
		\hfill (E002-35)
	\end{exe}


This strategy can be compared directly with the most common Modern Georgian strategy, using the general subordination conjunction \textit{rom}, also with the subordinated verb in the Subjunctive (see Example (\ref{purp-ex7a})).

\begin{exe}
		\ex\label{purp-ex7a}
        Standard Modern Georgian
        
			\gll moval	{{\normalfont[} \textbf{rom}}	{mogk'la {\normalfont]}}. \\
			I\_will\_come	\textbf{{\Subord}}	I\_would\_kill\_you \\	
			\trans `I will come in order to kill you.’ 
			\hfill (\cite[23]{hewitt87})
\end{exe}


The strategies pertaining to purpose clauses are summarised in \tabref{purp-table1}.

\begin{table}
	\begin{tabular}{lllll}
    \lsptoprule
		Chechen-Ingush  & {Tsova-Tush} & {Mod. Georgian} &  {Old Georgian} \\
		& & {Tush Georgian} & \\
		\midrule
		Infinitive & Infinitive & (Inflected {\Ptcp})  & (Inflected {\Vn}) \\
		& \textit{me} + {\Sbjv} & \textit{ro(m)} + {\Sbjv} & \\
        \lspbottomrule
	\end{tabular}
	\caption{Purpose clause types}
	\label{purp-table1}
\end{table}


\subsection{Conditional clauses}\label{condclause}\is{Conditional}\is{Conditional clauses}

Tsova-Tush has a dedicated verbal morpheme \textit{-ħe} to form conditional clauses, presented in more detail in \sectref{cond}. The reader is reminded that within Tsova-Tush, the  Conditional verb forms are best analysed as being finite, as (1) they co-occur with the morph \textit{-ra}, which, outside of the Conditional, only combines with finite forms, and (2) they can co-occur with person marking, which none of the non-finite forms (like the converbs used in other constructions in this chapter) are able to. Thus, morphologically, conditional forms behave like finite forms, but syntactically, they are converbs, as they are the head of an adjunct clause that does not contain a conjunction (see Examples (\ref{cond-ex1}) and (\ref{cond-ex2})).


	\begin{exe}
		\ex\label{cond-ex1}
		\gll {{\normalfont[} leɁ-e-\textbf{ħ}} 	{ħoⁿ {\normalfont]}},	as	ħo	šarn	v-ax-it-o-s.    \\
		wish-{\Npst}-\textbf{{\Cond}}	{\Ssg}.{\Dat}	{\Fsg}.{\Erg}	{\Ssg}	away	{\M}.{\Sg}-go-{\Caus}-{\Npst}-{\Fsg}.{\Erg}   \\
		\trans `If you want, I will let you go.’	
		\hfill (MM116-2.18)
	\end{exe}



	\begin{exe}
		\ex\label{cond-ex2}
		\gll at-uin-ši	duq	d-a,	{{\normalfont[} at-uinĭ}	wum	{d-a-\textbf{ħ} {\normalfont]}}.  \\
		grind-{\Ptcp}.{\Npst}-{\Pl}	many	{\D}-be	grind-{\Ptcp}-{\Npst}	something   {\D}-be-\textbf{{\Cond}}   \\
		\trans ‘We have many grinders, if anything needs to be ground.’
		\hfill (KK001-55.2)
	\end{exe}


Counterfactual conditional clauses are also constructed with a similar conditional form, with the subordinated verb in the Non-witnessed Remote Conditional (see Example (\ref{cond-ex3})). Additionally, the matrix verb is in the Pluperfect (a periphrastic form introduced in \sectref{periph}).\is{Counterfactual conditional}\is{Pluperfect}


	\begin{exe}
		\ex\label{cond-ex3}
		\gll {{\normalfont[} o}	din	{v-is-\textbf{noħer} {\normalfont]}}	so-g	ħal	ʕam-d-∅-it-en-d-a-r	oqu-s.  \\
		{\Dist}	alive	{\M}.{\Sg}-stay-\textbf{{\Nw}.{\Rem}.{\Cond}}	{\Fsg}-{\All}	{\Pv} study-{\D}-{\Tr}-{\Caus}-{\Ptcp}.{\Pst}-{\D}-be-{\Imprf}	{\Dist}.{\Obl}-{\Erg}    \\
		\trans `If he had been alive, he would have helped me study.’
		\hfill (E122-15)
	\end{exe}



	\begin{exe}
		\ex\label{cond-ex4}
		\gll {{\normalfont[} pxauzt'q'}	manat	xiɬ-\textbf{noħer}	{melot'-e-go {\normalfont]}},	ʕu-e-\u{g}	co	xiɬ-en-v-a-r.  \\
		hundred	manat	be.{\Pfv}-\textbf{{\Nw}.{\Rem}.{\Cond}}	bald-{\Obl}-{\Adess}	servant-{\Obl}-{\Trans}	{\Neg}	be.{\Pfv}-{\Ptcp}.{\Pst}-{\M}.{\Sg}-be-{\Imprf}    \\
		\trans `If the bald man had a hundred manat, I would not work as a servant.’	 \\
		\hfill (WS001-9.14)
	\end{exe}


The conditional suffix \textit{-ħe} can occur on both finite forms (\ref{cond-ex1}--\ref{cond-ex4}) as well as on the anteceding converb in \textit{-čeħ}. See (\ref{cond-ex5}), with a verb form in the Past Conditional.


	\begin{exe}
		\ex\label{cond-ex5}
		\gll {{\normalfont[} ejɬ-\textbf{če-ħer}}	dada-s	{{\normalfont[} me},	“gara	ču-tox,	{šuitt=en,” {\normalfont] ]}}    mzat	b-a-r	o	{{\normalfont[} ħatteɁ}	sak'er-g\u{o}	{xaɁ-aⁿ {\normalfont]}}. \\
		say-\textbf{{\Ante}-{\Cond}.{\Pst}}	father-{\Erg}	{\Subord}	{\Hort}	{\Pv}-hit	\textsc{interjection}={\Quot} ready	{\B}.{\Sg}-be-{\Imprf}	{\Dist}	immediately	neck-{\All}	sit-{\Inf} \\
		\trans `If the owner said: ``Come on, beat him, `shuitt','' [the dog] would jump on his neck immediately.’
		\hfill (E011-60)
	\end{exe}


The same morpheme can be found in Chechen and Ingush conditional clauses, such as in Example (\ref{cond-ex6}).


	\begin{exe}
		\ex\label{cond-ex6}
		Chechen
        
		\gll m\={u}s\={a}-na	xieta,	{{\normalfont[} sie-ga}	d\={\i}ʒalla	q\={a}ba-l\={u}	b-\={a}-c, {{\normalfont[} buolx}	ca	{xil-a-\textbf{ħ} {\normalfont] ]}}.	\\
		Musa-{\Dat}	think	{\Refl}-{\All}	family.{\Dat}	feed-{\Pot}	{\B}.{\Sg}-be-{\Neg}	work	{\Neg}	be-{\Prs}-\textbf{{\Cond}}	\\
		\trans `Musa believes he will not be able to feed his family if he doesn't have work’ 
		\hfill (\cite[64]{nichols94Che})
	\end{exe}


Georgian most commonly uses the conjunction \textit{tu} ‘if’ to form conditional clauses (see Example (\ref{cond-ex7})). A variety of tense-aspect forms can be used in both the matrix and the subordinate clause (for a more detailed description, see \textcite[73--86]{hewitt87}, and \cite{samushia2018subordination}).\largerpage


	\begin{exe}
		\ex\label{cond-ex7}
		Standard Modern Georgian
        
		\gll {\normalfont[} \textbf{tu}	sad-me	eguleboda	mo-sa-č'r-el-i   {ra-me {\normalfont]}}	ičkaroda.    \\
		{} \textbf{if}	where-{\Indf}	s/he\_was\_imagining	{\Pv}-{\Ptcp}.{\Fut}-cut-{\Ptcp}.{\Fut}-{\Nom}	what-{\Indf}	s/he\_was\_hurring	\\
		\trans `If s/he imagined something to cut anywhere, he used to hurry (there).’ \\
		\hfill (\cite[71]{hewitt87})
	\end{exe}


However, as in other types of subordinate clauses, the general conjunction \textit{rom} can be used (see Example (\ref{cond-ex8})). This type creates more vague conditionals, according to \textcite{hewitt87}. Here as well, many different tenses can signify different semantic nuance.

\begin{exe}
	\ex\label{cond-ex8} 	Standard Modern Georgian
    \begin{xlist}
	
		
		
			\ex\label{cond-ex8a}
			\gll k'arg-i	ikneboda,	{{\normalfont[} is}	\textbf{rom}	c'erdes	upro	{čkara {\normalfont]}.}  	\\
			good-{\Nom}	it\_would\_be	{\Dist}.{\Nom}	\textbf{{\Subord}}	he\_would\_write	more	quickly	\\
			\trans `It would be good, if he wrote more quickly.’ 
			\hfill (\cite[76]{hewitt87})
		
		
		
			\ex\label{cond-ex8b}
			\gll {{\normalfont[} \/{p}'ir-i}	c'q'l-it	\textbf{rom}	ara	mkondes	{savse {\normalfont]}}, vilap'arak'ebdi.  \\
			mouth-{\Nom}	water-{\Ins}	\textbf{{\Subord}}	{\Neg}	I\_would\_have	full	I\_would\_speak	\\
			\trans `If I didn’t have my mouth full of water, I’d speak.’ 
			\hfill (\cite[76]{hewitt87})
		
	\end{xlist}
\end{exe}

Neither strategy (with conjunction \textit{tu} or conjunction \textit{me}, the Tsova-Tush equivalent of \textit{rom}) has been found in the Tsova-Tush corpus, except for one occurrence (in a corpus of 250.000 tokens) of code-switched \textit{tu}. Thus, the strategies pertaining to conditional clauses are summarised in \tabref{cond-table1}.

\begin{table}
	\begin{tabular}{llll}
    \lsptoprule
		{Chechen-Ingush}  & {Tsova-Tush} & {Mod. Georgian} &  {Old Georgian} \\
		                        &                      & {Tush Georgian} & \\
		\midrule
		Conditional verb & Conditional verb &  &  \\
		& & Conjunction & Conjunction \\
		& & Conjunction \textit{ro(m)} & \\
        \lspbottomrule
	\end{tabular}
	\caption{Conditional clause types (both factual and counterfactual)}
	\label{cond-table1}
\end{table}


\subsection{Causal clauses}\label{caus}\is{Causal clause}


Tsova-Tush non-finite causal clauses can be formed with Oblique forms of participles, followed by the question particle \textit{=i} (compare Example (\ref{caus-ex1}) with English \textit{It’s dark, right (?), we can’t play outside.})


	\begin{exe}
		\ex\label{caus-ex1}
		\gll {{\normalfont[} uči\textsuperscript{n}}	{j-a-\textbf{ču=i} {\normalfont]}}	nʕaɁi	co	labc'-mak'	vai-n.  \\
		dark	{\J}-be({\Ptcp}.{\Npst})-\textbf{{\Obl}={\Q}}	out.{\Ess}	{\Neg}	play-{\Pot}({\Npst})	{\Fpl}.{\Incl}-{\Dat}    \\
		\trans `Because it is dark, we can’t play outside.’ 
		\hfill (\cite[204]{holiskygagua})

		\ex\label{caus-ex2}
		\gll co	xeɁ	so\textsuperscript{n},	{{\normalfont[} so\textsuperscript{n}}	k'aj	xet-ra-l\u{o},	{{\normalfont[} k'ac'k'o\textsuperscript{n}}	{j-a-\textbf{ču=i} {\normalfont] ]}}. \\
		{\Neg}	know	{\Fsg}.{\Dat}	{\Fsg}.{\Dat}	hot	find-{\Imprf}-{\Sbjv}	little	{\F}.{\Sg}-be({\Ptcp}.{\Npst})-\textbf{{\Obl}={\Q}}   \\
		\trans `I don’t know, whether I found it very hot, (just) because I was so little.’	 \\
		\hfill (MM222-1.4)
	\end{exe}


Another strategy involves a Verbal Noun\is{Verbal Noun} in the Contact case (see Examples (\ref{caus-ex3}) and (\ref{caus-ex4})).\is{Contact case}


	\begin{exe}
		\ex\label{caus-ex3}
		\gll {{\normalfont[} do\textsuperscript{n}}	co	{xiɬ-\textbf{r-e-x} {\normalfont]}} 	kuitad 	v-eɁ-n-as.    \\
		horse	{\Neg}	be.{\Pfv}-\textbf{{\Vn}-{\Obl}-{\Cont}}	on\_foot	{\M}.{\Sg}-come-{\Aor}-{\Fsg}.{\Erg}   \\
		\trans `Because I don’t have a horse, I came on foot.’ 
		\hfill (\cite[204]{holiskygagua})

		\ex\label{caus-ex4}
		\gll {{\normalfont[} c'q'e-ša-c'}	{j-itt-\textbf{r-e-x} {\normalfont]}}	axlu\u{g}-goħ	bos	b-aɬ-eⁿ.  \\
		once-two.{\Obl}-{\Mult}	{\J}-wash-\textbf{{\Vn}-{\Obl}-{\Cont}}	tunic-{\Adess}	colour	{\B}.{\Sg}-go\_out-{\Aor}    \\
		\trans `The tunic’s colour faded from washing it once or twice.’
		\hfill (KK002-1.263)
	\end{exe}


Chechen and Ingush, too, can form non-finite causal clauses, albeit with different verbal forms. Chechen uses the masdar form of the verb in the Dative (\ref{caus-ex5}), while Ingush features a specialised converb (\ref{caus-ex6}).


	\begin{exe}
		\ex\label{caus-ex5}
		Chechen
        
		\gll {{\normalfont[} aħamada}	mal\={\i}kina	t'\={a}ra	{tuox-\textbf{ar-na} {\normalfont]}},	i	j-uelx-aš	j-a-ra. \\
		Ahmed.{\Erg}	Malika.{\Dat}	palm	hit-\textbf{{\Vn}-{\Dat}}	{\Tsg}	{\F}.{\Sg}-cry-{\Simul}	{\F}.{\Sg}-be-{\Pst}	\\
		\trans `Because Ahmed slapped Malika, she was crying.’ 
		\hfill (\cite{good})
		
		\ex\label{caus-ex6}
		Ingush
        
		\gll {{\normalfont[} \={a}z}	d-erriga	urs-až	j\={a}škʲ\={a}=čɨ	cʕan	{ʕa-čɨ-d-exk-\textbf{andæ} {\normalfont]}}	urs-až	sixa	ærh-lu. \\
		{\Fsg}.{\Erg}	{\D}-all	knife-{\Pl}	drawer=in	together {\Pv}-{\Pv}-{\D}-put.{\Pl}-\textbf{{\Caus}}	knife-{\Pl}	fast	dull-{\Intr}.{\Prs}	\\
		\trans `Because I put all the knives together in the drawer, they get dull quickly.’  \\
		\hfill (\cite[608]{nichols11})
	\end{exe}


Another strategy to form causal clauses in Tsova-Tush, is to use a relative-type clause with the conjunction \textit{oqui-n=da{ɬ}a me}, which is the distal demonstrative pronoun in the dative, followed by a postposition ‘because of’, followed by the general subordinator \textit{me} (see Example (\ref{caus-ex7})). This constitutes a structural copy of Georgian \textit{im-it'om, rom} ‘because’, which consists of the distal demonstrative pronoun in the Oblique form, followed by a suffix \textit{-it'om} (that hitherto remains unetymologised) followed by the general subordinator \textit{rom}.


	\begin{exe}
		\ex\label{caus-ex7}
		\gll {{\normalfont[} \textbf{oqui-n=da{ɬ}a}}	\textbf{me}	lamu	že	aħ	d-ec'	{d-ett-aⁿ {\normalfont]}}. \\
		\textbf{{\Dist}.{\Obl}-{\Dat}=because\_of}	\textbf{{\Subord}}	mountain({\Ess})	sheep	{\Pv}	{\D}-must	{\D}-milk-{\Inf}  \\
		\trans `[…], because in the mountains, the sheep need to be milked.’
		\hfill (E002-31)
	\end{exe}


A similar construction features \textit{dax} / \textit{daxaɁ} / \textit{daxeɁ} instead of \textit{oquinda{ɬ}a} (Example (\ref{caus-ex8})), which is found already in the oldest textual material (corpus AS). While the morphological structure of \textit{oquinda{ɬ}a} closely resembles Georgian \textit{imit'om}, it seems less likely that \textit{dax} is a calque. 


	\begin{exe}
		\ex\label{caus-ex8}
		\gll albat	pxi	šar-lu\textsuperscript{n},	{{\normalfont[} \textbf{daxaɁ}}	\textbf{me}	ǯer	sk'ol-i	co	{ix-ra-s {\normalfont]}}.    \\
		probably	five	year.{\Obl}-{\Adjz}	\textbf{because\_of\_this}	\textbf{{\Subord}}	yet	school-{\Iness}	{\Neg}	go-{\Imprf}-{\Fsg}.{\Erg}    \\
		\trans `[I was] probably five years old, because I wasn’t yet going to school.’ \\
		\hfill (MM222-1.1)
	\end{exe}


Chechen, too, has the possibility to form causal clauses with a conjunction, although the construction is different, with the conjunction \textit{d\={e}la} ‘because’, cognate with Tsova-Tush \textit{da{ɬ}a}, in the final position of the subordinate clause (see Example (\ref{caus-ex9})).


	\begin{exe}
		\ex\label{caus-ex9}
		Chechen
        
		\gll mal\={\i}ka	j-uelx-aš	j-a-ra, {{\normalfont[} aħmada}	šiena	t'\={a}ra	tøx-\textbf{na}	{\textbf{d\={e}la} {\normalfont]}}.	\\
		Malika	{\F}.{\Sg}-cry-{\Simul}	{\F}.{\Sg}-be-{\Pst}	Ahmed.{\Erg}	{\Refl}.{\Dat}	palm	hit-\textbf{{\Ante}}	\textbf{because}	    \\
		\trans `Malika was crying, because Ahmed slapped her.’ 
		\hfill (\cite{good})
	\end{exe}


Now compare the Tsova-Tush constructions with (both standard and dialectal) Georgian \textit{imit'om ro(m)} ‘for that reason, that’, introducing causal clauses. \textit{imit'om} can be anywhere in the main clause (Example (\ref{caus-ex10a})), but usually it is not separated from \textit{rom}, at the very start of the subordinate clause, as in (\ref{cause-10b}).

\begin{exe} 
	\ex\label{caus-ex10} 
    		Standard Modern Georgian
            \begin{xlist}

		
			\ex\label{caus-ex10a}
			\gll q'ur-i	\textbf{imit'om}	ar	izrdeba,    {{\normalfont[} \textbf{rom}}	bevr-is	{gamgonea {\normalfont]}}	=o.	    \\
			ear-{\Nom}	\textbf{for\_that\_reason}	{\Neg}	it\_grows	\textbf{{\Subord}}	much-{\Gen}	it\_hears	={\Quot}	\\
			\trans `They say the ear does not grow because it hears a lot.’ \\
			\hfill (\cite[59]{hewitt87})
		
		
		
			\ex\label{cause-10b}
			\gll {{\normalfont[} rodesac}	kal-s	k'laven,	{{\normalfont[} \textbf{imit'om}}	\textbf{rom}	{kal-i=a {\normalfont] ]}}	es	sazogadoeb-is	p'roblema=a. \\		
			when	woman-{\Dat}	they\_kill\_sb	\textbf{for\_that\_reason}	\textbf{{\Subord}}	woman-{\Nom}={\Cop}	{\Prox}.{\Nom}	society-{\Gen}	problem={\Cop}		\\
			\trans `When a woman is killed because she is a woman, that’s a societal problem.’
			\hfill (S. Zurabishvili)
		
	\end{xlist}
\end{exe}

Old Georgian, too, used finite causal clauses introduced by a conjunction, although the conjunction here was usually \textit{rametu} ‘because’ (see Example (\ref{cause-ex11})).

	\begin{exe}
		\ex\label{cause-ex11}
		Old Georgian
        
		\gll {{\normalfont[} \textbf{rametu}}	igi=a	\u{g}mert-i	macxovari	{čwen-i {\normalfont]}}.  \\
		\textbf{because}	{\Dist}.{\Nom}={\Cop}	god-{\Nom}	animating	{\Fpl}.{\Poss}-{\Nom}	\\
		\trans `[…] for he is our animating God’ 
		\hfill (Ps. 94.7)
	\end{exe}


The strategies pertaining to causal clauses are summarised in \tabref{caus-table1}.

\begin{table}
	\small
	\begin{tabular}{lllll}
    \lsptoprule
		Chechen-Ingush  & {Tsova-Tush} & {Mod. Georgian} &  {Old Georgian} \\
		& & {Tush Georgian} & \\
		\midrule
		Converb, verbal noun & Participle, verbal noun &  &  \\
		Conjunction & Conjunction & Conjunction & Conjunction \\
        \lspbottomrule
	\end{tabular}
	\caption{Causal clause types}
	\label{caus-table1}
\end{table}


\subsection{Manner clauses}\label{man}\is{Manner clauses}

Tsova-Tush can form manner clauses with a conjunction formed from the interrogative \textit{moħ} ‘how’ with an added \textit{-e} (see Example (\ref{man-ex1})), similar to the relative pronoun (see \sectref{relpro}) and the temporal conjunctive (\ref{temp}):


	\begin{exe}
		\ex\label{man-ex1}
		\gll ħal	ħarč	xink'al	uišt'	{{\normalfont[} \textbf{moħ-e}}	nana-s	tec'-d-i-er	{ħo-g {\normalfont]}}.   \\
		{\Pv}	fold	khinkali	so.{\Dist}	\textbf{how-{\Rel}}	mother-{\Erg}	teach-{\D}-{\Tr}-{\Rem}	{\Ssg}-{\All} \\
		\trans `Fold up khinkali the way mother taught you.’	
		\hfill (BH037-46.1)
	\end{exe}


From my available sources, it is unclear how the other Nakh languages form manner clauses of this type. Ingush does have a cognate form used in a similar context, but the exact construction is different: \textit{muo} is used as a postposition of non-verbal phrases only (see Example (\ref{man-ex2a}) from \cite[514]{nichols11}). It seems that Example (\ref{man-ex2b}) can be qualified as containing two manner clauses (at least according to \textcite[621]{nichols11} herself). It makes use of a Simultaneous Converb in \textit{-až}, effectively not distinguishing it from a temporal adjunct clause. Data on Chechen is lacking.


\begin{exe}
	\ex\label{man-ex2}
    	Ingush
    \begin{xlist}
	
		
			\ex\label{man-ex2a}
			
			\gll fɨ	d-e-ž	l\={a}tt	ħo	{{\normalfont[} bežan}	\textbf{muo}	{dʕa-ett-\={a}? {\normalfont]}}.\\
			what	{\D}-do-{\Simul}	stand	{\Ssg}	cow	\textbf{like}	{\Pv}-stand\_up-{\Ante}\\
			\trans `What are you doing standing there like an idiot?’ (lit. `like a cow')\\
			\hfill (\cite[514]{nichols11})
		
		
		
			\ex\label{man-ex2b}
			%large gap between \gll and \trans in pdf
			\gll {{\normalfont[} \u{g}ālmaqa} čei yštta {\textbf{lūs-až} {\normalfont]}} lātt-ar, {{\normalfont[} caʕ} {{\normalfont[} k'æd-j-el-ča {\normalfont]}} šo-lla\u{g}-j-ar dʕa-t'ɨ=a {\textbf{j-uod-až} {\normalfont]}}.\\
			Kalmyk.{\Gen} tea thus \textbf{stir-{\Simul}} stand-{\Imprf} one tired-{\F}.{\Sg}-{\Intr}-{\Ante} two-{\Ord}-{\F}.{\Sg}-{\Nmlz} {\Pv}-on={\Add} \textbf{{\F}.{\Sg}-go-{\Simul}}\\
			\trans `They would stand stirring the Kalmyk tea, one woman replacing another when she got tired.'
			\hfill (\cite[621]{nichols11})
		
	\end{xlist}
\end{exe}


Modern Georgian, however, can form an exactly similar type of manner clause as in Tsova-Tush Example (\ref{man-ex1}), with a conjunction based on the interrogative (see Example (\ref{man-ex3})).


	\begin{exe}
		\ex\label{man-ex3}
		Standard Modern Georgian
        
		\gll q'velaperi	ise=a,	{{\normalfont[} \textbf{rogor-c}}	unda	{iq'os {\normalfont]}}.   \\
		everything	so.{\Dist}={\Cop}	\textbf{how-{\Rel}}	it\_should	it\_would\_be	\\
		\trans `Everything is as it should be.’ 
		\hfill (GNC: Ch. Amirejibi)
	\end{exe}


Another possibility to form Tsova-Tush manner clauses, is with a Verbal Noun\is{Verbal Noun} in the Instrumental case (see Examle (\ref{man-ex4})),\is{Instrumental case} which also exactly mirrors a parallel Modern Georgian construction (Example (\ref{man-ex5})).


	\begin{exe}
		\ex\label{man-ex4}
		\gll wux	lat'-o-d-∅-mak'-er	{{\normalfont[} badr-e-n}	t'q'uiħa	{eqq-\textbf{r-e-v} {\normalfont]}}.  \\
		what	help-{\Npst}-{\D}-{\Tr}-{\Pot}-{\Imprf}	child-{\Obl}-{\Dat}	behind	jump-\textbf{{\Vn}-{\Obl}-{\Ins}}    \\
		\trans `How could she have helped by jumping after the child?’	
		\hfill (E060-26)
	\end{exe}



	\begin{exe}
		\ex\label{man-ex5}
		Standard Modern Georgian
        
		\gll nišn-is	{{\normalfont[} mo-g-\textbf{eb-it} {\normalfont]}}	miaʒaxa	nik'o-m.	\\
		sign-{\Gen}	{\Pv}-retort-\textbf{{\Vn}-{\Ins}}	s/he\_shouted	Niko-{\Erg}	\\
		\trans `Niko shouted mockingly’
		\hfill (GNC: E. Akhvlediani)
	\end{exe}



\subsection{Summary}\label{adjunctsum}

For an overview of all strategies of forming adjunct clauses, consider \tabref{adjunct-table1}.


\begin{table}
\fittable{
	\begin{tabular}{llllll}
    \lsptoprule
		Type & {Chechen-Ingush}  & {Tsova-Tush} & {Mod. Georgian} & {Old Georgian} \\
		& & & {Tush Georgian} & \\
		\midrule
		Temporal & Converb & Converb & (Inflected {\Vn}) & \\
		& {\Vn} + Postpos. & {\Vn} + Postpos. & {\Vn} + Postpos.  &  \\
		& & Conjunction & Conjunction & Conjunction  \\	
		& & & Conjunction \textit{ro(m)} & \\
		\midrule
		Purpose & Infinitive & Infinitive &  (Inflected {\Ptcp}) & (Inflected {\Vn}) \\
		& & \textit{me} + {\Sbjv} & \textit{ro(m)} + {\Sbjv} & \\
		\midrule
		Conditional & Conditional verb & Conditional verb &   &  \\
		& & & Conjunction & Conjunction  \\
		& & & Conjunction \textit{ro(m)} & \\
		\midrule
		Causal & Converb, {\Vn} & Participle, {\Vn} &   &  \\
		& Conjunction & Conjunction & Conjunction & Conjunction \\
		\midrule
		Manner & Converb &  &  & \\
         &  & Conjunction & Conjunction & \\
		& & Inflected {\Vn} & Inflected {\Vn}  &  \\
        \lspbottomrule
	\end{tabular}
	}
	\caption{Adjunct clause types}
	\label{adjunct-table1}
\end{table}

The main intermediate conclusion to draw from \tabref{adjunct-table1}, is that whenever Tsova-Tush features a strategy involving a conjunction, Modern Standard and Tush Georgian can express the same clause type with a conjunction also. That is, in temporal, causal and manner clauses, both Tsova-Tush on the one hand and Modern Standard Georgian and Tush Georgian on the other feature conjunctions. In temporal clauses and causal clauses, Chechen, Ingush, and Old Georgian also show the possibility to use conjunctions, but the comparison is less striking, because of the similarity in form between the Tsova-Tush conjunction itself and its contemporary Georgian counterparts. To see this similarity, compare \tabref{adjunct-table2}, where the similar derivations in Tsova-Tush and Georgian are shown. To indicate the suppletion in the Tsova-Tush interrogatives ‘what’ and ‘who’, both the Nominative and an Oblique case are given. 

\begin{table}
	\begin{tabular}{lllll}
    \lsptoprule
		&  \multicolumn{2}{c}{{Tsova-Tush}} & \multicolumn{2}{c}{{Georgian}}\\\cmidrule(lr){2-3}\cmidrule(lr){4-5}
		&  {Interrogative} & {Relative} & {Interrogative} & {Relative} \\
		\midrule
		‘what’ ({\Nom}) & \textit{wu\textsuperscript{n}} & \textit{wun-e} & \textit{ra} & \textit{ra-c} \\
		‘what’ ({\Ins}) & \textit{st'e-v} & \textit{st'e-v-a} & \textit{r-it} & \textit{r-ita-c} \\
		‘who’ ({\Nom}) & \textit{me\textsuperscript{n}} & \textit{men-e/a} & \textit{vin} & \textit{vin-c} \\
		‘who’ ({\Dat}) & \textit{ħan-n} & \textit{ħan-n-a} & \textit{vi-s} & \textit{vi-sa-c} \\
		‘which’ & \textit{menux} & \textit{menux-a} & \textit{romel-i} & \textit{romel-i-c} \\
		‘when’ & \textit{maca\textsuperscript{n}} & \textit{macn-e} & \textit{rodis} & \textit{rodesa-c} \\
		‘how’ & \textit{moħ} & \textit{moħ-e} & \textit{rogor} & \textit{rogor-c} \\
		‘how much’ & \textit{meɬ} & \textit{meɬ-e} & \textit{ramden-i} & \textit{ramden-i-c} \\
		‘where’ & \textit{mičeħ} & \textit{mičħ-e} & \textit{sad} & \textit{sada-c} \\
		\midrule
		(additive particle) & \multicolumn{2}{c}{\textit{=e, =a}} & \multicolumn{2}{c}{\textit{=c}}\\
        \lspbottomrule
	\end{tabular}
	\caption{Derived relative pronouns and subordinating conjunctions in Tsova-Tush and Georgian}
	\label{adjunct-table2}
\end{table}

In addition to the table, compare again the Tsova-Tush and Georgian constructions for ‘because’. Even though Chechen-Ingush and Old Georgian also feature conjunctions in the clause type, the construction is not parallel to the Tsova\hyp Tush\slash Georgian one (Chechen \textit{d\={e}lla}, Old Georgian \textit{rametu}).

\ea 
\begin{minipage}[t]{.5\linewidth}
	Tsova-Tush\\
	\gll oqui-n-daɬa me\\
		 \Dist.{\Obl}-{\Dat}-because\_of {\Subord}\\
	\glt ‘because’
\end{minipage}%
\begin{minipage}[t]{.5\linewidth}
	Georgian\\
	\gll im-it'om rom\\
		 \Dist.{\Obl}-because\_of {\Subord}\\
	\glt ‘because’
 \end{minipage}
\z

On the basis of these parallels alone, one could hypothesise that Tsova-Tush borrowed from Standard Modern Georgian or from Tush Georgian:\pagebreak

\begin{itemize}
	\item The possibility to form finite adjunct clauses
	\item The possibility to form temporal, causal, and manner clauses using a conjunction 
	\item The template for forming adverbial conjunctions by adding the additive particle to an interrogative base
	\item The template for forming a conjunction ‘because’
	\item The possibility to form purpose clauses using a general subordinator and a finite form in the Subjunctive.
\end{itemize}

Finally, it has to be observed that the use of the above-mentioned Georgian general subordinating conjunction \textit{ro(m)} has not been copied in Tsova-Tush temporal or conditional clauses. In the context of adjunct subordination, we only observe its Tsova-Tush parallel \textit{me} in purpose clauses.

%proofread stop

\section{Complement clauses}\label{complement}\is{Complement clauses}\is{Georgian influence!Syntactic}

Most Tsova-Tush complement clauses are formed with Infinitives or Verbal Nouns on the one hand, or feature a finite verb in combination with the general subordinator \textit{me}, on the other. 



\subsection{Phasal verbs}\label{phas}\is{Phasal verbs}

Tsova-Tush forms complement clauses to the phasal verb ‘begin’ using the Infinitive\is{Infinitive} form of the verb. As is clear from Examples (\ref{phas-ex1}) and (\ref{phas-ex2}), the verb `begin' agrees in gender with its own intransitive subject, not with the Nominative argument in the subordinate clause.


	\begin{exe}
		\ex\label{phas-ex1}
		\gll {{\normalfont[} qor-i}	{d-aq'-\textbf{a\textsuperscript{n}} {\normalfont]}}	v-ol-v-ejl-nor. \\
		apple-{\Pl}	{\B}.{\Pl}-eat.{\Ipfv}-\textbf{{\Inf}}	{\M}.{\Sg}-begin-{\M}.{\Sg}.-{\Intr}-{\Nw}.{\Rem} \\
		\trans `He began to eat apples.’	
		\hfill (E058-66)

		\ex\label{phas-ex2}
		\gll je	ħal\u{o}	b-ol-b-ajl-n-atx	{{\normalfont[} načx}	{teg-j-∅-\textbf{a\textsuperscript{n}} {\normalfont]}}.   \\ 
		and	{\Pv}	{\M}.{\Pl}-begin-{\M}.{\Pl}-{\Intr}-{\Aor}-{\Fpl}.{\Erg}	cheese	make.{\Ipfv}-{\J}-{\Tr}-\textbf{{\Inf}}   \\
		\trans `And we began to make cheese.’
		\hfill (EK009-4.2)
	\end{exe}


Complement clauses to the verb \textit{maq'\u{o} b-aɬar} ‘stop’ use the Verbal Noun in the Dative case (see Examples (\ref{phas-ex3}) and (\ref{phas-ex4})).\is{Verbal Noun}\is{Dative case} This parallels the Dative case of the nominal object of the same verb in (\ref{phas-ex5}). The Dative case can be explained by the fact that this verb consists of a light verb construction (see \sectref{lvconstr}), and the Nominative argument slot is already filled by the lexical part of the verb \textit{maq'\u{o}} ‘freedom’ (B gender). The semantics of \textit{maq'\u{o} ba{ɬ}ar} developed from `give freedom' to `leave, let' (which is attested) to `stop'.


	\begin{exe}
		\ex\label{phas-ex3}
		\gll zorajš\u{\i}	k'aʒik'	xan-e	{{\normalfont[} čxindur}	{d-∅-\textbf{ar-e-n} {\normalfont]}}  maq'\u{o}	b-ajɬ-nor	mehir-e-s.	\\	
		very	little	time-{\Obl}({\Ess})	stocking	{\D}-do-\textbf{{\Vn}-{\Obl}-{\Dat}}	freedom	{\B}.{\Sg}-give-{\Nw}.{\Rem}	Mehir-obl-erg		\\
		\trans `She (apparently) stopped knitting for a while.’	
		\hfill (MM118-2.22)

		\ex\label{phas-ex4}
		\gll de\u{g}-e-v	{{\normalfont[} kot't'-d-al-\textbf{r-e-n} {\normalfont]}}	c'ʕajrkoⁿ	maq'\u{o}	b-aɬ-i\textsuperscript{n}.\\
		body-{\Obl}-{\Erg}	breathe-D-{\Intr}-\textbf{{\Vn}-{\Obl}-{\Dat}}	suddenly	freedom	{\B}.{\Sg}-give-{\Aor}   \\
		\trans `The body suddenly stopped breathing.’	
		\hfill (MM421-1.31)

		\ex\label{phas-ex5}
		\gll oqar	vašbi-ciⁿ	buħ-\textbf{e-n}	maq'	b-aɬ-iⁿ.  \\
		{\Dist}.{\Pl}	{\Recp}-{\Apudess}	fight-\textbf{{\Obl}-{\Dat}}	freedom	{\B}.{\Sg}-give-{\Aor}   \\
		\trans `They stopped fighting [lit. the fight] with each other.’	
		\hfill (Elicit.)
	\end{exe}


In both Chechen-Ingush and Georgian, using  non-finite clauses is the main strategy to form complement clauses to phasal verbs. In Example (\ref{phas-ex6}), Ingush uses an Infinitive  as complement to the matrix verb `begin' and a Simultaneous Converb as the complement to `stop'. Georgian (Example (\ref{phas-ex7})) uses Verbal Noun constructions for both phasal clauses.

\begin{exe}
	\ex Ingush\label{phas-ex6} \begin{xlist}
		
		
			\ex\label{phas-ex6a}
			\gll č\={a}rx	c'æxx\={a}	{{\normalfont[} qest-\textbf{a} {\normalfont]}}	j-uol-a-j-al-ar.	\\
			wheel	suddenly	turn-\textbf{{\Inf}}	{\J}-begin-{\Inf}-J-{\Intr}-{\Pst}	\\
			\trans `The wheel suddenly started turning.’ 
			\hfill (\cite[552]{nichols11})
		
		
		
			\ex\label{phas-ex6b}
			\gll ber	{{\normalfont[} cɨ}	{d-elx-\textbf{až} {\normalfont]}}	sac-ar. \\
			child	{\Neg}	{\D}-cry-\textbf{{\Simul}}	stop-{\Pst}	\\
			\trans `The baby stopped crying.’
			\hfill (\cite[581]{nichols11})
		
	\end{xlist}
	\pagebreak	
	\ex Standard Modern Georgian\label{phas-ex7} \begin{xlist}
		
		
			\ex\label{phas-ex7a}
			\gll man	šec'q'vit'a	{{\normalfont[} mic'a-ze}	{muša-ob-\textbf{a} {\normalfont]}}.    \\
			{\Tsg}.{\Erg}	s/he\_stopped	earth-{\Super}	work-{\Tm}-\textbf{{\Vn}}	\\
			\trans `S/he stopped working the land.’ 
			\hfill (Elicit.)
		
		
		
			\ex\label{phas-ex7b}
			\gll daircxvina	da	daic'q'o	{{\normalfont[} tit-eb-is}   ertmanet-ši	{xlart-v-\textbf{a} {\normalfont]}}.	\\
			s/he\_felt\_ashamed	and	s/he\_started	finger-{\Pl}-{\Gen}	{\Recp}-{\In}	entangle-{\Tm}-\textbf{{\Vn}}	\\
			\trans `S/he felt ashamed and started to interlace their fingers.’ \\
			\hfill (GNC: Ch. Amirejibi)
		
	\end{xlist}
\end{exe}

Old Georgian formed complement clauses to phasal verbs (as well as to desiderative and manipulative verbs) with a Verbal Noun in the Adverbial case (\cite{kobaidzevamling}). See \sectref{loanverb} on the accommodation of borrowed verbs using this same Old Georgian form. 


\subsection{Desiderative verbs} \label{desid}\is{Desiderative verbs}

In Tsova-Tush, complement clauses to desiderative verbs, too, can be formed using an Infinitive, as long as the matrix verb and the subordinate verb share the same subject (see Examples (\ref{desid-ex1}) and (\ref{desid-ex2})).


	\begin{exe}
		\ex\label{desid-ex1}
		\gll nan-e-n	{{\normalfont[} majq\u{\i}}	{j-aq'-\textbf{a\textsuperscript{n}} {\normalfont]}}	leɁ-\u{e}.    \\
		mother-{\Obl}-{\Dat}	bread	{\J}-eat-\textbf{{\Inf}}	wish-{\Npst}    \\
		\trans `Mother wants to eat.’
		\hfill (\cite[202]{holiskygagua})

		\ex\label{desid-ex2}
		\gll txo-n	{{\normalfont[} ʕam-d-∅-\textbf{a\textsuperscript{n}} {\normalfont]}}	d-ec'=in. \\
		{\Fpl}.{\Excl}-{\Dat}	learn-{\D}-{\Tr}-\textbf{{\Inf}}	{\D}-want={\Quot} \\
		\trans `{``}We want to study,'' (they say).'
		\hfill (E019-165)
	\end{exe}


In Chechen and Ingush, the same strategy can be observed, as illustrated by Example (\ref{desid-ex3}).


	\begin{exe}
		\ex\label{desid-ex3}
		Ingush\\
		\gll baq'onc\={a}	{{\normalfont[ [} sie}	{mal=\={u} {\normalfont]}}	{\textbf{x\={a}} {\normalfont]}}	biezam	b-ɨ	sɨ. \\
		really	{\Fsg}.{\Refl}	who={\M}.{\Sg}.be	\textbf{know.{\Inf}}	like	{\B}.{\Sg}-be.{\Prs}	{\Fsg}.{\Gen}	\\
		\trans `I would like to know who I really am.’ 
		\hfill (\cite[565]{nichols11})
	\end{exe}




A second strategy in Tsova-Tush involves the general subordinating conjunction \textit{me}, to introduce a finite complement clause to the same matrix verbs. This strategy is used when the subject of the subordinate clause differs from that of the matrix clause (see Examples (\ref{desid-ex4}) and (\ref{desid-ex5})).


	\begin{exe}
		\ex\label{desid-ex4}
		\gll leɁ	soⁿ	{{\normalfont[} \textbf{me}}	qe-ču-š-n=a	{d-ag-e-l {\normalfont]}.}    \\
		wish	{\Fsg}.{\Dat}	\textbf{{\Subord}}	other-{\Obl}-{\Pl}-{\Dat}={\Add}	{\D}-see-{\Npst}-{\Sbjv} \\
		\trans `I want others to see it too.’	
		\hfill (E29-11)
	\end{exe}



	\begin{exe}
		\ex\label{desid-ex5}
		\gll leɁ-er	{{\normalfont[} \textbf{me}}	sakartvelo	damouk'idebel	{gaqdad-j-∅-ora-l\u{o} {\normalfont]}},  \\
		wish-{\Imprf}	\textbf{{\Subord}}	Georgia	independent	become-{\J}-{\Tr}-{\Imprf}-{\Sbjv}   \\
		\trans `They wanted Georgia to become independent,’	
		\hfill (E146-4)
	\end{exe}


In these different-subject complement clauses to desiderative verbs, Chechen also requires a verb in the Subjunctive\is{Subjunctive} (Example \ref{desid-ex8}), while Ingush would instead have a non-finite verb, as in (\ref{desid-ex6}).


	\begin{exe}
		\ex\label{desid-ex8}
		Chechen
        
		\gll cun-na læɁ-a as saj nan-na \u{g}uo d-∅-ojla. \\
		{\Tsg}-{\Dat} want-{\Prs} {\Fsg}.{\Erg} {\Fsg}.{\Poss}.{\Refl} mother-{\Dat} help \textbf{{\D}-do-{\Sbjv}}\\
		\trans `S/he wants that I help my mother.’ 
		\hfill (\cite[63]{nichols94Che})
	\end{exe}



	\begin{exe}
		\ex\label{desid-ex6}
		Ingush
        
		\gll sɨ=Ɂa	{{\normalfont[} uquo}	{dieš-\textbf{ar} {\normalfont]}}	lou-ra. \\
		{\Fsg}.{\Gen}={\Add}	{\Tsg}.{\Erg}	study-\textbf{{\Vn}}	want-{\Imprf}	\\
		\trans `I too want him to study.’ 
		\hfill (\cite{nichols11})
	\end{exe}



Georgian, on the other hand, makes use of the general subordinator \textit{rom} to create complement clauses to desiderative verbs. A simple finite Optative\is{Optative} verb without any overt subordination can also be used in same-subject clauses (see Example (\ref{desid-ex7}a,b)).

\begin{exe}
	\ex\label{desid-ex7} 
    Standard Modern Georgian
    \begin{xlist}
		
		
			\ex\label{desid-ex7a}
			\gll me	ar	minda	{{\normalfont[} \textbf{rom}}	{dagcinodnen {\normalfont]}}.    \\
			{\Fsg}	{\Neg}	I\_want	\textbf{{\Subord}}	they\_would\_mock\_you	\\
			\trans `I don’t want them to mock you.’ 
			\hfill (GNC: R. Mishveladze)
		
		
		
			\ex\label{desid-ex7b}
			\gll ʒalian	minda,	{{\normalfont[} mand}	\textbf{rom}	{viq'o {\normalfont]}}.   \\
			very	I\_want	there.{\Med}	\textbf{{\Subord}}	I\_would\_be	\\
			\trans `I really want to be there with you.’ 
			\hfill (GNC: Ch. Amirejibi)
		
	\end{xlist}
\end{exe}

It has to be noted that Georgian is able to form complement clauses of this type with a verbal noun also.


\subsection{Cognitive verbs} \label{cognitive}\is{Cognitive verbs}

In Tsova-Tush, in order to express a complement to the verb \textit{xeɁar} `know that' (as opposed to `know how to' and `know + interrogative'), a finite clause introduced by the subordinating conjunction \textit{me} is used (see Examples (\ref{cog-ex1}) and (\ref{cog-ex2})).


	\begin{exe}
		\ex\label{cog-ex1}
		\gll xeɁ-ra-l	{{\normalfont[} \textbf{me}},	naq'a	badr-i	{bʕar-\u{g}-or {\normalfont]}}.    \\
		know-{\Imprf}-{\Sbjv}	\textbf{{\Subord}}	road.{\Obl}({\Ess})	child-{\Pl}	meet-{\Lv}-{\Imprf}    \\
		\trans `He (apparently) knew that children would meet him on the road.’	
		\hfill (E031-2)
	\end{exe}



	\begin{exe}
		\ex\label{cog-ex2}
		\gll ħoⁿ	xeɁ	{{\normalfont[} \textbf{me},}	{… {\normalfont]}}   \\
		{\Ssg}.{\Dat}	know	\textbf{{\Subord}}	\\
		\trans `You know that [in the year 1659, the Tush, Pshav, Khevsurs and the Kakhetian nobles made a treaty with the Cholokashvili's].
		\hfill (E104-3)
	\end{exe}


The verb \textit{mottar} ‘think, deem’ requires a similar pattern (Example (\ref{cog-ex3})).


	\begin{exe}
		\ex\label{cog-ex3}
		\gll o	mott	soⁿ	{{\normalfont[} \textbf{me}}	\u{g}az-iš	moq	{b-∅-o {\normalfont]}}.   \\
		{\Dist}	think	{\Fsg}.{\Dat}	\textbf{{\Subord}}	good-{\Adv}	song	{\B}.{\Sg}-do-{\Npst}    \\
		\trans `S/he, I think, sings well.’	
		\hfill (BH023-15.1)
	\end{exe}


The verb \textit{dak'lavar} `think' can take a clausal complement indicated with a quotative particle (Examples (\ref{cog-ex4}) and (\ref{cog-ex5})). The conjunction \textit{me} is optional, exactly mirroring complement clauses to speech verbs. 


	\begin{exe}
		\ex\label{cog-ex4}
		\gll dak'lav-er	st'ak'	{{\normalfont[} šakar}	{d-a=\textbf{en\u{o}} {\normalfont]}}	{{\normalfont[} b-aq'-oš {\normalfont]}}.   \\
		think.{\Pfv}-{\Imprf}	man	sugar	{\D}-be=\textbf{{\Quot}}	{\B}.{\Sg}-eat.{\Ipfv}-{\Simul} \\
		\trans `A man could think that it was sugar, when he ate it.’
		\hfill (E015-53)
	\end{exe}



	\begin{exe}
		\ex\label{cog-ex5}
		\gll dak'lav-iⁿ	{{\normalfont[} \textbf{me}}	ese-\u{g}=a	{d-axk'-o-d-∅=\textbf{uin} {\normalfont]}}.    \\
		think.{\Pfv}-{\Aor}	\textbf{{\Subord}}	here-{\Trans}=down	{\D}-take.{\Anim}-{\Npst}-{\D}-{\Tr}=\textbf{{\Quot}} \\
		\trans `He thought that they would take them (sheep) away this way.’	\hfill (E147-80)
	\end{exe}
	

In these types of complement clauses, Chechen and Ingush use an asyndetic finite verb (Subjunctive or Indicative). The Chechen verb \textit{xæɁa} in (\ref{cog-ex6}) is the cognate to Tsova-Tush \textit{xeɁ} in (\ref{cog-ex1}--\ref{cog-ex2}), and the verb \textit{mott} in the Ingush Example (\ref{cog-ex7}) is cognate to the Tsova-Tush verb \textit{mott} in (\ref{cog-ex3}).


	\begin{exe}
		\ex\label{cog-ex6}
		Chechen
        
		\gll s\={u}na	xæɁ-a,	{{\normalfont[} m\={u}s\={a}}	hinca=Ɂa	tx\={ø}-gaħ	v-u-jla      {\normalfont/}	{v-u {\normalfont]}}. \\
		{\Fsg}.{\Dat}	know-{\Npst}	Musa	now={\Add}	{\Fpl}-{\Adess}	{\M}.{\Sg}-be-{\Sbjv} /	{\M}.{\Sg}-be	\\
		\trans `I know Musa is still at our place.’ 
		\hfill (\cite[63]{nichols94Che})
	\end{exe}



	\begin{exe}
		\ex\label{cog-ex7}
		Ingush
        
		\gll {{\normalfont[} cħa}	kinaškʲa	j\={a}zd-ež	{v-oall {\normalfont]}}	mott	suona	ɨz.  \\
		one	book	write-{\Simul}	{\M}.{\Sg}-{\Lv}.{\Prog}	think	{\Fsg}.{\Dat}	{\Tsg}	    \\
		\trans `I think he's writing some kind of book.’ 
		\hfill (\cite[575]{nichols11})
	\end{exe}


In Standard Modern Georgian, as well as in the Tush dialect, these types of complement clauses are mainly expressed using the subordinator \textit{ro(m)} (see Examples (\ref{cog-ex8}) and (\ref{cog-ex9})).

\begin{exe}
	\ex\label{cog-ex8}
    Standard Modern Georgian
    \begin{xlist}
		
		
			\ex\label{cog-ex8a}
			\gll me	vici,	{{\normalfont[} \textbf{rom}}	srulkmnil-i	adamian-i	ar	{arsebobs {\normalfont]}},   \\
			{\Fsg}	I\_know	\textbf{{\Subord}}	perfected-{\Agr}	person-{\Nom}	{\Neg}	s/he\_exists	\\
			\trans `I know a perfect human does not exist.’ 
			\hfill (GNC: Ch. Amirejibi)
		
		
		
			\ex\label{cog-ex8b}
			\gll rat'om	ar	pikrobt,	{{\normalfont[} \textbf{rom}},	ikneb,	p'irikit	{moxdes {\normalfont]}}? \\	
			why	{\Neg}	y’all\_think	\textbf{{\Subord}}	perhaps	conversely	it\_would\_have\_happened	\\
			\trans `Why don’t you think that, perhaps, the opposite has happened?’ \\
			\hfill (GNC: R. Mishveladze)
		
	\end{xlist}
\end{exe}\il{Tush Georgian}

	\begin{exe}
		\ex\label{cog-ex9}
		Tush Georgian
        
		\gll mašin	ician,	{{\normalfont[} \textbf{ro}}	ar	{varga=o {\normalfont]}},    \\
		then	they\_know	\textbf{{\Subord}}	{\Neg}	it\_is\_of\_value={\Quot}	\\
		\trans `Then they know it isn’t any good.’	
		\hfill (TT058-1.174)
	\end{exe}




\subsection{Perception verbs} \label{perception}\is{Perception verbs}

Tsova-Tush perception verbs, too, take a complement clause introduced by the conjunction \textit{me} (see Examples (\ref{perc-ex1}\footnote{Original orthography of (\ref{perc-ex1}): Dagi cruen, me cḥain uirwas dakardie itt baḥ o\'{k}rui.}--\ref{perc-ex3})).


	\begin{exe}
		\ex\label{perc-ex1}
		\gll d-ag-iⁿ	cru-e-n,	{{\normalfont[} \textbf{me}}	cħajn	uirv-a-s	dak'ar-d-i-eⁿ	it't'	baħ	{okrui-\textsuperscript{n} {\normalfont]}}.  \\
		{\D}-see-{\Aor}	rogue-{\Obl}-{\Dat}	\textbf{{\Subord}}	one.{\Obl}	Jew-{\Obl}-{\Erg}	count-{\D}-{\Tr}-{\Aor}	ten	100\_pieces	gold-{\Gen}    \\
		\trans `The rogue saw, that a Jewish man was counting a thousand gold pieces.’ \\
		\hfill (AS008-11.3)
	\end{exe}



	\begin{exe}
		\ex\label{perc-ex2}
		\gll seⁿ	bader,	g-u	soⁿ	{{\normalfont[} \textbf{me}}	\u{g}azeⁿ	st'ak'	{v-a-ħ {\normalfont]}}.  \\
		{\Fsg}.{\Gen}	child	see-{\Npst}	{\Fsg}.{\Dat}	\textbf{{\Subord}}	good	man	{\M}.{\Sg}-be-{\Ssg} \\
		\trans `Son, I see that you are a nice fellow,’	
		\hfill (WS001-10.11)
	\end{exe}



	\begin{exe}
		\ex\label{perc-ex3}
		\gll e	mimin-e-n	xac'-eⁿ	{{\normalfont[} \textbf{me}}	cok'l-e-v	ise	lev-d-∅-or  {{\normalfont[} me}	so-x	sc'rap	{d-a=en\u{o}. {\normalfont] ]}}    \\
		{\Prox}	hawk-{\Obl}-{\Dat}	hear-{\Aor}	\textbf{{\Subord}}	fox-{\Obl}-{\Erg}	here    say.{\Ipfv}-{\D}-{\Tr}-{\Imprf}	{\Subord}	{\Fsg}-{\Cont}	fast	{\D}-be={\Quot}	\\
		\trans `And the falcon heard that the fox was saying ``he is faster than me.{''}' \\
		\hfill (BH085-13.1)
	\end{exe}


Chechen uses an asyndetic finite clause (\ref{perc-ex4}), similar to complement clauses of cognitive verbs (see \sectref{cognitive} above), while Ingush has innovated and extended the quotative particle \textit{ænna} (compare Tsova-Tush \textit{=en\u{o}} in Example (\ref{cog-ex4})) to also function as a more general subordinator, as in (\ref{perc-ex5}).


	\begin{exe}
		\ex\label{perc-ex4}
		Chechen
        
		\gll s\={u}na	xezza	m\={u}s\={a}-gara	{{\normalfont[} aħmada-na}	š\={a}	bʕarg-v-aⁿ  ca	{vieza {\normalfont]}}	b-\={o}x-uš. \\
		{\Fsg}.{\Dat}	heard	Musa-{\Abl}	Ahmed-{\Dat}	{\Refl}	eye-{\M}.{\Sg}-see    {\Neg}	like	{\B}.{\Sg}-say-{\Simul}	\\
		\trans `I heard from Musai that Ahmed hates him.’ 
		\hfill (\cite[62]{nichols94Che})
	\end{exe}



	\begin{exe}
		\ex\label{perc-ex5}
		Ingush
        
		\gll suona	{{\normalfont[} m\={u}s\={a}}	ʕa-v-iež-\={a}v	{\textbf{ænna} {\normalfont]}}	xaz-ar.	   \\
		{\Fsg}.{\Dat}	Musa	down-{\M}.{\Sg}-fall-{\Nw}.{\M}.{\Sg}	\textbf{{\Subord}}	hear-{\Pst}	\\
		\trans `I heard that Musa fell down. I heard that Musa had fallen down (and he had).’ 
		\hfill (\cite[544]{nichols11})
	\end{exe}


Just like with cognitive verbs, Modern Georgian uses the conjunction \textit{rom} to introduce complement clauses to perception verbs (see Examples (\ref{perc-ex6}a,b)).\pagebreak

\begin{exe}
	\ex\label{perc-ex6} 
    Standard Modern Georgian
    \begin{xlist}
		
		
			\ex\label{perc-ex6a}
			\gll da	naxa,	{{\normalfont[} \textbf{rom}}	sac'ol-ze	čacmul-i	{c'evs {\normalfont]}}. \\
			and	s/he\_saw	\textbf{{\Subord}}	bed-{\Super}	dressed-{\Nom}	s/he\_lies	\\
			\trans `And s/he saw that s/he was lying on the bed with his/her clothes on.’
			\hfill (GNC: B. Akunin)
		
		
		
			\ex\label{perc-ex6b}
			\gll xom	gaigone,	{{\normalfont[} \textbf{rom}}	čem-i	kal-i	čem-tan	{rčeba {\normalfont]}}.    \\
			{\Q}	you\_heard	\textbf{{\Subord}}	{\Fsg}.{\Poss}-{\Agr}	woman-{\Nom}	{\Fsg}.{\Obl}-{\Apud}	s/he\_remains	\\
			\trans `You have heard, right, that my wife stays with me?’ \\ 
			\hfill (GNC: A. Kazbegi)
		
	\end{xlist}
\end{exe}



\subsection{Summary}

Consider \tabref{comp-table1} for an overview of all strategies of forming complement clauses.

\begin{table}
\fittable{%
	\begin{tabular}{llllll}
    \lsptoprule
		Matrix verb & {Chechen-Ingush}  & {Tsova-Tush} & {Mod. Georgian} & {Old Georgian} \\
		& & & {Tush Georgian} & \\
		\midrule
		Phasal & Infinitive, Converb & Infinitive &  Verbal noun & Inflected {\Vn}    \\\addlinespace
		
		Desiderative & Infinitive & Infinitive  & Verbal noun & Inflected {\Vn}    \\
		(Same subj.) & & Asynd. finite & Asynd. finite & \\
		\addlinespace
		Desiderative & Verbal noun & & Verbal noun & Inflected {\Vn} \\
		(Diff. subj.) & & \textit{me} + {\Sbjv} & \textit{ro(m)} + {\Sbjv} & \\
		\addlinespace
		Cognitive & Asynd. finite & Conjunction \textit{me} & Conjunction \textit{ro(m)} &    \\ 
		\addlinespace
		Perception & Asynd. finite & Conjunction \textit{me} & Conjunction \textit{ro(m)} & \\
		& Particle \textit{ænna} & & & \\
        \lspbottomrule
	\end{tabular}}
	\caption{Complement clause strategies}
	\label{comp-table1}
\end{table}

It is clear from \tabref{comp-table1}, that Tsova-Tush complement clauses feature the conjunction \textit{me} whenever the same clause type in Georgian would feature the conjunction \textit{ro(m)}, i.e. in complements to cognitive verbs and perception verbs, as well as to different-subject desiderative verbs.


\section{Coordination and clause chaining} \label{coord}\is{Clause chaining}\is{Sequential}\is{Georgian influence!Syntactic}


Tsova-Tush features coordinating conjunctions to conjoin both phrases and clauses in addition to special verb forms that are used in clause chaining constructions.

\subsection{Clause chaining} \label{chaining}


In order to convey a sequence of same-subject events, Tsova-Tush employs the morph \textit{-e}. This morph is affixed directly after the Non-Past/Aorist/Non-Witnessed Aorist stem, before the morph \textit{-ra} and before any person marking. See \tabref{table-sequential}, where these verb forms, labeled Sequential in this work, are given side by side with their non-sequential, default counterparts. Only verb forms that are robustly attested in the corpus are given. Note that all phonological processes apply, most importantly word-final vowel deletion (see \sectref{processes}).

\begin{table}
	\small
	\tabcolsep=.75\tabcolsep
	\begin{tabular}{ll ll ll}
    \lsptoprule
		`press' & Pers. & \multicolumn{2}{c}{{Sequential}} & \multicolumn{2}{c}{{Default}} \\\cmidrule(lr){3-4}\cmidrule(lr){5-6}
		& & {morphs} & {surface} & {morphs} & {surface}\\
        \midrule
		Present & 1st  & \textit{ħač'q'-o-e-as} & \textit{ħač'q'oes} & \textit{ħač'q'-o-as} & \textit{ħač'q'os} \\
		& 3rd  & \textit{ħač'q'-o-e} & \textit{ħač'q'o} & \textit{ħač'q'-o} & \textit{ħač'q'\u{o}} \\
		Imperfect & 1st  & \textit{ħač'q'-o-e-ra-as} & \textit{ħač'q'oeras} & \textit{ħač'q'-o-ra-as} & \textit{ħač'q'oras} \\
		& 3rd  & \textit{ħač'q'-o-e-ra} & \textit{ħač'q'oer} & \textit{ħač'q'-o-ra} & \textit{ħač'q'or} \\
		Aorist & 1st  & \textit{ħač'q'-in-e-as} & \textit{ħač'q'ines} & \textit{ħač'q'-in-as} & \textit{ħeč'q'nas} \\
		& 3rd  &\textit{ħač'q'-in-e} & \textit{ħač'q'in} & \textit{ħač'q'-in} & \textit{ħač'q'iⁿ} \\
		NW Remote  & 1st  & \textit{ħač'q'-ino-e-ra-as} & \textit{ħeč'q'noeras} & \textit{ħač'q'-ino-ra-as} & \textit{ħeč'q'noras} \\
		\quad Past & 3rd  & \textit{ħač'q'-ino-e-ra} & \textit{ħeč'q'noer} & \textit{ħač'q'-ino-ra} & \textit{ħeč'q'nor} \\
		Imperative & \textsc{sg} & \textit{ħač'q'-a-e} & \textit{ħač'q'a} & \textit{ħač'q'-a} & \textit{ħač'q'} \\
		& \textsc{pl} & \textit{ħač'q'-a-e-t} & \textit{ħač'q'aet} & \textit{ħač'q'-a-t} & \textit{ħač'q'at} \\
		\lspbottomrule
	\end{tabular}
	\caption{Most frequent Tsova-Tush Sequential verb forms}
	\label{table-sequential}
\end{table}

As illustrated in \tabref{table-sequential}, these verb forms are finite, since they inflect for person and tense-aspect. This is in contrast to converbs (\sectref{temp}), which inflect only for relative tense and do not have person marking. It also contrasts sharply with clause chaining constructions in Chechen and Ingush, which only use converbs (i.e. non-finite verb forms) as verb forms in all but the last clause of the chain. See Examples (\ref{chain-ex08}--\ref{chain-ex09}) from \textcite[531--540]{nichols11} for Ingush and \textcite{good} for Chechen.\pagebreak


\begin{exe}
	\ex Ingush\label{chain-ex08}
	\begin{xlist}
			\ex\label{chain-ex08a}
			\gll pæt'mat-ā axča=Ɂa d-an-na, āra-v-æl-ar mūsā. \\
			Peatmat-{\Dat} money={\Add} {\D}-give-{\Ante} out-{\M}.{\Sg}-go-{\Pst} Musa \\
			\trans `Musa gave Peatmat money and went out.'
			\hfill (\cite[535]{nichols11})
		
		
		
			\ex\label{chain-ex08b}
			\gll mašen ħa=Ɂa j-ett-ā, ieza=Ɂa iez-ā, ʕa-j-æssa-j-æj. \\
			vehicle {\Pv}={\Add} {\J}-load-{\Ante} {\Redupl}={\Add} weigh-{\Ante} {\Pv}-{\J}-empty-{\J}-{\Tr}.{\Nw}.{\J} \\
			\trans `They loaded the truck, weighed it, and unloaded it.' \\
			\hfill (\cite[535]{nichols11})
		
		
	\end{xlist}

	\ex Chechen\label{chain-ex09}
	\begin{xlist}
			\ex\label{chain-ex09a}
			\gll aħmad žʕala=Ɂa iec-na v-ilx-ira. \\
			Ahmed dog={\Add} buy-{\Ante} {\M}.{\Sg}-cry-{\Pst} \\
			\trans `Ahmed bought a dog and cried.'
			\hfill (\cite[125]{good})
		
		
		
			\ex\label{chain-ex09b}
			\gll dō\u{g}a tøx-na āra=Ɂa j-el-la, tyka-na=Ɂa j-a\u{g}-na, c'a j-eɁ-ara malīka. \\
			lock hit-{\Ante} out={\Add} {\F}.{\Sg}-go-{\Ante} store-{\Dat}={\Add} {\F}.{\Sg}-come-{\Ante} home {\F}.{\Sg}-go-{\Pst} Malika \\
			\trans  `Having locked the door and gone out, Malika went to the store, and came home.'
			\hfill (\cite[140]{good})
		
		
		
		
	\end{xlist}
\end{exe}


The exact syntactic parameters of clause chaining constructions have to be investigated further, but one frequent construction involves a clause with a Sequential verb, followed by a clause that consists a preverb with the Additive enclitic \textit{=a}, followed by the verb, see Examples (\ref{chain-ex01}). Note that the Additive\is{Additive} clitic \textit{=a} in Tsova-Tush is added to the last clause, the one that contains a non-sequential verb form, while in Chechen and Ingush, the Additive clitic is found in the chained clauses with the non-finite verb form, see the Examples (\ref{chain-ex08}--\ref{chain-ex09}) above.

\begin{exe}
	\ex\label{chain-ex01}
	\begin{xlist}
		
		
			\ex\label{chain-ex01a}
			\gll  o kvevr gi-j-oll-\textbf{in} daħ\textbf{=a} v-ax-eⁿ. \\
			{\Dist} wine\_jar on\_back-{\J}-put-\textbf{{\Aor}.{\Seq}} {\Pv}\textbf{={\Add}} {\M}.{\Sg}-leave-{\Aor}\\
			\trans `He put the wine jar on his back and left,'
			\hfill (E181-141)
		
		
		
			\ex\label{chain-ex01b}
			\gll daħ v-ax-\textbf{en} ħal\textbf{=o} j-uc'-j-i-eⁿ o. \\
			{\Pv} {\M}.{\Sg}-leave-\textbf{{\Aor}.{\Seq}} {\Pv}\textbf{={\Add}} {\J}-fill-{\J}-{\Tr}-{\Aor} {\Dist} \\
			\trans `He left and filled it.'
			\hfill (E181-141)
		
		
	\end{xlist}
\end{exe}

Longer sequences (here without the Additive clitic) are also possible, see Example (\ref{chain-ex02}).

\begin{exe}
	\ex\label{chain-ex02}
	\begin{xlist}
		
		
			\ex\label{chain-ex02a}
			\gll ħal qall-\textbf{in}, so v-eɁ-\textbf{en}, o jaħo-g aɬ-iⁿ, me ... \\
			{\Pv} eat.{\Pfv}-\textbf{{\Aor}.{\Seq}} hither {\M}.{\Sg}-come-\textbf{{\Aor}.{\Seq}} {\Dist} girl-{\All} say-{\Aor} {\Subord} \\
			\trans `He ate it, returned, and told the girl, that [...]'
			\hfill (E181-148)
		
		
		
			\ex\label{chain-ex02b}
			\gll ču v-ax-\textbf{noer} p'adl-i. \\
			into {\M}.{\Sg}-go-\textbf{{\Nw}.{\Rem}.{\Seq}} basement-{\Iness} \\
			
			\gll o badr-i nʕajɁ=aɁ d-axk'-d-i-\textbf{noer}, kurcn-e-ⁿ badr-i. \\
			{\Dist} child-{\Pl} outside={\Emph} {\D}-come.{\Pl}-{\D}-{\Tr}-\textbf{{\Nw}.{\Rem}.{\Seq}}, weasel-{\Obl}-{\Gen} child-{\Pl} \\
			
			\gll laqiš ħal d-ik'-\textbf{noer}. \\
			high\_up up {\D}-take({\Anim})-\textbf{{\Nw}.{\Rem}.{\Seq}}\\
			
			\gll šuj ben ču xabž-d-i-nor. \\
			{\Refl}.{\Pl}.{\Gen} nest into sit\_down.{\Pfv}.{\Pl}-{\D}-{\Tr}-{\Nw}.{\Rem} \\
			
			\trans `He went into the basement, took the children out, the weasel young, took them up high and sat them down into their own nest.' \\
			\hfill (EK048-2.15)
		
		
	\end{xlist}
\end{exe}

The Sequential construction can also be used with Imperatives, see Examples in (\ref{chain-ex07}). Note that the \textit{-a} in these examples is not just the Imperative suffix \textit{-a}, but must contain the Sequential morpheme, since simple Imperative \textit{-a} would be apocopated (compare \tabref{table-sequential}).\largerpage

\begin{exe}
	\ex\label{chain-ex07}
	\begin{xlist}
		
		
			\ex\label{chain-ex07a}
			\gll kast'eⁿ, ħal\u{o} ħet'-a, kor lac-d-∅-eb e bader=ajn\u{o}. \\
			quick, {\Pv} run-{\Imp}(\Seq) in\_hand take-{\D}-{\Tr}-{\Imp} {\Prox} child={\Quot} \\
			\trans `Quickly, run and catch the baby!'
			\hfill (MM226-1.6)
		
		
		
			\ex\label{chain-ex07b}
			\gll k'ik'el c'e j-ʕaɁ-a, ejɬ-nor k'ruč'i-s, qen=geɁ šampur šampur-mak=a ott-d-∅-eb, ejɬ-nor.\\
			under fire {\J}-light.{\Pfv}-{\Imp}({\Seq}), say-{\Nw}.{\Rem} Kruchi-{\Erg} then={\Emph} skewer skewer-{\Superlat}={\Add} place-{\D}-{\Tr}-{\Imp} say-{\Nw}.{\Rem} \\
			\trans `{``}Kindle the fire,'' said Kruchi, ``and only then put the skewers on top of each other.{''}'
			\hfill (E058-93)
		
		
	\end{xlist}
\end{exe}


The precise parameters of Tsova-Tush Sequential constructions, such as the scope of illocutionary
operators (see e.g. \cite[56]{bickel2010clauselink}), remains to be investigated. It bears, however, strong resemblance to constructions in some Papuan languages termed ``cosubordination'' (\cite{olson81clauserelation,foleyvanvalin84}), but see \textcite{bickel2010clauselink,foley2010clauselink} for doubts about the cross-linguistic validity of this term.

\subsection{Coordination}\is{Coordination}\largerpage

Tsova-Tush coordinates different-subject clauses with the enclitic \textit{=(j)e} `and', see Example (\ref{chain-ex03}).


\begin{exe}
	\ex\label{chain-ex03}
	\begin{xlist}
		
		
			\ex\label{chain-ex03a}
			\gll  osi v-ax-er\textbf{=e} c'ova dax tiɬ-d-al-iⁿ. \\
			there.{\Dist}({\Ess}) {\M}.{\Sg}-live-{\Imprf}\textbf{=and} Tsova therefore be\_called-{\D}-{\Intr}-{\Aor} \\
			\trans `He was living there and because of him it was called Tsova.' \\
			\hfill (E288-86)
		
		
		
			\ex\label{chain-ex03b}
			\gll o vaħo-v o jaħi-ⁿ c'e j-ek-j-i-eⁿ mč'ekrat\textbf{=e}. \\
			{\Dist} boy-{\Erg} {\Dist} girl.{\Obl}-{\Gen} name {\J}-call-{\J}-{\Tr}-{\Aor} aloud\textbf{=and} \\
			
			\gll oqu-s t'q'oɁ oqui-n\textbf{=e}, vaša-x=a ħarč-eⁿ. \\
			{\Dist}.{\Obl}-{\Erg} again {\Dist}.{\Obl}-{\Gen}\textbf{=and} {\Recp}-{\Cont}={\Emph} wrap-{\Aor}\\
			\trans `That boy called out that girl's name aloud and she [did] the same with his, and they embraced each other.'
			\hfill (E142-48,49)
		
		
		
		
			\ex\label{chain-ex03c}
			\gll c'q'e ši ʕu v-ujt'-vanor\textbf{=e}, pħe ču b-epl-iš cħana-n dak'-d-eɁ-nor, me...\\
			once two shepherd {\M}.{\Sg}-go-{\M}.{\Sg}.{\Nw}.{\Imprf}\textbf{=and} village into {\M}.{\Pl}-go\_through-{\Simul} one.{\Obl}-{\Dat} hear-{\D}-come-{\Nw}.{\Rem} {\Subord}\\
			\trans `Once upon a time two shepherds were going on their way and when they entered the village, one of them remembered that [...]' \\
			\hfill (E058-35)
		
		
	\end{xlist}
\end{exe}

However, this same strategy has also been observed to be used for same-subject coordination, see Example (\ref{chain-ex04}).\largerpage[2]

\begin{exe}
	\ex\label{chain-ex04}
	\begin{xlist}
		
		
			\ex\label{chain-ex04a}
			\gll ħatteɁ ču b-ik'-nor\textbf{=e}, qal-maɬ-ar d-eɁ-nor. \\
			immediately into {\M}.{\Pl}-lead-{\Nw}.{\Rem}\textbf{=and} eat-drink-{\Vn} {\D}-bring-{\Nw}.{\Rem} \\
			\trans `He invited them at once, and brought them food and drink.' \\
			\hfill (E058-40)
		
			\ex\label{chain-ex04b}
			\gll  ħo inc se dad c'eni v-a-ħ\u{o}, ise toħ-ra-ħ\textbf{=e}, is-ħ=eɁ ħac'am-v-al-iⁿ-ħu=j\u{o}. \\
			{\Ssg} now {\Fsg}.{\Gen}.{\Obl} father house({\Ess}) {\M}.{\Sg}-be-{\Fsg}.{\Nom} here({\Ess}) sleep-{\Imprf}-{\Ssg}.{\Nom}\textbf{=and} here-{\Ess}={\Emph} wake-{\M}.{\Sg}-{\Intr}-{\Aor}-{\Ssg}.{\Nom}={\Quot}\\
			\trans `You are now in my father’s house, you slept here and woke up here.' \\
			(MM405-1.50)
		
		
	\end{xlist}
\end{exe}


Whatever the exact functional difference between the Sequential construction and the coordinating conjunction, it is clear that they are formally distinct. Historically, the conjunction \textit{=(j)e} may well have been the source of the Sequential suffix \textit{-e}, but in contemporary Tsova-Tush, they are bound to different combinatory rules: the conjunction \textit{=e} is located between clauses, and can be attached to any part of speech. These are often verbs, since the most common word order in Tsova-Tush is SOV, but this is not necessarily the case: see Example (\ref{chain-ex03}), where the clitic is hosted by an adverb and a demonstrative. On the other hand, the Sequential suffix \textit{-e} is clearly affixal, since it attaches only to verbs, and can be followed by other suffixes, such as the subject cross-referencing markers (see \sectref{person}) and the suffix \textit{-t} marking argument plurality on Imperatives (see \sectref{suffixpl}). Furthermore, the Sequential verb form as a whole is not necessarily the final word in a clause, see e.g. Example (\ref{chain-ex02b}). This last point is important; a Sequential suffix (converb or, as in Tsova-Tush, finite) can be distinguished from a conjunction in that the latter usually stands between clauses. This is not the case for a Sequential marker, see e.g. Example (\ref{chain-ex10}) from Kryz, a Lezgic language.\il{Kryz}

\begin{exe}
	\ex\label{chain-ex10}
	Kryz (Alik)
	
		\gll \u{g}i-xha-ci zina-\u{g} cura halav-ar šahar.c-a \u{g}a-č'id-zin. \\
		{\Pv}-wear-{\Seq} 1.{\Refl}-{\Super} other dress-{\Pl} city-{\In} {\Pv}-go\_out-{\Aor}.{\M}-1\\
		\trans `After changing my clothes, I went into the city.'
		\hfill (\cite[325]{authier2009kryzgrammar})
	
\end{exe}

Both constructions are found in the oldest Tsova-Tush sources. See Example (\ref{chain-ex05a}), where a Sequential construction is used, and Example (\ref{chain-ex05b}) for a construction with the conjunction. (For the use of pronouns and subject cross-reference markers in this period, see \sectref{person}.)



\begin{exe}
	\ex\label{chain-ex05}
	\begin{xlist}
		
		
			\ex\label{chain-ex05a}
			\gll as v-ax-\textbf{en} v-itt-v-ejl-n-\textbf{e}-s den-v-al-iⁿ so. \\
			{\Fsg}.{\Erg} {\M}.{\Sg}-go-\textbf{{\Aor}.{\Seq}} {\M}.{\Sg}-wash-{\M}.{\Sg}-{\Intr}-{\Aor}\textbf{-{\Seq}}-{\Fsg}.{\Nom} whole-{\M}.{\Sg}-{\Intr}-{\Aor} {\Fsg}.{\Nom} \\
			\trans `I went, washed myself and got better.'
			\hfill (AS004-1.12)
		
		
		
			\ex\label{chain-ex05b}
			\gll  xiɬ-eⁿ nic'q'li-š mac-ol osi-ħ \textbf{e} o=eɁ kott-v-al-iⁿ. \\
			become.{\Pfv}-{\Aor} strong-{\Adv} hunger-{\Nmlz} there-{\Ess} \textbf{and} {\Dist}={\Emph} disturb-{\M}.{\Sg}-{\Intr}-{\Aor} \\
			\trans `There was a big shortage there, and he too was affected.'
			\hfill (AS006-1.4)
		
		
	\end{xlist}
\end{exe}

Contrary to contemporary Tsova-Tush, that shows occasional use of the Conjunction construction to combine same-subject clauses (see Example (\ref{chain-ex04})), this use is not found in the  subcorpora dating from the 1850s (AS and IT). I therefore tentatively hypothesise that this use, i.e. using the coordinating conjunction \textit{=(j)e} to connect same-subject finite clauses, is a structural copy from Georgian. Georgian, namely, uses a coordinating conjunction \textit{da} to connect both same-subject (Example (\ref{chain-ex06a}--\ref{chain-ex06b})) and different-subject (\ref{chain-ex06c}) finite clauses.

\begin{exe}
	\ex\label{chain-ex06}
	Standard Modern Georgian
    \begin{xlist}
		
		
			\ex\label{chain-ex06a}
			\gll  abesaʒe c'amoxt'a \textbf{da} k'ar-i gaiǯaxuna. \\
			Abesadze s/he\_jumped\_up \textbf{and} door-{\Nom} s/he\_slammed \\
			\trans `Abesadze jumped up and slammed the door.'
			\hfill (GNC: N. Dumbaze)
		
		
		
			\ex\label{chain-ex06b}
			\gll c'avi\u{g}eb me am leks-s \textbf{da} sul a\u{g}ar moval am c're-ši. \\
			I\_will\_take {\Fsg} {\Prox}.{\Obl} poem-{\Dat} \textbf{and} at\_all not\_anymore I\_will\_come {\Prox}.{\Obl} circle-{\In}\\
			\trans `I will take this poem and will not come to this circle again.' \\
			\hfill (GNC: N. Dumbadze)
		
		
		
			\ex\label{chain-ex06c}
			\gll  baratašvil-i arsebobda \textbf{da} imit'om a\u{g}moačina ilia-m. \\
			Baratashvili-{\Nom} s/he\_existed \textbf{and} therefore s/he\_discovered\_it Ilia-{\Erg} \\
			\trans `Baratashvili existed and that's why Ilya discovered him.' \\
			\hfill (GNC: N. Dumbadze)
		
		
	\end{xlist}
\end{exe}

Note that the Additive particle \textit{=a} does not seem to be used for clausal coordination in the same way that \textit{=(j)e} is. The Additive particle is primarily used in clause chaining (see \sectref{chaining}) and to conjoin noun phrases (see \sectref{conjoined}).

\section{Summary}




In terms of basic description, this chapter has provided new insight into the following domains:

\begin{enumerate}
	\item An extensive description of different subordination strategies in Tsova-Tush is presented, for an overview, see \tabref{sub-table1}.
	
	\item In the domain of subordination, Tsova-Tush shows a peculiar profile. It shows the capacity to form both finite and non-finite subordinate clauses, and sometimes shows these multiple strategies for the same clause type. Non-finite constructions include participles, converbs, verbal nouns and infinitives, all of which are constructions shared with Ingush and Chechen (and Daghestanian languages), and are therefore considered archaisms. It has to be noted that in  complement clauses to cognitive and perception verbs, Chechen and Ingush feature finite verbs. Crucially, however, they are not the same constructions as we find in Tsova-Tush. 
	
	\item A very preliminary analysis has been attempted for Tsova-Tush coordination strategies, with a basic distribution of the Sequential suffix \textit{-e} for same-subject clause coordination, and the coordinating conjunction \textit{=(j)e} for different-subject clause coordination. 
	
\end{enumerate}


In terms of structural language contact, this chapter has shown the following parallels between Tsova-Tush and Georgian, which are most likely to be attributed to influence of the latter on the former language. Tsova-Tush borrowed from Georgian the following constructions:

\begin{itemize}
	\item The possibility to form finite adjunct clauses
	\item The possibility to form temporal, causal, and manner clauses using a conjunction 
	\item The template for forming adverbial conjunctions by adding the additive particle to an interrogative base
	\item The template for forming a conjunction `because'
	\item The possibility to form purpose clauses using a general subordinator and a finite form in the Subjunctive.
	\item  Using the coordinating conjunction \textit{=(j)e} to connect same-subject finite clauses is hypothesised to be a structural copy from Georgian.
\end{itemize}






\begin{table}
\fittable{%
	\begin{tabular}{lllll}
    \lsptoprule
		Type & {Chechen-}  & {Tsova-Tush} & {Mod. Georgian} & {Old Georgian} \\
		     & {Ingush}  &  & {Tush Georgian} & \\
		\midrule
		Relative & Participle & Participle & Participle & Participle \\
		&  & Pronoun & Pronoun & Pronoun \\
		&  & Conjunction \textit{me} & Conjunction \textit{ro(m)} & \\
		\addlinespace
		Temporal & Converb & Converb & (Inflected {\Vn}) & \\
		& {\Vn} + Postpos. & {\Vn} + Postpos. & {\Vn} + Postpos. & \\
		&  & Conjunction & Conjunction & Conjunction \\
		&  &  & Conjunction \textit{ro(m)} & \\
		\addlinespace
		Purpose & Infinitive & Infinitive & (Inflected {\Ptcp}) & Inflected {\Vn} \\
		&  & \textit{me} + {\Sbjv} & \textit{rom} + {\Sbjv} & \\
		\addlinespace
		Conditional & Cond. verb & Cond. verb & Conjunction & \\
		&  &  & Conjunction \textit{ro(m)} & \\
		\addlinespace
		Causal & Converb, {\Vn} & Participle, {\Vn} &  & \\
		&  & Conjunction & Conjunction & Conjunction \\
		\addlinespace
		Manner & Converb  & Conjunction & Conjunction & \\
		&  & Inflected {\Vn} & Inflected {\Vn} & \\
		\addlinespace
		Phasal & Infinitive, Converb & Infinitive & Verbal noun & Inflected {\Vn} \\
		\addlinespace
		Desiderative & Infinitive & Infinitive & Verbal noun & Inflected {\Vn} \\
		(Same subj.) &  &  & Asynd. finite & \\
		\addlinespace
		Desiderative & Verbal noun &  & Verbal noun & Inflected {\Vn} \\
		(Diff. subj.) &  & \textit{me} + {\Sbjv} & \textit{rom} + {\Sbjv} & \\
		\addlinespace
		Cognitive & Asynd. finite & Conjunction \textit{me} & Conjunction \textit{ro(m)} & \\
		\addlinespace
		Perception & Asynd. finite & Conjunction \textit{me} & Conjunction \textit{ro(m)} & \\
		& Particle \textit{ænna} &  &  &  \\
        \lspbottomrule
	\end{tabular}}
	\caption{Subordination strategies}
	\label{sub-table1}
\end{table}
