\addchap{Summary in Georgian -- \foreignlanguage{georgian}{ქართული მოკლე შინაარსი}}
\markboth{\foreignlanguage{english}{Summary in Georgian}}
		 {\foreignlanguage{georgian}{ქართული მოკლე შინაარსი}}
\begin{otherlanguage}{georgian}
    დასკვნა ქართულ ენაზე
	წოვათუშური  აღმოსავლეთ კავკასიური ენაა, რომელზეც დაახლოებით 300 ადამიანი საუბრობს აღმოსავლეთ საქართველოში, სოფელ ზემო ალვანში. იგი ნახურ ჯგუფს განეკუთვნება და ჩეჩნურისა და ინგუშურის  ახლომონათესავე ენაა. წოვათუშური განიცდის ქართულის მძლავრ გავლენას. წოვათუშურად მოლაპარაკეები ბილინგვები არიან, მშობლიური წოვათუშურის გარდა, საუბრობენ ქართულადაც და მათი ეთნიკური თვითაღქმა ქართულია. წინამდებარე დისერტაციას სამი მიზანი აქვს: (1) შეიტანოს წვლილი  წოვათუშური ენის ზოგადი აღწერის საქმეში და შეავსოს ის ხარვეზები, რაც არსებობს ამ ენის შესახებ აქამდე არსებულ სამეცნიერო ლიტერატურაში, რეფერენციული გრამატიკის არარსებობის პირობებში; (2) შეუპირისპიროს ერთმანეთს   წოვათუშური, ჩეჩნურ-ინგუშური და ქართული ეკვივალენტური კონსტრუქციები, რათა შესაძლებელი გახდეს ჰიპოთეზების ჩამოყალიბება იმის შესახებ, თუ რომელი კონსტრუქციაა საკუთრივ წოვათუშური და რომელი ქართულის გავლენით წარმოქმნილი; (3) წარმოადგინოს ენათა კონტაქტის ყველაზე სარწმუნო დიაქრონიული სურათი წოვათუშური ენის ისტორიული მონაცემებისა და ისტორიული სოციოლინგვისტური მდგომარეობის გათვალისწინებით.


	პირველ თავში მოცემულია წოვათუშური ენის ზოგადი დახასიათება (ნაწილი 1.2.1) და მის აღსანიშნად ხმარებული  ტერმინები, აქვე საუბარია აღმოსავლურ კავკასიურ ენათა ოჯახში წოვათუშური ენის ადგილისა და დიალექტური სახესხვაობების ხარისხის შესახებ. ამავე თავშია წარმოდგენილი ტიპოლოგიური მონახაზი, ამის შემდეგ კი წოვათუშური ენის კვლევის მოკლე ისტორია. ამას მოსდევს თუშეთის დახასიათება, წოვათუშების მოკლე ისტორია, მათი წეს-ჩვეულებების საბაზისო აღწერა და სოციოლინგვისტური სურათი. 1.3 ნაწილში  მოცემულია მსჯელობა წინამდებარე ნაშრომის მეთოდოლოგიაზე, მიმოხილულია ენათა კონტაქტის შესახებ არსებულ სამეცნიერო ლიტერატურაში წამოჭრილი აქტუალური საკითხები, ასევე, ძირითადი ემპირიული ბაზა და ნაშრომში გამოყენებული მეთოდები.


	მეორე თავი ეხება ფონეტიკისა და ფონოლოგიის საკითხებს. წარმოდგენილია ხმოვანთა და თანხმოვანთა სისტემა, აგრეთვე  აქამდე უცნობი ფონემა \textbf{ჳ} ([ɦ\textsuperscript{w}] ან [w̤]). განხილულია ყველაზე მნიშვნელოვანი ფონოლოგიური პროცესები, შემდეგ კი მასალა შედარებულია ქართულის მონაცემებთან. როგორც წოვათუშურის, ისე ქართულისათვის დამახასიათებელია ერთგვარი ფონოლოგიური მოვლენა, \textbf{რ}-ს დისიმილაცია:  \textbf{რ}-ს შემცველი ორი წოვათუშური სუფიქსი − აბლატიური ბრუნვის ბოლოსართი  \textbf{-რენ} და მრავლობითის მაწარმოებელი არქაული \textbf{-ერჩ}, \textbf{რ}-ს იცვლის \textbf{ლ}-დ, როდესაც დაერთვის \textbf{რ}-ს შემცველ ძირს. წოვათუშურის ეს მახასიათებელი, როგორც ამას 1850-იანი წლებიდან მოყოლებული დაკვირვება აჩვენებს, წარმოადგენს ქართულის ანალოგს როგორც დისტრიბუციის მხრივ, ისე სხვა დეტალებით. ქართული ენის წოვათუშურზე გავლენის კიდევ ერთი ნიშანი ამ  უკანასკნელში ბევრი ახალი თანხმოვანთკომპლექსის გაჩენაა. ეს განპირობებულია ქართულიდან დიდი რაოდენობით ნასესხები სიტყვებით, რომელთა თანხმოვანთკომპლექსები არ გამარტივებულა მსესხებელ ენაში. მოცემული თავი სრულდება  ნასესხები სიტყვების ფონოლოგიური ადაპტაციის შესახებ მსჯელობით. აქ ჩანს, რომ ნასესხობათა უფრო ძველი ფენა წოვათუშურში შეიცავს ფონემა -\textbf{ო}-ს, მაშინ როდესაც სალიტერატურო ქართულში მის ადგილას გვაქვს -\textbf{ვა}-,   ქართულის -\textbf{ხ}-ს ნაცვლად კი -\textbf{ჴ}- არის წარმოდგენილი. ეს მახასიათებლები მიუთითებენ, რომ ნასესხობების წყაროს ქართულის ჩრდილო-დასავლური დიალექტები წარმოდგენს, ყველაზე მეტად კი თუშური დიალექტი. დასახელებულ ელემენტთა შემცველი უფრო გვიანი ნასესხობები კი პირდაპირ სალიტერატურო ქართულიდანაა შესული. ამასთანავე, ქართულიდან ნასესხებ ისეთ სიტყვებში, რომლებშიც თანხმოვანთკომპლექსების პირველ წევრად \textbf{მ}- გვაქვს, იგი  იკარგება. სხვა თანხმოვანთკომპლექსები არ მარტივდება.


	მესამე თავში განხილულია სახელის ფლექსიისა და სახელური ფრაზის სტრუქტურის საკითხები. მოცემული თავი იწყება წოვათუშური ენის ბრუნვათა სისტემისა და ძირითადი გრამატიკული ბრუნვების განხილვით. წინამდებარე ნაშრომი იძლევა ლოკატიური ბრუნვების ახლებურ გაგებას, რომლებიც (გარდა “გვერდით”-სერიისა) მოიცავს დაღესტნური ენების ლოკატიური ბრუნვების მსგავს რთულ სუფიქსებიან სისტემას. აქვეა ბრუნების სხვადასხვა ტიპის გამოყოფის პირველი მცდელობა. ქვეთავი, რომელშიც განხილულია ნასესხები სიტყვების ადაპტაცია, გვიჩვენებს, რომ: (1) ქართულიდან ნასესხები სიტყვები ერთიანდება თანხმოვანფუძიანი სახელების ბრუნების ტიპში; (2) ნასესხები სიტყვების გრამატიკული კლასი განისაზღვრება იმავე ფონოლოგიური და სემანტიკური კრიტერიუმებით, რითაც საკუთრივ წოვათუშური სიტყვებისა; (3) ზედსართავი სახელების სესხება ხშირია, მაგრამ საკუთრივ წოვათუშური ზედსართავებისაგან განსხვავებით, ისინი ბრუნვაში შეთანხმებას არ გვიჩვენებს; (4) უფრო ადრეული სახელური ნასესხობები ერთიანდება Cუ-ბრუნების ტიპში, ადრე ნასესხები ზედსართავი სახელები კი დაირთავენ წოვათუშური ზედსართავების ბოლოსართ -ონ-ს, ამასთანავე, ირიბ ბრუნვებში, როგორც ბრუნვაში შეთანხმების ნიშანს, იღებენ -ჩო დაბოლოებას და ამგვარად, უფრო მეტად არიან ინტეგრირებული წოვათუშური ენის გრამატიკულ სისტემაში. მესამე თავი გრძელდება ყველაზე მნიშვნელოვანი მსაზღვრელების - ჩვენებითი ნაცვალსახელების, რიცხვითი სახელების, ზედსართავებისა და აღწერით. სპეციალური ქვეთავი ეძღვნება მოცემულ მსაზღვრელთა შეთანხმების ხარისხს. ამის შემდეგ განხილულია რთული სახელური ფრაზები, ნათესაობითი ბრუნვითა და შერწყმული ტიპის სახელური ფრაზებით. პირველად არის გაანალიზებული  მსაზღვრელ-საზღვრულის ისეთი კონსტრუქციები, სადაც არსებითი სახელის ირიბი ფორმა დაბოლოების გარეშეა გამოყენებული სხვა სახელის მსაზღვრელად. აღნიშნული თავი სრულდება ისეთ სახელურ ფრაზებზე მსჯელობით, რომელიც არ შეიცავს სახელს: ნაცვალსახელური ფრაზები და გაარსებითებული მსაზღვრელები. სტრუქტურული კონტაქტის მხრივ ეს თავი აჩვენებს, რომ: (1) ადგილობითი (Essive), “შორის”-ადგილობითი (Interessive), “შიდა”-ადგილობითი (Inessive), “ზედა”-ადგილობითი (Superessive), ყველა ამ ადგილობით ბრუნვებში წარმოდგენილი სახელები გამოიყენება მიმართულებითი  ფუნქციით. სხვა ნახურ ენებში მათი ამგვარი ფუნქცია არ დასტურდება, მაგრამ ქართულში რეგულარულად გამოიყენება, ამ უკანასკნელში მიმართულებით-ადგილობითი განსხვავება ცხადად გამოხატული არ არის სახელებში; (2) გარდა  საკუთარი სინთეზური ვარიანტებისა, ზედსართავი სახელის შედარებითი ხარისხის გადმოსაცემად წოვათუშურს გააჩნია ანალიზური კონსტრუქციებიც. ასეთი კონსტრუქციები არ მოიპოვება ჩეჩნურსა და ინგუშურში, მაგრამ სავალდებულოა ქართულში, ამასთანავე, ისინი ერთი და იმავე მორფემით იწარმოება ორივე ენაში, წოვათუშურში ქართულიდან ნასესხები მორფემა - \textbf{უფრო} გამოიყენება შესადარებლად. გარდა ამისა, უფროობითი ხარისხიც ანალიზური წარმოებისაა ჩეჩნურსა და ინგუშურში, წოვათუშურში კი  გამოიყენება სპეციალური მორფი \textbf{ჰ'ამახეჸ}, რომელიც ქართულის შესაბამისი ფუნქციის მორფემის \textbf{ყველაზე} კალკია. (3) როგორც წესი, ასზე მეტის აღმნიშვნელი რიცხვითი სახელები წოვათუშურში ქართულიდან არის ნასესხები. (4) ჩეჩნურ-ინგუშურის იმ  ჩვენებითი ნაცვალსახელების საპირისპიროდ, რომლებიც II პირთან ახლომყოფობას გამოხატავენ, წოვათუშურში   I-II პირთან არმყოფობის აღმნიშვნელი ო  გამოიყენება დეიქტურად ნეიტრალურ მესამე პირის ნაცვალსახელადაც. იგი  ქართული ის ნაცვალსახელის ზუსტ ანალოგს წარმოადგენს. (5) უარყოფითი ნაცვალსახელებით ნაწარმოები ფრაზები მათი ქართული შესატყვისების ანალოგიურია და განსხვავდება საკუთრივ წოვათუშური კონსტრუქციებისგან (მოიპოვება სხვა ნახურ ენებშიც), რომლებიც შედგება განუსაზღვრელი ნაცვალსახელისა და უარყოფისგან წინადადების დონეზე.


	მეოთხე თავი ზმნის ფლექსიას ეხება. მასში განხილულია ყველა წოვათუშური აფიქსი და აბლაუტი, ზმნურ ფორმაში წარმოდგენილი მარცხნიდან მარჯვნისკენ. ნომინატივში (იშვიათად ერგატივში) მდგარი აქტანტის აღმნიშნვნელი კლასის პრეფიქსი, არაპროდუქტიული მორფოლოგიური მოვლენები, როგორებიცაა ინფიქსითა და აბლაუტით გამოხატული ზმნის ძირის ასპექტი (ჩვეულებისამებრ ზმნის მრავლობითიობასთან დაკავშირებული), ფუძის მონაცვლეობა, რომელიც მასთან შეწყობილი პირების მრავლობითობას გამოხატავს, დრო-კილოთა ყველა სუფიქსი და სუბიექტის გამომხატველი აფიქსები. ეს თავი იძლება ახლებურ ხედვას შემდეგ საკითხებზე: (1) წოვათუშური განარჩევს ზმნურ კატეგორიას − უკვეობითს (Iamitive), მის გამოსახატავად დადებით წინადადებაში გამოყენებულია ბმული მორფემა “უკვე”,   უარყოფით წინადადებაში კი “აღარ”. (2) მანამდეც ცნობილი იყო, რომ ზმნის ფორმა, რომელსაც
	-\textbf{რალო} სუფიქსი დაერთვის, სრულ ასპექტსა და უნახავ მოქმედებას გამოხატავს, მაგრამ წინამდებარე ნაშრომში ჩვენ მას ვაძლევთ წარსული დროის კავშირებითი კილოს კვალიფიკაციას (რადგან მას შეუძლია კავშირებითი კილოს შინაარსის გამოხატვა სხვა კონტექსტში). მან აგრეთვე შეიძინა დამატებითი ფუნქცია − იმ თხრობაში, რომელიც სრულიად უნახავ მოქმედებას გადმოსცემს, იგი არასრულ ასპექტს გამოხატავს. ენათა სტურქტურული კონტაქტის მხრივ, წინამდებარე თავი აჩვენებს შემდეგს: (1) მიუხედავად იმისა, რომ ორივე ენაში გამოიყენება ნამყო სრულად წოდებული ზმნის ფორმები, ფაქტობრივი პირობითი კონსტრუქციები საკმაოდ განსხვავებულია წოვათუშურსა და ქართულში. ამიტომ უსაფუძვლო იქნებოდა ამაში ქართულის გავლენის დანახვა. (2) წოვათუშურში ჩამოყალიბდა სუბიექტური პირის ნიშნები (კოჯიმა 2019), რომლებსაც ობიექტური პირის კლიტიკურ ნაწილაკებთან ერთად შეუძლიათ გამოხატონ იგივე ინფორმაცია, რაც ქართულმა პოლიპერსონალურმა ფინიტურმა ზმნურმა ფორმამ. ნასესხებ ზმნებს რაც შეეხება, უშუალოდ ზმნურ ფორმებთან ერთად, წოვათუშურმა ქართულისაგან  ისესხა ასპექტის ზმნისწინებით გარჩევის საშუალება. ამასთანავე, მრავლობითობის გამოსახატავად ბრძანებითებსა და იმ ფორმებში, რომლებიც შეიცავენ მრავლობითი რიცხვის პირველი პირის ინკლუზიურ  ნაცვალსახელს, ისესხა სუფიქსი \textbf{-თ}. ეს ბმული მორფემის სესხების იშვიათ მოვლენას გვიჩვენებს.


	მეხუთე თავში სამი ურთიერთდაკავშირებული თემაა განხილული: ზმნის ვალენტობა და პირთა ასახვის მოდელები (ნომინატიური, ერგატიული და დატიური კონსტრუქციები)  მორფოლოგიურად რთული ზმნები და აქტანტური დერივაცია. პირველ რიგში, დახასიათებულია წოვათუშური ზმნის ვალენტობის ძირითადი მოდელები, აგრეთვე მოცემულია ვრცელი მსჯელობა წოვათუშურის გრამატიკაში ცნობილი და ხშირად განხილვადი საკითხის − ერთპირიანი ზმნების შესახებ, რომლებიც მოითხოვენ სუბიექტს ერგატიულ ბრუნვაში. შემოთავაზებულია  გარდაუვალი ზმნის ახალი ტიპოლოგია წოვათუშურისთვის. ასევე გამოთქმულია მოსაზრება, რომ შესაძლოა ამ ტიპის ზმნები არ წარმოადგენდეს ქართული ენის გავლენის შედეგს, რომლისთვისაც აგრეთვე დამახასიათებელია გარდაუვალი ზმნების იგივე კატეგორია (ლექსიკური ზმნების წყება, ასევე მრფოსინტაქსური მოდელები საკმაოდ განსხვავებულია). 5.3 ნაწილში კლასიფიცირებულია წოვათუშური რთული ზმნების სხვადასხვა სახეობები: მეშველზმნიანი კონსტრუქციები, მეშველზმნიანი კომპოზიტები (light verb constructions), ზმნური დვანდვა კომპოზიტები, რედუპლიკაციის შედეგად მიღებული და ნაზედსართავალი ზმნები. ზმნა \textbf{დ-ი} “კეთება” (\textbf{დ-} კლასის ნიშანია) გამოიყენება მეშველ ზმნად ფრაზებსა და კომპოზიტებში, \textbf{დ-ი} და \textbf{დ-ალ} აგრეთვე გამოიყენება  ზედსართავი სახელებისაგან გარდამავალი და გარდაუვალი ზმნების საწარმოებლად. იგივე ზმნები მონაწილეობენ აქტანტურ დერივაციაშიც, რომელიც განხილულია 5.4 ნაწილში. ამავე თავში მიმოხილულია მორფოლოგიური კაუზაციაც, გარდაუვალი კონსტრუქციების წარმოება \textbf{დ-ის} “დარჩენა” ზმნის გამოყენებით და პოტენციალისი. 5.5 ნაწილში წარმოდგენილია ნასესხები ზმნების ადაპტაციის საშუალებები  იმავე ზმნების −  გარდამავალი \textbf{დ-ი}-ისა და გარდაუვალი \textbf{დ-ალ}-ის გამოყენებით. ქართული ზმნები, რომლებიც მესამე კლასში ერთიანდება (ერთვალენტიანი ზმნები, რომლებიც პერფექტულ დროებს აწარმოებენ ზმნისწინის გარეშე, სემანტიკურად, ძირითადად შეიცავენ არამიზნიანი (atelic) ზმნებს, სინტაქსურად მოითხოვენ სუბიექტს ერგატიულ ბრუნვაში მეორე სერიაში), წოვათუშურში ადაპტირებულია ერთვალენტიანი გარდამავალი ზმნის მორფოსინტაქსით. ამ კონკრეტული ზმნების სესხებით წოვათუშურს ახლა გააჩნია გარდაუვალი ზმნების სერია, რომელიც გარდამავალი ზმნის მორფოლოგიით ხასიათდება (მეშველი ზმნა \textbf{დ-ი}). ზოგადი კლასის ნიშანი \textbf{დ-} (რომელიც არ აღნიშნავს ზმნასთან შეწყობილ არცერთ პირს), არამიზნიანი (atelic) სემანტიკა და ყველა სუბიექტური პირი ერგატიულ ბრუნვაში. ამასთანავე, ზმნები ნასესხებია ძველი ქართულის მასდარის ვითარებითი ბრუნვის ფორმით, რომელიც თანამედროვე ქართულში აღარ გამოიყენება, მაგრამ წოვათუშურში პროდუქტიულია, აქ მან ნასესხები ზმნების ადაპტირების ფუნქცია შეიძინა.


	მეექვსე თავში განხილულია რთული წინადადებების ნაწილების შეერთების საკითხი, უმთავრესად სუბორდინაცია. დამოკიდებული წინადადებების მთავარ ტიპებთან − მიმართებით, დამატებით, გარემობით დამოკიდებულ წინადადებებთან ერთად გაანალიზებულია ქართული ენის გავლენა აღნიშნულ უბანზე. წოვათუშურს გააჩნია როგორც ფინიტური, ისე არაფინიტური სუბორდინაციული დამოკიდებული წინადადების წარმოების საშუალება, ზოგჯერ კი ერთი და იმავე ტიპისათვის რამდენიმე სტრატეგიას იყენებს. არაფინიტური კონსტრუქციები მოიცავს მიმღეობებს, კონვერბებს, ნაზმნარ სახელებსა და ინფინიტივებს, ყველა კონსტრუქციას, რომელიც საერთოა ჩეჩნურ-ინგუშურთან (და დაღესტნურ ენებთან), ამის გათვალისწინებით, ისინი განხილულია არქაიზმებად. უნდა აღინიშნოს, რომ ზოგიერთ ტიპში, სახელდობრ, დამატებით დამოკიდებულ წინადადებაში, რომელიც შეიცავს გრძნობა-აღქმის ზმნებს, ჩეჩნური და ინგუშური იყენებს ფინიტურ ზმნებს, მაგრამ წოვათუშური ამ მხრივ განსხვავებულ ვითარებას გვიჩვენებს. წოვათუშურმა ქართულისაგან ისესხა დამატებითი დამოკიდებული წინადადების, აგრეთვე გარემოებითი, დროის, ვითარებისა და მიზეზის გარემოებითი დამოკიდებული წინადადების კავშირების გამოყენებით წარმოების საშუალებები.  მიმართებით სიტყვების წარმოქმნის ყალიბი - კითხვითი ნაცვალსახელისა და ენკლიტიკური კავშირის გამოყენებით. “რადგან” კავშირის  წარმოების ყალიბი, მიზნის გარემოებითი დამოკიდებული წინადადების წარმოების საშუალება ზოგადი მაქვემდებარებელი კავშირის საშუალებით, და კავშირებითის ფინიტური ფორმა. ამასთანავე, თანწყობის ნაწილში, წარმოდგენილია  ზმნის ფინიტური ფორმის, რომელსაც ჩვენ შედეგობითს, თანმიმდევრობითს ვუწოდებთ, პირველი აღწერის მცდელობა. იგი იწარმოება სუფიქს -\textbf{ე}-თი იმ დამოკიდებულ წინადადებაში, რომელსაც ერთი და იგივე სუბიექტი აქვს და მაჯგუფებელი კავშირი \textbf{ჲე} იმ წინადებაში, რომელსაც სხვადასხვა სუბიექტი აქვს. \textbf{ჲე}-ს გამოყენების ყველა საპირისპირო მაგალითი, მიჩნეულია ქართული ენის სტრუქტურულ კოპირებად.


	დაბოლოს, მეშვიდე თავი აჯამებს წოვათუშურში ენათა კონტაქტით  გამოწვეულ ყველა მაგალითს. ასევე გვაწვდის ინფორმაციას, რომელ პერიოდში მოხდა შესაბამისი ცვლილება. ემპირიული მასალა გვიჩვენებს, რომ მოგვიანო პერიოდში (მას შემდეგ, რაც გაიზარდა მოლაპარაკეთა შორის ბილინგვიზმის ხარისხი და სიმძლავრე), მორფოლოგიური მოდელების სესხების უფრო მეტი მაგალითია. ენათა კონტაქტის ინტენსიურობის ზრდასთან ერთად სინტაქსური მოდელების სესხების მაგალითებიც დასტურდება. ზმნის მრავლობითი რიცხვის მაწარმოებელი \textbf{-თ} მორფების სესხება მნიშვნელოვან გამონაკლისს წარმოდგენს, იგი დამოწმებულია ბილინგვიზმის შედარებით ადრეულ ეტაპზე. ენათა კონტაქტის შედეგად გამოწვეული ძირითადი ცვლილებები 1820 წლამდე უნდა მომხდარიყო, როდესაც წოვათუშთა და თუშეთის სხვა თემების მოსახლეობის რაოდენობა მნიშვნელოვნად არ განსხვავდებოდა, ბილინგვალიზმის დონე შედარებით დაბალი, ეთნიკური თვითიდენტიფიკაცია კი უკვე ქართული იყო.
\end{otherlanguage}
