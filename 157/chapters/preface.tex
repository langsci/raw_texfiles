\documentclass[output=paper]{LSP/langsci} 
\title{Preface and acknowledgements}
\author{Marian Klamer}
\abstract{\vspace*{-2\baselineskip}}
\ChapterDOI{10.5281/zenodo.569380}

\makeatletter
\renewcommand{\lsCollectionPaperTitle}{{%
  \renewcommand{\newlineTOC}{}
  \renewcommand{\newlineCover}{\\} 
  \\[-1\baselineskip]
% \vspace*{-2\baselineskip}
  \noindent{\LARGE ~}\\
  \bigskip  
  \noindent\@title}}
  
\renewcommand{\lsCollectionPaperTOC}{{%
  \iflsCollectionChapter%
    \protect\numberline{~}\fi
  \@title\newline{\normalfont\@author}}}
 \makeatother
\setcitation{Klamer, Marian}{Preface and acknowledgements}{1--3}
\maketitle

\rohead{}


\begin{document}
% \renewcommand{\lsCollectionPaperCitationText}{\bottomcitation}
% \abstract{}
% \rohead{Preface and acknowledgements}
% \renewcommand{\lsCollectionPaperTitle}{{%
% 	\renewcommand{\newlineTOC}{}
% 	\renewcommand{\newlineCover}{\\} 
% 	Preface and acknowledgements}}
% \maketitle
% \addcontentsline{toc}{chapter}{Preface and acknowledgements}

% \begin{document}
\noindent
This volume presents some of the results of the research project `Alor-Pantar languages: Origin and theoretical impact'. This project was one of the five collaborative research projects in the EuroCORES programme entitled `Better Analyses Based on Endangered Languages' (BABEL) which was funded by the European Science Foundation from 2009--2012. The `Alor-Pantar' project involved researchers from the University of Surrey (Dunstan Brown, Greville Corbett, Sebastian Fedden), the University of Alaska Fairbanks (Gary Holton, Laura Robinson), and Leiden University (Marian Klamer, Antoinette Schapper). František Kratochvíl (Nanyang Technological University) was an affiliated researcher. Brown, Corbett and Fedden (Surrey) were funded by the Arts and Humanities Research Council (UK) under grant AH/H500251/1; since April 2013,  Corbett, Brown and Fedden were funded by the Arts and Humanities Research Council (UK) under grant AH/K003194/1. Robinson was funded by the National Science Foundation (US), under BCS Grant No. 0936887. Schapper was funded by the Netherlands Organisation for Scientific Research (NWO) from 2009--2012. 
	
This volume represents the ``state-of-art'' of linguistic research in Alor-Pantar languages. Several chapters relate to work that has been published earlier, as explained in what follows. 

Chapter~2 builds on methodology described previously in \citet{HoltonEtAl2012} and \citet{RobinsonEtAl2012internal}, but the current chapter draws on new lexical data. In particular, the number of reconstructed proto-Alor-Pantar forms has been increased by 20\% over that reported in \citet{HoltonEtAl2012}, and many additional cognate sets have been identified. It also differs from Robinson and Holton (2012a) in that the latter work focuses on computational methodology, arguing for the superiority of using phylogenetic models with lexical characters over traditional approaches to subgrouping, whereas chapter 2 of this volume simply applies these tools to an updated data set, omitting the theoretical justification for the methodology. 

Chapter~3 revises and expands previous reconstructions within the larger Ti\-mor-Alor-Pantar family as published in \citet{HoltonEtAl2012} and \citet{SchapperEtAl2012}. Chapter~3 is new in considering the relatedness of Timor-Kisar languages with the Alor-Pantar languages, while Schapper et al. 2012 was limited to the study of the internal relatedness of the Timor-Kisar languages only.

Chapter~4 is an updated and significantly expanded revision of \citet{RobinsonEtAl2012reassessing}. It differs from the latter paper in that it includes a discussion of the typological profiles of the Timor-Alor-Pantar family and its putative relatives, and has also been updated to reflect new reconstructions, especially the proto-Timor-Alor-Pantar reconstructions that are given in chapter 3. 

Chapter~9 contains a discussion of plural words in five Alor-Pantar languages. The Kamang and Teiwa data were published earlier as conference proceedings \citep{SchapperEtAl2011plural}, but chapter 9 is able to revise and expand that earlier comparative work by taking into account data from three additional Alor-Pantar languages.

Chapter~10 is a newly written chapter on Alor-Pantar participant encoding, and summarizes the findings of \citet{FeddenEtAl2013} and \citet{FeddenEtAl2014}. In addition, it includes a discussion of specially made video clips that have been used to collect the pronominal data, and the field manual to work with the video clips is included in the Appendix.

Crucially, all the chapters in this volume rely on the latest, most complete and accurate data sets currently available. In this respect, they all differ significantly from any of the earlier publications, as these earlier works were written either before the EuroBABEL project had even begun (e.g. \citet{HoltonEtAl2012}, which is essentially a revised version of a conference paper presented in 2009), or while data collection and analysis in the project was still ongoing. Where there are any discrepancies between chapters in this volume and data that was published earlier, the content of the present volume prevails.

All the chapters in this volume have been reviewed single-blind, by both external and internal reviewers. I am grateful to the following colleagues for providing reviews and helpful comments on the various chapters (in alphabetical order): Dunstan Brown, Niclas Burenhult, Mary Dalrymple, Bethwyn Evans, Sebastian Fedden, Bill Foley, Jim Fox, Martin Haspelmath, Gary Holton, Andy Pawley, Laura C. Robinson, Hein Steinhauer, and Peter de Swart.
	
{\sloppy	
\printbibliography[heading=subbibliography,notkeyword=this]
}

% \end{document}

\renewcommand{\lsCollectionPaperTitle}{{%
  \renewcommand{\newlineTOC}{}
  \renewcommand{\newlineCover}{\\} 
  \\[-1\baselineskip]
% \vspace*{-2\baselineskip}
  \noindent{\LARGE Chapter \thechapter}\\
  \bigskip  
  \@title}}
  
  \renewcommand{\lsCollectionPaperTOC}{{%
  \iflsCollectionChapter%
    \protect\numberline{\thechapter}\fi
  \@title\newline{\normalfont\@author}}}

  \end{document}
  