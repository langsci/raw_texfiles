\chapter{Non-clause-level case marking}\label{extrasyn}
%\setcounter{exx}{0}

\section{Introduction}

This last chapter in the data-oriented part of this study is dedicated to a number of special contexts. 
All of these contexts have in common the fact that the case-marking of the noun is not based on its role at the clause level. 
The contexts studied in the previous chapters were clauses of some kind or even more complex constructions, e.g. the biclausal analysis of focus constructions (Chapter~\ref{emphaticS}). 
In all these contexts, the encoding of the subject or subject-like elements was investigated, including  the marking of one additional role, namely predicate nominals.  
In this chapter, the contexts are on a lower level, and the roles investigated cannot be considered to be subjects of any sort. 
Instead, all contexts encode roles that do not relate to verbal argument structure but are defined on a different level.
In the first context discussed in this chapter, attributive possession, this level is the noun phrase. 
The role of interest in this context is that of adnominal possessor. Attributive possession in general and the encoding of possessors are discussed in Section~\ref{AttrPoss}.
The two other roles to be discussed in this chapter are not defined by any syntactic relation at all but are rather defined entirely by the larger meta-linguistic or conversational context.
First, I discuss the form of a noun (most often a name) when addressing\is{terms of address} someone. 
Some languages have a dedicated case-form, a vocative\is{case!individual forms!vocative}, to be used in this function. 
A brief discussion of the grammar of address is provided in Section~\ref{Address}.
The other extra-syntactic form is the citation form\is{citation form} of a noun, which is used in meta-linguistic reference to a noun. 
It is often associated with the form used in dictionaries, but also for labeling things. This form is discussed in Section~\ref{Citation}. 

After introducing the three roles investigated in this chapter, Section~\ref{ExtraQuest} addresses the different coding-patterns to be distinguished here.
The subsequent sections provide data on the marking of extra-syntactic functions and structures below the level of the clause. 
As in the previous chapters, the data are divided by area and genealogical groupings. 
Starting with the African marked"=S languages, Section~\ref{ExtraNilo} discusses the Nilo-Saharan languages and Section~\ref{ExtraAfro} deals with the Afro-Asiatic ones. 
Data on the languages of the Pacific area are given in Section~\ref{ExtraPac}, while Section~\ref{ExtraNA} provides information on the North American languages.
Finally, a summary comparing the encoding strategies of extra-syntactic contexts and attributive possessors in all marked"=S languages is given in Section~\ref{ExtraSum}. 
 
\section{Attributive possessors}\label{AttrPoss}\is{possession!attributive|(}	

Attributive possessors modify a noun in a way similar to other nominal modifiers such as adjectives or quantifiers.
\footnote{Another strategy to encode possessive relationships is via predicative possessive constructions, and thus is analyzed at the level of the clause. 
This context, studied in much detail by \citet{Stassen:2009}, has already been discussed to some extent in Chapter~\ref{existpred}. 
In that chapter, I have explained that in the languages of my sample predicative possession is either encoded via a locational strategy, and hence with the same construction as locational predication, or via a transitive possessive verb with regular transitive case-marking on its arguments.}
They can be either realized as full nouns or as pronominal elements. 
If realized as pronominal elements, indicating number, person and/or gender of the possessor, the range of cross-linguistic coding strategies is very large. 
In many languages, the head-noun is marked with person agreement affixes if the possessor is not realized as an independent noun, either identical to the markers used for indexing on verbs or a different set of markers. 
Other languages use independent pronouns to encode grammatical features of possessors not realized as independent nouns. 
These can either be a special set of possessive pronouns or the same forms used in other pronominal contexts. 
The encoding of full-noun attributive possessors can be very different from the encoding of pronominal attributive possessors. 
As for the contexts studied here, I will focus on the encoding of possessors as full nouns rather than as pronominals. 
The pronominal coding properties will only be discussed when relevant. 
  
First, I will discuss the attributive possessive context in general.
A possessive contexts contains (at least) two entities, one that will be labeled as the \textsc{possessor} in the following and one that will be labeled as the \textsc{possessee}.\footnote{The Latin-derived term `possessum' is also commonly used in the linguistic literature. 
\citet[262]{Dixon:2010-2} introduces the roles R (possessor) and D (possessed) for the two nouns. 
However, since the label `R' is also used for ditransitive recipients, I will not use these abbreviations here to avoid confusion.}
The semantics of possessive constructions have been discussed extensively \citep[143--156]{Heine:1997}. 
It has been noted that most possessive constructions are not restricted to actual possession in the strict sense of one entity being the legal owner of another entity \citep[722]{Lyons:1977}. More often the possessive context expresses a more general association between two entities. 
Kinship terms and part-whole relations are the most common semantic domains to be expressed by possessive constructions, next to actual ownership \citep[263]{Dixon:2010-2}.

Further\is{possession!alienable versus inalienable|(}, many languages distinguish between so-called alienable possession, involving items that can easily be disposed of, and inalienable possession, involving items that are permanently possessed such as body-parts or kin \citep{Chappell:1996}. 
If a language distinguishes between these two types of possessives, each type has a dedicated construction; the two constructions might vary greatly in their means of expression. 
Due to the tight-knit relation between possessor and possessee in inalienable possession, the relationship is usually expressed using less material than with alienable possession. Strategies often associated with inalienable possession are mere juxtaposition and indexing on the possessee, while alienable possession is often expressed through genitives or free or bound linker morphemes between the two entities \citep[4--5]{Chappell:1996}.\is{possession!alienable versus inalienable|)}

\citet[32--40]{Croft:2003} presents a detailed analysis of the different kinds of marking found in possessive constructions.
He distinguishes between three basic types: `simple strategies', `relational strategies', and `indexing strategies'. 
He further notes that these distinctions might become blurred once a strategy becomes more grammaticalized. 
Not all the details of Croft's typology are relevant for the present study. 
Therefore, I will concentrate on the strategies and distinctions which are relevant for the present discussion.
Apart from pure positional marking (i.e. juxtaposition of possessor and possessee), head-marking and dependent-marking strategies can be distinguished. 
The dependent-marking strategy appears to be more common cross-linguistically. 
In this strategy, the possessor is marked for its role in the possessive construction, for example by a special inflectional case-form, which is often labeled as `genitive'\is{case!individual forms!genitive}. 
This terminology is indeed so common that \citet[104]{Payne:1997} refers to adnominal possessors as `genitive' irrespective of whether they are inflectionally marked or not. 
Apart from being fully fledged case-forms, the possessor in an attributive possessive construction can also be marked via possessive particles or distinct prepositions (cf. the English\il{English} \textit{of}-possessive). 
Head-marking attributive possessive constructions are often associated with inflectional markers on the possessee that agree with the possessor in person, number and the like. 
These markers are often only used when the possessor is not expressed as an independent noun. 
However, some languages use these markers in all possessive contexts.
Further\is{case!individual forms!antigenitive|(}, in some languages there is a special case-form used on the possessee sometimes called an `anti-genitive' \citep{Andersen:1991}. 
This marker differs from the affixal possessor agreement-system, since it is not inflected for any properties of the possessor, like person or number. 
Since this type of case-marking appears to be less common and does not occur in any of the traditional case-marking languages (like Latin, Greek or Sanskrit)  there is no common term for a case like this.
\citet[268]{Dixon:2010-2} proposes `pertensive' as a label.\is{case!individual forms!antigenitive|)} 
However, note that he restricts the use of the term `case' to clause level marking
and thus does not consider the genitive\is{case!individual forms!genitive} nor his newly coined `pertensive' to be cases altogether.\is{possession!attributive|)}

\section{Forms of address}\label{Address}\is{terms of address|(}

A special purpose form of the noun is the form used in addressing a person (or more seldom a thing). 
Latin grammar has the Vocative\is{case!individual forms!vocative}, traditionally regarded as a special case-form, and such a special form exists in a number of other languages. 
However, if there is no special case-form in this context, the address function is supposed to be passed over to the nominative\is{case!individual forms!nominative}  case, as \citet[184]{Jespersen:1992} noted. 
%Hjelmslev \citet[as cited and translated in][65]{JakobsonCase:1984} 

\citet{Daniel:2009}\is{case!individual forms!vocative|(} discuss the `vocative' as a member of case paradigms and also consider other means used to achieve the same function.
The function of addressing someone is often performed by intonation or other prosodic means. 
Lengthening of vowels or reduction of the noun stem are also commonly used, as well as vocative\is{case!individual forms!vocative} particles. 
These particles combine with the unmarked or nominative\is{case!individual forms!nominative}  case-form of a noun to form a kind of detached vocative\is{case!individual forms!vocative} according to \citet[630]{Daniel:2009}.  
They also find that the vocative\is{case!individual forms!vocative} seems to be derived from the nominative\is{case!individual forms!nominative} case in most languages (even if other case-forms are not). 
However, they conclude that it is actually the unmarked form of a noun, which often coincides with the nominative, that serves as a source for the vocative\is{case!individual forms!vocative}. 
In some cases, as they note, the vocative ``is even less marked than the nominative'' \citep[631]{Daniel:2009}.\is{case!individual forms!vocative|)}

By definition, nouns serving as terms of address are not integrated into the argument-structure of a sentence.
This is illustrated by the English\il{English} example in (\ref{EngVoc}) in which the term of address is co-referential with the subject of the sentence expressed via the second person pronoun.
Orthographic convention often separates these nouns from the remainder of the sentence by punctuation. 

\begin{exe}\ex\label{EngVoc} \textit{Do you hear me, John?}\end{exe}

In this study, I will restrict myself to the actual morphological shape of nouns used for address. 
This topic of research is extremely restricted in its scope. 
Other factors, especially concerning the prosody of terms of address, certainly need to be taken into account in order to get a full picture of this domain of grammar.\is{terms of address|)}


\section{Citation form}\label{Citation}\is{citation form|(}

The citation form of a noun is a meta-linguistic concept.
However, there are also actual speech situations in which such a form might prove useful. 
\citet[450]{Creissels:2009} lists labeling boxes or the like for their content or identifying persons by means of a passport as such contexts.
Furthermore, in societies without writing, such a form can be thought of as used in instructing language learners on how a specific item is called (in case language teaching is practiced at all in the particular society). 
These contexts cannot in all cases be interpreted as instances of elliptic nominal predication of the type `(This is an) X' since the form of citation and predicate nominals need not coincide, as \citet[450]{Creissels:2009} points out (also compare the data in Chapter~\ref{nompred} on this issue).
 
The concept of a citation form was discussed prominently by \citeauthor{Lyons:1977}, who defined it in the following way: 
\begin{quote}
By the citation-form of a lexeme is meant the form of the lexeme that is conventionally employed to refer to it in standard dictionaries and grammars of the language. [\dots] It is important to realize that the citation-form is indeed a form of the lexeme (being used for a particular reflexive or meta-linguistic purpose): it is not to be identified with the lexeme itself. \citep[19]{Lyons:1977}
\end{quote}

Lyons is careful to state that this meta-linguistic citation form may be different from the form speakers use in referring to a word. 
This distinction is probably more relevant for verbs than for nouns.
While for verbs a variety of different traditions of choosing one form over the other in dictionaries and the like exist (e.g. infinitive; 1st person singular, present, indicative, active), for nouns most often the nominative\is{case!individual forms!nominative}  case is chosen for this purpose \citep[40]{Aronoff:1994}.
The citation form is set apart from another meta-linguistic form by \citet[41]{Aronoff:1994}, the so-called `lexical representation', which in contrast to the citation form is an abstract form never realized at the surface level (cf. the contrast between Semitic consonantal roots and the citation form of the corresponding lexeme).

The naming function is connected closely to the nominative\is{case!individual forms!nominative}  case, as has already been discussed in the first chapter (Section~\ref{label}).
Though extra-syntactic functions do not play any role in modern grammatical theories, ancient grammarians put more emphasis on these uses when they chose to label the nominative\is{case!individual forms!nominative}  as ``\emph{onomastik\^e pt\^osis}, and to transpose this term into Latin as \emph{casus nominativus} `the case used to designate' '' \citep[450]{Creissels:2009}. 
However, as has been discussed there, the nominative in marked"=nominative languages most often does not fulfill this naming function, hence the suggestion by \citet{Melcuk:1997} and \citet{Creissels:2009} to abandon the use of the term `nominative' in these languages.
Extra-syntactic functions, and especially the form I refer to as `citation form' here, are one crucial aspect of \citeauthor{Creissels:2009}'s (\citeyear{Creissels:2009}) proposal of case-terminology. 
As in this study, he subsumes two functions under the label extra-syntactic use. 
The first one is the function of addressing someone (`function of call' in his terminology), which was discussed in the previous section as \textit{form of address}\is{terms of address}. 
His other form of extra-syntactic use is the function of `quotation and designation' \citep 450]{Creissels:2009}, which corresponds to my \textit{citation form}.\is{citation form|)}   

\section{Research questions}\label{ExtraQuest}

In the subsequent sections I will investigate how the contexts just outlined are encoded in the languages of my sample.
In short these contexts are:

\begin{itemize}
\item attributive possessors\is{possession!attributive} 
\item nouns used for addressing\is{terms of address} someone
\item nouns in the citation form\is{citation form}
\end{itemize}

\citet[590]{Lander:2009}\is{possession!attributive|(} notes that ``languages often code the possessor in a similar way to the marked participant in a transitive construction.''
This would mean that marked"=S languages should make strong use of the S-case (nominative or ab\-so\-lu\-tive) for marking attributive possessors.
While\is{case!individual forms!antigenitive|(} this pattern is for example found in Dinka\il{Dinka (Agar)} (\ref{DinGenEx})
\footnote{\citet{Andersen:1991} uses the following terminology: Genitive for the case that marks post-verbal (i.e. non-topical subjects) as well as attributive possessors, Antigenitive for the case that marks possessees.}
 the more common pattern in marked"=S languages is to either use a different overtly-coded form as exemplified by Boraana\il{Oromo (Boraana)} Oromo (\ref{BorGenEx}) or the zero-coded case-form like in Cocopa\il{Cocopa} (\ref{CocGenEx}). 
Furthermore, polyfunctionality of case-forms as attributive possessors and semantic cases such as local ablative or allative, another common pattern according to \citet[590]{Lander:2009},  does not seem to be found in the marked"=S languages.  

\begin{exe} 
\ex\label{DinGenEx}\langinfobreak{Dinka (Agar)}{Nilotic; Sudan}{\citealp[273]{Andersen:1991}}
\gll\textipa{dh\"*{\'O}ON} \textipa{\"*{\`e}} \textbf{\textipa{m\~*{\'a}ri\~*{\`a}al}} \textipa{\~*{\`a}-b\"*{\'O}}\\
boy.\antgen{}.\acc{} \partic{} Marial.\nom{} \decl{}-come\\
\glt `Marial's boy is coming.'
\end{exe}

\begin{exe}
\ex\label{BorGenEx}\langinfobreak{Oromo (Boraana)}{Eastern Cushitic; Ethiopia}{\citealp[35]{Stroomer:1995}}
\gll mina ciif\textbf{-aa}\\
house chief-\gen{}\\
\glt `the house of the chief'
\end{exe} 

\begin{exe}\ex\label{CocGenEx}\langinfobreak{Cocopa}{Yuman; California}{\citealp[165]{Crawford:1966}}
\gll\textbf{ap\'a} n\textsuperscript{y}aw\'a\\
man.\acc{} house.\acc{}\\
\glt `the man's house'
\end{exe}


Apart from the dependent-marking of attributive possessors, head-marking patterns are also found with some marked"=S languages.
Dinka\il{Dinka (Agar)}, as demonstrated above (\ref{DinGenEx}), not only uses the overtly coded Nominative\is{case!individual forms!nominative}  to mark the possessor in this context. 
There is also a special case-form labeled as the `antigenitive' (gloss: \antgen{}) on the possessee. 
A case-form like this is found in Arbore\il{Arbore} as well (\ref{ArbAntiGen}).

%\pagebreak

\begin{exe}\ex\label{ArbAntiGen}\langinfobreak{Arbore}{Eastern Cushitic; Ethiopia}{\citealp[151]{Hayward:1984}}
\gll gaydan\textbf{-ti} g\'eer\\
 hoe-\antgen{} old\_man\\
 \glt `(the) old man's hoe'
 \end{exe}\is{case!individual forms!antigenitive|)} 
 
Some languages of the Nilo-Saharan phylum have a more complex construction for attributive possession than the possessor and possessee in their respective case-forms.
These complex constructions insert an additional marker between the two nouns as illustrated in (\ref{TenGenEx}). 
When the marker serves as a preposition in other contexts, it is usually glossed correspondingly. 
Otherwise it may simply be referred to as particle, or with more language-specific terminology like the `associative marker' (glossed as \am{}) in Tennet\il{Tennet} (\ref{TenGenEx}).
I will uniformly refer to such markers as possessive marker (\poss{}), even if information on other uses is provided in the grammar.
If attributive possessors are in the zero-coded form of a noun, but combined with such a particle, one might argue that this particle serves as a kind of case-marking in a wider sense, and thus the noun is not zero-coded.
However, these markers do appear with both zero-coded and case-marked forms of a noun.
As such, the two systems (i.e. case-marking proper and particles) seem to be independent of one another, at least in the present sample (compare Section~\ref{ExtraNilo}, Table~\ref{NiloExtra}). 
If a possessive marker (\poss{}) is used in the relevant construction, this information will be provided in addition to the case-form of the attributive possessor in the discussion of the data.

%\pagebreak
\pagebreak

\begin{exe} 
\ex\label{TenGenEx}\langinfobreak{Tennet}{Surmic; Sudan}{\citealp[230]{Randal:1998}}
\gll\textipa{mana} \textipa{c\'I} \textipa{ongol\textbf{-\=*o}} \\
field \am{} elephant-\gen{}\\
\glt `elephant's field'  
\end{exe}
 
A\is{possession!alienable versus inalienable|(} distinction in encoding of alienable and inalienable possession is made by a few languages of the sample. 
As far as data are available, I will provide examples from both contexts.\is{possession!alienable versus inalienable|)}\is{possession!attributive|)}

For the next two roles, namely terms of address\is{terms of address} and citation form\is{citation form}, far fewer different patterns are to be expected since these are basically one word (or at least one phrase) items. 
In addition, neither role is treated explicitly in most grammars, and when they are, just in passing.
The\is{terms of address|(} basic distinction for terms of address is that between a dedicated form, often called Vocative\is{case!individual forms!vocative}, as in Gamo\il{Gamo} (\ref{GamVocEx}), or encoding via the basic zero-coded form of a noun, like in Nias\il{Nias} (\ref{NiaVocEx}). 
Other case-forms are rarely employed in this context. 
If other case-forms do occur in this context, they are usually restricted to a certain set of nouns. 
Free vocative\is{case!individual forms!vocative} particles are seldom found in the languages of my sample. 
If they do occur, these markers are optional and the noun can also be used without them to the same effect.

\begin{exe}\ex\label{GamVocEx}\langinfobreak{Gamo}{Omotic; Ethiopia}{\citep[282, 283]{Hompo:1990}}
\gll danna\textbf{-wu}!\\
judge-\voc{}\\
\glt `Oh, judge!'
\end{exe}

\begin{exe}\ex\label{NiaVocEx}\langinfobreak{Nias}{Sundic; Indonesia}{\citealp[59]{Brown:2001}} 
\gll Haiya ni-wa\"o-u ga,  \textbf{am\'a}?\\
what \pass{}-say-2\sg{}.\poss{} here father\\
\glt`What is it you want here, Sir?'
\end{exe}
\is{terms of address|)}

With\is{citation form|(} respect to the citation form of a noun, most grammars simply list this as one of the functions of the zero-coded case-form, without providing examples or discussing how this function was established in the research (e.g. whether it is a form actually used by the speakers, or something introduced by the linguist for some theoretical or practical reasons). 
This form seems to be most strongly correlated with the zero-coded case in the marked"=S languages. 
The few cases in which alternative forms for this function exist are usually an even more reduced form, such as a nominal stem.\is{citation form|)}

\section{Nilo-Saharan}\label{ExtraNilo}

Most\is{possession!attributive|(} Nilo-Saharan languages do not have a special case-form to mark attributive possessors.
Instead they usually use the accusative\is{case!individual forms!accusative} case-form in this context. 
The possessor is either just juxtaposed to the possessee, or additional material in form of a particle or preposition intervenes between the two nouns.\is{possession!attributive|)} 
Only in one language, Dinka\il{Dinka (Agar)}, is the Nominative case used to encode attributive possessors (as seen in (\ref{DinGenEx})). 
In the remaining languages, a special genitive case exists that is employed in this context.
Most grammars do not provide any information on terms of address\is{terms of address}, which is probably due to the lack of a dedicated form or construction for this context.
Only for Turkana\il{Turkana} is a special Vocative\is{case!individual forms!vocative} case mentioned. 
Finally, all Nilo-Saharan languages in my sample use the zero-coded accusative\is{case!individual forms!accusative} as the citation form of a noun.


%%%%%%%%%%%%%%%Genitive%%%%%%%%%%%%%%%%%%%%%%%%%%

\is{possession!attributive|(}In Datooga\il{Datooga} the possessed noun and the possessor are simply juxtaposed without any other overt marking of the possessive relationship. 
The possessor follows the possessee and is in the Accusative\is{case!individual forms!accusative} case (\ref{DatGen}), the possessee is marked for whichever grammatical relation it bears in the given sentence.

\begin{exe}\ex\label{DatGen}\langinfo{Datooga}{Nilotic; Tanzania}{\citealp[178, 179]{Kiessling:2007}}
\begin{xlist}
\ex\gll q\'a-b\'ar m\`ayd\'a d\^eed\textsubring{a}\\
3\sg{}-beat calf.\acc{}.\cs{} cow.\acc{}\\
\glt `He beat the cow's calf.'
\ex\gll q\'a-b\'ar m\'ayd\'a d\^eed\textsubring{a}\\
3\sg{}.beat calf.\nom{}.\cs{} cow.\acc{}\\
\glt `The cow's calf beat (him/her).'
\end{xlist}
\end{exe}

Similarly in Maa\il{Maa} (\ref{MaaGen}) and Nandi\il{Nandi} (\ref{NanGen}), possessors are in the accusative\is{case!individual forms!accusative} case and preceded by the possessee. 
However, the possessive relation is additionally marked by some extra material, namely, the so-called Genitive particle \emph{le/lo/loo} in Maa\il{Maa}, which inflects for gender and number of the possessor \citepalias[213]{Tucker:1955}, and a similar particle \emph{\textipa{a:p}} in Nandi\il{Nandi}. 

\begin{exe}\ex\label{MaaGen}\langinfo{Maa}{Nilotic; Kenya}{\citepalias[213]{Tucker:1955}}
\begin{xlist}
\ex\gll \'e-\'\i pot olcor\'e l\'o lay\'\i\'on\`i\\
3\sg{}-call friend.\acc{} \poss{} boy.\acc{}\\
\glt `He calls the friend of the boy.'

\ex\gll \'e-\'\i pot olc\'ore l\'o lay\'\i\'on\`i\\
3\sg{}-call friend.\nom{} \poss{} boy.\acc{}\\
\glt `The friend of the boy calls him.'
\end{xlist}
\end{exe}

\begin{exe}\ex\label{NanGen}\langinfobreak{Nandi}{Nilotic; Kenya}{\citealp[69]{Creider:1989}}
\gll \textipa{\'Imp\'are\^e:t} \textipa{a:p} \textipa{kipe:t}\\
field \poss{} Kibet\\
\glt `Kibet's field'
\end{exe}


In Murle\il{Murle}, attributive possessors are in a special case-form, dubbed as `Genitive' in accordance with traditional Latinate case naming conventions. 
If a modifier follows the Genitive noun, the case-ending is dropped \citep[53--54]{Arensen:1982}. 
As in the other Nilo-Saharan languages, possessors are preceded by their possessee. 
And as in Maa\il{Maa} and Nandi\il{Nandi} a particle (\emph{ci} or \emph{o}) intervenes between the two nouns (\ref{MurGen}).
This particle is also used to introduce relative clauses. 
Tennet\il{Tennet} (\ref{TenGen}) exhibits a similar pattern. 
Attributive possessors are in the Genitive case and are preceded by possessees, and the so-called `associative marker' intervenes between the two nouns. 
Note that in Tennet\il{Tennet} the Genitive case is identical to the Nominative\is{case!individual forms!nominative}  for some nouns \citep[225]{Randal:1998}. 

\begin{exe}
\ex\label{MurGen}\langinfobreak{Murle}{Surmic; Sudan}{\citealp[108]{Arensen:1982}}
\gll\textipa{cirlil-i} \textipa{agam} \textipa{idiN} \textipa{ci} \textipa{Naa\textbf{-o}}\\
kite-\nom{} grab meat \relativ{} woman-\gen{}\\
\glt `The kite grabs the meat of the woman' 
\end{exe}

\begin{exe}
\ex\label{TenGen}\langinfobreak{Tennet}{Surmic; Sudan}{\citealp[230]{Randal:1998}}
\gll\textipa{mana} \textipa{c\'I} \textipa{ongol\textbf{-\=*o}} \\
field \am{} elephant-\gen{}\\
\glt `elephant's field'  
\end{exe}

Turkana\il{Turkana} (\ref{TurGen}) also has a special Genitive\is{case!individual forms!genitive} case-form to encode most attributive possessors.
In this construction, as exemplified by \citet[266--268]{Dimmendaal:1982}, the possessee precedes the possessor, and a particle/preposition glossed `of' is inserted between the two nouns (\ref{TurGen}a). 
As exemplified below, the respective Accusative\is{case!individual forms!accusative} (\ref{TurGen}b) and Nominative (\ref{TurGen}c) case-forms differ in tone\is{case-marking!via tone} from the Genitive.
With kinship terms, a slightly different construction is used \citep[340]{Dimmendaal:1982}. 
The basic structure is similar to the construction discussed above, but the possessor is in the Accusative\is{case!individual forms!accusative} case and obligatorily followed by a pronominal (\ref{TurGenKin}). 
Also, in this construction a different particle/preposition is used.
\footnote{The noun \emph{\`\i-to\`o} `mother' is the only kinship term that uses the general possessive construction with the particle \emph{\`a} and the possessor in the Genitive case. However, it is still followed by the obligatory pronominal found with other kinship terms \citep[240]{Dimmendaal:1982}.} 

\pagebreak

\begin{exe}\ex\label{TurGen}\langinfo{Turkana}{Nilotic; Kenya}{\citealp[267, 384, 167]{Dimmendaal:1982}}
\begin{xlist}
\ex\gll\textipa{E-mu\textltailn En\`{}} \textipa{\`a} \textbf{\textipa{\`a-It\`E}} \textipa{na\`{}}\\
colour of \NC{}-cow.\gen{} this\\
\glt`the colour of this cow'

\ex\gll \textipa{\`e-l\`ep\`i\`{}} \textbf{\textipa{a-ItE\`{}}} \textipa{caaap}, \textipa{caaap}\\
3-milk-\asp{} \NC{}-cow.\acc{} \ideoph{} \ideoph{}\\
\glt`She milked the cow.' %[384]

\ex\gll \textipa{\`E-\`ON\`OrI-aa-n-\`a} \textbf{\textipa{a-\`It\`E}} \textipa{na\`{}}\\
3-brown-\hab{}-\sg{}-\Verb{} \NC{}-cow.\nom{} this\\
\glt `This cow is brownish.'
\end{xlist}
\end{exe}

\begin{exe}\ex\label{TurGenKin}\langinfo{Turkana}{}{\citealp[340]{Dimmendaal:1982}}
\begin{xlist}
\ex\gll\textipa{\`e-ya\`{}} \textipa{kEN\`{}} \textipa{k\`a} \textbf{\textipa{\`a-pa\`{}}} \textipa{kaN\`{}}\\
aunt.\acc{} his with \NC{}-father.\acc{} my\\
\glt `My father's aunt' %[340]

\ex\gll \textipa{a-m\`ot\r*{i}} \textipa{k\`a} \textbf{\textipa{\`e-ya\`{}}} \textipa{kaN\`{}}\\
pot.\acc{} with \NC{}-aunt.\acc{} my\\
\glt `my aunt's pot' %[340]
\end{xlist}
\end{exe}

Dinka\il{Dinka (Agar)} is exceptional compared with the other Nilo-Saharan languages in using the same case-form to encode (post-verbal) subjects and attributive possessors. 
In his (1991) paper, Andersen refers to this case-form as Genitive\is{case!individual forms!genitive} due to its property of marking adnominal possessors. 
However, the use to encode subjects (even though only if non-topical) sets this case apart from the Genitives of other Nilo-Saharan languages, which are not used to encode subjects at all. 
Another difference between Dinka\il{Dinka (Agar)} and other Nilo-Saharan languages is the special case-marking of the possessee in attributive possessive constructions. 
The possessee in these contexts is marked in the so-called `Antigenitive' (\ref{DinGen}a).
If this possessed noun serves as a possessor itself (\ref{DinGen}d) or is a post-verbal subject the special tonal form of Antigenitive-Nominative (or Antigenitive-Genitive in Andersen's terms) is used.  

\begin{exe}
\ex\label{DinGen}\langinfo{Dinka (Agar)}{Nilotic; Sudan}{\citealp[273]{Andersen:1991}}
\begin{xlist}
\raggedright
\ex\gll\textipa{dh\"*{\'O}ON} \textipa{\"*{\`e}} \textbf{\textipa{m\~*{\'a}ri\~*{\`a}al}} \textipa{\~*{\`a}-b\"*{\'O}}\\
boy.\antgen{}.\acc{} \partic{} Marial.\nom{} \decl{}-come\\
\glt `Marial's boy is coming.'

\ex\gll\textipa{m\~*{\`a}ri\~*{\`a}al} \textipa{\~*{\`a}-b\"*{\'O}}\\
 Marial.\acc{} \decl{}-come\\
\glt `Marial is coming.'

\ex\gll\textipa{dh\"*{\`O}Ok} \textipa{\~*{\`a}-th\~*{\'E}Et} \textipa{m\~*{\'a}ri\~*{\`a}al}\\
 boy.\acc{} \decl{}-beat.\nts{} Marial.\nom{}\\
\glt`Marial is beating the boy.'
 
\ex\gll\textipa{m\~*{\`o}c} \textipa{\~*{\`a}-y\"*{\^e}ep} \textipa{\"*{\`e}} \textipa{y\"*{\'e}m} \textipa{\"*{\`e}} \textipa{dh\"*{\`O}ON} \textbf{\textipa{\"*{\`e}}} \textipa{m\~*{\'a}ri\~*{\`a}al}\\
man.\acc{} \decl{}-cut.\antip{} \prep{} axe.\antgen{} \partic{} boy.\antgen{}.\nom{} \partic{} Marial.\nom{}\\
\glt `The man is cutting with Marial's boy's axe.' 
\end{xlist}
\end{exe}\is{possession!attributive|)}

%%%%%%%%%%%%%%%%%%%%Vocative:%%%%%%%%%%%%%%%%%%%%%
Special\is{terms of address|(} forms of address are not common in Nilo-Saharan languages. 
However, the whole topic of address is treated only scantily in the grammars, if treated at all.
This is not special about this language family but actually holds true for most grammars of the world's languages.
For Datooga\il{Datooga}, an example using a form of address is provided. 
The Accusative\is{case!individual forms!accusative} is used in this context (\ref{DatVoc}), like supposedly in most Nilo-Saharan marked"=S languages, although at the present moment this remains unknown. 
In contrast, Turkana\il{Turkana} has a special Vocative\is{case!individual forms!vocative} case-form, which is discussed by \citet[67, 268--269]{Dimmendaal:1982}.
The tonal shape of nouns used in address (\ref{TurVoc}) differs from other case-forms such as the Accusative\is{case!individual forms!accusative} for example, as exemplified in (\ref{TurNonVoc}).

\begin{exe}
\ex\label{DatVoc}\langinfobreak{Datooga}{}{\citealp[171]{Kiessling:2007}}
\raggedright
\gll gw\`a-y\'ee\textesh\`a h\'eew-\`\i{} b\'all\'aand\`a q\'amn\`aa g\`ay-d\'a-l\'\i k-\textltailn\`\i\\
3\sg{}-say bull.\nom{}-\dem{}.near.\sg{} boy.\acc{} now \fut{}-1\sg{}-swallow-2\sg{}.\obj{}\\
\glt `This bull said: ``Child, I'm going to swallow you now !'' '
\end{exe}


\begin{exe}\ex\label{TurVoc}\langinfo{Turkana}{}{\citealp[268, 269]{Dimmendaal:1982}}
\begin{xlist}
\ex\textipa{N\`I-dE\`{}}\glt `children!'

\ex\textipa{\`E-k\`a-tuk-\`o-n\r*i}\glt	`chief!'
\end{xlist}
\end{exe}

%{Turkana\il{Turkana}} \citep[75]{Dimmendaal:1982}
\begin{exe}
\ex\label{TurNonVoc}\langinfobreak{Turkana}{}{\citealp[75]{Dimmendaal:1982}}
\gll\textipa{NI-dE\'{}} \textipa{omwOn\'{}}\\
\NC{}-children.\acc{} four\\
\glt `There are four children.' %check reference
\end{exe}\is{terms of address|)}

%%%%%%%%%%%%%%%%%%Citation form:%%%%%%%%%%%%%%%%%%%
Finally\is{citation form|(}, all languages use the citation form as the Accusative\is{case!individual forms!accusative} case (or vice-versa).
In languages that mark the distinction between nominative and accusative\is{case!individual forms!accusative} via a tonal contrast, the tonal\is{case-marking!via tone} shape of a noun in its citation form is taken as one criterion to determine the Accusative\is{case!individual forms!accusative} as the basic form and the Nominative as the derived form.
This is discussed quite extensively for Dinka\il{Dinka (Agar)} \citep[273]{Andersen:1991} and Turkana\il{Turkana} \citep[66]{Dimmendaal:1982}.\is{citation form|)}

%%%%%%%%%%%%%%Other%%%%%%%%%%%%%%%%%%%%%%%%%%

%Dinka\il{Dinka (Agar)}:
%Case-formation discussed by \citet[9ff.]{Andersen:2002}

Table~\ref{NiloExtra} summarizes the data on non-clause-level case-marking in the Nilo-Saharan languages.
Concerning\is{possession!attributive|(} the possessor, all possible combinations of zero-coded Accusative\is{case!individual forms!accusative} vs. overtly coded Genitive and the presence vs. absence of a possessive marker in the attributive possessive construction are attested. 
Any kind of particle- or preposition-like coding is abbreviated as `\poss{}' in the tables in this chapter.  
Most languages use some type of overt marking, either case or possessive marker or both, in this context. 
Only Datooga\il{Datooga} has no possessive marker and no overt case-marking for this role. 
When considering only the actual case-marking, the data are split evenly between Accusative\is{case!individual forms!accusative} and Genitive forms, with a special use of the Nominative for this function in Dinka\il{Dinka (Agar)}.\is{possession!attributive|)}
On forms of address\is{terms of address}, little information can be found for the Nilo-Saharan languages. 
All that can be said is that there is some variation between the use of the Accusative\is{case!individual forms!accusative} (Datooga\il{Datooga}) and a special Vocative\is{case!individual forms!vocative} form (Turkana\il{Turkana}).
The citation form\is{citation form} is identical to the accusative\is{case!individual forms!accusative} for all languages.

\begin{table}[h,b,t]
\centering
\begin{tabular}{lccc}
\hline \hline
\bfseries language&\bfseries Possessor &\bfseries Address  &\bfseries Citation\\
\hline
Datooga\il{Datooga}&\acc{}&\acc{}&\acc{}\\
%\hdashline
Dinka\il{Dinka (Agar)}&\poss{} \textbf{\nom{}}&{-}&\acc{}\\
%\hdashline
Maa\il{Maa}& \poss{} \acc{}&{-}&{\acc{}}\\
%\hdashline
Murle\il{Murle}&\textbf{\gen{}}&{-}&\acc{}\\
%\hdashline
Nandi\il{Nandi}& \poss{} \acc{}&{-}&{\acc{}}\\
%\hdashline
Tennet\il{Tennet}&\textbf{\gen{}}&{-}&{\acc{}}\\
%\hdashline
Turkana\il{Turkana}& \poss{} \textbf{\gen{}}/ \poss{} \acc{}&\textbf{\voc{}}&\acc{}\\
\hline \hline
\end{tabular}
\caption{Non-clause-level case-marking in Nilo-Saharan}\label{NiloExtra}%\\
\end{table}


\section{Afro-Asiatic}\label{ExtraAfro}

Almost\is{possession!attributive|(} all Afro-Asiatic marked"=nominative languages have a special, overtly marked, genitive case-form to encode attributive possessors.
Apart from case-marking the possessive relation is expressed through juxtaposition of possessor and possessee without any additional marking through prepositions, particles or the like.\is{possession!attributive|)} 
Distinct vocative\is{terms of address} forms are found in quite a few languages of the sample. 
Moreover, the relation between citation form\is{citation form} of a noun and the accusative\is{case!individual forms!accusative} is not as straightforward as in the Nilo-Saharan languages.

%%%%%%%%%%%%Genitive%%%%%%%%%%%%%%%%%%%%%

In\is{possession!attributive|(} both dialects of Oromo discussed in this study, attributive possessors are in the Genitive case-form. 
This is illustrated by examples from Harar\il{Oromo (Harar)} (\ref{HarGen}) and Boraana\il{Oromo (Boraana)} (\ref{BorGen}a). 
\citet[50]{Owens:1982} provides additional data on focused possessors in Boraana\il{Oromo (Boraana)} Oromo, which like other focused constituents are in the Accusative\is{case!individual forms!accusative} case (\ref{BorGen}b). 


\begin{exe}\ex\label{HarGen}\langinfobreak{Oromo (Harar)}{Eastern Cushitic; Ethiopia}{\citealp[103]{Owens:1985}}
\gll\textipa{bif-n\'Ii} \textbf{\textipa{s\'ar\'ee}} \textipa{fakk\'ootaa}\\
color-\nom{} dog.\gen{} ugly\\
\glt `The dog's color is ugly.'
\end{exe}

\begin{exe}\ex\label{BorGen}\langinfo{Oromo (Boraana)}{Eastern Cushitic; Ethiopia}{\citealp[35]{Stroomer:1995}, \citealp[50]{Owens:1982}}
\begin{xlist}
\ex\gll mina ciif\textbf{-aa}\\
house chief-\gen{}\\
\glt `the house of the chief'

\ex\gll \textbf{nam} s\`un m\`ini \`isa dans\`a\\
man.\acc{} \dem{} house.\acc{} 3\sg{}.\mas{}.\poss{} good\\
\glt `As for the man, his house is good.' %\citep[50]{Owens:1982}
\end{xlist}
\end{exe} 

In Gamo\il{Gamo} (\ref{GamGen}), K'abeena\il{K'abeena} (\ref{KabGen}), Wolaytta\il{Wolaytta} (\ref{WolGen}), and Zayse\il{Zayse} (\ref{ZayGen}), the genitive\is{case!individual forms!genitive} case is used  to encode adnominal possessors as well. 
In all the languages, nominal possessors precede their possessees.
Except for the second degree of definiteness, the Gamo\il{Gamo} Genitive\is{case!individual forms!genitive} is identical to the Accusative\is{case!individual forms!accusative} case \citep[380]{Hompo:1990}.
\footnote{Gamo\il{Gamo} distinguishes between four degrees of definiteness. 
The exact usage is not clear to \citet[367]{Hompo:1990}, but the following description seem to hold more or less.
Degree 1 and 2 are indefinite, degree 3 is specific to definite and degree 4 is definite. 
The forms used to encode degree 1--3 of definiteness are interwoven with the markers of subject- and object-case.}
In Wolaytta\il{Wolaytta},  the Genitive\is{case!individual forms!genitive} case has two different forms according to \citet[217--218]{Lamberti:1997}. 
Either the bare noun stem is used (\ref{WolGen}a, b) or it is derived from the Accusative\is{case!individual forms!accusative} by lengthening the final vowel of that case-form (\ref{WolGen}c, d).
%The Zayse\il{Zayse} Genitive is marked via tone. 
%\emph{zikk\'ola} `eagle' and \emph{paNg\'e} `wing'.

%\pagebreak
\begin{exe}\ex\label{GamGen}\langinfo{Gamo}{Omotic; Ethiopia}{\citealp[368, 375]{Hompo:1990}}
\begin{xlist}
\ex\gll ka\`si giggiso-i ma\v c'\v c'\textbf{-\'a} o\`so-ko\\
food.\acc{} preparing-\nom{} woman-\gen{} work-\cop{}\\
\glt `Preparing food is (a) woman's task.'

\ex\gll issi ma\v c'\v c\textbf{-ai} iz-a goss-a-d-us\\
one woman-\nom{} 3\sg{}.\mas{}-\acc{} madden-\persm-\tns{}-\complx{}\\
\glt `A woman made him crazy.'
\end{xlist}
\end{exe}

\begin{exe}\ex\label{KabGen}\langinfo{K'abeena}{Eastern Cushitic; Ethiopia}{\citealp[115]{Crass:2005}} 
\begin{xlist} 
\ex\gll \textbf{manci}-'i bak'\'ulcut\textsuperscript{i} ba'o\\
husband.\gen{}-1\sg{}.\poss{} mule.\nom{} disappear.\pfv{}.3\sg{}.\fem{}\\
\glt `My husband's mule has disappeared.'\\
 original translation: `Das Maultier meines Ehemanns is verschwunden.'

\ex\gll \textbf{manco}-'i maalda mi 'aa'iyo?\\
wife.\gen{}-1\sg{}.\poss{} silver\_bracelet.\acc{} who.\nom{} take.\pfv{}.3\sg{}.\mas{}\\
\glt `Who took my wife's silver bracelet?'\\
 original translation: `Wer/Was hat das Silberarmband meiner Ehefrau weggenommen?'
\end{xlist}
\end{exe}  


\begin{exe}\ex\label{WolGen}\langinfo{Wolaytta}{Omotic; Ethiopia}{\citealp[217, 218]{Lamberti:1997}}
\begin{xlist}
\ex\gll \textbf{aliy\textsuperscript{a}} keetta\\
Ali.\gen{} house\\
\glt `Ali's house'

\ex\gll \textbf{kaawuwa} keetta-ta\\
king.\gen{} house-\pl{}\\
\glt `some houses of the king'

\ex\gll kaawuw\textbf{-aa} keetta-ta\\
king-\gen{} house-\pl{}\\
\glt `some houses of the king'

\ex\gll aliy\textbf{-aa} kusshiy\textsuperscript{a}\\
Ali-\gen{} hand\\
\glt `Ali's hand'
\end{xlist}
\end{exe}

%\pagebreak
\enlargethispage{2\baselineskip}
\begin{exe} \ex\label{ZayGen}\langinfobreak{Zayse}{Omotic; Ethiopia}{\citealp[251]{Hayward:1990}}
\gll \textbf{zikk\'ol\'a} \textipa{paNge}\\
eagle.\gen{} wing\\
\glt `(an) eagle's wing'
\end{exe}

{Arbore\il{Arbore}} is the only Afro-Asiatic language in my sample that exhibits a different pattern in the attributive possessive construction.
While the possessor is in the zero-coded Accusative\is{case!individual forms!accusative} case-form, the possessee is in the so-called Antigenitive\is{case!individual forms!antigenitive} form (\ref{ArbGen}).
In addition, the ordering of possessor and possessee is reversed in comparison to the other Afro-Asiatic languages. 
Instead of preceding the possessee, the possessor follows it.

%masculine possessee:
 
%\pagebreak
\begin{exe}\ex\label{ArbGen}\langinfo{Arbore}{Eastern Cushitic; Ethiopia}{\citealp[151]{Hayward:1984}}
\begin{xlist}
\ex\gll gaydan-ti \textbf{g\'eer}\\
 hoe-\antgen{} old\_man\\
\glt  `(the) old man's hoe'
 
%\ex\gll \textipa{\textglotstop aHan-ti} \textipa{n\'aag}\\
%calabash-\antgen{} boy\\
%`(the) boy's calabash.'

\ex\gll hiki\v c-i \textbf{h\'o\v g\v gattu(-t)}\\
axe-\antgen{} labourer\\
\glt `(the) labourer's axe

\ex\gll k'\'ub-a \textbf{neek'}\\
forelimb-\antgen{}  lion\\
\glt `(the) forelimb of a lion'
\end{xlist}
\end{exe}\is{possession!attributive|)}

%feminine possessee:
%{Arbore\il{Arbore}} \citep[152]{Hayward:1984}
%\begin{exe}\ex\label{ArbGenF} 
%\begin{xlist} \ex\gll beek'-e n\'aag(et)\\
%wound-e girl\\
%`(the) girl's wound'

%\ex\gll \textglotstop ilko \v c\'aar\\
%teeth leopard\\
%`(the) leopard's teeth' 
%\end{xlist}
%\end{exe}

%?
%\ea\gll se\textglotstop e \textglotstop \'argeri\\
%cow \textglotstop argeri.\abs{}\\
%`Argeri's cow.'
%\end{exe} \citet[236]{Hayward:1984}

%%%%%%%%%%%%%%%%%Vocative%%%%%%%%%%%%%%%

Gamo\il{Gamo}\is{terms of address|(} (\ref{GamVoc}) and Wolaytta\il{Wolaytta} (\ref{WolVoc}) have vocative\is{case!individual forms!vocative} case-affixes to mark terms of address. 
The endings vary with respect to number and gender of the addressee in Gamo\il{Gamo};  \emph{-o/-wu} is used with masculine or neuter nouns, \emph{-e} for feminine nouns and \emph{-t-o} for the plural \citep[382--383]{Hompo:1990}.
Wolaytta\il{Wolaytta} also has two different forms, namely \emph{-ow} and \emph{-ey}. 
Which factors influence the choice of one over the other is, however, not discussed by \citet[66]{Lamberti:1997}.

\begin{exe}\ex\label{GamVoc}\langinfo{Gamo}{}{\citealp[282, 283]{Hompo:1990}}
\begin{xlist}
\ex\gll danna-wu!\\
judge-\voc{}\\
\glt `Oh, judge!'

\ex\gll addez-o\\
man-\voc{}\\
\glt `Hey, man!'
\end{xlist}
\end{exe}

\begin{exe}\ex\label{WolVoc}\langinfobreak{Wolaytta}{}{\citealp[209]{Lamberti:1997}}
\gll aliy-ow, ta mat'aafa ekka!\\
Ali-\voc{} my book take\\
\glt `Ali, take my book.'
\end{exe}

For K'abeena\il{K'abeena} terms of address, a quite complex scenario is described by \citet{Crass:2005}.
The Vocative\is{case!individual forms!vocative} form is identical to the {Accusative\is{case!individual forms!accusative}} with personal names.
With nouns referring to relatives, the {Vocative}\is{case!individual forms!vocative} is identical to the citation form\is{citation form}, while with other nouns it is either identical to the {Genitive}\is{case!individual forms!genitive}, or else the Vocative\is{case!individual forms!vocative} is derived by affixation of the suffix \emph{-o}, and for at least one noun the {Vocative}\is{case!individual forms!vocative} is identical to the root. 
It is possible to distinguish the {Vocative}\is{case!individual forms!vocative} from identical case-forms by means of the interjection \emph{koo} (masculine) or \emph{tee}  (feminine) before the noun  \citep[95--96]{Crass:2005}.\is{terms of address|)} 

%%%%%%%%%%%%%%%%%%%Citation%%%%%%%%%%%%%

Gamo\il{Gamo}\is{citation form|(} has a quite complex system of case-marking, which distinguishes between four degrees of definiteness/individuation. 
The citation form corresponds to the so-called `first degree' Accusative\is{case!individual forms!accusative}, which is the least complex form of the paradigm. 
For this form, the Accusative and Genitive\is{case!individual forms!genitive} are the same \citep[370]{Hompo:1990}.
The Arbore\il{Arbore} citation form is identical to the basic form ({Accusative}\is{case!individual forms!accusative}) for most nouns. 
For some nouns, however, the citation form is a reduced version of the Accusative\is{case!individual forms!accusative}. 
According to \citet[133]{Hayward:1984} these nouns drop the second of two final consonants or reduce it to a glottal stop when used in isolation.
In K'abeena\il{K'abeena}, different forms of a noun are used in citation as well. 
The {Accusative\is{case!individual forms!accusative}} is the form used as the most basic pattern \citep[61]{Crass:2005}. 
For proper names the citation form deviates from the {Accusative} in some cases. %Note that the titles of \citeauthor{Crass:2005}'s texts \citeyearpar[332ff.]{Crass:2005} are all in \textsc{accusative} case. 
\citet[67--68]{Lamberti:1997}, in their grammar of  Wolaytta\il{Wolaytta}, list an `Absolutive' form and an `Object' case of a noun (distinct from other case-forms such as the Nominative). 
In some of the noun classes those two forms differ. 
This difference consists of the following contrast: the so-called Absolutive form has an voiceless vowel as its last segment, while the Object form has the voiced counterpart. 
The Absolutive form seems to refer to nouns used in isolation, i.e. the citation form, while the Object form corresponds to the Accusative\is{case!individual forms!accusative} in the traditional sense. 
\citeauthor{Lamberti:1997} do not comment on this alternation, but in the phonology section they state that unstressed final vowels always seem to be devoiced (51--52), thus the variation between Absolutive and Object form might be due to its context (especially stress assignment, of which is only little understood so far).\is{citation form|)} 


%%%%%%%%%%%%Other%%%%%%%%%%%%%%%%%%%

%Gamo\il{Gamo}:
%Case-formation: Oblique case ending are attached to the 1st degree accusative/first degree genitive \citep[370]{Hompo:1990}, however the Nominative (as well as the Accusative/Genitive) is derived from the root of the noun. 

A summary of all Afro-Asiatic marked"=S languages is provided in Table~\ref{AfroExtra}.
Most\is{possession!attributive|(} languages have an overtly coded genitive\is{case!individual forms!genitive} case to mark attributive possessors. 
The only exception is Arbore\il{Arbore}, which uses the zero-coded Accusative\is{case!individual forms!accusative} for this purpose.\is{possession!attributive|)} 
Three languages have a vocative\is{case!individual forms!vocative} case-form of the noun (Gamo\il{Gamo}, K'abeena\il{K'abeena} and Wolaytta\il{Wolaytta}), but in K'abeena\il{K'abeena} this form can only be distinguished from other case-forms for a subset of nouns\is{terms of address}. 
All languages have a citation form that is identical to the accusative\is{case!individual forms!accusative} case at least for some nouns. 
In K'abeena\il{K'abeena} and Arbore\il{Arbore}, sometimes the citation form\is{citation form} is a reduced variety of the Accusative\is{case!individual forms!accusative}. 
The exact distribution of the different forms is, however, poorly understood. 
Thus the possibility cannot be ruled out that there is no actual paradigmatic contrast between the two forms, and the variation is rather triggered by other factors such as morphophonological processes or prosody.

\begin{table}[b,t,h]
\centering
\begin{tabular}{lccc}
\hline \hline
\bfseries language&\bfseries Possessor &\bfseries Address  &\bfseries Citation\\
\hline
Arbore\il{Arbore}&{\acc{}}&{-}&{\acc{}}/reduced form\\
%\hdashline
Gamo\il{Gamo}&\textbf{\gen{}}&\textbf{\voc{}}&\acc{}\\
%\hdashline
K'abeena\il{K'abeena}&\textbf{\gen{}}&\acc{}/\textbf{\voc{}}/\textbf{\gen{}}&{\acc{}}/reduced form\\
%\hdashline
Oromo (Boraana\il{Oromo (Boraana)})&\textbf{\gen{}}&{-}&{\acc{}}\\
%\hdashline
Oromo (Harar\il{Oromo (Harar)})&\textbf{\gen{}}&{-}&{\acc{}}\\
%\hdashline
Wolaytta\il{Wolaytta}&\textbf{\gen{}}&\textbf{\voc{}}&\acc{}\\
%\hdashline
Zayse\il{Zayse}& \textbf{\gen{}}&{-}&\acc{}\\
\hline \hline
\end{tabular}
\caption{Non-clause-level case-marking in Afro-Asiatic}\label{AfroExtra}
\end{table}


%%%%%%%%%%%%%%%%%%%%%%%%%%%%%%%%%%%%%%%%%%
\section{Pacific}\label{ExtraPac}
%%%%%%%%%%%%%%%%%%%%%%%%%%%%%%%%%%%%%%%%%%

\is{possession!attributive|(}Attributive possessors are expressed by very different constructions in each of the marked"=S languages of the Pacific region.\is{possession!attributive|)}
Terms of address\is{terms of address} on the other hand are uniformly in the zero-coded form of a noun.
This is also the form usually employed in citation. 
However, in Nias\il{Nias} there is some variation between speakers as well as different nouns.
Both forms of a noun, the Mutated and Unmutated form, occur as citation forms\is{citation form}.
 

%%%%%%%%%%%%%%Genitive%%%%%%%%%%%%%%%%%%%%

\is{possession!attributive|(}There are two constructions to express nominal possessors in Aji\"e\il{Aji\"e}; one for inalienable and one for alienable possession.
Inalienable\is{possession!alienable versus inalienable} possession is expressed by mere juxtaposition of the possessor in the zero-coded form and the possessed item, with the possessor following the possessee (\ref{AijGen}a, b). In alienable possession, the possessor is preceded by the particle \emph{i}, and also follows the possessee (\ref{AijGen}c).
Since the morphological means of expression is identical to way the Nominative is encoded, this particle can actually be considered as part of a case paradigm, unlike similar markers in the Nilo-Saharan languages. 
I will therefore treat this particle, which is glossed as `of' by \citet{Lichtenberk:1978}, as a Genitive\is{case!individual forms!genitive} case-marker\is{case-marking!via adposition} (and have altered the glossing respectively), just as the particle \emph{na} is glossed as Nominative case. 

%\pagebreak
\begin{exe}\ex\label{AijGen}\langinfo{Aji\"e}{Oceanic; New Caledonia}{\citealp[85, 86]{Lichtenberk:1978}}
\begin{xlist}
\ex\gll\textipa{pwe} \textbf{\textipa{bwE\textglotstop}}\\
belly woman\\
\glt `the womans's belly' % \citet[85]{Lichtenberk:1978}\end{exe}

\ex\gll\textipa{karrO} \textbf{\textipa{kamO\textglotstop}}\\
body man\\
\glt `the man's body' %\citet[85]{Lichtenberk:1978} after \citet[17]{Leenhardt:1935}\end{exe}

\ex\gll\textipa{nev\~a} \textbf{\textipa{i}} \textipa{wi\textglotstop}\\
land \gen{} man\\
\glt `the man's land' %\citet[86]{Lichtenberk:1978}\end{exe} 
\end{xlist}
\end{exe}

In Nias\il{Nias} attributive possessors are in the Mutated form of the noun (\ref{NiaGen}) -- cf. the Unmutated form of the noun \emph{buaya} `crocodile'. %\citep[60]{Donohue.Brown:1999}.
They immediately follow their possessee. Pronominal possessors are expressed via person agreement suffixes. 
This construction is used for alienable\is{possession!alienable versus inalienable} and inalienable possession alike \citep[374]{Brown:2001}.

\begin{exe}
\ex\label{NiaGen}\langinfobreak{Nias}{Sundic; Indonesia}{\citealp[374]{Brown:2001}}
\gll telau mbuaya\\
head crocodile.\mut{}\\
\glt `the head of the crocodile'
\end{exe}

Savosavo\il{Savosavo} has a special case-form to express attributive possessors (among other functions): the Genitive\is{case!individual forms!genitive}.
The attributive possessive construction is illustrated in (\ref{SavGen}).
\enlargethispage{\baselineskip}
\begin{exe}\ex\label{SavGen}\langinfo{Savosavo}{Solomons East Papuan; Solomon Islands}{\citealp[132]{Wegener:2008}}
\raggedright
\gll ko tada lo\textbf{-va} ti=gho te pala-tu, bo kokoa\\
3\sg{}.\fem{}.\gen{} man 3.\sg{}.\mas{}-\gen{}.\mas{} tea=3\sg{}.\fem{}.\nom{} \emphat{} make.3\sg{}.\mas.\obj-\prs.\ipfv{} or 3\sg{}.\fem{}.\poss{}.\mas{}\\
\glt `Is she making her husband's tea or hers?'
\end{exe}\is{possession!attributive|)}

All\is{terms of address|(} Pacific marked"=S languages use the zero-coded form of a noun in addressing someone (i.e. Accusative\is{case!individual forms!accusative} or Unmutated form).
This is illustrated by the following examples from Aji\"e\il{Aji\"e} (\ref{AijVoc}), Nias\il{Nias} (\ref{NiaVoc}) and Savosavo\il{Savosavo} (\ref{SavVoc}).

\pagebreak
\begin{exe}\ex\label{AijVoc}\langinfobreak{Aji\"e}{}{\citealp[199]{Fontinelle:1961}}
\gll\textbf{\textipa{ngE:\textglotstop}} \textipa{pE-Bi} \textipa{para} \textipa{e-'kona} \dots\\
grandmother take-go \pl{} product-fish\\
%Grand-m\`ere prendre-aller \pl{} produit-p\^eche\\
\glt `Grandmother, take the fish.'\\
 original translation: `Grand-m\`ere, emporte les poissons.'
\end{exe}

%\pagebreak

\begin{exe}
\ex\label{NiaVoc}\langinfobreak{Nias}{}{\citealp[59]{Brown:2001}} 
\gll Haiya ni-wa\"o-u ga,  \textbf{am\'a}?\\
what \pass{}-say-2\sg{}.\poss{} here father\\
\glt `What is it you want here, Sir?'
\end{exe}

\begin{exe}\ex\label{SavVoc}\langinfobreak{Savosavo}{}{\citealp[127]{Wegener:2008}}
\raggedright
\gll \textbf{minister}, \textbf{secretary}, dulo bo-tu me=me kati ka zui so=gha=e me=na\\
minister secretary all go-\relativ{} 2\pl{}=\emphat{}.2\pl{} \cert{} already end \att{}=\pl{}=\emphat{} 2\pl{}=\nom{}\\
\glt `Minister, Secretary, you all who went, you will all be fired.'
\end{exe}\is{terms of address|)}

%%%%%%%%%%%%%%%%%Citation form%%%%%%%%%%%%%%%%%%%%%%%%%%%

Usually\is{citation form|(}, the zero-coded case-form (accusative\is{case!individual forms!accusative} or ergative\is{case!individual forms!ergative}) is considered to be the citation form of a noun in the marked"=S languages of the Pacific. 
For Nias\il{Nias} \citet[69]{Brown:2001} states that the ``unmutated form of a noun is usually its citation form'', but apparently some speakers also employ the Mutated form (Absolutive\is{case!individual forms!absolutive}) for citation (Lea Brown, p.c.). 
This behavior, which might be viewed as a reinterpretation of the different forms of the nouns, is especially frequent with a limited set of nouns.\is{citation form|)}

A summary of the Pacific date is given in Table~\ref{PacExtra}. 
Only Aji\"e\il{Aji\"e} makes a distinction between alienable\is{possession!alienable versus inalienable} and inalienable possession. 
This distinction is in accordance with the prediction by \citet[4--5]{Chappell:1996}, according to which inalienable possession tends to be expressed by mere juxtaposition of the (zero-coded form of the) noun, while alienable possession is expressed via overt coding through genitive\is{case!individual forms!genitive} case.
Nias\il{Nias} uses the overtly coded Absolutive\is{case!individual forms!absolutive} case (the so-called Mutated form) to code nominal possessors. 
This is one of the few examples supporting Lander's (\citeyear[590]{Lander:2009}) claim that this relation is encoded by the overtly marked transitive case-form. 
Terms of address are uniformly in the zero-coded accusative\is{case!individual forms!accusative}/ergative\is{case!individual forms!ergative} case in all marked"=S languages of this region.
Also the citation\is{citation form} form tends to be identical to the zero-coded form, but in Nias\il{Nias} some reorganization of the paradigm can possibly be observed.

\begin{table}[t,b,h]
\centering
\begin{tabular}{lccc}
\hline \hline
\bfseries language&\bfseries Possessor &\bfseries Address  &\bfseries Citation\\
\hline
Aji\"e\il{Aji\"e}&\textbf{\gen{}}/{\acc{}}&\acc{}&{\acc{}}\\
%\hdashline
Nias\il{Nias}&\textbf{\abs{}}&{\erg{}}&{\erg{}}/\textbf{\abs{}}\\
%\hdashline
Savosavo\il{Savosavo}&\textbf{\gen{}}&\acc{}&{\acc{}}\\
\hline \hline
\end{tabular}
\caption{Non-clause-level case-marking in the Pacific region}\label{PacExtra}
\end{table}


%%%%%%%%%%%%%%%%%%%%%%%%%%%%%%%%%%%%
\section{North America}\label{ExtraNA}
%%%%%%%%%%%%%%%%%%%%%%%%%%%%%%%%%%%%

\is{possession!attributive|(}The Yuman languages of North America use the zero-coded accusative\is{case!individual forms!accusative} case-form for encoding attributive possessors. 
Wappo\il{Wappo} and Maidu\il{Maidu}, the other marked"=S languages of this region, have a special genitive\is{case!individual forms!genitive} case for this purpose. 
In Wappo\il{Wappo}, however, the Genitive\is{case!individual forms!genitive} is only used for alienable possession\is{possession!alienable versus inalienable}, while inalienable possessors are encoded in the Accusative\is{case!individual forms!accusative}. 
Furthermore, the possessive relation is marked via juxtaposition of the two nouns (i.e. possessor and possessee) rather than by means of adpositions or particles. 
Possessor agreement marking is found on the possessee, which is optional in most languages if the possessor is expressed as a full noun.\is{possession!attributive|)} 
Terms of address are encoded in either the accusative\is{case!individual forms!accusative} form or via special vocative\is{case!individual forms!vocative} affixes. 
In Maidu\il{Maidu}, the Nominative\is{case!individual forms!nominative}  is sometimes employed in this context. 
Usually, the citation form\is{citation form} coincides with the accusative\is{case!individual forms!accusative} case of a noun, but for two languages there is a minor variation of this pattern.  

%%%%%%%%%%%%%%Genitive%%%%%%%%%%%%%%%%%
Mojave\il{Mojave}\is{possession!attributive|(} expresses attributive possession by preposing the noun referring to the possessor in its zero-coded form  to the possessee (\ref{MojGen}).
For alienable possession\is{possession!alienable versus inalienable}, the prefix \emph{n\textsuperscript{y}-} is inserted between person marker and noun stem. 
This marker may also appear with nouns which have a full-noun possessor \citep[16--18]{Munro:1976}.
%(\ref{MojGenPN})

\begin{exe}\ex\label{MojGen}\langinfo{Mojave}{}{\citealp[50, 18]{Munro:1976}}
\begin{xlist}\ex\gll \textipa{vidan\super{y}} \textbf{\textipa{john}} \textipa{n\super{y}-ava:-\v c}\\
this John.\acc{} \poss{}-house-\nom{}\\
\glt `This is John's house' 

\ex\gll \textbf{\textipa{k\super{w}aT@\textglotstop ide:}} \textipa{n\super{y}-ava:}\\
doctor.\acc{} \poss{}-house\\
\glt `the doctor's house' %\citep[18]{Munro:1976}
\end{xlist}
\end{exe}


%The regular free-pronouns can also occur in the position before the possessed noun.  ??????????how is distinguished between ny- alienable and ny- 1st person possessor?????

%
%{Mojave\il{Mojave}} \citet[18]{Munro:1976}
%\begin{exe}\ex\label{MojGenPN}
%\begin{xlist}\ex\gll\textipa{\textglotstop in\super{y}ep} \textipa{\textglotstop-n\super{y}ahmaruy}\\
%1\sg{}.\acc{} 1\sg{}-shoes\\
%\ex\gll\textipa{\textglotstop-n\super{y}ahmaruy}\\
%1\sg{}-shoes\\
%\ex\gll\textipa{\textglotstop in\super{y}ep} \textipa{n\super{y}ahmaruy}\\
%1\sg{}.\acc{} shoes\\
%\end{xlist}
%\end{exe}

The other Yuman languages behave in a parallel fashion:\enlargethispage{2\baselineskip}
in Cocopa\il{Cocopa} (\ref{CocGen}), Mesa Grande Diegue\~no\il{Diegue\~no (Mesa Grande)} (\ref{DieGen}), Jamul\il{Jamul Tiipay} Tiipay (\ref{JamGen})\footnote{The exact morphological structure of the last word in the Jamul Tiipay example, especially the function of the segments <ta>, is unclear. This is marked via the asterisk by \citet{Miller:2001}.}, and Maricopa\il{Maricopa} (\ref{MarGen}), the zero-coded possessor precedes the possessee.

\begin{exe}\ex\label{CocGen}\langinfobreak{Cocopa}{}{\citealp[165]{Crawford:1966}}
\gll \textbf{ap\'a} n\textsuperscript{y}aw\'a\\
     man.\acc{} house.\acc{}\\
\glt `the man's house'
\end{exe}

%Diegue\~no\il{Diegue\~no (Mesa Grande)}: Attr. Poss: zero-coded \citep[17]{Gorbet:1976}

\begin{exe}\ex\label{DieGen}\langinfobreak{Diegue\~no (Mesa Grande)}{}{\citealp[17]{Gorbet:1976}}
\gll \textbf{\textipa{k\super{w}sya:y}} \textipa{n\super{y}-kuci:} \\
doctor.\acc{} \poss{}-knife\\
\glt `the doctor's knife'
\end{exe}

%Havasupai\il{Havasupai}: [Attributive possessor:] Zero-coded form

\begin{exe}\ex\label{JamGen}\langinfobreak{Jamul Tiipay}{}{\citealp[152]{Miller:2001}}
\gll \textbf{Evelyn} nye-armewil uutak-x ta*paa-ch \dots\\
Evelyn.\acc{} \ali{}-car make\_open-\irr{} ta*be\_present-\ssbj{}\\
\glt `He was trying to break into Evelyn's car \dots'
\end{exe} 

\begin{exe}\ex\label{MarGen}\langinfo{Maricopa}{}{\citealp[31, 40]{Gordon:1986}}
\begin{xlist} 
\ex\gll \textbf{Bonnie} s'aw\\
Bonnie.\acc{} offspring.\acc{}\\
\glt `Bonnie's baby'

\ex\gll \textbf{Bonnie} s'aw ime\\
Bonnie.\acc{} offspring.\acc{} leg.\acc{}\\
\glt `Bonnie's baby's leg'

\ex\gll \textbf{'iipaa}-ny-a ny-va-ny-sh vtay-m\\
man-\dem{}-\augv{} \poss{}-house-\dem{}-\nom{} big-\rls{}\\
\glt `That man's house is big.'
\end{xlist}
\end{exe}

The same pattern is found in Havasupai\il{Havasupai} (\ref{HavGen}), Walapai\il{Walapai} (\ref{WalGen}), and Yavapai\il{Yavapai} (\ref{YavGen}), which form a distinct subgroup within the Yuman languages. 
For Yavapai\il{Yavapai}, this context is discussed in some more detail. 
The pattern of a zero-coded possessor is used for both inalienable\is{possession!alienable versus inalienable} possession (\ref{YavInal}) and alienable possession (\ref{YavAlien}).
 
%\pagebreak

\begin{exe}\ex\label{HavGen}\langinfo{Havasupai}{}{\citealp[57]{Kozlowski:1972}}
\begin{xlist} 
\ex \gll \textbf{jan} lwa\\
John.\acc{} wife\\
\glt `John's wife' %\citet[57]{Kozlowski:1972}\end{exe}

\ex\gll \textbf{pa} \~nu-hu\\
man.\acc{} \dem{}-head\\
\glt `the man's/his head' %\citet[57]{Kozlowski:1972}\end{exe}
\end{xlist}
\end{exe}

\begin{exe}\ex\label{WalGen}\langinfobreak{Walapai}{}{\citealp[76]{Watahomigie:2001}}
\gll \textbf{Joe} b\'ud-a-ch ya:d-i-k-yu\\
Joe.\acc{} 3.hat-\defsc{}-\nom{} 3.fly-suddenly-\ssbj{}-\aux{}\\
\glt `Joe's hat flew away.'
\end{exe}

\begin{exe}\ex\label{YavGen}\langinfo{Yavapai}{}{\citealp[60]{Kendall:1976}}
\begin{xlist}\ex\gll\label{YavInal}\textipa{kiTar-c} \textbf{\textipa{hamsi ktyo:ca}} \textipa{mpar} \textipa{ck\super{y}o:-k\~n}\\
dog-\nom{} hamsi\_ktyocha.\acc{} leg bite-\compl{}\\
\glt `A dog bit Hamsi-ketyocha's leg.'

\ex\gll\label{YavAlien}\textbf{\textipa{lupi}} \textipa{hanaq}\\
Lupe.\acc{} necklace\\
\glt `Lupe's necklace'
\end{xlist}
\end{exe}

In Wappo\il{Wappo}, two different constructions are used to encode alienable\is{possession!alienable versus inalienable} and inalienable possession.
Genitive\is{case!individual forms!genitive} marking is only used for alienable possession (\ref{WapGen}a), while in inalienable possession, the attributive possessor is zero-coded (\ref{WapGen}b).

\begin{exe}\ex\label{WapGen}\langinfo{Wappo}{}{\citealp[14, 15]{Thompsonetal:2006}}
\begin{xlist}
\ex\gll\textipa{ah} \textipa{ce} \textipa{met'e} \textipa{ce} \textipa{k'ew\textbf{-me\textglotstop}} \textipa{k'e\v su} \textipa{pa\textglotstop-is-ta\textglotstop}\\
1\sg{}.\nom{} \dem{} woman \dem{} man-\gen{} meat eat-\caus{}-\pst{}\\
\glt `I made the woman eat the man's meat' %\citet[14]{Thompsonetal:2006}

\ex\gll\textbf{\textipa{c'ic'a}} \textipa{khap-i} \textipa{ke\textglotstop te-khi\textglotstop}\\
bird.\acc{} wing-\nom{} broken-\stat{}\\
\glt `The bird's wing is broken.'% \citet[15]{Thompsonetal:2006}
\end{xlist}
\end{exe}

%
%Possessive pronouns for inalienably possessed nouns are the same as object/  Accusative personal pronouns (except 3\textsc{dual} -- different form, 1/2\textsc{dual} -- no personal pronoun); Possessive pronouns in alienable possession are derived from the inalienable form \citep[134f.]{Radin:1929}.

%However, in \citet[97f.]{Lietal:1977} the possessive pronouns are not mentioned (they were very probably no longer in use by that time).
%Radin establishes two sets of possessive pronouns -- alienable and inalienable. The alienable form can systematically be analyzed as being formed by the addition of the Genitive marker to the unalienable forms. In most cases the inalienable object pronoun is identical to the Accusative form of the personal pronouns. However Radin's set of possessive pronouns is larger than his person pronouns (containing some additional dual forms, mostly unattested in the alienable variant). In the analysis of \citet[14]{Thompsonetal:2006} the Accusative pronouns are regularly used for possessive function as well (adding the Genitive suffix for alienable possession).

%\medskip
%{Wappo\il{Wappo}} \citep[15, 16]{Thompsonetal:2006}
%\begin{exe}\ex\label{WapGenPN}
%\begin{xlist}
%\ex\gll\textipa{i} \textipa{yawe} \textipa{ah} \textipa{huhkal-ta\textglotstop}\\
%1\sg{} name 1\sg{}.\nom{} remember-\pst{}\\
%`I remembered my name' % \citet[15]{Thompsonetal:2006}

%\ex\gll\textipa{te} \textipa{phe\textglotstop-i} \textipa{tu\v c'a-khi\textglotstop}\\
%3\sg{} foot-\nom{} big-\stat{}\\
%`his foot is big' %\citet[15]{Thompsonetal:2006}

%\ex\gll\textipa{ah} \textipa{i-me\textglotstop} \textipa{t'ol} \textipa{oh-co:-ta\textglotstop}\\
%1\sg{}.\nom{} 1\sg{}-\gen{} hair \caus{}-black-\pst{}\\
%`I dyed my wig black' %\citet[16]{Thompsonetal:2006}
%\end{xlist}
%\end{exe}

In Maidu\il{Maidu}, attributive possessors are marked with {Genitive}\is{case!individual forms!genitive} case (\ref{MaiGen}).  
No distinction between alienable\is{possession!alienable versus inalienable} and inalienable possession is mentioned in the grammar. 
Only nouns which serve as subject, object, or location can be modified with a {Genitive} NP \citep[30--31]{Shipley:1964}.

\begin{exe}\ex \label{MaiGen}\langinfobreak{Maidu}{}{\citealp[31]{Shipley:1964}}
\raggedright
\gll w\'elk\textraiseglotstop et\textraiseglotstop i-m kyl\'okbe-m {\textglotstop}as w\'epa\textbf{-k} kyl\'e-m mac\textraiseglotstop \'oj-{\textglotstop}am\\
frog-\nom{} old\_woman-\nom{} \emphat{} coyote-\gen{} woman-\nom{} say-\pstpunc{}.3\\
\glt `They say that Frog Old Woman was Coyote's wife'
\end{exe}\is{possession!attributive|)}

%%%%%%%%%%%%%%%%%%%%%Vocative%%%%%%%%%%%%%%%%%%%%%

Many\is{terms of address|(} Yuman languages use the zero-coded Accusative\is{case!individual forms!accusative} case-form of a noun as a term of address.
Among these languages are Cocopa\il{Cocopa} (\ref{CocVoc}),  Mesa Grande Diegue\~no\il{Diegue\~no (Mesa Grande)} (\ref{DieVoc}), and Jamul\il{Jamul Tiipay} Tiipay (\ref{JamVoc}). 

%\pagebreak
\begin{exe}\ex\label{CocVoc}\langinfo{Cocopa}{}{\citealp[179]{Crawford:1966}}
\begin{xlist}
\ex\gll n\textsuperscript{y}c\'a\\
mother.\acc{}\\
\glt `Mother!'

\ex\gll xm\'ik\\
young\_man.\acc{}\\
\glt `Young man!'
\end{xlist}
\end{exe}

\begin{exe}\ex\label{DieVoc}\langinfo{Diegue\~no (Mesa Grande)}{}{\citealp[158]{Langdon:1970}}
\begin{xlist}
\ex\gll xawka \textbf{margarit} t\textschwa muwa=a\\
hello Margaret.\acc{} you\_are\_sitting=Q\\
\glt `Hello, Margaret, how are you?'

\ex\gll \textbf{may\textglotstop pay} k\textschwa y\textschwa wip!\\
you\_people.\acc{} you\_all\_listen\\
\glt `Listen, all you people!'
\end{xlist}
\end{exe}

\begin{exe}\ex\label{JamVoc}\langinfobreak{Jamul Tiipay}{}{\citealp[128]{Miller:1990}}
\gll\textbf{\textipa{perxaaw}} \textipa{maayich} \textipa{m-rar} \textipa{m-wa-ch-m-yu} \\
fox.\acc{} what 2-do 2-be\_sitting-\ssbj{}-2-be\\
\glt `What are you doing here, Fox?' 
\end{exe}

In some other Yuman languages, special forms are used in this context.
Walapai\il{Walapai} has two Vocative\is{case!individual forms!vocative} affixes, \emph{-\'e} for addressees near the speaker (proximal), and \emph{-\'o/-wo} for addressees who are out of sight (\ref{WalVoc}).
As for the citation form, \citet[129, footnote 3]{Munro:1976} notes that some Mojave\il{Mojave} speakers add a final schwa to nouns used for addressing.  %and \emph{humar-\textschwa} in \citet[49]{2Mojave\il{Mojave}}

\begin{exe}\ex\label{WalVoc}\langinfobreak{Walapai}{}{\citealp[56]{Watahomigie:2001}}
%\begin{xlist}
%\ex \gll nya misi:! gwe ma-ma:!\\
%1\sg{}.\poss{} girl.\voc{} thing 2/3-eat.\imp{}\\
%`My daughter! Eat!' %check page
%\ex 
\gll nya misi:\textbf{-ye}! Gwe ma-ma:-j-a!\\
1\sg{}.\poss{} girl-\pl{}.\voc{} thing 2$>$3-eat-\pl{}-\imp{}\\
\glt `My daughters! Eat!' %\citet[56]{Watahomigie:2001}
%\end{xlist}
\end{exe}

Information on terms of address in Wappo\il{Wappo} is provided by the earlier grammar by \citet{Radin:1929}. 
It is unclear whether this system was still used in the moribund stage of the language described by \citet{Thompsonetal:2006}.
Usually the zero-coded Accusative\is{case!individual forms!accusative} form is used for address.  
However, there is a tendency to use a different form, either by shortening stems with terminal vowels or by using the Nominative\is{case!individual forms!nominative}  \citep[130]{Radin:1929}. 
For a specific set of nouns, which \citet[130,~133]{Radin:1929} calls `relationship terms', a special Vocative\is{case!individual forms!vocative} form is used when the addressee is invisible or far away (\emph{-sta}). 
Maidu\il{Maidu} employs the Nominative\is{case!individual forms!nominative} as a form of address for all nouns except for a certain class of kinship terms. For these nouns the Accusative\is{case!individual forms!accusative} form is used instead \citep[30]{Shipley:1964}.\is{terms of address|)}

%%%%%%%%%%%%%%%Citation form%%%%%%%%%%%%%%%

All\is{citation form|(} Yuman languages use the Accusative\is{case!individual forms!accusative} form of a noun as the citation form. 
However, in Mojave\il{Mojave} another pattern is described, in which many speakers show a tendency to add \emph{-a} or \emph{-\textschwa} to any noun in isolation including the citation form \citep[129, footnote 3]{Munro:1976}.
In Wappo\il{Wappo} too, the Accusative\is{case!individual forms!accusative} form is used in citation \citep{Lietal:1977}.
Maidu\il{Maidu} uses the noun stem as a citation form, and this form is identical to the Accusative\is{case!individual forms!accusative} of a noun for all vowel final stems. 
Some speakers always use the object form as citation form, according to \citet[30]{Shipley:1964}.\is{citation form|)}

%\enlargethispage{\baselineskip}
\begin{table}[t,h]
\centering
\begin{tabular}{lccc}
\hline \hline
\bfseries language&\bfseries Possessor &\bfseries Address  &\bfseries Citation\\
\hline
Cocopa\il{Cocopa}&{\acc{}}&{\acc{}}&\acc{}\\
%\hdashline
Diegue\~no\il{Diegue\~no (Mesa Grande)} (Mesa Grande)&\acc{}&\acc{}&\acc{}\\
%\hdashline
Havasupai\il{Havasupai}&{\acc{}}&{-}&{\acc{}}\\
%\hdashline
Jamul\il{Jamul Tiipay} Tiipay&\acc{}&{\acc{}}&\acc{}\\
%\hdashline
Maidu\il{Maidu}&\textbf{\gen{}}&\textbf{\nom{}}/\acc{}&{\acc{}}/stem\\
%\hdashline
Mojave\il{Mojave}&\acc{}&\acc{}/-{\textschwa}&\acc{}/-{\textschwa}\\
%\hdashline
Walapai\il{Walapai}&\acc{}&\voc{}&\acc{}\\
%\hdashline
Wappo\il{Wappo}&\textbf{\gen{}}/\acc{}&\acc{}(/\voc{})&\acc{}\\
%\hdashline
Yavapai\il{Yavapai}&\acc{}&{-}&\acc{}\\
\hline \hline
\end{tabular}
\caption{Non-clause-level case-marking in North America}\label{NAExtra}
\end{table}


Table~\ref{NAExtra} summarizes the data from the North American languages. 
Except\is{possession!attributive|(} for Maidu\il{Maidu}, all languages use the accusative\is{case!individual forms!accusative} for attributive possessors. 
In Wappo\il{Wappo}\is{possession!alienable versus inalienable|(}, this construction is limited to inalienable possession, while alienable possession is expressed via Genitive\is{case!individual forms!genitive} case. 
This pattern nicely fits the prediction by \citet[4--5]{Chappell:1996}, according to which constructions that mark alienable possession are prone to use more overt material than constructions that mark inalienable possession\is{possession!alienable versus inalienable|)}.\is{possession!attributive|)} 
As a term of address\is{terms of address}, the accusative\is{case!individual forms!accusative} (most Yuman languages, some Maidu\il{Maidu} and Wappo\il{Wappo} nouns), special vocative\is{case!individual forms!vocative} forms (Walapai\il{Walapai}, and Wappo\il{Wappo}, with some restrictions), and the Nominative (Maidu\il{Maidu}, with some restrictions), are used.
All languages make use of the accusative\is{case!individual forms!accusative} form in citation to some extent. 
In Mojave\il{Mojave} and Maidu\il{Maidu}, there is some variation in the form used in citation\is{citation form} between different speakers. 


\section{Summary}\label{ExtraSum}


The final overview of the encoding of attributive possessors and extra"=syntactic functions in
marked"=S languages is given in Table~\vref{OverviewExtra}.
\begin{table}[t,b,h]
\centering
\begin{tabular}{lccc}
\hline \hline
\bfseries language&\bfseries Possessor &\bfseries Address  &\bfseries Citation\\
\hline
Aji\"e\il{Aji\"e}&\textbf{\gen{}}/{\acc{}}&\acc{}&{\acc{}}\\
%\hdashline
Arbore\il{Arbore}&{\acc{}}&{-}&{\acc{}}/reduced form\\
%\hdashline
Cocopa\il{Cocopa}&{\acc{}}&{\acc{}}&\acc{}\\
%\hdashline
Datooga\il{Datooga}&\acc{}&\acc{}&\acc{}\\
%\hdashline
Diegue\~no\il{Diegue\~no (Mesa Grande)} (Mesa Grande)&\acc{}&\acc{}&\acc{}\\
%\hdashline
Dinka\il{Dinka (Agar)}&\textbf{\nom{}}&{-}&\acc{}\\
%\hdashline
Gamo\il{Gamo}&\textbf{\gen{}}&\textbf{\voc{}}&\acc{}\\
%\hdashline
Havasupai\il{Havasupai}&{\acc{}}&{-}&{\acc{}}\\
%\hdashline
Jamul\il{Jamul Tiipay} Tiipay&\acc{}&{\acc{}}&\acc{}\\
%\hdashline
K'abeena\il{K'abeena}&\textbf{\gen{}}&\acc{}/\textbf{\gen{}}&{\acc{}}/reduced form\\
%\hdashline
Maa\il{Maa}&\acc{}&{-}&{\acc{}}\\
%\hdashline
Maidu\il{Maidu}&\textbf{\gen{}}&\textbf{\nom{}}/\acc{}&{\acc{}}/stem\\
%\hdashline
Mojave\il{Mojave}&\acc{}&\acc{}/-{\textschwa}&\acc{}/-{\textschwa}\\
%\hdashline
Murle\il{Murle}&\textbf{\gen{}}&{-}&\acc{}\\
%\hdashline
Nandi\il{Nandi}&\acc{}&{-}&{\acc{}}\\
%\hdashline
Nias\il{Nias}&\textbf{\abs{}}&{\erg{}}&{\erg{}}/\textbf{\abs{}}\\
%\hdashline
Oromo (Boraana\il{Oromo (Boraana)})&\textbf{\gen{}}&{-}&{\acc{}}\\
%\hdashline
Oromo (Harar\il{Oromo (Harar)})&\textbf{\gen{}}&{-}&{\acc{}}\\
%\hdashline
Savosavo\il{Savosavo}&\textbf{\gen{}}&\acc{}&{\acc{}}\\
%\hdashline
Tennet\il{Tennet}&\textbf{\gen{}}&{-}&{\acc{}}\\
%\hdashline
Turkana\il{Turkana}&\textbf{\gen{}}&\textbf{\voc{}}&\acc{}\\
%\hdashline
Walapai\il{Walapai}&\acc{}&\textbf{\voc{}}&\acc{}\\
%\hdashline
Wappo\il{Wappo}&\textbf{\gen{}}/\acc{}&\acc{}(/\textbf{\voc{}})&\acc{}\\
%\hdashline
Wolaytta\il{Wolaytta}&\textbf{\gen{}}&\textbf{\voc{}}&\acc{}\\
%\hdashline
Yavapai\il{Yavapai}&\acc{}&{-}&\acc{}\\
%\hdashline
Zayse\il{Zayse}& \textbf{\gen{}}&{-}&\acc{}\\
\hline \hline
\end{tabular}
\caption{Overview on the non-clause-level case-marking}\label{OverviewExtra}
\end{table}
About\is{possession!attributive|(} half the languages use the zero-coded form of a noun to encode attributive possessors; most of these languages can be found in North America. 
Roughly the other half has a dedicated genitive\is{case!individual forms!genitive} case for attributive possessors, which is distinct from the marking of the overtly coded transitive argument. 
Only two languages (Dinka\il{Dinka (Agar)} and Nias\il{Nias}) use the form corresponding to the transitive argument (A or P) which receives overt coding. 
This pattern was predicted to be quite common by \citet[590]{Lander:2009}, but this prediction has not been borne out by the marked"=S languages in my sample.\is{possession!attributive|)}
As\is{terms of address|(} for terms of address, the zero-coded form of a noun is also frequently used. 
In a number of languages of Africa (especially in Afro-Asiatic), and in Wappo\il{Wappo} at an earlier stage, special vocative\is{case!individual forms!vocative} forms exist(ed). 
In Maidu\il{Maidu} (Nominative\is{case!individual forms!nominative} ) and K'abeena\il{K'abeena} (Genitive\is{case!individual forms!genitive}), some other case-forms are employed in this context, but this is always limited to a specific set of nouns.\is{terms of address|)}
No case-form other than the zero-coded one is used as a citation form\is{citation form|(} of a noun in any of the languages, except for a reinterpretation of the relation between Mutated and Unmutated nouns in Nias\il{Nias}. 
Otherwise, if the form used in citation differs from the zero-coded case-form, it is a reduced form of the noun (Arbore\il{Arbore}, K'abeena\il{K'abeena}, Maidu\il{Maidu}). 
In sum, the correlation between zero-coded transitive case-form and citation form appears to be very strong. 
This finding indicates that there is no direct correlation between the nominative\is{case!individual forms!nominative}  case and the citation form of a noun by itself. 
Rather the relation is between the zero-coded form of a noun and the citation form. The zero-coded form, however, corresponds to the nominative in the majority of languages.\is{citation form|)}
	

