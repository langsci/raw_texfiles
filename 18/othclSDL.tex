\chapter{Subjects of non-basic clauses}\label{nonbasic}

%\setcounter{exx}{0}

%%%%%%%%%%%%%%%%%%%%%%%%%%%%%%%%%%%%%%%%%%%%%%%%
\section{Introduction} 
%%%%%%%%%%%%%%%%%%%%%%%%%%%%%%%%%%%%%%%%%%%%%%%%

In this chapter, case-marking in a number of non-basic clauses will be discussed. 
Under non-basic clauses, I subsume various types of dependent clauses\is{clause-type!dependent}, as well as clauses in which the number of arguments that a verb takes is changed through morphosyntactic processes. 
For the latter type one can distinguish between processes that decrease\is{valency-decreasing construction} the number of arguments -- passivization\is{valency-decreasing construction!passive} and antipassivization\is{valency-decreasing construction!antipassive} -- and those that increase\is{valency-increasing construction} the number of arguments, such as causativation. 
Since only the processes that decrease the number of arguments show any exceptional patterns (i.e. patterns that do not employ the S-case), only these contexts will be discussed here.
Basic clauses in contrast are defined here as consisting of a single predicate which has not undergone any argument-affecting derivation. 

Dependent\is{clause-type!dependent} clauses exhibit special marking strategies in many languages, far exceeding the domain of marked"=S. 
The verb-final word order found with German\il{German} dependent clauses as opposed to verb-second main clauses is one example of such special marking. 
Deviating patterns of case-marking are also found in this domain. 
The domain of dependent clauses can be subdivided into smaller domains, such as relative clauses\is{clause-type!relative clause} or adverbial clauses\is{clause-type!adverbial clause}. 
Each of the different clause-types potentially has its own distinct type of encoding. 
A brief discussion of different types of dependent clauses and their grammar is given in Section~\ref{DepCl}. 
However, this topic cannot be covered in depth here.

Next\is{valency-decreasing construction|(}, I will discuss  valency reducing operations (Section~\ref{ValDec}). 
The specific labels that are used for these constructions often carry strong implications about their formal encoding. 
The promotion of the logical object to subject status is usually seen as a prerequisite for labeling a construction as `passive'. 
The more neutral term `valency-decreasing operations' is more readily applicable to a wide range of phenomena that might not be captured by a more specific label such as passive\is{valency-decreasing construction!passive}.
Apart from formal marking properties, valency-decreasing constructions are also associated with a specific information structure. 
The passive, for example, puts attention on the patient argument. 
However, a similar communicative effect can also be achieved by other formal means. 
So-called `impersonal\is{valency-decreasing construction!impersonal construction} constructions' are a prime example of this. 
I will briefly discuss these in Section~\ref{ValDec} as well, yet, whether constructions of this type should be considered to be valency-decreasing is at least debatable.\is{valency-decreasing construction|)} 

%The last context discussed in this chapter are valency increasing operations. These are described in Section~\ref{ValIncr}. 
%There are quite a number of processes found in the languages of the world that can introduce new arguments, such as applicatives or the introduction of a beneficary, however, given the nature of this study, I am only interested in processes that introduce a new subject like argument. 
%Thus this section will be mainly concerned with causatives. 

The contexts studied in this chapter are associated with a rather formal register, such as written rather than spoken language.
A large number of passive\is{valency-decreasing construction!passive} and dependent clauses for example are typical indicators of written texts of an academic nature. 
For a number of languages of my sample, these types of register are not very elaborate or commonly used. 
Thus, the contexts of interest are often not  well represented in the description of a language or not even discussed at all. 

After introducing these different types of non-basic clauses in Sections~\ref{DepCl} and~\ref{ValDec}, 
the patterns of case-marking found in these contexts will be outlined in Section~\ref{QueNonBasic}. 
Subsequently, I will present data on the encoding of these contexts in the individual languages of North America (Section~\ref{NonBasicNA}), the Pacific region (Section~\ref{NonBasicPac}), the Nilo-Saharan (Section~\ref{NonBasicNilo}) and Afro-Asiatic (Sectio~n\ref{NonBasicAfro}) family. 
Finally, a summary of the data is provided in Section~\ref{NonBasicSum}.     

%%%%%%%%%%%%%%%%%%%%%%%%%%%%%%%%%%%%%%%%%%%%%%%%%%%%%
\section{Dependent clauses}\label{DepCl}\is{clause-type!dependent|(}
%%%%%%%%%%%%%%%%%%%%%%%%%%%%%%%%%%%%%%%%%%%%%%%%%%%%%

A\is{clause-type|(} common distinction between clause-types is the one between independent clauses that can stand on their own and subordinate or dependent clauses. 
The meaning of the latter clause-type is tied to another clause and thus they cannot be fully interpreted on their own.  
However, most languages differentiate between a number of different types of dependent clauses, which often differ according to their grammatical encoding.
For the present study the marking of subject arguments is the central aspect of grammatical encoding to be investigated.

Instead of making a binary distinction between main and dependent clauses, it is possible to establish a hierarchy of grammatical integration ranging from structures that constitute one fully integrated clause to two completely independent clauses \citep[307]{Payne:1997}. 
Structures which are typically considered to consist of one independent and one (or more) dependent clause(s) are located in the middle section of this continuum with relative clauses\is{clause-type!relative clause} being less grammatically integrated than, e.g. adverbial or even complement clauses. 
Payne's scale of grammatical integration of clauses is given in Figure~\ref{ClauseIntegration}, where the parts of the scale commonly referred to as dependent or subordinate clauses are set aside from the rest of the scale by a box in this version of the scale.
 
\begin{figure}[ht] \centering \fbox{
\begin{picture}(325,145)
\put(10,70){\makebox(15,55)[t]{\begin{sideways}{one clause}\end{sideways}}}

\put(50,70){\makebox(15,55)[t]{\begin{sideways}{serial verb}\end{sideways}}}

\put(85,70){\makebox(15,55)[t]{\begin{sideways}{complement}\end{sideways}}}
\put(95,70){\makebox(15,55)[t]{\begin{sideways}{clause}\end{sideways}}}

\put(125,70){\makebox(15,55)[t]{\begin{sideways}{adverbial}\end{sideways}}}
\put(135,70){\makebox(15,55)[t]{\begin{sideways}{clause}\end{sideways}}}

\put(170,70){\makebox(15,55)[t]{\begin{sideways}{clause chains}\end{sideways}}}

\put(210,70){\makebox(15,55)[t]{\begin{sideways}{relative clause}\end{sideways}}}

\put(250,70){\makebox(15,55)[t]{\begin{sideways}{coordination}\end{sideways}}}

\put(285,70){\makebox(15,55)[t]{\begin{sideways}{two separate}\end{sideways}}}
\put(295,70){\makebox(15,55)[t]{\begin{sideways}{clauses}\end{sideways}}}

\put(80,50){\line(1,0){150}}
\put(80,135){\line(1,0){150}}
\put(80,50){\line(0,1){85}}
\put(230,50){\line(0,1){85}}

\put(10,45){\line(1,0){305}}
\put(10,40){\line(0,1){10}}
\put(315,40){\line(0,1){10}}

\put(10,20){\makebox(55,15)[l]{High degree of}}
\put(10,8){\makebox(55,15)[l]{grammatical integration}}

\put(250,20){\makebox(65,15)[r]{No grammatical}}
\put(250,8){\makebox(65,15)[r]{integration}}

\end{picture}}
\caption{Level of integration of clauses \citep[after][307]{Payne:1997}}\label{ClauseIntegration} \end{figure}\is{clause-type|)}

The exact ordering of this continuum is not uncontroversial. 
While Payne locates the relative clause in the position which is closest to the `two separate clauses' end of the scale and thus adjacent to coordination (i.e. clause combining structures), \citet[238]{Adverbial}\is{clause-type!adverbial clause} state that, unlike relative or complement clauses, adverbial clauses ``are viewed as (hypotactic) clause combinations with respect to the main clause.'' 
They consider adverbial clauses to be subordinated to a lesser degree than the other two clause-types.   

In the following I will briefly introduce the three types of dependent clauses which are relevant for this study, namely relative clauses, adverbial clauses and complement clauses\is{clause-type!complement clause}.
Since only one of the languages of my sample has clause-chaining\is{clause-chaining}, namely Savosavo\il{Savosavo} \citep[286--297]{Wegener:2008}, I did not include this structure in my study. 
The three types of dependent clauses are sometimes also referred to as `adjectival', `adverbial' and `nominal' clauses corresponding to the part of speech that they resemble in function.

\textsc{Relative clauses}\is{clause-type!relative clause|(} modify an argument (and in some languages also other participants) of the main clause. 
They are often discussed among other nominal modifiers such as adjectives or demonstratives, especially in terms of word order typology \citep[cf.][]{WALS90}. 
Since the relative clause makes a statement about one of the participants of the main clause this participant is also an argument of the relative clause.\footnote{The term participant is used in a very broad sense here and may include the role of location or possessor in a given language.}
\citet[314]{Dixon:2010-2} refers to the argument shared between main and relative clauses as the \textsc{common argument} (CA), a term I will use in the following discussion. 
Languages differ with respect to whether the common argument is realized in the main clause, in the relative clause or in both. 
If there is only one instance of the common argument, there can be ambiguity with respect to the question in which clause the argument is located.
In this study, I am interested in the realization of the subject element of the relative clause. 
In situations in which the subject of the relative clause is the common argument, its case-form in the relative clause is often difficult to identify, since it will not be realized as an independent noun phrase in many languages. 
Therefore, the best example sentences for the purpose of this study are those in which the subject of the relative clause is not the common argument and is represented by a full NP (e.g. \emph{the book that my sister bought}). 
However, this type of relative clause could not be found in all languages due to lack of data or possibly ungrammaticality of this construction.
There is another caveat concerning the study of case-marking found in relative clauses. 
In some languages, the relative clause construction is actually a nominalization. 
Instead of having verbal marking -- either identical to the marking found in main clauses, or special dependent verb morphology -- the verb has nominalizing morphology. 
Though verbal arguments can be realized in nominalized structures, case-marking is usually not preserved but rather substituted by a genitive, for example.\is{clause-type!relative clause|)} 

\textsc{Adverbial clauses}\is{clause-type!adverbial clause|(} do not modify a single participant of the main clause, but rather modify the verb phrase or entire main clause. 
They establish a relation in terms of temporal structure or other factors such as presenting the reason for or desired goal of the action in the main clause. 
Based on their different functions, a large number of subtypes of adverbial clauses can be distinguished, such as temporal, locational, purposive or conditional clauses \citep[for a discussion of these different subtypes see][243--265]{Adverbial}. 
Main clauses and adverbial clauses do not necessarily share an argument between them, though they might.\footnote{The following English\il{English} examples are adverbial clauses having a different
  (\ref{EngAdvCl}.a) and co-referential subject (\ref{EngAdvCl}.b).
\eal\label{EngAdvCl}
\ex John served the meal, after Jack had brought the wine.
\ex John$_i$ served the meal, after he$_i$ had brought the wine.
\zllast
}%\enlargethispage{\baselineskip} 
Therefore instances of full NP subjects within the adverbial clauses are usually easy to find (provided the grammar discusses this type of clause at all).\is{clause-type!adverbial clause|)} 

\textsc{Complement clauses}\is{clause-type!complement clause|(} serve as arguments of the main clause -- or the matrix clause, as the non-complement clause in this construction is usually referred to.\footnote{The term matrix clause is often preferred, since in order to function as a grammatical sentence, the argument position filled by the complement clause would have to be filled first in most cases.
Some complement verbs, however, still form grammatical utterances when the complement is deleted, as illustrated in (\ref{EngComp1}.b) for English\il{English}. 

\eal\label{EngComp1} 
\ex John is scared  {\rm[}that he might lose his job{\rm]}.
\ex John is scared.
\zllast
} 
Though there are subject complement clauses, typical complement clauses function as the P argument of a complement taking verb. 
Such verbs can be subdivided into several semantic types such as verbs of utterance or desideratives \citep[an extensive discussion of the different types of complement verbs can be found in][120--145]{Complement}.
A language may distinguish between different types of complements. 
This type of dependent clause has its own argument structure and in many cases one of its arguments is co-referential with an argument of the main clause. 
The co-referential argument is often not realized in the complement clause. 
Case-marking in complement clauses is often special, and may also vary between different types. 
The English\il{English} examples in (\ref{EngComp}) demonstrate the different case-marking of the complement-internal subject in non-finite (\ref{EngComp}a) and finite (\ref{EngComp}b) complement clauses.

\begin{exe}\ex\label{EngComp}
\begin{xlist}
\ex \textit{She wants  {\rm[}\textbf{him} to leave.{\rm]}}
\ex \textit{She hopes  {\rm[}(that) \textbf{he} has already left.{\rm]}}
\end{xlist}
\end{exe}\is{clause-type!complement clause}

Complex syntactic structures like dependent clauses are a feature found more often in written than in spoken language. 
Many of the languages of my sample (and in fact the majority of the languages in the world) do not have a long tradition as a written language, if any. 
Therefore dependent structures are often underdescribed in grammars or lacking at all entirely, since they do not play a significant role in language use. 
Also a language might not have a distinct grammaticalized construction for encoding these structures. %For these reasons I lump together subordinate clauses as one context for most languages. However if the description of a languages distinguishes between for instance relative and adverbial clauses, this information will be included in the discussion of the data. 
Relative clauses and adverbial clauses are the two types of dependent clauses most likely to be treated in a grammar. 

Taking a closer look at the token frequency of dependent clauses in written versus spoken language, it becomes clear that the assumption formulated above (i.e. dependent structures are a characteristic feature of written language) is an oversimplification. 
\citet[139--141]{Biber:1998} demonstrate, based on English\il{English} corpus data, that the distribution of different types of dependent clauses varies greatly between different registers. 
While relative clauses are most frequently found in academic prose, causative adverbial clauses are for instance most commonly used in conversations. 
However, the basic observation that data on dependent clauses of all types are scarce in grammars of most under-described languages still remains valid.\is{clause-type!dependent|)}

%%%%%%%%%%%%%%%%%%%%%%%%%%%%%%%%%%%%%%%%%%%%%%%%%%%%%%%
\section{Valency-decreasing operations}\label{ValDec}\is{valency-decreasing construction|(}
%%%%%%%%%%%%%%%%%%%%%%%%%%%%%%%%%%%%%%%%%%%%%%%%%%%%%%%

Under the term `valency-decreasing operations', a variety of constructions is subsumed.  
All these operations have in common that fewer grammatical arguments are realized than in the corresponding basic clause. 
Passive\is{valency-decreasing construction!passive}, antipassive\is{valency-decreasing construction!antipassive}, and middle\is{valency-decreasing construction!middle} are typical instances of this type of operation.
Both formal and functional criteria are of interest when analyzing these voice alternations. 
A crucial formal aspect is the case-marking of arguments in these constructions (which is also the main focus of the present study), but also the pragmatic implications of these structures are taken into account. 
Especially if on formal criteria no passive structure can be identified in a language, functional criteria are often considered in order to identify the equivalent construction in a language. 
So-called `impersonal\is{valency-decreasing construction!impersonal construction} constructions' often have functions similar to those of prototypical passives. 
I will briefly discuss impersonal\is{valency-decreasing construction!impersonal construction} constructions, though their status as valency-decreasing operations is not unambiguous in languages with verbal person agreement. 
In the following, the different grammatical categories and constructions that are involved in the reduction of verbal valency will be introduced. 
 
The\is{valency-decreasing construction!passive|(} most common valency-decreasing operation from a cross-linguistic perspective is the \textsc{passive}. 
Passive constructions realize non-agent arguments as the grammatical subject of logically transitive verbs. 
All languages with passive constructions allow for patients to be promoted to grammatical subject status, but languages vary with respect to whether other semantic roles such as ditransitive recipients can be promoted as well. 
Prototypical {passives} have a number of formal properties which are not necessarily met by voice operations serving the same pragmatic functions as typical passives in a given language. 
If a construction meets all of the three following criteria, it constitutes a prototypical instance of a passive \citep[205]{WALS107, Payne:1997}.

\begin{enumerate}
\item Demotion of the A argument of the active counterpart to non-argument status.
\item Promotion of the P argument of the active counterpart to subject.
\item Morphosyntactic marking of the voice alternation on the verb or in the verb phrase (either through affixation of periphrastic means).
\end{enumerate}

If a construction does not meet all of these criteria, linguists differ strongly in whether they call a construction a passive or not. 
The last criterion -- verbal marking of the voice alternation -- is quite unproblematic in this respect. A deviation from the active clause in terms of verbal morphosyntax is considered a crucial criterion for identifying a distinct passive voice in a language  by some \citep{WALS107},
while others \citep{Haspelmath:1990,Dryer:1982} do not include morphosyntactic marking of passives as a necessary condition. 
The two criteria relating to the status of S and P arguments in passives as compared to basic clauses are more problematic. 
The subjecthood of the logical object is taken as a hard criterion for identifying passives by many linguists -- for instance by \citet{Munro:1976} on Mojave\il{Mojave}. 
Subjecthood is identified via case-marking and/or verbal agreement. In languages in which passive `subjects' deviate from the
standard subject-marking, such constructions can still unproblematically be included under the term valency-decreasing operation, however.\is{valency-decreasing construction!passive|)}

Apart\is{valency-decreasing construction!antipassive|(} from passives, there are two other voice operations that reduce the number of syntactic arguments in a clause: \textsc{Antipassive} and \textsc{Middle}.
The antipassive is a structure most commonly associated with languages of the ergative"=absolutive type.
In antipassive sentences a verb that has two semantic arguments only realizes the A argument of its usual argument structure. 
The P argument is not realized and the verb treats the remaining A argument syntactically like an S argument, marking it with absolutive\is{case!individual forms!absolutive} rather than ergative\is{case!individual forms!ergative} case for instance.\footnote{As with passive agents, the P argument can be overtly realized in some languages as a non-argument phrase, for example with a special oblique case.} 
However, the same label is nowadays applied to parallel constructions in languages with other alignment systems. 
For example, the Surmic language Tennet\il{Tennet} has both a passive\is{valency-decreasing construction!passive} (\ref{TenPassive}) and an antipassive construction (\ref{TenAntiPassive}).

\begin{exe}\ex\label{TenPassive}\langinfo{Tennet}{Surmic; Sudan}{\citealp[245]{Randal:1998}} 
\begin{xlist}
\ex\gll\textipa{\'{\=*a}-r\'{\=*u}h-w-\=*e} \textipa{\=*id\=*ong} \textipa{\'{\=*I}y\'{\=*o}k\=*o}\\
\ipfv{}-beat-\epen{}-\pass{} drum.\nom{} now\\
\glt `The drum is being beaten now.' 

\ex\gll\textipa{\'{\=*a}-r\'{\=*u}h} \textipa{enn\'e} \textipa{\'{\=*I}d\'{\=*o}ng} \textipa{\'{\=*I}y\'{\=*o}k{\=*o}}\\
\ipfv{}-beat 3\sg{}.\nom{} drum.\acc{} now\\
\glt `He is beating the drum now.' 
\end{xlist}
\end{exe}

\begin{exe}\ex\label{TenAntiPassive}\langinfo{Tennet}{}{\citealp[245]{Randal:1998}}
\begin{xlist}
\ex\gll\textipa{\'a-d\'ah-ye} \textipa{d\=*ol\'{\=*e}c}\\
\ipfv{}-eat-\antip{} child.\nom{}\\
\glt `The child is eating.' 

\ex\gll\textipa{\'a-d\'ah} \textipa{d\=*ol\'{\=*e}c} \textipa{\'ah\'at}\\
\ipfv{}-eat child.\nom{} asida.\acc{}\\
\glt `The child is eating asida.' 
\end{xlist}
\end{exe}\is{valency-decreasing construction!antipassive|)}

The\is{valency-decreasing construction!middle|(} last valency-decreasing operation that is relevant here is the middle.
The middle is often interpreted as the voice in between active and passive \citep[3--4]{Klaiman:1991}. 
In this construction, the role of the agent is not exactly downplayed, but rather the fact that an agent is involved is not considered. 
The semantic difference between active (\ref{EngVoice}a), passive (\ref{EngVoice}b) and middle (\ref{EngVoice}c) sentences is tentatively illustrated by the following English\il{English} examples. Note, however, that the middle as a distinct voice is only identified in a small set of languages and in the English\il{English} context the construction in (\ref{EngVoice}c) is rather described as an inchoative \citep[2--3]{Levin:1993}. 

\begin{exe}\ex\label{EngVoice}
\begin{xlist}
\ex \textit{The ball broke the vase.}
\ex \textit{The vase was broken (by the ball).}
\ex \textit{The vase broke.}
\end{xlist}
\end{exe}\is{valency-decreasing construction!middle|)}

A\is{valency-decreasing construction!detransitivizing|(} functionally less restrictive argument-reducing voice operation is usually referred to as a general \textsc{detransitivizing} operation.
Such an operation adds a special marker to a semantically transitive verb indicating that the verb is used as a syntactically intransitive verb. 
One of the arguments of the verb is deleted, but there are no syntactic restrictions regarding which argument of a transitive verb is not realized. 
Thus, basically any of the two arguments of a transitive verb could be deleted in a detransitivizing operation. 
However, there can be semantic restrictions or at least general tendencies for an individual verb on whether the A or P argument is deleted in this type of operation.\is{valency-decreasing construction!detransitivizing|)} 

Apart from the formal criteria listed above, the pragmatics of a construction are often also taken into account when identifying valency-decreasing operations.
Pragmatic criteria are for instance a central factor in the discussion of passives by \citet{Passives}. 
Even if a language does not have a passive construction (or any of the other voice operations discussed in this chapter), it will still have means to encode the same discourse functions associated with passives.
Syntactically transitive constructions can have an unspecified A argument in many languages.
The German\il{German} `man'-construction as in \emph{man spricht deutsch.}  `one speaks German (here)' is an instance of this.
Such constructions are often referred to as `impersonal\is{valency-decreasing construction!impersonal construction} constructions'.
In this type of construction the role of the agent is downplayed. 
The logical P argument of the construction remains in this position syntactically, but pragmatically it is the most salient argument. 
The logical A argument of this construction is unknown or irrelevant in the given situation. 
Therefore, the A argument is not realized lexically. 
From a syntactic point of view, these constructions are transitive, and thus do not decrease the number of arguments of a verb.
Therefore such constructions are not relevant for the further discussion in this chapter.\is{valency-decreasing construction|)}  

 
%%%%%%%%%%%%%%%%%%%%%%%%%%%%%%%%%%%%%%%%%%%%%%%%%%%%%%%%
%\section{Valency increasing operations}\label{ValIncr}
%%%%%%%%%%%%%%%%%%%%%%%%%%%%%%%%%%%%%%%%%%%%%%%%%%%%%%%%

%\citet{Payne:1997} lists `causatives', `applicatives' and `dative shift' as the mayor valency-increasing operations in the world's languages. 
%In causatives an argument is introduced, which though not necessarily being the direct agent of the predicate of a clause, does control and instigate the action (\ref{EngCaus}). Since control and instigation are common properties of subjects, causers are realized as grammatical subjects in the majority of the worlds languages. 

%\begin{exe}
%\ex\label{EngCaus}
%\begin{xlist}
%\ex The tree fell. 
%\ex Peter felled the tree. 
%\end{xlist}
%\end{exe}

%I will not be concerned here with the syntactic realization of the causative construction. If a language distinguishes between periphrastic and morphological causative, I will simply note so in the discussion of the data. My study focusses on case-marking and therefore such differences play no role.

%In the other valency increasing operations -- i.e. applicatives and dative shift -- the subject relation is not changed between the basic clause and the valency adjusted clause. Also, the arguments introduced through these devices do not have noteworthy subject properties.\footnote{Especially arguments introduced by applicatives do not resemble subjects, while for dative shift one might argue that the promoted argument, the goal or recipient, shares prototypical subject properties such as animacy, however, not the control property.} Therefore, these two operations are not further investigated in my sample of marked"=S languages. 


%%%%%%%%%%%%%%%%%%%%%%%%%%%%%%%%%%%%%%%%%%%%%%%%%%%%%%%%%%%%%%%%%
\section{Research questions}\label{QueNonBasic}
%%%%%%%%%%%%%%%%%%%%%%%%%%%%%%%%%%%%%%%%%%%%%%%%%%%%%%%%%%%%%%%%%

Case-marking in non-basic clauses does differ in a number of languages of the marked"=S type.
However, this is not an exclusive feature of the languages studied here.
Deviating patterns of case-assignment in non-basic clauses are also commonly found in other languages (i.e. non-marked-S languages).
In this section I will demonstrate (with examples from languages of the marked"=S type) how subjects of non-basic clauses behave differently from prototypical subjects of basic clauses. 
In most instances, the factor that differentiates between the clauses is the assignment of S-case to basic clause subjects, while non-basic subjects receive some other case-marking, usually the zero-case. 
However, there are often some other structural differences between the two types of structures such as in word order\is{word order}, or verbal indexing \is{verbal indexing}. 
I will note these differences when discussing the data. 
However, the main focus of this chapter is on the case-marking. 

In the following sections, I will discuss how marked"=S languages mark the case of the subject element of non-basic clauses. 
This includes the following: 

\begin{itemize}
\item case-marking of subjects in all types of dependent clauses\is{clause-type!dependent} (i.e. relative clauses\is{clause-type!relative clause}, adverbial clauses\is{clause-type!adverbial clause}, complement clauses)
%\item subjects introduced by  valency-increasing operations
\item (promoted) subjects of valency-decreasing operations\is{valency-decreasing construction}
\end{itemize}

A\is{clause-type!dependent|(} number of marked"=S languages do not mark subjects of dependent clauses in the same way they mark subjects of main clauses.
Wappo\il{Wappo}\is{clause-type!relative clause|(} relative clauses, for example, leave the internal subject of the relative clause  zero-coded (\ref{WapDepEx}b). 
It never receives Nominative case like it would in main clauses (\ref{WapDepEx}a). Similarly\is{clause-type!adverbial clause|(}, in adverbial (\ref{WapDepEx}c) and complement\is{clause-type!complement clause|(} clauses (\ref{WapDepEx}d) the subject remains zero-coded.

\begin{exe}\ex\label{WapDepEx} \langinfo{Wappo}{Wappo-Yukian; California}{\citealt[4, 117, 77]{Thompsonetal:2006}, \citealt[239]{Adverbial}}
\begin{xlist}
\ex\gll ce ew ce k'ew\textbf{-i} t'um-ta\textglotstop\\
\dem{} fish \dem{} man-\nom{} buy-\pst{}\\
\glt `That fish, the man bought (it).'

\ex \gll  {\rm[}ce \textbf{k'ew} ew t'um-ta{\rm]} cephi i naw-ta\textglotstop\\
          \hspaceThis{[}\dem{} man fish buy-\pst{}.\dep{} 3\sg{}.\nom{} 1\sg{}.\acc{} see-\pst{}\\
\glt `The man who bought the fish saw me.'

\ex\label{WapCondCl}\gll  {\rm[}\textipa{\textbf{te}} \textipa{ce} \textipa{ew} \textipa{t'ume} \textipa{cel'}{\rm]} \textipa{keye} \textipa{ah} \textipa{ce} \textipa{pa\textglotstop eh}\\
                          \hspaceThis{[}3\sg{}.\acc{} \dem{} fish buy.\dep{} \cond{} \opt{} 1\sg{}.\nom{} \dem{} eat.\Hyp{}\\
`If he had bought the fish, I would have eaten it' %\citet[77]{Thompsonetal:2006}

\ex\gll\textipa{ah}  {\rm[}\textbf{\textipa{te}} \textipa{\v sawo} \textipa{pa\textglotstop-tah}{\rm]} \textipa{hais-khi\textglotstop}\\
       1\sg{}.\nom{} \hspaceThis{[}3\sg{}.\acc{} bread.\acc{} eat-\pst{} know-\neu{}\_\fut{}\\
\glt `I know that he ate bread.' 
\end{xlist} 
\end{exe}\is{clause-type!adverbial clause|)}\is{clause-type!complement clause|)}

In other languages, however, subjects in different types of dependent clauses receive different kinds of case-marking.
While Murle\il{Murle} relative clauses mark their subjects in Nominative\is{case!individual forms!nominative}  case (\ref{MurDepEx}a), complement\is{clause-type!complement clause|(} clauses have zero-coded subjects (\ref{MurDepEx}b).
Mojave\il{Mojave} exhibits yet another pattern: subjects of relative clauses are zero-coded (\ref{MojDepEx}a) but subjects of adverbial clauses\is{clause-type!adverbial clause} are in the Nominative\is{case!individual forms!nominative}  case (\ref{MojDepEx}b). 

\begin{exe}\ex\label{MurDepEx}\langinfo{Murle}{Surmic; Sudan}{\citealp[112, 113]{Arensen:1982}}
\begin{xlist}
\ex\gll\textipa{kEEti} \textipa{naana} \textipa{kiziwan}  {\rm[}\textipa{o} \textipa{or} \textbf{\textipa{niina}}{\rm]}\\
skin.1\sg{} 1\sg{}.\nom{} buffalo.\acc{} \hspaceThis{[}which shoot.\pst{} 2\sg{}.\nom{}\\
\glt `I am skinning the buffalo which you shot.'

\ex\gll\textipa{kaga} \textipa{naana}  {\rm[}\textbf{\textipa{nOnnO}} \textipa{aak} \textipa{idiN}{\rm]}\\
know 1\sg{}.\nom{} \hspaceThis{[}3\sg{}.\acc{} cook meat.\acc{}\\
\glt `I know that she is cooking meat.'
\end{xlist}
%\end{exe}
\is{clause-type!complement clause|)}

%\begin{exe}
\ex\label{MojDepEx}\langinfo{Mojave}{Yuman; Arizona}{\citealp[188]{Munro:1976}, \citealp[144]{Munro.com:1980}}
\begin{xlist}
\ex\gll  {\rm[}\textbf{\textipa{Tin\super{y}a\textglotstop a:k}} \textipa{mat=k@h\super{w}el\super{y}} \textipa{k\super{w}-n\super{y}avay{\rm]}-n\super{y}-\v c} \textipa{\textglotstop-ahvay-n\super{y}} \textipa{i\v co:-k}\\
\hspaceThis{[}woman Parker \relativ{}-live-\dem{}-\nom{} 1-dress-\dem{} make-\tns{}\\
\glt `The woman who lives in Parker made my dress.'

\ex\gll  {\rm[}\textglotstop inye\textbf{-\v c} pap \textglotstop-akchoor-m{\rm]} judy\v c salyii-k\\
\hspaceThis{[}1\sg{}-\nom{} potato 1-peel-\dsbj{} Judy-\nom{} fry-\tns{}\\
\glt `After I peeled the potatoes, Judy fried them.' %[144]{Munro.com:1980}\end{xlist}
\end{xlist}
\end{exe}

Still other marked"=S languages do not show any difference in the case-marking of subjects of main and dependent clauses.
In the Harar\il{Oromo (Harar)} dialect of Oromo, subjects receive the regular Nominative\is{case!individual forms!nominative}  case-marking in relative (\ref{HarDepEx}a) as well as adverbial clauses\is{clause-type!adverbial clause} (\ref{HarDepEx}b).

\pagebreak
\begin{exe} \ex\label{HarDepEx}\langinfo{Oromo (Harar)}{Eastern Cushitic; Ethiopia}{\citealp[131, 143]{Owens:1985}}
\begin{xlist}
\ex\gll namicc-\'\i i  {\rm[}(xan) intal-t\'\i i is\'a bar\'eed-d\'uu{\rm]} \'ac jira\\
man-\nom{} \hspaceThis{[(}as girl-\nom{} 3\sg{}.\mas{}.\acc{} pretty-\fem{} there exist.3\sg{}.\mas{}\\
\glt `The man whose daughter is pretty is there.'

\ex\gll  {\rm[}hag\'a is\'\i in d'uf-t-\'u{\rm]} taa'-e\\
\hspaceThis{[}until 3.\sg{}.\fem{}.\nom{} come-\fem{}-\dep{} stay-\pst{}\\
\glt `He stayed until she came.'
%\citet[143]{Owens:1985}
\end{xlist}
\end{exe}\is{clause-type!relative clause|)} 
\is{clause-type!dependent|)}

I\is{valency-decreasing construction|(} will now turn to the different patterns of marking subjects in valency-de\-crea\-sing constructions.  
As already noted in Section~\ref{ValDec},  
Tennet\il{Tennet} has both a passive\is{valency-decreasing construction!passive} (\ref{TenVDO}a) and an antipassive\is{valency-decreasing construction!antipassive} (\ref{TenVDO}b). 
In both constructions the subject receives Nominative\is{case!individual forms!nominative}  case-marking. 

\begin{exe}\ex\label{TenVDO}\langinfo{Tennet}{Surmic; Sudan}{\citealp[245]{Randal:1998}}
\begin{xlist}
\ex\gll\textipa{\'{\=*a}-r\'{\=*u}h-w-\=*e} \textipa{\=*id\=*ong} \textipa{\'{\=*I}y\'{\=*o}k\=*o}\\
\ipfv{}-beat-\epen{}-\pass{} drum.\nom{} now\\
\glt `The drum is being beaten now.' 

\ex\gll\textipa{\'a-d\'ah-ye} \textipa{d\=*ol\'{\=*e}c}\\
\ipfv{}-eat-\antip{} child.\nom{}\\
\glt `The child is eating.' 
\end{xlist}
\end{exe}

In Maa\il{Maa}, three grammatical voices are distinguished: middle\is{valency-decreasing construction!middle|(}, antipassive\is{valency-decreasing construction!antipassive}, and the so-called impersonal passive.
In the middle (\ref{MaaVDO}a) and antipassive (\ref{MaaVDO}b) the subject is in the Nominative\is{case!individual forms!nominative}  case. 
In the impersonal\is{valency-decreasing construction!impersonal construction} passive (\ref{MaaVDO}c) on the other hand the subject is in the Accusative\is{case!individual forms!accusative} (i.e. zero-coded case)

\enlargethispage{\baselineskip}
\begin{exe}\ex\label{MaaVDO}\langinfo{Maa}{Nilotic; Kenya}{\citealp[ex.\,11, ex.\,13, ex.\,16, ex.\,15]{Payne:2007}} 
\begin{xlist}
\ex\gll\textipa{N-\'e-duN-o} \textipa{En-\textbf{\'am\`Uk\`E}}\\
\con{}-3-cut-\Mid{}.\Non{}\_\pfv{} \fem{}.\sg{}-sandal.\nom{}\\ %some error in this example
\glt `The shoe was cut.'

\ex\gll\textipa{n-\'e-ramat-\'Ish\`o} \textipa{Ol-\textbf{mUrran\'I}}\\
\con{}-3-tend\_livestock-\antip{} \mas{}.\sg{}-warrior.\nom{}\\
\glt `The warrior herds [e.g. cows].'

\ex\gll\textipa{E-tE-En-\'ak-\`I} \textipa{Ol-\textbf{ap\'urr\`on\`I}}\\
3-\prf{}-tie-\prf{}-\pass{} \mas{}.\sg{}-thief.\acc{}\\
\glt `The thief was arrested.'

\ex\gll\textipa{E-Ib\'UN-\'a} \textipa{I-s'IkarIn\'I}	\textipa{Ol-\textbf{ap\'urr\`on\`I}}\\
3-catch-\prf{} \pl{}-police.\nom{} \mas{}.\sg{}-thief.\acc{}\\
\glt `The policemen have arrested the thief.'
\end{xlist}
\end{exe}\is{valency-decreasing construction|)}
\is{valency-decreasing construction!middle|)}

%Finally turning to valency-increasing constructions, more precisely causatives, little interesting data can be found in marked"=S languages. 
%In general, there is not much variation in case-marking to be found with respect to the marking of causers.
%Most languages treat them like regular subject and mark them with the standard S-case, like the Nominative in Wappo\il{Wappo} (\ref{WapCausEx}).
%\citet[162]{Koenig:2008} states that a number of African marked"=S languages, among these Maa\il{Maa} and Datooga\il{Datooga}, use the Accusative case for participants introduced by valency-increasing operations. As such she lists causative, applicatives and benefactives, however, she does not provide an example that includes the causative to illustrate this behavior. She only provides examples of applicative and benefactive. In these constructions the newly introduced argument is not the subject of the clause, therefore Nominative case would not be expected on them anyways. 

%\medskip
%{Wappo\il{Wappo}} \citep[128,~129]{Thompsonetal:2006}
%\begin{exe}\ex\label{WapCausEx}
%\begin{xlist}\ex\gll\textipa{ah} \textipa{c'ani} \textipa{k'opa-tis-ta\textglotstop}\\
%1\sg{}.\nom{} ice melt-\caus{}-\pst{}\\
%`I melted the ice.'
%\ex\gll\textipa{c'an-ti} \textipa{k'opa-khi\textglotstop}\\
%ice-\nom{} melt-\stat{}\\
%`The ice has melted'
%\end{xlist}
%\end{exe}

Other than with the contexts of existential and locational predication (see Chapter~\ref{existpred}), which were encoded via identical constructions in most languages of my sample, the contexts studied in this chapter are typically encoded by a construction not shared with the other contexts.
Therefore the data in the following sections will be organized by the contexts rather than discussing all contexts for each language at the same time.
As in the previous chapters, the data are organized by geographical and genealogical groupings. 
Section~\ref{NonBasicNA} discusses the languages of North America. Data from the languages of the Pacific are given in Section~\ref{NonBasicPac}. 
And the languages of Africa are presented in Sections~\ref{NonBasicNilo} (Nilo-Saharan) and~\ref{NonBasicAfro} (Afro-Asiatic) respectively.

%%%%%%%%%%%%%%%%%%%%%%%%%%%%%%%
\section{North America}\label{NonBasicNA}
%%%%%%%%%%%%%%%%%%%%%%%%%%%%%%%

In a number of North American marked"=S languages, dependent clauses\is{clause-type!dependent}, and especially relative clauses\is{clause-type!relative clause}, do not mark their subjects with nominative case but leave them zero-coded. 
In Mojave\il{Mojave}, S arguments in valency-decreasing constructions are also left zero-coded. 
However, most languages of the region use the nominative\is{case!individual forms!nominative}  case in this context. 
Generally, the voice systems of the North American languages in my sample are not highly complex, judging from the available data. %Causative constructions behave predictable as well, i.e. they employ nominative case for causers.

In\is{clause-type!relative clause|(}  Wappo\il{Wappo}, S (\ref{WapSRel}) and A arguments (as demonstrated in example (\ref{WapDepEx}b) above)  in relative clauses have the zero-coded form. 
In main clauses these arguments are marked with the Nominative case-suffix \emph{-i} in contrast (\ref{WapSMain}).
Note that it is only subject case-marking via Nominative case which is absent from relative clauses. 
Case-marking of recipient arguments via Dative\is{case!individual forms!dative} case is preserved in relative clauses (\ref{WapDat}).

\begin{exe}\ex\label{WapRelS}\langinfo{Wappo}{Wappo-Yukian; California}{\citealp[117, 41]{Thompsonetal:2006}}
\begin{xlist} 
\ex\label{WapSRel} \gll {\rm [}ce \textbf{k'ew} kat'akh{\rm ]} cephi k'e\v su peh-khi\textglotstop\\
    \dem{} man laugh.\stat{}.\dep{} 3\sg{}.\nom{} deer look\_at-\stat{}\\
\glt `The man who laughed is looking at the deer.' %[117]
\ex\label{WapSMain} \gll hay\textbf{-i} ho\textglotstop -ta\textglotstop\\
     dog-\nom{} bark-\pst{}\\
\glt `The dog barked.' %[41]
\end{xlist}
\end{exe}

\enlargethispage{\baselineskip}

\begin{exe}\ex\label{WapDat}\langinfo{Wappo}{}{\citealp[117, 12]{Thompsonetal:2006}}
\begin{xlist} 
\ex\label{WapDatRel}\gll  {\rm[}mi ce k'ew\textbf{-thu} {taka\textglotstop} ma-hes-ta{\rm]} (ce) ah naw-ta\textglotstop\\
2\sg{} \dem{} man-\dat{} basket \dir{}-give-\pst{}.\dep{} \dem{} 1\sg{}.\nom{} see-\pst{}\\
\glt `I saw the man you gave the basket to.' %[117]

\ex\label{WapDatMain}\gll ce k'ew-i chica\textbf{-thu}  ew ma-hes-ta\textglotstop\\
\dem{} man-\nom{} bear-\dat{} fish \dir{}-give-\pst{}\\
\glt `The man gave the fish to the bear.' %[12]
\end{xlist}
\end{exe}\is{clause-type!relative clause|)}

But not only subjects of relative clauses are left without overt case-marking in this language.
Adverbial\is{clause-type!adverbial clause|(} clauses exhibit the same pattern, as exemplified by the following  temporal (\ref{WapTempCl}a) and conditional clauses (\ref{WapTempCl}b). 
Complement\is{clause-type!complement clause|(} clauses (\ref{WapTempCl}c) have zero-coded subjects as well.

\begin{exe}\ex\label{WapTempCl}\langinfo{Wappo}{}{\citealp[71, 77]{Thompsonetal:2006}, \citealp[239]{Adverbial}}
\begin{xlist}
%\ex\gll {\rm[}\textipa{mi} \textipa{mi-noma} \textipa{mu-leP a-cel'}{\rm]} \textipa{u\v ci} \textipa{ola} \textipa{miP} \textipa{hintolik-siP}\\
%2\sg{}.\acc{} 2\sg{}-home \dir{}-arrive-when night four 2\sg{}.\nom{} sleep-\fut{}\\
%`When you get to your home, you'll sleep for four nights' %\citet[71]{Thompsonetal:2006}
\ex\gll  {\rm[}\textipa{ce} \textipa{\textbf{layh}} \textipa{tu-lePa-cel'}{\rm]} \textipa{okal'te-lahkhiP}\\
\hspaceThis{[}\dem{} white\_person \dir{}-arrive-when talk.\ipfv{}-\Neg{}\\
\glt `When the white man comes, don't talk' %[71]{Thompsonetal:2006}

\ex\gll  {\rm[}\textipa{\textbf{mi}} \textipa{te} \textipa{o-me\textglotstop-is} \textipa{cel'}{\rm]} \textipa{keye} \textipa{\v choPe-lahkhih}\\
\hspaceThis{[}2\sg{}.\acc{} 3\sg{}.\acc{} \uop{}-feed-\caus{} \cond{} \opt{} die.\imp{}-\Neg{}.\Hyp{}\\
\glt `If you had fed it, it wouldn't have died' %\citet[77]{Thompsonetal:2006}

\ex\gll\textipa{ah}  {\rm[}\textbf{\textipa{te}} \textipa{\v sawo} \textipa{paP-tah}{\rm]} \textipa{hais-khiP}\\
1\sg{}.\nom{} \hspaceThis{[}3\sg{}.\acc{} bread.\acc{} eat-\pst{} know-\Non\_\fut{}\\
\glt `I know that he ate bread.' 
\end{xlist}
\end{exe}\is{clause-type!adverbial clause|)}\is{clause-type!complement clause|)}

With\is{clause-type!relative clause|(} respect to relative clauses the languages of the Yuman family exhibit a similar pattern.
In Mojave\il{Mojave}, for example, nouns serving as S (\ref{MojRelS}) or A argument (\ref{MojRelA}) of a relative clause are in the zero-coded form according to \citet[187--190]{Munro:1976}.

\begin{exe}\ex\label{MojRelS}\langinfo{Mojave}{Yuman; Arizona}{\citealp[188]{Munro:1976}}
\begin{xlist}
\ex\gll  {\rm[}\textipa{\textglotstop ava:} \textipa{k\super{w}-n\super{y}@m@savc{\rm]}-l\super{y}} \textipa{\textglotstop-iva-m}\\
\hspaceThis{[}house \relativ{}-white-\loc{} 1-sit-\tns{}\\
\glt `I am in the white house.'
\ex\gll \textipa{Pava:\textbf{-\v c}} \textipa{n\super{y}@m@sa:-m}\\
house-\nom{} white-\tns{}\\
\glt `The house is white.'
\end{xlist}
%\end{exe}

%\enlargethispage{2\baselineskip}
%\begin{exe}
\ex\label{MojRelA}\langinfo{Mojave}{}{\citealp[188]{Munro:1976}}
\begin{xlist}
\ex\gll  {\rm[}\textipa{hat\v coq} \textipa{po\v s} \textipa{k\super{w}-taver}{\rm]} \textipa{P-iyu:-p\v c}\\
\hspaceThis{[}dog.\acc{} cat.\acc{} \relativ{}-chase 1-see-\tns{}\\
\glt `I saw the dog that chased the cat.'

\ex\gll \textipa{hat\v coq\textbf{-\v c}} \textipa{po\v s} \textipa{taver-m}\\
dog-\nom{} cat.\acc{} chase-\tns{}\\
\glt `The dog chased the cat.'
\end{xlist}
%\begin{flushright}\citet[188]{Munro:1976}\end{flushright}
\end{exe}

\citet[333--334]{Dixon:2010-2} discusses relative clauses in Mojave\il{Mojave} based on Munro's data with a slightly different interpretation. 
Following Munro, he distinguishes between relative clauses in which the common argument is the subject of the relative clause and those in which it is not. 
In the former type, the verb of the relative clause is marked by the prefix \emph{k\textsuperscript{w}-} and according to Dixon's analysis the common argument is stated in the main clause and not realized as independent NP in this type of  relative clause. 
Accordingly, the CA is case-marked for its function in the main clause and not the relative clause. 
If the common argument does not function as the relative clause's subject, then the verb prefix is missing and the common argument is realized in the relative clause. 
In example (\ref{MojRelA}a) above, according to his analysis, the noun \emph{hat\v coq} `dog' is the P argument of  the main clause's predicate (`to see') and the Accusative\is{case!individual forms!accusative} form is thus expected.  

In the following example (\ref{MojRelSubj}), the CA serves as the subject of both main and relative clause. 
As is to be expected for subject relative clauses, the verb of the relative clause is marked by the prefix \emph{k\textsuperscript{w}-} (just like in (\ref{MojRelA}a)).  
However, the Nominative case is missing from the noun phrase \emph{\textschwa in\textsuperscript{y}a\textglotstop a:k-n\textsuperscript{y}} `that woman'.
Instead, the relative clauses as a whole is case-marked for the role. 
It is not clear to me how this can be explained in Dixon's analysis, which holds that in subject relative clauses the common argument is not realized in the main clause.
This behavior of marking the relative clause for the function the common argument bears in the main clause is also found in example (\ref{MojRelSub}), in which the relative clause is marked with Locative case.
This example is a non-subject relative clause. 
Accordingly, the subject of the relative clause is realized inside the relative clause since it is not an argument of the main clause. 
As in the examples of subject relative clauses above, the subject is zero-coded.\footnote{The relative clause in (\ref{MojRelS}) is also marked with Locative case. 
It functions as location in the main clause, but bears the subject relative prefix \emph{k\textsuperscript{w}-}. 
\citet{Munro:1976} does not comment on this.} 
\enlargethispage{\baselineskip}

\begin{exe}\ex\label{MojRelSubj}\langinfo{Mojave}{}{\citealp[188]{Munro:1976}}
\begin{xlist}
\ex\gll  {\rm[}\textipa{Tin\super{y}aPa:k} \textipa{mat=k@h\super{w}el\super{y}} \textipa{k\super{w}-n\super{y}avay{\rm]}-n\super{y}-\v c} \textipa{P-ahvay-n\super{y}} \textipa{i\v co:-k}\\
\hspaceThis{[}woman Parker \relativ{}-live-\dem{}-\nom{} 1-dress-\dem{} make-\tns{}\\
\glt `The woman who lives in Parker made my dress.'

\ex\gll\textipa{Tin\super{y}aPa:k-n\super{y}\textbf{-\v c}} \textipa{mat=k@h\super{w}el\super{y}} \textipa{n\super{y}avay-n\super{y}-k}\\
woman-\dem{}-\nom{} Parker live-\tns{}\\
\glt `The woman lives in Parker.'
\end{xlist}
%\end{exe}
\enlargethispage{\baselineskip}
%\begin{exe}
\ex\label{MojRelSub}\langinfo{Mojave}{}{\citealp[451]{Munro:1977}, \citealt[221]{Munro:1976}}
\begin{xlist}
\ex\gll  {\rm[}\textipa{\textglotstop-nakut} \textipa{\textglotstop ava} \textipa{u:\v co:}{\rm]}\textipa{-l\super{y}} \textipa{\textglotstop-navay-k}\\
\hspaceThis{[}1-father house make-\loc{} 1-live-\tns{}\\
\glt `I live in the house my father built.'

\ex\gll \textipa{Pin\super{y}ep} \textipa{P-nakut\textbf{-\v c}} \textipa{Pava:} \textipa{vidan\super{y}} \textipa{i\v co:-k}\\
me 1-father-\nom{} house this make-\tns{}\\
\glt `My father built this house.'
\end{xlist}
\end{exe}\is{clause-type!relative clause|)} 

% Subject of Rel Clause not argument of Main clause, still no Nom Case

%\begin{exe}\ex\gll [\textbf{\textipa{dean}} \textipa{masahay} \textipa{u:yu:-m]-n\super{y}-\v c} \textipa{iva:-k-@}\\
%Dean girl see-\Neg{}-\dem{}-\nom{} arrive-\tns{}-AUG\\
%`The girl Dean didn't see is here.'
%\citet[213]{Munro:1976}

%\begin{exe}\ex\gll tunay pi:pa \textglotstop-u:yu:-n\textsuperscript{y}-\v c ka\textglotstop a:k-k\\
%yesterday person 1-see-\dem{} dog bite-\dem{}-\nom{} cat kick-\tns{}\\
%`The man I saw yesterday who the dog bite kicked the cat.'\end{exe} unknown source, alignment messed up

Other\is{clause-type!adverbial clause|(} than relative clauses, Mojave\il{Mojave} adverbial clauses straightforwardly mark their subjects with the Nominative\is{case!individual forms!nominative} .
Case-marking of subjects of temporal clauses is illustrated by the examples in (\ref{MojTemp}).  

\begin{exe}\ex\label{MojTemp}\langinfo {Mojave}{}{\citealp[12]{Munro.com:1980}, \citealp[322]{Langdon.Munro:1979}}
\begin{xlist}
\ex\gll  {\rm[}\textglotstop iipa iiwa-ny\textbf{-\v c} nya-chalahop-m{\rm]} isma-mot-e\\
\hspaceThis{[}man heart-\dem{}-\nom{} when-empty-\dsbj{} sleep-\Neg{}-\fut{}\\
\glt `When a man is lonely, he can't sleep.'\\
lit.: `When a man's heart is empty, he can't sleep.' %[12]{Munro:1980}

\ex\gll  {\rm[}\textglotstop i:k\textsuperscript{w}i:\v c-v\textschwa\textbf{-\v c} n\textschwa k\textschwa mi\v c-m{\rm]} \textglotstop\textschwa-taly-\v c zu:pa:\\
\hspaceThis{[}men-\dem{}-\nom{} return.\pl{}-\dsbj{} 1-mother-\nom{} crack\_acorns\\
\glt `When the men came back, my mother cracked acorns.' %\citep[322]{Langdon.Munro:1979}
\end{xlist}
\end{exe}\is{clause-type!adverbial clause|)}

Complement\is{clause-type!complement clause|(} clauses are not discussed in any detail in the Mojave\il{Mojave} literature.  
\citet[232--234]{Munro:1976} notes that subordinate clauses with \emph{n\textsuperscript{y}a-} `when'/`if' and the switch reference/tense marker \emph{-k} and \emph{-m} have Nominative\is{case!individual forms!nominative}  subjects. 
While the when-and-if-clauses are of the adverbial type discussed above, the switch reference markers apparently are used in complementation, as in the following examples. 
Example (\ref{MojComp}a) marks the subjects of main (\emph{in\textsuperscript{y}e\v c} `I') and complement clause (\emph{Judy}) with Nominative\is{case!individual forms!nominative}  case.
In the second example, the switch reference marker is missing, however. 
Judging from the translation, this is an instance of complementation. 
In this example, the subject \emph{i\v cu:ra:v} `man' is not marked with the Nominative\is{case!individual forms!nominative}  case (\ref{MojComp}b), which it receives when the complement is realized as an independent main clause (\ref{MojComp}c). 
But this might be explained by the general possibility of Nominative case-marking to be
dropped\is{case-marking!optional} in Mojave\il{Mojave}, especially in fast speech.%
\enlargethispage{\baselineskip}

\begin{exe}\ex\label{MojComp}\langinfo{Mojave}{}{\citealp[274, 220]{Munro:1976}}
\begin{xlist}
\ex\gll\textipa{\textglotstop in\super{y} e\v c}  {\rm[}\textipa{judy\textbf{-\v c}} \textipa{iva:-p-m}{\rm]} \textipa{\textglotstop-su:paw-\v c}\\
1\sg{}.\nom{} \hspaceThis{[}Judy-\nom{} arrive-p-\dsbj{} 1-know-?\\
\glt `I know that Judy has arrived.'
%\begin{flushright}\citet[274]{Munro:1976}\end{flushright}

\ex\gll  {\rm[}pa \textglotstop i\v c u:ra:v{\rm]}-n\textsuperscript{y} \textglotstop-a\textglotstop a:v-k-e\\
\hspaceThis{[}man something hurt-\dem{} 1-hear-\tns{}-\augv{}\\
\glt `I know the man is sick.'
\ex\gll pa\textbf{-\v c} \textglotstop i\v c ira:v-k\\
man-\nom{} something hurt-\tns{}\\
\glt `The man is sick.'
\end{xlist} 
\end{exe}\is{clause-type!complement clause|)}

%\begin{exe}\ex
%\begin{xlist}\ex\gll \textipa{hatchoq-ch} \textipa{posh} \textipa{taver-m} \textipa{'-iyuu-m}\\
%dog-\nom{} cat chase-\dsbj{} 1-see-\tns{}\\
%`I saw the dog that chased the cat.'; `I saw the cat that the dog chased.'; `I saw the dog chase the cat.'; `The dog chased the cat and I saw it.'
%\ex\gll\textipa{'inye-ch} \textipa{hatchoq-ch} \textipa{posh} \textipa{taver-m} \textipa{'-iyuu-m}\\
%1\sg{}-\nom{} dog-\nom{} cat chase-\dsbj{} 1-see-\tns{}\\
%`I saw the dog that chased the cat.'; `I saw the cat that the dog chased.'; `I saw the dog chase the cat.'; `The dog chased the cat and I saw it.'
%\end{xlist}
%\end{exe}
%\begin{flushright}\citet[147]{Munro.com:1980}\end{flushright}

In general, the case-marking of dependent\is{clause-type!dependent} clauses exhibits parallel structures across the languages of the Yuman family.
For\is{clause-type!relative clause|(} {Jamul\il{Jamul Tiipay} Tiipay}, \citet[210]{Miller:2001} states that within a relative clause subjects always appear in the Accusative\is{case!individual forms!accusative} case (\ref{JamDep}a, b) while oblique noun phrases might be case-marked. 
Similarly to Mojave\il{Mojave}, adverbial clauses show a different pattern, cf. the purpose clause in (\ref{JamDep}c) that overtly marks its subject with Nominative\is{case!individual forms!nominative}  case.
In Yavapai\il{Yavapai} as well, subjects of  relative clauses are zero-coded (\ref{YavDep}a, b), while adverbial clauses mark their subjects with Nominative\is{case!individual forms!nominative}  case as exemplified by the conditionals in (\ref{YavDep}c). 
The same discussion as for Mojave\il{Mojave} could be held about the location of the common argument, i.e. whether it is in or outside the relative clause. 
In example (\ref{YavDep}a), the cat serves as the subject of the relative as well as main clause and thus the absence of case-marking cannot be explained on account of its role in the main clause (as proposed by Dixon for Mojave\il{Mojave}). 
Example (\ref{YavDep}b), on the other hand, is ambiguous since the coyote is the object of the main clause and as such would be expected to be zero-coded. 
Also, there is an alternative construction which is often translated as a relative clause into English (\ref{YavDep}d), in which the subject receives Nominative\is{case!individual forms!nominative}  case. \citet[221]{Kendall:1976} treats it as some kind of topicalization construction.

\begin{exe}\ex\label{JamDep}\langinfo{Jamul Tiipay}{Yuman; California}{\citealp[207, 210, 259]{Miller:2001}}
\begin{xlist}
\ex\gll  {\rm[}'iipay peya nye-kwe-'iny-pe{\rm]}-ch mespa\\
\hspaceThis{[}man this 3\textgreater 1-\sbj{}.\relativ{}-give-\dem{}-\nom{} die\\
\glt `The man who gave me this died.'

\ex\gll  {\rm[}leech Marii chshaak-pu{\rm]} mamwi-aa\\
\hspaceThis{[}milk Maria bring\_towards-\dem{} 2.do\_what-\question{}\\
\glt `What did you do with the milk Maria brought?'
%\citet[]{Miller:2001} %Subject of rel clause no arg of main clause

%\ex\gll [\textbf{nyaap} uuy\'awa] nyilly*wa \\
%1\sg{}.\acc{} know.ORS in\_it*be\_located \\
%somebody I know is in there.'%\citet[210]{Miller:2001}

\ex\gll  {\rm[}\textbf{maach} kaavaay peya me-llywa-x-ich{\rm]} uukwii\\
\hspaceThis{[}2\sg{}.\nom{} horse this 2-ride-\irr{}-\purp{} buy\\
\glt `I bought this horse for you to ride.' %\textbf{Jamul\il{Jamul Tiipay} Tiipai} \citep[259]{Miller:2001}

\end{xlist}
\end{exe}

%\enlargethispage{2\baselineskip}
\pagebreak
\begin{exe}\ex\label{YavDep}\langinfo{Yavapai}{Yuman; Arizona}{\citealp[51, 213, 24, 221]{Kendall:1976}}
\begin{xlist}
\ex\gll  {\rm[}\textipa{\~nmi} \textipa{vqi} \textipa{hma\~n} \textipa{k-ttmo:-c}{\rm]} \textipa{hma\~n} \textipa{hme-ha} \textipa{ck\super{y}o:-k\~n}\\
         \hspaceThis{[}cat female child \relativ{}-scratch-\nom{} child male\_that bite-\compl{}\\
\glt `The cat that scratched the girl bit the boy.'% Common argument is subject in main and rel clause, no Nom case-marking

\ex\gll  {\rm[}\textipa{kiTar q\super{w}ar} \textipa{qoleyaw} \textipa{k-ne:h-a}{\rm]} \textipa{\textglotstop-u:-k\~n}\\
         \hspaceThis{[}coyote chicken \relativ{}-kill-\tns{} 1-see-\compl{}\\
\glt `I saw the coyote that killed the chicken.' %\citet[213]{Kendall:1976}

\ex\gll  {\rm[}\textipa{vqi} \textipa{hma\~n\textbf{-c}} \textipa{\~nmi} \textipa{vhe:} \textipa{syo:m-kiTo}{\rm]} \textipa{\~nmi-c} \textipa{ttmo:-ha}\\
         \hspaceThis{[}woman child-\nom{} cat tail pull-\cond{} cat-\nom{} scratch-\fut{}\\
\glt `If the girl pulls the cat's tail, it will scratch (her).'
%\citet[24]{Kendall:1976}

%\ex\gll [\textipa{kopica-c} \textipa{hamsi} \textipa{\textglotstop i:\textglotstop i:} \textipa{ck\super{y}at-a} \textipa{\~n-om-kiTo}] \textipa{hamsi-c} \textipa{way alay-ha}\\
%Gobicha-\nom{} Hamsi wood cut-\tns{} when-not-\cond{} Hamsi-\nom{} angry-\fut{}\\
%`If Gobicha_i doesn't cut wood for Hamsi, she_i will get angry.'
%\citet[23]{Kendall:1976}\end{exe} 

\ex\gll\textipa{can\textbf{-c}} \textipa{k\super{w}e civiam-l} \textipa{wa-m} {\rm[}\textipa{\~nTa\textglotstop a} \textipa{pil-c} \textipa{kkav-k} \textipa{no:-km}{\rm]}\\
       John-\nom{} car-\ines{} sit-\allo{} \hspaceThis{[}that\_one\_visible Bill-\nom{} buy-\ego{} \fut{}-\icml{}\\
\glt `John is sitting in the car that Bill is going to buy.
%\citet[221]{Kendall:1976}\end{exe}

\end{xlist}
\end{exe}\is{clause-type!relative clause|)}

For other Yuman languages, there is only sparse information on dependent clauses\is{clause-type!adverbial clause|(}, usually consisting of just one or two odd examples without any discussion of their structure. 
%\citet[27]{Halpern:1997} notes that subordinate clauses in Yuma have Accusative subjects.
Among these languages is Cocopa\il{Cocopa}.
The example in (\ref{CocDep}a) clearly contains a dependent clause. 
However, its internal structure and type are relatively unclear.
The literal translation is probably something along the lines of `where the king's house is, he arrived at it', which could be interpreted as an adverbial locational clause. 
Whatever the exact semantic type of this clause is, the subject of the dependent clause is in the Nominative\is{case!individual forms!nominative}  case. 
Likewise, the subject is marked with the Nominative\is{case!individual forms!nominative}  in the temporal adverbial clause in (\ref{CocDep}b).
The Mesa Grande Diegue\~no\il{Diegue\~no (Mesa Grande)} example in (\ref{DieDep}) could likewise be interpreted as an adverbial clause, or maybe a relative clause\is{clause-type!relative clause}.
\citet[135]{Gorbet:1976} analyzes it as adverbial, but he notes that others might analyze it as a relative clause.
The subject of this dependent clause is zero-coded.

%Check glossing
\begin{exe}\ex\label{CocDep}\langinfo{Cocopa}{Yuman; California}{\citealp[191]{Crawford:1966}, \citealp[325]{Langdon.Munro:1979}}
\begin{xlist} 
\ex\gll  {\rm[}\textipa{r\'e} \textipa{n\super{y}aw\'a\textbf{-c}} \textipa{\textsubdot say\'a-m}{\rm]}, \textipa{n\super{y}\textsubdot s\'a-\textbarl \super{y}} \textipa{p-w\'amca}\\
         \hspaceThis{[}king 3\sg{}.house-\nom{} be\_there-\dsbj{} it-\loc{} 3-arrive\\
\glt `He arrived there at the king's house.' %\citet[191]{Crawford:1966}\end{exe}

\ex\gll  {\rm[}\textglotstop n\textsuperscript{y}a:\textbf{-\v c} l\v ca:\v s-m{\rm]} sa:m-t\textsuperscript{y} n\textsuperscript{y}-\textschwa wa:\v ca\\
  \hspaceThis{[}1\sg{}-\nom{} little-\dsbj{} Somerton-\loc{} 1-live\\
\glt `When I was little, we lived in Somerton.'
\end{xlist}
\end{exe} %\citep[325]{Langdon.Munro:1979}

%\pagebreak
\begin{exe}\ex\label{DieDep}\langinfobreak{Diegue\~no (Mesa Grande)}{Yuman; California}{\citealp[135]{Gorbet:1976}}
\gll  {\rm[}'xat n\textsuperscript{y}-cu:kuw{\rm]}-pu-i n\textsuperscript{y}i: w-Lic-x w-ma:w\\
\hspaceThis{[}dog 3$>$1-bite-\dem{}-\loc{} at\_all 3-bad-\irr{} 3-\Neg{}\\
\glt `the bite wasn't bad at all' (lit: `where the dog bit me, it wasn't bad at all') \end{exe}\is{clause-type!adverbial clause|)}

In Maidu\il{Maidu}, the discussion of complex sentences, i.e. those containing more than one clause, is very brief \citep[69--70]{Shipley:1964}. 
None of the given examples has an overt subject argument in the dependent clause and the case-marking on such arguments, should they occur, is not discussed.  

% this example does not appear to be anz subordinatd structure: 
%\medskip
%{Maidu\il{Maidu}}  \citep[58,~27]{Shipley:1964}
%\begin{exe} \ex \begin{xlist}
%\ex\gll mym m\'ajdy  \textglotstop as lol\'o m\'ej-k\textraiseglotstop as\\ 
%\dem{} person \emphat{} container give-1\sg{}\\
%`That's the man I gave the basket to.'

%\ex\gll mym kyl\'e-m \textglotstop as mym m\'ajdy lol\'o m\'ej-k\textraiseglotstop an\\
%\dem{} woman-\nom{} \emphat{} \dem{} man basket give-3\\
%`She gave him a basket.'

%\ex\gll mym m\'ajdy \textglotstop as lol\'o m\'ej-k\textraiseglotstop as\\
%\dem{} person container give-1\sg{}\\
%`I gave the man a basket.'
%\end{xlist} 
%\end{exe}


%%%%%%%%%%%%valency-decreasing-operations%%%%%%%%%%%%%%%%

Turning\is{valency-decreasing construction|(} to the investigation of valency-decreasing constructions, Mojave\il{Mojave} has an operation in which the logical subject is deleted.  
The single argument of the resulting clause (i.e. the logical object) remains in the zero-coded form. 
\citet{Munro:1976}\is{valency-decreasing construction!passive|(} therefore does not consider them to be subjects, although she still glosses the verbal marker that is found in this construction as passive. 
In addition `passive' verbs take the 1st and 2nd person object-agreement suffixes, and as such, agree with their subjects (\ref{MojPass}a), third person objects do not agree with the verb in any context in Mojave\il{Mojave}.
Comparing the `passive' clauses in example (\ref{MojPass}b) with (\ref{MojPass}c), in which the logical subject is not realized either, the `passive' morpheme on the verb appears to  make the logical object more central in the clause.

\begin{exe}\ex\label{MojPass}\langinfo{Mojave}{}{\citealp[241, 220]{Munro:1976}}
\begin{xlist}
\ex\gll \textipa{n\super{y}-tapi\textglotstop ipay-\v c-m}\\
1.\obj{}-save-\pass{}-\tns{}\\
\glt `I was saved.' 
\ex\gll \textipa{masahay-n\super{y}} \textipa{@ta:v-\v c-m}\\
girl-\dem{} hit-\pass{}-\tns{}\\
\glt `The girl got hit.' 
\ex\gll \textipa{masahay-n\super{y}} \textipa{@ta:v-k}\\
girl-\dem{} hit-\tns{}\\
\glt `(Someone) hit the girl.'
\end{xlist}
\end{exe}
%\begin{flushright}\citet[241]{Munro:1976}\end{flushright}


Two other Yuman languages, namely Yavapai\il{Yavapai} (\ref{YavPass}) and Havasupai\il{Havasupai} (\ref{HavPass}), %status of passive is not quite clear..
on the other hand, have clear passive constructions in which the logical object is promoted to syntactic Nominative-marked\is{case!individual forms!nominative}  subject.
As in Wappo\il{Wappo}, A arguments of passive clauses bear Nominative\is{case!individual forms!nominative}  case-marking (\ref{WapPass1}a,~\ref{WapPass2}a).

\begin{exe}\ex\label{YavPass}\langinfo{Yavapai}{}{\citealp[127]{Kendall:1976}}
\begin{xlist}
\ex\gll\textipa{hlo-v-\textbf{c}} \textipa{si:l-v-k\~n}\\
rabbit-\dem{}-\nom{}  fry-\pass{}-\compl{}\\
\glt `The rabbit was fried.' 
\ex\gll \textipa{Tala-\textbf{c}} \textipa{hlo} \textipa{si:l-k\~n}\\
Thala-\nom{} rabbit fry-\compl{}\\
\glt `Thala fried the rabbit.'
\end{xlist}
\end{exe}

%\pagebreak
\begin{exe}\ex\label{HavPass}\langinfo{Havasupai}{Yuman; Arizona}{\citealp[60, 61]{Kozlowski:1972}}
\begin{xlist}
\ex\gll wa-ha-\textbf{c} wi-v-m yo-v-c-a\\
house-\dem{}-\nom{} stone-\dem{}-\prtv{} make-\pass{}-\pl{}-\modal{}\\
\glt `The house is made of stone.' 

\ex\gll ah-\~nu-\textbf{c} mat-\~nu-m pay vtil-v-k-yu\\
water-\dem{}-\nom{} earth-\dem{}-on all lay-\pass{}-\ind{}-\aux{}\\
\glt `The water is lying all over the ground (over there). %\citet[61]{Kozlowski:1972}\end{exe}  %maybe ommit this example since the translation ios n ot clear
\end{xlist}
\end{exe}

\begin{exe}\ex\label{WapPass1}\langinfo{Wappo}{}{\citealp[79, 40]{Thompsonetal:2006}}
\begin{xlist}
\ex\gll\textipa{\v si\textglotstop ay\textbf{-i}} \textipa{mot'-khe\textglotstop}\\
stalk-\nom{} pile\_up-\pass{}\\
\glt `The stalks have been piled up'

\ex\gll \textbf{\textipa{ah}} \textipa{hol} \textipa{ko\.*to:mela} \textipa{te-k'e\v c'-ta\textglotstop}\\
1\sg{}.\nom{} tree big.\pl{} \dir{}-chop-\pst{}\\
\glt `I chopped down the big trees.'
\end{xlist}
\end{exe}

\begin{exe}\ex\label{WapPass2}\langinfo{Wappo}{}{\citealp[79, 46]{Thompsonetal:2006}}
\begin{xlist}
\ex\gll\textipa{\textbf{cephi}} \textipa{o\v say'-khe\textglotstop}\\
3\sg{}.\nom{} pay-\pass{}\\
\glt `S/he got paid'

\ex\gll \textbf{\textipa{ah}} \textipa{mi} \textipa{o-\v say'i-ya:mi\textglotstop}\\
1\sg{}.\nom{} 2\sg{}.\acc{} \uop{}-pay-\fut{}\\
\glt `I'm going to pay you.' %[46]
\end{xlist}
\end{exe}


%There seems to be some misalignment in the following example, the initial \emph{c} of the verb is most likely the Nominative suffix of the subject.

%{Havasupai\il{Havasupai}}
%\begin{exe}\ex\gll i c-k\textsuperscript{y}at-v-o-k-yu\\
%wood cut-\pass{}-PERF-\ind{}-\aux{}\\
%`The wood was cut up.' \citet[81]{Kozlowski:1972}\end{exe}

\citet{Shipley:1964} does not discuss any passive or passive-like constructions in his grammar of Maidu\il{Maidu}. 
Also, \citet{WALS107} lists Maidu\il{Maidu} as one of the languages in which a passive is absent\is{valency-decreasing construction!passive|)}, giving Shipley's work as reference.\is{valency-decreasing construction|)} 

%%%%Valency-increasing operatrions%%%%%%%%%

%\medskip
%Finally, I will discuss valency-increasing operations in the marked"=S languages of North America. 
%The Wappo\il{Wappo} causative construction encodes the newly introduced argument -- i.e. the causer -- in Nominative case (\ref{WapCaus}). The former subject of the the basic -- non-causative -- form of the verb is in the zero-coded Accusative case.

%\medskip
%{Wappo\il{Wappo}} \citep[128,~129]{Thompsonetal:2006}
%\begin{exe}\ex\label{WapCaus}
%\begin{xlist}\ex\gll\textipa{ah} \textipa{c'ani} \textipa{k'opa-tis-ta\textglotstop}\\
%1\sg{}.\nom{} ice melt-\caus{}-\pst{}\\
%`I melted the ice.'
%\ex\gll\textipa{c'an-ti} \textipa{k'opa-khi\textglotstop}\\
%ice-\nom{} melt-\stat{}\\
%`The ice has melted'

%\ex\gll\textipa{i} \textipa{ek'-i} \textipa{i} \textipa{kat'a-tis-ta\textglotstop}\\
%1\sg{} son-\nom{} 1\sg{}.\acc{} laugh-\caus{}-\pst{}\\
%`My son made me laugh'  %\citet[129]{Thompsonetal:2006}
%\end{xlist}
%\end{exe}

%\begin{exe}\ex\gll\textipa{cephi} \textipa{i} \textipa{oya\textglotstop} \textipa{ke\textglotstop-tis-ta\textglotstop}\\
%3\sg{}.\nom{} 1\sg{}.\acc{} pot break-\caus{}-\pst{}\\
%`S/he made me break the pot.' %\citet[128]{Thompsonetal:2006}
%\end{exe}

%Mojave\il{Mojave} uses a number of prefixes to derive causative transitive verbs from basic intransitive verbs \citep[238]{Munro:1976}, compare for example the transitive verb stem \emph{ipuy} `die' and its causative counterpart \emph{tapuy} `kill'. 
%In all these constructions the causer is in the Nominative form (\ref{MojCaus}).
%Similarly, in {Havasupai\il{Havasupai}} the Nominative case is used for subjects introduced by the causative \emph{c-} or benefactive \emph{-wo} (\ref{HavCaus}). Unfortunately \citet{Kozlowski:1972} does not provide a straightforward example with a causative. 

%\medskip
%{Mojave\il{Mojave}} \citep[104]{Munro.subj:1976}
%\begin{exe}\ex\label{MojCaus}
%\begin{xlist}\ex\gll \textipa{\textglotstop in\super{y}e\v c} \textipa{\textglotstop-ipuy-k}\\
%1\sg{}.\nom{} 1-die-\tns{}\\
%`I died.'
%\ex \gll\textipa{\textglotstop in\super{y}e\v c} \textipa{\textglotstop-t-apuy-k}\\
%1\sg{}.\nom{} 1-\caus{}-die-\tns{}\\
%`I killed him.'
%\end{xlist}
%\end{exe}

%{Havasupai\il{Havasupai}} \citep[86]{Kozlowski:1972}
%\begin{exe}\ex\label{HavCaus}\gll \~na-c tohovye pa u-wo-k-a-wi\\
%1-\nom{} cards man see-\ben{}-\ind{}-1-\aux{}\\
%`I'm showing him how to play cards.' \end{exe}

%For Jamul\il{Jamul Tiipay} Tiipay, \citet{Miller:2001} list a number of different causative constructions. In basic causative clauses the causer is marked as the subject by means of verbal affixes and (optional) Nominative case-marking (\ref{JamCaus}a,b). In this construction a special causative verb form is used (glossed as `make\_VERB'). Causers in what \citet[309]{Miller:2001} calls the `periphrastic causative construction' are also marked by Nominative case (\ref{JamCaus}c). This construction differs from the basic causative in employing the lexical verb `to make' rather than the causative form of a verb. In the `ornate periphrastic causative construction'(\ref{JamCaus}d) both the lexical verb `to make' and the causative form of the main verb are used. The causer is also in the Nominative case in this construction.

%\medskip
%{Jamul\il{Jamul Tiipay} Tiipay} \citep[164, 165, 310, 311]{Miller:2001}
%\begin{exe}\ex\label{JamCaus}
%\begin{xlist}
%\ex\gll peya-ch yay*me-taax\'aana\\
%this-\nom{} heart*3\textgreater 2-make\_good\\
%`This will make you feel better.'
%\begin{flushright}\citet[164]{Miller:2001}\end{flushright}

%\ex\gll puu-ch nyaap nye-sema'r\'aya\\
%that\_one-\nom{} 1.\acc{} 3\textgreater 1-make\_drunk\\
%`he got me drunk.'
%%\begin{flushright}\citet[165]{Miller:2001}\end{flushright}

%\ex\gll 'iipa peya-ch riik me-chaw-x\\
%man this-\nom{} be\_rich 3\textgreater 2-make-\irr{}\\
%`This man will make you rich.'
%% \begin{flushright}\citet[310]{Miller:2001}\end{flushright}

%\ex\gll 'iipa peya-ch taar\'\i ika me-chaw-x\\
%man this-\nom{} make\_rich  3\textgreater 2-make-\irr{}\\
%`This man will make you rich.'
%%\begin{flushright}\citet[311]{Miller:2001}\end{flushright}
%\end{xlist}
%\end{exe}

%There are no examples of causative sentences with overt subjects (causers) is Shipley's grammar of Maidu\il{Maidu}. In (\ref{MaiCaus}), the causee is in the zero-coded form and since the causer shows other subject properties such as verbal agreement, it is very likely that it would be marked with the Nominative suffix.

%\medskip
%{Maidu\il{Maidu}} \citep[41]{Shipley:1964}
%\begin{exe}\ex\label{MaiCaus} \gll m\'a d\'on-dom \textglotstop as t\'e \textglotstop yn\'o-ti-k\textraiseglotstop as\\
%hand grab-? \emphat{} child walk-\caus{}-1\sg{}\\
%`I walked the child, holding his hand (lit.: I caused the child to walk holding his hand.)\end{exe}

\begin{table}[ht]
\centering
\begin{tabular}{lcccc%c
}
\hline \hline
\bfseries language&\bfseries S rel&\bfseries S adv&\bfseries S compl&\bfseries S VDC%&\bfseries S VIC
\\
\hline
Cocopa\il{Cocopa}&{-}&\textbf{\nom{}}&{-}&{-}%&
\\
%\hdashline
Diegue\~no\il{Diegue\~no (Mesa Grande)} (Mesa Grande)&\acc{}&\acc{}&{-}&{-}%&
\\
%\hdashline
Havasupai\il{Havasupai}&{-}&{-}&{-}&\textbf{\nom{}}%&\textbf{\nom{}}
\\
%\hdashline
Jamul\il{Jamul Tiipay} Tiipay&\acc{}&\textbf{\nom{}}&{-}&{}%&\textbf{\nom{}}
\\
%\hdashline
Mojave\il{Mojave}&\acc{}&\textbf{\nom{}}&\textbf{\nom{}}&\acc{}%&\textbf{\nom{}}
\\
%\hdashline
Wappo\il{Wappo}&\acc{}&\acc{}&\acc{}&\textbf{\nom{}}%&\textbf{\nom{}}
\\
%\hdashline
Yavapai\il{Yavapai}&\acc{}&\textbf{\nom{}}&{-}&\textbf{\nom{}}%&{}
\\
\hline \hline
\end{tabular}		
\caption{Marking of subjects in non-basic clauses in the marked"=S languages of North America}\label{NANonBasic}
\end{table}

An overview of the marking of subjects in non-basic clauses is provided in Table~\ref{NANonBasic} for the languages of North America.\footnote{In the column headings the following abbreviations are used:  S = S-like/subject argument; rel = relative clause; adv = adverbial clause; compl = complement clause; VDC = valency-decreasing construction%; VIC = valency-increasing construction
} The most remarkable feature is the consistent absence of nominative case-marking for subjects of relative clauses.


%%%%%%%%%%%%%%%%%%%%%%%%%%%
\section{Pacific}\label{NonBasicPac}
%%%%%%%%%%%%%%%%%%%%%%%%%%%

Dependent\is{clause-type!dependent} clauses in the marked"=S languages of the Pacific exhibit quite a few interesting patterns with respect to case-marking.
The other non-basic clauses are not remarkable, though more detailed data on the Nias\il{Nias} passive\is{valency-decreasing construction!passive} could possibly  be very interesting.

%%%%%%%%%%%%%%%% Dependent clauses%%%%%%%%%%%%%%

Main\is{clause-type!relative clause|(} clauses and relative clauses in Nias\il{Nias} show opposite properties, with respect to the Mutated and Unmutated forms.
Compare the relativized S in (\ref{NiaRelS}a) with the main clause S in (\ref{NiaRelS}b). 
The same is true for the P argument, as can be seen by comparing the (a) and (b) sentences in examples (\ref{NiaRelP}--\ref{NiaRelP2}). 

\begin{exe} \ex\label{NiaRelS}\langinfo{Nias}{Sundic; Indonesia}{\citealp[414, 559]{Brown:2001}}
\begin{xlist}
\ex \gll nihs  {\rm[}si=ma=mate \textbf{fo'omo} mene{\ss}i{\rm]}\\
\hspaceThis{[}person \relativ{}=\compl{}=die wife yesterday\\
\glt `the man whose wife died yesterday.'

 \ex \gll mate \textbf{zibaya}-nia mene{\ss}i\\
die uncle.\mut{}-3\sg{}.\poss{} yesterday\\
\glt `His uncle died yesterday.' %559
\end{xlist} 
\end{exe}

\begin{exe} \ex\label{NiaRelP}\langinfo{Nias}{}{\citealp[414]{Brown:2001}}
\begin{xlist} 
\ex \gll Andrehe'e {nasu}     {\rm[}si=usu \textbf{ya'o}{\rm]}\\
         \dist{}   dog.\mut{} \hspaceThis{[}\relativ{}=bite 1\sg{}\\
\glt `That's the dog that bit me.'  %414

\ex \gll i-usu \textbf{ndrao} {asu}\\
3\sg{}.\rls{}-bite 1\sg{}.\mut{} dog\\
\glt `The dog bit me.' %414
\end{xlist}%\begin{flushright}\citet[414]{Brown:2001}\end{flushright}
\end{exe} 

%\pagebreak
\begin{exe} \ex\label{NiaRelP2}\langinfo{Nias}{}{\citealp[415]{Brown:2001}}
\begin{xlist}
\raggedright
\ex \gll Andrehe'e mbua              {\rm[}si=ma i-hal\"o \textbf{bua} \textbf{mbala} {andre}{\rm]}\\
        \dist{}   fruit\_tree.\mut{} \hspaceThis{[}\relativ{}=\pfv{} 3\sg{}.\rls{}-take fruit papaya \dist{}\\
\glt `That is the tree that he took those papaya from.' %415

\ex \gll i-hal\"o \textbf{mbua} \textbf{mbala} moroi ba mbua h\"o'\"o\\
3\sg{}.\rls{}-take fruit.\mut{} papaya.\mut{} come.from \loc{} fruit\_tree.\mut{} \dist{}\\
\glt `He took the papapya from the tree.'
\end{xlist}%\begin{flushright}\citet[415]{Brown:2001}\end{flushright}
\end{exe} 

The A argument of a relative clause is realized as a noun in the Mutated form (\ref{NiaRelA}), while in main clauses it would be Unmutated.\footnote{Another strategy for realizing the A argument of a relative clause is to have the A as a possessor \citep[420]{Brown:2001}. 

\begin{exe}\ex 
\begin{xlist}
\ex\gll u-fake zekhula ni-r\"okhi\textbf{-nia}\\
1\sg{}.\rls{}-use coconut.\mut{} \pass{}-grate-3\sg{}.\poss{}\\
\glt `I used the coconut which she grated.' %420

\ex\gll i-r\"okhi zekhula\\
3\sg{}.\rls{}-grate coconut.\mut{}\\
\glt `She grated the coconut.'
\end{xlist} 
\end{exe}}
The status of relative clause A arguments, however, is somewhat unclear. 
In relative clauses that have the internal A argument realized as an overt noun phrase,  the verb usually bears the prefix \emph{ni} glossed as passive\is{valency-decreasing construction!passive} (cf.~\ref{NiaA}a). 
The status of this passive is not completely clear. 
The passive morpheme appears predominantly within relative clauses, yet in some rare instances, is also used in independent main clauses according to Lea Brown (p.c.). 
I will return to this issue when discussing valency-reducing operations.

%\enlargethispage{\baselineskip}
\begin{exe}\ex\label{NiaA}\langinfo{Nias}{}{\citealp[422]{Brown:2001}}
\begin{xlist} 
\ex\label{NiaRelA}\gll Andrehe'e  nohi                 {\rm[}si=l\"ona ni-lau \textbf{nono} matua{\rm]}\\
                       \dist{}    coconut\_tree.\mut{} \hspaceThis{[}\relativ{}=\Neg{} \pass{}-climb child.\mut{} male\\
\glt`That is the coconut tree the boy did not climb.'

\ex\label{NiaMainA}\gll Ma=i-b\"ozi nasu \textbf{ono} matua ba ma=m-oloi ya\\
PERF=3\sg{}.\rls{}-hit dog.\mut{} child male \conj{} \pfv{}=\dyn{}-run 3\sg{}.\mut{}\\
\glt `The boy hit the dog and ran away.'
\end{xlist}% \begin{flushright}\citet[422]{Brown:2001}\end{flushright}
\end{exe}\is{clause-type!relative clause|)}


In\is{clause-type!adverbial clause|(} Aji\"e\il{Aji\"e}, subjects of purpose clauses (\ref{AijPurp}) and reason  clauses (\ref{AijReas}) are in the Nominative\is{case!individual forms!nominative}  case. 
\Citet[330]{Fontinelle:1976} gives this example with parentheses around the Nominative marker indicating its optionality, but she does not comment on this any further.

\begin{exe}\ex\label{AijDep}\langinfo{Aji\"e}{Oceanic; New Caledonia}{\citealp[113]{Lichtenberk:1978} after \citealp[330]{Fontinelle:1976} and \citealp[189, 190]{Fontinelle:1961}}
\begin{xlist}
\ex\label{AijPurp}\gll\textipa{na} \textipa{uu} \textipa{kwa\textglotstop}  {\rm[}\textipa{cE} \textipa{ki} \textipa{dii} \textbf{\textipa{na}} \textipa{ne\textbardotlessj 2\textglotstop}{\rm]}\\
                      3\sg{}       call\_for     rain                       \hspaceThis{[}\purp{} \Hyp{} wet \nom{} ground\\
\glt `He calls for rain so that the ground may be wet.' %\citet[113]{Lichtenberk:1978} \end{exe}

\ex\label{AijReas}\gll\textipa{gwe} \textipa{daa} \textipa{tuwiri}  {\rm[}\textipa{wE} \textipa{wi} \textipa{bomu} \textbf{\textipa{na}} \textipa{kOwi-\textltailn}{\rm]}  {\rm[}\textipa{wE} \textipa{wi} \textipa{Oi} \textipa{ne-l@\textglotstop} \textipa{Gi-\textltailn a} \textbf{\textipa{na}} \textipa{yiipu}{\rm]}\\
1\sg{}.\prosp{} \Neg{} touch \reas{} \dubt{} smell \nom{} hand-1\sg{} \reas{} \dubt{} eat \coll{}-braid \poss{}-1\sg{} \nom{} rat\\
\glt `I am not going to touch it because my hand might smell (and) because the rat might eat my braids.'%\citet[113]{Lichtenberk:1978} after \citet[189f.]{Fontinelle:1961} \end{exe}
\end{xlist}
\end{exe}\is{clause-type!adverbial clause|)}

For {Savosavo\il{Savosavo}}, information is provided for a large number of different types of dependent clauses \citep[254--286]{Wegener:2008}.
Several of these types allow for the optional or obligatory  realization of their subjects in the Genitive\is{case!individual forms!genitive} rather than the Nominative case.
%``Relative clauses are externally headed by the head of the NP, which they precede'' \citep[254]{Wegener:2008}
Relative\is{clause-type!relative clause|(} clauses always encode their subjects in the Genitive case. 
Other constituents of the relative clause are encoded like in independent clauses. 
The examples in (\ref{SavRel}) illustrate this pattern.
Adverbial clauses\is{clause-type!adverbial clause|(}, on the other hand, use either the
Nominative\is{case!individual forms!nominative}  or Genitive\is{case!individual forms!genitive} case
to mark their subjects, as seen in (\ref{SavAdv}). 

\begin{exe}\ex\label{SavRel}\langinfo{Savosavo}{Solomons East Papuan; Solomon Islands}{\citealp[257, 258]{Wegener:2008}}
\begin{xlist}
\ex\gll  {\rm[}lo fomu=gha \textbf{ze} pale-tu{\rm]} lo mavutu\\ 
         \hspaceThis{[}\deter{}.\pl{} form=\pl{} 3.\pl{}.\gen{} stay-\relativ{} \deter{}.\sg{} place \\
\glt`The place where the forms are.'% \citet[257]{Wegener:2008}

\ex\gll  {\rm[}lo lo-ma nyuba ko\textbf{-va} Honiara bo-tu{\rm]} lo mapa\\
         \hspaceThis{[}\deter{}.\sg{}.\mas{} 3\sg{}.\mas{}-\gen{}.\sg{}.\fem{} child 3\sg{}.\fem{}-\gen{}.\mas{} Honiara go-\relativ{} \deter{}.\sg{}.\mas{} person\\
\glt `the man whose daughter went to Honiara'%\citet[258]{Wegener:2008}
\end{xlist}
\end{exe}\is{clause-type!relative clause|)}%
%
%\pagebreak
\begin{exe}\ex\label{SavAdv}\langinfo{Savosavo}{}{\citealp[275, 272]{Wegener:2008}}
\begin{xlist}
\ex\gll  {\rm[}kokoroko\textbf{=na} ngia{\rm]} ze ka gholigholi tete=ghu=e lo tada=gha=na\\
         \hspaceThis{[}chicken=\nom{} cry.\simult{} 3\pl{}.\gen{} already scrape balance=\nmlz{}=\emphat{} \deter{}=\pl{} man=\pl{}=\nom{}\\
\glt `As the rooster crowed, they already scraped (coconuts), the men.' %\citet[275]{Wegener:2008}

\ex\gll pa  muzi=la              {\rm[}ko\textbf{-va} elu epi-atu{\rm]} lo sua=gha=na ngori-ngori(-i)\\
        one night=\loc{}.\mas{}  \hspaceThis{[}3\sg{}.\fem{}-\gen{}.\mas{} wake sit-\bg{}.\ipfv{} \deter{}.\pl{} giant=\pl{}=\nom{} \rdp{}-snore(-\fin{})\\
\glt`One night as she was still awake sitting there, the giants snored.' %\citet[275f.]{Wegener:2008}
%\ex\gll [zu \textbf{ave} tagha pale sua monei]=ve ave-ale\\
%but 1\pl.\excl{} up stay \att{} if\_only=1\pl{}.EX.\nom{} die-\irr{}\\
%`But if we had stayed up (i.e. above ground), we would have died.'% conditional: \citet[266]{Wegener:2008} check example uns case-marking with Claudia

\ex\gll te=lo                        ai            mau=na        zua    {\rm[}tulola\textbf{=ze} tei(-i)\,\dots{\rm]}\\
        \conj{}=3\sg{}.\mas{}.\nom{} 1\sg{}.\gen{} father=\nom{} ask \hspaceThis{[}then=3\pl{}.\nom{} say(-\fin{})\\
\glt`Then my father asked and then they said \dots' %Temporal clauses: \citet[272]{Wegener:2008}
\end{xlist}
\end{exe}\is{clause-type!adverbial clause|)}


%%%%%%%%%%%valency-decreasing operations%%%%%%%%%%%%%%%%%

I\is{valency-decreasing construction|(} will now turn to the discussion of valency-decreasing constructions.
None\is{valency-decreasing construction!passive|(} of the existing descriptions of Nias\il{Nias} gives an extensive discussion of the passive. 
The passive morpheme appears predominantly within relative clauses.
The passive subject is in the mutated form in (\ref{NiaPass}). 
In some rare instances, the passive is also used in independent main clauses according to Lea Brown\aimention{Brown, Lea} (p.c.), though unfortunately, I have no example sentence to demonstrate this behavior. 

%\pagebreak
\begin{exe}\ex\label{NiaPass}\langinfobreak{Nias}{}{\citealp[573]{Brown:2001}}
%\begin{xlist} 
%\ex\label{Nias\il{Nias}Pass1}\gll ya'o ni-be-ra si=a da'i manu\\
%1\sg{} \pass{}-make-3\pl{}.\poss{} \relativ{}=eat faeces.\mut{} chicken.\mut{}\\
%`I was humiliated' (lit. `I was made into a person who eats chicken scat')
%\ex
\gll ma=oya=ae \textbf{mbalatu} ni-n\"o\ss \"o-i-nia\\
\pfv{}=many=already knife.\mut{} \pass{}-make-\transitiv{}-3\sg{}.\poss{}\\
\glt `He had already made a lot of knifes' (lit. `The knifes made by him were already a lot')
%\end{xlist}
\end{exe}

The 3rd person possessive suffix on the passivized verb could indicate that this is an impersonal\is{valency-decreasing construction!impersonal construction} construction rather than a true passive.
Further, some passivized verbs have a transitivizer (\transitiv{}) affixed to their stem (\ref{NiaTR}a), which makes the whole situation even less transparent. But compare also (\ref{NiaTR}b), where the \transitiv{}-marker occurs also with the non-passivized form of the same verb. 

\begin{exe}\ex\label{NiaTR}\langinfo{Nias}{}{\citealp[556, 555]{Brown:2001}}
\begin{xlist}
\ex\gll ya'ia ni-bali-'\"o-ra sa\ss uyu\\
3\sg{} \pass{}-turn-\transitiv{}-3\pl{}.\poss{} slave\\
\glt `He was made a slave.'
\ex\gll la-bali-'\"o ya sa\ss uyu\\
3\pl{}.\rls{}-turn-\transitiv{} 3\sg{}.\mut{} slave\\
\glt `They made him a slave.'
\end{xlist}
\end{exe}\is{valency-decreasing construction!passive|)}

There\is{valency-decreasing construction!detransitivizing|(} is no specialized passive or antipassive construction in Savosavo\il{Savosavo}, but the detransitivizer \emph{-za} serves similar functions to those associated with passives and antipassives in other languages. When it is attached to transitive verbs, this results in a change in the argument structure of the verb. 
There are three possibilities for the nature of change in argument structure, the first (corresponding to a passive reading) being the most common \citep[171]{Wegener:2008}:

\begin{enumerate}
\item The subject is demoted and removed, the object is promoted to subject position.
\item The subject is unchanged, only the object is removed.
\item Both subject and object are removed and are replaced by a subject that is a semantic
cognate of the verb, e.g. `a shout' in case of a verb `to shout'
\end{enumerate} 

The following example illustrates the passive use of the detransitivizer. 
In this example the subject is in the Nominative\is{case!individual forms!nominative}  case (\ref{SavPass}).

\begin{exe}\ex\label{SavPass}\langinfo{Savosavo}{}{\citealp[171]{Wegener:2008}}
\begin{xlist}
\ex\gll lo karoti\textbf{=na} tozo-za-i\\
\deter{}.\sg{}.\mas{} carrot=\nom{} cut-\detr{}-\fin{}\\
\glt`The carrot is cut.'
\ex\gll karoti=lo te tozo-li(-i)\\
carrot=3.\sg{}.\mas{}.\nom{} \emphat{} cut-3\sg{}.\mas{}.\obj{}(-\fin{})\\
\glt`He cut (a) carrot.'
\end{xlist}
\end{exe}\is{valency-decreasing construction|)}
\is{valency-decreasing construction!detransitivizing|)}

%Nias\il{Nias} has a number of causative affixes which are used to derive causative verbs from intransitive verbs. The causer introduced in these constructions is in the unmutated form of the noun like regular transitive subjects would be (\ref{NiaCaus}). However, causers are only seldom realized as lexical items and usually only marked via verbal agreement.
% Savosavo\il{Savosavo} causative construction employes the verb \emph{l-au} `to take' combined with an intransitive verb \citep[194]{Wegener:2008}. Subjects introduced in this construction are in the Nominative case (\ref{SavCaus}).

%\medskip
%{Nias\il{Nias}} \citep[236]{Brown:2001}
%\begin{exe}\ex\label{NiaCaus}\gll i-f-a'ege ndraga \textbf{ba'e}\\
%3\sg{}.\rls{}-\caus{}-laugh 1\pl{}.\excl{}.\mut{} monkey\\
%`The monkey made us laugh.'
%\end{exe}

%{Savosavo\il{Savosavo}} \citep{Wegener:2008}
%\begin{exe}\ex\label{SavCaus}
%\begin{xlist}
%\ex\gll l-au sasi ze\textbf{=no}\\
%3\sg{}.\mas{}.\obj{}-take be\_wrong \partic{}=2\sg{}.\nom{}\\
%`You made it wrong.'% \citet[194]{Wegener:2008}

%\ex\gll pe bo kia, sika\textbf{=pe} ai lo lo mola l=au zaba-(a)le \dots\\
%2DU.\gen{} go if don't=2DU.\nom{} this \deter{}.\sg{}.\mas{} 3\sg{}.\mas{}.\gen{} canoe 3\sg{}.\mas{}.\obj{}-take become\_visible-\irr{}\\
%`When you (two) go, don't you let this canoe of his (the dead giant) be seen.'% \citet[269]{Wegener:2008} 
%\end{xlist}
%\end{exe} % another example on 289, 295, 311 

Table~\ref{PacNonBasic} summarizes the data just discussed. Similar to the North American languages, case-marking of subjects in relative clauses\is{clause-type!relative clause} is most interesting, although the patterns found in the Pacific are quite distinct from those found in North America.

\begin{table}[ht]
\centering
\begin{tabular}{lcccc%c
}
\hline \hline
\bfseries language&\bfseries S rel&\bfseries S adv&\bfseries S compl&\bfseries S VDC%&\bfseries S VIC
\\
\hline
Aji\"e\il{Aji\"e}&{-}&\textbf{\nom{}}&{-}&{-}%&{}
\\
%\hdashline
Nias\il{Nias}&\erg{}&{-}&{-}&\textbf{\abs{}}%&{\erg{}}
\\
%\hdashline
Savosavo\il{Savosavo}&\textbf{\gen{}}&\textbf{\nom{}/\gen{}}&{-}&\textbf{\nom{}}%&\textbf{\nom{}}
\\
\hline \hline
\end{tabular}
\caption{Subjects marking for non-basic clauses in the marked"=S languages of the Pacific}\label{PacNonBasic}%\\
\end{table}


%%%%%%%%%%%%%%%%%%%%%%%%%%%%%%
\section{Nilo-Saharan}\label{NonBasicNilo}
%%%%%%%%%%%%%%%%%%%%%%%%%%%%%%

The subject arguments of non-basic clauses are typically marked with the nominative\is{case!individual forms!nominative}  case in the Nilo-Saharan languages, though for each context there is at least one language behaving in an exceptional way.
Another interesting phenomenon is attested in P\"ari\il{P\"ari}, a language with a marked"=nominative system only with dependent clauses\is{clause-type!dependent}. 

The\is{alignment!splits|(} P\"ari\il{P\"ari} system exhibits a split within its alignment type, as defined in Chapter~\ref{method}. 
More precisely, it is split between different clause-types\is{clause-type!dependent|(}. 
In main clauses, P\"ari\il{P\"ari} has an ergative-pattern, yet in imperatives and most dependent clause-types the overt Ergative\is{case!individual forms!ergative} marker is also used for intransitive S \citep[316--319]{Andersen:1988}. 
Those clauses thus exhibit a marked"=nominative pattern, which \citet[316]{Andersen:1988} believes to be the source for the ergative pattern of main clauses in P\"ari\il{P\"ari}.
This split is not only limited to case-marking, but it is also found with the verbal indexing-system\is{verbal indexing} and word order\is{word order}.
The\is{clause-type!adverbial clause|(} examples in (\ref{ParMN}) illustrate the marked"=nominative pattern. 
The questions in (\ref{ParMN}a, b) are listed among the class of dependent clauses. 
Unfortunately, Andersen does not analyze the structure of the item glossed as `why'. 
Its complex structure and the fact that the whole structure is identified as a complex clause suggest that this item constitutes the main or matrix clause to the following subordinate clause. 
In the main clauses (\ref{ParMN}c,cd), this marking is indeed restricted to A arguments and not found on S arguments.\is{clause-type!dependent|)}

\begin{exe}\ex\label{ParMN}\langinfo{P\"ari}{Nilotic; Sudan}{\citealp[319, 318, 292]{Andersen:1988}}
\begin{xlist}
\ex\gll \textipa{p\`Ir N\`O} \textipa{dh\'aag\`O} \textipa{\`Ic\`\textomega Ol-y\'I} \textipa{\textltailn\`Ip\`Ond\`{}-\`E}\\
why woman call-3\sg{} child-\erg{}\\
\glt`Why did the child call the woman?'
\ex\gll\textipa{p\`Ir N\`O} \textipa{\`Ip\`22r} \textipa{c\'Ic\`\textomega-\^E}\\
why jump man-\erg{}\\
\glt`Why did the man jump?'
\ex\gll\textipa{\`ub\'ur} \textipa{\'a-t\'uuk\`{}}\\
Ubur \compl{}-play\\
\glt `Ubur played.'
\ex\gll\textipa{dh\'aag\`O} \textipa{\'a-y\`aa\textltailn} \textipa{\`ub\'ur-\`I}\\
woman \compl{}-insult Ubur-\erg{}\\
\glt `Ubur insulted the woman.'
\end{xlist}
\end{exe} %[319,~318,~292]

Among the subordinate clause-types that Andersen lists as having marked"=nominative coding are purposive clauses.  
This fits Dixon's expectations about this type of splits:

\begin{quote}
[\ldots{}] `purposive (= infinitival) clauses' normally refer to some attempt at controlled action; clauses of this kind generally have an A or S `agent' NP that is co-referential with some NP in their main clause [\dots] for this type of subordinate construction, we would surely expect S and A to be treated in the same way within the complement clause. 
\citep[101--102]{Dixon:1994}
\end{quote}\is{alignment!splits|)}\is{clause-type!adverbial clause|)}

Now I will return to the Nilo-Saharan languages which do exhibit marked"=nominative coding in main clauses.
In Tennet\il{Tennet}\is{clause-type!relative clause|(}, relative clause-internal subjects are in the Nominative\is{case!individual forms!nominative}  case (\ref{TenRel}).
Similarly in Nandi\il{Nandi}, subjects of relative clauses are marked with the Nominative\is{case!individual forms!nominative}  (\ref{NanRel}).
%show an interesting pattern with respect to case-marking. 
%All non-verbal items in relative clauses receive case-marking according to the case function of the head noun (\ref{NanRel}a).
%Relative-clause internal subjects receive Nominative marking independently of this (\ref{NanRel}b). Compare with the object relativezed clause (\ref{NanRel}c).
%This phenomenon resembles the so-called `case stacking' phenomenon found with many Australian languages  \citep{Nordlinger:1998,Nordlinger:2006} or `Suffixaufnahme' \citep{Plank:1995}. 
Maa\il{Maa} relative clauses, which have the structure V-AGR N [V-REL N], also employ the Nominative\is{case!individual forms!nominative}  case for subject-marking according to Tucker \& Mpaayei (1955:~chapter 12).% and the corresponding examples on page 231).  

\begin{exe}\ex\label{TenRel}\langinfo{Tennet}{Surmic; Sudan}{\citealp[259, 255]{Randal:1998}} 
\begin{xlist}
\ex\gll\textipa{k-\'I-c\'In-a} \textipa{ann\'a} \textipa{dh\'{\=*u}n\=*oc}  {\rm[}\textipa{c\'I} \textipa{b\=*al\=*i} \textipa{\'ak\'ati} \textipa{l\=*oh\'{\=*a}m\textbf{-i}}{\rm]}\\
       1-\pfv{}-see-1\sg{}     1\sg{}.\nom{}     waterbuck                  \hspaceThis{[}\am{} \pst{} \pfv{}.spear Loham-\nom{}\\
\glt `I saw the waterbuck that Loham speared.' 

\ex\gll\textipa{elegy\'e}  {\rm[}\textipa{c\'I-k} \textipa{\'uk} \textbf{\textipa{enn\'e}} \textipa{\'a-k\'at-a}{\rm]}\\
animals \hspaceThis{[}\am{}-\pl{} \pfv{}.go 3\sg{}.\nom{} \pfv{}-spear-{(pause)}\\
\glt 'the animals that he went and speared' 
\end{xlist}
\end{exe}

\begin{exe}\ex\label{NanRel}\langinfo{Nandi}{Nilotic; Kenya}{\citealp[134, 133]{Creider:1989}}
\begin{xlist}
\ex\gll\textipa{\'a-m\'ac-\'e} \textipa{ci:t\`a}  {\rm[}\textipa{ne} \textipa{k\`e:r-\'ey} \textbf{\textipa{te:ta}}{\rm]}\\
1\sg{}-want-\ipfv{} person \hspaceThis{[}\relativ{} 3-see-\ipfv{} cow.\nom{}\\
\glt `I want the person that the cow is looking at.'
%\citet[134]{Creider:1989}

\ex\gll\textipa{\'a-m\'ac-\'e} \textipa{ci:t\`a}  {\rm[}\textipa{ne} \textipa{k\`e:r-\'ey} \textipa{te:t\`a}{\rm]}\\
1\sg{}-want-\ipfv{} person \hspaceThis{[}\relativ{} 3-see-\ipfv{} cow\\
\glt `I want the person that is looking at the cow.'
%\citet[134]{Creider:1989}

%\ex\gll \textipa{m\^I:} \textipa{ci:ta} \textipa{n\`e} \textipa{ka:n\'e:t\'I:ntet}\\
%\cop{} person.\nom{} \relativ{}.\nom{} teacher.\nom{}\\
%`there is a person who is a teacher.'
% \citet[133]{Creider:1989}
\end{xlist}
\end{exe}

\citet[112--114]{Arensen:1982} distinguishes between what he calls `dependent' and `subordinate' clauses in his description of Murle\il{Murle}\il{Murle\il{Murle}}. 
All examples he lists as dependent clauses are relative clauses. 
Subjects inside the relative clause are in the Nominative\is{case!individual forms!nominative}  case (\ref{MurDep}a). 
The\is{clause-type!complement clause|(} clauses referred to as subordinate clauses by Arensen on the other hand mark their subjects with Accusative case (\ref{MurDep}b, c). 
All examples he presents are of the complement clause-type.

\begin{exe}\ex\label{MurDep}\langinfo{Murle}{Surmic; Sudan}{\citealp[112, 113, 114]{Arensen:1982}}
\begin{xlist}
\ex\gll\textipa{kEEti} \textipa{naana} \textipa{kiziwan}  {\rm[}\textipa{o} \textipa{or} \textbf{\textipa{niina}}{\rm]}\\
skin.1\sg{} 1\sg{}.\nom{} buffalo.\acc{} \hspaceThis{[}which shoot.\pst{} 2\sg{}.\nom{}\\
\glt `I am skinning the buffalo which you shot.'

\ex\gll\textipa{kaga} \textipa{naana}  {\rm[}\textbf{\textipa{nOnnO}} \textipa{aak} \textipa{idiN}{\rm]}\\
Know 1\sg{}.\nom{} \hspaceThis{[}3\sg{}.\acc{} cook meat.\acc{}\\
\glt `I know that she is cooking meat.'

\ex\gll\textipa{karOON} \textipa{naana}  {\rm[}\textbf{\textipa{Ol}} \textipa{kiliNliNit}{\rm]}\\
want.1\sg{} 1\sg{}.\nom{} \hspaceThis{[}people.\acc{} work\\
\glt `I want the people to work.'
\end{xlist}
\end{exe}\is{clause-type!complement clause|)}
 

Turkana\il{Turkana} relative clauses mark their subjects, when clause internal, in Nominative\is{case!individual forms!nominative}  case (\ref{TurRel}).
However, the subject is only realized in the relative clause if it is not the common argument  (\ref{TurRel}b).

\begin{exe}\ex\label{TurRel}\langinfo{Turkana}{Nilotic; Kenya}{\citealp[309, 314]{Dimmendaal:1982}}
\begin{xlist}
\ex\gll\textipa{e-dya\`{}}  {\rm[}\textipa{lo-w\`OOnI-k-a-IdEs-\`I} \textipa{a-yON\`{}}{\rm]}\\
boy.\acc{} \hspaceThis{[}that-other\_day-\transitiv{}-1\sg{}-hit-\asp{} 1\sg{}.\acc{}\\
\glt `The boy that hit me the other day.'

\ex\gll\textipa{e-dya\`{}}  {\rm[}\textipa{Nolo\`{}} \textipa{Nw\`OOn\r*{\`I}} \textipa{a-Id\`Es-I} \textbf{\textipa{a-y\`ON}}{\rm]}\\
boy.\acc{} \hspaceThis{[}that other\_day 1\sg{}-hit-\asp{} 1\sg{}.\nom{}\\
\glt `The boy that I hit the other day.'

\ex\gll\textipa{k-\`a-\`Id\`Es-I\`{}} \textipa{e-dy\`a} \textipa{a-yON\`{}} \textipa{Nw\`oon\r*{\`I}}\\
\transitiv{}-1\sg{}-hit-\asp{} boy.\nom{} 1\sg{}.\acc{} other\_day\\
\glt `The boy hit me the other day.'

\ex\gll\textipa{n\`a-mOnI\`{}}  {\rm[}\textipa{na-e-y\`a} \textipa{Ni-c\`om-in} \textipa{ka\`{}}  \textipa{NI-tOm-E\`{}}{\rm]}\\
in-forest \hspaceThis{[}that-3-be baboons.\nom{} with elephants.\acc{}\\
\glt `In the forest where there are baboons and elephants.'  %[314]
\end{xlist}
\end{exe}\is{clause-type!relative clause|)}

Other\is{clause-type!complement clause|(} dependent clauses behave differently from relative clauses with respect to case-marking. 
Most examples listed are of the complement clause type.
Like in relative clauses, the subject is only overtly realized in the complement clause when it is not identical to the subject of the matrix clause. 
Otherwise, the subject argument is only realized in the main clause in Turkana\il{Turkana} (\ref{TurDep}a).   
When occurring inside the complement clause, the subject is in topicalized position (i.e. before the verb of the dependent clause) and thus is in the Accusative\is{case!individual forms!accusative} case.  
\citet[374]{Dimmendaal:1982} argues that they are nonetheless part of the dependent rather than the matrix clause, since the matrix verb does not show any object agreement (\ref{TurDep}b).\footnote{In Turkana\il{Turkana} the marker \emph{k-} precedes the subject agreement affix if there is a first or second person object \citep[122]{Dimmendaal:1982}.}  

\begin{exe}\ex\label{TurDep}\langinfo{Turkana}{}{\citealp[374]{Dimmendaal:1982}}
\begin{xlist}
\ex\gll \textipa{\`a-sak-\`I} \textbf{\textipa{a-y\`ON}} \textipa{i-yoN\`{}} \textipa{akI-ar\`{}}\\
1\sg{}-want-\asp{} 1\sg{}.\nom{} 2\sg{}.\acc{} \Inf{}-kill\\
\glt `I want to kill you.'
 
\ex\gll \textipa{\`a-sak-\`I} \textipa{a-y\`ON} \textbf{\textipa{i-yoN\`{}}} \textipa{I-ar-\`I}\\
1\sg{}-want-\asp{} 1\sg{}.\nom{} 2\sg{}.\acc{} 2\sg{}-kill-\asp{}\\
\glt `I want you to kill it.'

\ex\gll\textipa{to-ryam-\r*{\`U}} \textipa{Nes\`I} \textbf{\textipa{\`a-pa\`{}}} \textipa{kEN\`{}} \textipa{E-ma\r*{\`U}}\\
3-find-\ventiv{} 3\sg{}.\nom{} \NC{}-father.\acc{} 3\sg{}.\poss{} 3-lack\\
\glt `He found his father was not there.'
\end{xlist}
\end{exe}\is{clause-type!complement clause|)}

In\is{valency-decreasing construction|(} the following paragraphs, valency-decreasing operations are discussed.
The single argument of the {Turkana\il{Turkana}} `impersonal active voice' (as Dimmendaal refers to the most passive-like construction) has Nominative\is{case!individual forms!nominative}  marking \citep[132--133]{Dimmendaal:1982}. 
In\is{valency-decreasing construction!passive|(} Murle\il{Murle}\il{Murle\il{Murle}}, the subject of passive sentences is in Nominative\is{case!individual forms!nominative}  case (\ref{MurPass}a).
And as seen already in Section~\ref{QueNonBasic}, Tennet\il{Tennet} passive (\ref{TenPass}) and antipassive\is{valency-decreasing construction!antipassive} subjects (\ref{TenAntiPass}) also are in the Nominative\is{case!individual forms!nominative}  case. 

\begin{exe}\ex\label{MurPass}\langinfo{Murle}{}{\citealp[140, 137]{Arensen:1982}}
\begin{xlist}
\ex\gll \textipa{ajuk-E} \textipa{EEt\textbf{-i}}\\
trow-\pass{} man-\nom{}\\
\glt `The man is thrown.' %\citet[140]{Arensen:1982} 

\ex\gll\textipa{ajuk} \textipa{EEt\textbf{-i}} \textipa{dila}\\
trow man-\nom{} spear.\acc{}\\
\glt `The man throws a spear.' %\citet[137]{Arensen:1982}
\end{xlist}
\end{exe}

\begin{exe}\ex\langinfo{Tennet}{}{\citealp[245]{Randal:1998}}
\begin{xlist}
\ex\label{TenPass}\gll\textipa{\'{\=*a}-r\'{\=*u}h-w-\=*e} \textbf{} \textipa{\'{\=*I}y\'{\=*o}k\=*o}\\
\ipfv{}-beat-\epen{}-\pass{} drum.\nom{} now\\
\glt `The drum is being beaten now.' 

\ex\label{TenAntiPass}\gll\textipa{\'a-d\'ah-ye} \textbf{\textipa{d\=*ol\'{\=*e}c}}\\
\ipfv{}-eat-\antip{} child.\nom{}\\
\glt `The child is eating.'  
\end{xlist}
\end{exe}

There are two processes which delete the logical subject of a sentence in Nandi\il{Nandi}. 
The first -- termed `stativization' by \citeauthor{Creider:1989} -- has the logical object as the surface (Nominative) subject and expression of an agent is not permitted (\ref{NanPass}a).
In the other process, which is actually referred to as `passivization', the agent is obligatorily deleted but the object gains no Nominative case-marking \citep[125--126]{Creider:1989}, as illustrated in (\ref{NanPass}b, c). 
In this construction, the verb receives invariant first person plural agreement, while the 3rd person stem form of the verb is chosen \citep[100]{Creider:1989}. `Impersonal\is{valency-decreasing construction!impersonal construction} construction' would probably be a better label for this construction, while the `stative' actually meets all criteria usually employed for a construction to be classified as a passive. \citet{Creider:1989} claim that the lack of an optional oblique phrase representing the logical subject disqualifies the construction from being considered a passive. 
However, this criterion is not widely used in cross-linguistic work on passives.\footnote{As \citet[100]{Creider:1989} note they chose the English passive, which allows for the oblique realization of logical subjects in passive clauses, as their model for a passive construction.}

\begin{exe}\ex\label{NanPass}\langinfo{Nandi}{}{\citealp[125, 126]{Creider:1989}}
\begin{xlist}\ex\gll\textipa{k\'a:-ko-y\`a:t-\'ak} \textbf{\textipa{ka:ri:k}}\\
\pst{}-3-open-\stat{}/\pass{} houses.\nom{}\\
\glt `The houses have been opened.'% \citet[125]{Creider:1989}

\ex\gll\textipa{k\'I:-ke:-s\`Ic} \textipa{kipe:t}  \textipa{k\'eny}\\
\pst{}-1\pl{}-bear.3 Kibet ago\\
\glt `Kibet was born long ago.'

\ex\gll\textipa{k\'I-y\^a:t-\'ey} \textipa{k\'urk\'e:t}\\
1\pl{}-open-\ipfv{}.3 door\\
\glt `The door is being opened.' 
\end{xlist}%\citet[126]{Creider:1989}
\end{exe}


Subjects of passive sentences are in the Accusative\is{case!individual forms!accusative} form in Maa\il{Maa} \citepalias[175]{Tucker:1955}.
\citet{Payne:2007}\is{valency-decreasing construction!antipassive|(} lists three kinds of verbal diathesis: middle\is{valency-decreasing construction!middle}, antipassive and impersonal\is{valency-decreasing construction!impersonal construction} passive.
In the middle (\ref{MaaPass}a) and antipassive (\ref{MaaPass}b), the subject is in the Nominative\is{case!individual forms!nominative}  case. 
In the impersonal passive on the other hand the subject is in the Accusative\is{case!individual forms!accusative} (i.e. zero-coded case), as shown in (\ref{MaaPass}c). 
Compare this with the active counterpart of the sentence (\ref{MaaPass}d).

\begin{exe}\ex\label{MaaPass}\langinfo{Maa}{}{\citealp[ex.11, ex.13, ex.16, ex.15]{Payne:2007}}
\begin{xlist}
\ex\gll\textipa{N-\'e-duN-o} \textipa{En-\textbf{\'am\`Uk\`E}}\\
\con{}-3-cut-\Mid{}.\Non\_\pfv{} \fem{}.\sg{}-sandal.\nom{}\\
\glt `The shoe was cut.'

\ex\gll\textipa{n-\'e-ramat-\'Ish\`o} \textipa{Ol-\textbf{mUrran\'I}}\\
\con{}-3-tend.livestock-\antip{} \mas{}.\sg{}-warrior.\nom{}\\
\glt `The warrior herds [e.g. cows].'

\ex\gll\textipa{E-tE-En-\'ak-\`I} \textipa{Ol-ap\'urr\`on\`I}\\
3-\pfv{}-tie-\pfv{}-\pass{} \mas{}.\sg{}-thief.\acc{}\\
\glt `The thief was arrested.'

\ex\gll\textipa{E-Ib\'UN-\'a} \textipa{I-s'IkarIn\'I}	\textipa{Ol-ap\'urr\`on\`I}\\
3-catch-\pfv{} \pl{}-police.\nom{} \mas{}.\sg{}-thief.\acc{}\\
\glt `The policemen have arrested the thief.'
\end{xlist}
\end{exe}\is{valency-decreasing construction|)}
\is{valency-decreasing construction!passive|)}
\is{valency-decreasing construction!antipassive|)}


%%%Valency increasing operations%%%%

%In {Tennet\il{Tennet}} causers are in the Nominative case (\ref{TenCaus}), the causative construction is of the periphrastic type and employs the verb `to give' in order to express causation. 
%For two more Nilo-Saharan languages there is information on causative subjects, though unfortunately without illustrative examples of the case-marking.
%In Turkana\il{Turkana} the causer in a causative construction receives Nominative case \citep[200]{Dimmendaal:1982}, while in relative clauses the causer is expressed in an instrumental phrase. 
%All participants introduced by verbal diathesis have the basic tonal pattern in {Maa\il{Maa}} \citep[666]{Koenig:2006}.

%\medskip
%{Tennet\il{Tennet}} \citep[240]{Randal:1998}
%\begin{exe}\ex\label{TenCaus}
%\begin{xlist}
%\ex\gll\textipa{a-\'any\'Ik} \textipa{l\=*ok\'{\=*u}l\'{\=*I}-i} \textipa{d\'{\=*o}l\^{\=*e}c} \textipa{k-\'a-t\'ar\'ar}\\
%\pfv{}-give Lokuli-\nom{} child 3\sbj{}-laugh\\
%`Lokuli made the child laugh.' 

%\ex\gll\textipa{a-ta\'ar\'ar} \textipa{d\=*ol\'{\=*e}c}\\
%\pfv{}-laugh child.\nom{}\\
%`The child laughed.' 
%\end{xlist}
%\end{exe}

All these findings are summarized in Table~\ref{NiloNonBasic}. 
Unlike from the previous sections, relative clauses in the Nilo-Saharan languages always employ the nominative\is{case!individual forms!nominative}  case to mark subjects. 
If any variation is found among dependent clauses, it is with complement clauses.

\begin{table}[ht]
\centering
\begin{tabular}{lcccc%c
}
\hline \hline
\bfseries language&\bfseries S rel&\bfseries S adv&\bfseries S compl&\bfseries S VDC%$&\bfseries S VIC
\\
\hline
Maa\il{Maa}&\textbf{\nom{}}&{-}&{-}&\acc{}/\textbf{\nom{}}%&\acc{}
\\
%\hdashline
Murle\il{Murle}&\textbf{\nom{}}&{-}&\acc{}&\textbf{\nom{}}%&{}
\\
%\hdashline
Nandi\il{Nandi}&\textbf{\nom{}}&{-}&{-}&\textbf{\nom{}}%&{}
\\
%\hdashline
P\"ari\il{P\"ari}&{-}&\textbf{\nom{}}&\textbf{\nom{}}&{-}%&
\\
%\hdashline
Tennet\il{Tennet}&\textbf{\nom{}}&{-}&{-}&\textbf{\nom{}}%&\textbf{\nom{}}
\\
%\hdashline
Turkana\il{Turkana}&\textbf{\nom{}}&{-}&\acc{} (topic)&\textbf{\nom{}}%&\textbf{\nom{}}
\\
\hline \hline
\end{tabular}
\caption{The marking of subjects in non-basic clauses in Nilo-Saharan}\label{NiloNonBasic}%\\
\end{table}


%%%%%%%%%%%%%%%%%%%%%%%%%%%%%
\section{Afro-Asiatic}\label{NonBasicAfro}
%%%%%%%%%%%%%%%%%%%%%%%%%%%%%

Non-basic clauses in the Afro-Asiatic languages exhibit little, if any, deviation from the general pattern of marking subjects with nominative\is{case!individual forms!nominative}  case.
Only the passive construction in Boraana\il{Oromo (Boraana)} Oromo might be different in this respect. 
Unfortunately, quite a few questions about the grammar of this construction remain unanswered. 
In general, more detailed information on non-basic clauses in the Afro-Asiatic languages would be very desirable.
Especially lacking is information about dependent clauses\is{clause-type!dependent} other than relative clauses\is{clause-type!relative clause}, since only few grammars treat this topic. 
%%%%%%%Dependent clauses%%%%%%%%%%%%%

Relative\is{clause-type!relative clause|(}\is{clause-type!adverbial clause|(} clauses in the Boraana\il{Oromo (Boraana)} dialect of Oromo mark subjects with Nominative\is{case!individual forms!nominative}  case (\ref{BorRel}).
In the Harar\il{Oromo (Harar)} dialect, the subject in dependent clauses is marked with Nominative\is{case!individual forms!nominative}  case as well.
This is true for relative clauses (\ref{HarRel}a, b) as well as adverbial\is{clause-type!adverbial clause} clauses (\ref{HarRel}c). 
The latter, however, do not have an overt subject inside the dependent clause in most cases.

\begin{exe}\ex\label{BorRel}\langinfo{Oromo (Boraana)}{Eastern Cushitic; Ethiopia}{\citealp[104]{Stroomer:1995}, \citealt[125]{OromoTexts}}
\begin{xlist}
\ex \gll nam\textbf{-ii} beesee hat-e is=aa\\
man-\nom{} money steal-3\sg{}.\mas{}.\pst{} him=\lin{}(\question{})\\
\glt `Is he the one that stole the money?'

\ex \gll Nam\textbf{-i} Diido ijeese Jaanjamtu'\\
person-\nom{} Diido killed Jaanjamtu\\
\glt `The people who killed Diido were the Janjamtu.' %\citet[125]{OromoTexts}
\end{xlist}
\end{exe}

\begin{exe} \ex\label{HarRel}\langinfo{Oromo (Harar)}{Eastern Cushitic; Ethiopia}{\citealp[131, 143]{Owens:1985}}
\begin{xlist}
\ex\gll intal\textbf{-t\'\i i} (taan) inn\'\i i arke \'ac jirti\\
girl-\nom{} (as) 3\sg{}.\mas{}.\nom{} saw there exist.3\sg{}.\fem{}\\
\glt`The girl he saw in there.'
\ex\gll namicc\textbf{-\'\i i} (xan) intal-t\'\i i is\'a bar\'eed-d\'uu \'ac jira\\
man-\nom{} (as) girl-\nom{} 3\sg{}.\mas{}.\acc{} pretty-\fem{} there exist.3\sg{}.\mas{}\\
\glt `The man whose daughter is pretty is there.'
\ex\gll hag\'a \textbf{is\'\i in} d'uf-t-\'u taa'-e\\
until 3.\sg{}.\fem{}.\nom{} come-\fem{}-\dep{} stay-\pst{}\\
\glt `He stayed until she came.'
%\citet[143]{Owens:1985}
\end{xlist}
\end{exe} \is{clause-type!adverbial clause|)}

Arbore\il{Arbore} relative clauses are discussed by \citet[314]{Hayward:1984}. 
In most of his examples there is no independent NP functioning as the subject of the relative clause. 
One of the few examples in which a subject is realized within the relative clause is given in (\ref{ArbDep}a). 
The subject is in the Nominative\is{case!individual forms!nominative}  case. 
Example\is{clause-type!complement clause|(} (\ref{ArbDep}b) appears to be a complement clause according to the translation. 
However, the structure of this example is not discussed by Hayward. 
In this case, the subject of the complement clause \emph{saal-t-\'atto} `that woman' is zero-coded. 
However, the referent coded by `that woman' is apparently topicalised in this construction. 
This fact could account for the absence of Nominative case-marking.

\begin{exe}\ex\label{ArbDep}\langinfo{Arbore}{Eastern Cushitic; Ethiopia}{\citealp[318, 321]{Hayward:1984}}
\begin{xlist}
\ex\gll maar-t-\'a \textbf{s[\textsubdot{e}\textglotstop\textsubdot{e}]} \textsubdot{d}a[l:-]e hunna ma \textsubdot{k}\'a[\textsubbridge{t}:\textsuperscript{h}]o\\
calf.\fem{}.\nom{} cow.\nom{} give\_birth-3\sg{}.\pfv{} ? who ? \\
\glt`The calf which the cow gave birth to has no strength.'
%\begin{exe}\ex\gll maar-t-\'a s\[\textsubdot{e}\textglotstop\textsubdot{e}\] \textsubdot{d}a\[l:-\]e hunna ma \textsubdot{k}\'a\[\textsubbridge{t}:\textsuperscript{h}\]o\\
%calf.\fem{}.\nom{} cow.\nom{} give\_birth-3\sg{}.\prf{} ? who ? \\
%`The calf which the cow gave birth to has no strength.'
%\end{exe} \citet[318]{Hayward:1984} %%Original version of example above, some formating problems check original source and tipa-manual

\ex\gll \textbf{saal-t-\'atto}, hatt-\'ay zaHate k'ub-\'a\textipa{N} k'ab-a\\
woman-\fem{}-\dist{} that-3\sg{} die.3\sg{} hand-INST have-? \\
\glt`That woman, I know that she died.' %[321]
\end{xlist}
\end{exe}	
\is{clause-type!complement clause|)}

In K'abeena\il{K'abeena} relative\is{clause-type!relative clause} clauses, the common argument is always realized in the main clause and is marked for its function there.  
There is no resumptive or relative pronoun, and as such, the common argument is gaped in the  relative clause. 
Other arguments within the relative clause get the same marking which they would receive in a main clause.
This means that subjects inside the relative clause are in the Nominative\is{case!individual forms!nominative}  case (\ref{KabDep}a).
Adverbial\is{clause-type!adverbial clause} clauses mark their subjects via Nominative\is{case!individual forms!nominative}  case as well (\ref{KabDep}b).
\enlargethispage{2\baselineskip}

\begin{exe}\ex\label{KabDep}\langinfo{K'abeena}{Eastern Cushitic; Ethiopia}{\citealp[287]{Crass:2005}}
\begin{xlist}
\raggedright
\ex\gll \textbf{n\'a'u}-n\textsuperscript{i} nassinoon-si c'uul\textsuperscript{u} laga\textprimstress{y}o-'n\textsuperscript{e}\\
1\pl{}.\nom{}-\emphat{} raise.\pfv{}.1\pl{}-3\sg{}.\mas{}.\obj{}.\relativ{} child.\nom{} leave.\pfv{}.3\sg{}.\mas{}-1\pl{}.\obj{}\\
\glt `The child which we raised has left us.'\\
original translation: `Das Kind, das wir selbst aufgezogen haben, hat uns verlassen.' %[287]

\ex\gll \textbf{got\textsuperscript{u}} wajjo-r\textsuperscript{a} hilikk'\textsuperscript{i} ke'yoomm\textsuperscript{i}\\
hyena.\nom{} scream.\pfv{}.3\sg{}.\mas{}-\tmp{} be\_shocked.\cvb{}.1\sg{} stand\_up.\pfv{}.1\sg{}\\
\begin{sloppypar}\glt  `When/After/Because a hyena (had) shrieked, I was shocked and stood up.' %\end{sloppypar}
\\
%\begin{sloppypar} 
original translation: `Als/Nachdem/Weil eine Hy\"{a}ne schrie/ ge\-schrie\-hen hatte, erschrak ich und stand auf.' \end{sloppypar} %[308]
\end{xlist}
\end{exe}

Finally, for Gamo\il{Gamo}, the two types of dependent clause on which information is provided are relative\is{clause-type!relative clause} clauses (\ref{GamDep}a) and complement\is{clause-type!complement clause} clauses (\ref{GamDep}b). 
Both mark subjects in the Nominative\is{case!individual forms!nominative}  case.

%\pagebreak
\begin{exe}\ex\label{GamDep}\langinfo{Gamo}{Omotic; Ethiopia}{\citealp[400, 361]{Hompo:1990}}
\begin{xlist}
\raggedright
\ex\gll {\rm [}nun\textbf{-i} be\textglotstop-i-d-a{\rm ]} misiri-y-aa pi\`sa o\`s-a-us\\
        \hspaceThis{[}1\pl{}-\nom{} see-\persm-\tns{}-\complx{} woman-\defsc{}-\nom{} basket.\acc{} make-\persm-\tns{}.\complx{}\\
\glt `The woman whom we saw makes baskets.'

\ex\gll tan-i {\rm [}nen\textbf{-i} ora\`s-a oi\v c-onta malaa{\rm ]} yotadis-\v sin\\
1\sg{}-\nom{} \hspaceThis{[}2\sg{}-\nom{} Oratsi-\acc{} ask-\Neg{}-\Inf{} that tell-\persm.\tns{}.\complx{}.\aux{}\\
\glt `I have told you not to ask Oratsi.'
\end{xlist}
\end{exe}\is{clause-type!relative clause|)}


%%%%%valency-decreasing operations%%%%%%%%%%%%%%%
Most\is{valency-decreasing construction|(}\is{valency-decreasing construction!passive|(} Afro-Asiatic languages for which voice alternations are discussed in the grammar have a construction labeled as passive.
However, these passive constructions do not exhibit identical properties across the languages, especially concerning the passivisation of non-P arguments. 
Passive subjects that correspond to direct objects in the active counterpart of a clause are mark\-ed with Nominative\is{case!individual forms!nominative}  case in K'abeena\il{K'abeena} (\ref{KabPass},~\ref{KabPass2}), Gamo\il{Gamo} (\ref{GamPass}a, b), and Harar\il{Oromo (Harar)} Oromo (\ref{HarPass}a, b).
%The subject of passive sentences is marked with nominative case if it is a promoted direct object in Gamo\il{Gamo}.
For Gamo\il{Gamo} and Harar\il{Oromo (Harar)} Oromo, the grammars provide additional information on the passivisation of ditransitive clauses. 
Gamo\il{Gamo} recipients or oblique marked participants which get promoted to subject of a passive sentence keep their original case-marking but gain verbal agreement \citep[394]{Hompo:1990}, as demonstrated in (\ref{GamPass}c). 
In Harar\il{Oromo (Harar)} Oromo both objects of ditransitives can be promoted to Nominative\is{case!individual forms!nominative}  marked subject (\ref{HarPass}b, c). 
Passive agents cannot be expressed in Harar\il{Oromo (Harar)} Oromo, while they can be realized as Locative phrases in K'abeena\il{K'abeena} (\ref{KabPass2}). 
For Gamo\il{Gamo}, no information on this topic is provided in the grammar.\enlargethispage{2\baselineskip}


\begin{exe}\ex\label{KabPass}\langinfo{K'abeena}{}{\citealp[275]{Crass:2005}}
\begin{xlist}
\ex\gll \textbf{'daliil\textsuperscript{i}} 'osa'lanto\\
Dalil.\nom{} laugh.\pass{}.\pfv{}.3\sg{}.\fem{}\\
\glt `Dalil was laughed at.'\\
original translation: `Dalil wurde ausgelacht.'
\ex\gll \textbf{'ilf\textsuperscript{u}} 'osa'lito\\
Ilfu.\nom{} laugh.\pfv{}.3\sg{}.\fem{}\\
\glt `Ilfu laughed (at Dalil).'\\ 
original translation: `Ilfu lachte.' or `Ilfu lachte (Dalil) aus.'
\end{xlist}
\end{exe}

%\pagebreak
\begin{exe}\ex\label{KabPass2}\langinfo{K'abeena}{}{\citealp[275]{Crass:2005}}
\begin{xlist}
\ex\gll \textbf{lal\textsuperscript{u}} faangaan\textsuperscript{i} 'aa'ammo\\
cattle.\nom{} thief.\loc{} take.\pass{}.\pfv{}.3\sg{}.\mas{}\\
\glt `The cattle was stolen by thieves.'\\
original translation: `Die Rinder wurden von Dieben gestohlen.'
\ex\gll \textbf{faangoo} lalu 'aa'ito\\
thief.\nom{} cattle.\acc{} take.\pfv{}.3\pl{}\\
\glt`Thieves have stolen the cattle.'\\
original translation: `Diebe haben die Rinder gestohlen.'
\end{xlist}
\end{exe}

\begin{exe}\ex\label{GamPass}\langinfo{Gamo}{}{\citealp[378, 394]{Hompo:1990}} 
\begin{xlist}
\ex\gll de\v sa-z\textbf{-ii} danna-z-a-s imme-ett-i-d-es\\
goat-\defsc{}-\nom{} judge-\defsc{}-\gen{}-\Recip{} give-\pass{}-\persm-\tns{}-\complx{}\\
\glt `The goat was given to the judge.' 

\ex\gll kawo-z-ii zall\textglotstop an\v ca-t-a-n wod'-ett-i-d-es\\
king-\defsc{}-\nom{} merchant-\pl{}-\acc{}-\loc{} kill-\pass{}-\persm-\tns{}-\complx{} \\
\glt`The king was killed by merchants.' %\citet[394]{Hompo:1990}

\ex \gll ta-s zar-ett-a-d-is\\
1\sg{}-\Recip{} answer-\pass{}-\persm-\tns{}-\complx{}\\
\glt`I was answered.'
\end{xlist}
\end{exe}

\begin{exe}\ex\label{HarPass}\langinfo{Oromo (Harar)}{}{\citealp[172, 173]{Owens:1985}}
\begin{xlist}
\ex\gll\textipa{makiin\'aa-n} \textipa{n\'i} \textipa{tolf-am-t-a}\\
car-\nom{} \foc{} repair-\pass{}-\fem{}-\ipfv{}\\
\glt `The car will be repaired.'  

\ex\gll\textipa{an} \textipa{hucc'\'u-n} \textipa{d'owwat-am-e}\\
1\sg{}.\nom{} clothes-1\sg{} deny-\pass{}-\pst{}\\
\glt `I was denied the clothes.'

\ex\gll\textipa{hucc'\'uu-n} \textipa{n\'a} \textipa{d'owwat-am-t-e}\\
clothes-\nom{} 1\sg{}.\acc{} deny-\pass{}-\fem{}-\pst{}\\
\glt `The clothes were denied me.' 
\end{xlist} 
\end{exe}

All examples of the passive in the Boraana\il{Oromo (Boraana)} dialect of Oromo mark their grammatical subject with the focus-marker\is{focus!overt marker} \emph{yaa} (\ref{BorPass}a, b, c).\footnote{As the following example demonstrates, at least pronouns can be in the Nominative case when marked by the focus-marker in Boraana\il{Oromo (Boraana)} Oromo. 
\citet[74]{Stroomer:1995} does not comment any further on case-marking with the focus-marker.

\begin{exe}\ex\gll \textbf{aani} yaa kalee billaa gabayaa bit-ad'd'-e\\
1\sg{}.\nom{} \foc{} yesterday knife market buy-\Mid-1\sg{}.\pst{}\\
\glt `Yesterday I bought a knife at the market.'
\zlast} 
% As is to be expected for focused subjects, they do not receive Nominative case-marking (\ref{BorPass}). 
% check this focus marker
\citet{Stroomer:1995} does not specify whether focus marking is obligatory for subjects of passives. 
Therefore, it is not clear whether passive subjects would receive Nominative case-marking in such a context (if the grammar of Boraana\il{Oromo (Boraana)} Oromo allows for it at all).
Compare also the impersonal\is{valency-decreasing construction!impersonal construction} construction in (\ref{BorImpers}), in which the the focus-marker is also used. %and the active sentence in (\ref{BorKill})

\begin{exe} \ex\label{BorPass}\langinfo{Oromo (Boraana)}{}{\citealp[74, 89, 90]{Stroomer:1995}}
\begin{xlist}\ex\gll fooni yaa d'aab-am-ani\\
meat \foc{} cook-\pass{}-3\pl{}.\pst{}\\
\glt `The meat has been cooked.'

\ex\gll mana yaa jaar-am-e\\
house \foc{} build-\pass{}-3\sg{}.\pst{}\\
\glt `The house has been built.'

\ex\gll sangaa yaa k'al-am-e\\
ox \foc{} slaughter-\pass{}-3\sg{}.\pst{}\\
\glt `The ox has been killed.'

\ex\label{BorImpers}\gll sangaa yaa ijees-ani\\
ox \foc{} kill-3\pl{}.\pst{}\\
\glt `They killed the ox' %\citep[90]{Stroomer:1995}

%\ex\label{BorKill}\gll namic=aa looni ijees-e\\
%man=\lin{} cow kill-3\sg{}.\mas{}.\pst{}\\
%`The man killed the cow.' %\citet[105]{Stroomer:1995}
\end{xlist} 
\end{exe}\is{valency-decreasing construction|)}\is{valency-decreasing construction!passive|)}  



%%%%%%%%%%%Valency increasing operations%%%%%%%%%%%%%%%%%%%%
%Causative constructions work similarly in all Afro-Asiatic languages on which data is available. %reformulate
%The newly introduced subject (i.e. the causer) receives Nominative case-marking while the the subject of the non-causative version of the clause is in the Accusative case. This is illustrated by the following examples from Kab'eena (\ref{KabCaus}), Gamo\il{Gamo} (\ref{GamCaus}) and the Harar\il{Oromo (Harar)} dialect (\ref{HarCaus}) as well as the Boraana\il{Oromo (Boraana)} dialect (\ref{BorCaus}) of Oromo.

%Causers introduced via cau\-sa\-tive structures are mark\-ed in Nominative case the original subject receives Accusative marking in K'abeena\il{K'abeena}.
%\medskip
%{K'abeena\il{K'abeena}} \citep[Cushitic; Ethiopia; ][275]{Crass:2005}
%\begin{exe}\ex\label{KabCaus}
%\begin{xlist}
%\ex\gll daliil\textsuperscript{i} 'ilfo 'osalsii\v s\v so\\
%Dalil.\nom{} Ilfu.\acc{} laugh.\caus{}.\pfv{}.3\sg{}.\mas{}\\
%`Dalil made Ilfu laugh.'\\
%Original translation: `Dalil brachte Ilfu zum lachen.'
%\ex \gll 'ilf\textsuperscript{u} 'osa'lito\\
%Ilfu.\nom{} laugh.\pfv{}.3\sg{}.\fem{}\\
%`Ilfu laughed.'\\
%Original translation: `Ilfu lachte.'
%\end{xlist}
%\end{exe}

%If a causer is introduced into a sentence in Gamo\il{Gamo}, it is marked with nominative, while the subject of the original sentence is marked with accusative case \citep[395]{Hompo:1990}
%{Gamo\il{Gamo}} \citep[395]{Hompo:1990}
%\begin{exe}\ex\label{GamCaus} \gll Tani iza nu afila me\v c-iss-$\emptyset$-and-is\\
%1\sg{}.\nom{} 3\sg{}.\acc{} our clothes.\acc{} wash-\caus{}-\persm-\tns{}-\complx{}\\
%`I will make him wash our clothes.'
%\end{exe}

%Causers are in Nominative case causees in Accusative case the Harar\il{Oromo (Harar)} dialect of Oromo.
%{Oromo (Harar\il{Oromo (Harar)})} \citep[99]{Owens:1985}
%\begin{exe}\ex\label{HarCaus}\gll\textipa{an} \textipa{m\'uus\'aa} \textipa{\'al\'Ii} \textipa{wek\'I-n} \textipa{gaafacc-iise}\\
%1\sg{}.\nom{} Musa Ali thing ask.1\sg{}-\caus{}\\
%`I made Musa ask Ali a question.' \end{exe} 

%In the Boraana\il{Oromo (Boraana)} Oromo causative construction, the newly introduced subjects (i.e. the causer) is in the Nominative case (\ref{BorCaus}).
%{Oromo (Boraana\il{Oromo (Boraana)})} \citep[49]{Owens:1982} 
%\begin{exe}\ex\label{BorCaus} \gll \`an\`in nam s\`un m\`idaan b\`icc\`iis\textsuperscript{e}\\
%1\sg{}.\nom{} man \dem{} grain buy.\caus{}\\
%`I made that man buy the grain.' \end{exe}

%Other, less clear examples:

%{Oromo (Boraana\il{Oromo (Boraana)})} \citep[131]{OromoTexts}\footnote{somnetti might be related to \emph{soba} `to decieve'}, \citep[74]{Stroomer:1995}
%\begin{exe} \ex \begin{xlist}
%\ex\gll gurba xana, somnetti, sa'a kana k'al-c'iifn\textsuperscript{a}, jed'a-n\textsuperscript{i}\\
%boy 1Pl.\poss{} ? cow this slaughter-\caus{}.? say-3\pl{}.\prs{}\\
%`They said: ``Let us deceive our brother and cause him to slaughter this cow.'' '

%\ex\gll isa yaayuu ciifa tol-s-ani\\
%3\sg{}.\propnoun \foc{} chief make-\caus{}-3\pl{}.\pst{}\\
%`They made him chief.'
%\end{xlist} %\begin{flushright}\citet[74]{Stroomer:1995}\end{flushright}
%\end{exe}  

Table~\vref{AfroNonBasic} summarizes the data. 
The Afro-Asiatic marked"=S languages make the most regular use of the nominative-case for encoding subjects in non-basic clauses. 
Only the Boraana\il{Oromo (Boraana)} Oromo passive\is{valency-decreasing construction!passive|)} seems to have a peculiar pattern.
However, very little is known about the structure of this construction.%
%\enlargethispage{2\baselineskip}

\begin{table}[htbp]
\centering
\begin{tabular}{lcccc%c
}
\hline \hline
\bfseries language&\bfseries S rel&\bfseries S adv&\bfseries S compl&\bfseries S VDC%&\bfseries S VIC
\\
\hline
Arbore\il{Arbore}&\textbf{\nom{}}&{-}&{-}&{-}%&{}
\\
%\hdashline
Gamo\il{Gamo}&\textbf{\nom{}}&{-}&\textbf{\nom{}}&\textbf{\nom{}}%&\textbf{\nom{}}
\\
%\hdashline
K'abeena\il{K'abeena}&\textbf{\nom{}}&\textbf{\nom{}}&{-}&\textbf{\nom{}}%&\textbf{\nom{}}
\\
%\hdashline
Oromo (Boraana\il{Oromo (Boraana)})&\textbf{\nom{}}&{-}&{-}&{\acc{}+\foc{}}?%&\textbf{\nom{}}
\\
%\hdashline
Oromo (Harar\il{Oromo (Harar)})&\textbf{\nom{}}&\textbf{\nom{}}&{-}&\textbf{\nom{}}%&\textbf{\nom{}}
\\
\hline \hline
\end{tabular}
\caption{Subject-marking of non-basic clauses in the Afro-Asiatic marked"=S languages}\label{AfroNonBasic}
\end{table}

\section{Summary}\label{NonBasicSum}

Non-basic clauses mark their subjects with regular S-case-marking\footnote{Remember that in the terminology established in Chapter~\ref{introduction}  the label S-case refers to the case-form that is used among other functions for marking the single argument of intransitive verbs (S). 
It corresponds to the nominative case in languages with nominative"=accusative alignment and the absolutive\is{case!individual forms!absolutive} in those with ergative"=absolutive alignment.} in most instances. 
However, some marked"=S languages employ non-standard subject case-marking for some of the roles discussed in this chapter. 
If a non-basic subject receives a different case-form than the S-case, this will usually be the zero-case.
An overview of the data of all marked"=S languages investigated in this chapter is provided in Table~\vref{OverviewNonBasCl}.
\begin{table}[htb]
\centering
\begin{tabular}{lcccc%c
}
\hline \hline
\bfseries language&\bfseries S rel&\bfseries S adv&\bfseries S compl&\bfseries S VDC%&\bfseries S VIC
\\
\hline
Aji\"e\il{Aji\"e}&{-}&\textbf{\nom{}}&{-}&{-}%&{}
\\
%\hdashline
Arbore\il{Arbore}&\textbf{\nom{}}&{-}&{-}&{-}%&{}
\\
%\hdashline
Cocopa\il{Cocopa}&{-}&\textbf{\nom{}}&{-}&-%&
\\
%\hdashline
Diegue\~no\il{Diegue\~no (Mesa Grande)} &\acc{}&\acc{}&{-}&-%&
\\
%\hdashline
Gamo\il{Gamo}&\textbf{\nom{}}&{-}&\textbf{\nom{}}&\textbf{\nom{}}%&\textbf{\nom{}}
\\
%\hdashline
Havasupai\il{Havasupai}&{-}&{-}&{-}&\textbf{\nom{}}%&\textbf{\nom{}}
\\
%\hdashline
Jamul\il{Jamul Tiipay} Tiipay&\acc{}&\textbf{\nom{}}&{-}&{}%&\textbf{\nom{}}
\\
%\hdashline
K'abeena\il{K'abeena}&\textbf{\nom{}}&\textbf{\nom{}}&{-}&\textbf{\nom{}}%&\textbf{\nom{}}
\\
%\hdashline
Maa\il{Maa}&\textbf{\nom{}}&{-}&{-}&\acc{}/\textbf{\nom{}}%&\acc{}
\\
%\hdashline
Maidu\il{Maidu}&{-}&{-}&{-}&{-}%&\textbf{\nom{}}?
\\
%\hdashline
Mojave\il{Mojave}&\acc{}&\textbf{\nom{}}&\textbf{\nom{}}&\acc{}%&\textbf{\nom{}}
\\
%\hdashline
Murle\il{Murle}&\textbf{\nom{}}&{-}&\acc{}&\textbf{\nom{}}%&{}
\\
%\hdashline
Nandi\il{Nandi}&\textbf{\nom{}}&{-}&{-}&\textbf{\nom{}}%&{}
\\
%\hdashline
Nias\il{Nias}&\erg{}&{-}&{-}&{\erg{}}%&\textbf{\abs{}}
\\
%\hdashline
Oromo (Boraana\il{Oromo (Boraana)})&\textbf{\nom{}}&{-}&{-}&{\acc{}+\foc{}}%&\textbf{\nom{}}
\\
%\hdashline
Oromo (Harar\il{Oromo (Harar)})&\textbf{\nom{}}&\textbf{\nom{}}&{-}&\textbf{\nom{}}%&\textbf{\nom{}}
\\
%\hdashline
P\"ari\il{P\"ari}&{-}&\textbf{\nom{}}&\textbf{\nom{}}&{-}%&
\\
%\hdashline
Savosavo\il{Savosavo}&\textbf{\gen{}}&\textbf{\nom{}/\gen{}}&{-}&\textbf{\nom{}}%&\textbf{\nom{}}
\\
%\hdashline
Tennet\il{Tennet}&\textbf{\nom{}}&{-}&{-}&\textbf{\nom{}}%&\textbf{\nom{}}
\\
%\hdashline
Turkana\il{Turkana}&\textbf{\nom{}}&{-}&\acc{} (topic)&\textbf{\nom{}}%&\textbf{\nom{}}
\\
%\hdashline
Wappo\il{Wappo}&\acc{}&\acc{}&\acc{}&\textbf{\nom{}}%&\textbf{\nom{}}
\\
%\hdashline
Yavapai\il{Yavapai}&\acc{}&\textbf{\nom{}}&{-}&\textbf{\nom{}}%&{}
\\
\hline \hline
\end{tabular}
\caption{Overview of the marking of subjects in non-basic clauses}\label{OverviewNonBasCl}
\end{table}

Atypical\is{clause-type!dependent|(} case-marking of subjects is most frequently found with dependent clauses. 
Within the domain of dependent clauses, relative clauses\is{clause-type!relative clause} are the most likely type of dependent clause to employ an exceptional case-form for the subject. 
This is particular obvious for the languages of North America, especially Wappo\il{Wappo} and the Yuman languages. 
While the Yuman languages only use zero-coding for subjects in relative clauses, Wappo\il{Wappo} does not mark any type of dependent subject with the Nominative. 
In addition atypical case-marking for dependent subjects is found in Nias\il{Nias}, which seems to reverse the marking relations in relative clauses; it is also found in Savosavo\il{Savosavo}, where Genitive marking is obligatorily (relative\is{clause-type!relative clause} clauses) or optionally used (adverbial\is{clause-type!adverbial clause} clauses) in dependent clauses. 
Also in Africa, some special patterns are found in this domain of grammar. 
Murle\il{Murle} uses the Accusative case for  subjects of complement\is{clause-type!complement clause} clauses. 
In Turkana\il{Turkana}, subjects of complement clauses (and all other arguments) obligatorily have to appear in the pre-verbal topic-position\is{topic} and thus receive Accusative case. 
While the Yuman languages use zero-coding for relative clause subjects and overt marking for other dependent clauses, Murle\il{Murle} and Turkana\il{Turkana} show the reverse. 
Accordingly, there does not appear to be a close association of any type of dependent clause with the case-marking pattern found in main clauses. 

The scale proposed for grammatical integration of clause-types \citep{Payne:1997} discussed in Section~\ref{DepCl} could serve as an indicator of how different types of clauses might be expected to behave.  
Payne's scale would suggest that relative clauses\is{clause-type!relative clause} should behave more like independent clauses than any other type of dependent clause. 
The data does not provide clear support for such a relation, although more languages would be needed to test for significant correlations. 
In addition, P\"ari\il{P\"ari} exhibits a marked"=nominative system only in dependent clauses, while other clauses have standard ergative"=absolutive alignment.\is{clause-type!dependent|)} 

Subjects\is{valency-decreasing construction|(} in valency-decreasing %and valency-increasing 
constructions typically employ the S-case. 
A notable exception to this general tendency is Maa\il{Maa}, where passive\is{valency-decreasing construction!passive|)} %and causative (among other valency-increasing constructions) 
employs the Accusative case to mark subjects. 
Also the `passive' constructions of Mojave\il{Mojave} and Boraana\il{Oromo (Boraana)} Oromo employ idiosyncratic marking of the subjects. \enlargethispage{\baselineskip}
However, these constructions (like the Nias\il{Nias} Passive) demand better understanding than presently available before drawing any conclusions from this behavior.\is{valency-decreasing construction|)}

		


