\author{Brett Reynolds and Peter Evans} %use this field for editors as well
\title{Negation in English and other languages}
% \subtitle{Add subtitle here if it exists}
\ISBNdigital{978-3-96110-500-7}
\ISBNhardcover{978-3-98554-129-4}
\BookDOI{10.5281/zenodo.14843487}
\typesetter{Brett Reynolds}
\proofreader{Katrin Menzel, Fabienne Michelet Pickave, Sune Gregersen, Brigitta Bence}
% \lsCoverTitleSizes{51.5pt}{17pt}
\renewcommand{\lsSeries}{classics} 
\renewcommand{\lsSeriesNumber}{8} 
\renewcommand{\lsID}{501}

\BackBody{Otto Jespersen's landmark study of negation provides a wide-ranging analysis of how languages express negative meaning. Drawing on an impressive array of historical texts and comparative examples, primarily from Germanic and Romance languages, Jespersen examines the forms, functions, and historical development of negative expressions. The work traces the evolution of negative markers, analyzes how negative prefixes modify word meanings, and reveals coherent patterns in how languages structure negative expressions.
 
Through meticulous analysis of authentic examples, Jespersen documents both common patterns and language-specific variations in negative expressions. His treatment of topics such as double negation, the distinction between special and nexal negation, and the various forms of negative particles provides a methodical account of negation's complexity. The work's enduring importance stems not only from its analysis of the cyclical renewal of negative markers (later termed “Jespersen's Cycle”) but from its comprehensive scope and detailed examination of negative expressions across multiple languages and historical periods.
 
This new critical edition makes this classic work accessible to modern readers while preserving its scholarly depth. The text has been completely re-typeset, with examples presented in contemporary numbered format and non-English examples given Leipzig-style glosses. A new introduction contextualizes Jespersen's achievement and demonstrates its continued significance for current linguistic research.}

\lsCoverTitleSizes{45pt}{15mm}

\patchcmd{\lsImpressum}{the authors}
  {%
    Brett Reynolds \& Peter Evans\\
    Introduction © \lsYear, Olli O. Silvennoinen%
  }{}{}

\makeatletter
\renewcommand{\frontcovertoptext}[3][white]{%
    \renewcommand{\newlineCover}{\\}
    \node [ font=\lsCoverTitleFont,
            below right = 10mm and 7.5mm of CoverColouredRectangleFront.north west,
            text width=#2,
            align=left
            ] (lspcls_covertitle) {\color{#1}\raggedright\@title\par};

    \ifx\@subtitle\empty  % Is there a subtitle? If no, just print the author.
    \node [ font=\lsCoverAuthorFont,
            right,
            below = 11.2mm of lspcls_covertitle.south,
            text width=#2
            ] {\color{#1}\nohyphens{%
            {\lsCoverAuthorFont Otto Jespersen\bigskip\bigskip\\}
            \lsEditorPrefix%
            \ResolveAffiliations[output in groups=false, 
                                 output affiliation=false, 
                                 orcid placement=none,
                                 output authors font=\lsCoverAuthorFont,
                                 separator between two=\\,
                                 separator between multiple=\\,
                                 separator between final two=\\]
                                 {\@author}\bigskip\bigskip\\}
            {\Large With an introduction by\par}
            {\lsCoverAuthorFont Olli O. Silvennoinen}
            };
    \else % If yes, create a node for subtitle and author
    \node [ font=\lsCoverSubTitleFont,
            below = 8mm of lspcls_covertitle.south,
            text width=#2,
            align=left
            ] (lspcls_coversubtitle) {\color{#1}\raggedright\@subtitle\par};
    \node [
            font=\lsCoverAuthorFont,
            right,
            below = 11.2mm of lspcls_coversubtitle.south,
            text width=#2
            ] {\color{#1}\nohyphens{%
            {\lsCoverAuthorFont Otto Jespersen\bigskip\bigskip\\}
            \lsEditorPrefix%
            \ResolveAffiliations[output in groups=false, 
                                 output affiliation=false, 
                                 orcid placement=none,
                                 output authors font=\lsCoverAuthorFont,
                                 separator between two=\\,
                                 separator between multiple=\\,
                                 separator between final two=\\]
                                 {\@author}\bigskip\bigskip\\}
            {\Large With an introduction by\par}
            {\lsCoverAuthorFont Olli O. Silvennoinen}
            };
    \fi
    }
\makeatother
