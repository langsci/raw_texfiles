\ChapterAndMark{Negative Prefixes} 
\label{ch:13}
\addsec{\textit{Un-}, \textit{in-}}\label{negative_un_in}
\il{English!un-@\textit{un-}|(}
\il{English!in-@\textit{in-}|(}

\is{productivity of prefixes}
\is{prefixes!negative|(}
\is{lexical strata|(}
The most important negative prefixes are \textit{un-} and \textit{in-}, both etymologically going back to the same Arian % PE: I wonder about glossing this kind of thing. One could point out that it's just a different spelling of "Aryan", and then add that this is, or was, a respectable term in philology, and not just Nazi-approved racist twaddle. I hadn't known this myself: Wikipedia enlightened me. But then again if I can educate myself via Wikipedia, our readers can do the same.....
%Brett: I'm inclined to leave it be.  % PE: OK
form, \textit{n-} (syllabic), reduced from the negative word \textit{ne} (which gave also the Greek \textit{a-} ``privativum'', % Peter: I think this means "the Greek privative prefix \textit{a}", but I'm not at all sure. If I'm right, then as OJ presents other prefixes with hyphens appended, shouldn't this too have a hyphen? 
%Brett: I think it should. I've added one. % PE: Thanks!
see p.~\pageref{prefix_a} below). % Closed the parenthesis and (I hope correctly) specified OJ's "below"
\textit{Un-} is the native English form, while 
\il{Latin!in-@\textit{in-}}\textit{in-} is the Latin form, known to the English through numerous 
\il{French!in-@\textit{in-}}French and Latin words, and to some extent also productive in English itself. A good deal of hesitation has prevailed between the two prefixes, though now in most cases one or the other has been definitely preferred. We shall speak first of the form, next of the choice between the two prefixes, and finally of their meaning.


\label{sound_change}\textit{In-}, according to the rules of Latin phonology, has the alternate forms \textit{ig-} as in \textit{ignoble}, \textit{il-} as in \textit{illiterate}, \textit{im-} as in \textit{impossible}, \textit{ir-} as in \textit{irreligious}. 


In a few words, the sound of a word is changed, when this prefix is added:

\bigskip

\begin{tabular}{@{}llll@{}}
 \textit{pious}& [paiəs]& \textit{impious}& [impiəs]\\
 \textit{finite}& [fainait]& \textit{infinite}& [infinit]\\ % Should the "fi" within "[infinit]" be prevented from forming a ligature?
 \textit{famous}& [feiməs]& \textit{infamous}& [infəməs]\\
\end{tabular}

\bigskip

In the last word, the signification too is changed (see p.~\pageref{meaning_change}).

\is{lexical change|(}
Pretty often \textit{un-} is preferred before the shorter word, and \textit{in-} before the longer word derived from it, which is generally also of a more learned nature; thus we have these contrasts, as in  (\ref{ex:13-01}). % ??? PE: Sorry, but this seems awkward. (It seems to suggest that (1) is an example of a pair within the tiny table, but it isn't.) OJ has simply "thus we have". This is awkward too. Can one of us come up with a better idea?
%BR: Thus, we have these contrasts, as illustrated by the following pairs and exemplified in (\ref{ex:13-01}).

\bigskip

\begin{tabular}{@{}ll@{}}
 \textit{unable}& \textit{inability}\\
 \textit{unjust}& \textit{injustice}\\
 \textit{unequal}& \textit{inequality}\\
\end{tabular}

\ea \label{ex:13-01} some excuse for incivility if I \textit{was} uncivil\hfill(\href{https://archive.org/details/prideprejudice00aust/page/238/mode/2up?q=%22some+excuse+for+incivility%22&view=theater}{Austen, \textit{Pride} 239}).
\z


\textit{Un-} is preferred where the word has a distinctly native ending, as in

\bigskip

\begin{tabular}{ll}
 \textit{ungrateful}& \textit{ingratitude}.\\
\end{tabular}
\is{lexical strata|)}

\bigskip

Hence also the following examples (\ref{ex:13-02}) of participles in \textit{-d} with \textit{un-}, while the adjectives in \textit{-able} have \textit{in-}.

\ea \label{ex:13-02}
\ea
all the \emph{unnumber'd} and \emph{innumerable} multitudes\hfill(\href{https://archive.org/details/cainmystery01byro/page/24/mode/2up?q=%22innumerable+multitudes%22&view=theater}{Byron, \textit{Cain} 1.1})
\ex
Their faces, \emph{undistinguished} and \emph{indistinguishable} in the crowd\\\hfill(\href{https://archive.org/details/johnmarvelassist00page_0/page/174/mode/2up?view=theater&q=undistinguished}{Page, \textit{Marvel} 175})
\ex
the two great fragments we possess of Shakespeare’s \emph{uncompleted} work are \emph{incomplete} simply because the labour {\dots} was cut short by his timeless death\hfill(\href{https://archive.org/details/studyofshakespea0000swin/page/212/mode/2up?q=%22fragments+we+possess%22&view=theater}{Swinburne, \textit{Shakespeare} 212}) % "two great" restored
\ex
\emph{unmitigated} and \emph{immitigable}\hfill(\href{https://archive.org/details/newgrubstreetnov01gissuoft/page/184/mode/2up?q=%22unmitigated%22&view=theater}{Gissing, \textit{Grub} 90})
\ex
after an \emph{unexplained}, but not \emph{inexplicable} delay\hfill(news 1917)
\z
\z
\is{lexical change|)}

\is{productivity of prefixes}
It should also be noted that while most of the \textit{in-} words are settled once for all, and have to be learned by children as wholes, there is always a possibility of forming new words on the spur of the moment with the prefix \textit{un-}, see, for instance the contrast in (\ref{ex:13-07}).

\ea \label{ex:13-07}the \emph{irresponsible} and \emph{unresponsive} powers\hfill(\href{https://archive.org/details/nojohnstreet00whitgoog/page/n307/mode/2up?q=%22irresponsible+and+unresponsive%22&view=theater}{Whiteing, \textit{Five} 267})
\z

Hence also the difference between \textit{unavoidable} from the existing verb \textit{avoid}, and \textit{inevitable}: there is no English verb \textit{evite}. % ??? PE: I've removed the asterisk that had preceded "evite": OJ doesn't seem to use an asterisk to mark any example of either "nonexistent" or "ill-formed/ungrammatical". If we used it here, we'd really have to use it elsewhere too -- and here it's hardly necessary as OJ says that the putative word doesn't exist.

In other instances we find \textit{un-} alternating with some other prefix in related words:

\bigskip

\begin{tabular}{ll}
 \textit{unfortunate}& \textit{misfortune}\\
 \textit{unsatisfactory}& \textit{dissatisfaction}\\
 \textit{uncomfortable}& \textit{discomfort}\\
\end{tabular}

\bigskip

\is{lexical change|(}
In a great many cases, the prefix \textit{un-} was formerly used, either alone or concurrently with \textit{in-}, where now the latter is exclusively used. Examples are:

\bigskip
%\emergencystretch=3em
\setlength{\leftskip}{1.55em}\noindent
    \textit{unactive} (\href{https://internetshakespeare.uvic.ca/doc/Cor_F1/index.html#tln-100}{Shakespeare}, \href{https://archive.org/details/poeticalworksofj00miltiala/page/466/mode/2up?ref=ol&view=theater&q=unactive}{Milton}), \textit{uncapable} (\href{https://internetshakespeare.uvic.ca/doc/Oth_F1/scene/4.2/index.html#tln-2945}{Shakespeare}, \href{https://archive.org/details/fartheradventure00defo/page/180/mode/2up?q=%22uncapable%22&view=theater}{Defoe}, \href{https://archive.org/details/bim_eighteenth-century_the-works-of-j-s-dd-_swift-jonathan_1735_3/page/62/mode/2up?view=theater&q=%22uncapable%22}{J. Swift}, \href{https://archive.org/details/spectatornewedre00addiuoft/page/294/mode/2up?q=%22uncapable%22&view=theater}{\textit{Spectator}}), \textit{unconstant} (\href{https://internetshakespeare.uvic.ca/doc/Lr_F1/scene/1.1/index.html#tln-325}{Shakespeare}, \href{https://archive.org/details/bim_early-english-books-1475-1640_campaspe-played-beefore-_lyly-john_1591/page/14/mode/2up?q=vnconstant&view=theater}{Lyly}), \textit{uncredible} (\href{https://archive.org/details/utopiasirthomas00robigoog/page/n285/mode/2up?q=vncredible&view=theater}{More}), \textit{uncurable} (\href{https://archive.org/details/utopiasirthomas00robigoog/page/n333/mode/2up?q=vncurable&view=theater}{More}, \href{https://internetshakespeare.uvic.ca/doc/2H6_F1/scene/3.1/index.html#tln-1585}{Shakespeare}), \textit{undecent} (\href{https://archive.org/details/bim_early-english-books-1475-1640_campaspe-played-beefore-_lyly-john_1591/page/n31/mode/2up?q=vndecent&view=theater}{Lyly}), \textit{undocile} (\href{https://ia800900.us.archive.org/29/items/compleatenglishg00deforich/compleatenglishg00deforich.pdf}{Defoe}), \textit{unhonest} (\href{https://archive.org/details/utopiasirthomas00robigoog/page/n203/mode/2up?q=vnhonest&view=theater}{More}), \textit{unmeasurable} (\href{https://internetshakespeare.uvic.ca/doc/Wiv_F1/scene/2.1/index.html#tln-645}{Shakespeare}), \textit{unnoble} (\href{https://archive.org/details/bim_early-english-books-1475-1640_campaspe-played-beefore-_lyly-john_1591/page/n15/mode/2up?q=vnnoble&view=theater}{Lyly}, \href{https://internetshakespeare.uvic.ca/doc/Ant_F1/scene/3.11/index.html#tln-2075}{Shakespeare}, \href{https://archive.org/details/b30527892_0005/page/2776/mode/2up?q=unnoble&view=theater}{Fletcher}), \textit{unnumerable} (\href{https://archive.org/details/utopiasirthomas00robigoog/page/n159/mode/2up?q=vnnumerable&view=theater}{More}), \textit{unperfect} (\href{https://internetshakespeare.uvic.ca/doc/Son_Q1/page/14/index.html#tln-330}{Shakespeare}, \href{https://www.kingjamesbibleonline.org/Psalms-139-16/}{\textit{AV}}), \textit{unplausible} (\href{https://archive.org/details/areopagitica00miltuoft/page/54/mode/2up?q=%22unplausible%22&view=theater}{Milton}), \textit{unpossible} (\href{https://archive.org/details/bim_early-english-books-1475-1640_campaspe-played-beefore-_lyly-john_1591/page/III/mode/2up?q=vnpossible&view=theater}{Lyly}, \href{https://internetshakespeare.uvic.ca/doc/R2_Q1/scene/2.2/index.html#tln-1075}{Shakespeare}, \href{https://archive.org/details/authorizedversio05wrig/page/124/mode/2up?q=vnpossible&view=theater}{\textit{AV}}, \href{https://archive.org/details/shestoopstoconqu03gold/page/30/mode/2up?q=%22parfectly%22&view=theater}{Goldsmith} (vulgar)), \textit{unproper} (\href{https://internetshakespeare.uvic.ca/doc/Oth_F1/scene/4.1/index.html#tln-2445}{Shakespeare}), \textit{unsatiable} (\href{https://archive.org/details/utopiasirthomas00robigoog/page/n161/mode/2up?q=vnsatiable&view=theater}{More}), \textit{unsatiate} (\href{https://internetshakespeare.uvic.ca/doc/R3_Q1/scene/3.5/index.html#tln-2170}{Shakespeare}), \textit{unsufferable} (\href{https://archive.org/details/journalofplaguey1881defo/page/296/mode/2up?view=theater&q=%22unsufferable%22}{Defoe}), \textit{unsufficient} (\href{https://archive.org/details/utopiasirthomas00robigoog/page/n205/mode/2up?q=vnsufficient&view=theater}{More}), \textit{untractable} (\href{https://ia800900.us.archive.org/29/items/compleatenglishg00deforich/compleatenglishg00deforich.pdf}{Defoe})\footnote{Jespersen specifies the authors who use these---as well as the words prefixed \textit{in-} in the following list---but not where the uses occur. Our choices are arbitrary. \eds}\\ % At the foot of page 84; normal search won't work as the word is printed with a ct ligature

\setlength{\leftskip}{0em}


Many of these, and similar \textit{un-} words, are still in use in dialects, see \citet{wright1905english6} and \citet[\href{https://www.gutenberg.org/cache/epub/47364/pg47364-images.html\#Page_31}{31}]{wright1913rustic}.
\is{lexical change|)}

Words in which \textit{in-} was formerly used while \textit{un-} is now recognized:

\bigskip
%\emergencystretch=3em
\setlength{\leftskip}{1.55em}\noindent \textit{incertain} (\href{https://internetshakespeare.uvic.ca/doc/WT_F1/scene/5.1/index.html#tln-2760}{Shakespeare}), 
 \textit{incharitable} (\href{https://internetshakespeare.uvic.ca/doc/Tmp_F1/scene/index.html#tln-50}{Shakespeare}), 
 \textit{inchaste} (\href{https://archive.org/details/worksofgeorgepee02peel/page/52/mode/2up?q=inchaste&view=theater}{Peele}), 
 \textit{infortunate} (\href{https://archive.org/details/cu31924013131614/page/36/mode/2up?q=infortunate&view=theater}{Kyd}, \href{https://internetshakespeare.uvic.ca/doc/Jn_F1/scene/2.1/index.html#tln-480}{Shakespeare}), 
 \textit{ingrateful} (\href{https://internetshakespeare.uvic.ca/doc/Jn_F1/scene/5.7/index.html#tln-2650}{Shakespeare}, \href{https://archive.org/details/poeticalworksofj00miltiala/page/524/mode/2up?ref=ol&view=theater&q=%22ingrateful%22}{Milton}), 
 \textit{insubstantial} (\href{https://internetshakespeare.uvic.ca/doc/Tmp_F1/scene/4.1/index.html#tln-1825}{Shakespeare})

\bigskip
\setlength{\leftskip}{0em}

(It is not, of course, pretended that these words occur only in the authors named; in most cases, it would be very easy to find examples in other writers as well.)

Both \textit{unfrequent} and \textit{infrequent} are in use, the latter, for instance in (\ref{ex:13-infrequent}). \textit{Unelegant} and \textit{unfirm} are rarer than \textit{inelegant} and \textit{infirm}.

\ea \label{ex:13-infrequent}
in not infrequent communication \hfill (\href{https://archive.org/details/greywigstoriesno00zang/page/274/mode/2up?q=%22not+infrequent%22&view=theater}{Zangwill, \textit{Mystery} 199})
\z

The distinction now made between \textit{human} and \textit{humane} is recent; \textit{inhuman} has the meaning corresponding to \textit{humane}, while the negative of \textit{human} is generally expressed by \textit{non-human}, rarely as in (\ref{ex:13-08}). 

\ea \label{ex:13-08} he was so unaffectedly \textit{unhuman} that he did not recognise the human intention and essence of that teaching\hfill(\href{https://archive.org/details/familiarstudies00stevuoft/page/168/mode/2up?view=theater&q=%22unaffectedly+unhuman%22}{Stevenson, \textit{Men} 166}) % Restored "and essence". The linked-to original has "recognize"; but it was published by Scribner's in NYC so this was probably just the conventional practice by which a US publisher would make a British text palatable for a US readership.
\z

Corresponding to \textit{apt} we have the Latin and French \textit{inept} with change of vowel and of meaning (`foolish') and the English formation \textit{unapt}; the corresponding substantives are \textit{ineptitude} and \textit{unaptness}, rarely as in (\ref{ex:13-09})---evidently with a sly innuendo of the other word.

\ea \label{ex:13-09}
women {\dots} their \textit{inaptitude} for reasoning\hfill(\href{https://archive.org/details/quintessenceofib00shawrich/page/10/mode/2up?q=inaptitude&view=theater}{Shaw, \textit{Ibsenism} 10})
\z

\textit{Inutterable} was in use in the 17th century (\href{https://archive.org/details/poeticalworksofj00miltiala/page/216/mode/2up?ref=ol&view=theater&q=%22abominable%22}{Milton}, etc.), but has been superseded by \textit{unutterable}; it has been revived, however, in one instance by Tennyson (\ref{ex:13-10}), no doubt to avoid two successive words beginning with \textit{un-}.

\ea \label{ex:13-10}
killed with inutterable unkindliness\hfill(\href{https://en.wikisource.org/wiki/Idylls_of_the_King/Merlin_and_Vivien}{\textit{Merlin}})
\z

Words beginning with \textit{in-} or \textit{im-} do not admit of the prefix \textit{in-}; hence \textit{un-} even in long and learned words like \textit{unimportant}, \textit{unintelligible}, \textit{unintentional}, \textit{uninterrupted}, etc. \textit{Unimmortal} (\href{https://archive.org/details/poeticalworksofj00miltiala/page/396/mode/2up?ref=ol&view=theater&q=unimmortal}{Milton, \textit{Lost} 10.611}) is rare. Note also \textit{disingenuous} (e.g. \href{https://archive.org/details/in.ernet.dli.2015.54181/page/n265/mode/2up?q=disingenuous&view=theater}{Shelley, \textit{Letters} 729}).

\bigskip
\is{homonyms and homographs|(}
\is{ambiguity!arising from prefix homophony/homography}
It is sometimes felt as an inconvenience that the negative prefix is identical in form with the (Latin) preposition \textit{in}. The verb \textit{inhabit} contains the latter; but \textit{inhabitable} is sometimes used with negative import, thus in (\ref{ex:13-11}). The ambiguity of this form leads to the use of two forms with \textit{un-}, a rarer one as in (\ref{ex:13-13}) (but the form \textit{inhabited} is used in the positive sense (\ref{ex:13-14})); and the more usual \textit{uninhabitable}, which is found in (\ref{ex:13-15}) and has now completely prevailed. The corresponding positive adjective (`what can be \textit{inhabited}') is \textit{habitable}. Ambiguities are also found in other similar adjectives, as seen by definitions in dictionaries: \textit{investigable} (1) that may be investigated, (2) incapable of being investigated; \textit{infusible} (1) that may be infused or poured in, (2) incapable of being fused or melted; \textit{invertible} (1) capable of being inverted, (2) incapable of being changed. \textit{Importable}, which is now used only as derived from \textit{import} (`capable of being imported') had formerly also the meaning `unbearable', and \textit{improvable} similarly had the meaning of `incapable of being proved', though it only retains that of `capable of being improved'. \textit{Inexistence} means (1) the condition of existing in something, and (2), rarely, the condition of not existing. Cf. \citet[\S140]{jespersen1905growth} %\href{https://archive.org/details/dli.csl.6897/page/n143/mode/2up?q=140&view=theater}{\textit{Growth and Structure of the English Language} §140}
for a few more examples.

\ea \label{ex:13-11}
\ea
In Ynde and abouten Ynde, ben mo than 5000 iles, gode and grete, that men duellen in, with outen tho that ben inhabitable\\`In Ind and about Ind, are more than 5000 isles, good and great, that men dwell in, excepting those that are not habitable'\hfill(\href{https://archive.org/details/b2935142x/page/160/mode/2up?view=theater&q=%22Ynde+and+abouten+Ynde%22}{\textit{Maundevile} 161})
\ex
Euen to the frozen ridges of the Alpes, Or any other ground inhabitable\hfill(\href{https://internetshakespeare.uvic.ca/doc/R2_F1/scene/index.html#tln-65}{Shakespeare, \textit{R2} 1.1.65})
\z
\z

\ea \label{ex:13-13}
the \textit{unhabitable} part of the world\hfill(\href{https://archive.org/details/lifeandstranges00dobsgoog/page/n181/mode/2up?q=%22unhabitable%22&view=theater}{Defoe, \textit{Robinson} 156})
\z

\ea \label{ex:13-14}
the island was \textit{inhabited}\hfill(\href{https://archive.org/details/lifeandstranges00dobsgoog/page/n213/mode/2up?q=%22was+inhabited%22&view=theater}{ibid 188})
\z 


\ea \label{ex:13-15}
Though this island seeme to be desert. {\dots} \textit{Vninhabitable}, and almost inaccessible\hfill(\href{https://internetshakespeare.uvic.ca/doc/Tmp_F1/scene/2.1/index.html#tln-710}{Shakespeare, \textit{Tp} 2.1.37})
\z

With regard to the employment and meaning of these two prefixes it is, first, important to note that their proper sphere is with adjectives and adverbs. They are found frequently with substantives, but exclusively with such as are derived from adjectives, e.g. \textit{unkindness}, \textit{injustice}, \textit{unimportance}, \textit{incomprehensibility}. Similarly \textit{unemployment}, which does not mean the same as \textit{non-employment}, % ??? PE: Yes it can mean this; or anyway it can in 2024 (offhand I'm not sure about 1917). Consider for example: "Long-term unemployment refers to an ongoing spell of unemployment that has lasted 27 continuous weeks or more." (US Bureau of Labor Statistics, "Labor Force Statistics from the Current Population Survey", https://www.bls.gov/cps/definitions.htm ). Currently lacking access to oed.com, I can't see how the OED dates this use of the word.
but refers to the number of unemployed. Cf. also the rare \textit{unproportion}, from \textit{proportionate}, in (\ref{ex:13-16}). \textit{Unfriend} (frequent in Scotch) % Does OJ use this here to mean "Scottish"?
%Brett: I've changed Sc. to Scotch
also smacks of \textit{unfriendly}; it is found in (\ref{ex:13-17}). Carlyle's ``Thinkers and \textit{unthinkers}'' (\href{https://archive.org/details/gri_33125008092856/page/n109/mode/2up?q=%22thinkers+and+unthinkers%22&view=theater}{\textit{Revolution} 107}) is a nonce-word. % Peter: Shall we either delete "Thinkers and" or change "is" to "uses" or similar?
%Brett: deleted "Thinkers and"

\ea \label{ex:13-16}
the wide unproportion between this slender company, and the boundless plain of sand\hfill(\href{https://www.gutenberg.org/files/43684/43684-h/43684-h.htm#page193}{Kinglake, \textit{Eothen} 178})
\z 

\ea \label{ex:13-17}
\ea
They were unfriends of mine\hfill(\href{https://archive.org/details/kimkipling01kipluoft/page/238/mode/2up?q=unfriends&view=theater}{Kipling, \textit{Kim} 202})
\ex
not distinguishing friend from unfriend\hfill(\href{https://archive.org/details/queensquairorsi00hewlgoog/page/30/mode/2up?q=%22distinguishing+friend%22&view=theater}{Hewlett, \textit{Quair} 30}) 
\z
\z


The negative prefixes \textit{un-} and \textit{in-} are not used with verbs, though \textit{un-} is very frequent with participles, because these are adjectival: \textit{undying}, \textit{unfinished}. (\textit{In-} with Latin participles, which in English are simply adjectives: \textit{inefficient}, \textit{imperfect}.) On the privative \textit{un-} with verbs see below p.~\pageref{privative_un}.

\is{prefixes!and depreciatory stems, negative}
Not all adjectives admit of having the negative prefix \textit{un-} or \textit{in-}, and it is not always easy to assign a reason why one adjective can take the prefix and another cannot. Still, the same general rule obtains in English as in other languages, that most adjectives with \textit{un-} or \textit{in-} have a depreciatory sense: we have \textit{unworthy}, \textit{undue}, \textit{imperfect}, etc., but it is not possible to form similar adjectives from \textit{wicked}, \textit{foolish}, or \textit{terrible}. \ia{van Ginneken, Jacobus}Van Ginneken (\citeyear[208 \S240]{vanginneken1907principes}) %\href{https://archive.org/details/principesdelingu00ginn/page/208/mode/2up?view=theater}{Van Ginneken (\textit{Principes de linguistique psychologique} 208)} ??? Peter: I understand that van/Van Ginneken must be listed as lowercase "van Ginneken". But this has brought about a bizarre result here. Is there some tweak that can be made to the bibliography entry, such that "van" is capitalized where even a common noun would be capitalized?
counted the words in 
\il{German!un-@\textit{un-}}\textit{un-} in a German dictionary and found that 98\% of the substantives and 85\% of the adjectives had ``une signification défavorable'' (`an unfavorable meaning'). \citet[\S567]{noreen1904vartsprak} 
%\href{https://archive.org/details/vrtsprknysvensk04noregoog/page/n593/mode/2up?q=567&view=theater}{Noreen (\textit{Vårt språk} 5.567)} 
found similar relations obtaining in 
\il{Swedish!\textit{un-}}Swedish.

\bigskip

\is{prefixes!negative with contrary sense, negative|(}
\is{prefixes!negative with non\hyp compositional results}
The modification in sense brought about by the addition of the prefix is generally that of a simple negative \textit{unworthy} = `not worthy', etc. The two terms are thus contradictory terms. But very often the prefix produces a ``contrary'' term or at any rate what approaches one: \textit{unjust} (and \textit{injustice}) generally imply the opposite of \textit{just} (\textit{justice}); \textit{unwise} means more than \textit{not wise} and approaches \textit{foolish}, \textit{unhappy} is not far from \textit{miserable}, etc. Still, in most cases we have only approximation, and \textit{unbeautiful} (which is not very common, but is used, for instance, by Carlyle, \href{https://archive.org/details/reminiscences0000thom_e9a0/page/92/mode/2up?q=%22unbeautiful%22&view=theater}{\textit{Reminiscences} 1.118}, Swinburne, \href{https://archive.org/details/lovescrosscurren00swinuoft/page/170/mode/2up?q=unbeautiful&view=theater}{\textit{Cross-currents} 187}, \href{https://archive.org/details/greywigstoriesno00zang/page/238/mode/2up?q=%22unbeautiful%22&view=theater}{Zangwill}, and others) is not so strong as \textit{ugly} or \textit{hideous}. Sometimes the use of the negative is restricted: \textit{unwell} refers only to health, and we could not speak of a book as \textit{unwell printed} (for \textit{badly}).\footnote{\textit{Unwell} has only ever been an adjective; this rules out \textit{unwell printed}. \eds} \textit{Unfair} is only used in the moral sense, not of outward looks.

\label{amoral}While \textit{immoral} means the opposite of \textit{moral}, i.e., what is contrary to (the received ideas of) morality, the necessity is sometimes felt of a term implying `having nothing to do with morality, standing outside the sphere of morality'; this is sometimes expressed by \textit{amoral} (thus frequently by the late ethnologist A. H. Keane), % https://en.wikipedia.org/wiki/Augustus_Henry_Keane suggests that one might reasonably refer to him as "late and unlamented"
sometimes by \textit{unmoral} (\ref{ex:13-19}). Cf. from French (\ref{ex:13-22}).

\ea \label{ex:13-19}
\ea There is a vast deal in life and letters both which is not immoral, but simply a-moral\hfill(\href{https://archive.org/details/memoriesportrait00sev/page/232/mode/2up?q=%22vast+deal+in%22&view=theater}{Stevenson, \textit{Memories}} (\href{https://archive.org/details/in.ernet.dli.2015.211152/page/n317/mode/2up?view=theater}{\textit{NED}, \textit{Amoral}})) 
\ex They [children] are naturally neither moral nor immoral---but merely unmoral. They are little savages, living in a civilized society that has not yet civilized them\hfill (\href{https://archive.org/details/beas00lind/page/134/mode/2up?q=%22naturally+neither+moral+nor+immoral%22&view=theater}{Lindsey, \textit{Beast} 779}) % Comma into dash
\ex the universe was unmoral and without concern for men\\\hfill(\href{https://www.gutenberg.org/files/1449/1449-h/1449-h.htm#link2HCH0031}{London, \textit{Valley} 255})%\footnote{Other editions have \textit{the universe was unmoral, that there was no God, no immortality}. \eds}
\z
\z

\ea \label{ex:13-22}
\gll Moralité, \emph{immoralité,} \emph{amoralité} --- tous ces mots ne veulent rien dire.\\
 morality immorality amorality {} all these words not want nothing say\\
\glt `Morality, immorality, amorality --- none of these words means anything.'
\\\hfill(\href{https://www.gutenberg.org/cache/epub/61876/pg61876-images.html}{Rolland, \textit{Foire} 130})
\z


As \textit{irreligious} is very often used as the opposite of \textit{religious}, Carlyle in one passage avoids this word, in speaking of University College, London, (\ref{ex:13-23}).

\ea \label{ex:13-23} ``it will be \emph{unreligious}, secretly \emph{anti-religious} all the same,'' said Irving to us\hfill(\href{https://archive.org/details/reminiscences0000thom_e9a0/page/232/mode/2up?q=%22will+be+unreligious%22&view=theater}{\textit{Reminiscences} 1.293})\footnote{The speaker is the Scottish clergyman Edward Irving (1792--1834). \eds} % OJ's "me" corrected to the "us" of Reminiscences; "anti-religious" re-hyphenated
\z

\is{lexical change|(}
\label{meaning_change}\textit{Infamous} has been separated from \textit{famous} as in sound (cf. p.~\pageref{sound_change}), so in sense; the negative of \textit{famous} is now rather \textit{unfamed}.

Other examples, in which the word with the negative prefix has been separated in sense from the simplex, are

\bigskip

\begin{tabular}{@{}ll@{}}
 \textit{different} & \textit{indifferent}\\
\textit{pertinent} & \textit{impertinent}
\end{tabular}

\bigskip

\textit{Invaluable} means `priceless', `very valuable' while the negative of \textit{valuable} is \textit{worthless}.
\is{lexical change|)}

\bigskip
\textit{Un-} (rarely \textit{in-}) may be prefixed to participial groups: \textit{unheard-of}, \textit{uncalled-for}, \textit{uncared-for} (\ref{ex:13-24}).

\ea \label{ex:13-24} the offer of the 872 moidores, which was \emph{indisposed of}\\\hfill(\href{https://archive.org/details/lifeandstranges00dobsgoog/page/n367/mode/2up?q=%22872%22&view=theater}{Defoe, \textit{Robinson} 341}) % "the offer of" restored
\z

To the same category may be referred (\ref{ex:13-25}).

\ea \label{ex:13-25}
\ea that the time was out of joint and life \emph{unworth living}\\\hfill(\href{https://archive.org/details/cu31924013586940/page/516/mode/2up?q=%22time+was+out+of+joint%22&view=theater}{Bennett, \textit{Wives} 2.235})
\ex were a generation of infants to grow up \emph{untaught to speak}\\\hfill(\href{https://archive.org/details/in.ernet.dli.2015.49512/page/n301/mode/2up?view=theater&q=%22untaught+to+speak%22}{Whitney, \textit{Studies} 1.286})
\ex you haue very rare, and \emph{vn-in-one-breath-vtter-able} skill\\\hfill(\href{https://archive.org/details/cu31924013130723/page/n89/mode/2up?q=%22very+rare%22&view=theater}{Jonson, \textit{Humour} 1.5}) % Orthographic touch-up
\z
\z

There is an interesting Scotch way of using the negative prefix 
\il{Scottish!\textit{on}}\textit{on-} (`un-') before participles, as in (\ref{ex:13-28}). This is sometimes mistakenly written \textit{ohn}, as if from 
\il{German!ohne@\textit{ohne}}German \textit{ohne}: (\ref{ex:13-29}).

\ea \label{ex:13-28}
I'm nae responsible to gae afore Sir Simon on-hed my papers upo' me \phantom{x} (`without having')\hfill(\href{https://archive.org/details/johnnygibbofgush00alex_0/page/234/mode/2up?q=%22nae+responsible%22&view=theater}{Alexander, \textit{Johnny} 235})
\z

\ea \label{ex:13-29}
ohn been ashamed
\hfill\citep[364 \textsc{on-}, \textit{pref.}]{wright1905english4}
%(\href{https://archive.org/details/englishdialectdi04wrig/page/346/mode/2up?view=theater&q=%22also+written+ohn%22}{\textit{English Dialect Dictionary}, \textsc{ON-}, \textit{pref.}})
\z

Instead of prefixing \textit{un-} to adjectives in \textit{-ful} it is usual to substitute \textit{-less} for \textit{-ful}, thus \textit{careless} corresponding to \textit{careful}, \textit{thoughtless}, \textit{hopeless}, \textit{useless}; but \textit{unfaithful}, \textit{unmerciful} are used by the side of \textit{faithless}, \textit{merciless}; \textit{unlawful} does not mean the same as \textit{lawless}; \textit{uneventful} and \textit{unsuccessful} are preferred to \textit{eventless} and \textit{successless}; \textit{unbeautiful} is used, but there is no \textit{beautiless}.
\is{prefixes!negative with contrary sense|)}
\il{English!un-@\textit{un-}|)}
\il{English!in-@\textit{in-}|)}

\addsec{\textit{Dis-}}
\il{English!dis-@\textit{dis-}|(}


\is{privative prefixes}
\is{prefixes!privative}
The prefix \textit{dis-} (from 
\il{Latin!dis-@\textit{dis-}}Latin) besides various other meanings also has that of a pure negative, as in \textit{dissimilar}, \textit{dishonest}, \textit{dispassionate}, \textit{disagree} (\textit{-able}), \textit{disuse}, \textit{dislike}, \textit{disbelieve} generally implying contrary rather than contradictory opposition, as is seen very distinctly in \textit{dissuade}, \textit{disadvise} (\ref{ex:13-30}), \textit{disreputable}, etc. Sometimes the prefix has the same privative meaning as \textit{un-} before verbs (see p.~\pageref{privative_un}), as in \textit{disburden}, \textit{disembarrass} (\ref{ex:13-31}); \textit{discover} has been specialized and differentiated from \textit{uncover}.

\ea \label{ex:13-30}
he \emph{disadvised} you from it\hfill(\href{https://archive.org/details/warden0000anth_w6p5/page/218/mode/2up?q=%22disadvised+you%22&view=theater}{Trollope, \textit{Warden} 231})
\z

\ea\label{ex:13-31}
\emph{diswhipped} Taskmaster\hfill(nonce-word, \href{https://archive.org/details/gri_33125008092856/page/n247/mode/2up?q=%22diswhipped&view=theater}{T. Carlyle, \textit{Revolution} 268})
\z

A difference is made between \textit{dis-} and \textit{un-} in (\ref{ex:13-32}), the former referring to egoism, the latter to more ideal motives. (In Ido the two would be \textit{sen-interesta ma ne sen-interesa}.)

\ea \label{ex:13-32} The entrance of a fresh and powerful neutral [U. S.], honestly \emph{disinterested} but not \emph{uninterested}\hfill(American news 1916)
\z

\is{ambiguity!arising from order of affixation}
As with \textit{in-} we have sometimes here a linguistic drawback arising from the ambiguity of the prefix. \textit{Dissociable} may be either the negative of \textit{sociable} (`unsociable') or derived from the verb \textit{dissociate} (`separable'); in the former case 
the \href{https://archive.org/details/oed03arch/page/n527/mode/2up?view=theater}{\textit{NED}} will pronounce a double [s], while Mr. Daniel Jones (\citeyear[136]{jones1917everyman}) %(\href{https://archive.org/details/in.ernet.dli.2015.93283/page/n183/mode/2up?view=theater&q=dissociable}{\textit{Pronouncing Dictionary}})
has single [s] in both, but pronounces the ending in the former [-ʃəbl], in the latter [ʃiəbl] or [ʃjəbl].% PE: To my surprise, OED3 does gives the meaning "separable", doesn't label it as rare (let alone as obsolete), and distinguishes its pronunciation from that of its homograph.

\textit{Disannul} means practically the same thing as \textit{annul} and thus contains a redundant negative (cf. Spanish \textit{desnudar}).
\il{English!dis-@\textit{dis-}|)}

\addsec{\textit{Non-}}
\il{English!non-@\textit{non-}|(}

A great many words (substantives, not so often adjectives) are formed with the 
\il{Latin!non-@\textit{non-}}Latin \textit{non-}, especially in those cases where no formations with \textit{un-} or \textit{in-} are available. Juridical terms are probably responsible for the extent to which this prefix has been made use of. Shakespeare has \textit{nonage}, \textit{non-payment}, \textit{non-performance}, \textit{non-regardance}, and \textit{non-suit}. It will be seen that \textit{non-} is chiefly used with action-nouns; but it is also frequent with agent-nouns, such as \textit{non-combatant}, \textit{non-belligerent}, \textit{non-communicant}, \textit{non-conductor}, cf. also \textit{non-conducting}, \textit{non-member}. See also (\ref{ex:13-33}).

\ea \label{ex:13-33}
\ea the \emph{non-arrival} of her own carriage\hfill(\href{https://archive.org/details/lifeadventuresofdickrich/page/60/mode/2up?q=%22non-arrival%22&view=theater}{Dickens, \textit{Nicholas} 50})
\ex in a \emph{non-natural} way\hfill(\href{https://archive.org/details/in.ernet.dli.2015.351554/page/n307/mode/2up?q=%22non-natural+way%22&view=theater}{Wells, \textit{Anticipations} 303})
\ex this tangled, \emph{nonunderstandable} conflict\hfill(\href{https://archive.org/details/valleyofmoon00londrich/page/n207/mode/2up?view=theater&q=%22this+tangled%22}{London, \textit{Valley} 199})
\ex their \emph{non-importation} resolutions\hfill(\href{https://archive.org/details/benfrankautobio00franrich/page/244/mode/2up?q=%22non-importation%22&view=theater}{MacDonald, \textit{Account} 245})
\ex the United States was born \emph{non-viable}\hfill(\href{https://archive.org/details/benfrankautobio00franrich/page/308/mode/2up?q=%22born+non-viable%22&view=theater}{ibid 309})
\ex a \emph{non-stopping} train
\z
\z
\il{English!non-@\textit{non-}|)}

\addsec{\textit{An-}, \textit{a-}}\label{prefix_a}
\il{English!an-@\textit{an-}|(}
\il{English!a-@\textit{a-}|(}

\il{English!a-/an- prefix@\textit{a-}/\textit{an-} prefix (Greek)}The Greek prefix \textit{an-} before a vowel, \textit{a-} before a consonant, etymologically identical with \textit{un-} and \textit{in-} (see p.~\pageref{negative_un_in}), is chiefly found in Greek words like \textit{anarchy}, \textit{amorphous}, \textit{achromatic}, but is also in rare instances used in English to form new words (from Latin roots), such as \textit{amoral} (above p.~\pageref{amoral}), \textit{asexual} in (\ref{ex:13-39}).

\ea \label{ex:13-39} the truly emancipated woman {\dots} is almost asexual\hfill(\href{https://archive.org/details/borninexileanov02gissgoog/page/246/mode/2up?q=%22emancipated+woman%22&view=theater}{Gissing, \textit{Born} 267}) % Adding dots to show an important excision by OJ
\z
\il{English!an-@\textit{an-}|)}
\il{English!a-@\textit{a-}|)}

\addsec{\textit{No-}}
\il{English!no-@\textit{no-}|(}

\is{pronouns, negative}
\textit{No} (the pronoun) is sometimes used as a kind of prefix, as illustrated in \citet[430 \S16.79]{jespersenMEG2} 
%\href{https://archive.org/details/jespersen-1954-a-modern-english-grammar-on-historical-principles-part-ii-syntax-first-volume/page/430/mode/2up?view=theater&q=%2216%2C79%22}{\textit{Modern English Grammar} 2.16.79}
with examples like \textit{no-education}, \textit{no-thoroughfare}, \textit{no-ball}, etc. Cf. also (\ref{ex:13-40}).

\ea \label{ex:13-40}
\ea must himself, with such \emph{no-faculty} as he has, begin governing\\\hfill(\href{https://archive.org/details/gri_33125008092856/page/n67/mode/2up?q=%22such+no-faculty%22&view=theater}{T. Carlyle, \textit{Revolution} 57}) % Restored context fore and aft
\ex The Constitution which will suit that? Alas, too clearly, a \emph{No-Constitution}, an Anarchy\hfill(\href{https://archive.org/details/gri_33125008092856/page/n185/mode/2up?q=%22constitution+which+will%22&view=theater}{ibid 199})
\ex there can be no settlement which is not a world-settlement. Even the \emph{no-settlement} which a stalemate would involve would be an \emph{unsettlement} of the whole world\\\hfill(The latter to the following prefix; Pollard, \textit{Prevention} 313)
\z
\z
\il{English!no-@\textit{no-}|)}

\addsec{The privative \textit{un-}}\label{privative_un}
\is{privative prefixes|(}
\is{prefixes!privative|(}
\il{English!un-@\textit{un-}|(}

\il{English!Old English!ond-@\textit{ond-}}
\il{English!Old English!and-@\textit{and-}}
\il{Greek!\textit{anti-}}
\il{German!ent-@\textit{ent-}}
Old English had the prefix \textit{ond-}, \textit{and-}, which was liable to lose its \textit{d} before a consonant; it corresponds etymologically to Greek \textit{anti-} and German \textit{ent-}. In \textit{answer} it is no longer felt as a prefix; and in \textit{dread} the only thing left of the prefix is \textit{d}: Old English \textit{ondrædan} (`advise against'), cf. German \textit{entraten} (`forego'), was felt as containing the preposition \textit{on}, and when that was subtracted, \textit{drædan} (`dread') remained \citep[182]{pogatscher1903fieldfare}.
%(\href{https://archive.org/details/angliazeitschrif14halluoft/page/182/mode/2up?view=theater&q=%22Altengl.+ondrcedan+%27f%C3%BCrchten%27%22}{Pogatscher, \textit{Etymologisches} 182}).

\is{productivity of prefixes}
\is{stress}
In other instances the prefix remained living, but the vowel was changed into \textit{u}, probably through influence from the negative prefix, (cf. \textit{unless}, 
\il{English!Middle English!on lesse@\textit{on lesse}}Middle English \textit{on lesse} (\textit{that}), where also the negative notion caused confusion with \textit{un-}). Thus the old 
\il{English!Middle English!on-@\textit{on-}}\textit{onbindan}, \textit{ontiegan} became \textit{unbindan}, \textit{untigan} in Ælfric, modern \textit{unbind}, \textit{untie}. The two prefixes are now different through stress, the negative words having even and the privative end stress. The privative \textit{un-} serves to make verbs, such as \textit{uncover} (`deprive of cover'), \textit{untie} (`loose'), % Peter: I was about to suggest correcting this to "loosen"; but the OED (1st ed) does have "loose" as a verb.
\textit{undress} (`take off dress'), \textit{undo} (`reverse what has been done, annul, untie'), \textit{unmask}, etc., also for instance \mbox{\textit{unman}} (`deprive of the qualities of a man'), \textit{unking} (`dethrone', Shakespeare), \textit{unlord}.

The following quotations may serve to illustrate the freedom with which new verbs are formed with this prefix: (\ref{ex:13-43}).

\ea\label{ex:13-43}
\ea she treads the path that she \emph{vntreads} againe\hfill(\href{https://internetshakespeare.uvic.ca/doc/Ven_Q1/stanza/151~155/index.html#tln-905}{Shakespeare, \textit{Ven} 908})
\ex \emph{Vn-sweare} faith sworne\hfill(\href{https://internetshakespeare.uvic.ca/doc/Jn_F1/scene/3.1/index.html#tln-1175}{Shakespeare, \textit{John} 3.1.245}) % Restored the form "Vn-sweare"
\ex thou hast \emph{vnwished} fiue thousand men\hfill(\href{https://internetshakespeare.uvic.ca/doc/H5_F1/scene/4.3/index.html#tln-2320}{Shakespeare, \textit{H5} 4.3.76})
\ex Then who created thee lamenting learne, When who can \emph{uncreate} thee thou shalt know\hfill(\href{https://archive.org/details/poeticalworksofj00miltiala/page/294/mode/2up?ref=ol&view=theater&q=%22Then+who+created+thee+lamenting%22}{Milton, \textit{Lost} 5.895})
\ex {[}he{]} wishes, he could \emph{unbeget} Those rebel sons\\\hfill(\href{https://archive.org/details/aurengzebetraged00dryd_1/page/2/mode/2up?q=unbeget&view=theater}{Dryden, \textit{Aureng-Zebe} 1}) % Not "these" but "those"
\ex to say or to \emph{unsay}, Whate'er you please\hfill(\href{https://archive.org/details/allforloveorworl00indryd/page/62/mode/2up?q=%22unfay%22&view=theater}{Dryden, \textit{All} 4})
\ex they were, as it were, alarmed, and \emph{un-alarmed} again\\\hfill(\href{https://archive.org/details/journalofplaguey1881defo/page/24/mode/2up?view=theater&q=%22alarmed%22}{Defoe, \textit{Journal} 25}) % Restoring the hyphen to "un-alarmed"
\ex before the end of the year {\dots}, I shall have my wings \emph{unbirdlim'd}\\\hfill(\href{https://archive.org/details/collectedletters0001edea/page/586/mode/2up?q=%22before+the+end+of+the+year%22&view=theater}{Coleridge, letter}) % ??? OJ describes this as "Coleridge, Letter 1800 (Campb. LVIII." (and he spells the word "un-birdlimed" and cuts ", if God grant me health,"). The link goes to Earl Leslie Griggs, ed., Collected Letters of Samuel Taylor Coleridge, vol 1 (published in 1956). Griggs credits books by J. Dykes Campbell, but none of these has a title that suggests it's a collection of letters. Moreover, Griggs says of this letter (which he numbers 332) "Hitherto unpublished". Was OJ citing a collection in the making?
\ex do not poison all My peace left, by \emph{unwishing} that thou wert A father\\\hfill(\href{https://archive.org/details/in.ernet.dli.2015.285363/page/n467/mode/2up?q=%22do+not+poison%22&view=theater}{Byron, \textit{Sardanapalus} 4.1})
\ex Death quite \emph{unfellows} us\hfill(\href{https://archive.org/details/auroraleighpoem00brow/page/182/mode/2up?q=%22death+quite+unfellows+us%22&view=theater}{E. B. Browning, \textit{Aurora} 170})
\ex it makes and \emph{unmakes} whole worlds\hfill(\href{https://archive.org/details/sartorresartus02unkngoog/page/108/mode/2up?view=theater&q=%22makes+and+unmakes%22}{T. Carlyle, \textit{Sartor} 82})
\ex {[}she{]} \emph{Unhandkerchiefs} one eye\hfill(\href{https://archive.org/details/LifeOnTheMississippi1883MarkTwain/page/n443/mode/2up?q=unhandkerchiefs&view=theater}{Twain, \textit{Mississippi} 190})
\z
\z

While infinitives and other pure verb-forms beginning with \textit{un-} can only be privatives, participles with the same beginning may be either negatives or privatives, the written and printed forms being identical in the two cases. Thus \mbox{\textit{uncovered}} may be [ˈʌnˈkʌvəd] `not covered' and [ʌnˈkʌvəd] `deprived of cover' \textit{unlocked} [ˈʌnˈlɔkt] `not locked' and [ʌnˈlɔkt] `opened'; similarly \textit{untied}, \textit{undressed}, \textit{unstrapped}, \textit{unbuttoned}, \textit{unharnessed}, \textit{unbridled}, \textit{unloaded}, \textit{unpacked}, etc.\largerpage

In some cases, it may be doubtful whether we have one or the other prefix, e.g.: (\ref{ex:13-55}).

\ea\label{ex:13-55}
\ea those \emph{unsexed} intellectuals\hfill(\href{https://archive.org/details/annveronicamoder0000hgwe/page/128/mode/2up?q=%22unsexed%22&view=theater}{Wells, \textit{Veronica} 124})
\ex all sorts of clothing, made and \emph{unmade}\hfill(\href{https://archive.org/details/personalhistory05dickgoog/page/n57/mode/2up?q=%22sorts+of+clothing%22&view=theater}{Dickens, \textit{David} 117})
\ex {[}an anonymous book{]} has been by some attributed to me---at which I ought to be much flattered and \emph{unflattered}\hfill(\href{https://www.darwinproject.ac.uk/letter/DCP-LETT-859.xml}{Darwin, \textit{Life} 1.333}) % I haven't looked in Life and Letters, but this website has "&", not "and"
\ex Love or \emph{unlove} me, \emph{Unknow} me or know, I am that which unloves me and loves\hfill(\href{https://archive.org/details/cu31924013555911/page/n93/mode/2up?q=%22unknow+me%22&view=theater}{Swinburne, \textit{Songs} 83})
\z
\z
\noindent (I reckon here also Swinburne's \textit{unlove} and \textit{unknow}, though according to the ordinary rules these should be only privatives.) % Peter: I've moved this from before to after the bunch of quotations: this seems less clunky. Rather than "I reckon here also", can we have "I include" (or of course some better alternative?
%Brett: OK
% PE: I've changed my mind (slightly). Let's retain "I reckon here also". It's awkward but it's understandable; and "slippery slope" and all that.....
%Brett: Better.

The two prefixes are brought together neatly in (\ref{ex:13-59}).

\ea \label{ex:13-59} If charity covers a multitude of sins, \emph{uncharitableness} has the advantage of \emph{uncovering} them\hfill(\href{https://archive.org/details/septimus00unkngoog/page/n233/mode/2up?q=%22charity+covers%22&view=theater}{Locke, \textit{Septimus} 246})
\z

Shakespeare and the Authorized Version have the illogical verb \textit{unloose} with confusion of \textit{untie} and \textit{loose}(\textit{n}). % ??? PE: Should we point to examples in WS and AV of this?

\is{stress}
From the privative verb \textit{to undress} is formed the substantive \textit{undress} (stress on the first syllable, \cite[175 \S5.72]{jespersenMEG1}) %\href{https://archive.org/details/a-modern-english-grammar-on-historical-principles.-jespersen-otto-1860-1943-hais/page/174/mode/2up?q=undress&view=theater}{\textit{Modern English Grammar} 1.5.72})
meaning `plain clothes' (not uniform), e.g. (\ref{ex:13-60}).

\ea \label{ex:13-60} in a military undress\hfill(\href{https://archive.org/details/cewaverleynovels03scotuoft/page/174/mode/2up?view=theater&q=%22undress%22}{Scott, \textit{Antiquary} 1.298}) % Restored "a", which OJ has deleted
\z

NB. The rules here given for stress of the two kinds of formations are probably too absolute; as a matter of fact there is a good deal of vacillation. Mr. Daniel Jones (\citeyear[488]{jones1917everyman}) % \href{https://archive.org/details/in.ernet.dli.2015.93283/page/n535/mode/2up?q=%22un-+An-%22&view=theater}{\textit{English Pronouncing Dictionary} 1917}
does not seem to recognize any distinction between the two prefixes. Most of the unphonetic pronouncing dictionaries give end-stress in all cases.

\is{privative prefixes|)}
\is{prefixes!privative|)}
\is{homonyms and homographs|)}
\is{prefixes!negative|)}
\il{English!un-@\textit{un-}|)}
