\ChapterAndMark{Negative Connectives} 
\label{ch:10}
\is{coordinated clauses|(}
\is{conjoined clauses|(}
\is{connectives!negative|(}
\is{neither...nor@`neither...nor', expressions corresponding to|(}


\il{English!nor@\textit{nor}}
It is, of course, possible to put two negative sentences together without any connective (\textit{he is not rich; his sister is not pretty}) or conversely joined by means of \textit{and} (\textit{he is not rich, and his sister is not pretty}); but when the two ideas have at least one element in common, it is usual to join them more closely by means of some negative connective: \textit{he is neither rich nor pretty} \vert ~\textit{neither he nor his sister is rich} \vert ~\textit{he neither eats nor drinks}. Negative connexions may be of various orders, which are here arranged according to a purely logical scheme: it would be impossible to arrange them historically, and nothing hinders the various types from coexisting in the same language. If we represent the two ideas to be connected as A and B, and understand by c a positive, and by nc a negative connective (while n is the ordinary negative without any connective force), we get the following seven types:

% Define a counter command that uses the gb4e numbering with parentheses
% \newcounter{tabexample}
% \setcounter{tabexample}{0}
% \newcommand{\tabexnum}{\refstepcounter{tabexample}(\thetabexample)}
\ea nc A~~~~nc B
\ex nc A~~~~nc\textsuperscript{2} B (c\textsuperscript{1} and c\textsuperscript{2} being different forms)
\ex nc A~~~~c B
\ex \phantom{nc }A~~~~nc B
\ex n\phantom{c} A~~~~nc B
\ex n\phantom{c} A~~~~nc\textsuperscript{1} B nc\textsuperscript{2}
\ex n\phantom{c} A ~~~~n B nc
\z


Not unfrequently an ordinary negative is found besides the negative connective.

What is here said about two ideas also applies to three or more, though we shall find in some cases simplifications like nc A, B, C, nc D instead of nc A nc B nc C nc D.

\bigskip\pagebreak
In the first three types the speaker from the very first makes the hearer expect a B after the A; in (4), (5), and (6) the connexion is indicated after A, but before B; and finally in (7) it is not till B has been spoken that the speaker thinks of showing that B is connected with A.

\is{connectives!and number}
\is{disjunction}
\il{English!neither ... nor@\textit{neither {\dots} nor}}
\is{either...or@`either...or', expressions corresponding to}
The connectives are often termed disjunctive, like (\textit{either} {\dots}) \textit{or}, but are really different and juxtapose rather than indicate an alternative; this is shown in the formation of Latin 
\il{Latin!neque ... neque@\textit{neque {\dots} neque}}\textit{neque {\dots} neque}, which are negative forms of \textit{que {\dots} que} (`both {\dots} and'), and it very often influences the number of the verb (\textit{neither he nor I \textsc{were}}), see \citet[\href{https://archive.org/details/jespersen-1954-a-modern-english-grammar-on-historical-principles-part-ii-syntax-first-volume/page/176/mode/2up?view=theater}{6.62}]{jespersenMEG2}. \textit{Neither {\dots} nor} thus is essentially different from \textit{either not {\dots} \ or not}, which gives the choice between two negative alternatives, as in (\ref{ex:10-01}).

\ea \label{ex:10-01}
he [Carlyle] either could not or would not think coherently\\\hfill(\href{https://archive.org/details/autobiographyher0000herb/page/380/mode/2up?q=%22either+could+not+or%22&view=theater}{Spencer, \textit{Autobiography} 1.380})
\z

\addsec{nc A nc B} \label{sec:type1}
\is{correlative negation|(}
\is{conjunctions!negative|(}
\is{connectives!duplication of|(}
\is{ne {\dots} ne@`ne {\dots} ne', expressions corresponding to}
The best-known examples of this type---the same connective before A and B---are Latin \textit{neque {\dots} neque} with 
\il{French!ni ... ni@\textit{ni {\dots} ni}}French, 
\il{Spanish!ni ... ni@\textit{ni {\dots} ni}}Spanish \textit{ni \dots~ ni}, 
\il{Italian!\textit{né {\dots} né}}Italian \textit{né \dots~ né}, \il{Romanian!\textit{nicĭ {\dots} nicĭ}}Rumanian \textit{nicĭ {\dots} nicĭ}, % ??? PE: The 1917 book has the "Rum." sample printed rather oddly and indistinctly. (Clearly the regular font didn't offer the wanted sort, which was borrowed from a different font.) But it looks to me less like ⟨nicī⟩ (with macron) than ⟨nicĭ⟩ (with breve). In the Wikipedia article on Romanian orthography I see no evidence for the use of either ⟨ī⟩ or ⟨ĭ⟩; but it is obvious that orthography was unstable in the 19th century and that Romanian has used the breve with other letters, whereas there's no suggestion that anyone has used the macron for any Romanian purpose.
and \il{Greek!\textit{oúte {\dots} oúte}}Greek \textit{oúte {\dots} oúte, \il{Greek!\textit{m{\'ē}te {\dots} m{\'ē}te}}m{\'ē}te {\dots} m{\'ē}te}. In the old Germanic languages we had correspondingly \il{Gothic!\textit{nih {\dots} nih}}Gothic \textit{nih {\dots} nih}, and (with a different word) \il{German!Old High German!noh ... noh@\textit{noh {\dots} noh}}Old High German (Tatian\footnote{St. Gallen, Stiftsbibliothek, MS Cod. Sang. 56: \textit{Evangelienharmonie des Tatian}, a ninth-century translation from the Latin of Tatian. \eds}) % This even has a DOI: 10.5076/e-codices-csg-0056
\textit{noh {\dots} noh}; but in 
\il{English!Old English!ne ... ne@\textit{ne {\dots} ne}}\il{English!Old English!ne@\textit{ne}}\textit{ne {\dots} ne} as found in \il{Old Norse!ne ... ne@\textit{ne {\dots} ne}}Old Norse, \il{Old Saxon!\textit{ne {\dots} ne}}Old Saxon,\footnote{Jespersen has \textit{OS}, which is not listed in any of his lists of abbreviations that we have seen. Old Saxon seems the likeliest interpretation. \eds}
and Old English the written form at any rate does not show us whether we have this type (\textit{ne} corresponding to Gothic \textit{nih}) or the unconnected use of two simple negatives, corresponding to Gothic \textit{ni {\dots} ni}; see on the latter \citet[\href{http://www.jstor.org/stable/40846532}{11ff}]{neckel1912germanischen}. There can be little doubt that the close similarity of the two words, one corresponding to \textit{ni} (Latin \textit{ne}) and the other to \textit{nih} (\textit{neque}), contributed to the disappearance of this type in these languages.

A late (1581) English example is (\ref{ex:10-02}).

\ea \label{ex:10-02}
they \emph{ne} could \emph{ne} would help the afflicted\hfill(\href{https://archive.org/details/oed6barch/page/n885/mode/2up?view=theater}{\textit{NED}, \textit{Ne} B \textit{conj.} 1~a})
\z

\il{English!Old English!ne@\textit{ne}}
\il{English!nother ... nother@\textit{nother {\dots} nother}}
There is another and fuller form of this type in English, namely \textit{nother {\dots} nother} (from \textit{ne} + \textit{ōhwæðer}), which was in use from the 13th century to the beginning of the Modern English period, e.g. (\ref{ex:10-03}). In the shortened form \textit{nor {\dots} nor} it was formerly extremely frequent, as in (\ref{ex:10-04}). This is found as an archaism even in the 19th century, e.g. (\ref{ex:10-05}).

\ea \label{ex:10-03}
whether they belyue well or no, nother the tyme dothe suffer us to discusse, nother it ys nowe necessarye.\hfill(\href{https://archive.org/details/utopiasirthomas00robigoog/page/n321/mode/2up?q=%22nother+the+tyme%22&view=theater}{More, \textit{Utopia} 211})
\z

\ea \label{ex:10-04}
Thou hast nor youth, nor age.\hfill(\href{https://internetshakespeare.uvic.ca/doc/MM_F1/scene/3.1/index.html#tln-1235}{Shakespeare, \textit{Meas} 3.1.32})
\z

\ea \label{ex:10-05}
Nor seeks nor finds he mortal blisses.\hfill(\href{https://archive.org/details/completepoeticalshel/page/220/mode/2up?view=theater&q=%22nor+seeks+nor+finds+he%22}{Shelley, \textit{Prometheus} 1.740})
\z
\is{connectives!duplication of|)}

\addsec{nc\textsuperscript{1} A nc\textsuperscript{2} B} \label{sec:type2}
The type nc1 A nc2 B that is, with two different connectives, both of them negative, has prevailed over nc A nc B in later stages of the Germanic languages. Thus we have Old Norse \il{Old Norse!hvartki (hvarki) ... ne@\textit{hvártki} (\textit{hvárki}) {\dots} \textit{né}}\textit{hvártki} (\textit{hvárki}) {\dots} \textit{né} (`neither {\dots} nor'); \textit{hvártki} corresponds to Gothic \textit{ni-hwaþar-hun} with dropping of the original negative \textit{ne}, the negative sense being attached to \textit{-gi} (\textit{ki}). In German we have 
\il{German!weder ... noch@\textit{weder {\dots} noch}}\textit{weder {\dots} noch}, in which similarly initial \textit{ne} has been dropped; \textit{weder} has quite lost the original pronominal value (`which of two') which \textit{whether} kept much longer in English.

\il{English!Old English!ne@\textit{ne}}
In English, on the other hand, the \textit{n}-element has never been lost, but is found both in the old formula \textit{nother} (\textit{nahwæðer}, \textit{nohwæðer}, \textit{nawðer}, \textit{nowðer}) {\dots} \textit{ne} and in the later (from the Middle English period) \textit{neither} (\textit{naiðer}, \textit{nayther}) {\dots} \textit{ne} as well as in the corresponding forms with \textit{nor} instead of \textit{ne}.

In the second member, the old \textit{ne} as in (\ref{ex:10-06}) was used archaically by \href{https://archive.org/details/spensersfaeriequ01spenuoft/page/12/mode/2up?q=%22+ne+%22}{Spenser} % Peter: OJ writes "Spencer" but clearly means Edmund Spenser, author of The Faerie Queene, a work that certainly does use "ne"  
and sometimes by his imitators (\href{https://archive.org/details/poeticalworksofw00shen2/page/262/mode/2up?view=theater&q=%22ne+superstition%22}{Shenstone, \textit{Schoolmistress}}; % PE: OJ writes "School-mistress" but I think "Schoolmistress" is the standard spelling within the title ("The Schoolmistress")
Byron, \textit{Childe Harold} \href{https://archive.org/details/childeharoldspi10byrogoog/page/n22/mode/2up?q=%22Ne+barrier%22&view=theater}{1} and \href{https://archive.org/details/childeharoldspi10byrogoog/page/n62/mode/2up?q=%22Ne*+citjr%27s%22&view=theater}{2}, etc.).

\ea \label{ex:10-06}
I shal neyther hate hym ne haue enuye at him\hfill(\href{https://archive.org/details/TheHistoryOfReynardTheFoxArber/page/n117/mode/2up?q=%22neyther+hate+hym%22&view=theater}{Caxton, \textit{Reynard} 88})
\z

Apart from this, the normal formula in the Modern English time is \textit{neither {\dots} nor} (\ref{ex:10-07}).

\ea \label{ex:10-07}
\ea
neither he nor his sister has come
\ex
he has neither wit nor money
\ex
I could neither run with speed, nor climb trees\hfill(\href{https://archive.org/details/bim_eighteenth-century_the-works-of-j-s-dd-_swift-jonathan_1735_3/page/338/mode/2up?view=theater&q=%22run+with+speed%22}{J. Swift, \textit{Travels} 336}) 
\ex
he neither loves nor hates her
\z
\z

Where there are more than two alternatives, it is not at all rare to omit the connective with the middle ones or one of them: (\ref{ex:10-11}).

\ea \label{ex:10-11}
\ea
thou hast neither heate, affection, limbe, nor beautie\\\hfill(\href{https://internetshakespeare.uvic.ca/doc/MM_F1/scene/3.1/index.html#tln-1240}{Shakespeare, \textit{Meas} 3.1.37})
\ex
I haue neyther writ nor words, nor worth, action nor vtterance, nor the power of speech\hfill(\href{https://internetshakespeare.uvic.ca/doc/JC_F1/scene/3.2/index.html#tln-1755}{Shakespeare, \textit{Cæs} 3.2.226}) % PE: Though OJ normally decapitalizes what in Shakespeare would not be capitalized in late ModE, he capitalizes "Action" and "Vterrance". I've decapitalized them in accordance with our general practice, which in this area accords with OJ's.
\z
\z

The conjunction may even be omitted poetically before all except the first alternative (\ref{ex:10-13}). This type, which is found only with more than two alternatives, has been placed here for convenience, but might have been given as an independent type: nc A B C D {\dots}.

\ea \label{ex:10-13}
\ea
Nor raine, winde, thunder, fire are my daughters\\\hfill(\href{https://internetshakespeare.uvic.ca/doc/Lr_F1/scene/3.2/#tln-1670}{Shakespeare, \textit{Lr} 3.2.15}) 
\ex
Neyther presse, coffer, chest, trunke, well, vault\\\hfill(\href{https://internetshakespeare.uvic.ca/doc/Wiv_F1/scene/4.2/index.html#tln-1950}{Shakespeare, \textit{Wiv} 4.2.62})
\ex
she loved him, as Nor brother, father, sister, daughter love\\\hfill(\href{https://archive.org/details/workslordbyron10unkngoog/page/414/mode/2up?view=theater&q=%22daughter+love%22}{Byron, \textit{Juan} 10.53})
% "she loved him," restored
\ex
connected In neither clime, time, blood, with her defender\hfill(\href{https://archive.org/details/workslordbyron10unkngoog/page/416/mode/2up?view=theater&q=%22neither+clime%22}{ibid 10.57})
\z
\z
\is{conjunctions!negative|)}

\addsec{nc A c B}
\is{connectives!positive|(}
\il{Danish!hverken ... eller@\textit{hverken {\dots} eller}}
Next, we come to the type: nc A c B. This is different from the preceding one in that the second connective is a positive one, the same as is used in alternatives like \textit{either {\dots} or, aut {\dots} aut, on {\dots} on, entweder {\dots} oder}. Here the negative force of nc is strong enough to work through A so as to infect B. This is the type in regular use in modern Scandinavian, as in Danish \textit{hverken {\dots} eller}, Swedish 
\il{Swedish!\textit{varken {\dots} eller}}\textit{varken {\dots} eller}. Examples: (\ref{ex:10-17}), etc.

\ea \label{ex:10-17}
 \ea 
 \gll han er hverken rig eller smuk\\
 he is neither rich nor handsome\\
 %\glt `He is neither rich nor handsome.'
 \ex 
 \gll hverken han eller hans søster er rig\\
 neither he nor his sister is rich\\
 %\glt `Neither he nor his sister is rich.'
 \ex 
 \gll han hverken spiser eller drikker\\
 he neither eats nor drinks\\
 %\glt `He neither eats nor drinks.' % PE: Another set of quotes where presenting both a word-for-word gloss and an idiomatic translation (identical to the gloss) seems unnecessary.
 %Brett: fixed
 \z
\z

\il{English!neither ... or @\textit{neither {\dots} or}}
\il{English!or@\textit{or}}
In English \textit{neither {\dots} or} is by no means uncommon (\ref{ex:10-18}), though now it has been generally discarded from literary writings through the influence of schoolmasters.

\ea \label{ex:10-18}
\ea
That you swerue not from the smallest article of it, Neither in time, matter, or other circumstance.\hfill(\href{https://internetshakespeare.uvic.ca/doc/MM_F1/scene/4.2/index.html#tln-1965}{Shakespeare, \textit{Meas} 4.2.108})\footnote{According to \citet[\textit{\href{https://www.perseus.tufts.edu/hopper/text?doc=Perseus\%3Atext\%3A1999.03.0079\%3Aentry\%3DNeither}{Neither}}]{schmidt1886}, only three or four times in Shakespeare. (Jespersen)}
\ex
they neither can speak, or attend to the discourses of others\\\hfill(\href{https://archive.org/details/bim_eighteenth-century_the-works-of-j-s-dd-_swift-jonathan_1735_3/page/198/mode/2up?view=theater&q=%22attend+to+the+discourses%22}{J. Swift, \textit{Travels} 199}) 
\ex
I had neither the strength or agility of a common Yahoo\hfill(\href{https://archive.org/details/bim_eighteenth-century_the-works-of-j-s-dd-_swift-jonathan_1735_3/page/338/mode/2up?view=theater&q=%22agility+of+a+common%22}{ibid 336})
\ex
many answers and replies which are neither witty, humorous, polite, or authentic\hfill(\href{https://archive.org/details/cu31924013200898/page/n41/mode/2up?q=%22replies+which+are+neither%22&view=theater}{J. Swift, \textit{Conversation} 6}) % Adding to the front, completing the NP
\ex
I neither saw, or desir'd to see any people\hfill(\href{https://archive.org/details/lifeandstranges00dobsgoog/page/n51/mode/2up?q=%22I+neither%22&view=theater}{Defoe, \textit{Robinson} 26}) % "People" capitalized
\ex
I neither had any business in the ship, or learn'd to do any.\hfill(\href{https://archive.org/details/lifeandstranges00dobsgoog/page/n43/mode/2up?q=%22I+neither%22&view=theater}{ibid 17}) % OJ provides the page number of the example but not the example itself
\ex
I lay a-bed all day, and neither eat or drank.\hfill(\href{https://archive.org/details/lifeandstranges00dobsgoog/page/n127/mode/2up?q=%22neither+eat%22&view=theater}{ibid 101}) % OJ provides the page number of the example but not the example itself
\ex
I would neither see it my self, or learn to know the blessing of it from my parents\hfill(\href{https://archive.org/details/lifeandstranges00dobsgoog/page/n131/mode/2up?q=%22but+I+would%22&view=theater}{ibid 106}) etc. % OJ provides the page number of the example but not the example itself
\ex
having neither sail, oar, or rudder\hfill(\href{https://archive.org/details/lifeandstranges00dobsgoog/page/n83/mode/2up?q=%22having+neither%22&view=theater}{ibid 58}) 
\ex
I had neither food, house, clothes, weapon, or place to fly to\hfill(\href{https://archive.org/details/lifeandstranges00dobsgoog/page/n107/mode/2up?q=%22neither+food%22&view=theater}{ibid 81}) % OJ provides the page number of the example but not the example itself
\ex
a cloak {\dots} , neither fit to defend the wearer from cold or from rain\\\hfill(\href{https://archive.org/details/scottsivanhoeedi0000amar/page/182/mode/2up?q=%22neither+fit+to+defend%22&view=theater}{Scott, \textit{Ivanhoe} 167}) % "of scanty dimensions" omitted
\ex
Am I neither to be obeyed as a master or a father? \\\hfill(\href{https://archive.org/details/cewaverleynovels03scotuoft/page/378/mode/2up?view=theater&q=%22neither+to+be+obeyed+as+a+master%22}{Scott, \textit{Antiquary} 2.36}) % This should appear on or around page 231 of the edition that we're linking to of the second half of the novel. But the example closest to 231 that I (PE) can find is "say nothing at all neither about somebodies or nobodies!" (243). Others, farther from 231, are: "But neither for laird or loon, gentle or semple, will I leave my ain house" (303); "a degree of inveteracy which, at such a distance of time, a mortal offence would neither have authorized or excited in any well-constituted mind" (308); "Am I neither to be obeyed as a master or a father?" (379); among which I think the last is the neatest. Suggestion: either (a) quote and link to the last, or (b) delete any mention.
%Brett: (a), please.  PE: Done.
\ex
thrifty men who neither fell into laggard relaxation of diligence or were stung by any madness of ambition\\\hfill(\href{https://archive.org/details/reminiscences0000thom_e9a0/page/58/mode/2up?q=%22thrifty+men%22&view=theater}{T. Carlyle, \textit{Reminiscences} 1.73}) % Removing a couple of commas not in the original
\ex
He neither wore on helm or shield The golden symbol of his kinglihood\hfill(\href{https://en.wikisource.org/wiki/Idylls_of_the_King/The_Coming_of_Arthur}{Tennyson, \textit{Coming}})
\ex
I am suffering neither from one or the other\hfill(\href{https://archive.org/details/dukeschildrennov00troluoft/page/226/mode/2up?q=%22am+suffering+neither%22&view=theater}{Trollope, \textit{Children} 2.140}) % Removing second "from": Trollope doesn't write it.
\z
\z


\il{English!or@\textit{or}}
Defoe, who very often has \textit{neither {\dots} or}, has the following sentences, which are interesting as showing the effect of distance: where \textit{neither} is near, \textit{or} suffices; where it is some distance back, the negative force has to be renewed: (\ref{ex:10-33}).

\il{English!nor@\textit{nor}}
\ea \label{ex:10-33}
\ea
I neither knew how to grind or to make meal of my corn, or indeed how to clean it and part it; nor if made into meal, how to make bread of it\hfill(\href{https://archive.org/details/lifeandstranges00dobsgoog/page/n163/mode/2up?q=%22neither+knew+how%22&view=theater}{\textit{Robinson} 138}) 
\ex
having neither weapons or cloaths, nor any food\hfill(\href{https://archive.org/details/lifeandstranges00dobsgoog/page/n317/mode/2up?q=%22neither+weapons%22&view=theater}{ibid 291}) % restored DD's spelling of "cloaths". (OJ's modernization is odd, as the edition of RC he quotes from has the exact same pagination as the edition I've linked to, and presumably is the same.)
\z
\z

\il{English!or@\textit{or}}
In (\ref{ex:10no-nor}) \textit{brother or sister} forms so to speak one idea (Ido\footnote{A derivative of Esperanto; Jespersen was an early and notable proponent. \eds} epicene \textit{frato}), hence \textit{nor} is not used between them. (\ref{ex:10no-nor2}) also shows that \textit{or} is preferred when two words are closely linked together; if we substitute \textit{nor}, we should be obliged to continue: \textit{nor how to speak}. A closely similar sentence is found in (\ref{ex:10no-nor3}); here if we substitute \textit{nor}, it will be necessary to repeat \textit{thy} before \textit{actual}; but if we change the word-order, it will be possible to say \textit{thou seest neither thy original nor actual infirmities}. (In other places Bunyan uses \textit{neither {\dots} nor}, thus (\ref{ex:10no-nor4}).)

\ea \label{ex:10no-nor}
neither she nor your brother or sister suspected a word of the matter\\\hfill(\href{https://archive.org/details/sensesensibility00austrich/page/226/mode/2up?q=%22neither+she+nor+your%22&view=theater}{Austen, \textit{Sense} 253})
\ex \label{ex:10no-nor2}
He knew neither how to walk or speak\hfill(news 1905) 
\ex \label{ex:10no-nor3}
\ea
they neither know how to do for, or speak to him\\\hfill(\href{https://archive.org/details/bunyanspilgrims00moffgoog/page/104/mode/2up?q=%22neither+know+how%22&view=theater}{Bunyan, \textit{Progress} 107})
\ex
thou neither seest thy original or actual infirmities\hfill(\href{https://archive.org/details/bunyanspilgrims00moffgoog/page/192/mode/2up?q=%22neither+seest+thy+original%22&view=theater}{ibid 204}) % No comma in source
\z
\ex \label{ex:10no-nor4}
\ea
There is there neither prayer, nor sign of repentance for sin\hfill(\href{https://archive.org/details/bunyanspilgrims00moffgoog/page/104/mode/2up?q=%22there+neither+Prayer%22&view=theater}{ibid 106}) 
\ex
they can neither call him Brother, nor Friend\hfill(\href{https://archive.org/details/bunyanspilgrims00moffgoog/page/104/mode/2up?q=%22can+neither+call+him%22&view=theater}{ibid 108}) % These are Bunyan's capitals; should it be "brother, nor friend"?
%Brett: keep
\z \z


\il{English!or@\textit{or}}
The use of \textit{or} after \textit{neither} cannot be separated from the use of \textit{or} after another negative, as in the following instances; it will be seen that \textit{or} is more natural in (\ref{ex:natural-or}), because the negative word can easily cover everything following, than in (\ref{ex:unnatural-or1}) or (\ref{ex:unnatural-or2}):

\ea \label{ex:natural-or}
 \ea
 Faustus vowes neuer to looke to heauen, Neuer to name God, or to pray to him\hfill(\href{https://babel.hathitrust.org/cgi/pt?id=uiuo.ark:/13960/t1pg73v3n&seq=128&q1=Faustus+vowes+neuer+to+looke}{Marlowe, \textit{Faustus} 718})
 \ex 
 Talke not of paradice or creation\\\hfill(ibid 729, \href{https://babel.hathitrust.org/cgi/pt?id=uiuo.ark:/13960/t1pg73v3n&seq=129&q1=paradise,+or+creation}{1616 edition}; but the \href{https://babel.hathitrust.org/cgi/pt?id=uiuo.ark:/13960/t1pg73v3n&seq=128&q1=paradise,+nor+creation}{1604 edition} has \textit{nor}) % OJ merely gives the line numbers and edition years, but not what Marlowe writes. (Marlowe capitalizes, "Paradice or Creation"; but I (PE) have done this in lowercase, following OJ's preference.)
 \ex
 he {\dots} lived alone, and never saw her, or inquired for her\\\hfill(\href{https://archive.org/details/dombeyson00dick_0/page/248/mode/2up?q=%22never+saw+her%22&view=theater}{Dickens, \textit{Dombey} 156}) % Added dots to show a cut by OJ; not "after" but "for"
 \ex
 She knew not what to think, or how to account for it.\\\hfill(\href{https://archive.org/details/prideprejudice00aust/page/310/mode/2up?q=%22knew+not+what+to+think%22&view=theater}{Austen, \textit{Pride} 310})
 \ex
 I haven't seen Palgrave yet or Woolner {\dots} I have not written to Browning yet or seen him\hfill(\href{https://archive.org/details/alfredlordtenny05tenngoog/page/116/mode/2up?q=%22palgrave+yet%22&view=theater}{Tennyson, diary})
 \ex
 Nobody was singing or shouting\hfill(\href{https://archive.org/details/mrbritlingseesi02unkngoog/page/n194/mode/2up?view=theater&q=%22singing+or+shouting%22}{Wells, \textit{Britling} 179})
 \z
\z
\pagebreak
\ea \label{ex:unnatural-or1}
 \ea
 a fruitful pleasant country, and no snow, no wolves, or any thing like them\hfill(\href{https://archive.org/details/lifeandstranges00dobsgoog/page/n385/mode/2up?q=%22any+thing+like%22&view=theater}{Defoe, \textit{Robinson} 359}) % "fruitful" restored
 \ex
 there were no looking-glasses or any bedroom signs about it\\\hfill(\href{https://archive.org/details/twelvestoriesand00well/page/216/mode/2up?q=%22no+looking-glasses%22&view=theater}{Wells, \textit{Stories} 70}) 
 \ex
 there were no clinging hands, or stolen looks, or any vow or promise\\\hfill(\href{https://archive.org/details/bwb_UE-390-059/page/270/mode/2up?view=theater&q=%22clinging+hands%22}{Parker, \textit{Right} 240})
 \z
\z
\ea \label{ex:unnatural-or2}
 \ea
 and not a hair of her head, or a fold of her dress, was stirred\\\hfill(\href{https://archive.org/details/personalhistory05dickgoog/page/n55/mode/2up?q=%22hair+of+her+head%22&view=theater}{Dickens, \textit{David} 114})
 \ex
 not a word was said, or a step taken\hfill(\href{https://archive.org/details/personalhistory05dickgoog/page/n59/mode/2up?q=%22word+was+said%22&view=theater}{ibid 125})
 \ex
 because your religion is not my religion or your God my God\\\hfill(\href{https://archive.org/details/christianstory00cainrich/page/112/mode/2up?q=%22your+religion+is+not%22&view=theater}{Caine, \textit{Christian} 95})
 \z
\z

Note also the change in \textit{No one supposes that the work is accomplished now or could be accomplished in one day} and \textit{{\dots} {\dots} {\dots} is accomplished now, nor could it be\hfill accomplished in one day}.

The continuation with \textit{hardly} is interesting in (\ref{ex:10-40}).

\ea \label{ex:10-40}
because he never trifled or talked gallantry with them, or paid them, indeed, hardly common attentions\hfill(\href{https://archive.org/details/essayseliacharle00lamb/page/100/mode/2up?q=%22never+trifled+or+talked+gallantry%22&view=theater}{Lamb, \textit{Elia} 1.155})
\z
\is{connectives!positive|)}
\is{correlative negation|)}

\addsec{A nc B}
\is{conjunctions!``looking before and after''}
\is{conjunctions!negative|(}
\il{English!Old English!ne@\textit{ne}}
A nc B, that is, a negative conjunction ``looking before and after'' and rendering both A and B negative, is comparatively frequent in Old Norse and Old English with \textit{ne}; from Wimmer's \textit{Læsebog} I quote (\ref{ex:10-43}); from Old English, (\ref{ex:10-45}). (The passages mentioned in \citet[\href{https://archive.org/details/sprachschatzdera00greiuoft/page/492/mode/2up?view=theater}{~493}]{grein_sprachschatz_1912}, are not parallel: in \href{http://ebeowulf.uky.edu/ebeo4.0/CD/main.html}{\textit{Beowulf} 1604} % Temporarily (?) unavailable 2 June 2024
%Brett: OJ has 1604.
%PE: The Kentucky website transcribes the MS as "wiston 7ne wendon" and (I think) numbers the line 1603
\textit{wiston ond ne wendon} must be understood `they wished, but did not think'; in \href{https://archive.org/details/cu31924013340348/page/n99/mode/2up?view=theater}{\textit{Andreas} 303} % A line number 
and \textit{Gu.} 671 % ??S I (PE) imagine that "Gu" is the long poem Guthlac, famously preserved within the "Exeter Book". OJ could have known of this via Codex Exoniensis (ed Benjamin Thorpe, 1842; https://archive.org/details/cu31924013337617/page/106/mode/2up?q=Guthlac ), but with no line numbering; or (more likely) via The Exeter Book (ed Israel Gollancz, 1895; https://archive.org/details/exeterbookanthol00goll/page/144/mode/2up ), and with line numbering (but no particular concentration of negation markings before the line that it numbers 671).
%Brett: OK  
% PE: Well, I wonder. If only there were plentiful negation around line 671. I'll make another attempt to dig around for some other edition or work.
the great number of preceding \textit{ne}'s account for the omission in one place, cf. above p.~\pageref{sec:type2}f.)

\ea \label{ex:10-43}
\ea 
 \gll kyks né dauðs nautka ek karls sonar\\ % Not "nautkak" but "nautka ek"
 %% SG: "nautkak" is just "nautka ek" with a cliticized 1SG pronoun; the former is probably the original manuscript reading, which was then split up in Wimmer's reader (a textbook for students after all)
 alive nor dead {enjoyed not} I farmer's son\\
 \glt `Whether alive or dead, I had no use for the farmer's son'
\hfill(\href{https://archive.org/details/oldnordisklaese01wimmgoog/page/n23/mode/2up?q=kyks&view=theater}{13})
 \ex 
 \gll hönd um þvær né höfuð kembir\\
 hand \textsc{part} washes nor head combs\\ % Not "ǫ" but "ö"; not "of" but "um"; not "þær" but "þvær".
 %% SG: <ǫ> and <ö> are just two alternative ways of representing the same Old Norse character. "of" and "um" are two variants of the same particle. It is not a negation, but a preverbal particle sometimes said to be more or less redundant. Not sure what the current communis opinio is, but there is a whole chapter on the particle in this book: https://benjamins.com/catalog/sigl.5.c3
 \glt `[He] neither washes his hand nor combs his hair'\hfill(\href{https://archive.org/details/oldnordisklaese01wimmgoog/page/n123/mode/2up?q=kembir&view=theater}{117}) % ??? PE: In my English, one combs one's hair (or possibly one's beard) but not one's head. And whatever one combs, the NP needs a determiner. How about `They [singular] neither wash their hands nor comb their hair'?
 %% SG: The sentence is about the god Vali, to the subject should be "he". And it's of course his own hand and head/hair -- possessive pronouns are not necessary with the subject's body parts, as they usually are in English (cf. Danish "jeg vasker hår" = 'I wash my hair')
 \z
\z

\ea \label{ex:10-45}
 \ea
 \gll suð ne norð\\
 south nor north\\
 \glt `neither south nor north'\hfill(\href{http://ebeowulf.uky.edu/ebeo4.0/CD/main.html}{\textit{Beowulf} 858a}) % Temporarily (?) unavailable 2 June 2024
 \ex 
 \gll wordum ne worcum\\
 by words nor by deeds\\
 \glt `neither by words nor by deeds'\hfill(\href{http://ebeowulf.uky.edu/ebeo4.0/CD/main.html}{ibid 1100a}) % Temporarily (?) unavailable 2 June 2024
\z
\z

See \citet[\href{https://archive.org/details/germanische-syntax-i-zu-den-negativen-s/page/n61/mode/2up}{55f}]{delbruck_negativen_1910}, % I (PE) suppose the book is what OJ elsewhere refers to as "Neg. Sätze", i.e. Germanische Syntax 1. Zu den negativen Sätzen (1910)
where also instances of Old High German \textit{noh} may be found: (\ref{ex:10-47}), etc. \citet[]{paul_deutsches_1908} has a few modern instances: (\ref{ex:10-49}). The examples show that Delbrück's restriction % I (PE) suppose in the book that OJ elsewhere calls "Neg. Sätze", i.e. Germanische Syntax 1. Zu den negativen Sätzen (1910)
to ``einem zweigliedrigen nominalen Ausdruck'' (`a nominal phrase with two heads') is too narrow; nor can I admit the correctness of his explanation that ``\textit{ni} erspart wurde, weil eine doppelte Negation in dem kurzen Satzstück als störend empfunde wurde'' (`\textit{ni} is not used, because a double negative in the short sentence is considered jarring'). \citet[]{neckel1912germanischen} says,
% On p.13 of "Zu den germanischen Negationen" (though with different orthography)
more convincingly: ``In solchen Ausdrücken steht \textit{ni}(\textit{h}) apò koinoû. Die unmittelbare Nachbarschaft mit beiden Gliedern erlaubt, es auf beide zu beziehen.'' (`In such expressions, \textit{ni}(\textit{h}) is used as a double negative. The close proximity of the two words allows it to be applied to both.') \is{prosiopesis}And then prosiopesis comes into play, too. 

\ea \label{ex:10-47}
\ea
\gll laba noh gizami\\
 sustenance nor salvation\\
\glt `neither sustenance nor salvation'
\ex
\gll kind noh quena\\
 child nor woman\\
\glt `neither child nor woman'
\z
\z

\ea \label{ex:10-49}
\ea
\gll in Wasser noch in Luft\\
 in water nor in air\\
\glt `neither in the water nor in the air'\hfill(\href{https://www.projekt-gutenberg.org/wieland/oberon/oberon83.html}{Wieland, \textit{Oberon}})
\ex
\gll da ich mich wegen eines Termins der Herausgabe noch sonst auf irgend eine Weise binden oder verpflichten kann\\ % Restored "oder verpflichten"
 where I myself because a deadline {of the} publication nor otherwise in any one way commit or oblige can\\
\glt `where I cannot commit myself to a publication deadline or in any other way'\hfill(\href{http://www.zeno.org/Literatur/M/Goethe,+Johann+Wolfgang/Briefe/1810}{Goethe, letter})
\z
\z

\il{English!nor@\textit{nor}|(}
In later English, though not often in quite recent times, we find \textit{nor} used in the same way without a preceding negative: (\ref{ex:10-51}).

\ea \label{ex:10-51}
\ea
my fadre nor I dyde hym neuer good\hfill(\href{https://archive.org/details/TheHistoryOfReynardTheFoxArber/page/n117/mode/2up?q=%22dyde+hym+neuer+good%22&view=theater}{Caxton, \textit{Reynard} 89}) % Not "fader" but "fadre"
\ex
for Iak nor for gill will I turne my face\hfill(\href{https://archive.org/details/towneleyplays71engl/page/32/mode/2up?q=%22turne+my+face%22&view=theater}{\textit{Noah} 336}) % OJ writes "Jak" but the source linked to has "Iak". In this source, each "ll" (pair of lowercase Ls) has a bar across it.
\ex
The king of England, nor the court of Fraunce, shall haue me from my gratious mothers side\hfill(\href{https://quod.lib.umich.edu/e/eebo/A07018.0001.001/1:2?rgn=div1;view=fulltext}{Marlowe, \textit{Edward} 1633})
\ex
so closely convaide that his new ladie nor any of her friendes know it\\\hfill(\href{https://archive.org/details/representativee02unkngoog/page/438/mode/2up?q=%22closely+convaide%22&view=theater}{\textit{Eastward} 439})
\ex
Tongue nor heart cannot conceiue, nor name thee\\\hfill(\href{https://internetshakespeare.uvic.ca/doc/Mac_F1/scene/2.3/index.html#tln-815}{Shakespeare, \textit{Mcb} 2.3.69})
\ex
they threatned, that the cage nor irons should serve their turn\\\hfill(\href{https://archive.org/details/bunyanspilgrims00moffgoog/page/122/mode/2up?q=%22cage+nor+irons%22&view=theater}{Bunyan, \textit{Progress} 127}) 
\ex
{}[they] were both strongly prepossessed that she nor her daughters were such kind of women\hfill(\href{https://archive.org/details/sensesensibility00austrich/page/200/mode/2up?q=%22both+strongly+prepossessed%22&view=theater}{Austen, \textit{Sense} 227}) % Subject is not "they" but "Fanny and Mrs. Ferrars"
\ex
She {\dots} struggled against this for an instant or two (maid nor nobody assisting)\hfill(\href{https://archive.org/details/reminiscences0000thom_e9a0/page/472/mode/2up?q=%22struggled+against+this%22&view=theater}{T. Carlyle, \textit{Reminiscences} 2.257}) % Dots added to show a major cut by OJ
\ex
My father, nor his father before him, ever saw it otherwise\\\hfill(\href{https://archive.org/details/AWonderBookForGirlsBoys/page/n201/mode/2up?q=%22nor+his+father+before+him%22&view=theater}{Hawthorne, \textit{Wonder} 126})
\z 
\z % OJ attributes the last quotation to Hawthorne's Tanglewood Tales. I (PE) looked in two editions of this and didn't find it there.

It will be seen that all these are examples of principal words (substantives or pronouns); it is very rare with verbs, as in (\ref{ex:10-60}), where \textit{no longer} shows that the negative notion is to be applied to both auxiliaries. Cf. also (\ref{ex:10-61}).

\ea \label{ex:10-60} but I can nor will stay no longer now\hfill(\href{https://archive.org/details/journaltostellae00swifuoft/page/116/mode/2up?q=%22can+nor+will+stay%22&view=theater}{J. Swift, \textit{Journal} 117})
\z

\ea \label{ex:10-61} he moved nor spoke, Nor changed his hue, nor raised his looks to meet The gaze of strangers\hfill(\href{https://archive.org/details/completepoeticalshel/page/82/mode/2up?view=theater&q=%22he+moved+nor+spoke%22}{Shelley, \textit{Revolt} 5.22}) % Not "locks" but "looks"; restored "to meet The gaze of strangers" to make the notion of raising looks less implausible
\z

On a different use of the same form (A nc B), where A is to be understood in a positive sense, see below p.~\pageref{another-way-of-using-it}.
\il{English!nor@\textit{nor}|)}
\is{conjunctions!negative|)}

\addsec{n A nc B} \label{sec:10.5}
\il{Danish!heller ikke@\textit{heller ikke}}
\is{no more@`no more', expressions corresponding to}
In this type the negativity of A is indicated, though not by means of a connective. The negative connective (nc) before B is the counterpart of \textit{also} or \textit{too}; and some languages, such as German, have no special connective for this purpose, but use the same adverb as in positive sentences (\il{German!auch nicht@\textit{auch nicht}}\textit{auch nicht}); in French the negative comparative 
\il{French!non plus@\textit{non plus}}\textit{non plus} is used either with or without the negative connective \textit{ni}. Danish has a special adverb used with some negative word, \textit{heller ikke}, \textit{heller ingen}, etc.; \textit{heller} (Old Norse \textit{heldr}) is an old comparative as in the French expression and signifies `rather, sooner'. In English, the same negative connectives are used as in the previous types, but in rather a different way; \textit{but no more} may also be used.\label{para:143}

Examples of this type: (\ref{ex:10-62}).
% ??? "type 5"?

\ea \label{ex:10-62}
\ea
I speake not this, that you should beare a good opinion of my knowledge {\dots}. \emph{neither} do I labor for a greater esteeme\\\hfill(\href{https://internetshakespeare.uvic.ca/doc/AYL_F1/scene/5.2/index.html#tln-2460}{Shakespeare, \textit{As} 5.2.61})
\il{English!nor@\textit{nor}|(}
\ex
My ventures are not in one bottome trusted {\dots}. \emph{nor} is my whole estate Vpon the fortune of this present yeere\hfill(\href{https://internetshakespeare.uvic.ca/doc/MV_F1/scene/index.html#tln-45}{Shakespeare, \textit{Merch} 1.1.43})
\ex
as yet he had not got rid thereof, \emph{nor} could he by any means get it off without help\hfill(\href{https://archive.org/details/bunyanspilgrims00moffgoog/page/36/mode/2up?q=%22got+rid+thereof%22&view=theater}{Bunyan, \textit{Progress} 17})
\ex
never attaching herself much to us, \emph{neither} us to her\\\hfill(\href{https://archive.org/details/praeterita01rusk/page/120/mode/2up?view=theater&q=%22never+attaching+herself%22}{Ruskin, \textit{Præterita} 1.120})
\ex
the royal Dane does not haunt his own murderer,---\emph{neither} does Arthur, King John; \emph{neither} Norfolk, King Richard II.; \emph{nor} Tybalt, Romeo\hfill(\href{https://archive.org/details/ruskinasliterary0000rusk/page/74/mode/2up?q=%22royal+dane+does+not+haunt%22&view=theater}{Ruskin, \textit{Fors}}) % OJ attributes this to "Ruskin F. 42". There's something wrong with that: it seems to have appeared in the 4th volume of Fors Clavigera, so he might have meant "Ruskin F. 4. 42" but then again he might have meant something else. I can't find vol 4 of FC on the web; I'm linking to a derivative.
\ex
Nothing, again, makes us think. {\dots} \emph{Nor}, I believe, are the facts ever so presented. {\dots} \emph{Neither}, lastly, do we receive the impression.{\dots}\\\hfill(\href{https://archive.org/details/shakespeareantra1905brad/page/28/mode/2up?q=%22facts+ever+so+presented%22&view=theater}{Bradley, \textit{Tragedy} 29}) % ", again," restored
\ex
She said nothing, \emph{neither} did he.\hfill(\href{https://archive.org/details/septimus00unkngoog/page/n181/mode/2up?q=%22she+said+nothing%22&view=theater}{Locke, \textit{Septimus} 186})
\z
\z

\textit{But neither} is used in the same way: (\ref{ex:10-69}). And \textit{nor} in the same sense is rarer: (\ref{ex:10-72}).

\ea \label{ex:10-69}
\ea
She had no great talents {\dots}; but neither had she any deficiency or vice\\\hfill(\href{https://archive.org/details/JaneEyre-CharlotteBronte/page/n95/mode/2up?q=%22she+had+no+great+talents%22&view=theater}{Brontë, \textit{Jane} 118})
\ex
He did not for a moment under-estimate the danger; but neither did he exaggerate its importance\hfill(\href{https://books.google.co.jp/books?id=qcScwG5bgxgC&pg=PP3&dq=%22justin+mccarthy%22+%22our+own+times%22&hl=en&newbks=1&newbks_redir=0&sa=X&redir_esc=y#v=snippet&q=%22did%20not%20for%20a%20moment%22&f=false}{McCarthy, \textit{Short history} 2.52}) % Note that there's only one "a" in "McCarthy", and note the hyphen in "under-estimate"
\ex
They were not exactly studious youths, but neither did they belong to the class that Godwin despised\hfill(\href{https://archive.org/details/borninexileanov02gissgoog/page/58/mode/2up?q=%22studious+youths%22&view=theater}{Gissing, \textit{Born} 63}) % "exactly" restored; "G" reverted to "Godwin"
\z
\z

\ea \label{ex:10-72}
Thackeray, for instance, didn't take a degree, \emph{and nor} did---oh, lots of others\hfill(\href{https://archive.org/details/cambridgetrifle01bangoog/page/n234/mode/2up?q=Thackeray&view=theater}{\textit{Trifles} 194})
\z

Very often the sentence introduced by \textit{neither} or \textit{nor} is added by a different speaker, as in (\ref{ex:10-73}).

\ea \label{ex:10-73}
 \ea
 ''Hath no man condemned thee?'' --- ``No man.'' --- ``\emph{Neither} doe I condemne thee''\hfill(\href{https://www.kingjamesbibleonline.org/1611_John-8-10/}{AV \textit{John} 8.10--11})
 \ex Did no one condemn you? {\dots} No one. \emph{Nor} do I condemn you.\\\hfill(in the 20th century translation) % I (PE) am puzzled by the definiteness of "the 20th c. translation". https://www.instonebrewer.com/TyndaleSites/Scriptures/www.innvista.com/scriptures/compare/story.htm may be of some interest; but the five versions that seem closest all postdate 1917 (unless I misunderstand something, which is likely).
 \z
\z

A repetition of the negation is very frequent in these sentences (\ref{ex:10-74}).

\ea \label{ex:10-74}
\ea
I neuer did repent for doing good, Nor shall not now\\\hfill(\href{https://internetshakespeare.uvic.ca/doc/MV_F1/scene/3.4/index.html#tln-1735}{Shakespeare, \textit{Merch} 3.4.11})
\ex
I know not loue (quoth he) nor will not know it\\\hfill(\href{https://internetshakespeare.uvic.ca/doc/Ven_Q1/page/22/index.html#tln-405}{Shakespeare, \textit{Ven} 409})
\ex
Nor they will not utter the other\hfill(\href{https://archive.org/details/essayscoloursofg00bacouoft/page/14/mode/2up?q=%22utter+the+other%22&view=theater}{Bacon, \textit{Parents}}; \\\hfill quoted \citet[86]{bogholm1906bacon} with other examples)
\ex
I don't quarrel at that, nor I don't think but your conversation was very innocent \hfill(\href{https://archive.org/details/in.ernet.dli.2015.219151/page/n165/mode/2up?q=%22quarrel+at+that%22&view=theater}{Congreve, \textit{Love} 231})% "very" restored
\ex
nor you shall not know till I see you again\hfill(\href{https://archive.org/details/journaltostellae00swifuoft/page/60/mode/2up?q=%22nor+you+shall+not+know%22&view=theater}{J. Swift, \textit{Journal} 61})
\ex
Steele {\dots}. came not, nor never did twice, since I knew him\hfill(\href{https://archive.org/details/journaltostellae00swifuoft/page/114/mode/2up?q=%22nor+never+did+twice%22&view=theater}{ibid 115})
\ex
Nor shall we not be tending towards that point\\\hfill(\href{https://en.wikisource.org/wiki/Page%3AThe_Prelude%2C_Wordsworth%2C_1850.djvu/247}{Wordsworth, \textit{Prelude} 8.451})
\ex
I have never told any one but you; nor I should not have mentioned it now, but {\dots}\hfill(\href{https://archive.org/details/liberamorisornew00hazlrich/page/14/mode/2up?q=%22have+never+told%22&view=theater}{Hazlitt, \textit{Liber} 15}) % Restored "but you"
\ex
I cannot live without you---nor I will not.\hfill(\href{https://archive.org/details/liberamorisornew00hazlrich/page/22/mode/2up?q=%22cannot+live%22&view=theater}{ibid 23})
\ex
I never saw anything like her, nor I never shall again\hfill(\href{https://archive.org/details/liberamorisornew00hazlrich/page/96/mode/2up?q=%22never+saw+anything%22&view=theater}{ibid 97})
\ex
For the life of them vanishes and is no more seen, Nor no more known\\\hfill(\href{https://archive.org/details/cu31924013555911/page/n53/mode/2up?q=%22them+vanishes%22&view=theater}{Swinburne, \textit{Songs} 42}; probably in imitation of Elizabethan English) % "El.E." is explained as "Elizabethan English" in MEG vol 7.
\z
\z

Bacon, according to \citet[85]{bogholm1906bacon} nearly always carries through the distinction \textit{neither} + verb + subject (\textit{neither do I say}) without \textit{not}, and \textit{nor} + subject + verb with \textit{not} or other negative (\textit{nor they will not utter}); it will be seen from my examples that the latter construction is the more frequent one with other writers as well.
\il{English!nor@\textit{nor}|)}


\il{English!no more@\textit{no more}}\is{no more@`no more', expressions corresponding to}
Instead of \textit{neither} or \textit{nor} we also have the combination \textit{no more} (cf. French above), used in a similar context but without the original meaning of negation: (\ref{ex:10-85}). The same with repeated negation (\ref{ex:10-87}). Cf. also (\ref{ex:10-88}).

\ea \label{ex:10-85}
\ea
You don't like Wagstaff? \emph{No more} do I much\footnote{This \textit{much} shows that \textit{no more} is used without any consciousness of its original meaning. (Jespersen)}\hfill(\href{https://archive.org/details/mrscaudlescurtai00jerruoft/page/78/mode/2up?q=%22no+more+do+I+much%22&view=theater}{Jerrold, \textit{Lectures} 60}) % "I" corrected to "You"; "W" disabbreviated to "Wagstaff"; "?" restored to the end of the first sentence (which Jerrold italicizes).
\ex
``Brown says you don't believe a gentleman can lick a cad {\dots}.'' --- ``\emph{No more} I do.''\hfill(\href{https://archive.org/details/tombrownatoxford00hughiala/page/138/mode/2up?q=%22Brown+says+you+don%27t%22&view=theater}{Hughes, \textit{Oxford} 133}) % Not OJ's "that" but "a gentleman can lick a cad, unless he is the biggest and strongest of the two" (incidentally, a negation-irrelevant but interesting use of the superlative).
\z
\z

\ea \label{ex:10-87}
``I would swear to speak ne'er a word to her to-day for't.'' --- ``By this light, \emph{no more} I will \emph{not}''\hfill(\href{https://archive.org/details/bim_eighteenth-century_epicne-or-the-silent-_jonson-ben_1776/page/n27/mode/2up?q=%22word+to+her%22&view=theater}{Jonson, \textit{Epicœne} 3.182}) % Restored " to-day for't". Two speakers, not one; therefore adding quotation marks and dash.
\z

\ea \label{ex:10-88}
\emph{Nor more} you wouldn't!\hfill(\href{https://archive.org/details/personalhistory05dickgoog/page/n63/mode/2up?q=%22nor+more%22&view=theater}{Dickens, \textit{David} 132} (vulgar))
\z

\addsec{n A nc\textsuperscript{1} B nc\textsuperscript{2}}
\is{connectives!supplementary}
This differs from type 5
% "type 5"? "type 10.5"? "the previous type"?
in having a supplementary connective placed after B. 

\textit{Nor} with subsequent (\textit{nother} or) \textit{neither}: (\ref{ex:10-89}).

\ea \label{ex:10-89}
\ea
nor so nother\hfill(\href{https://archive.org/details/utopiasirthomas00robigoog/page/n307/mode/2up?q=%22nor+so+nother%22&view=theater}{More, \textit{Utopia} 197})
\ex
``It is not for your health. {\dots} '' --- ``Nor for yours neither''\\\hfill(\href{https://internetshakespeare.uvic.ca/doc/JC_F1/scene/2.1/index.html#tln-875}{Shakespeare, \textit{Cæs} 2.1.327})
\ex
Loue no man in good earnest, nor no further in sport neyther\\\hfill(\href{https://internetshakespeare.uvic.ca/doc/AYL_F1/scene/1.2/index.html#tln-195}{Shakespeare, \textit{As} 1.2.31})
\ex
it stops but one breach of licence, nor that neither\\ \hfill(\href{https://archive.org/details/areopagitica00miltuoft/page/34/mode/2up?q=%22it+stops+but+one%22&view=theater}{Milton, \textit{Areopagitica} 34})
\ex
nor I don't know her if I see her; nor you neither\hfill(\href{https://archive.org/details/in.ernet.dli.2015.219151/page/n207/mode/2up?q=%22know+her+if%22&view=theater}{Congreve, \textit{Love} 4.1}) % "do not" corrected to "don't"
\ex
I can know nothing, nor themselves neither.\hfill(\href{https://archive.org/details/journaltostellae00swifuoft/page/364/mode/2up?q=%22can+know+nothing%22&view=theater}{J. Swift, \textit{Journal} 364})
\ex
I could not keep the toad from drinking himself, nor he would not let me go neither, nor Masham, who was with us.\hfill(\href{https://archive.org/details/journaltostellae00swifuoft/page/130/mode/2up?q=%22toad+from+drinking%22&view=theater}{ibid 130})
\z
\z

\addsec{n A n B nc}
\is{adjuncts, connective|(}
\is{afterthought words|(}
Here the connexion between the two negative ideas is not thought of until both have been fully expressed, and \textit{neither} comes as an afterthought at the very last. Examples: (\ref{ex:10-96}).

\ea \label{ex:10-96}
\ea
``Nay it makes nothing sir.'' --- ``If it marre nothing neither, The treason and you goe in peace away together.''\hfill(\href{https://internetshakespeare.uvic.ca/doc/LLL_F1/scene/4.3/index.html#tln-1530}{Shakespeare, \textit{LLL} 4.3.191}) % "Nay" restored; quotation marks added to show that there are two speakers
\ex
I'll not spend beyond it. {\dots} I'll ne're run in debt neither.\\\hfill(\href{https://ia800900.us.archive.org/29/items/compleatenglishg00deforich/compleatenglishg00deforich.pdf}{Defoe, \textit{Gentleman} 66}) % OJ omits "No! No! I abhorr to be dun'd." (which I (PE) find rather charming); dots added
\ex
They would not eat themselves, and would not let others eat neither.\\\hfill(\href{https://archive.org/details/fartheradventure00defo/page/44/mode/2up?q=%22would+not+eat%22&view=theater}{Defoe, \textit{Farther} 47})
\ex
they were too full of apprehensions of danger, to venture to go to sleep, though they could not tell what the danger was they had to fear neither\hfill(\href{https://archive.org/details/lifeandstranges00dobsgoog/page/n337/mode/2up?q=%22they+were+too+full+of%22&view=theater}{Defoe, \textit{Robinson} 312}) % OJ specifies the page number but doesn't provide the actual example.
\ex
To which the other making no answer. {\dots} Allworthy made no answer to this neither\hfill(\href{https://archive.org/details/bim_eighteenth-century_the-history-of-tom-jones_fielding-henry_1768_4/page/302/mode/2up?q=%22to+which+the+other%22&view=theater}{Fielding, \textit{Tom} 4.302})
\ex
blush not. {\dots} and do not laugh neither\hfill(\href{https://archive.org/details/scottsivanhoeedi0000amar/page/590/mode/2up?q=%22and+do+not+laugh%22&view=theater}{Scott, \textit{Ivanhoe} 481})
\ex
I hope, sister, things are not so very bad with you neither\\\hfill(\href{https://archive.org/details/mansfieldpark00aust_1/page/22/mode/2up?q=%22things+are+not+so+very+bad%22&view=theater}{Austen, \textit{Mansfield} 25}) % "sister" restored
\ex
I had no companions to quarrel with, neither\hfill(\href{https://archive.org/details/praeterita01rusk/page/52/mode/2up?view=theater&q=%22companions+to+quarrel%22}{Ruskin, \textit{Præterita} 1.53})
\ex
Fifteen feet thick, of not flowing, but flying water; not water, neither,---melted glacier rather\hfill(\href{https://archive.org/details/in.ernet.dli.2015.95622/page/n141/mode/2up?q=%22fifteen+feet+thick%22&view=theater}{ibid 2.130}; frequent in Ruskin,\\\hfill e.g. \href{https://archive.org/details/in.ernet.dli.2015.95622/page/n301/mode/2up?q=%22hearing+no+word%22&view=theater}{\textit{Præterita} 2.288}, \textit{Selections} 1.206, \href{https://archive.org/details/crownofwildolive00ruskiala/page/122/mode/2up?view=theater&q=%22not+those%22}{\textit{Crown} 201}) % In “Selections from the Writings of John Ruskin” (a single-volume cheap edition from 1861, which OJ certainly wasn’t using), this would be illustrated by “not a perfect point neither” https://archive.org/details/selectionsfromwr00rusk/page/60/mode/2up?q=neither&view=theater ; or “no dream, neither” https://archive.org/details/selectionsfromwr00rusk/page/382/mode/2up?q=neither&view=theater . ??S We need to look in the two (or more) -volume “Selections” of 1893.

\ex I did not come to recommend myself {\dots} and perhaps Miss Carew might not think it any great recommendation neither.\\\hfill(\href{https://archive.org/details/cashelbyronsprof00shawuoft/page/n173/mode/2up?q=%22come+to+recommend+myself%22&view=theater}{Shaw, \textit{Cashel} 147}) % "Carew" expanded as Shaw wrote it; "perhaps" restored; omission noted with dots
\z
\z

\is{repeated negation, taught avoidance of}
Instead of the afterthought-\textit{neither} which we have now seen so frequently in this chapter, most people now prefer \textit{either}, which seems to have come into use in the 19th century, probably through the war waged at schools against double negatives. Examples after negative expressions: (\ref{ex:10-106}).

\ea \label{ex:10-106}
\ea thy sex cannot help that either\hfill(\href{https://archive.org/details/chroniclesofcano03scot/page/240/mode/2up?view=theater&q=%22thy+sex+cannot%22}{Scott, \textit{Chronicles}}; via \href{https://archive.org/details/oed03arch/page/n819/mode/2up?view=theater}{\textit{NED}, \textit{Either} 5~b}) % provenance details added

\ex I {\dots} am unmoved by men's blame or their praise either.\\\hfill(\href{https://archive.org/details/cu31924013442631/page/132/mode/2up?q=%22am+unmoved+by%22&view=theater}{R. Browning, \textit{Andrea}})
\ex Poor chap, he had little enough to be cheery over either.\\\hfill(\href{https://archive.org/details/cu31924013342781/page/n195/mode/2up?q=%22cheery+over+either%22&view=theater}{Doyle, \textit{Letters} 180})
\ex Maud, tell the boy he needn't wait. You needn't either unless you like.\\\hfill(\href{https://archive.org/details/dododetailofday00bensuoft/page/12/mode/2up?view=theater&q=%22tell+the+boy%22}{E. F. Benson, \textit{Dodo} 10}) % OJ has "need not"; in the (different) edition I (PE) looked at, it's "needn't"
\z
\z

After a positive expression \textit{either} is used as an afterthought adverb to emphasize the existence of alternatives; the \textit{NED} has an example in \href{https://archive.org/details/oed03arch/page/n819/mode/2up?view=theater}{\textit{Either} 5~a} from about 1400; % Peter: Shall we just integrate into the main text what's now a footnote?
%Brett: done
Shakespeare has it once only: (\ref{ex:10-110}). Cf. also (\ref{ex:10-111}).

\ea \label{ex:10-110} ``Wilt thou set thy foote o'my necke?'' --- ``Or o'mine either?''\hfill(\href{https://internetshakespeare.uvic.ca/doc/TN_F1/scene/2.5/index.html#tln-1190}{\textit{Tw} 2.6.206})
\z

\ea \label{ex:10-111}
\ea ``A beautiful figure for a nutcracker; {\dots}'' --- ``Not ugly enough,'' said Tackleton. --- ``Or for a firebox, either'' {\dots} \hfill(\href{https://archive.org/details/cricketonhearthf00dickrich/page/46/mode/2up?q=%22beautiful+figure%22&view=theater}{Dickens, \textit{Cricket}}) % OJ (or his proximate source) cut this so ruthlessly as to imply that the speaker was remarking on a beautiful figure for a firebox -- but that's not what was said.
\ex Ah, if all my priests were but like them! or my people either!\\\hfill(\href{https://archive.org/details/hypatia00kinggoog/page/n342/mode/2up?q=%22if+all+my+priests%22&view=theater}{Kingsley, \textit{Hypatia} 274}) % not a ";" but a "!"
\z
\z

As this use after a positive expression is much older than that after a negative, \citet[\href{https://archive.org/details/p2englischephilo01storuoft/page/698/mode/2up?view=theater}{698}]{storm1896englische} cannot be right in believing that the former is ``übertragen'' (`carried over') from the latter.

\bigskip
It should be noted that we have very frequently sentences connecting with previous \textit{positive} sentences in the same ways as we have seen in the last three types: n A nc B; n A nc\textsuperscript{1} B nc\textsuperscript{2}; n A n B nc (pp. \pageref{para:143}--\pageref{para:147})\label{para:147} with negative ones. This generally serves to point out a contrast, but sometimes the logical connexion between the two sentences is very weak, and the final \textit{neither} then merely ``clinches the argument'' by making the negative very emphatic. In (\ref{ex:10-113}), we have two illustrations corresponding to types  n A nc B and n A n B nc.

\il{English!nor@\textit{nor}}
\ea \label{ex:10-113} Speake the speech as I pronounc'd it. {\dots} But if you mouth it, as many of your players do, I had as liue the town-cryer had spoke my lines: \emph{Nor} do not saw the ayre with your hand thus. {\dots} Be not too tame \emph{neyther}.\\\hfill(\href{https://internetshakespeare.uvic.ca/doc/Ham_F1/scene/3.2/index.html}{Shakespeare, \textit{Hml} 3.2.4ff})
\z

Other examples: (\ref{ex:10-114}). Cf. also the frequent literary formulas of transition \textit{Nor is this all} and \textit{Nor do we stop here}.

\ea \label{ex:10-114}
\ea
I sawe Marke Antony offer him a crowne; yet 'twas not a crowne neyther, 'twas one of these coronets\hfill(\href{https://internetshakespeare.uvic.ca/doc/JC_F1/scene/1.2/index.html#tln-335}{Shakespeare, \textit{Cæs} 1.2.233})
\ex
``The best thing I ever read is not yours, but Dr. Swift's on Vanbrugh''; which I do not reckon so very good neither.\hfill(\href{https://archive.org/details/journaltostellae00swifuoft/page/66/mode/2up?q=%22which+I+do+not+reckon+so+very+good%22&view=theater}{J. Swift, \textit{Journal} 66}) % Punctuation added, words added, to accord with the book.
\ex
There, I say, get you gone; no, I will not push you neither, but hand you on one side\hfill(\href{https://archive.org/details/journaltostellae00swifuoft/page/120/mode/2up?q=%22will+not+push+you+neither%22&view=theater}{ibid 121})
\ex
I resolv'd to run quite away from him. However, I did not act so hastily neither as my first heat of resolution prompted\\\hfill(\href{https://archive.org/details/lifeandstranges00dobsgoog/page/n31/mode/2up?q=%22run+quite+away%22&view=theater}{Defoe, \textit{Robinson} 5}) 
\ex
I travelled among unknown men, In lands beyond the sea; Nor, England! did I know till then What love I bore to thee.\\\hfill(\href{https://archive.org/details/poemsofwilliamwo01word/page/78/mode/2up?q=%22What+love+I+bore+to+thee%22&view=theater}{Wordsworth, \textit{Travelled}})
\z
\z
\is{adjuncts, connective|)}
\is{afterthought words|)}

\label{another-way-of-using-it}While this use of \textit{nor} is perfectly natural, there is another way of using it which is never found in prose though it is a favourite formula with some poets. \textit{Nor} here connects not two complete sentences, but only two verbs, of which the first is to be taken in a positive sense (cf. \cite[\href{https://archive.org/details/tennysonssprache00dybouoft/page/2/mode/2up?view=theater}{2}]{dyboski1907tennysons}). Thus (\ref{ex:10-119}).

\ea \label{ex:10-119}
\ea
Ida stood nor spoke \phantom{x} (`she stood and did not speak, she stood without speaking')\hfill(\href{https://en.m.wikisource.org/wiki/The_Princess;_a_medley/Canto_6}{Tennyson, \textit{Princess}})
\ex
He that gain'd a hundred fights, Nor ever [`and never'] lost an English gun\hfill(\href{https://en.wikisource.org/wiki/Maud,_and_other_poems/Ode_on_the_Death_of_the_Duke_of_Wellington}{Tennyson, \textit{Wellington}})
\ex
it concerns you that your knaves Pick up a manner nor discredit you \phantom{x} (`and (do) not')\hfill(\href{https://archive.org/details/cu31924013442631/page/124/mode/2up?q=%22concerns+you+that+your%22&view=theater}{R. Browning, \textit{Lippo}})
\ex
things we have passed Perhaps a hundred times nor cared to see\\\hfill(\href{https://archive.org/details/cu31924013442631/page/128/mode/2up?q=%22we+have+passed%22&view=theater}{ibid})
\ex
wait death nor be afraid!\hfill(\href{https://archive.org/details/cu31924013442631/page/638/mode/2up?q=%22wait+death%22&view=theater}{R. Browning, \textit{Rabbi} 19})
\z
\z

These instances may be compared with the Old Norse quotations given by \citet[\href{http://www.jstor.org/stable/40846532}{10}]{neckel1912germanischen}: (\ref{ex:10-124}), etc.

\ea \label{ex:10-124}
 \ea
 \gll sat hann, né hann svaf, ávalt\\
 sat he nor he slept always\\
 \glt `he sat and did not sleep'
 \ex
 \gll gumnum hollr, né gulli\\
 to-men loyal nor to-gold\\
 \glt `loyal to men, but not to gold'
 \z
\z

The negative connectives \textit{neither} and \textit{nor}, which we have treated in this chapter, are characteristic elements of idiomatic English; thus \textit{nor do I see any reason} is always preferred to \textit{and I see also no reason} (cf. the cause of this, above p.~\pageref{join-to-first}). In some few cases, however, we find \textit{also} in a negative sentence, but there is generally some special reason for its use, as in (\ref{ex:10-125}). % \footnote{Jespersen oddly fails to comment on the slightly reordered pattern \textit{and I also see no reason}. This was idiomatic at the time he was writing. \eds} % Erstwhile footnote commented out, because I'm neither happy with it nor happy without it. (I'll be interested to see if anyone who isn't privy to this LaTeX file [and so reads this comment of mine] asks "Didn't OJ mean 'I also see no reason'? But if he did, wasn't he plain wrong?"
In (\ref{ex:10-rooted}) the contrast is expressed more elegantly than in: \textit{but neither is any one rooted}.

\ea \label{ex:10-125}
\ea
But I must \emph{also not} forget that {\dots} \\(`also remember')\hfill(\href{https://archive.org/details/journalofplaguey1881defo/page/44/mode/2up?view=theater&q=%22must+also+not+forget%22}{Defoe, \textit{Journal} 44})
\ex
but then too was there \emph{not also} a national virtue? \\(`wasn't there a national virtue besides')\hfill(\href{https://archive.org/details/mrbritlingseesi02unkngoog/page/n130/mode/2up?view=theater&q=%22then+too+was+there%22}{Wells, \textit{Britling} 117})
\ex
``Everything may recover,'' he admitted. ``But \emph{also nothing} may recover'' \phantom{x} (\textit{also} meaning `there is another possibility')\hfill(\href{https://archive.org/details/mrbritlingseesi02unkngoog/page/n210/mode/2up?view=theater&q=%22everything+may+recover%22}{ibid 194}) % "he admitted" restored
\ex \label{ex:10-rooted}
No one is tied, but \emph{also no one} is rooted \\(`but on the other hand, no one')\hfill(\href{https://archive.org/details/fromjohnchinaman00dickiala/page/6/mode/2up?q=%22no+one+is+tied%22&view=theater}{Dickinson, \textit{Letters} 6})
\z
\z


In rare instances, a negative is put only with one of two (or three) verbs though it belongs to both (or all): (\ref{ex:10-129}).

\ea \label{ex:10-129}
He sette nat his benefice to hyre, And leet\footnote{\citet{skeat1900chaucer} explains \textit{and leet} as `and left (not)', adding ``We should now say---`\textit{Nor} left.' {\dots} \textit{Leet} is the [past tense] of \textit{leten}, to let alone, let go''. \eds} % Peter: (i) I've deleted the reference QUOTE (\href{https://archive.org/details/completeworksedi05chauuoft/completeworksedi05chauuoft}{46}) UNQUOTE from the end of this footnote. (ii) I've elaborated quite a bit of OJ's cryptic reference to Skeat.}
%Brett: OK
 his sheep encombred in the myre, And ran to London {\dots} But dwelte at hoom\\\hfill(\href{https://archive.org/details/completeworksofg04chauuoft/completeworksofg04chauuoft/page/14/mode/2up?q=%22He+sette+nat%22&view=theater}{Chaucer, \textit{Prologue} A~507})
\ex
Didst thou not write thy name in thine own blood and drewst the formall deed\hfill(\href{https://archive.org/details/merrydeviledmon01shakgoog/page/n29/mode/2up?q=%22Didst+thou+not%22&view=theater}{\textit{Devill} 524}) % The edition that's linked to modernizes the spelling throughout, which I (PE) suppose is the reason for "formal". Let's look for an edition with the original spelling.
%Brett: done
\ex
The winds play no longer and sing in the leaves\\ (`no longer play and sing')\hfill(\href{https://archive.org/details/poemsofwilliamco00cowp/page/376/mode/2up?q=%22winds+play+no+longer%22&view=theater}{Cowper, \textit{Poplar}})
\z

A frequent way of making one \textit{not} serve to negative two verbs is seen in \textit{The winds \textsc{do not play and sing} in the leaves} (`{\dots} are not playing and singing {\dots}'). 

\il{Danish!ikke@\textit{ikke}}
In Danish, \textit{ikke} sometimes is put only with the last of two verbs connected by means of \textit{og}, but only when their signification is closely related, as in (\ref{ex:10-132}). Otherwise \textit{ikke} has to be repeated (\ref{ex:10-133}). But if the first verb indicates only a more or less insignificant state or circumstance of the main action denoted by the second verb, \textit{ikke} is put with the first verb: (\ref{ex:10-134}). The explanation is that \textit{og} in this case is a disguised \textit{at}, originally followed by the infinitive, see \citet[\href{https://archive.org/details/dania_202407/page/166/mode/2up?view=theater}{167ff}]{jespersen1896dare} and \citet[\href{https://archive.org/details/dania_202407/page/248/mode/2up?view=theater}{249ff}]{siesbye1896bemerkninger}.

\ea\label{ex:10-132}
\gll jeg hykler og lyver ikke\\
 I pretend and lie not\\
\glt `I neither pretend nor lie'
\hfill(\href{https://archive.org/details/mgoldschmidtspo00unkngoog/page/n74/mode/2up?q=%22jeg+hykler+og+lyver+ikke%22&view=theater}{Goldschmidt, \textit{Levi} 8.60})
\z



\ea \label{ex:10-133}
\gll han spillede ikke klaver og sang ikke (heller)\\
 he played not piano and sang not either\\
\glt `he neither played the piano nor sang' % OJ provides no indication of where this comes from
\z

\ea \label{ex:10-134}
\ea
\gll sid ikke der og sov\\
 sit not there and sleep\\
\glt `do not sleep there'
%% SG: Or 'do not sleep there' – the verb 'sit' in this construction is, as OJ says, "a more or less insignificant state or circumstance"
\ex
\gll jeg går ikke hen og glemmer det\\
 I go not there and forget it\\
\glt `I won't forget it'
%% SG: The "go there" construction is an idiomatic way to express future/predictive meaning
\z
\z

\is{vividness of negation, cross-language}
\il{French!et@\textit{et}}
Where a positive and a negative sentence are combined, English uses the adversative conjunction \textit{but} (like Danish \textit{men}, German \textit{aber}), whereas French prefers \textit{et} (\ref{ex:10-136}). Negation thus is more vividly present in an English consciousness than in a French mind, since the combination of positive and negative is always felt as a contrast.

\ea \label{ex:10-136}
 \ea
 \ea I eat, but I don't drink
 \ex  the guard dies, but does not surrender 
 \z
\ex
 \ea je mange, et je ne bois pas
 \ex
 la garde meurt et ne se rend pas 
 \z
 \z
\z
\is{conjoined clauses|)}
\is{connectives!negative|)}
\is{coordinated clauses|)}
\is{neither...nor@`neither...nor', expressions corresponding to|)}
