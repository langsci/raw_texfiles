\ChapterAndMark{Double Negation} 
\label{ch:7}
\is{double negation|(}
\is{double negation!for affirmation|(}

\is{double negation!law of}
When logicians insist that ``two negatives make an affirmative'' their rule is not corroborated by actual usage in most languages. But it would be wrong to divide languages into some that follow this rule and others that do not, for on closer inspection we find that in spite of great differences between languages in this respect there are certain underlying principles that hold good for all languages. We shall deal first with those instances in which the rule of the logicians is observed; and afterwards with those in which the final result of two negatives is in itself negative.

First, it seems to be a universal rule in all languages that \textit{two negatives make an affirmative}, if both are special negatives attached to the same word; this generally happens in this way that \textit{not} is placed before some word of negative import or containing a negative prefix. But it should be noted that the double negative always modifies the idea, for the result of the whole expression is somewhat different from the simple idea expressed positively. Thus \textit{not without some doubt} is not exactly the same thing as \textit{with some doubt}; \textit{not uncommon} is weaker than \textit{common}, and \textit{not unhandsome} (\href{https://archive.org/details/lightthatfailed0000rudy_q6g8/page/258/mode/2up?q=%22for+he+was+not%22&view=theater}{Kipling, \textit{Light} 246}) than \textit{handsome}, the psychological reason being that the \textit{detour} through the two mutually destroying negatives weakens the \is{mental energy for comprehension}mental energy of the hearer and implies on the part of the speaker a certain hesitation absent from the blunt, outspoken \textit{common} or \textit{handsome}, as in \refp{ex:07-01}. Assertion by negative of opposite is a common feature of English as spoken in Ireland (see \cite[\href{https://archive.org/details/englishaswespeak00joycuoft/page/15/mode/2up?view=theater}{16}]{joyce1910english}):
\textit{this little rasher will do you no harm} meaning it will do you good; \textit{Paddy Walsh is no chicken now} meaning he is very old, etc. This is really on a par with \textit{not untragical}, \textit{not unentitled to speak}, \textit{not unpromptly}, etc. which abound in Carlyle \citep[\href{https://archive.org/details/englischestudien06leipuoft/page/388/mode/2up?view=theater}{388}]{krummacher1883notizen}; with him \textit{not without} has become quite a mannerism for which he is taken to task by Sterling (\href{https://archive.org/details/lifeofjohnsterli00carliala/page/146/mode/2up?q=%22not+without%22&view=theater}{letter}): \textit{not without ferocity}, \textit{not without result}, \textit{not without meditation}, etc. etc.

\ea \label{ex:07-01}
Tis not vnknowne to you, Anthonio \phantom{x} (`you are to some extent aware')\\\hfill(\href{https://internetshakespeare.uvic.ca/doc/MV_F1/scene/index.html#tln-130}{Shakespeare, \textit{Merch} 1.1.122})
\z

\is{indirect negation}
A special instance of this detour is Latin \il{Latin!non@\textit{non}}\textit{non-nunquam} (`not never'), \textit{non-nulli} (`not none'), on the meaning of which see p.~\pageref{08-non-nulli} (\chapref{ch:8}). % PE: OJ merely says chapter VIII, with no page number.

Next, the result is positive if we have a nexal negative in a sentence containing an implied negative, as in \textit{I do not deny}; this, of course, closely resembles the first case. Here belong such frequent French phrases as \textit{il n'était pas sans être frappé par la différence}; the meaning of the round-about expression is `you will readily understand that he was struck {\dots}'.

In this place should, perhaps, be mentioned the French \textit{il n'y a pas que ça} (`there is not only this'), which means the opposite of \textit{il n'y a que ça}, thus `there is more than this'. 

The negation of words like \textit{nobody} resulting in the meaning of `everybody' (\textit{nemo non videt}) will be treated in \chapref{ch:8} (p.~\pageref{08-negativing-nobody}f). % PE: OJ merely says chapter VIII.

Yet another way of affirming through a double negative is seen in \refp{ex:07-02}. But this hardly belongs in this chapter.

\ea \label{ex:07-02}
\ea For I am nothing, if not criticall\hfill(\href{https://internetshakespeare.uvic.ca/doc/Oth_F1/scene/2.1/index.html#tln-890}{Shakespeare, \textit{Oth} 2.1.120}) 

\ex The old Scots poets {\dots} were nothing if not plain-spoken\newline (`were plain-spoken to a high degree')\hfill(\href{https://archive.org/details/poetryrobertbur02burngoog/page/296/mode/2up?q=%22plain-spoken%22&view=theater}{Henley, \textit{Burns} 3.297}) % OJ ascribes this to "Henderson Burns". But Henley is the primary author of the book that this appears in (and it doesn't appear in Henderson's own biography of Burns).
\z
\z
\is{double negation!for affirmation|)}

\bigskip
\is{double negation!for negation}

If now we proceed to those cases in which \textit{a repeated negation means, not an affirmative, but a negative}, we shall do well to separate different classes in which the psychological explanation is not exactly the same.

\is{attraction!double|(}
\is{cumulative negation|(}
\AddSubSection{Double attraction}
In the first place we have instances of \textsc{double attraction}. % ??? PE: Should this be in small caps rather than italics?
Above we have seen the two tendencies, one to place the negative with the verb as nexal negative, and the other to amalgamate a negative element with some word capable of receiving a negative prefix. We have seen how now one, now the other of these tendencies prevails; but here we have to deal with those instances in which both are satisfied at once in popular speech, the result being sentences with double, or even treble or quadruple, negation.

\is{adverbs!negative|(}\il{English!Old English!ne@\textit{ne}|(}This was the regular idiom in Old English, so regular indeed that in the whole of \href{https://archive.org/details/anglosaxonversi00thorgoog}{\textit{Apollonius}} there is only one sentence containing \textit{ne} with the verb in which we have another word that might take \il{English!Old English!n-@\textit{n-}|(}\textit{n-} and does not---\textit{ne ondræt þu ðe æniges þinges} (`do not fear anything') (\href{https://archive.org/details/anglosaxonversi00thorgoog/page/n32/mode/2up?q=%22ne+ondrset+fu+t%24e+seniges%22&view=theater}{Appolonius: 22})%
---while there are nine instances of \textit{ne} + various forms of \il{English!Old English!nan@\textit{nan}}\textit{nan}, three of \textit{ne} + \il{English!Old English!naht@\textit{naht}}\textit{naht} (`nothing' or `not') and fifteen of \textit{ne} + some negative adverb beginning with \textit{n-} (\il{English!Old English!nahwar@\textit{nahwar}}\textit{nahwar} `nowhere', \il{English!Old English!naefre@\textit{næfre}}\textit{næfre} `never', \il{English!Old English!na@\textit{na}}\textit{na} `not', \il{English!Old English!nather@\textit{naðer}}\textit{naðer} `neither'). There are 40 instances of \textit{ne} or \textit{n-} with the verb without any other word that might take \mbox{\textit{n-}\il{English!Old English!n-@\textit{n-}},} and four of \textit{na} as special negative without any verb. In this text there are no instances of treble or quadruple negation, but these are by no means rare in Old English prose, as in  \refp{ex:07-03}. In the same way in Middle English, e.g. \refp{ex:07-05}.

\ea \label{ex:07-03}
\ea 
\gll \emph{n}an man \emph{n}yste \emph{n}an þing\\
no man {not knew}  no thing\\
\glt `no one new anything'

\ex \label{ex:07-04}
\gll \emph{n}e \emph{n}an neat \emph{n}yste nænne andan \emph{n}e \emph{n}ænne ege to oðrum\\
 not no animal {not knew} no malice nor no fear to others\\
\glt `no creature knew any malice nor any fear towards others'\\\hfill(\href{https://archive.org/details/analectaanglosax68thor/page/94/mode/2up?q=%22ne+nan+neat+nyste%22&view=theater}{Boethius, \textit{Orpheus}}) % What can "102. 7" refer to?
%Brett: I assume 102 refers to the page number in a particular edition of the Old English Boethius. 7 probably indicates the line number on that page.
%PE Yes, I'm sure that you're right. But unless (until?) we can point to it, I think "102.7" is unhelpful and therefore am changing this to "Orpheus"
\z\il{English!Old English!ne@\textit{ne}|)}
\z

\ea \label{ex:07-05}
\ea \il{English!Middle English!ne@\textit{ne}}
\gll He \emph{n}euere yet \emph{n}o vileynye \emph{n}e seyde In al his lyf unto \emph{n}o maner wight\\
 He never yet no villainy nor said in all his life unto no {manner of} person\\
\glt `He never yet spoke any villainy in all his life to any kind of person'\\\hfill(\href{https://archive.org/details/prologuefr89west00chau/page/2/mode/2up?view=theater&q=%22vileynye%22}{Chaucer, \textit{Prologue} A~70}) % Skeat's spelling is different but I (PE) haven't adjusted OJ's spelling
% ??? PE: Sorry but I don't understand "speak villainy".
\ex \il{English!Middle English!ne@\textit{ne}}
\gll \emph{n}e takeþ \emph{n}oþing to holde of \emph{n}oman \emph{n}e of \emph{n}o womman, \emph{n}e \emph{n}oiþer of þe seruauntz \emph{n}e bere non uncouþ tales\\
 nor takes anything to hold of {no man} nor of no woman, nor neither of the servants nor bear any uncouth tales\\
\glt `Nor takes anything to consider from no man nor woman, nor from the servants, nor to bear any uncouth tales'\hfill(\href{https://archive.org/details/englische-studien_1902_30/page/344/mode/2up?view=theater}{\textit{Recluse} 200}) % OJ's "th" changed to "þ" in a couple of places; a "V" changed to "U", in accordance with Paues
% ??? The translation is alien to me. I can make a variety of guesses about what it might mean, but I can't choose among them.
\z
\z
\is{adverbs!negative|)}

Early Modern English examples of double negation: \refp{ex:07-07}.

\ea \label{ex:07-07}
\ea the prouostis harneis was hole, and \emph{n}ought dammaged of \emph{n}othyng\\\hfill(\href{https://archive.org/details/caxtonsblanchard0000leon/page/48/mode/2up?q=%22harneis+was+hole%22&view=theater}{Caxton, \textit{Blanchardyn} 48}) % "prouostis" (="provost's"?) restored
\ex whan he coude \emph{n}owher \emph{n}one see\hfill(\href{https://archive.org/details/TheHistoryOfReynardTheFoxArber/page/n67/mode/2up?q=%22whan+he+coude%22&view=theater}{Caxton, \textit{Reynard} 38})
\ex \emph{n}e \emph{n}euer shal \emph{n}one be born fairer than she\hfill(\href{https://archive.org/details/TheHistoryOfReynardTheFoxArber/page/n113/mode/2up?q=%22neuer+shal+none%22&view=theater}{ibid 84})
\ex they \emph{n}euer make \emph{n}one [i.e. `no leagues'] with anye nacion\\\hfill(\href{https://archive.org/details/utopiasirthomas00robigoog/page/n349/mode/2up?q=%22they+neuer+make+none%22&view=theater}{More, \textit{Utopia} 238}) % PE: OJ had "\emph{n}one with anye nacion [none i.e. leagues]". In context, this meant "no league(s) with any nation". I've slightly reordered what OJ wrote.
\z
\z

In Elizabethan English this kind of repeated negation is comparatively rare; from Shakespeare I have only two instances (\ref{ex:07-11}; but I may, of course, have overlooked others). \citet{bogholm1906bacon} has one from Bacon (\href{https://archive.org/details/lettersandlifef08bacogoog/page/n219/mode/2up?q=%22never+no+violent%22&view=theater}{\textit{Observations}}): \textit{his lordship was \textsc{n}ever \textsc{n}o violent and transported man}. % "his lordship" and "and transported" restored

I cannot explain how it is that this particular redundancy seems to disappear for two centuries; it can hardly be accidental that I have no examples from the beginning of the seventeenth to the end of the eighteenth century, when \citet[\href{https://archive.org/details/anecdotesofengli00peggrich/page/80/mode/2up?q=luxuriance&view=theater}{80}]{pegge1814anecdotes} mentions this kind of ``luxuriance'' among the cockneys (\textit{I \textsc{don't} know \textsc{nothing} about it}) and says that he has heard in Yorkshire, \textit{No, I shall \textsc{not} do \textsc{no} such thing} and that a citizen is said to have enquired at a tavern, \textit{if \textsc{nobody} had seen \textsc{nothing} of \textsc{never-a} hat \textsc{nowhere's}?}

\ea \label{ex:07-11}
\ea I will \emph{not} budge for \emph{no} mans pleasure I\hfill(\href{https://internetshakespeare.uvic.ca/doc/Rom_F1/scene/3.1/index.html#tln-1485}{\textit{Rom} 3.1.58})
\ex I haue one heart, one bosome, and one truth, And that no woman has, \emph{n}or \emph{n}euer \emph{n}one Shall mistris be of it, saue I alone\hfill(\href{https://internetshakespeare.uvic.ca/doc/TN_F1/scene/3.1/index.html#tln-1370}{\textit{Tw} 3.1.171}) % OJ says "II. I" but actually 3.1/
\z
\z

Recent examples, put in the mouths of vulgar speakers (sometimes, no doubt, with some exaggeration of a tendency ridiculed at school, however natural in itself): \refp{ex:07-12}.\is{repeated negation, taught avoidance of|(}

\ea \label{ex:07-12}
\ea \emph{Nobody} \emph{never} went and hinted \emph{no} such a thing, said Peggotty\\\hfill(\href{https://archive.org/details/personalhistory05dickgoog/page/n15/mode/2up?q=%22went+and+hinted%22&view=theater}{Dickens, \textit{David} 19})
\ex all he [the butler] hopes is, he may \emph{never} hear of \emph{no} foreigner \emph{never} boning \emph{nothing} out of \emph{no} travelling chariot\hfill(\href{https://archive.org/details/dombeyson00dick_0/page/440/mode/2up?q=%22All+he+hopes+is%22&view=theater}{Dickens, \textit{Dombey} 279})
\ex We \emph{never} thought of \emph{nothing} wrong\hfill(\href{https://archive.org/details/dli.ministry.14127/page/493/mode/2up?q=%22never+thought%22&view=theater}{Thackeray, \textit{Pendennis} 3.85})
\ex there was \emph{niver} \emph{nobody} else gen [`gave'] me \emph{nothin'}\hfill(\href{https://archive.org/details/millonfloss0009geor/page/254/mode/2up?q=%22niver+nobody%27%22&view=theater}{Eliot, \textit{Mill} 1.327})
\ex I \emph{can't} do \emph{nothing} without my staff\hfill(\href{https://archive.org/details/dli.bengal.10689.8131/page/n31/mode/2up?q=%22nothing+without+my+staff%22&view=theater}{Hardy, \textit{Wessex} 23})
\ex you \emph{wont} care to spar with \emph{nobody} without youre well paid for it\\\hfill(\href{https://archive.org/details/cashelbyronsprof00shawuoft/page/n51/mode/2up?q=%22you+wont+care+to%22&view=theater}{Shaw, \textit{Cashel} 24}) % OJ's "like" corrected to GBS's "care"
\ex \emph{No} compensation \emph{nowhere} for being cut off innocent in the pride of youth and strength!\hfill(\href{https://archive.org/details/greywigstoriesno00zang/page/290/mode/2up?q=%22compensation+nowhere%22&view=theater}{Zangwill, \textit{Mystery} 209}) % Sentence completed.
\ex you \emph{won't} lose \emph{nothing} by it\hfill(\href{https://archive.org/details/memoirsofamerica0000robe/page/86/mode/2up?view=theater&q=%22nothing+by+it%22}{Herrick, \textit{Memoirs} 87})
\ex there \emph{won't} be \emph{no} hung jury\hfill(\href{https://archive.org/details/memoirsofamerica0000robe/page/88/mode/2up?view=theater&q=%22no+hung+jury%22}{ibid 89})
\z
\z

Cumulative negation exactly resembling that of Old English was very frequent in Middle High German, for example \refp{ex:07-21}.

\ea \label{ex:07-21}
\ea\il{German!Middle High German!en@\textit{en}|(}
\gll diz \emph{en}-mac nu \emph{nieman} bewarn\\ % "diz enmac nu nieman bewarn" (From Tristan IV. Die Entführung. https://www.hs-augsburg.de/~harsch/germanica/Chronologie/13Jh/Gottfried/got_tr04.html )
 this not-may now no-one prevent\\
\glt `now no one can prevent this'\hfill(\href{https://archive.org/details/germanische-syntax-i-zu-den-negativen-s/page/n13/mode/2up?q=%22nieman+bewarn%22&view=theater}{Delbrück, \textit{Syntax} 6}) 
\ex 
\gll \emph{nu} en-kan ich \emph{niemanne} gesagen\\
 now not-can I no-one say\\
\glt `now I cannot say [it] to anyone'\hfill(\href{https://archive.org/details/germanische-syntax-i-zu-den-negativen-s/page/n13/mode/2up?q=%22nu+enkan+ich+niemanne+gesagen%22&view=theater}{ibid 6})
\ex 
\gll ir ougen diu \emph{en}-wurden \emph{nie} {\dots} naz\\ % Added dots. According to https://www.hs-augsburg.de/~harsch/germanica/Chronologie/13Jh/Gottfried/got_tr02.html and https://archive.org/details/zeitschriftfrde01litegoog/page/n106/mode/2up?q=%22ir+ougen%22 , "ir ougen diu enwurden nie in allem disem leide naz" (From Tristan, II. Riwalin und Blanscheflur.)
 your eyes those not-become never {} wet\\
\glt `your eyes have never become wet'\hfill(\href{https://archive.org/details/germanische-syntax-i-zu-den-negativen-s/page/n13/mode/2up?q=%22ir+ougen+diu%22&view=theater}{ibid 6})
\z
\z\il{German!Middle High German!en@\textit{en}|)}


This was continued in later centuries, though as in English it was counteracted by schoolmasters. Luther (\href{https://books.google.co.jp/books?id=jcZTAAAAcAAJ&pg=PA314&lpg=PA314&dq=%22Wir+sind+niemand+nichts+schuldig%22+luther&source=bl&ots=8BpFlEDTnL&sig=ACfU3U2OKDJrrof3mO6B1CveMSSGBNvuKA&hl=en&sa=X&ved=2ahUKEwjF55f1kp6HAxXrhq8BHToTCg4Q6AF6BAgNEAM#v=onepage&q=%22Wir%20sind%20niemand%20nichts%20schuldig%22%20luther%20&f=false}{\textit{Bücher} 314}) has \textit{Wir sind \textsc{niemand nichts} schuldig} (`We owe no one anything at all') and Goethe (\href{https://archive.org/details/goethesfaust00goetuoft/page/234/mode/2up?q=%22dass+er+an+nichts+keinen+anteil+nimmt%22&view=theater}{\textit{Faust}~I}) \textit{Man sieht, dass er an \textsc{nichts keinen} anteil nimmt} (`One sees that he takes no interest at all in anything'), Schiller (\href{https://archive.org/details/schillerswallen07schigoog/page/226/mode/2up?q=\%22nirgend\%22&view=theater}{\textit{Tod} 3.15}) \textit{alles ist partei und \textsc{nirgend kein} richter} (`Everything is partisan, and nowhere is there any judge at all'), etc. \citep[\href{https://archive.org/details/sprachgebrauchu00andrgoog/page/n229/mode/2up?view=theater&q=\%22keinen+anteil+nimmt\%22}{209}]{andresen1892sprachgebrauch}. This is particularly frequent in vulgar language. In \citet[\href{https://archive.org/details/unseremutterspr03weisgoog/page/n88/mode/2up?view=theater}{78}]{weise1896unsere} I find the following:


\begin{quote}
  
Die Verneinung wird nachdrücklich wiederholt, damit sie recht ins Gewicht fällt. In Angelys Fest der Handwerker wird einem Gesellen auf die Frage: ‘Hat keener Schwamm?’ nicht geantwortet; als er dann aber der Frage die Form giebt: ‘Hat denn \emph{keener} \emph{keinen} Schwamm \emph{nich}?’ findet er Gehör. Doch kann einer der Anwesenden seinen Unwillen darüber nicht zurückhalten, dass er nicht gleich ordentlich deutsch geredet habe.
 %The negation is emphatically repeated {so that} it really in weight falls. In Angely's Festival of Craftsmen, becomes one journeyman on the question: `Has {no one} {a sponge}?' Nobody answers. As he then but the question the form gives `has then {no one} no sponge not finds he {a hearing} but can one the present those unwilling {about it} not refrain that he doesn't even properly German spoken have\\ % Although OJ writes "als er aber dann", the original reads "als er dann aber".
 
 `Negation is emphatically repeated to make it impactful. In Angely's Festival of the Craftsmen, a question from a journeyman, ``Does nobody have a sponge?'' goes unanswered; but when he reformulates it as ``Ain't nobody got no sponge?'' he finds an audience. Yet one of those present could not conceal his annoyance that proper German was not used from the start.'
\end{quote} % I (PE) suggest NOT presenting this as a numbered example but instead presenting the German original (not in italics), followed by a gloss in English -- in the same way that, in chapter 5, we treat quotations from Vondrák, Krüger, and Kühner (all on p 53 of OJ's printed book)
%Brett: done
\is{repeated negation, taught avoidance of|)}

In Danish similar expressions are extremely rare. Leonora Christina % "Leonora Christina" is PE's emendation of OJ's "El. Christine". Perhaps "Christine" is OJ's anglicization, but "El." mystifies PE. (Feel free to change it back; but if so, then alter and move the bibliography entry.)
writes \refp{ex:07-25}.

\ea \label{ex:07-25}
\gll saa hand kiøbte \emph{aldrig} \emph{intet} for mig\\
 so he bought never nothing for me\\
\glt `so he never bought anything for me'\hfill(\href{https://www.gutenberg.org/cache/epub/41072/pg41072-images.html#Side_151}{132})
\z

In French, \textit{nul} with \il{French!ne@\textit{ne}}\textit{ne} to the verb (\textit{\textsc{nul ne} vient} `no one comes at all'; \textit{on \textsc{ne} le voit \textsc{nulle} part} `you don't see him anywhere at all') is a case in point, though now it is hardly felt to be different from the corresponding usage with \textit{aucun}, which was originally positive (`some'), but has now acquired negative force (`(not) any'), as we have seen above.

In Spanish, repeated negation is not at all rare; I may quote \refp{ex:07-26}.

\ea \label{ex:07-26}
\ea\il{Spanish!ni@\textit{ni}}\il{Spanish!sin@\textit{sin}}\il{Spanish!nadie@\textit{nadie}}
\gll Estarémos, \emph{sin} que \emph{nadie,} \emph{Ni} aun el mismo sol, hoy sepa De nosotras\\ % Not OJ's "no sepa De nosotros" but instead -- according to both the linked-to source and https://upload.wikimedia.org/wikipedia/commons/5/50/El_alcalde_de_Zalamea_-_drama_escrito_en_verso_%28IA_elalcaldedeza00cald%29.pdf -- "hoy sepa De nosotras". Linked-to source has "áun"; but https://archive.org/details/elgarrotemasbien00cald/page/6/mode/2up?q=%22que+nadie%22&view=theater has "aun".
 we-will-be \emph{without} that \emph{anyone} \emph{nor} even the same sun today knows of us\\
\glt `We will be, without anyone, or even the sun itself, knowing of us today'\hfill(\href{https://archive.org/details/AGuichot05012/page/30/mode/2up?q=%22Estar%C3%A9mos%22&view=theater}{Calderón, \textit{Alcalde} 1.10})
\ex \il{Spanish!nunca@\textit{nunca}}\il{Spanish!no@\textit{no}}
\gll Aquí \emph{no} vienen casi \emph{nunca} soldados\\
 here \emph{not} come almost \emph{never} soldiers\\
\glt `Soldiers almost never come here'\hfill(\href{https://archive.org/details/Donaperfecta/page/n27/mode/2up?q=%22aqui+no+vienen%22&view=theater}{Pérez Galdós, \textit{Doña} 23}) % "Casi" ('almost') restored
\z
\z

Thus also in Slavonic languages, \citet[\href{https://archive.org/details/grundrissderver01delbgoog/page/526/mode/2up?view=theater&q=\%22serbische+Beispiele\%22}{526}]{delbruck1897vergleichende} gives among the other instances Serbian \refp{ex:07-serbian-answer}. For Russian, in the first few pages of \citet[\href{https://archive.org/details/manuelpourltuded00boye/page/n13/mode/2up?view=theater}{3f}]{boyer1905manuel}, I find: \refp{ex:07-28}, etc. % The cited book of course uses Cyrillic script. I (PE) haven't attempted to find any of these examples within the book.
%Brett: found
% ??? PE: I've added "For Russian,". Why so? Because OJ's 1917 book says: "in the first few pages of Boyer et Speranski, Manuel de la langue russe, I find"; and thus readers would have immediately known that these were from Russian. Feel free to replace my hint "For Russian," with something better; but really, some hint is needed.

\ea  \label{ex:07-serbian-answer}\il{Serbian!\textit{nikto}}
\gll i nikto mu ne mogaše odgovoriti riječi \\
and nobody {to him} no could answer words \\
\glt `and nobody could say a word in reply'
\z % PE: This example was previously (till late Sep '24) embedded in the paragraph above: too awkwardly, I thought. OJ glossed it, but was one word short. I got Google Translate to gloss it and have replaced OJ's gloss with Google Translate's.

\ea \label{ex:07-28}\il{Russian!ne@\textit{ne}|(}
\ea
\gll i nikomú zla ne dělaem\\
 and {to no one} evil not do\\
\glt `and we do no evil to anyone'
\ex 
\gll ničegó ne berët\\
 nothing not takes\\
\glt `he takes nothing'
\ex 
\gll ne daváĭ že mužikú ničegó\\
 not give-\textsc{imp} \textsc{emph} {to peasant} nothing\\
\glt `don't give the peasant anything'
\ex 
\gll Filipók ničegó ne skazál\\
 Filipok nothing not said\\
\glt `Filipok said nothing'\il{Russian!ne@\textit{ne}|)}
\ex \il{Russian!net@\textit{nět}}
\gll na kryl’čé nikogó nět\\
 on porch {no one} isn't\\
\glt `there is no one on the porch'
\z
\z

In Greek, repeated negation is very frequent, see any grammar. \citet[\href{https://archive.org/details/syntaxofgreeklan00madvuoft/page/198/mode/2up?q="Avev+\%CF\%84\%CE\%BF\%CF\%8D\%CF\%84\%CE\%BF\%CF\%85+\%CE\%BF\%E1\%BD\%90\%CE\%B4\%CE\%B5\%E1\%BD\%B6\%CF\%82+\%CE\%B5\%E1\%BC\%B0\%CF\%82+\%CE\%BF\%E1\%BD\%90\%CE\%B4\%E1\%BD\%B2\%CE\%BD+\%CE\%BF\%E1\%BD\%90\%CE\%B4\%CE\%B5\%CE\%BD\%E1\%BD\%B8\%CF\%82+\%E1\%BC\%82\%CE\%BD+\%E1\%BD\%91\%CE\%BC\%E1\%BF\%B6\%CE\%BD+\%CE\%BF\%E1\%BD\%90\%CE\%B4\%CE\%AD\%CF\%80\%CE\%BF\%CF\%84\%CE\%B5+\%CE\%B3\%CE\%AD\%CE\%BD\%CE\%BF\%CE\%B9\%CF\%84\%CE\%BF+\%E1\%BC\%84\%CE\%BE\%CE\%B9\%CE\%BF\%CF\%82"}{§209}]{madvig1873syntax} quotes for instance \refp{ex:07-33}.

\ea \label{ex:07-33}
\gll Áneu toútou oudeìs eis oudèn oudenòs àn humôn oudépote génoito áxios.\\
 without this no-one into nothing from-anyone \textsc{ptl} of-you never would-become worthy\\
\glt `Without this, none of you would ever become worthy of anything from anyone.'\hfill(Platon) % ??? Can we find where in Plato(n)?
\z
%Brett: https://www.perseus.tufts.edu/hopper/text?doc=Perseus%3Atext%3A1999.01.0173%3Atext%3DPhileb.%3Apage%3D19 But slightly different "alēthestata legeis, ō pai Kalliou: mē gar dunamenoi touto kata pantos henos kai homoiou kai tautou dran kai tou enantiou, hōs ho parelthōn logos emēnusen, oudeis eis ouden oudenos an hēmōn oudepote genoito axios."
% PE: ??? That does seem to be significantly different. Shall we leave it as OJ wrote it but add a footnote?

In Hungarian (Magyar) we have corresponding phenomena, see J. Szinnyei, \textit{Ungarische Sprachlehre} (\citeyear[§119]{szinnyei1912}): % PE ??? Replace "J. Szinnyei, \textit{Ungarische Sprachlehre} (\citeyear[§119]{szinnyei1912})" with "Szinnyei (\citeyear[§119]{szinnyei1912})"?
Negative pronouns like \il{Magyar!\textit{sënki}}\textit{sënki} (`nobody'), \il{Magyar!\textit{sëmmi}}\textit{sëmmi} (`nothing') and \is{pronominal adverbs, negative}pronominal adverbs like \il{Magyar!\textit{sëhol}}\textit{sëhol} (`nowhere'), \il{Magyar!\textit{sëhogy}}\textit{sëhogy} (`in no wise') are generally used in connexion with a negative particle or verbal form, e.g. \textit{\textsc{sënki sëm} volt ott} (or: \textit{\textsc{nëm} volt ott \textsc{sënki}}, `there was nobody there'); \textit{\textsc{sëmmit sëm} hallottam} (or: \textit{\textsc{nëm} hallottam \textsc{sëmmit}}, `I have heard nothing'). Sometimes there are three negative words in the same sentence: \textit{\textsc{nëm} felejtëk el \textsc{sëmmit sëm}} `I forget nothing'. Negative words begin with \textit{s-} or \textit{n-}.

Repeated negation is found in many other languages. \il{Bantu!Kongo!\textit{ka {\dots} ko}|(}I shall mention only a few examples from Bantu languages. In H. G. Guinness's \textit{Mosaic history and gospel story epitomised in the Congo language} (\citeyear{guinness1882mosaic}), % I (PE) have deleted publishing details; place and date appear in the bibliography.
I find, for example, \refp{ex:07-34}, etc. In D. Jones and S.~T. Plaatje, \textit{A Sechuana reader} (\citeyear[\href{https://archive.org/details/sechuanareaderin00joneuoft/page/14/mode/2up?view=theater}{15}]{jones1916sechuanareader}), % Again, simplified. Details are in the bibliography.
% ??? PE shorten "H. G. Guinness's" and "D. Jones and S.~T. Plaatje," to "Guinness," and "Jones and Plaatje,"?
a sentence is translated `not will-not you-be-destroyed by-nothing'; other examples occur on pp. 33, 41. % ??? PE: The former appears at the very end of p.33; it's glossed and translated at the very end of the first and second halves of p.32. The latter is presumably (very short) paragraph 10. I hesitate to add either as I think we'd have no choice but to use the original gloss, and I've no compelling reason to think that this is composed in the Leipzig style.

\ea \label{ex:07-34}
\ea 
\gll ka bena mambu mambiko\\
 not {there are} words evil.\NEG{}\\
\glt `there are no evil words'\hfill(\href{https://archive.org/details/mosaichistoryan00guingoog/page/n12/mode/2up?view=theater&q=ka+bena}{1})
\ex 
\gll Yetu katulendi kuba monako.\\
 we \NEG{}.can them see.\NEG{}\\
\glt `We cannot see them.'\hfill(\href{https://archive.org/details/mosaichistoryan00guingoog/page/n14/mode/2up?view=theater&q=katulendi+kuba}{2})
\ex 
\gll kavangidi kwandi wawubiko, kamonanga kwandi nganziko, kaba yelanga kwa-u ko\\
 \NEG{}.did he evil.\NEG{}, \NEG{}.feeling he pain.\NEG{}, \NEG{}.they sick they \NEG{}\\
\glt `he did no evil, he felt no pain, they were not sick'
\hfill(\href{https://archive.org/details/mosaichistoryan00guingoog/page/n14/mode/2up?view=theater&q=kavangidi+kwandi}{3})
\z
\z\il{Bantu!Kongo!\textit{ka {\dots} ko}|)}

Various explanations have been given of this phenomenon, but they mostly fail through not recognizing that this kind of repeated negation is really different from that found, for instance, when in Latin \il{Latin!non@\textit{non}}\textit{non} is followed by \il{Latin!ne@\textit{ne}}\textit{ne {\dots} quidem}; this will form our second class, but the explanation from ``supplementary negation'' (\textit{Ergänzungsnegation}), which is there all right, does not hold in the cases here considered. \ia{van Ginneken, Jacobus}Van Ginneken is right when he criticizes (\citeyear[\href{https://archive.org/details/principesdelingu00ginn/page/200/mode/2up?view=theater\&q=\%22Les\%20romanistes\%20ont\%20mis\%20en\%20usage\%22}{200}]{vanginneken1907principes}) the view of Romance scholars, who speak of a \is{half-negation}``half-negation'' (\textit{demi-négation})---an expression which may be more true of French \il{French!ne@\textit{ne}}\textit{ne} than of other negatives, but even there is not quite to the point. \ia{van Ginneken, Jacobus}Van Ginneken's own explanation is that ``negation in natural language is not logical negation, but the expression of a feeling of resistance''. He goes on to say:

\begin{quote}
L'adhésion négative logique ou mathématique (dont deux se compensent) est leur signification figurée, née seulement dans quelques centres de civilisation isolés; jamais et nulle part elle n'a pénétré dans le domaine populaire. % This is from the same page (200)

`The negative logical or mathematical adhesion (two of which compensate each other) is their figurative meaning, born only in a few isolated centers of civilization; never and nowhere has it penetrated the popular domain.' 
\end{quote}

\noindent It is true that if we look upon \textit{not}, etc., as expressing nothing but resistance, it is easy to see why such an element should be repeated over and over again in a sentence as the most effective way of resisting; but I very much doubt the primitivity of such an idea, and the theory looks suspiciously as having been invented, not from any knowledge of the natural mind of people in general, but from a desire to explain the grammatical phenomenon in question. I cannot imagine that when one of our primitive ancestors said \textit{he does not sleep}, he understood this as meaning `let us resist the idea of sleep in connexion with him'---or how is otherwise % ??? PE: "how is otherwise" strikes me as plain ungrammatical; I suspect it was a typo for "how otherwise is". Feel free to disagree; but if you agree then let's amend it.
the idea of resistance to come in here? I rather imagine he understood it exactly as we do nowadays.

But I quite agree with \ia{van Ginneken, Jacobus}van Ginneken, when he emphasizes the emotional character of repeated negation; already H. Ziemer, \textit{Junggrammatische Streifzüge} (\citeyear[\href{https://archive.org/details/junggrammatisch00ziemgoog/page/n163/mode/2up?view=theater\&q=\%22Der+sondernde\%22}{142}]{ziemer1883junggrammatische}) % Again, deliberately simplified: details in the bibliography.
% ??? PE: Shall we drop the "H." and the book title?
says in this connexion:

\begin{quote}
Der sondernde, unterscheidende Verstand blieb bei ihrer Bildung ganz aus dem Spiel; während das erregte Gefühl und der auf den Eindruck gerichtete Trieb frei schaltete.

`The separating, discriminating intellect remained completely out of play in their formation; while the excited feeling and the drive directed towards the impression operated freely.' % PE: Brett, do you understand this? I certainly don't....
%Brett: how's this new translation?
% PE: If I may quote Alice: Somehow it seems to fill my head with ideas — only I don't exactly know what they are!
\end{quote}

\noindent (though \citet{mourek1902ueber} is probably right when he says that the strengthening is a result, rather than the motive, of the repetition). I may also, like \ia{van Ginneken, Jacobus}van Ginneken, quote with approval Cauer's clever remark (\citeyear[\href{https://archive.org/details/grammaticamilit00cauegoog/page/50/mode/2up?q=\%22zugleich+vor+und+in+der+klammer\%22&view=theater}{50}]{cauer1903grammatica}): % Instead of linking from "remark", I (PE) would far prefer to link from the footnote -- but we've found that attempting this seems to bring LaTeX syntax errors.

\begin{quote}
das negative Vorzeichen ist, allerdings höchst unmathematisch, zugleich vor und in der Klammer gesetzt, indem sich die negative Stimmung über den ganzen Gedanken verbreitete % Cauer's book says "verbreitete" with an "e" on the end.

`the negative sign is placed, albeit in a highly unmathematical way, both in front of and in the bracket, so that the negative mood spreads over the entire idea'
\end{quote}

\is{Kantian categories}
\is{qualitative v. quantitative negation|(}
\label{sec:kant}There is one theory that has enjoyed a certain vogue of late years (though it is not mentioned by \ia{van Ginneken, Jacobus}van Ginneken) and which I must deal with a little more in detail. It was started by \ia{Gebauer, Jan}Gebauer\footnote{Jan Gebauer (1838–1907), scholar of the Czech language. \eds} with regard to Old Bohemian, but was made better known through Mourek's (\citeyear{mourek1902ueber}) work on negation in Middle High German % "(\textit{Königl. böhm. gesellschaft der wissenschaften} 1902)" moved out of body text
and has been faithfully repeated in the above-named (p.~\pageref{ch:preface}) works on Old English by \ia{Knörk, M.}\ia{Rauert, Matthäus}\ia{Einenkel, Eugen}Knörk, Rauert and Einenkel. These writers go back to \ia{Kant, Immanuel}Kant's %(\citeyear[\href{https://archive.org/details/in.ernet.dli.2015.149681/page/n99/mode/2up?q=\%22table+of+categories\%22}{B106}]{kant1787kritik}) %Brett: This is an English translation. I couldn't find the original.
table of categories, where the three categories of \textit{Position} (or \textit{Realität}), \textit{Negation}, \textit{Limitation} are ranged under the heading of \textit{Qualität}, while under the heading of \textit{Quantität} we find the three \textit{Einheit}, \textit{Vielheit}, \textit{Allheit} (`unity, plurality, totality'). % Peter: Better, I think, if both the three kinds of Qualität and the three kinds of Quantität were given in the single order German-then-English. The German terms for the former are Realität, Negation, Limitation; or anyway these are the three most conspicuous in Wikipedia (see https://en.wikipedia.org/wiki/Category_(Kant) and https://de.wikipedia.org/wiki/Kategorie_(Philosophie)#Immanuel_Kant ); I'm not sure why "position".
%Brett: Not sure what you're suggesting, but I think it sounds OK.
% PE: I'm embarrassed to realize that I too can't work out what it was that I was suggesting. For now at least, leave as is.
This leads to the distinction between qualitative and quantitative negation; in the former the verb and by that means the whole sentence (\textit{die ganze Aussage}) is negatived, while in the latter only one part of the sentence is negatived. As examples of qualitative negation are given \textit{the man is not truly happy} and \textit{my guests have not arrived}; of quantitative negation \textit{no man is truly happy, the man is never truly happy, the man is nowhere truly happy} (I translate \textit{der Mensch} as \textit{the man}, though perhaps the generic \textit{man} is meant) and \textit{none of my guests have arrived, I see nowhere any of my guests}. Now the supposition is that language started by having qualitative and quantitative negation separately, and that later the combination of both was arrived at in some languages, such as Middle High German and Old English, and this is looked upon as representing a higher and more logical stage.

\begin{quote}
Diese Art der Negation beruht auf der rein logischen Forderung, dass, wenn ein Satzteil quantitativ verneint auftritt, der ganze Inhalt des Satzes qualitativ verneint wird: Dies sei an einem Beispiel verdeutlicht: \textit{ne mæg \il{English!Old English!nan@\textit{nan}}nan man twam hlafordum hieran}. In diesem Satz wird ausgesagt, dass kein mensch zwei herren zugleich dienen kann. Wenn sich nun kein Mensch findet, der 2 Herren zugleich dienen kann, so kann eben nicht mehr von einem `können', sondern logischerweise nur von einem \il{German!nicht@\textit{nicht}}`nicht können' die Rede sein, daher in dem angeführten Satz ganz richtig bei \textit{mæg} `ne' steht.

`This type of negation is based on the purely logical requirement that if a part of a sentence appears quantitatively negatived, the entire content of the sentence is qualitatively negatived. This is illustrated by an example: \textit{ne mæg \il{English!Old English!nan@\textit{nan}}nan man twam hlafordum hieran}. This sentence states that no person can serve two masters at the same time. If there is no person who can serve two masters at the same time, then there can no longer be talk of a `can', but logically only of a `can't', therefore in the sentence quoted, `ne' is quite correct with \textit{mæg}.'\hfill\citep[76]{rauert1910negation} 
\end{quote}% ??S For those either in the US or using a VPN via which they can pretend that they're in the US, the PhD thesis in question should be freely available at https://catalog.hathitrust.org/api/volumes/oclc/8801471.html . This should really be checked, etc, I (PE) suppose. In the interim, I've attempted to recapitalize, more or less guessing which words are "substantives". 
% PE: I don't understand "\textit{mæg} `ne' it is quite correct". Is what's meant perhaps "\textit{ne} is quite correct with \textit{mæg}"?
%Brett: Yes, thanks!
%PE: I have (unenthusiastically) changed "negated" to "negatived".

\emergencystretch=3em
To this line of reasoning several observations naturally offer themselves. Kant's table of categories is not unobjectionable, and in \chapref{ch:8} (p.~\pageref{08-better-tripartition}) % PE: OJ merely says in chapter VIII; no page number(s)
I shall venture to propose an improvement on the tripartition of \textit{Einheit}, \textit{Vielheit}, \textit{Allheit}. \ia{Kant, Immanuel}Kant does not look upon negation as sometimes qualitative and sometimes quantitative, but thinks it always qualitative. It would seem to be more logical to consider it as always quantitative; for even in such a simple sentence as \textit{he does not sleep} we indicate the amount of sleep he obtains, though it is true that the amount is \( = 0 \). The true distinction between the two kinds of sentences cited does not, then, depend on two kinds of negation, as this is everywhere the same, but on two kinds of ideas negatived. In the so-called ``qualitative'' negation the idea negatived is in itself non-quantitative, while in the other it is in itself quantitative, for \textit{none}, \textit{never} and \textit{nowhere} negative \textit{one} (or \textit{any}), \textit{ever}, and \textit{anywhere} respectively, and these are all quantitative terms. But however this may be, it is curious here to find that language ranged highest that explicitly indicates the negativity of the sentence; containing a quantitative negation (a negatived quantity); for if it is logically self-evident that such sentences are in themselves negative, why should it need to be expressed? And if some nations are praised because they have reached this high stage of logical development that they have understood the distinction between qualitative and quantitative negation and have been able to combine both, it seems rather sad that they should later on have lost that faculty, as the Germans and the English have (at any rate the educated classes), for they say \textit{kein Mensch kann zwei Herren dienen} and \textit{no man can serve two masters}. Cf. also Delbrück's (\citeyear[36 ff]{delbruck_negativen_1910}) criticism of the same theory from partly different points of view, which I need not repeat here. % Presumably D's Germanische Syntax 1. Zu den negativen Sätzen; but I (PE) haven't yet (May '24) found this on the web
\is{qualitative v. quantitative negation|)}

We note incidentally the curious fact that the ``logically highest'' standpoint in this theory is exactly the reverse of what it was in van Ginneken's.\ia{van Ginneken, Jacobus}

\is{redundancy of expression|(}
My own pet theory is that neither is right; logically one negative suffices, but two or three in the same sentence cannot be termed illogical; they are simply a redundancy, that may be superfluous from a stylistic point of view, just as any repetition in a positive sentence (\textit{every and any, always and on all occasions}, etc.), but is otherwise unobjectionable. Double negation arises because under the influence of a strong feeling the two tendencies specified above, one to attract the negative to the verb as nexal negative, and the other to prefix it to some other word capable of receiving this element, may both be gratified in the same sentence. But repeated negation seems to become a habitual phenomenon only in those languages in which the ordinary negative element is comparatively small in regard to phonetic bulk, as \il{English!Old English!ne@\textit{ne}}\textit{ne} and \textit{n-} in Old English and Russian, \il{German!Middle High German!en@\textit{en}}\textit{en} and \il{German!Middle High German!n-@\textit{n-}}\textit{n-} in Middle High German, \textit{on} (sounded \textit{u}) in Greek, \textit{s-} or \textit{n-} in Magyar. The insignificance of these elements makes it desirable to multiply them so as to prevent their being overlooked. Hence also the comparative infrequency of this repetition in English and German, after the fuller negatives \textit{not} and \il{German!nicht@\textit{nicht}}\textit{nicht} have been thoroughly established---though, as already stated, the logic of the schools and the influence of Latin has had some share in restricting the tendency to this particular kind of redundancy. It might, however, finally be said that it requires greater \is{mental energy for comprehension}mental energy to content oneself with one negative, which has to be remembered during the whole length of the utterance both by the speaker and by the hearer, than to repeat the negative idea (and have it repeated) whenever an occasion offers itself.
\is{attraction!double|)}
\is{cumulative negation|)}
\is{redundancy of expression|)}


\AddSubSection{Resumptive negation}
\is{resumptive negation|(}
A second class comprises what may be termed \textsc{resumptive negation}, the characteristic of which is that after a negative sentence has been completed, something is added in a negative form with the obvious result that the negative effect is heightened. This is covered by Delbrück's expression ``Ergänzungsnegation'' (\citeyear[\href{https://archive.org/details/grundrissderver01delbgoog/page/534/mode/2up?view=theater\&q=\%22erg\%C3\%A4nzungsnegation\%22}{\S177}]{delbruck1897vergleichende}). In its pure form the supplementary negative is added outside the frame of the first sentence, generally as an afterthought, as in \textit{I shall never do it, not under any circumstances, not on any condition, neither at home nor abroad}, etc. A Danish example from Kierkegaard is \refp{ex:07-37}. But as no limits of sentences can be drawn with absolute certainty, the supplementary negative may be felt as belonging within the sentence, which accordingly comes to contain two negatives. This is the case in a popular Swedish idiom, in which the sentence begins and ends with \textit{inte}, as in \refp{ex:07-38}. Similarly in a Greek instance like \href{https://archive.org/details/homer-odyssey-loeb_202404/page/n85/mode/2up?q=%22+%CE%BF%E1%BD%90+%CE%B3%E1%BD%B0%CF%81+%E1%BD%80%CE%AF%CF%89+%CE%BF%E1%BD%94+%CF%83%CE%B5+%CE%B8%CE%B5%E1%BF%B6%CE%BD+%E1%BD%81+%E1%BC%80%CE%AD%CE%BA%CE%B7%CF%84%CE%B9+%CE%B3%CE%B5%CE%BD%CE%AD%CF%83%CE%B8%CE%B1%CE%B9+%CF%84%CE%B5+%CF%84%CF%81%CE%B1%CF%86%CE%AD%CE%BC%CE%B5%CE%BD%22}{\textit{Odyssey} 3.27}, where the second \il{Greek!\textit{ou}}\textit{ou} might be placed between two commas: \textit{ou gàr oíō Oú se theôn aékēti genésthai te traphémen te} (`For I believe that neither of gods nor of men will he ever be born and raised'). On account of the difficulty of telling whether we have two sentences or a sentence with a tag it may sometimes be doubtful whether we have to do with this or the preceding class, as in \href{https://internetshakespeare.uvic.ca/doc/AYL_F1/scene/2.4/index.html#tln-790}{Shakespeare, \textit{As You Like It} 2.4.8} \textit{I cannot goe no further}, which might be divided: \textit{I cannot go, no further}. 

\ea \label{ex:07-37}\il{Danish!ikke@\textit{ikke}|(}
\gll saa afskyeligt har aldrig, aldrig {nogensinde} (,) ikke den værste tyran handlet\\
 so abhorrently has never never ever {} not the worst tyrant acted\\
\glt `not even the worst tyrant has ever, ever acted so abhorrently'\\\hfill(\href{https://tekster.kb.dk/text/sks-tsa-txt-root#ss80}{\textit{Afhandlinger} 41}) % ??? PE: Could this instead be "so abhorrently has not even the worst tyrant never, ever acted"; or, better, "Not even the worst tyrant has ever, ever acted so abhorrently"? 
%% SG: The last suggestion probably catches the meaning best
\z
\is{resumptive negation|)}

\ea \label{ex:07-38}
\ea
\gll Inte ha vi några åsikter inte!\\
 not have we any opinions not\\
\glt `We do not have any opinions at all!'\hfill(\href{https://litteraturbanken.se/f%C3%B6rfattare/StrindbergA/titlar/RodaRummet/sida/227/etext}{Strindberg, \textit{Röda} 283})
\ex 
\gll Inte märkte han mig inte.\\
 not noticed he me not\\
\glt `He did not notice me at all.'\hfill(\href{https://litteraturbanken.se/f%C3%B6rfattare/W%C3%A4gnerE/titlar/Norrtullsligan/sida/156/etext}{Wägner, \textit{Norrtullsligan} 108})
\z
\z

The most important instances of this class are those in which \textit{not} is followed by a disjunctive combination with \textit{neither {\dots} nor} or a restrictive addition with \textit{not even}: \refp{ex:07-40} etc. Cf. also \refp{ex:07-43}.

\ea \label{ex:07-40}
\ea he cannot sleep, neither at night nor in the daytime
\ex he cannot sleep, not even after taking an opiate % Are these two "he cannot sleep" separate, or are they two halves of a single item?
\ex he had not the discretion neither to stop his ears, nor to know from whence those blasphemies came\hfill(\href{https://archive.org/details/bunyanspilgrims00moffgoog/page/86/mode/2up?q=%22discretion+neither%22&view=theater}{Bunyan, \textit{Progress} 80}) % Restored the end of the sentence
\z
\z

\ea \label{ex:07-43} You'll do no such thing---not until you've told me about the flat\\\hfill(\href{https://archive.org/details/septimus00unkngoog/page/n169/mode/2up?q=%22do+no+such+thing%22&view=theater}{Locke, \textit{Septimus} 174}) % Dash, not comma; "until", not "till"
\z

In the same way in other languages, e.g. Latin \textit{nan {\dots} neque {\dots} neque}, \il{Latin!non@\textit{non}}\textit{non {\dots} \il{Latin!ne@\textit{ne}}ne {\dots} quidem}, Greek \il{Greek!\textit{ou}}\textit{ou {\dots} oudé {\dots} oudé} etc. Examples are needless. (In Danish also with insertion of \textit{ikke} in the main sentence, \refp{ex:07-44}.)

\ea \label{ex:07-44}
\gll Jeg troer ikke, at hverken De eller jeg skal tage nogen bestemmelse\\
 I believe not that neither you nor I shall take any decision\\
\glt `I do not believe that either you or I should make any decision'\\\hfill(Christiansen, \textit{Fædreland} 135) % ??S I (PE) suppose that OJ's "Christiansen Fædrel." is Einar Christiansen's play Fædreland; but I can't find this on the web (in September '24) and therefore can't be sure.
%% SG: Yes, this is the source. The same sentence is quoted in ODS with ref. to the play and page number: https://ordnet.dk/ods/ordbog?query=hverken,4
\z

\is{repeated negation, taught avoidance of|(}
It is perhaps in consequence of the scholastic disinclination to repeated negation that some modern writers use \textit{even} instead of \textit{not even}, as in \refp{ex:07-45}.

\ea \label{ex:07-45}
I cannot give my Vivie up, even for your sake.\hfill(\href{https://archive.org/details/mrswarrensprofes00shawuoft/page/182/mode/2up?q=%22give+my+vivie%22&view=theater}{Shaw, \textit{Profession} 182})
\z

A few similar examples are given by Bøgholm, \textit{Anglia n. f.} 26.511. % ??S Calculating from the numbering of periodicals that are available, Anglia neue Folge [new series] 26 is dated 1915. But I (PE) can't find it on the web.

I am inclined to reckon among the cases of resumption (with the last negative originally outside the sentence) also the repetition \textit{it' ikke} or \textit{itik}, which in various phonetic forms is very frequent in Danish dialects (Seeland, % What is this? There’s no entry for it in either English- or Danish-language Wikipedia. Zealand is in Danish “Sjælland”, the Danish article about which does not mention “Seeland”. (OJ's impromptu Englishing of "Sjællandsk"?) 
%Brett: yes, I think so.
Fyn, some of the southern islands, some parts of Jutland); Feilberg also in his dictionary (\citeyear{Feilberg1886Bidrag}) % I (PE) have taken this from en:Wikipedia; hope it's accurate
quotes from various places in Jutland the combination \textit{ik hæjer it} and from Fjolde \textit{oller ek} (\textit{aldrig ikke}); for the exact phonetic form I refer to the dictionary).\footnote{Fyn is Denmark's third largest island, and the site of Odense; Fjolde (German name Viöl) is in Nordfriesland (Schleswig-Holstein). \eds}\is{repeated negation, taught avoidance of|)}

\il{Danish!nikke@\textit{nikke}}In colloquial Danish we have also an emphatic negative [\textit{gu gør jeg}] \textit{ikke nikke nej}, where \textit{nikke}, which is otherwise unknown, is a contamination of \textit{ikke} and \textit{nej}. In literature I have found this only in \refp{ex:07-46}.

\ea \label{ex:07-46}
\gll Pipmanden har delirium! {\dots} Gu' ha'de jeg ikke nikke nej!\\ % Not "havde" but "har". Ends with exclamation points, not periods. Dots to show the omission of "vrælede de." (The original capitalizes "Delirium", but I've kept it lowercase in accordance with OJ's taste.)
 {tweet man.\DEF{}} had {delirium} {} \textsc{emph} had I not not no\\
\glt `Pipman was delirious! {\dots} I most certainly was not!'\hfill(\href{https://archive.org/details/pelleerobrerenro03ande/page/18/mode/2up?q=%22delirium%22&view=theater}{Nexø, \textit{Pelle} 3.19}) % Unusually, I (PE) have fiddled with this translation (which previously and mysteriously included the word "nod").%Brett: Fiddle not! "nikke nej" literally means "nod no", but it's an idiomatic expression meaning "say no" or "refuse". But my natural translation was way off, so thank you! % PE: I sit corrected!
%% SG: The word "nikke" 'nod' is a homonym, it has nothing to do with this "nikke", which is simply a kind of reduplication of "ikke" 
%% SG: note also that Pipmanden is not 'pipe man' (that would be "Pibemanden") but a nickname meaning something like 'tweet/chirp man' ("pip" is the sound a small bird makes) -- I am not sure how this is best translated into English, maybe just "Tweety"? Perhaps check what the English translation uses? https://en.wikipedia.org/wiki/Pelle_the_Conqueror_(novel)
%% PE: The published translation ( https://gutenberg.org/cache/epub/7795/pg7795-images.html ) has "Pipman". I've changed from "Tweety" to "Pipman"; less because the published translation uses it, more because these days "Tweety" jarringly brings to mind "tweets" (and "X", and Musk...).  
\z\il{Danish!ikke@\textit{ikke}|)}


\is{adverbs!negative|(}
An English case of special interest is with \textit{hardly} (on the negative value of this see p.~\pageref{sec:incomplete-negation}) in combination with a preceding negative word, which is felt to be too absolute and is therefore softened down by the addition; the two negatives thus in this case neither neutralize nor strengthen one another. Examples (none in Shakespeare): \refp{ex:07-47}.\largerpage

\ea \label{ex:07-47}
\ea it {\dots} gave us not time hardly to say, O God\hfill(\href{https://archive.org/details/lifeandstranges00dobsgoog/page/n75/mode/2up?q=%22not+time+hardly%22&view=theater}{Defoe, \textit{Robinson} 50}) % Dots added to show a major cut by OJ
\ex and nobody hardly took notice of him\hfill(\href{https://archive.org/details/journaltostellae00swifuoft/page/372/mode/2up?q=%22nobody+hardly+took%22&view=theater}{J. Swift, \textit{Journal} 372})
\ex nothing hardly is welcome but childish fiction\\\hfill(\href{https://www.gutenberg.org/cache/epub/47790/pg47790-images.html#Page_87}{Cowper, letter, 19 Oct. 1781})
\ex I've never hardly known him to miss church before\hfill(\href{https://archive.org/details/adambede00eliouoft/page/n213/mode/2up?q=%22him+to+miss+church%22&view=theater}{Eliot, \textit{Adam} 197})
\ex that no one has hardly a right to examine the question of species who has not minutely described many\hfill(\href{https://www.darwinproject.ac.uk/letter/DCP-LETT-915.xml}{Darwin, \textit{Life} 2.39})
\ex I do not suppose I could be of hardly any use\hfill(\href{https://www.darwinproject.ac.uk/letter/DCP-LETT-2490.xml}{ibid 2.165}) % OJ gives the page number but not the quotation.
\ex ``Who was there?'' --- ``Nobody hardly''\hfill(\href{https://archive.org/details/returnofthenativ00harduoft/page/146/mode/2up?q=%22nobody+hardly%22&view=theater}{Hardy, \textit{Return} 192}) % Two speakers, thus restoring quotation marks
\ex nobody hardly understands criticism as badly as you do\\\hfill(\href{https://archive.org/details/quisantanovel00hopegoog/page/n130/mode/2up?q=%22hardly+understands+criticism%22&view=theater}{Hope, \textit{Quisanté} 119})
\ex you cant hardly tell who anyone is or isnt\hfill(\href{https://archive.org/details/doctorsdilemmage0000bern/page/260/mode/2up?q=%22tell+who+anyone+is%22&view=theater}{Shaw, \textit{Married} 194}) % OJ attributes this to The Doctor's Dilemma. It's not there, but Getting Married has been bound in the same volume as The Doctor's Dilemma. "Or isnt" has been restored.
\ex He wasn't changed at all hardly\hfill(\href{https://archive.org/details/stalkyandco015455mbp/page/n189/mode/2up?q=%22changed+at+all+hardly%22&view=theater}{Kipling, \textit{Stalky} 192})
\ex they don't seem hardly able to help it\hfill(\href{https://archive.org/details/wifeofsirisaacha00well/page/112/mode/2up?q=%22seem+hardly+able+to%22&view=theater}{Wells, \textit{Wife} 112})
\ex I don't hardly care to stay\hfill(\href{https://archive.org/details/thesetwain0000arno_p9h0/page/362/mode/2up?q=%22don%27t+hardly+care+to+stay%22&view=theater}{Bennett, \textit{Twain} 354})
\ex he couldn't hardly speak\hfill(\href{https://archive.org/details/hildalessways00benn_2/page/12/mode/2up?q=%22couldn%27t+hardly%22&view=theater}{Bennett, Hilda})
\z
\z

Examples of \textit{scarce}(\textit{ly}) after a negative: \refp{ex:07-59}.

\ea \label{ex:07-59}
\ea me not worthy scarce to touch Thy kind strong hand\\\hfill(\href{https://archive.org/details/tirstramlyoness00swingoog/page/n154/mode/2up?q=%22worthy+scarce%22&view=theater}{Swinburne, \textit{Pilgrimage} 137})
\ex There is not a yard of it, scarcely, that hasn't been \emph{made} by human hands\hfill(\href{https://archive.org/details/cu31924013567130/page/512/mode/2up?q=%22a+yard+of+it%22&view=theater}{Ward, \textit{Eleanor} 411}) % Ward italicizes "made"
\ex but no one scarcely could throw himself down\hfill(\href{https://archive.org/details/newsfromnowher00morr/page/154/mode/2up?q=%22but+no%22&view=theater}{Morris, \textit{News} 129})
\z
\z

\is{indirect negation}
\textit{Hardly} and \textit{scarcely} are also used after \textit{without} and other indirect negatives: \refp{ex:07-62}. \label{sec:indirect}

\ea \label{ex:07-62}
\ea The black, however, without hardly deigning A glance at that\\\hfill(\href{https://archive.org/details/workslordbyron10unkngoog/page/236/mode/2up?view=theater&q=%22without+hardly+deigning%22}{Byron, \textit{Juan} 5.66})
\ex without scarcely hearing a word\hfill(\href{https://archive.org/details/vanityfairanove03thacgoog/page/n293/mode/2up?q=%22hearing+a+word%22&view=theater}{Thackeray, \textit{Vanity} 476})
\ex refusing to acknowledge hardly any fiction that was not classic\\\hfill(\href{https://archive.org/details/pitepicofwheatde00norruoft/page/44/mode/2up?view=theater&q=%22refusing+to+acknowledge%22}{Norris, \textit{Pit} 52})
\ex I'll be dinged if I hardly know\hfill(\href{https://archive.org/details/toothpicktales00readgoog/page/n22/mode/2up?q=dinged&view=theater}{Read, \textit{Toothpick} 17})
\z
\z

Cf. also \refp{ex:07-66}.

\ea \label{ex:07-66}
\gll Og Edith og Gerhard trykkede hinanden i haanden --- uden at de knap vidste deraf.\\
and Edith and Gerhard pressed {each.other} in hand\DEF{} {} without that they barely knew thereof\\
\glt `And Edith and Gerhard pressed each other's hands---hardly aware that they were doing so.' \hfill (\href{https://tekster.kb.dk/text/adl-texts-drachmann14val-root#s297}{Drachmann, \textit{Forskrevet} 1.425})
\z
%% SG: Added "DEF" to the gloss of "haanden", it's the definite form. Not sure how to typeset the empty space after it

Some instances of double negation with words like \textit{nor} and \textit{neither}, which are not exactly analogues of those given here, will be found in the chapter on Negative connectives (p.~\pageref{ch:10}).
\is{adverbs!negative|)}


\AddSubSection{Paratactic negation}
\is{paratactic negation|(}
\is{verbs of negative import|(}
\label{paratactic}Closely connected with resumptive negation is what might perhaps be termed \textsc{paratactic negation}: a negative is placed in a clause dependent on a verb of negative import like `deny, forbid, hinder, doubt'. The clause here is in some way treated as an independent sentence, and the negative is expressed as if there had been no main sentence of that particular kind. It is well known how this develops in some languages to a fixed rule, especially if the negative employed has no longer its full negative force: I need only very briefly refer, for instance, to the Latin use of \il{Latin!ne@\textit{ne}}\textit{ne}, \il{Latin!quin@\textit{quin}}\textit{quin}, \il{Latin!quominus@\textit{quominus}}\textit{quominus}, and to the French insertion of \il{French!ne@\textit{ne}}\textit{ne} (which, by the way, is now disappearing like the other \textit{ne}'s). But even in languages which do not as a rule admit a negative in such clauses, it is by no means rare even in good writers, though generally looked upon as an error by grammarians, see for English e.g. \refp{ex:07-67}.

\ea \label{ex:07-67}
\ea You may deny that you were not the meane Of my Lord Hastings late imprisonment\hfill(\href{https://internetshakespeare.uvic.ca/doc/R3_F1/scene/1.3/index.html#tln-555}{Shakespeare, \textit{R3} 1.3.90})
\ex we have severely forbidden it to all our fellows \ {\dots} that they doe not shew any naturall worke\hfill(\href{https://archive.org/details/bim_early-english-books-1641-1700_new-atlantis_bacon-sir-francis_1658/page/32/mode/2up?q=%22forbidden+it%22&view=theater}{Bacon, \textit{Atlantis} 43.34}) % "Severely" restored. ??? "43.34" perhaps has meaning only to those people lucky enough to see the edition that OJ cites.
\ex What hinders in your own instance that you do not return to those habits\hfill(\href{https://archive.org/details/essayseliacharle00lamb/page/296/mode/2up?q=%22what+hinders%22&view=theater}{Lamb, \textit{Elia} 2.185})
\ex It never occurred to me for a moment to doubt that your work {\dots} {[}would{]} not advance our common object in the highest degree\\\hfill(\href{https://www.darwinproject.ac.uk/letter/DCP-LETT-5544.xml}{Darwin, \textit{Life} 3.69}) % "for a moment" restored. I didn't check Life and Letters; very likely it has "would", but the website has "wd" (with a superscripted "d")
\z
\z
 
Parallel instances from German may be found, for instance, in \textit{Sprachgebrauch und Sprachrichtigkeit im Deutschen} \citep[\href{https://archive.org/details/sprachgebrauchu00andrgoog/page/n223/mode/2up?view=theater}{209ff}]{andresen1892sprachgebrauch}. % ??? PE: Shall we cut the title of the book? 

Danish examples: \refp{ex:07-71}.

\ea \label{ex:07-71}
\ea
\gll forbøden, att ingen skulle lade mig faa naale\\
 {forbidden} that {no one} should let me get needles\\
\glt `[She had] forbidden that anyone should let me have needles'\\\hfill(\href{https://www.gutenberg.org/cache/epub/41072/pg41072-images.html#Side_70}{Leonora Christina 62}) % Here OJ calls the author "El. Christ."; earlier in the chapter, "El. Christine".
\ex 
\gll forhindre, att hun icke satte løgn sammen om mig\\
 prevent, that she not put lie together about me\\
\glt `prevent her from concocting lies about me'\hfill(\href{https://www.gutenberg.org/cache/epub/41072/pg41072-images.html#Side_97}{ibid 85})
\ex 
\gll efftersom quinden saa høyt haffde forsoeren icke att sige ded\\
 since woman.\DEF{} so loudly had forsworn not to say it\\
\glt `since the woman had so loudly sworn not to say it'\hfill(\href{https://www.gutenberg.org/cache/epub/41072/pg41072-images.html#Side_122}{ibid 107})
\ex 
\gll Hand næctede ded altiid, att ded icke war ham\\
 he denied it always that it not was him\\
\glt `He always denied that it was him'\hfill(\href{https://www.gutenberg.org/cache/epub/41072/pg41072-images.html#Side_137}{ibid 120})
\ex 
\gll forhindre {\dots} icke\\
 prevent {} not\\
\glt `prevent'\hfill(\href{https://www.gutenberg.org/cache/epub/41072/pg41072-images.html}{ibid 201, 213})
\ex\il{Danish!ikke@\textit{ikke}|(}
\gll for at hindre at misundelsens sæd ikke skal saaes iblandt os\\
 for to prevent that envy.\DEF.\POSS{} seed not shall be.sown among us\\
\glt `to prevent the seeds of envy from being sown among us'\\\hfill(\href{http://holbergsskrifter.dk/holberg-public/view?docId=skuespill%2FUlysses%2FUlysses.page&brand=&chunk.id=act2sc7&toc.id=act2&toc.depth=1}{Holberg, \textit{Ulysses} 2.7}; % Linked-to source has "icke" and "iblant"; could the difference be a matter of spelling reform?
%Brett: I believe so.
%% SG: Yes, Holberg uses an older spelling
also Holberg, \href{http://holbergsskrifter.dk/holberg-public/view?docId=Paars%2FPaars.page&brand=&chunk.id=book1song2&toc.id=book1&toc.depth=1}{\textit{Peder} 1.2}, \href{http://holbergsskrifter.dk/holberg-public/view?docId=Paars%2FPaars.page&brand=&chunk.id=book1song4&toc.id=book1&toc.depth=1}{\textit{Peder} 1.4}, etc.) % ??? What I (PE) read on pp 79-80 of P M Mitchell's A History of Danish Literature https://archive.org/details/historyofdanishl0000unse/page/n7/mode/2up suggests to me that this is not a work by anyone named Pedersen but is instead Holberg's pseudonymous work Peder Paars, which comprises four books, of which the first comprises five cantos. Book 1 canto 2 http://holbergsskrifter.dk/holberg-public/view?docId=Paars%2FPaars.page&brand=&chunk.id=book1song2&toc.id=book1&toc.depth=1 contains one token of "hindre" (within "forhindre"); book 1 canto 4 http://holbergsskrifter.dk/holberg-public/view?docId=Paars%2FPaars.page&brand=&chunk.id=book1song4&toc.id=book1&toc.depth=1 contains one of "hindre" (within "hindrer"). However, my optimistic guesswork really isn't good enough.
%% SG: This is surely a reference to Peder Paars, it's a well-known work and OJ's "Ped. P." would have been transparent to readers at the time. 

\ex 
\gll mine venner burde forhindre at ingen af mine digte, der kun vare poetiske misfostre, kom for lyset\\
 my friends should prevent that none of my poems which only were poetic miscarriages came to light\\
\glt `my friends should prevent any of my poems, which were only poetic miscarriages, from coming to light'\hfill(\href{https://tekster.kb.dk/text/adl-texts-andersen03val-root#idm140167596697696}{Andersen, \textit{Improvisatoren} 2.136})
\ex 
\gll alt skulde anvendes for at forebygge, at min lille pige ikke skulde blive koparret\\
 everything should be.used for to prevent that my little girl not should become pockmarked\\
\glt `everything should be done to prevent my little girl from becoming pockmarked'\hfill(\href{https://tekster.kb.dk/text/adl-texts-sibbern02val-root#s114}{Sibbern, \textit{Breve} 1.130})
\ex 
\gll at jeg af al magt skal stræbe {\dots} at bidrage til at afværge, at dette ikke skeer\\
 that I of all power shall strive {} to contribute to to prevent, that this not happens\\
\glt `that I shall strive with all my power {\dots} to contribute to preventing this from happening'\hfill(\href{https://archive.org/details/ieblikketno00kiergoog/page/n16/mode/2up?q=%22at+jeg+af%22&view=theater}{Kierkegaard, \textit{Øieblikket} 7})
\ex 
\gll vogtede hun sig for ikke at tale for meget om Carl\\
 guarded she herself for not to talk too much about Carl\\
\glt `she was careful not to talk too much about Carl'\hfill(\href{https://tekster.kb.dk/text/adl-texts-bang11val-shoot-workid54092#s163}{H. Bang, \textit{Fædra} 161}) % Linked source has "formeget" as a single word.
\z
\z

(Note here the difference between the usual Danish idiom (\ref{ex:07-new}a) and the corresponding English; cf. also (\ref{ex:07-new}b) and German (\ref{ex:07-new}c).)

\ea\label{ex:07-new}
\ea
\gll man må vogte sig for at overdrive\\
    one must guard self for to exaggerate\\
\glt `one must take care \emph{not} to exaggerate'
\ex
\gll jeg advarede ham mod at gøre det\\
    I warned him against to do it\\
\glt `I warned him \emph{not} to do it'
\ex
\gll ich warnte ihn, das zu tun\\
    I warned him that to do it\\
\glt `I warned him \emph{not} to do it'
\z
\z


In this connexion I must mention a Danish expression which is extremely frequent in colloquial speech, but which is invariably condemned as illogical and put down as one of the worst mistakes possible: \refp{ex:07-81}. 
\is{verbs of negative import|)}

\ea \label{ex:07-81}
\gll man kan ikke \emph{nægte} \emph{andet} \emph{end} \emph{at} hun er sød\\
one can not deny other than that she is sweet\\
\glt `One cannot deny that she is sweet'
\z

\is{conjunctions!negative|(}
\noindent This, of course, is illogical if analyzed with \textit{andet} as the sole object of \textit{nægte}: `one can deny nothing else except that she is sweet' but to the actual speech-instinct \textit{andet end {\dots} at hun} goes together as one indivisible whole constituting the object of \textit{nægte}; this is often marked by a pause before \textit{andet}, and \textit{andet-end-at} thus makes one negative conjunction comparable with Latin \il{Latin!quin@\textit{quin}}\textit{quin} or \il{Latin!quominus@\textit{quominus}}\textit{quominus}.

In the same way one hears, e.g. \refp{ex:07-82}. From Norwegian I have noted \refp{ex:07-85}.

\ea \label{ex:07-82}
\ea
\gll Der er ikke to meninger om, andet end (at) han er en dygtig mand\\
 there are not two opinions about other than that he is a competent man\\
\glt `There is no disagreement about his being a competent man' % ??? PE: Could "a competent man" not be simplified to "competent"?
\ex
\gll Der er ikke noget i vejen for, andet end at han skal nok gøre det\\
 there is not anything in way of other than that he shall indeed do it\\
\glt `There is nothing preventing him from indeed doing it'
\ex 
\gll Jeg kan ikke komme bort fra, andet end at han har ret\\
 I can not get away from other than that he is right\\
\glt `I cannot deny that he is right'
\z
\z

\ea \label{ex:07-85}
\gll og det var ikkje fritt, annat dei [draumar] tok hugen burt fraa boki med\\ % Reinserting the "j" in "ikkje", following the cited source
 and it was not free other they [dreams] took mind.\DEF{} away from book.\DEF{} with\\
\glt `and it was undeniable that they [i.e. dreams] took his attention away from the book'\hfill(\href{https://archive.org/details/bondestudentar00garborgnb/page/n39/mode/2up?q=%22var+ikkje+fritt%2C+annat%22&view=theater}{Garborg, \textit{Bondestudentar} 33}) % PE: "took his attention from the book"? "diverted him from the book"?
%Brett: Yes.
%% SG: The Norw. idiom "ikkje fritt" means "undeniable"
\z

\is{adverbs!negative|(}
The following quotations may serve to illustrate the transition of \textit{andet} (\textit{end}) to a negative conjunction or adverb: \refp{ex:07-86}--\refp{ex:07-86b}.

\ea \label{ex:07-86}
\ea
\gll det er ellerss wmweligt \emph{andet} \emph{end} \emph{at} han ey skall fare vild\\
 it is otherwise impossible other than that he not shall get lost\\
\glt `it is otherwise impossible, except that he should not get lost'\\\hfill(\href{https://www.google.co.jp/books/edition/Christiern_Pedersens_Danske_skrifter_bd/rslJAQAAMAAJ?hl=en&gbpv=1&dq=%22han+ey+skall+fare+vild%22&pg=PA493&printsec=frontcover}{Pedersen, \textit{Skrifter} 4.493}) % The linked-to source -- with the same page number (493) as what OJ cites and thus presumably the same edition -- has not "andet end at han" but (if I can decipher the Fraktur properly) instead "andet en ath han". What to do?
%Brett: AI says, "end" vs "en": This is likely a spelling variation. "En" is an older or dialectal form of "end" (than/but). "at" vs "ath": Again, this is a spelling variation. "Ath" is an older spelling of "at" (that).
\ex 
\gll Det er sgu da ikke \emph{andet} \emph{end} til at lee ad\\
 it is damn then not other than for to laugh at\\
\glt `It is damn well nothing but to laugh at'\hfill(\href{https://archive.org/details/ravnenfortlling00goldgoog/page/64/mode/2up?q=%22Det+er+fgu+da+iffe+Andet%22&view=theater}{Goldschmidt, \textit{Ravnen} 65})
\ex
\gll han bestilte ikke det, man kan tænke sig \emph{andet}, \emph{end} at drikke portvin\\
 he did not that one can imagine himself other than to drink port\\
\glt `He did nothing imaginable, except to drink port.'\\\hfill(\href{https://www.henrikpontoppidan.dk/text/kilder/artik/fiktive/pca008_vinterbillede.html}{Pontoppidan, \textit{Billeder} 155})
\largerpage[2]
\ex
\gll men det var umuligt \emph{annet} \emph{æn} i hennes omgang at komme til at gå for langt\\
 but it was impossible other than in her company to come to to go too far\\
\glt `but it was impossible, except in her company, to go too far'\\\hfill(Norwegian: \href{http://f9.no/ebok/filer/bjornson_det_flager_i_byen_og_paa_havnen.html}{Bjørnson, \textit{Flager} 432}) % Point out that this is in Norwegian?
%Brett: Yes, please do.   % PE: Done
\ex 
\gll Stodderen laa stille som en mus, \emph{andet} \emph{end} \emph{at} hun kunde høre ham trække vejret tungt\\
 bum.\DEF{} lay still as a mouse other than that she could hear him draw breath.\DEF{} heavily\\
\glt `The bum lay still as a mouse, except that she could hear him breathe heavily'\hfill(Grundtvig, \textit{Folkeæventyr} 65) % This does appear on p.90 of Danske Folkeeventyr (1908), whose list of contents says the particular story is "fra Sv. Grundtvigs Samlinger" ( https://www.kb.dk/e-mat/dod/130023202647-bw.pdf )
\z
\z

\ea \label{ex:07-86a}
\gll það var ekki að sjá á honum \il{Icelandic!\textit{annað}}\emph{annað} \il{Icelandic!\textit{en}}\emph{en} hann væri ungur maður\\
 it was not to see on him other than that was young man\\
\glt `There was nothing to see in him except that he was a young man'\\\hfill(Icelandic: \href{https://runeberg.org/anf/1890/0167.html}{Þorkelsson, \textit{Nekrolog} 163}) % Point out that this is Icelandic?
%Brett: Yes, please do.  % PE: Done
\z

\ea \label{ex:07-86b}
\ea
\gll De war ett got \emph{anned}\\ % The spelling in the first edition (which has different pagination from the edition that OJ cites) is not "got anned" but "godt annet"
 it was not good other\\
\glt `It wouldn't be good otherwise'\hfill(\href{https://books.google.co.jp/books?id=zggQAQAAIAAJ&printsec=frontcover&dq=inauthor:%22Steen+Steensen+Blicher%22&hl=ja&newbks=1&newbks_redir=0&sa=X&redir_esc=y#v=snippet&q=%22de%20war%20ett%20godt%22&f=false}{Jutlandic: Blicher, \textit{Bindstouw} 51})
%% SG: This is not standard Danish, but a Central Jutlandic dialect, which I think is worth adding for readers not familiar with the author (who is famous in Denmark but definitely not abroad). "ett" here is the negation corresponding to stand. Danish "ikke".
%% SG: And you can add a link to the online edition by the Royal Library instead if you want: https://tekster.kb.dk/text/adl-texts-blic03val-shoot-workid70257 (it uses the spelling "godt annet")
\ex 
\gll ``Maaske højesteretssagføreren kender mig?'' --- ``Bevares {\dots}, det vilde være mærkeligt \emph{andet}''\\
 maybe supreme.court.solicitor.\DEF{} knows me {} goodness {} it would be strange other\\ % Ellipsis added for "Fru Baltzer," (cut by OJ)
\glt ``Maybe the Supreme Court solicitor knows me?'' --- ``Goodness, it would be strange otherwise.''\hfill(\href{https://tekster.kb.dk/text/adl-texts-brandesed_05-root}{E. Brandes, \textit{Lykkens} 3})
\ex 
\gll begge dele har deres betydning, det kan man ikke sige \emph{andet}\\ % OJ writes "deres" but the source linked to (whose page number shows that it's not the edition that OJ cites) instead has "sin".
 both parts have their meaning that can one not say other\\
\glt `both things have their meaning, one cannot say otherwise'\\\hfill(\href{https://books.google.co.jp/books?redir_esc=y&hl=ja&id=G0ARAQAAMAAJ&q=begge+delle+har+deres#v=onepage&q=begge&f=false}{Giellerup, \textit{Romulus} 98})\il{Danish!ikke@\textit{ikke}}
\ex 
\gll Det er jeg vis på,--- det er umuligt \emph{andet}\\ % In the original, not "på" but "paa"
 that am I sure on fin it is impossible other\\
\glt `I am sure of that---it is impossible otherwise'\hfill(\href{https://tekster.kb.dk/text/adl-texts-gjellerup01-shoot-workid54089#idm139731117574624}{Giellerup, \textit{Minna} 311}) % On the title page of this edition of this book, his name is written "Gjellerup".
\z
\z
\is{adverbs!negative|)}

The related use of English \textit{but} (\textit{but that}, \textit{but what}) will be treated in chapter 12 (p.~\pageref{ch:12}).
\is{conjunctions!negative|)}
\is{paratactic negation|)}


\AddSubSection{\textit{Langtfra ikke} `far from \dots'}
\il{Danish!ikke@\textit{ikke}}\il{Danish!langtfra@\textit{langtfra}|(}There is a curious use of a seemingly superfluous negative in Danish, which cannot be explained exactly in the same way as any of the phenomena hitherto dealt with, namely \textit{langtfra ikke} (`far from it', literally `far from not'), which used to be the regular idiom in phrases like \textit{hun er langtfra ikke så køn som søsteren} (`she is nowhere near as pretty as her sister') from the time of Holberg\footnote{Ludvig Holberg (1684--1754) was most active as a writer from the 1720s to the 1750s. \eds} till the middle of the 19th century, when it was superseded by \textit{langtfra} without \textit{ikke}: \textit{hun er langtfra så køn som søsteren}; English here has the positive form, but often inserts the verbal substantive in \textit{-ing}: \textit{she is far from being as pretty as her sister}. (English does not always require \textit{being} after \textit{far from}: \textit{she is far from pretty}.) \textit{Langtfra ikke} would be explicable as an instance of blending (contamination) if it could be proved that \textit{langtfra} was used as in recent times before the rise of \textit{langtfra ikke}, but I have no material to decide this question. (Cf. \cite{levin1861dagbladet}.) % I've trimmed from this the initial of the author's first name (which OJ strangely renders as "J" rather than "I") and the place and year of publication: details can be found in the bibliography.
\il{Danish!langtfra@\textit{langtfra}|)}
\il{Danish!ikke@\textit{ikke}|)}


\AddSubSection{Instances of confusion}
\is{confusion of negation|(}
I collect here several partly heterogeneous instances of confusion in negative sentences, which I have found some difficulty in placing, either in this or in any other chapter. Such confusion will occur frequently, especially if two or more negative or half-negative words are combined, but more frequently, of course, in everyday speech than in printed literature. Shakespeare, in accordance with the popular character of Elizabethan plays, destined to be heard much more than to be read, pretty frequently indulges in such carelessness (see \cite[\href{https://archive.org/details/shakespearelexic02schmuoft/page/1420/mode/2up?view=theater}{1420}]{schmidt1886}, \textit{Shakespeare-lexicon}), % ??? PE: Cut the book title?
e.g. \refp{ex:07-96}.\largerpage[2]

\ea \label{ex:07-96}
\ea\il{English!less@\textit{less}|(}
wanted lesse impudence\\(`had less impudence' or `wanted impudence more') \\\hfill(\href{https://internetshakespeare.uvic.ca/doc/WT_F1/scene/3.2/index.html#tln-1230}{Shakespeare, \textit{Wint} 3.2.57})
\ex
a begger without less quality\\(`with less quality')\hfill(\href{https://internetshakespeare.uvic.ca/doc/Cym_F1/scene/1.5/index.html#tln-335}{Shakespeare, \textit{Cymb} 1.5.23}) % Corrected from "I. 4" to "1.5"
\ex
``Tullus Auffidious, is he within your walles?'' --- ``No, nor a man that feares you lesse then he'' \phantom{x} (`fears you more')\footnote{Rather than careless, this is a rhetorical use of understatement for emphasis. The speaker is saying that not only is Tullus Auffidious not within their walls, but neither is anyone who fears them less than he does---implying that Auffidious fears them very little. (\textit{That's lesser then a little}, which follows, reinforces this.) Thus ``The brave Auffidious is not there, and neither is anyone braver.'' \eds}\\\hfill(\href{https://internetshakespeare.uvic.ca/doc/Cor_F1/scene/1.4/index.html#tln-500}{Shakespeare, \textit{Cor} 1.4.14}) % ??? Has OJ perhaps misunderstood this? (Try looking at it in its context.)
%I believe you're correct.
\z
\z

\noindent A doubtful instance is \refp{ex:07-99}, for, as \citet[70]{koppel1899} remarks, % ??S Perhaps Richard Koppel, Verbesserungsvorschläge zu „Lear“ (aka Verbesserungsvorschläge den erläuterungen und der textlesung des „Lear“)
everything is correct, if we understand `you are \emph{still} less capable of valuing her than she is capable of scanting her duty'. But \refp{ex:07-100} evidently is a confusion of two ideas: \textit{thou art nothing less than {\dots}} and: \textit{thou art in nothing} (`in no respect') \textit{more than} {\dots}. 

\ea \label{ex:07-99}
You lesse know how to value her desert, Then she to scant her dutie\\\hfill(\href{https://internetshakespeare.uvic.ca/doc/Lr_F1/scene/2.2/index.html#tln-1415}{Shakespeare, \textit{Lr} 2.2.141}) % Uvic.ca assigns this to act 2, scene 2
\z

\ea \label{ex:07-100}
ile proue [\href{https://internetshakespeare.uvic.ca/doc/Lr_F1/page/25/#tln-3040}{folio}: make] it on thy heart Ere I tast bread, thou art in nothing lesse Then I haue heere proclaimd thee [a traitor]\hfill(\href{https://internetshakespeare.uvic.ca/doc/Lr_Q1/complete/index.html#tln-3040}{Shakespeare, \textit{Lr} 5.3.94}) % Peter: "tast" and "proclaimd" following the first quarto // "i.e." looks strange to me: shall we replace it with "sc.", simply delete it, or do something else? PE: PS I've simply deleted it
%Brett: fine
\z

In \refp{ex:07-101}, some editors change \textit{if not} into \textit{if that}, but this is not at all necessary: the sentence is meant to be continued: \textit{if not these suffice}, or: \textit{are strong enough}, but is then continued in a different way, as is very often the case in everyday speech.

\ea \label{ex:07-101}
if not the face of men, The sufferance of our soules, the times abuse; If these be motiues weake, breake off betimes\hfill(\href{https://internetshakespeare.uvic.ca/doc/JC_F1/scene/2.1/index.html#tln-745}{Shakespeare, \textit{Cæs} 2.1.114}) 
\z

Modern instances of a similar character: \refp{ex:07-102}.

\ea \label{ex:07-102}
\ea
He can have nothing to say to me that anybody need not hear  \phantom{x} (`that anybody may not hear'; `that it is necessary that nobody hears')\\\hfill(\href{https://archive.org/details/prideprejudice00aust/page/132/mode/2up?q=%22anybody+need+not+hear%22&view=theater}{Austen, \textit{Pride} 133})
\ex
There was none too poor or too remote not to feel an interest\\\hfill(news 1899)
\ex
A married man {\dots} cannot live at all in the position which I ought to occupy \emph{under less than} six hundred a year\hfill(\href{https://archive.org/details/lifelettersoftho0001huxl_m9n2/page/118/mode/2up?q=%22married+man%22&view=theater}{Huxley, letter}) % Dots to show a cut made by OJ
\il{English!less@\textit{less}|)}
\ex
You know what a weak softy the heir-apparent is. % OJ writes "he" but the original has "the heir-apparent".
If there was \emph{hardly any} mischief to be had he'd be in the thick of it \phantom{x}(`if there was any, even the slightest, mischief', `there was hardly any mischief, but {\dots}')\\\hfill(\href{https://books.google.co.jp/books?id=5Pw_AAAAYAAJ&printsec=frontcover&hl=ja#v=onepage&q&f=false}{Matthews, \textit{Son} 243}) % Note that the link is to the book as a whole. I can't find a way of getting a link to the specific page (which is visible if you scroll to it).
\z
\z

German instances of confusion have been collected by F. Polle (\textit{Wie denkt das Volk über die Sprache} \citeyear[\href{https://archive.org/details/wiedenktdasvolk01pollgoog/page/n27/mode/2up?view=theater}{14}]{polle1889}), % Year of publication removed: it's in the bibliography. 
% ??? PE Cut the "F." and the book title?
e.g. \refp{ex:07-106}. I remember seeing \refp{ex:07-108} in a notice in the Tirol, the meaning evidently being \il{German!nicht@\textit{nicht}|(}\textit{nicht weit} = \textit{unweit}.

\ea \label{ex:07-106}
\ea
\gll Wie wild er schon war, als er nur hörte, dass der Prinz dich jüngst \emph{nicht} \emph{ohne} \emph{Misfallen} gesehen!\\ % Removed one "s" from OJ's "missfallen", to accord with the source
 how wild he already was when he just heard that the prince you recently not without displeasure seen\\
\glt `How furious he was just on hearing that the prince had recently been ogling you!'\hfill(\href{https://archive.org/details/emiliagalottiein0000less_n9n0/page/51/mode/2up?q=%22wie+wild+er%22&view=theater}{Lessing, \textit{Emilia} 2.6})

\ex 
\gll Man \emph{versäume} \emph{nicht}, die günstige Gelegenheit \emph{unbenutzt} vorübergehen zu lassen.\\ % I (PE) can find passages by Karl May, Ludwig Rosen and Friedrich Schiller that include chunks of this, but I can't be sure that any one of these was chopped down to result in it.
 one neglect not, the favorable opportunity unused {pass by} to let\\
\glt `One should not let the favorable opportunity pass by unused'
\z
\z

\ea \label{ex:07-108}
\gll \emph{Nicht} \emph{unweit} von hier, in dem Walde {\dots}\\
 not {not-far} from here in the forest\\
\glt `Not far from here, in the forest {\dots}'
\z\il{German!nicht@\textit{nicht}|)}

\citet[\href{https://babel.hathitrust.org/cgi/pt?id=uiuo.ark:/13960/t4fn6hr38&seq=257}{241}]{siesbye1876småting}, and \citet[328]{mikkelsen1911ordföjningslære} collect some examples like \refp{ex:07-109}. 

\ea \label{ex:07-109}
\ea\il{Latin!non@\textit{non}}
\gll Invidus, iracundus, iners, vinosus, amator, Nemo adeo ferus est, ut non mitescere possit\\
 envious wrathful lazy given-to-wine lover {no one} so wild is that not soften can\\
\glt `Slave to envy, anger, sloth, wine, lewdness, No one is so savage that he cannot be tamed.' \phantom{x} (`{\dots} anyone can be tamed')\hfill(\href{https://archive.org/details/quintihoratiifl01goulgoog/page/n153/mode/2up?q=%22iracundus%22&view=theater}{Horace, \textit{Epistola}}) % Edition linked to has "àdeo" and one comma fewer
\ex 
\gll Musik, Rollen und Schuhe, Wäsche und italiänische Blumen {\dots}, keines verschmähte die Nachbarschaft des andern\\ 
 music rolls and shoes laundry and Italian flowers {} none disdained the neighborhood {of the} other\\
\glt `Music, rolls and shoes, laundry and Italian flowers {\dots}, none disdained the company of the others'\\ (`{\dots} all were harmonious')\hfill(\href{https://books.google.co.jp/books?id=IfEFAAAAQAAJ&printsec=frontcover&source=gbs_book_other_versions_r&redir_esc=y#v=snippet&q=musik%2C%20rollen&f=false}{Goethe, \textit{Lehrjahre}})
\ex \il{French!ne@\textit{ne}}
\gll Pistolets, sabres recourbés et coutelas, rien ne manquait pour lui donner l' apparence du plus expéditif {\dots} tueur d' hommes\\
 pistols sabres curved and knives nothing not {was missing} for him give the appearance {of the} most expedient {} killer of men\\
\glt `Pistols, curved sabres, and knives, nothing was missing to give him the appearance of the most efficient {\dots} killer of men.'\\ (`{\dots} everything gave him the appearance {\dots}')\hfill(\href{https://fr.wikisource.org/wiki/Page%3ASand_-_Consuelo_-_1856_-_tome_3.djvu/280}{Sand, \textit{Consuelo}})
\ex 
\gll sangene, indskrifterne, jordbærrene, intet blev glemt\\
 songs.\DEF{} inscriptions.\DEF{} strawberries.\DEF{} nothing was forgotten\\
\glt `The songs, the inscriptions, the strawberries, nothing was forgotten'\\(`{\dots} everything was provided') % Peter: I've no idea where this comes from.
\z
\z
But Mikkelsen's description is not quite correct, and the real explanation evidently is that the writer begins his sentence with the intention of continuing it in a positive form (\textit{the envious, angry {\dots} all can be mollified}, etc.) and then suddenly changes the form of his expression. Nor is it necessary, as Mikkelsen says, to have a whole series of words, as seen in \refp{ex:07-113}. Cf. Danish \refp{ex:07-114}.

\ea \label{ex:07-113}
 People, nobody, can do as they like in this world.\hfill(\href{https://archive.org/details/annveronicamoder0000hgwe/page/272/mode/2up?q=%22people%2C+nobody%22&view=theater}{Wells, \textit{Veronica} 258})
\z



\ea \label{ex:07-114}
\gll Mændene og endnu mindre kvinderne kender begrebet linned [i Japan]\\
 men.\DEF{} and even less women.\DEF{} know concept.\DEF{} linen in Japan\\
\glt `Few men, and even fewer women, are familiar with the concept of linen [in Japan]'\hfill(news 1915)
\z

\is{quantifiers!negative}
\is{comparison|(}
The confusion is somewhat similar to the one found when an enumeration of things that are wanting ends with \textit{no nothing} (\textit{no paper, no pen, no ink, no nothing}), which is meant as a negation of \textit{everything}: the origin of the phrase is, of course, to be explained from a desire to go on with \textit{no} + some other noun, but as the speaker can hit upon no more things to enumerate, he breaks off after \textit{no} and finishes with \textit{nothing}; \textit{no} thus is only seemingly an adjunct to \textit{nothing} \refp{ex:07-115}.

\ea \label{ex:07-115} no milk in the house! no nothing!\hfill(\href{https://archive.org/details/a582677302frouuoft/page/n233/mode/2up?q=%22no+milk%22&view=theater}{T. Carlyle, \textit{Life} 4.223})
\z

\is{auxiliary verbs}
\il{English!help@\textit{help}|(}\href{https://archive.org/details/newenglishdict05murrmiss/page/n231/mode/2up?view=theater}{\textit{NED}, \textit{Help} \textit{v.} 11~c} says ``Often erron. with negative omitted (\textit{can} instead of \textit{cannot}), e.g. \textit{I did not trouble myself more than I could help}~\vert~\textit{your name shall occur again as little as I can help}.'' But it would certainly be unidiomatic to say, as \citet[\href{https://archive.org/details/miscellaneousre01whatgoog/page/n314/mode/2up?view=theater}{296}]{whately1866miscellaneous} demands, \textit{more than I can \textsc{not} help}; % Whately italicizes "not"; retaining this indication of stress seems helpful (even if OJ ignores it)
the idiom is caused by the fact that every comparison with \textit{than} really implies a negative idea (\textit{he has more than necessary} implies `it is not necessary to have more', etc.) and it is on a par with the logic that is shown, for instance, in the French use of \il{French!ne@\textit{ne}}\textit{ne} (\textit{plus qu'il ne faut}) and in the dialectal \textit{nor} for `than'.

But there is some difficulty in explaining this meaning of \textit{help}; note that where in England it is usual to say \textit{I could not help admiring her}, Americans will often prefer the negative expression with \textit{but}: \textit{I could not help but admire her}. \il{English!help@\textit{help}|)}

\textit{Seldom or never} and \textit{seldom if ever} are blended into \textit{seldom or ever}, which is said to be frequent where the influence of the school is not strong; \href{https://archive.org/details/transactions-of-the-philological-society/page/n12/mode/2up?view=theater}{Ellis (\textit{Address}~12}) says \textit{Seldom or ever could I detect any approach to a labial}.
\is{comparison|)}
\is{confusion of negation|)}
\is{double negation|)}
