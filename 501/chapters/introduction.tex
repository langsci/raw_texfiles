\documentclass[output=chapter]{langscibook}
\title[Introduction]
      {An introduction to Otto Jespersen’s \textit{Negation in English and other languages} (1917)}
\author{Olli O. Silvennoinen\affiliation{Åbo Akademi University}}
\ChapterDOI{10.5281/zenodo.14843514} %will be filled in at production 
\abstract{}

\begin{document}
\maketitle
\label{ch:intro}
\section{Introduction}

Hardly any good introductory textbook to language change fails to present the following phenomenon:

\begin{quote}\is{history of negative expressions}
    The history of negative expressions in various languages makes us witness the following curious fluctuation: the original negative adverb\is{adverbs!negative} is first weakened\is{adverbs!weakening of negative}, then found insufficient and therefore strengthened, generally through some additional word, and this in its turn may be felt as the negative proper and may then in course of time be subject to the same developments as the original word.\hfill\hbox{(this volume, p. \pageref{p:first})}
\end{quote}

\is{Jespersen's cycle}
Thus opens the first chapter of Otto Jespersen’s \textit{Negation in English and other languages} (henceforth \textit{Negation}, following the convention of Jespersen’s autobiography, \cite{Jespersen1995}). Subsequent generations of linguists have learned to call this phenomenon Jespersen's Cycle, following \citet{Dahl1979}. There are many ironies related to the study of the Cycle, not the least of which is the fact that the book after whose author it was named almost did not happen. \textit{Negation} was published in 1917, during %at the time of
the First World War. Like many others of Jespersen’s publications, it is a by-product of what is probably his main work, the seven\hyp volume \textit{A Modern English grammar on historical principles} (\cite{Jespersen1909}, henceforth \textit{Modern English grammar}). The war had put this monumental project on hold – it was published by a company based in Heidelberg, Germany – and because of that, Jespersen had decided to publish the parts related to negation separately, augmented by observations on other languages.

The result is a slim volume of approximately 150 pages, but, as \citet[30]{McCawley1995} notes, it is ``strikingly comprehensive''. Hardly any detail of English negation, whether past or present, escapes Jespersen’s scrutiny, and he also makes a considerable number of observations on the negative expressions of other, mainly European languages, especially Danish and French. While the cyclical development of negative markers has dominated the discussions around Jespersen’s book, in reality \textit{Negation} is a much richer work, and the treatment of the cyclical renewal of negative markers is not the only valuable part of it – arguably, there are topics on which Jespersen succeeds rather better than in his analysis of Jespersen's Cycle.

This introduction aims to (i) place Otto Jespersen’s \textit{Negation} in its historical and intellectual context, (ii) present the structure and content of the work, %and 
(iii)~show some ways in which Jespersen’s \textit{Negation} has influenced subsequent research, and (iv) point out themes that recur in the book and which help us to understand both negation in general and %Otto 
Jespersen’s account of it in particular. Especially for the last aim, Jespersen’s work will be discussed in the light of more recent scholarship. %on the same topics. 

\section{Otto Jespersen}

Jens Otto Harry Jespersen was born in Randers, Denmark, on 16 July 1860. After graduating from grammar school in 1877, he initially continued a long family tradition and began to study law at the University of Copenhagen. By all accounts, this was an unhappy choice, and after four years he abandoned law. In September 1881, he began to study languages, with French as his main subject and English and Latin as subsidiary ones. \is{phonetics}In the first semester of his new studies, he attended \ia{Thomsen, Vilhelm}Vilhelm Thomsen’s lectures on phonetics, then a budding field outside of the largely historical mainstream of 19th-century linguistics. This was consequential in many ways: many of his early publications were to deal with phonetics and the role of spoken language in \is{second-language education}\is{foreign-language education}foreign language teaching. As a synchronic field, phonetics may also have contributed to Jespersen’s lifelong interest in the contemporary form of the languages that he studied, rather than only their historical development and classical texts. Finally, it was %Vilhelm 
Thomsen who in 1888 persuaded the freshly graduated Jespersen to consider writing his doctoral dissertation on English, a subject which was to have an opening at Copenhagen in a few years but without a clear candidate in sight to take on the role (in the letter in which this suggestion was made, Thomsen\ia{Thomsen, Vilhelm} also subtly lets on that the topic of said dissertation should be something other than phonetics). Jespersen took this up and in 1891, he defended his dissertation on English cases. Two years later in 1893, at the age of 33, Jespersen was appointed Professor of English Language and Literature at the University of Copenhagen. He held this position until 1925, when he turned 65, at the time unusually young %an unusually low age 
for retiring from the university. Even though he had retired, Jespersen continued to work on his research practically until the end of his life. He died in 1943, during Nazi Germany’s occupation of Denmark.

\is{Darwinist approach}\is{teleological approach}\is{functionalist approaches}\is{generative grammar}
Jespersen’s influence has been felt in various schools of linguistics. On the one hand, his Darwinist and teleological approach to language change \citep[see][]{Nielsen1989HF,McCawley1992} meshes well with the functionalist approaches that developed in the latter half of the 20th century in North America and Europe, since it points to the influence of communicative pressures on language structure \citep[1--5]{Givon1995}. On the other hand, Jespersen’s emphasis on universal principles underlying superficial variation across languages has been cited as an influence for mainstream generative grammar \citep[e.g.][]{Chomsky1975}. Jespersen has also been lauded by non-mainstream generativists, who have appreciated the empirical richness and theoretical depth of his studies; in an article analyzing Jespersen’s \textit{Negation}, \citet[37]{McCawley1995} even calls him ``in my opinion the most perceptive observer of human language who ever lived''. \is{corpus linguistics}Many corpus linguists recognize a precursor in Jespersen’s copious use of examples taken from real texts and in his attention to minor patterns in the data \citep[4--5]{Meyer2008}.

In spite of his wide-ranging influence, Jespersen is difficult to place in the intellectual landscape of his time. He was originally trained in the \is{Neogrammarian school}Neogrammarian school of historical\hyp comparative linguistics, but in his own work, he criticized the assumption of exceptionless sound laws \citep[344--345]{HovdhaugenEtAl2000} and sought to pay as much %equal 
attention to synchrony as to diachrony. Traditional \is{philology}philology is another clear intellectual debt, as shown in the carefully collected quotations from classic literary texts in his works, including \textit{Negation} -- but as a student, he campaigned against Latin as a compulsory subsidiary subject for modern language students at the University of Copenhagen, and as a professor he took part in finally abolishing it \citep[7]{Christophersen1989}. \is{structuralism}He was contemporaneous with structuralism and his ideas do resemble it in some respects -- but he was critical of %towards 
structuralists in his writings (\cite[346]{HovdhaugenEtAl2000}; \cite[119--132]{Koerner1999}). Similarly, structuralists were ambivalent towards his work, especially those aspects of it that could be seen as overly mentalistic; but at the same time, he had a strong influence on, for example, Bloomfield \citep[see][]{Falk1992}.

Jespersen’s iconoclasm and independence were also evident in his life beyond academia. In his farewell lecture given on his retirement as a professor in 1925, he criticized nationalism and declared conscription ``one of the nineteenth century’s most devilish inventions, something that has most potently contributed to detestable wars and with its systematic training in killing and its unqualified claim to obedience has had a demoralizing effect on many and many a young man'' \citep[3]{Jespersen1933}. On the other hand, his generally progressive and even radical outlook on politics has been somewhat overshadowed by the decidedly retrograde attitudes towards women and their language use displayed in his writings \citep[346--347]{HovdhaugenEtAl2000}.

It is fair to say that Jespersen had many, partly overlapping careers in linguistics, and he left a mark in virtually all the sub-fields that he worked in \citep[see][]{JuulNielsen1989,Falk1992,HovdhaugenEtAl2000}. His early work focused on phonetics, especially in Danish dialects but also in English. \is{language teaching}Another early focus was language teaching. Jespersen wrote his first school grammar of English while still an undergraduate. He also wrote other school \is{textbooks}textbooks and successfully promoted the use of modern teaching methods at the expense of the then prevalent grammar--translation method. He wrote on historical linguistics, producing both descriptive work and studies of a more theoretical nature. Furthermore, Jespersen was involved in the creation of artificial languages, contributing to Ido and creating Novial.

However, it is probably Jespersen’s work on grammar that has had the most profound influence on the field, and which is also the most prominent part of his published output as a scholar. His most important contribution in this field is the seven-volume \textit{Modern English grammar} \citep{Jespersen1909}, the last volume of which was published posthumously. As stated above, \textit{Negation} is a side product of the \textit{Modern English grammar}, as are many others of Jespersen’s publications over the four decades that he was writing it. Another work that merits mention is \textit{The philosophy of grammar} \citep{Jespersen1924}. To complement the largely descriptive nature of the \textit{Modern English grammar}, this work is more theoretical and is in many ways ahead of its time. The syntactic theory that is presented and developed in \textit{The philosophy of grammar} is used already in \textit{Negation}.

\section{\textit{Negation in English and other languages}}

\subsection{The context}

\textit{Negation} was published in 1917, when the First World War had been raging in Europe for three years. Denmark had managed to stay out of the war, and in the years preceding the publication of \textit{Negation}, Jespersen had largely been concerned with anti-war efforts together with other Scandinavian intellectuals and politicians. Much of the chapter in his autobiography dealing with the war years describes his participation in peace efforts rather lavishly funded by \ia{Ford, Henry}Henry Ford. In 1916, probably because of his position as professor of English, he was asked to go on a diplomatic mission to London to meet with journalists and politicians in order to prepare the ground for a peace deal, which would have been in Ford’s financial interest. %s. 
In his autobiography, Jespersen notes that the outcome of this trip was ``very slender'' and that he ``received the impression that a corresponding expedition to Germany had borne equally meagre fruit'' \citep[196]{Jespersen1995}. He also published a peace proposal of his own, suggesting the establishment of a United States of Europe, with Strasbourg as its Washington D.C., and translated Edward Carpenter’s anti-war pamphlet \textit{Never again} into Danish.

These aspects of his life are almost completely absent from the pages of \textit{Negation} (although the word \textit{non-belligerent} makes an appearance in \RefToRomanChapter{V}, as do example sentences that are thematically related to the war). In his autobiography, the publication of \textit{Negation} merits a very brief mention:

\begin{quote}
But I must not give the impression that during the war I was exclusively preoccupied with war and politics. Fortunately life had to go on regardless. Lectures and examinations had to be held, and were held. [...] I was still working on my large English grammar, but as the first volumes had appeared in Heidelberg and I could not expect to get a new volume published there in the very near future, I picked out some chapters which were particularly suited for comparison with other languages and reflections of a general linguistic and logical nature. This became the book \textit{Negation in English and Other Languages}, printed in the Historical-Philological Proceedings of the Royal Danish Academy of Sciences and Letters, 1917.\\\phantom{.}\hfill\citep[196--197]{Jespersen1995}
\end{quote}

\is{methodology, Jespersen's|(}\is{corpus linguistics}
When reading \textit{Negation} or its parent work, the \textit{Modern English grammar}, a striking feature is the large number of examples, most of them taken from authentic language use. This is particularly remarkable since Jespersen lived and worked many decades before computers, let alone linguistic corpora, were in common use. The secret to the examples is that Jespersen collected them systematically throughout his career \citep[247--251]{Jespersen1995}. Every example (together with a source reference) was written on a small piece of paper, which was also given a number according to a system that Jespersen himself noted was reminiscent of library classification. These examples were then arranged into boxes according to topic. The example that he gives in his autobiography is the English \textit{ing}-form: it formed class number 33. This category was divided into sub-categories such as \textit{ing}-form used as a noun (331), noun used as a plural (3312), and so on; the category from which the examples in \textit{Negation} have been taken, for anyone who may be %to those that are 
interested, was 63 \citep[250]{Jespersen1995}. If this sounds endearing, or possibly quixotic, it probably was, and at least among his colleagues, he seems to have been quite famous for the practice, as evidenced by this anecdote from his autobiography \citep[250]{Jespersen1995}:

\begin{quote}
One of my pupils is said to have once depicted my procedure like this: \mbox{Jespersen} gradually stuffs a quantity of slips into compartments in his drawers; when one compartment is full, a little bell rings, and another book is finished.
\end{quote}

This method allowed Jespersen considerable flexibility when dealing with his data: examples could be rearranged and reclassified, and when the time came to write things up, only the examples deemed best were included. \citet[94]{Francis1989} notes that Jespersen’s use of citations in the \textit{Modern English grammar} was influenced by the \textit{Oxford English dictionary} and that ``he intended to produce a grammatical counterpart'' to it.
\is{methodology, Jespersen's|(}

\subsection{The book and its contents}\largerpage

As \citet[29]{McCawley1995} notes, the structure of \textit{Negation} is ``rather peculiar'' for a modern reader. Just by looking at the table of contents, it is difficult to discern any narrative or guiding principle for how the contents are organized, as one might expect of a monograph published today. In the following, I will divide the book’s contents into three broad parts. The first part is diachronic, the other two mainly synchronic.

\is{Jespersen's cycle}
The first three chapters deal with the diachrony of negation. \RefToRomanChapter{I}, titled ``General tendencies'', is probably the most widely cited in the book. It introduces what has come to be known as Jespersen's Cycle, illustrating it with Latin and French, Old Norse and its descendants, as well as English. \RefToRomanChapter{II}, ``Strengthening of negatives'', continues with the theme, providing a list of etymological sources for negative strengtheners found in European languages. In \RefToRomanChapter{III}, ``Positive becomes negative'', the focus turns to the process by which positive elements come to be reanalysed as negatives.

What I consider to be the second part of \textit{Negation} consists of seven chapters. These chapters consider negation as a function from a synchronic perspective. \RefToRomanChapter{IV}, ``Indirect and incomplete negation'', treats two rather different \mbox{topics}. Under the rubric of ``indirect negation'', Jespersen discusses various constructions that imply a polarity value that is the opposite of their literal meaning; examples include questions that imply a negative statement (e.g.\ \textit{Am I the guardian of my brother?}) and conditional constructions (e.g.\ \textit{If I understand thee, I am a villain}, which implies `I am not a villain'). Since some of the examples are negative themselves, implying a positive (e.g.\ \textit{if it isn’t a pity} to imply `it is a pity'), a more appropriate title might be ``Polarity reversals'' or ``Negation and indirectness''. ``Incomplete negation'' on the other hand refers to such words as \textit{hardly} and \textit{few}. \RefToRomanChapter{V} is titled %called 
``Special and nexal negation''. Negation is \textit{nexal} if it applies to a nexus, i.e.\ a combination of two positive ideas; Jespersen’s example for this is \textit{He doesn’t come today}, which negates the combination of \textit{he} and \textit{coming today}. By contrast, negation is \textit{special} if it relates to only one idea; examples that he cites include \textit{never, unhappy, impossible, disorder} and \textit{non-belligerent}. The concept of nexus is central to Jespersen’s syntactic theorizing \citep{Jespersen1924,Francis1989}; I will return to it in \sectref{sec:nexalSpecial}. \RefToRomanChapter{VI} is on ``Negative attraction'', i.e.\ the locus of negation marking in a clause. In \RefToRomanChapter{VII} (``Double negation''), Jespersen returns to the presence of two negative expressions in one clause. \RefToRomanChapter{VIII} is entitled ``The meaning of negation''. Here, Jespersen discusses the interaction of negation with quantifiers and modals, treading ground that was later analysed under the notion of scalar implicature by \citet{Horn1989} and many others. \citet{McCawley1995} considers this chapter the weakest in the book. In \RefToRomanChapter{IX}, entitled ``Weakened negatives'', Jespersen turns to cases in which an explicit negative has little negative force (for example, \textit{Won’t you have a glass of beer?} is nearly equivalent to \textit{Will you have a glass of beer?}). \RefToRomanChapter{X} (``Negative connectives'') discusses negative connectives such as \textit{neither} and \textit{nor}.\largerpage

While the preceding chapters discuss negation using concepts that are at least broadly applicable across languages, in the last three chapters, the focus turns to negative forms that are specific to English. Chapters~\ref{ch:11} and~\ref{ch:12} in particular are also as much about diachrony as about synchrony. \RefToRomanChapter{XI} (``English verbal forms in \textit{n’t}'') discusses contracted forms in English, with a lot of emphasis placed on their pronunciation. \RefToRomanChapter{XII}, entitled \textit{But}, offers a historical treatment of the various negation-related meanings of the word. Finally, \RefToRomanChapter{XIII}, ``Negative prefixes'', is a detailed account of \textit{un-}, \textit{in-}, \textit{dis-}, \textit{non-} and \textit{a-} in English.

Put together, the chapters in \textit{Negation} offer a detailed panorama of negation in English and other European languages. While non-specialists sometimes think of negation as a rather boring corner of grammar, a simple reversal of truth value, Jespersen’s \textit{Negation} would hopefully convince them that negation is a rich source of interesting expression types that is continually in movement, and thus an appropriate topic for showcasing the blend of synchrony and diachrony that was typical of Jespersen’s work.

\section{Themes: What \textit{Negation} is and is not}

In this section, I will discuss some themes that cut across chapters in \textit{Negation} or that are important for understanding how it relates to other studies about negation, or by Jespersen. The selection of topics is not meant as exhaustive, perhaps not even as representative, and it necessarily reflects my personal biases and interests. I also try %have also tried 
to cover ground not wholly trodden by \citet{McCawley1995} in his article on \textit{Negation}, although some overlap is inevitable, notably on the distinction between nexal and special negation as well as Jespersen’s account of what has come to be known as pragmatic meaning.

The contents of this section are not intended as the final word on Jespersen, on \textit{Negation} or on negation (and even if they were so intended, they could not be that). Readers who are relatively new to the area can take this section as a collection of signposts, or of things to be on the lookout for. More advanced readers may wish to read the book for themselves first, and only come back later when they have formed their own opinions, to check if they agree.


\subsection{Grammaticalization and Jespersen's Cycle}%the Jespersen Cycle}
\label{sec:grammJesp}\is{Jespersen's cycle|(}

\is{grammaticalization|(}
For better or for worse, \textit{Negation} is probably best-known for its account of what is today known as Jespersen's Cycle, a name given to it by \citet{Dahl1979}. While Jespersen also illustrates the phenomenon with English and Scandinavian, the classic example is French. The number of stages that have been posited has varied from three to six \citep[37--39]{Auwera2009}, but the basic facts behind the varying numbers of stages are not in serious dispute. \Citet[39]{Auwera2009} argues that a six-stage model ``captures Jespersen (1917) better than the simpler schemes'', as shown in (\ref{tab:stages}) (see also \cite[]{Hansen2012}):
\il{French!ne@\textit{ne}|(}
\il{French!pas@\textit{pas}|(}
\il{Latin!non@\textit{non}|(}

\ea \label{tab:stages}
    \begin{tabular}[t]{ll}
    {Stages} & {Strategies} \\\addlinespace
    1 & non\textsubscript{\textsc{neg}} \\
    2 & ne\textsubscript{\textsc{neg}} \\
    3 & ne\textsubscript{\textsc{neg}}\dots pas\textsubscript{\textsc{x}} \\
    4 & ne\textsubscript{\textsc{neg}}\dots pas\textsubscript{\textsc{neg}} \\
    5 & ne\textsubscript{\textsc{x}}\dots pas\textsubscript{\textsc{neg}} \\
    6 & pas\textsubscript{\textsc{neg}}
    \end{tabular}
\z



Stage~1 is the non-reduced pre-verbal Latin negative marker \textit{non} (\textit{Non dico} `I do not say’). In stage~2, \textit{non} has \is{phonological reduction}phonetically reduced into \textit{ne} but remains a negative marker (\textit{Jeo ne dis}). Stage~3 represents the situation in Old French, whereby \textit{ne} was optionally accompanied by a variety of \is{minimizers}minimizers, such as \textit{mie} `crumb’ and \textit{pas} `step’ (\textit{Je ne dis {\op}pas})). Over time, adverb\is{adverbs!negative} \textit{pas} became more frequent than the other alternatives, and it became an obligatory part of the French standard negation construction (\textit{Je ne dis pas}). It was thus \is{reanalysis}reanalyzed as a \is{negators}negator in its own right rather than as a minimizer; this is stage~4. Once this was done, in stage~5, the relationship between \textit{ne} and \textit{pas} is recast: now \textit{pas} is the negative marker and \textit{ne} has another function, possibly to reinforce the negation -- indeed, the use of \textit{ne} in spoken French is now pragmatically special \citep{FonsecaGreber2007}. Colloquial French is largely in stage~6, in which \textit{pas} can be used to negate a clause on its own (\textit{Je dis pas}), the alternative which has been dominant in spoken varieties of French for decades if not longer. One language can occupy several stages at once; while spoken French is somewhere between stages~5 and~6, stage~2 constructions still remain in more formal registers of French for certain verbs, and the written standard is in stage~4.
\il{French!ne@\textit{ne}|)}
\il{French!pas@\textit{pas}|)}
\il{Latin!non@\textit{non}|)}

One myth needs to be dealt with right away: Jespersen was \textit{not} the first linguist to analyse this phenomenon, let alone observe it. In the 19th and early 20th centuries, most if not all western linguists had a thorough command of both French and Latin, and the etymological connection between Latin \textit{passum} and French \textit{pas} seems to have been common knowledge. A few years before Jespersen published his book, Antoine \citet{Meillet1912} had published an article that introduced the term \textsc{grammaticalization} and which also discusses the development of the French negative construction, albeit in much less detail than Jespersen. \Citet{Auwera2009} cites even earlier remarks on the topic, going back to the 19th century. Meillet also describes the cyclical nature of change in negatives in terms strongly reminiscent of Jespersen’s formulation as summarized above:

\begin{quote}
Les langues suivent ainsi une sorte de développement en spirale : elles ajoutent des mots accessoires pour obtenir une expression intense ; ces mots s’affaiblissent, se dégradent et tombent au niveau de simples outils grammaticaux ; on ajoute de nouveaux mots ou des mots différents en vue de l’expression ; l’affaiblissement recommence, et ainsi sans fin.

\medskip

\noindent
`Languages thus follow a kind of spiral development: they add accessory words to obtain an intense expression; these words weaken, are reduced and fall to the level of mere grammatical tools; new words are added, or different words are added for the sake of %depending on the 
expression; the weakening starts again, and so on %on it goes 
without end.'\hfill
\citep[130--148]{Meillet1921} [my translation]
\end{quote}

% consider “for the purpose of expression”
% consider “so on”

Not only did Meillet analyse the ``Jespersen Cycle'' earlier than Jespersen, he also seems to have done a better job in some respects. Jespersen’s account leans heavily on a phonetic explanation for the rise of \textit{pas}: the reason for strengthening \textit{ne} with \textit{pas} and other minimizers is its phonetic erosion, an explanation that may reflect Jespersen’s earlier interest in phonetics. By contrast, as the quote above %from Meillet shows, he 
shows, Meillet identified the speaker’s wish for \is{expressiveness}expressiveness as the real cause for the cycle. Later research into %of 
Jespersen's Cycle has found Meillet’s explanation to be closer to the facts \citep[see][]{Auwera2009}. \textit{Pas} originally occurred in contexts where the negated content was contextually accessible and thus its negation needed to be more forceful, given that the negation targeted something that was presumably accepted as true by the hearer \citep{HansenVisconti2009}. Similarly, the \is{double negation}double negation of Brazilian Portuguese is only felicitous in contexts where the negated content is activated \citep{Schwenter2005}.

Still, there might be something to Jespersen’s idea of \is{phonological reduction}phonetic weakening. \Citet{Auwera2009} points out that the phonetically reduced \textit{ne} may have been too inconspicuous for the kind of emphasis that is needed for negating contextually activated content. Thus, phonetic weakening could be integrated into the activation-based account of the Cycle.
\is{activation and accessibility}
\is{grammaticalization|)}

The name ``Jespersen Cycle'' is thus somewhat unfortunate, but it has stuck. Part of the reason is that, despite its flaws, Jespersen’s is the first relatively full account that demonstrates, rather than merely takes for granted, the parallel developments across different languages and analyses them in depth. Perhaps more importantly, a good alternative term is difficult to come by. The most obvious replacement, ``negative cycle'', is too vague because there are other cyclical developments related to negation, such as the negative existential cycle \is{Croft Cycle}(sometimes called the ``Croft Cycle'') and the quantifier cycle \citep{Croft1991,Hansen2012,AuweraKrasnoukhovaVossen2022}. Therefore, the term \textit{Jespersen Cycle} is probably to be preferred after all.

Jespersen only discusses single and \is{double negation}double negatives. Subsequent research has found that two is not the upper limit for the number of negative markers in standard negation: the cycle may start again in the double negation stage. \citet{VossenAuwera2014} cite the example of Lewo, shown in~(\ref{ex:lewo}), which they argue to have quadruple negation:

\ea Lewo (Austronesian; \citealp[405]{Early1994})\label{ex:lewo} 
   \\\gll pe-re a-pim re poli\\
   \textsc{neg-neg} 3\textsc{pl.sbj-r}.come \textsc{neg} \textsc{neg}\\
   \glt `They didn’t come.'
\z
\is{Jespersen's cycle|)}


\subsection{Nexal and special negation}
\label{sec:nexalSpecial}
\is{nexal negation!and special negation|(}

\is{junction}\is{rank}
The three main concepts in Jespersen’s theory of grammar are \textsc{rank}, \textsc{junction} and \textsc{nexus} \citep{Jespersen1924,Francis1989}. All these concepts relate to units that are at least potentially combinations of more than one word. \textsc{Rank} refers to levels of hierarchical structure in syntax: in \textit{extremely hot weather}, for example, what would now be called the \textsc{head} (\textit{weather}) is in Jespersen’s terms ``primary'', its dependent (\textit{hot}) ``secondary'' and the dependent’s dependent (\textit{extremely}) ``tertiary'' \citep[96]{Jespersen1924}. \textsc{Junction} and \textsc{nexus}, in turn, are ways of combining elements into hierarchically ordered wholes. In \textsc{junction}, elements are combined to form a single idea. In \textsc{nexus}, two elements are combined that ``must necessarily remain separate'' \citep[116]{Jespersen1924}. In \textit{The philosophy of grammar}, Jespersen’s go-to example for these two concepts is the pair of expressions \textit{the furiously barking dog} (junction) and \textit{The dog barks furiously} (nexus). In both, \textit{dog} is primary, \textit{bark/barking} secondary, and \textit{furiously} tertiary \citep[97,~114]{Jespersen1924}. In the prototypical case, the nexus contains a finite verb. \is{nexus types}However, there are also other kinds of nexus in English \citep[117--131]{Jespersen1924}: \textsc{infinitival nexus} (\textit{I heard \textsc{her sing}} combines `she’ and `sing’), \textsc{verbal substantive} (\textit{I heard of \textsc{the Doctor’s arrival}} combines `the Doctor’ and `arrive’), \textsc{verbless nexus} (\textit{the more fool I} combines `I’ and `the more fool’), \textsc{nexus-object} (\textit{I found \textsc{the cage empty}} combines `the cage’ and `empty’), \textsc{nexus subjunct} (\textit{We shall go, \textsc{weather permitting}} combines `weather’ and `permit’), and \textsc{nexus of deprecation} (\textit{Me dance!} combines `I’ and `dance’ and expresses an indirect negation).

Of Jespersen’s three concepts, only the nexus is drawn on explicitly in \textit{Negation}, when Jespersen makes a distinction between \textit{nexal} and \textit{special} negation in \RefToRomanChapter{V}. The distinction between nexal and special negation displays the insight that \is{clausal negation}clausal negation is quite different from negation of a constituent. However, the concepts are not entirely successful or clearly applied: for example, it is not clear why \il{English!never@\textit{never}}\textit{never} could not effect nexal negation, despite the etymology of \textit{never} as the negation of \textit{ever}.

Jespersen seems to oscillate between a \is{formal v. functional definitions}formal and a functional definition of nexal and special negation. On the one hand, they are meant as grammatical, and thus at least partly formal, categories: Jespersen writes accordingly that ``[i]n [Modern English] the use or non-use of the auxiliary \textit{do} serves in many, but not of course in all, cases to distinguish between nexal and special negation'' (this volume, p. \pageref{p:nexal-sp}). However, two pages before this statement, he has the example \textit{He doesn’t smoke cigars, only cigarettes}, which he seems to argue is special because the negation only applies to \textit{cigars}, not to the nexus of `he’ and `smokes cigars’. Here, the distinction between nexal and special negation seems to be purely functional. In \textit{The philosophy of grammar}, nexus is defined in purely notional terms. Additionally, it is difficult to square Jespersen’s discussion of nexal negation in \textit{Negation} with the wide variety of nexus types in \textit{The philosophy of grammar}, which were listed above. Many of these are not clausal at all and thus are not relevant to %for 
clausal negation.

In Jespersen’s defence, this area of English grammar has been difficult to make sense of, since syntactic expression and semantic interpretation do not always seem to go hand in hand. It is probably best to keep syntactic and semantic analyses conceptually distinct, so that the syntactic distinction between clausal and constituent negation need not correspond to the semantic distinction between negating a proposition or a term \citep[cf.][]{Klima1964}. As a further issue, Jespersen’s class of special negation is rather heterogeneous, and as \citet[33]{McCawley1995} notes, Jespersen’s generalizations of it do not really apply to all its members.

Subsequent typological research has applied the term \textit{special negation} in a different way, as the opposite of \textit{standard negation} (e.g.\ \citealp{Veselinova2013}). \textit{Standard negation} is ``the basic way(s) a language has for negating declarative verbal main clauses'' \citep[1]{Miestamo2005}; \textit{special negations} are any negative constructions that differ from the standard negation construction of the language in question. \mbox{Jespersen’s} special negations are a subclass (or, rather, several subclasses) of the special negations of typology. For example, a language that has a separate negative marker \is{attraction!double}for non-verbal predication, for existential predication or for subordinate clauses \citep[see][]{Miestamo2017} would be said to have a special negative construction even though, in Jespersen’s terms, that construction would apply to a kind of nexus.
\is{nexal negation!and special negation|)}

\subsection{Competing motivations}

\is{attraction!of negative to earlier position}
\is{negative first principle|(}
\is{position of negative|(}
Several of the chapters in \textit{Negation} are concerned with the placement of the negative marker, and Jespersen formulates several generalizations concerning this issue. The most famous of these is what \citet[449]{Horn1989} has called the Neg-First principle: all other things being equal, negative markers precede their focus. Another principle discussed by Jespersen is what might be called the \textit{neg-verbal} principle: \is{attraction!of negative to verb}negation is attracted to the finite verb. Thirdly, \is{attraction!of negative other than to verb}negation has a tendency to be attracted to \is{adverbs!negative}``any word that can easily be made negative'' (this volume p. \pageref{p:made-neg}). For example, \is{ambiguity!related to reason}\textit{We didn’t meet anybody} corresponds to 
\il{English!nobody@\textit{nobody}}\textit{We met nobody}. For the sake of completeness, let us call this tendency the principle of \is{neg-incorporation}\textit{Neg-Incorporation}.

\is{competing motivations|(}
\is{neg-verbal principle|(}
Pressures such as Neg-First, Neg-Verbal or Neg-Incorporation are functional in nature: they serve a communicative purpose. If there is a universalizing side to Jespersen’s \textit{Negation}, it probably lies in these principles. In the terms of present-day functionalism, Jespersen thus posits \textit{competing motivations} for the placement of the negative marker \citep{DuBois1985}. While Jespersen’s aims cannot be said to be typological or quantitative by today’s standards, at least the Neg-First and Neg-Verbal principles are borne out by later typological investigations: negative markers do tend to precede the verb that they are negating \citep[97--98, 101]{Dryer1992,Dryer2013a}, and the markers themselves are very commonly either verbal affixes or auxiliaries or, if they are independent words, belong to word classes that easily attach to verbs, such as particles \citep{Dahl1979,Dryer2013b}; see also \citet{MiestamoShagalSilvennoinen2022} for discussion of these findings from a competing motivations perspective.

Jespersen identifies a further competition of motivations in what \citet{McCawley1991} later called \is{contrast, function of}\textsc{contrastive negation}, i.e.\ ``combinations of affirmation and negation in which the focus of negation is replaced in the affirmative part of the expression'' \citep[10]{Silvennoinen2019}. Consider the pair~(\ref{ex:contrastive}a), which Jespersen cites from Oscar Wilde, and its near-equivalent in~(\ref{ex:contrastive}b):

\ea\label{ex:contrastive}
    \ea My ruin came not from too great individualism of life, \\but from too little.
    \ex My ruin didn’t come from too great individualism of life, \\but from too little.
\z\z

Contrastive negation allows the \is{attraction!of negative other than to verb}negation to be attached to the focus constituent rather than the nexus, as in (\ref{ex:contrastive}a). The nexal variant is also possible, however, as shown in (\ref{ex:contrastive}b). Contrastive negation has largely been ignored in the literature on negation post-1917. That Jespersen discusses it, and in several places, is further testament to his attention to seemingly small details of grammar.
\is{competing motivations|)}
\is{position of negative|)}
\is{neg-verbal principle|)}

\subsection{The limits of negation (and \textit{Negation}): Towards pragmatic meaning}

Many of the synchronic chapters in \textit{Negation} have to do with topics that would now fall under the rubric of pragmatics, or at the very least near it. In other words, they deal with uses of negation that are not literally negative, such as when Jespersen discusses \textsc{paratactic negatives} or double negatives that do not cancel one another (\RefToRomanChapter{VII}), or expressions that are not formally negative but have negative force, such as in the section on what Jespersen calls ``indirect negation'' (\RefToRomanChapter{IV}), or negative expressions whose precise type of negative meaning (contradictory or contrary) needs to be determined contextually (\RefToRomanChapter{VIII}). These sections have an unevenness to them: in my view, some of these passages are among the best in the whole book: the reader truly gets a feel for the almost surgical precision with which Jespersen could observe language use, putting construction after construction and example after example under the scalpel. Unfortunately, these chapters also bring to light the defects in the theoretical apparatus that he is using.

\is{pragmatics and pragmatic meaning|(}
Jespersen wrote \textit{Negation} several decades before developments in ordinary language philosophy, sociology and anthropology coalesced into the research programme that is today known as linguistic pragmatics. Even semantics was a rather tender branch of theoretical linguistics at the time. Yet, because of its multifarious semantics and rich contextual uses, negation requires a pragmatic approach, and Jespersen is clearly aware of this. The problem for him is that the kind of approach that he would have needed had not yet been developed.

\is{implicature|(}
Retrospective evaluation has its risks, but it looks as if Jespersen is grasping towards something like a Gricean implicature in some of his analyses \citep[cf.][]{Grice1975,Horn1989}. In \RefToRomanChapter{IV}, for example, Jespersen goes through various constructions in which the recipient makes a negative inference out of a positive expression, or vice versa. An example is the use of a question (\textit{Am I the guardian of my brother?}) to imply a negative (`I am not the guardian of my brother’). A standard Gricean account would treat this inference as a conversational implicature. However, some of the inferences are indirect indeed: for example, comparative constructions such as \textit{She is richer than you think} are claimed to imply `you do \textit{not} think that she is so rich as she really is’. While this is then connected to the \is{pleonastic negation}pleonastic negation in the corresponding French construction (\textit{Elle est plus riche que vous \textsc{ne} croyez}), the connections between the expressions of indirect negation and their inferences remain rather loose.

\is{scalar implicature|(}
Another problem relates to \RefToRomanChapter{VIII}, in which Jespersen treats what has since become known as scalar implicature. As in many other publications, Jespersen discusses scales through three terms, such as the ones in~(\ref{ex:scale}):

\ea\label{ex:scale}
\begin{tabular}[t]{lll}
A & B & C \\
all & some & none \\
everybody & somebody & nobody \\
always & sometimes & never \\
must & can & cannot
\end{tabular}
\z

\il{English!nobody@\textit{nobody}}
As \citet[35]{McCawley1995} puts it,

\begin{quote}
[Jespersen] offers generalizations about the meanings of combinations of negation with these \is{A, B, and C tripartition}three categories, for example, that the negations of A and C are in the category B, for which he gives equations such as ``not all, not everything = something'' and that A and C applied to a negation are equivalent to C and A respectively, e.g.\ ``\textit{Everybody was unkind} = \textit{Nobody was kind}''. [...] The above equations bring Jespersen perilously close to the absurd conclusion that A = C, that is, that \textit{everything} = \textit{nothing}.
\end{quote}

One solution for this conundrum, offered by \citet{Horn1972,Horn1989} and endlessly debated ever since, is to distinguish meaning from implicature. On such an account, a scalar term such as \textit{some} means `at least some' and conversationally implicates `no more than some'. Thus, the negation of an A term (`not all') can mean either B or C. Again, Jespersen is aware of the two readings, but the theoretical machinery is not yet in place.\footnote{There are %also (removed to avoid one word spilling onto the third line in the PDF.)
alternatives to the standard Gricean account developed by Horn. See e.g., \citet{Chierchia2017} and \citet{Sauerland2012} for reviews and arguments for a view contrary to the Gricean one.}
\is{pragmatics and pragmatic meaning|)}
\is{implicature|)}
\is{scalar implicature|)}

\section{Conclusion}

In this introduction, I have tried to show that Jespersen's \textit{Negation} can tell us much more than what happened to the French word \textit{pas}. Indeed, it is only when set against the backdrop of Jespersen's pragmatic conception of language that his account of Jespersen's Cycle can be properly understood and appreciated (and critiqued, as shown above). I have also tried to give the reader a glimpse of the man behind the book.

It is the fate of many classics that they end up more cited than read. I hope that this reissue will make this book available to new readers, and I also hope that, after reading \textit{Negation}, they will feel inspired to try out some of Jespersen's other books.

\printbibliography[heading=subbibliography]
\end{document}
