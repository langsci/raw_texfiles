\ChapterAndMark{The Meaning of Negation} 
\label{ch:8}
\is{excluded middle, law of}
\il{English!not@\textit{not}|(}
A linguistic negative generally changes a term into what logicians call the contradictory term (A and not-A comprising everything in existence) and is thus very different from a negative in the mathematical sense, where \(-4\) means a point as much below 0 as \(4\) (or \(+4\)) is above \(0\). We have, however, seen instances in which a negative changes a term into the ``contrary term'', as when \textit{he begins-not to sing} (for \textit{he begins not-to-sing}) comes to mean `he ceases singing' (p.~\pageref{para:not-to-sing}).

\is{inequality, expressions of|(}
If we say, according to the general rule, that \textit{not four} means `different from four', this should be taken with a certain qualification, for in practice it generally means, not whatever is above or below \(4\) in the scale, but only what is below \(4\), thus less than \(4\), something between \(4\) and \(0\), just as \textit{not everything} means something between everything and nothing (and as \textit{not good} means `inferior', but does not comprise `excellent'). Thus in \textit{He does not read three books in a year}; \textit{the hill is not two hundred feet high}; \textit{his income is not £200 a year}; \textit{he does not see her once a week}.

\il{English!none@\textit{none}}
\il{English!nothing@\textit{nothing}}
\il{English!Old English!nan@\textit{nan}}
\il{English!Old English!nanþing@\textit{nanþing}}
\is{not one@`not one', expressions meaning}
\is{nothing@`nothing', expressions meaning}
\il{German!k-ein@\textit{k-ein}}
This explains how \textit{not one} comes to be the natural expression in many languages for `none, no', and \textit{not one thing} for `nothing', as in Old English \textit{nan} = % ??? Peter: I don't know what the "=" means here. (I can guess, but the guess could well be wrong.)
%Brett: ¯\_(ツ)_/¯
\textit{ne-an}, whence \textit{none} and \textit{no}, Old English \textit{nanþing}, whence \textit{nothing}, Old Norse \il{Old Norse!eingi@\textit{eingi}}\textit{eingi}, whence Danish \textit{ingen}, German \textit{k-ein}, etc. Cf. also (\ref{ex:08-01}). See also p.~\pageref{para:kind-of-stronger-no}. In French similarly (\ref{ex:08-02}), etc.

\ea \label{ex:08-01}
That \emph{not one} life shall be destroy'd {\dots} That \emph{not a} worm is cloven in vain\\\hfill(\href{https://en.wikisource.org/wiki/In_Memoriam_(Tennyson)/Canto_53}{Tennyson, \textit{Memoriam} 53})
\z


\ea \label{ex:08-02}
\gll \emph{Pas} \emph{un} \emph{bruit} \emph{n'} interrompit le silence\\
 not a noise not interrupted the silence\\
\glt `Not a sound broke the silence'
\z

When \textit{not} \(+\) a numeral is exceptionally to be taken as `more than', the numeral has to be strongly stressed, and generally to be followed by a more exact indication: (\ref{ex:08-03}).

\ea \label{ex:08-03}
\ea the hill is not \emph{two} hundred feet high, but \emph{three} hundred
\ex his income is not 200, but at least 300 a year
\ex
Not one invention, but fifty---from a corkscrew to a machine-gun\\\hfill(\href{https://archive.org/details/septimus00unkngoog/page/n297/mode/2up?q=%22not+one+invention%22&view=theater}{Locke, \textit{Septimus} 321})
\ex
not once, but two or three times\hfill(\href{https://archive.org/details/lifeandstranges00dobsgoog/page/n367/mode/2up?q=%22not+once%2C+but%22&view=theater}{Defoe, \textit{Robinson} 342})
\ex
books that well merit to be pored over, not once but many a time\\\hfill(\href{https://archive.org/details/privatepapersofh0000geor/page/158/mode/2up?q=%22Books+that+well+merit%22&view=theater}{Gissing, \textit{Henry} 149})
\ex
he would bend to kiss her, not once, not once only\\\hfill(\href{https://archive.org/details/arundel00bens/page/202/mode/2up?q=%22would+bend+to+kiss+her%22&view=theater}{E. F. Benson, \textit{Arundel} 220})
\z
\z

But \textit{not once or twice} always means `several times', as in (\ref{ex:08-07}).

\ea \label{ex:08-07}
\ea
Not once or twice in our rough island-story, The path of duty was the way to glory\hfill(\href{https://en.wikisource.org/wiki/Maud,_and_other_poems/Ode_on_the_Death_of_the_Duke_of_Wellington}{Tennyson, \textit{Wellington}})
\ex
He bullied and punished me; not two or three times in the week, nor once or twice in a day, but continually.\hfill(\href{https://archive.org/details/JaneEyre-CharlotteBronte/page/n11/mode/2up?q=%22punished+me%22&view=theater}{Brontë, \textit{Jane} 4}) % Restored "and punished"; not "the day" but "a day"
\z
\z
\is{inequality, expressions of|)}

In Russian, on the other hand, \il{Russian!ne raz@\textit{ne raz}}\textit{ne raz} `not (a) time', thus really without a numeral, means `several times, sometimes' and in the same way \il{Russian!ne odin@\textit{ne odin}}\textit{ne odin} `not one' means `more than one'; corresponding phenomena are found in other languages as well, see \citet{schuchardt1894anaugust}, a valuable little article.
% PE: I've cut OJ's comment "(privately printed)": if it belongs anywhere, I think it belongs in the BibTeX entry; and the number of copies Worldcat shows are held in libraries suggests to me that it may not be correct.
%Brett: OK
\il{Russian!ni@\textit{ni}}\il{Russian!ni odin@\textit{ni odin}}
He rightly connects this with the use in Russian of the stronger negative \textit{ni} with a numeral to signify `less than', \textit{ni odin} `not even one'.

What the exact import is of a negative quantitative indication may in some instances depend on what is expected, or what is the direction of thought in each case. While the two sentences \textit{he spends £200 a year} and \textit{he lives on £200 a year} are practically synonymous, everything is changed if we add \textit{not}: \textit{he doesn't spend £200 a year} means `less than'; \textit{he doesn't live on £200 a year} means `more than'; because in the former case we expect an indication of a maximum, and in the latter of a minimum.

Or, perhaps, the explanation is rather this, that in the former sentence it does not matter whether we negative the nexus or the numeral (\textit{he does-not-spend £200}; \textit{he spends not-£200}), but in the latter it changes the whole meaning, for \textit{he does-not-live on 200} states the impossibility of living on so little, and \textit{he lives on not-200 a year} (which is rendered more idiomatic if we add an adverb: \il{English!not quite@\textit{not quite}}\textit{on not quite 200 a year}) states the possibility of living on less than 200. In the former sentence the numeral thus is not negatived at all. Compare also: \textit{he is not content with 200 a year} and \textit{he is content with not 200 a year}.

In the proverb \textit{Rome was not built in a day} (where \textit{a} is the old numeral and equals \textit{one}) the meaning also, of course, is that it took more than one day to build Rome. Thus also in (\ref{ex:08-09}).

\il{French!ne ... pas@\textit{ne ... pas}}
\ea \label{ex:08-09}
\gll On ne bâtit pas un art musical en un jour\\
 one not builds not an art musical in a day\\
\glt `One doesn't establish a musical tradition in a day.'\hfill(\href{https://www.gutenberg.org/cache/epub/61970/pg61970-images.html}{Rolland, \textit{Amies} 98}) % PE: https://dictionnaire.lerobert.com/definition/art-musical suggests to me that "art musical" is (in the lexical semantics sense!) compositional; but if it is, I don't understand it as the object of "bâtir". "a musical work of art"?
%Brett: Better? This translation reflects Christophe's (and by extension, Rolland's) vision of creating a musical culture that is deeply integrated with society and accessible to all, rather than an elite or isolated artistic pursuit.
% PE: Got it!
\z

Where a numeral is not used as a point in an ascending scale, its negative is really contradictory: \textit{the train doesn't start at seven} says nothing about the actual time of starting, which may be either before or after seven. But \textit{he won't be here at seven} implies `we can't expect him till after seven', because an arrival before 7 o'clock would naturally imply his being here also at that hour. 

\is{not at all@`not all', expressions corresponding to}
\il{English!not half@\textit{not half}}
As \textit{half} is a numeral, \textit{not half} generally means `less than half': \textit{the bottle is not half full}. In slang, \textit{not half bad} means, however, `not at all bad, quite good'. In the following quotation (\ref{ex:08-10}), \textit{not half-alive} (with strong stress on \textit{half}) means `more than half alive', as shown also by the continuation. In the same way, in rustic speech, \textit{she didn't half cry} means that she made a tremendous noise \citep[\href{https://www.gutenberg.org/cache/epub/47364/pg47364-images.html\#Page_117}{117}]{wright1913rustic}. 

\ea \label{ex:08-10}
At any rate she was not half alive; she was alive in every particle of herself\hfill(\href{https://archive.org/details/clayhanger01benngoog/page/238/mode/2up?q=%22rate+she+was+not+half%22&view=theater}{Bennett, \textit{Clayhanger} 1.286})
\z

\il{English!not quite@\textit{not quite}}
\textit{Not quite the average} generally means `below the average'; sometimes, however, \textit{average} is taken as a depreciating epithet, and then the negative may be appreciatory: (\ref{ex:08-11}).

\ea \label{ex:08-11}
Here is another piece of work which is not quite average; it is better than the average.\hfill(\href{https://archive.org/details/schoolsociety00dewerich/page/n71/mode/2up?q=%22here+is+another+piece+of+work%22&view=theater}{Dewey, \textit{School} 61})
\z
\il{English!not@\textit{not}|)}

\il{English!no less than@\textit{no less than}|(}
\is{comparatives, negativing|(}
\textit{Not above 30} means either 30 or less than 30. But \textit{less than 30} may in English be negatived in two ways: \textit{not less than 30} means either 30 or more than thirty, and \textit{no less than 30} means exactly 30, implying surprise or wonder at the high number. \textit{He has not less than ten children}---I am not certain of the exact number, but it is at least ten. \textit{He has no less than ten children}---he has ten, and isn't that a large family? In the same way with \textit{more}. On this distinction between \textit{not} and \textit{no} with comparatives, cf. \citet[\href{https://archive.org/details/jespersen-1954-a-modern-english-grammar-on-historical-principles-part-ii-syntax-first-volume/page/434/mode/2up?view=theater}{16.83ff}]{jespersenMEG2} and \citet[\href{https://archive.org/details/studiesinenglis00stofgoog/page/n106/mode/2up?view=theater}{87ff}]{stoffel1894studies}.% Originally "Cf on this distinction ... with comparatives Modern ..."; I've rearranged the sentence
%Brett: OK

In Latin both \textit{non magis quam} and \textit{non minus quam} are favourite expressions for equality, though of course used in different connexions (\ref{ex:08-12}).

\ea \label{ex:08-12}
\ea
\gll Cæsar non minus operibus pacis florebat quam rebus in bello gestis\\
 Caesar not less {with works} {of peace} flourished than deeds in war {carried out}\\
\glt `Caesar flourished no less through his peacetime achievements than through his deeds carried out in war'
\ex 
\gll Pericles non magis operibus pacis florebat quam rebus in bello gestis\\
 Pericles not more {with works} {of peace} flourished than deeds in war {carried out}\\
\glt `Pericles flourished no more through his peacetime achievements than through his wartime deeds'\hfill(\href{https://archive.org/details/grammaticamilit00cauegoog/page/50/mode/2up?view=theater&q=%22minus+operibus+pacis%22}{Cauer, \textit{Grammatica} 51})\footnote{These sentences about Caesar and Pericles are not direct quotations but rather Jespersen's imagined examples based on Cauer's discussion of comparative constructions where he suggests that such constructions could be applied to figures like Caesar and Pericles. \eds}
%  (i) Cauer doesn't directly say either of this pair. Rather -- I (PE) guestimate from my infinitesimal understanding of German -- he says that the former (minus its subject) may be said of Cäsar and the latter (ditto) of Perikles.
%Brett: Footnote added.
%  (ii) I (PE) don't understand "openings of peace". Or indeed "works of peace". ("Works during peace"? "Armistices"?) (iii) Not on p 52 but on p 51.
%Brett: Is this translation better?
% PE: Good!
\z
\z
\is{conjunctions!negative}
\il{English!no less than@\textit{no less than}|)}

\is{equality, negativing expressions of}
There is really no perfect negative corresponding to \textit{as rich as}, comprising both `richer' and `poorer', for \textit{not so rich as} (note the change of the first conjunction) excludes `richer' and means `less rich'.

We have already seen (p.~\pageref{para:little-a-little}) that \textit{a little} and \textit{little} differ, the former being a positive and the latter almost a negative term. We may arrange these terms (with \textit{a few} and \textit{few}) into a scale like this: 

\bigskip
%\begin{table}[h]
\centering
\begin{tabular}{rllll}\label{tab:quantitative_expressions}
1 & \textit{much} & \textit{much} (money) & \textit{many} (people) & \textit{very} (careless) \\ 
2 & \textit{a little} & \textit{a little} (money) & \textit{a few} (people) & \textit{a little} (careless) \\ 
3 & \textit{little} & \textit{little} (money) & \textit{few} (people) & \textit{little} (careless) \\ 
\end{tabular}
%\caption{Comparison of Quantitative Expressions}
\justifying

%\end{table} % What's above faithfully reproduces what's in the book. Is there a reason, which I (PE) don't grasp, why "people" is in parentheses but "money" and "careless" are not? And if there isn't, should we impose consistency one way or another?
%Brett: I've added parens throughout
\bigskip
\noindent only that \textit{little careless} is not quite idiomatic, as \textit{little} is not often used with depreciatory adjectives; cf. on the other hand, \textit{little intelligent}.

\label{ch8-not-a-little}Now if we try the negatives of these we discover that negativing 1 turns it into 3: \textit{not much} (\textit{money}) = `little (money)'; \textit{not many} (\textit{people}) = `few (people)'; \textit{not very intelligent} = `little intelligent'. But a negative 2 becomes nearly synonymous with 1 (or stands between 1 and 2): \textit{not a little} (\textit{money}) = `much (money)'; \textit{not a few} (\textit{people}) = `many (people)'; \textit{not a little intelligent} = `very intelligent'.

Examples of \textit{a few} and \textit{a little} negatived: (\ref{ex:08-14}).

\ea \label{ex:08-14}
\ea
I am solicited not by a few. And those of true condition \\(`by not a few')\hfill(\href{https://internetshakespeare.uvic.ca/doc/H8_F1/scene/1.2/index.html#tln-345}{Shakespeare, \textit{H8} 1.2.18})
\ex
Sister, it is not a little I haue to say, Of what most neerely appertaines to vs both\hfill(Shakespeare, \href{https://internetshakespeare.uvic.ca/doc/Lr_Q1/page/10/#tln-310}{\textit{Lr} quarto 1.1.286} (\href{https://internetshakespeare.uvic.ca/doc/Lr_F1/scene/1.1/index.html#tln-310}{folio}: \textit{not little})) % This is faithful to OJ; but I (PE) think it's a bit confusing and suggest rearranging it (of course providing the same information)
%Brett: fine
% PE: done
\ex
at which they were not a little sorry\hfill(\href{https://archive.org/details/bunyanspilgrims00moffgoog/page/146/mode/2up?q=%22not+a+little+sorry%22&view=theater}{Bunyan, \textit{Progress} 147}) 
\ex
that which did not a little amuse the merchandizers was {\dots}\hfill(\href{https://archive.org/details/bunyanspilgrims00moffgoog/page/118/mode/2up?q=%22not+a+little%22&view=theater}{ibid 124}) % OJ merely points to something on ibid 124; he doesn't write it out.
\ex
{}[it] gained me at once {\dots} the friendship of not a few whose friendship was worth having\hfill(\href{https://archive.org/details/myfirstbookexpe00jerogoog/page/n74/mode/2up?q=%22gained+me+at+once%22&view=theater}{Allen, \textit{Æsthetics} 46}) % dots where ", an unknown man," had been cut; subject is not "it" but "my poor little venture"
\ex
a phenomenon which puzzles me not a little\hfill(\href{https://archive.org/details/in.ernet.dli.2015.264111/page/n307/mode/2up?view=theater&q=%22phenomenon+which+puzzles%22}{Ruskin, \textit{Things}}) % OJ attributes this to "Ruskin Sel. 1. 410", but I (PE) don't find it in any edition I've seen of "Selections"
\z
\z

\is{intermediate terms, negativing}
While it seems to be usual in all languages to express \textit{contradictory} terms by means either of derivatives like those mentioned in chapter 5 (p.~\pageref{ch:5}ff) or of an adverb corresponding to \textit{not}, languages very often resort to separate roots to express the most necessary \is{contrary terms, negativing}\textit{contrary} terms. Hence such pairs as \textit{young--old}, \textit{good--bad}, \textit{big--small}, etc. Now, it is characteristic of such pairs that intermediate stages are found, which may be expressed negatively by \textit{neither young nor old}, etc.; the simple negation of one of the terms (for instance \textit{not young}) comprising both the intermediate and the other extreme. Sometimes a language creates a special expression for the intermediate stage, thus \textit{indifferent} in the comparatively recent sense of `neither good nor bad, what is between good and bad', \textit{medium-sized} between \textit{big} and \textit{small}. There may even be a whole long string of words with shades of meaning running into one another and partially overlapping, as in \textit{hot} (\textit{sweltering})\textit{---warm---tepid---lukewarm---mild---fresh---cool---chilly---cold---frosty---icy}. If one of these is negatived, the result is generally analogous to the negativing of a numeral: \textit{not lukewarm}, for instance, in most cases means less than lukewarm, i.e., cold or something between cold and lukewarm.

\is{A, B, and C tripartition|(}
\il{English!nothing@\textit{nothing}}
If we lengthen the series given above (\textit{much---a little---little}) in both directions, we get on the one hand \il{English!all@\textit{all}|(}\textit{all} (\textit{everything}), on the other hand \textit{nothing}. These are contrary terms, even in a higher degree than \textit{good} and \textit{bad} are, as both are absolute. Whatever comes in between them (thus all the three quantities mentioned above) is comprised in the term \textit{something}, and we may now arrange these terms in this way, denoting by \textit{A} and \textit{C} the two absolutes, and by \textit{B} the intermediate relative:\is{absolute terms, negativing}\footnote{Jespersen's use of ``(n.)'' here likely indicates the independent use of \textit{all} (e.g. \textit{\textsc{all} is lost}), distinguishing it from its use with a noun (e.g. \textit{\textsc{all record} is lost}). This notation is atypical for Jespersen, who usually prefers the term ``substantive" for such uses. \eds}


\bigskip
\il{English!nothing@\textit{nothing}}
{\centering
\begin{tabular}{c@{\hspace{0.5em}}c@{\hspace{0.5em}}c@{\hspace{0.5em}}c}
\emph{A} & & \emph{B} & \emph{C} \\
\textit{all} (n.) & \multirow{2}{*}{$\big\}$} & \multirow{2}{*}{\textit{something}} & \multirow{2}{*}{$\{$\,\textit{nothing}} \\
\textit{everything} & & & \\
\end{tabular}\par}

\bigskip
\noindent and correspondingly
\bigskip

\il{English!none@\textit{none}}
{\centering
\begin{tabular}{l@{\hspace{0.5em}}c@{\hspace{0.5em}}l@{\hspace{0.5em}}c@{\hspace{0.5em}}l}
\textit{all} (pl.) & \multirow{2}{*}{$\big\}$} & \textit{some} & \multirow{2}{*}{$\big\{$} & \textit{none} \\
\textit{everybody} & & \textit{somebody} & & \textit{nobody} \\
\textit{all girls} & $\}$ & \textit{some girls} (\textit{a girl}) & $\{$\ & \textit{no girl}(\textit{s}) \\
\textit{all the money} & $\}$ & \textit{some money} & $\{$\ & \textit{no money} \\
\end{tabular}
\label{tab:quantitative_terms}\par}

\bigskip
In exactly the same way we have the adverbs:
\bigskip

{\centering
\begin{tabular}{l@{\hspace{0.5em}}c@{\hspace{0.5em}}l@{\hspace{0.5em}}c@{\hspace{0.5em}}l}
\textit{always} & $\}$ & \textit{sometimes} & $\{$\ & \textit{never} \\
\textit{everywhere} & $\}$ & \textit{somewhere} & $\{$\ & \textit{nowhere} \\
\end{tabular}
\label{tab:qualitative_adverbs}\par}
\bigskip

\label{08-not-all}Let us now consider what the result is if we negative these terms. A negative \textit{A} means \textit{B}:

\begin{table}[H]
\begin{tabular}{l c l}
\emph{Negative A} & & \emph{means B}\\
\textit{not all}, \textit{not everything} & = & \textit{something}\\
\textit{not all}, \textit{not everybody} & = & \textit{some}\\
\textit{not all girls} & = & \textit{some girls}\\
\textit{not all the money} & = & \textit{some} (\textit{of the}) \textit{money}\\
\textit{not always} & = & \textit{sometimes}\\
\textit{not everywhere} & = & \textit{somewhere}\\
\end{tabular}
\end{table}

This amounts to saying that in negativing an \textit{A} it is the \textit{absolute} element of \textit{A} that is negatived. Thus always when the negative precedes the absolute word of the \textit{A}-class: (\ref{ex:08-20}).

\ea \label{ex:08-20}
\ea
We are not cotton-spinners all, But some love England and her honour yet\hfill(\href{https://books.google.st/books?id=TFCjuar4RRYC&pg=PA224&hl=pt-PT&source=gbs_toc_r&cad=2#v=onepage&q&f=false}{Tennyson, \textit{1852}})
\ex
They are not all of them fools
\ex
I do not look on every politician as a humbug
\ex
This change is not all gain\hfill(news 1917)
\ex
Not all Hugh's letters were concerned with these grim technicalities\\\hfill(\href{https://archive.org/details/mrbritlingseesi02unkngoog/page/n344/mode/2up?view=theater&q=%22technicalities%22}{Wells, \textit{Britling} 325}) % OJ omits "grim".
\ex
It seemed {\dots} that not all the pallor was due to the lamp\\\hfill(\href{https://archive.org/details/runningwater00masouoft/page/196/mode/2up?q=%22pallor+was+due+to+the+lamp%22&view=theater}{Mason, \textit{Water} 179}) % Dots to replace "to Chayne"
\ex
He is not always so sad
\ex
\gll Non omnis moriar\\
 Not all will-I-die\\
\glt `I shall not wholly die'\hfill(\href{https://archive.org/details/qhoratiflaccica00mlgoog/page/n175/mode/2up?q=%22non+omnis+moriar%22&view=theater}{Horace, \textit{Odes} 3.30}) % OJ doesn't specify the source; I (PE) thought readers might be interested
\z
\z


When a negatived \textit{all} in this sense is the subject, we may have the word-order \textit{not all} before the verb as in the sentences just quoted from Wells and Mason, or in the Danish and German proverbs (\ref{ex:08-28}).\il{English!all@\textit{all}|)} Or the subject may in some way be transposed so as to allow the negative to go with the verb, as in (\ref{ex:08-30}). Tobler %\href{https://archive.org/details/vermischtebeitr04toblgoog/page/n191/mode/2up?view=theater}
{quotes} the pair (\ref{ex:08-32}). %Brett: as far as I can tell, these are not quoted on the page cited, or even in the volume.
Cf. also (\ref{ex:08-34}).

\ea \label{ex:08-28}
\ea \il{Danish!ikke@\textit{ikke}}
\gll Ikke alt hvad der glimrer er guld\\
 not all what there glitters is gold\\\hfill(Danish)
\glt `Not all that glitters is gold'
\ex
\gll Nicht alles, was glänzt, ist Gold\\
 not all what glitters is gold\\\hfill(German)
\glt `Not all that glitters is gold'
\z
\z

\ea \label{ex:08-30}
\ea\il{Danish!ikke@\textit{ikke}}
\gll Det er ikke guld alt som glimrer\\
 it is not gold all which glitters\\\hfill(Danish, more usual form)
\glt `Not all that glitters is gold'
\ex
\gll Es ist nicht alles Gold, was glänzt\\
 it is not all gold what glitters\\\hfill(German)
\glt `Not all that glitters is gold'
\z
\z

\il{German!Middle High German!ez en-ist nicht@\textit{ez en-ist nicht}}
\ea \label{ex:08-32}
\ea
\gll ez en-ist nicht allez gold daz da glizzit\\
 it not-is not all gold that there glitters\\\hfill(Middle High German)
\glt `not all that glitters is gold' % Peter: Although Tobler capitalizes nouns in Modern German, he doesn't do so for MHG; see https://archive.org/details/vermischtebeitr04toblgoog/page/n195/mode/2up?view=theater&q=allez+g%C3%B6ld+daz+da+glizz%C3%BC
\ex
\gll n' est pas tout or quanqu' il reluit\\
 not is not all gold that-which it shines\\
\glt `not all that glitters is gold'\hfill(\href{https://archive.org/details/oeuvrescompltes01rute/page/246/mode/2up?q=reluit&view=theater}{Rutebeuf, \textit{Pharisian}})
\z
\z

\ea \label{ex:08-34}
\ea
\gll Es sind nicht alle frei, die ihrer Ketten spotten.\\
 they are not all free who their chains mock\\
\glt `Not all are free who mock their chains.'\hfill(\href{https://archive.org/details/nathanderweise02lessgoog/page/n216/mode/2up?q=%22Es+sind+nicht+alle+frei%22&view=theater}{Lessing, \textit{Nathan} 4.4}) % OJ attributes this to Schiller. A straightforward blooper, I (PE) fear. ??? Can we switch the order: "Not all who mock their chains are free"?
\ex 
\gll Es sind nicht alle Jäger, die das Horn gut blasen.\\
 they are not all hunters who the horn well blow\\
\glt `Not all are hunters who blow the horn well.'\hfill(proverb) % ??? PE: Can we switch the order: "Not all who blow the horn well are hunters"?
\z
\z

But very often \textit{all} is placed first for the sake of emphasis, and the negative is attracted to the verb in accordance with the general tendency mentioned above (p.~\pageref{para:nexal-negation-tendency}). This is often looked upon as illogical, but \citet[\href{https://archive.org/details/vermischtebeitr04toblgoog/page/n191/mode/2up?view=theater}{159ff}]{tobler1886vermischte}, in an instructive article on the French proverb \textit{Tout ce qui reluit n'est pas or}, rightly calls attention to the difference between sentences like \il{German!nicht@\textit{nicht}}\textit{nicht Mitglieder können ein\-ge\-führt werden} (`non-members may be introduced'), where only one member of a positive sentence is negative (what I call special negative) and the French proverb, where the negation is connected with the verb, ``dem Kern der Aussage'' (`the core of the statement'), and the expression consequently is 

\begin{quote}
ein im höchsten Grade angemessener, indem er besagt: von dem Subjekte ``alles Glänzende'' darf ``Gold sein'' nicht prädiziert werden

`a highly appropriate one, in that it says: ``to be gold'' cannot be predicated from the subject ``all that glitters'''
\end{quote}

English examples of this arrangement are very frequent (\ref{ex:08-36}).

\ea \label{ex:08-36}
\ea
but \emph{every man may nat} have the perfeccioun that ye seken\\\hfill(\href{https://archive.org/details/completeworksofg04chauuoft/completeworksofg04chauuoft/page/226/mode/2up?q=%22every+man+may+nat%22&view=theater}{Chaucer, \textit{Melibeus} B~2708})
\ex\il{English!all@\textit{all}|(}
\emph{All that glisters is not} gold\hfill(\href{https://internetshakespeare.uvic.ca/doc/MV_F1/scene/2.7/index.html#tln-1035}{Shakespeare, \textit{Merch} 2.7.65})
\ex
\emph{All's not} offence that indiscretion findes, And dotage termes so\\\hfill(\href{https://internetshakespeare.uvic.ca/doc/Lr_F1/scene/2.2/index.html#tln-1485}{Shakespeare, \textit{Lr} 2.2.199}) % uvic.ca assigns this to act 2 scene 2
\ex
All things are lawfull vnto mee, but \emph{all things are not} expedient\\\hfill(\href{https://www.kingjamesbibleonline.org/1611_1-Corinthians-6-12/}{AV \textit{1 Corinthians} 6.12})
\ex
\emph{every one cannot} make musick\hfill(\href{https://archive.org/details/bim_early-english-books-1641-1700_the-compleat-angler-_walton-isaac_1655/page/140/mode/2up?q=%22every+one%22&view=theater}{Walton, \textit{Angler} 106})
\ex
Thank Heaven, \emph{all scholars are not} like this\\\hfill(\href{https://archive.org/details/bim_eighteenth-century_sir-charles-grandison-_richardson-samuel_1780_1/page/74/mode/2up?q=%22all+scholars%22&view=theater}{Richardson, \textit{Grandison} 72}) % Restoring capital S of "Scholars"
\ex
\emph{every one is not} able to stem the temptations of public life\\\hfill(\href{https://archive.org/details/historyrasselas01johngoog/page/n157/mode/2up?q=%22not+able+to+stem%22&view=theater}{Johnson, \textit{Rasselas} 152}) % "temptation" corrected to "temptations"
\ex \il{English!may not@\textit{may not}}
As \emph{every person may not} be acquainted with this primeval pastime\\\hfill(\href{https://archive.org/details/TheVicarOfWakefield/page/n129/mode/2up?q=%22every+person+may+not%22&view=theater}{Goldsmith, \textit{Vicar}}) % Restored "primeval"
\ex
All is not lost\hfill(\href{https://archive.org/details/poeticalworksofj00miltiala/page/184/mode/2up?ref=ol&view=theater&q=%22all+is+not+lost%22}{Milton, \textit{Lost} 1.106} and \href{https://archive.org/details/completepoeticalshel/page/112/mode/2up?view=theater&q=%22all+is+not+lost%22}{Shelley, \textit{Revolt} 7.36})
\ex
But \emph{all men are not} born to reign\hfill(\href{https://archive.org/details/in.ernet.dli.2015.285363/page/n329/mode/2up?q=%22men+are+not+born+to+reign%22&view=theater}{Byron, \textit{Mazeppa} 7})
\ex
\emph{All Valentines are not} foolish\hfill(\href{https://archive.org/details/essayseliacharle00lamb/page/66/mode/2up?q=%22all+valentines+are+not%22&view=theater}{Lamb, \textit{Elia} 1.103})
\ex
\emph{All women are not} mothers of a boy, Though they live twice the length of my whole life\hfill(\href{https://archive.org/details/ringandbook17browgoog/page/300/mode/2up?q=%22All+women+are+not%22&view=theater}{R. Browning, \textit{Pompilia}})
\ex
Well, any fool can get up a Blue Book. Only---luckily for me---\emph{all the fools don't.}\hfill(\href{https://archive.org/details/marriageofwillia0000mrsh_i0u5/page/10/mode/2up?q=%22up+a+blue+book%22&view=theater}{Ward, \textit{Marriage} 16}) % Restored to Ward's original.
\ex
``Every one is lonely,'' he said, ``\emph{but every one does not} know it.''\\\hfill(\href{https://archive.org/details/shipsthatpassin00harr/page/110/mode/2up?view=theater&q=%22Every+one+is+lonely%22}{Harraden, \textit{Ships} 62}) % Restored "he said"
\ex
For each man kills the thing he loves, Yet \emph{each man does not} die.\\\hfill(\href{https://archive.org/details/balladofreadingg00wild/page/8/mode/2up?view=theater&q=%22each+man+kills%22}{Wilde, \textit{Gaol} 3})
\ex
\emph{All \textsc{our} men aren't} angels\hfill(\href{https://archive.org/details/mrbritlingseesi02unkngoog/page/n298/mode/2up?view=theater&q=%22all+our+men%22}{Wells, \textit{Britling} 281}) % OJ italicizes "our"
\z
\z\il{English!all@\textit{all}|)}

French examples from old and modern times have been collected by Tobler; I add from my own reading (\ref{ex:08-52}).

\il{French!tout@\textit{tout}|(}
\ea \label{ex:08-52}
\ea 
\gll \emph{Tout} \emph{le} \emph{monde} \emph{n'} \emph{a} \emph{pas} l' esprit de comprendre les chefs-d'œuvre.\\
 all the world not has not the spirit of understanding the masterpieces\\
\glt `Not everyone has the spirit to understand masterpieces.'\\\hfill(\href{https://archive.org/details/lesdeuxhritages02gogogoog/page/n97/mode/2up?q=%22tout+le+monde+n%27a%22&view=theater}{Mérimée, \textit{Héritages} 4.2}) % "chefs-d'œuvre" so hyphenated
\ex
\gll \emph{Tout} \emph{le} \emph{monde} \emph{n'} \emph{est} \emph{pas} fait pour l'art\\
 all the world not is not made for {the art}\\
\glt `Not everyone is made for art'\hfill(\href{https://www.gutenberg.org/cache/epub/61876/pg61876-images.html}{Rolland, \textit{Foire} 162})
\ex
\gll \emph{Tout} \emph{le} \emph{monde} \emph{ne} \emph{peut} \emph{pas} tirer le gros lot\\
 all the world not can not draw the big prize\\
\glt `Not everyone can win the jackpot'\hfill(\href{https://www.gutenberg.org/cache/epub/61876/pg61876-images.html}{ibid 295})
\z
\z
\il{French!tout@\textit{tout}|)}

\il{Danish!ikke@\textit{ikke}}
In Danish the same order is not at all rare: \textit{\textsc{Alt er ikke} tabt} (`all is not lost'), %%SG: shouln't the whole Danish sentence be italicized? \\ PE: Yes indeed. Done. (And thank you!)
etc. Note the positive continuation, which shows that `some' (or `many') is meant, in (\ref{ex:08-55}).

\ea \label{ex:08-55} \il{Danish!ikke@\textit{ikke}}
\gll Men \il{English!Old English!alle@\textit{alle}}\emph{alle} \emph{ere} \emph{ikke} saa vise, som Socrates, og indlade sig ofte ganske alvorligt med een, der gjør et dumt spørgsmaal.\\
 but all are not as wise as Socrates and engage themselves often quite seriously with one who makes a silly question\\
\glt `But not all are as wise as Socrates, and they often engage quite seriously with someone who asks a silly question.'\hfill(\href{https://tekster.kb.dk/text/sks-slv-txt-root#ss146}{Kierkegaard, \textit{Stadier} 138}) % One comma restored
\z

In German Tobler mentions the possibility of the same, e.g. (\ref{ex:08-56}).

\il{German!nicht@\textit{nicht}}
\ea \label{ex:08-56}
\gll \emph{Alle} Druckfehler können hier \emph{nicht} aufgezählt werden\\
 all printing-errors can here not listed be\\
\glt `Not all printing errors can be listed here'\hfill(\href{https://archive.org/details/vermischtebeitr04toblgoog/page/n195/mode/2up?view=theater&q=%22alle+druckfehler%22}{\textit{Beiträge} 162})
\z

With regard to Greek, \citet[\href{https://archive.org/details/griechischesprac00kr/page/296/mode/2up?view=theater}{§67}]{kruger1875griechische} insists on the distinction (\ref{ex:08-57}), but he admits exceptions for the sake of emphasis, especially with contrasts with \textit{mén} and \textit{dé}; \href{https://archive.org/details/griechischesprac00kr/page/304/mode/2up?view=theater&q=%22Havres+pe%26v+obx+7ABov%2C+Ap%C4%B1atos+de+%E2%80%9Car+Apraokos.+%22}{he quotes from Xenophon}: (\ref{ex:08-58}).

\ea \label{ex:08-57}
\ea
\gll ou pánta orthôs epoíēsen\\
 not all correctly did\\
\glt `he did not do everything correctly' \phantom{x}(but probably some things)
\ex
\gll pánta ouk orthôs epoíēsen\\
 all not correctly did\\
\glt `he did not do everything correctly' \phantom{x}(but rather wrongly)
\ex
\gll orthôs pánta ouk epoíēsen\\
 correctly all not did\\
\glt `he did not do everything correctly' \phantom{x}(but rather omitted some)
\z
\z

\ea \label{ex:08-58}
\gll Pántes mèn ouk êlthon, Ariaíos dè kaì Artáoxos.\\
 all indeed not came Ariaíos but and Artáoxos\\
\glt `Not all came, but Ariaíos and Artáoxos did.'
\z

\il{English!none@\textit{none}}
On the other hand, when a word of the A-class (\textit{all}, etc.) is placed in a sentence containing a special negative (or an implied negative), the result is the same as if we had the corresponding word from the C-class (\textit{none}, \textit{nobody}, etc.) and a positive word; thus the assertion is absolute: % OJ says not "word from the C-class" but "C-word"



\is{nothing@`nothing', expressions meaning}
\begin{table}[H]
    \centering
\begin{tabular}{l c l}
\emph{Negative A} & & \emph{means C}\\
\textit{all this is unnecessary} & = & `nothing is necessary'\\
\textit{everybody was unkind} & = & `nobody was kind'\\
\textit{he was always unkind} & = & `he was never kind'\\
\textit{everybody fails} & = & `nobody succeeds'\\
\textit{he forgets everything} & = & `he remembers nothing'\\
\end{tabular}
\end{table}


The same effect is rare when we have a nexal negative with one of the A-words; cf. (\ref{ex:08-59}). Tobler also has a few examples from French, thus (\ref{ex:08-60}). I know no English examples of this. 

\ea \label{ex:08-59}
\gll Tous ces gens -là ne sont pas humains.\\
 all these people there not are not human\\
\glt `None of these people is human.'\hfill(\href{https://www.gutenberg.org/cache/epub/61970/pg61970-images.html}{Rolland, \textit{Amies} 141})
\z

\il{French!tout@\textit{tout}|(}
\ea \label{ex:08-60}
\ea
\gll maxime usée et triviale que tout le monde sait, et que \emph{tout} \emph{le} \emph{monde} \emph{ne} \emph{pratique} \emph{pas}\\
 maxim worn-out and trivial that all the world knows and that all the world not practises not\\
\glt `a worn-out and trivial maxim that everyone knows, and that no one practises'\hfill(\href{https://archive.org/details/oeuvrescompltes01bruygoog/page/n137/mode/2up?q=%22maxime%22&view=theater}{La Bruyère, \textit{Caractères} 11.149}) % PE: Non-British verb spelling "practices" changed to "practises"
\ex
\gll \emph{Toute} \emph{jalousie} \emph{n'} \emph{est} \emph{point} exempte de quelque sorte d' envie {\dots} l'envie au contraire est quelquefois séparée de la jalousie.\\
 all jealousy not is {at all} exempt from some kind of envy {} envy {on the} contrary is sometimes separated from the jealousy\\
\glt `Jealousy is never free from some kind of envy; envy, though, is sometimes separate from jealousy.'\hfill(\href{https://archive.org/details/oeuvrescompltes01bruygoog/page/n107/mode/2up?q=%22Toute+jalousie%22&view=theater}{ibid 11.85}; \href{https://archive.org/details/vermischtebeitr04toblgoog/page/n195/mode/2up?view=theater&q=%22toute+jalousie%22}{Tobler, \textit{Beiträge} 162f}) % OJ's punctuation follows Tobler's; Tobler's does not follow La Bruyère's. I'm following La Bruyère's.
\z
\z

The difference between the two possible results of the negation of a word like \il{English!all@\textit{all}}\textit{all} is idiomatically expressed by the contrast between two adverbs, as seen in (\ref{ex:08-62}), result B, and (\ref{ex:08-63}), result C.

\ea \label{ex:08-62}
\ea
he is \emph{not altogether} happy 
\ex
I am \emph{not altogether} an asse\hfill(\href{https://internetshakespeare.uvic.ca/doc/Wiv_F1/index.html#tln-155}{Shakespeare, \textit{Wiv} 1.1.175}) % PE: OJ doesn't use italics in this one example; suspecting that this was a mere error, I've italicized it.
\ex
\gll \emph{pas} \emph{tout-à-fait}\\ 
 not completely\\\hfill(French)
%\glt `not completely' % PE: When, as in this and the following three examples, the idiomatic translation is identical to the word-by-word gloss, can't we omit the third line? (It strikes me as worse than a mere waste of space.)
%Brett: Makes sense. I've done it here but not looked for opportunities elsewhere.
% PE: Good. When we see opportunities elsewhere, let's give them the same treatment.
\ex\il{Danish!ikke@\textit{ikke}}
\gll \emph{ikke} \emph{helt}\\
 not entirely\\\hfill(Danish)
%\glt `not entirely'

\il{German!nicht@\textit{nicht}}
\ex
\gll \emph{nicht} \emph{ganz}\\
 not fully\\\hfill(German)
%\glt `not fully'
\z
\z


\il{French!pas du tout@\textit{pas du tout}}
\is{not at all@`not at all', expressions corresponding to}
\ea \label{ex:08-63} \il{English!not at all@\textit{not at all}}
\ea he is \emph{not at all} happy (he is \emph{not} happy \emph{at all})
\ex
\gll \emph{pas} \emph{du} \emph{tout}\\
 not of all\\\hfill(French)
\glt `not at all'

\ex\il{Danish!ikke@\textit{ikke}}
\gll \emph{slet} \emph{ikke}\\
 {at all} not\\\hfill(Danish)
\glt `not at all'

\ex
\gll \emph{gar} \emph{nicht}\\
 {at all} not\\\hfill(German)
\glt `not at all'
\z
\z
\il{French!tout@\textit{tout}|)}

Here should be mentioned words for `never' like German \textit{nimmer} and \textit{nie}, Old English \textit{nā}; but then the constituent \textit{ie}, \textit{ā} does not exclusively belong to class \textit{A}, but also to some extent to class \textit{B}.\footnote{Jespersen uses ``constituent'' % PE: I've put "constituent" in quotation marks
here to refer to the elements \textit{immer}, \textit{ie} and \textit{ā} within these words. In modern linguistic terminology, these might be described as morphemes or formatives. Jespersen's point is that, while words like \textit{nimmer}, \textit{nie}, and \textit{nā} mean `never', the elements that compose them are not exclusively associated with absolute negation (class A), but also have some relation to less absolute meanings (class B). \eds} % Added from Jespersen's Addenda.% I (PE) realize that I don't understand what this means. If we substituted a semicolon for the comma immediately preceding "but then the constituent", would the result express what OJ wants to say? (Is the problem that OJ is using the term "constituent" in some way that I don't understand?) ... By "constituent" does he perhaps mean what I'd call a "formative"?
%Brett: I see your point. I've made the semicolon switch "\textit{nā}; but" and added a footnote.

\is{intonation}
The effect of stress and tone in these cases is sometimes analogous to what we have seen with numerals; cf. Danish (\ref{ex:08-64}), which with strong stress and high tone on \textit{hele} may mean `he was only sick during part of the voyage', but otherwise means `not at all'.\is{not for nothing@`not for nothing', expressions corresponding to|(}

\ea \label{ex:08-64}
\gll han var ikke syg på hele rejsen\\
 he was not sick on whole trip.\DEF{}\\
\glt `he was not sick during the entire trip'
\z

\is{indirect negation}
A negative may, of course, be annulled by an indirect negative, as in (\ref{ex:08-65}).\largerpage[1.75]

\il{French!ne ... pas@\textit{ne ... pas}}\il{French!tout@\textit{tout}|(}
\ea \label{ex:08-65}
\gll Comment, vous me connaissez? dit-il. --- \emph{Comme} \emph{si} \emph{tout} \emph{le} \emph{monde} \emph{ne} \emph{se} \emph{connaissait} \emph{pas} à Paris!\\
 how you me know {said he} {} as if all the world not itself knew not in Paris\\
\glt `What, you know me? he said. --- As if everyone didn't know each other in Paris!'\hfill(\href{https://www.gutenberg.org/cache/epub/61970/pg61970-images.html}{Rolland, \textit{Amies} 142}) % Restored "!"
\z


\is{nothing@`nothing', expressions meaning|(}
It may perhaps be doubtful whether we have B or C as a result in the common phrase (\ref{ex:08-66}) (English \textit{I shouldn't like to do it for anything in the world} more often than \textit{{\dots} for all the world}). It is, however, more natural to take it to be an equivalent of `nothing', and in the corresponding French idiom \textit{rien} is used, see e.g. (\ref{ex:08-68}).

\ea \label{ex:08-66}
\ea\il{Danish!ikke@\textit{ikke}}
\gll Det gjorde jeg \emph{ikke} \emph{for} \emph{alt} \emph{i} \emph{verden}.\\
 that did I not for all in world\\\hfill(Danish)
\glt `I would not do that for anything in the world.'
\il{German!nicht@\textit{nicht}}
\ex
\gll Das täte ich \emph{um} \emph{alles} \emph{in} \emph{der} \emph{Welt} \emph{nicht}.\\
 that {would do} I for all in the world not\\\hfill(German)
\glt `I would not do that for anything in the world.'
\z
\ex \label{ex:08-68}
\gll {}[des mondains,] {\dots} \textit{qui} \emph{pour} \emph{rien} \emph{au} \emph{monde} \emph{n'} \emph{eussent} renoncé à l' honneur\\
 {of the} {worldly ones} {} who for nothing {in the} world not {would have} renounced to the honour\\
\glt `worldly ones, {\dots} who would not have renounced the honour for anything in the world'\hfill(\href{https://www.gutenberg.org/cache/epub/61876/pg61876-images.html}{Rolland, \textit{Foire} 83}) % Contra OJ, "qui" is part of the direct quotation
\z

\il{English!all@\textit{all}|(}There is a third possibility, when \textit{not} is for the sake of emphasis put before \textit{all} in the sense of `not even', though it should properly go with the verb as a nexal negative; \textit{all} here means `the sum of'. Cf. the distinction made in \citet[\href{https://archive.org/details/jespersen-1954-a-modern-english-grammar-on-historical-principles-part-ii-syntax-first-volume/page/130/mode/2up?view=theater}{5.4}]{jespersenMEG2} between \textit{all the boys of this form are stronger than their teacher} (if working together) and \textit{all the boys of this form are able to run faster than their teacher} (i.e., each separately). Thus (\ref{ex:08-69}). Cf. with nexal negative (\ref{ex:08-71}).

\ea \label{ex:08-69}
\ea
\emph{Not all} the water in the rough rude sea Can wash the balme from an anoynted king\hfill(\href{https://internetshakespeare.uvic.ca/doc/R2_F1/scene/3.2/index.html#tln-1405}{Shakespeare, \textit{R2} 3.2.54}) % Original has more capitals
\ex
\emph{Not all} the trying of Zora and all the Ladies Bountiful of Christendom could give her her heart's desire\hfill(\href{https://archive.org/details/septimus00unkngoog/page/n315/mode/2up?q=%22trying+of+zora%22&view=theater}{Locke, \textit{Septimus} 341})
\z
\z

\ea \label{ex:08-71}
\ea
On me, whose \emph{all not equals} Edwards moytie\\\hfill(\href{https://internetshakespeare.uvic.ca/doc/R3_F1/scene/1.2/index.html#tln-445}{Shakespeare, \textit{R3} 1.2.250}) % Folio capitalizes "All" and "Moytie"
\il{English!all@\textit{all}|)}
\ex
\gll \emph{toutes} les idées \emph{ne} \emph{comptent} \emph{guère}, quand on aime\\
 all the ideas not count hardly when one loves\\
\glt `all the ideas count for nothing, when one is in love'\\\hfill(\href{https://www.gutenberg.org/cache/epub/61970/pg61970-images.html}{Rolland, \textit{Maison} 193}) 
\z
\z
\is{nothing@`nothing', expressions meaning|)}
\il{French!tout@\textit{tout}|)}

\is{C-class, negativing}
\il{Latin!non nihil@\textit{non-nihil}}\il{Latin!non nemo@\textit{non-nemo}}\il{Latin!non nulli@\textit{non-nulli}}\il{Latin!non nunquam@\textit{non-nunquam}}
\label{08-non-nulli}If now we examine what results when a word belonging to the C-class is negatived, we shall see corresponding effects, only that immediate combinations are not frequent except in Latin, where \textit{non-nemo}, \textit{non-nulli} means `some', \textit{non-nihil} `something', \textit{non-nunquam} `sometimes'. Here thus the result clearly belongs to class B.

\is{not for nothing@`not for nothing', expressions corresponding to|(}
\il{English!not for nothing@\textit{not for nothing}}
\il{English!nothing@\textit{nothing}}
The same is true in the frequent idiom \textit{not for nothing} (`not in vain' or `to good purpose'), as in (\ref{ex:08-73}).

\ea \label{ex:08-73}
\ea
it was \emph{not for nothing} that my nose fell a bleeding on blacke monday last\hfill(\href{https://internetshakespeare.uvic.ca/doc/MV_F1/scene/2.5/index.html#tln-860}{Shakespeare, \textit{Merch} 2.5}) % OJ says 2.6.25; but it's in 2.5
\ex
\emph{Not for nothing} have I led the Pack\hfill(\href{https://archive.org/details/the-second-jungle-book-rudyard-kipling/page/n43/mode/2up?q=%22not+for+nothing%22&view=theater}{Kipling, \textit{Second} 66}) % "Pack" capitalized
\ex
She would \emph{not} have done so \emph{for nothing}\hfill(\href{https://archive.org/details/achangeair00hopegoog/page/n192/mode/2up?q=%22she+would+not+have+done+so%22&view=theater}{Hope, \textit{Change} 190})
\ex
He was \emph{not} the eldest son of his father \emph{for nothing}\\\hfill(\href{https://archive.org/details/shakespeare00rale/page/42/mode/2up?q=%22not+the+eldest%22&view=theater}{Raleigh, \textit{Shakespeare} 42})
\z
\z

In the same way in other languages: (\ref{ex:08-77}).

\ea \label{ex:08-77}
\ea\il{Danish!ikke@\textit{ikke}}
\gll Han er \emph{ikke} \emph{for} \emph{intet} (\emph{ikke} \emph{for} \emph{ingenting}) sin faers søn.\\
 he is not for nothing (not for nothing) his father.\POSS{} son\\
\glt `He was not the son of his father for nothing.'\hfill(Danish)
\il{French!ne ... pas@\textit{ne ... pas}}
\ex
\gll Ce \emph{n'} était \emph{pas} \emph{pour} \emph{rien} qu' elle avait ces yeux hardis\\
 it not was not for nothing that she had these eyes bold\\
\glt `It was not for nothing that she had those bold eyes'\\\hfill(\href{https://www.gutenberg.org/cache/epub/61876/pg61876-images.html}{Rolland, \textit{Révolte} 314})
\z
\z
\is{not for nothing@`not for nothing', expressions corresponding to|)}

It is more usual to place the two negatives in two sentences, % Peter: I'm surprised that he doesn't say "in two clauses"...
as in \textit{one cannot say that nothing is finer} (`something is finer') or in an infinitival combination, as in \refp{ex:08-79}, (`it is good to have some gods'). Here too the result belongs to class B. 

\ea \label{ex:08-79}
It's not good for a man to have no gods\hfill(\href{https://archive.org/details/septimus00unkngoog/page/n267/mode/2up?q=%22not+good+for+a+man%22&view=theater}{Locke, \textit{Septimus} 285})
\z

\label{08-negativing-nobody}Inversely if we begin with a word from class C and place the negative adverb after it. Thus again in Latin \textit{nemo non videt} (`everybody sees'); \textit{nihil non videt} (`he sees everything'); (\ref{ex:08-80}). The result belongs to class A.

\ea \label{ex:08-80}
\gll cum ipsum dicere nunquam sit non ineptum\\ % "Quum" changed to "cum"; "id" removed; "sit non" reversed
 since itself talking never is not foolish\\
\glt `since talking itself is always foolish'\hfill(\href{https://archive.org/details/deoratorelibri00cice/page/136/mode/2up?q=%22ipsum+dicere%22&view=theater}{Cicero, \textit{De oratore} 1.24.112}) 
\z


\is{nothing@`nothing', expressions meaning|(}
The same result is obtained when one of those words is followed by a word with a negative prefix or with implied negative meaning:

\bigskip

\begin{tabular}{@{}l c l@{}}
\emph{Negative Phrase} & & \emph{Positive Equivalent}\\
\textit{nothing is unnecessary} & = & `everything is necessary'\\
\textit{nobody was unkind} & = & `everybody was kind'\\
\textit{he was never unkind} & = & `he was always kind'\\
\textit{nobody fails} & = & `everybody succeeds'\\
\textit{he forgets nothing} & = & `he remembers everything'\\
\end{tabular}

\bigskip

When the negative is a separate word, the result is the same; but in English as in Danish such sentences are generally avoided because they are not always clear or readily understood: it is rare to find combinations like (\ref{ex:08-81}). There is, however no difficulty if the two negatives are placed in separate sentences, as in \textit{There was no one present that did not weep} (`everybody wept'); here \textit{that not} is often replaced by \textit{but}, \textit{but that}, \textit{but what}, see chapter 12 (p.~\pageref{ch:12}).\footnote{Each of Jespersen's examples \textit{one cannot say that nothing is finer} and \textit{There was no one present that did not weep} contains a main clause and a subordinate clause, not two separate sentences as his text suggests. This usage is inconsistent with both modern linguistic terminology and Jespersen's normal use of \textit{clause} and \textit{sentence}, e.g. in \citet[]{jespersenMEG2}. \eds} 
In Danish, (\ref{ex:08-83}), or, with a curious negative force of \il{Danish!jo@\textit{jo}}\textit{jo}: \textit{{\dots} som jo græd}. Similar constructions are frequent in other languages as well; cf. Dr. Johnson's epitaph on Goldsmith (\ref{ex:08-84}).

\ea \label{ex:08-81}
\ea
not a clerk in that house did not tremble before her \\
(`all the clerks trembled')\hfill(\href{https://archive.org/details/newcomes00unkngoog/page/n66/mode/2up?q=%22tremble+before%22&view=theater}{Thackeray, \textit{Newcomes} 55})
\ex
no other man but you would not have despised the woman \\
(`every other man would have despised')\hfill(\href{https://archive.org/details/septimus00unkngoog/page/n217/mode/2up?q=%22no+other+man+but+you%22&view=theater}{Locke, \textit{Septimus} 228})
\z
\z

\ea \label{ex:08-83}\il{Danish!ikke@\textit{ikke}}
\gll der var ingen tilstede, som ikke græd\\
 there was {no one} present who not cried\\
\glt `there was no one present who did not cry'
\z

\ea \label{ex:08-84}
\gll Nihil tetigit quod non ornavit\\
 nothing touched which not he-adorned\\
\glt `He touched nothing {[}= no kind of writing{]} that he did not adorn'\footnote{Johnson instead wrote \textit{nullum quod tetigit non ornavit} \citep[\href{https://archive.org/details/lifesamueljohns00baldgoog/page/n96/mode/2up?view=theater}{91--92}]{boswell1791life}, which appears unchanged on the memorial in Westminster Abbey \citep{westminsterabbey_goldsmith}. \eds}
\z

`Everything' is also the result in such combinations as (\ref{ex:08-85}).

\ea \label{ex:08-85}
\gll L'art est toujours pur; \emph{il} \emph{n'} \emph{y} \emph{a} \emph{rien} \emph{que} de chaste en lui.\\
 art is always pure there not there has nothing that of chaste in it\\
\glt `Art is always pure; there is nothing in it but chasteness.'\\\hfill(\href{https://fr.wikisource.org/wiki/Page%3ARolland_-_Jean-Christophe%2C_tome_5.djvu/145}{Rolland, \textit{Foire} 133})
\z
\is{nothing@`nothing', expressions meaning|)}

\is{impossibility|(}
\bigskip\emergencystretch=3em
The ordinary treatment of both A- and C-words when negatived may be brought under one general rule: when the absolute notion (A or C) is mentioned first, the absolute element prevails, and the result is the contrary notion (A {\dots} not = C; C {\dots} not = A). If, on the other hand, \textit{not} comes first, it negatives the absolute element, and the result is the intermediate relative (not A = B; not C = B). 

\label{08-better-tripartition}It seems to me that the tripartition here established, ---


\begin{enumerate}[label=\Alph*., noitemsep]
\item \textit{all}\il{English!all@\textit{all}}
\item \textit{some}
\item \textit{none},\il{English!none@\textit{none}}
\end{enumerate}

\noindent is logically preferable to the tripartition in Kant's famous table of categories, ---

\begin{enumerate}[label=\Alph*., noitemsep]
\item Allheit\is{Allheit}
\item Vielheit\is{Vielheit}
\item Einheit\is{Einheit}
\end{enumerate}

\noindent as \textit{many} (\textit{Vielheit}) and \textit{one} (\textit{Einheit}) are both of them comprised under `some'; Kant does not take `none' here, but unintelligibly places negation under the heading ``quality'', though it is clearly a quantitative category. (See on the confusion caused by these Kantian categories in some philologists' treatment of negation, p.~\pageref{sec:kant}ff.)

\bigskip\is{auxiliary verbs|(}
The following remarks may also be of some interest to the student of logic. We may establish another tripartition between

\begin{enumerate}[label=\Alph*., noitemsep]
\item necessity
\item possibility
\item impossibility,
\end{enumerate}

\noindent and if closely inspected, these three categories are found to be nothing else but special instances of our three categories above, for necessity really means that \emph{all} possibilities are comprised. Note now: % Peter: Yes, the printed book has "Note now". This seems odd to me. I wonder if it could be a printer's error for "Note how:"
%Brett: You may be right, but I don't think it warrants a note or a change.
\textit{not necessary} means `possible'; \textit{not impossible} `possible'; \textit{it is impossible not to see} `necessary'. The verbal expression for these three categories is: 

\begin{enumerate}[label=\Alph*., noitemsep]
\item \textit{must} (or, \textit{need})
\item \textit{can} (or, \textit{may})
\item \textit{cannot},
\end{enumerate}

\noindent and we see their interrelation in instances like these:

\bigskip

\textit{He \textsc{must} run} = \textit{he \textsc{cannot} but run} (\textit{\textsc{cannot} help running})

\textit{no one \textsc{can} deny} = \textit{every one \textsc{must} admit}

\textit{nobody \textsc{need} be present} = \textit{everybody \textsc{may} be absent}

\textit{he \textsc{cannot} succeed} = \textit{he \textsc{must} fail}

\textit{he \textsc{cannot} forget} = \textit{he \textsc{must} remember}

\bigskip

\il{Latin!non potest non@\textit{non potest non}}
In the same way we have the Latin expression for necessity \textit{non potest non amare}, and the corresponding French as in (\ref{ex:08-86}); even with \textit{ne plus}, (\ref{ex:08-88}). With indirect negation we have the same, (\ref{ex:08-89}), different from \textit{Pas moyen de faire la comparaison} (`impossible').

\il{French!ne ... pas@\textit{ne ... pas}}
\ea \label{ex:08-86}
\ea
\gll car il \emph{ne} \emph{pouvait} \emph{pas} \emph{ne pas} voir qu' ils se moquaient de lui\\
 because he not could not not see that they themselves mocked of him\\
\glt `because he could not but see that they were mocking him'\\\hfill(\href{https://www.gutenberg.org/cache/epub/61876/pg61876-images.html}{Rolland, \textit{Foire} 54})
\ex
\gll une variation {\dots} qui \emph{ne} \emph{peut} \emph{pas} \emph{n'} \emph{être} \emph{pas} ancienne\\
 a variation {} which not can not not be not ancient\\
\glt `a variation that cannot but be ancient'\hfill(\href{https://archive.org/details/caractresgn00meiluoft/page/50/mode/2up?q=%22n%27%C3%AAtre+pas+ancienne%22&view=theater}{Meillet, \textit{Caractères} 50})
\z
\z

\ea \label{ex:08-88}
\gll il l' entendait partout, il \emph{ne} \emph{pouvait} \emph{plus} \emph{ne} \emph{plus} l' entendre\\
 he it heard everywhere, he not could {no more} not {no more} it hear\\
\glt `he heard it everywhere, he could not but hear it'\hfill(\href{https://fr.wikisource.org/wiki/Page%3ARolland_-_Jean-Christophe%2C_tome_9.djvu/28}{Rolland, \textit{Buisson} 12}) % PE: "... he could not but hear it"? (This would save a line.)
%Brett: Done
\z

\ea \label{ex:08-89}
\gll Et le moyen de ne pas faire la comparaison!\\
 and the way of not not make the comparison\\
\glt `And how could one not make the comparison!' \phantom{x} (`you must')\hfill(\href{https://fr.wikisource.org/wiki/Page%3ARolland_-_Jean-Christophe%2C_tome_8.djvu/59}{ibid 49})
\z
\is{impossibility|)}

\is{deontic expressions|(}
If to the three categories just mentioned we add an element of will with regard to another being, the result is: 

\begin{enumerate}[label=\Alph*., noitemsep]
\item command
\item permission
\item prohibition
\end{enumerate}

\is{directives|(}
\is{imperatives}
But these three categories are not neatly separated in actual language, at any rate not in the forms of the verb, for the imperative is usually the only form available for A and B. Thus \textit{take that!} may have one of two distinct meanings, (A) a command: `you must take that', (B) a permission: `you may take that', with some intermediate shades of meaning (request, entreaty, prayer). Now a prohibition (C) means at the same time (1) a positive command to not (take that), and (2) the negative of a permission `you are not allowed to (take that)'; hence the possibility of using a negative imperative as a prohibitive: \textit{Don't take that! Don't you stir!}\is{A, B, and C tripartition|)} But hence also the disinclination in many languages to use a negative imperative, because that may be taken in a different and milder sense, as a polite request, or advice, not to, etc. And on the other hand formulas expressive at first of such mild requests may acquire the stronger signification of a prohibition. In Latin, the negative imperative is only found poetically (\ref{ex:08-90}), otherwise we have a paraphrase with \il{Latin!noli@\textit{noli}}\textit{noli} (\ref{ex:08-91}), or a subjunctive (\ref{ex:08-92}). In Spanish the latter has become the rule \il{Spanish!no@\textit{no}}(\textit{no vengas} `don't come').

\ea \label{ex:08-90}
Tu ne cede malis \phantom{x} (`do not yield')\hfill(\href{https://archive.org/details/virgil-aeneid-loeb/page/n529/mode/2up?q=%22tu+ne+cede+malis%22&view=theater}{Virgil, \textit{Aeneid} 6.95})
\z

\ea \label{ex:08-91}
Noli me tangere \phantom{x} (`don't touch me')\hfill(\href{https://archive.org/details/vulgatenewtesta00jerogoog/page/n171/mode/2up?view=theater&q=%22tangere%22}{\textit{Vulgate John} 20.17})
\z

\ea \label{ex:08-92} ne nos inducas in tentationem \phantom{x} (`lead us not into temptation')\\\hfill(\href{https://archive.org/details/vulgatenewtesta00jerogoog/page/n23/mode/2up?view=theater&q=%22inducas%22}{\textit{Vulgate Matthew} 6.13})
\z

In Danish, where \textit{Tag det ikke!} is generally employed to mean `I ask/advise you not to take it', a prohibition is expressed by \textit{La vær å tage det} (\il{Danish!lad vaere@\textit{lad være}}\textit{lad være at tage det}, literally `let be to take it'), % Peter: In single ' ', this should be understandable. But I don't understand it.
%Brett: How about with "literally", since the English meaning is already given.
% PE: Good
which has also the advantage of presenting the negative element first, or colloquially often by \il{Danish!ikke@\textit{ikke}}\textit{Ikke ta}(\textit{ge}) \textit{det!} (\textit{not} + infinitive), which, like the corresponding German formula (\textit{Nicht hinauslehnen} `don't lean out'), has developed through children's echo of the fuller sentence: \textit{Du må ikke tage det!} (\textit{Du darfst nicht hinauslehnen!} `you aren't allowed to lean out').\is{directives|)}

\is{jussives}
\il{Greek!\textit{mē}}
In other languages, separate verb-forms (``jussive'') have developed for prohibitions, or else negative adverbs distinct from the usual ones (cf. Greek \textit{mē}), see \citet[\href{https://archive.org/details/charakteristik-der-hauptsachlichsten/page/22/mode/2up?view=theater}{22}]{misteli1893charakteristik}.

\label{must-and-may} This will serve to explain some peculiarities in the use of English \textit{must} and \textit{may}. As we have seen, a prohibition means (1) a positive command to not {\dots}; thus: \textit{you must} (positive) \textit{not-take that} (negative); and (2) the negative of a permission: \textit{you may-not} (negative) \textit{take} (positive) \textit{that}. But in (1) we have the usual tendency to attract the negation to the auxiliary (see p.~\pageref{sec:neg-prefix}), and thus we get \textit{you mustn't take that}, which never has the sense of `it is not necessary for you to take that' (negative \textit{must}), but has become the ordinary prohibitive auxiliary. On the other hand, in (2) we have the competition with the usual combination of (positive) \textit{may} + negative infinitive, as in \il{English!may not@\textit{may not}|(}\textit{He may not be rich, but he is a gentleman}; this makes people shrink from \textit{may-not} in a prohibition, the more so as \textit{may} is felt to be weaker and more polite than the more brutal \textit{must}. The result is that to the positive \textit{we may walk on the grass} corresponds a negative `we mustn't walk on the grass'.

See on such semantic changes as a result of negatives 
\citet[38]{wellander1916ombetydelseutvecklingens}.

\is{epistemic expressions|(}
The old \textit{may not} in prohibitions, which was extremely common in Shakespeare, is now comparatively rare, except in questions implying a positive answer \il{English!mayn't@\textit{mayn't}|(}(\textit{mayn't I?} = `I suppose I may') and in close connexion with a positive \textit{may}, thus especially in answers. In our last quotation (\ref{ex:8-may_must_not}), it is probably put in for the sake of variation: (\ref{ex:08-93}). % ??? As ": (\ref{ex:08-93})" comes immediately after "probably put in for the sake of variation" it looks like a pointer to that particular quotation. But of course it isn't. Maybe "In our last quotation (\ref{ex:8-may_must_not}) among (\ref{ex:08-93}), it" ?

\ea \label{ex:08-93}
\ea
``I must needs after him {\dots}'' --- ``{\dots} stay with vs {\dots}'' --- ``I may not {\dots}''\\\hfill(\href{https://internetshakespeare.uvic.ca/doc/Lr_F1/scene/4.4/index.html#tln-2400}{Shakespeare, \textit{Lr} 4.4.16}) % OJ says 4.5 but the first folio has it in 4.4
\ex
such a one, as a man may not speake of, without he say sir reuerence\\\hfill(\href{https://internetshakespeare.uvic.ca/doc/Err_F1/scene/3.2/index.html#tln-880}{Shakespeare, \textit{Err} 3.2.92})
\ex
``You may not in, my lord.'' --- ``May we not''\hfill(\href{https://quod.lib.umich.edu/e/eebo/A07018.0001.001/1:2?rgn=div1;view=fulltext}{Marlowe, \textit{Edward} 939}) % It seems as if there's no question mark at the end
\ex
Mayn't my cousin stay with me?\hfill(\href{https://archive.org/details/in.ernet.dli.2015.219151/page/n187/mode/2up?q=%22mayn%27t+my+cousin%22&view=theater}{Congreve, \textit{Love} 249})
\ex
How it is that I appear before you in a shape that you can see, I may not tell.\hfill(\href{https://archive.org/details/christmascarol0000char_h5c8/page/36/mode/2up?q=%22I+may+not+tell%22&view=theater}{Dickens, \textit{Carol} 17}) % "in a shape that you can see" restored
\ex
Meanwhile, may n't I see the Dowager's?\hfill(\href{https://archive.org/details/dollydialogues00hope_0/page/88/mode/2up?view=theater&q=%22dowager%27s%22}{Hope, \textit{Dialogues} 59}) % "Meanwhile" restored; "mayn't" opened up as it had been; "dodges" de-garbled.
\ex
May not I accompany you {\dots}?\hfill(\href{https://archive.org/details/dollydialogues00hope_0/page/144/mode/2up?view=theater&q=%22may+not+I%22}{ibid 90}) % Dots to replace "to the ceremony"
\ex
``Perhaps I may kiss your hand?'' --- ``No, you may not''\\\hfill(\href{https://archive.org/details/returnofthenativ00harduoft/page/56/mode/2up?q=%22Perhaps+I+may%22&view=theater}{Hardy, \textit{Return} 73})
\ex
``May I tell you?'' --- ``No, you may not''\hfill(\href{https://archive.org/details/judgmentbookssto00bens/page/130/mode/2up?view=theater&q=%22may+i+tell+you%22}{E. F. Benson, \textit{Judgment} 164})
\ex
They may study maps beforehand {\dots} but they may not carry such helps. They must not go by beaten ways\hfill(\href{https://archive.org/details/modernutopi00well/page/302/mode/2up?view=theater&q=%22study+maps+beforehand%22}{Wells, \textit{Utopia} 302})
\ex \label{ex:8-may_must_not}
a Polish Jew must not leave the country, may not even quit his native town, unless it suits a paternal government that he should go elsewhere\hfill(\href{https://archive.org/details/vulturesnovel00merr/page/146/mode/2up?ref=ol&view=theater&q=%22polish+jew%22}{Merriman, \textit{Vultures} 175}) % The edition linked to (which isn't the one OJ cites) has small-"g" "government"; not "the" but "a" Polish Jew
\z
\z

\il{English!must not@\textit{must not}|(}
Positive \textit{may} and negative \textit{must not} are frequently found together: (\ref{ex:08-104}).

\ea \label{ex:08-104}
\ea
Your labour only may be sold; your soul must not.\hfill(\href{https://archive.org/details/afs9348.0001.001.umich.edu/page/93/mode/2up?view=theater&q=%22Your+labour+only+may+be+sold%22}{Ruskin, \textit{Time} 102})
\ex
Prose must be rhythmical, and it may be as much so as you will; but it must not be metrical. It may be anything, but it must not be verse.\\\hfill(\href{https://archive.org/details/essaysinartofwri00stevuoft/page/26/mode/2up?view=theater&q=%22rhythmical%22}{Stevenson, \textit{Art} 26}) % "Prose may" corrected to "Prose must", according to RLS
\ex
``I mustn't kiss your face,'' said he, ``but your hands I may kiss''\\\hfill(\href{https://archive.org/details/in.ernet.dli.2015.53170/page/n111/mode/2up?q=%22kiss+your+face%22&view=theater}{Hope, \textit{Rupert} 86}) % Restored "said he"
\ex
you may call me Dolly if you like; but you musnt call me child\\\hfill(\href{https://archive.org/details/dli.ministry.09828/page/251/mode/2up?q=%22may+call+me%22&view=theater}{Shaw, \textit{Never} 251})
\z
\z

\textit{May} is thus used even in tag questions after \textit{must not}: (\ref{ex:08-108}).

\ea \label{ex:08-108}
\ea
I must not tell, may I, Elinor?\hfill(\href{https://archive.org/details/sensesensibility00austrich/page/50/mode/2up?q=%22must+not+tell%22&view=theater}{Austen, \textit{Sense} 62})
\ex
``You mustn't marry more than one person at a time, may you, Peggotty?'' --- ``Certainly not'' {\dots} --- ``But if you marry a person, and the person dies, why then you may marry another person, mayn't you, Peggotty?'' --- ``You \emph{may} {\dots} if you choose, my dear.''\hfill(\href{https://archive.org/details/personalhistory05dickgoog/page/n13/mode/2up?q=%22marry+more%22&view=theater}{Dickens, \textit{David} 16}) % Restoring the vocatives, etc
\z
\z

On the other hand, \textit{must} begins to be used in tag questions, though it is not possible to ask \textit{Must I?} instead of \textit{May I?} (\ref{ex:08-110}).

\ea \label{ex:08-110}
\ea
I must not go any farther, I think, must I?\hfill(\href{https://archive.org/details/millonfloss0009geor/page/328/mode/2up?q=%22must+not+go%22&view=theater}{Eliot, \textit{Mill} 2.50}) % "further" corrected to "farther", and ", I think" restored
\ex
I suppose I must not romp too much now, must I?\\\hfill(\href{https://archive.org/details/prodigalson00caingoog/page/n144/mode/2up?view=theater&q=%22romp+too+much%22}{Caine, \textit{Prodigal} 136})
\z
\z
\is{deontic expressions|)}
\il{English!must not@\textit{must not}|)}

I may add here a few examples (\ref{ex:08-112}) of \textit{may} denoting possibility with a negative infinitive (\textit{you may not know} = `it is possible that you do not know'); in the first two quotations \textit{not} is attracted to the verb.

\ea \label{ex:08-112}
\ea
You mayn't know it, Brown, but {\dots} \hfill(\href{https://archive.org/details/tombrownatoxford00hughiala/page/228/mode/2up?q=%22you+mayn%27t+know+it%22&view=theater}{Hughes, \textit{Oxford} 222}) % Restoring "Brown"
\ex
What may be permissible to a scrubby little artist in Paris {\dots} mayn't be permitted to one who ought to know better.\hfill(\href{https://archive.org/details/wonderfulyear00lockuoft/page/284/mode/2up?q=%22scrubby+little+artist%22&view=theater}{Locke, \textit{Year} 269}) % Dots for "said Martin"
\il{English!mayn't@\textit{mayn't}|)}
\ex
newcomers whom they may not think quite good enough for them\\\hfill(\href{https://archive.org/details/widowershousesun00shaw/page/16/mode/2up?q=%22newcomers+whom%22&view=theater}{Shaw, \textit{Houses} 16})
\ex
I may not be an earl, but I have a perfect right to be useful.\\\hfill(\href{https://archive.org/details/dollydialogues00hope_0/page/146/mode/2up?view=theater&q=%22be+an+earl%22}{Hope, \textit{Dialogues} 91})
\z
\z

With \textit{may} we see another semantic change brought about by a negative: to the positive \textit{may, might} corresponds a negative \textit{cannot, could not} (not \textit{may not}, \textit{might not}): (\ref{ex:08-116}).\il{English!may not@\textit{may not}|)}

\ea \label{ex:08-116}
\ea
This \emph{cannot} do harm and \emph{may} do good\hfill(news 1917) % Unfortunately the example at https://books.google.co.jp/books?id=q1swAQAAMAAJ&pg=PA727&lpg=PA727&dq="this+cannot+do+harm+and+may+do+good"&source=bl&ots=6X-hIjAu6N&sig=ACfU3U32m7I3HfPcAOzf5N07CHaUuTXvCw&hl=en&sa=X&ved=2ahUKEwjmgNeIobSGAxUU1zgGHQ-lANcQ6AF6BAgLEAM#v=onepage&q="this cannot do harm and may do good"&f=false dates not from 1917 but from 1868.
\ex
I \emph{might} prudently, perhaps, but I \emph{could not} honestly, admit that charge [of careless writing]\hfill(\href{https://books.google.co.jp/books?id=gfRCAQAAMAAJ&pg=PA282&lpg=PA282&dq=%22I+might+prudently,+perhaps%22&source=bl&ots=5oIe7EGS_S&sig=ACfU3U0B7_Bz74gbktWEgeDnRsYs6tHXGg&hl=en&sa=X&ved=2ahUKEwjhw4jKtN6FAxVwl1YBHd8NBbcQ6AF6BAgJEAM#v=onepage&q=%22I%20might%20prudently%2C%20perhaps%22&f=false}{Cowper, letter, 27 February 1786})
\ex
Raphael's dialectic, too, though it \emph{might} silence her, \emph{could not} convince her.\hfill(\href{https://archive.org/details/hypatia00kinggoog/page/n442/mode/2up?q=%22might+silence+her%22&view=theater}{Kingsley, \textit{Hypatia} 357}) % Not "his" but "Raphael's", plus "too"
\ex
``He \emph{might} be a Turk,'' said Father Henaghan. --- ``No, he \emph{couldn't}.''\\\hfill(\href{https://archive.org/details/advofdrwhitty00birmiala/page/94/mode/2up?q=%22he+might+be+a+turk%22&view=theater}{Birmingham, \textit{Whitty} 94}) % "said Father Henaghan" restored
\z
\z
\is{epistemic expressions|)}
\is{auxiliary verbs|)}
