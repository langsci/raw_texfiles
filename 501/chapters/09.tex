\ChapterAndMark{Weakened Negatives} 
\label{ch:9}
\is{weakened negatives|(}
\is{interrogatives!negative}
\is{optionality of negative expression|(}
Negative words or formulas may in some combinations be used in such a way that the negative force is almost vanishing. There is scarcely any difference between questions like \textit{Will you have a glass of beer?} and \textit{Won't you have a glass of beer?}, because the real question is `Will you, or will you not, have {\dots}'; therefore, in offering one a glass, both formulas may be employed indifferently, though a marked tone of surprise can make the two sentences into distinct contrasts: \textit{Will you have a glass of beer?} then coming to mean `I am surprised at your wanting it', and \textit{Won't you have a glass of beer?} the reverse. (In this case, \textit{really} is often added.) 

\il{Danish!ikke@\textit{ikke}}
In the same way in Danish: \textit{Vil De ha et glas øl?} (`Will you have a glass of beer?') and \textit{Vil De ikke ha et glas øl?} (`Won't you have a glass of beer?'). A Dutch lady once told me how surprised she was at first in Denmark at having questions like \textit{Vil De ikke række mig saltet?} (`Will you not pass the salt?') asked her at a table in a boarding-house; she took the \textit{ikke} literally and did not pass the salt. \textit{Ikke} is also used in indirect (reported) questions, as in (\ref{ex:9-01}).\largerpage[-2]

\ea \label{ex:9-01}
 \gll saa har madammen bedt Giovanni, om han ikke vil passe lidt paa barnet\\
 so has lady.\DEF{} asked Giovanni if he not will look {a bit} on child.\DEF{}\\\normalsize
\glt `so, the lady has asked Giovanni if he wouldn't mind looking after the child a bit'\hfill(\href{https://www.kb.dk/e-mat/dod/130024712320-color.pdf}{Faber, \textit{Stegekjælderen} 28}) % A very bulky PDF; we should look for an alternative version
\z

A polite request is often expressed by saying \textit{Would} (or, \textit{Do}) \textit{you mind taking {\dots}} and, as \textit{mind} means `object to', the logical answer is \textit{no} (`I don't mind'), but very frequently \textit{yes} or some other positive reply (\textit{By all means!} etc.) is used, which corresponds to the implied positive request: (\ref{ex:9-02}).

\ea \label{ex:9-02}
\ea
Drummle: ``When you two fellows go home, do you mind leaving me behind here?'' --- Misquith:~``Not at all.'' --- Jayne:~``By all means.''\\\hfill(\href{https://archive.org/details/thesecondmrstanq00pineuoft/page/20/mode/2up?q=%22when+you+two+fellows%22&view=theater}{Pinero, \textit{Second} 21})
\ex ``Do you mind my asking you a question?'' {\dots} --- ``By all means!---What can I do?''\hfill(\href{https://archive.org/details/cu31924013567130/page/160/mode/2up?q=%22do+you+mind+my%22&view=theater}{Ward, \textit{Eleanor} 128}) % Showing an excision, etc
\z
\z


\is{not at all@`not at all', expressions corresponding to}
\textit{Not at all} is frequent as an idiomatic reply to phrases of politeness, which do not always contain words to which \textit{not at all} can be logically attached: (\ref{ex:9-04}).

\ea \label{ex:9-04}
\ea ``I'm sorry to give you so much trouble, Towlinson'' --- ``Not at all miss'' [does not negative the other's feeling sorry, but the giving trouble]\\\hfill(\href{https://archive.org/details/dombeyson00dick_0/page/58/mode/2up? q=%22I%27m+sorry+to+give%22&view=theater}{Dickens, \textit{Dombey} 32}) % Restored the vocatives
\ex ``I'm really excessively sorry, Dombey, that you should have so much trouble about it;'' to which Mr. Dombey answers, ``Not at all.''\\\hfill(\href{https://archive.org/details/dombeyson00dick_0/page/568/mode/2up?q=%22not+at+all%22&view=theater}{ibid 363}) % OJ points to this example but doesn't supply it.
\ex ``Oh, thank you very much for that!'' --- ``Not at all,'' I said, loftily. ``There is no reason why you should thank me.''\hfill(\href{https://archive.org/details/personalhistory05dickgoog/page/n157/mode/2up?q=%22loftily%22&view=theater}{Dickens, \textit{David} 355}) % Punctuation fixed (and "Oh" restored) according to the printed novel
\ex
Undershaft: ``My dear sir, I beg your pardon.'' --- Lomax:~``Not at all. Delighted, I assure you.'' {\dots} --- Undershaft:~``I beg your pardon.'' --- Stephen:~``Not at all''\hfill(\href{https://archive.org/details/johnbullsotheris00shawrich/page/204/mode/2up?q=%22my+dear+stephen%22&view=theater}{Shaw, \textit{Major} 205}) % Yes, two such exchanges on the same page; and yes, "Notatall" once and "Not at all" once. OJ attributes these, or one of them, to Shaw's "J", the abbreviation he uses for John Bull's Other Island. The closest example I can find in that play is "You won't mind me axin, will ye?" "Not at all." https://archive.org/details/johnbullsotheris0000shaw_o3k8/page/18/mode/2up?q=%22not+at+all%22 . The two plays (Major Barbara and John Bull's Other Island) were published within the same volume, which I suppose explains the confusion.
\ex Excuse me: I had a word to say to the servant. (Trench is heard replying ``Not at all'', Cokane ``Dont mention it, my dear sir.'')\\\hfill(\href{https://archive.org/details/widowershousesun00shaw/page/50/mode/2up?q=%22not+at+all%22}{Shaw, \textit{Houses} 48}) % "I had a word to say to the servant" restored. The edition linked to spells the word not "Dont" but "Don't"; the former is what OJ writes, citing a different and I imagine more authoritative edition. (Certainly Shaw liked to write "dont", "doesnt", etc.)
\z
\z

\is{exclamations|(}
\is{exclamative sentences, negative|(}
\is{intonation}
In exclamations, a \textit{not} is often used though no negative notion is really implied; this has developed from the use of a negative \emph{question} for a positive statement: \textit{How often have we not seen him?} for `we have often seen him'; \textit{What have we not suffered?} for `we have suffered everything' (or, `very much'). As an exclamation of this form is a weakened question (as shown also by the tone), we see that in these sentences (\ref{ex:9-09}) the import of the negation is also weakened, so that it really matters very little whether a \textit{not} is added or not, as illustrated clearly by the varied sentences in our first quotation (\ref{ex:stanley}).

\ea \label{ex:9-09}
\ea \label{ex:stanley}
What a long, long and true friendship was here sundered! Through what strange vicissitudes of life had they not followed me! What wild and varied scenes had we not seen together! What a noble fidelity these untutored souls had exhibited!\hfill(\href{https://archive.org/details/throughdarkconti1878stan2/page/482/mode/2up?q=%22what+a+long%22&view=theater}{Stanley, \textit{Dark} 2.482})
\ex What good to his country or himself might not a trader have done with such useful tho' ordinary qualifications?\hfill(\href{https://archive.org/details/spectatornewedre00addiuoft/page/166/mode/2up?q=%22what+good+to+his+country%22&view=theater}{\textit{Spectator} 166}) % Restored both "or himself" and "tho' ordinary", but not the 18th century capitalization
\ex Ah, my friends, what did I not feel at that moment!\hfill(\href{https://archive.org/details/TheStrandMagazineAnIllustratedMonthly/TheStrandMagazine1895aVol.IxJan-jun/page/n517/mode/2up?q=%22what+did+I+not%22&view=theater}{Doyle, \textit{How} 505}) % Not "friend" but "friends"
\ex How often have I not watched him {\dots}  How often have I not seen them coming back, tired as cats\hfill(\href{https://archive.org/details/motleyxx00gals/page/22/mode/2up?q=%22how+often+have+i+not%22&view=theater}{Galsworthy, \textit{Motley} 34})
\z
\z

Somewhat differently in (\ref{ex:9-13}).

\ea \label{ex:9-13}
\ea I don't know how long I should not have gone on grumbling\\\hfill(\href{https://archive.org/details/shipsthatpassin00harr/page/128/mode/2up?view=theater&q=%22I+don%27t+know+how+long%22}{Harraden, \textit{Ships} 71})
\ex no one could say how soon he might not come to himself\\\hfill(\href{https://archive.org/details/grandbabylonhote00bennuoft/page/158/mode/2up?q=%22No+one+could+say%22&view=theater}{Bennett, \textit{Babylon} 121})
\ex What Chaucer might not have produced had he lived ten years longer no one can endure to conjecture.\hfill(\href{https://archive.org/details/in.ernet.dli.2015.513623/page/23/mode/2up?view=theater}{Gosse, \textit{History} 23})% The OCR has gone very wrong here; one can't tell archive.org to highlight this or that within this sentence.
\z
\z

\il{Danish!ikke@\textit{ikke}}
In Danish exclamations \textit{ikke} is extremely frequent: (\ref{ex:9-16}).

\ea \label{ex:9-16}
\ea
 \gll Hvor var han dog ikke rar!\\
 how was he yet not nice\\
\glt `How nice he was!'
\ex
 \gll Hvor har ei da du liidt!\\
 how have not then you suffered\\
\glt `How you have suffered!"'\hfill(Paludan-Müller 6.380) % ??S Peter: What's this? I imagine that it's something by Frederik Paludan-Müller, but he was a poet and his collected poems were issued in a three-volume set. Line 380 of act 6 of some play of his? It could be Jens Paludan-Müller, but I don't see his collected works anywhere.
%Brett:¯\_(ツ)_/¯
%% SG: There are several different collections of Paludan-Müller's works. This could be vol. 6 of "Poetiske Skrifter" https://search.worldcat.org/title/162808102  or vol. 6 of "Poetiske skrifter i udvalg" https://search.worldcat.org/title/257220983
\ex
 \gll Hvilken større glæde kunde jeg ikke faa her paa jorden.\\
 which greater joy could I not get here on earth.\DEF{}\\
\glt `What greater joy could I possibly receive here on earth.'\hfill(\href{https://tekster.kb.dk/text/adl-texts-larsenk_11-root#s124}{AKC, letter})
\z
\z

\il{German!nicht@\textit{nicht}}
In German, \textit{nicht} was frequent in exclamations in the 18th century \textit{wie ungesucht war nicht der gang seines Glücks} (`How unsought was the course of his happiness'), now the positive form is preferred \citep[\href{https://archive.org/details/deutscheswrte00paul/page/382/mode/2up?view=theater}{383}]{paul_deutsches_1908}.
\is{exclamations|)}
\is{exclamative sentences, negative|)}

\is{concessive clauses and phrases|(}
\il{English!never so@\textit{never so}|(}
In concessive clauses and phrases, \textit{never} (\textit{so}) is often used concurrently with \textit{ever}, which seems to be gaining ground. (Cf. \citet[\href{https://archive.org/details/shakespeariangra0000edwi/page/44/mode/2up?view=theater}{§52}]{abbott1894shakespearian}; \citet[\href{https://archive.org/details/p2englischephilo01storuoft/page/702/mode/2up?view=theater}{702}]{storm1896englische}; \citet[\href{https://archive.org/details/queensenglishstr00alfo/page/62/mode/2up?q=\%22160\%22&view=theater}{62}]{alford1888queens}; \citet[88]{bogholm1906bacon}.)

\textit{Never so} after \textit{though} and \textit{if}: \refp{ex:9-19}, \refp{ex:9-19a}.

\ea \label{ex:9-19}
\ea For though his wyf be cristned never so whyte, She shal have nede to wasshe awey the rede\hfill(\href{https://archive.org/details/completeworksofg04chauuoft/completeworksofg04chauuoft/page/140/mode/2up?q=%22wyf+be+cristned%22&view=theater}{Chaucer, \textit{Lawe} B~355})
\ex 
he shall sterue for honger, {\dots} though the commen wealthe floryshe neuer so muche\hfill(\href{https://archive.org/details/utopiasirthomas00robigoog/page/n409/mode/2up?q=%22sterue+for+honger%22&view=theater}{More, \textit{Utopia} 299}) % Dots for omission
\ex they neuer so willingly offer them selfes therto\hfill(\href{https://archive.org/details/utopiasirthomas00robigoog/page/n163/mode/2up?q=%22neuer+so+willingly%22&view=theater}{ibid 54}) % OJ doesn't provide the example (he merely points to it); here it is.
\ex the numbre of shepe increase neuer so fast\hfill(\href{https://archive.org/details/utopiasirthomas00robigoog/page/n163/mode/2up?q=%22increase+neuer+so%22&view=theater}{ibid 55}) % OJ doesn't provide the example (he merely points to it); here it is.
\ex thoughe leagues be neuer so faythfully obserued and kept {\dots} \hfill(\href{https://archive.org/details/utopiasirthomas00robigoog/page/n351/mode/2up?q=%22leagues+be+neuer+so%22&view=theater}{ibid 241}) % OJ doesn't provide the example (he merely points to it); here it is.
\ex If I make my handes neuer so cleane, yet shalt thou plunge me in the ditch\hfill(\href{https://www.kingjamesbibleonline.org/1611_Job-9-30/}{AV \textit{Job} 9.30})
\ex any deceased author, though never so famous in his life time\\\hfill(\href{https://archive.org/details/areopagitica00miltuoft/page/32/mode/2up?q=%22any+deceased+author%22&view=theater}{Milton, \textit{Areopagitica} 30})
\ex had I but seen a priest (though never so sordid and debauched in his life,)\hfill(\href{https://archive.org/details/graceaboundingto00buny/page/14/mode/2up?q=%22seen+a+priest%22&view=theater}{Bunyan, \textit{Grace} 11}) % Yes, "life"-comma-parenthesis is the order.
\z
\z

\ea \label{ex:9-19a}
\ea yf it were neuer so muche\hfill(\href{https://archive.org/details/utopiasirthomas00robigoog/page/n147/mode/2up?q=%22were+neuer+so+muche%22&view=theater}{More, \textit{Utopia} 38})
\ex if thou dost intend Neuer so little shew of loue to her, Thou shalt abide it\hfill(\href{https://internetshakespeare.uvic.ca/doc/MND_F1/scene/3.2/index.html#tln-1370}{Shakespeare, \textit{Mids} 3.2.34})
\z
\z

\is{inverted word order}
It is very frequent in clauses with inverted word-order and no conjunction: (\ref{ex:9-29}). Cf. also  (\ref{ex:9-39}).

\ea \label{ex:9-29}
\ea were she never so glad, Hir loking was not foly sprad\\\hfill(\href{https://archive.org/details/minorpoemsedite00chau/page/44/mode/2up?q=%22loking+was+not%22&view=theater}{Chaucer, \textit{Duchesse} 873})
\ex be hit never so derke, Me thinketh I se her ever-mo\hfill(\href{https://archive.org/details/minorpoemsedite00chau/page/44/mode/2up?q=%22hit+never+so+derke%22&view=theater}{ibid 913}) % Jespersen points to the example but doesn't provide it.
\ex Me thoghte no-thing mighte me greve, Were my sorwes never so smerte\hfill(\href{https://archive.org/details/minorpoemsedite00chau/page/52/mode/2up?q=%22were+my+sorwes%22&view=theater}{ibid 1106}) % Jespersen points to the example but doesn't provide it. Change from 1107 to 1106 as I (PE) think it better to start on the preceding line
\ex A wower be he neuer so poore Must play and sing before his bestbeloues doore\hfill(\href{https://archive.org/details/roisterdoister00udalgoog/page/n54/mode/2up?view=theater&q=%22wower+be+he%22}{\textit{Roister} 48})
\ex they thinke it not lawfull to touch him with mannes hande, be he neuer so vityous\hfill(\href{https://archive.org/details/utopiasirthomas00robigoog/page/n397/mode/2up?q=%22neuer+so+vityous%22&view=theater}{More, \textit{Utopia} 286}) % Restored "with mannes hande"
\ex and creepe time nere so slow, Yet it shall come, for me to doe thee good\hfill(\href{https://internetshakespeare.uvic.ca/doc/Jn_F1/scene/3.2/index.html#tln-1330}{Shakespeare, \textit{John} 3.3.31})
\ex wisest men Have err'd {\dots} And shall again, pretend they ne're so wise\\\hfill(\href{https://archive.org/details/poeticalworksofj00miltiala/page/514/mode/2up?ref=ol&view=theater&q=%22wisest+men%22}{Milton, \textit{Samson} 210}) % This finishes on line 212 but starts on line 210.
\ex forgive her all her sins, be they never so many\hfill(\href{https://archive.org/details/bim_eighteenth-century_the-history-of-tom-jones_fielding-henry_1768_4/page/300/mode/2up?q=%22forgive+her+all%22&view=theater}{Fielding, \textit{Tom} 4.301})
\ex go they never so glibly\hfill(\href{https://archive.org/details/forsclavigeralet12rusk/page/n145/mode/2up?view=theater&q=%22never+so+glibly%22}{Ruskin, \textit{Fors} 95})
\ex there was a sullen silence, which Paul could not charm away, charm he never so wisely\hfill(\href{https://archive.org/details/sowersnovel00merr/page/286/mode/2up?q=%22sullen+silence+which+paul%22&view=theater}{Merriman, \textit{Sowers} 179}) % comma restored
\z
\ex \label{ex:9-39}
\ea lette neuer so little a gappe be open, And {\dots} the worst shall be spoken\\\hfill(\href{https://archive.org/details/roisterdoister00udalgoog/page/n86/mode/2up?view=theater&q=%22lette+neuer%22}{\textit{Roister} 81})
\ex curb her never so little, she kicks up, and you're flung in a ditch\\\hfill(\href{https://archive.org/details/shestoopstoconqu03gold/page/60/mode/2up?q=%22curb+her+never+so+little%22&view=theater}{Goldsmith, \textit{Stoops}})
\z
\z

Other examples of \textit{never so}: (\ref{ex:9-41}).

\ea \label{ex:9-41}
\ea Thou {\dots} wilt know againe, Being ne're so little vrg'd another way\\\hfill(\href{https://internetshakespeare.uvic.ca/doc/R2_F1/scene/5.1/index.html#tln-2320}{Shakespeare, \textit{R2} 5.1.64}) % Apostrophe rather than "e" in "vrg'd"; "thou" on an earlier line
\ex There will not again be any man, never so great, whom his fellowmen will take for a god.\hfill(\href{https://archive.org/details/heroesheroworshi00carl/page/38/mode/2up?q=%22there+will+not+again+be%22&view=theater}{T. Carlyle, \textit{Heroes} 39}) % "fellow-/men" occurs across a line break.
\ex
{}[the pain] ceased, except when the wounded limb was meddled with never so little\hfill(\href{https://archive.org/details/reminiscences0000thom_e9a0/page/472/mode/2up?q=%22wounded+limb%22&view=theater}{T. Carlyle, \textit{Reminiscences} 2.258}) 
\ex I have heard a hundred anecdotes about William Hazlitt \dots; yet cannot by never so much cross-questioning even form to myself the smallest notion of how it really stood with him.\hfill(\href{https://archive.org/details/thomascarlylehis00frouiala/page/122/mode/2up?view=theater&q=%22hundred+anecdotes%22}{T. Carlyle, \textit{Life} 2.209}) % dots for ", for example"
\ex Private men keep their promises, never so trivial.\\\hfill(\href{https://quod.lib.umich.edu/e/emerson/4957107.0005.001/132:6.7}{Emerson, \textit{Traits} 308})
\z
\z
\il{English!never so@\textit{never so}|)}

\il{English!ever so@\textit{ever so}}
Some examples of \textit{ever so} (\ref{ex:9-46}) may serve to show that the signification is exactly the same as of the negative phrase.

\ea \label{ex:9-46}
\ea Every man desired to put off death for sometime longer, let it approach ever so late\hfill(\href{https://archive.org/details/bim_eighteenth-century_the-works-of-j-s-dd-_swift-jonathan_1735_3/page/270/mode/2up?view=theater&q=%22desired+to+put%22}{J. Swift, \textit{Travels} 271}) % "for sometimes longer" restored; "Man" and "Death" both capitalized in the original
\ex There is something of farce in all these mournings, let them be ever so serious.\hfill(\href{https://archive.org/details/journaltostellae00swifuoft/page/492/mode/2up?q=%22something+of+farce%22&view=theater}{J. Swift, \textit{Journal} 492})
\ex Pray write to me a good-humoured letter immediately, let it be ever so short\hfill(\href{https://archive.org/details/journaltostellae00swifuoft/page/544/mode/2up?q=%22good-humoured%22&view=theater}{ibid 545}) % "to" added, in accordance with the cited book
\ex The honest man, tho' e'er sae poor, Is king o' men for a' that.\\\hfill(\href{https://archive.org/details/selectedpoems00burn/page/226/mode/2up?view=theater}{Burns, \textit{Man}})
\ex how easily my watchful reason, if ever so slightly provoked, would drag me back to life\hfill(\href{https://www.gutenberg.org/files/43684/43684-h/43684-h.htm#page113}{Kinglake, \textit{Eothen} 113})
% OJ omits "watchful".
\ex A chance of being useful, in ever so little a way\hfill(\href{https://archive.org/details/crownofwildolive00ruskiala/page/70/mode/2up?view=theater&q=%22chance+of+being+useful%22}{Ruskin, \textit{Crown} 68}) % “in ever so humble a way” appears in https://books.google.co.jp/books?id=VMzYEAAAQBAJ&pg=PA59&lpg=PA59&dq=%22chance+of+being+useful,+in+ever+so+humble+a+way%22&source=bl&ots=VY4ZTseOaQ&sig=ACfU3U1SDs6a07WAPU6wcAhSXPxPD4Mx6g&hl=en&sa=X&ved=2ahUKEwjJsvr16L2FAxXIrlYBHQ1cDygQ6AF6BAgIEAM#v=onepage&q=%22chance%20of%20being%20useful%2C%20in%20ever%20so%20humble%20a%20way%22&f=false , which Google tells us is from vol 6 of The Works of John Ruskin “Reprint of the original, first published in 1873”. I (PE) don’t want to link to this, as although the particular page is reproduced clearly, the scan has many missing pages. Most other versions, including the one I link to, have “in ever so little a way”. So for now we have the appearance of OJ misquoting “little” as “humble”. This isn’t fair to OJ. I suggest discreetly changing “humble” to “little”.
%Brett: done
\ex no one will be vexed or uneasy, linger I ever so late\hfill(\href{https://archive.org/details/privatepapersofh0000geor/page/10/mode/2up?q=%22No+one+will+be+vexed%22&view=theater}{Gissing, \textit{Henry} 8})% "or uneasy" restored
\z
\z\largerpage[2]

\il{Danish!ikke@\textit{ikke}}
In Danish concessive clauses with \textit{om} we may similarly use either \textit{aldrig} or \textit{nok}:  (\ref{ex:9-53}). The negative purport of \textit{aldrig} is here so little felt that one may even sometimes find \textit{ikke} after it (\ref{ex:9-54}).% Peter: Interesting that OJ gives this as an example of Danish though it's by an author who, it seems, is normally regarded as Norwegian.

\ea \label{ex:9-53}
 \ea  \gll Jeg gør det ikke, om han så ber mig aldrig så meget om det\\
 I do it not if he then asks me never so much about it\\
 \glt `I won't do it, no matter how much he asks me to'
 \ex \gll {\dots} om han så ber mig nok så meget om det\\
 {} if he then asks me enough so much about it\\
 \glt `{\dots} no matter how much he asks me to'
\z
\z
\is{optionality of negative expression|)}

\ea \label{ex:9-54}
 \gll Det er så, om hun så aldrig så meget ikke ved om det.\\
 it is so if she then never so much not knows about it\\
\glt `It is so, however ignorant she may be of it.'\hfill(\href{https://www.bokselskap.no/boker/lucie/xix}{Skram, \textit{Lucie} 193})
\z
\is{concessive clauses and phrases|)}

\is{unlimitedness, expressions of|(}
\il{Russian!ni@\textit{ni}}
In Russian, \textit{ni} after a relative (interrogative) pronoun has the same generalizing effect as English \textit{-ever}: (\ref{ex:9-55}).

\ea \label{ex:9-55}
 \ea  \gll kto by ni sprosil\\
 who would not asked\\
 \glt `whoever asked'
 \ex  \gll kak ni dumal\\
 how not thought\\
 \glt `however much he thought'\hfill\citep[152]{pedersen1916russisk} % PE: Should be a BibTeX reference, I think. Although yes, it is only used as a source of examples.
 %Brett: done
 % ??? PE: I've completely changed my mind. I don't understand why it should be a BibTex reference or why I thought so earlier. May I change it back?
 \z
\z

\il{Danish!ikke@\textit{ikke}}
In the Scandinavian languages, there is a curious way of using \textit{ikke for aldrig det} in the signification `not for the whole world': (\ref{ex:9-56}). Rarely without \textit{ikke}: (\ref{ex:9-61}).

\ea \label{ex:9-56}
\ea \gll Ak! jeg tør ikke spørge et menneske om noget, ikke for aldrig det\\
 alas I dare not ask a person about anything not for never it\\\normalsize
 \glt `Alas! I wouldn't dare ask anyone about anything, not for anything in the world'\hfill(\href{https://tekster.kb.dk/text/sks-slv-txt-root#ss239}{Kierkegaard, \textit{Stadier} 234})
 
 \ex  \gll Man vilde ikke have gjort det samme, ``ikke for aldrig det''\\
 one would not have done the same not for never it\\
 \glt `One would not have done the same, not for anything in the world'\\\hfill(\href{https://tekster.kb.dk/text/adl-texts-goldschmidt03-root#idm139686922076496}{Goldschmidt, \textit{Hjemløs} 1.48}) % restored quotation marks
 
 \ex  \gll A vel ikke træk kjowlen aa ham faar aalle de\\ % OJ has "vel"; but the first edition (which is not the edition that OJ cites, as it has different pagination) has "vil".
 I well not pull coat.\DEF{} off him for all it\\
 \glt `I surely wouldn't take the coat off him for anything'\\\hfill(Jutlandic; \href{https://books.google.co.jp/books?id=zggQAQAAIAAJ&printsec=frontcover&dq=inauthor:%22Steen+Steensen+Blicher%22&hl=ja&newbks=1&newbks_redir=0&sa=X&redir_esc=y#v=onepage&q=%22faar%20aalle%22&f=false}{Blicher, \textit{Bindstouw} 48})
 
 \ex  \gll hun vilde ikke truffet toldinspektøren i nattrøye for aldrig det\\
 she would not met {customs.inspector} in nightgown for never it\\
 \glt `not for anything would she have met the customs inspector in her nightgown'\hfill(Norwegian; \href{https://archive.org/details/naarsolgaarnedfo00liej/page/4/mode/2up?q=%22hun+vilde+ikke+truffet%22&view=theater}{Lie, \textit{Sol} 5})
 
 \ex  \gll Han ville icke sälja den för aldrig det\\
 he would not sell it for never it\\
 \glt `He would not sell it for all the world'\\\hfill(Swedish; \href{https://litteraturbanken.se/f%C3%B6rfattare/StrindbergA/titlar/UtopierIVerkligheten1885/sida/51/faksimil}{Strindberg, \textit{Utopier} 51}) % OJ says 52; actually the quote starts on 51 and extends to 52.
\z
\z
\is{unlimitedness, expressions of|)}


\ea \label{ex:9-61}
  \gll Han vilde have givet aldrig det for at kunne have bekæmpet sin uro.\\
 he would have given never it for to can have fought his dread\\
 \glt `He would have given anything to be able to quell his dread.'\\\hfill(\href{https://tekster.kb.dk/text/adl-texts-larsenk_08-root#s139}{Larsen, \textit{Punkt} 138})
\z

\is{no@`no', expressions corresponding to}
\is{yes@`yes', expressions corresponding to}
\il{English!nay@\textit{nay}}\il{English!yea@\textit{yea}}
\il{Old Norse!nei@\textit{nei}}
\label{ch10-nay}Among weakened negatives should also be mentioned \textit{nay} (Old Norse \textit{nei}): when one has used a weak expression and finds that a stronger might be properly applied, the addition is partly a contradiction, partly a confirmation, as going further in the same direction. Hence, both \textit{nay} and \textit{yea} may be used in the same sense (note that both were in Middle English and early Modern English less strong than \textit{no} and \textit{yes}, respectively). Thus: (\ref{ex:9-62}).

\ea \label{ex:9-62}
\ea We are betroathd: nay more, our mariage how're With all the cunning manner of our flight Determin'd of\hfill(\href{https://internetshakespeare.uvic.ca/doc/TGV_F1/scene/2.4/index.html#tln-830}{Shakespeare, \textit{Gent} 2.4.179}) % Only one "r" in "mariage"; "With all the cunning manner of our flight" restored; not "Determin'd on" but "Determin'd of"; a stray "(" ignored. (Yes, https://firstfolio.bodleian.ox.ac.uk/text/46 agrees with the uvic.ca page.)
\ex threatned me To strike me, spurne me, nay, to kill me too\\\hfill(\href{https://internetshakespeare.uvic.ca/doc/MND_F1/scene/3.2/index.html#tln-1345}{Shakespeare, \textit{Mids} 3.2.313})
\ex I should be as bad, nay worse, than I was before\\\hfill(\href{https://archive.org/details/bunyanspilgrims00moffgoog/page/180/mode/2up?q=%22should+be+as+bad%22&view=theater}{Bunyan, \textit{Progress} 189})
\ex The Mediterranean Sea {\dots} the chief, nay, almost the one sea of history\\\hfill(\href{https://archive.org/details/ourcolonialexpan00seel/page/50/mode/2up?q=%22the+chief%2C+nay%22&view=theater}{Seeley, \textit{Expansion} 89})
\z
\z

Cf. \textit{yea} (\ref{ex:9-66}).

\ea \label{ex:9-66}
here I tender it for him in the Court, Yea, twice the summe, if that will not suffice\hfill(\href{https://internetshakespeare.uvic.ca/doc/MV_F1/scene/4.1/index.html#tln-2120}{Shakespeare, \textit{Merch} 4.1.210})
\z

\textit{Nay} is preserved with the old negative meaning in connexion with \textit{say}, probably for the sake of the rime: (\ref{ex:9-67}).

\ea \label{ex:9-67}
\ea no one had the right to say him nay\hfill(\href{https://www.gutenberg.org/cache/epub/57710/pg57710-images.html}{Ridge, \textit{Son} 64})
\ex
With no one to say him nay\hfill(\href{https://archive.org/details/bwb_UE-390-059/page/86/mode/2up?view=theater&q=%22with+no+one+to+say+him+nay%22}{Parker, \textit{Right} 77})
\z % Peter: The square brackets [ ] are of course OJ's. I suppose that they signal a digression. If indeed they do, they're odd. The whole thing could be a footnote -- but numbered examples don't fare well in footnotes. Shall we simply delete the opening and closing bracket?
%Brett: done
\z

\il{Danish!ja@\textit{ja}}\il{Danish!ne@\textit{ne}}
In Danish, both \textit{ja} and \textit{nej} may be used in correcting or pointing out a statement: \refp{ex:09-new}.

\ea \label{ex:09-new}
\ea
\gll han er millionær, nej mangemillionær\\
    he is millionaire no multi-millionaire\\
\glt `he is a millionaire, no [correction:] a multi-millionaire'
\ex
\gll han er millionær, ja mangemillionær\\  
    he is millionaire yes multi-millionaire\\ 
\glt `he is a millionaire, yes [emphasizing:] a multi-millionaire'
\z
\z

\bigskip
A weakened negative is also found in the colloquial exaggeration \textit{no time} (or humorously \textit{less than no time}) meaning `a very short time': (\ref{ex:9-69}).\largerpage[2]

\ea \label{ex:9-69}
\ea Gip {\dots} got it in no time.\hfill(\href{https://archive.org/details/HGWellsTwelveStoriesAndADream/page/n51/mode/2up?q=%22in+no+time%22}{Wells, \textit{Stories} 17}) % Linked to a nasty-looking version; let's find something better
\ex The news will filter through the town in no time.\hfill(\href{https://archive.org/details/in.ernet.dli.2015.53170/page/n263/mode/2up?q=%22news+will+filter%22&view=theater}{Hope, \textit{Rupert} 203})
\ex And all this in five minutes less than no time at all\\\hfill(\href{https://www.gutenberg.org/files/39270/39270-h/39270-h.htm#page170}{Sterne, \textit{Tristram} 3.38}) % Renumbered for volume and chapter.
\z
\z

\is{unlimitedness, expressions of}
\il{English!no end@\textit{no end}}
A different case is found with \textit{no end}, which is used colloquially for `an infinite quantity', i.e., `very much' or `very many'; in recent times, this is even found where no quantity is thought of: \textit{no end of a fine fellow} (`a very fine fellow'), \textit{no end of a man} (`a real man' or `a great man'): (\ref{ex:9-72}).

\ea \label{ex:9-72}
\ea the Alderman had sealed it with a very large coat of arms and no end of wax\hfill(\href{https://archive.org/details/chimes00dick/page/78/mode/2up?q=%22large+coat%22&view=theater}{Dickens, \textit{Chimes} 101})
\ex Every man {\dots} must make no end of melancholy reflections\\\hfill(\href{https://archive.org/details/in.ernet.dli.2015.61330/page/n223/mode/2up?q=%22melancholy+reflections%22&view=theater}{Thackeray, \textit{Snobs} 128}) % Thackeray doesn't write "everybody"; and dots for the omission of a relative clause.
\ex I have sometimes no end of trouble to get rid of the alliteration.\\\hfill(\href{https://archive.org/details/alfredlordtenny05tenngoog/page/n558/mode/2up?q=%22no+end+of+trouble%22&view=theater}{Tennyson, comment})
\ex Parliament had passed no end of laws against it\\\hfill(\href{https://books.google.co.jp/books?id=9PjnkXA5jOkC&pg=PP5&dq=%22justin+mccarthy%22+%22our+own+times%22&hl=en&newbks=1&newbks_redir=0&sa=X&ved=2ahUKEwiE2YnrmaCFAxVsoa8BHcesDZwQ6AF6BAgKEAI#v=snippet&q=%22parliament%20had%20passed%20no%20end%22&f=false}{McCarthy, \textit{History} 2.402})
\ex We'll take an interest in the house. We'll take no end of interest in the house!\hfill(\href{https://archive.org/details/stalkyandco015455mbp/page/n115/mode/2up?q=%22interest+in+the+house%22&view=theater}{Kipling, \textit{Stalky} 119})
\ex I'm doing a lot of work. No end of work---more than I've ever done.\\\hfill(\href{https://archive.org/details/newgrubstreetnov01gissuoft/page/198/mode/2up?q=%22doing+a+lot+of+work+%22&view=theater}{Gissing, \textit{Grub} 96})
\ex Mrs. Horrocks has had no end of a good time.\hfill(\href{https://archive.org/details/dramaticworksofs02hank/page/12/mode/2up?q=%22has+had+no+end+of%22&view=theater}{Hankin, \textit{Works} 2.16})
\ex there'll be no end of a scandal\hfill(\href{https://archive.org/details/dramaticworksofs02hank/page/188/mode/2up?q=%22no+end%22&view=theater}{ibid 2.187}% OJ doesn't provide the example but points to "2.167"; I (PE) find nothing there and suppose that this was a typo for "2.187"
%also \href{}{ibid 3.107}
) % There's no example on 3.107. (Or on 2.107, or anywhere in volume 1 https://archive.org/details/dramaticworkswit01hankuoft/page/n9/mode/2up . There is an example on 3.114: https://archive.org/details/dramaticworkswit03hankuoft/page/114/mode/2up?q=%22no+end%22 ) I suggest that we either supply that or simply remove this second ibid. (Now DELETED.)
\ex he got up and followed (in no end of a maze one would think)\\\hfill(\href{https://archive.org/details/lovescrosscurren00swinuoft/page/172/mode/2up?q=%22followed%2C+in+no%22&view=theater}{Swinburne, \textit{Cross-currents} 188}) % OJ made quite a mess of this one
\ex They'll make me out no end of a fine fellow\hfill(\href{https://archive.org/details/marriageofwillia0000mrsh_i0u5/page/12/mode/2up?q=%22no+end+of+a+fine%22&view=theater}{Ward, \textit{Marriage} 17})
\ex I feel no end of a man\hfill(\href{https://archive.org/details/magistratefarcei00pinerich/page/38/mode/2up?q=%22feel+no+end+of%22&view=theater}{Pinero, \textit{Magistrate} 38})
\ex this beastly scrape of Theophila's has been no end of a shocker for me\\\hfill(\href{https://archive.org/details/benefitdoubtaco00pinegoog/page/n27/mode/2up?q=%22beastly+scrape%22&view=theater}{Pinero, \textit{Benefit} 12})
\ex We're no end of moral reformers\hfill(\href{https://archive.org/details/stalkyandco015455mbp/page/n167/mode/2up?q=%22end+of+moral%22&view=theater}{Kipling, \textit{Stalky} 171})
\ex About noon there was no end of a snowstorm\hfill(\href{https://archive.org/details/stalkyandco015455mbp/page/n267/mode/2up?q=%22about+noon+there%22&view=theater}{ibid 272})
\ex I sent him no end of an official stinger\hfill(\href{https://archive.org/details/stalkyandco015455mbp/page/n279/mode/2up?q=%22sent+him+no+end+of%22&view=theater}{ibid 284})
\ex you ought to make no end of a good hitter in time {\dots} It was a jolly good rod, and quite fresh, with no end of buds on\\\hfill(\href{https://archive.org/details/lovescrosscurren00swinuoft/page/30/mode/2up?q=%22you+ought+to+make%22&view=theater}{Swinburne, \textit{Cross-currents} 43})
\z
\z
\is{weakened negatives|)}
